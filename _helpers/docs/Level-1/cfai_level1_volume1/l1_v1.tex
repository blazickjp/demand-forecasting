\documentclass[10pt]{article}
\usepackage[utf8]{inputenc}
\usepackage[T1]{fontenc}
\usepackage{amsmath}
\usepackage{amsfonts}
\usepackage{amssymb}
\usepackage[version=4]{mhchem}
\usepackage{stmaryrd}
\usepackage{hyperref}
\hypersetup{colorlinks=true, linkcolor=blue, filecolor=magenta, urlcolor=cyan,}
\urlstyle{same}
\usepackage{graphicx}
\usepackage[export]{adjustbox}
\graphicspath{ {./images/} }
\usepackage{multirow}
\usepackage{fixltx2e}

\title{QUANTITATIVE METHODS }


\author{by Pamela Peterson Drake, PhD, CFA.}
\date{}


\begin{document}
\maketitle
\section{CFA $^{\circledR}$ Program Curriculum 2023 • LEVEL 1 • VOLUME 1}
@2022 by CFA Institute. All rights reserved. This copyright covers material written expressly for this volume by the editor/s as well as the compilation itself. It does not cover the individual selections herein that first appeared elsewhere. Permission to reprint these has been obtained by CFA Institute for this edition only. Further reproductions by any means, electronic or mechanical, including photocopying and recording, or by any information storage or retrieval systems, must be arranged with the individual copyright holders noted.

CFA $^{\circ}$, Chartered Financial Analyst ${ }^{\circ}$, AIMR-PPS $^{\circ}$, and GIPS ${ }^{\circ}$ are just a few of the trademarks owned by CFA Institute. To view a list of CFA Institute trademarks and the Guide for Use of CFA Institute Marks, please visit our website at \href{http://www.cfainstitute.org}{www.cfainstitute.org}.

This publication is designed to provide accurate and authoritative information in regard to the subject matter covered. It is sold with the understanding that the publisher is not engaged in rendering legal, accounting, or other professional service. If legal advice or other expert assistance is required, the services of a competent professional should be sought.

All trademarks, service marks, registered trademarks, and registered service marks are the property of their respective owners and are used herein for identification purposes only.

ISBN 978-1-950157-96-9 (paper)

ISBN 978-1-953337-23-8 (ebook)

2022

\section{CONTENTS}
How to Use the CFA Program Curriculum

Errata

Designing Your Personal Study Program

CFA Institute Learning Ecosystem (LES)

Feedback

Feedback

\section{Quantitative Methods}
\section{Learning Module 1}
The Time Value of Money

3

Introduction 3

Interest Rates $\quad 4$

Future Value of a Single Cash Flow $\quad 6$

$\begin{array}{lr}\text { Non-Annual Compounding (Future Value) } & 10\end{array}$

Continuous Compounding $\quad 12$

$\begin{array}{lr}\text { Stated and Effective Rates } & 14\end{array}$

$\begin{array}{lr}\text { A Series of Cash Flows } & 15\end{array}$

$\begin{array}{lr}\text { Equal Cash Flows-Ordinary Annuity } & 15\end{array}$

Unequal Cash Flows $\quad 16$

Present Value of a Single Cash Flow $\quad 17$

Non-Annual Compounding (Present Value) 19

Present Value of a Series of Equal and Unequal Cash Flows $\quad 21$

$\begin{array}{ll}\text { The Present Value of a Series of Equal Cash Flows } & 21\end{array}$

The Present Value of a Series of Unequal Cash Flows $\quad 25$

Present Value of a Perpetuity $\quad 26$

$\begin{array}{ll}\text { Present Values Indexed at Times Other than } \boldsymbol{t}=0 & 27\end{array}$

Solving for Interest Rates, Growth Rates, and Number of Periods 28

$\begin{array}{lr}\text { Solving for Interest Rates and Growth Rates } & 29\end{array}$

$\begin{array}{ll}\text { Solving for the Number of Periods } & 31\end{array}$

Solving for Size of Annuity Payments $\quad 32$

Present and Future Value Equivalence and the Additivity Principle 36

$\begin{array}{ll}\text { The Cash Flow Additivity Principle } & 38\end{array}$

$\begin{array}{ll}\text { Summary } & 39\end{array}$

Practice Problems $\quad 40$

$\begin{array}{ll}\text { Solutions } & 45\end{array}$

Learning Module $2 \quad$ Organizing, Visualizing, and Describing Data 59

Introduction 59

Data Types $\quad 60$

$\begin{array}{ll}\text { Numerical versus Categorical Data } & 61\end{array}$

Cross-Sectional versus Time-Series versus Panel Data $\quad 63$

$\begin{array}{lr}\text { Structured versus Unstructured Data } & 64\end{array}$

Data Summarization $\quad 68$

Organizing Data for Quantitative Analysis $\quad 68$

$\begin{array}{ll}\text { Summarizing Data Using Frequency Distributions } & 71\end{array}$

Summarizing Data Using a Contingency Table 77 Data Visualization $\quad 82$

$\begin{array}{lr}\text { Histogram and Frequency Polygon } & 82\end{array}$

$\begin{array}{lr}\text { Bar Chart } & 84\end{array}$

$\begin{array}{lr}\text { Tree-Map } & 87\end{array}$

Word Cloud $\quad 88$

$\begin{array}{lr}\text { Line Chart } & 90\end{array}$

$\begin{array}{lr}\text { Scatter Plot } & 92\end{array}$

$\begin{array}{lr}\text { Heat Map } & 96\end{array}$

Guide to Selecting among Visualization Types 98

$\begin{array}{lr}\text { Measures of Central Tendency } & 100\end{array}$

The Arithmetic Mean 101

$\begin{array}{lr}\text { The Median } & 105\end{array}$

The Mode 106

$\begin{array}{lr}\text { Other Concepts of Mean } & 107\end{array}$

Quantiles 116

Quartiles, Quintiles, Deciles, and Percentiles $\quad 117$

Quantiles in Investment Practice $\quad 122$

Measures of Dispersion $\quad 123$

The Range 123

The Mean Absolute Deviation $\quad 124$

Sample Variance and Sample Standard Deviation $\quad 125$

Downside Deviation and Coefficient of Variation $\quad 128$

$\begin{array}{lr}\text { Coefficient of Variation } & 131\end{array}$

The Shape of the Distributions 133

The Shape of the Distributions: Kurtosis 136

$\begin{array}{lr}\text { Correlation between Two Variables } & 139\end{array}$

$\begin{array}{ll}\text { Properties of Correlation } & 140\end{array}$

Limitations of Correlation Analysis $\quad 143$

Summary 146

Practice Problems 151

$\begin{array}{ll}\text { Solutions } & 164\end{array}$

\section{Learning Module 3}
$\begin{array}{lr}\text { Probability Concepts } & 173\end{array}$

Probability Concepts and Odds Ratios $\quad 174$

Probability, Expected Value, and Variance $\quad 174$

$\begin{array}{lr}\text { Conditional and Joint Probability } & 179\end{array}$

Expected Value and Variance $\quad 191$

Portfolio Expected Return and Variance of Return 197

$\begin{array}{ll}\text { Covariance Given a Joint Probability Function } & 202\end{array}$

Bayes' Formula 206

Bayes' Formula $\quad 206$

$\begin{array}{lr}\text { Principles of Counting } & 212\end{array}$

$\begin{array}{ll}\text { Summary } & 218\end{array}$

$\begin{array}{lr}\text { References } & 220\end{array}$

$\begin{array}{ll}\text { Practice Problems } & 221\end{array}$

$\begin{array}{ll}\text { Solutions } & 228\end{array}$ Discrete and Continuous Uniform Distribution $\quad 241$

Continuous Uniform Distribution $\quad 243$

Binomial Distribution $\quad 246$

Normal Distribution $\quad 254$

The Normal Distribution $\quad 254$

Probabilities Using the Normal Distribution $\quad 258$

$\begin{array}{ll}\text { Standardizing a Random Variable } & 260\end{array}$

$\begin{array}{ll}\text { Probabilities Using the Standard Normal Distribution } & 260\end{array}$

$\begin{array}{ll}\text { Applications of the Normal Distribution } & 260\end{array}$

Lognormal Distribution and Continuous Compounding 266

The Lognormal Distribution $\quad 266$

Continuously Compounded Rates of Return $\quad 269$

$\begin{array}{ll}\text { Student's } t-\text {, Chi-Square, and } F \text {-Distributions } & 272\end{array}$

Student's $t$-Distribution $\quad 272$

$\begin{array}{lr}\text { Chi-Square and } F \text {-Distribution } & 274\end{array}$

Monte Carlo Simulation $\quad 279$

Summary 285

Practice Problems $\quad 288$

Solutions $\quad 296$

Learning Module 5

Sampling and Estimation $\quad 303$

$\begin{array}{lr}\text { Introduction } & 304\end{array}$

$\begin{array}{ll}\text { Sampling Methods } & 304\end{array}$

Simple Random Sampling $\quad 305$

Stratified Random Sampling $\quad 306$

$\begin{array}{ll}\text { Cluster Sampling } & 308\end{array}$

$\begin{array}{ll}\text { Non-Probability Sampling } & 309\end{array}$

Sampling from Different Distributions $\quad 313$

The Central Limit Theorem and Distribution of the Sample Mean 315

The Central Limit Theorem $\quad 315$

$\begin{array}{ll}\text { Standard Error of the Sample Mean } & 317\end{array}$

$\begin{array}{ll}\text { Point Estimates of the Population Mean } & 320\end{array}$

$\begin{array}{ll}\text { Point Estimators } & 320\end{array}$

Confidence Intervals for the Population Mean and Sample Size Selection $\quad 324$

$\begin{array}{ll}\text { Selection of Sample Size } & 330\end{array}$

Resampling $\quad 332$

Sampling Related Biases 335

Data Snooping Bias $\quad 336$

Sample Selection Bias $\quad 337$

Look-Ahead Bias $\quad 339$

$\begin{array}{lr}\text { Time-Period Bias } & 340\end{array}$

Summary 341

$\begin{array}{lr}\text { Practice Problems } & 344 \\ \text { Solutions } & 349\end{array}$

$\begin{array}{ll}\text { Solutions } & 349\end{array}$

Why Hypothesis Testing? Implications from a Sampling Distribution 355

The Process of Hypothesis Testing $\quad 356$

Stating the Hypotheses $\quad 357$

Two-Sided vs. One-Sided Hypotheses 357

Selecting the Appropriate Hypotheses 358

Identify the Appropriate Test Statistic $\quad 359$

$\begin{array}{lr}\text { Test Statistics } & 359\end{array}$

Identifying the Distribution of the Test Statistic $\quad 360$

$\begin{array}{ll}\text { Specify the Level of Significance } & 360\end{array}$

State the Decision Rule $\quad 362$

Determining Critical Values $\quad 363$

Decision Rules and Confidence Intervals $\quad 364$

Collect the Data and Calculate the Test Statistic $\quad 365$

Make a Decision $\quad 366$

Make a Statistical Decision $\quad 366$

Make an Economic Decision $\quad 366$

Statistically Significant but Not Economically Significant? 366

The Role of $p$-Values $\quad 367$

$\begin{array}{ll}\text { Multiple Tests and Significance Interpretation } & 370\end{array}$

$\begin{array}{ll}\text { Tests Concerning a Single Mean } & 373\end{array}$

Test Concerning Differences between Means with Independent Samples 377

Test Concerning Differences between Means with Dependent Samples 379

Testing Concerning Tests of Variances $\quad 383$

Tests of a Single Variance $\quad 383$

Test Concerning the Equality of Two Variances (F-Test) 387

Parametric vs. Nonparametric Tests 392

Uses of Nonparametric Tests $\quad 393$

Nonparametric Inference: Summary 393

$\begin{array}{ll}\text { Tests Concerning Correlation } & 394\end{array}$

Parametric Test of a Correlation 395

Tests Concerning Correlation: The Spearman Rank Correlation

Coefficient

397

Test of Independence Using Contingency Table Data 399

$\begin{array}{ll}\text { Summary } & 404\end{array}$

$\begin{array}{lr}\text { References } & 407\end{array}$

Practice Problems 408

Solutions 419

Learning Module $7 \quad$ Introduction to Linear Regression 429

Simple Linear Regression $\quad 429$

Estimating the Parameters of a Simple Linear Regression 432

The Basics of Simple Linear Regression $\quad 432$

Estimating the Regression Line 433

Interpreting the Regression Coefficients 436

Cross-Sectional vs. Time-Series Regressions 437

Assumptions of the Simple Linear Regression Model 440

Assumption 1: Linearity $\quad 440$

Assumption 2: Homoskedasticity 442

Assumption 3: Independence $\quad 444$ Assumption 4: Normality 445

Analysis of Variance $\quad 447$

Breaking down the Sum of Squares Total into Its Components $\quad 448$

$\begin{array}{lr}\text { Measures of Goodness of Fit } & 449\end{array}$

ANOVA and Standard Error of Estimate in Simple Linear Regression $\quad 450$

Hypothesis Testing of Linear Regression Coefficients 453

Hypothesis Tests of the Slope Coefficient 453

Hypothesis Tests of the Intercept 456

Hypothesis Tests of Slope When Independent Variable Is an Indicator Variable 457

Test of Hypotheses: Level of Significance and $\boldsymbol{p}$-Values $\quad 459$

$\begin{array}{ll}\text { Prediction Using Simple Linear Regression and Prediction Intervals } & 460\end{array}$

Functional Forms for Simple Linear Regression 464

The Log-Lin Model 465

The Lin-Log Model $\quad 466$

$\begin{array}{lr}\text { The Log-Log Model } & 468\end{array}$

$\begin{array}{lr}\text { Selecting the Correct Functional Form } & 469\end{array}$

Summary 471

$\begin{array}{lr}\text { Practice Problems } & 474\end{array}$

$\begin{array}{ll}\text { Solutions } & 488\end{array}$

Appendices 493

\section{How to Use the CFA Program Curriculum}
The CFA Program exams measure your mastery of the core knowledge, skills, and abilities required to succeed as an investment professional. These core competencies are the basis for the Candidate Body of Knowledge $\left(\mathrm{CBOK}^{\mathrm{m}}\right)$ ). The CBOK consists of four components:

\begin{itemize}
  \item A broad outline that lists the major CFA Program topic areas (www. \href{http://cfainstitute.org/programs/cfa/curriculum/cbok}{cfainstitute.org/programs/cfa/curriculum/cbok})

  \item Topic area weights that indicate the relative exam weightings of the top-level topic areas (\href{http://www.cfainstitute.org/programs/cfa/curriculum}{www.cfainstitute.org/programs/cfa/curriculum})

  \item Learning outcome statements (LOS) that advise candidates about the specific knowledge, skills, and abilities they should acquire from curriculum content covering a topic area: LOS are provided in candidate study sessions and at the beginning of each block of related content and the specific lesson that covers them. We encourage you to review the information about the LOS on our website (\href{http://www.cfainstitute.org/programs/cfa/curriculum/}{www.cfainstitute.org/programs/cfa/curriculum/} study-sessions), including the descriptions of LOS "command words" on the candidate resources page at \href{http://www.cfainstitute.org}{www.cfainstitute.org}.

  \item The CFA Program curriculum that candidates receive upon exam registration

\end{itemize}

Therefore, the key to your success on the CFA exams is studying and understanding the CBOK. You can learn more about the CBOK on our website: www.cfainstitute. org/programs/cfa/curriculum/cbok.

The entire curriculum, including the practice questions, is the basis for all exam questions and is selected or developed specifically to teach the knowledge, skills, and abilities reflected in the CBOK.

\section{ERRATA}
The curriculum development process is rigorous and includes multiple rounds of reviews by content experts. Despite our efforts to produce a curriculum that is free of errors, there are instances where we must make corrections. Curriculum errata are periodically updated and posted by exam level and test date online on the Curriculum Errata webpage (\href{http://www.cfainstitute.org/en/programs/submit-errata}{www.cfainstitute.org/en/programs/submit-errata}). If you believe you have found an error in the curriculum, you can submit your concerns through our curriculum errata reporting process found at the bottom of the Curriculum Errata webpage.

\section{DESIGNING YOUR PERSONAL STUDY PROGRAM}
An orderly, systematic approach to exam preparation is critical. You should dedicate a consistent block of time every week to reading and studying. Review the LOS both before and after you study curriculum content to ensure that you have mastered the applicable content and can demonstrate the knowledge, skills, and abilities described by the LOS and the assigned reading. Use the LOS self-check to track your progress and highlight areas of weakness for later review.

Successful candidates report an average of more than 300 hours preparing for each exam. Your preparation time will vary based on your prior education and experience, and you will likely spend more time on some study sessions than on others.

\section{CFA INSTITUTE LEARNING ECOSYSTEM (LES)}
Your exam registration fee includes access to the CFA Program Learning Ecosystem (LES). This digital learning platform provides access, even offline, to all of the curriculum content and practice questions and is organized as a series of short online lessons with associated practice questions. This tool is your one-stop location for all study materials, including practice questions and mock exams, and the primary method by which CFA Institute delivers your curriculum experience. The LES offers candidates additional practice questions to test their knowledge, and some questions in the LES provide a unique interactive experience.

\section{FEEDBACK}
Please send any comments or feedback to \href{mailto:info@cfainstitute.org}{info@cfainstitute.org}, and we will review your suggestions carefully.

\section{Quantitative Methods}
\section{LEARNING MODULE
1}
\section{The Time Value of Money}
by Richard A. DeFusco, PhD, CFA, Dennis W. McLeavey, DBA, CFA, Jerald E. Pinto, PhD, CFA, and David E. Runkle, PhD, CFA.

Richard A. DeFusco, PhD, CFA, is at the University of Nebraska-Lincoln (USA). Dennis W. McLeavey, DBA, CFA, is at the University of Rhode Island (USA). Jerald E. Pinto, PhD, CFA, is at CFA Institute (USA). David E. Runkle, PhD, CFA, is at Jacobs Levy Equity Management (USA).

\section{LEARNING OUTCOME}
\begin{center}
\begin{tabular}{c|l}
Mastery & The candidate should be able to: \\
\hline
$\square$ & $\begin{array}{l}\text { interpret interest rates as required rates of return, discount rates, or } \\ \text { opportunity costs } \\ \text { explain an interest rate as the sum of a real risk-free rate and } \\ \text { premiums that compensate investors for bearing distinct types of } \\ \text { risk } \\ \text { calculate and interpret the future value (FV) and present value (PV) } \\ \text { of a single sum of money, an ordinary annuity, an annuity due, a } \\ \text { perpetuity (PV only), and a series of unequal cash flows } \\ \text { demonstrate the use of a time line in modeling and solving time } \\ \text { value of money problems } \\ \text { calculate the solution for time value of money problems with } \\ \text { different frequencies of compounding } \\ \text { calculate and interpret the effective annual rate, given the stated } \\ \text { annual interest rate and the frequency of compounding }\end{array}$ \\
$\square$ &  \\
\end{tabular}
\end{center}

\section{INTRODUCTION}
As individuals, we often face decisions that involve saving money for a future use, or borrowing money for current consumption. We then need to determine the amount we need to invest, if we are saving, or the cost of borrowing, if we are shopping for a loan. As investment analysts, much of our work also involves evaluating transactions with present and future cash flows. When we place a value on any security, for example, we are attempting to determine the worth of a stream of future cash flows. To carry out all the above tasks accurately, we must understand the mathematics of time value of money problems. Money has time value in that individuals value a given amount of money more highly the earlier it is received. Therefore, a smaller amount of money now may be equivalent in value to a larger amount received at a future date. The time value of money as a topic in investment mathematics deals with equivalence relationships between cash flows with different dates. Mastery of time value of money concepts and techniques is essential for investment analysts.

The reading ${ }^{1}$ is organized as follows: Section 2 introduces some terminology used throughout the reading and supplies some economic intuition for the variables we will discuss. Section 3 tackles the problem of determining the worth at a future point in time of an amount invested today. Section 4 addresses the future worth of a series of cash flows. These two sections provide the tools for calculating the equivalent value at a future date of a single cash flow or series of cash flows. Sections 5 and 6 discuss the equivalent value today of a single future cash flow and a series of future cash flows, respectively. In Section 7, we explore how to determine other quantities of interest in time value of money problems.

\section{INTEREST RATES}
interpret interest rates as required rates of return, discount rates, or opportunity costs

explain an interest rate as the sum of a real risk-free rate and premiums that compensate investors for bearing distinct types of risk

In this reading, we will continually refer to interest rates. In some cases, we assume a particular value for the interest rate; in other cases, the interest rate will be the unknown quantity we seek to determine. Before turning to the mechanics of time value of money problems, we must illustrate the underlying economic concepts. In this section, we briefly explain the meaning and interpretation of interest rates.

Time value of money concerns equivalence relationships between cash flows occurring on different dates. The idea of equivalence relationships is relatively simple. Consider the following exchange: You pay $\$ 10,000$ today and in return receive $\$ 9,500$ today. Would you accept this arrangement? Not likely. But what if you received the $\$ 9,500$ today and paid the $\$ 10,000$ one year from now? Can these amounts be considered equivalent? Possibly, because a payment of $\$ 10,000$ a year from now would probably be worth less to you than a payment of $\$ 10,000$ today. It would be fair, therefore, to discount the $\$ 10,000$ received in one year; that is, to cut its value based on how much time passes before the money is paid. An interest rate, denoted $r$, is a rate of return that reflects the relationship between differently dated cash flows. If $\$ 9,500$ today and $\$ 10,000$ in one year are equivalent in value, then $\$ 10,000-\$ 9,500=\$ 500$ is the required compensation for receiving $\$ 10,000$ in one year rather than now. The interest rate-the required compensation stated as a rate of return-is $\$ 500 / \$ 9,500$ $=0.0526$ or 5.26 percent.

Interest rates can be thought of in three ways. First, they can be considered required rates of return - that is, the minimum rate of return an investor must receive in order to accept the investment. Second, interest rates can be considered discount rates. In the example above, 5.26 percent is that rate at which we discounted the $\$ 10,000$ future amount to find its value today. Thus, we use the terms "interest rate" and "discount rate" almost interchangeably. Third, interest rates can be considered opportunity costs.

1 Examples in this reading and other readings in quantitative methods at Level I were updated in 2018 by Professor Sanjiv Sabherwal of the University of Texas, Arlington. An opportunity cost is the value that investors forgo by choosing a particular course of action. In the example, if the party who supplied $\$ 9,500$ had instead decided to spend it today, he would have forgone earning 5.26 percent on the money. So we can view 5.26 percent as the opportunity cost of current consumption.

Economics tells us that interest rates are set in the marketplace by the forces of supply and demand, where investors are suppliers of funds and borrowers are demanders of funds. Taking the perspective of investors in analyzing market-determined interest rates, we can view an interest rate $r$ as being composed of a real risk-free interest rate plus a set of four premiums that are required returns or compensation for bearing distinct types of risk:

$r=$ Real risk-free interest rate + Inflation premium + Default risk premium + Liquidity premium + Maturity premium

\begin{itemize}
  \item The real risk-free interest rate is the single-period interest rate for a completely risk-free security if no inflation were expected. In economic theory, the real risk-free rate reflects the time preferences of individuals for current versus future real consumption.

  \item The inflation premium compensates investors for expected inflation and reflects the average inflation rate expected over the maturity of the debt. Inflation reduces the purchasing power of a unit of currency-the amount of goods and services one can buy with it. The sum of the real risk-free interest rate and the inflation premium is the nominal risk-free interest rate. ${ }^{2}$ Many countries have governmental short-term debt whose interest rate can be considered to represent the nominal risk-free interest rate in that country. The interest rate on a 90-day US Treasury bill (T-bill), for example, represents the nominal risk-free interest rate over that time horizon. ${ }^{3}$ US T-bills can be bought and sold in large quantities with minimal transaction costs and are backed by the full faith and credit of the US government.

  \item The default risk premium compensates investors for the possibility that the borrower will fail to make a promised payment at the contracted time and in the contracted amount.

  \item The liquidity premium compensates investors for the risk of loss relative to an investment's fair value if the investment needs to be converted to cash quickly. US T-bills, for example, do not bear a liquidity premium because large amounts can be bought and sold without affecting their market price. Many bonds of small issuers, by contrast, trade infrequently after they are issued; the interest rate on such bonds includes a liquidity premium reflecting the relatively high costs (including the impact on price) of selling a position.

  \item The maturity premium compensates investors for the increased sensitivity of the market value of debt to a change in market interest rates as maturity is extended, in general (holding all else equal). The difference between the

\end{itemize}

2 Technically, 1 plus the nominal rate equals the product of 1 plus the real rate and 1 plus the inflation rate. As a quick approximation, however, the nominal rate is equal to the real rate plus an inflation premium. In this discussion we focus on approximate additive relationships to highlight the underlying concepts. 3 Other developed countries issue securities similar to US Treasury bills. The French government issues BTFs or negotiable fixed-rate discount Treasury bills (Bons du Trésor àtaux fixe et à intérêts précomptés) with maturities of up to one year. The Japanese government issues a short-term Treasury bill with maturities of 6 and 12 months. The German government issues at discount both Treasury financing paper (Finanzierungsschätze des Bundes or, for short, Schätze) and Treasury discount paper (Bubills) with maturities up to 24 months. In the United Kingdom, the British government issues gilt-edged Treasury bills with maturities ranging from 1 to 364 days. The Canadian government bond market is closely related to the US market; Canadian Treasury bills have maturities of 3,6 , and 12 months. The Time Value of Money

interest rate on longer-maturity, liquid Treasury debt and that on short-term Treasury debt reflects a positive maturity premium for the longer-term debt (and possibly different inflation premiums as well).

Using this insight into the economic meaning of interest rates, we now turn to a discussion of solving time value of money problems, starting with the future value of a single cash flow.

FUTURE VALUE OF A SINGLE CASH FLOW

calculate and interpret the future value (FV) and present value (PV) of a single sum of money, an ordinary annuity, an annuity due, a perpetuity (PV only), and a series of unequal cash flows demonstrate the use of a time line in modeling and solving time value of money problems

In this section, we introduce time value associated with a single cash flow or lump-sum investment. We describe the relationship between an initial investment or present value (PV), which earns a rate of return (the interest rate per period) denoted as $r$, and its future value (FV), which will be received $N$ years or periods from today.

The following example illustrates this concept. Suppose you invest $\$ 100(\mathrm{PV}=$ \$100) in an interest-bearing bank account paying 5 percent annually. At the end of the first year, you will have the $\$ 100$ plus the interest earned, $0.05 \times \$ 100=\$ 5$, for a total of $\$ 105$. To formalize this one-period example, we define the following terms:

$$
\begin{gathered}
\mathrm{PV}=\text { present value of the investment } \\
\mathrm{FV}_{N}=\text { future value of the investment } N \text { periods from today } \\
r=\text { rate of interest per period } \\
\text { For } N=1 \text {, the expression for the future value of amount } \mathrm{PV} \text { is } \\
\mathrm{FV}_{1}=\mathrm{PV}(1+r)
\end{gathered}
$$

For this example, we calculate the future value one year from today as $\mathrm{FV}_{1}=\$ 100(1.05)$ $=\$ 105$.

Now suppose you decide to invest the initial $\$ 100$ for two years with interest earned and credited to your account annually (annual compounding). At the end of the first year (the beginning of the second year), your account will have $\$ 105$, which you will leave in the bank for another year. Thus, with a beginning amount of $\$ 105$ $(\mathrm{PV}=\$ 105)$, the amount at the end of the second year will be $\$ 105(1.05)=\$ 110.25$. Note that the $\$ 5.25$ interest earned during the second year is 5 percent of the amount invested at the beginning of Year 2.

Another way to understand this example is to note that the amount invested at the beginning of Year 2 is composed of the original $\$ 100$ that you invested plus the $\$ 5$ interest earned during the first year. During the second year, the original principal again earns interest, as does the interest that was earned during Year 1. You can see how the original investment grows:

Original investment $\$ 100.00$

Interest for the first year $(\$ 100 \times 0.05)$ 5.00

Interest for the second year based on original investment $(\$ 100 \times 0.05)$ Interest for the second year based on interest earned in the first year $(0.05 \times$ $\$ 5.00$ interest on interest)

Total

0.25

$\$ 110.25$

The $\$ 5$ interest that you earned each period on the $\$ 100$ original investment is known as simple interest (the interest rate times the principal). Principal is the amount of funds originally invested. During the two-year period, you earn $\$ 10$ of simple interest. The extra $\$ 0.25$ that you have at the end of Year 2 is the interest you earned on the Year 1 interest of $\$ 5$ that you reinvested.

The interest earned on interest provides the first glimpse of the phenomenon known as compounding. Although the interest earned on the initial investment is important, for a given interest rate it is fixed in size from period to period. The compounded interest earned on reinvested interest is a far more powerful force because, for a given interest rate, it grows in size each period. The importance of compounding increases with the magnitude of the interest rate. For example, $\$ 100$ invested today would be worth about $\$ 13,150$ after 100 years if compounded annually at 5 percent, but worth more than $\$ 20$ million if compounded annually over the same time period at a rate of 13 percent.

To verify the $\$ 20$ million figure, we need a general formula to handle compounding for any number of periods. The following general formula relates the present value of an initial investment to its future value after $N$ periods:

$$
\mathrm{FV}_{N}=\mathrm{PV}(1+r)^{N}
$$

where $r$ is the stated interest rate per period and $N$ is the number of compounding periods. In the bank example, $\mathrm{FV}_{2}=\$ 100(1+0.05)^{2}=\$ 110.25$. In the 13 percent investment example, $\mathrm{FV}_{100}=\$ 100(1.13)^{100}=\$ 20,316,287.42$.

The most important point to remember about using the future value equation is that the stated interest rate, $r$, and the number of compounding periods, $N$, must be compatible. Both variables must be defined in the same time units. For example, if $N$ is stated in months, then $r$ should be the one-month interest rate, unannualized.

A time line helps us to keep track of the compatibility of time units and the interest rate per time period. In the time line, we use the time index $t$ to represent a point in time a stated number of periods from today. Thus the present value is the amount available for investment today, indexed as $t=0$. We can now refer to a time $N$ periods from today as $t=N$. The time line in Exhibit 1 shows this relationship.

Exhibit 1: The Relationship between an Initial Investment, PV, and Its Future

Value, FV

\begin{center}
\includegraphics[max width=\textwidth]{2023_05_04_cff39ee44f77d6514e1bg-017}
\end{center}

In Exhibit 1, we have positioned the initial investment, $\mathrm{PV}$, at $t=0$. Using Equation 2, we move the present value, $\mathrm{PV}$, forward to $t=N$ by the factor $(1+r)^{N}$. This factor is called a future value factor. We denote the future value on the time line as FV and position it at $t=N$. Suppose the future value is to be received exactly 10 periods from today's date $(N=10)$. The present value, PV, and the future value, FV, are separated in time through the factor $(1+r)^{10}$.

The fact that the present value and the future value are separated in time has important consequences:

\begin{itemize}
  \item We can add amounts of money only if they are indexed at the same point in time.

  \item For a given interest rate, the future value increases with the number of periods.

  \item For a given number of periods, the future value increases with the interest rate.

\end{itemize}

To better understand these concepts, consider three examples that illustrate how to apply the future value formula.

\section{EXAMPLE 1}
The Future Value of a Lump Sum with Interim Cash

Reinvested at the Same Rate

\begin{enumerate}
  \item You are the lucky winner of your state's lottery of $\$ 5$ million after taxes. You invest your winnings in a five-year certificate of deposit (CD) at a local financial institution. The CD promises to pay 7 percent per year compounded annually. This institution also lets you reinvest the interest at that rate for the duration of the CD. How much will you have at the end of five years if your money remains invested at 7 percent for five years with no withdrawals?
\end{enumerate}

\section{Solution:}
To solve this problem, compute the future value of the $\$ 5$ million investment using the following values in Equation 2:

$$
\begin{aligned}
& \mathrm{PV}=\$ 5,000,000 \\
& r=7 \%=0.07 \\
& N=5 \\
& \mathrm{FV}_{N}=\mathrm{PV}(1+r)^{N} \\
& =\$ 5,000,000(1.07)^{5} \\
& =\$ 5,000,000(1.402552) \\
& =\$ 7,012,758.65
\end{aligned}
$$

At the end of five years, you will have $\$ 7,012,758.65$ if your money remains invested at 7 percent with no withdrawals.

In this and most examples in this reading, note that the factors are reported at six decimal places but the calculations may actually reflect greater precision. For example, the reported 1.402552 has been rounded up from 1.40255173 (the calculation is actually carried out with more than eight decimal places of precision by the calculator or spreadsheet). Our final result reflects the higher number of decimal places carried by the calculator or spreadsheet. ${ }^{4}$

\begin{enumerate}
  \setcounter{enumi}{3}
  \item We could also solve time value of money problems using tables of interest rate factors. Solutions using tabled values of interest rate factors are generally less accurate than solutions obtained using calculators or spreadsheets, so practitioners prefer calculators or spreadsheets.
\end{enumerate}

\section{EXAMPLE 2}
\section{The Future Value of a Lump Sum with No Interim Cash}
\begin{enumerate}
  \item An institution offers you the following terms for a contract: For an investment of $¥ 2,500,000$, the institution promises to pay you a lump sum six years from now at an 8 percent annual interest rate. What future amount can you expect?
\end{enumerate}

\section{Solution:}
Use the following data in Equation 2 to find the future value:

$$
\begin{aligned}
& \mathrm{PV}=¥ 2,500,000 \\
& r=8 \%=0.08 \\
& N=6 \\
& \mathrm{FV}_{N}=\mathrm{PV}(1+r)^{N} \\
& =¥ 2,500,000(1.08)^{6} \\
& =¥ 2,500,000(1.586874) \\
& =¥ 3,967,186
\end{aligned}
$$

You can expect to receive $¥ 3,967,186$ six years from now.

Our third example is a more complicated future value problem that illustrates the importance of keeping track of actual calendar time.

\section{EXAMPLE 3}
\section{The Future Value of a Lump Sum}
\begin{enumerate}
  \item A pension fund manager estimates that his corporate sponsor will make a $\$ 10$ million contribution five years from now. The rate of return on plan assets has been estimated at 9 percent per year. The pension fund manager wants to calculate the future value of this contribution 15 years from now, which is the date at which the funds will be distributed to retirees. What is that future value?
\end{enumerate}

\section{Solution:}
By positioning the initial investment, PV, at $t=5$, we can calculate the future value of the contribution using the following data in Equation 2:

$$
\begin{aligned}
& \mathrm{PV}=\$ 10 \text { million } \\
& r=9 \%=0.09 \\
& N=10 \\
& \mathrm{FV}_{N}=\mathrm{PV}(1+r)^{N} \\
& =\$ 10,000,000(1.09)^{10} \\
& =\$ 10,000,000(2.367364) \\
& =\$ 23,673,636.75
\end{aligned}
$$

This problem looks much like the previous two, but it differs in one important respect: its timing. From the standpoint of today $(t=0)$, the future amount of $\$ 23,673,636.75$ is 15 years into the future. Although the future value is 10 years from its present value, the present value of $\$ 10$ million will not be received for another five years. Exhibit 2: The Future Value of a Lump Sum, Initial Investment Not at $\boldsymbol{t}=0$

\begin{center}
\includegraphics[max width=\textwidth]{2023_05_04_cff39ee44f77d6514e1bg-020}
\end{center}

As Exhibit 2 shows, we have followed the convention of indexing today as $t=0$ and indexing subsequent times by adding 1 for each period. The additional contribution of $\$ 10$ million is to be received in five years, so it is indexed as $t=5$ and appears as such in the figure. The future value of the investment in 10 years is then indexed at $t=15$; that is, 10 years following the receipt of the $\$ 10$ million contribution at $t=5$. Time lines like this one can be extremely useful when dealing with more-complicated problems, especially those involving more than one cash flow.

In a later section of this reading, we will discuss how to calculate the value today of the $\$ 10$ million to be received five years from now. For the moment, we can use Equation 2. Suppose the pension fund manager in Example 3 above were to receive $\$ 6,499,313.86$ today from the corporate sponsor. How much will that sum be worth at the end of five years? How much will it be worth at the end of 15 years?

$$
\begin{aligned}
& \mathrm{PV}=\$ 6,499,313.86 \\
& r=9 \%=0.09 \\
& N=5 \\
& \mathrm{FV}_{N}=\mathrm{PV}(1+r)^{N} \\
& =\$ 6,499,313.86(1.09)^{5} \\
& =\$ 6,499,313.86(1.538624) \\
& =\$ 10,000,000 \text { at the five-year mark } \\
& \text { and } \\
& \mathrm{PV}=\$ 6,499,313.86 \\
& r=9 \%=0.09 \\
& N=15 \\
& \mathrm{FV}_{N}=\mathrm{PV}(1+r)^{N} \\
& =\$ 6,499,313.86(1.09)^{15} \\
& =\$ 6,499,313.86(3.642482) \\
& =\$ 23,673,636.74 \text { at the } 15 \text {-year mark }
\end{aligned}
$$

These results show that today's present value of about $\$ 6.5$ million becomes $\$ 10$ million after five years and $\$ 23.67$ million after 15 years.

\section{NON-ANNUAL COMPOUNDING (FUTURE VALUE)}
calculate the solution for time value of money problems with different frequencies of compounding In this section, we examine investments paying interest more than once a year. For instance, many banks offer a monthly interest rate that compounds 12 times a year. In such an arrangement, they pay interest on interest every month. Rather than quote the periodic monthly interest rate, financial institutions often quote an annual interest rate that we refer to as the stated annual interest rate or quoted interest rate. We denote the stated annual interest rate by $r_{s}$. For instance, your bank might state that a particular CD pays 8 percent compounded monthly. The stated annual interest rate equals the monthly interest rate multiplied by 12 . In this example, the monthly interest rate is $0.08 / 12=0.0067$ or 0.67 percent. $^{5}$ This rate is strictly a quoting convention because $(1+0.0067)^{12}=1.083$, not 1.08 ; the term $\left(1+r_{s}\right)$ is not meant to be a future value factor when compounding is more frequent than annual.

With more than one compounding period per year, the future value formula can be expressed as

$$
\mathrm{FV}_{N}=\mathrm{PV}\left(1+\frac{r_{s}}{m}\right)^{m N}
$$

where $r_{s}$ is the stated annual interest rate, $m$ is the number of compounding periods per year, and $N$ now stands for the number of years. Note the compatibility here between the interest rate used, $r_{s} / m$, and the number of compounding periods, $m N$. The periodic rate, $r_{s} / m$, is the stated annual interest rate divided by the number of compounding periods per year. The number of compounding periods, $m N$, is the number of compounding periods in one year multiplied by the number of years. The periodic rate, $r_{s} / m$, and the number of compounding periods, $m N$, must be compatible.

\section{EXAMPLE 4}
\section{The Future Value of a Lump Sum with Quarterly Compounding}
\begin{enumerate}
  \item Continuing with the CD example, suppose your bank offers you a CD with a two-year maturity, a stated annual interest rate of 8 percent compounded quarterly, and a feature allowing reinvestment of the interest at the same interest rate. You decide to invest $\$ 10,000$. What will the $\mathrm{CD}$ be worth at maturity?
\end{enumerate}

\section{Solution:}
Compute the future value with Equation 3 as follows:

$$
\begin{aligned}
& \mathrm{PV}=\$ 10,000 \\
& r_{s}=8 \%=0.08 \\
& m=4 \\
& r_{s} / m=0.08 / 4=0.02 \\
& N=2 \\
& m N=4(2)=8 \text { interest periods } \\
& \mathrm{FV}_{N}=\mathrm{PV}\left(1+\frac{r_{s}}{m}\right)^{m N} \\
& =\$ 10,000(1.02)^{8} \\
& =\$ 10,000(1.171659) \\
& =\$ 11,716.59
\end{aligned}
$$

At maturity, the CD will be worth $\$ 11,716.59$.

5 To avoid rounding errors when using a financial calculator, divide 8 by 12 and then press the $\% i$ key, rather than simply entering 0.67 for $\% i$, so we have $(1+0.08 / 12)^{12}=1.083000$. The future value formula in Equation 3 does not differ from the one in Equation 2. Simply keep in mind that the interest rate to use is the rate per period and the exponent is the number of interest, or compounding, periods.

\section{EXAMPLE 5}
\section{The Future Value of a Lump Sum with Monthly Compounding}
\begin{enumerate}
  \item An Australian bank offers to pay you 6 percent compounded monthly. You decide to invest $\mathrm{A} \$ 1$ million for one year. What is the future value of your investment if interest payments are reinvested at 6 percent?
\end{enumerate}

\section{Solution:}
Use Equation 3 to find the future value of the one-year investment as follows:

$$
\begin{aligned}
& \mathrm{PV}=\mathrm{A} \$ 1,000,000 \\
& r_{s}=6 \%=0.06 \\
& m=12 \\
& r_{s} / m=0.06 / 12=0.0050 \\
& N=1 \\
& m N=12(1)=12 \text { interest periods } \\
& \mathrm{FV}_{N}=\mathrm{PV}\left(1+\frac{r_{s}}{m}\right)^{m N} \\
& =\mathrm{A} \$ 1,000,000(1.005)^{12} \\
& =\mathrm{A} \$ 1,000,000(1.061678) \\
& =\mathrm{A} \$ 1,061,677.81
\end{aligned}
$$

If you had been paid 6 percent with annual compounding, the future amount would be only A $\$ 1,000,000(1.06)=\mathrm{A} \$ 1,060,000$ instead of A $\$ 1,061,677.81$ with monthly compounding.

\section{CONTINUOUS COMPOUNDING}
calculate and interpret the effective annual rate, given the stated annual interest rate and the frequency of compounding calculate the solution for time value of money problems with different frequencies of compounding

The preceding discussion on compounding periods illustrates discrete compounding, which credits interest after a discrete amount of time has elapsed. If the number of compounding periods per year becomes infinite, then interest is said to compound continuously. If we want to use the future value formula with continuous compounding, we need to find the limiting value of the future value factor for $m \rightarrow \infty$ (infinitely many compounding periods per year) in Equation 3. The expression for the future value of a sum in $N$ years with continuous compounding is

$$
\mathrm{FV}_{N}=\mathrm{PV} e^{r_{s} N}
$$

The term $e^{r_{s} N}$ is the transcendental number $e \approx 2.7182818$ raised to the power $r_{s} N$. Most financial calculators have the function $e^{x}$.

\section{EXAMPLE 6}
\section{The Future Value of a Lump Sum with Continuous Compounding}
Suppose a $\$ 10,000$ investment will earn 8 percent compounded continuously for two years. We can compute the future value with Equation 4 as follows:

$$
\begin{aligned}
& \mathrm{PV}=\$ 10,000 \\
& r_{s}=8 \%=0.08 \\
& N=2 \\
& \mathrm{FV}_{N}=\mathrm{PV} e^{r_{s} N} \\
& =\$ 10,000 e^{0.08(2)} \\
& =\$ 10,000(1.173511) \\
& =\$ 11,735.11
\end{aligned}
$$

With the same interest rate but using continuous compounding, the $\$ 10,000$ investment will grow to $\$ 11,735.11$ in two years, compared with $\$ 11,716.59$ using quarterly compounding as shown in Example 4.

Exhibit 3 shows how a stated annual interest rate of 8 percent generates different ending dollar amounts with annual, semiannual, quarterly, monthly, daily, and continuous compounding for an initial investment of $\$ 1$ (carried out to six decimal places).

As Exhibit 3 shows, all six cases have the same stated annual interest rate of 8

\begin{center}
\begin{tabular}{|c|c|c|c|c|c|}
\hline
\multirow{2}{*}{$\frac{\text { Frequency }}{\text { Annual }}$} & \multirow{2}{*}{$\frac{\boldsymbol{r}_{\boldsymbol{s}} / \boldsymbol{m}}{8 \% / 1=8 \%}$} & \multirow{2}{*}{$\frac{m N}{1 \times 1=1}$} & \multicolumn{3}{|c|}{Future Value of \$1} \\
\hline
 &  &  & $\$ 1.00(1.08)$ & $=$ & $\$ 1.08$ \\
\hline
Semiannual & $8 \% / 2=4 \%$ & $2 \times 1=2$ & $\$ 1.00(1.04)^{2}$ & $=$ & $\$ 1.081600$ \\
\hline
Quarterly & $8 \% / 4=2 \%$ & $4 \times 1=4$ & $\$ 1.00(1.02)^{4}$ & $=$ & $\$ 1.082432$ \\
\hline
Monthly & $8 \% / 12=0.6667 \%$ & $12 \times 1=12$ & $\$ 1.00(1.006667)^{12}$ & $=$ & $\$ 1.083000$ \\
\hline
Daily & $8 \% / 365=0.0219 \%$ & $365 \times 1=365$ & $\$ 1.00(1.000219)^{365}$ & $=$ & $\$ 1.083278$ \\
\hline
Continuous &  &  & $\$ 1.00 e^{0.08(1)}$ & $=$ & $\$ 1.083287$ \\
\hline
\end{tabular}
\end{center}

percent; they have different ending dollar amounts, however, because of differences in the frequency of compounding. With annual compounding, the ending amount is $\$ 1.08$. More frequent compounding results in larger ending amounts. The ending dollar amount with continuous compounding is the maximum amount that can be earned with a stated annual rate of 8 percent.

\section{Exhibit 3: The Effect of Compounding Frequency on Future Value}
Exhibit 3 also shows that a $\$ 1$ investment earning 8.16 percent compounded annually grows to the same future value at the end of one year as a $\$ 1$ investment earning 8 percent compounded semiannually. This result leads us to a distinction between the stated annual interest rate and the effective annual rate (EAR). ${ }^{6}$ For an 8 percent stated annual interest rate with semiannual compounding, the EAR is 8.16 percent.

6 Among the terms used for the effective annual return on interest-bearing bank deposits are annual percentage yield (APY) in the United States and equivalent annual rate (EAR) in the United Kingdom. By contrast, the annual percentage rate (APR) measures the cost of borrowing expressed as a yearly The Time Value of Money

\section{Stated and Effective Rates}
The stated annual interest rate does not give a future value directly, so we need a formula for the EAR. With an annual interest rate of 8 percent compounded semiannually, we receive a periodic rate of 4 percent. During the course of a year, an investment of $\$ 1$ would grow to $\$ 1(1.04)^{2}=\$ 1.0816$, as illustrated in Exhibit 3 . The interest earned on the $\$ 1$ investment is $\$ 0.0816$ and represents an effective annual rate of interest of 8.16 percent. The effective annual rate is calculated as follows:

$$
\mathrm{EAR}=(1+\text { Periodic interest rate })^{m}-1
$$

The periodic interest rate is the stated annual interest rate divided by $m$, where $m$ is the number of compounding periods in one year. Using our previous example, we can solve for EAR as follows: $(1.04)^{2}-1=8.16$ percent.

The concept of EAR extends to continuous compounding. Suppose we have a rate of 8 percent compounded continuously. We can find the EAR in the same way as above by finding the appropriate future value factor. In this case, a $\$ 1$ investment would grow to $\$ 1 e^{0.08(1.0)}=\$ 1.0833$. The interest earned for one year represents an effective annual rate of 8.33 percent and is larger than the 8.16 percent EAR with semiannual compounding because interest is compounded more frequently. With continuous compounding, we can solve for the effective annual rate as follows:

$$
\mathrm{EAR}=e^{r_{s}-1}
$$

We can reverse the formulas for EAR with discrete and continuous compounding to find a periodic rate that corresponds to a particular effective annual rate. Suppose we want to find the appropriate periodic rate for a given effective annual rate of 8.16 percent with semiannual compounding. We can use Equation 5 to find the periodic rate:

$$
\begin{aligned}
& 0.0816=(1+\text { Periodic rate })^{2}-1 \\
& 1.0816=(1+\text { Periodic rate })^{2} \\
& (1.0816)^{1 / 2}-1=\text { Periodic rate } \\
& (1.04)-1=\text { Periodic rate } \\
& 4 \%=\text { Periodic rate }
\end{aligned}
$$

To calculate the continuously compounded rate (the stated annual interest rate with continuous compounding) corresponding to an effective annual rate of 8.33 percent, we find the interest rate that satisfies Equation 6:

$$
\begin{aligned}
0.0833 & =e^{r_{s}-1} \\
1.0833 & =e^{r_{s}}
\end{aligned}
$$

To solve this equation, we take the natural logarithm of both sides. (Recall that the natural $\log$ of $e^{r_{s}}$ is $\ln e^{r_{s}}=r_{s}$.) Therefore, $\ln 1.0833=r_{s}$, resulting in $r_{s}=8$ percent. We see that a stated annual rate of 8 percent with continuous compounding is equivalent to an EAR of 8.33 percent.

rate. In the United States, the APR is calculated as a periodic rate times the number of payment periods per year and, as a result, some writers use APR as a general synonym for the stated annual interest rate. Nevertheless, APR is a term with legal connotations; its calculation follows regulatory standards that vary internationally. Therefore, "stated annual interest rate" is the preferred general term for an annual interest rate that does not account for compounding within the year.

\section{A SERIES OF CASH FLOWS}
calculate and interpret the future value (FV) and present value (PV) of a single sum of money, an ordinary annuity, an annuity due, a perpetuity (PV only), and a series of unequal cash flows demonstrate the use of a time line in modeling and solving time value of money problems

In this section, we consider series of cash flows, both even and uneven. We begin with a list of terms commonly used when valuing cash flows that are distributed over many time periods.

\begin{itemize}
  \item An annuity is a finite set of level sequential cash flows.

  \item An ordinary annuity has a first cash flow that occurs one period from now (indexed at $t=1$ ).

  \item An annuity due has a first cash flow that occurs immediately (indexed at $t$ $=0)$.

  \item A perpetuity is a perpetual annuity, or a set of level never-ending sequential cash flows, with the first cash flow occurring one period from now.

\end{itemize}

\section{Equal Cash Flows-Ordinary Annuity}
Consider an ordinary annuity paying 5 percent annually. Suppose we have five separate deposits of $\$ 1,000$ occurring at equally spaced intervals of one year, with the first payment occurring at $t=1$. Our goal is to find the future value of this ordinary annuity after the last deposit at $t=5$. The increment in the time counter is one year, so the last payment occurs five years from now. As the time line in Exhibit 4 shows, we find the future value of each $\$ 1,000$ deposit as of $t=5$ with Equation 2, $\mathrm{FV}_{N}=\mathrm{PV}(1+r)^{N}$. The arrows in Exhibit 4 extend from the payment date to $t=5$. For instance, the first $\$ 1,000$ deposit made at $t=1$ will compound over four periods. Using Equation 2, we find that the future value of the first deposit at $t=5$ is $\$ 1,000(1.05)^{4}=\$ 1,215.51$. We calculate the future value of all other payments in a similar fashion. (Note that we are finding the future value at $t=5$, so the last payment does not earn any interest.) With all values now at $t=5$, we can add the future values to arrive at the future value of the annuity. This amount is $\$ 5,525.63$.

Exhibit 4: The Future Value of a Five-Year Ordinary Annuity

\begin{center}
\includegraphics[max width=\textwidth]{2023_05_04_cff39ee44f77d6514e1bg-025}
\end{center}

We can arrive at a general annuity formula if we define the annuity amount as $A$, the number of time periods as $N$, and the interest rate per period as $r$. We can then define the future value as

$$
\mathrm{FV}_{N}=A\left[(1+r)^{N-1}+(1+r)^{N-2}+(1+r)^{N-3}+\ldots+(1+r)^{1}+(1+r)^{0}\right]
$$

which simplifies to

$$
\mathrm{FV}_{N}=A\left[\frac{(1+r)^{N}-1}{r}\right]
$$

The term in brackets is the future value annuity factor. This factor gives the future value of an ordinary annuity of $\$ 1$ per period. Multiplying the future value annuity factor by the annuity amount gives the future value of an ordinary annuity. For the ordinary annuity in Exhibit 4, we find the future value annuity factor from Equation 7 as

$$
\left[\frac{(1.05)^{5}-1}{0.05}\right]=5.525631
$$

With an annuity amount $A=\$ 1,000$, the future value of the annuity is $\$ 1,000(5.525631)$ $=\$ 5,525.63$, an amount that agrees with our earlier work.

The next example illustrates how to find the future value of an ordinary annuity using the formula in Equation 7.

\section{EXAMPLE 7}
\section{The Future Value of an Annuity}
\begin{enumerate}
  \item Suppose your company's defined contribution retirement plan allows you to invest up to $€ 20,000$ per year. You plan to invest $€ 20,000$ per year in a stock index fund for the next 30 years. Historically, this fund has earned 9 percent per year on average. Assuming that you actually earn 9 percent a year, how much money will you have available for retirement after making the last payment?
\end{enumerate}

\section{Solution:}
Use Equation 7 to find the future amount:

$A=€ 20,000$

$r=9 \%=0.09$

$N=30$

FV annuity factor $=\frac{(1+r)^{N}-1}{r}=\frac{(1.09)^{30}-1}{0.09}=136.307539$

$\mathrm{FV}_{N}=€ 20,000(136.307539)$

$=€ 2,726,150.77$

Assuming the fund continues to earn an average of 9 percent per year, you will have $€ 2,726,150.77$ available at retirement.

\section{Unequal Cash Flows}
In many cases, cash flow streams are unequal, precluding the simple use of the future value annuity factor. For instance, an individual investor might have a savings plan that involves unequal cash payments depending on the month of the year or lower savings during a planned vacation. One can always find the future value of a series of unequal cash flows by compounding the cash flows one at a time. Suppose you have the five cash flows described in Exhibit 5, indexed relative to the present $(t=0)$.

Exhibit 5: A Series of Unequal Cash Flows and Their Future Values at 5 Percent

\begin{center}
\begin{tabular}{lcccc}
\hline
Time & Cash Flow (\$) & \multicolumn{3}{c}{Future Value at Year 5} \\
\hline
$t=1$ & 1,000 & $\$ 1,000(1.05)^{4}=$ & $\$ 1,215.51$ &  \\
$t=2$ & 2,000 & $\$ 2,000(1.05)^{3}=$ & $\$ 2,315.25$ &  \\
$t=3$ & 4,000 & $\$ 4,000(1.05)^{2}$ & $=$ & $\$ 4,410.00$ \\
$t=4$ & 5,000 & $\$ 5,000(1.05)^{1}$ & $=$ & $\$ 5,250.00$ \\
$t=5$ & 6,000 & $\$ 6,000(1.05)^{0}$ & $=$ & $\$ 6,000.00$ \\
 &  & Sum & $=$ & $\$ 19,190.76$ \\
\hline
\end{tabular}
\end{center}

All of the payments shown in Exhibit 5 are different. Therefore, the most direct approach to finding the future value at $t=5$ is to compute the future value of each payment as of $t=5$ and then sum the individual future values. The total future value at Year 5 equals $\$ 19,190.76$, as shown in the third column. Later in this reading, you will learn shortcuts to take when the cash flows are close to even; these shortcuts will allow you to combine annuity and single-period calculations.

\section{PRESENT VALUE OF A SINGLE CASH FLOW}
calculate and interpret the future value (FV) and present value (PV) of a single sum of money, an ordinary annuity, an annuity due, a perpetuity (PV only), and a series of unequal cash flows demonstrate the use of a time line in modeling and solving time value of money problems

Just as the future value factor links today's present value with tomorrow's future value, the present value factor allows us to discount future value to present value. For example, with a 5 percent interest rate generating a future payoff of $\$ 105$ in one year, what current amount invested at 5 percent for one year will grow to $\$ 105$ ? The answer is $\$ 100$; therefore, $\$ 100$ is the present value of $\$ 105$ to be received in one year at a discount rate of 5 percent.

Given a future cash flow that is to be received in $N$ periods and an interest rate per period of $r$, we can use the formula for future value to solve directly for the present value as follows:

$$
\begin{aligned}
& \mathrm{FV}_{N}=\mathrm{PV}(1+r)^{N} \\
& \mathrm{PV}=\mathrm{FV}_{N}\left[\frac{1}{(1+r)^{N}}\right] \\
& \mathrm{PV}=\mathrm{FV}_{N}(1+r)^{-N}
\end{aligned}
$$

We see from Equation 8 that the present value factor, $(1+r)^{-N}$, is the reciprocal of the future value factor, $(1+r)^{N}$.

\section{EXAMPLE 8}
\section{The Present Value of a Lump Sum}
\begin{enumerate}
  \item An insurance company has issued a Guaranteed Investment Contract (GIC) that promises to pay $\$ 100,000$ in six years with an 8 percent return rate. What amount of money must the insurer invest today at 8 percent for six years to make the promised payment?
\end{enumerate}

\section{Solution:}
We can use Equation 8 to find the present value using the following data:

$$
\begin{aligned}
& \mathrm{FV}_{N}=\$ 100,000 \\
& r=8 \%=0.08 \\
& N=6 \\
& \mathrm{PV}=\mathrm{FV}_{N}(1+r)^{-N} \\
& =\$ 100,000\left[\frac{1}{(1.08)^{6}}\right] \\
& =\$ 100,000(0.6301696) \\
& =\$ 63,016.96
\end{aligned}
$$

We can say that $\$ 63,016.96$ today, with an interest rate of 8 percent, is equivalent to $\$ 100,000$ to be received in six years. Discounting the $\$ 100,000$ makes a future $\$ 100,000$ equivalent to $\$ 63,016.96$ when allowance is made for the time value of money. As the time line in Exhibit 6 shows, the $\$ 100,000$ has been discounted six full periods.

Exhibit 6: The Present Value of a Lump Sum to Be Received at Time $\boldsymbol{t}=6$

\begin{center}
\includegraphics[max width=\textwidth]{2023_05_04_cff39ee44f77d6514e1bg-028}
\end{center}

\section{EXAMPLE 9}
The Projected Present Value of a More Distant Future Lump Sum

\begin{enumerate}
  \item Suppose you own a liquid financial asset that will pay you $\$ 100,000$ in 10 years from today. Your daughter plans to attend college four years from today, and you want to know what the asset's present value will be at that time. Given an 8 percent discount rate, what will the asset be worth four years from today?
\end{enumerate}

\section{Solution:}
The value of the asset is the present value of the asset's promised payment. At $t=4$, the cash payment will be received six years later. With this information, you can solve for the value four years from today using Equation 8:

$$
\begin{aligned}
& \mathrm{FV}_{N}=\$ 100,000 \\
& r=8 \%=0.08 \\
& N=6 \\
& \mathrm{PV}=\mathrm{FV}_{N}(1+r)^{-N} \\
& =\$ 100,000 \frac{1}{(1.08)^{6}} \\
& =\$ 100,000(0.6301696) \\
& =\$ 63,016.96
\end{aligned}
$$

\section{Exhibit 7: The Relationship between Present Value and Future Value}
\begin{center}
\includegraphics[max width=\textwidth]{2023_05_04_cff39ee44f77d6514e1bg-029}
\end{center}

The time line in Exhibit 7 shows the future payment of $\$ 100,000$ that is to be received at $t=10$. The time line also shows the values at $t=4$ and at $t=0$. Relative to the payment at $t=10$, the amount at $t=4$ is a projected present value, while the amount at $t=0$ is the present value (as of today).

Present value problems require an evaluation of the present value factor, $(1+r)^{-N}$. Present values relate to the discount rate and the number of periods in the following ways:

\begin{itemize}
  \item For a given discount rate, the farther in the future the amount to be received, the smaller that amount's present value.

  \item Holding time constant, the larger the discount rate, the smaller the present value of a future amount.

\end{itemize}

\section{NON-ANNUAL COMPOUNDING (PRESENT VALUE)}
calculate the solution for time value of money problems with different frequencies of compounding

Recall that interest may be paid semiannually, quarterly, monthly, or even daily. To handle interest payments made more than once a year, we can modify the present value formula (Equation 8) as follows. Recall that $r_{s}$ is the quoted interest rate and equals the periodic interest rate multiplied by the number of compounding periods in each year. In general, with more than one compounding period in a year, we can express the formula for present value as

$$
\mathrm{PV}=\mathrm{FV}_{N}\left(1+\frac{r_{s}}{m}\right)^{-m N}
$$

where

$$
\begin{aligned}
& m=\text { number of compounding periods per year } \\
& r_{s}=\text { quoted annual interest rate } \\
& N=\text { number of years }
\end{aligned}
$$

The formula in Equation 9 is quite similar to that in Equation 8. As we have already noted, present value and future value factors are reciprocals. Changing the frequency of compounding does not alter this result. The only difference is the use of the periodic interest rate and the corresponding number of compounding periods.

The following example illustrates Equation 9.

\section{EXAMPLE 10}
\section{The Present Value of a Lump Sum with Monthly Compounding}
\begin{enumerate}
  \item The manager of a Canadian pension fund knows that the fund must make a lump-sum payment of $\mathrm{C} \$ 5$ million 10 years from now. She wants to invest an amount today in a GIC so that it will grow to the required amount. The current interest rate on GICs is 6 percent a year, compounded monthly. How much should she invest today in the GIC?
\end{enumerate}

\section{Solution:}
Use Equation 9 to find the required present value:

$$
\begin{aligned}
& \mathrm{FV}_{N}=\mathrm{C} \$ 5,000,000 \\
& r_{s}=6 \%=0.06 \\
& m=12 \\
& r_{s} / m=0.06 / 12=0.005 \\
& N=10 \\
& m N=12(10)=120 \\
& \mathrm{PV}=\mathrm{FV}_{N}\left(1+\frac{r_{s}}{m}\right)^{-m N} \\
& =\mathrm{C} \$ 5,000,000(1.005)^{-120} \\
& =\mathrm{C} \$ 5,000,000(0.549633) \\
& =\mathrm{C} \$ 2,748,163.67
\end{aligned}
$$

In applying Equation 9, we use the periodic rate (in this case, the monthly rate) and the appropriate number of periods with monthly compounding (in this case, 10 years of monthly compounding, or 120 periods).

\section{PRESENT VALUE OF A SERIES OF EQUAL AND UNEQUAL CASH FLOWS}
calculate and interpret the future value (FV) and present value (PV) of a single sum of money, an ordinary annuity, an annuity due, a perpetuity (PV only), and a series of unequal cash flows demonstrate the use of a time line in modeling and solving time value of money problems

Many applications in investment management involve assets that offer a series of cash flows over time. The cash flows may be highly uneven, relatively even, or equal. They may occur over relatively short periods of time, longer periods of time, or even stretch on indefinitely. In this section, we discuss how to find the present value of a series of cash flows.

\section{The Present Value of a Series of Equal Cash Flows}
We begin with an ordinary annuity. Recall that an ordinary annuity has equal annuity payments, with the first payment starting one period into the future. In total, the annuity makes $N$ payments, with the first payment at $t=1$ and the last at $t=N$. We can express the present value of an ordinary annuity as the sum of the present values of each individual annuity payment, as follows:

$$
\mathrm{PV}=\frac{A}{(1+r)}+\frac{A}{(1+r)^{2}}+\frac{A}{(1+r)^{3}}+\ldots+\frac{A}{(1+r)^{N-1}}+\frac{A}{(1+r)^{N}}
$$

where

$A=$ the annuity amount

$r=$ the interest rate per period corresponding to the frequency of annuity payments (for example, annual, quarterly, or monthly)

$N=$ the number of annuity payments

Because the annuity payment $(A)$ is a constant in this equation, it can be factored out as a common term. Thus the sum of the interest factors has a shortcut expression:

$$
\mathrm{PV}=A\left[\frac{1-\frac{1}{(1+r)^{N}}}{r}\right]
$$

In much the same way that we computed the future value of an ordinary annuity, we find the present value by multiplying the annuity amount by a present value annuity factor (the term in brackets in Equation 11).

\section{EXAMPLE 11}
\section{The Present Value of an Ordinary Annuity}
\begin{enumerate}
  \item Suppose you are considering purchasing a financial asset that promises to pay $€ 1,000$ per year for five years, with the first payment one year from now. The required rate of return is 12 percent per year. How much should you pay for this asset?
\end{enumerate}

\section{Solution:}
To find the value of the financial asset, use the formula for the present value of an ordinary annuity given in Equation 11 with the following data:

$$
\begin{aligned}
& A=€ 1,000 \\
& r=12 \%=0.12 \\
& N=5 \\
& \mathrm{PV}=A\left[\frac{1-\frac{1}{(1+r)^{N}}}{r}\right] \\
& =€ 1,000\left[\frac{1-\frac{1}{(1.12)^{5}}}{0.12}\right] \\
& =€ 1,000(3.604776) \\
& =€ 3,604.78
\end{aligned}
$$

The series of cash flows of $€ 1,000$ per year for five years is currently worth $€ 3,604.78$ when discounted at 12 percent.

Keeping track of the actual calendar time brings us to a specific type of annuity with level payments: the annuity due. An annuity due has its first payment occurring today $(t=0)$. In total, the annuity due will make $N$ payments. Exhibit 8 presents the time line for an annuity due that makes four payments of $\$ 100$.

\section{Exhibit 8: An Annuity Due of $\$ 100$ per Period}
\begin{center}
\includegraphics[max width=\textwidth]{2023_05_04_cff39ee44f77d6514e1bg-032}
\end{center}

As Exhibit 8 shows, we can view the four-period annuity due as the sum of two parts: a $\$ 100$ lump sum today and an ordinary annuity of $\$ 100$ per period for three periods. At a 12 percent discount rate, the four $\$ 100$ cash flows in this annuity due example will be worth $\$ 340.18$. ${ }^{7}$

Expressing the value of the future series of cash flows in today's dollars gives us a convenient way of comparing annuities. The next example illustrates this approach.

7 There is an alternative way to calculate the present value of an annuity due. Compared to an ordinary annuity, the payments in an annuity due are each discounted one less period. Therefore, we can modify Equation 11 to handle annuities due by multiplying the right-hand side of the equation by $(1+r)$ :

$\operatorname{PV}($ Annuity due $)=A\left\{\left[1-(1+r)^{-N}\right] / r\right\}(1+r)$

\section{EXAMPLE 12}
\section{An Annuity Due as the Present Value of an Immediate Cash Flow Plus an Ordinary Annuity}
\begin{enumerate}
  \item You are retiring today and must choose to take your retirement benefits either as a lump sum or as an annuity. Your company's benefits officer presents you with two alternatives: an immediate lump sum of $\$ 2$ million or an annuity with 20 payments of $\$ 200,000$ a year with the first payment starting today. The interest rate at your bank is 7 percent per year compounded annually. Which option has the greater present value? (Ignore any tax differences between the two options.)
\end{enumerate}

\section{Solution:}
To compare the two options, find the present value of each at time $t=0$ and choose the one with the larger value. The first option's present value is $\$ 2$ million, already expressed in today's dollars. The second option is an annuity due. Because the first payment occurs at $t=0$, you can separate the annuity benefits into two pieces: an immediate $\$ 200,000$ to be paid today $(t=0)$ and an ordinary annuity of $\$ 200,000$ per year for 19 years. To value this option, you need to find the present value of the ordinary annuity using Equation 11 and then add $\$ 200,000$ to it.

$$
\begin{aligned}
& A=\$ 200,000 \\
& N=19 \\
& r=7 \%=0.07 \\
& \mathrm{PV}=A\left[\frac{1-\frac{1}{(1+r)^{N}}}{r}\right] \\
& =\$ 200,000\left[\frac{1-\frac{1}{(1.07)^{19}}}{0.07}\right] \\
& =\$ 200,000(10.335595) \\
& =\$ 2,067,119.05
\end{aligned}
$$

The 19 payments of $\$ 200,000$ have a present value of $\$ 2,067,119.05$. Adding the initial payment of $\$ 200,000$ to $\$ 2,067,119.05$, we find that the total value of the annuity option is $\$ 2,267,119.05$. The present value of the annuity is greater than the lump sum alternative of $\$ 2$ million.

We now look at another example reiterating the equivalence of present and future values.

\section{EXAMPLE 13}
\section{The Projected Present Value of an Ordinary Annuity}
\begin{enumerate}
  \item A German pension fund manager anticipates that benefits of $€ 1$ million per year must be paid to retirees. Retirements will not occur until 10 years from now at time $t=10$. Once benefits begin to be paid, they will extend until $t=39$ for a total of 30 payments. What is the present value of the pension liability if the appropriate annual discount rate for plan liabilities is 5 percent compounded annually?
\end{enumerate}

Solution:

This problem involves an annuity with the first payment at $t=10$. From the perspective of $t=9$, we have an ordinary annuity with 30 payments. We can compute the present value of this annuity with Equation 11 and then look at it on a time line.

$$
\begin{aligned}
& A=€ 1,000,000 \\
& r=5 \%=0.05 \\
& N=30 \\
& \mathrm{PV}=A\left[\frac{1-\frac{1}{(1+r)^{N}}}{r}\right] \\
& =€ 1,000,000\left[\frac{1-\frac{1}{(1.05)^{30}}}{0.05}\right] \\
& =€ 1,000,000(15.372451) \\
& =€ 15,372,451.03
\end{aligned}
$$

Exhibit 9: The Present Value of an Ordinary Annuity with First Payment at Time $\boldsymbol{t}=10$ (in Millions)

\begin{center}
\includegraphics[max width=\textwidth]{2023_05_04_cff39ee44f77d6514e1bg-034}
\end{center}

On the time line, we have shown the pension payments of $€ 1$ million extending from $t=10$ to $t=39$. The bracket and arrow indicate the process of finding the present value of the annuity, discounted back to $t=9$. The present value of the pension benefits as of $t=9$ is $€ 15,372,451.03$. The problem is to find the present value today (at $t=0$ ).

Now we can rely on the equivalence of present value and future value. As Exhibit 9 shows, we can view the amount at $t=9$ as a future value from the vantage point of $t=0$. We compute the present value of the amount at $t=9$ as follows:

$\mathrm{FV}_{N}=€ 15,372,451.03$ (the present value at $t=9$ )

$N=9$

$r=5 \%=0.05$

$\mathrm{PV}=\mathrm{FV}_{N}(1+r)^{-N}$

$=€ 15,372,451.03(1.05)^{-9}$

$=€ 15,372,451.03(0.644609)$

$=€ 9,909,219.00$

The present value of the pension liability is $€ 9,909,219.00$. Example 13 illustrates three procedures emphasized in this reading:

\begin{itemize}
  \item finding the present or future value of any cash flow series;

  \item recognizing the equivalence of present value and appropriately discounted future value; and

  \item keeping track of the actual calendar time in a problem involving the time value of money.

\end{itemize}

\section{The Present Value of a Series of Unequal Cash Flows}
When we have unequal cash flows, we must first find the present value of each individual cash flow and then sum the respective present values. For a series with many cash flows, we usually use a spreadsheet. Exhibit 10 lists a series of cash flows with the time periods in the first column, cash flows in the second column, and each cash flow's present value in the third column. The last row of Exhibit 10 shows the sum of the five present values.

Exhibit 10: A Series of Unequal Cash Flows and Their

Present Values at 5 Percent

\begin{center}
\begin{tabular}{|c|c|c|c|c|}
\hline
\multirow{2}{*}{$\frac{\text { Time Period }}{1}$} & \multirow{2}{*}{$\frac{\text { Cash Flow (\$) }}{1,000}$} & \multicolumn{3}{|c|}{Present Value at Year 0} \\
\hline
 &  & $\$ 1,000(1.05)^{-1}$ & $=$ & $\$ 952.38$ \\
\hline
2 & 2,000 & $\$ 2,000(1.05)^{-2}$ & $=$ & $\$ 1,814.06$ \\
\hline
3 & 4,000 & $\$ 4,000(1.05)^{-3}$ & $=$ & $\$ 3,455.35$ \\
\hline
4 & 5,000 & $\$ 5,000(1.05)^{-4}$ & $=$ & $\$ 4,113.51$ \\
\hline
\multirow[t]{2}{*}{5} & 6,000 & $\$ 6,000(1.05)^{-5}$ & $=$ & $\$ 4,701.16$ \\
\hline
 &  & Sum & $=$ & $\$ 15,036.46$ \\
\hline
\end{tabular}
\end{center}

We could calculate the future value of these cash flows by computing them one at a time using the single-payment future value formula. We already know the present value of this series, however, so we can easily apply time-value equivalence. The future value of the series of cash flows from Table $2, \$ 19,190.76$, is equal to the single $\$ 15,036.46$ amount compounded forward to $t=5$ :

$$
\begin{aligned}
& \mathrm{PV}=\$ 15,036.46 \\
& N=5 \\
& r=5 \%=0.05 \\
& \mathrm{FV}_{N}=\mathrm{PV}(1+r)^{N} \\
& =\$ 15,036.46(1.05)^{5} \\
& =\$ 15,036.46(1.276282) \\
& =\$ 19,190.76
\end{aligned}
$$

\section{PRESENT VALUE OF A PERPETUITY}
calculate and interpret the future value (FV) and present value (PV) of a single sum of money, an ordinary annuity, an annuity due, a perpetuity (PV only), and a series of unequal cash flows

Consider the case of an ordinary annuity that extends indefinitely. Such an ordinary annuity is called a perpetuity (a perpetual annuity). To derive a formula for the present value of a perpetuity, we can modify Equation 10 to account for an infinite series of cash flows:

$$
\mathrm{PV}=A \sum_{t=1}^{\infty}\left[\frac{1}{(1+r)^{t}}\right]
$$

As long as interest rates are positive, the sum of present value factors converges and

$$
\mathrm{PV}=\frac{A}{r}
$$

To see this, look back at Equation 11, the expression for the present value of an ordinary annuity. As $N$ (the number of periods in the annuity) goes to infinity, the term $1 /(1+r)^{N}$ approaches 0 and Equation 11 simplifies to Equation 13 . This equation will reappear when we value dividends from stocks because stocks have no predefined life span. (A stock paying constant dividends is similar to a perpetuity.) With the first payment a year from now, a perpetuity of $\$ 10$ per year with a 20 percent required rate of return has a present value of $\$ 10 / 0.2=\$ 50$.

Equation 13 is valid only for a perpetuity with level payments. In our development above, the first payment occurred at $t=1$; therefore, we compute the present value as of $t=0$.

Other assets also come close to satisfying the assumptions of a perpetuity. Certain government bonds and preferred stocks are typical examples of financial assets that make level payments for an indefinite period of time.

\section{EXAMPLE 14}
\section{The Present Value of a Perpetuity}
\begin{enumerate}
  \item The British government once issued a type of security called a consol bond, which promised to pay a level cash flow indefinitely. If a consol bond paid $\pounds 100$ per year in perpetuity, what would it be worth today if the required rate of return were 5 percent?
\end{enumerate}

\section{Solution:}
To answer this question, we can use Equation 13 with the following data:

$$
\begin{aligned}
& A=\pounds 100 \\
& r=5 \%=0.05 \\
& \mathrm{PV}=A / r \\
& =\pounds 100 / 0.05 \\
& =\pounds 2,000
\end{aligned}
$$

The bond would be worth $\pounds 2,000$.

\section{Present Values Indexed at Times Other than $t=0$}
In practice with investments, analysts frequently need to find present values indexed at times other than $t=0$. Subscripting the present value and evaluating a perpetuity beginning with $\$ 100$ payments in Year 2 , we find $\mathrm{PV}_{1}=\$ 100 / 0.05=\$ 2,000$ at a 5 percent discount rate. Further, we can calculate today's PV as $\mathrm{PV}_{0}=\$ 2,000 / 1.05=\$ 1,904.76$.

Consider a similar situation in which cash flows of $\$ 6$ per year begin at the end of the 4th year and continue at the end of each year thereafter, with the last cash flow at the end of the 10th year. From the perspective of the end of the third year, we are facing a typical seven-year ordinary annuity. We can find the present value of the annuity from the perspective of the end of the third year and then discount that present value back to the present. At an interest rate of 5 percent, the cash flows of $\$ 6$ per year starting at the end of the fourth year will be worth $\$ 34.72$ at the end of the third year $(t=3)$ and $\$ 29.99$ today $(t=0)$

The next example illustrates the important concept that an annuity or perpetuity beginning sometime in the future can be expressed in present value terms one period prior to the first payment. That present value can then be discounted back to today's present value.

\section{EXAMPLE 15}
\section{The Present Value of a Projected Perpetuity}
\begin{enumerate}
  \item Consider a level perpetuity of $\pounds 100$ per year with its first payment beginning at $t=5$. What is its present value today (at $t=0$ ), given a 5 percent discount rate?
\end{enumerate}

\section{Solution:}
First, we find the present value of the perpetuity at $t=4$ and then discount that amount back to $t=0$. (Recall that a perpetuity or an ordinary annuity has its first payment one period away, explaining the $t=4$ index for our present value calculation.)

i. Find the present value of the perpetuity at $t=4$ :

$$
\begin{aligned}
& A=\pounds 100 \\
& r=5 \%=0.05 \\
& \mathrm{PV}=A / r \\
& =\pounds 100 / 0.05 \\
& =\pounds 2,000
\end{aligned}
$$

ii. Find the present value of the future amount at $t=4$. From the perspective of $t=0$, the present value of $\pounds 2,000$ can be considered a future value. Now we need to find the present value of a lump sum:

$$
\begin{aligned}
& \mathrm{FV}_{N}=\pounds 2,000(\text { the present value at } t=4) \\
& r=5 \%=0.05 \\
& N=4 \\
& \mathrm{PV}=\mathrm{FV}_{N}(1+r)^{-N} \\
& =\pounds 2,000(1.05)^{-4} \\
& =\pounds 2,000(0.822702) \\
& =\pounds 1,645.40
\end{aligned}
$$

Today's present value of the perpetuity is $\pounds 1,645.40$. The Time Value of Money

As discussed earlier, an annuity is a series of payments of a fixed amount for a specified number of periods. Suppose we own a perpetuity. At the same time, we issue a perpetuity obligating us to make payments; these payments are the same size as those of the perpetuity we own. However, the first payment of the perpetuity we issue is at $t=5$; payments then continue on forever. The payments on this second perpetuity exactly offset the payments received from the perpetuity we own at $t=5$ and all subsequent dates. We are left with level nonzero net cash flows at $t=1,2,3$, and 4 . This outcome exactly fits the definition of an annuity with four payments. Thus we can construct an annuity as the difference between two perpetuities with equal, level payments but differing starting dates. The next example illustrates this result.

\section{EXAMPLE 16}
The Present Value of an Ordinary Annuity as the Present Value of a Current Minus Projected Perpetuity

\begin{enumerate}
  \item Given a 5 percent discount rate, find the present value of a four-year ordinary annuity of $\pounds 100$ per year starting in Year 1 as the difference between the following two level perpetuities:
\end{enumerate}

Perpetuity $1 \quad \pounds 100$ per year starting in Year 1 (first payment at $t=1$ )

Perpetuity $2 \quad \pounds 100$ per year starting in Year 5 (first payment at $t=5$ )

\section{Solution:}
If we subtract Perpetuity 2 from Perpetuity 1 , we are left with an ordinary annuity of $\pounds 100$ per period for four years (payments at $t=1,2,3,4$ ). Subtracting the present value of Perpetuity 2 from that of Perpetuity 1 , we arrive at the present value of the four-year ordinary annuity:

$$
\begin{aligned}
& \mathrm{PV}_{0}(\text { Perpetuity 1) }=\pounds 100 / 0.05=\pounds 2,000 \\
& \mathrm{PV}_{4}(\text { Perpetuity 2) }=\pounds 100 / 0.05=\pounds 2,000 \\
& \mathrm{PV}_{0}\left(\text { Perpetuity 2) }=\pounds 2,000 /(1.05)^{4}=\pounds 1,645.40\right. \\
& \mathrm{PV}_{0}(\text { Annuity })=\mathrm{PV}_{0}(\text { Perpetuity } 1)-\mathrm{PV}_{0}(\text { Perpetuity 2) } \\
&= \pounds 2,000-\pounds 1,645.40 \\
&= \pounds 354.60
\end{aligned}
$$

The four-year ordinary annuity's present value is equal to $\pounds 2,000-\pounds 1,645.40$ $=\pounds 354.60$.

SOLVING FOR INTEREST RATES, GROWTH RATES, AND NUMBER OF PERIODS

calculate and interpret the future value (FV) and present value (PV) of a single sum of money, an ordinary annuity, an annuity due, a perpetuity (PV only), and a series of unequal cash flows In the previous examples, certain pieces of information have been made available. For instance, all problems have given the rate of interest, $r$, the number of time periods, $N$, the annuity amount, $A$, and either the present value, PV, or future value, FV. In real-world applications, however, although the present and future values may be given, you may have to solve for either the interest rate, the number of periods, or the annuity amount. In the subsections that follow, we show these types of problems.

\section{Solving for Interest Rates and Growth Rates}
Suppose a bank deposit of $€ 100$ is known to generate a payoff of $€ 111$ in one year. With this information, we can infer the interest rate that separates the present value of $€ 100$ from the future value of $€ 111$ by using Equation 2, $\mathrm{FV}_{N}=\mathrm{PV}(1+r)^{N}$, with $N$ = 1. With PV, FV, and $N$ known, we can solve for $r$ directly:

$$
\begin{aligned}
1+r & =\mathrm{FV} / \mathrm{PV} \\
1+r & =€ 111 / € 100=1.11 \\
r & =0.11, \text { or } 11 \%
\end{aligned}
$$

The interest rate that equates $€ 100$ at $t=0$ to $€ 111$ at $t=1$ is 11 percent. Thus we can state that $€ 100$ grows to $€ 111$ with a growth rate of 11 percent.

As this example shows, an interest rate can also be considered a growth rate. The particular application will usually dictate whether we use the term "interest rate" or "growth rate." Solving Equation 2 for $r$ and replacing the interest rate $r$ with the growth rate $g$ produces the following expression for determining growth rates:

$$
g=\left(\mathrm{FV}_{N} / \mathrm{PV}\right)^{1 / N}-1
$$

Below are two examples that use the concept of a growth rate.

\section{EXAMPLE 17}
\section{Calculating a Growth Rate (1)}
Hyundai Steel, the first Korean steelmaker, was established in 1953. Hyundai Steel's sales increased from $\# 14,146.4$ billion in 2012 to $\# 19,166.0$ billion in 2017. However, its net profit declined from W796.4 billion in 2012 to $\# 727.5$ billion in 2017. Calculate the following growth rates for Hyundai Steel for the five-year period from the end of 2012 to the end of 2017 :

\begin{enumerate}
  \item Sales growth rate.
\end{enumerate}

\section{Solution to 1:}
To solve this problem, we can use Equation $14, g=\left(\mathrm{FV}_{N} / \mathrm{PV}\right)^{1 / N}-1$. We denote sales in 2012 as PV and sales in 2017 as $\mathrm{FV}_{5}$. We can then solve for the growth rate as follows:

$$
\begin{aligned}
& g=\sqrt[5]{\# 19,166.0 / \# 14,146.4}-1 \\
& =\sqrt[5]{1.354832}-1 \\
& =1.062618-1 \\
& =0.062618 \text { or about } 6.3 \%
\end{aligned}
$$

The calculated growth rate of about 6.3 percent a year shows that Hyundai Steel's sales grew during the 2012-2017 period. 2. Net profit growth rate.

\section{Solution to 2:}
In this case, we can speak of a positive compound rate of decrease or a negative compound growth rate. Using Equation 14, we find

$$
\begin{aligned}
& g=\sqrt[5]{\# 727.5 / \# 796.4}-1 \\
& =\sqrt[5]{0.913486}-1 \\
& =0.982065-1 \\
& =-0.017935 \text { or about }-1.8 \%
\end{aligned}
$$

In contrast to the positive sales growth, the rate of growth in net profit was approximately -1.8 percent during the $2012-2017$ period.

\section{EXAMPLE 18}
\section{Calculating a Growth Rate (2)}
\begin{enumerate}
  \item Toyota Motor Corporation, one of the largest automakers in the world, had consolidated vehicle sales of 8.96 million units in 2018 (fiscal year ending 31 March 2018). This is substantially more than consolidated vehicle sales of 7.35 million units six years earlier in 2012. What was the growth rate in number of vehicles sold by Toyota from 2012 to 2018 ?
\end{enumerate}

Solution:

Using Equation 14, we find

$$
\begin{aligned}
& g=\sqrt[6]{8.96 / 7.35}-1 \\
& =\sqrt[6]{1.219048}-1 \\
& =1.033563-1 \\
& =0.033563 \text { or about } 3.4 \%
\end{aligned}
$$

The rate of growth in vehicles sold was approximately 3.4 percent during the 2012-2018 period. Note that we can also refer to 3.4 percent as the compound annual growth rate because it is the single number that compounds the number of vehicles sold in 2012 forward to the number of vehicles sold in 2018. Exhibit 11 lists the number of vehicles sold by Toyota from 2012 to 2018.

\section{Exhibit 11: Number of Vehicles Sold, 2012-2018}
\begin{center}
\begin{tabular}{lccc}
\hline
Year & $\begin{array}{c}\text { Number of Vehicles Sold } \\ \text { (Millions) }\end{array}$ & $(\mathbf{1}+\boldsymbol{g})_{\boldsymbol{t}}$ & $\boldsymbol{t}$ \\
\hline
2012 & 7.35 &  & 0 \\
2013 & 8.87 & $8.87 / 7.35=1.206803$ & 1 \\
2014 & 9.12 & $9.12 / 8.87=1.028185$ & 2 \\
2015 & 8.97 & $8.97 / 9.12=0.983553$ & 3 \\
2016 & 8.68 & $8.68 / 8.97=0.967670$ & 4 \\
2017 & 8.97 & $8.97 / 8.68=1.033410$ & 5 \\
2018 & 8.96 & $8.96 / 8.97=0.998885$ & 6 \\
\hline
\end{tabular}
\end{center}

Source: \href{http://www.toyota.com}{www.toyota.com}.

Exhibit 11 also shows 1 plus the one-year growth rate in number of vehicles sold. We can compute the 1 plus six-year cumulative growth in number of vehicles sold from 2012 to 2018 as the product of quantities $(1+$ one-year growth rate). We arrive at the same result as when we divide the ending number of vehicles sold, 8.96 million, by the beginning number of vehicles sold, 7.35 million:

$$
\begin{aligned}
& \frac{8.96}{7.35}=\left(\frac{8.87}{7.35}\right)\left(\frac{9.12}{8.87}\right)\left(\frac{8.97}{9.12}\right)\left(\frac{8.68}{8.97}\right)\left(\frac{8.97}{8.68}\right)\left(\frac{8.96}{8.97}\right) \\
= & \left(1+g_{1}\right)\left(1+g_{2}\right)\left(1+g_{3}\right)\left(1+g_{4}\right)\left(1+g_{5}\right)\left(1+g_{6}\right)
\end{aligned}
$$

$1.219048=(1.206803)(1.028185)(0.983553)(0.967670)(1.033410)(0.998885)$

The right-hand side of the equation is the product of 1 plus the one-year growth rate in number of vehicles sold for each year. Recall that, using Equation 14, we took the sixth root of 8.96/7.35 $=1.219048$. In effect, we were solving for the single value of $g$ which, when compounded over six periods, gives the correct product of 1 plus the one-year growth rates. ${ }^{8}$

In conclusion, we do not need to compute intermediate growth rates as in Exhibit 11 to solve for a compound growth rate $g$. Sometimes, however the intermediate growth rates are interesting or informative. For example, most of the 21.9 percent increase in vehicles sold by Toyota from 2012 to 2018 occurred in 2013 as sales increased by 20.7 percent from 2012 to 2013. Elsewhere in Toyota Motor's disclosures, the company noted that all regions except Europe showed a substantial increase in sales in 2013. We can also analyze the variability in growth rates when we conduct an analysis as in Exhibit 11. Sales continued to increase in 2014 but then declined in 2015 and 2016. Sales then increased but the sales in 2017 and 2018 are about the same as in 2015.

The compound growth rate is an excellent summary measure of growth over multiple time periods. In our Toyota Motors example, the compound growth rate of 3.4 percent is the single growth rate that, when added to 1 , compounded over six years, and multiplied by the 2012 number of vehicles sold, yields the 2018 number of vehicles sold.

\section{Solving for the Number of Periods}
In this section, we demonstrate how to solve for the number of periods given present value, future value, and interest or growth rates.

8 The compound growth rate that we calculate here is an example of a geometric mean, specifically the geometric mean of the growth rates. We define the geometric mean in the reading on statistical concepts.

\section{EXAMPLE 19}
The Number of Annual Compounding Periods Needed for an Investment to Reach a Specific Value

\begin{enumerate}
  \item You are interested in determining how long it will take an investment of $€ 10,000,000$ to double in value. The current interest rate is 7 percent compounded annually. How many years will it take $€ 10,000,000$ to double to $€ 20,000,000 ?$
\end{enumerate}

Solution:

Use Equation 2, $\mathrm{FV}_{N}=\mathrm{PV}(1+r)^{N}$, to solve for the number of periods, $N$, as follows:

$$
\begin{aligned}
& (1+r)^{N}=\mathrm{FV}_{N} / \mathrm{PV}=2 \\
& N \ln (1+r)=\ln (2) \\
& N=\ln (2) / \ln (1+r) \\
& =\ln (2) / \ln (1.07)=10.24
\end{aligned}
$$

With an interest rate of 7 percent, it will take approximately 10 years for the initial $€ 10,000,000$ investment to grow to $€ 20,000,000$. Solving for $N$ in the expression $(1.07)^{N}=2.0$ requires taking the natural logarithm of both sides and using the rule that $\ln \left(x^{N}\right)=N \ln (x)$. Generally, we find that $N=[\ln (\mathrm{FV} /$ PV $)] / \ln (1+r)$. Here, $N=\ln (€ 20,000,000 / € 10,000,000) / \ln (1.07)=\ln (2) /$ $\ln (1.07)=10.24 .9$

\section{SOLVING FOR SIZE OF ANNUITY PAYMENTS}
calculate and interpret the future value (FV) and present value (PV) of a single sum of money, an ordinary annuity, an annuity due, a perpetuity (PV only), and a series of unequal cash flows demonstrate the use of a time line in modeling and solving time value of money problems

In this section, we discuss how to solve for annuity payments. Mortgages, auto loans, and retirement savings plans are classic examples of applications of annuity formulas.

9 To quickly approximate the number of periods, practitioners sometimes use an ad hoc rule called the Rule of 72: Divide 72 by the stated interest rate to get the approximate number of years it would take to double an investment at the interest rate. Here, the approximation gives $72 / 7=10.3$ years. The Rule of 72 is loosely based on the observation that it takes 12 years to double an amount at a 6 percent interest rate, giving $6 \times 12=72$. At a 3 percent rate, one would guess it would take twice as many years, $3 \times 24=72$.

\section{EXAMPLE 20}
\section{Calculating the Size of Payments on a Fixed-Rate Mortgage}
\begin{enumerate}
  \item You are planning to purchase a $\$ 120,000$ house by making a down payment of $\$ 20,000$ and borrowing the remainder with a 30 -year fixed-rate mortgage with monthly payments. The first payment is due at $t=1$. Current mortgage interest rates are quoted at 8 percent with monthly compounding. What will your monthly mortgage payments be?
\end{enumerate}

\section{Solution:}
The bank will determine the mortgage payments such that at the stated periodic interest rate, the present value of the payments will be equal to the amount borrowed (in this case, $\$ 100,000$ ). With this fact in mind, we can use Equation 11, PV $=A\left[\frac{1-\frac{1}{(1+r)^{N}}}{r}\right]$, to solve for the annuity amount, $A$, as the present value divided by the present value annuity factor:

$\mathrm{PV}=\$ 100,000$

$r_{s}=8 \%=0.08$

$m=12$

$r_{s} / m=0.08 / 12=0.006667$

$N=30$

$m N=12 \times 30=360$

Present value annuity factor $=\frac{1-\frac{1}{\left[1+\left(r_{s} / m\right)\right]^{m N}}}{r_{s} / m}=\frac{1-\frac{1}{(1.006667)^{360}}}{0.006667}$

$=136.283494$

$A=\mathrm{PV} /$ Present value annuity factor

$=\$ 100,000 / 136.283494$

$=\$ 733.76$

The amount borrowed, $\$ 100,000$, is equivalent to 360 monthly payments of $\$ 733.76$ with a stated interest rate of 8 percent. The mortgage problem is a relatively straightforward application of finding a level annuity payment.

Next, we turn to a retirement-planning problem. This problem illustrates the complexity of the situation in which an individual wants to retire with a specified retirement income. Over the course of a life cycle, the individual may be able to save only a small amount during the early years but then may have the financial resources to save more during later years. Savings plans often involve uneven cash flows, a topic we will examine in the last part of this reading. When dealing with uneven cash flows, we take maximum advantage of the principle that dollar amounts indexed at the same point in time are additive-the cash flow additivity principle.

\section{EXAMPLE 21}
\section{The Projected Annuity Amount Needed to Fund a}
 Future-Annuity Inflow\begin{enumerate}
  \item Jill Grant is 22 years old (at $t=0$ ) and is planning for her retirement at age 63 (at $t=41$ ). She plans to save $\$ 2,000$ per year for the next 15 years $(t=1$ to $t=15$ ). She wants to have retirement income of $\$ 100,000$ per year for 20 years, with the first retirement payment starting at $t=41$. How much must Grant save each year from $t=16$ to $t=40$ in order to achieve her retirement goal? Assume she plans to invest in a diversified stock-and-bond mutual fund that will earn 8 percent per year on average.
\end{enumerate}

Solution:

To help solve this problem, we set up the information on a time line. As Exhibit 12 shows, Grant will save $\$ 2,000$ (an outflow) each year for Years 1 to 15 . Starting in Year 41, Grant will start to draw retirement income of $\$ 100,000$ per year for 20 years. In the time line, the annual savings is recorded in parentheses (\$2) to show that it is an outflow. The problem is to find the savings, recorded as $X$, from Year 16 to Year 40 .

\section{Exhibit 12: Solving for Missing Annuity Payments (in Thousands)}
\begin{center}
\includegraphics[max width=\textwidth]{2023_05_04_cff39ee44f77d6514e1bg-044}
\end{center}

Solving this problem involves satisfying the following relationship: the present value of savings (outflows) equals the present value of retirement income (inflows). We could bring all the dollar amounts to $t=40$ or to $t=15$ and solve for $X$.

Let us evaluate all dollar amounts at $t=15$ (we encourage the reader to repeat the problem by bringing all cash flows to $t=40$ ). As of $t=15$, the first payment of $X$ will be one period away (at $t=16$ ). Thus we can value the stream of $X \mathrm{~s}$ using the formula for the present value of an ordinary annuity. This problem involves three series of level cash flows. The basic idea is that the present value of the retirement income must equal the present value of Grant's savings. Our strategy requires the following steps:

\begin{enumerate}
  \item Find the future value of the savings of $\$ 2,000$ per year and index it at $t$ $=15$. This value tells us how much Grant will have saved.

  \item Find the present value of the retirement income at $t=15$. This value tells us how much Grant needs to meet her retirement goals (as of $t$ $=15$ ). Two substeps are necessary. First, calculate the present value of the annuity of $\$ 100,000$ per year at $t=40$. Use the formula for the present value of an annuity. (Note that the present value is indexed at $t$ $=40$ because the first payment is at $t=41$.) Next, discount the present value back to $t=15$ (a total of 25 periods). 3. Now compute the difference between the amount Grant has saved (Step 1) and the amount she needs to meet her retirement goals (Step 2). Her savings from $t=16$ to $t=40$ must have a present value equal to the difference between the future value of her savings and the present value of her retirement income.

\end{enumerate}

Our goal is to determine the amount Grant should save in each of the 25 years from $t=16$ to $t=40$. We start by bringing the $\$ 2,000$ savings to $t=15$, as follows:

$$
\begin{aligned}
& A=\$ 2,000 \\
& r=8 \%=0.08 \\
& N=15 \\
& \mathrm{FV}=A\left[\frac{(1+r)^{N}-1}{r}\right] \\
& =\$ 2,000\left[\frac{(1.08)^{15}-1}{0.08}\right] \\
& =\$ 2,000(27.152114) \\
& =\$ 54,304.23
\end{aligned}
$$

At $t=15$, Grant's initial savings will have grown to $\$ 54,304.23$.

Now we need to know the value of Grant's retirement income at $t=15$. As stated earlier, computing the retirement present value requires two substeps. First, find the present value at $t=40$ with the formula in Equation 11 ; second, discount this present value back to $t=15$. Now we can find the retirement income present value at $t=40$ :

$$
\begin{aligned}
& A=\$ 100,000 \\
& r=8 \%=0.08 \\
& N=20 \\
& \mathrm{PV}=A\left[\frac{1-\frac{1}{(1+r)^{N}}}{r}\right] \\
& =\$ 100,000\left[\frac{1-\frac{1}{(1.08)^{20}}}{0.08}\right] \\
& =\$ 100,000(9.818147) \\
& =\$ 981,814.74
\end{aligned}
$$

The present value amount is as of $t=40$, so we must now discount it back as a lump sum to $t=15$ :

$\mathrm{FV}_{N}=\$ 981,814.74$

$N=25$

$r=8 \%=0.08$

$\mathrm{PV}=\mathrm{FV}_{N}(1+r)^{-N}$

$=\$ 981,814.74(1.08)^{-25}$

$=\$ 981,814.74(0.146018)$

$=\$ 143,362.53$

Now recall that Grant will have saved $\$ 54,304.23$ by $t=15$. Therefore, in present value terms, the annuity from $t=16$ to $t=40$ must equal the difference between the amount already saved $(\$ 54,304.23)$ and the amount required for retirement $(\$ 143,362.53)$. This amount is equal to $\$ 143,362.53$ $\$ 54,304.23=\$ 89,058.30$. Therefore, we must now find the annuity payment, $A$, from $t=16$ to $t=40$ that has a present value of $\$ 89,058.30$. We find the annuity payment as follows: $\mathrm{PV}=\$ 89,058.30$

$r=8 \%=0.08$

$N=25$

Present value annuity factor $=\left[\frac{1-\frac{1}{(1+r)^{N}}}{r}\right]$

$=\left[\frac{1-\frac{1}{(1.08)^{25}}}{0.08}\right]$

$=10.674776$

$A=\mathrm{PV} /$ Present value annuity factor

$=\$ 89,058.30 / 10.674776$

$=\$ 8,342.87$

Grant will need to increase her savings to $\$ 8,342.87$ per year from $t=16$ to $t$ $=40$ to meet her retirement goal of having a fund equal to $\$ 981,814.74$ after making her last payment at $t=40$.

\section{PRESENT AND FUTURE VALUE EQUIVALENCE AND THE ADDITIVITY PRINCIPLE}
calculate and interpret the future value (FV) and present value (PV) of a single sum of money, an ordinary annuity, an annuity due, a perpetuity (PV only), and a series of unequal cash flows demonstrate the use of a time line in modeling and solving time value of money problems

As we have demonstrated, finding present and future values involves moving amounts of money to different points on a time line. These operations are possible because present value and future value are equivalent measures separated in time. Exhibit 13 illustrates this equivalence; it lists the timing of five cash flows, their present values at $t=0$, and their future values at $t=5$.

To interpret Exhibit 13, start with the third column, which shows the present values. Note that each $\$ 1,000$ cash payment is discounted back the appropriate number of periods to find the present value at $t=0$. The present value of $\$ 4,329.48$ is exactly equivalent to the series of cash flows. This information illustrates an important point: A lump sum can actually generate an annuity. If we place a lump sum in an account that earns the stated interest rate for all periods, we can generate an annuity that is equivalent to the lump sum. Amortized loans, such as mortgages and car loans, are examples of this principle.

\section{Exhibit 13: The Equivalence of Present and Future Values}
\begin{center}
\begin{tabular}{|c|c|c|c|c|c|c|c|}
\hline
\multirow{2}{*}{$\frac{\text { Time }}{1}$} & \multirow{2}{*}{$\frac{\text { Cash Flow (\$) }}{1,000}$} & \multicolumn{3}{|c|}{Present Value at $t=0$} & \multicolumn{3}{|c|}{Future Value at $t=5$} \\
\hline
 &  & $\$ 1,000(1.05)^{-1}$ & $=$ & $\$ 952.38$ & $\$ 1,000(1.05)^{4}$ & $=$ & $\$ 1,215.5$ \\
\hline
2 & 1,000 & $\$ 1,000(1.05)^{-2}$ & $=$ & $\$ 907.03$ & $\$ 1,000(1.05)^{3}$ & $=$ & $\$ 1,157.63$ \\
\hline
3 & 1,000 & $\$ 1,000(1.05)^{-3}$ & $=$ & $\$ 863.84$ & $\$ 1,000(1.05)^{2}$ & $=$ & $\$ 1,102.50$ \\
\hline
\end{tabular}
\end{center}

\begin{center}
\begin{tabular}{|c|c|c|c|c|c|c|c|}
\hline
\multirow{2}{*}{$\frac{\text { Time }}{4}$} & \multirow{2}{*}{$\frac{\text { Cash Flow (\$) }}{1,000}$} & \multicolumn{3}{|c|}{Present Value at $t=0$} & \multicolumn{3}{|c|}{Future Value at $t=5$} \\
\hline
 &  & $\$ 1,000(1.05)^{-4}$ & $=$ & $\$ 822.70$ & $\$ 1,000(1.05)^{1}$ & $=$ & $\$ 1,050.00$ \\
\hline
\multirow[t]{2}{*}{5} & 1,000 & $\$ 1,000(1.05)^{-5}$ & $=$ & $\$ 783.53$ & $\$ 1,000(1.05)^{0}$ & $=$ & $\$ 1,000.00$ \\
\hline
 &  & Sum: &  & $\$ 4,329.48$ & Sum: &  & $\$ 5,525.64$ \\
\hline
\end{tabular}
\end{center}

To see how a lump sum can fund an annuity, assume that we place $\$ 4,329.48$ in the bank today at 5 percent interest. We can calculate the size of the annuity payments by using Equation 11 . Solving for $A$, we find

$$
\begin{aligned}
& A=\frac{\mathrm{PV}}{\frac{1-\left[1 /(1+r)^{N}\right]}{r}} \\
= & \frac{\$ 4,329.48}{\frac{1-\left[1 /(1.05)^{5}\right]}{0.05}} \\
= & \$ 1,000
\end{aligned}
$$

Exhibit 14 shows how the initial investment of $\$ 4,329.48$ can actually generate five $\$ 1,000$ withdrawals over the next five years.

To interpret Exhibit 14, start with an initial present value of $\$ 4,329.48$ at $t=0$. From $t=0$ to $t=1$, the initial investment earns 5 percent interest, generating a future value of $\$ 4,329.48(1.05)=\$ 4,545.95$. We then withdraw $\$ 1,000$ from our account, leaving $\$ 4,545.95-\$ 1,000=\$ 3,545.95$ (the figure reported in the last column for time period 1). In the next period, we earn one year's worth of interest and then make a $\$ 1,000$ withdrawal. After the fourth withdrawal, we have $\$ 952.38$, which earns 5 percent. This amount then grows to $\$ 1,000$ during the year, just enough for us to make the last withdrawal. Thus the initial present value, when invested at 5 percent for five years, generates the $\$ 1,000$ five-year ordinary annuity. The present value of the initial investment is exactly equivalent to the annuity.

Now we can look at how future value relates to annuities. In Exhibit 13, we reported that the future value of the annuity was $\$ 5,525.64$. We arrived at this figure by compounding the first $\$ 1,000$ payment forward four periods, the second $\$ 1,000$ forward three periods, and so on. We then added the five future amounts at $t=5$. The annuity is equivalent to $\$ 5,525.64$ at $t=5$ and $\$ 4,329.48$ at $t=0$. These two dollar measures are thus equivalent. We can verify the equivalence by finding the present value of $\$ 5,525.64$, which is $\$ 5,525.64 \times(1.05)^{-5}=\$ 4,329.48$. We found this result

\begin{center}
\begin{tabular}{|c|c|c|c|c|c|c|}
\hline
$\begin{array}{l}\text { Time } \\ \text { Period }\end{array}$ & $\begin{array}{l}\text { Amount Available } \\ \text { at the Beginning of } \\ \text { the Time Period (\$) }\end{array}$ & Ending Amoun & for & ithdrawal & Withdrawal (\$) & $\begin{array}{l}\text { Amount Available } \\ \text { after Withdrawal (\$ }\end{array}$ \\
\hline
1 & $4,329.48$ & $\$ 4,329.48(1.05)$ & $=$ & $\$ 4,545.95$ & 1,000 & $3,545.95$ \\
\hline
2 & $3,545.95$ & $\$ 3,545.95(1.05)$ & $=$ & $\$ 3,723.25$ & 1,000 & $2,723.25$ \\
\hline
3 & $2,723.25$ & $\$ 2,723.25(1.05)$ & $=$ & $\$ 2,859.41$ & 1,000 & $1,859.41$ \\
\hline
4 & $1,859.41$ & $\$ 1,859.41(1.05)$ & $=$ & $\$ 1,952.38$ & 1,000 & 952.38 \\
\hline
5 & 952.38 & $\$ 952.38(1.05)$ & $=$ & $\$ 1,000$ & 1,000 & 0 \\
\hline
\end{tabular}
\end{center}

above when we showed that a lump sum can generate an annuity.

\section{Exhibit 14: How an Initial Present Value Funds an Annuity}
To summarize what we have learned so far: A lump sum can be seen as equivalent to an annuity, and an annuity can be seen as equivalent to its future value. Thus present values, future values, and a series of cash flows can all be considered equivalent as long as they are indexed at the same point in time. The Time Value of Money

\section{The Cash Flow Additivity Principle}
The cash flow additivity principle-the idea that amounts of money indexed at the same point in time are additive-is one of the most important concepts in time value of money mathematics. We have already mentioned and used this principle; this section provides a reference example for it.

Consider the two series of cash flows shown on the time line in Exhibit 15. The series are denoted $\mathrm{A}$ and $\mathrm{B}$. If we assume that the annual interest rate is 2 percent, we can find the future value of each series of cash flows as follows. Series A's future value is $\$ 100(1.02)+\$ 100=\$ 202$. Series B's future value is $\$ 200(1.02)+\$ 200=\$ 404$. The future value of $(A+B)$ is $\$ 202+\$ 404=\$ 606$ by the method we have used up to this point. The alternative way to find the future value is to add the cash flows of each series, A and B (call it $A+B$ ), and then find the future value of the combined cash flow, as shown in Exhibit 15.

The third time line in Exhibit 15 shows the combined series of cash flows. Series A has a cash flow of $\$ 100$ at $t=1$, and Series B has a cash flow of $\$ 200$ at $t=1$. The combined series thus has a cash flow of $\$ 300$ at $t=1$. We can similarly calculate the cash flow of the combined series at $t=2$. The future value of the combined series $(A+B)$ is $\$ 300(1.02)+\$ 300=\$ 606$-the same result we found when we added the future values of each series.

The additivity and equivalence principles also appear in another common situation. Suppose cash flows are $\$ 4$ at the end of the first year and $\$ 24$ (actually separate payments of $\$ 4$ and $\$ 20$ ) at the end of the second year. Rather than finding present values of the first year's $\$ 4$ and the second year's $\$ 24$, we can treat this situation as a $\$ 4$ annuity for two years and a second-year $\$ 20$ lump sum. If the discount rate were 6 percent, the $\$ 4$ annuity would have a present value of $\$ 7.33$ and the $\$ 20$ lump sum a present value of $\$ 17.80$, for a total of $\$ 25.13$.

\section{Exhibit 15: The Additivity of Two Series of Cash Flows}
\begin{center}
\includegraphics[max width=\textwidth]{2023_05_04_cff39ee44f77d6514e1bg-048}
\end{center}

\section{SUMMARY}
In this reading, we have explored a foundation topic in investment mathematics, the time value of money. We have developed and reviewed the following concepts for use in financial applications:

\begin{itemize}
  \item The interest rate, $r$, is the required rate of return; $r$ is also called the discount rate or opportunity cost.

  \item An interest rate can be viewed as the sum of the real risk-free interest rate and a set of premiums that compensate lenders for risk: an inflation premium, a default risk premium, a liquidity premium, and a maturity premium.

  \item The future value, FV, is the present value, PV, times the future value factor, $(1+r)^{N}$.

  \item The interest rate, $r$, makes current and future currency amounts equivalent based on their time value.

  \item The stated annual interest rate is a quoted interest rate that does not account for compounding within the year.

  \item The periodic rate is the quoted interest rate per period; it equals the stated annual interest rate divided by the number of compounding periods per year.

  \item The effective annual rate is the amount by which a unit of currency will grow in a year with interest on interest included.

  \item An annuity is a finite set of level sequential cash flows.

  \item There are two types of annuities, the annuity due and the ordinary annuity. The annuity due has a first cash flow that occurs immediately; the ordinary annuity has a first cash flow that occurs one period from the present (indexed at $t=1$ ).

  \item On a time line, we can index the present as 0 and then display equally spaced hash marks to represent a number of periods into the future. This representation allows us to index how many periods away each cash flow will be paid.

  \item Annuities may be handled in a similar approach as single payments if we use annuity factors rather than single-payment factors.

  \item The present value, PV, is the future value, FV, times the present value factor, $(1+r)^{-N}$.

  \item The present value of a perpetuity is $A / r$, where $A$ is the periodic payment to be received forever.

  \item It is possible to calculate an unknown variable, given the other relevant variables in time value of money problems.

  \item The cash flow additivity principle can be used to solve problems with uneven cash flows by combining single payments and annuities.

\end{itemize}

\section{PRACTICE PROBLEMS}
\begin{enumerate}
  \item The table below gives current information on the interest rates for two two-year and two eight-year maturity investments. The table also gives the maturity, liquidity, and default risk characteristics of a new investment possibility (Investment 3). All investments promise only a single payment (a payment at maturity). Assume that premiums relating to inflation, liquidity, and default risk are constant across all time horizons.
\end{enumerate}

\begin{center}
\begin{tabular}{lcccc}
\hline
Investment & Maturity (in Years) & Liquidity & Default Risk & Interest Rate (\%) \\
\hline
1 & 2 & High & Low & 2.0 \\
2 & 2 & Low & Low & 2.5 \\
3 & Low & Low & $r_{3}$ &  \\
4 & 7 & High & Low & 4.0 \\
5 & 8 & Low & High & 6.5 \\
\hline
\end{tabular}
\end{center}

Based on the information in the above table, address the following:

A. Explain the difference between the interest rates on Investment 1 and Investment 2.

B. Estimate the default risk premium.

C. Calculate upper and lower limits for the interest rate on Investment $3, r_{3}$.

\begin{enumerate}
  \setcounter{enumi}{1}
  \item The nominal risk-free rate is best described as the sum of the real risk-free rate and a premium for:
A. maturity.
B. liquidity.
C. expected inflation.

  \item Which of the following risk premiums is most relevant in explaining the difference in yields between 30-year bonds issued by the US Treasury and 30-year bonds issued by a small private issuer?
A. Inflation
B. Maturity
C. Liquidity

  \item The value in six years of $\$ 75,000$ invested today at a stated annual interest rate of 7\% compounded quarterly is closest to:
A. $\$ 112,555$
B. $\$ 113,330$.
C. $\$ 113,733$

  \item A bank quotes a stated annual interest rate of $4.00 \%$. If that rate is equal to an effective annual rate of $4.08 \%$, then the bank is compounding interest:

\end{enumerate}

A. daily. B. quarterly.

C. semiannually.

\begin{enumerate}
  \setcounter{enumi}{5}
  \item Given a $€ 1,000,000$ investment for four years with a stated annual rate of $3 \%$ compounded continuously, the difference in its interest earnings compared with the same investment compounded daily is closest to:
A. $€ 1$.
B. $€ 6$.
C. $€ 455$.

  \item A couple plans to set aside $\$ 20,000$ per year in a conservative portfolio projected to earn 7 percent a year. If they make their first savings contribution one year from now, how much will they have at the end of 20 years?

  \item Two years from now, a client will receive the first of three annual payments of $\$ 20,000$ from a small business project. If she can earn 9 percent annually on her investments and plans to retire in six years, how much will the three business project payments be worth at the time of her retirement?

  \item A saver deposits the following amounts in an account paying a stated annual rate of $4 \%$, compounded semiannually:

\end{enumerate}

\begin{center}
\begin{tabular}{lc}
\hline
Year & End of Year Deposits (\$) \\
\hline
1 & 4,000 \\
2 & 8,000 \\
3 & 7,000 \\
4 & 10,000 \\
\hline
\end{tabular}
\end{center}

At the end of Year 4, the value of the account is closest to:
A. $\$ 30,432$
B. $\$ 30,447$
C. $\$ 31,677$

\begin{enumerate}
  \setcounter{enumi}{9}
  \item To cover the first year's total college tuition payments for his two children, a father will make a $\$ 75,000$ payment five years from now. How much will he need to invest today to meet his first tuition goal if the investment earns 6 percent annually?

  \item Given the following timeline and a discount rate of $4 \%$ a year compounded annually, the present value (PV), as of the end of Year $5\left(\mathrm{PV}_{5}\right)$, of the cash flow received at the end of Year 20 is closest to:

\end{enumerate}

\includegraphics[max width=\textwidth, center]{2023_05_04_cff39ee44f77d6514e1bg-051}
A. $\$ 22,819$.
B. $\$ 27,763$.
C. $\$ 28,873$. 12. A client requires $\pounds 100,000$ one year from now. If the stated annual rate is $2.50 \%$ compounded weekly, the deposit needed today is closest to:
A. $\pounds 97,500$.
B. $\pounds 97,532$.
C. $\pounds 97,561$.

\begin{enumerate}
  \setcounter{enumi}{12}
  \item A client can choose between receiving 10 annual $\$ 100,000$ retirement payments, starting one year from today, or receiving a lump sum today. Knowing that he can invest at a rate of 5 percent annually, he has decided to take the lump sum. What lump sum today will be equivalent to the future annual payments?

  \item You are considering investing in two different instruments. The first instrument will pay nothing for three years, but then it will pay $\$ 20,000$ per year for four years. The second instrument will pay $\$ 20,000$ for three years and $\$ 30,000$ in the fourth year. All payments are made at year-end. If your required rate of return on these investments is 8 percent annually, what should you be willing to pay for:

\end{enumerate}

A. The first instrument?

B. The second instrument (use the formula for a four-year annuity)?

\begin{enumerate}
  \setcounter{enumi}{14}
  \item Suppose you plan to send your daughter to college in three years. You expect her to earn two-thirds of her tuition payment in scholarship money, so you estimate that your payments will be $\$ 10,000$ a year for four years. To estimate whether you have set aside enough money, you ignore possible inflation in tuition payments and assume that you can earn 8 percent annually on your investments. How much should you set aside now to cover these payments?

  \item An investment pays $€ 300$ annually for five years, with the first payment occurring today. The present value (PV) of the investment discounted at a $4 \%$ annual rate is closest to:

\end{enumerate}

A. $€ 1,336$.

B. $€ 1,389$.

C. $€ 1,625$.

\begin{enumerate}
  \setcounter{enumi}{16}
  \item At a $5 \%$ interest rate per year compounded annually, the present value (PV) of a 10 -year ordinary annuity with annual payments of $\$ 2,000$ is $\$ 15,443.47$. The PV of a 10-year annuity due with the same interest rate and payments is closest to:
A. $\$ 14,708$.
B. $\$ 16,216$.
C. $\$ 17,443$

  \item Grandparents are funding a newborn's future university tuition costs, estimated at $\$ 50,000$ /year for four years, with the first payment due as a lump sum in 18 years. Assuming a 6\% effective annual rate, the required deposit today is closest to:

\end{enumerate}

A. $\$ 60,699$.

B. $\$ 64,341$. C. $\$ 68,201$.

\begin{enumerate}
  \setcounter{enumi}{18}
  \item The present value (PV) of an investment with the following year-end cash flows (CF) and a $12 \%$ required annual rate of return is closest to:
\end{enumerate}

\begin{center}
\begin{tabular}{lc}
\hline
Year & Cash Flow (€) \\
\hline
1 & 100,000 \\
2 & 150,000 \\
5 & $-10,000$ \\
\hline
\end{tabular}
\end{center}

A. $€ 201,747$.
B. $€ 203,191$.
C. $€ 227,573$.

\begin{enumerate}
  \setcounter{enumi}{19}
  \item A perpetual preferred stock makes its first quarterly dividend payment of $\$ 2.00$ in five quarters. If the required annual rate of return is $6 \%$ compounded quarterly, the stock's present value is closest to:
A. $\$ 31$.
B. $\$ 126$.
C. $\$ 133$.

  \item A sweepstakes winner may select either a perpetuity of $\pounds 2,000$ a month beginning with the first payment in one month or an immediate lump sum payment of $\pounds 350,000$. If the annual discount rate is $6 \%$ compounded monthly, the present value of the perpetuity is:
A. less than the lump sum.
B. equal to the lump sum.
C. greater than the lump sum.

  \item For a lump sum investment of $¥ 250,000$ invested at a stated annual rate of $3 \%$ compounded daily, the number of months needed to grow the sum to $¥ 1,000,000$ is closest to:
A. 555
B. 563 .
C. 576 .

  \item An investment of $€ 500,000$ today that grows to $€ 800,000$ after six years has a stated annual interest rate closest to:
A. $7.5 \%$ compounded continuously.
B. $7.7 \%$ compounded daily.
C. $8.0 \%$ compounded semiannually.

  \item A client plans to send a child to college for four years starting 18 years from now. Having set aside money for tuition, she decides to plan for room and board also. She estimates these costs at $\$ 20,000$ per year, payable at the beginning of each year, by the time her child goes to college. If she starts next year and makes 17 payments into a savings account paying 5 percent annually, what annual payments must she make?

  \item A couple plans to pay their child's college tuition for 4 years starting 18 years from now. The current annual cost of college is $C \$ 7,000$, and they expect this cost to rise at an annual rate of 5 percent. In their planning, they assume that they can earn 6 percent annually. How much must they put aside each year, starting next year, if they plan to make 17 equal payments?

  \item A sports car, purchased for $\pounds 200,000$, is financed for five years at an annual rate of $6 \%$ compounded monthly. If the first payment is due in one month, the monthly payment is closest to:
A. $\pounds 3,847$.
B. $\pounds 3,867$.
C. $\pounds 3,957$.

  \item Given a stated annual interest rate of $6 \%$ compounded quarterly, the level amount that, deposited quarterly, will grow to $\pounds 25,000$ at the end of 10 years is closest to:
A. $\pounds 461$.
B. $\pounds 474$.
C. $\pounds 836$.

  \item A client invests $€ 20,000$ in a four-year certificate of deposit (CD) that annually pays interest of $3.5 \%$. The annual CD interest payments are automatically reinvested in a separate savings account at a stated annual interest rate of $2 \%$ compounded monthly. At maturity, the value of the combined asset is closest to:
A. $€ 21,670$.
B. $€ 22,890$.
C. $€ 22,950$.

\end{enumerate}

\section{SOLUTIONS}
1.

A. Investment 2 is identical to Investment 1 except that Investment 2 has low liquidity. The difference between the interest rate on Investment 2 and Investment 1 is 0.5 percentage point. This amount represents the liquidity premium, which represents compensation for the risk of loss relative to an investment's fair value if the investment needs to be converted to cash quickly.

B. To estimate the default risk premium, find the two investments that have the same maturity but different levels of default risk. Both Investments 4 and 5 have a maturity of eight years. Investment 5 , however, has low liquidity and thus bears a liquidity premium. The difference between the interest rates of Investments 5 and 4 is 2.5 percentage points. The liquidity premium is 0.5 percentage point (from Part A). This leaves $2.5-0.5=2.0$ percentage points that must represent a default risk premium reflecting Investment 5's high default risk.

C. Investment 3 has liquidity risk and default risk comparable to Investment 2, but with its longer time to maturity, Investment 3 should have a higher maturity premium. The interest rate on Investment $3, r_{3}$, should thus be above 2.5 percent (the interest rate on Investment 2). If the liquidity of Investment 3 were high, Investment 3 would match Investment 4 except for Investment 3's shorter maturity. We would then conclude that Investment 3's interest rate should be less than the interest rate on Investment 4, which is 4 percent. In contrast to Investment 4, however, Investment 3 has low liquidity. It is possible that the interest rate on Investment 3 exceeds that of Investment 4 despite 3's shorter maturity, depending on the relative size of the liquidity and maturity premiums. However, we expect $r_{3}$ to be less than 4.5 percent, the expected interest rate on Investment 4 if it had low liquidity. Thus 2.5 percent $<r_{3}<4.5$ percent.

\begin{enumerate}
  \setcounter{enumi}{1}
  \item $\mathrm{C}$ is correct. The sum of the real risk-free interest rate and the inflation premium is the nominal risk-free rate.

  \item C is correct. US Treasury bonds are highly liquid, whereas the bonds of small issuers trade infrequently and the interest rate includes a liquidity premium. This liquidity premium reflects the relatively high costs (including the impact on price) of selling a position.

  \item $\mathrm{C}$ is correct, as shown in the following (where $\mathrm{FV}$ is future value and PV is present value):

\end{enumerate}

$$
\begin{aligned}
& \mathrm{FV}=\mathrm{PV}\left(1+\frac{r_{s}}{m}\right)^{m N} \\
& \mathrm{FV}_{6}=\$ 75,000\left(1+\frac{0.07}{4}\right)^{(4 \times 6)} \\
& \mathrm{FV}_{6}=\$ 113,733.21
\end{aligned}
$$

\begin{enumerate}
  \setcounter{enumi}{4}
  \item A is correct. The effective annual rate (EAR) when compounded daily is $4.08 \%$. EAR $=(1+\text { Periodic interest rate })^{m}-1$ $\mathrm{EAR}=(1+0.04 / 365)^{365}-1$
\end{enumerate}

$\operatorname{EAR}=(1.0408)-1=0.04081 \approx 4.08 \%$

\begin{enumerate}
  \setcounter{enumi}{5}
  \item B is correct. The difference between continuous compounding and daily compounding is
\end{enumerate}

$€ 127,496.85-€ 127,491.29=€ 5.56$, or $\approx € 6$, as shown in the following calculations.

With continuous compounding, the investment earns (where PV is present value)

$$
\begin{aligned}
& \text { PV } e^{r_{s} N}-\mathrm{PV}=€ 1,000,000 e^{0.03(4)}-€ 1,000,000 \\
& =€ 1,127,496.85-€ 1,000,000 \\
& =€ 127,496.85
\end{aligned}
$$

With daily compounding, the investment earns:

$€ 1,000,000(1+0.03 / 365)^{365(4)}-€ 1,000,000=€ 1,127,491.29-€ 1,000,000=$ $€ 127,491.29$.

7.

i. Draw a time line.

\begin{center}
\includegraphics[max width=\textwidth]{2023_05_04_cff39ee44f77d6514e1bg-056(1)}
\end{center}

ii. Identify the problem as the future value of an annuity.

iii. Use the formula for the future value of an annuity.

$$
\begin{aligned}
& \mathrm{FV}_{N}=A\left[\frac{(1+r)^{N}-1}{r}\right] \\
& =\$ 20,000\left[\frac{(1+0.07)^{20}-1}{0.07}\right] \\
& =\$ 819,909.85
\end{aligned}
$$

\begin{center}
\includegraphics[max width=\textwidth]{2023_05_04_cff39ee44f77d6514e1bg-056}
\end{center}

$$
\mathrm{FV}=\$ 819,909.85
$$

iv. Alternatively, use a financial calculator.

\begin{center}
\begin{tabular}{lc}
\hline
$\begin{array}{l}\text { Notation Used } \\ \text { on Most Calculators }\end{array}$ & $\begin{array}{c}\text { Numerical Va } \\ \text { for This Prob }\end{array}$ \\
\hline
$N$ & 20 \\
$\% i$ & 7 \\
PV & $\mathrm{n} / \mathrm{a}(=0)$ \\
FV compute & $X$ \\
\end{tabular}
\end{center}

\begin{center}
\begin{tabular}{lc}
\hline
$\begin{array}{lc}\text { Notation Used } \\ \text { on Most Calculators }\end{array}$ & $\begin{array}{l}\text { Numerical Value } \\ \text { for This Problem }\end{array}$ \\
\hline
PMT & $\$ 20,000$ \\
\hline
\end{tabular}
\end{center}

Enter 20 for $N$, the number of periods. Enter 7 for the interest rate and 20,000 for the payment size. The present value is not needed, so enter 0 . Calculate the future value. Verify that you get $\$ 819,909.85$ to make sure you have mastered your calculator's keystrokes.

In summary, if the couple sets aside $\$ 20,000$ each year (starting next year), they will have $\$ 819,909.85$ in 20 years if they earn 7 percent annually.

8.

i. Draw a time line.

\begin{center}
\includegraphics[max width=\textwidth]{2023_05_04_cff39ee44f77d6514e1bg-057}
\end{center}

ii. Recognize the problem as the future value of a delayed annuity. Delaying the payments requires two calculations.

iii. Use the formula for the future value of an annuity (Equation 7).

$\mathrm{FV}_{N}=A\left[\frac{(1+r)^{N}-1}{r}\right]$

to bring the three $\$ 20,000$ payments to an equivalent lump sum of $\$ 65,562.00$ four years from today.

\begin{center}
\begin{tabular}{lc}
\hline
$\begin{array}{l}\text { Notation Used } \\ \text { on Most Calculators }\end{array}$ & $\begin{array}{c}\text { Numerical Value } \\ \text { for This Problem }\end{array}$ \\
\hline
$N$ & 3 \\
$\% i$ & 9 \\
PV & $\mathrm{n} / \mathrm{a}(=0)$ \\
FV compute & $X$ \\
PMT & $\$ 20,000$ \\
\hline
\end{tabular}
\end{center}

iv. Use the formula for the future value of a lump sum (Equation 2), $\mathrm{FV}_{N}=$ $\mathrm{PV}(1+r)^{N}$, to bring the single lump sum of $\$ 65,562.00$ to an equivalent lump sum of $\$ 77,894.21$ six years from today.

Notation Used on Most Calculators

\begin{center}
\includegraphics[max width=\textwidth]{2023_05_04_cff39ee44f77d6514e1bg-057(1)}
\end{center}

$N$

$\% i$

$\mathrm{PV}$

FV compute Numerical Value for This Problem

2

9

$\$ 65,562.00$

$X$

\begin{center}
\includegraphics[max width=\textwidth]{2023_05_04_cff39ee44f77d6514e1bg-058(2)}
\end{center}

In summary, your client will have $\$ 77,894.21$ in six years if she receives three yearly payments of $\$ 20,000$ starting in Year 2 and can earn 9 percent annually on her investments.

\begin{enumerate}
  \setcounter{enumi}{8}
  \item B is correct. To solve for the future value of unequal cash flows, compute the future value of each payment as of Year 4 at the semiannual rate of $2 \%$, and then sum the individual future values, as follows:
\end{enumerate}

\begin{center}
\begin{tabular}{lccc}
\hline
Year & End of Year Deposits (\$) & Factor & Future Value (\$) \\
\hline
1 & 4,000 & $(1.02)^{6}$ & $4,504.65$ \\
2 & 8,000 & $(1.02)^{4}$ & $8,659.46$ \\
3 & 7,000 & $(1.02)^{2}$ & $7,282.80$ \\
4 & 10,000 & $(1.02)^{0}$ & $10,000.00$ \\
 &  & Sum $=$ & $30,446.91$ \\
\hline
\end{tabular}
\end{center}

10.

i. Draw a time line.

\begin{center}
\includegraphics[max width=\textwidth]{2023_05_04_cff39ee44f77d6514e1bg-058(1)}
\end{center}

ii. Identify the problem as the present value of a lump sum.

iii. Use the formula for the present value of a lump sum.

$$
\begin{aligned}
& \mathrm{PV}=\mathrm{FV}_{N}(1+r)^{-N} \\
& =\$ 75,000(1+0.06)^{-5} \\
& =\$ 56,044.36
\end{aligned}
$$

\begin{center}
\includegraphics[max width=\textwidth]{2023_05_04_cff39ee44f77d6514e1bg-058}
\end{center}

In summary, the father will need to invest $\$ 56,044.36$ today in order to have $\$ 75,000$ in five years if his investments earn 6 percent annually.

\begin{enumerate}
  \setcounter{enumi}{10}
  \item B is correct. The PV in Year 5 of a $\$ 50,000$ lump sum paid in Year 20 is $\$ 27,763.23$ (where FV is future value):
\end{enumerate}

$$
\mathrm{PV}=\mathrm{FV}_{N}(1+r)^{-N}
$$

$$
\begin{aligned}
& \mathrm{PV}=\$ 50,000(1+0.04)^{-15} \\
& P V=\$ 27,763.23
\end{aligned}
$$

\begin{enumerate}
  \setcounter{enumi}{11}
  \item B is correct because $\pounds 97,531$ represents the present value (PV) of $\pounds 100,000$ received one year from today when today's deposit earns a stated annual rate of $2.50 \%$ and interest compounds weekly, as shown in the following equation (where FV is future value):
\end{enumerate}

$$
\begin{aligned}
& \mathrm{PV}=\mathrm{FV}_{N}\left(1+\frac{r_{s}}{m}\right)^{-m N} \\
& \mathrm{PV}=\pounds 100,000\left(1+\frac{0.025}{52}\right)^{-52} \\
& \mathrm{PV}=\pounds 97,531.58 .
\end{aligned}
$$

13.

i. Draw a time line for the 10 annual payments.

\begin{center}
\includegraphics[max width=\textwidth]{2023_05_04_cff39ee44f77d6514e1bg-059(1)}
\end{center}

ii. Identify the problem as the present value of an annuity.

iii. Use the formula for the present value of an annuity.

$$
\begin{aligned}
& \mathrm{PV}=A\left[\frac{1-\frac{1}{(1+r)^{N}}}{r}\right] \\
& =\$ 100,000\left[\frac{1-\frac{1}{(1+0.05)^{10}}}{0.05}\right] \\
& =\$ 772,173.49
\end{aligned}
$$

\begin{center}
\includegraphics[max width=\textwidth]{2023_05_04_cff39ee44f77d6514e1bg-059}
\end{center}

$$
\mathrm{PV}=\$ 772,173.49
$$

iv. Alternatively, use a financial calculator.

\begin{center}
\begin{tabular}{lc}
$\begin{array}{l}\text { Notation Used } \\ \text { on Most Calculators }\end{array}$ & $\begin{array}{c}\text { Numerical V } \\ \text { for This Prob }\end{array}$ \\
\hline
$N$ & 10 \\
$\% i$ & 5 \\
PV compute & $X$ \\
FV & n/a $(=0)$ \\
\end{tabular}
\end{center}

\begin{center}
\begin{tabular}{lc}
\hline
$\begin{array}{l}\text { Notation Used } \\ \text { on Most Calculators }\end{array}$ & $\begin{array}{l}\text { Numerical Value } \\ \text { for This Problem }\end{array}$ \\
\hline
PMT & $\$ 100,000$ \\
\hline
\end{tabular}
\end{center}

In summary, the present value of 10 payments of $\$ 100,000$ is $\$ 772,173.49$ if the first payment is received in one year and the rate is 5 percent compounded annually. Your client should accept no less than this amount for his lump sum payment.

14.

A. To evaluate the first instrument, take the following steps:

i. Draw a time line.

\begin{center}
\includegraphics[max width=\textwidth]{2023_05_04_cff39ee44f77d6514e1bg-060(1)}
\end{center}

$\mathrm{PV}_{3}=A\left[\frac{1-\frac{1}{(1+r)^{N}}}{r}\right]$

$=\$ 20,000\left[\frac{1-\frac{1}{(1+0.08)^{4}}}{0.08}\right]$

$=\$ 66,242.54$

$\mathrm{PV}_{0}=\frac{\mathrm{PV}_{3}}{(1+r)^{N}}=\frac{\$ 66,242.54}{1.08^{3}}=\$ 52,585.46$

ii. You should be willing to pay $\$ 52,585.46$ for this instrument.

B. To evaluate the second instrument, take the following steps:

i. Draw a time line.

\begin{center}
\includegraphics[max width=\textwidth]{2023_05_04_cff39ee44f77d6514e1bg-060}
\end{center}

The time line shows that this instrument can be analyzed as an ordinary annuity of $\$ 20,000$ with four payments (valued in Step ii below) and a $\$ 10,000$ payment to be received at $t=4$ (valued in Step iii below).

$\mathrm{PV}=A\left[\frac{1-\frac{1}{(1+r)^{N}}}{r}\right]$

$=\$ 20,000\left[\frac{1-\frac{1}{(1+0.08)^{4}}}{0.08}\right]$

$=\$ 66,242.54$

$\mathrm{PV}=\frac{\mathrm{FV}_{4}}{(1+r)^{N}}=\frac{\$ 10,000}{(1+0.08)^{4}}=\$ 7,350.30$

ii. $\quad$ Total $=\$ 66,242.54+\$ 7,350.30=\$ 73,592.84$ You should be willing to pay $\$ 73,592.84$ for this instrument.

15.

i. Draw a time line.

\begin{center}
\includegraphics[max width=\textwidth]{2023_05_04_cff39ee44f77d6514e1bg-061(1)}
\end{center}

ii. Recognize the problem as a delayed annuity. Delaying the payments requires two calculations.

iii. Use the formula for the present value of an annuity (Equation 11).

$\mathrm{PV}=A\left[\frac{1-\frac{1}{(1+r)^{N}}}{r}\right]$

to bring the four payments of $\$ 10,000$ back to a single equivalent lump sum of $\$ 33,121.27$ at $t=2$. Note that we use $t=2$ because the first annuity payment is then one period away, giving an ordinary annuity.

\begin{center}
\begin{tabular}{lc}
\hline
$\begin{array}{l}\text { Notation Used } \\ \text { on Most Calculators }\end{array}$ & $\begin{array}{l}\text { Numerical Value } \\ \text { for This Problem }\end{array}$ \\
\hline
$N$ & 4 \\
$\% i$ & 8 \\
PV compute & $X$ \\
PMT & $\$ 10,000$ \\
\hline
\end{tabular}
\end{center}

iv. Then use the formula for the present value of a lump sum (Equation 8), PV $=\mathrm{FV}_{N}(1+r)^{-N}$, to bring back the single payment of $\$ 33,121.27$ (at $t=2$ ) to an equivalent single payment of $\$ 28,396.15$ (at $t=0$ ).

\begin{center}
\begin{tabular}{lc}
$\begin{array}{l}\text { Notation Used } \\ \text { on Most Calculators }\end{array}$ & $\begin{array}{c}\text { Numerical Value } \\ \text { for This Problem }\end{array}$ \\
\hline
$N$ & 2 \\
$\% i$ & 8 \\
PV compute & $X$ \\
FV & $\$ 33,121.27$ \\
PMT & $\mathrm{n} / \mathrm{a}(=0)$ \\
\hline
\end{tabular}
\end{center}

\begin{center}
\includegraphics[max width=\textwidth]{2023_05_04_cff39ee44f77d6514e1bg-061}
\end{center}

In summary, you should set aside $\$ 28,396.15$ today to cover four payments of $\$ 10,000$ starting in three years if your investments earn a rate of 8 percent annually.

\begin{enumerate}
  \setcounter{enumi}{15}
  \item B is correct, as shown in the following calculation for an annuity (A) due: $\mathrm{PV}=A\left[\frac{1-\frac{1}{(1+r)^{N}}}{r}\right](1+r)$
\end{enumerate}

where $\mathrm{A}=€ 300, r=0.04$, and $N=5$.

$P V=€ 300\left[\frac{1-\frac{1}{(1+.04)^{5}}}{.04}\right]$

$\mathrm{PV}=€ 1,388.97$, or $\approx € 1,389$

\begin{enumerate}
  \setcounter{enumi}{16}
  \item B is correct.
\end{enumerate}

The present value of a 10 -year annuity (A) due with payments of $\$ 2,000$ at a $5 \%$ discount rate is calculated as follows:

$\mathrm{PV}=A\left[\frac{1-\frac{1}{(1+r)^{N}}}{r}\right]+\$ 2,000$

$\mathrm{PV}=\$ 2,000\left[\frac{1-\frac{1}{(1+0.05)^{9}}}{0.05}\right]+\$ 2,000$

$\mathrm{PV}=\$ 16,215.64$

Alternatively, the PV of a 10-year annuity due is simply the PV of the ordinary annuity multiplied by 1.05 :

$$
\begin{aligned}
& \mathrm{PV}=\$ 15,443.47 \times 1.05 \\
& \mathrm{PV}=\$ 16,215.64 .
\end{aligned}
$$

\begin{enumerate}
  \setcounter{enumi}{17}
  \item B is correct. First, find the present value (PV) of an ordinary annuity in Year 17 that represents the tuition costs:
\end{enumerate}

$$
\begin{aligned}
& \$ 50,000\left[\frac{1-\frac{1}{(1+0.06)^{4}}}{0.06}\right] \\
& =\$ 50,000 \times 3.4651 \\
& =\$ 173,255.28 .
\end{aligned}
$$

Then, find the PV of the annuity in today's dollars (where FV is future value):

$$
\begin{aligned}
& \mathrm{PV}_{0}=\frac{\mathrm{FV}}{(1+0.06)^{17}} \\
& \mathrm{PV}_{0}=\frac{\$ 173,255.28}{(1+0.06)^{17}} \\
& \mathrm{PV}_{0}=\$ 64,340.85 \approx \$ 64,341 .
\end{aligned}
$$

\begin{enumerate}
  \setcounter{enumi}{18}
  \item B is correct, as shown in the following table.
\end{enumerate}

\begin{center}
\begin{tabular}{lccc}
\hline
Year & $\begin{array}{c}\text { Cash Flow } \\ (\boldsymbol{\epsilon})\end{array}$ & $\begin{array}{c}\text { Formula } \\ \text { CF } \times(1+\boldsymbol{r})^{t}\end{array}$ & $\begin{array}{c}\text { PV at } \\ \text { Year 0 }\end{array}$ \\
\hline
1 & 100,000 & $100,000(1.12)^{-1}=$ & $89,285.71$ \\
2 & 150,000 & $150,000(1.12)^{-2}=$ & $119,579.08$ \\
5 & $-10,000$ & $-10,000(1.12)^{-5}=$ & $-5,674.27$ \\
\end{tabular}
\end{center}

\begin{center}
\begin{tabular}{lccc}
\hline
Year & $\begin{array}{c}\text { Cash Flow } \\ (\boldsymbol{\epsilon})\end{array}$ & $\begin{array}{c}\text { Formula } \\ \text { CF } \times(\mathbf{1}+r)^{t}\end{array}$ & $\begin{array}{c}\text { PV at } \\ \text { Year } \mathbf{0}\end{array}$ \\
\hline
 &  & $203,190.52$ &  \\
\hline
\end{tabular}
\end{center}

\begin{enumerate}
  \setcounter{enumi}{19}
  \item B is correct. The value of the perpetuity one year from now is calculated as: $\mathrm{PV}=\mathrm{A} / r$, where $\mathrm{PV}$ is present value, $\mathrm{A}$ is annuity, and $r$ is expressed as a quarterly required rate of return because the payments are quarterly.
\end{enumerate}

$\mathrm{PV}=\$ 2.00 /(0.06 / 4)$

$\mathrm{PV}=\$ 133.33$

The value today is (where $\mathrm{FV}$ is future value)

$\mathrm{PV}=\mathrm{FV}_{N}(1+r)^{-N}$

$\mathrm{PV}=\$ 133.33(1+0.015)^{-4}$

$\mathrm{PV}=\$ 125.62 \approx \$ 126$

\begin{enumerate}
  \setcounter{enumi}{20}
  \item C is correct. As shown below, the present value (PV) of a $\pounds 2,000$ per month perpetuity is worth approximately $\pounds 400,000$ at a $6 \%$ annual rate compounded monthly. Thus, the present value of the annuity (A) is worth more than the lump sum offers.
\end{enumerate}

$\mathrm{A}=\pounds 2,000$

$r=(6 \% / 12)=0.005$

$\mathrm{PV}=(\mathrm{A} / r)$

$\mathrm{PV}=(\pounds 2,000 / 0.005)$

$\mathrm{PV}=\pounds 400,000$

\begin{enumerate}
  \setcounter{enumi}{21}
  \item A is correct. The effective annual rate (EAR) is calculated as follows:
\end{enumerate}

$\mathrm{EAR}=(1+\text { Periodic interest rate })^{m}-1$

$\mathrm{EAR}=(1+0.03 / 365)^{365}-1$

$\mathrm{EAR}=(1.03045)-1=0.030453 \approx 3.0453 \%$.

Solving for $N$ on a financial calculator results in (where $\mathrm{FV}$ is future value and $\mathrm{PV}$ is present value):

$(1+0.030453)^{N}=\mathrm{FV}_{N} / \mathrm{PV}=¥ 1,000,000 / ¥ 250,000$

$=46.21$ years, which multiplied by 12 to convert to months results in 554.5 , or $\approx$ 555 months.

\begin{enumerate}
  \setcounter{enumi}{22}
  \item $\mathrm{C}$ is correct, as shown in the following (where FV is future value and PV is present value):
\end{enumerate}

If:

$\mathrm{FV}_{N}=\mathrm{PV}\left(1+\frac{r_{s}}{m}\right)^{m N}$,

Then:

$$
\begin{aligned}
& \left(\frac{\mathrm{FV}_{N}}{\mathrm{PV}}\right)^{\frac{1}{m N}}-1=\frac{r_{s}}{m} \\
& \left(\frac{800,000}{500,000}\right)^{\frac{1}{2 \times 6}}-1=\frac{r_{s}}{2} \\
& r_{s}=0.07988 \text { (rounded to } 8.0 \% \text { ). }
\end{aligned}
$$

24.

i. Draw a time line.

\begin{center}
\includegraphics[max width=\textwidth]{2023_05_04_cff39ee44f77d6514e1bg-064}
\end{center}

ii. Recognize that you need to equate the values of two annuities.

iii. Equate the value of the four $\$ 20,000$ payments to a single payment in Period 17 using the formula for the present value of an annuity (Equation 11), with $r=0.05$. The present value of the college costs as of $t=17$ is $\$ 70,919$.

$$
\text { PV }=\$ 20,000\left[\frac{1-\frac{1}{(1.05)^{4}}}{0.05}\right]=\$ 70,919
$$

\begin{center}
\begin{tabular}{lc}
\hline
$\begin{array}{l}\text { Notation Used } \\ \text { on Most Calculators }\end{array}$ & $\begin{array}{c}\text { Numerical Value } \\ \text { for This Problem }\end{array}$ \\
\hline
$N$ & 4 \\
$\% i$ & 5 \\
PV compute & $X$ \\
FV & $\mathrm{n} / \mathrm{a}(=0)$ \\
PMT & $\$ 20,000$ \\
\hline
\end{tabular}
\end{center}

iv. Equate the value of the 17 investments of $X$ to the amount calculated in Step iii, college costs as of $t=17$, using the formula for the future value of an annuity (Equation 7). Then solve for $X$.

$\$ 70,919=\left[\frac{(1.05)^{17}-1}{0.05}\right]=25.840366 X$

$X=\$ 2,744.50$

Notation Used on Most Calculators

N

$\% i$

PV

FV Numerical Value for This Problem

17

5

$\mathrm{n} / \mathrm{a}(=0)$

$\$ 70,919$

\begin{center}
\begin{tabular}{ll}
\hline
$\begin{array}{l}\text { Notation Used } \\ \text { on Most Calculators }\end{array}$ & $\begin{array}{l}\text { Numerical Value } \\ \text { for This Problem }\end{array}$ \\
\hline
\end{tabular}
\end{center}

PMT compute

$X$

\begin{center}
\includegraphics[max width=\textwidth]{2023_05_04_cff39ee44f77d6514e1bg-065(1)}
\end{center}

In summary, your client will have to save $\$ 2,744.50$ each year if she starts next year and makes 17 payments into a savings account paying 5 percent annually.

25.

i. Draw a time line.

\begin{center}
\includegraphics[max width=\textwidth]{2023_05_04_cff39ee44f77d6514e1bg-065}
\end{center}

ii. Recognize that the payments in Years $18,19,20$, and 21 are the future values of a lump sum of $C \$ 7,000$ in Year 0 .

iii. With $r=5 \%$, use the formula for the future value of a lump sum (Equation 2), $\mathrm{FV}_{N}=\mathrm{PV}(1+\mathrm{r})^{N}$, four times to find the payments. These future values are shown on the time line below.

\begin{center}
\includegraphics[max width=\textwidth]{2023_05_04_cff39ee44f77d6514e1bg-065(2)}
\end{center}

iv. Using the formula for the present value of a lump sum $(r=6 \%)$, equate the four college payments to single payments as of $t=17$ and add them together. $C \$ 16,846(1.06)^{-1}+C \$ 17,689(1.06)^{-2}+C \$ 18,573(1.06)^{-3}+$ $\mathrm{C} \$ 19,502(1.06)^{-4}=\mathrm{C} \$ 62,677$

v. Equate the sum of $C \$ 62,677$ at $t=17$ to the 17 payments of $X$, using the formula for the future value of an annuity (Equation 7). Then solve for $X$.

$\mathrm{C} \$ 62,677=X\left[\frac{(1.06)^{17}-1}{0.06}\right]=28.21288 X$

$X=\mathrm{C} \$ 2,221.58$

Notation Used

Numerical Value

on Most Calculators

for This Problem

$N$

\begin{center}
\begin{tabular}{lc}
\hline
$\begin{array}{l}\text { Notation Used } \\ \text { on Most Calculators }\end{array}$ & $\begin{array}{c}\text { Numerical Value } \\ \text { for This Problem }\end{array}$ \\
\hline
PV & $\mathrm{n} / \mathrm{a}(=0)$ \\
FV & $\mathrm{C} \$ 62,677$ \\
PMT compute & $X$ \\
\hline
\end{tabular}
\end{center}

\begin{center}
\includegraphics[max width=\textwidth]{2023_05_04_cff39ee44f77d6514e1bg-066(1)}
\end{center}

In summary, the couple will need to put aside $\mathrm{C} \$ 2,221.58$ each year if they start next year and make 17 equal payments.

\begin{enumerate}
  \setcounter{enumi}{25}
  \item B is correct, calculated as follows (where A is annuity and PV is present value):
\end{enumerate}

$$
\begin{aligned}
& \mathrm{A}=(\mathrm{PV} \text { of annuity }) /\left[\frac{1-\frac{1}{\left(1+r_{s} / m\right)^{m N}}}{r_{s} / m}\right] \\
& =(\pounds 200,000) /\left[\frac{1-\frac{1}{\left(1+r_{s} / m\right)^{m N}}}{r_{s} / m}\right] \\
& (\pounds 200,000) /\left[\frac{1-\frac{1}{(1+0.06 / 12)^{12(5)}}}{0.06 / 12}\right] \\
& =(\pounds 200,000) / 51.72556 \\
& =\pounds 3,866.56
\end{aligned}
$$

\begin{enumerate}
  \setcounter{enumi}{26}
  \item A is correct. To solve for an annuity (A) payment, when the future value (FV), interest rate, and number of periods is known, use the following equation:
\end{enumerate}

$\mathrm{FV}=\mathrm{A}\left[\frac{\left(1+\frac{r_{s}}{m}\right)^{m N}-1}{\frac{r}{m}}\right]$

$$
\pounds 25,000=\mathrm{A}\left[\frac{\left(1+\frac{0.06}{4}\right)^{4 \times 10}-1}{\frac{0.06}{4}}\right]
$$

$\mathrm{A}=\pounds 460.68$

\begin{enumerate}
  \setcounter{enumi}{27}
  \item B is correct, as the following cash flows show:
\end{enumerate}

\begin{center}
\includegraphics[max width=\textwidth]{2023_05_04_cff39ee44f77d6514e1bg-066}
\end{center}

The four annual interest payments are based on the CD's $3.5 \%$ annual rate.

The first payment grows at $2.0 \%$ compounded monthly for three years (where FV is future value): $\mathrm{FV}_{N}=€ 700\left(1+\frac{0.02}{12}\right)^{3 \times 12}$

$\mathrm{FV}_{N}=743.25$

The second payment grows at $2.0 \%$ compounded monthly for two years:

$\mathrm{FV}_{N}=€ 700\left(1+\frac{0.02}{12}\right)^{2 \times 12}$

$\mathrm{FV}_{N}=728.54$

The third payment grows at $2.0 \%$ compounded monthly for one year:

$\mathrm{FV}_{N}=€ 700\left(1+\frac{0.02}{12}\right)^{1 \times 12}$

$\mathrm{FV}_{N}=714.13$

The fourth payment is paid at the end of Year 4. Its future value is $€ 700$.

The sum of all future value payments is as follows:

\begin{center}
\begin{tabular}{|c|c|}
\hline
$€ 20,000.00$ & $\mathrm{CD}$ \\
\hline
$€ 743.25$ & First payment's $F V$ \\
\hline
$€ 728.54$ & Second payment's $F V$ \\
\hline
$€ 714.13$ & Third payment's $F V$ \\
\hline
$€ 700.00$ & Fourth payment's $F V$ \\
\hline
$€ 22,885.92$ & Total $F V$ \\
\hline
\end{tabular}
\end{center}

\section{LEARNING MODULE 2}
\section{Organizing, Visualizing, and Describing Data}
by Pamela Peterson Drake, PhD, CFA, and Jian Wu, PhD.

Pamela Peterson Drake, PhD, CFA, is at James Madison University (USA). Jian Wu, PhD, is at State Street (USA).

\section{LEARNING OUTCOME}
\begin{center}
\begin{tabular}{|c|c|}
\hline
Mastery & The candidate should be able to: \\
\hline
$\square$ & identify and compare data types \\
\hline
$\square$ & describe how data are organized for quantitative analysis \\
\hline
$\square$ & interpret frequency and related distributions \\
\hline
$\square$ & interpret a contingency table \\
\hline
$\square$ & $\begin{array}{l}\text { describe ways that data may be visualized and evaluate uses of } \\ \text { specific visualizations }\end{array}$ \\
\hline
$\square$ & describe how to select among visualization types \\
\hline
$\square$ & calculate and interpret measures of central tendency \\
\hline
$\square$ & $\begin{array}{l}\text { evaluate alternative definitions of mean to address an investment } \\ \text { problem }\end{array}$ \\
\hline
$\square$ & calculate quantiles and interpret related visualizations \\
\hline
$\square$ & calculate and interpret measures of dispersion \\
\hline
$\square$ & calculate and interpret target downside deviation \\
\hline
$\square$ & interpret skewness \\
\hline
$\square$ & interpret kurtosis \\
\hline
$\square$ & interpret correlation between two variables \\
\hline
\end{tabular}
\end{center}

\section{INTRODUCTION}
Data have always been a key input for securities analysis and investment management, but the acceleration in the availability and the quantity of data has also been driving the rapid evolution of the investment industry. With the rise of big data and machine learning techniques, investment practitioners are embracing an era featuring large volume, high velocity, and a wide variety of data-allowing them to explore and exploit this abundance of information for their investment strategies.

While this data-rich environment offers potentially tremendous opportunities for investors, turning data into useful information is not so straightforward. Organizing, cleaning, and analyzing data are crucial to the development of successful investment strategies; otherwise, we end up with "garbage in and garbage out" and failed investments. It is often said that $80 \%$ of an analyst's time is spent on finding, organizing, cleaning, and analyzing data, while just $20 \%$ of her/his time is taken up by model development. So, the importance of having a properly organized, cleansed, and well-analyzed dataset cannot be over-emphasized. With this essential requirement met, an appropriately executed data analysis can detect important relationships within data, uncover underlying structures, identify outliers, and extract potentially valuable insights. Utilizing both visual tools and quantitative methods, like the ones covered in this reading, is the first step in summarizing and understanding data that will be crucial inputs to an investment strategy.

This reading provides a foundation for understanding important concepts that are an indispensable part of the analytical tool kit needed by investment practitioners, from junior analysts to senior portfolio managers. These basic concepts pave the way for more sophisticated tools that will be developed as the quantitative methods topic unfolds and that are integral to gaining competencies in the investment management techniques and asset classes that are presented later in the CFA curriculum.

Section 2 covers core data types, including continuous and discrete numerical data, nominal and ordinal categorical data, and structured versus unstructured data. Organizing data into arrays and data tables and summarizing data in frequency distributions and contingency tables are discussed in Section 3. Section 4 introduces the important topic of data visualization using a range of charts and graphics to summarize, explore, and better understand data. Section 5 covers the key measures of central tendency, including several variants of mean that are especially useful in investments. Quantiles and their investment applications are the focus of Section 6. Key measures of dispersion are discussed in Section 7. The shape of data distributions-specifically, skewness and kurtosis-are covered in Sections 8 and 9, respectively. Section 10 provides a graphical introduction to covariance and correlation between two variables. The reading concludes with a Summary.

\section{DATA TYPES}
identify and compare data types

describe how data are organized for quantitative analysis

Data can be defined as a collection of numberpanel datas, characters, words, and text-as well as images, audio, and video-in a raw or organized format to represent facts or information. To choose the appropriate statistical methods for summarizing and analyzing data and to select suitable charts for visualizing data, we need to distinguish among different data types. We will discuss data types under three different perspectives of classifications: numerical versus categorical data; cross-sectional vs. time-series vs. panel data; and structured vs. unstructured data.

\section{Numerical versus Categorical Data}
From a statistical perspective, data can be classified into two basic groups: numerical data and categorical data.

\section{Numerical Data}
Numerical data are values that represent measured or counted quantities as a number and are also called quantitative data. Numerical (quantitative) data can be split into two types: continuous data and discrete data.

Continuous data are data that can be measured and can take on any numerical value in a specified range of values. For example, the future value of a lump-sum investment measures the amount of money to be received after a certain period of time bearing an interest rate. The future value could take on a range of values depending on the time period and interest rate. Another common example of continuous data is the price returns of a stock that measures price change over a given period in percentage terms.

Discrete data are numerical values that result from a counting process. So, practically speaking, the data are limited to a finite number of values. For example, the frequency of discrete compounding, $m$, counts the number of times that interest is accrued and paid out in a given year. The frequency could be monthly $(m=12)$, quarterly $(m=4)$, semi-yearly $(m=2)$, or yearly $(m=1)$.

\section{Categorical Data}
Categorical data (also called qualitative data) are values that describe a quality or characteristic of a group of observations and therefore can be used as labels to divide a dataset into groups to summarize and visualize. Usually they can take only a limited number of values that are mutually exclusive. Examples of categorical data for classifying companies include bankrupt vs. not bankrupt and dividends increased vs. no dividend action.

Nominal data are categorical values that are not amenable to being organized in a logical order. An example of nominal data is the classification of publicly listed stocks into 11 sectors, as shown in Exhibit 1, that are defined by the Global Industry Classification Standard (GICS). GICS, developed by Morgan Stanley Capital International (MSCI) and Standard \& Poor's (S\&P), is a four-tiered, hierarchical industry classification system consisting of 11 sectors, 24 industry groups, 69 industries, and 158 sub-industries. Each sector is defined by a unique text label, as shown in the column named "Sector."

Exhibit 1: Equity Sector Classification by GICS

Sector

(Text Label)

\begin{center}
\begin{tabular}{ll}
\hline
Energy & 10 \\
Materials & 15 \\
Industrials & 20 \\
Consumer Discretionary & 25 \\
Consumer Staples & 30 \\
Health Care & 35 \\
Financials & 40 \\
Information Technology & 45 \\
\end{tabular}
\end{center}

\begin{center}
\begin{tabular}{lc}
\hline
$\begin{array}{l}\text { Sector } \\ \text { (Text Label) }\end{array}$ & $\begin{array}{c}\text { Code } \\ \text { (Numerical Label) }\end{array}$ \\
\hline
Communication Services & 50 \\
Utilities & 55 \\
Real Estate & 60 \\
\hline
\end{tabular}
\end{center}

Source: S\&P Global Market Intelligence.

Text labels are a common format to represent nominal data, but nominal data can also be coded with numerical labels. As shown below, the column named "Code" contains a corresponding GICS code of each sector as a numerical value. However, the nominal data in numerical format do not indicate ranking, and any arithmetic operations on nominal data are not meaningful. In this example, the energy sector with the code 10 does not represent a lower or higher rank than the real estate sector with the code 60 . Often, financial models, such as regression models, require input data to be numerical; so, nominal data in the input dataset must be coded numerically before applying an algorithm (that is, a process for problem solving) for performing the analysis. This would be mainly to identify the category (here, sector) in the model.

Ordinal data are categorical values that can be logically ordered or ranked. For example, the Morningstar and Standard \& Poor's star ratings for investment funds are ordinal data in which one star represents a group of funds judged to have had relatively the worst performance, with two, three, four, and five stars representing groups with increasingly better performance or quality as evaluated by those firms.

Ordinal data may also involve numbers to identify categories. For example, in ranking growth-oriented investment funds based on their five-year cumulative returns, we might assign the number 1 to the top performing $10 \%$ of funds, the number 2 to next best performing $10 \%$ of funds, and so on; the number 10 represents the bottom performing $10 \%$ of funds. Despite the fact that categories represented by ordinal data can be ranked higher or lower compared to each other, they do not necessarily establish a numerical difference between each category. Importantly, such investment fund ranking tells us nothing about the difference in performance between funds ranked 1 and 2 compared with the difference in performance between funds ranked 3 and 4 or 9 and 10.

Having discussed different data types from a statistical perspective, it is important to note that at first glance, identifying data types may seem straightforward. In some situations, where categorical data are coded in numerical format, they should be distinguished from numerical data. A sound rule of thumb: Meaningful arithmetic operations can be performed on numerical data but not on categorical data.

\section{EXAMPLE 1}
\section{Identifying Data Types (I)}
Identify the data type for each of the following kinds of investment-related information:

\begin{enumerate}
  \item Number of coupon payments for a corporate bond. As background, a corporate bond is a contractual obligation between an issuing corporation (i.e., borrower) and bondholders (i.e., lenders) in which the issuer agrees to pay interest-in the form of fixed coupon payments-on specified dates, typi- cally semi-annually, over the life of the bond (i.e., to its maturity date) and to repay principal (i.e., the amount borrowed) at maturity.
\end{enumerate}

\section{Solution to 1}
Number of coupon payments are discrete data. For example, a newly-issued 5-year corporate bond paying interest semi-annually (quarterly) will make 10 (20) coupon payments during its life. In this case, coupon payments are limited to a finite number of values; so, they are discrete.

\begin{enumerate}
  \setcounter{enumi}{1}
  \item Cash dividends per share paid by a public company. Note that cash dividends are a distribution paid to shareholders based on the number of shares owned.
\end{enumerate}

\section{Solution to 2}
Cash dividends per share are continuous data since they can take on any non-negative values.

\begin{enumerate}
  \setcounter{enumi}{2}
  \item Credit ratings for corporate bond issues. As background, credit ratings gauge the bond issuer's ability to meet the promised payments on the bond. Bond rating agencies typically assign bond issues to discrete categories that are in descending order of credit quality (i.e., increasing probability of non-payment or default).
\end{enumerate}

\section{Solution to 3}
Credit ratings are ordinal data. A rating places a bond issue in a category, and the categories are ordered with respect to the expected probability of default. But arithmetic operations cannot be done on credit ratings, and the difference in the expected probability of default between categories of highly rated bonds, for example, is not necessarily equal to that between categories of lowly rated bonds.

\begin{enumerate}
  \setcounter{enumi}{3}
  \item Hedge fund classification types. Note that hedge funds are investment vehicles that are relatively unconstrained in their use of debt, derivatives, and long and short investment strategies. Hedge fund classification types group hedge funds by the kind of investment strategy they pursue.
\end{enumerate}

\section{Solution to 4}
Hedge fund classification types are nominal data. Each type groups together hedge funds with similar investment strategies. In contrast to credit ratings for bonds, however, hedge fund classification schemes do not involve a ranking. Thus, such classification schemes are not ordinal data.

\section{Cross-Sectional versus Time-Series versus Panel Data}
Another data classification standard is based on how data are collected, and it categorizes data into three types: cross-sectional, time series, and panel.

Prior to the description of the data types, we need to explain two data-related terminologies: variable and observation. A variable is a characteristic or quantity that can be measured, counted, or categorized and is subject to change. A variable can also be called a field, an attribute, or a feature. For example, stock price, market capitalization, dividend and dividend yield, earnings per share (EPS), and price-to-earnings ratio $(\mathrm{P} / \mathrm{E})$ are basic data variables for the financial analysis of a public company. An observation is the value of a specific variable collected at a point in time or over a specified period of time. For example, last year DEF, Inc. recorded EPS of $\$ 7.50$. This value represented a 15\% annual increase.

Cross-sectional data are a list of the observations of a specific variable from multiple observational units at a given point in time. The observational units can be individuals, groups, companies, trading markets, regions, etc. For example, January inflation rates (i.e., the variable) for each of the euro-area countries (i.e., the observational units) in the European Union for a given year constitute cross-sectional data.

Time-series data are a sequence of observations for a single observational unit of a specific variable collected over time and at discrete and typically equally spaced intervals of time, such as daily, weekly, monthly, annually, or quarterly. For example, the daily closing prices (i.e., the variable) of a particular stock recorded for a given month constitute time-series data.

Panel data are a mix of time-series and cross-sectional data that are frequently used in financial analysis and modeling. Panel data consist of observations through time on one or more variables for multiple observational units. The observations in panel data are usually organized in a matrix format called a data table. Exhibit 2 is an example of panel data showing quarterly earnings per share (i.e., the variable) for three companies (i.e., the observational units) in a given year by quarter. Each column is a time series of data that represents the quarterly EPS observations from $\mathrm{Q} 1$ to $\mathrm{Q} 4$ of a specific company, and each row is cross-sectional data that represent the EPS of all three companies of a particular quarter.

\section{Exhibit 2: Earnings per Share in Euros of Three Eurozone Companies in a}
 Given Year\begin{center}
\begin{tabular}{lccc}
\hline
Time Period & Company A & Company B & Company C \\
\hline
Q1 & 13.53 & 0.84 & -0.34 \\
Q2 & 4.36 & 0.96 & 0.08 \\
Q3 & 13.16 & 0.79 & -2.72 \\
Q4 & 12.95 & 0.19 & 0.09 \\
\hline
\end{tabular}
\end{center}

\section{Structured versus Unstructured Data}
Categorizing data into structured and unstructured types is based on whether or not the data are in a highly organized form.

Structured data are highly organized in a pre-defined manner, usually with repeating patterns. The typical forms of structured data are one-dimensional arrays, such as a time series of a single variable, or two-dimensional data tables, where each column represents a variable or an observation unit and each row contains a set of values for the same columns. Structured data are relatively easy to enter, store, query, and analyze without much manual processing. Typical examples of structured company financial data are:

\begin{itemize}
  \item Market data: data issued by stock exchanges, such as intra-day and daily closing stock prices and trading volumes.

  \item Fundamental data: data contained in financial statements, such as earnings per share, price to earnings ratio, dividend yield, and return on equity.

  \item Analytical data: data derived from analytics, such as cash flow projections or forecasted earnings growth. Unstructured data, in contrast, are data that do not follow any conventionally organized forms. Some common types of unstructured data are text-such as financial news, posts in social media, and company filings with regulators-and also audio/ video, such as managements' earnings calls and presentations to analysts.

\end{itemize}

Unstructured data are a relatively new classification driven by the rise of alternative data (i.e., data generated from unconventional sources, like electronic devices, social media, sensor networks, and satellites, but also by companies in the normal course of business) and its growing adoption in the financial industry. Unstructured data are typically alternative data as they are usually collected from unconventional sources. By indicating the source from which the data are generated, such data can be classified into three groups:

\begin{itemize}
  \item Produced by individuals (i.e., via social media posts, web searches, etc.);

  \item Generated by business processes (i.e., via credit card transactions, corporate regulatory filings, etc.); and

  \item Generated by sensors (i.e., via satellite imagery, foot traffic by mobile devices, etc.).

\end{itemize}

Unstructured data may offer new market insights not normally contained in data from traditional sources and may provide potential sources of returns for investment processes. Unlike structured data, however, utilizing unstructured data in investment analysis is challenging. Typically, financial models are able to take only structured data as inputs; therefore, unstructured data must first be transformed into structured data that models can process.

Exhibit 3 shows an excerpt from Form 10-Q (Quarterly Report) filed by Company XYZ with the US Securities and Exchange Commission (SEC) for the fiscal quarter ended 31 March 20XX. The form is an unstructured mix of text and tables, so it cannot be directly used by computers as input to financial models. The SEC has utilized eXtensible Business Reporting Language (XBRL) to structure such data. The data extracted from the XBRL submission can be organized into five tab-delimited TXT format files that contain information about the submission, including taxonomy tags (i.e., financial statement items), dates, units of measure (uom), values (i.e., for the tag items), and more-making it readable by computer. Exhibit 4 shows an excerpt from one of the now structured data tables downloaded from the SEC's EDGAR (Electronic Data Gathering, Analysis, and Retrieval) database.

Exhibit 3: Excerpt from 10-Q of Company XYZ for Fiscal Quarter Ended 31 March 20XX

Company XYZ

Form 10-Q

Fiscal Quarter Ended 31 March 20XX

Table of Contents

Part I

\begin{center}
\begin{tabular}{llc}
\hline
 &  & Page \\
Item 1 & Financial Statements & 1 \\
Item 2 & Management's Discussion and Analysis of Financial & 21 \\
 & Condition and Results of Operations & 32 \\
Item 3 & Quantitative and Qualitative Disclosures About Market Risk & 32 \\
Item 4 & Controls and Procedures & 33 \\
\hline
Part II &  & 33 \\
\hline
\end{tabular}
\end{center}

\begin{center}
\includegraphics[max width=\textwidth]{2023_05_04_cff39ee44f77d6514e1bg-076}
\end{center}

Source: EDGAR.

Exhibit 4: Structured Data Extracted from Form 10-Q of Company XYZ for Fiscal Quarter Ended 31 March $20 X X$

\begin{center}
\begin{tabular}{|c|c|c|c|c|}
\hline
adsh & $\operatorname{tag}$ & ddate & uom & value \\
\hline
$0000320193-19-000066$ & RevenueFromContractWithCustomerExcludingAssessedTax & $20 \times X 0331$ & USD & $\$ 58,015,000,000$ \\
\hline
$0000320193-19-000066$ & GrossProfit & $20 \times X 0331$ & USD & $\$ 21,821,000,000$ \\
\hline
$0000320193-19-000066$ & OperatingExpenses & $20 \times X 0331$ & USD & $\$ 8,406,000,000$ \\
\hline
$0000320193-19-000066$ & OperatingIncomeLoss & $20 \times X 0331$ & USD & $\$ 13,415,000,000$ \\
\hline
$0000320193-19-000066$ & NetIncomeLoss & $20 \times X 0331$ & USD & $\$ 11,561,000,000$ \\
\hline
\end{tabular}
\end{center}

Source: EDGAR.

\section{EXAMPLE 2}
\section{Identifying Data Types (II)}
\begin{enumerate}
  \item Which of the following is most likely to be structured data?
\end{enumerate}

A. Social media posts where consumers are commenting on what they think of a company's new product.

B. Daily closing prices during the past month for all companies listed on Japan's Nikkei 225 stock index.

C. Audio and video of a CFO explaining her company's latest earnings announcement to securities analysts.

\section{Solution to 1}
B is correct as daily closing prices constitute structured data. A is incorrect as social media posts are unstructured data. $\mathrm{C}$ is incorrect as audio and video are unstructured data.

\begin{enumerate}
  \setcounter{enumi}{1}
  \item Which of the following statements describing panel data is most accurate?
\end{enumerate}

A. It is a sequence of observations for a single observational unit of a specific variable collected over time at discrete and equally spaced intervals.

B. It is a list of observations of a specific variable from multiple observational units at a given point in time.

C. It is a mix of time-series and cross-sectional data that are frequently used in financial analysis and modeling.

\section{Solution to 2}
$\mathrm{C}$ is correct as it most accurately describes panel data. A is incorrect as it describes time-series data. $\mathrm{B}$ is incorrect as it describes cross-sectional data.

\begin{enumerate}
  \setcounter{enumi}{2}
  \item Which of the following data series is least likely to be sortable by values?
\end{enumerate}

A. Daily trading volumes for stocks listed on the Shanghai Stock Exchange.

B. EPS for a given year for technology companies included in the S\&P 500 Index.

C. Dates of first default on bond payments for a group of bankrupt European manufacturing companies.

\section{Solution to 3}
$\mathrm{C}$ is correct as dates are ordinal data that can be sorted by chronological order but not by value. A and $\mathrm{B}$ are incorrect as both daily trading volumes and earnings per share (EPS) are numerical data, so they can be sorted by values.

\begin{enumerate}
  \setcounter{enumi}{3}
  \item Which of the following best describes a time series?
\end{enumerate}

A. Daily stock prices of the XYZ stock over a 60-month period.

B. Returns on four-star rated Morningstar investment funds at the end of the most recent month.

C. Stock prices for all stocks in the FTSE100 on 31 December of the most recent calendar year.

\section{Solution to 4}
A is correct since a time series is a sequence of observations of a specific variable (XYZ stock price) collected over time (60 months) and at discrete intervals of time (daily). B and C are both incorrect as they are cross-sectional data.

\section{Data Summarization}
Given the wide variety of possible formats of raw data, which are data available in their original form as collected, such data typically cannot be used by humans or computers to directly extract information and insights. Organizing data into a one-dimensional array or a two-dimensional array is typically the first step in data analytics and modeling. In this section, we will illustrate the construction of these typical data organization formats. We will also introduce two useful tools that can efficiently summarize one-variable and two-variable data: frequency distributions and contingency tables, respectively. Both of them can give us a quick snapshot of the data and allow us to find patterns in the data and associations between variables.

\section{ORGANIZING DATA FOR QUANTITATIVE ANALYSIS}
$\square \quad$ describe how data are organized for quantitative analysis

Quantitative analysis and modeling typically require input data to be in a clean and formatted form, so raw data are usually not suitable for use directly by analysts. Depending upon the number of variables, raw data can be organized into two typical formats for quantitative analysis: one-dimensional arrays and two-dimensional rectangular arrays.

A one-dimensional array is the simplest format for representing a collection of data of the same data type, so it is suitable for representing a single variable. Exhibit 5 is an example of a one-dimensional array that shows the closing price for the first 10 trading days for $\mathrm{ABC}$ Inc. stock after the company went public. Closing prices are time-series data collected at daily intervals, so it is natural to organize them into a time-ordered sequence. The time-series format also facilitates future data updates to the existing dataset. In this case, closing prices for future trading sessions can be easily added to the end of the array with no alteration of previously formatted data.

More importantly, in contrast to compiling the data randomly in an unorganized manner, organizing such data by its time-series nature preserves valuable information beyond the basic descriptive statistics that summarize central tendency and spread variation in the data's distribution. For example, by simply plotting the data against time, we can learn whether the data demonstrate any increasing or decreasing trends over time or whether the time series repeats certain patterns in a systematic way over time. Exhibit 5: One-Dimensional Array: Daily Closing Price of ABC Inc. Stock

\begin{center}
\begin{tabular}{lc}
\hline
Observation by Day & Stock Price (\$) \\
\hline
1 & 57.21 \\
2 & 58.26 \\
3 & 58.64 \\
4 & 56.19 \\
5 & 54.78 \\
6 & 54.26 \\
7 & 56.88 \\
8 & 54.74 \\
9 & 52.42 \\
10 & 50.14 \\
\hline
\end{tabular}
\end{center}

A two-dimensional rectangular array (also called a data table) is one of the most popular forms for organizing data for processing by computers or for presenting data visually for consumption by humans. Similar to the structure in an Excel spreadsheet, a data table is comprised of columns and rows to hold multiple variables and multiple observations, respectively. When a data table is used to organize the data of one single observational unit (i.e., a single company), each column represents a different variable (feature or attribute) of that observational unit, and each row holds an observation for the different variables; successive rows represent the observations for successive time periods. In other words, observations of each variable are a time-series sequence that is sorted in either ascending or descending time order. Consequently, observations of different variables must be sorted and aligned to the same time scale. Example 3 shows how to organize a raw dataset for a company collected online into a machine-readable data table.

\section{EXAMPLE 3}
\section{Organizing a Company's Raw Data into a Data Table}
\begin{enumerate}
  \item Suppose you are conducting a valuation analysis of $\mathrm{ABC}$ Inc., which has been listed on the stock exchange for two years. The metrics to be used in your valuation include revenue, earnings per share (EPS), and dividends paid per share (DPS). You have retrieved the last two years of ABC's quarterly data from the exchange's website, which is shown in Exhibit 6. The data available online are pre-organized into a tabular format, where each column represents a fiscal year and each row represents a particular quarter with values of the three measures clustered together.
\end{enumerate}

\section{Exhibit 6: Metrics of ABC Inc. Retrieved Online}
\begin{center}
\begin{tabular}{lcc}
\hline
Fiscal Quarter & $\begin{array}{c}\text { Year 1 } \\ \text { (Fiscal Year) }\end{array}$ & $\begin{array}{c}\text { Year 2 } \\ \text { (Fiscal Year) }\end{array}$ \\
\hline
March &  &  \\
Revenue & $\$ 3,784(\mathrm{M})$ & $\$ 4,097(\mathrm{M})$ \\
EPS & 1.37 & -0.34 \\
\end{tabular}
\end{center}

\begin{center}
\begin{tabular}{lcc}
\hline
Fiscal Quarter & $\begin{array}{c}\text { Year 1 } \\ \text { (Fiscal Year) }\end{array}$ & $\begin{array}{c}\text { Year 2 } \\ \text { (Fiscal Year) }\end{array}$ \\
\hline
DPS & N/A & N/A \\
June &  &  \\
Revenue & $\$ 4,236(\mathrm{M})$ & $\$ 5,905(\mathrm{M})$ \\
EPS & 1.78 & 3.89 \\
DPS & N/A & 0.25 \\
September &  &  \\
Revenue & $\$ 4,187(\mathrm{M})$ & $\$ 4,997(\mathrm{M})$ \\
EPS & -3.38 & -2.88 \\
DPS & N/A & 0.25 \\
December &  &  \\
Revenue & $\$ 3,889(\mathrm{M})$ & $\$ 4,389(\mathrm{M})$ \\
EPS & -8.66 & -3.98 \\
DPS & N/A & 0.25 \\
\hline
\end{tabular}
\end{center}

Use the data to construct a two-dimensional rectangular array (i.e., data table) with the columns representing the metrics for valuation and the observations arranged in a time-series sequence.

\section{Solution:}
To construct a two-dimensional rectangular array, we first need to determine the data table structure. The columns have been specified to represent the three valuation metrics (i.e., variables): revenue, EPS and DPS. The rows should be the observations for each variable in a time ordered sequence. In this example, the data for the valuation measures will be organized in the same quarterly intervals as the raw data retrieved online, starting from $\mathrm{Q} 1$ Year 1 to Q4 Year 2. Then, the observations from the original table can be placed accordingly into the data table by variable name and by filing quarter. Exhibit 7 shows the raw data reorganized in the two-dimensional rectangular array (by date and associated valuation metric), which can now be used in financial analysis and is readable by a computer.

It is worth pointing out that in case of missing values while organizing data, how to handle them depends largely on why the data are missing. In this example, dividends (DPS) in the first five quarters are missing because ABC Inc. did not authorize (and pay) any dividends. So, filling the dividend column with zeros is appropriate. If revenue, EPS, and DPS of a given quarter are missing due to particular data source issues, however, these missing values cannot be simply replaced with zeros; this action would result in incorrect interpretation. Instead, the missing values might be replaced with the latest available data or with interpolated values, depending on how the data will be consumed or modeled.

Exhibit 7: Data Table for ABC Inc.

\begin{center}
\begin{tabular}{lccc}
\hline
 & Revenue (\$ Million) & EPS (\$) & DPS (\$) \\
\hline
Q1 Year 1 & 3,784 & 1.37 & 0 \\
Q2 Year 1 & 4,236 & 1.78 & 0 \\
\end{tabular}
\end{center}

\begin{center}
\begin{tabular}{lccc}
\hline
 & Revenue (\$ Million) & EPS (\$) & DPS (\$) \\
\hline
Q3 Year 1 & 4,187 & -3.38 & 0 \\
Q4 Year 1 & 3,889 & -8.66 & 0 \\
Q1 Year 2 & 4,097 & -0.34 & 0 \\
Q2 Year 2 & 5,905 & 3.89 & 0.25 \\
Q3 Year 2 & 4,997 & -2.88 & 0.25 \\
Q4 Year 2 & 4,389 & -3.98 & 0.25 \\
\hline
\end{tabular}
\end{center}

\section{SUMMARIZING DATA USING FREQUENCY DISTRIBUTIONS}
interpret frequency and related distributions

We now discuss various tabular formats for describing data based on the count of observations. These tables are a necessary step toward building a true visualization of a dataset. Later, we shall see how bar charts, tree-maps, and heat maps, among other graphic tools, are used to visualize important properties of a dataset.

A frequency distribution (also called a one-way table) is a tabular display of data constructed either by counting the observations of a variable by distinct values or groups or by tallying the values of a numerical variable into a set of numerically ordered bins. It is an important tool for initially summarizing data by groups or bins for easier interpretation.

Constructing a frequency distribution of a categorical variable is relatively straightforward and can be stated in the following two basic steps:

\begin{enumerate}
  \item Count the number of observations for each unique value of the variable.

  \item Construct a table listing each unique value and the corresponding counts, and then sort the records by number of counts in descending or ascending order to facilitate the display.

\end{enumerate}

Exhibit 8 shows a frequency distribution of a portfolio's stock holdings by sectors (the variables), which are defined by GICS. The portfolio contains a total of 479 stocks that have been individually classified into 11 GICS sectors (first column). The stocks are counted by sector and are summarized in the second column, absolute frequency. The absolute frequency, or simply the raw frequency, is the actual number of observations counted for each unique value of the variable (i.e., each sector). Often it is desirable to express the frequencies in terms of percentages, so we also show the relative frequency (in the third column), which is calculated as the absolute frequency of each unique value of the variable divided by the total number of observations. The relative frequency provides a normalized measure of the distribution of the data, allowing comparisons between datasets with different numbers of total observations.

\section{Exhibit 8: Frequency Distribution for a Portfolio by Sector}
\begin{center}
\begin{tabular}{lcc}
\hline
$\begin{array}{l}\text { Sector } \\ \text { (Variable) }\end{array}$ & $\begin{array}{c}\text { Absolute } \\ \text { Frequency }\end{array}$ & $\begin{array}{c}\text { Relative } \\ \text { Frequency }\end{array}$ \\
\hline
Industrials & 73 & $15.2 \%$ \\
Information Technology & 69 & $14.4 \%$ \\
Financials & 67 & $14.0 \%$ \\
Consumer Discretionary & 62 & $12.9 \%$ \\
Health Care & 54 & $11.3 \%$ \\
Consumer Staples & 33 & $6.9 \%$ \\
Real Estate & 30 & $6.3 \%$ \\
Energy & 29 & $6.1 \%$ \\
Utilities & 26 & $5.4 \%$ \\
Materials & 26 & $5.4 \%$ \\
Communication Services & 10 & $2.1 \%$ \\
Total & 479 & $\mathbf{1 0 0 . 0 \%}$ \\
\hline
\end{tabular}
\end{center}

A frequency distribution table provides a snapshot of the data, and it facilitates finding patterns. Examining the distribution of absolute frequency in Exhibit 8, we see that the largest number of stocks (73), accounting for $15.2 \%$ of the stocks in the portfolio, are held in companies in the industrials sector. The sector with the least number of stocks (10) is communication services, which represents just $2.1 \%$ of the stocks in the portfolio.

It is also easy to see that the top four sectors (i.e., industrials, information technology, financials, and consumer discretionary) have very similar relative frequencies, between $15.2 \%$ and $12.9 \%$. Similar relative frequencies, between $6.9 \%$ and $5.4 \%$, are also seen among several other sectors. Note that the absolute frequencies add up to the total number of stocks in the portfolio (479), and the sum of the relative frequencies should be equal to $100 \%$.

Frequency distributions also help in the analysis of large amounts of numerical data. The procedure for summarizing numerical data is a bit more involved than that for summarizing categorical data because it requires creating non-overlapping bins (also called intervals or buckets) and then counting the observations falling into each bin. One procedure for constructing a frequency distribution for numerical data can be stated as follows:

\begin{enumerate}
  \item Sort the data in ascending order.

  \item Calculate the range of the data, defined as Range = Maximum value Minimum value.

  \item Decide on the number of bins $(k)$ in the frequency distribution.

  \item Determine bin width as Range/ $k$.

  \item Determine the first bin by adding the bin width to the minimum value. Then, determine the remaining bins by successively adding the bin width to the prior bin's end point and stopping after reaching a bin that includes the maximum value.

  \item Determine the number of observations falling into each bin by counting the number of observations whose values are equal to or exceed the bin minimum value yet are less than the bin's maximum value. The exception is in the last bin, where the maximum value is equal to the last bin's maximum, and therefore, the observation with the maximum value is included in this bin's count.

  \item Construct a table of the bins listed from smallest to largest that shows the number of observations falling into each bin.

\end{enumerate}

In Step 4, when rounding the bin width, round up (rather than down) to ensure that the final bin includes the maximum value of the data.

These seven steps are basic guidelines for constructing frequency distributions. In practice, however, we may want to refine the above basic procedure. For example, we may want the bins to begin and end with whole numbers for ease of interpretation. Another practical refinement that promotes interpretation is to start the first bin at the nearest whole number below the minimum value.

As this procedure implies, a frequency distribution groups data into a set of bins, where each bin is defined by a unique set of values (i.e., beginning and ending points). Each observation falls into only one bin, and the total number of bins covers all the values represented in the data. The frequency distribution is the list of the bins together with the corresponding measures of frequency.

To illustrate the basic procedure, suppose we have 12 observations sorted in ascending order (Step 1):

$-4.57,-4.04,-1.64,0.28,1.34,2.35,2.38,4.28,4.42,4.68,7.16$, and 11.43.

The minimum observation is -4.57 , and the maximum observation is +11.43 . So, the range is $+11.43-(-4.57)=16$ (Step 2) .

If we set $k=4($ Step 3$)$, then the bin width is $16 / 4=4($ Step 4$)$.

Exhibit 9 shows the repeated addition of the bin width of 4 to determine the endpoint for each of the bins (Step 5).

\section{Exhibit 9: Determining Endpoints of the Bins}
\begin{center}
\begin{tabular}{ccccc}
\hline
-4.57 & + & 4.0 & $=$ & -0.57 \\
-0.57 & + & 4.0 & $=$ & 3.43 \\
3.43 & + & 4.0 & $=$ & 7.43 \\
7.40 & + & 4.0 & $=$ & 11.43 \\
\hline
\end{tabular}
\end{center}

Thus, the bins are [ -4.57 to -0.57$)$, [-0.57 to 3.43), [3.43 to 7.43 ), and [7.43 to 11.43], where the notation $[-4.57$ to -0.57$)$ indicates $-4.57 \leq$ observation $<-0.57$. The parentheses indicate that the endpoints are not included in the bins, and the square brackets indicate that the beginning points and the last endpoint are included in the bin. Exhibit 10 summarizes Steps 5 through 7.

\section{Exhibit 10: Frequency Distribution}
\begin{center}
\begin{tabular}{|c|c|c|c|c|}
\hline
Bi &  &  &  & Absolute Frequency \\
\hline
A & -4.57 & $\leq$ observation $<$ & -0.57 & 3 \\
\hline
B & -0.57 & $\leq$ observation $<$ & 3.43 & 4 \\
\hline
C & 3.43 & $\leq$ observation $<$ & 7.43 & 4 \\
\hline
D & 7.43 & $\leq$ observation $\leq$ & 11.43 & 1 \\
\hline
\end{tabular}
\end{center}

Note that the bins do not overlap, so each observation can be placed uniquely into one bin, and the last bin includes the maximum value. We turn to these issues in discussing the construction of frequency distributions for daily returns of the fictitious Euro-Asia-Africa (EAA) Equity Index. The dataset of daily returns of the EAA Equity Index spans a five-year period and consists of 1,258 observations with a minimum value of $-4.1 \%$ and a maximum value of $5.0 \%$. Thus, the range of the data is $5 \%-(-4.1 \%)=9.1 \%$, approximately. [The mean daily return-mean as a measure of central tendency will be discussed shortly-is 0.04\%.]

The decision on the number of bins $(k)$ into which we should group the observations often involves inspecting the data and exercising judgment. How much detail should we include? If we use too few bins, we will summarize too much and may lose pertinent characteristics. Conversely, if we use too many bins, we may not summarize enough and may introduce unnecessary noise.

We can establish an appropriate value for $k$ by evaluating the usefulness of the resulting bin width. A large number of empty bins may indicate that we are attempting to over-organize the data to present too much detail. Starting with a relatively small bin width, we can see whether or not the bins are mostly empty and whether or not the value of $k$ associated with that bin width is too large. If the bins are mostly empty, implying that $k$ is too large, we can consider increasingly larger bins (i.e., smaller values of $k$ ) until we have a frequency distribution that effectively summarizes the distribution.

Suppose that for ease of interpretation we want to use a bin width stated in whole rather than fractional percentages. In the case of the daily EAA Equity Index returns, a $1 \%$ bin width would be associated with $9.1 / 1=9.1$ bins, which can be rounded up to $k=10$ bins. That number of bins will cover a range of $1 \% \times 10=10 \%$. By constructing the frequency distribution in this manner, we will also have bins that end and begin at a value of $0 \%$, thereby allowing us to count the negative and positive returns in the data. Without too much work, we have found an effective way to summarize the data.

Exhibit 11 shows the frequency distribution for the daily returns of the EAA Equity Index using return bins of $1 \%$, where the first bin includes returns from $-5.0 \%$ to $-4.0 \%$ (exclusive, meaning $<-4 \%$ ) and the last bin includes daily returns from $4.0 \%$ to $5.0 \%$ (inclusive, meaning $\leq 5 \%$ ). Note that to facilitate interpretation, the first bin starts at the nearest whole number below the minimum value (so, at $-5.0 \%$ ).

Exhibit 11 includes two other useful ways to present the data (which can be computed in a straightforward manner once we have established the absolute and relative frequency distributions): the cumulative absolute frequency and the cumulative relative frequency. The cumulative absolute frequency cumulates (meaning, adds up) the absolute frequencies as we move from the first bin to the last bin. Similarly, the cumulative relative frequency is a sequence of partial sums of the relative frequencies. For the last bin, the cumulative absolute frequency will equal the number observations in the dataset $(1,258)$, and the cumulative relative frequency will equal $100 \%$.

\section{Exhibit 11: Frequency Distribution for Daily Returns of EAA Equity Index}
\begin{center}
\begin{tabular}{lcccc}
\hline
Return & $\begin{array}{c}\text { Absolute } \\ \text { Bin } \\ \text { Frequency }\end{array}$ & $\begin{array}{c}\text { Relative } \\ \text { Frequency } \\ (\%)\end{array}$ & $\begin{array}{c}\text { Cumulative } \\ \text { Absolute } \\ \text { Frequency }\end{array}$ & $\begin{array}{c}\text { Cumulative } \\ \text { Relative } \\ \text { Frequency (\%) }\end{array}$ \\
\hline
-5.0 to -4.0 & 1 & 0.08 & 1 & 0.08 \\
-4.0 to -3.0 & 7 & 0.56 & 31 & 0.64 \\
-3.0 to -2.0 & 23 & 1.83 & 108 & 2.46 \\
-2.0 to -1.0 & 77 & 6.12 & 578 & 8.59 \\
-1.0 to 0.0 & 470 & 37.36 & 1,133 & 95.95 \\
0.0 to 1.0 & 555 & 44.12 & 1,243 & 98.81 \\
\end{tabular}
\end{center}

\begin{center}
\begin{tabular}{lcccc}
\hline
$\begin{array}{l}\text { Return } \\ \text { Bin } \\ (\%)\end{array}$ & $\begin{array}{c}\text { Absolute } \\ \text { Frequency }\end{array}$ & $\begin{array}{c}\text { Relative } \\ \text { Frequency } \\ (\%)\end{array}$ & $\begin{array}{c}\text { Cumulative } \\ \text { Absolute } \\ \text { Frequency }\end{array}$ & $\begin{array}{c}\text { Cumulative } \\ \text { Relative } \\ \text { Frequency (\%) }\end{array}$ \\
\hline
2.0 to 3.0 & 13 & 1.03 & 1,256 & 99.84 \\
3.0 to 4.0 & 0.08 & 1,257 & 99.92 &  \\
4.0 to 5.0 & 1 & 0.08 & 1,258 & 100.00 \\
\hline
\end{tabular}
\end{center}

As Exhibit 11 shows, the absolute frequencies vary widely, ranging from 1 to 555 . The bin encompassing returns between $0 \%$ and $1 \%$ has the most observations (555), and the corresponding relative frequency tells us these observations account for $44.12 \%$ of the total number of observations. The frequency distribution gives us a sense of not only where most of the observations lie but also whether the distribution is evenly spread. It is easy to see that the vast majority of observations $(37.36 \%+44.12 \%=$ $81.48 \%$ ) lie in the middle two bins spanning $-1 \%$ to $1 \%$. We can also see that not many observations are greater than $3 \%$ or less than $-4 \%$. Moreover, as there are bins with $0 \%$ as ending or beginning points, we are able to count positive and negative returns in the data. Looking at the cumulative relative frequency in the last column, we see that the bin of $-1 \%$ to $0 \%$ shows a cumulative relative frequency of $45.95 \%$. This indicates that $45.95 \%$ of the observations lie below the daily return of $0 \%$ and that $54.05 \%$ of the observations are positive daily returns.

It is worth noting that other than being summarized in tables, frequency distributions also can be effectively represented in visuals, which will be discussed shortly in the section on data visualization.

\section{EXAMPLE 4}
\section{Constructing a Frequency Distribution of Country Index Returns}
\begin{enumerate}
  \item Suppose we have the annual equity index returns of a given year for 18 different countries, as shown in Exhibit 12, and we are asked to summarize the data.
\end{enumerate}

\section{Exhibit 12: Annual Equity Index Returns for 18 Countries}
\begin{center}
\begin{tabular}{ll}
\hline
Market & Index Return (\%) \\
\hline
Country A & 7.7 \\
Country B & 8.5 \\
Country C & 9.1 \\
Country D & 5.5 \\
Country E & 7.1 \\
Country F & 9.9 \\
Country G & 6.2 \\
Country H & 6.8 \\
Country I & 7.5 \\
Country J & 8.9 \\
Country K & 7.4 \\
Country L & 8.6 \\
Country M & 9.6 \\
\end{tabular}
\end{center}

\begin{center}
\begin{tabular}{lc}
\hline
Market & Index Return (\%) \\
\hline
Country N & 7.7 \\
Country O & 6.8 \\
Country P & 6.1 \\
Country Q & 8.8 \\
Country R & 7.9 \\
\hline
\end{tabular}
\end{center}

Construct a frequency distribution table from these data and state some key findings from the summarized data.

\section{Solution:}
The first step in constructing a frequency distribution table is to sort the return data in ascending order:

\begin{center}
\begin{tabular}{ll}
\hline
Market & Index Return (\%) \\
\hline
Country D & 5.5 \\
Country P & 6.1 \\
Country G & 6.2 \\
Country H & 6.8 \\
Country O & 6.8 \\
Country E & 7.1 \\
Country K & 7.4 \\
Country I & 7.5 \\
Country A & 7.7 \\
Country N & 7.7 \\
Country R & 7.9 \\
Country B & 8.5 \\
Country L & 8.6 \\
Country Q & 8.8 \\
Country J & 8.9 \\
Country C & 9.9 \\
Country M & 9.6 \\
Country F & 9.9 \\
\hline
\end{tabular}
\end{center}

The second step is to calculate the range of the data, which is $9.9 \%-5.5 \%=$ $4.4 \%$.

The third step is to decide on the number of bins. Here, we will use $k=5$. The fourth step is to determine the bin width. Here, it is $4.4 \% / 5=0.88 \%$, which we will round up to $1.0 \%$.

The fifth step is to determine the bins, which are as follows:

$$
\begin{aligned}
& 5.0 \%+1.0 \%=6.0 \% \\
& 6.0 \%+1.0 \%=7.0 \% \\
& 7.0 \%+1.0 \%=8.0 \% \\
& 8.0 \%+1.0 \%=9.0 \% \\
& 9.0 \%+1.0 \%=10.0 \%
\end{aligned}
$$

For ease of interpretation, the first bin is set to begin with the nearest whole number $(5.0 \%)$ below the minimum value $(5.5 \%)$ of the data series. The sixth step requires counting the return observations falling into each bin, and the seventh (last) step is use these results to construct the final frequency distribution table.

Exhibit 13 presents the frequency distribution table, which summarizes the data in Exhibit 12 into five bins spanning $5 \%$ to $10 \%$. Note that with 18 countries, the relative frequency for one observation is calculated as $1 / 18=$ $5.56 \%$.

\section{Exhibit 13: Frequency Distribution of Equity Index Returns}
\begin{center}
\begin{tabular}{lcccc}
\hline
Return & $\begin{array}{c}\text { Absolute } \\ \text { Frequency }\end{array}$ & $\begin{array}{c}\text { Relative } \\ \text { Frequency (\%) }\end{array}$ & $\begin{array}{c}\text { Cumulative } \\ \text { Absolute } \\ \text { Frequency }\end{array}$ & $\begin{array}{c}\text { Cumulative } \\ \text { Relative } \\ \text { Frequency (\%) }\end{array}$ \\
\hline
5.0 to 6.0 & 1 & 5.56 & 1 & 5.56 \\
6.0 to 7.0 & 4 & 22.22 & 5 & 27.78 \\
7.0 to 8.0 & 6 & 33.33 & 11 & 61.11 \\
8.0 to 9.0 & 4 & 22.22 & 15 & 83.33 \\
9.0 to 10.0 & 3 & 16.67 &  & 100.00 \\
\hline
\end{tabular}
\end{center}

As Exhibit 13 shows, there is substantial variation in these equity index returns. One-third of the observations fall in the 7.0 to $8.0 \%$ bin, making it the bin with the most observations. Both the 6.0 to $7.0 \%$ bin and the 8.0 to $9.0 \%$ bin hold four observations each, accounting for $22.22 \%$ of the total number of the observations, respectively. The two remaining bins have fewer observations, one or three observations, respectively.

\section{SUMMARIZING DATA USING A CONTINGENCY TABLE}
interpret a contingency table

We have shown that the frequency distribution table is a powerful tool to summarize data for one variable. How can we summarize data for two variables simultaneously? A contingency table provides a solution to this question.

A contingency table is a tabular format that displays the frequency distributions of two or more categorical variables simultaneously and is used for finding patterns between the variables. A contingency table for two categorical variables is also known as a two-way table. Contingency tables are constructed by listing all the levels (i.e., categories) of one variable as rows and all the levels of the other variable as columns in the table. A contingency table having $R$ levels of one variable in rows and $C$ levels of the other variable in columns is referred to as an $R \times C$ table. Note that each variable in a contingency table must have a finite number of levels, which can be either ordered (ordinal data) or unordered (nominal data). Importantly, the data displayed in the cells of the contingency table can be either a frequency (count) or a relative frequency (percentage) based on either overall total, row totals, or column totals.

Exhibit 14 presents a $5 \times 3$ contingency table that summarizes the number of stocks (i.e., frequency) in a particular portfolio of 1,000 stocks by two variables, sector and company market capitalization. Sector has five levels, with each one being a GICS-defined sector. Market capitalization (commonly referred to as "market cap") is defined for a company as the number of shares outstanding times the price per share. The stocks in this portfolio are categorized by three levels of market capitalization: large cap, more than $\$ 10$ billion; mid cap, $\$ 10$ billion to $\$ 2$ billion; and small cap, less than $\$ 2$ billion.

\section{Exhibit 14: Portfolio Frequencies by Sector and Market Capitalization}
Market Capitalization Variable

(3 Levels)

\begin{center}
\begin{tabular}{lcccc}
\hline
Sector Variable (5 Levels) & Small & Mid & Large & Total \\
\hline
Communication Services & 55 & 35 & 20 & 110 \\
Consumer Staples & 50 & 30 & 30 & 110 \\
Energy & 175 & 95 & 20 & 290 \\
Health Care & 275 & 105 & 55 & 435 \\
Utilities & 20 & 25 & 10 & 55 \\
\hline
Total & $\mathbf{5 7 5}$ & $\mathbf{2 9 0}$ & $\mathbf{1 3 5}$ & $\mathbf{1 , 0 0 0}$ \\
\hline
\end{tabular}
\end{center}

The entries in the cells of the contingency table show the number of stocks of each sector with a given level of market cap. For example, there are 275 small-cap health care stocks, making it the portfolio's largest subgroup in terms of frequency. These data are also called joint frequencies because you are joining one variable from the row (i.e., sector) and the other variable from the column (i.e., market cap) to count observations. The joint frequencies are then added across rows and across columns, and these corresponding sums are called marginal frequencies. For example, the marginal frequency of health care stocks in the portfolio is the sum of the joint frequencies across all three levels of market cap, so $435(=275+105+55)$. Similarly, adding the joint frequencies of small-cap stocks across all five sectors gives the marginal frequency of small-cap stocks of 575 (= $55+50+175+275+20)$.

Clearly, health care stocks and small-cap stocks have the largest marginal frequencies among sector and market cap, respectively, in this portfolio. Note the marginal frequencies represent the frequency distribution for each variable. Finally, the marginal frequencies for each variable must sum to the total number of stocks (overall total) in the portfolio-here, 1,000 (shown in the lower right cell).

Similar to the one-way frequency distribution table, we can express frequency in percentage terms as relative frequency by using one of three options. We can divide the joint frequencies by: a) the total count; b) the marginal frequency on a row; or c) the marginal frequency on a column.

Exhibit 15 shows the contingency table using relative frequencies based on total count. It is readily apparent that small-cap health care and energy stocks comprise the largest portions of the total portfolio, at 27.5\% $(=275 / 1,000)$ and 17.5\% $(=175 / 1,000)$, respectively, followed by mid-cap health care and energy stocks, at $10.5 \%$ and $9.5 \%$, respectively. Together, these two sectors make up nearly three-quarters of the portfolio $(43.5 \%+29.0 \%=72.5 \%)$.

\section{Exhibit 15: Relative Frequencies as Percentage of Total}
Market Capitalization Variable

(3 Levels)

\begin{center}
\begin{tabular}{lcccc}
\hline
Sector Variable (5 Levels) & Small & Mid & Large & Total \\
\hline
Communication Services & $5.5 \%$ & $3.5 \%$ & $2.0 \%$ & $11.0 \%$ \\
Consumer Staples & $5.0 \%$ & $3.0 \%$ & $3.0 \%$ & $11.0 \%$ \\
Energy & $17.5 \%$ & $9.5 \%$ & $2.0 \%$ & $29.0 \%$ \\
Health Care & $27.5 \%$ & $10.5 \%$ & $5.5 \%$ & $43.5 \%$ \\
Utilities & $2.0 \%$ & $2.5 \%$ & $1.0 \%$ & $5.5 \%$ \\
\hline
Total & $\mathbf{5 7 . 5 \%}$ & $\mathbf{2 9 . 0 \%}$ & $\mathbf{1 3 . 5 \%}$ & $\mathbf{1 0 0 \%}$ \\
\hline
\end{tabular}
\end{center}

Exhibit 16 shows relative frequencies based on marginal frequencies of market cap (i.e., columns). From this perspective, it is clear that the health care and energy sectors dominate the other sectors at each level of market capitalization: $78.3 \%(=275 / 575+$ 175/575), 69.0\% (=105/290 + 95/290), and 55.6\% $(=55 / 135+20 / 135)$, for small, mid, and large caps, respectively. Note that there may be a small rounding error difference between these results and the numbers shown in Exhibit 15.

\section{Exhibit 16: Relative Frequencies: Sector as Percentage of Market Cap}
\section{Market Capitalization Variable}
(3 Levels)

\begin{center}
\begin{tabular}{lcccc}
\hline
Sector Variable (5 Levels) & Small & Mid & Large & Total \\
\hline
Communication Services & $9.6 \%$ & $12.1 \%$ & $14.8 \%$ & $11.0 \%$ \\
Consumer Staples & $8.7 \%$ & $10.3 \%$ & $22.2 \%$ & $11.0 \%$ \\
Energy & $30.4 \%$ & $32.8 \%$ & $14.8 \%$ & $29.0 \%$ \\
Health Care & $47.8 \%$ & $36.2 \%$ & $40.7 \%$ & $43.5 \%$ \\
Utilities & $3.5 \%$ & $8.6 \%$ & $7.4 \%$ & $5.5 \%$ \\
\hline
Total & $100.0 \%$ & $100.0 \%$ & $100.0 \%$ & $\mathbf{1 0 0 . 0 \%}$ \\
\hline
\end{tabular}
\end{center}

In conclusion, the findings from these contingency tables using frequencies and relative frequencies indicate that in terms of the number of stocks, the portfolio can be generally described as a small- to mid-cap-oriented health care and energy sector portfolio that also includes stocks of several other defensive sectors.

As an analytical tool, contingency tables can be used in different applications. One application is for evaluating the performance of a classification model (in this case, the contingency table is called a confusion matrix). Suppose we have a model for classifying companies into two groups: those that default on their bond payments and those that do not default. The confusion matrix for displaying the model's results will be a $2 \times 2$ table showing the frequency of actual defaults versus the model's predicted frequency of defaults. Exhibit 17 shows such a confusion matrix for a sample of 2,000 non-investment-grade bonds. Using company characteristics and other inputs, the model correctly predicts 300 cases of bond defaults and 1,650 cases of no defaults.

\section{Exhibit 17: Confusion Matrix for Bond Default Prediction Model}
\begin{center}
\begin{tabular}{lccc}
\hline
Predicted & \multicolumn{2}{l}{Actual Default} &  \\
\hline
Default & Yes & No & Total \\
\hline
Yes & 300 & 40 & 340 \\
No & 10 & 1,650 & $\mathbf{1 , 6 6 0}$ \\
Total & $\mathbf{3 1 0}$ & $\mathbf{1 , 6 9 0}$ & $\mathbf{2 , 0 0 0}$ \\
\hline
\end{tabular}
\end{center}

We can also observe that this classification model incorrectly predicts default in 40 cases where no default actually occurred and also incorrectly predicts no default in 10 cases where default actually did occur. Later in the CFA Program curriculum you will learn how to construct a confusion matrix, how to calculate related model performance metrics, and how to use them to evaluate and tune a classification model.

Another application of contingency tables is to investigate potential association between two categorical variables. For example, revisiting Exhibit 14, one may ask whether the distribution of stocks by sectors is independent of the levels of market capitalization? Given the dominance of small-cap and mid-cap health care and energy stocks, the answer is likely, no.

One way to test for a potential association between categorical variables is to perform a chi-square test of independence. Essentially, the procedure involves using the marginal frequencies in the contingency table to construct a table with expected values of the observations. The actual values and expected values are used to derive the chi-square test statistic. This test statistic is then compared to a value from the chi-square distribution for a given level of significance. If the test statistic is greater than the chi-square distribution value, then there is evidence to reject the claim of independence, implying a significant association exists between the categorical variables. The following example describes how a contingency table is used to set up this test of independence.

\section{EXAMPLE 5}
\section{Contingency Tables and Association between Two Categorical Variables}
Suppose we randomly pick 315 investment funds and classify them two ways: by fund style, either a growth fund or a value fund; and by risk level, either low risk or high risk. Growth funds primarily invest in stocks whose earnings are expected to grow at a faster rate than earnings for the broad stock market. Value funds primarily invest in stocks that appear to be undervalued relative to their fundamental values. Risk here refers to volatility in the return of a given investment fund, so low (high) volatility implies low (high) risk. The data are summarized in a $2 \times 2$ contingency table shown in Exhibit 18 .

Exhibit 18: Contingency Table by Investment Fund Style and Risk Level

\begin{center}
\begin{tabular}{lcc}
\hline
 & Low Risk & High Risk \\
\hline
Growth & 73 & 26 \\
Value & 183 & 33 \\
\hline
\end{tabular}
\end{center}

\begin{enumerate}
  \item Calculate the number of growth funds and number of value funds out of the total funds.
\end{enumerate}

\section{Solution to 1}
The task is to calculate the marginal frequencies by fund style, which is done by adding joint frequencies across the rows. Therefore, the marginal frequency for growth is $73+26=99$, and the marginal frequency for value is $183+33=216$

\begin{enumerate}
  \setcounter{enumi}{1}
  \item Calculate the number of low-risk and high-risk funds out of the total funds.
\end{enumerate}

\section{Solution to 2}
The task is to calculate the marginal frequencies by fund risk, which is done by adding joint frequencies down the columns. Therefore, the marginal frequency for low risk is $73+183=256$, and the marginal frequency for high risk is $26+33=59$

\begin{enumerate}
  \setcounter{enumi}{2}
  \item Describe how the contingency table is used to set up a test for independence between fund style and risk level.
\end{enumerate}

\section{Solution to 3}
Based on the procedure mentioned for conducting a chi-square test of independence, we would perform the following three steps.

Step 1: Add the marginal frequencies and overall total to the contingency table. We have also included the relative frequency table for observed values.

Exhibit 19: Observed Marginal Frequencies and Relative

Frequencies

\begin{center}
\begin{tabular}{|c|c|c|c|c|c|c|c|}
\hline
\multicolumn{4}{|c|}{Observed Values} & \multicolumn{4}{|c|}{Observed Values} \\
\hline
 & $\begin{array}{l}\text { Low } \\ \text { Risk }\end{array}$ & $\begin{array}{l}\text { High } \\ \text { Risk }\end{array}$ &  &  & $\begin{array}{l}\text { Low } \\ \text { Risk }\end{array}$ & $\begin{array}{l}\text { High } \\ \text { Risk }\end{array}$ &  \\
\hline
Growth & 73 & 26 & 99 & Growth & $74 \%$ & $26 \%$ & $100 \%$ \\
\hline
\multirow[t]{2}{*}{Value} & 183 & 33 & 216 & Value & $85 \%$ & $15 \%$ & $100 \%$ \\
\hline
 & 256 & 59 & 315 &  &  &  &  \\
\hline
\end{tabular}
\end{center}

Step 2: Use the marginal frequencies in the contingency table to construct a table with expected values of the observations. To determine expected values for each cell, multiply the respective row total by the respective column total, then divide by the overall total. So, for $c_{e l l}^{i, j}$ (in ith row and $j$ th column):

\begin{center}
\includegraphics[max width=\textwidth]{2023_05_04_cff39ee44f77d6514e1bg-091}
\end{center}

For example,

Expected value for Growth/Low Risk is: $(99 \times 256) / 315=80.46$; and

Expected value for Value/High Risk is: $(216 \times 59) / 315=40.46$.

The table of expected values (and accompanying relative frequency table) are:

\section{Exhibit 20: Expected Marginal Frequencies and Relative}
Frequencies

\begin{center}
\begin{tabular}{|c|c|c|c|c|c|c|c|}
\hline
\multicolumn{4}{|c|}{Observed Values} & \multicolumn{4}{|c|}{Observed Values} \\
\hline
 & $\begin{array}{l}\text { Low } \\ \text { Risk }\end{array}$ & $\begin{array}{l}\text { High } \\ \text { Risk }\end{array}$ &  &  & $\begin{array}{l}\text { Low } \\ \text { Risk }\end{array}$ & $\begin{array}{l}\text { High } \\ \text { Risk }\end{array}$ &  \\
\hline
Growth & 80.457 & 18.543 & 99 & Growth & $81 \%$ & $19 \%$ & $100 \%$ \\
\hline
\multirow[t]{2}{*}{Value} & 175.543 & 40.457 & 216 & Value & $81 \%$ & $19 \%$ & $100 \%$ \\
\hline
 & 256 & 59 & 315 &  &  &  &  \\
\hline
\end{tabular}
\end{center}

Step 3: Use the actual values and the expected values of observation counts to derive the chi-square test statistic, which is then compared to a value from the chi-square distribution for a given level of significance. If the test statistic is greater than the chi-square distribution value, then there is evidence of a significant association between the categorical variables.

\section{DATA VISUALIZATION}
describe ways that data may be visualized and evaluate uses of specific visualizations

describe how to select among visualization types

Visualization is the presentation of data in a pictorial or graphical format for the purpose of increasing understanding and for gaining insights into the data. As has been said, "a picture is worth a thousand words." In this section, we discuss a variety of charts that are useful for understanding distributions, making comparisons, and exploring potential relationships among data. Specifically, we will cover visualizing frequency distributions of numerical and categorical data by using plots that represent multi-dimensional data for discovering relationships and by interpreting visuals that display unstructured data.

\section{Histogram and Frequency Polygon}
A histogram is a chart that presents the distribution of numerical data by using the height of a bar or column to represent the absolute frequency of each bin or interval in the distribution.

To construct a histogram from a continuous variable, we first need to split the data into bins and summarize the data into a frequency distribution table, such as the one we constructed in Exhibit 11. In a histogram, the $y$-axis generally represents the absolute frequency or the relative frequency in percentage terms, while the $x$-axis usually represents the bins of the variable. Using the frequency distribution table in Exhibit 11, we plot the histogram of daily returns of the EAA Equity Index, as shown in Exhibit 21. The bars are of equal width, representing the bin width of $1 \%$ for each return interval. The bars are usually drawn with no spaces in between, but small gaps can also be added between adjacent bars to increase readability, as in this exhibit. In this case, the height of each bar represents the absolute frequency for each return bin. A quick glance can tell us that the return bin $0 \%$ to $1 \%$ (exclusive) has the highest frequency, with more than 500 observations ( 555 , to be exact), and it is represented by the tallest bar in the histogram.

An advantage of the histogram is that it can effectively present a large amount of numerical data that has been grouped into a frequency distribution and can allow a quick inspection of the shape, center, and spread of the distribution to better understand it. For example, in Exhibit 21, despite the histogram of daily EAA Equity Index returns appearing bell-shaped and roughly symmetrical, most bars to the right side of the origin (i.e., zero) are taller than those on the left side, indicating that more observations lie in the bins in positive territory. Remember that in the earlier discussion of this return distribution, it was noted that $54.1 \%$ of the observations are positive daily returns.

As mentioned, histograms can also be created with relative frequencies-the choice of using absolute versus relative frequency depends on the question being answered. An absolute frequency histogram best answers the question of how many items are in each bin, while a relative frequency histogram gives the proportion or percentage of the total observations in each bin.

Exhibit 21: Histogram Overlaid with Frequency Polygon for Daily Returns of EAA Equity Index

\begin{center}
\includegraphics[max width=\textwidth]{2023_05_04_cff39ee44f77d6514e1bg-093}
\end{center}

Another graphical tool for displaying frequency distributions is the frequency polygon. To construct a frequency polygon, we plot the midpoint of each return bin on the $x$-axis and the absolute frequency for that bin on the $y$-axis. We then connect neighboring points with a straight line. Exhibit 21 shows the frequency polygon that overlays the histogram. In the graph, for example, the return interval $1 \%$ to $2 \%$ (exclusive) has a frequency of 110 , so we plot the return-interval midpoint of $0.5 \%$ (which is $1.50 \%$ on the $x$-axis) and a frequency of 110 (on the $y$-axis). Importantly, the frequency polygon can quickly convey a visual understanding of the distribution since it displays frequency as an area under the curve.

Another form for visualizing frequency distributions is the cumulative frequency distribution chart. Such a chart can plot either the cumulative absolute frequency or the cumulative relative frequency on the $y$-axis against the upper limit of the interval. The cumulative frequency distribution chart allows us to see the number or the percentage of the observations that lie below a certain value. To construct the cumulative frequency distribution, we graph the returns in the fourth (i.e., Cumulative Absolute Frequency) or fifth (i.e., Cumulative Relative Frequency) column of Exhibit 11 against the upper limit of each return interval.

Exhibit 22 presents the graph of the cumulative absolute frequency distribution for the daily returns on the EAA Equity Index. Notice that the cumulative distribution tends to flatten out when returns are extremely negative or extremely positive because the frequencies in these bins are quite small. The steep slope in the middle of Exhibit 22 reflects the fact that most of the observations-[(470+555)/1,258], or $81.5 \%$-lie in the neighborhood of $-1.0 \%$ to $1.0 \%$.

\section{Exhibit 22: Cumulative Absolute Frequency Distribution of Daily Returns of}
 EAA Equity Index\begin{center}
\includegraphics[max width=\textwidth]{2023_05_04_cff39ee44f77d6514e1bg-094}
\end{center}

\section{Bar Chart}
As we have demonstrated, the histogram is an efficient graphical tool to present the frequency distribution of numerical data. The frequency distribution of categorical data can be plotted in a similar type of graph called a bar chart. In a bar chart, each bar represents a distinct category, with the bar's height proportional to the frequency of the corresponding category.

Similar to plotting a histogram, the construction of a bar chart with one categorical variable first requires a frequency distribution table summarized from the variable. Note that the bars can be plotted vertically or horizontally. In a vertical bar chart, the $y$-axis still represents the absolute frequency or the relative frequency. Different from the histogram, however, is that the $x$-axis in a bar chart represents the mutually exclusive categories to be compared rather than bins that group numerical data.

For example, using the marginal frequencies for the five GICS sectors shown in the last column in Exhibit 14, we plot a horizontal bar chart in Exhibit 23 to show the frequency of stocks by sector in the portfolio. The bars are of equal width to represent each sector, and sufficient space should be between adjacent bars to separate them from each other. Because this is a horizontal bar chart-in this case, the $x$-axis shows the absolute frequency and the $y$-axis represents the sectors-the length of each bar represents the absolute frequency of each sector. Since sectors are nominal data with no logical ordering, the bars representing sectors may be arranged in any order. However, in the particular case where the categories in a bar chart are ordered by frequency in descending order and the chart includes a line displaying cumulative relative frequency, then it is called a Pareto Chart. The chart is often used to highlight dominant categories or the most important groups.

Bar charts provide a snapshot to show the comparison between categories of data. As shown in Exhibit 23, the sector in which the portfolio holds most stocks is the health care sector, with 435 stocks, followed by the energy sector, with 290 stocks. The sector in which the portfolio has the least number of stocks is utilities, with 55 stocks. To compare categories more accurately, in some cases we may add the frequency count to the right end of each bar (or the top end of each bar in the case of a vertical bar chart).

\section{Exhibit 23: Frequency by Sector for Stocks in a Portfolio}
\begin{center}
\includegraphics[max width=\textwidth]{2023_05_04_cff39ee44f77d6514e1bg-095}
\end{center}

The bar chart shown in Exhibit 23 can present the frequency distribution of only one categorical variable. In the case of two categorical variables, we need an enhanced version of the bar chart, called a grouped bar chart (also known as a clustered bar chart), to show joint frequencies. Using the joint frequencies by sector and by level of market capitalization given in Exhibit 14, for example, we show how a grouped bar chart is constructed in Exhibit 24. While the $y$-axis still represents the same categorical variable (the distinct GICS sectors as in Exhibit 23), in Exhibit 24 three bars are clustered side-by-side within the same sector to represent the three respective levels of market capitalization. The bars within each cluster should be colored differently to distinguish between them, but the color schemes for the sub-groups must be identical across the sector clusters, as shown by the legend at the upper right of Exhibit 24. Additionally, the bars in each sector cluster must always be placed in the same order throughout the chart. It is easy to see that the small-cap heath care stocks are the sub-group with the highest frequency (275), and we can also see that small-cap stocks are the largest sub-group within each sector-except for utilities, where mid cap is the largest.

Exhibit 24: Frequency by Sector and Level of Market Capitalization for Stocks in a Portfolio

\begin{center}
\includegraphics[max width=\textwidth]{2023_05_04_cff39ee44f77d6514e1bg-096}
\end{center}

An alternative form for presenting the joint frequency distribution of two categorical variables is a stacked bar chart. In the vertical version of a stacked bar chart, the bars representing the sub-groups are placed on top of each other to form a single bar. Each subsection of the bar is shown in a different color to represent the contribution of each sub-group, and the overall height of the stacked bar represents the marginal frequency for the category. Exhibit 24 can be replotted in a stacked bar chart, as shown in Exhibit 25. Exhibit 25: Frequency by Sector and Level of Market Capitalization in a Stacked Bar Chart

\begin{center}
\includegraphics[max width=\textwidth]{2023_05_04_cff39ee44f77d6514e1bg-097}
\end{center}

We have shown that the frequency distribution of categorical data can be clearly and efficiently presented by using a bar chart. However, it is worth noting that applications of bar charts may be extended to more general cases when categorical data are associated with numerical data. For example, suppose we want to show a company's quarterly profits over the past one year. In this case, we can plot a vertical bar chart where each bar represents one of the four quarters in a time order and its height indicates the value of profits for that quarter.

\section{Tree-Map}
In addition to bar charts and grouped bar charts, another graphical tool for displaying categorical data is a tree-map. It consists of a set of colored rectangles to represent distinct groups, and the area of each rectangle is proportional to the value of the corresponding group. For example, referring back to the marginal frequencies by GICS sector in Exhibit 14, we plot a tree-map in Exhibit 26 to represent the frequency distribution by sector for stocks in the portfolio. The tree-map clearly shows that health care is the sector with the largest number of stocks in the portfolio, which is represented by the rectangle with the largest area. Exhibit 26: Tree-Map for Frequency Distribution by Sector in a Portfolio

\begin{center}
\includegraphics[max width=\textwidth]{2023_05_04_cff39ee44f77d6514e1bg-098}
\end{center}

Note that this example also depicts one more categorical variable (i.e., level of market capitalization). The tree-map can represent data with additional dimensions by displaying a set of nested rectangles. To show the joint frequencies of sub-groups by sector and level of market capitalization, as given in Exhibit 14, we can split each existing rectangle for sector into three sub-rectangles to represent small-cap, mid-cap, and large-cap stocks, respectively. In this case, the area of each nested rectangle would be proportional to the number of stocks in each market capitalization sub-group. The exhibit clearly shows that small-cap health care is the sub-group with the largest number of stocks. It is worth noting a caveat for using tree-maps: Tree-maps become difficult to read if the hierarchy involves more than three levels.

\section{Word Cloud}
So far, we have shown how to visualize the frequency distribution of numerical data or categorical data. However, can we find a chart to depict the frequency of unstructured data-particularly, textual data? A word cloud (also known as tag cloud) is a visual device for representing textual data. A word cloud consists of words extracted from a source of textual data, with the size of each distinct word being proportional to the frequency with which it appears in the given text. Note that common words (e.g., "a," "it," "the") are generally stripped out to focus on key words that convey the most meaningful information. This format allows us to quickly perceive the most frequent terms among the given text to provide information about the nature of the text, including topic and whether or not the text conveys positive or negative news. Moreover, words conveying different sentiment may be displayed in different colors. For example, "profit" typically indicates positive sentiment so might be displayed in green, while "loss" typically indicates negative sentiment and may be shown in red.

Exhibit 27 is an excerpt from the Management's Discussion and Analysis (MDA) section of the 10- $Q$ filing for QXR Inc. for the quarter ended 31 March 20XX. Taking this text, we can create a word cloud, as shown in Exhibit 28. A quick glance at the word cloud tells us that the following words stand out (i.e., they were used most frequently in the MDA text): "billion," "revenue," “year," “income," “growth," and "financial." Note that specific words, such as "income" and "growth," typically convey positive sentiment, as contrasted with such words as "loss" and "decline," which typically convey negative sentiment. In conclusion, word clouds are a useful tool for visualizing textual data that can facilitate understanding the topic of the text as well as the sentiment it may convey.

Exhibit 27: Excerpt of MDA Section in Form 10-Q of QXR Inc. for Quarter Ended 31 March 20XX

\section{MANAGEMENT'S DISCUSSION AND ANALYSIS OF FINANCIAL CONDITION AND RESULTS OF OPERATIONS}
Please read the following discussion and analysis of our financial condition and results of operations together with our consolidated financial statements and related notes included under Part I, Item 1 of this Quarterly Report on Form 10-Q

\section{Executive Overview of Results}
Below are our key financial results for the three months ended March 31, 20XX (consolidated unless otherwise noted):

\begin{itemize}
  \item Revenues of $\$ 36.3$ billion and revenue growth of $17 \%$ year over year, constant currency revenue growth of $19 \%$ year over year.

  \item Major segment revenues of $\$ 36.2$ billion with revenue growth of $17 \%$ year over year and other segments' revenues of $\$ 170$ million with revenue growth of $13 \%$ year over year.

  \item Revenues from the United States, EMEA, APAC, and Other Americas were $\$ 16.5$ billion, $\$ 11.8$ billion, $\$ 6.1$ billion, and $\$ 1.9$ billion, respectively.

  \item Cost of revenues was $\$ 16.0$ billion, consisting of TAC of $\$ 6.9$ billion and other cost of revenues of $\$ 9.2$ billion. Our TAC as a percentage of advertising revenues were $22 \%$.

  \item Operating expenses (excluding cost of revenues) were $\$ 13.7$ billion, including the EC AFS fine of $\$ 1.7$ billion.

  \item Income from operations was $\$ 6.6$ billion

  \item Other income (expense), net, was $\$ 1.5$ billion.

  \item Effective tax rate was $18 \%$

  \item Net income was $\$ 6.7$ billion with diluted net income per share of $\$ 9.50$.

  \item Operating cash flow was $\$ 12.0$ billion.

  \item Capital expenditures were $\$ 4.6$ billion. Exhibit 28: Word Cloud Visualizing Excerpted Text in MDA Section in Form 10-Q of QXR Inc.

\end{itemize}

\begin{center}
\includegraphics[max width=\textwidth]{2023_05_04_cff39ee44f77d6514e1bg-100}
\end{center}

\section{Line Chart}
A line chart is a type of graph used to visualize ordered observations. Often a line chart is used to display the change of data series over time. Note that the frequency polygon in Exhibit 21 and the cumulative frequency distribution chart in Exhibit 22 are also line charts but used particularly in those instances for representing data frequency distributions.

Constructing a line chart is relatively straightforward: We first plot all the data points against horizontal and vertical axes and then connect the points by straight line segments. For example, to show the 10-day daily closing prices of ABC Inc. stock presented in Exhibit 5, we first construct a chart with the $x$-axis representing time (in days) and the $y$-axis representing stock price (in dollars). Next, plot each closing price as points against both axes, and then use straight line segments to join the points together, as shown in Exhibit 29.

An important benefit of a line chart is that it facilitates showing changes in the data and underlying trends in a clear and concise way. This helps to understand the current data and also helps with forecasting the data series. In Exhibit 29, for example, it is easy to spot the price changes over the first 10 trading days since ABC's initial public offering (IPO). We see that the stock price peaked on Day 3 and then traded lower. Following a partial recovery on Day 7 , it declined steeply to around $\$ 50$ on Day 10. In contrast, although the one-dimensional data array table in Exhibit 5 displays the same values as the line chart, the data table by itself does not provide a quick snapshot of changes in the data or facilitate understanding underlying trends. This is why line charts are helpful for visualization, particularly in cases of large amounts of data (i.e., hundreds, or even thousands, of data points).

\section{Exhibit 29: Daily Closing Prices of ABC Inc.s Stock and Its Sector Index}
\begin{center}
\includegraphics[max width=\textwidth]{2023_05_04_cff39ee44f77d6514e1bg-101}
\end{center}

A line chart is also capable of accommodating more than one set of data points, which is especially helpful for making comparisons. We can add a line to represent each group of data (e.g., a competitor's stock price or a sector index), and each line would have a distinct color or line pattern identified in a legend. For example, Exhibit 29 also includes a plot of ABC's sector index (i.e., the sector index for which ABC stock is a member, like health care or energy) over the same period. The sector index is displayed with its own distinct color to facilitate comparison. Note also that because the sector index has a different range (approximately 6,230 to 6,390) than ABCs' stock (\$50 to $\$ 59$ per share), we need a secondary $y$-axis to correctly display the sector index, which is on the right-hand side of the exhibit.

This comparison can help us understand whether ABC's stock price movement over the period is due to potential mispricing of its share issuance or instead due to industry-specific factors that also affect its competitors' stock prices. The comparison shows that over the period, the sector index moved in a nearly opposite trend versus ABC's stock price movement. This indicates that the steep decline in ABC's stock price is less likely attributable to sector-specific factors and more likely due to potential over-pricing of its IPO or to other company-specific factors.

When an observational unit (here, ABC Inc.) has more than two features (or variables) of interest, it would be useful to show the multi-dimensional data all in one chart to gain insights from a more holistic view. How can we add an additional dimension to a two-dimensional line chart? We can replace the data points with varying-sized bubbles to represent a third dimension of the data. Moreover, these bubbles may even be color-coded to present additional information. This version of a line chart is called a bubble line chart.

Exhibit 7, for example, presented three types of quarterly data for ABC Inc. for use in a valuation analysis. We would like to plot two of them, revenue and earnings per share (EPS), over the two-year period. As shown in Exhibit 30, with the $x$-axis representing time (i.e., quarters) and the $y$-axis representing revenue in millions of dollars, we can plot the revenue data points against both axes to form a typical line chart. Next, each marker representing a revenue data point is replaced by a circular bubble with its size proportional to the magnitude of the EPS in the corresponding quarter. Moreover, the bubbles are colored in a binary scheme with green representing profits and red representing losses. In this way, the bubble line chart reflects the changes for both revenue and EPS simultaneously, and it also shows whether the EPS represents a profit or a loss.

\section{Exhibit 30: Quarterly Revenue and EPS of ABC Incorporated}
\begin{center}
\includegraphics[max width=\textwidth]{2023_05_04_cff39ee44f77d6514e1bg-102}
\end{center}

As depicted, ABC's earning were quite volatile during its initial two years as a public company. Earnings started off as a profit of $\$ 1.37 /$ share but finished the first year with a big loss of $-\$ 8.66 /$ share, during which time revenue experienced only small fluctuations. Furthermore, while revenues and earnings both subsequently recovered sharply-peaking in Q2 of Year 2-revenues then declined, and the company returned to significant losses $(-3.98 /$ share) by the end of Year 2.

\section{Scatter Plot}
A scatter plot is a type of graph for visualizing the joint variation in two numerical variables. It is a useful tool for displaying and understanding potential relationships between the variables.

A scatter plot is constructed with the $x$-axis representing one variable and the $y$-axis representing the other variable. It uses dots to indicate the values of the two variables for a particular point in time, which are plotted against the corresponding axes. Suppose an analyst is investigating potential relationships between sector index returns and returns for the broad market, such as the S\&P 500 Index. Specifically, he or she is interested in the relative performance of two sectors, information technology (IT) and utilities, compared to the market index over a specific five-year period. The analyst has obtained the sector and market index returns for each month over the five years under investigation and plotted the data points in the scatter plots, shown in Exhibit 31 for IT versus the S\&P 500 returns and in Exhibit 32 for utilities versus the S\&P 500 returns.

Despite their relatively straightforward construction, scatter plots convey lots of valuable information. First, it is important to inspect for any potential association between the two variables. The pattern of the scatter plot may indicate no apparent relationship, a linear association, or a non-linear relationship. A scatter plot with randomly distributed data points would indicate no clear association between the two variables. However, if the data points seem to align along a straight line, then there may exist a significant relationship among the variables. A positive (negative) slope for the line of data points indicates a positive (negative) association, meaning the variables move in the same (opposite) direction. Furthermore, the strength of the association can be determined by how closely the data points are clustered around the line. Tight (loose) clustering signals a potentially stronger (weaker) relationship.

Exhibit 31: Scatter Plot of Information Technology Sector Index Return vs.

S\&P 500 Index Return

\begin{center}
\includegraphics[max width=\textwidth]{2023_05_04_cff39ee44f77d6514e1bg-103(1)}
\end{center}

Exhibit 32: Scatter Plot of Utilities Sector Index Return vs. S\&P 500 Index Return

Information Technology

\begin{center}
\includegraphics[max width=\textwidth]{2023_05_04_cff39ee44f77d6514e1bg-103}
\end{center}

Examining Exhibit 31, we can see the returns of the IT sector are highly positively associated with S\&P 500 Index returns because the data points are tightly clustered along a positively sloped line. Exhibit 32 tells a different story for relative performance of the utilities sector and S\&P 500 index returns: The data points appear to be distributed in no discernable pattern, indicating no clear relationship among these variables. Second, observing the data points located toward the ends of each axis, which represent the maximum or minimum values, provides a quick sense of the data range. Third, assuming that a relationship among the variables is apparent, inspecting the scatter plot can help to spot extreme values (i.e., outliers). For example, an outlier data point is readily detected in Exhibit 31, as indicated by the arrow. As you will learn later in the CFA Program curriculum, finding these extreme values and handling them with appropriate measures is an important part of the financial modeling process.

Scatter plots are a powerful tool for finding patterns between two variables, for assessing data range, and for spotting extreme values. In practice, however, there are situations where we need to inspect for pairwise associations among many variablesfor example, when conducting feature selection from dozens of variables to build a predictive model.

A scatter plot matrix is a useful tool for organizing scatter plots between pairs of variables, making it easy to inspect all pairwise relationships in one combined visual. For example, suppose the analyst would like to extend his or her investigation by adding another sector index. He or she can use a scatter plot matrix, as shown in Exhibit 33, which now incorporates four variables, including index returns for the S\&P 500 and for three sectors: IT, utilities, and financials.

\section{Exhibit 33: Pairwise Scatter Plot Matrix}
\begin{center}
\includegraphics[max width=\textwidth]{2023_05_04_cff39ee44f77d6514e1bg-105}
\end{center}

The scatter plot matrix contains each combination of bivariate scatter plot (i.e., S\&P 500 vs. each sector, IT vs. utilities, IT vs. financials, and financials vs. utilities) as well as univariate frequency distribution histograms for each variable plotted along the diagonal. In this way, the scatter plot matrix provides a concise visual summary of each variable and of potential relationships among them. Importantly, the construction of the scatter plot matrix is typically a built-in function in most major statistical software packages, so it is relatively easy to implement. It is worth pointing out that the upper triangle of the matrix is the mirror image of the lower triangle, so the compact form of the scatter plot matrix that uses only the lower triangle is also appropriate.

With the addition of the financial sector, the bottom panel of Exhibit 33 reveals the following additional information, which can support sector allocation in the portfolio construction process:

\begin{itemize}
  \item Strong positive relationship between returns of financial and S\&P 500;

  \item Positive relationship between returns of financial and IT; and - No clear relationship between returns of financial and utilities.

\end{itemize}

It is important to note that despite their usefulness, scatter plots and scatter plot matrixes should not be considered as a substitute for robust statistical tests; rather, they should be used alongside such tests for best results.

\section{Heat Map}
A heat map is a type of graphic that organizes and summarizes data in a tabular format and represents them using a color spectrum. For example, given a portfolio, we can create a contingency table that summarizes the joint frequencies of the stock holdings by sector and by level of market capitalization, as in Exhibit 34.

\section{Exhibit 34: Frequencies by Sector and Market Capitalization in Heat Map}
\begin{center}
\includegraphics[max width=\textwidth]{2023_05_04_cff39ee44f77d6514e1bg-106}
\end{center}

Cells in the chart are color-coded to differentiate high values from low values by using the color scheme defined in the color spectrum on the right side of the chart. As shown by the heat map, this portfolio has the largest exposure (in terms of number of stocks) to small- and mid-cap energy stocks. It has substantial exposures to large-cap communications services, mid-cap consumer staples, and small-cap utilities; however, exposure to the health care sector is limited. In sum, the heat map reveals this portfolio to be relatively well-diversified among sectors and market-cap levels. Besides their use in displaying frequency distributions, heat maps are commonly used for visualizing the degree of correlation among different variables.

\section{EXAMPLE 6}
\section{Evaluating Data Visuals}
\begin{enumerate}
  \item You have a cumulative absolute frequency distribution graph (similar to the one in Exhibit 22) of daily returns over a five-year period for an index of Asian equity markets.
\end{enumerate}

Interpret the meaning of the slope of such a graph.

\section{Solution 1}
The slope of the graph of a cumulative absolute frequency distribution reflects the change in the number of observations between two adjacent return bins. A steep (flat) slope indicates a large (small) change in the frequency of observations between adjacent return bins.

\begin{enumerate}
  \setcounter{enumi}{1}
  \item You are creating a word cloud for a visual representation of text on a company's quarterly earnings announcements over the past three years. The word cloud uses font size to indicate word frequency. This particular company has experienced both quarterly profits and losses during the period under investigation.
\end{enumerate}

Describe how the word cloud might be used to convey information besides word frequency.

\section{Solution 2}
Color can add an additional dimension to the information conveyed in the word cloud. For example, red can be used for "losses" and other words conveying negative sentiment, and green can be used for "profit" and other words indicative of positive sentiment.

\begin{enumerate}
  \setcounter{enumi}{2}
  \item You are examining a scatter plot of monthly stock returns, similar to the one in Exhibit 31, for two technology companies: one is a hardware manufacturer, and the other is a software developer. The scatter plot shows a strong positive association among their returns.
\end{enumerate}

Describe what other information the scatter plot can provide.

\section{Solution 3}
Besides the sign and degree of association of the stocks' returns, the scatter plot can provide a visual representation of whether the association is linear or non-linear, the maximum and minimum values for the return observations, and an indication of which observations may have extreme values (i.e., are potential outliers).

\begin{enumerate}
  \setcounter{enumi}{3}
  \item You are reading a vertical bar chart displaying the sales of a company over the past five years. The sales of the first four years seem nearly flat as the corresponding bars are nearly the same height, but the bar representing the sales of the most recent year is approximately three times as high as the other bars.
\end{enumerate}

Explain whether we can conclude that the sales of the fifth year tripled compared to sales in the earlier years.

\section{Solution 4}
Typically, the heights of bars in a vertical bar chart are proportional to the values that they represent. However, if the graph is using a truncated $y$-axis (i.e., one that does not start at zero), then values are not accurately represented by the height of bars. Therefore, we need to examine the $y$-axis of the bar chart before concluding that sales in the fifth year were triple the sales of the prior years.

\section{Guide to Selecting among Visualization Types}
We have introduced and discussed a variety of different visualization types that are regularly used in investment practice. When it comes to selecting a chart for visualizing data, the intended purpose is the key consideration: Is it for exploring and/or presenting distributions or relationships, or is it for making comparisons? Given your intended purpose, the best selection is typically the simplest visual that conveys the message or achieves the specific goal. Exhibit 35 presents a flow chart for facilitating selection among the visualization types we have discussed. Finally, note that some visualization types, such as bar chart and heat map, may be suitable for several different purposes.

\section{Exhibit 35: Flow Chart of Selecting Visualization Types}
\begin{center}
\includegraphics[max width=\textwidth]{2023_05_04_cff39ee44f77d6514e1bg-108}
\end{center}

Data visualization is a powerful tool to show data and gain insights into data. However, we need to be cautious that a graph could be misleading if data are mispresented or the graph is poorly constructed. There are numerous different ways that may lead to a misleading graph. We list four typical pitfalls here that analysts should avoid.

First, an improper chart type is selected to present data, which would hinder the accurate interpretation of data. For example, to investigate the correlation between two data series, we can construct a scatter plot to visualize the joint variation between two variables. In contrast, plotting the two data series separately in a line chart would make it rather difficult to examine the relationship.

Second, data are selectively plotted in favor of the conclusion an analyst intends to draw. For example, data presented for an overly short time period may appear to show a trend that is actually noise-that is, variation within the data's normal range if examining the data over a longer time period. So, presenting data for too short a time window may mistakenly point to a non-existing trend.

Third, data are improperly plotted in a truncated graph that has a $y$-axis that does not start at zero. In some situations, the truncated graph can create the false impression of significant differences when there is actually only a small difference. For example, suppose a vertical bar chart is used to compare annual revenues of two companies, one with $\$ 9$ billion and the other with $\$ 10$ billion. If the $y$-axis starts at $\$ 8$ billion, then the bar heights would inaccurately imply that the latter company's revenue is twice the former company's revenue. Last, but not least, is the improper scaling of axes. For example, given a line chart, setting a higher than necessary maximum on the $y$-axis tends to compress the graph into an area close to the $x$-axis. This causes the graph to appear to be less steep and less volatile than if it was properly plotted. In sum, analysts need to avoid these misuses of visualization when charting data and must ensure the ethical use of data visuals.

\section{EXAMPLE 7}
\section{Selecting Visualization Types}
\begin{enumerate}
  \item A portfolio manager plans to buy several stocks traded on a small emerging market exchange but is concerned whether the market can provide sufficient liquidity to support her purchase order size. As the first step, she wants to analyze the daily trading volumes of one of these stocks over the past five years.
\end{enumerate}

Explain which type of chart can best provide a quick view of trading volume for the given period.

\section{Solution to 1}
The five-year history of daily trading volumes contains a large amount of numerical data. Therefore, a histogram is the best chart for grouping these data into frequency distribution bins and for showing a quick snapshot of the shape, center, and spread of the data's distribution.

\begin{enumerate}
  \setcounter{enumi}{1}
  \item An analyst is building a model to predict stock market downturns. According to the academic literature and his practitioner knowledge and expertise, he has selected 10 variables as potential predictors. Before continuing to construct the model, the analyst would like to get a sense of how closely these variables are associated with the broad stock market index and whether any pair of variables are associated with each other.
\end{enumerate}

Describe the most appropriate visual to select for this purpose.

\section{Solution to 2}
To inspect for a potential relationship between two variables, a scatter plot is a good choice. But with 10 variables, plotting individual scatter plots is not an efficient approach. Instead, utilizing a scatter plot matrix would give the analyst a good overview in one comprehensive visual of all the pairwise associations between the variables.

\begin{enumerate}
  \setcounter{enumi}{2}
  \item Central Bank members meet regularly to assess the economy and decide on any interest rate changes. Minutes of their meetings are published on the Central Bank's website. A quantitative researcher wants to analyze the meeting minutes for use in building a model to predict future economic growth.
\end{enumerate}

Explain which type of chart is most appropriate for creating an overview of the meeting minutes.

\section*{Solution to 3 }
Since the meeting minutes consist of textual data, a word cloud would be the most suitable tool to visualize the textual data and facilitate the researcher's understanding of the topic of the text as well as the sentiment, positive or negative, it may convey.

\begin{enumerate}
  \setcounter{enumi}{3}
  \item A private investor wants to add a stock to her portfolio, so she asks her financial adviser to compare the three-year financial performances (by quarter) of two companies. One company experienced consistent revenue and earnings growth, while the other experienced volatile revenue and earnings growth, including quarterly losses.
\end{enumerate}

Describe the chart the adviser should use to best show these performance differences.

\section{Solution to 4}
The best chart for making this comparison would be a bubble line chart using two different color lines to represent the quarterly revenues for each company. The bubble sizes would then indicate the magnitude of each company's quarterly earnings, with green bubbles signifying profits and red bubbles indicating losses.

\section{MEASURES OF CENTRAL TENDENCY}
So far, we have discussed methods we can use to organize and present data so that they are more understandable. The frequency distribution of an asset return series, for example, reveals much about the nature of the risks that investors may encounter in a particular asset. Although frequency distributions, histograms, and contingency tables provide a convenient way to summarize a series of observations, these methods are just a first step toward describing the data. In this section, we discuss the use of quantitative measures that explain characteristics of data. Our focus is on measures of central tendency and other measures of location. A measure of central tendency specifies where the data are centered. Measures of central tendency are probably more widely used than any other statistical measure because they can be computed and applied relatively easily. Measures of location include not only measures of central tendency but other measures that illustrate the location or distribution of data.

In the following subsections, we explain the common measures of central tendencythe arithmetic mean, the median, the mode, the weighted mean, the geometric mean, and the harmonic mean. We also explain other useful measures of location, including quartiles, quintiles, deciles, and percentiles.

A statistic is a summary measure of a set of observations, and descriptive statistics summarize the central tendency and spread variation in the distribution of data. If the statistic summarizes the set of all possible observations of a population, we refer to the statistic as a parameter. If the statistic summarizes a set of observations that is a subset of the population, we refer to the statistic as a sample statistic, often leaving off the word "sample" and simply referring to it as a statistic. While measures of central tendency and location can be calculated for populations and samples, our focus is on sample measures (i.e., sample statistics) as it is rare that an investment manager would be dealing with an entire population of data.

\section{The Arithmetic Mean}
Analysts and portfolio managers often want one number that describes a representative possible outcome of an investment decision. The arithmetic mean is one of the most frequently used measures of the center of data.

Definition of Arithmetic Mean. The arithmetic mean is the sum of the values of the observations divided by the number of observations.

\section{The Sample Mean}
The sample mean is the arithmetic mean or arithmetic average computed for a sample. As you will see, we use the terms "mean" and "average" interchangeably. Often, we cannot observe every member of a population; instead, we observe a subset or sample of the population.

Sample Mean Formula. The sample mean or average, $\bar{X}$ (read "X-bar"), is the arithmetic mean value of a sample:

$$
\bar{X}=\frac{\sum_{i=1}^{n} x_{i}}{n}
$$

where $n$ is the number of observations in the sample.

Equation 2 tells us to sum the values of the observations $\left(X_{i}\right)$ and divide the sum by the number of observations. For example, if a sample of market capitalizations for six publicly traded Australian companies contains the values (in AUD billions) 35, 30, $22,18,15$, and 12, the sample mean market cap is 132/6 = A\$22 billion. As previously noted, the sample mean is a statistic (that is, a descriptive measure of a sample).

Means can be computed for individual units or over time. For instance, the sample might be the return on equity (ROE) in a given year for a sample of 25 companies in the FTSE Eurotop 100, an index of Europe's 100 largest companies. In this case, we calculate the mean ROE in that year as an average across 25 individual units. When we examine the characteristics of some units at a specific point in time (such as ROE for the FTSE Eurotop 100), we are examining cross-sectional data; the mean of these observations is the cross-sectional mean. If the sample consists of the historical monthly returns on the FTSE Eurotop 100 for the past five years, however, then we have time-series data; the mean of these observations is the time-series mean. We will examine specialized statistical methods related to the behavior of time series in the reading on time-series analysis.

Except in cases of large datasets with many observations, we should not expect any of the actual observations to equal the mean; sample means provide only a summary of the data being analyzed. Also, although in some cases the number of values below the mean is quite close to the number of values above the mean, this need not be the case. As an analyst, you will often need to find a few numbers that describe the characteristics of the distribution, and we will consider more later. The mean is generally the statistic that you use as a measure of the typical outcome for a distribution. You can then use the mean to compare the performance of two different markets. For example, you might be interested in comparing the stock market performance of investments in Asia Pacific with investments in Europe. You can use the mean returns in these markets to compare investment results.

\section{EXAMPLE 8}
\section{Calculating a Cross-Sectional Mean}
\begin{enumerate}
  \item Suppose we want to examine the performance of a sample of selected stock indexes from 11 different countries. The 52-week percentage change is reported in Exhibit 36 for Year 1, Year 2, and Year 3 for the sample of indexes.
\end{enumerate}

Exhibit 36: Annual Returns for Years 1 to 3 for Selected

Countries' Stock Indexes

\begin{center}
\begin{tabular}{lccc}
\hline
 & \multicolumn{2}{c}{} & 52-Week Return (\%) \\
\hline
Index & Year 1 & Year 2 & Year 3 \\
\hline
Country A & -15.6 & -5.4 & 6.1 \\
Country B & 7.8 & 6.3 & -1.5 \\
Country C & 5.3 & 1.2 & 3.5 \\
Country D & -2.4 & -3.1 & 6.2 \\
Country E & -4.0 & -3.0 & 3.0 \\
Country F & 5.4 & 5.2 & -1.0 \\
Country G & 12.7 & 6.7 & -1.2 \\
Country H & 3.5 & 4.3 & 3.4 \\
Country I & 6.2 & 7.8 & 3.2 \\
Country J & 8.1 & 4.1 & -0.9 \\
Country K & 11.5 & 3.4 & 1.2 \\
\hline
\end{tabular}
\end{center}

\begin{center}
\includegraphics[max width=\textwidth]{2023_05_04_cff39ee44f77d6514e1bg-112}
\end{center}

Using the data provided, calculate the sample mean return for the 11 indexes for each year.

\section{Solution:}
For Year 3, the calculation applies Equation 2 to the returns for Year 3: (6.1

$-1.5+3.5+6.2+3.0-1.0-1.2+3.4+3.2-0.9+1.2) / 11=22.0 / 11=2.0 \%$. Using a similar calculation, the sample mean is $3.5 \%$ for Year 1 and $2.5 \%$ for Year 2.

\section{Properties of the Arithmetic Mean}
The arithmetic mean can be likened to the center of gravity of an object. Exhibit 37 expresses this analogy graphically by plotting nine hypothetical observations on a bar. The nine observations are $2,4,4,6,10,10,12,12$, and 12 ; the arithmetic mean is $72 / 9=8$. The observations are plotted on the bar with various heights based on their frequency (that is, 2 is one unit high, 4 is two units high, and so on). When the bar is placed on a fulcrum, it balances only when the fulcrum is located at the point on the scale that corresponds to the arithmetic mean.

\section{Exhibit 37: Center of Gravity Analogy for the Arithmetic Mean}
\begin{center}
\includegraphics[max width=\textwidth]{2023_05_04_cff39ee44f77d6514e1bg-113}
\end{center}

As analysts, we often use the mean return as a measure of the typical outcome for an asset. As in Example 8, however, some outcomes are above the mean and some are below it. We can calculate the distance between the mean and each outcome, which is the deviation. Mathematically, it is always true that the sum of the deviations around the mean equals 0 . We can see this by using the definition of the arithmetic mean shown in Equation 2, multiplying both sides of the equation by $n: n \bar{X}=\sum_{i=1}^{n} X_{i}$. The
sum of the deviations from the mean is calculated as follows:

$$
\sum_{i=1}^{n}\left(X_{i}-\bar{X}\right)=\sum_{i=1}^{n} X_{i}-\sum_{i=1}^{n} \bar{X}=\sum_{i=1}^{n} X_{i}-n \bar{X}=0 .
$$

Deviations from the arithmetic mean are important information because they indicate risk. The concept of deviations around the mean forms the foundation for the more complex concepts of variance, skewness, and kurtosis, which we will discuss later.

A property and potential drawback of the arithmetic mean is its sensitivity to extreme values, or outliers. Because all observations are used to compute the mean and are given equal weight (i.e., importance), the arithmetic mean can be pulled sharply upward or downward by extremely large or small observations, respectively. For example, suppose we compute the arithmetic mean of the following seven numbers: $1,2,3,4,5,6$, and 1,000 . The mean is $1,021 / 7=145.86$, or approximately 146 . Because the magnitude of the mean, 146, is so much larger than most of the observations (the first six), we might question how well it represents the location of the data. Perhaps the most common approach in such cases is to report the median, or middle value, in place of or in addition to the mean.

\section{Outliers}
In practice, although an extreme value or outlier in a financial dataset may just represent a rare value in the population, it may also reflect an error in recording the value of an observation or an observation generated from a different population from that producing the other observations in the sample. In the latter two cases, in particular, the arithmetic mean could be misleading. So, what do we do? The first step is to examine the data, either by inspecting the sample observations if the sample is not too large or by using visualization approaches. Once we are comfortable that we have identified and eliminated errors (that is, we have "cleaned" the data), we can then address what to do with extreme values in the sample. When dealing with a sample that has extreme values, there may be a possibility of transforming the variable (e.g., a log transformation) or of selecting another variable that achieves the same purpose. However, if alternative model specifications or variable transformations are not possible, then here are three options for dealing with extreme values:

Option $1 \quad$ Do nothing; use the data without any adjustment.

Option $2 \quad$ Delete all the outliers.

Option $3 \quad$ Replace the outliers with another value.

The first option is appropriate if the values are legitimate, correct observations, and it is important to reflect the whole of the sample distribution. Outliers may contain meaningful information, so excluding or altering these values may reduce valuable information. Further, because identifying a data point as extreme leaves it up to the judgment of the analyst, leaving in all observations eliminates that need to judge a value as extreme.

The second option excludes the extreme observations. One measure of central tendency in this case is the trimmed mean, which is computed by excluding a stated small percentage of the lowest and highest values and then computing an arithmetic mean of the remaining values. For example, a $5 \%$ trimmed mean discards the lowest $2.5 \%$ and the highest $2.5 \%$ of values and computes the mean of the remaining $95 \%$ of values. A trimmed mean is used in sports competitions when judges' lowest and highest scores are discarded in computing a contestant's score.

The third option involves substituting values for the extreme values. A measure of central tendency in this case is the winsorized mean. It is calculated by assigning a stated percentage of the lowest values equal to one specified low value and a stated percentage of the highest values equal to one specified high value, and then it computes a mean from the restated data. For example, a $95 \%$ winsorized mean sets the bottom $2.5 \%$ of values equal to the value at or below which $2.5 \%$ of all the values lie (as will be seen shortly, this is called the "2.5th percentile" value) and the top $2.5 \%$ of values equal to the value at or below which $97.5 \%$ of all the values lie (the "97.5th percentile" value).

In Exhibit 38, we show the differences among these options for handling outliers using daily returns for the fictitious Euro-Asia-Africa (EAA) Equity Index in Exhibit 11.

\section{Exhibit 38: Handling Outliers: Daily Returns to an Index}
Consider the fictitious EAA Equity Index. Using daily returns on the EAA Equity Index for the period of five years, consisting of 1,258 trading days, we can see the effect of trimming and winsorizing the data:

\begin{center}
\begin{tabular}{lccc}
\hline
 & $\begin{array}{c}\text { Arithmetic } \\ \text { Mean } \\ \mathbf{( \% )}\end{array}$ & $\begin{array}{c}\text { Trimmed Mean } \\ {[\text { Trimmed 5\%] }}\end{array}$ & $\begin{array}{c}\text { Winsorized Mean } \\ \text { [95\%] } \\ \mathbf{( \% )}\end{array}$ \\
\hline
Mean & 0.035 & 0.048 & 0.038 \\
Number of Observations & 1,258 & 1,194 & 1,258 \\
\hline
\end{tabular}
\end{center}

The trimmed mean eliminates the lowest $2.5 \%$ of returns, which in this sample is any daily return less than $-1.934 \%$, and it eliminates the highest $2.5 \%$, which in this sample is any daily return greater than $1.671 \%$. The result of this trimming is that the mean is calculated using 1,194 observations instead of the original sample's 1,258 observations.

The winsorized mean substitutes $-1.934 \%$ for any return below -1.934 and substitutes $1.671 \%$ for any return above 1.671 . The result in this case is that the trimmed and winsorized means are above the arithmetic mean.

\section{The Median}
A second important measure of central tendency is the median.

Definition of Median. The median is the value of the middle item of a set of items that has been sorted into ascending or descending order. In an odd-numbered sample of $n$ items, the median is the value of the item that occupies the $(n+1) / 2$ position. In an even-numbered sample, we define the median as the mean of the values of items occupying the $n / 2$ and $(n+2) / 2$ positions (the two middle items).

Suppose we have a return on assets (in \%) for each of three companies: $0.0,2.0$, and 2.1. With an odd number of observations $(n=3)$, the median occupies the $(n+$ $1) / 2=4 / 2=2$ nd position. The median is $2.0 \%$. The value of $2.0 \%$ is the "middlemost" observation: One lies above it, and one lies below it. Whether we use the calculation for an even- or odd-numbered sample, an equal number of observations lie above and below the median. A distribution has only one median.

A potential advantage of the median is that, unlike the mean, extreme values do not affect it. For example, if a sample consists of the observations of 1, 2, 3, 4, 5, 6 and 1,000 , the median is 4 . The median is not influenced by the extremely large outcome of 1,000 . In other words, the median is affected less by outliers than the mean and therefore is useful in describing data that follow a distribution that is not symmetric, such as revenues.

The median, however, does not use all the information about the size of the observations; it focuses only on the relative position of the ranked observations. Calculating the median may also be more complex. To do so, we need to order the observations from smallest to largest, determine whether the sample size is even or odd, and then on that basis, apply one of two calculations. Mathematicians express this disadvantage by saying that the median is less mathematically tractable than the mean.

We use the data from Exhibit 36 to demonstrate finding the median, reproduced in Exhibit 39 in ascending order of the return for Year 3, with the ranked position from 1 (lowest) to 11 (highest) indicated. Because this sample has 11 observations, the median is the value in the sorted array that occupies the $(11+1) / 2=6$ th position. Country E's index occupies the sixth position and is the median. The arithmetic mean for Year 3 for this sample of indexes is $2.0 \%$, whereas the median is 3.0. Exhibit 39: Returns on Selected Country Stock Indexes for Year 3 in Ascending Order

\begin{center}
\begin{tabular}{lcc}
\hline
Index & $\begin{array}{c}\text { Year 3 } \\ \text { Return (\%) }\end{array}$ & Position \\
\hline
Country B & -1.5 & 1 \\
Country G & -1.2 & 2 \\
Country F & -1.0 & 3 \\
Country J & -0.9 & 4 \\
Country K & 1.2 & 5 \\
Country E & 3.0 & 6 \\
Country I & 3.2 & 7 \\
Country H & 3.4 & 8 \\
Country C & 3.5 & 9 \\
Country A & 6.1 & 10 \\
Country D & 6.2 & 11 \\
\hline
\end{tabular}
\end{center}

\begin{center}
\includegraphics[max width=\textwidth]{2023_05_04_cff39ee44f77d6514e1bg-116}
\end{center}

If a sample has an even number of observations, the median is the mean of the two values in the middle. For example, if our sample in Exhibit 39 had 12 indexes instead of 11 , the median would be the mean of the values in the sorted array that occupy the sixth and the seventh positions.

\section{The Mode}
The third important measure of central tendency is the mode.

Definition of Mode. The mode is the most frequently occurring value in a distribution.

A distribution can have more than one mode, or even no mode. When a distribution has a single value that is most frequently occurring, the distribution is said to be unimodal. If a distribution has two most frequently occurring values, then it has two modes and is called bimodal. If the distribution has three most frequently occurring values, then it is trimodal. When all the values in a dataset are different, the distribution has no mode because no value occurs more frequently than any other value. Stock return data and other data from continuous distributions may not have a modal outcome. When such data are grouped into bins, however, we often find an interval (possibly more than one) with the highest frequency: the modal interval (or intervals). Consider the frequency distribution of the daily returns for the EAA Equity Index over five years that we looked at in Exhibit 11. A histogram for the frequency distribution of these daily returns is shown in Exhibit 40. The modal interval always has the highest bar in the histogram; in this case, the modal interval is 0.0 to $0.9 \%$, and this interval has 493 observations out of a total of 1,258 observations.

Notice that this histogram in Exhibit 40 looks slightly different from the one in Exhibit 11, since this one has 11 bins and follows the seven-step procedure exactly. Thus, the bin width is 0.828 [= $(5.00--4.11) / 11]$, and the first bin begins at the minimum value of $-4.11 \%$. It was noted previously that for ease of interpretation, in practice bin width is often rounded up to the nearest whole number; the first bin can start at the nearest whole number below the minimum value. These refinements and the use of 10 bins were incorporated into the histogram in Exhibit 11, which has a modal interval of $0.0 \%$ to $1.0 \%$.

\section{Exhibit 40: Histogram of Daily Returns on the EAA Equity Index}
\begin{center}
\includegraphics[max width=\textwidth]{2023_05_04_cff39ee44f77d6514e1bg-117}
\end{center}

The mode is the only measure of central tendency that can be used with nominal data. For example, when we categorize investment funds into different styles and assign a number to each style, the mode of these categorized data is the most frequent investment fund style.

\section{Other Concepts of Mean}
Earlier we explained the arithmetic mean, which is a fundamental concept for describing the central tendency of data. An advantage of the arithmetic mean over two other measures of central tendency, the median and mode, is that the mean uses all the information about the size of the observations. The mean is also relatively easy to work with mathematically.

However, other concepts of mean are very important in investments. In the following sections, we discuss such concepts.

\section{The Weighted Mean}
The concept of weighted mean arises repeatedly in portfolio analysis. In the arithmetic mean, all sample observations are equally weighted by the factor $1 / n$. In working with portfolios, we often need the more general concept of weighted mean to allow for different (i.e., unequal) weights on different observations.

To illustrate the weighted mean concept, an investment manager with $\$ 100$ million to invest might allocate $\$ 70$ million to equities and $\$ 30$ million to bonds. The portfolio, therefore, has a weight of 0.70 on stocks and 0.30 on bonds. How do we calculate the return on this portfolio? The portfolio's return clearly involves an averaging of the returns on the stock and bond investments. The mean that we compute, however, must reflect the fact that stocks have a $70 \%$ weight in the portfolio and bonds have a $30 \%$ weight. The way to reflect this weighting is to multiply the return on the stock investment by 0.70 and the return on the bond investment by 0.30 , then sum the two results. This sum is an example of a weighted mean. It would be incorrect to take an arithmetic mean of the return on the stock and bond investments, equally weighting the returns on the two asset classes.

Weighted Mean Formula. The weighted mean $\bar{X}_{w}$ (read " $X$-bar sub- $w$ "), for a set of observations $X_{1}, X_{2}, \ldots, X_{n}$ with corresponding weights of $w_{1}, w_{2}$, ..., $w_{n}$, is computed as:

$$
\bar{X}_{w}=\sum_{i=1}^{n} w_{i} X_{i}
$$

where the sum of the weights equals 1 ; that is, $\sum w_{i}=1$.

In the context of portfolios, a positive weight rejresents an asset held long and a negative weight represents an asset held short.

The formula for the weighted mean can be compared to the formula for the arithmetic mean. For a set of observations $X_{1}, X_{2}, \ldots, X_{n}$, let the weights $w_{1}, w_{2}, \ldots, w_{n}$ all equal $1 / n$. Under this assumption, the formula for the weighted mean is $(1 / n) \sum_{i=1}^{n} X_{i}$. This is the formula for the arithmetic mean. Therefore, the arithmetic mean is a special case of the weighted mean in which all the weights are equal.

\section{EXAMPLE 9}
\section{Calculating a Weighted Mean}
\begin{enumerate}
  \item Using the country index data shown in Exhibit 36, consider a portfolio that consists of three funds that track three countries' indexes: County C, Country G, and Country K. The portfolio weights and index returns are as follows:
\end{enumerate}

\begin{center}
\begin{tabular}{|c|c|c|c|c|}
\hline
\multirow[b]{2}{*}{Index Tracked by Fund} & \multirow{2}{*}{$\begin{array}{c}\text { Allocation } \\
\text { (\%) }\end{array}$} & \multicolumn{3}{|c|}{Annual Return (\%)} \\
\hline
 &  & Year 1 & Year 2 & Year 3 \\
\hline
Country C & $25 \%$ & 5.3 & 1.2 & 3.5 \\
\hline
Country G & $45 \%$ & 12.7 & 6.7 & -1.2 \\
\hline
Country K & $30 \%$ & 11.5 & 3.4 & 1.2 \\
\hline
\end{tabular}
\end{center}

Using the information provided, calculate the returns on the portfolio for each year.

\section{Solution}
Converting the percentage asset allocation to decimal form, we find the mean return as the weighted average of the funds' returns. We have:

$$
\begin{aligned}
\text { Mean portfolio return for Year } 1 & =0.25(5.3)+0.45(12.7)+0.30(11.5) \\
& =10.50 \% \\
\text { Mean portfolio return for Year } 2 & =0.25(1.2)+0.45(6.7)+0.30(3.4) \\
& =4.34 \% \\
\text { Mean portfolio return for Year } 3 & =0.25(3.5)+0.45(-1.2)+0.30(1.2) \\
& =0.70 \%
\end{aligned}
$$

This example illustrates the general principle that a portfolio return is a weighted sum. Specifically, a portfolio's return is the weighted average of the returns on the assets in the portfolio; the weight applied to each asset's return is the fraction of the portfolio invested in that asset.

Market indexes are computed as weighted averages. For market-capitalization weighted indexes, such as the CAC-40 in France, the TOPIX in Japan, or the S\&P 500 in the United States, each included stock receives a weight corresponding to its market value divided by the total market value of all stocks in the index.

Our illustrations of weighted mean use past data, but they might just as well use forward-looking data. When we take a weighted average of forward-looking data, the weighted mean is the expected value. Suppose we make one forecast for the year-end level of the S\&P 500 assuming economic expansion and another forecast for the year-end level of the S\&P 500 assuming economic contraction. If we multiply the first forecast by the probability of expansion and the second forecast by the probability of contraction and then add these weighted forecasts, we are calculating the expected value of the S\&P 500 at year-end. If we take a weighted average of possible future returns on the S\&P 500, where the weights are the probabilities, we are computing the S\&P 500's expected return. The probabilities must sum to 1 , satisfying the condition on the weights in the expression for weighted mean, Equation 3.

\section{The Geometric Mean}
The geometric mean is most frequently used to average rates of change over time or to compute the growth rate of a variable. In investments, we frequently use the geometric mean to either average a time series of rates of return on an asset or a portfolio or to compute the growth rate of a financial variable, such as earnings or sales. The geometric mean is defined by the following formula.

Geometric Mean Formula. The geometric mean, $\bar{X}_{G}$, of a set of observations $X_{1}, X_{2}, \ldots, X_{n}$ is:

$$
\bar{X}_{G}=\sqrt[n]{X_{1} X_{2} X_{3} \ldots X_{n}} \text { with } X_{i} \geq 0 \text { for } i=1,2, \ldots, n .
$$

Equation 4 has a solution, and the geometric mean exists only if the product under the square root sign is non-negative. Therefore, we must impose the restriction that all the observations $X_{i}$ are greater than or equal to zero. We can solve for the geometric mean directly with any calculator that has an exponentiation key (on most calculators, $\left.y^{x}\right)$. We can also solve for the geometric mean using natural logarithms. Equation 4 can also be stated as

$$
\ln \bar{X}_{G}=\frac{1}{n} \ln \left(X_{1} X_{2} X_{3} \ldots X_{n}\right)
$$

or, because the logarithm of a product of terms is equal to the sum of the logarithms of each of the terms, as

$$
\ln \bar{X}_{G}=\frac{\sum_{i=1}^{n} \ln X_{i}}{n}
$$

\includegraphics[max width=\textwidth, center]{2023_05_04_cff39ee44f77d6514e1bg-120}
for this step is $e^{x}$ ).

Risky assets can have negative returns up to $-100 \%$ (if their price falls to zero), so we must take some care in defining the relevant variables to average in computing a geometric mean. We cannot just use the product of the returns for the sample and then take the $n$th root because the returns for any period could be negative. We must recast the returns to make them positive. We do this by adding 1.0 to the returns expressed as decimals, where $R_{t}$ represents the return in period $t$. The term $\left(1+R_{t}\right)$ represents the year-ending value relative to an initial unit of investment at the beginning of the year. As long as we use $\left(1+R_{t}\right)$, the observations will never be negative because the biggest negative return is $-100 \%$. The result is the geometric mean of 1 $+R_{t}$ b by then subtracting 1.0 from this result, we obtain the geometric mean of the individual returns $R_{t}$.

An equation that summarizes the calculation of the geometric mean return, $R_{G}$, is a slightly modified version of Equation 4 in which $X_{i}$ represents " $1+$ return in decimal form." Because geometric mean returns use time series, we use a subscript $t$ indexing time as well. We calculate one plus the geometric mean return as:

$$
1+R_{G}=\sqrt[T]{\left(1+R_{1}\right)\left(1+R_{2}\right) \ldots\left(1+R_{T}\right)} .
$$

We can represent this more compactly as:

$$
1+R_{G}=\left[\prod_{t=1}^{T}\left(1+R_{t}\right)\right]^{\frac{1}{T}},
$$

where the capital Greek letter 'pi,' $\Pi$, denotes the arithmetical operation of multiplication of the $T$ terms. Once we subtract one, this becomes the formula for the geometric mean return.

For example, the returns on Country B's index are given in Exhibit 36 as 7.8, 6.3, and $-1.5 \%$. Putting the returns into decimal form and adding 1.0 produces 1.078 , 1.063, and 0.985. Using Equation 4, we have $\sqrt[3]{(1.078)(1.063)(0.985)}=\sqrt[3]{1.128725}$ $=1.041189$. This number is 1 plus the geometric mean rate of return. Subtracting 1.0 from this result, we have $1.041189-1.0=0.041189$, or approximately $4.12 \%$. This is lower than the arithmetic mean for County B's index of $4.2 \%$.

Geometric Mean Return Formula. Given a time series of holding period returns $R_{t}, t=1,2, \ldots, T$, the geometric mean return over the time period spanned by the returns $R_{1}$ through $R_{T}$ is:

$$
R_{G}=\left[\prod_{t=1}^{T}\left(1+R_{t}\right)\right]^{\frac{1}{T}}-1 .
$$

We can use Equation 5 to solve for the geometric mean return for any return data series. Geometric mean returns are also referred to as compound returns. If the returns being averaged in Equation 5 have a monthly frequency, for example, we may call the geometric mean monthly return the compound monthly return. The next example illustrates the computation of the geometric mean while contrasting the geometric and arithmetic means.

\section{EXAMPLE 10}
\section{Geometric and Arithmetic Mean Returns}
\begin{enumerate}
  \item Using the data in Exhibit 36, calculate the arithmetic mean and the geometric mean returns over the three years for each of the three stock indexes: those of Country D, Country E, and Country F.
\end{enumerate}

\section{Solution}
The arithmetic mean returns calculations are:

\begin{center}
\begin{tabular}{lcccccc}
\hline
 & \multicolumn{2}{c}{Annual Return (\%)} &  & Sum &  &  \\
\cline { 2 - 5 }
 & Year 1 & Year 2 & Year 3 &  & $\sum_{i=1}^{3} R_{i}$ & $\begin{array}{c}\text { Arithmetic } \\ \text { Mean }\end{array}$ \\
\hline
Country D & -2.4 & -3.1 & 6.2 & 0.7 & 0.233 &  \\
Country E & -4.0 & -3.0 & 3.0 & -4.0 & -1.333 &  \\
Country F & 5.4 & 5.2 & -1.0 & 9.6 & 3.200 &  \\
\hline
\end{tabular}
\end{center}

Geometric mean returns calculations are:

\begin{center}
\begin{tabular}{|c|c|c|c|c|c|c|}
\hline
 & \multicolumn{3}{|c|}{$\begin{array}{c}1 \text { + Return in Decimal } \\
\text { Form } \\
\left(1+R_{t}\right)\end{array}$} & \multirow{2}{*}{$\begin{array}{l}\text { Product } \\
\prod_{t}^{T}\left(1+R_{t}\right)\end{array}$} & \multirow{2}{*}{[\textbackslash prod\_\{t\}\^{}\{3\}(1+R\_\{t\})]$^{\frac{1}{3}}$} & \multirow{2}{*}{$\begin{array}{l}\text { Geometric } \\
\text { mean } \\
\text { return (\%) }\end{array}$} \\
\hline
 & Year 1 & Year 2 & Year 3 &  &  \\
\hline
Country D & 0.976 & 0.969 & 1.062 & 1.00438 & 1.00146 & 0.146 \\
\hline
Country E & 0.960 & 0.970 & 1.030 & 0.95914 & 0.98619 & -1.381 \\
\hline
Country F & 1.054 & 1.052 & 0.990 & 1.09772 & 1.03157 & 3.157 \\
\hline
\end{tabular}
\end{center}

In Example 10, the geometric mean return is less than the arithmetic mean return for each country's index returns. In fact, the geometric mean is always less than or equal to the arithmetic mean. The only time that the two means will be equal is when there is no variability in the observations-that is, when all the observations in the series are the same.

In general, the difference between the arithmetic and geometric means increases with the variability within the sample; the more disperse the observations, the greater the difference between the arithmetic and geometric means. Casual inspection of the returns in Exhibit 36 and the associated graph of means suggests a greater variability for Country A's index relative to the other indexes, and this is confirmed with the greater deviation of the geometric mean return $(-5.38 \%)$ from the arithmetic mean return $(-4.97 \%)$, as we show in Exhibit 41. How should the analyst interpret these results? Exhibit 41: Arithmetic and Geometric Mean Returns for Country Stock Indexes: Years 1 to 3

Country

\begin{center}
\includegraphics[max width=\textwidth]{2023_05_04_cff39ee44f77d6514e1bg-122}
\end{center}

The geometric mean return represents the growth rate or compound rate of return on an investment. One unit of currency invested in a fund tracking the Country B index at the beginning of Year 1 would have grown to $(1.078)(1.063)(0.985)=1.128725$ units of currency, which is equal to 1 plus the geometric mean return compounded over three periods: $[1+0.041189]^{3}=1.128725$, confirming that the geometric mean is the compound rate of return. With its focus on the profitability of an investment over a multi-period horizon, the geometric mean is of key interest to investors. The arithmetic mean return, focusing on average single-period performance, is also of interest. Both arithmetic and geometric means have a role to play in investment management, and both are often reported for return series.

For reporting historical returns, the geometric mean has considerable appeal because it is the rate of growth or return we would have to earn each year to match the actual, cumulative investment performance. Suppose we purchased a stock for $€ 100$ and two years later it was worth $€ 100$, with an intervening year at $€ 200$. The geometric mean of $0 \%$ is clearly the compound rate of growth during the two years, which we can confirm by compounding the returns: $[(1+1.00)(1-0.50)]^{1 / 2}-1=$ $0 \%$. Specifically, the ending amount is the beginning amount times $\left(1+R_{G}\right)^{2}$. The geometric mean is an excellent measure of past performance.

The arithmetic mean, which is $[100 \%+-50 \%] / 2=25 \%$ in the above example, can distort our assessment of historical performance. As we noted previously, the arithmetic mean is always greater than or equal to the geometric mean. If we want to estimate the average return over a one-period horizon, we should use the arithmetic mean because the arithmetic mean is the average of one-period returns. If we want to estimate the average returns over more than one period, however, we should use the geometric mean of returns because the geometric mean captures how the total returns are linked over time. In a forward-looking context, a financial analyst calculating expected risk premiums may find that the weighted mean is appropriate, with the probabilities of the possible outcomes used as the weights.

Dispersion in cash flows or returns causes the arithmetic mean to be larger than the geometric mean. The more dispersion in the sample of returns, the more divergence exists between the arithmetic and geometric means. If there is zero variance in a sample of observations, the geometric and arithmetic return are equal.

\section{The Harmonic Mean}
The arithmetic mean, the weighted mean, and the geometric mean are the most frequently used concepts of mean in investments. A fourth concept, the harmonic mean, $\bar{X}_{H}$, is another measure of central tendency. The harmonic mean is appropriate in cases in which the variable is a rate or a ratio. The terminology "harmonic" arises from its use of a type of series involving reciprocals known as a harmonic series.

Harmonic Mean Formula. The harmonic mean of a set of observations $X_{1}$, $X_{2}, \ldots, X_{n}$ is:

$\bar{X}_{H}=\frac{n}{\sum_{i=1}^{n}\left(1 / X_{i}\right)}$ with $X_{i}>0$ for $i=1,2, \ldots, n$.

The harmonic mean is the value obtained by summing the reciprocals of the observations-terms of the form $1 / X_{i}$-then averaging that sum by dividing it by the number of observations $n$, and, finally, taking the reciprocal of the average.

The harmonic mean may be viewed as a special type of weighted mean in which an observation's weight is inversely proportional to its magnitude. For example, if there is a sample of observations of $1,2,3,4,5,6$, and 1,000 , the harmonic mean is 2.8560 . Compared to the arithmetic mean of 145.8571 , we see the influence of the outlier (the 1,000 ) to be much less than in the case of the arithmetic mean. So, the harmonic mean is quite useful as a measure of central tendency in the presence of outliers.

The harmonic mean is used most often when the data consist of rates and ratios, such as P/Es. Suppose three peer companies have P/Es of 45,15 , and 15. The arithmetic mean is 25 , but the harmonic mean, which gives less weight to the $\mathrm{P} / \mathrm{E}$ of 45 , is 19.3.

\section{EXAMPLE 11}
\section{Harmonic Mean Returns and the Returns on Selected Country Stock Indexes}
Using data in Exhibit 36, calculate the harmonic mean return over the 2016-2018 period for three stock indexes: Country D, Country E, and Country F.

\section{Calculating the Harmonic Mean for the Indexes}
\begin{center}
\begin{tabular}{|c|c|c|c|c|c|c|}
\hline
\multirow[b]{2}{*}{Index} & \multicolumn{3}{|c|}{$\begin{array}{l}\text { Inverse of } 1+\text { Return, or } \\
\qquad \frac{1}{\left(1+X_{i}\right)} \\
\text { where } X_{i} \text { is the return } \\
\text { in decimal form }\end{array}$} & \multirow{2}{*}{$\sum_{i}^{n} 1 / X_{i}$} & \multirow{2}{*}{$\frac{n}{\sum_{i}^{n} 1 / x_{i}}$} & \multirow{2}{*}{$\begin{array}{l}\text { Harmonic } \\
\text { Mean (\%) }\end{array}$} \\
\hline
 & Year 1 & Year 2 & Year 3 &  &  \\
\hline
Country D & 1.02459 & 1.03199 & 0.94162 & 2.99820 & 1.00060 & 0.05999 \\
\hline
Country E & 1.04167 & 1.03093 & 0.97087 & 3.04347 & 0.98572 & -1.42825 \\
\hline
Country F & 0.94877 & 0.95057 & 1.01010 & 2.90944 & 1.03113 & 3.11270 \\
\hline
\end{tabular}
\end{center}

Comparing the three types of means, we see the arithmetic mean is higher than the geometric mean return, and the geometric mean return is higher than the harmonic mean return. We can see the differences in these means in the following graph:

\section{Harmonic, Geometric, and Arithmetic Means of Selected Country Indexes}
\begin{center}
\includegraphics[max width=\textwidth]{2023_05_04_cff39ee44f77d6514e1bg-124}
\end{center}

The harmonic mean is a relatively specialized concept of the mean that is appropriate for averaging ratios ("amount per unit") when the ratios are repeatedly applied to a fixed quantity to yield a variable number of units. The concept is best explained through an illustration. A well-known application arises in the investment strategy known as cost averaging, which involves the periodic investment of a fixed amount of money. In this application, the ratios we are averaging are prices per share at different purchase dates, and we are applying those prices to a constant amount of money to yield a variable number of shares. An illustration of the harmonic mean to cost averaging is provided in Example 12.

\section{EXAMPLE 12}
\section{Cost Averaging and the Harmonic Mean}
\begin{enumerate}
  \item Suppose an investor purchases $€ 1,000$ of a security each month for $n=2$ months. The share prices are $€ 10$ and $€ 15$ at the two purchase dates. What is the average price paid for the security?
\end{enumerate}

Purchase in the first month $=€ 1,000 / € 10=100$ shares

Purchase in the second month $=€ 1,000 / € 15=66.67$ shares

The purchases are 166.67 shares in total, and the price paid per share is $€ 2,000 / 166.67=€ 12$.

The average price paid is in fact the harmonic mean of the asset's prices at the purchase dates. Using Equation 6, the harmonic mean price is $2 /[(1 / 10)$ $+(1 / 15)]=€ 12$. The value $€ 12$ is less than the arithmetic mean purchase price $(€ 10+€ 15) / 2=€ 12.5$.

\section{Solution:}
However, we could find the correct value of $€ 12$ using the weighted mean formula, where the weights on the purchase prices equal the shares purchased at a given price as a proportion of the total shares purchased. In our example, the calculation would be (100/166.67)€10.00 + $(66.67 / 166.67) € 15.00=€ 12$. If we had invested varying amounts of money at each date, we could not use the harmonic mean formula. We could, however, still use the weighted mean formula.

Since they use the same data but involve different progressions in their respective calculations (that is, arithmetic, geometric, and harmonic progressions) the arithmetic, geometric, and harmonic means are mathematically related to one another. While we will not go into the proof of this relationship, the basic result follows:

Arithmetic mean $\times$ Harmonic mean $=$ Geometric mean $^{2}$.

However, the key question is: Which mean to use in what circumstances?

\section{EXAMPLE 13}
\section{Calculating the Arithmetic, Geometric, and Harmonic Means for $P / E s$}
Each year in December, a securities analyst selects her 10 favorite stocks for the next year. Exhibit 42 gives the $P / E$, the ratio of share price to projected earnings per share (EPS), for her top-10 stock picks for the next year.

\section{Exhibit 42: Analyst's 10 Favorite Stocks for Next Year}
\begin{center}
\begin{tabular}{lc}
\hline
Stock & P/E \\
\hline
Stock 1 & 22.29 \\
Stock 2 & 15.54 \\
Stock 3 & 9.38 \\
Stock 4 & 15.12 \\
Stock 5 & 10.72 \\
Stock 6 & 14.57 \\
Stock 7 & 7.20 \\
Stock 8 & 7.97 \\
Stock 9 & 10.34 \\
Stock 10 & 8.35 \\
\hline
\end{tabular}
\end{center}

For these 10 stocks,

\begin{enumerate}
  \item Calculate the arithmetic mean P/E.
\end{enumerate}

\section{Solution to 1:}
The arithmetic mean is $121.48 / 10=12.1480$.

\begin{enumerate}
  \setcounter{enumi}{1}
  \item Calculate the geometric mean $\mathrm{P} / \mathrm{E}$.
\end{enumerate}

\section{Solution to 2:}
The geometric mean is $e^{24.3613 / 10}=11.4287$.

\begin{enumerate}
  \setcounter{enumi}{2}
  \item Calculate the harmonic mean P/E.
\end{enumerate}

\section{Solution to 3:}
The harmonic mean is $10 / 0.9247=10.8142$. A mathematical fact concerning the harmonic, geometric, and arithmetic means is that unless all the observations in a dataset have the same value, the harmonic mean is less than the geometric mean, which, in turn, is less than the arithmetic mean. The choice of which mean to use depends on many factors, as we describe in Exhibit 43:

\begin{itemize}
  \item Are there outliers that we want to include?

  \item Is the distribution symmetric?

  \item Is there compounding?

  \item Are there extreme outliers?

\end{itemize}

\section{Exhibit 43: Deciding Which Central Tendency Measure to Use}
\begin{center}
\includegraphics[max width=\textwidth]{2023_05_04_cff39ee44f77d6514e1bg-126}
\end{center}

\section{QUANTILES}
calculate quantiles and interpret related visualizations

Having discussed measures of central tendency, we now examine an approach to describing the location of data that involves identifying values at or below which specified proportions of the data lie. For example, establishing that 25, 50, and $75 \%$ of the annual returns on a portfolio are at or below the values $-0.05,0.16$, and 0.25 , respectively, provides concise information about the distribution of portfolio returns. Statisticians use the word quantile (or fractile) as the most general term for a value at or below which a stated fraction of the data lies. In the following section, we describe the most commonly used quantiles-quartiles, quintiles, deciles, and percentiles-and their application in investments.

\section{Quartiles, Quintiles, Deciles, and Percentiles}
We know that the median divides a distribution of data in half. We can define other dividing lines that split the distribution into smaller sizes. Quartiles divide the distribution into quarters, quintiles into fifths, deciles into tenths, and percentiles into hundredths. Given a set of observations, the $y$ th percentile is the value at or below which $y \%$ of observations lie. Percentiles are used frequently, and the other measures can be defined with respect to them. For example, the first quartile $\left(Q_{1}\right)$ divides a distribution such that $25 \%$ of the observations lie at or below it; therefore, the first quartile is also the 25th percentile. The second quartile $\left(Q_{2}\right)$ represents the 50th percentile, and the third quartile $\left(Q_{3}\right)$ represents the 75th percentile (i.e., 75\% of the observations lie at or below it). The interquartile range (IQR) is the difference between the third quartile and the first quartile, or IQR $=Q_{3}-Q_{1}$.

When dealing with actual data, we often find that we need to approximate the value of a percentile. For example, if we are interested in the value of the 75th percentile, we may find that no observation divides the sample such that exactly $75 \%$ of the observations lie at or below that value. The following procedure, however, can help us determine or estimate a percentile. The procedure involves first locating the position of the percentile within the set of observations and then determining (or estimating) the value associated with that position.

Let $P_{y}$ be the value at or below which $y \%$ of the distribution lies, or the $y$ th percentile. (For example, $P_{18}$ is the point at or below which $18 \%$ of the observations lie; this implies that $100-18=82 \%$ of the observations are greater than $P_{18}$.) The formula for the position (or location) of a percentile in an array with $n$ entries sorted in ascending order is:

$$
L_{y}=(n+1) \frac{y}{100}
$$

where $y$ is the percentage point at which we are dividing the distribution, and $L_{y}$ is the location $(L)$ of the percentile $\left(P_{y}\right)$ in the array sorted in ascending order. The value of $L_{y}$ may or may not be a whole number. In general, as the sample size increases, the percentile location calculation becomes more accurate; in small samples it may be quite approximate.

To summarize:

\begin{itemize}
  \item When the location, $L_{y}$, is a whole number, the location corresponds to an actual observation. For example, if we are determining the third quartile $\left(Q_{3}\right)$ in a sample of size $n=11$, then $L_{y}$ would be $L_{75}=(11+1)(75 / 100)=$ 9 , and the third quartile would be $P_{75}=X_{9}$, where $X_{i}$ is defined as the value of the observation in the $i$ th $\left(i=L_{75}\right.$, so 9 th), position of the data sorted in ascending order.

  \item When $L_{y}$ is not a whole number or integer, $L_{y}$ lies between the two closest integer numbers (one above and one below), and we use linear interpolation between those two places to determine $P_{y}$. Interpolation means estimating an unknown value on the basis of two known values that surround it (i.e., lie above and below it); the term "linear" refers to a straight-line estimate.

\end{itemize}

Example 14 illustrates the calculation of various quantiles for the daily return on the EAA Equity Index.

\section{EXAMPLE 14}
\section{Percentiles, Quintiles, and Quartiles for the EAA Equity Index}
Using the daily returns on the fictitious EAA Equity Index over five years and ranking them by return, from lowest to highest daily return, we show the return bins from 1 (the lowest 5\%) to 20 (the highest 5\%) as follows:

\section{Exhibit 44: EAA Equity Index Daily Returns Grouped by Size of}
 Return\begin{center}
\begin{tabular}{|c|c|c|c|c|}
\hline
\multirow[b]{2}{*}{Bin} & \multirow{2}{*}{$\begin{array}{c}\text { Cumulative } \\
\text { Percentage } \\
\text { of Sample } \\
\text { Trading Days } \\
\text { (\%) }\end{array}$} & \multicolumn{2}{|c|}{Daily Return (\%) Between*} & \multirow[b]{2}{*}{$\begin{array}{c}\text { Number of } \\
\text { Observations }\end{array}$} \\
\hline
 &  & Lower Bound & Upper Bound \\
\hline
1 & 5 & -4.108 & -1.416 & 63 \\
\hline
2 & 10 & -1.416 & -0.876 & 63 \\
\hline
3 & 15 & -0.876 & -0.629 & 63 \\
\hline
4 & 20 & -0.629 & -0.432 & 63 \\
\hline
5 & 25 & -0.432 & -0.293 & 63 \\
\hline
6 & 30 & -0.293 & -0.193 & 63 \\
\hline
7 & 35 & -0.193 & -0.124 & 62 \\
\hline
8 & 40 & -0.124 & -0.070 & 63 \\
\hline
9 & 45 & -0.070 & -0.007 & 63 \\
\hline
10 & 50 & -0.007 & 0.044 & 63 \\
\hline
11 & 55 & 0.044 & 0.108 & 63 \\
\hline
12 & 60 & 0.108 & 0.173 & 63 \\
\hline
13 & 65 & 0.173 & 0.247 & 63 \\
\hline
14 & 70 & 0.247 & 0.343 & 62 \\
\hline
15 & 75 & 0.343 & 0.460 & 63 \\
\hline
16 & 80 & 0.460 & 0.575 & 63 \\
\hline
17 & 85 & 0.575 & 0.738 & 63 \\
\hline
18 & 90 & 0.738 & 0.991 & 63 \\
\hline
19 & 95 & 0.991 & 1.304 & 63 \\
\hline
20 & 100 & 1.304 & 5.001 & 63 \\
\hline
\end{tabular}
\end{center}

Note that because of the continuous nature of returns, it is not likely for a return to fall on the boundary for any bin other than the minimum $(\operatorname{Bin}=1)$ and maximum $(\operatorname{Bin}=20)$

\begin{enumerate}
  \item Identify the 10th and 90th percentiles.
\end{enumerate}

\section{Solution to 1}
The 10th and 90th percentiles correspond to the bins or ranked returns that include $10 \%$ and $90 \%$ of the daily returns, respectively. The 10th percentile corresponds to the return of $-0.876 \%$ (and includes returns of that much and lower), and the 90th percentile corresponds to the return of $0.991 \%$ (and lower). 2. Identify the first, second, and third quintiles.

\section{Solution to 2}
The first quintile corresponds to the lowest $20 \%$ of the ranked data, or $-0.432 \%$ (and lower).

The second quintile corresponds to the lowest $40 \%$ of the ranked data, or $-0.070 \%$ (and lower).

The third quintile corresponds to the lowest $60 \%$ of the ranked data, or $0.173 \%$ (and lower).

\begin{enumerate}
  \setcounter{enumi}{2}
  \item Identify the first and third quartiles.
\end{enumerate}

\section{Solution to 3}
The first quartile corresponds to the lowest $25 \%$ of the ranked data, or $-0.293 \%$ (and lower).

The third quartile corresponds to the lowest $75 \%$ of the ranked data, or $0.460 \%$ (and lower).

\begin{enumerate}
  \setcounter{enumi}{3}
  \item Identify the median.
\end{enumerate}

\section{Solution to 4}
The median is the return for which $50 \%$ of the data lies on either side, which is $0.044 \%$, the highest daily return in the 10 th bin out of 20 .

\begin{enumerate}
  \setcounter{enumi}{4}
  \item Calculate the interquartile range.
\end{enumerate}

\section{Solution to 5}
The interquartile range is the difference between the third and first quartiles, $0.460 \%$ and $-0.293 \%$, or $0.753 \%$.

One way to visualize the dispersion of data across quartiles is to use a diagram, such as a box and whisker chart. A box and whisker plot consists of a "box" with "whiskers" connected to the box, as shown in Exhibit 45. The "box" represents the lower bound of the second quartile and the upper bound of the third quartile, with the median or arithmetic average noted as a measure of central tendency of the entire distribution. The whiskers are the lines that run from the box and are bounded by the "fences," which represent the lowest and highest values of the distribution.

\section{Exhibit 45: Box and Whisker Plot}
\begin{center}
\includegraphics[max width=\textwidth]{2023_05_04_cff39ee44f77d6514e1bg-129}
\end{center}

There are several variations for box and whisker displays. For example, for ease in detecting potential outliers, the fences of the whiskers may be a function of the interquartile range instead of the highest and lowest values like that in Exhibit 45.

In Exhibit 45, visually, the interquartile range is the height of the box and the fences are set at extremes. But another form of box and whisker plot typically uses 1.5 times the interquartile range for the fences. Thus, the upper fence is 1.5 times the interquartile range added to the upper bound of $Q_{3}$, and the lower fence is 1.5 times the interquartile range subtracted from the lower bound of $Q_{2}$. Observations beyond the fences (i.e., outliers) may also be displayed.

We can see the role of outliers in such a box and whisker plot using the EAA Equity Index daily returns, as shown in Exhibit 46. Referring back to Exhibit 44 (Example 13), we know:

\begin{itemize}
  \item The maximum and minimum values of the distribution are 5.001 and -4.108 , respectively, while the median (50th percentile) value is 0.044 .

  \item The interquartile range is $0.753[=0.460-(-0.293)]$, and when multiplied by 1.5 and added to the $Q_{3}$ upper bound of 0.460 gives an upper fence of 1.589 $[=(1.5 \times 0.753)+0.460]$.

  \item The lower fence is determined in a similar manner, using the $Q_{2}$ lower bound, to be $-1.422[=-(1.5 \times 0.753)+(-0.293)]$.

\end{itemize}

As noted, any observation above (below) the upper (lower) fence is deemed to be an outlier.

\section{Exhibit 46: Box and Whisker Chart for EAA Equity Index Daily Returns}
\begin{center}
\includegraphics[max width=\textwidth]{2023_05_04_cff39ee44f77d6514e1bg-130}
\end{center}

\section{EXAMPLE 15}
\section{Quantiles}
Consider the results of an analysis focusing on the market capitalizations of a sample of 100 firms:

\begin{center}
\begin{tabular}{|c|c|c|c|c|}
\hline
\multirow[b]{2}{*}{Bin} & \multirow{2}{*}{$\begin{array}{c}\text { Cumulative } \\
\text { Percentage of } \\
\text { Sample (\%) }\end{array}$} & \multicolumn{2}{|c|}{$\begin{array}{l}\text { Market Capitalization } \\
\text { (in billions of } € \text { ) }\end{array}$} & \multirow{2}{*}{$\begin{array}{c}\text { Number of } \\
\text { Observations }\end{array}$} \\
\hline
 &  & Lower Bound & Upper Bound \\
\hline
1 & 5 & 0.28 & 15.45 & 5 \\
\hline
2 & 10 & 15.45 & 21.22 & 5 \\
\hline
3 & 15 & 21.22 & 29.37 & 5 \\
\hline
4 & 20 & 29.37 & 32.57 & 5 \\
\hline
5 & 25 & 32.57 & 34.72 & 5 \\
\hline
6 & 30 & 34.72 & 37.58 & 5 \\
\hline
7 & 35 & 37.58 & 39.90 & 5 \\
\hline
8 & 40 & 39.90 & 41.57 & 5 \\
\hline
9 & 45 & 41.57 & 44.86 & 5 \\
\hline
10 & 50 & 44.86 & 46.88 & 5 \\
\hline
11 & 55 & 46.88 & 49.40 & 5 \\
\hline
12 & 60 & 49.40 & 51.27 & 5 \\
\hline
13 & 65 & 51.27 & 53.58 & 5 \\
\hline
14 & 70 & 53.58 & 56.66 & 5 \\
\hline
15 & 75 & 56.66 & 58.34 & 5 \\
\hline
16 & 80 & 58.34 & 63.10 & 5 \\
\hline
17 & 85 & 63.10 & 67.06 & 5 \\
\hline
18 & 90 & 67.06 & 73.00 & 5 \\
\hline
19 & 95 & 73.00 & 81.62 & 5 \\
\hline
20 & 100 & 81.62 & 96.85 & 5 \\
\hline
\end{tabular}
\end{center}

Using this information, answer the following five questions.

\begin{enumerate}
  \item The tenth percentile corresponds to observations in bins:
A. 2 .
B. 1 and 2 .
C. 19 and 20 .
\end{enumerate}

\section{Solution to 1}
$B$ is correct because the tenth percentile corresponds to the lowest $10 \%$ of the observations in the sample, which are in bins 1 and 2.

\begin{enumerate}
  \setcounter{enumi}{1}
  \item The second quintile corresponds to observations in bins:
A. 8
B. $5,6,7$, and 8 .
C. $6,7,8,9$, and 10 .
\end{enumerate}

\section{Solution to 2}
$B$ is correct because the second quintile corresponds to the second $20 \%$ of observations. The first $20 \%$ consists of bins 1 through 4 . The second $20 \%$ of observations consists of bins 5 through 8 . 3. The fourth quartile corresponds to observations in bins:
A. 17.
B. $17,18,19$, and 20 .
C. $16,17,18,19$, and 20 .

\section{Solution to 3}
$\mathrm{C}$ is correct because a quartile consists of $25 \%$ of the data, and the last $25 \%$ of the 20 bins are 16 through 20.

\begin{enumerate}
  \setcounter{enumi}{3}
  \item The median is closest to:
A. 44.86 .
B. 46.88 .
C. 49.40 .
\end{enumerate}

\section{Solution to 4}
$B$ is correct because this is the center of the 20 bins. The market capitalization of 46.88 is the highest value of the 10th bin and the lowest value of the 11th bin.

\begin{enumerate}
  \setcounter{enumi}{4}
  \item The interquartile range is closest to:
A. 20.76 .
B. 23.62 .
C. 25.52 .
\end{enumerate}

\section{Solution to 5}
$\mathrm{B}$ is correct because the interquartile range is the difference between the lowest value in the second quartile and the highest value in the third quartile. The lowest value of the second quartile is 34.72 , and the highest value of the third quartile is 58.34 . Therefore, the interquartile range is $58.34-34.72$ $=23.62$.

\section{Quantiles in Investment Practice}
In this section, we briefly discuss the use of quantiles in investments. Quantiles are used in portfolio performance evaluation as well as in investment strategy development and research.

Investment analysts use quantiles every day to rank performance-for example, the performance of portfolios. The performance of investment managers is often characterized in terms of the percentile or quartile in which they fall relative to the performance of their peer group of managers. The Morningstar investment fund star rankings, for example, associate the number of stars with percentiles of performance relative to similar-style investment funds.

Another key use of quantiles is in investment research. For example, analysts often refer to the set of companies with returns falling below the 10th percentile cutoff point as the bottom return decile. Dividing data into quantiles based on some characteristic allows analysts to evaluate the impact of that characteristic on a quantity of interest. For instance, empirical finance studies commonly rank companies based on the market value of their equity and then sort them into deciles. The first decile contains the portfolio of those companies with the smallest market values, and the tenth decile contains those companies with the largest market values. Ranking companies by decile allows analysts to compare the performance of small companies with large ones.

\section{MEASURES OF DISPERSION}
calculate and interpret measures of dispersion

Few would disagree with the importance of expected return or mean return in investments: The mean return tells us where returns, and investment results, are centered. To more completely understand an investment, however, we also need to know how returns are dispersed around the mean. Dispersion is the variability around the central tendency. If mean return addresses reward, then dispersion addresses risk.

In this section, we examine the most common measures of dispersion: range, mean absolute deviation, variance, and standard deviation. These are all measures of absolute dispersion. Absolute dispersion is the amount of variability present without comparison to any reference point or benchmark.

These measures are used throughout investment practice. The variance or standard deviation of return is often used as a measure of risk pioneered by Nobel laureate Harry Markowitz. Other measures of dispersion, mean absolute deviation and range, are also useful in analyzing data.

\section{The Range}
We encountered range earlier when we discussed the construction of frequency distributions. It is the simplest of all the measures of dispersion.

Definition of Range. The range is the difference between the maximum and minimum values in a dataset:

Range $=$ Maximum value - Minimum value .

As an illustration of range, consider Exhibit 35, our example of annual returns for countries' stock indexes. The range of returns for Year 1 is the difference between the returns of Country G's index and Country A's index, or $12.7-(-15.6)=28.3 \%$. The range of returns for Year 3 is the difference between the returns for the County $\mathrm{D}$ index and the Country B index, or $6.2-(-1.5)=7.7 \%$.

An alternative definition of range specifically reports the maximum and minimum values. This alternative definition provides more information than does the range as defined in Equation 8. In other words, in the above-mentioned case for Year 1 , the range is reported as "from $12.7 \%$ to $-15.6 \% . "$

One advantage of the range is ease of computation. A disadvantage is that the range uses only two pieces of information from the distribution. It cannot tell us how the data are distributed (that is, the shape of the distribution). Because the range is the difference between the maximum and minimum returns, it can reflect extremely large or small outcomes that may not be representative of the distribution.

\section{The Mean Absolute Deviation}
Measures of dispersion can be computed using all the observations in the distribution rather than just the highest and lowest. But how should we measure dispersion? Our previous discussion on properties of the arithmetic mean introduced the notion of distance or deviation from the mean $\left(X_{i}-\bar{X}\right)$ as a fundamental piece of information used in statistics. We could compute measures of dispersion as the arithmetic average of the deviations around the mean, but we would encounter a problem: The deviations around the mean always sum to 0 . If we computed the mean of the deviations, the result would also equal 0 . Therefore, we need to find a way to address the problem of negative deviations canceling out positive deviations.

One solution is to examine the absolute deviations around the mean as in the mean absolute deviation. This is also known as the average absolute deviation.

Mean Absolute Deviation Formula. The mean absolute deviation (MAD) for a sample is:

$\operatorname{MAD}=\frac{\sum_{i=1}^{n}\left|X_{i}-\bar{X}\right|}{n}$

where $\bar{X}$ is the sample mean, $n$ is the number of observations in the sample, and the || indicate the absolute value of what is contained within these bars.

In calculating MAD, we ignore the signs of the deviations around the mean. For example, if $X_{i}=-11.0$ and $\bar{X}=4.5$, the absolute value of the difference is $\mid-11.0-$ $4.5|=|-15.5 \mid=15.5$. The mean absolute deviation uses all of the observations in the sample and is thus superior to the range as a measure of dispersion. One technical drawback of MAD is that it is difficult to manipulate mathematically compared with the next measure we will introduce, sample variance. Example 16 illustrates the use of the range and the mean absolute deviation in evaluating risk.

\section{EXAMPLE 16}
\section{Mean Absolute Deviation for Selected Countries' Stock Index Returns}
\begin{enumerate}
  \item Using the country stock index returns in Exhibit 35, calculate the mean absolute deviation of the index returns for each year. Note the sample mean returns $(\bar{X})$ are $3.5 \%, 2.5 \%$, and $2.0 \%$ for Years 1,2 , and 3 , respectively.
\end{enumerate}

Solution

\begin{center}
\begin{tabular}{|c|c|c|c|}
\hline
 & \multicolumn{3}{|c|}{$\begin{array}{l}\text { Absolute Value of Deviation from the Mean } \\
\qquad\left|X_{i}-\bar{X}\right|\end{array}$} \\
\hline
 & Year 1 & Year 2 & Year 3 \\
\hline
Country A & 19.1 & 7.9 & 4.1 \\
\hline
Country B & 4.3 & 3.8 & 3.5 \\
\hline
Country C & 1.8 & 1.3 & 1.5 \\
\hline
Country D & 5.9 & 5.6 & 4.2 \\
\hline
Country E & 7.5 & 5.5 & 1.0 \\
\hline
Country F & 1.9 & 2.7 & 3.0 \\
\hline
Country G & 9.2 & 4.2 & 3.2 \\
\hline
Country H & 0.0 & 1.8 & 1.4 \\
\hline
\end{tabular}
\end{center}

\begin{center}
\begin{tabular}{|c|c|c|c|}
\hline
 & \multicolumn{3}{|c|}{$\begin{array}{l}\text { Absolute Value of Deviation from the Mean } \\
\qquad\left|X_{i}-\bar{X}\right|\end{array}$} \\
\hline
 & Year 1 & Year 2 & Year 3 \\
\hline
Country I & 2.7 & 5.3 & 1.2 \\
\hline
Country J & 4.6 & 1.6 & 2.9 \\
\hline
Country K & $\underline{8.0}$ & $\underline{0.9}$ & $\underline{0.8}$ \\
\hline
Sum & $\underline{65.0}$ & $\underline{40.6}$ & $\underline{26.8}$ \\
\hline
MAD & 5.91 & 3.69 & 2.44 \\
\hline
\end{tabular}
\end{center}

For Year 3, for example, the sum of the absolute deviations from the arithmetic mean $(\bar{X}=2.0)$ is 26.8 . We divide this by 11 , with the resulting MAD of 2.44 .

\section{Sample Variance and Sample Standard Deviation}
The mean absolute deviation addressed the issue that the sum of deviations from the mean equals zero by taking the absolute value of the deviations. A second approach to the treatment of deviations is to square them. The variance and standard deviation, which are based on squared deviations, are the two most widely used measures of dispersion. Variance is defined as the average of the squared deviations around the mean. Standard deviation is the positive square root of the variance. The following discussion addresses the calculation and use of variance and standard deviation.

\section{Sample Variance}
In investments, we often do not know the mean of a population of interest, usually because we cannot practically identify or take measurements from each member of the population. We then estimate the population mean using the mean from a sample drawn from the population, and we calculate a sample variance or standard deviation.

Sample Variance Formula. The sample variance, $s^{2}$, is:

$$
s^{2}=\frac{\sum_{i=1}^{n}\left(X_{i}-\bar{X}\right)^{2}}{n-1}
$$

where $\bar{X}$ is the sample mean and $n$ is the number of observations in the sample.

Given knowledge of the sample mean, we can use Equation 10 to calculate the sum of the squared differences from the mean, taking account of all $n$ items in the sample, and then to find the mean squared difference by dividing the sum by $n-1$. Whether a difference from the mean is positive or negative, squaring that difference results in a positive number. Thus, variance takes care of the problem of negative deviations from the mean canceling out positive deviations by the operation of squaring those deviations.

For the sample variance, by dividing by the sample size minus 1 (or $n-1$ ) rather than $n$, we improve the statistical properties of the sample variance. In statistical terms, the sample variance defined in Equation 10 is an unbiased estimator of the population variance (a concept covered later in the curriculum on sampling). The quantity $n-1$ is also known as the number of degrees of freedom in estimating the population variance. To estimate the population variance with $s^{2}$, we must first calculate the sample mean, which itself is an estimated parameter. Therefore, once we have computed the sample mean, there are only $n-1$ independent pieces of information from the sample; that is, if you know the sample mean and $n-1$ of the observations, you could calculate the missing sample observation.

\section{Sample Standard Deviation}
Because the variance is measured in squared units, we need a way to return to the original units. We can solve this problem by using standard deviation, the square root of the variance. Standard deviation is more easily interpreted than the variance because standard deviation is expressed in the same unit of measurement as the observations. By taking the square root, we return the values to the original unit of measurement. Suppose we have a sample with values in euros. Interpreting the standard deviation in euros is easier than interpreting the variance in squared euros.

Sample Standard Deviation Formula. The sample standard deviation, $s$, is:

$$
s=\sqrt{\frac{\sum_{i=1}^{n}\left(X_{i}-\bar{X}\right)^{2}}{n-1}},
$$

where $\bar{X}$ is the sample mean and $n$ is the number of observations in the sample.

To calculate the sample standard deviation, we first compute the sample variance. We then take the square root of the sample variance. The steps for computing the sample variance and the standard deviation are provided in Exhibit 47.

\section{Exhibit 47: Steps to Calculate Sample Standard Deviation and Variance}
\begin{center}
\begin{tabular}{llc}
\hline
Step & \multicolumn{1}{c}{Description} & Notation \\
\hline
1 & Calculate the sample mean & $\bar{X}$ \\
2 & Calculate the deviations from the sample mean & $\left(X_{i}-\bar{X}\right)$ \\
3 & $\begin{array}{l}\text { Calculate each observation's squared deviation from the } \\ \text { sample mean }\end{array}$ & $\left(X_{i}-\bar{X}\right)^{2}$ \\
4 & Sum the squared deviations from the mean & $\sum_{i=1}^{n}\left(X_{i}-\bar{X}\right)^{2}$ \\
5 & $\begin{array}{l}\text { Divide the sum of squared deviations from the mean by } \\ n-1 . \text { This is the variance }\left(s^{2}\right) .\end{array}$ & $\frac{\sum_{i=1}^{n}\left(X_{i}-\bar{X}\right)^{2}}{n-1}$ \\
6 & $\begin{array}{l}\text { Take the square root of the sum of the squared deviations } \\ \text { divided by } n-1 . \text { This is the standard deviation }(s) .\end{array}$ & $\sqrt{\frac{\sum_{i=1}^{n}\left(X_{i}-\bar{X}\right)^{2}}{n-1}}$ \\
\hline
\end{tabular}
\end{center}

We illustrate the process of calculating the sample variance and standard deviation in Example 17 using the returns of the selected country stock indexes presented in Exhibit 35 .

\section{EXAMPLE 17}
\section{Calculating Sample Variance and Standard Deviation for Returns on Selected Country Stock Indexes}
\begin{enumerate}
  \item Using the sample information on country stock indexes in Exhibit 35, calculate the sample variance and standard deviation of the sample of index returns for Year 3 .
\end{enumerate}

\section{Solution}
\begin{center}
\begin{tabular}{lccc}
\hline
Index & $\begin{array}{c}\text { Sample } \\ \text { Observation }\end{array}$ & $\begin{array}{c}\text { Deviation from the } \\ \text { Sample Mean }\end{array}$ & $\begin{array}{c}\text { Squared } \\ \text { Deviation }\end{array}$ \\
\hline
Country A & 6.1 & 4.1 & 16.810 \\
Country B & -1.5 & -3.5 & 12.250 \\
Country C & 3.5 & 1.5 & 2.250 \\
Country D & 6.2 & 4.2 & 17.640 \\
Country E & 3.0 & 1.0 & 1.000 \\
Country F & -1.0 & -3.0 & 9.000 \\
Country G & -1.2 & -3.2 & 10.240 \\
Country H & 3.4 & 1.4 & 1.960 \\
Country I & 3.2 & 1.2 & 1.440 \\
Country J & -0.9 & -2.9 & 8.410 \\
Country K & 1.2 & -0.8 & 0.640 \\
Sum & $\mathbf{2 2 . 0}$ & $\mathbf{0 . 0}$ & $\mathbf{8 1 . 6 4 0}$ \\
\hline
\end{tabular}
\end{center}

Sample variance $=81.640 / 10=8.164$

Sample standard deviation $=\sqrt{8.164}=2.857$

In addition to looking at the cross-sectional standard deviation as we did in Example 17, we could also calculate the standard deviation of a given country's returns across time (that is, the three years). Consider Country F, which has an arithmetic mean return of $3.2 \%$. The sample standard deviation is calculated as:

$$
\begin{aligned}
& \sqrt{\frac{(0.054-0.032)^{2}+(0.052-0.032)^{2}+(-0.01-0.032)^{2}}{2}} \\
= & \sqrt{\frac{0.000484+0.000400+0.001764}{2}} \\
= & \sqrt{0.001324} \\
= & 3.6387 \% .
\end{aligned}
$$

Because the standard deviation is a measure of dispersion about the arithmetic mean, we usually present the arithmetic mean and standard deviation together when summarizing data. When we are dealing with data that represent a time series of percentage changes, presenting the geometric mean-representing the compound rate of growth-is also very helpful.

\section{Dispersion and the Relationship between the Arithmetic and the Geometric Means}
We can use the sample standard deviation to help us understand the gap between the arithmetic mean and the geometric mean. The relation between the arithmetic mean $(\bar{X})$ and geometric mean $\left(\bar{X}_{G}\right)$ is:

$$
\bar{X}_{G} \approx \bar{X}-\frac{s^{2}}{2} .
$$

In other words, the larger the variance of the sample, the wider the difference between the geometric mean and the arithmetic mean.

Using the data for Country $F$ from Example 8, the geometric mean return is 3.1566\%, the arithmetic mean return is $3.2 \%$, and the factor $s^{2} / 2$ is $0.001324 / 2=0.0662 \%$ :
$3.1566 \% \approx 3.2 \%-0.0662 \%$
$3.1566 \% \approx 3.1338 \%$.

This relation informs us that the more disperse or volatile the returns, the larger the gap between the geometric mean return and the arithmetic mean return.

\section{DOWNSIDE DEVIATION AND COEFFICIENT OF VARIATION}
calculate and interpret target downside deviation

An asset's variance or standard deviation of returns is often interpreted as a measure of the asset's risk. Variance and standard deviation of returns take account of returns above and below the mean, or upside and downside risks, respectively. However, investors are typically concerned only with downside risk-for example, returns below the mean or below some specified minimum target return. As a result, analysts have developed measures of downside risk.

In practice, we may be concerned with values of return (or another variable) below some level other than the mean. For example, if our return objective is $6.0 \%$ annually (our minimum acceptable return), then we may be concerned particularly with returns below $6.0 \%$ a year. The $6.0 \%$ is the target. The target downside deviation, also referred to as the target semideviation, is a measure of dispersion of the observations (here, returns) below the target. To calculate a sample target semideviation, we first specify the target. After identifying observations below the target, we find the sum of the squared negative deviations from the target, divide that sum by the total number of observations in the sample minus 1 , and, finally, take the square root.

Sample Target Semideviation Formula. The target semideviation, $s_{\text {Target }}$, is:

$s_{\text {Target }}=\sqrt{\sum_{\text {for all } X_{i} \leq B}^{n} \frac{\left(X_{i}-B\right)^{2}}{n-1}}$

where $B$ is the target and $n$ is the total number of sample observations. We illustrate this in Example 18.

\section{EXAMPLE 18}
\section{Calculating Target Downside Deviation}
Suppose the monthly returns on a portfolio are as shown:

\section{Monthly Portfolio Returns}
Month

Return (\%)

January

5

February 3

\begin{center}
\begin{tabular}{ll}
\hline
Month & Return (\%) \\
\hline
March & -1 \\
April & -4 \\
May & 4 \\
June & 2 \\
July & 0 \\
August & 4 \\
September & 3 \\
October & 0 \\
November & 6 \\
December & 5 \\
\hline
\end{tabular}
\end{center}

\begin{enumerate}
  \item Calculate the target downside deviation when the target return is $3 \%$.
\end{enumerate}

\section{Solution to 1}
\begin{center}
\begin{tabular}{|c|c|c|c|c|}
\hline
Month & Observation & $\begin{array}{c}\text { Deviation } \\ \text { from the 3\% } \\ \text { Target }\end{array}$ & $\begin{array}{c}\text { Deviations } \\ \text { below the } \\ \text { Target }\end{array}$ & $\begin{array}{c}\text { Squared } \\ \text { Deviations } \\ \text { below the } \\ \text { Target }\end{array}$ \\
\hline
January & 5 & 2 & - & - \\
\hline
February & 3 & 0 & - & - \\
\hline
March & -1 & -4 & -4 & 16 \\
\hline
April & -4 & -7 & -7 & 49 \\
\hline
May & 4 & 1 & - & - \\
\hline
June & 2 & -1 & -1 & 1 \\
\hline
July & 0 & -3 & -3 & 9 \\
\hline
August & 4 & 1 & - & - \\
\hline
September & 3 & 0 & - & - \\
\hline
October & 0 & -3 & -3 & 9 \\
\hline
November & 6 & 3 & - & - \\
\hline
December & 5 & 2 & - & - \\
\hline
Sum &  &  &  & 84 \\
\hline
\end{tabular}
\end{center}

Target semideviation $=\sqrt{\frac{84}{11}}=2.7634 \%$

\begin{enumerate}
  \setcounter{enumi}{1}
  \item If the target return were $4 \%$, would your answer be different from that for question 1? Without using calculations, explain how would it be different?
\end{enumerate}

\section{Solution to 2}
If the target return is higher, then the existing deviations would be larger and there would be several more values in the deviations and squared deviations below the target; so, the target semideviation would be larger.

How does the target downside deviation relate to the sample standard deviation? We illustrate the differences between the target downside deviation and the standard deviation in Example 19, using the data in Example 18.

\section{EXAMPLE 19}
\section{Comparing the Target Downside Deviation with the Standard Deviation}
\begin{enumerate}
  \item Given the data in Example 18, calculate the sample standard deviation.
\end{enumerate}

\section{Solution to 1}
\begin{center}
\begin{tabular}{lccc}
\hline
Month & Observation & $\begin{array}{c}\text { Deviation from } \\ \text { the mean }\end{array}$ & $\begin{array}{c}\text { Squared } \\ \text { deviation }\end{array}$ \\
\hline
January & 5 & 2.75 & 7.5625 \\
February & 3 & 0.75 & 0.5625 \\
March & -1 & -3.25 & 10.5625 \\
April & -4 & -6.25 & 39.0625 \\
May & 4 & 1.75 & 3.0625 \\
June & 2 & -0.25 & 0.0625 \\
July & 0 & -2.25 & 5.0625 \\
August & 4 & 1.75 & 3.0625 \\
September & 3 & 0.75 & 0.5625 \\
October & 0 & -2.25 & 5.0625 \\
November & 6 & 3.75 & 14.0625 \\
December & 5 & 2.75 & 7.5625 \\
Sum & 27 &  & 96.2500 \\
\hline
\end{tabular}
\end{center}

The sample standard deviation is $\sqrt{\frac{96.2500}{11}}=2.958 \%$.

\begin{enumerate}
  \setcounter{enumi}{1}
  \item Given the data in Example 18, calculate the target downside deviation if the target is $2 \%$.
\end{enumerate}

Solution to 2

\begin{center}
\begin{tabular}{lcccc}
\hline
 & Observation & $\begin{array}{c}\text { Deviation } \\ \text { from the 2\% } \\ \text { Target }\end{array}$ & $\begin{array}{c}\text { Deviations } \\ \text { below the } \\ \text { Target }\end{array}$ & $\begin{array}{c}\text { Squared } \\ \text { Deviations } \\ \text { below the } \\ \text { Target }\end{array}$ \\
\hline
Month & 5 & 3 & - & - \\
\hline
January & 3 & 1 & - & - \\
February & -1 & -3 & -3 & 9 \\
March & -4 & -6 & -6 & 36 \\
April & 4 & 2 & - & - \\
May & 2 & 0 & - & - \\
June & 0 & -2 & -2 & 4 \\
July & 4 & 2 & - & - \\
August & 3 & 1 & - & - \\
September & 0 & -2 & -2 & 4 \\
October & 6 & 4 & - & - \\
November & 5 & 3 & - & - \\
December &  &  & $\mathbf{5 3}$ &  \\
Sum &  &  &  &  \\
\end{tabular}
\end{center}

The target semideviation with $2 \%$ target $=\sqrt{\frac{53}{11}}=2.195 \%$.

\begin{enumerate}
  \setcounter{enumi}{2}
  \item Compare the standard deviation, the target downside deviation if the target is $2 \%$, and the target downside deviation if the target is $3 \%$.
\end{enumerate}

\section{Solution to 3}
The standard deviation is based on the deviation from the mean, which is $2.25 \%$. The standard deviation includes all deviations from the mean, not just those below it. This results in a sample standard deviation of $2.958 \%$.

Considering just the four observations below the $2 \%$ target, the target semideviation is $2.195 \%$. It is less than the sample standard deviation since target semideviation captures only the downside risk (i.e., deviations below the target). Considering target semideviation with a $3 \%$ target, there are now five observations below $3 \%$, so the target semideviation is higher, at $2.763 \%$.

\section{Coefficient of Variation}
We noted earlier that the standard deviation is more easily interpreted than variance because standard deviation uses the same units of measurement as the observations. We may sometimes find it difficult to interpret what standard deviation means in terms of the relative degree of variability of different sets of data, however, either because the datasets have markedly different means or because the datasets have different units of measurement. In this section, we explain a measure of relative dispersion, the coefficient of variation that can be useful in such situations. Relative dispersion is the amount of dispersion relative to a reference value or benchmark.

The coefficient of variation is helpful in such situations as that just described (i.e., datasets with markedly different means or different units of measurement).

Coefficient of Variation Formula. The coefficient of variation, CV, is the ratio of the standard deviation of a set of observations to their mean value:

$\mathrm{CV}=s / \bar{X}$

where $s$ is the sample standard deviation and $\bar{X}$ is the sample mean.

When the observations are returns, for example, the coefficient of variation measures the amount of risk (standard deviation) per unit of reward (mean return). An issue that may arise, especially when dealing with returns, is that if $\bar{X}$ is negative, the statistic is meaningless.

The CV may be stated as a multiple (e.g., 2 times) or as a percentage (e.g., 200\%). Expressing the magnitude of variation among observations relative to their average size, the coefficient of variation permits direct comparisons of dispersion across different datasets. Reflecting the correction for scale, the coefficient of variation is a scale-free measure (that is, it has no units of measurement).

We illustrate the usefulness of coefficient of variation for comparing datasets with markedly different standard deviations using two hypothetical samples of companies in Example 20.

\section{EXAMPLE 20}
\section{Coefficient of Variation of Returns on Assets}
Suppose an analyst collects the return on assets (in percentage terms) for ten companies for each of two industries:

\begin{center}
\begin{tabular}{lcc}
\hline
Company & Industry A & Industry B \\
\hline
1 & -5 & -10 \\
2 & -3 & -9 \\
3 & -1 & -7 \\
4 & 2 & -3 \\
5 & 4 & 1 \\
6 & 6 & 3 \\
7 & 7 & 5 \\
8 & 9 & 18 \\
9 & 10 & 20 \\
10 & 11 & 22 \\
\hline
\end{tabular}
\end{center}

These data can be represented graphically as the following:

\begin{center}
\includegraphics[max width=\textwidth]{2023_05_04_cff39ee44f77d6514e1bg-142}
\end{center}

\begin{enumerate}
  \item Calculate the average return on assets (ROA) for each industry.
\end{enumerate}

\section{Solution to 1}
The arithmetic mean for both industries is the sum divided by 10 , or $40 / 10$ $=4 \%$

\begin{enumerate}
  \setcounter{enumi}{1}
  \item Calculate the standard deviation of ROA for each industry.
\end{enumerate}

\section{Solution to 2}
The standard deviation using Equation 11 for Industry A is 5.60, and for Industry B the standard deviation is 12.12.

\begin{enumerate}
  \setcounter{enumi}{2}
  \item Calculate the coefficient of variation of ROA for each industry.
\end{enumerate}

\section{Solution to 3}
The coefficient of variation for Industry A $=5.60 / 4=1.40$. The coefficient of variation for Industry $B=12.12 / 4=3.03$.

Though the two industries have the same arithmetic mean ROA, the dispersion is different-with Industry B's returns on assets being much more disperse than those of Industry A. The coefficients of variation for these two industries reflects this, with Industry $B$ having a larger coefficient of variation. The interpretation is that the risk per unit of mean return is more than two times $(2.16=3.03 / 1.40)$ greater for Industry B compared to Industry A.

\section{THE SHAPE OF THE DISTRIBUTIONS}
interpret skewness
interpret kurtosis

Mean and variance may not adequately describe an investment's distribution of returns. In calculations of variance, for example, the deviations around the mean are squared, so we do not know whether large deviations are likely to be positive or negative. We need to go beyond measures of central tendency and dispersion to reveal other important characteristics of the distribution. One important characteristic of interest to analysts is the degree of symmetry in return distributions.

If a return distribution is symmetrical about its mean, each side of the distribution is a mirror image of the other. Thus, equal loss and gain intervals exhibit the same frequencies. If the mean is zero, for example, then losses from $-5 \%$ to $-3 \%$ occur with about the same frequency as gains from $3 \%$ to $5 \%$.

One of the most important distributions is the normal distribution, depicted in Exhibit 48. This symmetrical, bell-shaped distribution plays a central role in the mean-variance model of portfolio selection; it is also used extensively in financial risk management. The normal distribution has the following characteristics:

\begin{itemize}
  \item Its mean, median, and mode are equal.

  \item It is completely described by two parameters-its mean and variance (or standard deviation).

\end{itemize}

But with any distribution other than a normal distribution, more information than the mean and variance is needed to characterize its shape. Exhibit 48: The Normal Distribution

\begin{center}
\includegraphics[max width=\textwidth]{2023_05_04_cff39ee44f77d6514e1bg-144}
\end{center}

A distribution that is not symmetrical is skewed. A return distribution with positive skew has frequent small losses and a few extreme gains. A return distribution with negative skew has frequent small gains and a few extreme losses. Exhibit 49 shows continuous positively and negatively skewed distributions. The continuous positively skewed distribution shown has a long tail on its right side; the continuous negatively skewed distribution shown has a long tail on its left side.

For a continuous positively skewed unimodal distribution, the mode is less than the median, which is less than the mean. For the continuous negatively skewed unimodal distribution, the mean is less than the median, which is less than the mode. For a given expected return and standard deviation, investors should be attracted by a positive skew because the mean return lies above the median. Relative to the mean return, positive skew amounts to limited, though frequent, downside returns compared with somewhat unlimited, but less frequent, upside returns.

\section{Exhibit 49: Properties of Skewed Distributions}
\section{A. Positively Skewed}
Density of Probability

\begin{center}
\includegraphics[max width=\textwidth]{2023_05_04_cff39ee44f77d6514e1bg-145(1)}
\end{center}

Mode Median Mean

B. Negatively Skewed

Density of Probability

\begin{center}
\includegraphics[max width=\textwidth]{2023_05_04_cff39ee44f77d6514e1bg-145}
\end{center}

Mean Median Mode

Skewness is the name given to a statistical measure of skew. (The word "skewness" is also sometimes used interchangeably for "skew.") Like variance, skewness is computed using each observation's deviation from its mean. Skewness (sometimes referred to as relative skewness) is computed as the average cubed deviation from the mean standardized by dividing by the standard deviation cubed to make the measure free of scale. A symmetric distribution has skewness of 0 , a positively skewed distribution has positive skewness, and a negatively skewed distribution has negative skewness, as given by this measure.

We can illustrate the principle behind the measure by focusing on the numerator. Cubing, unlike squaring, preserves the sign of the deviations from the mean. If a distribution is positively skewed with a mean greater than its median, then more than half of the deviations from the mean are negative and less than half are positive. However, for the sum of the cubed deviations to be positive, the losses must be small and likely and the gains less likely but more extreme. Therefore, if skewness is positive, the average magnitude of positive deviations is larger than the average magnitude of negative deviations.

The approximation for computing sample skewness when $n$ is large (100 or more) is:

Skewness $\approx\left(\frac{1}{n}\right) \frac{\sum_{i=1}^{n}\left(X_{i}-\bar{X}\right)^{3}}{s^{3}}$. A simple example illustrates that a symmetrical distribution has a skewness measure equal to 0 . Suppose we have the following data: $1,2,3,4,5,6,7,8$, and 9 . The mean outcome is 5 , and the deviations are $-4,-3,-2,-1,0,1,2,3$, and 4 . Cubing the deviations yields $-64,-27,-8,-1,0,1,8,27$, and 64 , with a sum of 0 . The numerator of skewness (and so skewness itself) is thus equal to 0 , supporting our claim.

As you will learn as the CFA Program curriculum unfolds, different investment strategies may tend to introduce different types and amounts of skewness into returns.

\section{The Shape of the Distributions: Kurtosis}
In the previous section, we discussed how to determine whether a return distribution deviates from a normal distribution because of skewness. Another way in which a return distribution might differ from a normal distribution is its relative tendency to generate large deviations from the mean. Most investors would perceive a greater chance of extremely large deviations from the mean as increasing risk.

Kurtosis is a measure of the combined weight of the tails of a distribution relative to the rest of the distribution-that is, the proportion of the total probability that is outside of, say, 2.5 standard deviations of the mean. A distribution that has fatter tails than the normal distribution is referred to as leptokurtic or fat-tailed; a distribution that has thinner tails than the normal distribution is referred to as being platykurtic or thin-tailed; and a distribution similar to the normal distribution as concerns relative weight in the tails is called mesokurtic. A fat-tailed (thin-tailed) distribution tends to generate more-frequent (less-frequent) extremely large deviations from the mean than the normal distribution.

Exhibit 50 illustrates a fat-tailed distribution. It has fatter tails than the normal distribution. By construction, the fat-tailed and normal distributions in this exhibit have the same mean, standard deviation, and skewness. Note that this fat-tailed distribution is more likely than the normal distribution to generate observations in the tail regions defined by the intersection of graphs near a standard deviation of about \textbackslash pm 2.5 . This fat-tailed distribution is also more likely to generate observations that are near the mean, defined here as the region \textbackslash pm 1 standard deviation around the mean. In compensation, to have probabilities sum to 1 , this distribution generates fewer observations in the regions between the central region and the two tail regions.

Exhibit 50: Fat-Tailed Distribution Compared to the Normal Distribution

\begin{center}
\includegraphics[max width=\textwidth]{2023_05_04_cff39ee44f77d6514e1bg-146}
\end{center}

The calculation for kurtosis involves finding the average of deviations from the mean raised to the fourth power and then standardizing that average by dividing by the standard deviation raised to the fourth power. A normal distribution has kurtosis of 3.0, so a fat-tailed distribution has a kurtosis of above 3 and a thin-tailed distribution of below 3.0.

Excess kurtosis is the kurtosis relative to the normal distribution. For a large sample size ( $n=100$ or more), sample excess kurtosis $\left(K_{E}\right)$ is approximately as follows:

$$
K_{E} \approx\left[\left(\frac{1}{n}\right) \frac{\sum_{i=1}^{n}\left(X_{i}-\bar{X}\right)^{4}}{s^{4}}\right]-3 .
$$

As with skewness, this measure is free of scale. Many statistical packages report estimates of sample excess kurtosis, labeling this as simply "kurtosis."

Excess kurtosis thus characterizes kurtosis relative to the normal distribution. A normal distribution has excess kurtosis equal to 0 . A fat-tailed distribution has excess kurtosis greater than 0 , and a thin-tailed distribution has excess kurtosis less than 0 . A return distribution with positive excess kurtosis-a fat-tailed return distribution-has more frequent extremely large deviations from the mean than a normal distribution.

Summarizing:

\begin{center}
\begin{tabular}{|c|c|c|c|}
\hline
If kurtosis is ... & $\begin{array}{l}\text { then excess } \\ \text { kurtosis is ... }\end{array}$ & $\begin{array}{l}\text { Therefore, the } \\ \text { distribution is ... }\end{array}$ & $\begin{array}{c}\text { And we refer to } \\ \text { the distribution as } \\ \text { being ... }\end{array}$ \\
\hline
above 3.0 & above 0 . & $\begin{array}{l}\text { fatter-tailed than the } \\ \text { normal distribution. }\end{array}$ & $\begin{array}{c}\text { fat-tailed } \\ \text { (leptokurtic). }\end{array}$ \\
\hline
equal to 3.0 & equal to 0 . & $\begin{array}{l}\text { similar in tails to the nor- } \\ \text { mal distribution. }\end{array}$ & mesokurtic. \\
\hline
less than 3.0 & less than 0 . & $\begin{array}{c}\text { thinner-tailed than the } \\ \text { normal distribution. }\end{array}$ & $\begin{array}{l}\text { thin-tailed } \\ \text { (platykurtic). }\end{array}$ \\
\hline
\end{tabular}
\end{center}

Most equity return series have been found to be fat-tailed. If a return distribution is fat-tailed and we use statistical models that do not account for the distribution, then we will underestimate the likelihood of very bad or very good outcomes. Using the data on the daily returns of the fictitious EAA Equity Index, we see the skewness and kurtosis of these returns in Exhibit 51.

\section{Exhibit 51: Skewness and Kurtosis of EAA Equity Index Daily Returns}
\begin{center}
\begin{tabular}{lc}
\hline
 & Daily Return (\%) \\
\hline
Arithmetic mean & 0.0347 \\
Standard deviation & 0.8341 \\
 &  \\
Skewness & Measure of Symmetry \\
Excess kurtosis & -0.4260 \\
\hline
\end{tabular}
\end{center}

We can see this graphically, comparing the distribution of the daily returns with a normal distribution with the same mean and standard deviation: Number of Observations

\begin{center}
\includegraphics[max width=\textwidth]{2023_05_04_cff39ee44f77d6514e1bg-148}
\end{center}

Using both the statistics and the graph, we see the following:

\begin{itemize}
  \item The distribution is negatively skewed, as indicated by the negative calculated skewness of -0.4260 and the influence of observations below the mean of $0.0347 \%$.

  \item The highest frequency of returns occurs within the -0.5 to 0.0 standard deviations from the mean (i.e., negatively skewed).

  \item The distribution is fat-tailed, as indicated by the positive excess kurtosis of 3.7962. We can see fat tails, a concentration of returns around the mean, and fewer observations in the regions between the central region and the two-tail regions.

\end{itemize}

\section{EXAMPLE 21}
\section{Interpreting Skewness and Kurtosis}
Consider the daily trading volume for a stock for one year, as shown in the graph below. In addition to the count of observations within each bin or interval, the number of observations anticipated based on a normal distribution (given the sample arithmetic average and standard deviation) is provided in the chart as well. The average trading volume per day for this stock in this year is 8.6 million shares, and the standard deviation is 4.9 million shares.

\section{Histogram of Daily Trading Volume for a Stock for One Year}
\begin{center}
\includegraphics[max width=\textwidth]{2023_05_04_cff39ee44f77d6514e1bg-149}
\end{center}

\begin{enumerate}
  \item Describe whether or not this distribution is skewed. If so, what could account for this situation?
\end{enumerate}

\section{Solution to 1}
The distribution appears to be skewed to the right, or positively skewed. This is likely due to: (1) no possible negative trading volume on a given trading day, so the distribution is truncated at zero; and (2) greater-than-typical trading occurring relatively infrequently, such as when there are company-specific announcements.

The actual skewness for this distribution is 2.1090 , which supports this interpretation.

\begin{enumerate}
  \setcounter{enumi}{1}
  \item Describe whether or not this distribution displays kurtosis. How would you make this determination?
\end{enumerate}

\section{Solution to 2}
The distribution appears to have excess kurtosis, with a right-side fat tail and with maximum shares traded in the 4.6 to 6.1 million range, exceeding what is expected if the distribution was normally distributed. There are also fewer observations than expected between the central region and the tail.

The actual excess kurtosis for this distribution is 5.2151, which supports this interpretation.

\section{CORRELATION BETWEEN TWO VARIABLES}
interpret correlation between two variables Now that we have some understanding of sample variance and standard deviation, we can more formally consider the concept of correlation between two random variables that we previously explored visually in the scatter plots in Section 4. Correlation is a measure of the linear relationship between two random variables.

The first step is to consider how two variables vary together, their covariance.

Definition of Sample Covariance. The sample covariance $\left(s_{X Y}\right)$ is a measure of how two variables in a sample move together:

$$
s_{X Y}=\frac{\sum_{i=1}^{n}\left(X_{i}-\bar{X}\right)\left(Y_{i}-\bar{Y}\right)}{n-1}
$$

Equation 14 indicates that the sample covariance is the average value of the product of the deviations of observations on two random variables $\left(X_{i}\right.$ and $\left.Y i\right)$ from their sample means. If the random variables are returns, the units would be returns squared. Also, note the use of $n-1$ in the denominator, which ensures that the sample covariance is an unbiased estimate of population covariance.

Stated simply, covariance is a measure of the joint variability of two random variables. If the random variables vary in the same direction-for example, $X$ tends to be above its mean when $Y$ is above its mean, and $X$ tends to be below its mean when $Y$ is below its mean-then their covariance is positive. If the variables vary in the opposite direction relative to their respective means, then their covariance is negative.

By itself, the size of the covariance measure is difficult to interpret as it is not normalized and so depends on the magnitude of the variables. This brings us to the normalized version of covariance, which is the correlation coefficient.

Definition of Sample Correlation Coefficient. The sample correlation coefficient is a standardized measure of how two variables in a sample move together. The sample correlation coefficient $\left(r_{X Y}\right)$ is the ratio of the sample covariance to the product of the two variables' standard deviations:

$$
r_{X Y}=\frac{s_{X Y}}{s_{X} s_{Y}}
$$

Importantly, the correlation coefficient expresses the strength of the linear relationship between the two random variables.

\section{Properties of Correlation}
We now discuss the correlation coefficient, or simply correlation, and its properties in more detail, as follows:

\begin{enumerate}
  \item Correlation ranges from -1 and +1 for two random variables, $X$ and $Y$ :
\end{enumerate}

$-1 \leq r_{X Y} \leq+1$.

\begin{enumerate}
  \setcounter{enumi}{1}
  \item A correlation of 0 (uncorrelated variables) indicates an absence of any linear (that is, straight-line) relationship between the variables.

  \item A positive correlation close to +1 indicates a strong positive linear relationship. A correlation of 1 indicates a perfect linear relationship.

  \item A negative correlation close to -1 indicates a strong negative (that is, inverse) linear relationship. A correlation of -1 indicates a perfect inverse linear relationship. We will make use of scatter plots, similar to those used previously in our discussion of data visualization, to illustrate correlation. In contrast to the correlation coefficient, which expresses the relationship between two data series using a single number, a scatter plot depicts the relationship graphically. Therefore, scatter plots are a very useful tool for the sensible interpretation of a correlation coefficient.

\end{enumerate}

Exhibit 52 shows examples of scatter plots. Panel A shows the scatter plot of two variables with a correlation of +1 . Note that all the points on the scatter plot in Panel A lie on a straight line with a positive slope. Whenever variable $X$ increases by one unit, variable $Y$ increases by two units. Because all of the points in the graph lie on a straight line, an increase of one unit in $X$ is associated with exactly a two-unit increase in $Y$, regardless of the level of $X$. Even if the slope of the line were different (but positive), the correlation between the two variables would still be +1 as long as all the points lie on that straight line. Panel B shows a scatter plot for two variables with a correlation coefficient of -1 . Once again, the plotted observations all fall on a straight line. In this graph, however, the line has a negative slope. As $X$ increases by one unit, $Y$ decreases by two units, regardless of the initial value of $X$.

\section{Exhibit 52: Scatter Plots Showing Various Degrees of Correlation}
A. Variables With a Correlation of +1 Variable $Y$

\begin{center}
\includegraphics[max width=\textwidth]{2023_05_04_cff39ee44f77d6514e1bg-151(1)}
\end{center}

C. Variables With a Correlation of 0 Variable $Y$

\begin{center}
\includegraphics[max width=\textwidth]{2023_05_04_cff39ee44f77d6514e1bg-151}
\end{center}

B. Variables With a Correlation of -1 Variable $Y$

\begin{center}
\includegraphics[max width=\textwidth]{2023_05_04_cff39ee44f77d6514e1bg-151(2)}
\end{center}

D. Variables With a Strong Nonlinear Association

Variable $Y$

\begin{center}
\includegraphics[max width=\textwidth]{2023_05_04_cff39ee44f77d6514e1bg-151(3)}
\end{center}

Panel C shows a scatter plot of two variables with a correlation of 0 ; they have no linear relation. This graph shows that the value of variable $X$ tells us nothing about the value of variable $Y$. Panel D shows a scatter plot of two variables that have a non-linear relationship. Because the correlation coefficient is a measure of the linear association between two variables, it would not be appropriate to use the correlation coefficient in this case.

Example 22 is meant to reinforce your understanding of how to interpret covariance and correlation.

\section{EXAMPLE 22}
\section{Interpreting the Correlation Coefficient}
Consider the statistics for the returns over twelve months for three funds, A, B, and C, shown in Exhibit 53.

Exhibit 53

\begin{center}
\begin{tabular}{lccc}
\hline
 & Fund A & Fund B & Fund C \\
\hline
Arithmetic average & 2.9333 & 3.2250 & 2.6250 \\
Standard deviation & 2.4945 & 2.4091 & 3.6668 \\
\hline
\end{tabular}
\end{center}

The covariances are represented in the upper-triangle (shaded area) of the matrix shown in Exhibit 54.

\section{Exhibit 54}
\begin{center}
\begin{tabular}{lccc}
\hline
 & Fund A & Fund B & Fund C \\
\hline
Fund A & 6.2224 & 5.7318 & -3.6682 \\
Fund B &  & 5.8039 & -2.3125 \\
Fund C &  &  & 13.4457 \\
\hline
\end{tabular}
\end{center}

The covariance of Fund A and Fund B returns, for example, is 5.7318.

Why show just the upper-triangle of this matrix? Because the covariance of Fund $A$ and Fund $B$ returns is the same as the covariance of Fund B and Fund A returns.

The diagonal of the matrix in Exhibit 54 is the variance of each fund's return. For example, the variance of Fund A returns is 6.2224, but the covariance of Fund $A$ and Fund B returns is 5.7138 .

The correlations among the funds' returns are given in Exhibit 55, where the correlations are reported in the upper-triangle (shaded area) of the matrix. Note that the correlation of a fund's returns with itself is +1 , so the diagonal in the correlation matrix consists of 1.000 .

\section{Exhibit 55}
\begin{center}
\begin{tabular}{lccc}
\hline
 & Fund A & Fund B & Fund C \\
\hline
Fund A & 1.0000 & 0.9538 & -0.4010 \\
Fund B &  & 1.0000 & -0.2618 \\
Fund C &  &  & 1.0000 \\
\hline
\end{tabular}
\end{center}

\begin{enumerate}
  \item Interpret the correlation between Fund A's returns and Fund B's returns.
\end{enumerate}

\section{Solution to 1}
The correlation of Fund $A$ and Fund $B$ returns is 0.9538 , which is positive and close to 1.0. This means that when returns of Fund $A$ tend to be above their mean, Fund B's returns also tend to be above their mean. Graphically, we would observe a positive, but not perfect, linear relationship between the returns for the two funds.

\begin{enumerate}
  \setcounter{enumi}{1}
  \item Interpret the correlation between Fund A's returns and Fund C's returns.
\end{enumerate}

\section{Solution to 2}
The correlation of Fund A's returns and Fund C's returns is -0.4010, which indicates that when Fund A's returns are above their mean, Fund B's returns tend to be below their mean. This implies a negative slope when graphing the returns of these two funds, but it would not be a perfect inverse relationship.

\begin{enumerate}
  \setcounter{enumi}{2}
  \item Describe the relationship of the covariance of these returns and the correlation of returns.
\end{enumerate}

\section{Solution to 3}
There are two negative correlations: Fund A returns with Fund C returns, and Fund B returns with Fund C returns. What determines the sign of the correlation is the sign of the covariance, which in each of these cases is negative. When the covariance between fund returns is positive, such as between Fund A and Fund B returns, the correlation is positive. This follows from the fact that the correlation coefficient is the ratio of the covariance of the two funds' returns to the product of their standard deviations.

\section{Limitations of Correlation Analysis}
Exhibit 52 illustrates that correlation measures the linear association between two variables, but it may not always be reliable. Two variables can have a strong nonlinear relation and still have a very low correlation. For example, the relation $Y=(X-4)^{2}$ is a nonlinear relation contrasted to the linear relation $Y=2 X-4$. The nonlinear relation between variables $X$ and $Y$ is shown in Panel D. Below a level of 4 for $X, Y$ increases with decreasing values of $X$. When $X$ is 4 or greater, however, $Y$ increases whenever $X$ increases. Even though these two variables are perfectly associated, there is no linear association between them (hence, no meaningful correlation)

Correlation may also be an unreliable measure when outliers are present in one or both of the variables. As we have seen, outliers are small numbers of observations at either extreme (small or large) of a sample. The correlation may be quite sensitive to outliers. In such a situation, we should consider whether it makes sense to exclude those outlier observations and whether they are noise or news. As a general rule, we must determine whether a computed sample correlation changes greatly by removing outliers. We must also use judgment to determine whether those outliers contain information about the two variables' relationship (and should thus be included in the correlation analysis) or contain no information (and should thus be excluded). If they are to be excluded from the correlation analysis, as we have seen previously, outlier observations can be handled by trimming or winsorizing the dataset. Importantly, keep in mind that correlation does not imply causation. Even if two variables are highly correlated, one does not necessarily cause the other in the sense that certain values of one variable bring about the occurrence of certain values of the other.

Moreover, with visualizations too, including scatter plots, we must be on guard against unconsciously making judgments about causal relationships that may or may not be supported by the data.

The term spurious correlation has been used to refer to: 1 ) correlation between two variables that reflects chance relationships in a particular dataset; 2) correlation induced by a calculation that mixes each of two variables with a third variable; and 3) correlation between two variables arising not from a direct relation between them but from their relation to a third variable.

As an example of the chance relationship, consider the monthly US retail sales of beer, wine, and liquor and the atmospheric carbon dioxide levels from 2000-2018. The correlation is 0.824 , indicating that there is a positive relation between the two. However, there is no reason to suspect that the levels of atmospheric carbon dioxide are related to the retail sales of beer, wine, and liquor.

As an example of the second kind of spurious correlation, two variables that are uncorrelated may be correlated if divided by a third variable. For example, consider a cross-sectional sample of companies' dividends and total assets. While there may be a low correlation between these two variables, dividing each by market capitalization may increase the correlation.

As an example of the third kind of spurious correlation, height may be positively correlated with the extent of a person's vocabulary, but the underlying relationships are between age and height and between age and vocabulary.

Investment professionals must be cautious in basing investment strategies on high correlations. Spurious correlations may suggest investment strategies that appear profitable but actually would not be, if implemented.

A further issue is that correlation does not tell the whole story about the data. Consider Anscombe's Quartet, discussed in Exhibit 56, where very dissimilar graphs can be developed with variables that have the same mean, same standard deviation, and same correlation.

\section{Exhibit 56: Anscombe's Quartet}
Francis Anscombe, a British statistician, developed datasets that illustrate why just looking at summary statistics (that is, mean, standard deviation, and correlation) does not fully describe the data. He created four datasets (designated I, II, III, and IV), each with two variables, $X$ and $Y$, such that:

\begin{itemize}
  \item The $X \mathrm{~s}$ in each dataset have the same mean and standard deviation, 9.00 and 3.32 , respectively.

  \item The $Y \mathrm{~s}$ in each dataset have the same mean and standard deviation, 7.50 and 2.03, respectively.

  \item The $X \mathrm{~s}$ and $Y \mathrm{~s}$ in each dataset have the same correlation of 0.82 .

\end{itemize}

\begin{center}
\begin{tabular}{lccccccccc}
\hline
 &  & I &  & II & \multicolumn{2}{l}{III} &  & IV \\
\hline
Observation & $\boldsymbol{X}$ & $\boldsymbol{Y}$ & $\boldsymbol{X}$ & $\boldsymbol{Y}$ & $\boldsymbol{X}$ & $\boldsymbol{Y}$ & $\boldsymbol{X}$ & $\boldsymbol{Y}$ \\
\hline
1 & 10 & 8.04 & 10 & 9.14 & 10 & 7.46 & 8 & 6.6 \\
2 & 8 & 6.95 & 8 & 8.14 & 8 & 6.77 & 8 & 5.8 \\
3 & 13 & 7.58 & 13 & 8.74 & 13 & 12.74 & 8 & 7.7 \\
4 & 9 & 8.81 & 9 & 8.77 & 9 & 7.11 & 8 & 8.8 \\
\end{tabular}
\end{center}

\begin{center}
\begin{tabular}{|c|c|c|c|c|c|c|c|c|}
\hline
\multirow[b]{2}{*}{Observation} & \multicolumn{2}{|c|}{I} & \multicolumn{2}{|c|}{II} & \multicolumn{2}{|c|}{III} & \multicolumn{2}{|c|}{IV} \\
\hline
 & $x$ & $Y$ & $X$ & $\boldsymbol{Y}$ & $X$ & $Y$ & $x$ & $Y$ \\
\hline
5 & 11 & 8.33 & 11 & 9.26 & 11 & 7.81 & 8 & 8.5 \\
\hline
6 & 14 & 9.96 & 14 & 8.1 & 14 & 8.84 & 8 & 7 \\
\hline
7 & 6 & 7.24 & 6 & 6.13 & 6 & 6.08 & 8 & 5.3 \\
\hline
8 & 4 & 4.26 & 4 & 3.1 & 4 & 5.39 & 19 & 13 \\
\hline
9 & 12 & 10.8 & 12 & 9.13 & 12 & 8.15 & 8 & 5.6 \\
\hline
10 & 7 & 4.82 & 7 & 7.26 & 7 & 6.42 & 8 & 7.9 \\
\hline
11 & 5 & 5.68 & 5 & 4.74 & 5 & 5.73 & 8 & 6.9 \\
\hline
$N$ & 11 & 11 & 11 & 11 & 11 & 11 & 11 & 11 \\
\hline
Mean & 9.00 & 7.50 & 9.00 & 7.50 & 9.00 & 7.50 & 9.00 & 7.50 \\
\hline
$\begin{array}{l}\text { Standard } \\ \text { deviation }\end{array}$ & 3.32 & 2.03 & 3.32 & 2.03 & 3.32 & 2.03 & 3.32 & 2.03 \\
\hline
Correlation & \multicolumn{2}{|c|}{0.82} & \multicolumn{2}{|c|}{0.82} & \multicolumn{2}{|c|}{0.82} & \multicolumn{2}{|c|}{0.82} \\
\hline
\end{tabular}
\end{center}

While the $X$ variable has the same values for I, II, and III in the quartet of datasets, the $\mathrm{Y}$ variables are quite different, creating different relationships. The four datasets are:

I An approximate linear relationship between $X$ and $Y$.

II A curvilinear relationship between $X$ and $Y$.

III A linear relationship except for one outlier.

IV A constant $X$ with the exception of one outlier.

Depicting the quartet visually,
\includegraphics[max width=\textwidth, center]{2023_05_04_cff39ee44f77d6514e1bg-156}

The bottom line? Knowing the means and standard deviations of the two variables, as well as the correlation between them, does not tell the entire story. Source: Francis John Anscombe, "Graphs in Statistical Analysis," The American Statistician 27 (February 1973): 17-21.

\section{SUMMARY}
In this reading, we have presented tools and techniques for organizing, visualizing, and describing data that permit us to convert raw data into useful information for investment analysis.

\begin{itemize}
  \item Data can be defined as a collection of numbers, characters, words, and text-as well as images, audio, and video-in a raw or organized format to represent facts or information.

  \item From a statistical perspective, data can be classified as numerical data and categorical data. Numerical data (also called quantitative data) are values that represent measured or counted quantities as a number. Categorical data (also called qualitative data) are values that describe a quality or characteristic of a group of observations and usually take only a limited number of values that are mutually exclusive. - Numerical data can be further split into two types: continuous data and discrete data. Continuous data can be measured and can take on any numerical value in a specified range of values. Discrete data are numerical values that result from a counting process and therefore are limited to a finite number of values.

  \item Categorical data can be further classified into two types: nominal data and ordinal data. Nominal data are categorical values that are not amenable to being organized in a logical order, while ordinal data are categorical values that can be logically ordered or ranked.

  \item Based on how they are collected, data can be categorized into three types: cross-sectional, time series, and panel. Time-series data are a sequence of observations for a single observational unit on a specific variable collected over time and at discrete and typically equally spaced intervals of time.

\end{itemize}

Cross-sectional data are a list of the observations of a specific variable from multiple observational units at a given point in time. Panel data are a mix of time-series and cross-sectional data that consists of observations through time on one or more variables for multiple observational units.

\begin{itemize}
  \item Based on whether or not data are in a highly organized form, they can be classified into structured and unstructured types. Structured data are highly organized in a pre-defined manner, usually with repeating patterns. Unstructured data do not follow any conventionally organized forms; they are typically alternative data as they are usually collected from unconventional sources.

  \item Raw data are typically organized into either a one-dimensional array or a two-dimensional rectangular array (also called a data table) for quantitative analysis.

  \item A frequency distribution is a tabular display of data constructed either by counting the observations of a variable by distinct values or groups or by tallying the values of a numerical variable into a set of numerically ordered bins. Frequency distributions permit us to evaluate how data are distributed.

  \item The relative frequency of observations in a bin (interval or bucket) is the number of observations in the bin divided by the total number of observations. The cumulative relative frequency cumulates (adds up) the relative frequencies as we move from the first bin to the last, thus giving the fraction of the observations that are less than the upper limit of each bin.

  \item A contingency table is a tabular format that displays the frequency distributions of two or more categorical variables simultaneously. One application of contingency tables is for evaluating the performance of a classification model (using a confusion matrix). Another application of contingency tables is to investigate a potential association between two categorical variables by performing a chi-square test of independence.

  \item Visualization is the presentation of data in a pictorial or graphical format for the purpose of increasing understanding and for gaining insights into the data.

  \item A histogram is a bar chart of data that have been grouped into a frequency distribution. A frequency polygon is a graph of frequency distributions obtained by drawing straight lines joining successive midpoints of bars representing the class frequencies. - A bar chart is used to plot the frequency distribution of categorical data, with each bar representing a distinct category and the bar's height (or length) proportional to the frequency of the corresponding category. Grouped bar charts or stacked bar charts can present the frequency distribution of multiple categorical variables simultaneously.

  \item A tree-map is a graphical tool to display categorical data. It consists of a set of colored rectangles to represent distinct groups, and the area of each rectangle is proportional to the value of the corresponding group. Additional dimensions of categorical data can be displayed by nested rectangles.

  \item A word cloud is a visual device for representing textual data, with the size of each distinct word being proportional to the frequency with which it appears in the given text.

  \item A line chart is a type of graph used to visualize ordered observations and often to display the change of data series over time. A bubble line chart is a special type of line chart that uses varying-sized bubbles as data points to represent an additional dimension of data.

  \item A scatter plot is a type of graph for visualizing the joint variation in two numerical variables. It is constructed by drawing dots to indicate the values of the two variables plotted against the corresponding axes. A scatter plot matrix organizes scatter plots between pairs of variables into a matrix format to inspect all pairwise relationships between more than two variables in one combined visual.

  \item A heat map is a type of graphic that organizes and summarizes data in a tabular format and represents it using a color spectrum. It is often used in displaying frequency distributions or visualizing the degree of correlation among different variables.

  \item The key consideration when selecting among chart types is the intended purpose of visualizing data (i.e., whether it is for exploring/presenting distributions or relationships or for making comparisons).

  \item A population is defined as all members of a specified group. A sample is a subset of a population.

  \item A parameter is any descriptive measure of a population. A sample statistic (statistic, for short) is a quantity computed from or used to describe a sample.

  \item Sample statistics-such as measures of central tendency, measures of dispersion, skewness, and kurtosis-help with investment analysis, particularly in making probabilistic statements about returns.

  \item Measures of central tendency specify where data are centered and include the mean, median, and mode (i.e., the most frequently occurring value).

  \item The arithmetic mean is the sum of the observations divided by the number of observations. It is the most frequently used measure of central tendency.

  \item The median is the value of the middle item (or the mean of the values of the two middle items) when the items in a set are sorted into ascending or descending order. The median is not influenced by extreme values and is most useful in the case of skewed distributions.

  \item The mode is the most frequently observed value and is the only measure of central tendency that can be used with nominal data. A distribution may be unimodal (one mode), bimodal (two modes), trimodal (three modes), or have even more modes. - A portfolio's return is a weighted mean return computed from the returns on the individual assets, where the weight applied to each asset's return is the fraction of the portfolio invested in that asset.

  \item The geometric mean, $\bar{X}_{G}$, of a set of observations $X_{1}, X_{2}, \ldots, X_{n}$, is $\bar{X}_{G}=$ $\sqrt[n]{X_{1} X_{2} X_{3} \ldots X_{n}}$, with $X_{i} \geq 0$ for $i=1,2, \ldots, n$. The geometric mean is especially important in reporting compound growth rates for time-series data. The geometric mean will always be less than an arithmetic mean whenever there is variance in the observations.

  \item The harmonic mean, $\bar{X}_{H}$, is a type of weighted mean in which an observation's weight is inversely proportional to its magnitude.

  \item Quantiles-such as the median, quartiles, quintiles, deciles, and percentiles-are location parameters that divide a distribution into halves, quarters, fifths, tenths, and hundredths, respectively.

  \item A box and whiskers plot illustrates the interquartile range (the "box") as well as a range outside of the box that is based on the interquartile range, indicated by the "whiskers."

  \item Dispersion measures-such as the range, mean absolute deviation (MAD), variance, standard deviation, target downside deviation, and coefficient of variation-describe the variability of outcomes around the arithmetic mean.

  \item The range is the difference between the maximum value and the minimum value of the dataset. The range has only a limited usefulness because it uses information from only two observations.

  \item The MAD for a sample is the average of the absolute deviations of observations from the mean, $\frac{\sum_{i=1}^{n}\left|X_{i}-\bar{X}\right|}{n}$, where $\bar{X}$ is the sample mean and $n$ is the number of observations in the sample.

  \item The variance is the average of the squared deviations around the mean, and the standard deviation is the positive square root of variance. In computing sample variance $\left(s^{2}\right)$ and sample standard deviation $(s)$, the average squared deviation is computed using a divisor equal to the sample size minus 1 .

  \item The target downside deviation, or target semideviation, is a measure of the risk of being below a given target. It is calculated as the square root of the average squared deviations from the target, but it includes only those obser-

\end{itemize}

\begin{center}
\includegraphics[max width=\textwidth]{2023_05_04_cff39ee44f77d6514e1bg-159}
\end{center}

\begin{itemize}
  \item The coefficient of variation, CV, is the ratio of the standard deviation of a set of observations to their mean value. By expressing the magnitude of variation among observations relative to their average size, the CV permits direct comparisons of dispersion across different datasets. Reflecting the correction for scale, the CV is a scale-free measure (i.e., it has no units of measurement).

  \item Skew or skewness describes the degree to which a distribution is asymmetric about its mean. A return distribution with positive skewness has frequent small losses and a few extreme gains compared to a normal distribution. A return distribution with negative skewness has frequent small gains and a few extreme losses compared to a normal distribution. Zero skewness indicates a symmetric distribution of returns.

  \item Kurtosis measures the combined weight of the tails of a distribution relative to the rest of the distribution. A distribution with fatter tails than the normal distribution is referred to as fat-tailed (leptokurtic); a distribution with thinner tails than the normal distribution is referred to as thin-tailed (platykurtic). Excess kurtosis is kurtosis minus 3, since 3 is the value of kurtosis for all normal distributions.

  \item The correlation coefficient is a statistic that measures the association between two variables. It is the ratio of covariance to the product of the two variables' standard deviations. A positive correlation coefficient indicates that the two variables tend to move together, whereas a negative coefficient indicates that the two variables tend to move in opposite directions. Correlation does not imply causation, simply association. Issues that arise in evaluating correlation include the presence of outliers and spurious correlation.

\end{itemize}

\section{PRACTICE PROBLEMS}
\begin{enumerate}
  \item Published ratings on stocks ranging from 1 (strong sell) to 5 (strong buy) are examples of which measurement scale?
A. Ordinal
B. Continuous
C. Nominal

  \item Data values that are categorical and not amenable to being organized in a logical order are most likely to be characterized as:
A. ordinal data.
B. discrete data.
C. nominal data.

  \item Which of the following data types would be classified as being categorical?
A. Discrete
B. Nominal
C. Continuous

  \item A fixed-income analyst uses a proprietary model to estimate bankruptcy probabilities for a group of firms. The model generates probabilities that can take any value between 0 and 1 . The resulting set of estimated probabilities would most likely be characterized as:
A. ordinal data.
B. discrete data.
C. continuous data.

  \item An analyst uses a software program to analyze unstructured data-specifically, management's earnings call transcript for one of the companies in her research coverage. The program scans the words in each sentence of the transcript and then classifies the sentences as having negative, neutral, or positive sentiment. The resulting set of sentiment data would most likely be characterized as:
A. ordinal data.
B. discrete data.
C. nominal data.

\end{enumerate}

\section{The following information relates to questions}
6-7

An equity analyst gathers total returns for three country equity indexes over the past four years. The data are presented below.

\begin{center}
\begin{tabular}{lccc}
\hline
Time Period & Index A & Index B & Index C \\
\hline
Year $\boldsymbol{t}-\mathbf{3}$ & $15.56 \%$ & $11.84 \%$ & $-4.34 \%$ \\
Year $\mathbf{t}-\mathbf{2}$ & $-4.12 \%$ & $-6.96 \%$ & $9.32 \%$ \\
Year $\boldsymbol{t}-\mathbf{1}$ & $11.19 \%$ & $10.29 \%$ & $-12.72 \%$ \\
Year $\boldsymbol{t}$ & $8.98 \%$ & $6.32 \%$ & $21.44 \%$ \\
\hline
\end{tabular}
\end{center}

\begin{enumerate}
  \setcounter{enumi}{5}
  \item Each individual column of data in the table can be best characterized as:
A. panel data.
B. time-series data.
C. cross-sectional data.

  \item Each individual row of data in the table can be best characterized as:
A. panel data.
B. time-series data.
C. cross-sectional data.

  \item A two-dimensional rectangular array would be most suitable for organizing a collection of raw:
A. panel data.
B. time-series data.
C. cross-sectional data.

  \item In a frequency distribution, the absolute frequency measure:

\end{enumerate}

A. represents the percentages of each unique value of the variable.

B. represents the actual number of observations counted for each unique value of the variable.

C. allows for comparisons between datasets with different numbers of total observations.

\begin{enumerate}
  \setcounter{enumi}{9}
  \item An investment fund has the return frequency distribution shown in the following exhibit.
\end{enumerate}

\begin{center}
\begin{tabular}{lc}
\hline
Return Interval (\%) & Absolute Frequency \\
\hline
-10.0 to -7.0 & 3 \\
-7.0 to -4.0 & 7 \\
-4.0 to -1.0 & 10 \\
-1.0 to +2.0 & 12 \\
+2.0 to +5.0 & 23 \\
+5.0 to +8.0 & 5 \\
\hline
\end{tabular}
\end{center}

Which of the following statements is correct?
A. The relative frequency of the bin " -1.0 to +2.0 " is $20 \%$.
B. The relative frequency of the bin " +2.0 to +5.0 " is $23 \%$.
C. The cumulative relative frequency of the bin “ +5.0 to +8.0 " is $91.7 \%$.

\begin{enumerate}
  \setcounter{enumi}{10}
  \item An analyst is using the data in the following exhibit to prepare a statistical report.
\end{enumerate}

\section{Portfolio's Deviations from Benchmark Return for a 12-Year Period (\%)}
\begin{center}
\begin{tabular}{lccc}
\hline
Year 1 & 2.48 & Year 7 & -9.19 \\
Year 2 & -2.59 & Year 8 & -5.11 \\
Year 3 & 9.47 & Year 9 & 1.33 \\
Year 4 & -0.55 & Year 10 & 6.84 \\
Year 5 & -1.69 & Year 11 & 3.04 \\
Year 6 & -0.89 & Year 12 & 4.72 \\
\hline
\end{tabular}
\end{center}

The cumulative relative frequency for the bin $-1.71 \% \leq x<2.03 \%$ is closest to:
A. 0.250 .
B. 0.333 .
C. 0.583 .

\section{The following information relates to questions}
 12-13A fixed-income portfolio manager creates a contingency table of the number of bonds held in her portfolio by sector and bond rating. The contingency table is presented here:

\begin{center}
\begin{tabular}{lccc}
\hline
 &  & Bond Rating &  \\
\hline
Sector & A & AA & AAA \\
\hline
Communication Services & 25 & 32 & 27 \\
Consumer Staples & 30 & 25 & 25 \\
Energy & 100 & 85 & 30 \\
Health Care & 200 & 100 & 63 \\
Utilities & 22 & 28 & 14 \\
\hline
\end{tabular}
\end{center}

\begin{enumerate}
  \setcounter{enumi}{11}
  \item The marginal frequency of energy sector bonds is closest to:
A. 27 .
B. 85 .
C. 215 .

  \item The relative frequency of AA rated energy bonds, based on the total count, is closest to:
A. $10.5 \%$
B. $31.5 \%$.
C. $39.5 \%$.

  \item The following is a frequency polygon of monthly exchange rate changes in the US dollar/Japanese yen spot exchange rate for a four-year period. A positive change represents yen appreciation (the yen buys more dollars), and a negative change represents yen depreciation (the yen buys fewer dollars).

\end{enumerate}

Exhibit 1: Monthly Changes in the US Dollar/Japanese Yen Spot Exchange Rate

\begin{center}
\includegraphics[max width=\textwidth]{2023_05_04_cff39ee44f77d6514e1bg-164}
\end{center}

Based on the chart, yen appreciation:

A. occurred more than $50 \%$ of the time.

B. was less frequent than yen depreciation.

C. in the 0.0 to 2.0 interval occurred $20 \%$ of the time.

\begin{enumerate}
  \setcounter{enumi}{14}
  \item A bar chart that orders categories by frequency in descending order and includes a line displaying cumulative relative frequency is referred to as a:
A. Pareto Chart.
B. grouped bar chart.
C. frequency polygon.

  \item Which visualization tool works best to represent unstructured, textual data?
A. Tree-Map
B. Scatter plot C. Word cloud

  \item A tree-map is best suited to illustrate:

\end{enumerate}

A. underlying trends over time.

B. joint variations in two variables.

C. value differences of categorical groups.

\begin{enumerate}
  \setcounter{enumi}{17}
  \item A line chart with two variables-for example, revenues and earnings per shareis best suited for visualizing:
A. the joint variation in the variables.
B. underlying trends in the variables over time.
C. the degree of correlation between the variables.

  \item A heat map is best suited for visualizing the:
A. frequency of textual data.
B. degree of correlation between different variables.
C. shape, center, and spread of the distribution of numerical data.

  \item Which valuation tool is recommended to be used if the goal is to make comparisons of three or more variables over time?
A. Heat map
B. Bubble line chart
C. Scatter plot matrix

\end{enumerate}

\section{The following information relates to questions 21-22}
The following histogram shows a distribution of the S\&P 500 Index annual returns for a 50-year period:

\begin{center}
\includegraphics[max width=\textwidth]{2023_05_04_cff39ee44f77d6514e1bg-166}
\end{center}

\begin{enumerate}
  \setcounter{enumi}{20}
  \item The bin containing the median return is:
A. $3 \%$ to $8 \%$.
B. $8 \%$ to $13 \%$.
C. $13 \%$ to $18 \%$.

  \item Based on the previous histogram, the distribution is best described as being:
A. unimodal.
B. bimodal.
C. trimodal.

  \item The annual returns for three portfolios are shown in the following exhibit. Portfolios $\mathrm{P}$ and $\mathrm{R}$ were created in Year 1, Portfolio $\mathrm{Q}$ in Year 2.

\end{enumerate}

\begin{center}
\begin{tabular}{lccccc}
\hline
 & \multicolumn{4}{c}{Annual Portfolio Returns (\%)} &  \\
\hline
 & Year 1 & Year 2 & Year 3 & Year 4 & Year 5 \\
\hline
Portfolio P & -3.0 & 4.0 & 5.0 & 3.0 & 7.0 \\
Portfolio Q &  & -3.0 & 6.0 & 4.0 & 8.0 \\
Portfolio R & 1.0 & -1.0 & 4.0 & 4.0 & 3.0 \\
\hline
\end{tabular}
\end{center}

The median annual return from portfolio creation to Year 5 for:

A. Portfolio $\mathrm{P}$ is $4.5 \%$.

B. Portfolio $\mathrm{Q}$ is $4.0 \%$. C. Portfolio $\mathrm{R}$ is higher than its arithmetic mean annual return.

\begin{enumerate}
  \setcounter{enumi}{23}
  \item At the beginning of Year $X$, an investor allocated his retirement savings in the asset classes shown in the following exhibit and earned a return for Year $\mathrm{X}$ as also shown.
\end{enumerate}

\begin{center}
\begin{tabular}{lcc}
\hline
Asset Class & $\begin{array}{c}\text { Asset Allocation } \\ \text { (\%) }\end{array}$ & Asset Class Return for Year $\mathbf{X}$ (\%) \\
\hline
Large-cap US equities & 20.0 & 8.0 \\
Small-cap US equities & 40.0 & 12.0 \\
Emerging market equities & 25.0 & -3.0 \\
High-yield bonds & 15.0 & 4.0 \\
\hline
\end{tabular}
\end{center}

The portfolio return for Year $\mathrm{X}$ is closest to:
A. $5.1 \%$.
B. $5.3 \%$.
C. $6.3 \%$.

\begin{enumerate}
  \setcounter{enumi}{24}
  \item The following exhibit shows the annual returns for Fund Y.
\end{enumerate}

\begin{center}
\begin{tabular}{lc}
\hline
 & Fund $\mathbf{Y}(\%)$ \\
\hline
Year 1 & 19.5 \\
Year 2 & -1.9 \\
Year 3 & 19.7 \\
Year 4 & 35.0 \\
Year 5 & 5.7 \\
\hline
\end{tabular}
\end{center}

The geometric mean return for Fund $\mathrm{Y}$ is closest to:
A. $14.9 \%$.
B. $15.6 \%$.
C. $19.5 \%$.

\begin{enumerate}
  \setcounter{enumi}{25}
  \item A portfolio manager invests $€ 5,000$ annually in a security for four years at the prices shown in the following exhibit.
\end{enumerate}

\begin{center}
\begin{tabular}{lc}
\hline
 & Purchase Price of Security ( $€$ per unit) \\
\hline
Year 1 & 62.00 \\
Year 2 & 76.00 \\
Year 3 & 84.00 \\
Year 4 & 90.00 \\
\hline
\end{tabular}
\end{center}

The average price is best represented as the:
A. harmonic mean of $€ 76.48$.
B. geometric mean of $€ 77.26$.
C. arithmetic average of $€ 78.00$. 27. When analyzing investment returns, which of the following statements is correct?

A. The geometric mean will exceed the arithmetic mean for a series with non-zero variance.

B. The geometric mean measures an investment's compound rate of growth over multiple periods.

C. The arithmetic mean measures an investment's terminal value over multiple periods.

\section{The following information relates to questions 28-32}
A fund had the following experience over the past 10 years:

\begin{center}
\begin{tabular}{lc}
\hline
Year & Return \\
\hline
1 & $4.5 \%$ \\
2 & $6.0 \%$ \\
3 & $1.5 \%$ \\
4 & $-2.0 \%$ \\
5 & $0.0 \%$ \\
6 & $4.5 \%$ \\
7 & $3.5 \%$ \\
8 & $2.5 \%$ \\
9 & $5.5 \%$ \\
10 & $4.0 \%$ \\
\hline
\end{tabular}
\end{center}

\begin{enumerate}
  \setcounter{enumi}{27}
  \item The arithmetic mean return over the 10 years is closest to:
A. $2.97 \%$.
B. $3.00 \%$.
C. $3.33 \%$.

  \item The geometric mean return over the 10 years is closest to:
A. $2.94 \%$
B. $2.97 \%$.
C. $3.00 \%$.

  \item The harmonic mean return over the 10 years is closest to:
A. $2.94 \%$.
B. $2.97 \%$
C. $3.00 \%$. 31. The standard deviation of the 10 years of returns is closest to:
A. $2.40 \%$.
B. $2.53 \%$.
C. $7.58 \%$.

  \item The target semideviation of the returns over the 10 years if the target is $2 \%$ is closest to:
A. $1.42 \%$.
B. $1.50 \%$.
C. $2.01 \%$.

\end{enumerate}

\section{The following information relates to questions}
\section{3-34}
\begin{center}
\includegraphics[max width=\textwidth]{2023_05_04_cff39ee44f77d6514e1bg-169}
\end{center}

\begin{enumerate}
  \setcounter{enumi}{32}
  \item The median is closest to:
A. 34.51 .
B. 100.49 .
C. 102.98 .

  \item The interquartile range is closest to:
A. 13.76 .
B. 25.74 .
C. 34.51 .

\end{enumerate}

\section{The following information relates to questions 35-36}
The following exhibit shows the annual MSCI World Index total returns for a 10 -year period.

\begin{center}
\begin{tabular}{lccc}
\hline
Year 1 & $15.25 \%$ & Year 6 & $30.79 \%$ \\
Year 2 & $10.02 \%$ & Year 7 & $12.34 \%$ \\
Year 3 & $20.65 \%$ & Year 8 & $-5.02 \%$ \\
Year 4 & $9.57 \%$ & Year 9 & $16.54 \%$ \\
Year 5 & $-40.33 \%$ & Year 10 & $27.37 \%$ \\
\hline
\end{tabular}
\end{center}

\begin{enumerate}
  \setcounter{enumi}{34}
  \item The fourth quintile return for the MSCI World Index is closest to:
A. $20.65 \%$.
B. $26.03 \%$.
C. $27.37 \%$.

  \item For Year 6-Year 10, the mean absolute deviation of the MSCI World Index total returns is closest to:
A. $10.20 \%$.
B. $12.74 \%$.
C. $16.40 \%$.

  \item Annual returns and summary statistics for three funds are listed in the following exhibit:

\end{enumerate}

\begin{center}
\begin{tabular}{lccc}
\hline
 & \multicolumn{3}{c}{Annual Returns (\%)} \\
\hline
Year & Fund ABC & Fund XYZ & Fund PQR \\
\hline
Year 1 & -20.0 & -33.0 & -14.0 \\
Year 2 & 23.0 & -12.0 & -18.0 \\
Year 3 & -14.0 & -12.0 & 6.0 \\
Year 4 & 5.0 & -8.0 & -2.0 \\
Year 5 & -14.0 & 11.0 & 3.0 \\
Mean &  &  &  \\
Standard deviation & -4.0 & -10.8 & -5.0 \\
\hline
\end{tabular}
\end{center}

The fund with the highest absolute dispersion is:

A. Fund $\mathrm{PQR}$ if the measure of dispersion is the range.

B. Fund XYZ if the measure of dispersion is the variance.

C. Fund $A B C$ if the measure of dispersion is the mean absolute deviation.

\begin{enumerate}
  \setcounter{enumi}{37}
  \item The average return for Portfolio A over the past twelve months is $3 \%$, with a stan- dard deviation of $4 \%$. The average return for Portfolio B over this same period is also $3 \%$, but with a standard deviation of $6 \%$. The geometric mean return of Portfolio $\mathrm{A}$ is $2.85 \%$. The geometric mean return of Portfolio B is:
\end{enumerate}

A. less than $2.85 \%$.

B. equal to $2.85 \%$.

C. greater than $2.85 \%$.

\begin{enumerate}
  \setcounter{enumi}{38}
  \item The mean monthly return and the standard deviation for three industry sectors are shown in the following exhibit.
\end{enumerate}

Standard Deviation of Return

Sector

Mean Monthly Return (\%)

(\%)

\begin{center}
\begin{tabular}{lll}
\hline
Utilities (UTIL) & 2.10 & 1.23 \\
Materials (MATR) & 1.25 & 1.35 \\
Industrials (INDU) & 3.01 & 1.52 \\
\hline
\end{tabular}
\end{center}

Based on the coefficient of variation, the riskiest sector is:

A. utilities.

B. materials.

C. industrials.

\section{The following information relates to questions}
 40-42An analyst examined a cross-section of annual returns for 252 stocks and calculated the following statistics:

\begin{center}
\begin{tabular}{lc}
Arithmetic Average & $9.986 \%$ \\
Geometric Mean & $9.909 \%$ \\
Variance & 0.001723 \\
Skewness & 0.704 \\
Excess Kurtosis & 0.503 \\
\hline
\end{tabular}
\end{center}

\begin{enumerate}
  \setcounter{enumi}{39}
  \item The coefficient of variation is closest to:
A. 0.02 .
B. 0.42
C. 2.41

  \item This distribution is best described as:
A. negatively skewed.
B. having no skewness.
C. positively skewed. 42. Compared to the normal distribution, this sample's distribution is best described as having tails of the distribution with:
A. less probability than the normal distribution.
B. the same probability as the normal distribution.
C. more probability than the normal distribution.

  \item An analyst calculated the excess kurtosis of a stock's returns as -0.75 . From this information, we conclude that the distribution of returns is:
A. normally distributed.
B. thin-tailed compared to the normal distribution.
C. fat-tailed compared to the normal distribution.

  \item A correlation of 0.34 between two variables, $X$ and $Y$, is best described as:

\end{enumerate}

A. changes in $X$ causing changes in $Y$.

B. a positive association between $X$ and $Y$.

C. a curvilinear relationship between $X$ and $Y$.

\begin{enumerate}
  \setcounter{enumi}{44}
  \item Which of the following is a potential problem with interpreting a correlation coefficient?
\end{enumerate}

A. Outliers

B. Spurious correlation

C. Both outliers and spurious correlation

The following information relates to questions 46-47

An analyst is evaluating the tendency of returns on the portfolio of stocks she manages to move along with bond and real estate indexes. She gathered monthly data on returns and the indexes:

\begin{center}
\begin{tabular}{lccc}
\hline
 & Returns (\%) &  &  \\
\hline
Portfolio Returns & $\begin{array}{c}\text { Bond Index } \\ \text { Returns }\end{array}$ & $\begin{array}{c}\text { Real Estate Index } \\ \text { Returns }\end{array}$ &  \\
\hline
Arithmetic average & 5.5 & 3.2 & 7.8 \\
Standard deviation & 8.2 & 3.4 & 10.3 \\
\hline
\end{tabular}
\end{center}

Portfolio Returns and

Bond Index Returns Portfolio Returns and Real Estate Index Returns 46. Without calculating the correlation coefficient, the correlation of the portfolio returns and the bond index returns is:
A. negative.
B. zero.
C. positive.

\begin{enumerate}
  \setcounter{enumi}{46}
  \item Without calculating the correlation coefficient, the correlation of the portfolio returns and the real estate index returns is:
A. negative.
B. zero.
C. positive.

  \item Consider two variables, $A$ and $B$. If variable $A$ has a mean of -0.56 , variable $B$ has a mean of 0.23 , and the covariance between the two variables is positive, the correlation between these two variables is:
A. negative.
B. zero.
C. positive.

\end{enumerate}

\section{SOLUTIONS}
\begin{enumerate}
  \item A is correct. Ordinal scales sort data into categories that are ordered with respect to some characteristic and may involve numbers to identify categories but do not assure that the differences between scale values are equal. The buy rating scale indicates that a stock ranked 5 is expected to perform better than a stock ranked 4, but it tells us nothing about the performance difference between stocks ranked 4 and 5 compared with the performance difference between stocks ranked 1 and 2 , and so on.

  \item C is correct. Nominal data are categorical values that are not amenable to being organized in a logical order. A is incorrect because ordinal data are categorical data that can be logically ordered or ranked. B is incorrect because discrete data are numerical values that result from a counting process; thus, they can be ordered in various ways, such as from highest to lowest value.

  \item B is correct. Categorical data (or qualitative data) are values that describe a quality or characteristic of a group of observations and therefore can be used as labels to divide a dataset into groups to summarize and visualize. The two types of categorical data are nominal data and ordinal data. Nominal data are categorical values that are not amenable to being organized in a logical order, while ordinal data are categorical values that can be logically ordered or ranked. A is incorrect because discrete data would be classified as numerical data (not categorical data). $\mathrm{C}$ is incorrect because continuous data would be classified as numerical data (not categorical data).

  \item C is correct. Continuous data are data that can be measured and can take on any numerical value in a specified range of values. In this case, the analyst is estimating bankruptcy probabilities, which can take on any value between 0 and 1 . Therefore, the set of bankruptcy probabilities estimated by the analyst would likely be characterized as continuous data. A is incorrect because ordinal data are categorical values that can be logically ordered or ranked. Therefore, the set of bankruptcy probabilities would not be characterized as ordinal data. B is incorrect because discrete data are numerical values that result from a counting process, and therefore the data are limited to a finite number of values. The proprietary model used can generate probabilities that can take any value between 0 and 1; therefore, the set of bankruptcy probabilities would not be characterized as discrete data.

  \item A is correct. Ordinal data are categorical values that can be logically ordered or ranked. In this case, the classification of sentences in the earnings call transcript into three categories (negative, neutral, or positive) describes ordinal data, as the data can be logically ordered from positive to negative. B is incorrect because discrete data are numerical values that result from a counting process. In this case, the analyst is categorizing sentences (i.e., unstructured data) from the earnings call transcript as having negative, neutral, or positive sentiment. Thus, these categorical data do not represent discrete data. $\mathrm{C}$ is incorrect because nominal data are categorical values that are not amenable to being organized in a logical order. In this case, the classification of unstructured data (i.e., sentences from the earnings call transcript) into three categories (negative, neutral, or positive) describes ordinal (not nominal) data, as the data can be logically ordered from positive to negative.

  \item B is correct. Time-series data are a sequence of observations of a specific variable collected over time and at discrete and typically equally spaced intervals of time, such as daily, weekly, monthly, annually, and quarterly. In this case, each column is a time series of data that represents annual total return (the specific variable) for a given country index, and it is measured annually (the discrete interval of time). A is incorrect because panel data consist of observations through time on one or more variables for multiple observational units. The entire table of data is an example of panel data showing annual total returns (the variable) for three country indexes (the observational units) by year. $\mathrm{C}$ is incorrect because cross-sectional data are a list of the observations of a specific variable from multiple observational units at a given point in time. Each row (not column) of data in the table represents cross-sectional data.

  \item C is correct. Cross-sectional data are observations of a specific variable from multiple observational units at a given point in time. Each row of data in the table represents cross-sectional data. The specific variable is annual total return, the multiple observational units are the three countries' indexes, and the given point in time is the time period indicated by the particular row. A is incorrect because panel data consist of observations through time on one or more variables for multiple observational units. The entire table of data is an example of panel data showing annual total returns (the variable) for three country indexes (the observational units) by year. B is incorrect because time-series data are a sequence of observations of a specific variable collected over time and at discrete and typically equally spaced intervals of time, such as daily, weekly, monthly, annually, and quarterly. In this case, each column (not row) is a time series of data that represents annual total return (the specific variable) for a given country index, and it is measured annually (the discrete interval of time).

  \item A is correct. Panel data consist of observations through time on one or more variables for multiple observational units. A two-dimensional rectangular array, or data table, would be suitable here as it is comprised of columns to hold the variable(s) for the observational units and rows to hold the observations through time. $\mathrm{B}$ is incorrect because a one-dimensional (not a two-dimensional rectangular) array would be most suitable for organizing a collection of data of the same data type, such as the time-series data from a single variable. $\mathrm{C}$ is incorrect because a one-dimensional (not a two-dimensional rectangular) array would be most suitable for organizing a collection of data of the same data type, such as the same variable for multiple observational units at a given point in time (cross-sectional data).

  \item B is correct. In a frequency distribution, the absolute frequency, or simply the raw frequency, is the actual number of observations counted for each unique value of the variable. A is incorrect because the relative frequency, which is calculated as the absolute frequency of each unique value of the variable divided by the total number of observations, presents the absolute frequencies in terms of percentages. $\mathrm{C}$ is incorrect because the relative (not absolute) frequency provides a normalized measure of the distribution of the data, allowing comparisons between datasets with different numbers of total observations.

  \item A is correct. The relative frequency is the absolute frequency of each bin divided by the total number of observations. Here, the relative frequency is calculated as: $(12 / 60) \times 100=20 \%$. B is incorrect because the relative frequency of this bin is $(23 / 60) \times 100=38.33 \%$. C is incorrect because the cumulative relative frequency of the last bin must equal 100\%.

  \item $C$ is correct. The cumulative relative frequency of a bin identifies the fraction of observations that are less than the upper limit of the given bin. It is determined by summing the relative frequencies from the lowest bin up to and including the given bin. The following exhibit shows the relative frequencies for all the bins of the data from the previous exhibit:

\end{enumerate}

\begin{center}
\begin{tabular}{ccccc}
\hline
$\begin{array}{c}\text { Lower Limit } \\ \text { (\%) }\end{array}$ & $\begin{array}{c}\text { Upper Limit } \\ (\%)\end{array}$ & $\begin{array}{c}\text { Absolute } \\ \text { Frequency }\end{array}$ & $\begin{array}{c}\text { Relative } \\ \text { Frequency }\end{array}$ & $\begin{array}{c}\text { Cumulative Relative } \\ \text { Frequency }\end{array}$ \\
\hline
$-9.19 \leq$ & $<-5.45$ & 1 & 0.083 & 0.083 \\
$-5.45 \leq$ & $<-1.71$ & 2 & 0.167 & 0.250 \\
$-1.71 \leq$ & $<2.03$ & 4 & 0.333 & 0.583 \\
$2.03 \leq$ & $<5.77$ & 3 & 0.250 & 0.833 \\
$5.77 \leq$ & $\leq 9.47$ & 2 & 0.167 & 1.000 \\
\hline
\end{tabular}
\end{center}

The bin $-1.71 \% \leq x<2.03 \%$ has a cumulative relative frequency of 0.583 .

\begin{enumerate}
  \setcounter{enumi}{11}
  \item $\mathrm{C}$ is correct. The marginal frequency of energy sector bonds in the portfolio is the sum of the joint frequencies across all three levels of bond rating, so 100 $+85+30=215$. A is incorrect because 27 is the relative frequency for energy sector bonds based on the total count of 806 bonds, so $215 / 806=26.7 \%$, not the marginal frequency. B is incorrect because 85 is the joint frequency for AA rated energy sector bonds, not the marginal frequency.

  \item A is correct. The relative frequency for any value in the table based on the total count is calculated by dividing that value by the total count. Therefore, the relative frequency for AA rated energy bonds is calculated as $85 / 806=10.5 \%$.

\end{enumerate}

$B$ is incorrect because $31.5 \%$ is the relative frequency for AA rated energy bonds, calculated based on the marginal frequency for all AA rated bonds, so $85 /(32+$ $25+85+100+28$ ), not based on total bond counts. $C$ is incorrect because $39.5 \%$ is the relative frequency for AA rated energy bonds, calculated based on the marginal frequency for all energy bonds, so $85 /(100+85+30)$, not based on total bond counts.

\begin{enumerate}
  \setcounter{enumi}{13}
  \item A is correct. Twenty observations lie in the interval " 0.0 to 2.0," and six observations lie in the "2.0 to 4.0 " interval. Together, they represent $26 / 48$, or $54.17 \%$, of all observations, which is more than $50 \%$.

  \item A is correct. A bar chart that orders categories by frequency in descending order and includes a line displaying cumulative relative frequency is called a Pareto Chart. A Pareto Chart is used to highlight dominant categories or the most important groups. B is incorrect because a grouped bar chart or clustered bar chart is used to present the frequency distribution of two categorical variables. $\mathrm{C}$ is incorrect because a frequency polygon is used to display frequency distributions.

  \item $\mathrm{C}$ is correct. A word cloud, or tag cloud, is a visual device for representing unstructured, textual data. It consists of words extracted from text with the size of each word being proportional to the frequency with which it appears in the given text. A is incorrect because a tree-map is a graphical tool for displaying and comparing categorical data, not for visualizing unstructured, textual data. B is incorrect because a scatter plot is used to visualize the joint variation in two numerical variables, not for visualizing unstructured, textual data.

  \item $\mathrm{C}$ is correct. A tree-map is a graphical tool used to display and compare categorical data. It consists of a set of colored rectangles to represent distinct groups, and the area of each rectangle is proportional to the value of the corresponding group. A is incorrect because a line chart, not a tree-map, is used to display the change in a data series over time. $B$ is incorrect because a scatter plot, not a tree-map, is used to visualize the joint variation in two numerical variables.

  \item B is correct. An important benefit of a line chart is that it facilitates showing changes in the data and underlying trends in a clear and concise way. Often a line chart is used to display the changes in data series over time. A is incorrect because a scatter plot, not a line chart, is used to visualize the joint variation in two numerical variables. $C$ is incorrect because a heat map, not a line chart, is used to visualize the values of joint frequencies among categorical variables.

  \item B is correct. A heat map is commonly used for visualizing the degree of correlation between different variables. A is incorrect because a word cloud, or tag cloud, not a heat map, is a visual device for representing textual data with the size of each distinct word being proportional to the frequency with which it appears in the given text. $\mathrm{C}$ is incorrect because a histogram, not a heat map, depicts the shape, center, and spread of the distribution of numerical data.

  \item B is correct. A bubble line chart is a version of a line chart where data points are replaced with varying-sized bubbles to represent a third dimension of the data. A line chart is very effective at visualizing trends in three or more variables over time. A is incorrect because a heat map differentiates high values from low values and reflects the correlation between variables but does not help in making comparisons of variables over time. $\mathrm{C}$ is incorrect because a scatterplot matrix is a useful tool for organizing scatterplots between pairs of variables, making it easy to inspect all pairwise relationships in one combined visual. However, it does not help in making comparisons of these variables over time.

  \item $C$ is correct. Because 50 data points are in the histogram, the median return would be the mean of the $50 / 2=25$ th and $(50+2) / 2=26$ th positions. The sum of the return bin frequencies to the left of the $13 \%$ to $18 \%$ interval is 24 . As a result, the 25th and 26th returns will fall in the $13 \%$ to $18 \%$ interval.

  \item $\mathrm{C}$ is correct. The mode of a distribution with data grouped in intervals is the interval with the highest frequency. The three intervals of $3 \%$ to $8 \%, 18 \%$ to $23 \%$, and $28 \%$ to $33 \%$ all have a high frequency of 7.

  \item $\mathrm{C}$ is correct. The median of Portfolio $\mathrm{R}$ is $0.8 \%$ higher than the mean for Portfolio R.

  \item C is correct. The portfolio return must be calculated as the weighted mean return, where the weights are the allocations in each asset class:

\end{enumerate}

$(0.20 \times 8 \%)+(0.40 \times 12 \%)+(0.25 \times-3 \%)+(0.15 \times 4 \%)=6.25 \%$, or $\approx 6.3 \%$.

\begin{enumerate}
  \setcounter{enumi}{24}
  \item A is correct. The geometric mean return for Fund $Y$ is found as follows:
\end{enumerate}

Fund $Y=[(1+0.195) \times(1-0.019) \times(1+0.197) \times(1+0.350) \times(1+0.057)]$

$(1 / 5)-1$

$=14.9 \%$.

\begin{enumerate}
  \setcounter{enumi}{25}
  \item A is correct. The harmonic mean is appropriate for determining the average price per unit. It is calculated by summing the reciprocals of the prices, then averaging that sum by dividing by the number of prices, then taking the reciprocal of the average:
\end{enumerate}

$4 /[(1 / 62.00)+(1 / 76.00)+(1 / 84.00)+(1 / 90.00)]=€ 76.48$

\begin{enumerate}
  \setcounter{enumi}{26}
  \item B is correct. The geometric mean compounds the periodic returns of every period, giving the investor a more accurate measure of the terminal value of an investment. 28. B is correct. The sum of the returns is $30.0 \%$, so the arithmetic mean is $30.0 \% / 10$ $=3.0 \%$

  \item B is correct.

\end{enumerate}

\begin{center}
\begin{tabular}{lcc}
\hline
Year & Return & 1+ Return \\
\hline
1 & $4.5 \%$ & 1.045 \\
2 & $6.0 \%$ & 1.060 \\
3 & $1.5 \%$ & 1.015 \\
4 & $-2.0 \%$ & 0.980 \\
5 & $0.0 \%$ & 1.000 \\
6 & $4.5 \%$ & 1.045 \\
7 & $3.5 \%$ & 1.035 \\
8 & $2.5 \%$ & 1.025 \\
9 & $5.5 \%$ & 1.055 \\
10 & $4.0 \%$ & 1.040 \\
\hline
\end{tabular}
\end{center}

The product of the $1+$ Return is 1.3402338 .

Therefore, $\bar{X}_{G}=\sqrt[10]{1.3402338}-1=2.9717 \%$.

\begin{enumerate}
  \setcounter{enumi}{29}
  \item A is correct.
\end{enumerate}

\begin{center}
\begin{tabular}{lccc}
\hline
Year & Return & 1+ Return & $\mathbf{1 / ( 1 + R e t u r n ) ~}$ \\
\hline
1 & $4.5 \%$ & 1.045 & 0.957 \\
2 & $6.0 \%$ & 1.060 & 0.943 \\
3 & $1.5 \%$ & 1.015 & 0.985 \\
4 & $-2.0 \%$ & 0.980 & 1.020 \\
5 & $0.0 \%$ & 1.000 & 1.000 \\
6 & $4.5 \%$ & 1.045 & 0.957 \\
7 & $3.5 \%$ & 1.035 & 0.966 \\
8 & $2.5 \%$ & 1.025 & 0.976 \\
9 & $5.5 \%$ & 1.055 & 0.948 \\
10 & $4.0 \%$ & 1.040 & 0.962 \\
Sum &  & $\mathbf{9} \% .714$ &  \\
\hline
\end{tabular}
\end{center}

The harmonic mean return $=(n /$ Sum of reciprocals $)-1=(10 / 9.714)-1$.

The harmonic mean return $=2.9442 \%$.

\begin{enumerate}
  \setcounter{enumi}{30}
  \item B is correct.
\end{enumerate}

\begin{center}
\begin{tabular}{lccc}
\hline
Year & Return & Deviation & Deviation Squared \\
\hline
1 & $4.5 \%$ & 0.0150 & 0.000225 \\
2 & $6.0 \%$ & 0.0300 & 0.000900 \\
3 & $1.5 \%$ & -0.0150 & 0.000225 \\
4 & $-2.0 \%$ & -0.0500 & 0.002500 \\
5 & $0.0 \%$ & -0.0300 & 0.000900 \\
6 & $4.5 \%$ & 0.0150 & 0.000225 \\
\end{tabular}
\end{center}

\begin{center}
\begin{tabular}{lccc}
\hline
Year & Return & Deviation & Deviation Squared \\
\hline
7 & $3.5 \%$ & 0.0050 & 0.000025 \\
8 & $2.5 \%$ & -0.0050 & 0.000025 \\
9 & $5.5 \%$ & 0.0250 & 0.000625 \\
10 & $4.0 \%$ & $\underline{0.0100}$ & $\underline{0.000100}$ \\
Sum & 0.0000 & $\mathbf{0 . 0 0 5 7 5 0}$ &  \\
\hline
\end{tabular}
\end{center}

The standard deviation is the square root of the sum of the squared deviations divided by $n-1$ :

$s=\sqrt{\frac{0.005750}{9}}=2.5276 \%$.

\begin{enumerate}
  \setcounter{enumi}{31}
  \item B is correct.
\end{enumerate}

\begin{center}
\begin{tabular}{lcc}
\hline
Year & Return & $\begin{array}{r}\text { Deviation Squared } \\ \text { below Target of 2\% }\end{array}$ \\
\hline
1 & $4.5 \%$ &  \\
2 & $6.0 \%$ & 0.000025 \\
3 & $1.5 \%$ & 0.001600 \\
4 & $-2.0 \%$ & 0.000400 \\
5 & $0.0 \%$ &  \\
6 & $4.5 \%$ &  \\
7 & $3.5 \%$ &  \\
8 & $2.5 \%$ &  \\
9 & $5.5 \%$ & 0.002025 \\
10 & $4.0 \%$ &  \\
Sum &  &  \\
\hline
\end{tabular}
\end{center}

The target semi-deviation is the square root of the sum of the squared deviations from the target, divided by $n-1$ :

$s_{\text {Target }}=\sqrt{\frac{0.002025}{9}}=1.5 \%$.

\begin{enumerate}
  \setcounter{enumi}{32}
  \item B is correct. The median is indicated within the box, which is the 100.49 in this diagram.

  \item $\mathrm{C}$ is correct. The interquartile range is the difference between 114.25 and 79.74, which is 34.51 .

  \item B is correct. Quintiles divide a distribution into fifths, with the fourth quintile occurring at the point at which $80 \%$ of the observations lie below it. The fourth quintile is equivalent to the 80th percentile. To find the $y$ th percentile $\left(P_{y}\right)$, we first must determine its location. The formula for the location $\left(L_{y}\right)$ of a $y$ th percentile in an array with $n$ entries sorted in ascending order is $L_{y}=(n+1) \times$ $(y / 100)$. In this case, $n=10$ and $y=80 \%$, so

\end{enumerate}

$L_{80}=(10+1) \times(80 / 100)=11 \times 0.8=8.8$.

With the data arranged in ascending order $(-40.33 \%,-5.02 \%, 9.57 \%, 10.02 \%$, $12.34 \%, 15.25 \%, 16.54 \%, 20.65 \%, 27.37 \%$, and $30.79 \%$ ), the 8.8th position would be between the 8th and 9th entries, 20.65\% and $27.37 \%$, respectively. Using linear

$$
\begin{aligned}
& \text { interpolation, } P_{80}=X_{8}+\left(L_{y}-8\right) \times\left(X_{9}-X_{8}\right) \\
& P_{80}=20.65+(8.8-8) \times(27.37-20.65) \\
& =20.65+(0.8 \times 6.72)=20.65+5.38 \\
& =26.03 \% .
\end{aligned}
$$

\begin{enumerate}
  \setcounter{enumi}{35}
  \item A is correct. The formula for mean absolute deviation (MAD) is
\end{enumerate}

$\mathrm{MAD}=\frac{\sum_{i=1}^{n}\left|X_{i}-\bar{X}\right|}{n}$

Column 1: Sum annual returns and divide by $n$ to find the arithmetic mean $(\bar{X})$ of $16.40 \%$.

Column 2: Calculate the absolute value of the difference between each year's

\begin{center}
\begin{tabular}{|c|c|c|c|}
\hline
\multicolumn{3}{|c|}{Column 1} & \multirow{2}{*}{$\frac{\text { Column } 2}{\left|X_{i}-\bar{X}\right|}$} \\
\hline
Year & Return &  \\
\hline
Year 6 & $30.79 \%$ &  & $14.39 \%$ \\
\hline
Year 7 & $12.34 \%$ &  & $4.06 \%$ \\
\hline
Year 8 & $-5.02 \%$ &  & $21.42 \%$ \\
\hline
Year 9 & $16.54 \%$ &  & $0.14 \%$ \\
\hline
Year 10 & $27.37 \%$ &  & $10.97 \%$ \\
\hline
Sum: & $82.02 \%$ & Sum: & $50.98 \%$ \\
\hline
$n:$ & 5 & $n:$ & 5 \\
\hline
$\bar{X}$ & $16.40 \%$ & MAD: & $10.20 \%$ \\
\hline
\end{tabular}
\end{center}

return and the mean from Column 1. Sum the results and divide by $n$ to find the MAD.

These calculations are shown in the following exhibit:

\begin{enumerate}
  \setcounter{enumi}{36}
  \item $C$ is correct. The mean absolute deviation (MAD) of Fund ABC's returns is greater than the MAD of both of the other funds.
\end{enumerate}

MAD $=\frac{\sum_{i=1}^{n}\left|X_{i}-\bar{X}\right|}{n}$, where $\bar{X}$ is the arithmetic mean of the series.

MAD for Fund $\mathrm{ABC}=$

$\frac{|-20-(-4)|+|23-(-4)|+|-14-(-4)|+|5-(-4)|+|-14-(-4)|}{5}=14.4 \%$.

MAD for Fund XYZ =

$\frac{|-33-(-10.8)|+|-12-(-10.8)|+|-12-(-10.8)|+|-8-(-10.8)|+|11-(-10.8)|}{5}$
$=9.8 \%$.

MAD for Fund $\mathrm{PQR}=$

$\frac{|-14-(-5)|+|-18-(-5)|+|6-(-5)|+|-2-(-5)|+|3-(-5)|}{5}=8.8 \%$.

$A$ and $B$ are incorrect because the range and variance of the three funds are as follows:

\begin{center}
\begin{tabular}{lccc}
\hline
 & Fund ABC & Fund XYZ & Fund PQR \\
\hline
Range & $43 \%$ & $44 \%$ & $24 \%$ \\
Variance & 317 & 243 & 110 \\
\hline
\end{tabular}
\end{center}

The numbers shown for variance are understood to be in "percent squared" terms so that when taking the square root, the result is standard deviation in percentage terms. Alternatively, by expressing standard deviation and variance in decimal form, one can avoid the issue of units. In decimal form, the variances for Fund ABC, Fund XYZ, and Fund PQR are $0.0317,0.0243$, and 0.0110 , respectively.

\begin{enumerate}
  \setcounter{enumi}{37}
  \item A is correct. The more disperse a distribution, the greater the difference between the arithmetic mean and the geometric mean.

  \item B is correct. The coefficient of variation (CV) is the ratio of the standard deviation to the mean, where a higher $C V$ implies greater risk per unit of return.

\end{enumerate}

$$
\begin{aligned}
& \mathrm{CV}_{\text {UTIL }}=\frac{s}{\bar{X}}=\frac{1.23 \%}{2.10 \%}=0.59 . \\
& \mathrm{CV}_{M A T R}=\frac{s}{\bar{X}}=\frac{1.35 \%}{1.25 \%}=1.08 . \\
& \mathrm{CV}_{I N D U}=\frac{s}{\bar{X}}=\frac{1.52 \%}{3.01 \%}=0.51 .
\end{aligned}
$$

\begin{enumerate}
  \setcounter{enumi}{39}
  \item B is correct. The coefficient of variation is the ratio of the standard deviation to the arithmetic average, or $\sqrt{0.001723} / 0.09986=0.416$.

  \item $C$ is correct. The skewness is positive, so it is right-skewed (positively skewed).

  \item $C$ is correct. The excess kurtosis is positive, indicating that the distribution is "fat-tailed"; therefore, there is more probability in the tails of the distribution relative to the normal distribution.

  \item B is correct. The distribution is thin-tailed relative to the normal distribution because the excess kurtosis is less than zero.

  \item B is correct. The correlation coefficient is positive, indicating that the two series move together.

  \item $C$ is correct. Both outliers and spurious correlation are potential problems with interpreting correlation coefficients.

  \item $\mathrm{C}$ is correct. The correlation coefficient is positive because the covariation is positive.

  \item A is correct. The correlation coefficient is negative because the covariation is negative.

  \item $C$ is correct. The correlation coefficient is positive because the covariance is positive. The fact that one or both variables have a negative mean does not affect the sign of the correlation coefficient.

\end{enumerate}

\textbackslash section\{LEARNING MODULE

\begin{center}
\includegraphics[max width=\textwidth]{2023_05_04_cff39ee44f77d6514e1bg-183(1)}
\end{center}

\section{Probability Concepts}
by Richard A. DeFusco, PhD, CFA, Dennis W. McLeavey, DBA, CFA, Jerald E. Pinto, PhD, CFA, and David E. Runkle, PhD, CFA.

Richard A. DeFusco, PhD, CFA, is at the University of Nebraska-Lincoln (USA). Dennis W. McLeavey, DBA, CFA, is at the University of Rhode Island (USA). Jerald E. Pinto, PhD, CFA, is at CFA Institute (USA). David E. Runkle, PhD, CFA, is at Jacobs Levy Equity Management (USA).

\section{LEARNING OUTCOME}
\begin{center}
\includegraphics[max width=\textwidth]{2023_05_04_cff39ee44f77d6514e1bg-183}
\end{center}

\section{PROBABILITY CONCEPTS AND ODDS RATIOS}
define a random variable, an outcome, and an event
identify the two defining properties of probability, including
mutually exclusive and exhaustive events, and compare and contrast
empirical, subjective, and a priori probabilities
describe the probability of an event in terms of odds for and against
the event

Investment decisions are made in a risky environment. The tools that allow us to make decisions with consistency and logic in this setting are based on probability concepts. This reading presents the essential probability tools needed to frame and address many real-world problems involving risk. These tools apply to a variety of issues, such as predicting investment manager performance, forecasting financial variables, and pricing bonds so that they fairly compensate bondholders for default risk. Our focus is practical. We explore the concepts that are most important to investment research and practice. Among these are independence, as it relates to the predictability of returns and financial variables; expectation, as analysts continually look to the future in their analyses and decisions; and variability, variance or dispersion around expectation, as a risk concept important in investments. The reader will acquire specific skills and competencies in using these probability concepts to understand risks and returns on investments.

\section{Probability, Expected Value, and Variance}
The probability concepts and tools necessary for most of an analyst's work are relatively few and straightforward but require thought to apply. This section presents the essentials for working with probability, expectation, and variance, drawing on examples from equity and fixed income analysis.

An investor's concerns center on returns. The return on a risky asset is an example of a random variable.

\begin{itemize}
  \item Definition of Random Variable. A random variable is a quantity whose future outcomes are uncertain.

  \item Definition of Outcome. An outcome is a possible value of a random variable.

\end{itemize}

Using Exhibit 1 as an example, a portfolio manager may have a return objective of $10 \%$ a year. The portfolio manager's focus at the moment may be on the likelihood of earning a return that is less than $10 \%$ over the next year. Ten percent is a particular value or outcome of the random variable "portfolio return." Although we may be concerned about a single outcome, frequently our interest may be in a set of outcomes. The concept of "event" covers both.

\begin{itemize}
  \item Definition of Event. An event is a specified set of outcomes.
\end{itemize}

\section{Exhibit 1: Visualizing Probability}
\begin{center}
\includegraphics[max width=\textwidth]{2023_05_04_cff39ee44f77d6514e1bg-185}
\end{center}

An event can be a single outcome-for example, the portfolio earns a return of (exactly) $10 \%$. We can capture the portfolio manager's concerns by defining another event as the portfolio earns a return below $10 \%$. This second event, referring as it does to all possible returns greater than or equal to $-100 \%$ (the worst possible return, losing all the money in the portfolio) but less than $10 \%$, contains an infinite number of outcomes. To save words, it is common to use a capital letter in italics to represent a defined event. We could define $A=$ the portfolio earns a return of $10 \%$ and $B=$ the portfolio earns a return below $10 \%$.

To return to the portfolio manager's concern, how likely is it that the portfolio will earn a return below $10 \%$ ? The answer to this question is a probability: a number between 0 and 1 that measures the chance that a stated event will occur. If the probability is 0.65 that the portfolio earns a return below $10 \%$, there is a $65 \%$ chance of that event happening. If an event is impossible, it has a probability of 0 . If an event is certain to happen, it has a probability of 1 . If an event is impossible or a sure thing, it is not random at all. So, 0 and 1 bracket all the possible values of a probability.

To reiterate, a probability can be thought of as the likelihood that something will happen. If it has a probability of 1 , it is likely to happen $100 \%$ of the time, and if it has a probably of 0 , it is likely to never happen. Some people think of probabilities as akin to relative frequencies. If something is expected to happen 30 times out of 100 , the probability is 0.30 . The probability is the number of ways that an (equally likely) event can happen divided by the total number of possible outcomes.

Probability has two properties, which together constitute its definition.

\begin{itemize}
  \item Definition of Probability. The two defining properties of a probability are:
\end{itemize}

\begin{enumerate}
  \item The probability of any event $E$ is a number between 0 and $1: 0 \leq P(E) \leq 1$.

  \item The sum of the probabilities of any set of mutually exclusive and exhaustive events equals 1 .

\end{enumerate}

$P$ followed by parentheses stands for "the probability of (the event in parentheses)," as in $P(E)$ for "the probability of event $E$." We can also think of $P$ as a rule or function that assigns numerical values to events consistent with Properties 1 and 2.

In the above definition, the term mutually exclusive means that only one event can occur at a time; exhaustive means that the events cover all possible outcomes. Referring back to Exhibit 1 , the events $A=$ the portfolio earns a return of $10 \%$ and $B=$ the portfolio earns a return below $10 \%$ are mutually exclusive because $A$ and $B$ cannot both occur at the same time. For example, a return of $8.1 \%$ means that $B$ has occurred and $A$ has not occurred. Although events $A$ and $B$ are mutually exclusive, they are not exhaustive because they do not cover outcomes such as a return of $11 \%$. Suppose we define a third event: $C=$ the portfolio earns a return above $10 \%$. Clearly, $A, B$, and $C$ are mutually exclusive and exhaustive events. Each of $P(A), P(B)$, and $P(C)$ is a number between 0 and 1 , and $P(A)+P(B)+P(C)=1$.

Earlier, to illustrate a concept, we assumed a probability of 0.65 for a portfolio earning less than $10 \%$, without justifying the particular assumption. We also talked about using assigned probabilities of outcomes to calculate the probability of events, without explaining how such a probability distribution might be estimated. Making actual financial decisions using inaccurate probabilities could have grave consequences. How, in practice, do we estimate probabilities? This topic is a field of study in itself, but there are three broad approaches to estimating probabilities. In investments, we often estimate the probability of an event as a relative frequency of occurrence based on historical data. This method produces an empirical probability. For example, suppose you noted that 51 of the 60 stocks in a particular large-cap equity index pay dividends. The empirical probability of the stocks in the index paying a dividend is $P($ stock is dividend paying $)=51 / 60=0.85$.

Relationships must be stable through time for empirical probabilities to be accurate. We cannot calculate an empirical probability of an event not in the historical record or a reliable empirical probability for a very rare event. In some cases, then, we may adjust an empirical probability to account for perceptions of changing relationships. In other cases, we have no empirical probability to use at all. We may also make a personal assessment of probability without reference to any particular data. Another type of probability is a subjective probability, one drawing on personal or subjective judgment. Subjective probabilities are of great importance in investments. Investors, in making buy and sell decisions that determine asset prices, often draw on subjective probabilities.

For many well-defined problems, we can deduce probabilities by reasoning about the problem. The resulting probability is an a priori probability, one based on logical analysis rather than on observation or personal judgment. Because a priori and empirical probabilities generally do not vary from person to person, they are often grouped as objective probabilities.

For examples of the three types of probabilities, suppose you want to estimate the probability of flipping a coin and getting exactly two heads out of five flips. For the empirical probability, you do the experiment 100 times (five flips each time) and find that you get two heads 33 times. The empirical probability would be $33 / 100=0.33$. For a subjective judgement, you think the probability is somewhere between 0.25 and 0.50 , so you split the difference and choose 0.375 . For the a priori probability, you assume that the binomial probability function (discussed later in the curriculum) applies, and the mathematical probability of two heads out of five flips is 0.3125 .

Another way of stating probabilities often encountered in investments is in terms of odds-for instance, "the odds for $E$ " or the "odds against E." A probability is the fraction of the time you expect an event to occur, and the odds for an event is the probability that an event will occur divided by the probability that the event will not occur. Consider a football team that has a 0.25 probability of winning the World Cup, and a 0.75 probability of losing. The odds for winning are $0.25 / 0.75=0.33$ (and the odds for losing are $0.75 / 0.25=3.0$ ). If another team has a 0.80 probability of winning, the odds for winning would be $0.80 / 0.20=4.0$. If, for a third team, the probability of winning was 0.50 , the odds are even: odds $=0.50 / 0.50=1$. If the probability is low, the odds are very close to the probability. For example, if the probability of winning is 0.05 , the odds for winning are $0.05 / 0.95=0.0526$.

\section{EXAMPLE 1}
\section{Odds of Passing a Quantitative Methods Investment}
\section{Course}
Two of your colleagues are taking a quantitative methods investment course.

\begin{enumerate}
  \item If your first colleague has a 0.40 probability of passing, what are his odds for passing?
\end{enumerate}

\section{Solution for 1:}
The odds are the probability of passing divided by the probability of not passing. The odds are $0.40 / 0.60=2 / 3 \approx 0.667$.

\begin{enumerate}
  \setcounter{enumi}{1}
  \item If your second colleague has odds of passing of 4 to 1 , what is the probability of her passing?
\end{enumerate}

\section{Solution for 2:}
The odds = Probability (passing) $/$ Probability (not passing). If $Y=$ Probability of passing, then $4=Y /(1-Y)$. Solving for $Y$, we get 0.80 as the probability of passing.

We interpret probabilities stated in terms of odds as follows:

\begin{itemize}
  \item Probability Stated as Odds. Given a probability $P(E)$,
\end{itemize}

\begin{enumerate}
  \item Odds for $\boldsymbol{E}=P(E) /[1-P(E)]$. The odds for $E$ are the probability of $E$ divided by 1 minus the probability of $E$. Given odds for $E$ of " $a$ to $b$," the implied probability of $E$ is $a /(a+b)$.
\end{enumerate}

In the example, the statement that your second colleague's odds of passing the exam are 4 to 1 means that the probability of the event is $4 /(4+1)=4 / 5$ $=0.80$.

\begin{enumerate}
  \setcounter{enumi}{1}
  \item Odds against $\boldsymbol{E}=[1-P(E)] / P(E)$, the reciprocal of odds for $E$. Given odds against $E$ of " $a$ to $b$," the implied probability of $E$ is $b /(a+b)$.
\end{enumerate}

In the example, if the odds against your second colleague passing the exam are 1 to 4 , this means that the probability of the event is $1 /(4+1)=1 / 5=$ 0.20 .

To further explain odds for an event, if $P(E)=1 / 8$, the odds for $E$ are $(1 / 8) /(7 / 8)$ $=(1 / 8)(8 / 7)=1 / 7$, or " 1 to 7 ." For each occurrence of $E$, we expect seven cases of non-occurrence; out of eight cases in total, therefore, we expect $E$ to happen once, and the probability of $E$ is $1 / 8$. In wagering, it is common to speak in terms of the odds against something, as in Statement 2. For odds of "15 to 1" against $E$ (an implied probability of $E$ of $1 / 16$ ), a $\$ 1$ wager on $E$, if successful, returns $\$ 15$ in profits plus the $\$ 1$ staked in the wager. We can calculate the bet's anticipated profit as follows:

Win: $\quad$ Probability $=1 / 16$; Profit $=\$ 15$

Loss: $\quad$ Probability $=15 / 16$; Profit $=-\$ 1$

Anticipated profit $=(1 / 16)(\$ 15)+(15 / 16)(-\$ 1)=\$ 0$

Weighting each of the wager's two outcomes by the respective probability of the outcome, if the odds (probabilities) are accurate, the anticipated profit of the bet is $\$ 0$.

\section{EXAMPLE 2}
\section{Profiting from Inconsistent Probabilities}
\begin{enumerate}
  \item You are examining the common stock of two companies in the same industry in which an important antitrust decision will be announced next week. The first company, SmithCo Corporation, will benefit from a governmental decision that there is no antitrust obstacle related to a merger in which it is involved. You believe that SmithCo's share price reflects a 0.85 probability of such a decision. A second company, Selbert Corporation, will equally benefit from a "go ahead" ruling. Surprisingly, you believe Selbert stock reflects only a 0.50 probability of a favorable decision. Assuming your analysis is correct, what investment strategy would profit from this pricing discrepancy?
\end{enumerate}

Consider the logical possibilities. One is that the probability of 0.50 reflected in Selbert's share price is accurate. In that case, Selbert is fairly valued, but SmithCo is overvalued, because its current share price overestimates the probability of a "go ahead" decision. The second possibility is that the probability of 0.85 is accurate. In that case, SmithCo shares are fairly valued, but Selbert shares, which build in a lower probability of a favorable decision, are undervalued. You diagram the situation as shown in Exhibit 2.

\section{Exhibit 2: Worksheet for Investment Problem}
\begin{center}
\begin{tabular}{lcc}
\hline
 & True Probability of a "Go Ahead" Decision &  \\
\hline
0.50 & $\mathbf{0 . 8 5}$ &  \\
\hline
SmithCo & Shares Overvalued & Shares Fairly Valued \\
Selbert & Shares Fairly Valued & Shares Undervalued \\
Strategy & Short-Sell Smith / & Sell Smith / \\
 & Buy Selbert & Buy Selbert \\
\hline
\end{tabular}
\end{center}

The 0.50 probability column shows that Selbert shares are a better value than SmithCo shares. Selbert shares are also a better value if a 0.85 probability is accurate. Thus, SmithCo shares are overvalued relative to Selbert shares.

Your investment actions depend on your confidence in your analysis and on any investment constraints you face (such as constraints on selling stock short). Selling short or shorting stock means selling borrowed shares in the hope of repurchasing them later at a lower price. A conservative strategy would be to buy Selbert shares and reduce or eliminate any current position in SmithCo. The most aggressive strategy is to short SmithCo stock (relatively overvalued) and simultaneously buy the stock of Selbert (relatively undervalued). The prices of SmithCo and Selbert shares reflect probabilities that are not consistent. According to one of the most important probability results for investments, the Dutch Book Theorem, inconsistent probabilities create profit opportunities. In our example, investors' buy and sell decisions exploit the inconsistent probabilities to eliminate the profit opportunity and inconsistency.

\section*{CONDITIONAL AND JOINT PROBABILITY }
To understand the meaning of a probability in investment contexts, we need to distinguish between two types of probability: unconditional and conditional. Both unconditional and conditional probabilities satisfy the definition of probability stated earlier, but they are calculated or estimated differently and have different interpretations. They provide answers to different questions.

The probability in answer to the straightforward question "What is the probability of this event $A$ ?" is an unconditional probability, denoted $P(A)$. Suppose the question is "What is the probability that the stock earns a return above the risk-free rate (event $A$ )?" The answer is an unconditional probability that can be viewed as the ratio of two quantities. The numerator is the sum of the probabilities of stock returns above the risk-free rate. Suppose that sum is 0.70 . The denominator is 1 , the sum of the probabilities of all possible returns. The answer to the question is $P(A)=0.70$.

Contrast the question "What is the probability of $A$ ?" with the question "What is the probability of $A$, given that $B$ has occurred?" The probability in answer to this last question is a conditional probability, denoted $P(A \mid B)$ (read: "the probability of $A$ given $B^{\prime \prime}$ ).

Suppose we want to know the probability that the stock earns a return above the risk-free rate (event $A$ ), given that the stock earns a positive return (event $B$ ). With the words "given that," we are restricting returns to those larger than $0 \%$-a new element in contrast to the question that brought forth an unconditional probability. The conditional probability is calculated as the ratio of two quantities. The numerator is the sum of the probabilities of stock returns above the risk-free rate; in this particular case, the numerator is the same as it was in the unconditional case, which we gave as 0.70. The denominator, however, changes from 1 to the sum of the probabilities for all outcomes (returns) above $0 \%$. Suppose that number is 0.80 , a larger number than 0.70 because returns between 0 and the risk-free rate have some positive probability of occurring. Then $P(A \mid B)=0.70 / 0.80=0.875$. If we observe that the stock earns a positive return, the probability of a return above the risk-free rate is greater than the unconditional probability, which is the probability of the event given no other information. To review, an unconditional probability is the probability of an event without any restriction (i.e., a standalone probability). A conditional probability, in contrast, is a probability of an event given that another event has occurred.

To state an exact definition of conditional probability, we first need to introduce the concept of joint probability. Suppose we ask the question "What is the probability of both $A$ and $B$ happening?" The answer to this question is a joint probability, denoted $P(A B)$ (read: "the probability of $A$ and $B$ "). If we think of the probability of $A$ and the probability of $B$ as sets built of the outcomes of one or more random variables, the joint probability of $A$ and $B$ is the sum of the probabilities of the outcomes they have in common. For example, consider two events: the stock earns a return above the risk-free rate $(A)$ and the stock earns a positive return $(B)$. The outcomes of $A$ are contained within (a subset of) the outcomes of $B$, so $P(A B)$ equals $P(A)$. We can now state a formal definition of conditional probability that provides a formula for calculating it.

\begin{itemize}
  \item Definition of Conditional Probability. The conditional probability of $A$ given that $B$ has occurred is equal to the joint probability of $A$ and $B$ divided by the probability of $B$ (assumed not to equal 0).
\end{itemize}

$$
P(A \mid B)=P(A B) / P(B), P(B) \neq 0
$$

For example, suppose $B$ happens half the time, $P(B)=0.50$, and $A$ and $B$ both happen $10 \%$ of the time, $P(A B)=0.10$. What is the probability that $A$ happens, given that $B$ happens? That is $P(A \mid B)=P(A B) / P(B)=0.10 / 0.50=0.20$. Sometimes we know the conditional probability $P(A \mid B)$ and we want to know the joint probability $P(A B)$. We can obtain the joint probability from the following multiplication rule for probabilities, Equation 1 rearranged.

\begin{itemize}
  \item Multiplication Rule for Probability. The joint probability of $A$ and $B$ can be expressed as
\end{itemize}

$$
P(A B)=P(A \mid B) P(B)
$$

With the same numbers above, if $B$ happens $50 \%$ of the time, and the probability of $A$ given that $B$ happens is $20 \%$, the joint probability of $A$ and $B$ happening is $P(A B)$ $=P(A \mid B) P(B)=0.20 \times 0.50=0.10$.

\section{EXAMPLE 3}
\section{Conditional Probabilities and Predictability of Mutual Fund Performance (1)}
An analyst conducts a study of the returns of 200 mutual funds over a two-year period. For each year, the total returns for the funds were ranked, and the top $50 \%$ of funds were labeled winners; the bottom $50 \%$ were labeled losers. Exhibit 3 shows the percentage of those funds that were winners in two consecutive years, winners in one year and then losers in the next year, losers then winners, and finally losers in both years. The winner-winner entry, for example, shows that $66 \%$ of the first-year winner funds were also winners in the second year. The four entries in the table can be viewed as conditional probabilities.

Exhibit 3: Persistence of Returns: Conditional Probability for Year 2 Performance Given Year 1 Performance

\begin{center}
\begin{tabular}{lcc}
\hline
 & Year 2 Winner & Year 2 Loser \\
\hline
Year 1 Winner & $66 \%$ & $34 \%$ \\
Year 1 Loser & $34 \%$ & $66 \%$ \\
\hline
\end{tabular}
\end{center}

Based on the data in Exhibit 3, answer the following questions:

\begin{enumerate}
  \item State the four events needed to define the four conditional probabilities.
\end{enumerate}

\section{Solution to 1:}
The four events needed to define the conditional probabilities are as follows:

Fund is a Year 1 winner

Fund is a Year 1 loser Fund is a Year 2 loser

Fund is a Year 2 winner

\begin{enumerate}
  \setcounter{enumi}{1}
  \item State the four entries of the table as conditional probabilities using the form $P($ this event $\mid$ that event $)=$ number.
\end{enumerate}

\section{Solution to 2:}
From Row 1:

$P($ fund is a Year 2 winner $\mid$ fund is a Year 1 winner $)=0.66$

$P($ fund is a Year 2 loser $\mid$ fund is a Year 1 winner $)=0.34$

From Row 2:

$P($ fund is a Year 2 winner $\mid$ fund is a Year 1 loser $)=0.34$

$P($ fund is a Year 2 loser $\mid$ fund is a Year 1 loser $)=0.66$

\begin{enumerate}
  \setcounter{enumi}{2}
  \item Are the conditional probabilities in Question 2 empirical, a priori, or subjective probabilities?
\end{enumerate}

\section{Solution to 3:}
These probabilities are calculated from data, so they are empirical probabilities.

\begin{enumerate}
  \setcounter{enumi}{3}
  \item Using information in the table, calculate the probability of the event a fund is a loser in both Year 1 and Year 2. (Note that because 50\% of funds are categorized as losers in each year, the unconditional probability that a fund is labeled a loser in either year is 0.5.)
\end{enumerate}

\section{Solution to 4:}
The estimated probability is 0.33 . Let $A$ represent the event that a fund is a Year 2 loser, and let $B$ represent the event that the fund is a Year 1 loser. Therefore, the event $A B$ is the event that a fund is a loser in both Year 1 and Year 2. From Exhibit $3, P(A \mid B)=0.66$ and $P(B)=0.50$. Thus, using Equation 2, we find that

$$
P(A B)=P(A \mid B) P(B)=0.66(0.50)=0.33
$$

or a probability of 0.33 . Note that Equation 2 states that the joint probability of $A$ and $B$ equals the probability of $A$ given $B$ times the probability of $B$. Because $P(A B)=P(B A)$, the expression $P(A B)=P(B A)=P(B \mid A) P(A)$ is equivalent to Equation 2 .

When we have two events, $A$ and $B$, that we are interested in, we often want to know the probability that either $A$ or $B$ occurs. Here the word "or" is inclusive, meaning that either $A$ or $B$ occurs or that both $A$ and $B$ occur. Put another way, the probability of $A$ or $B$ is the probability that at least one of the two events occurs. Such probabilities are calculated using the addition rule for probabilities.

\begin{itemize}
  \item Addition Rule for Probabilities. Given events $A$ and $B$, the probability that $A$ or $B$ occurs, or both occur, is equal to the probability that $A$ occurs, plus the probability that $B$ occurs, minus the probability that both $A$ and $B$ occur.
\end{itemize}

$P(A$ or $B)=P(A)+P(B)-P(A B)$ If we think of the individual probabilities of $A$ and $B$ as sets built of outcomes of one or more random variables, the first step in calculating the probability of $A$ or $B$ is to sum the probabilities of the outcomes in $A$ to obtain $P(A)$. If $A$ and $B$ share any outcomes, then if we now added $P(B)$ to $P(A)$, we would count twice the probabilities of those shared outcomes. So we add to $P(A)$ the quantity $[P(B)-P(A B)]$, which is the probability of outcomes in $B$ net of the probability of any outcomes already counted when we computed $P(A)$. Exhibit 4 illustrates this process; we avoid double-counting the outcomes in the intersection of $A$ and $B$ by subtracting $P(A B)$. As an example of the calculation, if $P(A)=0.50, P(B)=0.40$, and $P(A B)=0.20$, then $P(A$ or $B)=0.50$ $+0.40-0.20=0.70$. Only if the two events $A$ and $B$ were mutually exclusive, so that $P(A B)=0$, would it be correct to state that $P(A$ or $B)=P(A)+P(B)$.

\section{Exhibit 4: Addition Rule for Probabilities}
\begin{center}
\includegraphics[max width=\textwidth]{2023_05_04_cff39ee44f77d6514e1bg-192}
\end{center}

Example 4 illustrates the relation between empirical frequencies and unconditional, conditional, and joint probabilities as well as the multiplication and addition rules for probability.

\section{EXAMPLE 4}
\section{Frequencies and Probability Concepts}
\begin{enumerate}
  \item Analysts often discuss the frequencies of events as well as their probabilities. In Exhibit 5, there are 150 cells, each representing one trading day. Outcome $A$, one of the 80 trading days when the stock market index increased, is represented by the dark-shaded rectangle with 80 cells. Outcome $B$, one the 30 trading days when interest rates decreased, is represented by the light -bordered rectangle with 30 cells. The overlap between these two rectangles, when both events $A$ and $B$ occurred-the stock market index increased, and interest rates decreased-happened 15 times and is represented by the intermediate-shaded rectangle.
\end{enumerate}

\section{Exhibit 5: Frequencies for Two Events}
\begin{center}
\includegraphics[max width=\textwidth]{2023_05_04_cff39ee44f77d6514e1bg-193}
\end{center}

\begin{itemize}
  \item The frequency of $A$ (stock market index increased) is 80 and has an unconditional probability $P(A)=80 / 150=0.533$.

  \item The frequency of $B$ (interest rates decreased) is 30 and has an unconditional probability $P(B)=30 / 150=0.20$.

  \item The frequency of $A$ and $B$ (stock market index increased, and interest rates decreased) is 15 and has a joint probability, $P(A B)=15 / 150=$ 0.10

\end{itemize}

The frequency of $A$ or $B$ (stock market index increased or interest rates decreased is 95 , and $P(A$ or $B)$ is $95 / 150=0.633$. Using the addition rule for probabilities, the probability of $A$ or $B$ is $P(A$ or $B)=P(A)+P(B)-P(A B)=$ $80 / 150+30 / 150-15 / 150=95 / 150=0.633$. The probability of not $A$ or $B$ (stock market index did not increase or interest rates did not decrease) $=1$ $95 / 150=55 / 150=0.367$

The conditional probability of $A$ given $B, P(A \mid B)$, stock market index increased given that interest rates decreased, was $15 / 30=0.50$, which is also $P(A \mid B)=P(A B) / P(B)=(15 / 150) /(30 / 150)=0.10 / 0.20=0.50$

The next example shows how much useful information can be obtained using the probability rules presented to this point.

\section{EXAMPLE 5}
\section{Probability of a Limit Order Executing}
You have two buy limit orders outstanding on the same stock. A limit order to buy stock at a stated price is an order to buy at that price or lower. A number of vendors, including an internet service that you use, supply the estimated probability that a limit order will be filled within a stated time horizon, given the current stock price and the price limit. One buy order (Order 1 ) was placed at a price limit of $\$ 10$. The probability that it will execute within one hour is 0.35 . The second buy order (Order 2 ) was placed at a price limit of $\$ 9.75$; it has a 0.25 probability of executing within the same one-hour time frame. 1. What is the probability that either Order 1 or Order 2 will execute?

\section{Solution to 1:}
The probability is 0.35 . The two probabilities that are given are $P$ (Order 1 executes $)=0.35$ and $P$ (Order 2 executes $)=0.25$. Note that if Order 2 executes, it is certain that Order 1 also executes because the price must pass through $\$ 10$ to reach $\$ 9.75$. Thus,

$P($ Order 1 executes $\mid$ Order 2 executes $)=1$

and using the multiplication rule for probabilities,

$P($ Order 1 executes and Order 2 executes $)=P($ Order 1 executes $\mid$

Order 2 executes $) P($ Order 2 executes $)=1(0.25)=0.25$

To answer the question, we use the addition rule for probabilities:

$P($ Order 1 executes or Order 2 executes $)=P($ Order 1 executes $)$

$+P($ Order 2 executes $)-P($ Order 1 executes and Order 2 executes $)$

$=0.35+0.25-0.25=0.35$

Note that the outcomes for which Order 2 executes are a subset of the outcomes for which Order 1 executes. After you count the probability that Order 1 executes, you have counted the probability of the outcomes for which Order 2 also executes. Therefore, the answer to the question is the probability that Order 1 executes, 0.35 .

\begin{enumerate}
  \setcounter{enumi}{1}
  \item What is the probability that Order 2 executes, given that Order 1 executes?
\end{enumerate}

\section{Solution to 2:}
If the first order executes, the probability that the second order executes is 0.714. In the solution to Part 1 , you found that $P$ (Order 1 executes and Order 2 executes $)=P($ Order 1 executes $\mid$ Order 2 executes $) P($ Order 2 executes $)=$ $1(0.25)=0.25$. An equivalent way to state this joint probability is useful here: $P($ Order 1 executes and Order 2 executes $)=0.25$

$=P($ Order 2 executes $\mid$ Order 1 executes $) P($ Order 1 executes $)$

Because $P$ (Order 1 executes $)=0.35$ was a given, you have one equation with one unknown:

$$
0.25=P(\text { Order } 2 \text { executes } \mid \text { Order } 1 \text { executes })(0.35)
$$

You conclude that $P($ Order 2 executes $\mid$ Order 1 executes $)=0.25 / 0.35=$ 0.714 . You can also use Equation 1 to obtain this answer.

The concepts of independence and dependence are of great interest to investment analysts. These concepts bear on such basic investment questions as which financial variables are useful for investment analysis, whether asset returns can be predicted, and whether superior investment managers can be selected based on their past records.

Two events are independent if the occurrence of one event does not affect the probability of occurrence of the other event.

\begin{itemize}
  \item Definition of Independent Events. Two events $A$ and $B$ are independent if and only if $P(A \mid B)=P(A)$ or, equivalently, $P(B \mid A)=P(B)$.
\end{itemize}

The logic of independence is clear: $A$ and $B$ are independent if the conditional probability of $A$ given $B, P(A \mid B)$, is the same as the unconditional probability of $A$, $P(A)$. Independence means that knowing $B$ tells you nothing about $A$. For an example of independent events, suppose that event $A$ is the bankruptcy of Company $A$, and event $B$ is the bankruptcy of Company B. If the probability of bankruptcy of Company $\mathrm{A}$ is $P(A)=0.20$, and the probability of bankruptcy of Company A given that Company B goes bankrupt is the same, $P(A \mid B)=0.20$, then event $A$ is independent of event $B$.

When two events are not independent, they are dependent: The probability of occurrence of one is related to the occurrence of the other. If we are trying to forecast one event, information about a dependent event may be useful, but information about an independent event will not be useful. For example, suppose an announcement is released that a biotech company will be acquired at an attractive price by another company. If the prices of pharmaceutical companies increase as a result of this news, the companies' stock prices are not independent of the biotech takeover announcement event. For a different example, if two events are mutually exclusive, then knowledge that one event has occurred gives us information that the other (mutually exclusive) event cannot occur.

When two events are independent, the multiplication rule for probabilities, Equation 2, simplifies because $P(A \mid B)$ in that equation then equals $P(A)$.

\begin{itemize}
  \item Multiplication Rule for Independent Events. When two events are independent, the joint probability of $A$ and $B$ equals the product of the individual probabilities of $A$ and $B$.
\end{itemize}

$$
P(A B)=P(A) P(B)
$$

Therefore, if we are interested in two independent events with probabilities of 0.75 and 0.50 , respectively, the probability that both will occur is $0.375=0.75(0.50)$. The multiplication rule for independent events generalizes to more than two events; for example, if $A, B$, and $C$ are independent events, then $P(A B C)=P(A) P(B) P(C)$.

\section{EXAMPLE 6}
\section{BankCorp's Earnings per Share (1)}
As part of your work as a banking industry analyst, you build models for forecasting earnings per share of the banks you cover. Today you are studying BankCorp. The historical record shows that in $55 \%$ of recent quarters, BankCorp's EPS has increased sequentially, and in $45 \%$ of quarters, EPS has decreased or remained unchanged sequentially. At this point in your analysis, you are assuming that changes in sequential EPS are independent.

Earnings per share for $2 \mathrm{Q}$ :Year 1 (that is, EPS for the second quarter of Year 1) were larger than EPS for $1 \mathrm{Q}$ :Year 1.

\begin{enumerate}
  \item What is the probability that 3Q:Year 1 EPS will be larger than $2 Q: Y e a r 1 E P S$ (a positive change in sequential EPS)?
\end{enumerate}

\section{Solution to 1:}
Under the assumption of independence, the probability that 3Q:Year 1 EPS will be larger than 2Q:Year 1 EPS is the unconditional probability of positive change, 0.55 . The fact that $2 \mathrm{Q}$ :Year $1 \mathrm{EPS}$ was larger than $1 \mathrm{Q}$ : Year $1 \mathrm{EPS}$ is not useful information, because the next change in EPS is independent of the prior change. 2. What is the probability that EPS decreases or remains unchanged in the next two quarters?

\section{Solution to 2:}
Assuming independence, the probability is $0.2025=0.45(0.45)$.

The following example illustrates how difficult it is to satisfy a set of independent criteria even when each criterion by itself is not necessarily stringent.

\section{EXAMPLE 7}
\section{Screening Stocks for Investment}
You have developed a stock screen-a set of criteria for selecting stocks. Your investment universe (the set of securities from which you make your choices) is 905 large- and medium-cap US equities, specifically all stocks that are members of the S\&P 500 and S\&P 400 Indexes. Your criteria capture different aspects of the stock selection problem; you believe that the criteria are independent of each other, to a close approximation.

\begin{center}
\begin{tabular}{lcc}
\hline
Criterion & $\begin{array}{c}\text { Number of stocks } \\ \text { meeting criterion }\end{array}$ & $\begin{array}{c}\text { Fraction of stocks } \\ \text { meeting criterion }\end{array}$ \\
\hline
First valuation criterion & 556 & 0.614 \\
Second valuation criterion & 489 & 0.540 \\
Analyst coverage criterion & 600 & 0.663 \\
Profitability criterion & 490 & 0.541 \\
Financial strength criterion & 313 & 0.346 \\
\hline
\end{tabular}
\end{center}

How many stocks do you expect to pass your screen?

Only 37 stocks out of 905 should pass through your screen. If you define five events-the stock passes the first valuation criterion, the stock passes the second valuation criterion, the stock passes the analyst coverage criterion, the company passes the profitability criterion, the company passes the financial strength criterion (say events $A, B, C, D$, and $E$, respectively)-then the probability that a stock will pass all five criteria, under independence, is

$$
P(A B C D E)=P(A) P(B) P(C) P(D) P(E)=(0.614)(0.540)(0.663)(0.541)(0.346)=
$$

0.0411

Although only one of the five criteria is even moderately strict (the strictest lets $34.6 \%$ of stocks through), the probability that a stock can pass all five criteria is only 0.0411 , or about $4 \%$. If the criteria are independent, the size of the list of candidate investments is expected to be $0.0411(905)=37$ stocks.

An area of intense interest to investment managers and their clients is whether records of past performance are useful in identifying repeat winners and losers. The following example shows how this issue relates to the concept of independence.

\section{EXAMPLE 8}
\section{Conditional Probabilities and Predictability of Mutual Fund Performance (2)}
\begin{enumerate}
  \item The purpose of the mutual fund study introduced in Example 3 was to address the question of repeat mutual fund winners and losers. If the status of a fund as a winner or a loser in one year is independent of whether it is a winner in the next year, the practical value of performance ranking is questionable. Using the four events defined in Example 3 as building blocks, we can define the following events to address the issue of predictability of mutual fund performance:
\end{enumerate}

Fund is a Year 1 winner and fund is a Year 2 winner

Fund is a Year 1 winner and fund is a Year 2 loser

Fund is a Year 1 loser and fund is a Year 2 winner

Fund is a Year 1 loser and fund is a Year 2 loser

In Part 4 of Example 3, you calculated that

$P($ fund is a Year 2 loser and fund is a Year 1 loser $)=0.33$

If the ranking in one year is independent of the ranking in the next year, what will you expect $P$ (fund is a Year 2 loser and fund is a Year 1 loser) to be? Interpret the empirical probability 0.33.

By the multiplication rule for independent events, $P$ (fund is a Year 2 loser and fund is a Year 1 loser $)=P($ fund is a Year 2 loser $) P($ fund is a Year 1 loser). Because $50 \%$ of funds are categorized as losers in each year, the unconditional probability that a fund is labeled a loser in either year is 0.50 . Thus $P($ fund is a Year 2 loser $) P($ fund is a Year 1 loser $)=0.50(0.50)=0.25$. If the status of a fund as a loser in one year is independent of whether it is a loser in the prior year, we conclude that $P$ (fund is a Year 2 loser and fund is a Year 1 loser $)=0.25$. This probability is a priori because it is obtained from reasoning about the problem. You could also reason that the four events described above define categories and that if funds are randomly assigned to the four categories, there is a 1/4 probability of fund is a Year 1 loser and fund is a Year 2 loser. If the classifications in Year 1 and Year 2 were dependent, then the assignment of funds to categories would not be random. The empirical probability of 0.33 is above 0.25 . Is this apparent predictability the result of chance? Further analysis would be necessary to determine whether these results would allow you to reject the hypothesis that investment returns are independent between Year 1 and Year 2.

In many practical problems, we logically analyze a problem as follows: We formulate scenarios that we think affect the likelihood of an event that interests us. We then estimate the probability of the event, given the scenario. When the scenarios (conditioning events) are mutually exclusive and exhaustive, no possible outcomes are left out. We can then analyze the event using the total probability rule. This rule explains the unconditional probability of the event in terms of probabilities conditional on the scenarios. The total probability rule is stated below for two cases. Equation 5 gives the simplest case, in which we have two scenarios. One new notation is introduced: If we have an event or scenario $S$, the event not-S, called the complement of $S$, is written $S^{\mathrm{C}}$. Note that $P(S)+P\left(S^{\mathrm{C}}\right)=1$, as either $S$ or not- $S$ must occur. Equation 6 states the rule for the general case of $n$ mutually exclusive and exhaustive events or scenarios.

\begin{itemize}
  \item Total Probability Rule.
\end{itemize}

$$
\begin{aligned}
& P(A)=P(A S)+P\left(A S^{C}\right) \\
& =P(A \mid S) P(S)+P\left(A \mid S^{C}\right) P\left(S^{C}\right) \\
& P(A)=P\left(A S_{1}\right)+P\left(A S_{2}\right)+\ldots+P\left(A S_{n}\right) \\
& =P\left(A \mid S_{1}\right) P\left(S_{1}\right)+P\left(A \mid S_{2}\right) P\left(S_{2}\right)+\ldots+P\left(A \mid S_{n}\right) P\left(S_{n}\right)
\end{aligned}
$$

where $S_{1}, S_{2}, \ldots, S_{n}$ are mutually exclusive and exhaustive scenarios or events.

Equation 6 states the following: The probability of any event $[P(A)]$ can be expressed as a weighted average of the probabilities of the event, given scenarios [terms such $\left.P\left(A \mid S_{1}\right)\right]$; the weights applied to these conditional probabilities are the respective probabilities of the scenarios [terms such as $P\left(S_{1}\right)$ multiplying $P\left(A \mid S_{1}\right)$ ], and the scenarios must be mutually exclusive and exhaustive. Among other applications, this rule is needed to understand Bayes' formula, which we discuss later.

Exhibit 6 is a visual representation of the total probability rule. Panel A illustrates Equation 5 for the total probability rule when there are two scenarios $(S$ and its complement $S^{C}$ ). For two scenarios, the probabilities of $S$ and $S^{C}$ sum to 1 , and the probability of $A$ is a weighted average where the probability of $A$ in each scenario is weighted by the probability of each scenario. Panel B of Exhibit 6 illustrates Equation 6 for the total probability rule when there are $n$ scenarios. The scenarios are mutually exclusive and exhaustive, and the sum of the probabilities for the scenarios is 1. Like the two-scenario case, the probability of $A$ given the $n$-scenarios is a weighted average of the conditional probabilities of $A$ in each scenario, using as weights the probability of each scenario.

\section{Exhibit 6: The Total Probability Rule for Two Scenarios and for $n$ Scenarios}
\section{A. Total Probability Rule for Two Scenarios (S and SC)}
\begin{center}
\includegraphics[max width=\textwidth]{2023_05_04_cff39ee44f77d6514e1bg-199}
\end{center}

$P(S)+P\left(S^{C}\right)=1 ; P(A)=P(A S)+P\left(A S^{C}\right)=P(A \mid S) P(S)+P\left(A \mid S^{C}\right) P\left(S^{C}\right)$

\section{B. Total Probability Rule for $n$ Scenarios}
\begin{center}
\includegraphics[max width=\textwidth]{2023_05_04_cff39ee44f77d6514e1bg-199(1)}
\end{center}

$S_{1}, S_{2}, \ldots S_{n}$ are mutually exclusive and exhaustive scenarios, such that

$$
\begin{aligned}
& \sum_{i=1}^{n} P\left(S_{i}\right)=1 \\
P(A) & =P\left(A S_{1}\right)+P\left(A S_{2}\right)+\ldots+P\left(A S_{n}\right) \\
& =P\left(A \mid S_{1}\right) P\left(S_{1}\right)+P\left(A \mid S_{2}\right) P\left(S_{2}\right)+\ldots+P\left(A \mid S_{n}\right) P\left(S_{n}\right)
\end{aligned}
$$

In the next example, we use the total probability rule to develop a consistent set of views about BankCorp's earnings per share.

\section{EXAMPLE 9}
\section{BankCorp's Earnings per Share (2)}
You are continuing your investigation into whether you can predict the direction of changes in BankCorp's quarterly EPS. You define four events:

\begin{center}
\begin{tabular}{lc}
\hline
Event & Probability \\
\hline
$A=$ Change in sequential EPS is positive next quarter & 0.55 \\
$A^{\mathrm{C}}=$ Change in sequential EPS is 0 or negative next quarter & 0.45 \\
$S=$ Change in sequential EPS is positive in the prior quarter & 0.55 \\
$S^{\mathrm{C}}=$ Change in sequential EPS is 0 or negative in the prior quarter & 0.45 \\
\hline
\end{tabular}
\end{center}

On inspecting the data, you observe some persistence in EPS changes: Increases tend to be followed by increases, and decreases by decreases. The first probability estimate you develop is $P$ (change in sequential EPS is positive next quarter | change in sequential EPS is 0 or negative in the prior quarter) $=$ $P\left(A \mid S^{\mathrm{C}}\right)=0.40$. The most recent quarter's EPS (2Q:Year 1) is announced, and the change is a positive sequential change (the event $S$ ). You are interested in forecasting EPS for 3Q:Year 1.

\begin{enumerate}
  \item Write this statement in probability notation: "the probability that the change in sequential EPS is positive next quarter, given that the change in sequential EPS is positive the prior quarter."
\end{enumerate}

\section{Solution to 1:}
In probability notation, this statement is written $P(A \mid S)$.

\begin{enumerate}
  \setcounter{enumi}{1}
  \item Calculate the probability in Part 1 . (Calculate the probability that is consistent with your other probabilities or beliefs.)
\end{enumerate}

\section{Solution to 2:}
The probability is 0.673 that the change in sequential EPS is positive for 3Q:Year 1, given the positive change in sequential EPS for 2Q:Year 1, as shown below.

According to Equation 5, $P(A)=P(A \mid S) P(S)+P\left(A \mid S^{\mathrm{C}}\right) P\left(S^{\mathrm{C}}\right)$. The values of the probabilities needed to calculate $P(A \mid S)$ are already known: $P(A)=0.55$, $P(S)=0.55, P\left(S^{\mathrm{C}}\right)=0.45$, and $P\left(A \mid S^{\mathrm{C}}\right)=0.40$. Substituting into Equation 5 ,

$$
0.55=P(A \mid S)(0.55)+0.40(0.45)
$$

Solving for the unknown, $P(A \mid S)=[0.55-0.40(0.45)] / 0.55=0.672727$, or 0.673 .

You conclude that $P$ (change in sequential EPS is positive next quarter | change in sequential EPS is positive the prior quarter) $=0.673$. Any other probability is not consistent with your other estimated probabilities. Reflecting the persistence in EPS changes, this conditional probability of a positive EPS change, 0.673, is greater than the unconditional probability of an EPS increase, 0.55 .

\section{EXPECTED VALUE AND VARIANCE}
calculate and interpret the expected value, variance, and standard deviation of random variables

explain the use of conditional expectation in investment applications

interpret a probability tree and demonstrate its application to investment problems

The expected value of a random variable is an essential quantitative concept in investments. Investors continually make use of expected values-in estimating the rewards of alternative investments, in forecasting EPS and other corporate financial variables and ratios, and in assessing any other factor that may affect their financial position. The expected value of a random variable is defined as follows:

\begin{itemize}
  \item Definition of Expected Value. The expected value of a random variable is the probability-weighted average of the possible outcomes of the random variable. For a random variable $X$, the expected value of $X$ is denoted $E(X)$.
\end{itemize}

Expected value (for example, expected stock return) looks either to the future, as a forecast, or to the "true" value of the mean (the population mean). We should distinguish expected value from the concepts of historical or sample mean. The sample mean also summarizes in a single number a central value. However, the sample mean presents a central value for a particular set of observations as an equally weighted average of those observations. In sum, the contrast is forecast versus historical, or population versus sample.

\section{EXAMPLE 10}
\section{BankCorp's Earnings per Share (3)}
You continue with your analysis of BankCorp's EPS. In Exhibit 7, you have recorded a probability distribution for BankCorp's EPS for the current fiscal year.

\section{Exhibit 7: Probability Distribution for BankCorp's EPS}
\begin{center}
\begin{tabular}{cc}
\hline
Probability & EPS (\$) \\
\hline
0.15 & 2.60 \\
0.45 & 2.45 \\
0.24 & 2.20 \\
0.16 & 2.00 \\
\hline
1.00 &  \\
\hline
\end{tabular}
\end{center}

What is the expected value of BankCorp's EPS for the current fiscal year? Following the definition of expected value, list each outcome, weight it by its probability, and sum the terms.

$E(\mathrm{EPS})=0.15(\$ 2.60)+0.45(\$ 2.45)+0.24(\$ 2.20)+0.16(\$ 2.00)=\$ 2.3405$

The expected value of EPS is $\$ 2.34$.

An equation that summarizes your calculation in Example 10 is

$$
E(X)=P\left(X_{1}\right) X_{1}+P\left(X_{2}\right) X_{2}+\ldots+P\left(X_{n}\right) X_{n}=\sum_{i=1}^{n} P\left(X_{i}\right) X_{i}
$$

where $X_{i}$ is one of $n$ possible outcomes of the random variable $X$.

The expected value is our forecast. Because we are discussing random quantities, we cannot count on an individual forecast being realized (although we hope that, on average, forecasts will be accurate). It is important, as a result, to measure the risk we face. Variance and standard deviation measure the dispersion of outcomes around the expected value or forecast.

\begin{itemize}
  \item Definition of Variance. The variance of a random variable is the expected value (the probability-weighted average) of squared deviations from the random variable's expected value:
\end{itemize}

$$
\sigma^{2}(X)=E\left\{[X-E(X)]^{2}\right\}
$$

The two notations for variance are $\sigma^{2}(X)$ and $\operatorname{Var}(X)$.

Variance is a number greater than or equal to 0 because it is the sum of squared terms. If variance is 0 , there is no dispersion or risk. The outcome is certain, and the quantity $X$ is not random at all. Variance greater than 0 indicates dispersion of outcomes. Increasing variance indicates increasing dispersion, all else equal. Variance of $X$ is a quantity in the squared units of $X$. For example, if the random variable is return in percent, variance of return is in units of percent squared. Standard deviation is easier to interpret than variance because it is in the same units as the random variable. If the random variable is return in percent, standard deviation of return is also in units of percent. In the following example, when the variance of returns is stated as a percent or amount of money, to conserve space, we may suppress showing the unit squared.

\begin{itemize}
  \item Definition of Standard Deviation. Standard deviation is the positive square root of variance.
\end{itemize}

The best way to become familiar with these concepts is to work examples.

\section{EXAMPLE 11}
\section{BankCorp's Earnings per Share (4)}
In Example 10, you calculated the expected value of BankCorp's EPS as \$2.34, which is your forecast. Using the probability distribution of EPS from Exhibit 6, you want to measure the dispersion around your forecast. What are the variance and standard deviation of BankCorp's EPS for the current fiscal year?

The order of calculation is always expected value, then variance, then standard deviation. Expected value has already been calculated. Following the definition of variance above, calculate the deviation of each outcome from the mean or expected value, square each deviation, weight (multiply) each squared deviation by its probability of occurrence, and then sum these terms.

$$
\begin{aligned}
& \sigma^{2}(\mathrm{EPS})=P(\$ 2.60)[\$ 2.60-E(\mathrm{EPS})]^{2}+P(\$ 2.45)[\$ 2.45-E(\mathrm{EPS})]^{2} \\
& +P(\$ 2.20)[\$ 2.20-E(\mathrm{EPS})]^{2}+P(\$ 2.00)[\$ 2.00-E(\mathrm{EPS})]^{2} \\
= & 0.15(2.60-2.34)^{2}+0.45(2.45-2.34)^{2} \\
+ & 0.24(2.20-2.34)^{2}+0.16(2.00-2.34)^{2} \\
= & 0.01014+0.005445+0.004704+0.018496=0.038785
\end{aligned}
$$

Standard deviation is the positive square root of 0.038785 :

$$
\sigma(\mathrm{EPS})=0.038785^{1 / 2}=0.196939 \text {, or approximately } 0.20 \text {. }
$$

$$
\begin{aligned}
\sigma^{2}(X) & =P\left(X_{1}\right)\left[X_{1}-E(X)\right]^{2}+P\left(X_{2}\right)\left[X_{2}-E(X)\right]^{2} \\
& +\ldots+P\left(X_{n}\right)\left[X_{n}-E(X)\right]^{2}=\sum_{i=1}^{n} P\left(X_{i}\right)\left[X_{i}-E(X)\right]^{2}
\end{aligned}
$$

where $X_{i}$ is one of $n$ possible outcomes of the random variable $X$.

In investments, we make use of any relevant information available in making our forecasts. When we refine our expectations or forecasts, we are typically making adjustments based on new information or events; in these cases, we are using conditional expected values. The expected value of a random variable $X$ given an event or scenario $S$ is denoted $E(X \mid S)$. Suppose the random variable $X$ can take on any one of $n$ distinct outcomes $X_{1}, X_{2}, \ldots, X_{n}$ (these outcomes form a set of mutually exclusive and exhaustive events). The expected value of $X$ conditional on $S$ is the first outcome, $X_{1}$, times the probability of the first outcome given $S, P\left(X_{1} \mid S\right)$, plus the second outcome, $X_{2}$, times the probability of the second outcome given $S, P\left(X_{2} \mid S\right)$, and so forth.

$$
E(X \mid S)=P\left(X_{1} \mid S\right) X_{1}+P\left(X_{2} \mid S\right) X_{2}+\ldots+P\left(X_{n} \mid S\right) X_{n}
$$

We will illustrate this equation shortly.

Parallel to the total probability rule for stating unconditional probabilities in terms of conditional probabilities, there is a principle for stating (unconditional) expected values in terms of conditional expected values. This principle is the total probability rule for expected value.

\begin{itemize}
  \item Total Probability Rule for Expected Value.
\end{itemize}

$$
\begin{aligned}
& E(X)=E(X \mid S) P(S)+E\left(X \mid S^{C}\right) P\left(S^{C}\right) \\
& E(X)=E\left(X \mid S_{1}\right) P\left(S_{1}\right)+E\left(X \mid S_{2}\right) P\left(S_{2}\right)+\ldots+E\left(X \mid S_{n}\right) P\left(S_{n}\right)
\end{aligned}
$$

where $S_{1}, S_{2}, \ldots, S_{n}$ are mutually exclusive and exhaustive scenarios or events.

The general case, Equation 12, states that the expected value of $X$ equals the expected value of $X$ given Scenario 1, $E\left(X \mid S_{1}\right)$, times the probability of Scenario 1, $P\left(S_{1}\right)$, plus the expected value of $X$ given Scenario 2, $E\left(X \mid S_{2}\right)$, times the probability of Scenario 2, $P\left(S_{2}\right)$, and so forth.

To use this principle, we formulate mutually exclusive and exhaustive scenarios that are useful for understanding the outcomes of the random variable. This approach was employed in developing the probability distribution of BankCorp's EPS in Examples 10 and 11 , as we now discuss.

The earnings of BankCorp are interest rate sensitive, benefiting from a declining interest rate environment. Suppose there is a 0.60 probability that BankCorp will operate in a declining interest rate environment in the current fiscal year and a 0.40 probability that it will operate in a stable interest rate environment (assessing the chance of an increasing interest rate environment as negligible). If a declining interest rate environment occurs, the probability that EPS will be $\$ 2.60$ is estimated at 0.25 , and the probability that EPS will be $\$ 2.45$ is estimated at 0.75 . Note that 0.60 , the probability of declining interest rate environment, times 0.25 , the probability of $\$ 2.60$ EPS given a declining interest rate environment, equals 0.15 , the (unconditional) probability of $\$ 2.60$ given in the table in Exhibit 7. The probabilities are consistent. Also, 0.60(0.75) $=0.45$, the probability of $\$ 2.45$ EPS given in Exhibit 7 . The probability tree diagram in Exhibit 8 shows the rest of the analysis.

\section{Exhibit 8: BankCorp's Forecasted EPS}
\begin{center}
\includegraphics[max width=\textwidth]{2023_05_04_cff39ee44f77d6514e1bg-204}
\end{center}

A declining interest rate environment points us to the node of the tree that branches off into outcomes of $\$ 2.60$ and $\$ 2.45$. We can find expected EPS given a declining interest rate environment as follows, using Equation 10:

$E(\mathrm{EPS} \mid$ declining interest rate environment $)=0.25(\$ 2.60)+0.75(\$ 2.45)$

$$
=\$ 2.4875
$$

If interest rates are stable,

$E(\mathrm{EPS} \mid$ stable interest rate environment $)=0.60(\$ 2.20)+0.40(\$ 2.00)$

$$
=\$ 2.12
$$

Once we have the new piece of information that interest rates are stable, for example, we revise our original expectation of EPS from $\$ 2.34$ downward to $\$ 2.12$. Now using the total probability rule for expected value,

$\mathrm{E}(\mathrm{EPS})$

$=\mathrm{E}(\mathrm{EPS} \mid$ declining interest rate environment $) \mathrm{P}($ declining interest rateenvironment $)$

$+\mathrm{E}(\mathrm{EPS} \mid$ stable interest rate environment) $\mathrm{P}($ stable interest rateenvironment)

So, $E(\mathrm{EPS})=\$ 2.4875(0.60)+\$ 2.12(0.40)=\$ 2.3405$ or about $\$ 2.34$.

This amount is identical to the estimate of the expected value of EPS calculated directly from the probability distribution in Example 10. Just as our probabilities must be consistent, so must our expected values, unconditional and conditional; otherwise our investment actions may create profit opportunities for other investors at our expense.

To review, we first developed the factors or scenarios that influence the outcome of the event of interest. After assigning probabilities to these scenarios, we formed expectations conditioned on the different scenarios. Then we worked backward to formulate an expected value as of today. In the problem just worked, EPS was the event of interest, and the interest rate environment was the factor influencing EPS.

We can also calculate the variance of EPS given each scenario:

$\sigma^{2}(\mathrm{EPS} \mid$ declining interest rate environment $)$

$=P(\$ 2.60 \mid$ declining interest rate environment $)$

$\times[\$ 2.60-E(\mathrm{EPS} \mid \text { declining interest rate environment })]^{2}$

$+P(\$ 2.45 \mid$ declining interest rate environment $)$ $\times[\$ 2.45-E(\mathrm{EPS} \mid \text { declining interest rate environment })]^{2}$

$=0.25(\$ 2.60-\$ 2.4875)^{2}+0.75(\$ 2.45-\$ 2.4875)^{2}=0.004219$

Similarly, $\sigma^{2}$ (EPS $\mid$ stable interest rate environment) is found to be equal to $=0.60(\$ 2.20-\$ 2.12)^{2}+0.40(\$ 2.00-\$ 2.12)^{2}=0.0096$

These are conditional variances, the variance of EPS given a declining interest rate environment and the variance of EPS given a stable interest rate environment. The relationship between unconditional variance and conditional variance is a relatively advanced topic. The main points are 1) that variance, like expected value, has a conditional counterpart to the unconditional concept and 2) that we can use conditional variance to assess risk given a particular scenario.

\section{EXAMPLE 12}
\section{BankCorp's Earnings per Share (5)}
Continuing with BankCorp, you focus now on BankCorp's cost structure. One model, a simple linear regression model, you are researching for BankCorp's operating costs is

$$
\widehat{Y}=a+b X
$$

where $\hat{Y}$ is a forecast of operating costs in millions of dollars and $X$ is the number of branch offices. $\widehat{Y}$ represents the expected value of $Y$ given $X$, or $E(Y \mid X)$. You interpret the intercept $a$ as fixed costs and $b$ as variable costs. You estimate the equation as

$$
\widehat{Y}=12.5+0.65 X
$$

BankCorp currently has 66 branch offices, and the equation estimates operating costs as $12.5+0.65(66)=\$ 55.4$ million. You have two scenarios for growth, pictured in the tree diagram in Exhibit 9.

\section{Exhibit 9: BankCorp's Forecasted Operating Costs}
\begin{center}
\includegraphics[max width=\textwidth]{2023_05_04_cff39ee44f77d6514e1bg-205}
\end{center}

\begin{enumerate}
  \item Compute the forecasted operating costs given the different levels of operating costs, using $\widehat{Y}=12.5+0.65 X$. State the probability of each level of the number of branch offices. These are the answers to the questions in the terminal boxes of the tree diagram.
\end{enumerate}

\section{Solution to 1:}
Using $\widehat{Y}=12.5+0.65 X$, from top to bottom, we have Operating Costs

Probability

$\widehat{Y}=12.5+0.65(125)=\$ 93.75$ million

$0.80(0.50)=0.40$

$\widehat{Y}=12.5+0.65(100)=\$ 77.50$ million

$0.80(0.50)=0.40$

$\widehat{Y}=12.5+0.65(80)=\$ 64.50$ million

$0.20(0.85)=0.17$

$\widehat{Y}=12.5+0.65(70)=\$ 58.00$ million

$0.20(0.15)=0.03$

Sum $=1.00$

\begin{enumerate}
  \setcounter{enumi}{1}
  \item Compute the expected value of operating costs under the high growth scenario. Also calculate the expected value of operating costs under the low growth scenario.
\end{enumerate}

\section{Solution to 2:}
Dollar amounts are in millions.

$E($ operating costs $\mid$ high growth $)=0.50(\$ 93.75)+0.50(\$ 77.50)$

$=\$ 85.625$

$E($ operating costs $\mid$ low growth $)=0.85(\$ 64.50)+0.15(\$ 58.00)$

$=\$ 63.525$

\begin{enumerate}
  \setcounter{enumi}{2}
  \item Answer the question in the initial box of the tree: What are BankCorp's expected operating costs?
\end{enumerate}

\section{Solution to 3:}
Dollar amounts are in millions.

$$
\begin{aligned}
& E(\text { operating costs })=E(\text { operating costs } \mid \text { high growth }) P(\text { high growth }) \\
& +E(\text { operating costs } \mid \text { low growth }) P(\text { low growth }) \\
& =\$ 85.625(0.80)+\$ 63.525(0.20)=\$ 81.205
\end{aligned}
$$

BankCorp's expected operating costs are $\$ 81.205$ million.

In this section, we have treated random variables such as EPS as standalone quantities. We have not explored how descriptors such as expected value and variance of EPS may be functions of other random variables. Portfolio return is one random variable that is clearly a function of other random variables, the random returns on the individual securities in the portfolio. To analyze a portfolio's expected return and variance of return, we must understand these quantities are a function of characteristics of the individual securities' returns. Looking at the variance of portfolio return, we see that the way individual security returns move together or covary is key. So, next we cover portfolio expected return, variance of return, and importantly, covariance and correlation.

\textbackslash section\{PORTFOLIO EXPECTED RETURN AND VARIANCE OF RETURN

calculate and interpret the expected value, variance, standard deviation, covariances, and correlations of portfolio returns

Modern portfolio theory makes frequent use of the idea that investment opportunities can be evaluated using expected return as a measure of reward and variance of return as a measure of risk. In this section, we will develop an understanding of portfolio expected return and variance of return, which are functions of the returns on the individual portfolio holdings. To begin, the expected return on a portfolio is a weighted average of the expected returns on the securities in the portfolio, using exactly the same weights. When we have estimated the expected returns on the individual securities, we immediately have portfolio expected return.

\begin{itemize}
  \item Calculation of Portfolio Expected Return. Given a portfolio with $n$ securities, the expected return on the portfolio $(E(R p))$ is a weighted average of the expected returns $\left(R_{1}\right.$ to $\left.R_{n}\right)$ on the component securities using their respective weights $\left(w_{1}\right.$ to $\left.w_{n}\right)$ :
\end{itemize}

$$
\begin{aligned}
& E\left(R_{p}\right)=E\left(w_{1} R_{1}+w_{2} R_{2}+\ldots+w_{n} R_{n}\right) \\
= & w_{1} E\left(R_{1}\right)+w_{2} E\left(R_{2}\right)+\ldots+w_{n} E\left(R_{n}\right)
\end{aligned}
$$

Suppose we have estimated expected returns on assets in the three-asset portfolio shown in Exhibit 10.

\section{Exhibit 10: Weights and Expected Returns}
\begin{center}
\begin{tabular}{lcc}
\hline
Asset Class & Weight & Expected Return (\%) \\
\hline
S\&P 500 & 0.50 & 13 \\
US long-term corporate bonds & 0.25 & 6 \\
MSCI EAFE & 0.25 & 15 \\
\hline
\end{tabular}
\end{center}

We calculate the expected return on the portfolio as $11.75 \%$ :

$$
\begin{aligned}
& E\left(R_{p}\right)=w_{1} E\left(R_{1}\right)+w_{2} E\left(R_{2}\right)+w_{3} E\left(R_{3}\right) \\
= & 0.50(13 \%)+0.25(6 \%)+0.25(15 \%)=11.75 \%
\end{aligned}
$$

Here we are interested in portfolio variance of return as a measure of investment risk. Accordingly, portfolio variance is $\sigma^{2}\left(R_{p}\right)=E\left\{\left[R_{p}-E\left(R_{p}\right)\right]^{2}\right\}$, which is variance in a forward-looking sense. To implement this definition of portfolio variance, we use information about the individual assets in the portfolio, but we also need the concept of covariance. To avoid notational clutter, we write $E R_{p}$ for $E\left(R_{p}\right)$.

\begin{itemize}
  \item Definition of Covariance. Given two random variables $R_{i}$ and $R_{j}$, the covariance between $R_{i}$ and $R_{j}$ is
\end{itemize}

$$
\operatorname{Cov}\left(R_{i}, R_{j}\right)=E\left[\left(R_{i}-E R_{i}\right)\left(R_{j}-E R_{j}\right)\right]
$$

Alternative notations are $\sigma\left(R_{i}, R_{j}\right)$ and $\sigma_{i j}$. Equation 14 states that the covariance between two random variables is the probability-weighted average of the cross-products of each random variable's deviation from its own expected value. The above measure is the population covariance and is forward-looking. The sample covariance between two random variables $R_{i}$ and $R_{j}$, based on a sample of past data of size $n$ is

$$
\operatorname{Cov}\left(R_{i}, R_{j}\right)=\sum_{n=1}^{n}\left(R_{i, t}-\bar{R}_{i}\right)\left(R_{j, t}-\bar{R}_{j}\right) /(n-1)
$$

Start with the definition of variance for a three-asset portfolio and see how it decomposes into three variance terms and six covariance terms. Dispensing with the derivation, the result is Equation 16:

$$
\begin{aligned}
& \sigma^{2}\left(R_{p}\right)=E\left[\left(R_{p}-E R_{p}\right)^{2}\right] \\
= & E\left\{\left[w_{1} R_{1}+w_{2} R_{2}+w_{3} R_{3}-E\left(w_{1} R_{1}+w_{2} R_{2}+w_{3} R_{3}\right)\right]^{2}\right\} \\
= & E\left\{\left[w_{1} R_{1}+w_{2} R_{2}+w_{3} R_{3}-w_{1} E R_{1}-w_{2} E R_{2}-w_{3} E R_{3}\right]^{2}\right\}
\end{aligned}
$$

(using Equation 13)

$$
\begin{gathered}
=w_{1}^{2} \sigma^{2}\left(R_{1}\right)+w_{1} w_{2} \operatorname{Cov}\left(R_{1}, R_{2}\right)+w_{1} w_{3} \operatorname{Cov}\left(R_{1}, R_{3}\right) \\
+w_{1} w_{2} \operatorname{Cov}\left(R_{1}, R_{2}\right)+w_{2}^{2} \sigma^{2}\left(R_{2}\right)+w_{2} w_{3} \operatorname{Cov}\left(R_{2}, R_{3}\right) \\
+w_{1} w_{3} \operatorname{Cov}\left(R_{1}, R_{3}\right)+w_{2} w_{3} \operatorname{Cov}\left(R_{2}, R_{3}\right)+w_{3}^{2} \sigma^{2}\left(R_{3}\right)
\end{gathered}
$$

Noting that the order of variables in covariance does not matter, for example, $\operatorname{Cov}\left(R_{2}, R_{1}\right)$ $=\operatorname{Cov}\left(R_{1}, R_{2}\right)$, and that diagonal variance terms $\sigma^{2}\left(R_{1}\right), \sigma^{2}\left(R_{2}\right)$, and $\sigma^{2}\left(R_{3}\right)$ can be expressed as $\operatorname{Cov}\left(R_{1}, R_{1}\right), \operatorname{Cov}\left(R_{2}, R_{2}\right)$, and $\operatorname{Cov}\left(R_{3}, R_{3}\right)$, respectively, the most compact way to state Equation 16 is $\sigma^{2}\left(R_{p}\right)=\sum_{i=1}^{3} \sum_{j=1}^{3} w_{i} w_{j} \operatorname{Cov}\left(R_{i}, R_{j}\right)$. Moreover, this expression generalizes for a portfolio of any size $n$ to

$$
\sigma^{2}\left(R_{p}\right)=\sum_{i=1}^{n} \sum_{j=1}^{n} w_{i} w_{j} \operatorname{Cov}\left(R_{i}, R_{j}\right)
$$

We see from Equation 16 that individual variances of return constitute part, but not all, of portfolio variance. The three variances are actually outnumbered by the six covariance terms off the diagonal. If there are 20 assets, there are 20 variance terms and 20(20) - 20 $=380$ off-diagonal covariance terms. A first observation is that as the number of holdings increases, covariance becomes increasingly important, all else equal.

The covariance terms capture how the co-movements of returns affect portfolio variance. From the definition of covariance, we can establish two essential observations about covariance.

\begin{enumerate}
  \item We can interpret the sign of covariance as follows:
\end{enumerate}

Covariance of returns is negative if, when the return on one asset is above its expected value, the return on the other asset tends to be below its expected value (an average inverse relationship between returns).

Covariance of returns is 0 if returns on the assets are unrelated.

Covariance of returns is positive when the returns on both assets tend to be on the same side (above or below) their expected values at the same time (an average positive relationship between returns). 2. The covariance of a random variable with itself (own covariance) is its own variance: $\operatorname{Cov}(R, R)=E\{[R-E(R)][R-E(R)]\}=E\left\{[R-E(R)]^{2}\right\}=\sigma^{2}(R)$.

Exhibit 11 summarizes the inputs for portfolio expected return and variance of return. A complete list of the covariances constitutes all the statistical data needed to compute portfolio variance of return as shown in the covariance matrix in Panel B.

\section{Exhibit 11: Inputs to Portfolio Expected Return and Variance}
\section{A. Inputs to Portfolio Expected Return}
\begin{center}
\begin{tabular}{llll}
\hline
Asset & A & B & C \\
\hline
 & $E\left(R_{A}\right)$ & $E\left(R_{B}\right)$ & $E\left(R_{C}\right)$ \\
\hline
\end{tabular}
\end{center}

B. Covariance Matrix: The Inputs to Portfolio Variance of Return

\begin{center}
\begin{tabular}{llll}
Asset & $\mathrm{A}$ & $\mathrm{B}$ & $\mathrm{C}$ \\
A & $\operatorname{Cov}\left(\boldsymbol{R}_{\mathrm{A}}, R_{\mathrm{A}}\right)$ & $\operatorname{Cov}\left(R_{\mathrm{A}}, R_{\mathrm{B}}\right)$ & $\operatorname{Cov}\left(R_{\mathrm{A}}, R_{\mathrm{C}}\right)$ \\
$\mathrm{B}$ & $\operatorname{Cov}\left(R_{\mathrm{B}}, R_{\mathrm{A}}\right)$ & $\operatorname{Cov}\left(\boldsymbol{R}_{\mathrm{B}}, \boldsymbol{R}_{\mathrm{B}}\right)$ & $\operatorname{Cov}\left(R_{\mathrm{B}}, R_{\mathrm{C}}\right)$ \\
$\mathrm{C}$ & $\operatorname{Cov}\left(R_{\mathrm{C}}, R_{\mathrm{A}}\right)$ & $\operatorname{Cov}\left(R_{\mathrm{C}}, R_{\mathrm{B}}\right)$ & $\operatorname{Cov}\left(\boldsymbol{R}_{\mathrm{C}}, \boldsymbol{R}_{\mathrm{C}}\right)$ \\
\hline
\end{tabular}
\end{center}

With three assets, the covariance matrix has $3^{2}=3 \times 3=9$ entries, but the diagonal terms, the variances (bolded in Exhibit 11), are treated separately from the off-diagonal terms. So there are $9-3=6$ covariances, excluding variances. But $\operatorname{Cov}\left(R_{\mathrm{B}}, R_{\mathrm{A}}\right)=$ $\operatorname{Cov}\left(R_{\mathrm{A}}, R_{\mathrm{B}}\right), \operatorname{Cov}\left(R_{\mathrm{C}}, R_{\mathrm{A}}\right)=\operatorname{Cov}\left(R_{\mathrm{A}}, R_{\mathrm{C}}\right)$, and $\operatorname{Cov}\left(R_{\mathrm{C}}, R_{\mathrm{B}}\right)=\operatorname{Cov}\left(R_{\mathrm{B}}, R_{\mathrm{C}}\right)$. The covariance matrix below the diagonal is the mirror image of the covariance matrix above the diagonal, so you only need to use one (i.e., either below or above the diagonal). As a result, there are only $6 / 2=3$ distinct covariance terms to estimate. In general, for $n$ securities, there are $n(n-1) / 2$ distinct covariances and $n$ variances to estimate.

Suppose we have the covariance matrix shown in Exhibit 12. We will be working in returns stated as percents, and the table entries are in units of percent squared $\left(\%^{2}\right)$. The terms $38 \%^{2}$ and $400 \%^{2}$ are 0.0038 and 0.0400 , respectively, stated as decimals; correctly working in percents and decimals leads to identical answers.

\section{Exhibit 12: Covariance Matrix}
\begin{center}
\begin{tabular}{lccc}
\hline
 & S\&P 500 & $\begin{array}{c}\text { US Long-Term } \\ \text { Corporate Bonds }\end{array}$ & $\begin{array}{c}\text { MSCI } \\ \text { EAFE }\end{array}$ \\
\hline
S\&P 500 & 400 & 45 & 189 \\
US long-term corporate bonds & 45 & 81 & 38 \\
MSCI EAFE & 189 & 38 & 441 \\
\hline
\end{tabular}
\end{center}

Taking Equation 16 and grouping variance terms together produces the following:

$$
\begin{aligned}
& \sigma^{2}\left(R_{p}\right)=w_{1}^{2} \sigma^{2}\left(R_{1}\right)+w_{2}^{2} \sigma^{2}\left(R_{2}\right)+w_{3}^{2} \sigma^{2}\left(R_{3}\right)+2 w_{1} w_{2} \operatorname{Cov}\left(R_{1}, R_{2}\right) \\
& +2 w_{1} w_{3} \operatorname{Cov}\left(R_{1}, R_{3}\right)+2 w_{2} w_{3} \operatorname{Cov}\left(R_{2}, R_{3}\right) \\
= & (0.50)^{2}(400)+(0.25)^{2}(81)+(0.25)^{2}(441) \\
+ & 2(0.50)(0.25)(45)+2(0.50)(0.25)(189) \\
+ & 2(0.25)(0.25)(38) \\
= & 100+5.0625+27.5625+11.25+47.25+4.75=195.875
\end{aligned}
$$

The variance is 195.875. Standard deviation of return is $195.875^{1 / 2}=14 \%$. To summarize, the portfolio has an expected annual return of $11.75 \%$ and a standard deviation of return of $14 \%$.

Looking at the first three terms in the calculation above, their sum $(100+5.0625+$ 27.5625 ) is 132.625 , the contribution of the individual variances to portfolio variance. If the returns on the three assets were independent, covariances would be 0 and the standard deviation of portfolio return would be $132.625^{1 / 2}=11.52 \%$ as compared to $14 \%$ before, so a less risky portfolio. If the covariance terms were negative, then a negative number would be added to 132.625 , so portfolio variance and risk would be even smaller, while expected return would not change. For the same expected portfolio return, the portfolio has less risk. This risk reduction is a diversification benefit, meaning a risk-reduction benefit from holding a portfolio of assets. The diversification benefit increases with decreasing covariance. This observation is a key insight of modern portfolio theory. This insight is even more intuitively stated when we can use the concept of correlation.

\begin{itemize}
  \item Definition of Correlation. The correlation between two random variables, $R_{i}$ and $R_{j}$, is defined as $\rho\left(R_{i}, R_{j}\right)=\operatorname{Cov}\left(R_{i}, R_{j}\right) /\left[\sigma\left(R_{i}\right) \sigma\left(R_{j}\right)\right]$. Alternative notations are $\operatorname{Corr}\left(R_{i}, R_{j}\right)$ and $\rho_{i j}$.
\end{itemize}

The above definition of correlation is forward-looking because it involves dividing the forward-looking covariance by the product of forward-looking standard deviations. Frequently, covariance is substituted out using the relationship $\operatorname{Cov}\left(R_{i} R_{j}\right)=\rho\left(R_{i} R_{j}\right)$ $\sigma\left(R_{i}\right) \sigma\left(R_{j}\right)$. Like covariance, the correlation coefficient is a measure of linear association. However, the division in the definition makes correlation a pure number (without a unit of measurement) and places bounds on its largest and smallest possible values, which are +1 and -1 , respectively.

If two variables have a strong positive linear relation, then their correlation will be close to +1 . If two variables have a strong negative linear relation, then their correlation will be close to -1 . If two variables have a weak linear relation, then their correlation will be close to 0 . Using the above definition, we can state a correlation matrix from data in the covariance matrix alone. Exhibit 13 shows the correlation matrix.

\section{Exhibit 13: Correlation Matrix of Returns}
\begin{center}
\begin{tabular}{|c|c|c|c|}
\hline
 & S\&P 500 & $\begin{array}{l}\text { US Long-Term } \\ \text { Corporate Bonds }\end{array}$ & MSCI EAFE \\
\hline
S\&P 500 & 1.00 & 0.25 & 0.45 \\
\hline
US long-term corporate bonds & 0.25 & 1.00 & 0.20 \\
\hline
MSCI EAFE & 0.45 & 0.20 & 1.00 \\
\hline
\end{tabular}
\end{center}

For example, the covariance between long-term bonds and MSCI EAFE is 38, from Exhibit 12. The standard deviation of long-term bond returns is $81^{1 / 2}=9 \%$, that of MSCI EAFE returns is $441^{1 / 2}=21 \%$, from diagonal terms in Exhibit 12. The correlation $\rho\left(R_{\text {long-term bonds }}, R_{\text {EAFE }}\right)$ is $38 /[(9 \%)(21 \%)]=0.201$, rounded to 0.20 . The correlation of the S\&P 500 with itself equals 1: The calculation is its own covariance divided by its standard deviation squared.

\section{EXAMPLE 13}
\section{Portfolio Expected Return and Variance of Return with Varying Portfolio Weights}
Anna Cintara is constructing different portfolios from the following two stocks:

\section{Exhibit 14: Description of Two-Stock Portfolio}
\begin{center}
\begin{tabular}{lcc}
\hline
 & Stock 1 & Stock 2 \\
\hline
Expected return & $4 \%$ & $8 \%$ \\
Standard deviation & $6 \%$ & $15 \%$ \\
Current portfolio weights & 0.40 & 0.60 \\
Correlation between returns & 0.30 &  \\
\hline
\end{tabular}
\end{center}

\begin{enumerate}
  \item Calculate the covariance between the returns on the two stocks.
\end{enumerate}

\section{Solution to 1:}
The correlation between two stock returns is $\rho\left(R_{i}, R_{j}\right)=\operatorname{Cov}\left(R_{i}, R_{j}\right) /\left[\sigma\left(R_{i}\right)\right.$ $\left.\sigma\left(R_{j}\right)\right]$, so the covariance is $\operatorname{Cov}\left(R_{i}, R_{j}\right)=\rho\left(R_{i}, R_{j}\right) \sigma\left(R_{i}\right) \sigma\left(R_{j}\right)$. For these two stocks, the covariance is $\operatorname{Cov}\left(R_{1}, R_{2}\right)=\rho\left(R_{1}, R_{2}\right) \sigma\left(R_{1}\right) \sigma\left(R_{2}\right)=0.30$ (6) (15) = 27.

\begin{enumerate}
  \setcounter{enumi}{1}
  \item What is the portfolio expected return and standard deviation if Cintara puts $100 \%$ of her investment in Stock $1\left(w_{1}=1.00\right.$ and $\left.w_{2}=0.00\right)$ ? What is the portfolio expected return and standard deviation if Cintara puts $100 \%$ of her investment in Stock $2\left(w_{1}=0.00\right.$ and $\left.w_{2}=1.00\right)$ ?
\end{enumerate}

\section{Solution to 2:}
If the portfolio is $100 \%$ invested in Stock 1 , the portfolio has an expected return of $4 \%$ and a standard deviation of $6 \%$. If the portfolio is $100 \%$ invested in Stock 2, the portfolio has an expected return of $8 \%$ and a standard deviation of $15 \%$

\begin{enumerate}
  \setcounter{enumi}{2}
  \item What are the portfolio expected return and standard deviation using the current portfolio weights?
\end{enumerate}

\section{Solution to 3:}
For the current $40 / 60$ portfolio, the expected return is

$$
E\left(R_{p}\right)=w_{1} E\left(R_{1}\right)+\left(1-w_{1}\right) E\left(R_{2}\right)=0.40(4 \%)+0.60(8 \%)=6.4 \%
$$

The portfolio variance and standard deviation are

$$
\begin{aligned}
& \sigma^{2}\left(R_{p}\right)=w_{1}^{2} \sigma^{2}\left(R_{1}\right)+w_{2}^{2} \sigma^{2}\left(R_{2}\right)+2 w_{1} w_{2} \operatorname{Cov}\left(R_{1}, R_{2}\right) \\
= & \left({ }^{0.40}\right)^{2}(36)+\left({ }^{0.60}\right)^{2}(225)+2(0.40)(0.60)(27) \\
= & 5.76+81+12.94=99.72 \\
& \sigma\left(R_{p}\right)=99.72^{1 / 2}=9.99 \%
\end{aligned}
$$

\begin{enumerate}
  \setcounter{enumi}{3}
  \item Calculate the expected return and standard deviation of the portfolios when $w_{1}$ goes from 0.00 to 1.00 in 0.10 increments (and $w_{2}=1-w_{1}$ ). Place the results (stock weights, portfolio expected return, and portfolio standard deviation) in a table, and then sketch a graph of the results with the standard deviation on the horizontal axis and expected return on the vertical axis.
\end{enumerate}

Solution to 4:

The portfolio expected returns, variances, and standard deviations for the different sets of portfolio weights are given in the following table. Three of the rows are already computed in the Solutions to 2 and 3, and the other rows are computed using the same expected return, variance, and standard deviation formulas as in the Solution to 3 :

\begin{center}
\begin{tabular}{lcccc}
\hline
Stock 1 & $\begin{array}{c}\text { Stock 2 } \\ \text { weight }\end{array}$ & $\begin{array}{c}\text { Expected } \\ \text { return (\%) }\end{array}$ & $\begin{array}{c}\text { Variance } \\ \mathbf{( \%}^{\mathbf{2})}\end{array}$ & $\begin{array}{c}\text { Standard } \\ \text { deviation (\%) }\end{array}$ \\
\hline
1.00 & 0.00 & 4.00 & 36.00 & 6.00 \\
0.90 & 0.10 & 4.40 & 36.27 & 6.02 \\
0.80 & 0.20 & 4.80 & 40.68 & 6.38 \\
0.70 & 0.30 & 5.20 & 49.23 & 7.02 \\
0.60 & 0.40 & 5.60 & 61.92 & 7.87 \\
0.50 & 0.50 & 6.00 & 78.75 & 8.87 \\
0.40 & 0.60 & 6.40 & 99.72 & 9.99 \\
0.30 & 0.70 & 6.80 & 124.83 & 11.17 \\
0.20 & 0.80 & 7.20 & 154.08 & 12.41 \\
0.10 & 0.90 & 8.00 & 187.47 & 13.69 \\
0.00 & 1.00 &  & 225.00 & 15.00 \\
\hline
\end{tabular}
\end{center}

The graph of the expected return and standard deviation is

\begin{center}
\includegraphics[max width=\textwidth]{2023_05_04_cff39ee44f77d6514e1bg-212}
\end{center}

\section{COVARIANCE GIVEN A JOINT PROBABILITY FUNCTION}
calculate and interpret the covariances of portfolio returns using the joint probability function How do we estimate return covariance and correlation? Frequently, we make forecasts on the basis of historical covariance or use other methods such as a market model regression based on historical return data. We can also calculate covariance using the joint probability function of the random variables, if that can be estimated. The joint probability function of two random variables $X$ and $Y$, denoted $P(X, Y)$, gives the probability of joint occurrences of values of $X$ and $Y$. For example, $P(X=3, Y=2)$, is the probability that $X$ equals 3 and $Y$ equals 2 .

Suppose that the joint probability function of the returns on BankCorp stock $\left(R_{A}\right)$ and the returns on NewBank stock $\left(R_{B}\right)$ has the simple structure given in Exhibit 15 .

Exhibit 15: Joint Probability Function of BankCorp and NewBank Returns (Entries Are Joint Probabilities)

\begin{center}
\begin{tabular}{lccc}
\hline
 & $\boldsymbol{R}_{\boldsymbol{B}}=\mathbf{2 0 \%}$ & $\boldsymbol{R}_{\boldsymbol{B}}=\mathbf{1 6 \%}$ & $\boldsymbol{R}_{\boldsymbol{B}}=\mathbf{1 0 \%}$ \\
\hline
$R_{A}=25 \%$ & 0.20 & 0 & 0 \\
$R_{A}=12 \%$ & 0 & 0.50 & 0 \\
$R_{A}=10 \%$ & 0 & 0 & 0.30 \\
\hline
\end{tabular}
\end{center}

The expected return on BankCorp stock is $0.20(25 \%)+0.50(12 \%)+0.30(10 \%)=$ $14 \%$. The expected return on NewBank stock is $0.20(20 \%)+0.50(16 \%)+0.30(10 \%)=$ $15 \%$. The joint probability function above might reflect an analysis based on whether banking industry conditions are good, average, or poor. Exhibit 16 presents the calculation of covariance.

\section{Exhibit 16: Covariance Calculations}
\begin{center}
\begin{tabular}{lcccc}
$\begin{array}{l}\text { Banking } \\ \text { Industry } \\ \text { Condition }\end{array}$ & $\begin{array}{c}\text { Deviations } \\ \text { BankCorp }\end{array}$ & $\begin{array}{c}\text { Deviations } \\ \text { NewBank }\end{array}$ & $\begin{array}{c}\text { Product of } \\ \text { Deviations }\end{array}$ & $\begin{array}{c}\text { Probability of } \\ \text { Condition }\end{array}$ \\
\hline
Good & $25-14$ & $20-15$ & 55 & 0.20 \\
Probability-Weighted &  &  &  &  \\
Product &  &  &  &  \\
\end{tabular}
\end{center}

Note: Expected return for BankCorp is $14 \%$ and for NewBank, $15 \%$.

The first and second columns of numbers show, respectively, the deviations of BankCorp and NewBank returns from their mean or expected value. The next column shows the product of the deviations. For example, for good industry conditions, $(25-14)(20-15)$ $=11(5)=55$. Then, 55 is multiplied or weighted by 0.20 , the probability that banking industry conditions are good: $55(0.20)=11$. The calculations for average and poor banking conditions follow the same pattern. Summing up these probability-weighted products, we find $\operatorname{Cov}\left(R_{A}, R_{B}\right)=16$.

A formula for computing the covariance between random variables $R_{A}$ and $R_{B}$ is

$\operatorname{Cov}\left(R_{A}, R_{B}\right)=\sum_{i} \sum_{j} P\left(R_{A, i}, R_{B, j}\right)\left(R_{A, i}-E R_{A}\right)\left(R_{B, j}-E R_{B}\right)$

The formula tells us to sum all possible deviation cross-products weighted by the appropriate joint probability. Next, we take note of the fact that when two random variables are independent, their joint probability function simplifies.

\begin{itemize}
  \item Definition of Independence for Random Variables. Two random variables $X$ and $Y$ are independent if and only if $P(X, Y)=P(X) P(Y)$.
\end{itemize}

For example, given independence, $P(3,2)=P(3) P(2)$. We multiply the individual probabilities to get the joint probabilities. Independence is a stronger property than uncorrelatedness because correlation addresses only linear relationships. The following condition holds for independent random variables and, therefore, also holds for uncorrelated random variables.

\begin{itemize}
  \item Multiplication Rule for Expected Value of the Product of Uncorrelated Random Variables. The expected value of the product of uncorrelated random variables is the product of their expected values.
\end{itemize}

$E(X Y)=E(X) E(Y)$ if $X$ and $Y$ are uncorrelated.

Many financial variables, such as revenue (price times quantity), are the product of random quantities. When applicable, the above rule simplifies calculating expected value of a product of random variables.

\section{EXAMPLE 14}
\section{Covariances and Correlations of Security Returns}
\begin{enumerate}
  \item Isabel Vasquez is reviewing the correlations between four of the asset classes in her company portfolio. In Exhibit 17, she plots 24 recent monthly returns for large-cap US stocks versus for large-cap world ex-US stocks (Panel 1) and the 24 monthly returns for intermediate-term corporate bonds versus long-term corporate bonds (Panel 2). Vasquez presents the returns, variances, and covariances in decimal form instead of percentage form. Note the different ranges of their vertical axes (Return \%).
\end{enumerate}

\section{Exhibit 17: Monthly Returns for Four Asset Classes}
\begin{center}
\includegraphics[max width=\textwidth]{2023_05_04_cff39ee44f77d6514e1bg-215(1)}
\end{center}

B. Corporate Bond Monthly Returns Return (\%)

\begin{center}
\includegraphics[max width=\textwidth]{2023_05_04_cff39ee44f77d6514e1bg-215}
\end{center}

Selected data for the four asset classes are shown in Exhibit 18.

\section{Exhibit 18: Selected Data for Four Asset Classes}
\begin{center}
\begin{tabular}{|l|c|c|c|c|}
\hline
\multicolumn{1}{|c|}{Asset Classes} & $\begin{array}{c}\text { Large-Cap } \\ \text { US } \\ \text { Equities }\end{array}$ & $\begin{array}{c}\text { World } \\ \text { (ex US) } \\ \text { Equities }\end{array}$ & $\begin{array}{c}\text { Intermediate } \\ \text { Corp Bonds }\end{array}$ & $\begin{array}{c}\text { Long-Term } \\ \text { Corp Bonds }\end{array}$ \\
\hline
Variance & 0.001736 & 0.001488 & 0.000174 & 0.000699 \\
\hline
Standard deviation & 0.041668 & 0.038571 & 0.013180 & 0.026433 \\
\hline
Covariance & \multicolumn{2}{|c|}{0.001349} & \multicolumn{2}{|c|}{0.000318} \\
\hline
Correlation & \multicolumn{2}{|c|}{0.87553} & \multicolumn{2}{c|}{0.95133} \\
\hline
\end{tabular}
\end{center}

Vasquez noted, as shown in Exhibit 18, that although the two equity classes had much greater variances and covariance than the two bond classes, the correlation between the two equity classes was lower than the correlation between the two bond classes. She also noted that although long-term bonds were more volatile (higher variance) than intermediate-term bonds, long- and intermediate-term bond returns still had a high correlation.

\section{BAYES' FORMULA}
calculate and interpret an updated probability using Bayes' formula

A topic that is often useful in solving investment problems is Bayes' formula: what probability theory has to say about learning from experience.

\section{Bayes' Formula}
When we make decisions involving investments, we often start with viewpoints based on our experience and knowledge. These viewpoints may be changed or confirmed by new knowledge and observations. Bayes' formula is a rational method for adjusting our viewpoints as we confront new information. Bayes' formula and related concepts have been applied in many business and investment decision-making contexts.

Bayes' formula makes use of Equation 6, the total probability rule. To review, that rule expressed the probability of an event as a weighted average of the probabilities of the event, given a set of scenarios. Bayes' formula works in reverse; more precisely, it reverses the "given that" information. Bayes' formula uses the occurrence of the event to infer the probability of the scenario generating it. For that reason, Bayes' formula is sometimes called an inverse probability. In many applications, including those illustrating its use in this section, an individual is updating his/her beliefs concerning the causes that may have produced a new observation.

\begin{itemize}
  \item Bayes' Formula. Given a set of prior probabilities for an event of interest, if you receive new information, the rule for updating your probability of the event is
\end{itemize}

Updated probability of event given the new information

$=\frac{\text { Probability of the new information given event }}{\text { Unconditional probability of the new information }} \times$ Prior probability of event

In probability notation, this formula can be written concisely as

$P($ Event $\mid$ Information $)=\frac{P(\text { Information } \mid \text { Event })}{P(\text { Information })} P($ Event $)$

Consider the following example using frequencies-which may be more straightforward initially than probabilities-for illustrating and understanding Bayes' formula. Assume a hypothetical large-cap stock index has 500 member firms, of which 100 are technology firms, and 60 of these had returns of $>10 \%$, and 40 had returns of $\leq 10 \%$. Of the 400 non-technology firms in the index, 100 had returns of $>10 \%$, and $300 \mathrm{had}$ returns of $\leq 10 \%$. The tree map in Exhibit 19 is useful for visualizing this example, which is summarized in the table in Exhibit 20.

\section{Exhibit 19: Tree Map for Visualizing Bayes' Formula Using Frequencies}
\begin{center}
\includegraphics[max width=\textwidth]{2023_05_04_cff39ee44f77d6514e1bg-217}
\end{center}

Exhibit 20: Summary of Returns for Tech and Non-Tech Firms in Hypothetical Large-Cap Equity Index

\begin{center}
\begin{tabular}{lccc}
\hline
 & \multicolumn{2}{c}{Type of Firm in Stock Index} &  \\
\cline { 2 - 4 }
Rate of Return (R) & Non-Tech & Tech & Total \\
\hline
$R>10 \%$ & 100 & 60 & 160 \\
$R \leq 10 \%$ & 300 & 40 & $\mathbf{3 4 0}$ \\
Total & $\mathbf{4 0 0}$ & $\mathbf{1 0 0}$ & $\mathbf{5 0 0}$ \\
\hline
\end{tabular}
\end{center}

What is the probability a firm is a tech firm given that it has a return of $>10 \%$ or $P($ tech $\mid R>10 \%)$ ? Looking at the frequencies in the tree map and in the table, we can see many empirical probabilities, such as the following:

\begin{itemize}
  \item $P($ tech $)=100 / 500=0.20$,

  \item $P($ non-tech $)=400 / 500=0.80$,

  \item $P(R>10 \% \mid$ tech $)=60 / 100=0.60$,

  \item $P(R>10 \% \mid$ non-tech $)=100 / 400=0.25$,

  \item $P(R>10 \%)=160 / 500=0.32$, and, finally,

  \item $P($ tech $\mid R>10 \%)=60 / 160=0.375$. This probability is the answer to our initial question.

\end{itemize}

Without looking at frequencies, let us use Bayes' formula to find the probability that a firm has a return of $>10 \%$ and then the probability that a firm with a return of $>10 \%$ is a tech firm, $P($ tech $\mid R>10 \%)$. First,

$$
\begin{aligned}
& P(R>10 \%)=P(R>10 \% \mid \text { tech }) \times P(\text { tech })+P(R>10 \% \mid \text { non-tech }) \times P(\text { non-tech }) \\
& =0.60 \times 0.20+0.25 \times 0.80=0.32 .
\end{aligned}
$$

Now we can implement the Bayes' formula answer to our question:

$$
P(\text { tech } \mid R>10 \%)=\frac{P(R>10 \% \mid \text { tech }) \times P(\text { tech })}{P(R>10 \%)}=\frac{0.60 \times 0.20}{0.32}=0.375
$$

The probability that a firm with a return of $>10 \%$ is a tech firm is 0.375 , which is impressive because the probability that a firm is a tech firm (from the whole sample) is only 0.20 . In sum, it can be readily seen from the tree map and the underlying frequency data (Exhibits 19 and 20, respectively) or from the probabilities in Bayes' formula that there are 160 firms with $R>10 \%$, and 60 of them are tech firms, so $P($ tech $|R\rangle 10 \%)=60 / 160=.375$.

Users of Bayesian statistics do not consider probabilities (or likelihoods) to be known with certainty but that these should be subject to modification whenever new information becomes available. Our beliefs or probabilities are continually updated as new information arrives over time.

To further illustrate Bayes' formula, we work through an investment example that can be adapted to any actual problem. Suppose you are an investor in the stock of DriveMed, Inc. Positive earnings surprises relative to consensus EPS estimates often result in positive stock returns, and negative surprises often have the opposite effect. DriveMed is preparing to release last quarter's EPS result, and you are interested in which of these three events happened: last quarter's EPS exceeded the consensus EPS estimate, last quarter's EPS exactly met the consensus EPS estimate, or last quarter's EPS fell short of the consensus EPS estimate. This list of the alternatives is mutually exclusive and exhaustive.

On the basis of your own research, you write down the following prior probabilities (or priors, for short) concerning these three events:

\begin{itemize}
  \item $P(E P S$ exceeded consensus $)=0.45$

  \item $P(E P S$ met consensus $)=0.30$

  \item $P(E P S$ fell short of consensus $)=0.25$

\end{itemize}

These probabilities are "prior" in the sense that they reflect only what you know now, before the arrival of any new information.

The next day, DriveMed announces that it is expanding factory capacity in Singapore and Ireland to meet increased sales demand. You assess this new information. The decision to expand capacity relates not only to current demand but probably also to the prior quarter's sales demand. You know that sales demand is positively related to EPS. So now it appears more likely that last quarter's EPS will exceed the consensus.

The question you have is, "In light of the new information, what is the updated probability that the prior quarter's EPS exceeded the consensus estimate?"

Bayes' formula provides a rational method for accomplishing this updating. We can abbreviate the new information as DriveMed expands. The first step in applying Bayes' formula is to calculate the probability of the new information (here: DriveMed expands), given a list of events or scenarios that may have generated it. The list of events should cover all possibilities, as it does here. Formulating these conditional probabilities is the key step in the updating process. Suppose your view, based on research of DriveMed and its industry, is

$$
\begin{aligned}
& P(\text { DriveMed expands } \mid \text { EPS exceeded consensus })=0.75 \\
& P(\text { DriveMed expands } \mid \text { EPS met consensus })=0.20 \\
& P(\text { DriveMed expands } \mid E P S \text { fell short of consensus })=0.05
\end{aligned}
$$

Conditional probabilities of an observation (here: DriveMed expands) are sometimes referred to as likelihoods. Again, likelihoods are required for updating the probability. Next, you combine these conditional probabilities or likelihoods with your prior probabilities to get the unconditional probability for DriveMed expanding, $P($ DriveMed expands), as follows:

$$
\begin{aligned}
& P(\text { DriveMed expands }) \\
&= P(\text { DriveMed expands } \mid \text { EPS exceeded consensus }) \\
& \times P(\text { EPS exceeded consensus }) \\
&+ P(\text { DriveMed expands } \mid \text { EPS met consensus }) \\
& \quad \times P(E P S \text { met consensus }) \\
&+ P(\text { DriveMed expands } \mid \text { EPS fell short of consensus }) \\
& \quad \times P(\text { EPS fell short of consensus }) \\
&= 0.75(0.45)+0.20(0.30)+0.05(0.25)=0.41, \text { or } 41 \%
\end{aligned}
$$

This is Equation 6, the total probability rule, in action. Now you can answer your question by applying Bayes' formula:

$$
\begin{aligned}
& P(\text { EPS exceeded consensus } \mid \text { DriveMed expands }) \\
= & \frac{P(\text { DriveMed expands } \mid \text { EPS exceeded consensus })}{P(\text { DriveMed expands })} P(\text { EPS exceeded consensus }) \\
= & (0.75 / 0.41)(0.45)=1.829268(0.45)=0.823171
\end{aligned}
$$

Prior to DriveMed's announcement, you thought the probability that DriveMed would beat consensus expectations was $45 \%$. On the basis of your interpretation of the announcement, you update that probability to $82.3 \%$. This updated probability is called your posterior probability because it reflects or comes after the new information.

The Bayes' calculation takes the prior probability, which was $45 \%$, and multiplies it by a ratio-the first term on the right-hand side of the equal sign. The denominator of the ratio is the probability that DriveMed expands, as you view it without considering (conditioning on) anything else. Therefore, this probability is unconditional. The numerator is the probability that DriveMed expands, if last quarter's EPS actually exceeded the consensus estimate. This last probability is larger than unconditional probability in the denominator, so the ratio (1.83 roughly) is greater than 1 . As a result, your updated or posterior probability is larger than your prior probability. Thus, the ratio reflects the impact of the new information on your prior beliefs.

\section{EXAMPLE 15}
\section{Inferring Whether DriveMed's EPS Met Consensus EPS}
You are still an investor in DriveMed stock. To review the givens, your prior probabilities are $P(E P S$ exceeded consensus $)=0.45, P(E P S$ met consensus $)=0.30$, and $P(E P S$ fell short of consensus $)=0.25$. You also have the following conditional probabilities:

$$
\begin{aligned}
& P(\text { DriveMed expands } \mid \text { EPS exceeded consensus })=0.75 \\
& P(\text { DriveMed expands } \mid \text { EPS met consensus })=0.20 \\
& P(\text { DriveMed expands } \mid \text { EPS fell short of consensus })=0.05
\end{aligned}
$$

Recall that you updated your probability that last quarter's EPS exceeded the consensus estimate from $45 \%$ to $82.3 \%$ after DriveMed announced it would expand. Now you want to update your other priors.

What is your estimate of the probability $P(E P S$ exceeded consensus | DriveMed expands)? 1. Update your prior probability that DriveMed's EPS met consensus.

\section{Solution to 1:}
The probability is $P(E P S$ met consensus $\mid$ DriveMed expands $)=$ $\frac{P(\text { DriveMed expands } \mid \text { EPS met consensus })}{P(\text { DriveMed expands })} P($ EPS met consensus $)$

The probability $P($ DriveMed expands) is found by taking each of the three conditional probabilities in the statement of the problem, such as $P$ (DriveMed expands $\mid$ EPS exceeded consensus); multiplying each one by the prior probability of the conditioning event, such as $P(E P S$ exceeded consensus); then adding the three products. The calculation is unchanged from the problem in the text above: $P($ DriveMed expands $)=0.75(0.45)+0.20(0.30)$ $+0.05(0.25)=0.41$, or $41 \%$. The other probabilities needed, $P($ DriveMed expands $\mid E P S$ met consensus $)=0.20$ and $P(E P S$ met consensus $)=0.30$, are givens. So

$P(E P S$ met consensus $\mid$ DriveMed expands $)$

$=[P($ DriveMed expands $\mid$ EPS met consensus $) / P($ DriveMed expands $)] P(E P S$ met consensus)

$=(0.20 / 0.41)(0.30)=0.487805(0.30)=0.146341$

After taking account of the announcement on expansion, your updated probability that last quarter's EPS for DriveMed just met consensus is $14.6 \%$ compared with your prior probability of $30 \%$.

\begin{enumerate}
  \setcounter{enumi}{1}
  \item Update your prior probability that DriveMed's EPS fell short of consensus.
\end{enumerate}

Solution to 2:

$P($ DriveMed expands) was already calculated as $41 \%$. Recall that $P($ DriveMed expands $\mid$ EPS fell short of consensus $)=0.05$ and $P($ EPS fell short of consensus) $=0.25$ are givens.

$$
\begin{aligned}
& P(\text { EPS fell short of consensus } \mid \text { DriveMed expands }) \\
& =[P(\text { DriveMed expands } \mid \text { EPS fell short of consensus }) / \\
& P(\text { DriveMed expands })] P(\text { EPS fell short of consensus }) \\
& =(0.05 / 0.41)(0.25)=0.121951(0.25)=0.030488
\end{aligned}
$$

As a result of the announcement, you have revised your probability that DriveMed's EPS fell short of consensus from 25\% (your prior probability) to $3 \%$.

\begin{enumerate}
  \setcounter{enumi}{2}
  \item Show that the three updated probabilities sum to 1. (Carry each probability to four decimal places.)
\end{enumerate}

\section{Solution to 3:}
The sum of the three updated probabilities is

$P($ EPS exceeded consensus $\mid$ DriveMed expands $)+P($ EPS met consensus $\mid$

DriveMed expands $)+P($ EPS fell short of consensus $\mid$ DriveMed expands $)$

$=0.8232+0.1463+0.0305=1.0000$

The three events (EPS exceeded consensus, EPS met consensus, EPS fell short of consensus) are mutually exclusive and exhaustive: One of these events or statements must be true, so the conditional probabilities must sum to 1 . Whether we are talking about conditional or unconditional probabilities, whenever we have a complete set of distinct possible events or outcomes, the probabilities must sum to 1 . This calculation serves to check your work.

\begin{enumerate}
  \setcounter{enumi}{3}
  \item Suppose, because of lack of prior beliefs about whether DriveMed would meet consensus, you updated on the basis of prior probabilities that all three possibilities were equally likely: $P(E P S$ exceeded consensus $)=P(E P S$ met consensus $)=P(E P S$ fell short of consensus $)=1 / 3$.
\end{enumerate}

\section{Solution to 4:}
Using the probabilities given in the question,

$$
\begin{aligned}
& P(\text { DriveMed expands }) \\
& =P(\text { DriveMed expands } \mid \text { EPS exceeded consensus }) \\
& P(\text { EPS exceeded consensus })+P(\text { DriveMed expands } \mid \\
& E P S \text { met consensus }) P(\text { EPS met consensus })+P(\text { DriveMed expands } \mid \\
& E P S \text { fell short of consensus }) P(\text { EPS fell short of consensus }) \\
= & 0.75(1 / 3)+0.20(1 / 3)+0.05(1 / 3)=1 / 3
\end{aligned}
$$

Not surprisingly, the probability of DriveMed expanding is $1 / 3$ because the decision maker has no prior beliefs or views regarding how well EPS performed relative to the consensus estimate.

Now we can use Bayes' formula to find $P(E P S$ exceeded consensus $\mid$ DriveMed expands $)=[P($ DriveMed expands $\mid$ EPS exceeded consensus $) / P($ DriveMed expands $)] P($ EPS exceeded consensus $)=[(0.75 /(1 / 3)](1 / 3)$ $=0.75$ or $75 \%$. This probability is identical to your estimate of $P$ (DriveMed expands | EPS exceeded consensus).

When the prior probabilities are equal, the probability of information given an event equals the probability of the event given the information. When a decision maker has equal prior probabilities (called diffuse priors), the probability of an event is determined by the information.

Example 16 shows how Bayes' formula is used in credit granting where the probability of payment given credit information is higher than the probability of payment without the information.

\section{EXAMPLE 16}
\section{Bayes' Formula and the Probability of Payment}
\begin{enumerate}
  \item Jake Bronson is predicting the probability that consumer finance applicants granted credit will repay in a timely manner (i.e., their accounts will not become "past due"). Using Bayes' formula, he has structured the problem as
\end{enumerate}

$P($ Event $\mid$ Information $)=\frac{P(\text { Information } \mid \text { Event })}{P(\text { Information })} P($ Event $)$,

where the event $(A)$ is "timely repayment" and the information $(B)$ is having a "good credit report."

Bronson estimates that the unconditional probability of receiving timely payment, $P(A)$, is 0.90 and that the unconditional probability of having a good credit report, $P(B)$, is 0.80 . The probability of having a good credit report given that borrowers paid on time, $P(B \mid A)$, is 0.85 .

What is the probability that applicants with good credit reports will repay in a timely manner?
A. 0.720
B. 0.944
C. 0.956

Solution:

The correct answer is $C$. The probability of timely repayment given a good credit report, $P(A \mid B)$, is

$$
P(A \mid B)=\frac{P(B \mid A)}{P(B)} P(A)=\frac{0.85}{0.80} \times 0.90=0.956
$$

\section{PRINCIPLES OF COUNTING}
identify the most appropriate method to solve a particular counting problem and analyze counting problems using factorial, combination, and permutation concepts

The first step in addressing a question often involves determining the different logical possibilities. We may also want to know the number of ways that each of these possibilities can happen. In the back of our mind is often a question about probability. How likely is it that I will observe this particular possibility? Records of success and failure are an example. For instance, the counting methods presented in this section have been used to evaluate a market timer's record. We can also use the methods in this section to calculate what we earlier called a priori probabilities. When we can assume that the possible outcomes of a random variable are equally likely, the probability of an event equals the number of possible outcomes favorable for the event divided by the total number of outcomes.

In counting, enumeration (counting the outcomes one by one) is of course the most basic resource. What we discuss in this section are shortcuts and principles. Without these shortcuts and principles, counting the total number of outcomes can be very difficult and prone to error. The first and basic principle of counting is the multiplication rule.

\begin{itemize}
  \item Multiplication Rule for Counting. If one task can be done in $n_{1}$ ways, and a second task, given the first, can be done in $n_{2}$ ways, and a third task, given the first two tasks, can be done in $n_{3}$ ways, and so on for $k$ tasks, then the number of ways the $k$ tasks can be done is $\left(n_{1}\right)\left(n_{2}\right)\left(n_{3}\right) \ldots\left(n_{k}\right)$.
\end{itemize}

Exhibit 21 illustrates the multiplication rule where, for example, we have three steps in an investment decision process. In the first step, stocks are classified two ways, as domestic or foreign (represented by dark- and light-shaded circles, respectively). In the second step, stocks are assigned to one of four industries in our investment universe: consumer, energy, financial, or technology (represented by four circles with progressively darker shades, respectively). In the third step, stocks are classified three ways by size: small-cap, mid-cap, and large-cap (represented by light-, medium-, and dark-shaded circles, respectively). Because the first step can be done in two ways, the second in four ways, and the third in three ways, using the multiplication rule, we can carry out the three steps in $(2)(4)(3)=24$ different ways.

Exhibit 21: Investment Decision Process Using Multiplication Rule: $n_{1}=2$, $n_{2}=4, n_{3}=3$

\begin{center}
\includegraphics[max width=\textwidth]{2023_05_04_cff39ee44f77d6514e1bg-223}
\end{center}

Step 3: Small-Cap Step 3: Mid-Cap Step 3: Large-Cap

Another illustration is the assignment of members of a group to an equal number of positions. For example, suppose you want to assign three security analysts to cover three different industries. In how many ways can the assignments be made? The first analyst can be assigned in three different ways. Then two industries remain. The second analyst can be assigned in two different ways. Then one industry remains. The third and last analyst can be assigned in only one way. The total number of different assignments equals $(3)(2)(1)=6$. The compact notation for the multiplication we have just performed is 3! (read: 3 factorial). If we had $n$ analysts, the number of ways we could assign them to $n$ tasks would be

$$
n !=n(n-1)(n-2)(n-3) \ldots 1
$$

or $\boldsymbol{n}$ factorial. (By convention, 0 ! $=1$.) To review, in this application, we repeatedly carry out an operation (here, job assignment) until we use up all members of a group (here, three analysts). With $n$ members in the group, the multiplication formula reduces to $n$ factorial.

The next type of counting problem can be called labeling problems. ${ }^{1}$ We want to give each object in a group a label, to place it in a category. The following example illustrates this type of problem.

A mutual fund guide ranked 18 bond mutual funds by total returns for the last year. The guide also assigned each fund one of five risk labels: high risk (four funds), above-average risk (four funds), average risk (three funds), below-average risk (four funds), and low risk (three funds); as $4+4+3+4+3=18$, all the funds are accounted for. How many different ways can we take 18 mutual funds and label 4 of them high risk, 4 above-average risk, 3 average risk, 4 below-average risk, and 3 low risk, so that each fund is labeled?

1 This discussion follows Kemeny, Schleifer, Snell, and Thompson (1972) in terminology and approach. The answer is close to 13 billion. We can label any of 18 funds high risk (the first slot), then any of 17 remaining funds, then any of 16 remaining funds, then any of 15 remaining funds (now we have 4 funds in the high risk group); then we can label any of 14 remaining funds above-average risk, then any of 13 remaining funds, and so forth. There are 18! possible sequences. However, order of assignment within a category does not matter. For example, whether a fund occupies the first or third slot of the four funds labeled high risk, the fund has the same label (high risk). Thus, there are 4 ! ways to assign a given group of four funds to the four high risk slots. Making the same argument for the other categories, in total there are (4!)(4!)(3!)(4!)(3!) equivalent sequences. To eliminate such redundancies from the 18 ! total, we divide 18 ! by (4!)(4!) (3!)(4!)(3!). We have 18!/[(4!)(4!)(3!)(4!)(3!)] = 18!/[(24)(24)(6)(24)(6)] = 12,864,852,000. This procedure generalizes as follows.

\begin{itemize}
  \item Multinomial Formula (General Formula for Labeling Problems). The number of ways that $n$ objects can be labeled with $k$ different labels, with $n_{1}$ of the first type, $n_{2}$ of the second type, and so on, with $n_{1}+n_{2}+\ldots+n_{k}=n$, is given by
\end{itemize}

$$
\frac{n !}{n_{1} ! n_{2} ! \ldots n_{k} !}
$$

The multinomial formula with two different labels $(k=2)$ is especially important. This special case is called the combination formula. A combination is a listing in which the order of the listed items does not matter. We state the combination formula in a traditional way, but no new concepts are involved. Using the notation in the formula below, the number of objects with the first label is $r=n_{1}$ and the number with the second label is $n-r=n_{2}$ (there are just two categories, so $n_{1}+n_{2}=n$ ). Here is the formula:

\begin{itemize}
  \item Combination Formula (Binomial Formula). The number of ways that we can choose $r$ objects from a total of $n$ objects, when the order in which the $r$ objects are listed does not matter, is
\end{itemize}

$$
{ }_{n} C_{r}=\left(\begin{array}{l}
n \\
r
\end{array}\right)=\frac{n !}{(n-r) ! r !}
$$

Here ${ }_{n} C_{r}$ and $\left(\begin{array}{l}n \\ r\end{array}\right)$ are shorthand notations for $n ! /(n-r) ! r !$ (read: $n$ choose $r$, or $n$ combination $r$.

If we label the $r$ objects as belongs to the group and the remaining objects as does not belong to the group, whatever the group of interest, the combination formula tells us how many ways we can select a group of size $r$. We can illustrate this formula with the binomial option pricing model. (The binomial pricing model is covered later in the CFA curriculum. The only intuition we are concerned with here is that a number of different pricing paths can end up with the same final stock price.) This model describes the movement of the underlying asset as a series of moves, price up (U) or price down (D). For example, two sequences of five moves containing three up moves, such as UUUDD and UDUUD, result in the same final stock price. At least for an option with a payoff dependent on final stock price, the number but not the order of up moves in a sequence matters. How many sequences of five moves belong to the group with three up moves? The answer is 10 , calculated using the combination formula ("5 choose 3 "):

$$
\begin{aligned}
& { }_{5} C_{3}=5 ! /[(5-3) ! 3 !] \\
& =[(5)(4)(3)(2)(1)] /[(2)(1)(3)(2)(1)]=120 / 12=10 \text { ways }
\end{aligned}
$$

A useful fact can be illustrated as follows: ${ }_{5} C_{3}=5 ! /(2 ! 3 !)$ equals ${ }_{5} C_{2}=5 ! /(3 ! 2 !)$, as $3+$ $2=5 ;{ }_{5} C_{4}=5 ! /(1 ! 4 !)$ equals ${ }_{5} C_{1}=5 ! /(4 ! 1 !)$, as $4+1=5$. This symmetrical relationship can save work when we need to calculate many possible combinations. Suppose jurors want to select three companies out of a group of five to receive the first-, second-, and third-place awards for the best annual report. In how many ways can the jurors make the three awards? Order does matter if we want to distinguish among the three awards (the rank within the group of three); clearly the question makes order important. On the other hand, if the question were "In how many ways can the jurors choose three winners, without regard to place of finish?" we would use the combination formula.

To address the first question above, we need to count ordered listings such as first place, New Company; second place, Fir Company; third place, Well Company. An ordered listing is known as a permutation, and the formula that counts the number of permutations is known as the permutation formula. A more formal definition states that a permutation is an ordered subset of $n$ distinct objects.

\begin{itemize}
  \item Permutation Formula. The number of ways that we can choose $r$ objects from a total of $n$ objects, when the order in which the $r$ objects are listed does matter, is
\end{itemize}

$$
{ }_{n} P_{r}=\frac{n !}{(n-r) !}
$$

So the jurors have ${ }_{5} P_{3}=5 ! /(5-3) !=[(5)(4)(3)(2)(1)] /[(2)(1)]=120 / 2=60$ ways in which they can make their awards. To see why this formula works, note that [(5) $(4)(3)(2)(1)] /[(2)(1)]$ reduces to (5)(4)(3), after cancellation of terms. This calculation counts the number of ways to fill three slots choosing from a group of five companies, according to the multiplication rule of counting. This number is naturally larger than it would be if order did not matter (compare 60 to the value of 10 for " 5 choose 3" that we calculated above). For example, first place, Well Company; second place, Fir Company; third place, New Company contains the same three companies as first place, New Company; second place, Fir Company; third place, Well Company. If we were concerned only with award winners (without regard to place of finish), the two listings would count as one combination. But when we are concerned with the order of finish, the listings count as two permutations.

\section{EXAMPLE 17}
\section{Permutations and Combinations for Two Out of Four Outcomes}
\begin{enumerate}
  \item There are four balls numbered 1, 2, 3, and 4 in a basket. You are running a contest in which two of the four balls are selected at random from the basket. To win, a player must have correctly chosen the numbers of the two randomly selected balls. Suppose the winning numbers are numbers 1 and 3. If the player must choose the balls in the same order in which they are drawn, she wins if she chose 1 first and 3 second. On the other hand, if order is not important, the player wins if the balls drawn are 1 and then 3 or if the balls drawn are 3 and then 1 . The number of possible outcomes for permutations and combinations of choosing 2 out of 4 items is illustrated in Exhibit 22. If order is not important, for choosing 2 out of 4 items, the winner wins twice as often. Exhibit 22: Permutations and Combinations for Choosing 2 out of 4 Items
\end{enumerate}

\begin{center}
\begin{tabular}{|c|c|}
\hline
Permutations: Order matters & Combinations: Order does not matter \\
\hline
\includegraphics[max width=\textwidth]{2023_05_04_cff39ee44f77d6514e1bg-226(1)}
 & $\begin{array}{l}\text { List of all possible outcomes: } \\ \left(\begin{array}{lll}1 & 2\end{array}\right)\left(\begin{array}{ll}2 & 3\end{array}\right)\left(\begin{array}{ll}3 & 4\end{array}\right) \\ \left(\begin{array}{ll}1 & 3\end{array}\right)\left(\begin{array}{ll}2 & 4\end{array}\right) \\ \left(\begin{array}{ll}1 & 4\end{array}\right)\end{array}$ \\
\hline
$\begin{array}{l}\text { Number of permutations: } \\ { }_{n} P_{r}=\frac{n !}{\left(n 4 !^{r}\right) !} \\ { }_{4} P_{2}=\frac{4 \times 3 \times 2 \times 1}{(4-2) !}=\frac{4 \times 1}{2 \times 1}=12\end{array}$ & \includegraphics[max width=\textwidth]{2023_05_04_cff39ee44f77d6514e1bg-226}
 \\
\hline
\end{tabular}
\end{center}

If order is important, the number of permutations (possible outcomes) is much larger than the number of combinations when order is not important.

\section{EXAMPLE 18}
\section{Reorganizing the Analyst Team Assignments}
\begin{enumerate}
  \item Gehr-Flint Investors classifies the stocks in its investment universe into 11 industries and is assigning each research analyst one or two industries. Five of the industries have been assigned, and you are asked to cover two industries from the remaining six.
\end{enumerate}

How many possible pairs of industries remain?
A. 12
B. 15
C. 36

Solution:

B is correct. The number of combinations of selecting two industries out of six is equal to

$$
{ }_{n} C_{r}=\left[\frac{n}{r}\right]=\frac{n !}{(n-r) ! r !}=\frac{6 !}{4 ! 2 !}=15
$$

The number of possible combinations for picking two industries out of six is 15.

\section{EXAMPLE 19}
\section{Australian Powerball Lottery}
To win the Australian Powerball jackpot, you must match the numbers of seven balls pulled from a basket (the balls are numbered 1 through 35) plus the number of the Powerball (numbered 1 through 20). The order in which the seven balls are drawn is not important. The number of combinations of matching 7 out of 35 balls is

$$
{ }_{n} C_{r}=\left[\frac{n}{r}\right]=\frac{n !}{(n-r) ! r !}=\frac{35 !}{28 ! 7 !}=6,724,520
$$

The number of combinations for picking the Powerball, 1 out of 20 , is

$$
{ }_{n} C_{r}=\left[\frac{n}{r}\right]=\frac{n !}{(n-r) ! r !}=\frac{20 !}{19 ! 1 !}=20
$$

The number of ways to pick the seven balls plus the Powerball is

$$
{ }_{35} C_{7} \times{ }_{20} C_{1}=6,724,520 \times 20=134,490,400
$$

Your probability of winning the Australian Powerball with one ticket is 1 in $134,490,400$.

Exhibit 23 is a flow chart that may help you apply the counting methods we have presented in this section.

\section{Exhibit 23: Summary of Counting Methods}
\begin{center}
\includegraphics[max width=\textwidth]{2023_05_04_cff39ee44f77d6514e1bg-227}
\end{center}

\section{SUMMARY}
In this reading, we have discussed the essential concepts and tools of probability. We have applied probability, expected value, and variance to a range of investment problems.

\begin{itemize}
  \item A random variable is a quantity whose outcome is uncertain.

  \item Probability is a number between 0 and 1 that describes the chance that a stated event will occur.

  \item An event is a specified set of outcomes of a random variable.

  \item Mutually exclusive events can occur only one at a time. Exhaustive events cover or contain all possible outcomes.

  \item The two defining properties of a probability are, first, that $0 \leq P(E) \leq 1$ (where $P(E)$ denotes the probability of an event $E$ ) and, second, that the sum of the probabilities of any set of mutually exclusive and exhaustive events equals 1.

  \item A probability estimated from data as a relative frequency of occurrence is an empirical probability. A probability drawing on personal or subjective judgment is a subjective probability. A probability obtained based on logical analysis is an a priori probability.

  \item A probability of an event $E, P(E)$, can be stated as odds for $E=P(E) /[1$ $P(E)]$ or odds against $E=[1-P(E)] / P(E)$.

  \item Probabilities that are inconsistent create profit opportunities, according to the Dutch Book Theorem.

  \item A probability of an event not conditioned on another event is an unconditional probability. The unconditional probability of an event $A$ is denoted $P(A)$. Unconditional probabilities are also called marginal probabilities.

  \item A probability of an event given (conditioned on) another event is a conditional probability. The probability of an event $A$ given an event $B$ is denoted $P(A \mid B)$, and $P(A \mid B)=P(A B) / P(B), P(B) \neq 0$.

  \item The probability of both $A$ and $B$ occurring is the joint probability of $A$ and $B$, denoted $P(A B)$.

  \item The multiplication rule for probabilities is $P(A B)=P(A \mid B) P(B)$.

  \item The probability that $A$ or $B$ occurs, or both occur, is denoted by $P(A$ or $B)$.

  \item The addition rule for probabilities is $P(A$ or $B)=P(A)+P(B)-P(A B)$.

  \item When events are independent, the occurrence of one event does not affect the probability of occurrence of the other event. Otherwise, the events are dependent.

  \item The multiplication rule for independent events states that if $A$ and $B$ are independent events, $P(A B)=P(A) P(B)$. The rule generalizes in similar fashion to more than two events.

  \item According to the total probability rule, if $S_{1}, S_{2}, \ldots, S_{n}$ are mutually exclusive and exhaustive scenarios or events, then $P(A)=P\left(A \mid S_{1}\right) P\left(S_{1}\right)+P\left(A \mid S_{2}\right)$ $P\left(S_{2}\right)+\ldots+P\left(A \mid S_{n}\right) P\left(S_{n}\right)$.

  \item The expected value of a random variable is a probability-weighted average of the possible outcomes of the random variable. For a random variable $X$, the expected value of $X$ is denoted $E(X)$. - The total probability rule for expected value states that $E(X)=E\left(X \mid S_{1}\right)$ $P\left(S_{1}\right)+E\left(X \mid S_{2}\right) P\left(S_{2}\right)+\ldots+E\left(X \mid S_{n}\right) P\left(S_{n}\right)$, where $S_{1}, S_{2}, \ldots, S_{n}$ are mutually exclusive and exhaustive scenarios or events.

  \item The variance of a random variable is the expected value (the probability-weighted average) of squared deviations from the random variable's expected value $E(X)$ : $\sigma^{2}(X)=E\left\{[X-E(X)]^{2}\right\}$, where $\sigma^{2}(X)$ stands for the variance of $X$.

  \item Variance is a measure of dispersion about the mean. Increasing variance indicates increasing dispersion. Variance is measured in squared units of the original variable.

  \item Standard deviation is the positive square root of variance. Standard deviation measures dispersion (as does variance), but it is measured in the same units as the variable.

  \item Covariance is a measure of the co-movement between random variables.

  \item The covariance between two random variables $R_{i}$ and $R_{j}$ in a forward-looking sense is the expected value of the cross-product of the deviations of the two random variables from their respective means: $\operatorname{Cov}\left(R_{i}, R_{j}\right)=E\left\{\left[R_{i}-E\left(R_{i}\right)\right]\left[R_{j}-E\left(R_{j}\right)\right]\right\}$. The covariance of a random variable with itself is its own variance.

  \item The historical or sample covariance between two random variables $R_{i}$ and $R_{j}$ based on a sample of past data of size $n$ is the average value of the product of the deviations of observations on two random variables from their sample means:

\end{itemize}

$\operatorname{Cov}\left(R_{i}, R_{j}\right)=\sum_{n=1}^{n}\left(R_{i, t}-\bar{R}_{i}\right)\left(R_{j, t}-\bar{R}_{j}\right) /(n-1)$

\begin{itemize}
  \item Correlation is a number between -1 and +1 that measures the co-movement (linear association) between two random variables: $\rho\left(R_{i}, R_{j}\right)=\operatorname{Cov}\left(R_{i}, R_{j}\right) /$ $\left[\sigma\left(R_{i}\right) \sigma\left(R_{j}\right)\right]$.

  \item If two variables have a very strong (inverse) linear relation, then the absolute value of their correlation will be close to $1(-1)$. If two variables have a weak linear relation, then the absolute value of their correlation will be close to 0 .

  \item If the correlation coefficient is positive, the two variables are positively related; if the correlation coefficient is negative, the two variables are inversely related.

  \item To calculate the variance of return on a portfolio of $n$ assets, the inputs needed are the $n$ expected returns on the individual assets, $n$ variances of return on the individual assets, and $n(n-1) / 2$ distinct covariances.

  \item Portfolio variance of return is $\sigma^{2}\left(R_{p}\right)=\sum_{i=1}^{n} \sum_{j=1}^{n} w_{i} w_{j} \operatorname{Cov}\left(R_{i}, R_{j}\right)$.

  \item The calculation of covariance in a forward-looking sense requires the specification of a joint probability function, which gives the probability of joint occurrences of values of the two random variables.

  \item When two random variables are independent, the joint probability function is the product of the individual probability functions of the random variables.

  \item Bayes' formula is a method for updating probabilities based on new information. - Bayes' formula is expressed as follows: Updated probability of event given the new information $=[$ Probability of the new information given event $) /$ (Unconditional probability of the new information)] $\times$ Prior probability of event.

  \item The multiplication rule of counting says, for example, that if the first step in a process can be done in 10 ways, the second step, given the first, can be done in 5 ways, and the third step, given the first two, can be done in 7 ways, then the steps can be carried out in (10)(5)(7) $=350$ ways.

  \item The number of ways to assign every member of a group of size $n$ to $n$ slots is $n !=n(n-1)(n-2)(n-3) \ldots 1$. (By convention, $0 !=1$.)

  \item The number of ways that $n$ objects can be labeled with $k$ different labels, with $n_{1}$ of the first type, $n_{2}$ of the second type, and so on, with $n_{1}+n_{2}+\ldots$ $+n_{k}=n$, is given by $n ! /\left(n_{1} ! n_{2} ! \ldots n_{k} !\right)$. This expression is the multinomial formula.

  \item A special case of the multinomial formula is the combination formula. The number of ways to choose $r$ objects from a total of $n$ objects, when the order in which the $r$ objects are listed does not matter, is

\end{itemize}

${ }_{n} C_{r}=\left(\begin{array}{l}n \\ r\end{array}\right)=\frac{n !}{(n-r) ! r !}$

\begin{itemize}
  \item The number of ways to choose $r$ objects from a total of $n$ objects, when the order in which the $r$ objects are listed does matter, is
\end{itemize}

$$
{ }_{n} P_{r}=\frac{n !}{(n-r) !}
$$

This expression is the permutation formula.

\section{REFERENCES}
Kemeny, John G., Arthur Schleifer, J. Laurie Snell, Gerald L. Thompson. 1972. Finite Mathematics with Business Applications, 2nd edition. Englewood Cliffs: Prentice-Hall.

\section{PRACTICE PROBLEMS}
\begin{enumerate}
  \item In probability theory, exhaustive events are best described as the set of events that:
A. have a probability of zero.
B. are mutually exclusive.
C. include all potential outcomes.

  \item Which probability estimate most likely varies greatly between people?
A. An a priori probability
B. An empirical probability
C. A subjective probability

  \item If the probability that Zolaf Company sales exceed last year's sales is 0.167 , the odds for exceeding sales are closest to:
A. 1 to 5 .
B. 1 to 6 .
C. 5 to 1 .

  \item After six months, the growth portfolio that Rayan Khan manages has outperformed its benchmark. Khan states that his odds of beating the benchmark for the year are 3 to 1 . If these odds are correct, what is the probability that Khan's portfolio will beat the benchmark for the year?
A. 0.33
B. 0.67
C. 0.75

  \item Suppose that $5 \%$ of the stocks meeting your stock-selection criteria are in the telecommunications (telecom) industry. Also, dividend-paying telecom stocks are $1 \%$ of the total number of stocks meeting your selection criteria. What is the probability that a stock is dividend paying, given that it is a telecom stock that has met your stock selection criteria?

  \item You are using the following three criteria to screen potential acquisition targets from a list of 500 companies:

\end{enumerate}

Criterion

Fraction of the 500 Companies Meeting the Criterion

$\begin{array}{lr}\text { Product lines compatible } & 0.20\end{array}$

$\begin{array}{ll}\text { Company will increase combined sales growth rate } & 0.45\end{array}$

$\begin{array}{ll}\text { Balance sheet impact manageable } & 0.78\end{array}$

If the criteria are independent, how many companies will pass the screen?

\begin{enumerate}
  \setcounter{enumi}{6}
  \item Florence Hixon is screening a set of 100 stocks based on two criteria (Criterion 1 and Criterion 2). She set the passing level such that $50 \%$ of the stocks passed each screen. For these stocks, the values for Criterion 1 and Criterion 2 are not independent but are positively related. How many stocks should pass Hixon's two screens?
\end{enumerate}

A. Less than 25

B. 25

C. More than 25

\begin{enumerate}
  \setcounter{enumi}{7}
  \item You apply both valuation criteria and financial strength criteria in choosing stocks. The probability that a randomly selected stock (from your investment universe) meets your valuation criteria is 0.25 . Given that a stock meets your valuation criteria, the probability that the stock meets your financial strength criteria is 0.40 . What is the probability that a stock meets both your valuation and financial strength criteria?

  \item The probability of an event given that another event has occurred is a:
A. joint probability.
B. marginal probability.
C. conditional probability.

  \item After estimating the probability that an investment manager will exceed his benchmark return in each of the next two quarters, an analyst wants to forecast the probability that the investment manager will exceed his benchmark return over the two-quarter period in total. Assuming that each quarter's performance is independent of the other, which probability rule should the analyst select?
A. Addition rule
B. Multiplication rule
C. Total probability rule

  \item Which of the following is a property of two dependent events?

\end{enumerate}

A. The two events must occur simultaneously.

B. The probability of one event influences the probability of the other event.

C. The probability of the two events occurring is the product of each event's probability.

\begin{enumerate}
  \setcounter{enumi}{11}
  \item Which of the following best describes how an analyst would estimate the expected value of a firm using the scenarios of bankruptcy and non-bankruptcy? The analyst would use:
\end{enumerate}

A. the addition rule.

B. conditional expected values.

C. the total probability rule for expected value.

\begin{enumerate}
  \setcounter{enumi}{12}
  \item Suppose the prospects for recovering principal for a defaulted bond issue depend on which of two economic scenarios prevails. Scenario 1 has probability 0.75 and will result in recovery of $\$ 0.90$ per $\$ 1$ principal value with probability 0.45 , or in recovery of $\$ 0.80$ per $\$ 1$ principal value with probability 0.55 . Scenario 2 has probability 0.25 and will result in recovery of $\$ 0.50$ per $\$ 1$ principal value with probability 0.85 , or in recovery of $\$ 0.40$ per $\$ 1$ principal value with probability 0.15 .
\end{enumerate}

A. Compute the probability of each of the four possible recovery amounts: $\$ 0.90$, $\$ 0.80$, $\$ 0.50$, and $\$ 0.40$.

B. Compute the expected recovery, given the first scenario.

C. Compute the expected recovery, given the second scenario.

D. Compute the expected recovery.

E. Graph the information in a probability tree diagram.

\begin{enumerate}
  \setcounter{enumi}{13}
  \item An analyst developed two scenarios with respect to the recovery of $\$ 100,000$ principal from defaulted loans:
\end{enumerate}

\begin{center}
\begin{tabular}{lccc}
\hline
Scenario & $\begin{array}{c}\text { Probability } \\ \text { of Scenario (\%) }\end{array}$ & $\begin{array}{c}\text { Amount } \\ \text { Recovered (\$) }\end{array}$ & $\begin{array}{c}\text { Probability } \\ \text { of Amount (\%) }\end{array}$ \\
\hline
1 & 40 & 50,000 & 60 \\
2 & 30,000 & 40 &  \\
 & 60 & 80,000 & 90 \\
 & 60,000 & 10 &  \\
\hline
\end{tabular}
\end{center}

The amount of the expected recovery is closest to:
A. $\$ 36,400$.
B. $\$ 55,000$.
C. $\$ 63,600$.

\begin{enumerate}
  \setcounter{enumi}{14}
  \item The probability distribution for a company's sales is:
\end{enumerate}

\begin{center}
\begin{tabular}{lc}
\hline
Probability & Sales (\$ millions) \\
\hline
0.05 & 70 \\
0.70 & 40 \\
0.25 & 25 \\
\hline
\end{tabular}
\end{center}

The standard deviation of sales is closest to:
A. $\$ 9.81$ million.
B. $\$ 12.20$ million.
C. $\$ 32.40$ million.

\begin{enumerate}
  \setcounter{enumi}{15}
  \item US and Spanish bonds have return standard deviations of 0.64 and 0.56 , respectively. If the correlation between the two bonds is 0.24 , the covariance of returns is closest to:
A. 0.086 .
B. 0.335 .
C. 0.390 . 17. The covariance of returns is positive when the returns on two assets tend to:
A. have the same expected values.
B. be above their expected value at different times.
C. be on the same side of their expected value at the same time.

  \item Which of the following correlation coefficients indicates the weakest linear relationship between two variables?
A. -0.67
B. -0.24
C. 0.33

  \item An analyst develops the following covariance matrix of returns:

\end{enumerate}

\begin{center}
\begin{tabular}{lcc}
\hline
 & Hedge Fund & Market Index \\
\hline
Hedge fund & 256 & 110 \\
Market index & 110 & 81 \\
\hline
\end{tabular}
\end{center}

The correlation of returns between the hedge fund and the market index is closest to:
A. 0.005 .
B. 0.073 .
C. 0.764 .

\begin{enumerate}
  \setcounter{enumi}{19}
  \item All else being equal, as the correlation between two assets approaches +1.0 , the diversification benefits:
A. decrease.
B. stay the same.
C. increase.

  \item Given a portfolio of five stocks, how many unique covariance terms, excluding variances, are required to calculate the portfolio return variance?
A. 10
B. 20
C. 25

  \item Which of the following statements is most accurate? If the covariance of returns between two assets is 0.0023 , then:
A. the assets' risk is near zero.
B. the asset returns are unrelated.
C. the asset returns have a positive relationship. 23. A two-stock portfolio includes stocks with the following characteristics:

\end{enumerate}

\begin{center}
\begin{tabular}{|c|c|c|}
\hline
 & Stock 1 & Stock 2 \\
\hline
Expected return & $7 \%$ & $10 \%$ \\
\hline
Standard deviation & $12 \%$ & $25 \%$ \\
\hline
Portfolio weights & 0.30 & 0.70 \\
\hline
Correlation & \multicolumn{2}{|c|}{0.20} \\
\hline
\end{tabular}
\end{center}

What is the standard deviation of portfolio returns?
A. $14.91 \%$
B. $18.56 \%$
C. $21.10 \%$

\begin{enumerate}
  \setcounter{enumi}{23}
  \item Lena Hunziger has designed the three-asset portfolio summarized below:
\end{enumerate}

\begin{center}
\begin{tabular}{|c|c|c|c|}
\hline
 & Asset 1 & Asset 2 & Asset 3 \\
\hline
Expected return & $5 \%$ & $6 \%$ & $7 \%$ \\
\hline
Portfolio weight & 0.20 & 0.30 & 0.50 \\
\hline
\multicolumn{4}{|c|}{Variance-Covariance Matrix} \\
\hline
 & Asset 1 & Asset 2 & Asset 3 \\
\hline
Asset 1 & 196 & 105 & 140 \\
\hline
Asset 2 & 105 & 225 & 150 \\
\hline
Asset 3 & 140 & 150 & 400 \\
\hline
\end{tabular}
\end{center}

Hunziger estimated the portfolio return to be $6.3 \%$. What is the portfolio standard deviation?
A. $13.07 \%$
B. $13.88 \%$
C. $14.62 \%$

\begin{enumerate}
  \setcounter{enumi}{24}
  \item An analyst produces the following joint probability function for a foreign index (FI) and a domestic index (DI).
\end{enumerate}

\begin{center}
\begin{tabular}{lccc}
\hline
 & $\boldsymbol{R}_{\boldsymbol{D I}}=\mathbf{3 0} \%$ & $\boldsymbol{R}_{\boldsymbol{D I}}=\mathbf{2 5 \%}$ & $\boldsymbol{R}_{\boldsymbol{D I}}=\mathbf{1 5 \%}$ \\
\hline
$R_{F I}=25 \%$ & 0.25 &  &  \\
$R_{F I}=15 \%$ &  & 0.50 &  \\
$R_{F I}=10 \%$ &  & 0.25 &  \\
\hline
\end{tabular}
\end{center}

The covariance of returns on the foreign index and the returns on the domestic index is closest to:
A. 26.39 .
B. 26.56 .
C. 28.12 .

\begin{enumerate}
  \setcounter{enumi}{25}
  \item You have developed a set of criteria for evaluating distressed credits. Companies that do not receive a passing score are classed as likely to go bankrupt within 12 months. You gathered the following information when validating the criteria:
\end{enumerate}

\begin{itemize}
  \item Forty percent of the companies to which the test is administered will go bankrupt within 12 months: $P$ (non-survivor $)=0.40$.

  \item Fifty-five percent of the companies to which the test is administered pass it: $P($ pass test $)=0.55$.

  \item The probability that a company will pass the test given that it will subsequently survive 12 months, is $0.85: P($ pass test $\mid$ survivor $)=0.85$.

\end{itemize}

A. What is $\mathrm{P}($ pass test $\mid$ non-survivor $)$ ?

B. Using Bayes' formula, calculate the probability that a company is a survivor, given that it passes the test; that is, calculate $P$ (survivor | pass test).

C. What is the probability that a company is a non-survivor, given that it fails the test?

D. Is the test effective?

\begin{enumerate}
  \setcounter{enumi}{26}
  \item An analyst estimates that $20 \%$ of high-risk bonds will fail (go bankrupt). If she applies a bankruptcy prediction model, she finds that $70 \%$ of the bonds will receive a "good" rating, implying that they are less likely to fail. Of the bonds that failed, only $50 \%$ had a "good" rating. Use Bayes' formula to predict the probability of failure given a "good" rating. (Hint, let $P(A)$ be the probability of failure, $P(B)$ be the probability of a "good" rating, $P(B \mid A)$ be the likelihood of a "good" rating given failure, and $P(A \mid B)$ be the likelihood of failure given a "good" rating.)
A. $5.7 \%$
B. $14.3 \%$
C. $28.6 \%$

  \item In a typical year, $5 \%$ of all CEOs are fired for "performance" reasons. Assume that CEO performance is judged according to stock performance and that $50 \%$ of stocks have above-average returns or "good" performance. Empirically, 30\% of all CEOs who were fired had "good" performance. Using Bayes' formula, what is the probability that a CEO will be fired given "good" performance? (Hint, let $P(A)$ be the probability of a CEO being fired, $P(B)$ be the probability of a "good" performance rating, $P(B \mid A)$ be the likelihood of a "good" performance rating given that the CEO was fired, and $P(A \mid B)$ be the likelihood of the CEO being fired given a "good" performance rating.)
A. $1.5 \%$
B. $2.5 \%$
C. $3.0 \%$

  \item A manager will select 20 bonds out of his universe of 100 bonds to construct a portfolio. Which formula provides the number of possible portfolios?

\end{enumerate}

A. Permutation formula

B. Multinomial formula

C. Combination formula

\begin{enumerate}
  \setcounter{enumi}{29}
  \item A firm will select two of four vice presidents to be added to the investment com- mittee. How many different groups of two are possible?
A. 6
B. 12
C. 24

  \item From an approved list of 25 funds, a portfolio manager wants to rank 4 mutual funds from most recommended to least recommended. Which formula is most appropriate to calculate the number of possible ways the funds could be ranked?

\end{enumerate}

A. Permutation formula

B. Multinomial formula

C. Combination formula

\begin{enumerate}
  \setcounter{enumi}{31}
  \item Himari Fukumoto has joined a new firm and is selecting mutual funds in the firm's pension plan. If 10 mutual funds are available, and she plans to select four, how many different sets of mutual funds can she choose?
A. 210
B. 720
C. 5,040
\end{enumerate}

\section{The following information relates to questions}
\section{3-35}
Gerd Sturm wants to sponsor a contest with a $\$ 1$ million prize. The winner must pick the stocks that will be the top five performers next year among the 30 stocks in a well-known large-cap stock index. He asks you to estimate the chances that contestants can win the contest.

\begin{enumerate}
  \setcounter{enumi}{32}
  \item What are the chances of winning if the contestants must pick the five stocks in the correct order of their total return? If choosing five stocks randomly, a contestant's chance of winning is one out of:
A. 142,506 .
B. $17,100,720$.
C. $24,300,000$.

  \item What are the chances of winning if the contestants must pick the top five stocks without regard to order? If choosing five stocks randomly, a contestant's chance of winning is one out of:
A. 142,506
B. $17,100,720$
C. $24,300,000$.

  \item Sturm asks, "Can we trust these probabilities of winning?"

\end{enumerate}

\section{SOLUTIONS}
\begin{enumerate}
  \item $C$ is correct. The term "exhaustive" means that the events cover all possible outcomes.

  \item C is correct. A subjective probability draws on personal or subjective judgment that may be without reference to any particular data.

  \item A is correct. Given odds for $E$ of $a$ to $b$, the implied probability of $E=a /(a+b)$. Stated in terms of odds $a$ to $b$ with $a=1, b=5$, the probability of $E=1 /(1+5)$ $=1 / 6=0.167$. This result confirms that a probability of 0.167 for beating sales is odds of 1 to 5 .

  \item $C$ is correct. The odds for beating the benchmark $=P$ (beating benchmark) $/[1-$ $P($ beating benchmark)]. Let $P(A)=P$ (beating benchmark). Odds for beating the benchmark $=P(A) /[1-P(A)]$.

\end{enumerate}

$3=P(A) /[1-P(A)]$

Solving for $P(A)$, the probability of beating the benchmark is 0.75 .

\begin{enumerate}
  \setcounter{enumi}{4}
  \item Use this equation to find this conditional probability: $P$ (stock is dividend paying $\mid$ telecom stock that meets criteria $)=P$ (stock is dividend paying and telecom stock that meets criteria $) / P($ telecom stock that meets criteria $)=0.01 / 0.05=0.20$.

  \item According to the multiplication rule for independent events, the probability of a company meeting all three criteria is the product of the three probabilities. Labeling the event that a company passes the first, second, and third criteria, $A$, $B$, and $C$, respectively, $P(A B C)=P(A) P(B) P(C)=(0.20)(0.45)(0.78)=0.0702$. As a consequence, $(0.0702)(500)=35.10$, so 35 companies pass the screen.

  \item $C$ is correct. Let event $A$ be a stock passing the first screen (Criterion 1) and event $B$ be a stock passing the second screen (Criterion 2). The probability of passing each screen is $P(A)=0.50$ and $P(B)=0.50$. If the two criteria are independent, the joint probability of passing both screens is $P(A B)=P(A) P(B)=0.50$ $\times 0.50=0.25$, so 25 out of 100 stocks would pass both screens. However, the two criteria are positively related, and $P(A B) \neq 0.25$. Using the multiplication rule for probabilities, the joint probability of $A$ and $B$ is $P(A B)=P(A \mid B) P(B)$. If the two criteria are not independent, and if $P(B)=0.50$, then the contingent probability of $P(A \mid B)$ is greater than 0.50 . So the joint probability of $P(A B)=P(A \mid B) P(B)$ is greater than 0.25 . More than 25 stocks should pass the two screens.

  \item Use the equation for the multiplication rule for probabilities $P(A B)=P(A \mid B)$ $P(B)$, defining $A$ as the event that a stock meets the financial strength criteria and defining $B$ as the event that $a$ stock meets the valuation criteria. Then $P(A B)$ $=P(A \mid B) P(B)=0.40 \times 0.25=0.10$. The probability that a stock meets both the financial and valuation criteria is 0.10 .

  \item $\mathrm{C}$ is correct. A conditional probability is the probability of an event given that another event has occurred.

  \item B is correct. Because the events are independent, the multiplication rule is most appropriate for forecasting their joint probability. The multiplication rule for independent events states that the joint probability of both $A$ and $B$ occurring is $P(A B)=P(A) P(B)$.

  \item B is correct. The probability of the occurrence of one is related to the occurrence of the other. If we are trying to forecast one event, information about a dependent event may be useful.

  \item $\mathrm{C}$ is correct. The total probability rule for expected value is used to estimate an expected value based on mutually exclusive and exhaustive scenarios.

  \item 
\end{enumerate}

A. Outcomes associated with Scenario 1: With a 0.45 probability of a $\$ 0.90$ recovery per $\$ 1$ principal value, given Scenario 1, and with the probability of Scenario 1 equal to 0.75 , the probability of recovering $\$ 0.90$ is $0.45(0.75)$ $=0.3375$. By a similar calculation, the probability of recovering $\$ 0.80$ is $0.55(0.75)=0.4125$.

Outcomes associated with Scenario 2: With a 0.85 probability of a $\$ 0.50$ recovery per $\$ 1$ principal value, given Scenario 2, and with the probability of Scenario 2 equal to 0.25 , the probability of recovering $\$ 0.50$ is $0.85(0.25)$ $=0.2125$. By a similar calculation, the probability of recovering $\$ 0.40$ is $0.15(0.25)=0.0375$.

B. $E($ recovery $\mid$ Scenario 1$)=0.45(\$ 0.90)+0.55(\$ 0.80)=\$ 0.845$

C. $E($ recovery $\mid$ Scenario 2) $=0.85(\$ 0.50)+0.15(\$ 0.40)=\$ 0.485$

D. $E($ recovery $)=0.75(\$ 0.845)+0.25(\$ 0.485)=\$ 0.755$

E.

\begin{center}
\includegraphics[max width=\textwidth]{2023_05_04_cff39ee44f77d6514e1bg-239}
\end{center}

\begin{enumerate}
  \setcounter{enumi}{13}
  \item $C$ is correct. If Scenario 1 occurs, the expected recovery is $60 \%(\$ 50,000)+40 \%$ $(\$ 30,000)=\$ 42,000$, and if Scenario 2 occurs, the expected recovery is $90 \%$ $(\$ 80,000)+10 \%(\$ 60,000)=\$ 78,000$. Weighting by the probability of each scenario, the expected recovery is $40 \%(\$ 42,000)+60 \%(\$ 78,000)=\$ 63,600$. Alternatively, first calculating the probability of each amount occurring, the expected recovery is $(40 \%)(60 \%)(\$ 50,000)+(40 \%)(40 \%)(\$ 30,000)+(60 \%)(90 \%)(\$ 80,000)+$ $(60 \%)(10 \%)(\$ 60,000)=\$ 63,600$.

  \item A is correct. The analyst must first calculate expected sales as $0.05 \times \$ 70+0.70$ $\times \$ 40+0.25 \times \$ 25=\$ 3.50$ million $+\$ 28.00$ million $+\$ 6.25$ million $=\$ 37.75$ million.

\end{enumerate}

After calculating expected sales, we can calculate the variance of sales:

$\sigma^{2}$ (Sales $)=P(\$ 70)[\$ 70-E(\text { Sales })]^{2}+P(\$ 40)[\$ 40-E(\text { Sales })]^{2}+P(\$ 25)$ $\left[\$ 25-E(\text { Sales) }]^{2}\right.$

$$
=0.05(\$ 70-37.75)^{2}+0.70(\$ 40-37.75)^{2}+0.25(\$ 25-37.75)^{2}
$$

$=\$ 52.00$ million $+\$ 3.54$ million $+\$ 40.64$ million $=\$ 96.18$ million.

The standard deviation of sales is thus $\sigma=(\$ 96.18)^{1 / 2}=\$ 9.81$ million.

\begin{enumerate}
  \setcounter{enumi}{15}
  \item A is correct. The covariance is the product of the standard deviations and correlation using the formula $\operatorname{Cov}(U S$ bond returns, Spanish bond returns) $=\sigma(U S$ bonds) $\times \sigma$ (Spanish bonds) $\times \rho$ (US bond returns, Spanish bond returns $)=0.64 \times$ $0.56 \times 0.24=0.086$.

  \item $\mathrm{C}$ is correct. The covariance of returns is positive when the returns on both assets tend to be on the same side (above or below) their expected values at the same time, indicating an average positive relationship between returns.

  \item B is correct. Correlations near +1 exhibit strong positive linearity, whereas correlations near -1 exhibit strong negative linearity. A correlation of 0 indicates an absence of any linear relationship between the variables. The closer the correlation is to 0 , the weaker the linear relationship.

  \item $C$ is correct. The correlation between two random variables $R_{i}$ and $R_{j}$ is defined as $\rho\left(R_{i}, R_{j}\right)=\operatorname{Cov}\left(R_{i}, R_{j}\right) /\left[\sigma\left(R_{i}\right) \sigma\left(R_{j}\right)\right]$. Using the subscript $i$ to represent hedge funds and the subscript $j$ to represent the market index, the standard deviations are $\sigma\left(R_{i}\right)=256^{1 / 2}=16$ and $\sigma\left(R_{j}\right)=81^{1 / 2}=9$. Thus, $\rho\left(R_{i}, R_{j}\right)=\operatorname{Cov}\left(R_{i}, R_{j}\right) /\left[\sigma\left(R_{i}\right) \sigma\left(R_{j}\right)\right]=$ $110 /(16 \times 9)=0.764$

  \item A is correct. As the correlation between two assets approaches +1 , diversification benefits decrease. In other words, an increasingly positive correlation indicates an increasingly strong positive linear relationship and fewer diversification benefits.

  \item A is correct. A covariance matrix for five stocks has $5 \times 5=25$ entries. Subtracting the 5 diagonal variance terms results in 20 off-diagonal entries. Because a covariance matrix is symmetrical, only 10 entries are unique $(20 / 2=10)$.

  \item $C$ is correct. The covariance of returns is positive when the returns on both assets tend to be on the same side (above or below) their expected values at the same time.

  \item $\mathrm{B}$ is correct. The covariance between the returns for the two stocks is $\operatorname{Cov}\left(R_{1}, R_{2}\right)$ $=\rho\left(R_{1}, R_{2}\right) \sigma\left(R_{1}\right) \sigma\left(R_{2}\right)=0.20(12)(25)=60$. The portfolio variance is

\end{enumerate}

$$
\begin{aligned}
& \sigma^{2}\left(R_{p}\right)=w_{1}^{2} \sigma^{2}\left(R_{1}\right)+w_{2}^{2} \sigma^{2}\left(R_{2}\right)+2 w_{1} w_{2} \operatorname{Cov}\left(R_{1}, R_{2}\right) \\
= & (0.30)^{2}(12)^{2}+(0.70)^{2}(25)^{2}+2(0.30)(0.70)(60) \\
= & 12.96+306.25+25.2=344.41
\end{aligned}
$$

The portfolio standard deviation is

$$
\sigma^{2}\left(R_{p}\right)=344.41^{1 / 2}=18.56 \%
$$

\begin{enumerate}
  \setcounter{enumi}{23}
  \item $C$ is correct. For a three-asset portfolio, the portfolio variance is
\end{enumerate}

$$
\begin{aligned}
& \sigma^{2}\left(R_{p}\right)=w_{1}^{2} \sigma^{2}\left(R_{1}\right)+w_{2}^{2} \sigma^{2}\left(R_{2}\right)+w_{3}^{2} \sigma^{2}\left(R_{3}\right)+2 w_{1} w_{2} \operatorname{Cov}\left(R_{1}, R_{2}\right) \\
& +2 w_{1} w_{3} \operatorname{Cov}\left(R_{1}, R_{3}\right)+2 w_{2} w_{3} \operatorname{Cov}\left(R_{2}, R_{3}\right) \\
= & (0.20)^{2}(196)+(0.30)^{2}(225)+(0.50)^{2}(400)+2(0.20)(0.30)(105) \\
+ & (2)(0.20)(0.50)(140)+(2)(0.30)(0.50)(150) \\
= & 7.84+20.25+100+12.6+28+45=213.69
\end{aligned}
$$

The portfolio standard deviation is

$$
\sigma^{2}\left(R_{p}\right)=213.69^{1 / 2}=14.62 \%
$$

\begin{enumerate}
  \setcounter{enumi}{24}
  \item B is correct. The covariance is 26.56 , calculated as follows. First, expected returns are
\end{enumerate}

$$
\begin{aligned}
& E\left(R_{F I}\right)=(0.25 \times 25)+(0.50 \times 15)+(0.25 \times 10) \\
& =6.25+7.50+2.50=16.25 \text { and } \\
& E\left(R_{D I}\right)=(0.25 \times 30)+(0.50 \times 25)+(0.25 \times 15) \\
& =7.50+12.50+3.75=23.75 .
\end{aligned}
$$

Covariance is

$$
\begin{aligned}
& \operatorname{Cov}\left(R_{F I}, R_{D I}\right)=\sum_{i} \sum_{j} P\left(R_{F I, i}, R_{D I, j}\right)\left(R_{F I, i}-E R_{F I}\right)\left(R_{D I, j}-E R_{D I}\right) \\
& =0.25[(25-16.25)(30-23.75)]+0.50[(15-16.25)(25-23.75)]+ \\
& 0.25[(10-16.25)(15-23.75)] \\
& =13.67+(-0.78)+13.67=26.56 .
\end{aligned}
$$

26.

A. We can set up the equation using the total probability rule:

$P($ pass test $)=P($ pass test $\mid$ survivor $) P($ survivor $)$

$+P($ pass test $\mid$ non-survivor $) P($ non-survivor $)$

We know that $P($ survivor $)=1-P($ non-survivor $)=1-0.40=0.60$. Therefore, $P($ pass test $)=0.55=0.85(0.60)+P($ pass test $\mid$ non-survivor $)(0.40)$.

Thus, $P($ pass test $\mid$ non-survivor $)=[0.55-0.85(0.60)] / 0.40=0.10$.

$P($ survivor $\mid$ pass test $)=[P($ pass test $\mid$ survivor $) / P($ pass test $)] P($ survivor $)$

$=(0.85 / 0.55) 0.60=0.927273$

B. The information that a company passes the test causes you to update your probability that it is a survivor from 0.60 to approximately 0.927 .

C. According to Bayes' formula, $P($ non-survivor $\mid$ fail test $)=[P($ fail test $\mid$ non-survivor $) / P($ fail test $)] P($ non-survivor $)=[P($ fail test $\mid$ non-survivor)/0.45]0.40.

We can set up the following equation to obtain $P($ fail test $\mid$ non-survivor $)$ :

$P($ fail test $)=P($ fail test $\mid$ non-survivor $) P($ non-survivor $)$

$+P($ fail test $\mid$ survivor $) P($ survivor $)$

$0.45=P($ fail test $\mid$ non-survivor $) 0.40+0.15(0.60)$

where $P($ fail test $\mid$ survivor $)=1-P($ pass test $\mid$ survivor $)=1-0.85=0.15$. So $P($ fail test $\mid$ non-survivor $)=[0.45-0.15(0.60)] / 0.40=0.90$.

Using this result with the formula above, we find $P$ (non-survivor $\mid$ fail test $)$ $=[0.90 / 0.45] 0.40=0.80$. Seeing that a company fails the test causes us to update the probability that it is a non-survivor from 0.40 to 0.80 . D. A company passing the test greatly increases our confidence that it is a survivor. A company failing the test doubles the probability that it is a non-survivor. Therefore, the test appears to be useful.

\begin{enumerate}
  \setcounter{enumi}{26}
  \item B is correct. With Bayes' formula, the probability of failure given a "good" rating is
\end{enumerate}

$P(A \mid B)=\frac{P(B \mid A)}{P(B)} P(A)$

where

$P(A)=0.20=$ probability of failure

$P(B)=0.70=$ probability of a "good" rating

$P(B \mid A)=0.50=$ probability of a "good" rating given failure

With these estimates, the probability of failure given a "good" rating is

$P(A \mid B)=\frac{P(B \mid A)}{P(B)} P(A)=\frac{0.50}{0.70} \times 0.20=0.143$

If the analyst uses the bankruptcy prediction model as a guide, the probability of failure declines from $20 \%$ to $14.3 \%$.

\begin{enumerate}
  \setcounter{enumi}{27}
  \item C is correct. With Bayes' formula, the probability of the CEO being fired given a "good" rating is
\end{enumerate}

$P(A \mid B)=\frac{P(B \mid A)}{P(B)} P(A)$

where

$P(A)=0.05=$ probability of the CEO being fired

$P(B)=0.50=$ probability of a "good" rating

$P(B \mid A)=0.30=$ probability of a "good" rating given that the CEO is fired

With these estimates, the probability of the CEO being fired given a "good" rating is

$P(A \mid B)=\frac{P(B \mid A)}{P(B)} P(A)=\frac{0.30}{0.50} \times 0.05=0.03$

Although 5\% of all CEOs are fired, the probability of being fired given a "good" performance rating is $3 \%$.

\begin{enumerate}
  \setcounter{enumi}{28}
  \item $\mathrm{C}$ is correct. The combination formula provides the number of ways that $r$ objects can be chosen from a total of $n$ objects, when the order in which the $r$ objects are listed does not matter. The order of the bonds within the portfolio does not matter.

  \item A is correct. The answer is found using the combination formula

\end{enumerate}

$$
{ }_{n} C_{r}=\left(\begin{array}{l}
n \\
r
\end{array}\right)=\frac{n !}{(n-r) ! r !}
$$

Here, $n=4$ and $r=2$, so the answer is $4 ! /[(4-2) ! 2 !]=24 /[(2) \times(2)]=6$. This result can be verified by assuming there are four vice presidents, VP1-VP4. The six possible additions to the investment committee are VP1 and VP2, VP1 and VP3, VP1 and VP4, VP2 and VP3, VP2 and VP4, and VP3 and VP4.

\begin{enumerate}
  \setcounter{enumi}{30}
  \item A is correct. The permutation formula is used to choose $r$ objects from a total of $n$ objects when order matters. Because the portfolio manager is trying to rank the four funds from most recommended to least recommended, the order of the funds matters; therefore, the permutation formula is most appropriate.

  \item A is correct. The number of combinations is the number of ways to pick four mutual funds out of 10 without regard to order, which is

\end{enumerate}

${ }_{n} C_{r}=\frac{n !}{(n-r) ! r !}$

${ }_{10} C_{4}=\frac{10 !}{(10-4) ! 4 !}=\frac{10 \times 9 \times 8 \times 7}{4 \times 3 \times 2 \times 1}=210$

\begin{enumerate}
  \setcounter{enumi}{32}
  \item B is correct. The number of permutations is the number of ways to pick five stocks out of 30 in the correct order.
\end{enumerate}

${ }_{n} P_{r}=\frac{n !}{(n-r) ! r !}$

${ }_{30} P_{5}=\frac{30 !}{(30-5) !}=\frac{30 !}{25 !}=30 \times 29 \times 28 \times 27 \times 26=17,100,720$

The contestant's chance of winning is one out of $17,100,720$.

\begin{enumerate}
  \setcounter{enumi}{33}
  \item A is correct. The number of combinations is the number of ways to pick five stocks out of 30 without regard to order.
\end{enumerate}

${ }_{n} C_{r}=\frac{n !}{(n-r) ! r !}$

${ }_{30} C_{5}=\frac{30 !}{(30-5) ! 5 !}=\frac{30 \times 29 \times 28 \times 27 \times 26}{5 \times 4 \times 3 \times 2 \times 1}=142,506$

The contestant's chance of winning is one out of 142,506.

\begin{enumerate}
  \setcounter{enumi}{34}
  \item This contest does not resemble a usual lottery. Each of the 30 stocks does not have an equal chance of having the highest returns. Furthermore, contestants may have some favored investments, and the 30 stocks will not be chosen with the same frequencies. To guard against more than one person selecting the winners correctly, Sturm may wish to stipulate that if there is more than one winner, the winners will share the $\$ 1$ million prize.
\end{enumerate}

\section*{LEARNING MODULE 
 4 }
\section{Common Probability Distributions}


\section{LEARNING OUTCOME}
\begin{center}
\begin{tabular}{c|l}
Mastery & The candidate should be able to: \\
$\square$ & $\begin{array}{l}\text { define a probability distribution and compare and contrast discrete } \\ \text { and continuous random variables and their probability functions } \\ \text { calculate and interpret probabilities for a random variable given its } \\ \text { cumulative distribution function } \\ \text { describe the properties of a discrete uniform random variable, and } \\ \text { calculate and interpret probabilities given the discrete uniform } \\ \text { distribution function } \\ \text { describe the properties of the continuous uniform distribution, and } \\ \text { calculate and interpret probabilities given a continuous uniform } \\ \text { distribution } \\ \text { describe the properties of a Bernoulli random variable and a } \\ \text { binomial random variable, and calculate and interpret probabilities } \\ \text { given the binomial distribution function } \\ \text { explain the key properties of the normal distribution } \\ \square\end{array} \square$ \\
$\square$ & $\begin{array}{l}\text { contrast a multivariate distribution and a univariate distribution, and } \\ \text { explain the role of correlation in the multivariate normal distribution } \\ \text { calculate the probability that a normally distributed random variable } \\ \text { lies inside a given interval } \\ \text { explain how to standardize a random variable } \\ \text { calculate and interpret probabilities using the standard normal } \\ \text { distribution } \\ \text { define shortfall risk, calculate the safety-first ratio, and identify an } \\ \text { optimal portfolio using Roy's safety-first criterion }\end{array}$ \\
$\square$ &  \\
\end{tabular}
\end{center}

\section{LEARNING OUTCOME}
\begin{center}
\begin{tabular}{c|l}
Mastery & The candidate should be able to: \\
\hline
$\square$ & $\begin{array}{l}\text { explain the relationship between normal and lognormal distributions } \\ \text { and why the lognormal distribution is used to model asset prices } \\ \text { calculate and interpret a continuously compounded rate of return, } \\ \text { given a specific holding period return } \\ \text { describe the properties of the Student's } t \text {-distribution, and calculate } \\ \text { and interpret its degrees of freedom } \\ \text { describe the properties of the chi-square distribution and the } \\ \text { F-distribution, and calculate and interpret their degrees of freedom } \\ \text { describe Monte Carlo simulation }\end{array}$ \\
$\square$ &  \\
\end{tabular}
\end{center}

\begin{center}
\includegraphics[max width=\textwidth]{2023_05_04_cff39ee44f77d6514e1bg-246}
\end{center}

\section{DISCRETE RANDOM VARIABLES}
define a probability distribution and compare and contrast discrete and continuous random variables and their probability functions calculate and interpret probabilities for a random variable given its cumulative distribution function

Probabilities play a critical role in investment decisions. Although we cannot predict the future, informed investment decisions are based on some kind of probabilistic thinking. An analyst may put probability estimates behind the success of her high-conviction or low-conviction stock recommendations. Risk managers would typically think in probabilistic terms: What is the probability of not achieving the target return, or what kind of losses are we facing with high likelihood over the relevant time horizon? Probability distributions also underpin validating trade signal-generating models: For example, does earnings revision play a significant role in forecasting stock returns?

In nearly all investment decisions, we work with random variables. The return on a stock and its earnings per share are familiar examples of random variables. To make probability statements about a random variable, we need to understand its probability distribution. A probability distribution specifies the probabilities associated with the possible outcomes of a random variable.

In this reading, we present important facts about seven probability distributions and their investment uses. These seven distributions-the uniform, binomial, normal, lognormal, Student's $t$-, chi-square, and $F$-distributions-are used extensively in investment analysis. Normal and binomial distributions are used in such basic valuation models as the Black-Scholes-Merton option pricing model, the binomial option pricing model, and the capital asset pricing model. Student's $t$-, chi-square, and $F$-distributions are applied in validating statistical significance and in hypothesis testing. With the working knowledge of probability distributions provided in this reading, you will be better prepared to study and use other quantitative methods, such as regression analysis, time-series analysis, and hypothesis testing. After discussing probability distributions, we end with an introduction to Monte Carlo simulation, a computer-based tool for obtaining information on complex investment problems. We start by defining basic concepts and terms, then illustrate the operation of these concepts through the simplest distribution, the uniform distribution, and then address probability distributions that have more applications in investment work but also greater complexity.

\section{Discrete Random Variables}
A random variable is a quantity whose future outcomes are uncertain. The two basic types of random variables are discrete random variables and continuous random variables. A discrete random variable can take on at most a countable (possibly infinite) number of possible values. For example, a discrete random variable $X$ can take on a limited number of outcomes $x_{1}, x_{2}, \ldots, x_{n}$ ( $n$ possible outcomes), or a discrete random variable $Y$ can take on an unlimited number of outcomes $y_{1}, y_{2}, \ldots$ (without end). The number of "yes" votes at a corporate board meeting, for example, is a discrete variable that is countable and finite (from 0 to the voting number of board members). The number of trades at a stock exchange is also countable but is infinite, since there is no limit to the number of trades by the market participants. Note that $X$ refers to the random variable, and $x$ refers to an outcome of $X$. We subscript outcomes, as in $x_{1}$ and $x_{2}$, when we need to distinguish among different outcomes in a list of outcomes of a random variable. Since we can count all the possible outcomes of $X$ and $Y$ (even if we go on forever in the case of $Y$ ), both $X$ and $Y$ satisfy the definition of a discrete random variable.

In contrast, we cannot count the outcomes of a continuous random variable. We cannot describe the possible outcomes of a continuous random variable $Z$ with a list $z_{1}, z_{2}, \ldots$, because the outcome $\left(z_{1}+z_{2}\right) / 2$, not in the list, would always be possible. The volume of water in a glass is an example of a continuous random variable since we cannot "count" water on a discrete scale but can only measure its volume. In finance, unless a variable exhibits truly discrete behavior-for example, a positive or negative earnings surprise or the number of central bank board members voting for a rate hike-it is practical to work with a continuous distribution in many cases. The rate of return on an investment is an example of such a continuous random variable.

In working with a random variable, we need to understand its possible outcomes. For example, a majority of the stocks traded on the New Zealand Stock Exchange are quoted in increments of NZ\$0.01. Quoted stock price is thus a discrete random variable with possible values NZ $\$ 0, \mathrm{NZ} \$ 0.01$, NZ $\$ 0.02, \ldots$, but we can also model stock price as a continuous random variable (as a lognormal random variable, to look ahead). In many applications, we have a choice between using a discrete or a continuous distribution. We are usually guided by which distribution is most efficient for the task we face. This opportunity for choice is not surprising, because many discrete distributions can be approximated with a continuous distribution, and vice versa. In most practical cases, a probability distribution is only a mathematical idealization, or approximate model, of the relative frequencies of a random variable's possible outcomes.

Every random variable is associated with a probability distribution that describes the variable completely. We can view a probability distribution in two ways. The basic view is the probability function, which specifies the probability that the random variable takes on a specific value: $P(X=x)$ is the probability that a random variable $X$ takes on the value $x$. For a discrete random variable, the shorthand notation for the probability function (sometimes referred to as the "probability mass function") is $p(x)$ $=P(X=x)$. For continuous random variables, the probability function is denoted $f(x)$ and called the probability density function (pdf), or just the density.

A probability function has two key properties (which we state, without loss of generality, using the notation for a discrete random variable):

\begin{itemize}
  \item $0 \leq p(x) \leq 1$, because probability is a number between 0 and 1 . - The sum of the probabilities $p(x)$ over all values of $X$ equals 1 . If we add up the probabilities of all the distinct possible outcomes of a random variable, that sum must equal 1 .
\end{itemize}

We are often interested in finding the probability of a range of outcomes rather than a specific outcome. In these cases, we take the second view of a probability distribution, the cumulative distribution function (cdf). The cumulative distribution function, or distribution function for short, gives the probability that a random variable $X$ is less than or equal to a particular value $x, P(X \leq x)$. For both discrete and continuous random variables, the shorthand notation is $F(x)=P(X \leq x)$. How does the cumulative distribution function relate to the probability function? The word "cumulative" tells the story. To find $F(x)$, we sum up, or accumulate, values of the probability function for all outcomes less than or equal to $x$. The function of the cdf is parallel to that of cumulative relative frequency.

We illustrate the concepts of probability density functions and cumulative distribution functions with an empirical example using daily returns (i.e., percentage changes) of the fictitious Euro-Asia-Africa (EAA) Equity Index. This dataset spans five years and consists of 1,258 observations, with a minimum value of $-4.1 \%$, a maximum value of $5.0 \%$, a range of $9.1 \%$, and a mean daily return of $0.04 \%$.

Exhibit 1 depicts the histograms, representing pdfs, and empirical cdfs (i.e., accumulated values of the bars in the histograms) based on daily returns of the EAA Equity Index. Panels A and B represent the same dataset; the only difference is the histogram bins used in Panel A are wider than those used in Panel B, so naturally Panel B has more bins. Note that in Panel A, we divided the range of observed daily returns $(-5 \%$ to $5 \%$ ) into 10 bins, so we chose the bin width to be $1.0 \%$. In Panel B, we wanted a more granular histogram with a narrower range, so we divided the range into 20 bins, resulting in a bin width of $0.5 \%$. Panel A gives a sense of the observed range of daily index returns, whereas Panel B is much more granular, so it more closely resembles continuous pdf and cdf graphs.

\section{Exhibit 1: PDFs and CDFs of Daily Returns for EAA Equity Index}
\begin{center}
\includegraphics[max width=\textwidth]{2023_05_04_cff39ee44f77d6514e1bg-249}
\end{center}

\section{EXAMPLE 1}
Using PDFs and CDFs of Discrete and Continuous Random Variables to Calculate Probabilities

\section{Discrete Random Variables: Rolling a Die}
The example of rolling a six-sided die is an easy and intuitive way to illustrate a discrete random variable's pdf and cdf. 1. What is the probability that you would roll the number 3 ?

\section{Solution to 1:}
Assuming the die is fair, rolling any number from 1 to 6 has a probability of $1 / 6$ each, so the chance of rolling the number 3 would also equal $1 / 6$. This outcome represents the pdf of this game; the pdf at number 3 takes a value of $1 / 6$. In fact, it takes a value of $1 / 6$ at every number from 1 to 6 .

\begin{enumerate}
  \setcounter{enumi}{1}
  \item What is the probability that you would roll a number less than or equal to 3 ?
\end{enumerate}

\section{Solution to 2:}
Answering this question involves the cdf of the die game. Three possible events would satisfy our criterion-namely, rolling 1,2 , or 3 . The probability of rolling any of these numbers would be $1 / 6$, so by accumulating them from 1 to 3 , we get a probability of $3 / 6$, or $1 / 2$. The cdf of rolling a die takes a value of $1 / 6$ at number $1,3 / 6$ (or $50 \%$ ) at number 3 , and $6 / 6$ (or $100 \%$ ) at number 6.

\section{Continuous Random Variables: EAA Index Return}
\begin{enumerate}
  \setcounter{enumi}{2}
  \item We use the EAA Index to illustrate the pdf and cdf for a continuously distributed random variable. Since daily returns can take on any numbers within a reasonable range rather than having discrete outcomes, we represent the pdf in the context of bins (as shown in the following table).
\end{enumerate}

\begin{center}
\begin{tabular}{rclc}
\hline
Bin & PDF & Bin & PDF \\
\hline
$-5 \%$ to $-4 \%$ & $0.1 \%$ & $0 \%$ to $1 \%$ & $44.1 \%$ \\
$-4 \%$ to $-3 \%$ & $0.6 \%$ & $1 \%$ to $2 \%$ & $8.8 \%$ \\
$-3 \%$ to $-2 \%$ & $1.8 \%$ & $2 \%$ to $3 \%$ & $1.0 \%$ \\
$-2 \%$ to $-1 \%$ & $6.1 \%$ & $3 \%$ to $4 \%$ & $0.1 \%$ \\
$-1 \%$ to $0 \%$ & $37.4 \%$ & $4 \%$ to $5 \%$ & $0.1 \%$ \\
\hline
\end{tabular}
\end{center}

In our sample, we did not find any daily returns below $-5 \%$, and we found only $0.1 \%$ of the total observations between $-5 \%$ and $-4 \%$. In the next bin, $-4 \%$ to $-3 \%$, we found $0.6 \%$ of the total observations, and so on.

If this empirical pdf is a guide for the future, what is the probability that we will see a daily return less than $-2 \%$ ?

\section{Solution to 3:}
We must calculate the cdf up to -2\%. The answer is the sum of the pdfs of the first three bins (see the shaded rectangle in the table provided); $0.1 \%+$ $0.6 \%+1.8 \%=2.5 \%$. So, the probability that we will see a daily return less than $-2 \%$ is $2.5 \%$.

Next, we illustrate these concepts with examples and show how we use discrete and continuous distributions. We start with the simplest distribution, the discrete uniform distribution.

\textbackslash section\{DISCRETE AND CONTINUOUS UNIFORM DISTRIBUTION

\begin{center}
\includegraphics[max width=\textwidth]{2023_05_04_cff39ee44f77d6514e1bg-251}
\end{center}

The simplest of all probability distributions is the discrete uniform distribution. Suppose that the possible outcomes are the integers (whole numbers) 1-8, inclusive, and the probability that the random variable takes on any of these possible values is the same for all outcomes (that is, it is uniform). With eight outcomes, $p(x)=1 / 8$, or 0.125 , for all values of $X(X=1,2,3,4,5,6,7,8)$; this statement is a complete description of this discrete uniform random variable. The distribution has a finite number of specified outcomes, and each outcome is equally likely. Exhibit 2 summarizes the two views of this random variable, the probability function and the cumulative distribution function, with Panel A in tabular form and Panel B in graphical form.

\section{Exhibit 2: PDF and DCF for Discrete Uniform Random Variable}
\section{A. Probability Function and Cumulative Distribution Function for a Discrete Uniform}
 Random Variable\begin{center}
\begin{tabular}{lcc}
\hline
$\boldsymbol{X}=\boldsymbol{x}$ & $\begin{array}{c}\text { Probability Function } \\ \boldsymbol{p}(\boldsymbol{x})=\boldsymbol{P}(\boldsymbol{X}=\boldsymbol{x})\end{array}$ & $\begin{array}{c}\text { Cumulative Distribution Function } \\ \boldsymbol{F}(\boldsymbol{x})=\boldsymbol{P}(\boldsymbol{X} \leq \boldsymbol{x})\end{array}$ \\
\hline
1 & 0.125 & 0.125 \\
2 & 0.125 & 0.250 \\
3 & 0.125 & 0.375 \\
4 & 0.125 & 0.500 \\
5 & 0.125 & 0.625 \\
6 & 0.125 & 0.750 \\
7 & 0.125 & 0.875 \\
8 & 0.125 & 1.000 \\
\hline
\end{tabular}
\end{center}

B. Graph of PDF and CDF for Discrete Uniform Random Variable

\begin{center}
\includegraphics[max width=\textwidth]{2023_05_04_cff39ee44f77d6514e1bg-252}
\end{center}

$(x)$

We can use the table in Panel A to find three probabilities: $P(X \leq 7), P(4 \leq X \leq$ 6), and $P(4<X \leq 6)$. The following examples illustrate how to use the cdf to find the probability that a random variable will fall in any interval (for any random variable, not only the uniform one). The results can also be gleaned visually from the graph in Panel B.

\begin{itemize}
  \item The probability that $X$ is less than or equal to $7, P(X \leq 7)$, is the next-to-last entry in the third column: 0.875 , or $87.5 \%$.

  \item To find $P(4 \leq X \leq 6)$, we need to find the sum of three probabilities: $p(4)$, $p(5)$, and $p(6)$. We can find this sum in two ways. We can add $p(4), p(5)$, and $p$ (6) from the second column. Or we can calculate the probability as the difference between two values of the cumulative distribution function:

\end{itemize}

$F(6)=P(X \leq 6)=p(6)+p(5)+p(4)+p(3)+p(2)+p(1)$,

and

$F(3)=P(X \leq 3)=p(3)+p(2)+p(1)$

So

$P(4 \leq X \leq 6)=F(6)-F(3)=p(6)+p(5)+p(4)=3 / 8$.

So, we calculate the second probability as $F(6)-F(3)=3 / 8$. This can be seen as the shaded area under the step function cdf graph in Panel B.

\begin{itemize}
  \item The third probability, $P(4<X \leq 6)$, the probability that $X$ is less than or equal to 6 but greater than 4 , is $p(5)+p(6)$. We compute it as follows, using the cdf:
\end{itemize}

$P(4<X \leq 6)=P(X \leq 6)-P(X \leq 4)=F(6)-F(4)=p(6)+p(5)=2 / 8$.

So we calculate the third probability as $F(6)-F(4)=2 / 8$.

Suppose we want to check that the discrete uniform probability function satisfies the general properties of a probability function given earlier. The first property is $0 \leq$ $p(x) \leq 1$. We see that $p(x)=1 / 8$ for all $x$ in the first column of in Panel A. [Note that $p(x)$ equals 0 for numbers $x$ that are not in that column, such as -14 or 12.215.] The first property is satisfied. The second property is that the probabilities sum to 1 . The entries in the second column of Panel A do sum to 1 .

The cdf has two other characteristic properties:

\begin{itemize}
  \item The cdf lies between 0 and 1 for any $x: 0 \leq F(x) \leq 1$.

  \item As $x$ increases, the cdf either increases or remains constant.

\end{itemize}

Check these statements by looking at the third column in the table in Panel A and at the graph in Panel B.

We now have some experience working with probability functions and cdfs for discrete random variables. Later, we will discuss Monte Carlo simulation, a methodology driven by random numbers. As we will see, the uniform distribution has an important technical use: It is the basis for generating random numbers, which, in turn, produce random observations for all other probability distributions.

\section{Continuous Uniform Distribution}
The continuous uniform distribution is the simplest continuous probability distribution. The uniform distribution has two main uses. As the basis of techniques for generating random numbers, the uniform distribution plays a role in Monte Carlo simulation. As the probability distribution that describes equally likely outcomes, the uniform distribution is an appropriate probability model to represent a particular kind of uncertainty in beliefs in which all outcomes appear equally likely.

The pdf for a uniform random variable is

$f(x)=\left\{\begin{array}{l}\frac{1}{b-a} \text { for } a \leq x \leq b \\ 0 \text { otherwise }\end{array}\right.$

For example, with $a=0$ and $b=8, f(x)=1 / 8$, or 0.125 . We graph this density in Exhibit 3.

\section{Exhibit 3: Probability Density Function for a Continuous Uniform}
Distribution

\begin{center}
\includegraphics[max width=\textwidth]{2023_05_04_cff39ee44f77d6514e1bg-253}
\end{center}

The graph of the density function plots as a horizontal line with a value of 0.125 . What is the probability that a uniform random variable with limits $a=0$ and $b=8$ is less than or equal to 3 , or $F(3)=P(X \leq 3)$ ? When we were working with the discrete uniform random variable with possible outcomes $1,2, \ldots, 8$, we summed individual probabilities: $p(1)+p(2)+p(3)=0.375$. In contrast, the probability that a continuous uniform random variable or any continuous random variable assumes any given fixed value is 0 . To illustrate this point, consider the narrow interval 2.510-2.511. Because that interval holds an infinity of possible values, the sum of the probabilities of values in that interval alone would be infinite if each individual value in it had a positive probability. To find the probability $F(3)$, we find the area under the curve graphing the pdf, between 0 and 3 on the $x$-axis (shaded area in Exhibit 3 ). In calculus, this operation is called integrating the probability function $f(x)$ from 0 to 3 . This area under the curve is a rectangle with base $3-0=3$ and height $1 / 8$. The area of this rectangle equals base times height: $3(1 / 8)=3 / 8$, or 0.375 . So $F(3)=3 / 8$, or 0.375 .

The interval from 0 to 3 is three-eighths of the total length between the limits of 0 and 8 , and $F(3)$ is three-eighths of the total probability of 1 . The middle line of the expression for the cdf captures this relationship:

$$
F(x)=\left\{\begin{array}{l}
0 \text { for } x<a \\
\frac{x-a}{b-a} \\
1 \text { for } x<b
\end{array} \text { for } a \leq x \leq b .\right.
$$

For our problem, $F(x)=0$ for $x \leq 0, F(x)=x / 8$ for $0<x<8$, and $F(x)=1$ for $x \geq 8$. Exhibit 4 shows a graph of this cdf.

\section{Exhibit 4: Continuous Uniform Cumulative Distribution}
\begin{center}
\includegraphics[max width=\textwidth]{2023_05_04_cff39ee44f77d6514e1bg-254}
\end{center}

The mathematical operation that corresponds to finding the area under the curve of a pdf $f(x)$ from $a$ to $b$ is the definite integral of $f(x)$ from $a$ to $b$ :

$$
P(a \leq X \leq b)=\int_{a}^{b} f(x) d x,
$$

where $\int d x$ is the symbol for summing $\int$ over small changes $d x$ and the limits of integration $(a$ and $b)$ can be any real numbers or $-\infty$ and $+\infty$. All probabilities of continuous random variables can be computed using Equation 1. For the uniform distribution example considered previously, $F(7)$ is Equation 1 with lower limit $a=0$ and upper limit $b=7$. The integral corresponding to the cdf of a uniform distribution reduces to the three-line expression given previously. To evaluate Equation 1 for nearly all other continuous distributions, including the normal and lognormal, we rely on spreadsheet functions, computer programs, or tables of values to calculate probabilities. Those tools use various numerical methods to evaluate the integral in Equation 1.

Recall that the probability of a continuous random variable equaling any fixed point is 0 . This fact has an important consequence for working with the cumulative distribution function of a continuous random variable: For any continuous random variable $X, P(a \leq X \leq b)=P(a<X \leq b)=P(a \leq X<b)=P(a<X<b)$, because the probabilities at the endpoints $a$ and $b$ are 0 . For discrete random variables, these relations of equality are not true, because for them probability accumulates at points.

\section{EXAMPLE 2}
\section{Probability That a Lending Facility Covenant Is Breached}
You are evaluating the bonds of a below-investment-grade borrower at a low point in its business cycle. You have many factors to consider, including the terms of the company's bank lending facilities. The contract creating a bank lending facility such as an unsecured line of credit typically has clauses known as covenants. These covenants place restrictions on what the borrower can do. The company will be in breach of a covenant in the lending facility if the interest coverage ratio, EBITDA/interest, calculated on EBITDA over the four trailing quarters, falls below 2.0. EBITDA is earnings before interest, taxes, depreciation, and amortization. Compliance with the covenants will be checked at the end of the current quarter. If the covenant is breached, the bank can demand immediate repayment of all borrowings on the facility. That action would probably trigger a liquidity crisis for the company. With a high degree of confidence, you forecast interest charges of $\$ 25$ million. Your estimate of EBITDA runs from $\$ 40$ million on the low end to $\$ 60$ million on the high end.

Address two questions (treating projected interest charges as a constant):

\begin{enumerate}
  \item If the outcomes for EBITDA are equally likely, what is the probability that EBITDA/interest will fall below 2.0, breaching the covenant?
\end{enumerate}

\section{Solution to 1:}
EBITDA/interest is a continuous uniform random variable because all outcomes are equally likely. The ratio can take on values between $1.6=(\$ 40$ million)/(\$25 million) on the low end and $2.4=$ ( $\$ 60$ million $/ \$ 25$ million) on the high end. The range of possible values is $2.4-1.6=0.8$. The fraction of possible values falling below 2.0 , the level that triggers default, is the distance between 2.0 and 1.6 , or 0.40 ; the value 0.40 is one-half the total length of 0.8 , or $0.4 / 0.8=0.50$. So, the probability that the covenant will be breached is $50 \%$.

\begin{enumerate}
  \setcounter{enumi}{1}
  \item Estimate the mean and standard deviation of EBITDA/interest. For a continuous uniform random variable, the mean is given by $\mu=(a+b) / 2$ and the variance is given by $\sigma^{2}=(b-a)^{2} / 12$.
\end{enumerate}

\section{Solution to 2:}
In Solution 1, we found that the lower limit of EBITDA/interest is 1.6. This lower limit is $a$. We found that the upper limit is 2.4. This upper limit is $b$. Using the formula given previously,

$\mu=(a+b) / 2=(1.6+2.4) / 2=2.0$.

The variance of the interest coverage ratio is

$$
\sigma^{2}=(b-a)^{2} / 12=(2.4-1.6)^{2} / 12=0.053333 .
$$

The standard deviation is the positive square root of the variance, 0.230940 $=(0.053333)^{1 / 2}$. However, the standard deviation is not particularly useful as a risk measure for a uniform distribution. The probability that lies within various standard deviation bands around the mean is sensitive to different specifications of the upper and lower. Here, a one standard deviation interval around the mean of 2.0 runs from 1.769 to 2.231 and captures $0.462 / 0.80$ $=0.5775$, or $57.8 \%$, of the probability. A two standard deviation interval runs from 1.538 to 2.462 , which extends past both the lower and upper limits of the random variable.

\section{BINOMIAL DISTRIBUTION}
describe the properties of a Bernoulli random variable and a binomial random variable, and calculate and interpret probabilities given the binomial distribution function

In many investment contexts, we view a result as either a success or a failure or as binary (twofold) in some other way. When we make probability statements about a record of successes and failures or about anything with binary outcomes, we often use the binomial distribution. What is a good model for how a stock price moves over time? Different models are appropriate for different uses. Cox, Ross, and Rubinstein (1979) developed an option pricing model based on binary moves-price up or price down-for the asset underlying the option. Their binomial option pricing model was the first of a class of related option pricing models that have played an important role in the development of the derivatives industry. That fact alone would be sufficient reason for studying the binomial distribution, but the binomial distribution has uses in decision making as well.

The building block of the binomial distribution is the Bernoulli random variable, named after the Swiss probabilist Jakob Bernoulli (1654-1704). Suppose we have a trial (an event that may repeat) that produces one of two outcomes. Such a trial is a Bernoulli trial. If we let $Y$ equal 1 when the outcome is success and $Y$ equal 0 when the outcome is failure, then the probability function of the Bernoulli random variable $Y$ is

$$
p(1)=P(Y=1)=p
$$

and

$p(0)=P(Y=0)=1-p$,

where $p$ is the probability that the trial is a success.

In $n$ Bernoulli trials, we can have 0 to $n$ successes. If the outcome of an individual trial is random, the total number of successes in $n$ trials is also random. A binomial random variable $X$ is defined as the number of successes in $n$ Bernoulli trials. A binomial random variable is the sum of Bernoulli random variables $Y_{i}$, where $i=1,2, \ldots, n$ :

$$
X=Y_{1}+Y_{2}+\ldots+Y_{n},
$$

where $Y_{i}$ is the outcome on the $i$ th trial (1 if a success, 0 if a failure). We know that a Bernoulli random variable is defined by the parameter $p$. The number of trials, $n$, is the second parameter of a binomial random variable. The binomial distribution makes these assumptions:

\begin{itemize}
  \item The probability, $p$, of success is constant for all trials.

  \item The trials are independent.

\end{itemize}

The second assumption has great simplifying force. If individual trials were correlated, calculating the probability of a given number of successes in $n$ trials would be much more complicated.

Under these two assumptions, a binomial random variable is completely described by two parameters, $n$ and $p$. We write

$X \sim B(n, p)$

which we read as " $X$ has a binomial distribution with parameters $n$ and $p$ " You can see that a Bernoulli random variable is a binomial random variable with $n=1$ : $Y \sim B(1, p)$.

Now we can find the general expression for the probability that a binomial random variable shows $x$ successes in $n$ trials (also known as the probability mass function). We can think in terms of a model of stock price dynamics that can be generalized to allow any possible stock price movements if the periods are made extremely small. Each period is a Bernoulli trial: With probability $p$, the stock price moves up; with probability $1-p$, the price moves down. A success is an up move, and $x$ is the number of up moves or successes in $n$ periods (trials). With each period's moves independent and $p$ constant, the number of up moves in $n$ periods is a binomial random variable. We now develop an expression for $P(X=x)$, the probability function for a binomial random variable.

Any sequence of $n$ periods that shows exactly $x$ up moves must show $n-x$ down moves. We have many different ways to order the up moves and down moves to get a total of $x$ up moves, but given independent trials, any sequence with $x$ up moves must occur with probability $p^{x}(1-p)^{n-x}$. Now we need to multiply this probability by the number of different ways we can get a sequence with $x$ up moves. Using a basic result in counting, there are

$$
\frac{n !}{(n-x) ! x !}
$$

different sequences in $n$ trials that result in $x$ up moves (or successes) and $n-x$ down moves (or failures). Recall that for positive integers $n, n$ factorial $(n !)$ is defined as $n(n-1)(n-2) \ldots 1$ (and $0 !=1$ by convention). For example, $5 !=(5)(4)(3)(2)(1)=$ 120. The combination formula $n ! /[(n-x) ! x !]$ is denoted by

(read " $n$ combination $x$ " or " $n$ choose $x$ "). For example, over three periods, exactly three different sequences have two up moves: $u u d, u d u$, and $d u u$. We confirm this by

$$
\left(\begin{array}{l}
3 \\
2
\end{array}\right)=\frac{3 !}{(3-2) ! 2 !}=\frac{(3)(2)(1)}{(1)(2)(1)}=3 \text {. }
$$

If, hypothetically, each sequence with two up moves had a probability of 0.15 , then the total probability of two up moves in three periods would be $3 \times 0.15=0.45$. This example should persuade you that for $X$ distributed $B(n, p)$, the probability of $x$ successes in $n$ trials is given by

$$
p(x)=P(X=x)=\left(\begin{array}{l}
n \\
x
\end{array}\right) p^{x}(1-p)^{n-x}=\frac{n !}{(n-x) ! x !} p^{x}(1-p)^{n-x} .
$$

Some distributions are always symmetric, such as the normal, and others are always asymmetric or skewed, such as the lognormal. The binomial distribution is symmetric when the probability of success on a trial is 0.50 , but it is asymmetric or skewed otherwise.

We illustrate Equation 2 (the probability function) and the cdf through the symmetrical case by modeling the behavior of stock price movements on four consecutive trading days in a binomial tree framework. Each day is an independent trial. The stock moves up with constant probability $p$ (the up transition probability); if it moves up, $u$ is 1 plus the rate of return for an up move. The stock moves down with constant probability $1-p$ (the down transition probability); if it moves down, $d$ is 1 plus the rate of return for a down move. The binomial tree is shown in Exhibit 5, where we now associate each of the $n=4$ stock price moves with time indexed by $t$; the shape of the graph suggests why it is a called a binomial tree. Each boxed value from which successive moves or outcomes branch out in the tree is called a node. The initial node, at $t=0$, shows the beginning stock price, $S$. Each subsequent node represents a potential value for the stock price at the specified future time.

\section{Exhibit 5: A Binomial Model of Stock Price Movement}
\begin{center}
\includegraphics[max width=\textwidth]{2023_05_04_cff39ee44f77d6514e1bg-258}
\end{center}

We see from the tree that the stock price at $t=4$ has five possible values: uuuuS, uuudS, uuddS, $d d d u S$, and $d d d d S$. The probability that the stock price equals any one of these five values is given by the binomial distribution. For example, four sequences of moves result in a final stock price of иuиdS: These are иuиd, ииdu, иduи, and диuи. These sequences have three up moves out of four moves in total; the combination formula confirms that the number of ways to get three up moves (successes) in four periods (trials) is $4 ! /(4-3) ! 3 !=4$. Next, note that each of these sequences-uииd, ииdu, uduu, and $d u u u$-has probability $p^{3}(1-p)^{1}$, which equals $0.0625\left(=0.50^{3} \times 0.50^{1}\right)$. So, $P\left(S_{4}=u u u d S\right)=4\left[p^{3}(1-p)\right]$, or 0.25 , where $S_{4}$ indicates the stock's price after four moves. This is shown numerically in Panel A of Exhibit 6, in the line indicating three up moves in $x$, as well as graphically in Panel $\mathrm{B}$, as the height of the bar above $x=3$. Note that in Exhibit 6, columns 5 and 6 in Panel A show the pdf and cdf, respectively, for this binomial distribution, and in Panel B, the pdf and cdf are represented by the bars and the line graph, respectively.

\section{Exhibit 6: PDF and CDF of Binomial Probabilities for Stock Price Movements}
A. Binomial Probabilities, $n=4$ and $p=0.50$

\begin{center}
\begin{tabular}{|c|c|c|c|c|c|}
\hline
$\begin{array}{l}\text { Col. } 1 \\ \text { Number of Up } \\ \text { Moves, } \\ x\end{array}$ & $\begin{array}{c}\text { Col. } 2 \\ \text { Implied Number } \\ \text { of Down Moves, } \\ n-x\end{array}$ & $\begin{array}{c}\text { Col. } 3^{A} \\ \text { Number of Possible } \\ \text { Ways to Reach } x \text { Up } \\ \text { Moves }\end{array}$ & $\begin{array}{c}\text { Col. } 4^{B} \\ \text { Probability for } \\ \text { Each Way, } \\ p(x)\end{array}$ & $\begin{array}{l}\text { Col. } 5^{C} \\ \text { Probability for } \\ \qquad x p(x)\end{array}$ & $\begin{array}{c}\text { Col. } 6 \\ F(x)=P(X \leq x)\end{array}$ \\
\hline
0 & 4 & 1 & 0.0625 & 0.0625 & 0.0625 \\
\hline
1 & 3 & 4 & 0.0625 & 0.2500 & 0.3125 \\
\hline
2 & 2 & 6 & 0.0625 & 0.3750 & 0.6875 \\
\hline
3 & 1 & 4 & 0.0625 & 0.2500 & 0.9375 \\
\hline
\multirow[t]{2}{*}{4} & 0 & 1 & 0.0625 & 0.0625 & 1.0000 \\
\hline
 &  &  &  & 1.0000 &  \\
\hline
\end{tabular}
\end{center}

A: Column $3=n ! /[(n-x) ! x !]$

B: Column $4=p^{x}(1-p)^{n-x}$

C: Column $5=$ Column $3 \times$ Column 4

\section{B. Graphs of Binomial PDF and CDF}
\begin{center}
\includegraphics[max width=\textwidth]{2023_05_04_cff39ee44f77d6514e1bg-259}
\end{center}

PDF $\cdots . . .$.$) CDF$

To be clear, the binomial random variable in this application is the number of up moves. Final stock price distribution is a function of the initial stock price, the number of up moves, and the size of the up moves and down moves. We cannot say that stock price itself is a binomial random variable; rather, it is a function of a binomial random variable, as well as of $u$ and $d$, and initial price, $S$. This richness is actually one key to why this way of modeling stock price is useful: It allows us to choose values of these parameters to approximate various distributions for stock price (using a large number of time periods). One distribution that can be approximated is the lognormal, an important continuous distribution model for stock price that we will discuss later. The flexibility extends further. In the binomial tree shown in Exhibit 5 , the transition probabilities are the same at each node: $p$ for an up move and $1-p$ for a down move. That standard formula describes a process in which stock return volatility is constant over time. Derivatives experts, however, sometimes model changing volatility over time using a binomial tree in which the probabilities for up and down moves differ at different nodes.

\section{EXAMPLE 3}
\section{A Trading Desk Evaluates Block Brokers}
Blocks are orders to sell or buy that are too large for the liquidity ordinarily available in dealer networks or stock exchanges. Your firm has known interests in certain kinds of stock. Block brokers call your trading desk when they want to sell blocks of stocks that they think your firm may be interested in buying. You know that these transactions have definite risks. For example, if the broker's client (the seller) has unfavorable information on the stock or if the total amount he or she is selling through all channels is not truthfully communicated, you may see an immediate loss on the trade. Your firm regularly audits the performance of block brokers by calculating the post-trade, market-risk-adjusted returns on stocks purchased from block brokers. On that basis, you classify each trade as unprofitable or profitable. You have summarized the performance of the brokers in a spreadsheet, excerpted in the following table for November of last year. The broker names are coded BB001 and BB002.

\section{Block Trading Gains and Losses}
\begin{center}
\begin{tabular}{|c|c|c|}
\hline
 & Profitable Trades & Losing Trades \\
\hline
BB001 & 3 & 9 \\
\hline
BB002 & 5 & 3 \\
\hline
\end{tabular}
\end{center}

You now want to evaluate the performance of the block brokers, and you begin with two questions:

\begin{enumerate}
  \item If you are paying a fair price on average in your trades with a broker, what should be the probability of a profitable trade?
\end{enumerate}

\section{Solution to 1:}
If the price you trade at is fair, then $50 \%$ of the trades you do with a broker should be profitable.

\begin{enumerate}
  \setcounter{enumi}{1}
  \item Did each broker meet or miss that expectation on probability?
\end{enumerate}

\section{Solution to 2:}
Your firm has logged $3+9=12$ trades with block broker BB001. Since 3 of the 12 trades were profitable, the portion of profitable trades was $3 / 12$, or $25 \%$. With broker BB002, the portion of profitable trades was $5 / 8$, or $62.5 \%$. The rate of profitable trades with broker BB001 of $25 \%$ clearly missed your performance expectation of $50 \%$. Broker BB002, at $62.5 \%$ profitable trades, exceeded your expectation.

\begin{enumerate}
  \setcounter{enumi}{2}
  \item You also realize that the brokers' performance has to be evaluated in light of the sample sizes, and for that you need to use the binomial probability function (Equation 2).
\end{enumerate}

Under the assumption that the prices of trades were fair,

A. calculate the probability of three or fewer profitable trades with broker BB001. B. calculate the probability of five or more profitable trades with broker BB002.

\section{Solution to 3:}
A. For broker BB001, the number of trades (the trials) was $n=12$, and 3 were profitable. You are asked to calculate the probability of three or fewer profitable trades, $F(3)=p(3)+p(2)+p(1)+p(0)$.

Suppose the underlying probability of a profitable trade with BB001 is $p=0.50$. With $n=12$ and $p=0.50$, according to Equation 2 the probability of three profitable trades is

$$
\begin{aligned}
& p(3)=\left(\begin{array}{l}
n \\
x
\end{array}\right) p^{x}(1-p)^{n-x}=\left(\begin{array}{l}
12 \\
3
\end{array}\right)\left(0.50^{3}\right)\left(0.50^{9}\right) \\
& =\frac{12 !}{(12-3) ! 3 !} 0.50^{12}=220(0.000244)=0.053711 \text {. }
\end{aligned}
$$

The probability of exactly 3 profitable trades out of 12 is $5.4 \%$ if broker BB001 were giving you fair prices. Now you need to calculate the other probabilities:

$p(2)=[12 ! /(12-2) ! 2 !]\left(0.50^{2}\right)\left(0.50^{10}\right)=66(0.000244)=0.016113$.

$p(1)=[12 ! /(12-1) ! 1 !]\left(0.50^{1}\right)\left(0.50^{11}\right)=12(0.000244)=0.00293$

$p(0)=[12 ! /(12-0) ! 0 !]\left(0.50^{0}\right)\left(0.50^{12}\right)=1(0.000244)=0.000244$.

Adding all the probabilities, $F(3)=0.053711+0.016113+0.00293+$ $0.000244=0.072998$, or $7.3 \%$. The probability of making 3 or fewer profitable trades out of 12 would be $7.3 \%$ if your trading desk were getting fair prices from broker BB001.

B. For broker BB002, you are assessing the probability that the underlying probability of a profitable trade with this broker was $50 \%$, despite the good results. The question was framed as the probability of making five or more profitable trades if the underlying probability is $50 \%: 1-$ $F(4)=p(5)+p(6)+p(7)+p(8)$. You could calculate $F(4)$ and subtract it from 1 , but you can also calculate $p(5)+p(6)+p(7)+p(8)$ directly.

You begin by calculating the probability that exactly five out of eight trades would be profitable if BB002 were giving you fair prices:

$$
\begin{aligned}
& p(5)=\left(\begin{array}{l}
8 \\
5
\end{array}\right)\left(0.50^{5}\right)\left(0.50^{3}\right) \\
& =56(0.003906)=0.21875 .
\end{aligned}
$$

The probability is about $21.9 \%$. The other probabilities are as follows: $p(6)=28(0.003906)=0.109375$.

$p(7)=8(0.003906)=0.03125$.

$p(8)=1(0.003906)=0.003906$.

So, $p(5)+p(6)+p(7)+p(8)=0.21875+0.109375+0.03125+$ $0.003906=0.363281$, or $36.3 \%$. A $36.3 \%$ probability is substantial; the underlying probability of executing a fair trade with BB002 might well have been 0.50 despite your success with BB002 in November of last year. If one of the trades with BB002 had been reclassified from profitable to unprofitable, exactly half the trades would have been profitable. In summary, your trading desk is getting at least fair prices from BB002; you will probably want to accumulate additional evidence before concluding that you are trading at better-than-fair prices.

The magnitude of the profits and losses in these trades is another important consideration. If all profitable trades had small profits but all unprofitable trades had large losses, for example, you might lose money on your trades even if the majority of them were profitable.

Two descriptors of a distribution that are often used in investments are the mean and the variance (or the standard deviation, the positive square root of variance). Exhibit 7 gives the expressions for the mean and variance of binomial random variables.

\section{Exhibit 7: Mean and Variance of Binomial Random}
Variables

\begin{center}
\begin{tabular}{lcl}
\hline
 & Mean & Variance \\
\hline
Bernoulli, $B(1, p)$ & $P$ & $p(1-p)$ \\
Binomial, $B(n, p)$ & $N p$ & $n p(1-p)$ \\
\hline
\end{tabular}
\end{center}

Because a single Bernoulli random variable, $Y \sim B(1, p)$, takes on the value 1 with probability $p$ and the value 0 with probability $1-p$, its mean or weighted-average outcome is $p$. Its variance is $p(1-p)$. A general binomial random variable, $B(n, p)$, is the sum of $n$ Bernoulli random variables, and so the mean of a $B(n, p)$ random variable is $n p$. Given that a $B(1, p)$ variable has variance $p(1-p)$, the variance of a $B(n, p)$ random variable is $n$ times that value, or $n p(1-p)$, assuming that all the trials (Bernoulli random variables) are independent. We can illustrate the calculation for two binomial random variables with differing probabilities as follows:

\begin{center}
\begin{tabular}{lcc}
\hline
Random Variable & Mean & Variance \\
\hline
$B(n=5, p=0.50)$ & $2.50=5(0.50)$ & $1.25=5(0.50)(0.50)$ \\
$B(n=5, p=0.10)$ & $0.50=5(0.10)$ & $0.45=5(0.10)(0.90)$ \\
\hline
\end{tabular}
\end{center}

For a $B(n=5, p=0.50)$ random variable, the expected number of successes is 2.5 , with a standard deviation of $1.118=(1.25)^{1 / 2}$; for a $B(n=5, p=0.10)$ random variable, the expected number of successes is 0.50 , with a standard deviation of $0.67=(0.45)^{1 / 2}$.

\section{EXAMPLE 4}
\section{The Expected Number of Defaults in a Bond Portfolio}
Suppose as a bond analyst you are asked to estimate the number of bond issues expected to default over the next year in an unmanaged high-yield bond portfolio with 25 US issues from distinct issuers. The credit ratings of the bonds in the portfolio are tightly clustered around Moody's B2/Standard \& Poor's B, meaning that the bonds are speculative with respect to the capacity to pay interest and repay principal. The estimated annual default rate for $\mathrm{B} 2 / \mathrm{B}$ rated bonds is $10.7 \%$.

\begin{enumerate}
  \item Over the next year, what is the expected number of defaults in the portfolio, assuming a binomial model for defaults?
\end{enumerate}

\section{Solution to 1:}
For each bond, we can define a Bernoulli random variable equal to 1 if the bond defaults during the year and zero otherwise. With 25 bonds, the expected number of defaults over the year is $n p=25(0.107)=2.675$, or approximately 3.

\begin{enumerate}
  \setcounter{enumi}{1}
  \item Estimate the standard deviation of the number of defaults over the coming year.
\end{enumerate}

\section{Solution to 2:}
The variance is $n p(1-p)=25(0.107)(0.893)=2.388775$. The standard deviation is $(2.388775)^{1 / 2}=1.55$. Thus, a two standard deviation confidence interval $( \pm 3.10)$ about the expected number of defaults $(\approx 3)$, for example, would run from approximately 0 to approximately 6 .

\begin{enumerate}
  \setcounter{enumi}{2}
  \item Critique the use of the binomial probability model in this context.
\end{enumerate}

\section{Solution to 3:}
An assumption of the binomial model is that the trials are independent. In this context, a trial relates to whether an individual bond issue will default over the next year. Because the issuing companies probably share exposure to common economic factors, the trials may not be independent. Nevertheless, for a quick estimate of the expected number of defaults, the binomial model may be adequate.

\section*{NORMAL DISTRIBUTION }
In this section, we focus on the two most important continuous distributions in investment work, the normal and lognormal.

\section{The Normal Distribution}
The normal distribution may be the most extensively used probability distribution in quantitative work. It plays key roles in modern portfolio theory and in several risk management technologies. Because it has so many uses, the normal distribution must be thoroughly understood by investment professionals.

The role of the normal distribution in statistical inference and regression analysis is vastly extended by a crucial result known as the central limit theorem. The central limit theorem states that the sum (and mean) of a large number of independent random variables (with finite variance) is approximately normally distributed.

The French mathematician Abraham de Moivre (1667-1754) introduced the normal distribution in 1733 in developing a version of the central limit theorem. As Exhibit 8 shows, the normal distribution is symmetrical and bell-shaped. The range of possible outcomes of the normal distribution is the entire real line: all real numbers lying between $-\infty$ and $+\infty$. The tails of the bell curve extend without limit to the left and to the right. Exhibit 8: PDFs of Two Different Normal Distributions

\begin{center}
\includegraphics[max width=\textwidth]{2023_05_04_cff39ee44f77d6514e1bg-265}
\end{center}

The defining characteristics of a normal distribution are as follows:

\begin{itemize}
  \item The normal distribution is completely described by two parameters-its mean, $\mu$, and variance, $\sigma^{2}$. We indicate this as $X \sim N\left(\mu, \sigma^{2}\right)$ (read " $X$ follows a normal distribution with mean $\mu$ and variance $\sigma^{2 ")}$. We can also define a normal distribution in terms of the mean and the standard deviation, $\sigma$ (this is often convenient because $\sigma$ is measured in the same units as $X$ and $\mu$ ). As a consequence, we can answer any probability question about a normal random variable if we know its mean and variance (or standard deviation).

  \item The normal distribution has a skewness of 0 (it is symmetric). The normal distribution has a kurtosis of 3; its excess kurtosis (kurtosis - 3.0) equals 0 . As a consequence of symmetry, the mean, the median, and the mode are all equal for a normal random variable.

  \item A linear combination of two or more normal random variables is also normally distributed.

\end{itemize}

The foregoing bullet points and descriptions concern a single variable or univariate normal distribution: the distribution of one normal random variable. A univariate distribution describes a single random variable. A multivariate distribution specifies the probabilities for a group of related random variables. You will encounter the multivariate normal distribution in investment work and readings and should know the following about it.

When we have a group of assets, we can model the distribution of returns on each asset individually or on the assets as a group. "As a group" implies that we take account of all the statistical interrelationships among the return series. One model that has often been used for security returns is the multivariate normal distribution. A multivariate normal distribution for the returns on $n$ stocks is completely defined by three lists of parameters:

\begin{itemize}
  \item the list of the mean returns on the individual securities ( $n$ means in total);

  \item the list of the securities' variances of return ( $n$ variances in total); and

  \item the list of all the distinct pairwise return correlations: $n(n-1) / 2$ distinct correlations in total. The need to specify correlations is a distinguishing feature of the multivariate normal distribution in contrast to the univariate normal distribution.

\end{itemize}

The statement "assume returns are normally distributed" is sometimes used to mean a joint normal distribution. For a portfolio of 30 securities, for example, portfolio return is a weighted average of the returns on the 30 securities. A weighted average is a linear combination. Thus, portfolio return is normally distributed if the individual security returns are (joint) normally distributed. To review, in order to specify the normal distribution for portfolio return, we need the means, the variances, and the distinct pairwise correlations of the component securities.

With these concepts in mind, we can return to the normal distribution for one random variable. The curves graphed in Exhibit 8 are the normal density function:

$$
f(x)=\frac{1}{\sigma \sqrt{2 \pi}} \exp \left(\frac{-(x-\mu)^{2}}{2 \sigma^{2}}\right) \text { for }-\infty<x<+\infty .
$$

The two densities graphed in Exhibit 8 correspond to a mean of $\mu=0$ and standard deviations of $\sigma=1$ and $\sigma=2$. The normal density with $\mu=0$ and $\sigma=1$ is called the standard normal distribution (or unit normal distribution). Plotting two normal distributions with the same mean and different standard deviations helps us appreciate why standard deviation is a good measure of dispersion for the normal distribution: Observations are much more concentrated around the mean for the normal distribution with $\sigma=1$ than for the normal distribution with $\sigma=2$.

Exhibit 9 illustrates the relationship between the pdf (density function) and cdf (distribution function) of the standard normal distribution ( mean $=0$, standard deviation $=1$ ). Most of the time, we associate a normal distribution with the "bell curve," which, in fact, is the probability density function of the normal distribution, depicted in Panel A. The cumulative distribution function, depicted in Panel B, in fact plots the size of the shaded areas of the pdfs. Let's take a look at the third row: In Panel A, we have shaded the bell curve up to $x=0$, the mean of the standard normal distribution. This shaded area corresponds to $50 \%$ in the cdf graph, as seen in Panel B, meaning that $50 \%$ of the observations of a normally distributed random variable would be equal or less than the mean.

\section{Exhibit 9: Density and Distribution Functions of the Standard Normal}
Distribution

A. PDFs
\includegraphics[max width=\textwidth, center]{2023_05_04_cff39ee44f77d6514e1bg-267}

B. CDFs
\includegraphics[max width=\textwidth, center]{2023_05_04_cff39ee44f77d6514e1bg-267(1)}

Although not literally accurate, the normal distribution can be considered an approximate model for asset returns. Nearly all the probability of a normal random variable is contained within three standard deviations of the mean. For realistic values of mean return and return standard deviation for many assets, the normal probability of outcomes below $-100 \%$ is very small.

Whether the approximation is useful in a given application is an empirical question. For example, Fama (1976) and Campbell, Lo, and MacKinlay (1997) showed that the normal distribution is a closer fit for quarterly and yearly holding period returns on a diversified equity portfolio than it is for daily or weekly returns. A persistent departure from normality in most equity return series is kurtosis greater than 3 , the fat-tails problem. So when we approximate equity return distributions with the normal distribution, we should be aware that the normal distribution tends to underestimate the probability of extreme returns.

Fat tails can be modeled, among other things, by a mixture of normal random variables or by a Student's $t$-distribution (which we shall cover shortly). In addition, since option returns are skewed, we should be cautious in using the symmetrical normal distribution to model the returns on portfolios containing significant positions in options. The normal distribution is also less suitable as a model for asset prices than as a model for returns. An asset price can drop only to 0 , at which point the asset becomes worthless. As a result, practitioners generally do not use the normal distribution to model the distribution of asset prices but work with the lognormal distribution, which we will discuss later.

\section{Probabilities Using the Normal Distribution}
Having established that the normal distribution is the appropriate model for a variable of interest, we can use it to make the following probability statements:

\begin{itemize}
  \item Approximately $50 \%$ of all observations fall in the interval $\mu \pm(2 / 3) \sigma$.

  \item Approximately $68 \%$ of all observations fall in the interval $\mu \pm \sigma$.

  \item Approximately $95 \%$ of all observations fall in the interval $\mu \pm 2 \sigma$.

  \item Approximately $99 \%$ of all observations fall in the interval $\mu \pm 3 \sigma$.

\end{itemize}

One, two, and three standard deviation intervals are illustrated in Exhibit 10. The intervals indicated are easy to remember but are only approximate for the stated probabilities. More precise intervals are $\mu \pm 1.96 \sigma$ for $95 \%$ of the observations and $\mu$ $\pm 2.58 \sigma$ for $99 \%$ of the observations.

\section{Exhibit 10: Units of Standard Deviation}
\begin{center}
\includegraphics[max width=\textwidth]{2023_05_04_cff39ee44f77d6514e1bg-268}
\end{center}

In general, we do not observe the mean or the standard deviation of the distribution of the whole population, so we need to estimate them from an observable sample. We estimate the population mean, $\mu$, using the sample mean, $\bar{X}$ (sometimes denoted as $\widehat{\mu})$, and estimate the population standard deviation, $\sigma$, using the sample standard deviation, $s$ (sometimes denoted as $\hat{\sigma}$ ).

\section{EXAMPLE 5}
\section{Calculating Probabilities from the Normal Distribution}
The chief investment officer of Fund XYZ would like to present some investment return scenarios to the Investment Committee, so she asks your assistance with some indicative numbers. Assuming daily asset returns are normally distributed, she would like to know the following:

\section{Note on Answering Questions 1-4:}
Normal distribution-related functions are part of spreadsheets, R, Python, and all statistical packages. Here, we use Microsoft Excel functions to answer these questions. When we speak in terms of "number of standard deviations above or below the mean," we are referring to the standard normal distribution (i.e., mean of 0 and standard deviation of 1), so it is best to use Excel's "=NORM.S.DIST(Z, 0 or 1)" function. "Z" represents the distance in number of standard deviations away from the mean, and the second parameter of the function is either 0 (Excel returns pdf value) or 1 (Excel returns cdf value).

\begin{enumerate}
  \item What is the probability that returns would be less than or equal to 1 standard deviation below the mean?
\end{enumerate}

\section{Solution to 1:}
To answer Question 1, we need the normal cdf value (so, set the second parameter equal to 1 ) that is associated with a $Z$ value of -1 (i.e., one standard deviation below the mean). Thus, "=NORM.S.DIST $(-1,1)$ " returns 0.1587 , or $15.9 \%$.

\begin{enumerate}
  \setcounter{enumi}{1}
  \item What is the probability that returns would be between +1 and -1 standard deviation around the mean?
\end{enumerate}

\section{Solution to 2:}
Here, we need to calculate the area under the normal pdf within the range of the mean \textbackslash pm 1 standard deviation. The area under the $\mathrm{pdf}$ is the cdf, so we must calculate the difference between the cdf one standard deviation above the mean and the cdf one standard deviation below the mean. Note that “=NORM.S.DIST $(1,1)$ " returns 0.8413 , or $84.1 \%$, which means that $84.1 \%$ of all observations of a normally distributed random variable would fall below the mean plus one standard deviation. We already calculated $15.9 \%$ for the probability that observations for such a variable would fall less than one standard deviation below the mean in the Solution to 1 , so the answer here is $84.1 \%-15.9 \%=68.3 \%$.

\begin{enumerate}
  \setcounter{enumi}{2}
  \item What is the probability that returns would be less than or equal to -2 standard deviations below the mean?
\end{enumerate}

\section{Solution to 3:}
Similar to Solution 1, use the Excel function “=NORM.S.DIST $(-2,1)$ " which returns a probability of 0.0228 , or $2.3 \%$.

\begin{enumerate}
  \setcounter{enumi}{3}
  \item How far (in terms of standard deviation) must returns fall below the mean for the probability to equal $95 \%$ ?
\end{enumerate}

\section{Solution to 4:}
This question is a typical way of phrasing "value at risk." In statistical terms, we want to know the lowest return value below which only $5 \%$ of the observations would fall. Thus, we need to find the $Z$ value for which the normal cdf would be $5 \%$ probability. To do this, we use the inverse of the cdf function -that is, "=NORM.S.INV(0.05)"-which results in -1.6449 , or -1.64 . In other words, only $5 \%$ of the observations should fall below the mean minus 1.64 standard deviations, or equivalently, $95 \%$ of the observations should exceed this threshold. There are as many different normal distributions as there are choices for mean $(\mu)$ and variance $\left(\sigma^{2}\right)$. We can answer all the previous questions in terms of any normal distribution. Spreadsheets, for example, have functions for the normal cdf for any specification of mean and variance. For the sake of efficiency, however, we would like to refer all probability statements to a single normal distribution. The standard normal distribution (the normal distribution with $\mu=0$ and $\sigma=1$ ) fills that role.

\section{Standardizing a Random Variable}
There are two steps in standardizing a normal random variable $X$ : Subtract the mean of $X$ from $X$ and then divide that result by the standard deviation of $X$ (this is also known as computing the $Z$-score). If we have a list of observations on a normal random variable, $X$, we subtract the mean from each observation to get a list of deviations from the mean and then divide each deviation by the standard deviation. The result is the standard normal random variable, $Z(Z$ is the conventional symbol for a standard normal random variable). If we have $X \sim N\left(\mu, \sigma^{2}\right)$ (read " $X$ follows the normal distribution with parameters $\mu$ and $\left.\sigma^{2 "}\right)$, we standardize it using the formula

$$
Z=(X-\mu) / \sigma
$$

Suppose we have a normal random variable, $X$, with $\mu=5$ and $\sigma=1.5$. We standardize $X$ with $Z=(X-5) / 1.5$. For example, a value $X=9.5$ corresponds to a standardized value of 3, calculated as $Z=(9.5-5) / 1.5=3$. The probability that we will observe a value as small as or smaller than 9.5 for $X \sim N(5,1.5)$ is exactly the same as the probability that we will observe a value as small as or smaller than 3 for $Z \sim N(0,1)$.

\section{Probabilities Using the Standard Normal Distribution}
We can answer all probability questions about $X$ using standardized values. We generally do not know the population mean and standard deviation, so we often use the sample mean $\bar{X}$ for $\mu$ and the sample standard deviation $s$ for $\sigma$. Standard normal probabilities are computed with spreadsheets, statistical and econometric software, and programming languages. Tables of the cumulative distribution function for the standard normal random variable are also readily available.

To find the probability that a standard normal variable is less than or equal to 0.24 , for example, calculate NORM.S.DIST $(0.24,1)=0.5948$; thus, $P(Z \leq 0.24)=0.5948$, or $59.48 \%$. If we want to find the probability of observing a value 1.65 standard deviations below the mean, calculate NORM.S.DIST $(-1.65,1)=0.04947$, or roughly $5 \%$.

The following are some of the most frequently referenced values when using the normal distribution, and for these values, =NORM.S.INV(Probability) is a convenient Excel function:

\begin{itemize}
  \item The 90th percentile point is 1.282, or NORM.S.INV(0.90)=1.28155. Thus, only $10 \%$ of values remain in the right tail beyond the mean plus 1.28 standard deviations;
\end{itemize}

\includegraphics[max width=\textwidth, center]{2023_05_04_cff39ee44f77d6514e1bg-270}
which means that $P(Z \leq 1.65)=N(1.65)=0.95$, or $95 \%$, and $5 \%$ of values remain in the right tail. The 5th percentile point, in contrast, is $\operatorname{NORM.S.INV}(0.05)=-1.64485$-that is, the same number as for $95 \%$, but with a negative sign.

\begin{itemize}
  \item Note the difference between the use of a percentile point when dealing with one tail rather than two tails. We used 1.65 because we are concerned with the $5 \%$ of values that lie only on one side, the right tail. If we want to cut off both the left and right $5 \%$ tails, then $90 \%$ of values would stay within the mean \textbackslash pm 1.65 standard deviations range. - The 99th percentile point is $2.327: P(Z \leq 2.327)=N(2.327)=0.99$, or $99 \%$, and $1 \%$ of values remain in the right tail.
\end{itemize}

\section{EXAMPLE 6}
\section{Probabilities for a Common Stock Portfolio}
Assume the portfolio mean return is $12 \%$ and the standard deviation of return estimate is $22 \%$ per year. Note also that if $X$ is portfolio return, the standardized portfolio return is $Z=(X-\bar{X}) / s=(X-12 \%) / 22 \%$. We use this expression throughout the solutions.

You want to calculate the following probabilities, assuming that a normal distribution describes returns.

\begin{enumerate}
  \item What is the probability that portfolio return will exceed $20 \%$ ?
\end{enumerate}

\section{Solution to 1:}
For $X=20 \%, Z=(20 \%-12 \%) / 22 \%=0.363636$. You want to find $P(Z>$ $0.363636)$. First, note that $P(Z>x)=P(Z \geq x)$ because the normal distribution is a continuous distribution. Also, recall that $P(Z \geq x)=1.0-P(Z \leq x)$ or $1-N(x)$. Next, NORM.S.DIST $(0.363636,1)=0.64194$, so, $1-0.6419=$ 0.3581 . Therefore, the probability that portfolio return will exceed $20 \%$ is about $36 \%$ if your normality assumption is accurate.

\begin{enumerate}
  \setcounter{enumi}{1}
  \item What is the probability that portfolio return will be between $12 \%$ and $20 \%$ ? In other words, what is $P(12 \% \leq$ portfolio return $\leq 20 \%)$ ?
\end{enumerate}

\section{Solution to 2:}
$P(12 \% \leq$ Portfolio return $\leq 20 \%)=N(Z$ corresponding to $20 \%)-N(Z$ corresponding to $12 \%)$. For the first term, $Z=(20 \%-12 \%) / 22 \%=0.363636$, and $N(0.363636)=0.6419$ (as in Solution 1). To get the second term immediately, note that $12 \%$ is the mean, and for the normal distribution, $50 \%$ of the probability lies on either side of the mean. Therefore, $N(Z$ corresponding to $12 \%)$ must equal $50 \%$. So $P(12 \% \leq$ Portfolio return $\leq 20 \%)=0.6419-0.50=$ 0.1419 , or approximately $14 \%$.

\begin{enumerate}
  \setcounter{enumi}{2}
  \item You can buy a one-year T-bill that yields $5.5 \%$. This yield is effectively a one-year risk-free interest rate. What is the probability that your portfolio's return will be equal to or less than the risk-free rate?
\end{enumerate}

\section{Solution to 3:}
If $X$ is portfolio return, then we want to find $P$ (Portfolio return $\leq 5.5 \%)$. For $X=5.5 \%, Z=(5.5 \%-12 \%) / 22 \%=-0.2955$. Using NORM.S.DIST $(-0.2955,1)=0.3838$, we see an approximately $38 \%$ chance the portfolio's return will be equal to or less than the risk-free rate.

Next, we will briefly discuss and illustrate the concept of the central limit theorem, according to which the sum (as well as the mean) of a set of independent, identically distributed random variables with finite variances is normally distributed, whatever distribution the random variables follow.

To illustrate this concept, consider a sample of 30 observations of a random variable that can take a value of just $-100,0$, or 100 , with equal probability. Clearly, this sample is drawn from a simple discrete uniform distribution, where the possible values of $-100,0$, and 100 each have $1 / 3$ probability. We randomly pick 10 elements of this sample and calculate the sum of these elements, and then we repeat this process a total of 100 times. The histogram in Exhibit 11 shows the distribution of these sums: The underlying distribution is a very simple discrete uniform distribution, but the sums converge toward a normal distribution.

Exhibit 11: Central Limit Theorem: Sums of Elements from Discrete Uniform Distribution Converge to Normal Distribution

\begin{center}
\includegraphics[max width=\textwidth]{2023_05_04_cff39ee44f77d6514e1bg-272}
\end{center}

\section{APPLICATIONS OF THE NORMAL DISTRIBUTION}
$$
\begin{aligned}
& \text { define shortfall risk, calculate the safety-first ratio, and identify an } \\
& \text { optimal portfolio using Roy's safety-first criterion }
\end{aligned}
$$

Modern portfolio theory (MPT) makes wide use of the idea that the value of investment opportunities can be meaningfully measured in terms of mean return and variance of return. In economic theory, mean-variance analysis holds exactly when investors are risk averse; when they choose investments so as to maximize expected utility, or satisfaction; and when either (1) returns are normally distributed or (2) investors have quadratic utility functions, a concept used in economics for a mathematical representation of attitudes toward risk and return. Mean-variance analysis, however, can still be useful-that is, it can hold approximately-when either Assumption 1 or 2 is violated. Because practitioners prefer to work with observables, such as returns, the proposition that returns are at least approximately normally distributed has played a key role in much of MPT.

To illustrate this concept, assume an investor is saving for retirement, and although her goal is to earn the highest real return possible, she believes that the portfolio should at least achieve real capital preservation over the long term. Assuming a long-term expected inflation rate of $2 \%$, the minimum acceptable return would be $2 \%$. Exhibit 12 compares three investment alternatives in terms of their expected returns and standard deviation of returns. The probability of falling below $2 \%$ is calculated on basis of the assumption of normally distributed returns. In the table, we see that Portfolio II, which combines the highest expected return and the lowest volatility, has the lowest probability of earning less than $2 \%$ (or equivalently, the highest probability of earning at least 2\%). This can also be seen in Panel B, where Portfolio II has the smallest shaded area to the left of $2 \%$ (the probability of earning less than the minimum acceptable return).

\section{Exhibit 12: Probability of Earning a Minimum Acceptable Return}
\begin{center}
\begin{tabular}{lccc}
\hline
Portfolio & I & II & II \\
\hline
Expected return & $5 \%$ & $8 \%$ & $5 \%$ \\
Standard deviation of return & $8 \%$ & $8 \%$ & $12 \%$ \\
Probability of earning $<2 \%[P(x<2)]$ & $37.7 \%$ & $24.6 \%$ & $41.7 \%$ \\
Probability of earning $\geq 2 \%[P(x \geq 2)]$ & $62.3 \%$ & $75.4 \%$ & $58.3 \%$ \\
\hline
\end{tabular}
\end{center}

\section{A. Portfolio I}
\begin{center}
\includegraphics[max width=\textwidth]{2023_05_04_cff39ee44f77d6514e1bg-273}
\end{center}

\section{B. Portfolio II}
\begin{center}
\includegraphics[max width=\textwidth]{2023_05_04_cff39ee44f77d6514e1bg-273(2)}
\end{center}

C. Portfolio III

\begin{center}
\includegraphics[max width=\textwidth]{2023_05_04_cff39ee44f77d6514e1bg-273(1)}
\end{center}

Mean-variance analysis generally considers risk symmetrically in the sense that standard deviation captures variability both above and below the mean. An alternative approach evaluates only downside risk. We discuss one such approach, safety-first rules, because they provide an excellent illustration of the application of normal distribution theory to practical investment problems. Safety-first rules focus on shortfall risk, the risk that portfolio value (or portfolio return) will fall below some minimum acceptable level over some time horizon. The risk that the assets in a defined benefit plan will fall below plan liabilities is an example of a shortfall risk.

Suppose an investor views any return below a level of $R_{L}$ as unacceptable. Roy's safety-first criterion (Roy 1952) states that the optimal portfolio minimizes the probability that portfolio return, $R_{P}$, will fall below the threshold level, $R_{L}$. In symbols, the investor's objective is to choose a portfolio that minimizes $P\left(R_{P}<R_{L}\right)$. When portfolio returns are normally distributed, we can calculate $P\left(R_{P}<R_{L}\right)$ using the number of standard deviations that $R_{L}$ lies below the expected portfolio return, $E\left(R_{P}\right)$. The portfolio for which $E\left(R_{P}\right)-R_{L}$ is largest relative to standard deviation minimizes $P\left(R_{P}<\right.$ $R_{L}$ ). Therefore, if returns are normally distributed, the safety-first optimal portfolio maximizes the safety-first ratio (SFRatio):

$$
\text { SFRatio }=\left[\mathrm{E}\left(R_{P}\right)-R_{L}\right] / \sigma_{P} .
$$

The quantity $\mathrm{E}\left(R_{P}\right)-R_{L}$ is the distance from the mean return to the shortfall level. Dividing this distance by $\sigma_{P}$ gives the distance in units of standard deviation. There are two steps in choosing among portfolios using Roy's criterion (assuming normality):

\begin{enumerate}
  \item Calculate each portfolio's SFRatio.

  \item Choose the portfolio with the highest SFRatio.

\end{enumerate}

For a portfolio with a given safety-first ratio, the probability that its return will be less than $R_{L}$ is $N$ (-SFRatio), and the safety-first optimal portfolio has the lowest such probability. For example, suppose an investor's threshold return, $R_{L}$, is $2 \%$. He is presented with two portfolios. Portfolio 1 has an expected return of $12 \%$, with a standard deviation of $15 \%$. Portfolio 2 has an expected return of $14 \%$, with a standard deviation of 16\%. The SFRatios, using Equation 5, are 0.667 $=(12-2) / 15$ and $0.75=(14-2) / 16$ for Portfolios 1 and 2, respectively. For the superior Portfolio 2, the probability that portfolio return will be less than $2 \%$ is $N(-0.75)=1-N(0.75)=1-0.7734=0.227$, or about $23 \%$, assuming that portfolio returns are normally distributed.

You may have noticed the similarity of the SFRatio to the Sharpe ratio. If we substitute the risk-free rate, $R_{F}$, for the critical level $R_{L}$, the SFRatio becomes the Sharpe ratio. The safety-first approach provides a new perspective on the Sharpe ratio: When we evaluate portfolios using the Sharpe ratio, the portfolio with the highest Sharpe ratio is the one that minimizes the probability that portfolio return will be less than the risk-free rate (given a normality assumption).

\section{EXAMPLE 7}
\section{The Safety-First Optimal Portfolio for a Client}
You are researching asset allocations for a client in Canada with a $C \$ 800,000$ portfolio. Although her investment objective is long-term growth, at the end of a year she may want to liquidate $\mathrm{C} \$ 30,000$ of the portfolio to fund educational expenses. If that need arises, she would like to be able to take out the $C \$ 30,000$ without invading the initial capital of $\mathrm{C} \$ 800,000$. The table below shows three alternative allocations.

Mean and Standard Deviation for Three Allocations (in

Percent)

\begin{center}
\begin{tabular}{lccc}
\hline
Allocation & A & B & C \\
\hline
Expected annual return & 25 & 11 & 14 \\
Standard deviation of return & 27 & 8 & 20 \\
\hline
\end{tabular}
\end{center}

Address these questions (assume normality for Questions 2 and 3):

\begin{enumerate}
  \item Given the client's desire not to invade the $C \$ 800,000$ principal, what is the shortfall level, $R_{L}$ ? Use this shortfall level to answer Question 2.
\end{enumerate}

\section{Solution to 1:}
Because $C \$ 30,000 / C \$ 800,000$ is $3.75 \%$, for any return less than $3.75 \%$ the client will need to invade principal if she takes out $C \$ 30,000$. So, $R_{L}=3.75 \%$.

\begin{enumerate}
  \setcounter{enumi}{1}
  \item According to the safety-first criterion, which of the three allocations is the best?
\end{enumerate}

\section{Solution to 2:}
To decide which of the three allocations is safety-first optimal, select the alternative with the highest ratio $\left[\mathrm{E}\left(R_{P}\right)-R_{L}\right] / \sigma_{P}$ :

Allocation A: $0.787037=(25-3.75) / 27$

Allocation B: $0.90625=(11-3.75) / 8$

Allocation $\mathrm{C}: 0.5125=(14-3.75) / 20$

Allocation B, with the largest ratio (0.90625), is the best alternative according to the safety-first criterion.

\begin{enumerate}
  \setcounter{enumi}{2}
  \item What is the probability that the return on the safety-first optimal portfolio will be less than the shortfall level?
\end{enumerate}

\section{Solution to 3:}
To answer this question, note that $P\left(R_{B}<3.75\right)=N(-0.90625)$. We can round 0.90625 to 0.91 for use with tables of the standard normal cdf. First, we calculate $N(-0.91)=1-N(0.91)=1-0.8186=0.1814$, or about $18.1 \%$. Using a spreadsheet function for the standard normal cdf on -0.90625 without rounding, we get 0.182402 , or about $18.2 \%$. The safety-first optimal portfolio has a roughly $18 \%$ chance of not meeting a $3.75 \%$ return threshold. This can be seen in the following graphic, where Allocation B has the smallest area under the distribution curve to the left of $3.75 \%$.

\begin{center}
\includegraphics[max width=\textwidth]{2023_05_04_cff39ee44f77d6514e1bg-275}
\end{center}

Several points are worth noting. First, if the inputs were slightly different, we could get a different ranking. For example, if the mean return on B were $10 \%$ rather than $11 \%$, Allocation A would be superior to B. Second, if meeting the $3.75 \%$ return threshold were a necessity rather than a wish, $C \$ 830,000$ in one year could be modeled as a liability. Fixed-income strategies, such as cash flow matching, could be used to offset or immunize the $\mathrm{C} \$ 830,000$ quasi-liability.

In many investment contexts besides Roy's safety-first criterion, we use the normal distribution to estimate a probability. Another arena in which the normal distribution plays an important role is financial risk management. Financial institutions, such as investment banks, security dealers, and commercial banks, have formal systems to measure and control financial risk at various levels, from trading positions to the overall risk for the firm. Two mainstays in managing financial risk are value at risk (VaR) and stress testing/scenario analysis. Stress testing and scenario analysis refer to a set of techniques for estimating losses in extremely unfavorable combinations of events or scenarios. Value at risk $(\mathrm{VaR})$ is a money measure of the minimum value of losses expected over a specified time period (for example, a day, a quarter, or a year) at a given level of probability (often 0.05 or 0.01 ). Suppose we specify a one-day time horizon and a level of probability of 0.05 , which would be called a $95 \%$ one-day VaR. If this VaR equaled $€ 5$ million for a portfolio, there would be a 0.05 probability that the portfolio would lose $€ 5$ million or more in a single day (assuming our assumptions were correct). One of the basic approaches to estimating VaR, the variance-covariance or analytical method, assumes that returns follow a normal distribution.

\section*{LOGNORMAL DISTRIBUTION AND CONTINUOUS COMPOUNDING }
\begin{abstract}
explain the relationship between normal and lognormal distributions and why the lognormal distribution is used to model asset prices calculate and interpret a continuously compounded rate of return, given a specific holding period return
\end{abstract}

\section{The Lognormal Distribution}
Closely related to the normal distribution, the lognormal distribution is widely used for modeling the probability distribution of share and other asset prices. For example, the lognormal distribution appears in the Black-Scholes-Merton option pricing model. The Black-Scholes-Merton model assumes that the price of the asset underlying the option is lognormally distributed.

A random variable $Y$ follows a lognormal distribution if its natural logarithm, $\ln$ $Y$, is normally distributed. The reverse is also true: If the natural logarithm of random variable $Y$, $\ln Y$, is normally distributed, then $Y$ follows a lognormal distribution. If you think of the term lognormal as "the log is normal," you will have no trouble remembering this relationship.

The two most noteworthy observations about the lognormal distribution are that it is bounded below by 0 and it is skewed to the right (it has a long right tail). Note these two properties in the graphs of the pdfs of two lognormal distributions in Exhibit 13. Asset prices are bounded from below by 0 . In practice, the lognormal distribution has been found to be a usefully accurate description of the distribution of prices for many financial assets. However, the normal distribution is often a good approximation for returns. For this reason, both distributions are very important for finance professionals.

\section{Exhibit 13: Two Lognormal Distributions}
\begin{center}
\includegraphics[max width=\textwidth]{2023_05_04_cff39ee44f77d6514e1bg-277}
\end{center}

Like the normal distribution, the lognormal distribution is completely described by two parameters. Unlike the other distributions we have considered, a lognormal distribution is defined in terms of the parameters of a different distribution. The two parameters of a lognormal distribution are the mean and standard deviation (or variance) of its associated normal distribution: the mean and variance of $\ln Y$, given that $Y$ is lognormal. Remember, we must keep track of two sets of means and standard deviations (or variances): the mean and standard deviation (or variance) of the associated normal distribution (these are the parameters) and the mean and standard deviation (or variance) of the lognormal variable itself.

To illustrate this relationship, we simulated 1,000 scenarios of yearly asset returns, assuming that returns are normally distributed with $7 \%$ mean and $12 \%$ standard deviation. For each scenario $i$, we converted the simulated continuously compounded returns $\left(r_{i}\right)$ to future asset prices with the formula Price $(1 \text { year later })_{i}=\$ 1 \mathrm{x} \exp \left(r_{i}\right)$, where exp is the exponential function and assuming that the asset's price is $\$ 1$ today. In Exhibit 14, Panel A shows the distribution of the simulated returns together with the fitted normal pdf, whereas Panel B shows the distribution of the corresponding future asset prices together with the fitted lognormal pdf. Again, note that the lognormal distribution of future asset prices is bounded below by 0 and has a long right tail. Exhibit 14: Simulated Returns (Normal PDF) and Asset Prices (Lognormal PDF)

\section{A. Normal PDF}
\begin{center}
\includegraphics[max width=\textwidth]{2023_05_04_cff39ee44f77d6514e1bg-278}
\end{center}

\section{B. Lognormal PDF}
\begin{center}
\includegraphics[max width=\textwidth]{2023_05_04_cff39ee44f77d6514e1bg-278(1)}
\end{center}

The expressions for the mean and variance of the lognormal variable itself are challenging. Suppose a normal random variable $X$ has expected value $\mu$ and variance $\sigma^{2}$. Define $Y=\exp (X)$. Remember that the operation indicated by $\exp (X)$ or $e^{X}$ (where $e$ $\approx 2.7183)$ is the opposite operation from taking logs. Because $\ln Y=\ln [\exp (X)]=X$ is normal (we assume $X$ is normal), $Y$ is lognormal. What is the expected value of $Y$ $=\exp (X)$ ? A guess might be that the expected value of $Y$ is $\exp (\mu)$. The expected value is actually $\exp \left(\mu+0.50 \sigma^{2}\right)$, which is larger than $\exp (\mu)$ by a factor of $\exp \left(0.50 \sigma^{2}\right)>1$. To get some insight into this concept, think of what happens if we increase $\sigma^{2}$. The distribution spreads out; it can spread upward, but it cannot spread downward past 0 . As a result, the center of its distribution is pushed to the right: The distribution's mean increases.

The expressions for the mean and variance of a lognormal variable are summarized below, where $\mu$ and $\sigma^{2}$ are the mean and variance of the associated normal distribution (refer to these expressions as needed, rather than memorizing them):

\begin{itemize}
  \item $\operatorname{Mean}\left(\mu_{L}\right)$ of a lognormal random variable $=\exp \left(\mu+0.50 \sigma^{2}\right)$. - Variance $\left(\sigma_{L}^{2}\right)$ of a lognormal random variable $=\exp \left(2 \mu+\sigma^{2}\right) \times\left[\exp \left(\sigma^{2}\right)\right.$ $-1]$.
\end{itemize}

\section{Continuously Compounded Rates of Return}
We now explore the relationship between the distribution of stock return and stock price. In this section, we show that if a stock's continuously compounded return is normally distributed, then future stock price is necessarily lognormally distributed. Furthermore, we show that stock price may be well described by the lognormal distribution even when continuously compounded returns do not follow a normal distribution. These results provide the theoretical foundation for using the lognormal distribution to model asset prices.

To outline the presentation that follows, we first show that the stock price at some future time $T, S_{T}$, equals the current stock price, $S_{0}$, multiplied by $e$ raised to power $r_{0, T}$, the continuously compounded return from 0 to $T$; this relationship is expressed as $S_{T}=S_{0} \exp \left(r_{0, T}\right)$. We then show that we can write $r_{0, T}$ as the sum of shorter-term continuously compounded returns and that if these shorter-period returns are normally distributed, then $r_{0, T}$ is normally distributed (given certain assumptions) or approximately normally distributed (not making those assumptions). As $S_{T}$ is proportional to the $\log$ of a normal random variable, $S_{T}$ is lognormal.

To supply a framework for our discussion, suppose we have a series of equally spaced observations on stock price: $S_{0}, S_{1}, S_{2}, \ldots, S_{T}$. Current stock price, $S_{0}$, is a known quantity and thus is nonrandom. The future prices (such as $S_{1}$ ), however, are random variables. The price relative, $S_{1} / S_{0}$, is an ending price, $S_{1}$, over a beginning price, $S_{0}$; it is equal to 1 plus the holding period return on the stock from $t=0$ to $t=1$ :

$$
S_{1} / S_{0}=1+R_{0,1} \text {. }
$$

For example, if $S_{0}=\$ 30$ and $S_{1}=\$ 34.50$, then $S_{1} / S_{0}=\$ 34.50 / \$ 30=1.15$. Therefore, $R_{0,1}=0.15$, or $15 \%$. In general, price relatives have the form

$S_{t+1} / S_{t}=1+R_{t, t+1}$,

where $R_{t, t+1}$ is the rate of return from $t$ to $t+1$.

An important concept is the continuously compounded return associated with a holding period return, such as $R_{0,1}$. The continuously compounded return associated with a holding period return is the natural logarithm of 1 plus that holding period return, or equivalently, the natural logarithm of the ending price over the beginning price (the price relative). Note that here we are using lowercase $r$ to refer specifically to continuously compounded returns. For example, if we observe a one-week holding period return of 0.04 , the equivalent continuously compounded return, called the one-week continuously compounded return, is $\ln (1.04)=0.039221 ; € 1.00$ invested for one week at 0.039221 continuously compounded gives $€ 1.04$, equivalent to a $4 \%$ one-week holding period return. The continuously compounded return from $t$ to $t+1$ is

$$
r_{t, t+1}=\ln \left(S_{t+1} / S_{t}\right)=\ln \left(1+R_{t, t+1}\right) .
$$

For our example, $r_{0,1}=\ln \left(S_{1} / S_{0}\right)=\ln \left(1+R_{0,1}\right)=\ln (\$ 34.50 / \$ 30)=\ln (1.15)=$ 0.139762. Thus, $13.98 \%$ is the continuously compounded return from $t=0$ to $t=1$. The continuously compounded return is smaller than the associated holding period return. If our investment horizon extends from $t=0$ to $t=T$, then the continuously compounded return to $T$ is

$$
r_{0, T}=\ln \left(S_{T} / S_{0}\right) .
$$

Applying the function exp to both sides of the equation, we have $\exp \left(r_{0, T}\right)=$ $\exp \left[\ln \left(S_{T} / S_{0}\right)\right]=S_{T} / S_{0}$, so

$S_{T}=S_{0} \exp \left(r_{0, T}\right)$. We can also express $S_{T} / S_{0}$ as the product of price relatives:

$S_{T} / S_{0}=\left(S_{T} / S_{T-1}\right)\left(S_{T-1} / S_{T-2}\right) \ldots\left(S_{1} / S_{0}\right)$.

Taking logs of both sides of this equation, we find that the continuously compounded return to time $T$ is the sum of the one-period continuously compounded returns:

$$
r_{0, T}=r_{T-1, T}+r_{T-2, T-1}+\ldots+r_{0,1} .
$$

Using holding period returns to find the ending value of a $\$ 1$ investment involves the multiplication of quantities ( 1 + holding period return). Using continuously compounded returns involves addition (as shown in Equation 7).

A key assumption in many investment applications is that returns are independently and identically distributed (i.i.d.). Independence captures the proposition that investors cannot predict future returns using past returns. Identical distribution captures the assumption of stationarity, a property implying that the mean and variance of return do not change from period to period.

Assume that the one-period continuously compounded returns (such as $r_{0,1}$ ) are i.i.d. random variables with mean $\mu$ and variance $\sigma^{2}$ (but making no normality or other distributional assumption). Then,

$$
E\left(r_{0, T}\right)=E\left(r_{T-1, T}\right)+E\left(r_{T-2, T-1}\right)+\ldots+E\left(r_{0,1}\right)=\mu T
$$

(we add up $\mu$ for a total of $T$ times), and

$\sigma^{2}\left(r_{0, T}\right)=\sigma^{2} T$

(as a consequence of the independence assumption). The variance of the $T$ holding period continuously compounded return is $T$ multiplied by the variance of the one-period continuously compounded return; also, $\sigma\left(r_{0, T}\right)=\sigma \sqrt{T}$. If the one-period continuously compounded returns on the right-hand side of Equation 7 are normally distributed, then the $T$ holding period continuously compounded return, $r_{0, T}$, is also normally distributed with mean $\mu T$ and variance $\sigma^{2} T$. This relationship is so because a linear combination of normal random variables is also normal. But even if the one-period continuously compounded returns are not normal, their sum, $r_{0, T}$, is approximately normal according to the central limit theorem. Now compare $S_{T}=$ $S_{0} \exp \left(r_{0, T}\right)$ to $Y=\exp (X)$, where $X$ is normal and $Y$ is lognormal (as we discussed previously). Clearly, we can model future stock price $S_{T}$ as a lognormal random variable because $r_{0, T}$ should be at least approximately normal. This assumption of normally distributed returns is the basis in theory for the lognormal distribution as a model for the distribution of prices of shares and other assets.

Continuously compounded returns play a role in many asset pricing models, as well as in risk management. Volatility measures the standard deviation of the continuously compounded returns on the underlying asset; by convention, it is stated as an annualized measure. In practice, we very often estimate volatility using a historical series of continuously compounded daily returns. We gather a set of daily holding period returns and then use Equation 6 to convert them into continuously compounded daily returns. We then compute the standard deviation of the continuously compounded daily returns and annualize that number using Equation 9 .

To compute the standard deviation of a set (or sample) of $n$ returns, we sum the squared deviation of each return from the mean return and then divide that sum by $n-1$. The result is the sample variance. Taking the square root of the sample variance gives the sample standard deviation. Annualizing is typically done on the basis of 250 days in a year, the approximate number of days markets are open for trading. Thus if daily volatility were 0.01 , we would state volatility (on an annual basis) as $0.01 \sqrt{250}=0.1581$. Example 8 illustrates the estimation of volatility for the shares of Astra International.

\section{EXAMPLE 8}
\section{Volatility of Share Price}
Suppose you are researching Astra International (Indonesia Stock Exchange: ASII) and are interested in Astra's price action in a week in which international economic news had significantly affected the Indonesian stock market. You decide to use volatility as a measure of the variability of Astra shares during that week. The following shows closing prices during that week.

\section{Astra International Daily Closing Prices}
\begin{center}
\begin{tabular}{ll}
\hline
Day & Closing Price (IDR) \\
\hline
Monday & 6,950 \\
Tuesday & 7,000 \\
Wednesday & 6,850 \\
Thursday & 6,600 \\
Friday & 6,350 \\
\hline
\end{tabular}
\end{center}

Use the data provided to do the following:

\begin{enumerate}
  \item Estimate the volatility of Astra shares. (Annualize volatility on the basis of 250 days in a year.)
\end{enumerate}

\section{Solution to 1:}
First, use Equation 6 to calculate the continuously compounded daily returns; then, find their standard deviation in the usual way. In calculating sample variance, to get sample standard deviation, the divisor is sample size minus 1.

$\ln (7,000 / 6,950)=0.007168$

$\ln (6,850 / 7,000)=-0.021661$

$\ln (6,600 / 6,850)=-0.037179$

$\ln (6,350 / 6,600)=-0.038615$

$\operatorname{Sum}=-0.090287$

Mean $=-0.022572$

Variance $=0.000452$

Standard deviation $=0.021261$

The standard deviation of continuously compounded daily returns is 0.021261. Equation 9 states that $\hat{\sigma}\left(r_{0, T}\right)=\hat{\sigma} \sqrt{T}$. In this example, $\hat{\sigma}$ is the sample standard deviation of one-period continuously compounded returns. Thus, $\hat{\sigma}$ refers to 0.021261 . We want to annualize, so the horizon $T$ corresponds to one year. Because $\hat{\sigma}$ is in days, we set $T$ equal to the number of trading days in a year (250).

We find that annualized volatility for Astra stock that week was $33.6 \%$, calculated as $0.021261 \sqrt{250}=0.336165$.

Note that the sample mean, -0.022572 , is a possible estimate of the mean, $\mu$, of the continuously compounded one-period or daily returns. The sample mean can be translated into an estimate of the expected continuously compounded annual return using Equation 8: $\hat{\mu} T=-0.022572$ (250) (using 250 to be consistent with the calculation of volatility). But four observations are far too few to estimate expected returns. The variability in the daily returns overwhelms any information about expected return in a series this short.

\begin{enumerate}
  \setcounter{enumi}{1}
  \item Identify the probability distribution for Astra share prices if continuously compounded daily returns follow the normal distribution.
\end{enumerate}

\section{Solution to 2:}
Astra share prices should follow the lognormal distribution if the continuously compounded daily returns on Astra shares follow the normal distribution.

We have shown that the distribution of stock price is lognormal, given certain assumptions. What are the mean and variance of $S_{T}$ if $S_{T}$ follows the lognormal distribution? Earlier we gave bullet-point expressions for the mean and variance of a lognormal random variable. In the bullet-point expressions, the $\hat{\mu}$ and $\widehat{\sigma}^{2}$ would refer, in the context of this discussion, to the mean and variance of the $T$ horizon (not the one-period) continuously compounded returns (assumed to follow a normal distribution), compatible with the horizon of $S_{T}$. Related to the use of mean and variance (or standard deviation), previously we used those quantities to construct intervals in which we expect to find a certain percentage of the observations of a normally distributed random variable. Those intervals were symmetric about the mean. Can we state similar symmetric intervals for a lognormal random variable? Unfortunately, we cannot; because the lognormal distribution is not symmetric, such intervals are more complicated than for the normal distribution, and we will not discuss this specialist topic here.

\section{STUDENT'S T-, CHI-SQUARE, AND F-DISTRIBUTIONS}
describe the properties of the Student's $t$-distribution, and calculate and interpret its degrees of freedom

describe the properties of the chi-square distribution and the $F$-distribution, and calculate and interpret their degrees of freedom

\section{Student's $t$-Distribution}
To complete the review of probability distributions commonly used in finance, we discuss Student's $t$-, chi-square, and $F$-distributions. Most of the time, these distributions are used to support statistical analyses, such as sampling, testing the statistical significance of estimated model parameters, or hypothesis testing. In addition, Student's $t$-distribution is also sometimes used to model asset returns in a manner similar to that of the normal distribution. However, since the $t$-distribution has "longer tails," it may provide a more reliable, more conservative downside risk estimate.

The standard $t$-distribution is a symmetrical probability distribution defined by a single parameter known as degrees of freedom (df), the number of independent variables used in defining sample statistics, such as variance, and the probability distributions they measure. Each value for the number of degrees of freedom defines one distribution in this family of distributions. We will shortly compare $t$-distributions with the standard normal distribution, but first we need to understand the concept of degrees of freedom. We can do so by examining the calculation of the sample variance,

$$
s^{2}=\frac{\sum_{i=1}^{n}\left(X_{i}-\bar{X}\right)^{2}}{n-1} .
$$

Equation 10 gives the unbiased estimator of the sample variance that we use. The term in the denominator, $n-1$, which is the sample size minus 1 , is the number of degrees of freedom in estimating the population variance when using Equation 10. We also use $n-1$ as the number of degrees of freedom for determining reliability factors based on the $t$-distribution. The term "degrees of freedom" is used because in a random sample, we assume that observations are selected independently of each other. The numerator of the sample variance, however, uses the sample mean. How does the use of the sample mean affect the number of observations collected independently for the sample variance formula? With a sample size of 10 and a mean of $10 \%$, for example, we can freely select only 9 observations. Regardless of the 9 observations selected, we can always find the value for the 10th observation that gives a mean equal to $10 \%$. From the standpoint of the sample variance formula, then, there are nine degrees of freedom. Given that we must first compute the sample mean from the total of $n$ independent observations, only $n-1$ observations can be chosen independently for the calculation of the sample variance. The concept of degrees of freedom comes up frequently in statistics, and you will see it often later in the CFA Program curriculum.

Suppose we sample from a normal distribution. The ratio $z=(\bar{X}-\mu) /(\sigma / \sqrt{n}$ ) is distributed normally with a mean of 0 and standard deviation of 1 ; however, the ratio $t=(\bar{X}-\mu) /(s / \sqrt{n})$ follows the $t$-distribution with a mean of 0 and $n-1$ degrees of freedom. The ratio represented by $t$ is not normal because $t$ is the ratio of two random variables, the sample mean and the sample standard deviation. The definition of the standard normal random variable involves only one random variable, the sample mean. As degrees of freedom increase (i.e., as sample size increases), however, the $t$-distribution approaches the standard normal distribution. Exhibit 15 shows the probability density functions for the standard normal distribution and two $t$-distributions, one with $\mathrm{df}=2$ and one with $\mathrm{df}=8$.

\section{Exhibit 15: Student's $t$-Distributions vs. Standard Normal Distribution}
\begin{center}
\includegraphics[max width=\textwidth]{2023_05_04_cff39ee44f77d6514e1bg-283}
\end{center}

Of the three distributions shown in Exhibit 15, the standard normal distribution has tails that approach zero faster than the tails of the two $t$-distributions. The $t$-distribution is also symmetrically distributed around its mean value of zero, just like the normal distribution. As the degrees of freedom increase, the $t$-distribution approaches the standard normal distribution. The $t$-distribution with $\mathrm{df}=8$ is closer to the standard normal distribution than the $t$-distribution with $\mathrm{df}=2$.

Beyond plus and minus four standard deviations from the mean, the area under the standard normal distribution appears to approach 0; both $t$-distributions, however, continue to show some area under each curve beyond four standard deviations. The $t$-distributions have fatter tails, but the tails of the $t$-distribution with $\mathrm{df}=8$ more closely resemble the normal distribution's tails. As the degrees of freedom increase, the tails of the $t$-distribution become less fat.

Probabilities for the $t$-distribution can be readily computed with spreadsheets, statistical software, and programming languages. As an example of the latter, see the final sidebar at the end of this section for sample code in the $\mathrm{R}$ programming language.

\section{Chi-Square and F-Distribution}
The chi-square distribution, unlike the normal and $t$-distributions, is asymmetrical. Like the $t$-distribution, the chi-square distribution is a family of distributions. The chi-square distribution with $k$ degrees of freedom is the distribution of the sum of the squares of $k$ independent standard normally distributed random variables; hence, this distribution does not take on negative values. A different distribution exists for each possible value of degrees of freedom, $n-1$ ( $n$ is sample size).

Like the chi-square distribution, the $F$-distribution is a family of asymmetrical distributions bounded from below by 0 . Each $F$-distribution is defined by two values of degrees of freedom, called the numerator and denominator degrees of freedom.

The relationship between the chi-square and $F$-distributions is as follows: If $X_{1}^{2}$ is one chi-square random variable with $m$ degrees of freedom and $\chi_{2}^{2}$ is another chi-square random variable with $n$ degrees of freedom, then $F=\left(x_{1}^{2} / m\right) /\left(x_{2}^{2} / n\right)$ follows an $F$-distribution with $m$ numerator and $n$ denominator degrees of freedom.

Chi-square and $F$-distributions are asymmetric, and as shown in Exhibit 16, the domain of their pdfs are positive numbers. Like Student's $t$-distribution, as the degrees of freedom of the chi-square distribution increase, the shape of its pdf becomes more similar to a bell curve (see Panel A). For the $F$-distribution, as both the numerator $\left(\mathrm{df}_{1}\right)$ and the denominator $\left(\mathrm{df}_{2}\right)$ degrees of freedom increase, the density function will also become more bell curve-like (see Panel B).

\section{Exhibit 16: PDFs of Chi-Square and F-Distributions}
\section{A. Chi-Square Distributions}
\begin{center}
\includegraphics[max width=\textwidth]{2023_05_04_cff39ee44f77d6514e1bg-285(1)}
\end{center}

B. F-Distributions

\begin{center}
\includegraphics[max width=\textwidth]{2023_05_04_cff39ee44f77d6514e1bg-285}
\end{center}

As for typical investment applications, Student's $t$, chi-square, and $F$-distributions are the basis for test statistics used in performing various types of hypothesis tests on portfolio returns, such as those summarized in Exhibit 17. Exhibit 17: Student's $t$, Chi-Square, and F-Distributions: Basis for Hypothesis Tests of Investment Returns

\begin{center}
\begin{tabular}{lcl}
\hline
Distribution & Test Statistic & \multicolumn{1}{c}{Hypothesis Tests of Returns} \\
\hline
Student's $t$ & $t$-Statistic & $\begin{array}{l}\text { Tests of a single population mean, of differences } \\ \text { between two population means, of mean difference } \\ \text { between paired (dependent) populations, and of pop- } \\ \text { ulation correlation coefficient }\end{array}$ \\
Chi-square & $\begin{array}{c}\text { Chi-square } \\ \text { statistic } \\ \text { F-statistic }\end{array}$ & $\begin{array}{l}\text { Test of variance of a normally distributed population } \\ \text { Test of equality of variances of two normally dis- } \\ \text { tributed populations from two independent random } \\ \text { samples }\end{array}$ \\
\hline
\end{tabular}
\end{center}

\section{EXAMPLE 9}
Probabilities Using Student's-t, Chi-Square, and F-Distributions

\begin{enumerate}
  \item Of the distributions we have covered in this reading, which can take values that are only positive numbers (i.e., no negative values)?
\end{enumerate}

\section{Solution to 1:}
Of the probability distributions covered in this reading, the domains of the pdfs of the lognormal, the chi-square, and the $F$-distribution are only positive numbers.

\begin{enumerate}
  \setcounter{enumi}{1}
  \item Interpret the degrees of freedom for a chi-square distribution, and describe how a larger value of df affects the shape of the chi-square pdf.
\end{enumerate}

\section{Solution to 2:}
A chi-square distribution with $k$ degrees of freedom is the distribution of the sum of the squares of $k$ independent standard normally distributed random variables. The greater the degrees of freedom, the more symmetrical and bell curve-like the pdf becomes.

\begin{enumerate}
  \setcounter{enumi}{2}
  \item Generate cdf tables in Excel for values 1,2, and 3 for the following distributions: standard normal, Student's $t-(\mathrm{df}=5)$, chi-square $(\mathrm{df}=5)$, and $F$-distribution $\left(\mathrm{df}_{1}=5, \mathrm{df}_{2}=1\right)$. Then, calculate the distance from the mean for probability $(p)=90 \%, 95 \%$, and $99 \%$ for each distribution.
\end{enumerate}

\section{Solution to 3:}
In Excel, we can calculate cdfs using the NORM.S.DIST(value,1), T.DIST(value,DF,1), CHISQ.DIST(value,DF,1), and F.DIST(value,DF1,DF2,1) functions for the standard normal, Student's $t$-, chi-square, and F-distributions, respectively. At the end of this question set, we also show code snippets in the $\mathrm{R}$ language for generating cdfs for the requested values. For values 1, 2, and 3, the following are the results using the Excel functions:

\section{CDF Values Using Different Probability Distributions}
\begin{center}
\begin{tabular}{|c|c|c|c|c|}
\hline
Value & Normal & $\begin{array}{l}\text { Student's } t \\ (\text { df = 5) }\end{array}$ & $\begin{array}{c}\text { Chi-Square } \\ (d f=5)\end{array}$ & $\begin{array}{c}F \\ \left(d f_{1}=5\right. \\ \left.d f_{2}=1\right)\end{array}$ \\
\hline
1 & $84.1 \%$ & $81.8 \%$ & $3.7 \%$ & $36.3 \%$ \\
\hline
2 & $97.7 \%$ & $94.9 \%$ & $15.1 \%$ & $51.1 \%$ \\
\hline
3 & $999 \%$ & $98.5 \%$ & $30.0 \%$ & $58.9 \%$ \\
\hline
\end{tabular}
\end{center}

To calculate distances from the mean given probability $p$, we must use the inverse of the distribution functions: NORM.S.INV(p), T.INV(p,DF), CHISQ.INV(p,DF), and F.INV(p,DF1,DF2), respectively. At the end of this question set, we also show code snippets in the $\mathrm{R}$ language for calculating distances from the mean for the requested probabilities. The results using the inverse functions and the requested probabilities are as follows:

\section{Distance from the Mean for a Given Probability (p)}
\begin{center}
\begin{tabular}{|c|c|c|c|c|}
\hline
Probability & Normal & $\begin{array}{l}\text { Student's } t \\ (\text { df }=5)\end{array}$ & $\begin{array}{l}\text { Chi-Square } \\ \qquad(d f=5)\end{array}$ & $\begin{array}{c}F \\ \left(d f_{1}=5\right. \\ \left.d f_{2}=1\right)\end{array}$ \\
\hline
$90 \%$ & 1.28 & 1.48 & 9.24 & 57.24 \\
\hline
$95 \%$ & 1.64 & 2.02 & 11.07 & 230.16 \\
\hline
$99 \%$ & 2.33 & 3.36 & 15.09 & $5,763.65$ \\
\hline
\end{tabular}
\end{center}

\begin{enumerate}
  \setcounter{enumi}{3}
  \item You fit a Student's $t$-distribution to historically observed returns of stock market index ABC. Your best fit comes with five degrees of freedom. Compare this Student's $t$-distribution $(\mathrm{df}=5)$ to a standard normal distribution on the basis of your answer to Question 3 .
\end{enumerate}

\section{Solution to 4:}
Student's $t$-distribution with df of 5 has longer tails than the standard normal distribution. For probabilities $90 \%, 95 \%$, and $99 \%$, such $t$-distributed random variables would fall farther away from their mean $(1.48,2.02$, and 3.36 standard deviations, respectively) than a normally distributed random variable $(1.28,1.64$, and 2.33 standard deviations, respectively).

\section{R CODE FOR PROBABILITIES INVOLVING STUDENT'S T-, CHI-SQUARE, AND F-DISTRIBUTIONS}
For those of you with a knowledge of (or interest in learning) readily accessible computer code to find probabilities involving Student's $t$-, chi-square, and $F$-distributions, you can try out the following program. Specifically, this program uses code in the $\mathrm{R}$ language to solve for the answers to Example 9, which you have just completed. Good luck and have fun!

\begin{center}
\includegraphics[max width=\textwidth]{2023_05_04_cff39ee44f77d6514e1bg-288}
\end{center}

\begin{center}
\includegraphics[max width=\textwidth]{2023_05_04_cff39ee44f77d6514e1bg-289(1)}
\end{center}

\section{MONTE CARLO SIMULATION}
\begin{center}
\includegraphics[max width=\textwidth]{2023_05_04_cff39ee44f77d6514e1bg-289}
\end{center}

After gaining an understanding of probability distributions, we now learn about a technique in which probability distributions play an integral role. The technique is called Monte Carlo simulation, and in finance it involves the use of computer software to represent the operation of a complex financial system. A characteristic feature of Monte Carlo simulation is the generation of a large number of random samples from a specified probability distribution or distributions to represent the role of risk in the system.

Monte Carlo simulation is widely used to estimate risk and return in investment applications. In this setting, we simulate the portfolio's profit and loss performance for a specified time horizon. Repeated trials within the simulation (each trial involving a draw of random observations from a probability distribution) produce a simulated frequency distribution of portfolio returns from which performance and risk measures are derived. Another important use of Monte Carlo simulation in investments is as a tool for valuing complex securities for which no analytic pricing formula is available. For other securities, such as mortgage-backed securities with complex embedded options, Monte Carlo simulation is also an important modeling resource. Since we control the assumptions when we carry out a simulation, we can run a model for valuing such securities through a Monte Carlo simulation to examine the model's sensitivity to a change in key assumptions.

To understand the technique of Monte Carlo simulation, we present the process as a series of steps; these can be viewed as providing an overview rather than a detailed recipe for implementing a Monte Carlo simulation in its many varied applications. To illustrate the steps, we use Monte Carlo simulation to value a contingent claim security (a security whose value is based on some other underlying security) for which no analytic pricing formula is available. For our purposes, such a contingent claim security has a value at its maturity equal to the difference between the underlying stock price at that maturity and the average stock price during the life of the contingent claim or $\$ 0$, whichever is greater. For instance, if the final underlying stock price is $\$ 34$ and the average value over the life of the claim is $\$ 31$, the value of the contingent claim at its maturity is $\$ 3$ (the greater of $\$ 34-\$ 31=\$ 3$ and $\$ 0$ ).

Assume that the maturity of the claim is one year from today; we will simulate stock prices in monthly steps over the next 12 months and will generate 1,000 scenarios to evaluate this claim. The payoff diagram of this contingent claim security is depicted in Panel A of Exhibit 18, a histogram of simulated average and final stock prices is shown in Panel B, and a histogram of simulated payoffs of the contingent claim is presented in Panel C.

The payoff diagram (Panel A) is a snapshot of the contingent claim at maturity. If the stock's final price is less than or equal to its average over the life of the contingent claim, then the payoff would be zero. However, if the final price exceeds the average price, the payoff is equal to this difference. Panel B shows histograms of the simulated final and average stock prices. Note that the simulated final price distribution is wider than the simulated average price distribution. Also, note that the contingent claim's value depends on the difference between the final and average stock prices, which cannot be directly inferred from these histograms. Exhibit 18: Payoff Diagram, Histogram of Simulated Average and Final Stock Prices, and Histogram of Simulated Payoffs for Contingent Claim

\section{A. Contingent Claim Payoff Diagram}
\begin{center}
\includegraphics[max width=\textwidth]{2023_05_04_cff39ee44f77d6514e1bg-291}
\end{center}

\section{B. Histogram of Simulated Average and Final Stock Prices}
 Number of Trials\begin{center}
\includegraphics[max width=\textwidth]{2023_05_04_cff39ee44f77d6514e1bg-291(1)}
\end{center}

\section{Histogram of Simulated Contingent Claim Payoffs}
\begin{center}
\includegraphics[max width=\textwidth]{2023_05_04_cff39ee44f77d6514e1bg-291(2)}
\end{center}

Finally, Panel C shows the histogram of the contingent claim's simulated payoffs. In 654 of 1,000 total trials, the final stock price was less than or equal to the average price, so in $65.4 \%$ of the trials the contingent claim paid off zero. In the remaining $34.6 \%$ of the trials, however, the claim paid the positive difference between the final and average prices, with the maximum payoff being $\$ 11$. The process flowchart in Exhibit 19 shows the steps for implementing the Monte Carlo simulation for valuing this contingent claim. Steps 1 through 3 of the process describe specifying the simulation; Steps 4 through 7 describe running the simulation.

\section{Exhibit 19: Steps in Implementing the Monte Carlo Simulation}
\begin{center}
\includegraphics[max width=\textwidth]{2023_05_04_cff39ee44f77d6514e1bg-292}
\end{center}

The mechanics of implementing the Monte Carlo simulation for valuing the contingent claim using the seven-step process are described as follows:

\begin{enumerate}
  \item Specify the quantity of interest in terms of underlying variables. Here the quantity of interest is the contingent claim value, and the underlying variable is the stock price. Then, specify the starting value(s) of the underlying variable(s).
\end{enumerate}

We use $C_{i T}$ to represent the value of the claim at maturity, $T$. The subscript $i$ in $\mathrm{C}_{i T}$ indicates that $\mathrm{C}_{i T}$ is a value resulting from the $i$ th simulation trial, each simulation trial involving a drawing of random values (an iteration of Step 4).

\begin{enumerate}
  \setcounter{enumi}{1}
  \item Specify a time grid. Take the horizon in terms of calendar time and split it into a number of subperiods-say, $K$ in total. Calendar time divided by the number of subperiods, $K$, is the time increment, $\Delta t$. In our example, calendar time is one year and $K$ is 12 , so $\Delta t$ equals one month.

  \item Specify distributional assumptions for the key risk factors that drive the underlying variables. For example, stock price is the underlying variable for the contingent claim, so we need a model for stock price movement. We choose the following model for changes in stock price, where $Z_{k}$ stands for the standard normal random variable: $\Delta$ Stock price $=(\mu \times$ Prior stock price $\times \Delta t)+\left(\sigma \times\right.$ Prior stock price $\left.\times Z_{k}\right)$.

\end{enumerate}

The term $Z_{k}$ is the key risk factor in the simulation. Through our choice of $\mu$ (mean) and $\sigma$ (standard deviation), we control the distribution of the stock price variable. Although this example has one key risk factor, a given simulation may have multiple key risk factors.

\begin{enumerate}
  \setcounter{enumi}{3}
  \item Using a computer program or spreadsheet function, draw $K$ random values of each risk factor. In our example, the spreadsheet function would produce a draw of $K(=12)$ values of the standard normal variable $Z_{k}: Z_{1}, Z_{2}, Z_{3}, \ldots$ ., $Z_{K}$. We will discuss generating standard normal random numbers (or, in fact, random numbers with any kind of distribution) after describing the sequence of simulation steps.

  \item Convert the standard normal random numbers generated in Step 4 into stock price changes ( $\Delta$ Stock price) by using the model of stock price dynamics from Step 3. The result is $K$ observations on possible changes in stock price over the $K$ subperiods (remember, $K=12$ ). An additional calculation is needed to convert those changes into a sequence of $K$ stock prices, with the initial stock price as the starting value over the $K$ subperiods. Another calculation produces the average stock price during the life of the contingent claim (the sum of $K$ stock prices divided by $K$ ).

  \item Compute the value of the contingent claim at maturity, $\mathrm{C}_{i T}$, and then calculate its present value, $C_{i 0}$, by discounting this terminal value using an appropriate interest rate as of today. (The subscript $i$ in $C_{i 0}$ stands for the $i$ th simulation trial, as it does in $\mathrm{C}_{i T}$.) We have now completed one simulation trial.

  \item Iteratively go back to Step 4 until the specified number of trials, $I$, is completed. Finally, produce summary values and statistics for the simulation. The quantity of interest in our example is the mean value of $\mathrm{C}_{i 0}$ for the total number of simulation trials $(I=1,000)$. This mean value is the Monte Carlo estimate of the value of our contingent claim.

\end{enumerate}

In Step 4 of our example, a computer function produced a set of random observations on a standard normal random variable. Recall that for a uniform distribution, all possible numbers are equally likely. The term random number generator refers to an algorithm that produces uniformly distributed random numbers between 0 and 1. In the context of computer simulations, the term random number refers to an observation drawn from a uniform distribution. For other distributions, the term "random observation" is used in this context.

It is a remarkable fact that random observations from any distribution can be produced using the uniform distribution with endpoints 0 and 1 . The technique for producing random observations is known as the inverse transformation method. As a generalist, you do not need to address the technical details of converting random numbers into random observations, but you do need to know that random observations from any distribution can be generated using a uniform random variable.

Exhibit 20 provides a visual representation of the workings of the inverse transformation method. In essence, the randomly generated uniform number (0.30) lying on the continuous uniform probability density function, pdf, bounded by 0 and 1 (Panel A), is mapped onto the inverted cumulative density function, cdf, bounded by 0 and 1 , of any distribution from which random observations are desired; here, we use the standard normal distribution (Panel B). The point on the given distribution's cdf is then mapped onto its pdf (Panel $C$ ), and the random observation is thereby identified $(-0.5244)$. Note that in actuality, the random observation can be read off the $y$-axis in Panel B, but we include Panel C here to reinforce the intuition on how inverse transformation works. This method for generating random observations for the standard normal distribution is the same one used in the Monte Carlo simulation-based valuation of the contingent claim just described.

Exhibit 20: Inverse Transformation: Random Number from Uniform

Distribution (PDF) Mapped to CDF and PDF of Standard Normal

Distribution to Produce Random Observation

A. Uniform Random Number Is Generated

Probability Density

\begin{center}
\includegraphics[max width=\textwidth]{2023_05_04_cff39ee44f77d6514e1bg-294}
\end{center}

B. Random Number Mapped to Inverted Standard Normal CDF

\begin{center}
\includegraphics[max width=\textwidth]{2023_05_04_cff39ee44f77d6514e1bg-294(2)}
\end{center}

\section{Point on Inverted CDF Mapped to Standard Normal PDF}
Probability

\begin{center}
\includegraphics[max width=\textwidth]{2023_05_04_cff39ee44f77d6514e1bg-294(1)}
\end{center}

In Example 10, we continue with the application of Monte Carlo simulation to value another type of contingent claim.

\section{EXAMPLE 10}
\section{Valuing a Lookback Contingent Claim Using Monte Carlo}
 Simulation\begin{enumerate}
  \item A standard lookback contingent claim on stock has a value at maturity equal to (Value of the stock at maturity - Minimum value of stock during the life of the claim prior to maturity) or $\$ 0$, whichever is greater. If the minimum value reached prior to maturity was $\$ 20.11$ and the value of the stock at maturity is $\$ 23$, for example, the contingent claim is worth $\$ 23-\$ 20.11=$ $\$ 2.89$.
\end{enumerate}

Briefly discuss how you might use Monte Carlo simulation in valuing a lookback contingent claim.

\section{Solution:}
We previously described how to use Monte Carlo simulation to value a certain type of contingent claim. Just as we can calculate the average value of the stock over a simulation trial to value that claim, for a lookback contingent claim, we can also calculate the minimum value of the stock over a simulation trial. Then, for a given simulation trial, we can calculate the terminal value of the claim, given the minimum value of the stock for the simulation trial. We can then discount this terminal value back to the present to get the value of the claim today $(t=0)$. The average of these $t=0$ values over all simulation trials is the Monte Carlo simulated value of the lookback contingent claim.

Finally, it is important to note that Monte Carlo simulation is a complement to analytical methods. It provides only statistical estimates, not exact results. Analytical methods, where available, provide more insight into cause-and-effect relationships. However, as financial product innovations proceed, the applications for Monte Carlo simulation in investment management continue to grow.

\section{SUMMARY}
In this reading, we have presented the most frequently used probability distributions in investment analysis and Monte Carlo simulation.

\begin{itemize}
  \item A probability distribution specifies the probabilities of the possible outcomes of a random variable.

  \item The two basic types of random variables are discrete random variables and continuous random variables. Discrete random variables take on at most a countable number of possible outcomes that we can list as $x_{1}, x_{2}, \ldots$ In contrast, we cannot describe the possible outcomes of a continuous random variable $Z$ with a list $z_{1}, z_{2}, \ldots$, because the outcome $\left(z_{1}+z_{2}\right) / 2$, not in the list, would always be possible. - The probability function specifies the probability that the random variable will take on a specific value. The probability function is denoted $p(x)$ for a discrete random variable and $f(x)$ for a continuous random variable. For any probability function $p(x), 0 \leq p(x) \leq 1$, and the sum of $p(x)$ over all values of $X$ equals 1 .

  \item The cumulative distribution function, denoted $F(x)$ for both continuous and discrete random variables, gives the probability that the random variable is less than or equal to $x$.

  \item The discrete uniform and the continuous uniform distributions are the distributions of equally likely outcomes.

  \item The binomial random variable is defined as the number of successes in $n$ Bernoulli trials, where the probability of success, $p$, is constant for all trials and the trials are independent. A Bernoulli trial is an experiment with two outcomes, which can represent success or failure, an up move or a down move, or another binary (twofold) outcome.

  \item A binomial random variable has an expected value or mean equal to $n p$ and variance equal to $n p(1-p)$.

  \item A binomial tree is the graphical representation of a model of asset price dynamics in which, at each period, the asset moves up with probability $p$ or down with probability $(1-p)$. The binomial tree is a flexible method for modeling asset price movement and is widely used in pricing options.

  \item The normal distribution is a continuous symmetric probability distribution that is completely described by two parameters: its mean, $\mu$, and its variance, $\sigma^{2}$.

  \item A univariate distribution specifies the probabilities for a single random variable. A multivariate distribution specifies the probabilities for a group of related random variables.

  \item To specify the normal distribution for a portfolio when its component securities are normally distributed, we need the means, the standard deviations, and all the distinct pairwise correlations of the securities. When we have those statistics, we have also specified a multivariate normal distribution for the securities.

  \item For a normal random variable, approximately $68 \%$ of all possible outcomes are within a one standard deviation interval about the mean, approximately $95 \%$ are within a two standard deviation interval about the mean, and approximately $99 \%$ are within a three standard deviation interval about the mean.

  \item A normal random variable, $X$, is standardized using the expression $Z=(X$ $-\mu) / \sigma$, where $\mu$ and $\sigma$ are the mean and standard deviation of $X$. Generally, we use the sample mean, $\bar{X}$, as an estimate of $\mu$ and the sample standard deviation, $s$, as an estimate of $\sigma$ in this expression.

  \item The standard normal random variable, denoted $Z$, has a mean equal to 0 and variance equal to 1 . All questions about any normal random variable can be answered by referring to the cumulative distribution function of a standard normal random variable, denoted $N(x)$ or $N(z)$.

  \item Shortfall risk is the risk that portfolio value or portfolio return will fall below some minimum acceptable level over some time horizon.

  \item Roy's safety-first criterion, addressing shortfall risk, asserts that the optimal portfolio is the one that minimizes the probability that portfolio return falls below a threshold level. According to Roy's safety-first criterion, if returns are normally distributed, the safety-first optimal portfolio $P$ is the one that maximizes the quantity $\left[\mathrm{E}\left(R_{P}\right)-R_{L}\right] / \sigma_{P}$, where $R_{L}$ is the minimum acceptable level of return.

  \item A random variable follows a lognormal distribution if the natural logarithm of the random variable is normally distributed. The lognormal distribution is defined in terms of the mean and variance of its associated normal distribution. The lognormal distribution is bounded below by 0 and skewed to the right (it has a long right tail).

  \item The lognormal distribution is frequently used to model the probability distribution of asset prices because it is bounded below by zero.

  \item Continuous compounding views time as essentially continuous or unbroken; discrete compounding views time as advancing in discrete finite intervals.

  \item The continuously compounded return associated with a holding period is the natural $\log$ of 1 plus the holding period return, or equivalently, the natural log of ending price over beginning price.

  \item If continuously compounded returns are normally distributed, asset prices are lognormally distributed. This relationship is used to move back and forth between the distributions for return and price. Because of the central limit theorem, continuously compounded returns need not be normally distributed for asset prices to be reasonably well described by a lognormal distribution.

  \item Student's $t$-, chi-square, and $F$-distributions are used to support statistical analyses, such as sampling, testing the statistical significance of estimated model parameters, or hypothesis testing.

  \item The standard $t$-distribution is a symmetrical probability distribution defined by degrees of freedom (df) and characterized by fat tails. As df increase, the $t$-distribution approaches the standard normal distribution.

  \item The chi-square distribution is asymmetrical, defined by degrees of freedom, and with $k \mathrm{df}$ is the distribution of the sum of the squares of $k$ independent standard normally distributed random variables, so it does not take on negative values. A different distribution exists for each value of $\mathrm{df}, n-1$.

  \item The $F$-distribution is a family of asymmetrical distributions bounded from below by 0 . Each $F$-distribution is defined by two values of degrees of freedom, the numerator $\mathrm{df}$ and the denominator $\mathrm{df}$. If $X_{1}^{2}$ is one chi-square random variable with $m \mathrm{df}$ and $\chi_{2}^{2}$ is another chi-square random variable with $n$ df, then $F=\left(\chi_{1}^{2} / m\right) /\left(\chi_{2}^{2} / n\right)$ follows an $F$-distribution with $m$ numerator df and $n$ denominator $\mathrm{df}$

  \item Monte Carlo simulation involves the use of a computer to represent the operation of a complex financial system. A characteristic feature of Monte Carlo simulation is the generation of a large number of random samples from specified probability distributions to represent the operation of risk in the system. Monte Carlo simulation is used in planning, in financial risk management, and in valuing complex securities. Monte Carlo simulation is a complement to analytical methods but provides only statistical estimates, not exact results.

  \item Random observations from any distribution can be produced using the uniform random variable with endpoints 0 and 1 via the inverse transformation method. The randomly generated uniform random number is mapped onto the inverted cdf of any distribution from which random observations are desired. The point on the given distribution's cdf is then mapped onto its pdf, and the random observation is identified.

\end{itemize}

\section{PRACTICE PROBLEMS}
\begin{enumerate}
  \item A European put option on stock conveys the right to sell the stock at a prespecified price, called the exercise price, at the maturity date of the option. The value of this put at maturity is (exercise price - stock price) or $\$ 0$, whichever is greater. Suppose the exercise price is $\$ 100$ and the underlying stock trades in increments of $\$ 0.01$. At any time before maturity, the terminal value of the put is a random variable.
\end{enumerate}

A. Describe the distinct possible outcomes for terminal put value. (Think of the put's maximum and minimum values and its minimum price increments.)

B. Is terminal put value, at a time before maturity, a discrete or continuous random variable?

C. Letting $Y$ stand for terminal put value, express in standard notation the probability that terminal put value is less than or equal to $\$ 24$. No calculations or formulas are necessary.

\begin{enumerate}
  \setcounter{enumi}{1}
  \item Which of the following is a continuous random variable?
\end{enumerate}

A. The value of a futures contract quoted in increments of $\$ 0.05$

B. The total number of heads recorded in 1 million tosses of a coin

C. The rate of return on a diversified portfolio of stocks over a three-month period

\begin{enumerate}
  \setcounter{enumi}{2}
  \item $X$ is a discrete random variable with possible outcomes $X=\{1,2,3,4\}$. Three functions- $f(x), g(x)$, and $h(x)$-are proposed to describe the probabilities of the outcomes in $X$.
\end{enumerate}

\begin{center}
\begin{tabular}{lccc}
\hline
 & \multicolumn{3}{c}{Probability Function} \\
\hline
$\boldsymbol{X}=\boldsymbol{x}$ & $\boldsymbol{f}(\boldsymbol{x})=\boldsymbol{P}(\boldsymbol{X}=\boldsymbol{x})$ & $\boldsymbol{g}(\boldsymbol{x})=\boldsymbol{P}(\boldsymbol{X}=\boldsymbol{x})$ & $\boldsymbol{h}(\boldsymbol{x})=\boldsymbol{P}(\boldsymbol{X}=\boldsymbol{x})$ \\
\hline
1 & -0.25 & 0.20 & 0.20 \\
2 & 0.25 & 0.25 & 0.25 \\
3 & 0.50 & 0.50 & 0.30 \\
4 & 0.25 & 0.05 & 0.35 \\
\hline
\end{tabular}
\end{center}

The conditions for a probability function are satisfied by:
A. $f(x)$.
B. $g(x)$.
C. $h(x)$.

\begin{enumerate}
  \setcounter{enumi}{3}
  \item The value of the cumulative distribution function $F(x)$, where $x$ is a particular outcome, for a discrete uniform distribution:
A. sums to 1 .
B. lies between 0 and 1 .
C. decreases as $x$ increases. 5. In a discrete uniform distribution with 20 potential outcomes of integers $1-20$, the probability that $X$ is greater than or equal to 3 but less than $6, P(3 \leq X<6)$, is:
A. 0.10 .
B. 0.15
C. 0.20 .

  \item You are forecasting sales for a company in the fourth quarter of its fiscal year. Your low-end estimate of sales is $€ 14$ million, and your high-end estimate is $€ 15$ million. You decide to treat all outcomes for sales between these two values as equally likely, using a continuous uniform distribution.

\end{enumerate}

A. What is the expected value of sales for the fourth quarter?

B. What is the probability that fourth-quarter sales will be less than or equal to $€ 14,125,000 ?$

\begin{enumerate}
  \setcounter{enumi}{6}
  \item The cumulative distribution function for a discrete random variable is shown in the following table.
\end{enumerate}

\begin{center}
\begin{tabular}{lc}
\hline
$\boldsymbol{X}=\boldsymbol{x}$ & $\begin{array}{c}\text { Cumulative Distribution Function } \\ \boldsymbol{F}(\boldsymbol{x})=\boldsymbol{P}(\boldsymbol{X} \leq \boldsymbol{x})\end{array}$ \\
\hline
1 & 0.15 \\
2 & 0.25 \\
3 & 0.50 \\
4 & 0.60 \\
5 & 0.95 \\
6 & 1.00 \\
\hline
\end{tabular}
\end{center}

The probability that $X$ will take on a value of either 2 or 4 is closest to:
A. 0.20
B. 0.35 .
C. 0.85

\begin{enumerate}
  \setcounter{enumi}{7}
  \item A random number between zero and one is generated according to a continuous uniform distribution. What is the probability that the first number generated will have a value of exactly 0.30 ?
A. $0 \%$
B. $30 \%$
C. $70 \%$

  \item Define the term "binomial random variable." Describe the types of problems for which the binomial distribution is used.

  \item For a binomial random variable with five trials and a probability of success on each trial of 0.50 , the distribution will be:
A. skewed.
B. uniform. C. symmetric.

  \item Over the last 10 years, a company's annual earnings increased year over year seven times and decreased year over year three times. You decide to model the number of earnings increases for the next decade as a binomial random variable. For Parts B, C, and D of this problem, assume the estimated probability is the actual probability for the next decade.

\end{enumerate}

A. What is your estimate of the probability of success, defined as an increase in annual earnings?

B. What is the probability that earnings will increase in exactly 5 of the next 10 years?

C. Calculate the expected number of yearly earnings increases during the next 10 years.

D. Calculate the variance and standard deviation of the number of yearly earnings increases during the next 10 years.

E. The expression for the probability function of a binomial random variable depends on two major assumptions. In the context of this problem, what must you assume about annual earnings increases to apply the binomial distribution in Part B? What reservations might you have about the validity of these assumptions?

\begin{enumerate}
  \setcounter{enumi}{11}
  \item A portfolio manager annually outperforms her benchmark $60 \%$ of the time. Assuming independent annual trials, what is the probability that she will outperform her benchmark four or more times over the next five years?
A. 0.26
B. 0.34
C. 0.48

  \item You are examining the record of an investment newsletter writer who claims a $70 \%$ success rate in making investment recommendations that are profitable over a one-year time horizon. You have the one-year record of the newsletter's seven most recent recommendations. Four of those recommendations were profitable. If all the recommendations are independent and the newsletter writer's skill is as claimed, what is the probability of observing four or fewer profitable recommendations out of seven in total?

  \item If the probability that a portfolio outperforms its benchmark in any quarter is 0.75 , the probability that the portfolio outperforms its benchmark in three or fewer quarters over the course of a year is closest to:
A. 0.26
B. 0.42
C. 0.68

  \item Which of the following events can be represented as a Bernoulli trial?
A. The flip of a coin
B. The closing price of a stock C. The picking of a random integer between 1 and 10

  \item A stock is priced at $\$ 100.00$ and follows a one-period binomial process with an up move that equals 1.05 and a down move that equals 0.97 . If 1 million Bernoulli trials are conducted and the average terminal stock price is $\$ 102.00$, the probability of an up move $(p)$ is closest to:
A. 0.375 .
B. 0.500 .
C. 0.625

  \item A call option on a stock index is valued using a three-step binomial tree with an up move that equals 1.05 and a down move that equals 0.95 . The current level of the index is $\$ 190$, and the option exercise price is $\$ 200$. If the option value is positive when the stock price exceeds the exercise price at expiration and $\$ 0$ otherwise, the number of terminal nodes with a positive payoff is:
A. one.
B. two.
C. three.

  \item State the approximate probability that a normal random variable will fall within the following intervals:

\end{enumerate}

A. Mean plus or minus one standard deviation.

B. Mean plus or minus two standard deviations.

C. Mean plus or minus three standard deviations.

\begin{enumerate}
  \setcounter{enumi}{18}
  \item In futures markets, profits or losses on contracts are settled at the end of each trading day. This procedure is called marking to market or daily resettlement. By preventing a trader's losses from accumulating over many days, marking to market reduces the risk that traders will default on their obligations. A futures markets trader needs a liquidity pool to meet the daily mark to market. If liquidity is exhausted, the trader may be forced to unwind his position at an unfavorable time.
\end{enumerate}

Suppose you are using financial futures contracts to hedge a risk in your portfolio. You have a liquidity pool (cash and cash equivalents) of $\lambda$ dollars per contract and a time horizon of $T$ trading days. For a given size liquidity pool, $\lambda$, Kolb, Gay, and Hunter developed an expression for the probability stating that you will exhaust your liquidity pool within a $T$-day horizon as a result of the daily marking to market. Kolb et al. assumed that the expected change in futures price is 0 and that futures price changes are normally distributed. With $\sigma$ representing the standard deviation of daily futures price changes, the standard deviation of price changes over a time horizon to day $T$ is $\sigma \sqrt{T}$, given continuous compounding. With that background, the Kolb et al. expression is

Probability of exhausting liquidity pool $=2[1-N(x)]$,

where $x=\lambda /(\sigma \sqrt{T})$. Here, $x$ is a standardized value of $\lambda . N(x)$ is the standard normal cumulative distribution function. For some intuition about $1-N(x)$ in the expression, note that the liquidity pool is exhausted if losses exceed the size of the liquidity pool at any time up to and including $T$; the probability of that event happening can be shown to be proportional to an area in the right tail of a standard normal distribution, $1-N(x)$.

Using the Kolb et al. expression, answer the following questions:

A. Your hedging horizon is five days, and your liquidity pool is $\$ 2,000$ per contract. You estimate that the standard deviation of daily price changes for the contract is $\$ 450$. What is the probability that you will exhaust your liquidity pool in the five-day period?

B. Suppose your hedging horizon is 20 days but all the other facts given in Part A remain the same. What is the probability that you will exhaust your liquidity pool in the 20-day period?

\begin{enumerate}
  \setcounter{enumi}{19}
  \item Which of the following is characteristic of the normal distribution?
A. Asymmetry
B. Kurtosis of 3
C. Definitive limits or boundaries

  \item Which of the following assets most likely requires the use of a multivariate distribution for modeling returns?

\end{enumerate}

A. A call option on a bond

B. A portfolio of technology stocks

C. A stock in a market index

\begin{enumerate}
  \setcounter{enumi}{21}
  \item The total number of parameters that fully characterizes a multivariate normal distribution for the returns on two stocks is:
A. 3 .
B. 4 .
C. 5 .

  \item A portfolio has an expected mean return of $8 \%$ and standard deviation of $14 \%$. The probability that its return falls between $8 \%$ and $11 \%$ is closest to:
A. $8.5 \%$
B. $14.8 \%$.
C. $58.3 \%$.

  \item A portfolio has an expected return of $7 \%$, with a standard deviation of $13 \%$. For an investor with a minimum annual return target of $4 \%$, the probability that the portfolio return will fail to meet the target is closest to:
A. $33 \%$.
B. $41 \%$.
C. $59 \%$.

  \item Which parameter equals zero in a normal distribution?
A. Kurtosis
B. Skewness
C. Standard deviation

  \item An analyst develops the following capital market projections.

\end{enumerate}

\begin{center}
\begin{tabular}{lcc}
\hline
 & Stocks & Bonds \\
\hline
Mean Return & $10 \%$ & $2 \%$ \\
Standard Deviation & $15 \%$ & $5 \%$ \\
\hline
\end{tabular}
\end{center}

Assuming the returns of the asset classes are described by normal distributions, which of the following statements is correct?

A. Bonds have a higher probability of a negative return than stocks.

B. On average, $99 \%$ of stock returns will fall within two standard deviations of the mean.

C. The probability of a bond return less than or equal to $3 \%$ is determined using a $Z$-score of 0.25 .

\begin{enumerate}
  \setcounter{enumi}{26}
  \item A client has a portfolio of common stocks and fixed-income instruments with a current value of $\pounds 1,350,000$. She intends to liquidate $\pounds 50,000$ from the portfolio at the end of the year to purchase a partnership share in a business. Furthermore, the client would like to be able to withdraw the $\pounds 50,000$ without reducing the initial capital of $\pounds 1,350,000$. The following table shows four alternative asset allocations.
\end{enumerate}

\section{Mean and Standard Deviation for Four Allocations (in Percent)}
\begin{center}
\begin{tabular}{lcccc}
\hline
 & A & B & $\mathbf{C}$ & $\mathbf{D}$ \\
\hline
Expected annual return & 16 & 12 & 10 & 9 \\
Standard deviation of return & 24 & 17 & 12 & 11 \\
\hline
\end{tabular}
\end{center}

Address the following questions (assume normality for Parts B and C):

A. Given the client's desire not to invade the $\pounds 1,350,000$ principal, what is the shortfall level, $R_{L}$ ? Use this shortfall level to answer Part B.

B. According to the safety-first criterion, which of the allocations is the best?

C. What is the probability that the return on the safety-first optimal portfolio will be less than the shortfall level, $R_{L}$ ?

\begin{enumerate}
  \setcounter{enumi}{27}
  \item A client holding a $\pounds 2,000,000$ portfolio wants to withdraw $\pounds 90,000$ in one year without invading the principal. According to Roy's safety-first criterion, which of the following portfolio allocations is optimal?
\end{enumerate}

\begin{center}
\begin{tabular}{cccc}
\hline
 & Allocation A & Allocation B & Allocation C \\
\hline
Expected annual return & $6.5 \%$ & $7.5 \%$ & $8.5 \%$ \\
\hline
\end{tabular}
\end{center}

\begin{center}
\begin{tabular}{lccc}
\hline
 & Allocation A & Allocation B & Allocation C \\
\hline
Standard deviation of returns & $8.35 \%$ & $10.21 \%$ & $14.34 \%$ \\
\hline
\end{tabular}
\end{center}

A. Allocation A

B. Allocation B

C. Allocation C

\begin{enumerate}
  \setcounter{enumi}{28}
  \item The weekly closing prices of Mordice Corporation shares are as follows:
\end{enumerate}

\begin{center}
\begin{tabular}{lc}
\hline
Date & Closing Price (€) \\
\hline
1 August & 112 \\
8 August & 160 \\
15 August & 120 \\
\hline
\end{tabular}
\end{center}

The continuously compounded return of Mordice Corporation shares for the period August 1 to August 15 is closest to:
A. $6.90 \%$.
B. $7.14 \%$.
C. $8.95 \%$.

\begin{enumerate}
  \setcounter{enumi}{29}
  \item In contrast to normal distributions, lognormal distributions:
\end{enumerate}

A. are skewed to the left.

B. have outcomes that cannot be negative.

C. are more suitable for describing asset returns than asset prices.

\begin{enumerate}
  \setcounter{enumi}{30}
  \item The lognormal distribution is a more accurate model for the distribution of stock prices than the normal distribution because stock prices are:
\end{enumerate}

A. symmetrical.

B. unbounded.

C. non-negative.

\begin{enumerate}
  \setcounter{enumi}{31}
  \item The price of a stock at $t=0$ is $\$ 208.25$ and at $t=1$ is $\$ 186.75$. The continuously compounded rate of return for the stock from $t=0$ to $t=1$ is closest to:
A. $-10.90 \%$.
B. $-10.32 \%$.
C. $11.51 \%$.

  \item Which one of the following statements about Student's $t$-distribution is false?

\end{enumerate}

A. It is symmetrically distributed around its mean value, like the normal distribution.

B. It has shorter (i.e., thinner) tails than the normal distribution. C. As its degrees of freedom increase, Student's $t$-distribution approaches the normal distribution.

\begin{enumerate}
  \setcounter{enumi}{33}
  \item Which one of the following statements concerning chi-square and $F$-distributions is false?
\end{enumerate}

A. They are both asymmetric distributions.

B. As their degrees of freedom increase, the shapes of their pdfs become more bell curve-like.

C. The domains of their pdfs are positive and negative numbers.

35.

A. Define Monte Carlo simulation, and explain its use in investment management.

B. Compared with analytical methods, what are the strengths and weaknesses of Monte Carlo simulation for use in valuing securities?

\begin{enumerate}
  \setcounter{enumi}{35}
  \item A Monte Carlo simulation can be used to:
A. directly provide precise valuations of call options.
B. simulate a process from historical records of returns.
C. test the sensitivity of a model to changes in assumptions-for example, on distributions of key variables.

  \item A limitation of Monte Carlo simulation is:
A. its failure to do "what if" analysis.
B. that it requires historical records of returns.
C. its inability to independently specify cause-and-effect relationships.

\end{enumerate}

\section{SOLUTIONS}
1.

A. The put's minimum value is $\$ 0$. The put's value is $\$ 0$ when the stock price is at or above $\$ 100$ at the maturity date of the option. The put's maximum value is $\$ 100=\$ 100$ (the exercise price) - $\$ 0$ (the lowest possible stock price). The put's value is $\$ 100$ when the stock is worthless at the option's maturity date. The put's minimum price increments are $\$ 0.01$. The possible outcomes of terminal put value are thus $\$ 0.00, \$ 0.01, \$ 0.02, \ldots, \$ 100$.

B. The price of the underlying has minimum price fluctuations of $\$ 0.01$ : These are the minimum price fluctuations for terminal put value. For example, if the stock finishes at $\$ 98.20$, the payoff on the put is $\$ 100-\$ 98.20=\$ 1.80$. We can specify that the nearest values to $\$ 1.80$ are $\$ 1.79$ and $\$ 1.81$. With a continuous random variable, we cannot specify the nearest values. So, we must characterize terminal put value as a discrete random variable.

C. The probability that terminal put value is less than or equal to $\$ 24$ is $P(Y \leq$ 24), or $F(24)$ in standard notation, where $F$ is the cumulative distribution function for terminal put value.

\begin{enumerate}
  \setcounter{enumi}{1}
  \item $\mathrm{C}$ is correct. The rate of return is a random variable because the future outcomes are uncertain, and it is continuous because it can take on an unlimited number of outcomes.

  \item B is correct. The function $g(x)$ satisfies the conditions of a probability function. All of the values of $g(x)$ are between 0 and 1 , and the values of $g(x)$ all sum to 1 .

  \item B is correct. The value of the cumulative distribution function lies between 0 and 1 for any $\mathrm{x}: 0 \leq F(x) \leq 1$.

  \item B is correct. The probability of any outcome is $0.05, P(1)=1 / 20=0.05$. The probability that $X$ is greater than or equal to 3 but less than 6 is expressed as $P(3 \leq X<$ 6) $=P(3)+P(4)+P(5)=0.05+0.05+0.05=0.15$.

  \item 
\end{enumerate}

A. The expected value of fourth-quarter sales is $€ 14,500,000$, calculated as $(€ 14,000,000+€ 15,000,000) / 2$. With a continuous uniform random variable, the mean or expected value is the midpoint between the smallest and largest values.

B. The probability that fourth-quarter sales will be less than or equal to $€ 14,125,000$ is 0.125 , or $12.5 \%$, calculated as $(€ 14,125,000-€ 14,000,000) /$ $(€ 15,000,000-€ 14,000,000)$.

\begin{enumerate}
  \setcounter{enumi}{6}
  \item A is correct. The probability that $X$ will take on a value of 4 or less is $F(4)=P(X$ $\leq 4)=p(1)+p(2)+p(3)+p(4)=0.60$. The probability that $X$ will take on a value of 3 or less is $F(3)=P(X \leq 3)=p(1)+p(2)+p(3)=0.50$. So, the probability that $X$ will take on a value of 4 is $F(4)-F(3)=p(4)=0.10$. The probability of $X=2$ can be found using the same logic: $F(2)-F(1)=p(2)=0.25-0.15=0.10$. The probability of $X$ taking on a value of 2 or 4 is $p(2)+p(4)=0.10+0.10=0.20$.

  \item A is correct. The probability of generating a random number equal to any fixed point under a continuous uniform distribution is zero. 9. A binomial random variable is defined as the number of successes in $n$ Bernoulli trials (a trial that produces one of two outcomes). The binomial distribution is used to make probability statements about a record of successes and failures or about anything with binary (twofold) outcomes.

  \item $\mathrm{C}$ is correct. The binomial distribution is symmetric when the probability of success on a trial is 0.50 , but it is asymmetric or skewed otherwise. Here, it is given that $p=0.50$.

  \item 
\end{enumerate}

A. The probability of an earnings increase (success) in a year is estimated as $7 / 10=0.70$, or $70 \%$, based on the record of the past 10 years.

B. The probability that earnings will increase in 5 of the next 10 years is about $10.3 \%$. Define a binomial random variable $X$, counting the number of earnings increases over the next 10 years. From Part A, the probability of an earnings increase in a given year is $p=0.70$ and the number of trials (years) is $n=10$. Equation 2 gives the probability that a binomial random variable has $x$ successes in $n$ trials, with the probability of success on a trial equal to $p:$

$P(X=x)=\left(\begin{array}{l}n \\ x\end{array}\right) p^{x}(1-p)^{n-x}=\frac{n !}{(n-x) ! x !} p^{x}(1-p)^{n-x}$.

For this example,

$$
\begin{aligned}
& \left({ }_{5} 0\right) 0.7^{5} 0.3^{10-5}=\frac{10 !}{(10-5) ! 5 !} 0.7^{5} 0.3^{10-5} \\
= & 252 \times 0.16807 \times 0.00243=0.102919 .
\end{aligned}
$$

We conclude that the probability that earnings will increase in exactly 5 of the next 10 years is 0.1029 , or approximately $10.3 \%$.

C. The expected number of yearly increases is $\mathrm{E}(X)=n p=10 \times 0.70=7$.

D. The variance of the number of yearly increases over the next 10 years is $\sigma^{2}$ $=n p(1-p)=10 \times 0.70 \times 0.30=2.1$. The standard deviation is 1.449 (the positive square root of 2.1).

E. You must assume that (1) the probability of an earnings increase (success) is constant from year to year and (2) earnings increases are independent trials. If current and past earnings help forecast next year's earnings, Assumption 2 is violated. If the company's business is subject to economic or industry cycles, neither assumption is likely to hold.

\begin{enumerate}
  \setcounter{enumi}{11}
  \item B is correct. To calculate the probability of four years of outperformance, use the formula
\end{enumerate}

$p(x)=P(X=x)=\left(\begin{array}{l}n \\ x\end{array}\right) p^{x}(1-p)^{n-x}=\frac{n !}{(n-x) ! x !} p^{x}(1-p)^{n-x}$.

Using this formula to calculate the probability in four of five years, $n=5, x=4$, and $p=0.60$.

Therefore,

$p(4)=\frac{5 !}{(5-4) ! 4 !} 0.6^{4}(1-0.6)^{5-4}=[120 / 24](0.1296)(0.40)=0.2592$.

$p(5)=\frac{5 !}{(5-5) ! 5 !} 0.6^{5}(1-0.6)^{5-5}=[120 / 120](0.0778)(1)=0.0778$.

The probability of outperforming four or more times is $p(4)+p(5)=0.2592+$ $0.0778=0.3370$. 13. The observed success rate is $4 / 7=0.571$, or $57.1 \%$. The probability of four or fewer successes is $F(4)=p(4)+p(3)+p(2)+p(1)+p(0)$, where $p(4), p(3), p(2), p(1)$, and $p(0)$ are, respectively, the probabilities of $4,3,2,1$, and 0 successes, according to the binomial distribution with $n=7$ and $p=0.70$. We have the following probabilities:

$p(4)=(7 ! / 4 ! 3 !)\left(0.70^{4}\right)\left(0.30^{3}\right)=35(0.006483)=0.226895$.

$p(3)=(7 ! / 3 ! 4 !)\left(0.70^{3}\right)\left(0.30^{4}\right)=35(0.002778)=0.097241$

$p(2)=(7 ! / 2 ! 5 !)\left(0.70^{2}\right)\left(0.30^{5}\right)=21(0.001191)=0.025005$.

$p(1)=(7 ! / 1 ! 6 !)\left(0.70^{1}\right)\left(0.30^{6}\right)=7(0.000510)=0.003572$.

$p(0)=(7 ! / 0 ! 7 !)\left(0.70^{0}\right)\left(0.30^{7}\right)=1(0.000219)=0.000219$.

Summing all these probabilities, you conclude that $F(4)=0.226895+0.097241+$ $0.025005+0.003572+0.000219=0.352931$, or $35.3 \%$.

\begin{enumerate}
  \setcounter{enumi}{13}
  \item $C$ is correct. The probability that the performance is at or below the expectation is calculated by finding $F(3)=p(3)+p(2)+p(1)+p(0)$ using the formula:
\end{enumerate}

$p(x)=P(X=x)=\left(\begin{array}{l}n \\ x\end{array}\right) p^{x}(1-p)^{n-x}=\frac{n !}{(n-x) ! x !} p^{x}(1-p)^{n-x}$.

Using this formula,

$p(3)=\frac{4 !}{(4-3) ! 3 !} 0.75^{3}(1-0.75)^{4-3}=[24 / 6](0.42)(0.25)=0.42$.

$p(2)=\frac{4 !}{(4-2) ! 2 !} 0.75^{2}(1-0.75)^{4-2}=[24 / 4](0.56)(0.06)=0.20$.

$p(1)=\frac{4 !}{(4-1) ! 1 !} 0.75^{1}(1-0.75)^{4-1}=[24 / 6](0.75)(0.02)=0.06$.

$p(0)=\frac{4 !}{(4-0) ! 0 !} 0.75^{0}(1-0.75)^{4-0}=[24 / 24](1)(0.004)=0.004$.

Therefore,

$F(3)=p(3)+p(2)+p(1)+p(0)=0.42+0.20+0.06+0.004=0.684$, or approximately $68 \%$.

\begin{enumerate}
  \setcounter{enumi}{14}
  \item A is correct. A trial, such as a coin flip, will produce one of two outcomes. Such a trial is a Bernoulli trial.

  \item $\mathrm{C}$ is correct. The probability of an up move $(p)$ can be found by solving the equation $(p) u S+(1-p) d S=(p) 105+(1-p) 97=102$. Solving for $p$ gives $8 p=5$, so $p$ $=0.625$.

  \item A is correct. Only the top node value of $\$ 219.9488$ exceeds $\$ 200$.

\end{enumerate}

\begin{center}
\includegraphics[max width=\textwidth]{2023_05_04_cff39ee44f77d6514e1bg-309}
\end{center}

18.

A. Approximately $68 \%$ of all outcomes of a normal random variable fall within plus or minus one standard deviation of the mean.

B. Approximately $95 \%$ of all outcomes of a normal random variable fall within plus or minus two standard deviations of the mean.

C. Approximately $99 \%$ of all outcomes of a normal random variable fall within plus or minus three standard deviations of the mean.

19.

A. The probability of exhausting the liquidity pool is $4.7 \%$. First, calculate $x=\lambda /(\sigma \sqrt{T})=\$ 2,000 /(\$ 450 \sqrt{5})=1.987616$. By using Excel's NORM.S.DIST() function, we get NORM.S.DIST(1.987616) $=0.9766$. Thus, the probability of exhausting the liquidity pool is $2[1-N(1.99)]=2(1-$ $0.9766)=0.0469$, or about $4.7 \%$.

B. The probability of exhausting the liquidity pool is now $32.2 \%$. The calculation follows the same steps as those in Part A. We calculate $x=\lambda /(\sigma$ $\sqrt{T})=\$ 2,000 /(\$ 450 \sqrt{20})=0.993808$. By using Excel's NORM.S.DIST () function, we get NORM.S.DIST $(0.993808)=0.8398$. Thus, the probability of exhausting the liquidity pool is $2[1-N(0.99)]=2(1-0.8398)=0.3203$, or about $32.0 \%$. This is a substantial probability that you will run out of funds to meet marking to market.

In their paper, Kolb et al. called the probability of exhausting the liquidity pool the probability of ruin, a traditional name for this type of calculation.

\begin{enumerate}
  \setcounter{enumi}{19}
  \item $\mathrm{B}$ is correct. The normal distribution has a skewness of 0 , a kurtosis of 3 , and a mean, median, and mode that are all equal.

  \item B is correct. Multivariate distributions specify the probabilities for a group of related random variables. A portfolio of technology stocks represents a group of related assets. Accordingly, statistical interrelationships must be considered, resulting in the need to use a multivariate normal distribution.

  \item C is correct. A bivariate normal distribution (two stocks) will have two means, two variances, and one correlation. A multivariate normal distribution for the returns on $n$ stocks will have $n$ means, $n$ variances, and $n(n-1) / 2$ distinct correlations.

  \item A is correct. $P(8 \% \leq$ Portfolio return $\leq 11 \%)=N(Z$ corresponding to $11 \%)-N(Z$ corresponding to $8 \%)$. For the first term, NORM.S.DIST $((11 \%-8 \%) / 14 \%)=$ $58.48 \%$. To get the second term immediately, note that $8 \%$ is the mean, and for the normal distribution, $50 \%$ of the probability lies on either side of the mean. Therefore, $N(Z$ corresponding to $8 \%)$ must equal $50 \%$. So, $P(8 \% \leq$ Portfolio return $\leq 11 \%)=0.5848-0.50=0.0848$, or approximately $8.5 \%$. 24. B is correct. By using Excel's NORM.S.DIST() function, we get NORM.S.DIST $((4 \%-7 \%) / 13 \%)=40.87 \%$. The probability that the portfolio will underperform the target is about $41 \%$.

  \item B is correct. A normal distribution has a skewness of zero (it is symmetrical around the mean). A non-zero skewness implies asymmetry in a distribution.

  \item A is correct. The chance of a negative return falls in the area to the left of $0 \%$ under a standard normal curve. By standardizing the returns and standard deviations of the two assets, the likelihood of either asset experiencing a negative return may be determined: $Z$-score (standardized value) $=(X-\mu) / \sigma$.

\end{enumerate}

$Z$-score for a bond return of $0 \%=(0-2) / 5=-0.40$.

$Z$-score for a stock return of $0 \%=(0-10) / 15=-0.67$.

For bonds, a $0 \%$ return falls 0.40 standard deviations below the mean return of $2 \%$. In contrast, for stocks, a $0 \%$ return falls 0.67 standard deviations below the mean return of $10 \%$. A standard deviation of 0.40 is less than a standard deviation of 0.67 . Negative returns thus occupy more of the left tail of the bond distribution than the stock distribution. Thus, bonds are more likely than stocks to experience a negative return.

27.

A. Because $\pounds 50,000 / \pounds 1,350,000$ is $3.7 \%$, for any return less than $3.7 \%$ the client will need to invade principal if she takes out $\pounds 50,000$. So $R_{L}=3.7 \%$.

B. To decide which of the allocations is safety-first optimal, select the alternative with the highest ratio $\left[\mathrm{E}\left(R_{P}\right)-R_{L}\right] / \sigma_{P}$ :

Allocation A: $0.5125=(16-3.7) / 24$.

Allocation B: $0.488235=(12-3.7) / 17$.

Allocation C: $0.525=(10-3.7) / 12$.

Allocation D: $0.481818=(9-3.7) / 11$.

Allocation C, with the largest ratio (0.525), is the best alternative according to the safety-first criterion.

C. To answer this question, note that $\left.P\left(R_{C}<3.7\right)=N(0.037-0.10) / 0.12\right)$ $=N(-0.525)$. By using Excel's NORM.S.DIST() function, we get NORM.S.DIST $((0.037-0.10) / 0.12)=29.98 \%$, or about $30 \%$. The safety-first optimal portfolio has a roughly $30 \%$ chance of not meeting a $3.7 \%$ return threshold.

\begin{enumerate}
  \setcounter{enumi}{27}
  \item B is correct. Allocation B has the highest safety-first ratio. The threshold return level, $R_{L}$, for the portfolio is $\pounds 90,000 / \pounds 2,000,000=4.5 \%$; thus, any return less than $R_{L}=4.5 \%$ will invade the portfolio principal. To compute the allocation that is safety-first optimal, select the alternative with the highest ratio:
\end{enumerate}

$\frac{\left[\mathrm{E}\left(R_{P}-R_{L}\right)\right]}{\sigma_{P}}$.

Allocation $\mathrm{A}=\frac{6.5-4.5}{8.35}=0.240$.

Allocation $B=\frac{7.5-4.5}{10.21}=0.294$. Allocation $\mathrm{C}=\frac{8.5-4.5}{14.34}=0.279$

\begin{enumerate}
  \setcounter{enumi}{28}
  \item A is correct. The continuously compounded return of an asset over a period is equal to the natural log of the asset's price change during the period. In this case, $\ln (120 / 112)=6.90 \%$

  \item B is correct. By definition, lognormal random variables cannot have negative values.

  \item C is correct. A lognormal distributed variable has a lower bound of zero. The lognormal distribution is also right skewed, which is a useful property in describing asset prices.

  \item A is correct. The continuously compounded return from $t=0$ to $t=1$ is $r_{0,1}=$ $\ln \left(S_{1} / S_{0}\right)=\ln (186.75 / 208.25)=-0.10897=-10.90 \%$

  \item A is correct, since it is false. Student's $t$-distribution has longer (fatter) tails than the normal distribution and, therefore, it may provide a more reliable, more conservative downside risk estimate.

  \item $C$ is correct, since it is false. Both chi-square and $F$-distributions are bounded from below by zero, so the domains of their pdfs are restricted to positive numbers.

  \item 
\end{enumerate}

A. Monte Carlo simulation involves the use of computer software to represent the operation of a complex financial system. A characteristic feature of Monte Carlo simulation is the generation of a large number of random samples from a specified probability distribution (or distributions) to represent the role of risk in the system. Monte Carlo simulation is widely used to estimate risk and return in investment applications. In this setting, we simulate the portfolio's profit and loss performance for a specified time horizon. Repeated trials within the simulation produce a simulated frequency distribution of portfolio returns from which performance and risk measures are derived. Another important use of Monte Carlo simulation in investments is as a tool for valuing complex securities for which no analytic pricing formula is available. It is also an important modeling resource for securities with complex embedded options.

B. Strengths: Monte Carlo simulation can be used to price complex securities for which no analytic expression is available, particularly European-style options.

Weaknesses: Monte Carlo simulation provides only statistical estimates, not exact results. Analytic methods, when available, provide more insight into cause-and-effect relationships than does Monte Carlo simulation.

\begin{enumerate}
  \setcounter{enumi}{35}
  \item $C$ is correct. A characteristic feature of Monte Carlo simulation is the generation of a large number of random samples from a specified probability distribution or distributions to represent the role of risk in the system. Therefore, it is very useful for investigating the sensitivity of a model to changes in assumptions-for example, on distributions of key variables. 37. C is correct. Monte Carlo simulation is a complement to analytical methods. Monte Carlo simulation provides statistical estimates and not exact results. Analytical methods, when available, provide more insight into cause-and-effect relationships.
\end{enumerate}

\section{LEARNING MODULE}
\begin{center}
\includegraphics[max width=\textwidth]{2023_05_04_cff39ee44f77d6514e1bg-313}
\end{center}

\section{Sampling and Estimation}
by Richard A. DeFusco, PhD, CFA, Dennis W. McLeavey, DBA, CFA, Jerald E. Pinto, PhD, CFA, and David E. Runkle, PhD, CFA.

Richard A. DeFusco, PhD, CFA, is at the University of Nebraska-Lincoln (USA). Dennis W. McLeavey, DBA, CFA, is at the University of Rhode Island (USA). Jerald E. Pinto, PhD, CFA, is at CFA Institute (USA). David E. Runkle, PhD, CFA, is at Jacobs Levy Equity Management (USA).

\section{LEARNING OUTCOME}
\begin{center}
\begin{tabular}{|c|c|}
\hline
Mastery & The candidate should be able to: \\
\hline
\includegraphics[max width=\textwidth]{2023_05_04_cff39ee44f77d6514e1bg-313(1)}
 & $\begin{array}{l}\text { compare and contrast probability samples with non-probability } \\ \text { samples and discuss applications of each to an investment problem } \\ \text { explain sampling error } \\ \text { compare and contrast simple random, stratified random, cluster, } \\ \text { convenience, and judgmental sampling } \\ \text { explain the central limit theorem and its importance } \\ \text { calculate and interpret the standard error of the sample mean } \\ \text { identify and describe desirable properties of an estimator } \\ \text { contrast a point estimate and a confidence interval estimate of a } \\ \text { population parameter } \\ \text { calculate and interpret a confidence interval for a population mean, } \\ \text { given a normal distribution with } 1 \text { ) a known population variance, } \\ \text { 2) an unknown population variance, or } 3 \text { ) an unknown population } \\ \text { variance and a large sample size } \\ \text { describe the use of resampling (bootstrap, jackknife) to estimate the } \\ \text { sampling distribution of a statistic } \\ \text { describe the issues regarding selection of the appropriate sample } \\ \text { size, data snooping bias, sample selection bias, survivorship bias, } \\ \text { look-ahead bias, and time-period bias }\end{array}$ \\
\hline
\end{tabular}
\end{center}

\section{INTRODUCTION}
Each day, we observe the high, low, and close of stock market indexes from around the world. Indexes such as the S\&P 500 Index and the Nikkei 225 Stock Average are samples of stocks. Although the S\&P 500 and the Nikkei do not represent the populations of US or Japanese stocks, we view them as valid indicators of the whole population's behavior. As analysts, we are accustomed to using this sample information to assess how various markets from around the world are performing. Any statistics that we compute with sample information, however, are only estimates of the underlying population parameters. A sample, then, is a subset of the population-a subset studied to infer conclusions about the population itself.

We introduce and discuss sampling-the process of obtaining a sample. In investments, we continually make use of the mean as a measure of central tendency of random variables, such as return and earnings per share. Even when the probability distribution of the random variable is unknown, we can make probability statements about the population mean using the central limit theorem. We discuss and illustrate this key result. Following that discussion, we turn to statistical estimation. Estimation seeks precise answers to the question "What is this parameter's value?"

The central limit theorem and estimation, the core of the body of methods presented in the sections that follow, may be applied in investment applications. We often interpret the results for the purpose of deciding what works and what does not work in investments. We will also discuss the interpretation of statistical results based on financial data and the possible pitfalls in this process.

\section{SAMPLING METHODS}
compare and contrast probability samples with non-probability samples and discuss applications of each to an investment problem explain sampling error

compare and contrast simple random, stratified random, cluster, convenience, and judgmental sampling

In this section, we present the various methods for obtaining information on a population (all members of a specified group) through samples (part of the population). The information on a population that we try to obtain usually concerns the value of a parameter, a quantity computed from or used to describe a population of data. When we use a sample to estimate a parameter, we make use of sample statistics (statistics, for short). A statistic is a quantity computed from or used to describe a sample of data.

We take samples for one of two reasons. In some cases, we cannot possibly examine every member of the population. In other cases, examining every member of the population would not be economically efficient. Thus, savings of time and money are two primary factors that cause an analyst to use sampling to answer a question about a population.

There are two types of sampling methods: probability sampling and non-probability sampling.Probability sampling gives every member of the population an equal chance of being selected. Hence it can create a sample that is representative of the population. In contrast, non-probability sampling depends on factors other than probability considerations, such as a sampler's judgment or the convenience to access data. Consequently, there is a significant risk that non-probability sampling might generate a non-representative sample. In general, all else being equal, probability sampling can yield more accuracy and reliability compared with non-probability sampling.

We first focus on probability sampling, particularly the widely used simple random sampling and stratified random sampling. We then turn our attention to non-probability sampling.

\section{Simple Random Sampling}
Suppose a telecommunications equipment analyst wants to know how much major customers will spend on average for equipment during the coming year. One strategy is to survey the population of telecom equipment customers and inquire what their purchasing plans are. In statistical terms, the characteristics of the population of customers' planned expenditures would then usually be expressed by descriptive measures such as the mean and variance. Surveying all companies, however, would be very costly in terms of time and money.

Alternatively, the analyst can collect a representative sample of companies and survey them about upcoming telecom equipment expenditures. In this case, the analyst will compute the sample mean expenditure, $\bar{X}$, a statistic. This strategy has a substantial advantage over polling the whole population because it can be accomplished more quickly and at lower cost.

Sampling, however, introduces error. The error arises because not all the companies in the population are surveyed. The analyst who decides to sample is trading time and money for sampling error.

When an analyst chooses to sample, he must formulate a sampling plan. A sampling plan is the set of rules used to select a sample. The basic type of sample from which we can draw statistically sound conclusions about a population is the simple random sample (random sample, for short).

\begin{itemize}
  \item Definition of Simple Random Sample. A simple random sample is a subset of a larger population created in such a way that each element of the population has an equal probability of being selected to the subset.
\end{itemize}

The procedure of drawing a sample to satisfy the definition of a simple random sample is called simple random sampling. How is simple random sampling carried out? We need a method that ensures randomness-the lack of any pattern-in the selection of the sample. For a finite (limited) population, the most common method for obtaining a random sample involves the use of random numbers (numbers with assured properties of randomness). First, we number the members of the population in sequence. For example, if the population contains 500 members, we number them in sequence with three digits, starting with 001 and ending with 500. Suppose we want a simple random sample of size 50. In that case, using a computer random-number generator or a table of random numbers, we generate a series of three-digit random numbers. We then match these random numbers with the number codes of the population members until we have selected a sample of size 50 . Simple random sampling is particularly useful when data in the population is homogeneous-that is, the characteristics of the data or observations (e.g., size or region) are broadly similar. We will see that if this condition is not satisfied other types of sampling may be more appropriate.

Sometimes we cannot code (or even identify) all the members of a population. We often use systematic sampling in such cases. With systematic sampling, we select every $k$ th member until we have a sample of the desired size. The sample that results from this procedure should be approximately random. Real sampling situations may require that we take an approximately random sample. Suppose the telecommunications equipment analyst polls a random sample of telecom equipment customers to determine the average equipment expenditure. The sample mean will provide the analyst with an estimate of the population mean expenditure. The mean obtained from the sample this way will differ from the population mean that we are trying to estimate. It is subject to error. An important part of this error is known as sampling error, which comes from sampling variation and occurs because we have data on only a subset of the population.

\begin{itemize}
  \item Definition of Sampling Error. Sampling error is the difference between the observed value of a statistic and the quantity it is intended to estimate as a result of using subsets of the population.
\end{itemize}

A random sample reflects the properties of the population in an unbiased way, and sample statistics, such as the sample mean, computed on the basis of a random sample are valid estimates of the underlying population parameters.

A sample statistic is a random variable. In other words, not only do the original data from the population have a distribution but so does the sample statistic.

This distribution is the statistic's sampling distribution.

\begin{itemize}
  \item Definition of Sampling Distribution of a Statistic. The sampling distribution of a statistic is the distribution of all the distinct possible values that the statistic can assume when computed from samples of the same size randomly drawn from the same population.
\end{itemize}

In the case of the sample mean, for example, we refer to the "sampling distribution of the sample mean" or the distribution of the sample mean. We will have more to say about sampling distributions later in this reading. Next, however, we look at another sampling method that is useful in investment analysis.

\section{Stratified Random Sampling}
The simple random sampling method just discussed may not be the best approach in all situations. One frequently used alternative is stratified random sampling.

\begin{itemize}
  \item Definition of Stratified Random Sampling. In stratified random sampling, the population is divided into subpopulations (strata) based on one or more classification criteria. Simple random samples are then drawn from each stratum in sizes proportional to the relative size of each stratum in the population. These samples are then pooled to form a stratified random sample.
\end{itemize}

In contrast to simple random sampling, stratified random sampling guarantees that population subdivisions of interest are represented in the sample. Another advantage is that estimates of parameters produced from stratified sampling have greater precision-that is, smaller variance or dispersion-than estimates obtained from simple random sampling.

Bond indexing is one area in which stratified sampling is frequently applied. Indexing is an investment strategy in which an investor constructs a portfolio to mirror the performance of a specified index. In pure bond indexing, also called the full-replication approach, the investor attempts to fully replicate an index by owning all the bonds in the index in proportion to their market value weights. Many bond indexes consist of thousands of issues, however, so pure bond indexing is difficult to implement. In addition, transaction costs would be high because many bonds do not have liquid markets. Although a simple random sample could be a solution to the cost problem, the sample would probably not match the index's major risk factorsinterest rate sensitivity, for example. Because the major risk factors of fixed-income portfolios are well known and quantifiable, stratified sampling offers a more effective approach. In this approach, we divide the population of index bonds into groups of similar duration (interest rate sensitivity), cash flow distribution, sector, credit quality, and call exposure. We refer to each group as a stratum or cell (a term frequently used in this context). Then, we choose a sample from each stratum proportional to the relative market weighting of the stratum in the index to be replicated.

\section{EXAMPLE 1}
\section{Bond Indexes and Stratified Sampling}
Suppose you are the manager of a portfolio of bonds indexed to the Bloomberg Barclays US Government/Credit Index, meaning that the portfolio returns should be similar to those of the index. You are exploring several approaches to indexing, including a stratified sampling approach. You first distinguish among agency bonds, US Treasury bonds, and investment-grade corporate bonds. For each of these three groups, you define 10 maturity intervals -1 to 2 years, 2 to 3 years, 3 to 4 years, 4 to 6 years, 6 to 8 years, 8 to 10 years, 10 to 12 years, 12 to 15 years, 15 to 20 years, and 20 to 30 years-and also separate the bonds with coupons (annual interest rates) of $6 \%$ or less from the bonds with coupons of more than $6 \%$.

\begin{enumerate}
  \item How many cells or strata does this sampling plan entail?
\end{enumerate}

\section{Solution to 1:}
We have 3 issuer classifications, 10 maturity classifications, and 2 coupon classifications. So, in total, this plan entails $3(10)(2)=60$ different strata or cells.

\begin{enumerate}
  \setcounter{enumi}{1}
  \item If you use this sampling plan, what is the minimum number of issues the indexed portfolio can have?
\end{enumerate}

\section{Solution to 2:}
One cannot have less than 1 issue for each cell, so the portfolio must include at least 60 issues.

\begin{enumerate}
  \setcounter{enumi}{2}
  \item Suppose that in selecting among the securities that qualify for selection within each cell, you apply a criterion concerning the liquidity of the security's market. Is the sample obtained random? Explain your answer.
\end{enumerate}

\section{Solution to 3:}
Applying any additional criteria to the selection of securities for the cells, not every security that might be included has an equal probability of being selected. As a result, the sampling is not random. In practice, indexing using stratified sampling usually does not strictly involve random sampling because the selection of bond issues within cells is subject to various additional criteria. Because the purpose of sampling in this application is not to make an inference about a population parameter but rather to index a portfolio, lack of randomness is not in itself a problem in this application of stratified sampling.

\section{Cluster Sampling}
Another sampling method, cluster sampling, also requires the division or classification of the population into subpopulation groups, called clusters. In this method, the population is divided into clusters, each of which is essentially a mini-representation of the entire populations Then certain clusters are chosen as a whole using simple random sampling. If all the members in each sampled cluster are sampled, this sample plan is referred to as one-stage cluster sampling. If a subsample is randomly selected from each selected cluster, then the plan is referred as two-stage cluster sampling. Exhibit 1 (bottom right panel) shows how cluster sampling works and how it compares with the other probability sampling methods.

\section{Exhibit 1: Probability Sampling}
\begin{center}
\includegraphics[max width=\textwidth]{2023_05_04_cff39ee44f77d6514e1bg-318(1)}
\end{center}

Stratified sample

\begin{center}
\includegraphics[max width=\textwidth]{2023_05_04_cff39ee44f77d6514e1bg-318}
\end{center}

Systematic sample
\includegraphics[max width=\textwidth, center]{2023_05_04_cff39ee44f77d6514e1bg-318(2)}

A major difference between cluster and stratified random samples is that in cluster sampling, a whole cluster is regarded as a sampling unit and only sampled clusters are included. In stratified random sampling, however, all the strata are included and only specific elements within each stratum are then selected as sampling units.

Cluster sampling is commonly used for market survey, and the most popular version identifies clusters based on geographic parameters. For example, a research institute is looking to survey if individual investors in the United States are bullish, bearish, or neutral on the stock market. It would be impossible to carry out the research by surveying all the individual investors in the country. The two-stage cluster sampling is a good solution in this case. At the first stage, a researcher can group the population by states and all the individual investors of each state represent a cluster. A handful of the clusters are then randomly selected. At the second stage, a simple random sample of individual investors is selected from each sampled cluster to conduct the survey.

Compared with other probability sampling methods, given equal sample size, cluster sampling usually yields lower accuracy because a sample from a cluster might be less representative of the entire population. Its major advantage, however, is offering the most time-efficient and cost-efficient probability sampling plan for analyzing a vast population.

\section{Non-Probability Sampling}
Non-probability sampling methods rely not on a fixed selection process but instead on a researcher's sample selection capabilities. We introduce two major types of non-probability sampling methods here.

\begin{itemize}
  \item Convenience Sampling: In this method, an element is selected from the population based on whether or not it is accessible to a researcher or on how easy it is for a researcher to access the element. Because the samples are selected conveniently, they are not necessarily representative of the entire population, and hence the level of the sampling accuracy could be limited. But the advantage of convenience sampling is that data can be collected quickly at a low cost. In situations such as the preliminary stage of research or in circumstances subject to cost constraints, convenience sampling is often used as a time-efficient and cost-effective sampling plan for a small-scale pilot study before testing a large-scale and more representative sample.

  \item Judgmental Sampling: This sampling process involves selectively handpicking elements from the population based on a researcher's knowledge and professional judgment. Sample selection under judgmental sampling could be affected by the bias of the researcher and might lead to skewed results that do not represent the whole population. In circumstances where there is a time constraint, however, or when the specialty of researchers is critical to select a more representative sample than by using other probability or non-probability sampling methods, judgmental sampling allows researchers to go directly to the target population of interest. For example, when auditing financial statements, seasoned auditors can apply their sound judgment to select accounts or transactions that can provide sufficient audit coverage. Example 2 illustrates an application of these sampling methods.

\end{itemize}

\section{EXAMPLE 2}
\section{Demonstrating the Power of Sampling}
To demonstrate the power of sampling, we conduct two sampling experiments on a large dataset. The full dataset is the "population," representing daily returns of the fictitious Euro-Asia-Africa (EAA) Equity Index. This dataset spans a five-year period and consists of 1,258 observations of daily returns with a minimum value of $-4.1 \%$ and a maximum value of $5.0 \%$.

First, we calculate the mean daily return of the EAA Equity Index (using the population).

By taking the average of all the data points, the mean of the entire daily return series is computed as $0.035 \%$.

\section{First Experiment: Random Sampling}
The sample size $m$ is set to $5,10,20,50,100,200,500$, and 1,000. At each sample size, we run random sampling multiple times $(N=100)$ to collect 100 samples to compute mean absolute error. The aim is to compute and plot the mean error versus the sample size.

For a given sample size $m$, we use the following procedure to compute mean absolute error in order to measure sampling error:

\begin{enumerate}
  \item Randomly draw $m$ observations from the entire daily return series to form a sample. 2. Compute the mean of this sample.

  \item Compute the absolute error, the difference between the sample's mean and the population mean. Because we treat the whole five-year daily return series as a population, the population mean is $0.035 \%$ as we computed in Solution 1.

  \item We repeat the previous three steps a hundred times $(N=100)$ to collect 100 samples of the same size $m$ and compute the absolute error of each sample.

  \item Compute the mean of the 100 absolute errors as the mean absolute error for the sample size $m$.

\end{enumerate}

By applying this procedure, we compute mean absolute errors for eight different sample sizes: $m=5,10,20,50,100,200,500$, and 1000. Exhibit 2 summarizes the results.

\section{Exhibit 2: Mean Absolute Error of Random Sampling}
\begin{center}
\begin{tabular}{lcccccccc}
\hline
$\begin{array}{l}\text { Sample } \\ \text { size }\end{array}$ & $\mathbf{5}$ & $\mathbf{1 0}$ & $\mathbf{2 0}$ & $\mathbf{5 0}$ & $\mathbf{1 0 0}$ & $\mathbf{2 0 0}$ & $\mathbf{5 0 0}$ & $\mathbf{1 , 0 0 0}$ \\
\hline
$\begin{array}{l}\text { Mean } \\ \text { absolute } \\ \text { error }\end{array}$ & $0.297 \%$ & $0.218 \%$ & $0.163 \%$ & $0.091 \%$ & $0.063 \%$ & $0.039 \%$ & $0.019 \%$ & $0.009 \%$ \\
\hline
\end{tabular}
\end{center}

Mean absolute errors are plotted against sample size in Exhibit 3. The plot shows that the error quickly shrinks as the sample size increases. It also indicates that a minimum sample size is needed to limit sample error and achieve a certain level of accuracy. After a certain size, however (e.g., 200 to 400 in this case), there is little incremental benefit from adding more observations.

Exhibit 3: Mean Absolute Error of Random Sampling vs. Sample Size

\begin{center}
\includegraphics[max width=\textwidth]{2023_05_04_cff39ee44f77d6514e1bg-320}
\end{center}

\section{Second Experiment: Stratified Random Sampling}
We now conduct stratified random sampling by dividing daily returns into groups by year. The sample size $m$ is again set to $5,10,20,50,100,200,500$, and 1,000. At each sample size, run random sampling multiple times $(N=100)$ to collect 100 samples to compute mean absolute error. We follow the same steps as before, except for the first step. Rather than running a simple random sampling, we conduct stratified random sampling-that is, randomly selecting subsamples of equal number from daily return groups by year to generate a full sample. For example, for a sample of 50, 10 data points are randomly selected from daily returns of each year from 2014 to 2018, respectively. Exhibit 4 summarizes the results.

\section{Exhibit 4: Mean Absolute Error of Stratified Random Sampling}
\begin{center}
\begin{tabular}{lcccccccc}
\hline
$\begin{array}{l}\text { Sample } \\ \text { size }\end{array}$ & $\mathbf{5}$ & $\mathbf{1 0}$ & $\mathbf{2 0}$ & $\mathbf{5 0}$ & $\mathbf{1 0 0}$ & $\mathbf{2 0 0}$ & $\mathbf{5 0 0}$ & $\mathbf{1 , 0 0 0}$ \\
\hline
$\begin{array}{l}\text { Mean } \\ \text { absolute } \\ \text { error }\end{array}$ & $0.294 \%$ & $0.205 \%$ & $0.152 \%$ & $0.083 \%$ & $0.071 \%$ & $0.051 \%$ & $0.025 \%$ & $0.008 \%$ \\
\end{tabular}
\end{center}

Mean absolute errors are plotted against sample size in Exhibit 5. Similar to random sampling, the plot shows rapid shrinking of errors with increasing sample size, but this incremental benefit diminishes after a certain sample size is reached.

Exhibit 5: Mean Absolute Error of Stratified Random Sampling vs.

Sample Size

\begin{center}
\includegraphics[max width=\textwidth]{2023_05_04_cff39ee44f77d6514e1bg-321}
\end{center}

The sampling methods introduced here are summarized in the diagram in Exhibit 6 .

\section{Exhibit 6: Summary of Sampling Methods}
\begin{center}
\includegraphics[max width=\textwidth]{2023_05_04_cff39ee44f77d6514e1bg-322}
\end{center}

\section{EXAMPLE 3}
An analyst is studying research and development (R\&D) spending by pharmaceutical companies around the world. She considers three sampling methods for understanding a company's level of R\&D. Method 1 is to simply use all the data available to her from an internal database that she and her colleagues built while researching several dozen representative stocks in the sector. Method 2 involves relying on a commercial database provided by a data vendor. She would select every fifth pharmaceutical company on the list to pull the data. Method 3 is to first divide pharmaceutical companies in the commercial database into three groups according to the region where a company is headquartered (e.g., Asia, Europe, or North America) and then randomly select a subsample of companies from each group, with the sample size proportional to the size of its associated group in order to form a complete sample.

\begin{enumerate}
  \item Method 1 is an example of:
A. simple random sampling.
B. stratified random sampling.
C. convenience sampling.
\end{enumerate}

\section{Solution to 1:}
$\mathrm{C}$ is correct. The analyst selects the data from the internal database because they are easy and convenient to access. 2. Method 2 is an example of:
A. judgmental sampling.
B. systematic sampling.
C. cluster sampling.

\section{Solution to 2:}
B is correct. The sample elements are selected with a fixed interval $(k=5)$ from the large population provided by data vendor.

\begin{enumerate}
  \setcounter{enumi}{2}
  \item Method 3 is an example of:
A. simple random sampling.
B. stratified random sampling.
C. cluster sampling.
\end{enumerate}

\section{Solution to 3:}
$\mathrm{B}$ is correct. The population of pharmaceutical companies is divided into three strata by region to perform random sampling individually.

\section{Sampling from Different Distributions}
In practice, other than selecting appropriate sampling methods, we also need to be careful when sampling from a population that is not under one single distribution. Example 4 illustrates the problems that can arise when sampling from more than one distribution.

\section{EXAMPLE 4}
\section{Calculating Sharpe Ratios: One or Two Years of Quarterly Data}
\begin{enumerate}
  \item Analysts often use the Sharpe ratio to evaluate the performance of a managed portfolio. The Sharpe ratio is the average return in excess of the risk-free rate divided by the standard deviation of returns. This ratio measures the return of a fund or a security over and above the risk-free rate (the excess return) earned per unit of standard deviation of return.
\end{enumerate}

To compute the Sharpe ratio, suppose that an analyst collects eight quarterly excess returns (i.e., total return in excess of the risk-free rate). During the first year, the investment manager of the portfolio followed a low-risk strategy, and during the second year, the manager followed a high-risk strategy. For each of these years, the analyst also tracks the quarterly excess returns of some benchmark against which the manager will be evaluated. For each of the two years, the Sharpe ratio for the benchmark is 0.21 . Exhibit 7 gives the calculation of the Sharpe ratio of the portfolio. Exhibit 7: Calculation of Sharpe Ratios: Low-Risk and High-Risk Strategies

\begin{center}
\begin{tabular}{lcc}
\hline
Quarter/Measure & $\begin{array}{c}\text { Year 1 } \\ \text { Excess Returns }\end{array}$ & $\begin{array}{c}\text { Year 2 } \\ \text { Excess Returns }\end{array}$ \\
\hline
Quarter 1 & $-3 \%$ & $-12 \%$ \\
Quarter 2 & 5 & 20 \\
Quarter 3 & -3 & -12 \\
Quarter 4 & 5 & 20 \\
Quarterly average & $1 \%$ & $4 \%$ \\
Quarterly standard deviation & $4.62 \%$ & $18.48 \%$ \\
\end{tabular}
\end{center}

Sharpe ratio $=0.22=1 / 4.62=4 / 18.48$

For the first year, during which the manager followed a low-risk strategy, the average quarterly return in excess of the risk-free rate was $1 \%$ with a standard deviation of $4.62 \%$. The Sharpe ratio is thus $1 / 4.62=0.22$. The second year's results mirror the first year except for the higher average return and volatility. The Sharpe ratio for the second year is $4 / 18.48=0.22$. The Sharpe ratio for the benchmark is 0.21 during the first and second years. Because larger Sharpe ratios are better than smaller ones (providing more return per unit of risk), the manager appears to have outperformed the benchmark.

Now, suppose the analyst believes a larger sample to be superior to a small one. She thus decides to pool the two years together and calculate a Sharpe ratio based on eight quarterly observations. The average quarterly excess return for the two years is the average of each year's average excess return. For the two-year period, the average excess return is $(1+4) / 2=2.5 \%$ per quarter. The standard deviation for all eight quarters measured from the sample mean of $2.5 \%$ is $12.57 \%$. The portfolio's Sharpe ratio for the two-year period is now $2.5 / 12.57=0.199$; the Sharpe ratio for the benchmark remains 0.21 . Thus, when returns for the two-year period are pooled, the manager appears to have provided less return per unit of risk than the benchmark and less when compared with the separate yearly results.

The problem with using eight quarters of return data is that the analyst has violated the assumption that the sampled returns come from the same population. As a result of the change in the manager's investment strategy, returns in Year 2 followed a different distribution than returns in Year 1. Clearly, during Year 1, returns were generated by an underlying population with lower mean and variance than the population of the second year. Combining the results for the first and second years yielded a sample that was representative of no population. Because the larger sample did not satisfy model assumptions, any conclusions the analyst reached based on the larger sample are incorrect. For this example, she was better off using a smaller sample than a larger sample because the smaller sample represented a more homogeneous distribution of returns.

\textbackslash section\{THE CENTRAL LIMIT THEOREM AND DISTRIBUTION OF THE SAMPLE MEAN

explain the central limit theorem and its importance

calculate and interpret the standard error of the sample mean

Earlier we presented a telecommunications equipment analyst who decided to sample in order to estimate mean planned capital expenditures by the customers of telecom equipment vendors. Supposing that the sample is representative of the underlying population, how can the analyst assess the sampling error in estimating the population mean? Viewed as a formula that takes a function of the random outcomes of a random variable, the sample mean is itself a random variable with a probability distribution. That probability distribution is called the statistic's sampling distribution. To estimate how closely the sample mean can be expected to match the underlying population mean, the analyst needs to understand the sampling distribution of the mean. Fortunately, we have a result, the central limit theorem, that helps us understand the sampling distribution of the mean for many of the estimation problems we face.

\section{The Central Limit Theorem}
To explain the central limit theorem, we will revisit the daily returns of the fictitious Euro-Asia-Africa Equity Index shown earlier. The dataset (the population) consists of daily returns of the index over a five-year period. The 1,258 return observations have a population mean of $0.035 \%$.

We conduct four different sets of random sampling from the population. We first draw a random sample of 10 daily returns and obtain a sample mean. We repeat this exercise 99 more times, drawing a total of 100 samples of 10 daily returns. We plot the sample mean results in a histogram, as shown in the top left panel of Exhibit 8 . We then repeat the process with a larger sample size of 50 daily returns. We draw 100 samples of 50 daily returns and plot the results (the mean returns) in the histogram shown in the top right panel of Exhibit 8. We then repeat the process for sample sizes of 100 and 300 daily returns, respectively, again drawing 100 samples in each case. These results appear in the bottom left and bottom right panels of Exhibit 8. Looking at all four panels together, we observe that the larger the sample size, the more closely the histogram follows the shape of normal distribution.

\section{Exhibit 8: Sampling Distribution with Increasing Sample Size}
A. Sample Size $n=10$

\begin{center}
\includegraphics[max width=\textwidth]{2023_05_04_cff39ee44f77d6514e1bg-326(1)}
\end{center}

C. Sample Size $n=100$

\begin{center}
\includegraphics[max width=\textwidth]{2023_05_04_cff39ee44f77d6514e1bg-326}
\end{center}

B. Sample Size $n=50$

\begin{center}
\includegraphics[max width=\textwidth]{2023_05_04_cff39ee44f77d6514e1bg-326(3)}
\end{center}

D. Sample Size $n=300$

\begin{center}
\includegraphics[max width=\textwidth]{2023_05_04_cff39ee44f77d6514e1bg-326(2)}
\end{center}

The results of this exercise show that as we increase the size of a random sample, the distribution of the sample means tends towards a normal distribution. This is a significant outcome and brings us to the central limit theorem concept, one of the most practically useful theorems in probability theory. It has important implications for how we construct confidence intervals and test hypotheses. Formally, it is stated as follows:

\begin{itemize}
  \item The Central Limit Theorem. Given a population described by any probability distribution having mean $\mu$ and finite variance $\sigma^{2}$, the sampling distribution of the sample mean $\bar{X}$ computed from random samples of size $n$ from this population will be approximately normal with mean $\mu$ (the population mean) and variance $\sigma^{2} / n$ (the population variance divided by $n$ ) when the sample size $n$ is large.
\end{itemize}

Consider what the expression $\sigma^{2} / n$ signifies. Variance $\left(\sigma^{2}\right)$ stays the same, but as $n$ increases, the size of the fraction decreases. This dynamic suggests that it becomes progressively less common to obtain a sample mean that is far from the population mean with a larger sample size. For example, if we randomly pick returns of five stocks (trading on a market that features more than 1,000 stocks) on a particular day, their mean return is likely to be quite different from the market return. If we pick 100 stocks, the sample mean will be much closer to the market return. If we pick 1,000 stocks, the sample mean will likely be very close to the market return. The central limit theorem allows us to make quite precise probability statements about the population mean by using the sample mean, whatever the distribution of the population (so long as it has finite variance), because the sample mean follows an approximate normal distribution for large-size samples. The obvious question is, "When is a sample's size large enough that we can assume the sample mean is normally distributed?" In general, when sample size $n$ is greater than or equal to 30 , we can assume that the sample mean is approximately normally distributed. When the underlying population is very non-normal, a sample size well in excess of 30 may be required for the normal distribution to be a good description of the sampling distribution of the mean.

\section{EXAMPLE 5}
A research analyst makes two statements about repeated random sampling:

Statement 1 When repeatedly drawing large samples from datasets, the sample means are approximately normally distributed.

Statement 2 The underlying population from which samples are drawn must be normally distributed in order for the sample mean to be normally distributed.

\begin{enumerate}
  \item Which of the following best describes the analyst's statements?
\end{enumerate}

A. Statement 1 is false; Statement 2 is true.

B. Both statements are true.

C. Statement 1 is true; Statement 2 is false.

\section{Solution:}
$\mathrm{C}$ is correct. According to the central limit theorem, Statement 1 is true. Statement 2 is false because the underlying population does not need to be normally distributed in order for the sample mean to be normally distributed.

\section{Standard Error of the Sample Mean}
The central limit theorem states that the variance of the distribution of the sample mean is $\sigma^{2} / n$. The positive square root of variance is standard deviation. The standard deviation of a sample statistic is known as the standard error of the statistic. The standard error of the sample mean is an important quantity in applying the central limit theorem in practice.

\begin{itemize}
  \item Definition of the Standard Error of the Sample Mean. For sample mean $\bar{X}$ calculated from a sample generated by a population with standard deviation $\sigma$, the standard error of the sample mean is given by one of two expressions:
\end{itemize}

$\sigma_{\bar{X}}=\frac{\sigma}{\sqrt{n}}$

when we know $\sigma$, the population standard deviation, or by

$s_{X}=\frac{s}{\sqrt{n}}$

when we do not know the population standard deviation and need to use the sample standard deviation, $s$, to estimate it. In practice, we almost always need to use Equation 2. The estimate of $s$ is given by the square root of the sample variance, $s^{2}$, calculated as follows:

$$
s^{2}=\frac{\sum_{i=1}^{n}\left(X_{i}-\bar{X}\right)^{2}}{n-1}
$$

It is worth noting that although the standard error is the standard deviation of the sampling distribution of the parameter, "standard deviation" in general and "standard error" are two distinct concepts, and the terms are not interchangeable. Simply put, standard deviation measures the dispersion of the data from the mean, whereas standard error measures how much inaccuracy of a population parameter estimate comes from sampling. The contrast between standard deviation and standard error reflects the distinction between data description and inference. If we want to draw conclusions about how spread out the data are, standard deviation is the term to quote. If we want to find out how precise the estimate of a population parameter from sampled data is relative to its true value, standard error is the metric to use.

We will soon see how we can use the sample mean and its standard error to make probability statements about the population mean by using the technique of confidence intervals. First, however, we provide an illustration of the central limit theorem's force.

\section{EXAMPLE 6}
\section{The Central Limit Theorem}
It is remarkable that the sample mean for large sample sizes will be distributed normally regardless of the distribution of the underlying population. To illustrate the central limit theorem in action, we specify in this example a distinctly non-normal distribution and use it to generate a large number of random samples of size 100 . We then calculate the sample mean for each sample and observe the frequency distribution of the calculated sample means. Does that sampling distribution look like a normal distribution?

We return to the telecommunications analyst studying the capital expenditure plans of telecom businesses. Suppose that capital expenditures for communications equipment form a continuous uniform random variable with a lower bound equal to $\$ 0$ and an upper bound equal to $\$ 100$-for short, call this a uniform $(0,100)$ random variable. The probability function of this continuous uniform random variable has a rather simple shape that is anything but normal. It is a horizontal line with a vertical intercept equal to $1 / 100$. Unlike a normal random variable, for which outcomes close to the mean are most likely, all possible outcomes are equally likely for a uniform random variable.

To illustrate the power of the central limit theorem, we conduct a Monte Carlo simulation to study the capital expenditure plans of telecom businesses. In this simulation, we collect 200 random samples of the capital expenditures of 100 companies (200 random draws, each consisting of the capital expenditures of 100 companies with $n=100$ ). In each simulation trial, 100 values for capital expenditure are generated from the uniform $(0,100)$ distribution. For each random sample, we then compute the sample mean. We conduct 200 simulation trials in total. Because we have specified the continuous random distribution generating the samples, we know that the population mean capital expenditure is equal to $(\$ 0+\$ 100$ million)/2 $=\$ 50$ million (i.e., $\mu=(a+b) / 2)$; the population variance of capital expenditures is equal to $(100-0)^{2} / 12=833.33$ (i.e., $\sigma^{2}$ $\left.=(b-a)^{2} / 12\right)$; thus, the standard deviation is $\$ 28.87$ million and the standard error is $28.87 / \sqrt{100}=2.887$ under the central limit theorem. The results of this Monte Carlo experiment are tabulated in Exhibit 9 in the form of a frequency distribution. This distribution is the estimated sampling distribution of the sample mean.

Exhibit 9: Frequency Distribution: 200 Random Samples of a Uniform $(0,100)$ Random Variable

\begin{center}
\begin{tabular}{lc}
$\begin{array}{l}\text { Range of Sample Means } \\ \text { (\$ million) }\end{array}$ & Absolute Frequency \\
\hline
$42.5 \leq \bar{X}<44$ & 1 \\
$44 \leq \bar{X}<45.5$ & 22 \\
$45.5 \leq \bar{X}<47$ & 39 \\
$47 \leq \bar{X}<48.5$ & 4 \\
$48.5 \leq \bar{X}<50$ & 23 \\
$50 \leq \bar{X}<51.5$ & 12 \\
$51.5 \leq \bar{X}<53$ & 12 \\
$53 \leq \bar{X}<54.5$ & 23 \\
$54.5 \leq \bar{X}<56$ & 12 \\
$56 \leq \bar{X}<57.5$ &  \\
\end{tabular}
\end{center}

Note: $\bar{X}$ is the mean capital expenditure for each sample.

The frequency distribution can be described as bell-shaped and centered close to the population mean of 50 . The most frequent, or modal, range, with 41 observations, is 48.5 to 50 . The overall average of the sample means is $\$ 49.92$, with a standard error equal to $\$ 2.80$. The calculated standard error is close to the value of 2.887 given by the central limit theorem. The discrepancy between calculated and expected values of the mean and standard deviation under the central limit theorem is a result of random chance (sampling error).

In summary, although the distribution of the underlying population is very non-normal, the simulation has shown that a normal distribution well describes the estimated sampling distribution of the sample mean, with mean and standard error consistent with the values predicted by the central limit theorem.

To summarize, according to the central limit theorem, when we sample from any distribution, the distribution of the sample mean will have the following properties as long as our sample size is large:

\begin{itemize}
  \item The distribution of the sample mean $\bar{X}$ will be approximately normal.

  \item The mean of the distribution of $\bar{X}$ will be equal to the mean of the population from which the samples are drawn.

  \item The variance of the distribution of $\bar{X}$ will be equal to the variance of the population divided by the sample size. We next discuss the concepts and tools related to estimating the population parameters, with a special focus on the population mean. We focus on the population mean because analysts are more likely to meet interval estimates for the population mean than any other type of interval estimate.

\end{itemize}

\section{POINT ESTIMATES OF THE POPULATION MEAN}
identify and describe desirable properties of an estimator

Statistical inference traditionally consists of two branches, hypothesis testing and estimation. Hypothesis testing addresses the question "Is the value of this parameter (say, a population mean) equal to some specific value (0, for example)?" In this process, we have a hypothesis concerning the value of a parameter, and we seek to determine whether the evidence from a sample supports or does not support that hypothesis. The topic of hypothesis testing will be discussed later.

The second branch of statistical inference, and what we focus on now, is estimation. Estimation seeks an answer to the question "What is this parameter's (for example, the population mean's) value?" In estimating, unlike in hypothesis testing, we do not start with a hypothesis about a parameter's value and seek to test it. Rather, we try to make the best use of the information in a sample to form one of several types of estimates of the parameter's value. With estimation, we are interested in arriving at a rule for best calculating a single number to estimate the unknown population parameter (a point estimate). In addition to calculating a point estimate, we may also be interested in calculating a range of values that brackets the unknown population parameter with some specified level of probability (a confidence interval). We first discuss point estimates of parameters and then turn our attention to the formulation of confidence intervals for the population mean.

\section{Point Estimators}
An important concept introduced here is that sample statistics viewed as formulas involving random outcomes are random variables. The formulas that we use to compute the sample mean and all the other sample statistics are examples of estimation formulas or estimators. The particular value that we calculate from sample observations using an estimator is called an estimate. An estimator has a sampling distribution; an estimate is a fixed number pertaining to a given sample and thus has no sampling distribution. To take the example of the mean, the calculated value of the sample mean in a given sample, used as an estimate of the population mean, is called a point estimate of the population mean. As we have seen earlier, the formula for the sample mean can and will yield different results in repeated samples as different samples are drawn from the population.

In many applications, we have a choice among a number of possible estimators for estimating a given parameter. How do we make our choice? We often select estimators because they have one or more desirable statistical properties. Following is a brief description of three desirable properties of estimators: unbiasedness (lack of bias), efficiency, and consistency.

\begin{itemize}
  \item Unbiasedness. An unbiased estimator is one whose expected value (the mean of its sampling distribution) equals the parameter it is intended to estimate. For example, as shown in Exhibit 10 of the sampling distribution of the sample mean, the expected value of the sample mean, $\bar{X}$, equals $\mu$, the population mean, so we say that the sample mean is an unbiased estimator (of the population mean). The sample variance, $s^{2}$, calculated using a divisor of $n-1$ (Equation 3), is an unbiased estimator of the population variance, $\sigma^{2}$. If we were to calculate the sample variance using a divisor of $n$, the estimator would be biased: Its expected value would be smaller than the population variance. We would say that sample variance calculated with a divisor of $n$ is a biased estimator of the population variance.
\end{itemize}

\section{Exhibit 10: Unbiasedness of an Estimator}
\begin{center}
\includegraphics[max width=\textwidth]{2023_05_04_cff39ee44f77d6514e1bg-331}
\end{center}

Whenever one unbiased estimator of a parameter can be found, we can usually find a large number of other unbiased estimators. How do we choose among alternative unbiased estimators? The criterion of efficiency provides a way to select from among unbiased estimators of a parameter.

\begin{itemize}
  \item Efficiency. An unbiased estimator is efficient if no other unbiased estimator of the same parameter has a sampling distribution with smaller variance.
\end{itemize}

To explain the definition, in repeated samples we expect the estimates from an efficient estimator to be more tightly grouped around the mean than estimates from other unbiased estimators. For example, Exhibit 11 shows the sampling distributions of two different estimators of the population mean. Both estimators $A$ and $B$ are unbiased because their expected values are equal to the population mean $\left(\bar{X}_{A}=\bar{X}_{B}=\mu\right)$, but estimator $A$ is more efficient because it shows smaller variance. Efficiency is an important property of an estimator. Sample mean $\bar{X}$ is an efficient estimator of the population mean; sample variance $s^{2}$ is an efficient estimator of $\sigma^{2}$.

\section{Exhibit 11: Efficiency of an Estimator}
\begin{center}
\includegraphics[max width=\textwidth]{2023_05_04_cff39ee44f77d6514e1bg-332}
\end{center}

Recall that a statistic's sampling distribution is defined for a given sample size. Different sample sizes define different sampling distributions. For example, the variance of sampling distribution of the sample mean is smaller for larger sample sizes. Unbiasedness and efficiency are properties of an estimator's sampling distribution that hold for any size sample. An unbiased estimator is unbiased equally in a sample of size 100 and in a sample of size 1,000. In some problems, however, we cannot find estimators that have such desirable properties as unbiasedness in small samples. In this case, statisticians may justify the choice of an estimator based on the properties of the estimator's sampling distribution in extremely large samples, the estimator's so-called asymptotic properties. Among such properties, the most important is consistency.

\begin{itemize}
  \item Consistency. A consistent estimator is one for which the probability of estimates close to the value of the population parameter increases as sample size increases.
\end{itemize}

Somewhat more technically, we can define a consistent estimator as an estimator whose sampling distribution becomes concentrated on the value of the parameter it is intended to estimate as the sample size approaches infinity. The sample mean, in addition to being an efficient estimator, is also a consistent estimator of the population mean: As sample size $n$ goes to infinity, its standard error, $\sigma / \sqrt{n}$, goes to 0 and its sampling distribution becomes concentrated right over the value of population mean, $\mu$. Exhibit 12 illustrates the consistency of the sample mean, in which the standard error of the estimator narrows as the sample size increases. To summarize, we can think of a consistent estimator as one that tends to produce more and more accurate estimates of the population parameter as we increase the sample's size. If an estimator is consistent, we may attempt to increase the accuracy of estimates of a population parameter by calculating estimates using a larger sample. For an inconsistent estimator, however, increasing sample size does not help to increase the probability of accurate estimates.

\section{Exhibit 12: Consistency of an Estimator}
\begin{center}
\includegraphics[max width=\textwidth]{2023_05_04_cff39ee44f77d6514e1bg-333(1)}
\end{center}

It is worth noting that in a Big Data world, consistency is much more crucial than efficiency, because the accuracy of a population parameter's estimates can be increasingly improved with the availability of more sample data. In addition, given a big dataset, a biased but consistent estimator can offer considerably reduced error. For example, $s^{2} / n$ is a biased estimator of variance. As $n$ goes to infinity, the distinction between $s^{2} / n$ and the unbiased estimator $s^{2} /(n-1)$ diminish to zero.

\section{EXAMPLE 7}
\section{Exhibit 13: Sampling Distributions of an Estimator}
\begin{center}
\includegraphics[max width=\textwidth]{2023_05_04_cff39ee44f77d6514e1bg-333}
\end{center}

\begin{enumerate}
  \item Exhibit 13 plots several sampling distributions of an estimator for the population mean, and the vertical dash line represents the true value of population mean.
\end{enumerate}

Which of the following statements best describes the estimator's properties?

A. The estimator is unbiased. B. The estimator is biased and inconsistent.

C. The estimator is biased but consistent.

Solution:

$\mathrm{C}$ is correct. The chart shows three sampling distributions of the estimator at different sample sizes $(n=50,200$, and 1,000). We can observe that the means of each sampling distribution-that is, the expected value of the estimator-deviates from the population mean, so the estimator is biased. As the sample size increases, however, the mean of the sampling distribution draws closer to the population mean with smaller variance. So, it is a consistent estimator.

\section*{CONFIDENCE INTERVALS FOR THE POPULATION MEAN AND SAMPLE SIZE SELECTION }
When we need a single number as an estimate of a population parameter, we make use of a point estimate. However, because of sampling error, the point estimate is not likely to equal the population parameter in any given sample. Often, a more useful approach than finding a point estimate is to find a range of values that we expect to bracket the parameter with a specified level of probability-an interval estimate of the parameter. A confidence interval fulfills this role.

\begin{itemize}
  \item Definition of Confidence Interval. A confidence interval is a range for which one can assert with a given probability $1-\alpha$, called the degree of confidence, that it will contain the parameter it is intended to estimate. This interval is often referred to as the $100(1-\alpha) \%$ confidence interval for the parameter.
\end{itemize}

The endpoints of a confidence interval are referred to as the lower and upper confidence limits. In this reading, we are concerned only with two-sided confidence intervals-confidence intervals for which we calculate both lower and upper limits.

Confidence intervals are frequently given either a probabilistic interpretation or a practical interpretation. In the probabilistic interpretation, we interpret a 95\% confidence interval for the population mean as follows. In repeated sampling, 95\% of such confidence intervals will, in the long run, include or bracket the population mean. For example, suppose we sample from the population 1,000 times, and based on each sample, we construct a $95 \%$ confidence interval using the calculated sample mean. Because of random chance, these confidence intervals will vary from each other, but we expect $95 \%$, or 950, of these intervals to include the unknown value of the population mean. In practice, we generally do not carry out such repeated sampling. Therefore, in the practical interpretation, we assert that we are $95 \%$ confident that a single $95 \%$ confidence interval contains the population mean. We are justified in making this statement because we know that $95 \%$ of all possible confidence intervals constructed in the same manner will contain the population mean. The confidence intervals that we discuss in this reading have structures similar to the following basic structure:

\begin{itemize}
  \item Construction of Confidence Intervals. A $100(1-\alpha) \%$ confidence interval for a parameter has the following structure:
\end{itemize}

Point estimate \textbackslash pm Reliability factor $\times$ Standard error

where

Point estimate $=$ a point estimate of the parameter (a value of a sample statistic)

Reliability factor $=$ a number based on the assumed distribution of the point estimate and the degree of confidence $(1-\alpha)$ for the confidence interval Standard error $=$ the standard error of the sample statistic providing the point estimate

The quantity "Reliability factor $\times$ Standard error" is sometimes called the precision of the estimator; larger values of the product imply lower precision in estimating the population parameter.

The most basic confidence interval for the population mean arises when we are sampling from a normal distribution with known variance. The reliability factor in this case is based on the standard normal distribution, which has a mean of 0 and a variance of 1 . A standard normal random variable is conventionally denoted by $Z$. The notation $z_{\alpha}$ denotes the point of the standard normal distribution such that $\alpha$ of the probability remains in the right tail. For example, 0.05 or $5 \%$ of the possible values of a standard normal random variable are larger than $z_{0.05}=1.65$. Similarly, 0.025 or $2.5 \%$ of the possible values of a standard normal random variable are larger than $z_{0.025}=1.96$.

Suppose we want to construct a $95 \%$ confidence interval for the population mean and, for this purpose, we have taken a sample of size 100 from a normally distributed population with known variance of $\sigma^{2}=400$ (so, $\sigma=20$ ). We calculate a sample mean of $\bar{X}=25$. Our point estimate of the population mean is, therefore, 25 . If we move 1.96 standard deviations above the mean of a normal distribution, 0.025 or $2.5 \%$ of the probability remains in the right tail; by symmetry of the normal distribution, if we move 1.96 standard deviations below the mean, 0.025 or $2.5 \%$ of the probability remains in the left tail. In total, 0.05 or $5 \%$ of the probability is in the two tails and 0.95 or $95 \%$ lies in between. So, $z_{0.025}=1.96$ is the reliability factor for this $95 \%$ confidence interval. Note the relationship $100(1-\alpha) \%$ for the confidence interval and the $z_{\alpha / 2}$ for the reliability factor. The standard error of the sample mean, given by Equation 1, is $\sigma_{\bar{X}}=20 / \sqrt{100}=2$. The confidence interval, therefore, has a lower limit of $\bar{X}-1.96$ $\sigma_{\bar{X}}=25-1.96(2)=25-3.92=21.08$. The upper limit of the confidence interval is $\bar{X}+1.96 \sigma_{\bar{X}}=25+1.96(2)=25+3.92=28.92$. The $95 \%$ confidence interval for the population mean spans 21.08 to 28.92 .

\begin{itemize}
  \item Confidence Intervals for the Population Mean (Normally Distributed Population with Known Variance). A 100(1- $\alpha) \%$ confidence interval for population mean $\mu$ when we are sampling from a normal distribution with known variance $\sigma^{2}$ is given by
\end{itemize}

$$
\bar{X} \pm z_{\alpha / 2} \frac{\sigma}{\sqrt{n}}
$$

The reliability factors for the most frequently used confidence intervals are as follows.

\begin{itemize}
  \item Reliability Factors for Confidence Intervals Based on the Standard Normal Distribution. We use the following reliability factors when we construct confidence intervals based on the standard normal distribution:

  \item $90 \%$ confidence intervals: Use $z_{0.05}=1.65$

  \item $95 \%$ confidence intervals: Use $z_{0.025}=1.96$

  \item $99 \%$ confidence intervals: Use $z_{0.005}=2.58$

\end{itemize}

These reliability factors highlight an important fact about all confidence intervals. As we increase the degree of confidence, the confidence interval becomes wider and gives us less precise information about the quantity we want to estimate.

Exhibit 14 demonstrates how a confidence interval works. We again use the daily returns of the fictitious Euro-Asia-Africa Equity Index shown earlier. The dataset consists of 1,258 observations with a population mean of $0.035 \%$ and a population standard deviation of $0.834 \%$. We conduct random sampling from the population 1,000 times, drawing a sample of a hundred daily returns $(n=100)$ each time.

We construct a histogram of the sample means, shown in Exhibit 14. The shape appears to be that of a normal distribution, in line with the central limit theorem. We next pick one random sample to construct confidence intervals around its sample mean. The mean of the selected sample is computed to be $0.103 \%$ (as plotted with a solid line). Next we construct $99 \%, 95 \%$, and $50 \%$ confidence intervals around that sample mean. We use Equation 4, to compute the upper and lower bounds of each pair of confidence intervals and plot these bounds in dashed lines.

The resulting chart shows that confidence intervals narrow with decreasing confidence level, and vice versa. For example, the narrowest confidence interval in the chart corresponds to the lowest confidence level of $50 \%$-that is, we are only $50 \%$ confident that the population mean falls within the $50 \%$ confidence interval around the sample mean. Importantly, as shown by Equation 4, given a fixed confidence level, the confidence interval narrows with smaller population deviation and greater sample size, indicating higher estimate accuracy.

\section{Exhibit 14: Illustration of Confidence Intervals}
\begin{center}
\includegraphics[max width=\textwidth]{2023_05_04_cff39ee44f77d6514e1bg-337}
\end{center}

In practice, the assumption that the sampling distribution of the sample mean is at least approximately normal is frequently reasonable, either because the underlying distribution is approximately normal or because we have a large sample and the central limit theorem applies. Rarely do we know the population variance in practice, however. When the population variance is unknown but the sample mean is at least approximately normally distributed, we have two acceptable ways to calculate the confidence interval for the population mean. We will soon discuss the more conservative approach, which is based on Student's $t$-distribution (the $t$-distribution, for short and covered earlier). In investment literature, it is the most frequently used approach in both estimation and hypothesis tests concerning the mean when the population variance is not known, whether sample size is small or large.

A second approach to confidence intervals for the population mean, based on the standard normal distribution, is the $z$-alternative. It can be used only when sample size is large. (In general, a sample size of 30 or larger may be considered large.) In contrast to the confidence interval given in Equation 4, this confidence interval uses the sample standard deviation, $s$, in computing the standard error of the sample mean (Equation 2).

\begin{itemize}
  \item Confidence Intervals for the Population Mean-The $z$-Alternative (Large Sample, Population Variance Unknown). A 100(1- $\alpha) \%$ confidence interval for population mean $\mu$ when sampling from any distribution with unknown variance and when sample size is large is given by
\end{itemize}

$$
\bar{X} \pm z_{\alpha / 2} \frac{s}{\sqrt{n}}
$$

Because this type of confidence interval appears quite often, we illustrate its calculation in Example 8.

\section{EXAMPLE 8}
\section{Confidence Interval for the Population Mean of Sharpe Ratios-z-Statistic}
\begin{enumerate}
  \item Suppose an investment analyst takes a random sample of US equity mutual funds and calculates the average Sharpe ratio. The sample size is 100 , and the average Sharpe ratio is 0.45 . The sample has a standard deviation of 0.30 . Calculate and interpret the $90 \%$ confidence interval for the population mean of all US equity mutual funds by using a reliability factor based on the standard normal distribution.
\end{enumerate}

The reliability factor for a $90 \%$ confidence interval, as given earlier, is $z_{0.05}=$ 1.65. The confidence interval will be

$$
\bar{X} \pm z_{0.05} \frac{s}{\sqrt{n}}=0.45 \pm 1.65 \frac{0.30}{\sqrt{100}}=0.45 \pm 1.65(0.03)=0.45 \pm 0.0495
$$

The confidence interval spans 0.4005 to 0.4995 , or 0.40 to 0.50 , carrying two decimal places. The analyst can say with $90 \%$ confidence that the interval includes the population mean.

In this example, the analyst makes no specific assumption about the probability distribution describing the population. Rather, the analyst relies on the central limit theorem to produce an approximate normal distribution for the sample mean.

As Example 8 shows, even if we are unsure of the underlying population distribution, we can still construct confidence intervals for the population mean as long as the sample size is large because we can apply the central limit theorem.

We now turn to the conservative alternative, using the $t$-distribution, for constructing confidence intervals for the population mean when the population variance is not known. For confidence intervals based on samples from normally distributed populations with unknown variance, the theoretically correct reliability factor is based on the $t$-distribution. Using a reliability factor based on the $t$-distribution is essential for a small sample size. Using a $t$ reliability factor is appropriate when the population variance is unknown, even when we have a large sample and could use the central limit theorem to justify using a $z$ reliability factor. In this large sample case, the $t$-distribution provides more-conservative (wider) confidence intervals.

Suppose we sample from a normal distribution. The ratio $z=(\bar{X}-\mu) /(\sigma / \sqrt{n})$ is distributed normally with a mean of 0 and standard deviation of 1 ; however, the ratio $t=(\bar{X}-\mu) /(s / \sqrt{n})$ follows the $t$-distribution with a mean of 0 and $n-1$ degrees of freedom. The ratio represented by $t$ is not normal because $t$ is the ratio of two random variables, the sample mean and the sample standard deviation. The definition of the standard normal random variable involves only one random variable, the sample mean.

Values for the $t$-distribution are available from Excel, using the function T.INV(p,DF). For each degree of freedom, five values are given: $t_{0.10}, t_{0.05}, t_{0.025}, t_{0.01}$, and $t_{0.005}$. The values for $t_{0.10}, t_{0.05}, t_{0.025}, t_{0.01}$, and $t_{0.005}$ are such that, respectively, $0.10,0.05,0.025$, 0.01 , and 0.005 of the probability remains in the right tail, for the specified number of degrees of freedom. For example, for $\mathrm{df}=30, t_{0.10}=1.310, t_{0.05}=1.697, t_{0.025}=$ $2.042, t_{0.01}=2.457$, and $t_{0.005}=2.750$. We now give the form of confidence intervals for the population mean using the $t$-distribution.

\begin{itemize}
  \item Confidence Intervals for the Population Mean (Population Variance Unknown) $\boldsymbol{t} \boldsymbol{t}$-Distribution. If we are sampling from a population with unknown variance and either of the conditions below holds:

  \item the sample is large, or

  \item the sample is small, but the population is normally distributed, or approximately normally distributed,

\end{itemize}

then a $100(1-\alpha) \%$ confidence interval for the population mean $\mu$ is given by

$$
\bar{X} \pm t_{\alpha / 2} \frac{s}{\sqrt{n}}
$$

where the number of degrees of freedom for $t_{\alpha / 2}$ is $n-1$ and $n$ is the sample size.

Example 9 reprises the data of Example 8 but uses the $t$-statistic rather than the $z$-statistic to calculate a confidence interval for the population mean of Sharpe ratios.

\section{EXAMPLE 9}
\section{Confidence Interval for the Population Mean of Sharpe Ratios-t-Statistic}
As in Example 8, an investment analyst seeks to calculate a $90 \%$ confidence interval for the population mean Sharpe ratio of US equity mutual funds based on a random sample of 100 US equity mutual funds. The sample mean Sharpe ratio is 0.45 , and the sample standard deviation of the Sharpe ratios is 0.30 . Now recognizing that the population variance of the distribution of Sharpe ratios is unknown, the analyst decides to calculate the confidence interval using the theoretically correct $t$-statistic.

Because the sample size is $100, \mathrm{df}=99$. Using the Excel function $\operatorname{T.INV}(0.05,99)$, $t_{0.05}=1.66$. This reliability factor is slightly larger than the reliability factor $z_{0.05}$ $=1.65$ that was used in Example 8. The confidence interval will be

$$
\bar{X} \pm t_{0.05} \frac{s}{\sqrt{n}}=0.45 \pm 1.66 \frac{0.30}{\sqrt{100}}=0.45 \pm 1.66(0.03)=0.45 \pm 0.0498 .
$$

The confidence interval spans 0.4002 to 0.4998 , or 0.40 to 0.50 , carrying two decimal places. To two decimal places, the confidence interval is unchanged from the one computed in Example 8.

Exhibit 15 summarizes the various reliability factors that we have used.

\section{Exhibit 15: Basis of Computing Reliability Factors}
\begin{center}
\begin{tabular}{lcc}
\hline
Sampling from & $\begin{array}{c}\text { Statistic for Small } \\ \text { Sample Size }\end{array}$ & $\begin{array}{c}\text { Statistic for Large } \\ \text { Sample Size }\end{array}$ \\
\hline
$\begin{array}{l}\text { Normal distribution with known } \\ \text { variance }\end{array}$ & $z$ & $z$ \\
$\begin{array}{l}\text { Normal distribution with } \\ \text { unknown variance }\end{array}$ & $t$ &  \\
$\begin{array}{l}\text { Non-normal distribution with } \\ \text { known variance }\end{array}$ & not available &  \\
\end{tabular}
\end{center}

\begin{center}
\begin{tabular}{lcc}
\hline
Sampling from & $\begin{array}{c}\text { Statistic for Small } \\ \text { Sample Size }\end{array}$ & $\begin{array}{c}\text { Statistic for Large } \\ \text { Sample Size }\end{array}$ \\
\hline
$\begin{array}{l}\text { Non-normal distribution with } \\ \text { unknown variance }\end{array}$ & not available & $t^{*}$ \\
\hline
\end{tabular}
\end{center}

*Use of $z$ also acceptable.

Exhibit 16 shows a flowchart that helps determine what statistics should be used to produce confidence intervals under different conditions.

\section{Exhibit 16: Determining Statistics for Confidence Intervals}
\begin{center}
\includegraphics[max width=\textwidth]{2023_05_04_cff39ee44f77d6514e1bg-340}
\end{center}

\section{Selection of Sample Size}
What choices affect the width of a confidence interval? To this point we have discussed two factors that affect width: the choice of statistic $(t$ or $z)$ and the choice of degree of confidence (affecting which specific value of $t$ or $z$ we use). These two choices determine the reliability factor. (Recall that a confidence interval has the structure Point estimate \textbackslash pm Reliability factor $\times$ Standard error.)

The choice of sample size also affects the width of a confidence interval. All else equal, a larger sample size decreases the width of a confidence interval. Recall the expression for the standard error of the sample mean:

Standard error of the sample mean $=\frac{\text { Sample standard deviation }}{\sqrt{\text { Sample size }}}$

We see that the standard error varies inversely with the square root of sample size. As we increase sample size, the standard error decreases and consequently the width of the confidence interval also decreases. The larger the sample size, the greater precision with which we can estimate the population parameter.

At a given degree of confidence $(1-\alpha)$, we can determine the sample size needed to obtain a desired width for a confidence interval. Define $E=$ Reliability factor $\mathrm{x}$ Standard error; then $2 E$ is the confidence interval's width. The smaller $E$ is, the smaller the width of the confidence interval. Accordingly, the sample size to obtain a desired value of $E$ at a given degree of confidence $(1-\alpha)$ can be derived as $n=[(t \times \mathrm{s}) / E]^{2}$. It is worth noting that appropriate sample size is also needed for performing a valid power analysis and determining the minimum detectable effect in hypothesis testing, concepts that will be covered at a later stage.

All else equal, larger samples are good, in that sense. In practice, however, two considerations may operate against increasing sample size. First, as we saw earlier concerning the Sharpe ratio, increasing the size of a sample may result in sampling from more than one population. Second, increasing sample size may involve additional expenses that outweigh the value of additional precision. Thus three issues that the analyst should weigh in selecting sample size are the need for precision, the risk of sampling from more than one population, and the expenses of different sample sizes.

\section{EXAMPLE 10}
\section{A Money Manager Estimates Net Client Inflows}
A money manager wants to obtain a 95\% confidence interval for fund inflows and outflows over the next six months for his existing clients. He begins by calling a random sample of 10 clients and inquiring about their planned additions to and withdrawals from the fund. The manager then computes the change in cash flow for each client sampled as a percentage change in total funds placed with the manager. A positive percentage change indicates a net cash inflow to the client's account, and a negative percentage change indicates a net cash outflow from the client's account. The manager weights each response by the relative size of the account within the sample and then computes a weighted average.

As a result of this process, the money manager computes a weighted average of $5.5 \%$. Thus, a point estimate is that the total amount of funds under management will increase by $5.5 \%$ in the next six months. The standard deviation of the observations in the sample is $10 \%$. A histogram of past data looks fairly close to normal, so the manager assumes the population is normal.

\begin{enumerate}
  \item Calculate a $95 \%$ confidence interval for the population mean and interpret your findings.
\end{enumerate}

\section{Solution to 1:}
Because the population variance is unknown and the sample size is small, the manager must use the $t$-statistic in Equation 6 to calculate the confidence interval. Based on the sample size of $10, \mathrm{df}=n-1=10-1=9$. For a 95\% confidence interval, he needs to use the value of $t_{0.025}$ for $\mathrm{df}=9$. This value is 2.262, using Excel function T.INV(0.025,9). Therefore, a 95\% confidence interval for the population mean is

$$
\begin{aligned}
& \bar{X} \pm t_{0.025} \frac{s}{\sqrt{n}}=5.5 \% \pm 2.262 \frac{10 \%}{\sqrt{10}} \\
& =5.5 \% \pm 2.262(3.162) \\
& =5.5 \% \pm 7.15 \%
\end{aligned}
$$

The confidence interval for the population mean spans $-1.65 \%$ to $+12.65 \%$. The manager can be confident at the $95 \%$ level that this range includes the population mean. 2. The manager decides to see what the confidence interval would look like if he had used a sample size of 20 or 30 and found the same mean (5.5\%) and standard deviation (10\%).

Compute the confidence interval for sample sizes of 20 and 30. For the sample size of 30, use Equation 6 .

\section{Solution to 2:}
Exhibit 17 gives the calculations for the three sample sizes.

Exhibit 17: The 95\% Confidence Interval for Three Sample Sizes

\begin{center}
\begin{tabular}{lcccc}
\hline
Distribution & $\begin{array}{c}\mathbf{9 5 \%} \\ \text { Confidence Interval }\end{array}$ & $\begin{array}{c}\text { Lower } \\ \text { Bound }\end{array}$ & $\begin{array}{c}\text { Upper } \\ \text { Bound }\end{array}$ & $\begin{array}{c}\text { Relative } \\ \text { Size }\end{array}$ \\
\hline
$t(n=10)$ & $5.5 \% \pm 2.262(3.162)$ & $-1.65 \%$ & $12.65 \%$ & $100.0 \%$ \\
$t(n=20)$ & $5.5 \% \pm 2.093(2.236)$ & 0.82 & 10.18 & 65.5 \\
$t(n=30)$ & $5.5 \% \pm 2.045(1.826)$ & 1.77 & 9.23 & 52.2 \\
\hline
\end{tabular}
\end{center}

\begin{enumerate}
  \setcounter{enumi}{2}
  \item Interpret your results from Parts 1 and 2.
\end{enumerate}

\section{Solution to 3:}
The width of the confidence interval decreases as we increase the sample size. This decrease is a function of the standard error becoming smaller as $n$ increases. The reliability factor also becomes smaller as the number of degrees of freedom increases. The last column of Exhibit 17 shows the relative size of the width of confidence intervals based on $n=10$ to be $100 \%$. Using a sample size of 20 reduces the confidence interval's width to $65.5 \%$ of the interval width for a sample size of 10 . Using a sample size of 30 cuts the width of the interval almost in half. Comparing these choices, the money manager would obtain the most precise results using a sample of 30 .

\section{RESAMPLING}
describe the use of resampling (bootstrap, jackknife) to estimate the sampling distribution of a statistic

Earlier, we demonstrated how to find the standard error of the sample mean, which can be computed using Equation 4 based on the central limit theorem. We now introduce a computational tool called resampling, which repeatedly draws samples from the original observed data sample for the statistical inference of population parameters. Bootstrap, one of the most popular resampling methods, uses computer simulation for statistical inference without using an analytical formula such as a $z$-statistic or $t$-statistic.

The idea behind bootstrap is to mimic the process of performing random sampling from a population, similar to what we have shown earlier, to construct the sampling distribution of the sample mean. The difference lies in the fact that we have no knowledge of what the population looks like, except for a sample with size $n$ drawn from the population. Because a random sample offers a good representation of the population, we can simulate sampling from the population by sampling from the observed sample. In other words, the bootstrap mimics the process by treating the randomly drawn sample as if it were the population.

The right-hand side of Exhibit 18 illustrates the process. In bootstrap, we repeatedly draw samples from the original sample, and each resample is of the same size as the original sample. Note that each item drawn is replaced for the next draw (i.e., the identical element is put back into the group so that it can be drawn more than once). Assuming we are looking to find the standard error of sample mean, we take many resamples and then compute the mean of each resample. Note that although some items may appear several times in the resamples, other items may not appear at all.

\section{Exhibit 18: Bootstrap Resampling}
\begin{center}
\includegraphics[max width=\textwidth]{2023_05_04_cff39ee44f77d6514e1bg-343}
\end{center}

Subsequently, we construct a sampling distribution with these resamples. The bootstrap sampling distribution (right-hand side of Exhibit 18) will approximate the true sampling distribution. We estimate the standard error of the sample mean using Equation 7. Note that to distinguish the foregoing resampling process from other types of resampling, it is often called model-free resampling or non-parametric resampling.

$$
s_{\bar{X}}=\sqrt{\frac{1}{B-1} \sum_{b=1}^{B}\left(\hat{\theta}_{b}-\bar{\theta}\right)^{2}}
$$

where

$S_{\bar{X}}$ is the estimate of the standard error of the sample mean

$B$ denotes the number of resamples drawn from the original sample.

$\hat{\theta}_{b}$ denotes the mean of a resample

$\bar{\theta}$ denotes the mean across all the resample means

Bootstrap is one of the most powerful and widely used tools for statistical inference. As we have explained, it can be used to estimate the standard error of sample mean. Similarly, bootstrap can be used to find the standard error or construct confidence intervals for the statistic of other population parameters, such as the median, which does not apply to the previously discussed methodologies. Compared with conventional statistical methods, bootstrap does not rely on an analytical formula to estimate the distribution of the estimators. It is a simple but powerful method for any complicated estimators and particularly useful when no analytical formula is available. In addition, bootstrap has potential advantages in accuracy. Given these advantages, bootstrap can be applied widely in finance, such as for historical simulations in asset allocation or in gauging an investment strategy's performance against a benchmark.

\section{EXAMPLE 11}
\section{Bootstrap Resampling Illustration}
The following table displays a set of 12 monthly returns of a rarely traded stock, shown in Column A. Our aim is to calculate the standard error of the sample mean. Using the bootstrap resampling method, a series of bootstrap samples, labelled as "resamples" (with replacement) are drawn from the sample of 12 returns. Notice how some of the returns from data sample in Column A feature more than once in some of the resamples (for example, 0.055 features twice in Resample 1).

\begin{center}
\begin{tabular}{|c|c|c|c|c|c|}
\hline
Column A & $\begin{array}{c}\text { Resample } \\ 1\end{array}$ & $\begin{array}{c}\text { Resample } \\ 2\end{array}$ & $\begin{array}{c}\text { Resample } \\ 3\end{array}$ &  & $\begin{array}{c}\text { Resample } \\ 1,000\end{array}$ \\
\hline
-0.096 & 0.055 & -0.096 & -0.033 & $\ldots .$. & -0.072 \\
\hline
-0.132 & -0.033 & 0.055 & -0.132 & $\ldots$. & 0.255 \\
\hline
-0.191 & 0.255 & 0.055 & -0.157 & $\ldots .$. & 0.055 \\
\hline
-0.096 & -0.033 & -0.157 & 0.255 & $\ldots$. & 0.296 \\
\hline
0.055 & 0.255 & -0.096 & -0.132 & $\ldots .$. & 0.055 \\
\hline
-0.053 & -0.157 & -0.053 & -0.191 & $\ldots .$. & -0.096 \\
\hline
-0.033 & -0.053 & -0.096 & 0.055 & $\ldots$. & 0.296 \\
\hline
0.296 & -0.191 & -0.132 & 0.255 & $\ldots .$. & -0.132 \\
\hline
0.055 & -0.132 & -0.132 & 0.296 & $\ldots .$. & 0.055 \\
\hline
-0.072 & -0.096 & 0.055 & -0.096 & $\ldots$. & -0.096 \\
\hline
0.255 & 0.055 & -0.072 & 0.055 & $\ldots .$. & -0.191 \\
\hline
-0.157 & -0.157 & -0.053 & -0.157 & $\ldots .$. & 0.055 \\
\hline
$\begin{array}{l}\text { Sample } \\ \text { mean }\end{array}$ & -0.019 & -0.060 & 0.001 & $\ldots$. & 0.040 \\
\hline
\end{tabular}
\end{center}

Drawing 1,000 such samples, we obtain 1,000 sample means. The mean across all resample means is -0.01367 . The sum of squares of the differences between each sample mean and the mean across all resample means $\left(\sum_{b=1}^{B}\left(\hat{\theta}_{b}-\bar{\theta}\right)^{2}\right)$ is 1.94143 . Using Equation 7, we calculate an estimate of the standard error of the sample mean:

$$
s_{\bar{X}}=\sqrt{\frac{1}{B-1} \sum_{b=1}^{B}\left(\hat{\theta}_{b}-\bar{\theta}\right)^{2}}=\sqrt{\frac{1}{999} \times 1.94143}=0.04408
$$

Jackknife is another resampling technique for statistical inference of population parameters. Unlike bootstrap, which repeatedly draws samples with replacement, jackknife samples are selected by taking the original observed data sample and leaving out one observation at a time from the set (and not replacing it). Jackknife method is often used to reduce the bias of an estimator, and other applications include finding the standard error and confidence interval of an estimator. According to its computation procedure, we can conclude that jackknife produces similar results for every run, whereas bootstrap usually gives different results because bootstrap resamples are randomly drawn. For a sample of size $n$, jackknife usually requires $n$ repetitions, whereas with bootstrap, we are left to determine how many repetitions are appropriate.

\section{EXAMPLE 12}
An analyst in a real estate investment company is researching the housing market of the Greater Boston area. From a sample of collected house sale price data in the past year, she estimates the median house price of the area. To find the standard error of the estimated median, she is considering two options:

option 1 The standard error of the sample median can be given by $\frac{s}{\sqrt{n}}$, where $s$ denotes the sample standard deviation and $n$ denotes the sample size.

option 2 Apply the bootstrap method to construct the sampling distribution of the sample median, and then compute the standard error by using Equation 7 .

\begin{enumerate}
  \item Which of the following statements is accurate?
\end{enumerate}

A. Option 1 is suitable to find the standard error of the sample median.

B. Option 2 is suitable to find the standard error of the sample median.

C. Both options are suitable to find the standard error of the sample median.

\section{Solution:}
B is correct. Option 1 is valid for estimating the standard error of the sample mean but not for that of the sample median, which is not based on the given formula. Thus, both $\mathrm{A}$ and $\mathrm{C}$ are incorrect. The bootstrap method is a simple way to find the standard error of an estimator even if no analytical formula is available or it is too complicated.

Having covered many of the fundamental concepts of sampling and estimation, we are in a good position to focus on sampling issues of special concern to analysts. The quality of inferences depends on the quality of the data as well as on the quality of the sampling plan used. Financial data pose special problems, and sampling plans frequently reflect one or more biases. The next section discusses these issues.

\section{SAMPLING RELATED BIASES}
describe the issues regarding selection of the appropriate sample size, data snooping bias, sample selection bias, survivorship bias, look-ahead bias, and time-period bias

We have already seen that the selection of sample period length may raise the issue of sampling from more than one population. There are, in fact, a range of challenges to valid sampling that arise in working with financial data. In this section we discuss several such sampling-related issues: data snooping bias, sample selection group of biases (including survivorship bias), look-ahead bias, and time-period bias. All of these issues are important for point and interval estimation and hypothesis testing. As we will see, if the sample is biased in any way, then point and interval estimates and any other conclusions that we draw from the sample will be in error.

\section{Data Snooping Bias}
Data snooping relates to overuse of the same or related data in ways that we shall describe shortly. Data snooping bias refers to the errors that arise from such misuse of data. Investment strategies that reflect data snooping biases are often not successful if applied in the future. Nevertheless, both investment practitioners and researchers in general have frequently engaged in data snooping. Analysts thus need to understand and guard against this problem.

Data snooping is the practice of determining a model by extensive searching through a dataset for statistically significant patterns (that is, repeatedly "drilling" in the same data until finding something that appears to work). In exercises involving statistical significance, we set a significance level, which is the probability of rejecting the hypothesis we are testing when the hypothesis is in fact correct. Because rejecting a true hypothesis is undesirable, the investigator often sets the significance level at a relatively small number, such as 0.05 or $5 \%$.

Suppose we test the hypothesis that a variable does not predict stock returns, and we test in turn 100 different variables. Let us also suppose that in truth, none of the 100 variables has the ability to predict stock returns. Using a $5 \%$ significance level in our tests, we would still expect that 5 out of 100 variables would appear to be significant predictors of stock returns because of random chance alone. We have mined the data to find some apparently significant variables. In essence, we have explored the same data again and again until we found some after-the-fact pattern or patterns in the dataset. This is the sense in which data snooping involves overuse of data. If we were to report only the significant variables without also reporting the total number of variables tested that were unsuccessful as predictors, we would be presenting a very misleading picture of our findings. Our results would appear to be far more significant than they actually were, because a series of tests such as the one just described invalidates the conventional interpretation of a given significance level (such as 5\%), according to the theory of inference. Datasets in the Big Data space are often blindly used to make statistical inferences without a proper hypothesis testing framework, which may lead to inferring higher-than-justified significance.

How can we investigate the presence of data snooping bias? Typically we can split the data into three separate datasets: the training dataset, the validation dataset, and the test dataset. The training dataset is used to build a model and fit the model parameters. The validation dataset is used to evaluate the model fit while tuning the model parameters. The test dataset is to provide an out-of-sample test to evaluate the final model fit. If a variable or investment strategy is the result of data snooping, it should generally not be significant in out-of-sample tests.

A variable or investment strategy that is statistically and economically significant in out-of-sample tests, and that has a plausible economic basis, may be the basis for a valid investment strategy. Caution is still warranted, however. The most crucial out-of-sample test is future investment success. It should be noted that if the strategy becomes known to other investors, prices may adjust so that the strategy, however well tested, does not work in the future. To summarize, the analyst should be aware that many apparently profitable investment strategies may reflect data snooping bias and thus be cautious about the future applicability of published investment research results.

\section{UNTANGLING THE EXTENT OF DATA SNOOPING}
To assess the significance of an investment strategy, we need to know how many unsuccessful strategies were tried not only by the current investigator but also by previous investigators using the same or related datasets. Much research, in practice, closely builds on what other investigators have done, and so reflects intergenerational data mining (McQueen and Thorley, 1999) that involves using information developed by previous researchers using a dataset to guide current research using the same or a related dataset. Analysts have accumulated many observations about the peculiarities of many financial datasets, and other analysts may develop models or investment strategies that will tend to be supported within a dataset based on their familiarity with the prior experience of other analysts. As a consequence, the importance of those new results may be overstated. Research has suggested that the magnitude of this type of data-mining bias may be considerable.

McQueen and Thorley (1999) explored data mining in the context of the popular Motley Fool "Foolish Four" investment strategy, a version of the Dow Dividend Strategy tuned by its developers to exhibit an even higher arithmetic mean return than the original Dow Dividend Strategy. The Foolish Four strategy claimed to show significant investment returns over 20 years starting in 1973, and its proponents claimed that the strategy should have similar returns in the future. McQueen and Thorley highlighted the data-mining issues in that research and presented two signs that can warn analysts about the potential existence of data mining:

\begin{itemize}
  \item Too much digging/too little confidence. The testing of many variables by the researcher is the "too much digging" warning sign of a data-mining problem. Although the number of variables examined may not be reported, we should look closely for verbal hints that the researcher searched over many variables. The use of terms such as "we noticed (or noted) that" or "someone noticed (or noted) that," with respect to a pattern in a dataset, should raise suspicions that the researchers were trying out variables based on their own or others' observations of the data.

  \item No story/no future. The absence of an explicit economic rationale for a variable or trading strategy is the "no story" warning sign of a data-mining problem. Without a plausible economic rationale or story for why a variable should work, the variable is unlikely to have predictive power. What if we do have a plausible economic explanation for a significant variable? McQueen and Thorley caution that a plausible economic rationale is a necessary but not a sufficient condition for a trading strategy to have value. As we mentioned earlier, if the strategy is publicized, market prices may adjust to reflect the new information as traders seek to exploit it; as a result, the strategy may no longer work.

\end{itemize}

\section{Sample Selection Bias}
When researchers look into questions of interest to analysts or portfolio managers, they may exclude certain stocks, bonds, portfolios, or periods from the analysis for various reasons-perhaps because of data availability. When data availability leads to certain assets being excluded from the analysis, we call the resulting problem sample selection bias. For example, you might sample from a database that tracks only companies currently in existence. Many mutual fund databases, for instance, provide historical information about only those funds that currently exist. Databases that report historical balance sheet and income statement information suffer from the same sort of bias as the mutual fund databases: Funds or companies that are no longer in business do not appear there. So, a study that uses these types of databases suffers from a type of sample selection bias known as survivorship bias.

The issue of survivorship bias has also been raised in relation to international indexes, particularly those representing less established markets. Some of these markets have suffered complete loss of value as a result of hyperinflation, nationalization or confiscation of industries, or market failure. Measuring the performance of markets or particular investments that survive over time will overstate returns from investing. There is, of course, no way of determining in advance which markets will fail or survive.

Survivorship bias sometimes appears when we use both stock price and accounting data. For example, many studies in finance have used the ratio of a company's market price to book equity per share (i.e., the price-to-book ratio, $\mathrm{P} / \mathrm{B}$ ) and found that $\mathrm{P} / \mathrm{B}$ is inversely related to a company's returns. P/B is also used to create many popular value and growth indexes. The "value" indexes, for example, would include companies trading on relatively low $\mathrm{P} / \mathrm{B}$. If the database that we use to collect accounting data excludes failing companies, however, a survivorship bias might result. It can be argued that failing stocks would be expected to have low returns and low P/Bs. If we exclude failing stocks, then those stocks with low P/Bs that are included in the index will have returns that are higher on average than if all stocks with low P/Bs were included. As shown in Exhibit 19, without failing stocks (shown in the bottom left part), we can fit a line with a negative slope indicating that $\mathrm{P} / \mathrm{B}$ is inversely related to a company's stock return. With all the companies included, however, the fitted line (horizontal, dotted) shows an insignificant slope coefficient.

This bias would then be responsible for some of the traditional findings of an inverse relationship between average return and P/B. Researchers should be aware of any biases potentially inherent in a sample.

\section{Exhibit 19: Survivorship Bias}
\begin{center}
\includegraphics[max width=\textwidth]{2023_05_04_cff39ee44f77d6514e1bg-348}
\end{center}

P/BV Surviving Stocks ± P/BV Failing Stocks

\section{DELISTINGS AND BIAS}
A sample can also be biased because of the removal (or delisting) of a company's stock from an exchange. For example, the Center for Research in Security Prices at the University of Chicago is a major provider of return data used in academic research. When a delisting occurs, CRSP attempts to collect returns for the delisted company. Many times, however, it cannot do so because of the difficulty involved; CRSP must simply list delisted company returns as missing. A study in the Journal of Finance by Shumway and Warther (1999) documented the bias caused by delisting for CRSP NASDAQ return data. The authors showed that delistings associated with poor company performance (e.g., bankruptcy) are missed more often than delistings associated with good or neutral company performance (e.g., merger or moving to another exchange). In addition, delistings occur more frequently for small companies.

Sample selection bias occurs even in markets where the quality and consistency of the data are quite high. Newer asset classes such as hedge funds may present even greater problems of sample selection bias. Hedge funds are a heterogeneous group of investment vehicles typically organized so as to be free from regulatory oversight. In general, hedge funds are not required to publicly disclose performance (in contrast to, say, mutual funds). Hedge funds themselves decide whether they want to be included in one of the various databases of hedge fund performance. Hedge funds with poor track records clearly may not wish to make their records public, creating a problem of self-selection bias in hedge fund databases. Further, as pointed out by Fung and Hsieh (2002), because only hedge funds with good records will volunteer to enter a database, in general, overall past hedge fund industry performance will tend to appear better than it really is. Furthermore, many hedge fund databases drop funds that go out of business, creating survivorship bias in the database. Even if the database does not drop defunct hedge funds, in the attempt to eliminate survivorship bias, the problem remains of hedge funds that stop reporting performance because of poor results or because successful funds no longer want new cash inflows. In some circumstances, implicit selection bias may exist because of a threshold enabling self-selection. For example, compared with smaller exchanges, the NYSE has higher stock listing requirements. Choosing NYSE-listed stocks may introduce an implicit quality bias into the analysis. Although the bias is less obvious, it is important for generalizing findings.

A variation of selection bias is backfill bias. For example, when a new hedge fund is added to a given index, the fund's past performance may be backfilled into the index's database, even though the fund was not included in the database in the previous year. Usually a new fund starts contributing data after a period of good performance, so adding the fund's instant history into the index database may inflate the index performance.

\section{Look-Ahead Bias}
A test design is subject to look-ahead bias if it uses information that was not available on the test date. For example, tests of trading rules that use stock market returns and accounting balance sheet data must account for look-ahead bias. In such tests, a company's book value per share is commonly used to construct the $\mathrm{P} / \mathrm{B}$ variable. Although the market price of a stock is available for all market participants at the same point in time, fiscal year-end book equity per share might not become publicly available until sometime in the following quarter. One solution to mitigate the look-ahead bias is to use point-in-time (PIT) data when possible. PIT data is stamped with the date when it was recorded or released. In the previous example, the PIT data of P/B would be accompanied with the date of company filing date or press release date, rather than the end date of the fiscal quarter the P/B data represents. It is worth noting that the look-ahead bias could also be implicitly introduced. For example, when normalizing input data by deducting the mean and dividing it by standard deviation, we must ensure that the standard deviation of the training data is used as the proxy for standard deviation in validation and test data sets. Using standard deviation of validation or test data to normalize them will implicitly introduce a look-ahead bias as the variance of future data is inappropriately used.

\section{Time-Period Bias}
A test design is subject to time-period bias if it is based on a period that may make the results period specific. A short time series is likely to give period-specific results that may not reflect a longer period. A long time series may give a more accurate picture of true investment performance; its disadvantage lies in the potential for a structural change occurring during the time frame that would result in two different return distributions. In this situation, the distribution that would reflect conditions before the change differs from the distribution that would describe conditions after the change. Regime changes, such as low versus high volatility regimes or low versus high interest rate regimes, are highly influential to asset classes. Inferences based on data influenced by one regime, and thus not appropriately distributed, should account for how the regime may bias the inferences.

\section{EXAMPLE 13}
\section{Biases in Investment Research}
An analyst is reviewing the empirical evidence on historical equity returns in the Eurozone (European countries that use the euro). She finds that value stocks (i.e., those with low P/Bs) outperformed growth stocks (i.e., those with high P/Bs) in recent periods. After reviewing the Eurozone market, the analyst wonders whether value stocks might be attractive in the United Kingdom. She investigates the performance of value and growth stocks in the UK market for a 10-year period. To conduct this research, the analyst does the following:

\begin{itemize}
  \item obtains the current composition of the Financial Times Stock Exchange (FTSE) All Share Index, a market-capitalization-weighted index;

  \item eliminates the companies that do not have December fiscal year-ends;

  \item uses year-end book values and market prices to rank the remaining universe of companies by P/Bs at the end of the year;

  \item based on these rankings, divides the universe into 10 portfolios, each of which contains an equal number of stocks;

  \item calculates the equal-weighted return of each portfolio and the return for the FTSE All Share Index for the 12 months following the date each ranking was made; and

  \item subtracts the FTSE returns from each portfolio's returns to derive excess returns for each portfolio.

\end{itemize}

She discusses the research process with her supervisor, who makes two comments:

\begin{itemize}
  \item The proposed process may introduce survivorship bias into her analysis.

  \item The proposed research should cover a longer period. 1. Which of the following best describes the supervisor's first comment?

\end{itemize}

A. The comment is false. The proposed method is designed to avoid survivorship bias.

B. The comment is true, because she is planning to use the current list of FTSE stocks rather than the actual list of stocks that existed at the start of each year.

C. The comment is true, because the test design uses information unavailable on the test date.

\section{Solution to 1:}
$B$ is correct because the research design is subject to survivorship bias if it fails to account for companies that have gone bankrupt, merged, or otherwise departed the database. Using the current list of FTSE stocks rather than the actual list of stocks that existed at the start of each year means that the computation of returns excluded companies removed from the index. The performance of the portfolios with the lowest $\mathrm{P} / \mathrm{B}$ is subject to survivorship bias and may be overstated. At some time during the testing period, those companies not currently in existence were eliminated from testing. They would probably have had low prices (and low $\mathrm{P} / \mathrm{Bs}$ ) and poor returns.

$A$ is incorrect because the method is not designed to avoid survivorship bias. $C$ is incorrect because the fact that the test design uses information unavailable on the test date relates to look-ahead bias. A test design is subject to look-ahead bias if it uses information unavailable on the test date. This bias would make a strategy based on the information appear successful, but it assumes perfect forecasting ability.

\begin{enumerate}
  \setcounter{enumi}{1}
  \item What bias is the supervisor concerned about when making the second comment?
\end{enumerate}

A. Time period bias, because the results may be period specific

B. Look-ahead bias, because the bias could be reduced or eliminated if one uses a longer period

C. Survivorship bias, because the bias would become less relevant over longer periods

\section{Solution to 2:}
A is correct. A test design is subject to time-period bias if it is based on a period that may make the results period specific. Although the research covered a period of 10 years, that period may be too short for testing an anomaly. Ideally, an analyst should test market anomalies over several market cycles to ensure that results are not period specific. This bias can favor a proposed strategy if the period chosen was favorable to the strategy.

\section{SUMMARY}
In this reading, we have presented basic concepts and results in sampling and estimation. We have also emphasized the challenges faced by analysts in appropriately using and interpreting financial data. As analysts, we should always use a critical eye when evaluating the results from any study. The quality of the sample is of the utmost importance: If the sample is biased, the conclusions drawn from the sample will be in error.

\begin{itemize}
  \item To draw valid inferences from a sample, the sample should be random.

  \item In simple random sampling, each observation has an equal chance of being selected. In stratified random sampling, the population is divided into subpopulations, called strata or cells, based on one or more classification criteria; simple random samples are then drawn from each stratum.

  \item Stratified random sampling ensures that population subdivisions of interest are represented in the sample. Stratified random sampling also produces more-precise parameter estimates than simple random sampling.

  \item Convenience sampling selects an element from the population on the basis of whether or not it is accessible to a researcher or how easy it is to access. Because convenience sampling presents the advantage of collecting data quickly at a low cost, it is a suitable sampling plan for small-scale pilot studies.

  \item Judgmental sampling may yield skewed results because of the bias of researchers, but its advantages lie in the fact that in some circumstances, the specialty of researchers and their judgmental can lead them directly to the target population of interest within time constraints.

  \item The central limit theorem states that for large sample sizes, for any underlying distribution for a random variable, the sampling distribution of the sample mean for that variable will be approximately normal, with mean equal to the population mean for that random variable and variance equal to the population variance of the variable divided by sample size.

  \item Based on the central limit theorem, when the sample size is large, we can compute confidence intervals for the population mean based on the normal distribution regardless of the distribution of the underlying population. In general, a sample size of 30 or larger can be considered large.

  \item An estimator is a formula for computing a sample statistic used to estimate a population parameter. An estimate is a particular value that we calculate from a sample by using an estimator.

  \item Because an estimator or statistic is a random variable, it is described by some probability distribution. We refer to the distribution of an estimator as its sampling distribution. The standard deviation of the sampling distribution of the sample mean is called the standard error of the sample mean.

  \item The desirable properties of an estimator are unbiasedness (the expected value of the estimator equals the population parameter), efficiency (the estimator has the smallest variance), and consistency (the probability of accurate estimates increases as sample size increases).

  \item The two types of estimates of a parameter are point estimates and interval estimates. A point estimate is a single number that we use to estimate a parameter. An interval estimate is a range of values that brackets the population parameter with some probability.

  \item A confidence interval is an interval for which we can assert with a given probability $1-\alpha$, called the degree of confidence, that it will contain the parameter it is intended to estimate. This measure is often referred to as the $100(1-\alpha) \%$ confidence interval for the parameter.

  \item A 100(1- $1 \%$ confidence interval for a parameter has the following structure: Point estimate \textbackslash pm Reliability factor $\times$ Standard error, where the reliability factor is a number based on the assumed distribution of the point estimate and the degree of confidence $(1-\alpha)$ for the confidence interval and where standard error is the standard error of the sample statistic providing the point estimate.

  \item A $100(1-\alpha) \%$ confidence interval for population mean $\mu$ when sampling from a normal distribution with known variance $\sigma^{2}$ is given by $\bar{X} \pm \mathrm{z}_{\alpha / 2} \frac{\sigma}{\sqrt{n}}$, where $z_{\alpha / 2}$ is the point of the standard normal distribution such that $\alpha / 2$ remains in the right tail.

  \item A random sample of size $n$ is said to have $n-1$ degrees of freedom for estimating the population variance, in the sense that there are only $n-1$ independent deviations from the mean on which to base the estimate.

  \item A $100(1-\alpha) \%$ confidence interval for the population mean $\mu$ when sampling from a normal distribution with unknown variance (a $t$-distribution confidence interval) is given by $\bar{X} \pm t_{\alpha / 2}(s / \sqrt{n})$, where $t_{\alpha / 2}$ is the point of the $t$-distribution such that $\alpha / 2$ remains in the right tail and $s$ is the sample standard deviation. This confidence interval can also be used, because of the central limit theorem, when dealing with a large sample from a population with unknown variance that may not be normal.

  \item We may use the confidence interval $\bar{X} \pm z_{\alpha / 2}(s / \sqrt{n})$ as an alternative to the $t$-distribution confidence interval for the population mean when using a large sample from a population with unknown variance. The confidence interval based on the $z$-statistic is less conservative (narrower) than the corresponding confidence interval based on a $t$-distribution.

  \item Bootstrap and jackknife are simple but powerful methods for statistical inference, and they are particularly useful when no analytical formula is available. Bootstrap constructs the sampling distribution of an estimator by repeatedly drawing samples from the original sample to find standard error and confidence interval. Jackknife draws repeated samples while leaving out one observation at a time from the set, without replacing it.

  \item Three issues in the selection of sample size are the need for precision, the risk of sampling from more than one population, and the expenses of different sample sizes.

  \item Data snooping bias comes from finding models by repeatedly searching through databases for patterns.

  \item Sample selection bias occurs when data availability leads to certain assets being excluded from the analysis, we call the resulting problem

  \item Survivorship bias is a subset of sample selection bias and occurs if companies are excluded from the analysis because they have gone out of business or because of reasons related to poor performance.

  \item Self-selection bias reflects the ability of entities to decide whether or not they wish to report their attributes or results and be included in databases or samples. Implicit selection bias is one type of selection bias introduced through the presence of a threshold that filters out some unqualified members. A subset of selection bias is backfill bias, in which past data, not reported or used before, is backfilled into an existing database.

  \item Look-ahead bias exists if the model uses data not available to market participants at the time the market participants act in the model.

  \item Time-period bias is present if the period used makes the results period specific or if the period used includes a point of structural change.

\end{itemize}

\section{PRACTICE PROBLEMS}
\begin{enumerate}
  \item Perkiomen Kinzua, a seasoned auditor, is auditing last year's transactions for Conemaugh Corporation. Unfortunately, Conemaugh had a very large number of transactions last year, and Kinzua is under a time constraint to finish the audit. He decides to audit only the small subset of the transaction population that is of interest and to use sampling to create that subset.
\end{enumerate}

The most appropriate sampling method for Kinzua to use is:

A. judgmental sampling.

B. systematic sampling.

C. convenience sampling.

\begin{enumerate}
  \setcounter{enumi}{1}
  \item Which one of the following statements is true about non-probability sampling?
\end{enumerate}

A. There is significant risk that the sample is not representative of the population.

B. Every member of the population has an equal chance of being selected for the sample.

C. Using judgment guarantees that population subdivisions of interest are represented in the sample.

\begin{enumerate}
  \setcounter{enumi}{2}
  \item The best approach for creating a stratified random sample of a population involves:
\end{enumerate}

A. drawing an equal number of simple random samples from each subpopulation.

B. selecting every $k$ th member of the population until the desired sample size is reached.

C. drawing simple random samples from each subpopulation in sizes proportional to the relative size of each subpopulation.

\begin{enumerate}
  \setcounter{enumi}{3}
  \item Although he knows security returns are not independent, a colleague makes the claim that because of the central limit theorem, if we diversify across a large number of investments, the portfolio standard deviation will eventually approach zero as $n$ becomes large. Is he correct?

  \item Why is the central limit theorem important?

  \item What is wrong with the following statement of the central limit theorem?

\end{enumerate}

Central Limit Theorem. "If the random variables $X_{1}, X_{2}, X_{3}, \ldots, X_{n}$ are a random sample of size $n$ from any distribution with finite mean $\mu$ and variance $\sigma^{2}$, then the distribution of $\bar{X}$ will be approximately normal, with a standard deviation of $\sigma / \sqrt{n}$,"

\begin{enumerate}
  \setcounter{enumi}{6}
  \item Peter Biggs wants to know how growth managers performed last year. Biggs assumes that the population cross-sectional standard deviation of growth manager returns is $6 \%$ and that the returns are independent across managers.
\end{enumerate}

A. How large a random sample does Biggs need if he wants the standard deviation of the sample means to be $1 \%$ ?

B. How large a random sample does Biggs need if he wants the standard deviation of the sample means to be $0.25 \%$ ?

\begin{enumerate}
  \setcounter{enumi}{7}
  \item A population has a non-normal distribution with mean $\mu$ and variance $\sigma^{2}$. The sampling distribution of the sample mean computed from samples of large size from that population will have:
\end{enumerate}

A. the same distribution as the population distribution.

B. its mean approximately equal to the population mean.

C. its variance approximately equal to the population variance.

\begin{enumerate}
  \setcounter{enumi}{8}
  \item A sample mean is computed from a population with a variance of 2.45 . The sample size is 40 . The standard error of the sample mean is closest to:
A. 0.039 .
B. 0.247 .
C. 0.387 .

  \item An estimator with an expected value equal to the parameter that it is intended to estimate is described as:
A. efficient.
B. unbiased.
C. consistent.

  \item If an estimator is consistent, an increase in sample size will increase the:
A. accuracy of estimates.
B. efficiency of the estimator.
C. unbiasedness of the estimator.

  \item Petra Munzi wants to know how value managers performed last year. Munzi estimates that the population cross-sectional standard deviation of value manager returns is $4 \%$ and assumes that the returns are independent across managers.

\end{enumerate}

A. Munzi wants to build a 95\% confidence interval for the population mean return. How large a random sample does Munzi need if she wants the $95 \%$ confidence interval to have a total width of $1 \%$ ?

B. Munzi expects a cost of about $\$ 10$ to collect each observation. If she has a $\$ 1,000$ budget, will she be able to construct the confidence interval she wants?

\begin{enumerate}
  \setcounter{enumi}{12}
  \item Find the reliability factors based on the $t$-distribution for the following confidence intervals for the population mean $(\mathrm{df}=$ degrees of freedom, $n=$ sample size):
\end{enumerate}

A. A $99 \%$ confidence interval, $\mathrm{df}=20$ B. A $90 \%$ confidence interval, $\mathrm{df}=20$

C. A $95 \%$ confidence interval, $n=25$

D. A $95 \%$ confidence interval, $n=16$

\begin{enumerate}
  \setcounter{enumi}{13}
  \item Assume that monthly returns are normally distributed with a mean of $1 \%$ and a sample standard deviation of $4 \%$. The population standard deviation is unknown. Construct a 95\% confidence interval for the sample mean of monthly returns if the sample size is 24 .

  \item Explain the differences between constructing a confidence interval when sampling from a normal population with a known population variance and sampling from a normal population with an unknown variance.

  \item For a two-sided confidence interval, an increase in the degree of confidence will result in:

\end{enumerate}

A. a wider confidence interval.

B. a narrower confidence interval.

C. no change in the width of the confidence interval.

\begin{enumerate}
  \setcounter{enumi}{16}
  \item For a sample size of 17 , with a mean of 116.23 and a variance of 245.55 , the width of a $90 \%$ confidence interval using the appropriate $t$-distribution is closest to:
A. 13.23 .
B. 13.27 .
C. 13.68 .

  \item For a sample size of 65 with a mean of 31 taken from a normally distributed population with a variance of 529 , a $99 \%$ confidence interval for the population mean will have a lower limit closest to:
A. 23.64 .
B. 25.41 .
C. 30.09 .

  \item An increase in sample size is most likely to result in a:
A. wider confidence interval.
B. decrease in the standard error of the sample mean.
C. lower likelihood of sampling from more than one population.

  \item Otema Chi has a spreadsheet with 108 monthly returns for shares in Marunou Corporation. He writes a software program that uses bootstrap resampling to create 200 resamples of this Marunou data by sampling with replacement. Each resample has 108 data points. Chi's program calculates the mean of each of the 200 resamples, and then it calculates that the mean of these 200 resample means is 0.0261 . The program subtracts 0.0261 from each of the 200 resample means, squares each of these 200 differences, and adds the squared differences together. The result is 0.835 . The program then calculates an estimate of the standard error of the sample mean.

\end{enumerate}

The estimated standard error of the sample mean is closest to:
A. 0.0115
B. 0.0648
C. 0.0883

\begin{enumerate}
  \setcounter{enumi}{20}
  \item Compared with bootstrap resampling, jackknife resampling:
A. is done with replacement.
B. usually requires that the number of repetitions is equal to the sample size.
C. produces dissimilar results for every run because resamples are randomly drawn.

  \item Suppose we take a random sample of 30 companies in an industry with 200 companies. We calculate the sample mean of the ratio of cash flow to total debt for the prior year. We find that this ratio is $23 \%$. Subsequently, we learn that the population cash flow to total debt ratio (taking account of all 200 companies) is $26 \%$. What is the explanation for the discrepancy between the sample mean of $23 \%$ and the population mean of $26 \%$ ?

\end{enumerate}

A. Sampling error.

B. Bias.

C. A lack of consistency.

\begin{enumerate}
  \setcounter{enumi}{22}
  \item Alcorn Mutual Funds is placing large advertisements in several financial publications. The advertisements prominently display the returns of 5 of Alcorn's 30 funds for the past 1-, 3-, 5-, and 10-year periods. The results are indeed impressive, with all of the funds beating the major market indexes and a few beating them by a large margin. Is the Alcorn family of funds superior to its competitors?

  \item Julius Spence has tested several predictive models in order to identify undervalued stocks. Spence used about 30 company-specific variables and 10 market-related variables to predict returns for about 5,000 North American and European stocks. He found that a final model using eight variables applied to telecommunications and computer stocks yields spectacular results. Spence wants you to use the model to select investments. Should you? What steps would you take to evaluate the model?

  \item A report on long-term stock returns focused exclusively on all currently publicly traded firms in an industry is most likely susceptible to:
A. look-ahead bias.
B. survivorship bias.
C. intergenerational data mining.

  \item Which sampling bias is most likely investigated with an out-of-sample test?
A. Look-ahead bias
B. Data-mining bias C. Sample selection bias

  \item Which of the following characteristics of an investment study most likely indicates time-period bias?

\end{enumerate}

A. The study is based on a short time-series.

B. Information not available on the test date is used.

C. A structural change occurred prior to the start of the study's time series.

\section{SOLUTIONS}
\begin{enumerate}
  \item A is correct. With judgmental sampling, Kinzua will use his knowledge and professional judgment as a seasoned auditor to select transactions of interest from the population. This approach will allow Kinzua to create a sample that is representative of the population and that will provide sufficient audit coverage. Judgmental sampling is useful in cases that have a time constraint or in which the specialty of researchers is critical to select a more representative sample than by using other probability or non-probability sampling methods. Judgement sampling, however, entails the risk that Kinzua is biased in his selections, leading to skewed results that are not representative of the whole population.

  \item A is correct. Because non-probability sampling is dependent on factors other than probability considerations, such as a sampler's judgment or the convenience to access data, there is a significant risk that non-probability sampling might generate a non-representative sample

  \item C is correct. Stratified random sampling involves dividing a population into subpopulations based on one or more classification criteria. Then, simple random samples are drawn from each subpopulation in sizes proportional to the relative size of each subpopulation. These samples are then pooled to form a stratified random sample.

  \item No. First the conclusion on the limit of zero is wrong; second, the support cited for drawing the conclusion (i.e., the central limit theorem) is not relevant in this context.

  \item In many instances, the distribution that describes the underlying population is not normal or the distribution is not known. The central limit theorem states that if the sample size is large, regardless of the shape of the underlying population, the distribution of the sample mean is approximately normal. Therefore, even in these instances, we can still construct confidence intervals (and conduct tests of inference) as long as the sample size is large (generally $n \geq 30$ ).

  \item The statement makes the following mistakes:

\end{enumerate}

\begin{itemize}
  \item Given the conditions in the statement, the distribution of $\bar{X}$ will be approximately normal only for large sample sizes.

  \item The statement omits the important element of the central limit theorem that the distribution of $\bar{X}$ will have mean $\mu$.

\end{itemize}

7.

A. The standard deviation or standard error of the sample mean is $\sigma_{\bar{X}}=\sigma /$ $\sqrt{n}$. Substituting in the values for $\sigma_{\bar{X}}$ and $\sigma$, we have $1 \%=6 \% / \sqrt{n}$, or $\sqrt{n}=6$. Squaring this value, we get a random sample of $n=36$

B. As in Part A, the standard deviation of sample mean is $\sigma_{\bar{X}}=\sigma / \sqrt{n}$.

Substituting in the values for $\sigma_{X}$ and $\sigma$, we have $0.25 \%=6 \% / \sqrt{n}$, or $\sqrt{n}=24$. Squaring this value, we get a random sample of $n=576$, which is substantially larger than for Part A of this question.

\begin{enumerate}
  \setcounter{enumi}{7}
  \item B is correct. Given a population described by any probability distribution (normal or non-normal) with finite variance, the central limit theorem states that the sampling distribution of the sample mean will be approximately normal, with the mean approximately equal to the population mean, when the sample size is large.

  \item B is correct. Taking the square root of the known population variance to determine the population standard deviation $(\sigma)$ results in

\end{enumerate}

$\sigma=\sqrt{2.45}=1.565$

The formula for the standard error of the sample mean $\left(\sigma_{X}\right)$, based on a known sample size $(n)$, is

$\sigma_{X}=\frac{\sigma}{\sqrt{n}}$

Therefore,

$\sigma_{X}=\frac{1.565}{\sqrt{40}}=0.247$

\begin{enumerate}
  \setcounter{enumi}{9}
  \item B is correct. An unbiased estimator is one for which the expected value equals the parameter it is intended to estimate.

  \item A is correct. A consistent estimator is one for which the probability of estimates close to the value of the population parameter increases as sample size increases. More specifically, a consistent estimator's sampling distribution becomes concentrated on the value of the parameter it is intended to estimate as the sample size approaches infinity.

  \item 
\end{enumerate}

A. Assume the sample size will be large and thus the $95 \%$ confidence interval for the population mean of manager returns is $\bar{X} \pm 1.96 s_{\bar{X}}$, where $s_{\bar{X}}=s /$ $\sqrt{n}$. Munzi wants the distance between the upper limit and lower limit in the confidence interval to be $1 \%$, which is

$$
\left(\bar{X}+1.96 s_{\bar{X}}\right)-\left(\bar{X}-1.96 s_{\bar{X}}\right)=1 \%
$$

Simplifying this equation, we get $2\left(1.96 s_{\bar{X}}\right)=1 \%$. Finally, we have $3.92 s_{\bar{X}}=$ $1 \%$, which gives us the standard deviation of the sample mean, $s_{\bar{X}}=0.255 \%$. The distribution of sample means is $s_{\bar{X}}=s / \sqrt{n}$. Substituting in the values for $s_{\bar{X}}$ and $s$, we have $0.255 \%=4 \% / \sqrt{n}$, or $\sqrt{n}=15.69$. Squaring this value, we get a random sample of $n=246$.

B. With her budget, Munzi can pay for a sample of up to 100 observations, which is far short of the 246 observations needed. Munzi can either proceed with her current budget and settle for a wider confidence interval or she can raise her budget (to around $\$ 2,460$ ) to get the sample size for a $1 \%$ width in her confidence interval.

13.

A. For a $99 \%$ confidence interval, the reliability factor we use is $t_{0.005}$; for $\mathrm{df}=$ 20 , this factor is 2.845 .

B. For a $90 \%$ confidence interval, the reliability factor we use is $t_{0.05}$; for $\mathrm{df}=$ 20 , this factor is 1.725 .

C. Degrees of freedom equals $n-1$, or in this case $25-1=24$. For a $95 \%$ confidence interval, the reliability factor we use is $t_{0.025}$; for $\mathrm{df}=24$, this factor is 2.064 .

D. Degrees of freedom equals $16-1=15$. For a $95 \%$ confidence interval, the reliability factor we use is $t_{0.025}$; for $\mathrm{df}=15$, this factor is 2.131 . 14. Because this is a small sample from a normal population and we have only the sample standard deviation, we use the following model to solve for the confidence interval of the population mean:

$$
\bar{X} \pm t_{\alpha / 2} \frac{s}{\sqrt{n}}
$$

where we find $t_{0.025}$ (for a 95\% confidence interval) for $\mathrm{df}=n-1=24-1=23$; this value is 2.069 . Our solution is $1 \% \pm 2.069(4 \%) / \sqrt{24}=1 \% \pm 2.069(0.8165)=1 \%$ \textbackslash pm 1.69 . The $95 \%$ confidence interval spans the range from $-0.69 \%$ to $+2.69 \%$.

\begin{enumerate}
  \setcounter{enumi}{14}
  \item If the population variance is known, the confidence interval is
\end{enumerate}

$$
\bar{X} \pm z_{\alpha / 2} \frac{\sigma}{\sqrt{n}}
$$

The confidence interval for the population mean is centered at the sample mean, $\bar{X}$. The population standard deviation is $\sigma$, and the sample size is $n$. The population standard deviation divided by the square root of $n$ is the standard error of the estimate of the mean. The value of $z$ depends on the desired degree of confidence. For a 95\% confidence interval, $z_{0.025}=1.96$ and the confidence interval estimate is

$\bar{X} \pm 1.96 \frac{\sigma}{\sqrt{n}}$

If the population variance is not known, we make two changes to the technique used when the population variance is known. First, we must use the sample standard deviation instead of the population standard deviation. Second, we use the $t$-distribution instead of the normal distribution. The critical $t$-value will depend on degrees of freedom $n-1$. If the sample size is large, we have the alternative of using the $z$-distribution with the sample standard deviation.

\begin{enumerate}
  \setcounter{enumi}{15}
  \item A is correct. As the degree of confidence increases (e.g., from $95 \%$ to $99 \%$ ), a given confidence interval will become wider. A confidence interval is a range for which one can assert with a given probability $1-\alpha$, called the degree of confidence, that it will contain the parameter it is intended to estimate.

  \item B is correct. The confidence interval is calculated using the following equation:

\end{enumerate}

$$
\bar{X} \pm t_{\alpha / 2} \frac{s}{\sqrt{n}}
$$

Sample standard deviation $(s)=\sqrt{245.55}=15.670$.

For a sample size of 17 , degrees of freedom equal 16 , so $t_{0.05}=1.746$.

The confidence interval is calculated as

$116.23 \pm 1.746 \frac{15.67}{\sqrt{17}}=116.23 \pm 6.6357$

Therefore, the interval spans 109.5943 to 122.8656 , meaning its width is equal to approximately 13.271 . (This interval can be alternatively calculated as $6.6357 \times 2$.)

\begin{enumerate}
  \setcounter{enumi}{17}
  \item A is correct. To solve, use the structure of Confidence interval = Point estimate \textbackslash pm Reliability factor $\times$ Standard error, which, for a normally distributed population with known variance, is represented by the following formula:
\end{enumerate}

$$
\bar{X} \pm z_{\alpha / 2} \frac{\sigma}{\sqrt{n}}
$$

For a 99\% confidence interval, use $z_{0.005}=2.58$.

Also, $\sigma=\sqrt{529}=23$.

Therefore, the lower limit $=31-2.58 \frac{23}{\sqrt{65}}=23.6398$.

\begin{enumerate}
  \setcounter{enumi}{18}
  \item B is correct. All else being equal, as the sample size increases, the standard error of the sample mean decreases and the width of the confidence interval also decreases. 20. B is correct.
\end{enumerate}

The estimate of the standard error of the sample mean with bootstrap resampling is calculated as follows:

$s_{\bar{X}}=\sqrt{\frac{1}{B-1} \sum_{b=1}^{B}\left(\hat{\theta}_{b}-\bar{\theta}\right)^{2}}=\sqrt{\frac{1}{200-1} \sum_{b=1}^{200}\left(\hat{\theta}_{b}-0.0261\right)^{2}}=\sqrt{\frac{1}{199} \times 0.835}$

$s_{\bar{X}}=0.0648$

\begin{enumerate}
  \setcounter{enumi}{20}
  \item B is correct. For a sample of size $n$, jackknife resampling usually requires $n$ repetitions. In contrast, with bootstrap resampling, we are left to determine how many repetitions are appropriate.

  \item A is correct. The discrepancy arises from sampling error. Sampling error exists whenever one fails to observe every element of the population, because a sample statistic can vary from sample to sample. As stated in the reading, the sample mean is an unbiased estimator, a consistent estimator, and an efficient estimator of the population mean. Although the sample mean is an unbiased estimator of the population mean - the expected value of the sample mean equals the population mean-because of sampling error, we do not expect the sample mean to exactly equal the population mean in any one sample we may take.

  \item No, we cannot say that Alcorn Mutual Funds as a group is superior to competitors. Alcorn Mutual Funds' advertisement may easily mislead readers because the advertisement does not show the performance of all its funds. In particular, Alcorn Mutual Funds is engaging in sample selection bias by presenting the investment results from its best-performing funds only.

  \item Spence may be guilty of data mining. He has used so many possible combinations of variables on so many stocks, it is not surprising that he found some instances in which a model worked. In fact, it would have been more surprising if he had not found any. To decide whether to use his model, you should do two things: First, ask that the model be tested on out-of-sample data-that is, data that were not used in building the model. The model may not be successful with out-of-sample data. Second, examine his model to make sure that the relationships in the model make economic sense, have a story, and have a future.

  \item B is correct. A report that uses a current list of stocks does not account for firms that failed, merged, or otherwise disappeared from the public equity market in previous years. As a consequence, the report is biased. This type of bias is known as survivorship bias.

  \item B is correct. An out-of-sample test is used to investigate the presence of data-mining bias. Such a test uses a sample that does not overlap the time period of the sample on which a variable, strategy, or model was developed.

  \item A is correct. A short time series is likely to give period-specific results that may not reflect a longer time period.

\end{enumerate}

\section{LEARNING MODULE
6}
\section{Hypothesis Testing}
by Pamela Peterson Drake, PhD, CFA.

Pamela Peterson Drake, PhD, CFA, is at James Madison University (USA).

\section{LEARNING OUTCOME}
\begin{center}
\begin{tabular}{c|l}
Mastery & The candidate should be able to: \\
\hline
$\square$ & $\begin{array}{l}\text { define a hypothesis, describe the steps of hypothesis testing, } \\ \text { and describe and interpret the choice of the null and alternative } \\ \text { hypotheses } \\ \text { compare and contrast one-tailed and two-tailed tests of hypotheses } \\ \text { explain a test statistic, Type I and Type II errors, a significance level, } \\ \text { how significance levels are used in hypothesis testing, and the power } \\ \text { of a test } \\ \text { explain a decision rule and the relation between confidence intervals } \\ \text { and hypothesis tests, and determine whether a statistically significant } \\ \text { result is also economically meaningful } \\ \text { explain and interpret the p-value as it relates to hypothesis testing } \\ \text { describe how to interpret the significance of a test in the context of } \\ \text { multiple tests } \\ \text { identify the appropriate test statistic and interpret the results for a } \\ \text { hypothesis test concerning the population mean of both large and } \\ \text { small samples when the population is normally or approximately } \\ \text { normally distributed and the variance is (1) known or (2) unknown } \\ \text { identify the appropriate test statistic and interpret the results for a } \\ \text { hypothesis test concerning the equality of the population means of } \\ \text { two at least approximately normally distributed populations based } \\ \text { on independent random samples with equal assumed variances } \\ \text { identify the appropriate test statistic and interpret the results for } \\ \text { a hypothesis test concerning the mean difference of two normally } \\ \text { distributed populations } \\ \text { identify the appropriate test statistic and interpret the results for a } \\ \text { hypothesis test concerning (1) the variance of a normally distributed } \\ \text { population and (2) the equality of the variances of two normally } \\ \text { distributed populations based on two independent random samples }\end{array}$ \\
$\square$ &  \\
\end{tabular}
\end{center}

\section{LEARNING OUTCOME}
\begin{center}
\begin{tabular}{c|l}
Mastery & The candidate should be able to: \\
$\square$ & $\begin{array}{l}\text { compare and contrast parametric and nonparametric tests, and } \\ \text { describe situations where each is the more appropriate type of test } \\ \text { explain parametric and nonparametric tests of the hypothesis that } \\ \text { the population correlation coefficient equals zero, and determine } \\ \text { whether the hypothesis is rejected at a given level of significance } \\ \text { explain tests of independence based on contingency table data }\end{array}$ \\
$\square$ &  \\
\end{tabular}
\end{center}

\begin{center}
\includegraphics[max width=\textwidth]{2023_05_04_cff39ee44f77d6514e1bg-364}
\end{center}

\section{INTRODUCTION}
define a hypothesis, describe the steps of hypothesis testing, and describe and interpret the choice of the null and alternative hypotheses

\section{Why Hypothesis Testing?}
Faced with an overwhelming amount of data, analysts must deal with the task of wrangling those data into something that provides a clearer picture of what is going on. Consider an analyst evaluating the returns on two investments over 33 years, as we show in Exhibit 1.

\section{Exhibit 1: Returns for Investments One and Two over 33 Years}
\begin{center}
\includegraphics[max width=\textwidth]{2023_05_04_cff39ee44f77d6514e1bg-364(1)}
\end{center}

Although "a picture is worth a thousand words," what can we actually glean from this plot? Can we tell if each investment's returns are different from an average of 5\%? Can we tell whether the returns are different for Investment One and Investment Two? Can we tell whether the standard deviations of the two investments are each different from 2\%? Can we tell whether the variability is different for the two investments? For these questions, we need to have more precise tools than simply a plot over time. What we need is a set of tools that aid us in making decisions based on the data.

We use the concepts and tools of hypothesis testing to address these questions. Hypothesis testing is part of statistical inference, the process of making judgments about a larger group (a population) based on a smaller group of observations (that is, a sample).

\section{Implications from a Sampling Distribution}
Consider a set of 1,000 asset returns with a mean of $6 \%$ and a standard deviation of $2 \%$. If we draw a sample of returns from this population, what is the chance that the mean of this sample will be $6 \%$ ? What we know about sampling distributions is that how close any given sample mean will be to the population mean depends on the sample size, the variability within the population, and the quality of our sampling methodology.

For example, suppose we draw a sample of 30 observations and the sample mean is $6.13 \%$. Is this close enough to $6 \%$ to alleviate doubt that the sample is drawn from a population with a mean of 6\%? Suppose we draw another sample of 30 and find a sample mean of $4.8 \%$. Does this bring into doubt whether the population mean is 6\%? If we keep drawing samples of 30 observations from this population, we will get a range of possible sample means, as we show in Exhibit 2 for 100 different samples of size 30 from this population, with a range of values from 5.06 to $7.03 \%$. All these sample means are a result of sampling from the 1,000 asset returns.

Exhibit 2: Distribution of Sample Means of 100 Samples Drawn from a Population of 1,000 Returns

\begin{center}
\includegraphics[max width=\textwidth]{2023_05_04_cff39ee44f77d6514e1bg-365}
\end{center}

As you can see in Exhibit 2, a sample mean that is quite different from the population mean can occur; this situation is not as likely as drawing a sample with a mean closer to the population mean, but it can still happen. In hypothesis testing, we test to see whether a sample statistic is likely to come from a population with the hypothesized value of the population parameter.

The concepts and tools of hypothesis testing provide an objective means to gauge whether the available evidence supports the hypothesis. After applying a statistical test of a hypothesis, we should have a clearer idea of the probability that a hypothesis is true or not, although our conclusion always stops short of certainty.

The main focus of this reading is on the framework of hypothesis testing and tests concerning mean, variance, and correlation, three quantities frequently used in investments.

\section{THE PROCESS OF HYPOTHESIS TESTING}
compare and contrast one-tailed and two-tailed tests of hypotheses

Hypothesis testing is part of the branch of statistics known as statistical inference. In statistical inference, there is estimation and hypothesis testing. Estimation involves point estimates and interval estimates. Consider a sample mean, which is a point estimate, that we can use to form a confidence interval. In hypothesis testing, the focus is examining how a sample statistic informs us about a population parameter. A hypothesis is a statement about one or more populations that we test using sample statistics.

The process of hypothesis testing begins with the formulation of a theory to organize and explain observations. We judge the correctness of the theory by its ability to make accurate predictions-for example, to predict the results of new observations. If the predictions are correct, we continue to maintain the theory as a possibly correct explanation of our observations. Risk plays a role in the outcomes of observations in finance, so we can only try to make unbiased, probability-based judgments about whether the new data support the predictions. Statistical hypothesis testing fills that key role of testing hypotheses when there is uncertainty. When an analyst correctly formulates the question into a testable hypothesis and carries out a test of hypotheses, the use of well-established scientific methods supports the conclusions and decisions made on the basis of this test.

We organize this introduction to hypothesis testing around the six steps in Exhibit 3, which illustrate the standard approach to hypothesis testing.

\section{Exhibit 3: The Process of Hypothesis Testing}
\begin{center}
\includegraphics[max width=\textwidth]{2023_05_04_cff39ee44f77d6514e1bg-367}
\end{center}

\section{Stating the Hypotheses}
For each hypothesis test, we always state two hypotheses: the null hypothesis (or null), designated $H_{0}$, and the alternative hypothesis, designated $H_{a}$. For example, our null hypothesis may concern the value of a population mean, $\mu$, in relation to one possible value of the mean, $\mu_{0}$. As another example, our null hypothesis may concern the population variance, $\sigma^{2}$, compared with a possible value of this variance, $\sigma_{0}{ }^{2}$. The null hypothesis is a statement concerning a population parameter or parameters considered to be true unless the sample we use to conduct the hypothesis test gives convincing evidence that the null hypothesis is false. In fact, the null hypothesis is what we want to reject. If there is sufficient evidence to indicate that the null hypothesis is not true, we reject it in favor of the alternative hypothesis.

Importantly, the null and alternative hypotheses are stated in terms of population parameters, and we use sample statistics to test these hypotheses.

\section{Two-Sided vs. One-Sided Hypotheses}
Suppose we want to test whether the population mean return is equal to $6 \%$. We would state the hypotheses as

$$
H_{0}: \mu=6
$$

and the alternative as

$$
H_{a}: \mu \neq 6 \text {. }
$$

What we just created was a two-sided hypothesis test. We are testing whether the mean is equal to 6\%; it could be greater than or less than that because we are simply asking whether the mean is different from $6 \%$. If we find that the sample mean is far enough away from the hypothesized value, considering the risk of drawing a sample that is not representative of the population, then we would reject the null in favor of the alternative hypothesis.

What if we wanted to test whether the mean is greater than $6 \%$. This presents us with a one-sided hypothesis test, and we specify the hypotheses as follows $H_{0}: \mu \leq 6$

$H_{a}: \mu>6$.

If we find that the sample mean is greater than the hypothesized value of 6 by a sufficient margin, then we would reject the null hypothesis. Why is the null hypothesis stated with a "ड" sign? First, if the sample mean is less than or equal to $6 \%$, this would not support the alternative. Second, the null and alternative hypotheses must be mutually exclusive and collectively exhaustive; in other words, all possible values are contained in either the null or the alternative hypothesis.

Despite the different ways to formulate hypotheses, we always conduct a test of the null hypothesis at the point of equality; for example, $\mu=\mu_{0}$. Whether the null is $H_{0}: \mu=\mu_{0}, H_{0}: \mu \leq \mu_{0}$, or $H_{0}: \mu \geq \mu_{0}$, we actually test $\mu=\mu_{0}$. The reasoning is straightforward: Suppose the hypothesized value of the mean is 6 . Consider $H_{0}: \mu \leq 6$, with a "greater than" alternative hypothesis, $H_{a}: \mu>6$. If we have enough evidence to reject $H_{0}: \mu=6$ in favor of $H_{a}: \mu>6$, we definitely also have enough evidence to reject the hypothesis that the parameter $\mu$ is some smaller value, such as 4.5 or 5 .

Using hypotheses regarding the population mean as an example, the three possible formulations of hypotheses are as follows:

Two-sided alternative: $H_{0}: \mu=\mu_{0}$ versus $H_{a}: \mu \neq \mu_{0}$

One-sided alternative (right side): $H_{0:} \mu \leq \mu_{0}$ versus $H_{a:} \mu>\mu_{0}$

One-sided alternative (left side): $H_{0:} \mu \geq \mu_{0}$ versus $H_{a:} \mu<\mu_{0}$

The reference to the side (right or left) refers to where we reject the null in the probability distribution. For example, if the alternative is $\mathrm{H}_{a}: \mu>6$, this means that we will reject the null hypothesis if the sample mean is sufficiently higher than (or on the right side of the distribution of) the hypothesized value.

Importantly, the calculation to test the null hypothesis is the same for all three formulations. What is different for the three formulations is how the calculation is evaluated to decide whether to reject the null.

\section{Selecting the Appropriate Hypotheses}
How do we choose the null and alternative hypotheses? The null is what we are hoping to reject. The most common alternative is the "not equal to" hypothesis. However, economic or financial theory may suggest a one-sided alternative hypothesis. For example, if the population parameter is the mean risk premium, financial theory may argue that this risk premium is positive. Following the principle of stating the alternative as the "hoped for" condition and using $\mu_{r p}$ for the population mean risk premium, we formulate the following hypotheses:

$$
H_{0}: \mu_{r p} \leq 0 \text { versus } H_{a}: \mu_{r p}>0
$$

Note that the sign in the alternative hypotheses reflects the belief of the researcher more strongly than a two-sided alternative hypothesis. However, the researcher may sometimes select a two-sided alternative hypothesis to emphasize an attitude of neutrality when a one-sided alternative hypothesis is also reasonable. Typically, the easiest way to formulate the hypotheses is to specify the alternative hypothesis first and then specify the null.

\section{EXAMPLE 1}
\section{Specifying the Hypotheses}
\begin{enumerate}
  \item An analyst suspects that in the most recent year excess returns on stocks have fallen below $5 \%$. She wants to study whether the excess returns are less than $5 \%$. Designating the population mean as $\mu$, which hypotheses are most appropriate for her analysis?
\end{enumerate}

A. $H_{0}: \mu=5$ versus $H_{a}: \mu \neq 5$

B. $H_{0}: \mu>5$ versus $H_{a}: \mu<5$

C. $H_{0}: \mu<5$ versus $H_{a}: \mu>5$

\section{Solution}
$B$ is correct. The null hypothesis is what she wants to reject in favor of the alternative, which is that population mean excess return is less than $5 \%$. This is a one-sided (left-side) alternative hypothesis.

\section{IDENTIFY THE APPROPRIATE TEST STATISTIC}
explain a test statistic, Type I and Type II errors, a significance level, how significance levels are used in hypothesis testing, and the power of a test

A test statistic is a value calculated on the basis of a sample that, when used in conjunction with a decision rule, is the basis for deciding whether to reject the null hypothesis.

\section{Test Statistics}
The focal point of our statistical decision is the value of the test statistic. The test statistic that we use depends on what we are testing. As an example, let us examine the test of a population mean risk premium. Consider the sample mean, $\bar{X}$, calculated from a sample of returns drawn from the population. If the population standard deviation is known, the standard error of the distribution of sample means, $\sigma_{\bar{X}}$, is the ratio of the population standard deviation to the square root of the sample size:

$$
\sigma_{\bar{X}}=\frac{\sigma}{\sqrt{n}}
$$

The test statistic for the test of the mean when the population variance is known is a $z$-distributed (that is, normally distributed) test statistic:

$$
z=\frac{\bar{X}_{r p}-\mu_{0}}{\sigma / \sqrt{n}}
$$

If the hypothesized value of the mean population risk premium is 6 (that is, $\mu_{0}=6$ ), we calculate this as $z=\frac{\bar{X}_{r p}-6}{\sigma / \sqrt{n}}$. If, however, the hypothesized value of the mean risk premium is zero (that is, $\mu_{0}=0$ ), we can simplify this test statistic as

$$
z=\frac{\bar{X}_{r p}}{\sigma / \sqrt{n}}
$$

Notably, the key to hypothesis testing is identifying the appropriate test statistic for the hypotheses and the underlying distribution of the population.

\section{Identifying the Distribution of the Test Statistic}
Following the identification of the appropriate test statistic, we must be concerned with the distribution of the test statistic. We show examples of the test statistics and their corresponding distributions in Exhibit 4.

\section{Exhibit 4: Test Statistics and Their Distributions}
What We Want to Test

Test of a single mean

Test of the difference in means

Test of the mean of differences

Test of a single variance

Test of the difference in variances

Test of a correlation

Test of independence (categorical data) Test Statistic

$$
t=\frac{\bar{X}-\mu_{0}}{s / \sqrt{n}}
$$\$\$

t=\textbackslash frac\{\textbackslash left(\textbackslash bar\{X\}\textit{\{1\}-\textbackslash bar\{X\}}\{2\}\textbackslash right)-\textbackslash left(\textbackslash mu\_\{1\}-\textbackslash mu\_\{2\}\textbackslash right)\}\{\textbackslash sqrt\{\textbackslash frac\{s\_\{p\}\textsuperscript{\{2\}\}\{n\_\{1\}\}+\textbackslash frac\{s\_\{p\}}\{2\}\}\{n\_\{2\}\}\}\}

\$\$

$t$-Distributed of the Statistic

Degrees of Freedom

$t$-Distributed

$n-1$

$t=\frac{\bar{d}-\mu_{d 0}}{s_{\bar{d}}}$

$t$-Distributed

$n-1$

$X^{2}=\frac{s^{2}(n-1)}{\sigma_{0}^{2}}$

Chi-square distributed $\quad n-1$

$F=\frac{s_{1}^{2}}{s_{2}^{2}}$

F-distributed

$n_{1}-1, n_{2}-1$

$t=\frac{r \sqrt{n-2}}{\sqrt{1-r^{2}}}$

$t$-Distributed

$n-2$

Chi-square distributed $\quad(r-1)(c-1)$

Note: $\mu_{0}, \mu_{\mathrm{d} 0}$, and $\sigma_{0}^{2}$ denote hypothesized values of the mean, mean difference, and variance, respectively. The $\bar{x}, \bar{d}, s^{2}, s$, and $r$ denote for a sample the mean, mean of the differences, variance, standard deviation, and correlation, respectively, with subscripts indicating the sample, if appropriate. The sample size is indicated as $n$, and the subscript indicates the sample, if appropriate. $O_{i j}$ and $E_{i j}$ are observed and expected frequencies, respectively, with $r$ indicating the number of rows and $c$ indicating the number of columns in the contingency table.

\section{SPECIFY THE LEVEL OF SIGNIFICANCE}
The level of significance reflects how much sample evidence we require to reject the null hypothesis. The required standard of proof can change according to the nature of the hypotheses and the seriousness of the consequences of making a mistake. There are four possible outcomes when we test a null hypothesis, as shown in Exhibit 5. A Type I error is a false positive (reject when the null is true), whereas a Type II error is a false negative (fail to reject when the null is false).

\section{Exhibit 5: Correct and Incorrect Decisions in Hypothesis Testing}
\begin{center}
\begin{tabular}{lcc}
\hline
\multicolumn{2}{l}{} & \multicolumn{2}{c}{True Situation} \\
\hline
Decision & $\boldsymbol{H}_{0}$ True & $\boldsymbol{H}_{\mathbf{0}}$ False &  \\
\hline
Correct decision: & Type II error: &  &  \\
Fail to reject $H_{0}$ & Do not reject a true null &  &  \\
hypothesis. & Fail to reject a false null &  &  \\
hypothesis. &  &  &  \\
Feject $H_{0}$ & Type I error: negative &  &  \\
Reject a true null hypothesis. & Correct decision: &  &  \\
False positive & Ralse null hypothesis. &  &  \\
\end{tabular}
\end{center}

When we make a decision in a hypothesis test, we run the risk of making either a Type I or a Type II error. As you can see in Exhibit 5, these errors are mutually exclusive: If we mistakenly reject the true null, we can only be making a Type I error; if we mistakenly fail to reject the false null, we can only be making a Type II error.

Consider a test of a hypothesis of whether the mean return of a population is equal to 6\%. How far away from 6\% could a sample mean be before we believe it to be different from 6\%, the hypothesized population mean? We are going to tolerate sample means that are close to $6 \%$, but we begin doubting that the population mean is equal to $6 \%$ when we calculate a sample mean that is much different from 0.06. How do we determine "much different"? We do this by setting a risk tolerance for a Type I error and determining the critical value or values at which we believe that the sample mean is much different from the population mean. These critical values depend on (1) the alternative hypothesis, whether one sided or two sided, and (2) the probability distribution of the test statistic, which, in turn, depends on the sample size and the level of risk tolerance in making a Type I error.

The probability of a Type I error in testing a hypothesis is denoted by the lowercase Greek letter alpha, $\alpha$. This probability is also known as the level of significance of the test, and its complement, $(1-\alpha)$, is the confidence level. For example, a level of significance of $5 \%$ for a test means that there is a $5 \%$ probability of rejecting a true null hypothesis and corresponds to the $95 \%$ confidence level.

Controlling the probabilities of the two types of errors involves a trade-off. All else equal, if we decrease the probability of a Type I error by specifying a smaller significance level (say, 1\% rather than 5\%), we increase the probability of making a Type II error because we will reject the null less frequently, including when it is false. Both Type I and Type II errors are risks of being wrong. Whether to accept more of one type versus the other depends on the consequences of the errors, such as costs. The only way to reduce the probabilities of both types of errors simultaneously is to increase the sample size, $n$.

Quantifying the trade-off between the two types of errors in practice is challenging because the probability of a Type II error is itself difficult to quantify because there may be many different possible false hypotheses. Because of this, we specify only $\alpha$, the probability of a Type I error, when we conduct a hypothesis test.

Whereas the significance level of a test is the probability of incorrectly rejecting the true null, the power of a test is the probability of correctly rejecting the null-that is, the probability of rejecting the null when it is false. The power of a test is, in fact, the complement of the Type II error. The probability of a Type II error is often denoted by the lowercase Greek letter beta, $\beta$. We can classify the different probabilities in Exhibit 6 to reflect the notation that is often used.

\section{Exhibit 6: Probabilities Associated with Hypothesis Testing Decisions}
\begin{center}
\begin{tabular}{lcc}
\hline
\multicolumn{2}{c}{True Situation} &  \\
\hline
Decision & $\boldsymbol{H}_{0}$ True & $\boldsymbol{H}_{0}$ False \\
\hline
Fail to reject $H_{0}$ & Confidence level & $\beta$ \\
Reject $H_{0}$ & $(1-\alpha)$ & Power of the test \\
Level of significance & $\alpha$ & $(1-\beta)$ \\
\hline
\end{tabular}
\end{center}

The standard approach to hypothesis testing involves choosing the test statistic with the most power and then specifying a level of significance. It is more appropriate to specify this significance level prior to calculating the test statistic because if we specify it after calculating the test statistic, we may be influenced by the result of the calculation. The researcher is free to specify the probability of a Type I error, but the most common are $10 \%, 5 \%$, and $1 \%$.

\section{EXAMPLE 2}
\section{Significance Level}
\begin{enumerate}
  \item If a researcher selects a $5 \%$ level of significance for a hypothesis test, the confidence level is:
A. $2.5 \%$.
B. $5 \%$.
C. $95 \%$.
\end{enumerate}

\section{Solution}
$\mathrm{C}$ is correct. The $5 \%$ level of significance (i.e., probability of a Type I error) corresponds to $1-0.05=0.95$, or a $95 \%$ confidence level (i.e., probability of not rejecting a true null hypothesis). The level of significance is the complement to the confidence level; in other words, they sum to 1.00 , or $100 \%$.

\section{STATE THE DECISION RULE}
explain a decision rule and the relation between confidence intervals and hypothesis tests, and determine whether a statistically significant result is also economically meaningful

The fourth step in hypothesis testing is stating the decision rule. Before any sample is drawn and before a test statistic is calculated, we need to set up a decision rule: When do we reject the null hypothesis, and when do we not? The action we take is based on comparing the calculated test statistic with a specified value or values, which we refer to as critical values. The critical value or values we choose are based on the level of significance and the probability distribution associated with the test statistic. If we find that the calculated value of the test statistic is more extreme than the critical value or values, then we reject the null hypothesis; we say the result is statistically significant. Otherwise, we fail to reject the null hypothesis; there is not sufficient evidence to reject the null hypothesis.

\section{Determining Critical Values}
For a two-tailed test, we indicate two critical values, splitting the level of significance, $\alpha$, equally between the left and right tails of the distribution. Using a $z$-distributed (standard normal) test statistic, for example, we would designate these critical values as $\pm z_{\alpha / 2}$. For a one-tailed test, we indicate a single rejection point using the symbol for the test statistic with a subscript indicating the specified probability of a Type I error - for example, $z_{\alpha}$

As we noted in our discussion of Exhibit 2, it is possible to draw a sample that has a mean different from the true population mean. In fact, it is likely that a given sample mean is different from the population mean because of sampling error. The issue becomes whether a given sample mean is far enough away from what is hypothesized to be the population mean that there is doubt about whether the hypothesized population mean is true. Therefore, we need to decide how far is too far for a sample mean in comparison to a population mean. That is where the critical values come into the picture.

Suppose we are using a $z$-test and have chosen a $5 \%$ level of significance. In Exhibit 7, we illustrate two tests at the $5 \%$ significance level using a $z$-statistic: A two-sided alternative hypothesis test in Panel A and a one-sided alternative hypothesis test in Panel B, with the white area under the curve indicating the confidence level and the shaded areas indicating the significance level. In Panel A, if the null hypothesis that $\mu=\mu_{0}$ is true, the test statistic has a $2.5 \%$ chance of falling in the left rejection region and a $2.5 \%$ chance of falling in the right rejection region. Any calculated value of the test statistic that falls in either of these two regions causes us to reject the null hypothesis at the $5 \%$ significance level.

We determine the cut-off values for the reject and fail-to-reject regions on the basis of the distribution of the test statistic. For a test statistic that is normally distributed, we determine these cut-off points on the basis of the area under the normal curve; with a Type I error (i.e., level of significance, $\alpha$ ) of $5 \%$ and a two-sided test, there is $2.5 \%$ of the area on either side of the distribution. This results in rejection points of -1.960 and +1.960 , dividing the distribution between the rejection and fail-to-reject regions. Similarly, if we have a one-sided test involving a normal distribution, we would have $5 \%$ area under the curve with a demarcation of 1.645 for a right-side test, as we show in Exhibit 7 (or -1.645 for a left-side test).

\section{Exhibit 7: Decision Criteria Using a 5\% Level of Significance}
A. $H_{0}: \mu=\mu_{0}$ versus $H_{a}: \mu \neq \mu_{0}$

\begin{center}
\includegraphics[max width=\textwidth]{2023_05_04_cff39ee44f77d6514e1bg-374}
\end{center}

B. $H_{0}: \mu \leq \mu_{0}$ versus $H_{a}: \mu>\mu_{0}$

\begin{center}
\includegraphics[max width=\textwidth]{2023_05_04_cff39ee44f77d6514e1bg-374(1)}
\end{center}

$\leftarrow$ null hypothesis Reject the null hypothesis $\rightarrow$

\section{Determining the cut-off points using programming}
The programs in Microsoft Excel, Python, and R differ slightly, depending on whether the user specifies the area to the right or the left of the cut-off in the code:

\begin{center}
\begin{tabular}{llll}
\hline
Cut-off for ... & \multicolumn{1}{c}{Excel} & \multicolumn{1}{c}{Python} & \multicolumn{1}{c}{R} \\
\hline
Right tail, 2.5\% & NORM.S.INV(0.975) & norm.ppf(.975) & qnorm(.025,lower.tail=FALSE) \\
Left tail, 2.5\% & NORM.S.INV(0.025) & norm.ppf(.025) & qnorm(.025,lower.tail=TRUE) \\
Right tail, 5\% & NORM.S.INV(0.95) & norm.ppf(.95) & qnorm(.05,lower.tail=FALSE) \\
Left tail, 5\% & NORM.S.INV(0.05) & norm.ppf(.05) & qnorm(.05,lower.tail=TRUE) \\
\hline
\end{tabular}
\end{center}

For Python, install scipy.stats and import: from scipy.stats import norm.

\section{Decision Rules and Confidence Intervals}
Exhibit 7 provides an opportunity to highlight the relationship between confidence intervals and hypothesis tests. A 95\% confidence interval for the population mean, $\mu$, based on sample mean, $\bar{X}$, is given by:

$$
\left\{\bar{X}-1.96 \frac{\sigma}{\sqrt{n}}, \bar{X}+1.96 \frac{\sigma}{\sqrt{n}}\right\}
$$

or, more compactly, $\bar{X} \pm 1.96 \frac{\sigma}{\sqrt{n}}$.

Now consider the conditions for rejecting the null hypothesis:

$$
\frac{\bar{X}-\mu_{0}}{\sigma / \sqrt{n}}<-1.96 \text { or } \frac{\bar{X}-\mu_{0}}{\sigma / \sqrt{n}}>1.96, \text { where } z=\frac{\bar{X}-\mu_{0}}{\sigma / \sqrt{n}}
$$

As you can see by comparing these conditions with the confidence interval, we can address the question of whether $\bar{X}$ is far enough away from $\mu_{0}$ by either comparing the calculated test statistic with the critical values or comparing the hypothesized population parameter $\left(\mu=\mu_{0}\right)$ with the bounds of the confidence interval, as we show in Exhibit 8. Thus, a significance level in a two-sided hypothesis test can be interpreted in the same way as a $(1-\alpha)$ confidence interval.

\section{Exhibit 8: Making a Decision Based on Critical Values and Confidence}
 Intervals for a Two-Sided Alternative Hypothesis\begin{center}
\begin{tabular}{lll}
\hline
Method & Procedure & Decision \\
\hline
$\mathbf{1}$ & $\begin{array}{l}\text { Compare the calculated test } \\ \text { statistic with the critical } \\ \text { values. }\end{array}$ & $\begin{array}{l}\text { If the calculated test statistic is less than the lower } \\ \text { critical value or greater than the upper critical } \\ \text { value, reject the null hypothesis. }\end{array}$ \\
\hline
$\mathbf{2}$ & $\begin{array}{l}\text { Compare the calculated test } \\ \text { statistic with the bounds of the hypothesized value of the population param- } \\ \text { the confidence interval. }\end{array}$ & $\begin{array}{l}\text { eter under the null is outside the corresponding } \\ \text { confidence interval, the null hypothesis is rejected. }\end{array}$ \\
\hline
\end{tabular}
\end{center}

\section{Collect the Data and Calculate the Test Statistic}
The fifth step in hypothesis testing is collecting the data and calculating the test statistic. The quality of our conclusions depends on not only the appropriateness of the statistical model but also the quality of the data we use in conducting the test. First, we need to ensure that the sampling procedure does not include biases, such as sample selection or time bias. Second, we need to cleanse the data, checking inaccuracies and other measurement errors in the data. Once assured that the sample is unbiased and accurate, the sample information is used to calculate the appropriate test statistic.

\section{EXAMPLE 3}
\section{Using a Confidence Interval in Hypothesis Testing}
\begin{enumerate}
  \item Consider the hypotheses $H_{0}: \mu=3$ versus $H_{a}: \mu \neq 3$. If the confidence interval based on sample information has a lower bound of 2.75 and an upper bound of 4.25 , the most appropriate decision is:
\end{enumerate}

A. reject the null hypothesis.

B. accept the null hypothesis.

C. fail to reject the null hypothesis.

\section{Solution}
$\mathrm{C}$ is correct. Since the hypothesized population mean $(\mu=3)$ is within the bounds of the confidence interval $(2.75,4.25)$, the correct decision is to fail to reject the null hypothesis. It is only when the hypothesized value is outside these bounds that the null hypothesis is rejected. Note that the null hypothesis is never accepted; either the null is rejected on the basis of the evidence or there is a failure to reject the null hypothesis.

\section{MAKE A DECISION}
explain a decision rule and the relation between confidence intervals and hypothesis tests, and determine whether a statistically significant result is also economically meaningful

\section{Make a Statistical Decision}
The sixth step in hypothesis testing is making the decision. Consider a test of the mean risk premium, comparing the population mean with zero. If the calculated $z$-statistic is 2.5 and with a two-sided alternative hypothesis and a $5 \%$ level of significance, we reject the null hypothesis because 2.5 is outside the bounds of \textbackslash pm 1.96 . This is a statistical decision: The evidence indicates that the mean risk premium is not equal to zero.

\section{Make an Economic Decision}
Another part of the decision making is making the economic or investment decision. The economic or investment decision takes into consideration not only the statistical decision but also all pertinent economic issues. If, for example, we reject the null that the risk premium is zero in favor of the alternative hypothesis that the risk premium is greater than zero, we have found evidence that the US risk premium is different from zero. The question then becomes whether this risk premium is economically meaningful. On the basis of these considerations, an investor might decide to commit funds to US equities. A range of non-statistical considerations, such as the investor's tolerance for risk and financial position, might also enter the decision-making process.

\section{Statistically Significant but Not Economically Significant?}
We frequently find that slight differences between a variable and its hypothesized value are statistically significant but not economically meaningful. For example, we may be testing an investment strategy and reject a null hypothesis that the mean return to the strategy is zero based on a large sample. In the case of a test of the mean, the smaller the standard error of the mean, the larger the value of the test statistic and the greater the chance the null will be rejected, all else equal. The standard error decreases as the sample size, $n$, increases, so that for very large samples, we can reject the null for small departures from it. We may find that although a strategy provides a statistically significant positive mean return, the results may not be economically significant when we account for transaction costs, taxes, and risk. Even if we conclude that a strategy's results are economically meaningful, we should explore the logic of why the strategy might work in the future before implementing it. Such considerations cannot be incorporated into a hypothesis test.

\section{EXAMPLE 4}
\section{Decisions and Significance}
\begin{enumerate}
  \item An analyst is testing whether there are positive risk-adjusted returns to a trading strategy. He collects a sample and tests the hypotheses of $H_{0}: \mu \leq 0 \%$ versus $H_{a}: \mu>0 \%$, where $\mu$ is the population mean risk-adjusted return. The mean risk-adjusted return for the sample is $0.7 \%$. The calculated $t$-statistic is 2.428 , and the critical $t$-value is 2.345 . He estimates that the transaction costs are $0.3 \%$. The results are most likely:
\end{enumerate}

A. statistically and economically significant.

B. statistically significant but not economically significant.

C. economically significant but not statistically significant.

\section{Solution}
$\mathrm{A}$ is correct. The results indicate that the mean risk-adjusted return is greater than $0 \%$ because the calculated test statistic of 2.428 is greater than the critical value of 2.245 . The results are also economically significant because the risk-adjusted return exceeds the transaction cost associated with this strategy by $0.4 \%(=0.7-0.3)$.

\section{THE ROLE OF $P$-VALUES}
explain and interpret the $p$-value as it relates to hypothesis testing

Analysts, researchers, and statistical software often report the $p$-value associated with hypothesis tests. The $\boldsymbol{p}$-value is the area in the probability distribution outside the calculated test statistic; for a two-sided test, this is the area outside \textbackslash pm the calculated test statistic, but for a one-sided test, this is the area outside the calculated test statistic on the appropriate side of the probability distribution. We illustrated in Exhibit 7 the rejection region, which corresponds to the probability of a Type I error. However, the $p$-value is the area under the curve (so, the probability) associated with the calculated test statistic. Stated another way, the $p$-value is the smallest level of significance at which the null hypothesis can be rejected.

Consider the calculated $z$-statistic of 2.33 in a two-sided test: The $p$-value is the area in the $z$-distribution that lies outside \textbackslash pm 2.33 . Calculation of this area requires a bit of calculus, but fortunately statistical programs and other software calculate the $p$-value for us. For the value of the test statistic of 2.33 , the $p$-value is approximately 0.02 , or $2 \%$. Using Excel, we can get the precise value of 0.019806 [(1-NORM.S.DIST(2.33,TRUE))*2]. We can reject the null hypothesis because we were willing to tolerate up to $5 \%$ outside the calculated value. The smaller the $p$-value, the stronger the evidence against the null hypothesis and in favor of the alternative hypothesis; if the $p$-value is less than the level of significance, we reject the null hypothesis.

We illustrate the comparison of the level of significance and the $p$-value in Exhibit 9. The fail-to-reject region is determined by the critical values of \textbackslash pm 1.96 , as we saw in Exhibit 7. There is $5 \%$ of the area under the distribution in the rejection regions- $2.5 \%$ on the left side, $2.5 \%$ on the right. But now we introduce the area outside the calculated test statistic. For the calculated $z$-statistic of 2.33 , there is 0.01 , or $1 \%$, of the area under the normal distribution above 2.33 and $1 \%$ of the area below -2.33 (or, in other words, $98 \%$ of the area between \textbackslash pm 2.33$)$. Since we are willing to tolerate a $5 \%$ Type I error, we reject the null hypothesis in the case of a calculated test statistic of 2.33 because there is a $p$-value of $2 \%$; there is $2 \%$ of the distribution outside the calculated test statistic.

\section{Exhibit 9: Comparison of the Level of Significance and the $p$-Value}
\begin{center}
\includegraphics[max width=\textwidth]{2023_05_04_cff39ee44f77d6514e1bg-378}
\end{center}

What if we are testing a one-sided alternative hypothesis? We focus solely on the area outside the calculated value on the side indicated by the alternative hypothesis. For example, if we are testing an alternative hypothesis that the population mean risk premium is greater than zero, calculate a $z$-statistic of 2.5 , and have a level of significance of $5 \%$, the $p$-value is the area in the probability distribution that is greater than 2.5. This area is 0.00621 , or $0.621 \%$. Since this is less than what we tolerate if the $\alpha$ is $5 \%$, we reject the null hypothesis.

Consider a population of 1,000 assets that has a mean return of $6 \%$ and a standard deviation of $2 \%$. Suppose we draw a sample of 50 returns and test whether the mean is equal to $6 \%$, calculating the $p$-value for the calculated $z$-statistic. Then, suppose we repeat this process, draw 1,000 different samples, and, therefore, get 1,000 different calculated $z$-statistics and 1,000 different $p$-values. If we use a $5 \%$ level of significance, we should expect to reject the true null $5 \%$ of the time; if we use a $10 \%$ level of significance, we should expect to reject the true null $10 \%$ of the time.

Now suppose that with this same population, whose mean return is $6 \%$, we test the hypothesis that the population mean is $7 \%$, with the same standard deviation. As before, we draw 1,000 different samples and calculate 1,000 different $p$-values. If we use a $5 \%$ level of significance and a two-sided alternative hypothesis, we should expect to reject this false null hypothesis on the basis of the power of the test.

Putting this together, consider the histograms of $p$-values for the two different tests in Exhibit 10. The first bin is the $p$-values of $5 \%$ or less. What we see is that with the true null hypothesis of $6 \%$, we reject the null approximately $5 \%$ of the time. For the false null hypothesis, that the mean is equal to $7 \%$, we reject the null approximately 0.973 , or $97.3 \%$, of the time. Exhibit 10: Distribution of $p$-values for 1,000 Different Samples of Size 50

Drawn from a Population with a Mean of $6 \%$

\begin{center}
\includegraphics[max width=\textwidth]{2023_05_04_cff39ee44f77d6514e1bg-379(1)}
\end{center}

False Null Mean $=0.07 \quad$ True Null Mean $=0.06$

The $p$-values for the true null hypothesis are generally uniformly distributed between $0 \%$ and $100 \%$ because under the null hypothesis, there is a $5 \%$ chance of the $p$-values being less than $5 \%$, a $10 \%$ chance being less than $10 \%$, and so on. Why is it not completely uniform? Because we took 1,000 samples of 50 ; taking more samples or larger samples would result in a more uniform distribution of $p$-values. When looking at the $p$-values for the false null hypothesis in Exhibit 10, we see that this is not a uniform distribution; rather, there is a peak around $0 \%$ and very little elsewhere. You can see the difference in the $p$-values for two false hypothesized means of $6.5 \%$ and $7 \%$ in Exhibit 11. It shows that the further the false hypothesis is away from the truth (i.e., mean of $6 \%$ ), the greater the power of the test and the better the ability to detect the false hypothesis.

Software, such as Excel, Python, and R, is available for calculating $p$-values for most distributions.

Exhibit 11: Comparison of the Distribution of $p$-Values for the False Null

Hypotheses $H_{0}: \mu=7 \%$ and $H_{0}: \mu=6.5 \%$

\begin{center}
\includegraphics[max width=\textwidth]{2023_05_04_cff39ee44f77d6514e1bg-379}
\end{center}

False Null Mean $=0.07 \quad$ False Null Mean $=0.065$

\section{EXAMPLE 5}
\section{Making a Decision Using $p$-Values}
\begin{enumerate}
  \item An analyst is testing the hypotheses $H_{0}: \sigma^{2}=0.01$ versus $H_{a}: \sigma^{2} \neq 0.01$. Using software, she determines that the $p$-value for the test statistic is 0.03 , or $3 \%$. Which of the following statements are correct?
\end{enumerate}

A. Reject the null hypothesis at both the $1 \%$ and $5 \%$ levels of significance.

B. Reject the null hypothesis at the $5 \%$ level but not at the $1 \%$ level of significance.

C. Fail to reject the null hypothesis at both the $1 \%$ and $5 \%$ levels of significance.

\section{Solution}
$B$ is correct. Rejection of the null hypothesis requires that the $p$-value be less than the level of significance. On the basis of this requirement, the null is rejected at the $5 \%$ level of significance but not at the $1 \%$ level of significance.

\section{MULTIPLE TESTS AND SIGNIFICANCE INTERPRETATION}
describe how to interpret the significance of a test in the context of multiple tests

A Type I error is the risk of rejection of a true null hypothesis. Another way of phrasing this is that it is a false positive result; that is, the null is rejected (the positive), yet the null is true (hence, a false positive). The expected portion of false positives is the false discovery rate (FDR). In the previous example of drawing 1,000 samples of 50 observations each, it is the case that there are samples in which we reject the true null hypothesis of a population mean of $6 \%$. If we draw enough samples with a level of significance of 0.05 , approximately 0.05 of the time you will reject the null hypothesis, even if the null is true. In other words, if you run 100 tests and use a $5 \%$ level of significance, you get five false positives, on average. This is referred to as the multiple testing problem.

The false discovery approach to testing requires adjusting the $p$-value when you have a series of tests. The idea of adjusting for the likelihood of significant results being false positives was first introduced by Benjamini and Hochberg (BH) in 1995. What they proposed is that the researcher rank the $p$-values from the various tests, from lowest to highest, and then make the following comparison, starting with the lowest $p$-value (with $k=1), p(1)$ :

$$
p(1) \leq \alpha \frac{\text { Rank of } i}{\text { Number of tests }} \text {. }
$$

This comparison is repeated, such that $k$ is determined by the highest ranked $p(k)$ for which this is a true statement. If, say, $k$ is 4 , then the first four tests (ranked on the basis of the lowest $p$-values) are said to be significant.

Suppose we test the hypothesis that the population mean is equal to $6 \%$ and repeat the sampling process by drawing 20 samples and calculating 20 test statistics; the six test statistics with the lowest $p$-values are shown in. Using a significance level of $5 \%$, if we simply relied on each test and its $p$-value, then there are five tests in which we would reject the null. However, using the BH criteria, only one test is considered significant, as shown in Exhibit 12.

\section{Exhibit 12: Applying the Benjamini and Hochberg Criteria}
\begin{center}
\begin{tabular}{|c|c|c|c|c|c|}
\hline
(1) & (2) & (3) & (4) & (5) & (6) \\
\hline
$\overline{\mathbf{X}}$ & $\begin{array}{l}\text { Calculated } \\ \text { z-Statistic }\end{array}$ & $p$-Value & $\begin{array}{l}\text { Rank of } p \text {-Value } \\ \text { (lowest to highest) }\end{array}$ & $a \frac{\text { Rank of } \mathbf{i}}{\text { Number of tests }}$ & $\begin{array}{c}\text { Is value in (3) less than or } \\ \text { equal to value in (5)? }\end{array}$ \\
\hline
0.0664 & 3.1966 & 0.0014 & 1 & 0.0025 & Yes \\
\hline
0.0645 & 2.2463 & 0.0247 & 2 & 0.0050 & No \\
\hline
0.0642 & 2.0993 & 0.0358 & 3 & 0.0075 & No \\
\hline
0.0642 & 2.0756 & 0.0379 & 4 & 0.0100 & No \\
\hline
0.0641 & 2.0723 & 0.0382 & 5 & 0.0125 & No \\
\hline
0.0637 & 1.8627 & 0.0625 & 6 & 0.0150 & No \\
\hline
\end{tabular}
\end{center}

Note: Level of significance $=5 \%$,

So, what is the conclusion from looking at $p$-values and the multiple testing problem?

\begin{itemize}
  \item First, if we sample, test, and find a result that is not statistically significant, this result is not wrong; in fact, the null hypothesis may well be true.

  \item Second, if the power of the test is low or the sample size is small, we should be cautious because there is a good chance of a false positive.

  \item Third, when we perform a hypothesis test and determine the critical values, these values are based on the assumption that the test is run once. Running multiple tests on data risks data snooping, which may result in spurious results. Determine the dataset and perform the test, but do not keep performing tests repeatedly to search out statistically significant results, because you may, by chance, find them (i.e., false positives).

  \item Fourth, in very large samples, we will find that nearly every test is significant. The approach to use to address this issue is to draw different samples; if the results are similar, the results are more robust.

\end{itemize}

\section{EXAMPLE 6}
\section{False Discovery and Multiple Tests}
A researcher is examining the mean return on assets of publicly traded companies that constitute an index of 2,000 mid-cap stocks and is testing hypotheses concerning whether the mean is equal to $15 \%: H_{0}: \mu_{R O A}=15$ versus $H a: \mu_{R O A}$ $\neq 15$. She uses a $10 \%$ level of significance and collects a sample of 50 firms. She wants to examine the robustness of her analysis, so she repeats the collection and test of the return on assets 30 times. The results for the samples with the five lowest $p$-values are given in Exhibit 13. Exhibit 13: Five Lowest $p$-Values of the 30 Samples Tested

\begin{center}
\begin{tabular}{lccc}
\hline
Sample & $\begin{array}{c}\text { Calculated } \\ \text { z-Statistic }\end{array}$ & $\boldsymbol{p}$-Value & $\begin{array}{c}\text { Ranked } \\ \boldsymbol{p} \text {-Value }\end{array}$ \\
\hline
1 & 3.203 & 0.00136 & 1 \\
5 & 3.115 & 0.00184 & 2 \\
14 & 2.987 & 0.00282 & 3 \\
25 & 2.143 & 0.03211 & 4 \\
29 & 1.903 & 0.05704 & 5 \\
\hline
\end{tabular}
\end{center}

\begin{enumerate}
  \item Of the 30 samples tested, how many should the researcher expect, on average, to have $p$-values less than the level of significance?
\end{enumerate}

\section{Solution to 1}
Of the 30 samples tested, she should expect $30 \times 0.10=3$ to have significant results just by chance. Consider why she ended up with more than three. Three is based on large sample sizes and large numbers of samples. Using a limited sample size (i.e., 50) and number of samples (i.e., 30), there is a risk of a false discovery with repeated samples and tests.

\begin{enumerate}
  \setcounter{enumi}{1}
  \item What are the corrected $p$-values based on her selected level of significance, and what is the effect on her decision?
\end{enumerate}

\section{Solution to 2}
Applying the $\mathrm{BH}$ criteria, the researcher determines the adjusted $\mathrm{p}$-values shown in Exhibit 14.

\section{Exhibit 14: Adjusted $p$-Values for Five Lowest $p$-Values from 30 Samples Tested}
\begin{center}
\begin{tabular}{lcccc}
\hline
$\begin{array}{l}\text { Calculated } \\ \text { z-Statistic }\end{array}$ & $\begin{array}{c}\text { Rank of } \\ \boldsymbol{p} \text {-Value }\end{array}$ & $\begin{array}{c}\text { Rank of } \mathbf{i} \\ \text { Number of tests }\end{array}$ & $\begin{array}{c}\text { Is } \boldsymbol{p} \text {-value less than or } \\ \text { equal to adjusted } \boldsymbol{p} \text {-value? }\end{array}$ &  \\
\hline
3.203 & 0.00136 & 1 & 0.00333 & Yes \\
3.115 & 0.00184 & 2 & 0.00667 & Yes \\
2.987 & 0.00282 & 3 & 0.01000 & Yes \\
2.143 & 0.03211 & 4 & 0.01333 & No \\
1.903 & 0.05704 & 5 & 0.01667 &  \\
\hline
\end{tabular}
\end{center}

On the basis of the results in Exhibit 14, there are three samples with $p$-values less than their adjusted $p$-values. So, the number of significant sample results is the same as would be expected from chance, given the $10 \%$ level of significance. The researcher concludes that the results for the samples with Ranks 4 and 5 are false discoveries, and she has not uncovered any evidence from her testing that supports rejecting the null hypothesis.

\section{TESTS CONCERNING A SINGLE MEAN}
identify the appropriate test statistic and interpret the results for a hypothesis test concerning the population mean of both large and small samples when the population is normally or approximately normally distributed and the variance is (1) known or (2) unknown

Hypothesis tests concerning the mean are among the most common in practice. The sampling distribution of the mean when the population standard deviation is unknown is $t$-distributed, and when the population standard deviation is known, it is normally distributed, or $z$-distributed. Since the population standard deviation is unknown in almost all cases, we will focus on the use of a $t$-distributed test statistic.

The $t$-distribution is a probability distribution defined by a single parameter known as degrees of freedom (df). Like the standard normal distribution, a $t$-distribution is symmetrical with a mean of zero, but it has a standard deviation greater than 1 and generally fatter tails. As the number of degrees of freedom increases with the sample size, the $t$-distribution approaches the standard normal distribution.

For hypothesis tests concerning the population mean of a normally distributed population with unknown variance, the theoretically correct test statistic is the $t$-statistic. What if a normal distribution does not describe the population? The $t$-statistic is robust to moderate departures from normality, except for outliers and strong skewness. When we have large samples, departures of the underlying distribution from the normal case are of increasingly less concern. The sample mean is approximately normally distributed in large samples according to the central limit theorem, whatever the distribution describing the population. A traditional rule of thumb is that the normal distribution is used in cases when the sample size is larger than 30, but the more precise testing uses the $t$-distribution. Moreover, with software that aids us in such testing, we do not need to resort to rules of thumb.

If the population sampled has unknown variance, then the test statistic for hypothesis tests concerning a single population mean, $\mu$, is

$$
t_{n-1}=\frac{\bar{X}-\mu_{0}}{s / \sqrt{n}}
$$

where

$$
\begin{aligned}
\bar{X} & =\text { sample mean } \\
\mu_{0} & =\text { hypothesized value of the population mean } \\
s & =\text { sample standard deviation } \\
n & =\text { sample size } \\
s_{\bar{X}} & =s / \sqrt{n}=\text { estimate of the sample mean standard error }
\end{aligned}
$$

This test statistic is $t$-distributed with $n-1$ degrees of freedom, which we can write as $t_{n-1}$. For simplicity, we often drop the subscript $n-1$ because each particular test statistic has specified degrees of freedom, as we presented in Exhibit 4.

Consider testing whether Investment One's returns (from Exhibit 1) are different from 6\%; that is, we are testing $H_{0}: \mu=6$ versus $H_{a}: \mu \neq 6$. If the calculated $t$-distributed test statistic is outside the bounds of the critical values based on the level of significance, we will reject the null hypothesis in favor of the alternative. If we have a sample size of 33, there are $n-1=32$ degrees of freedom. At a $5 \%$ significance level (two tailed) and 32 degrees of freedom, the critical $t$-values are \textbackslash pm 2.037 . We can determine the critical values from software:

\begin{itemize}
  \item Excel [T.INV $(0.025,32)$ and T.INV $(0.975,32)]$

  \item $\mathrm{R}[\mathrm{qt}(\mathrm{c}(.025, .975), 32)]$

  \item Python [from scipy.stats import t and t.ppf(.025,32) and t.ppf(.975,32)]

\end{itemize}

Suppose that the sample mean return is $5.2990 \%$ and the sample standard deviation is $1.4284 \%$. The calculated test statistic is

$t=\frac{5.2990-6}{1.4284 \sqrt{33}}=-2.8192$

with 32 degrees of freedom. The calculated value is less than -2.037 , so we reject the null that the population mean is 6\%, concluding that it is different from $6 \%$.

\section{EXAMPLE 7}
\section{Risk and Return Characteristics of an Equity Mutual Fund}
Suppose you are analyzing Sendar Equity Fund, a midcap growth fund that has been in existence for 24 months. During this period, it has achieved a mean monthly return of $1.50 \%$, with a sample standard deviation of monthly returns of $3.60 \%$. Given its level of market risk and according to a pricing model, this mutual fund was expected to have earned a $1.10 \%$ mean monthly return during that time period. Assuming returns are normally distributed, are the actual results consistent with an underlying or population mean monthly return of $1.10 \%$ ?

\begin{enumerate}
  \item Test the hypothesis using a $5 \%$ level of significance.
\end{enumerate}

\section{Solution to 1}
Step 1 State the hypotheses.

Step 2 Identify the appropriate test statistic.

Step 3 Specify the level of significance.

Step $4 \quad$ State the decision rule.

Step 5 Calculate the test statistic.

Step 6 Make a decision. $H_{0}: \mu=1.1 \%$ versus $H_{a}: \mu \neq 1.1 \%$

$t=\frac{\bar{X}-\mu_{0}}{s / \sqrt{n}}$

with $24-1=23$ degrees of freedom.

$\alpha=5 \%$ (two tailed).

Critical $t$-values $= \pm 2.069$.

Reject the null if the calculated $t$-statistic is less than -2.069 , and reject the null if the calculated $t$-statistic is greater than +2.069 .

Excel

Lower: T.INV(0.025,23)

Upper: $\operatorname{T.INV}(0.975,23)$

$\mathbf{R} \operatorname{qt}(\mathrm{c}(.025, .975), 23)$

Python from scipy.stats import $t$

Lower: t.ppf(.025,23)

Upper: t.ppf $(.975,23)$

$t=\frac{1.5-1.1}{3.6 / \sqrt{24}}=0.54433$

Fail to reject the null hypothesis because the calculated $t$-statistic falls between the two critical values. There is not sufficient evidence to indicate that the population mean monthly return is different from $1.10 \%$. Test the hypothesis using the $95 \%$ confidence interval.

\section{Solution to 2}
The $95 \%$ confidence interval is $\bar{X} \pm$ Critical value $\left(\frac{s}{\sqrt{n}}\right)$, so

$$
\begin{aligned}
& \{1.5-2.069(3.6 / \sqrt{24}), 1.5+2.069(3.6 / \sqrt{24})\} \\
& \{1.5-1.5204,1.5+1.5204\} \\
& \{-0.0204,3.0204\}
\end{aligned}
$$

The hypothesized value of $1.1 \%$ is within the bounds of the $95 \%$ confidence interval, so we fail to reject the null hypothesis.

We stated previously that when population variance is not known, we use a $t$-test for tests concerning a single population mean. Given at least approximate normality, the $t$-distributed test statistic is always called for when we deal with small samples and do not know the population variance. For large samples, the central limit theorem states that the sample mean is approximately normally distributed, whatever the distribution of the population. The $t$-statistic is still appropriate, but an alternative test may be more useful when sample size is large.

For large samples, practitioners sometimes use a $z$-test in place of a $t$-test for tests concerning a mean. The justification for using the $z$-test in this context is twofold. First, in large samples, the sample mean should follow the normal distribution at least approximately, as we have already stated, fulfilling the normality assumption of the $z$-test. Second, the difference between the rejection points for the $t$-test and $z$-test becomes quite small when the sample size is large. Since the $t$-test is readily available as statistical program output and theoretically correct for unknown population variance, we present it as the test of choice.

In a very limited number of cases, we may know the population variance; in such cases, the $z$-statistic is theoretically correct. In this case, the appropriate test statistic is what we used earlier (Equation 2):

$$
z=\frac{\bar{X}-\mu_{0}}{\sigma / \sqrt{n}}
$$

In cases of large samples, a researcher may use the $z$-statistic, substituting the sample standard deviation $(s)$ for the population standard deviation $(\sigma)$ in the formula. When we use a $z$-test, we usually refer to a rejection point in Exhibit 15.

\section{Exhibit 15: Critical Values for Common Significance Levels for the Standard}
 Normal DistributionReject the Null if...

\begin{center}
\begin{tabular}{lccc}
$\begin{array}{l}\text { Level of } \\ \text { Significance }\end{array}$ & Alternative & $\begin{array}{c}\text { below the } \\ \text { Critical Value }\end{array}$ & $\begin{array}{c}\text { above the } \\ \text { Critical Value }\end{array}$ \\
\hline
0.01 & Two sided: $H_{0}: \mu=\mu_{0}, H_{a}: \mu \neq \mu_{0}$ & -2.576 & 2.576 \\
 & One sided: $H_{0}: \mu \leq \mu_{0}, H_{a}: \mu>\mu_{0}$ & 2.326 &  \\
 & One sided: $H_{0}: \mu \geq \mu_{0}, H_{a}: \mu<\mu_{0}$ & -2.326 & 1.960 \\
Two sided: $H_{0}: \mu=\mu_{0}, H_{a}: \mu \neq \mu_{0}$ & -1.960 & 1.645 &  \\
One sided: $H_{0}: \mu \leq \mu_{0}, H_{a}: \mu>\mu_{0}$ &  &  &  \\
One sided: $H_{0}: \mu \geq \mu_{0}, H_{a}: \mu<\mu_{0}$ & -1.645 &  &  \\
\end{tabular}
\end{center}

\section{EXAMPLE 8}
\section{Testing the Returns on the ACE High Yield Index}
\begin{enumerate}
  \item Suppose we want to test whether the daily return in the ACE High Yield Total Return Index is different from zero. Collecting a sample of 1,304 daily returns, we find a mean daily return of $0.0157 \%$, with a standard deviation of $0.3157 \%$.

  \item Test whether the mean daily return is different from zero at the $5 \%$ level of significance.

  \item Using the $z$-distributed test statistic as an approximation, test whether the mean daily return is different from zero at the $5 \%$ level of significance.

\end{enumerate}

\section{Solution to 1}
Step 1 State the hypotheses.

Step 2 Identify the appropriate test statistic.

Step 3 Specify the level of significance.

Step $4 \quad$ State the decision rule.

Step 5 Calculate the test statistic.

Step 6 Make a decision. $H_{0}: \mu=0 \%$ versus $H_{a}: \mu \neq 0 \%$

$t=\frac{\bar{X}-\mu_{0}}{s / \sqrt{n}}$

with $1,304-1=1,303$ degrees of freedom.

$\alpha=5 \%$.

Critical $t$-values $= \pm 1.962$.

Reject the null if the calculated $t$-statistic is less than -1.962 , and reject the null if the calculated $t$-statistic is greater than +1.962 .

Excel

Lower: T.INV(0.025,1303)

Upper: $\mathrm{T} . \operatorname{INV}(0.975,1303)$

R $\operatorname{qt}(\mathrm{c}(.025, .975), 1303)$

Python from scipy.stats import $\mathrm{t}$

Lower: t.ppf(.025,1303)

Upper: t.ppf $(.975,1303)$

$t=\frac{0.0157-0}{0.3157 / \sqrt{1,304}}=1.79582$

Fail to reject the null because the calculated $t$-statistic falls between the two critical values. There is not sufficient evidence to indicate that the mean daily return is different from $0 \%$.

Solution to 2

Step 1 State the hypotheses.

Step 2 Identify the appropriate test statistic.

Step 3 Specify the level of significance. $H_{0}: \mu=0 \%$ versus $H_{a}: \mu \neq 0 \%$

$z=\frac{\bar{X}-\mu_{0}}{s / \sqrt{n}}$

with $1,304-1=1,303$ degrees of freedom.

$\alpha=5 \%$. Step 4 State the decision rule. $\quad$ Critical $t$-values $= \pm 1.960$.

Reject the null if the calculated $z$-statistic is less than -1.960 , and reject the null if the calculated $z$-statistic is greater than +1.960 .

\begin{center}
\includegraphics[max width=\textwidth]{2023_05_04_cff39ee44f77d6514e1bg-387}
\end{center}

Fail to reject the null because the calculated $z$-statistic falls between the two critical values. There is not sufficient evidence to indicate that the mean daily return is different from $0 \%$.

Step 5 Calculate the test statistic.

Step 6 Make a decision.

\textbackslash section\{TEST CONCERNING DIFFERENCES BETWEEN MEANS WITH INDEPENDENT SAMPLES

identify the appropriate test statistic and interpret the results for a hypothesis test concerning the equality of the population means of two at least approximately normally distributed populations based on independent random samples with equal assumed variances

We often want to know whether a mean value-for example, a mean return-differs for two groups. Is an observed difference due to chance or to different underlying values for the mean? We test this by drawing a sample from each group. When it is reasonable to believe that the samples are from populations at least approximately normally distributed and that the samples are also independent of each other, we use the test of the differences in the means. We may assume that population variances are equal or unequal. However, our focus in discussing the test of the differences of means is using the assumption that the population variances are equal. In the calculation of the test statistic, we combine the observations from both samples to obtain a pooled estimate of the common population variance.

Let $\mu_{1}$ and $\mu_{2}$ represent, respectively, the population means of the first and second populations. Most often we want to test whether the population means are equal or whether one is larger than the other. Thus, we formulate the following sets of hypotheses:

Two sided:

$H_{0}: \mu_{1}-\mu_{2}=0$ versus $H_{a}: \mu_{1}-\mu_{2} \neq 0$,

or, equivalently, $H_{0}: \mu_{1}=\mu_{2}$ versus $H_{a}: \mu_{1} \neq \mu_{2}$

One sided (right side):

$H_{0}: \mu_{1}-\mu_{2} \leq 0$ versus $H_{a}: \mu_{1}-\mu_{2}>0$,

or, equivalently,

$H_{0}: \mu_{1} \leq \mu_{2}$ versus $H_{a}: \mu_{1}>\mu_{2}$

One sided (left side):

$H_{0}: \mu_{1}-\mu_{2} \geq 0$ versus $H_{a}: \mu_{1}-\mu_{2}<0$,

or, equivalently,

$H_{0}: \mu_{1} \geq \mu_{2}$ versus $H_{a}: \mu_{1}<\mu_{2}$

We can, however, formulate other hypotheses, where the difference is something other than zero, such as $H_{0}: \mu_{1}-\mu_{2}=2$ versus $H_{a}: \mu_{1}-\mu_{2} \neq 2$. The procedure is the same.

When we can assume that the two populations are normally distributed and that the unknown population variances are equal, we use a $t$-distributed test statistic based on independent random samples:

$$
t=\frac{\left(\bar{X}_{1}-\bar{X}_{2}\right)-\left(\mu_{1}-\mu_{2}\right)}{\sqrt{\frac{s_{p}^{2}}{n_{1}}+\frac{s_{p}^{2}}{n_{2}}}},
$$

Where $s_{p}^{2}=\frac{\left(n_{1}-1\right) s_{1}^{2}+\left(n_{2}-1\right) s_{2}^{2}}{n_{1}+n_{2}-2}$ is a pooled estimator of the common variance. As you can see, the pooled estimate is a weighted average of the two samples' variances, with the degrees of freedom for each sample as the weight. The number of degrees of freedom for this $t$-distributed test statistic is $n_{1}+n_{2}-2$.

\section{EXAMPLE 9}
\section{Returns on the ACE High Yield Index Compared for Two}
 PeriodsContinuing the example of the returns in the ACE High Yield Total Return Index, suppose we want to test whether these returns, shown in Exhibit 16, are different for two different time periods, Period 1 and Period 2.

Exhibit 16: Descriptive Statistics for ACE High Yield Total Return Index for Periods 1 and 2

\begin{center}
\begin{tabular}{lll}
\hline
 & Period 1 & Period 2 \\
\hline
Mean & $0.01775 \%$ & $0.01134 \%$ \\
Standard deviation & $0.31580 \%$ & $0.38760 \%$ \\
Sample size & 445 days & 859 days \\
\hline
\end{tabular}
\end{center}

Note that these periods are of different lengths and the samples are independent; that is, there is no pairing of the days for the two periods.

Test whether there is a difference between the mean daily returns in Period 1 and in Period 2 using a 5\% level of significance. Step 1 State the hypotheses.

Step 2 Identify the appropriate test statistic.

Step 3 Specify the level of significance.

Step 4 State the decision rule.

Step 5 Calculate the test statistic.

Step 6 Make a decision.

TEST CONCERNING DIFFERENCES BETWEEN MEANS WITH DEPENDENT SAMPLES

identify the appropriate test statistic and interpret the results for a hypothesis test concerning the mean difference of two normally distributed populations

When we compare two independent samples, we use a $t$-distributed test statistic that uses the difference in the means and a pooled variance. An assumption for the validity of those tests is that the samples are independent-that is, unrelated to each other. When we want to conduct tests on two means based on samples that we believe are dependent, we use the test of the mean of the differences.

The $t$-test in this section is based on data arranged in paired observations, and the test itself is sometimes referred to as the paired comparisons test. Paired observations are observations that are dependent because they have something in common. For example, we may be concerned with the dividend policy of companies before and after a change in the tax law affecting the taxation of dividends. We then have pairs of observations for the same companies; these are dependent samples because we have pairs of the sample companies before and after the tax law change. We may test a hypothesis about the mean of the differences that we observe across companies. For example, we may be testing whether the mean returns earned by two investment strategies were equal over a study period. The observations here are dependent in the sense that there is one observation for each strategy in each month, and both observations depend on underlying market risk factors. What is being tested are the differences, and the paired comparisons test assumes that the differences are normally distributed. By calculating a standard error based on differences, we can use a $t$-distributed test statistic to account for correlation between the observations.

How is this test of paired differences different from the test of the differences in means in independent samples? The test of paired comparisons is more powerful than the test of the differences in the means because by using the common element (such as the same periods or companies), we eliminate the variation between the samples that could be caused by something other than what we are testing.

Suppose we have observations for the random variables $X_{A}$ and $X_{B}$ and that the samples are dependent. We arrange the observations in pairs. Let $d_{i}$ denote the difference between two paired observations. We can use the notation $d_{i}=x_{A i}-x_{B i}$, where $x_{\mathrm{A} i}$ and $x_{\mathrm{B} i}$ are the $i$ th pair of observations, $i=1,2, \ldots, n$, on the two variables. Let $\mu_{d}$ stand for the population mean difference. We can formulate the following hypotheses, where $\mu_{d 0}$ is a hypothesized value for the population mean difference:

Two sided: $H_{0}: \mu_{d}=\mu_{d 0}$ versus $H_{a}: \mu_{d} \neq \mu_{d 0}$

One sided (right side): $H_{0}: \mu_{d} \leq \mu_{d 0}$ versus $H_{a}: \mu_{d}>\mu_{d 0}$

One sided (left side) $H_{0}: \mu_{d} \geq \mu_{d 0}$ versus $H_{a}: \mu_{d}<\mu_{d 0}$

In practice, the most commonly used value for $\mu_{d 0}$ is zero.

We are concerned with the case of normally distributed populations with unknown population variances, and we use a $t$-distributed test statistic. We begin by calculating $\bar{d}$ , the sample mean difference, or the mean of the differences, $d_{i}$ :

$$
\bar{d}=\frac{1}{n} \sum_{i=1}^{n} d_{i}
$$

where $n$ is the number of pairs of observations. The sample standard deviation, $s_{d}$, is the standard deviation of the differences, and the standard error of the mean differences is $s_{\bar{d}}=\frac{s_{d}}{\sqrt{n}}$.

When we have data consisting of paired observations from samples generated by normally distributed populations with unknown variances, the $t$-distributed test statistic is

$$
t=\frac{\bar{d}-\mu_{d 0}}{s_{\bar{d}}}
$$

with $n-1$ degrees of freedom, where $n$ is the number of paired observations.

For example, suppose we want to see if there is a difference between the returns for Investments One and Two (from Exhibit 1), for which we have returns in each of 33 years. Using a 1\% level of significance, the critical values for a two-sided hypothesis test are \textbackslash pm 2.7385 . Lining up these returns by the years and calculating the differences, we find a sample mean difference $(\bar{d})$ of $0.10353 \%$ and a standard deviation of these differences $\left(s_{d}\right)$ of $2.35979 \%$. Therefore, the calculated $t$-statistic for testing whether the mean of the differences is equal to zero is

$$
t=\frac{0.10353-0}{2.35979 / \sqrt{33}}=0.25203
$$

with 32 degrees of freedom. In this case, we fail to reject the null because the $t$-statistic falls within the bounds of the two critical values. We conclude that there is not sufficient evidence to indicate that the returns for Investment One and Investment Two are different.

Importantly, if we think of the differences between the two samples as a single sample, then the test of the mean of differences is identical to the test of a single sample mean.

\section{EXAMPLE 10}
\section{Testing for the Mean of the Differences}
In Exhibit 17, we report the quarterly returns for a three-year period for two actively managed portfolios specializing in precious metals. The two portfolios are similar in risk and had nearly identical expense ratios. A major investment services company rated Portfolio B more highly than Portfolio A. In investigating the portfolios' relative performance, suppose we want to test the hypothesis that the mean quarterly return on Portfolio $A$ is equal to the mean quarterly return on Portfolio B during the three-year period. Since the two portfolios share essentially the same set of risk factors, their returns are not independent, so a paired comparisons test is appropriate. Use a $10 \%$ level of significance.

Exhibit 17: Quarterly Returns for Two Actively Managed Precious Metals Portfolios

\begin{center}
\begin{tabular}{lcccc}
\hline
Year & Quarter & $\begin{array}{c}\text { Portfolio A } \\ (\%)\end{array}$ & $\begin{array}{c}\text { Portfolio B } \\ (\%)\end{array}$ & $\begin{array}{c}\text { Difference } \\ \text { (Portfolio A - Portfolio B) }\end{array}$ \\
\hline
1 & 1 & 4.50 & 0.50 & 4.00 \\
1 & 2 & -4.10 & -3.10 & -1.00 \\
1 & 3 & -14.50 & -16.80 & 2.30 \\
1 & 4 & -6.50 & 1.28 &  \\
2 & 1 & 12.00 & -2.00 & 14.00 \\
2 & -7.97 & -8.96 & 0.99 &  \\
2 & -14.01 & -10.01 & -4.00 &  \\
2 & 4.11 & -6.31 & 10.42 &  \\
3 & 2.34 & -5.00 & 7.34 &  \\
3 & 26.36 & 12.77 & 13.59 &  \\
3 & 10.72 & 9.23 & 1.49 &  \\
3 & 3.60 & 1.20 & 2.40 &  \\
\hline
Average &  & 1.46 & -2.94 & 4.40083 \\
Standard deviation & 11.18 & 7.82 & 5.47434 &  \\
\hline
 &  &  &  &  \\
\hline
\end{tabular}
\end{center}

Using this sample information, we can summarize the test as follows:

Step 1 State the hypotheses.

Step 2 Identify the appropriate test statistic.

$$
\begin{aligned}
& H_{0}: \mu_{d 0}=0 \text { versus } H_{a}: \mu_{d 0} \neq 0 \\
& t=\frac{\bar{d}-\mu d 0}{s_{\bar{d}}}
\end{aligned}
$$

Step 3 State the decision rule.

Step 4 Calculate the test statistic. With $12-1=11$ degrees of freedom, the critical values are \textbackslash pm 1.796 .

We reject the null hypothesis if the calculated test statistic is below -1.796 or above +1.796 .

Excel

Lower: T.INV $(0.05,11)$

Upper: T.INV $(0.95,11)$

$\mathbf{R}$ qt(c $(.05, .95), 11)$

Python from scipy.stats import $t$

Lower: t.ppf $(.05,11)$

Upper: t.ppf $(.95,11)$

$\bar{d}=4.40083$

$s_{\bar{d}}=\frac{5.47434}{\sqrt{12}}=1.58031$

$t=\frac{4.40083-0}{1.58031}=2.78480$

Reject the null hypothesis because the calculated $t$-statistic falls outside the bounds of the two critical values. There is sufficient evidence to indicate that the mean of the differences of returns is not zero.

The following example illustrates the application of this test to evaluate two competing investment strategies.

\section{EXAMPLE 11}
\section{A Comparison of the Returns of Two Indexes}
\begin{enumerate}
  \item Suppose we want to compare the returns of the ACE High Yield Index with those of the ACE BBB Index. We collect data over 1,304 days for both indexes and calculate the means and standard deviations as shown in Exhibit 18.
\end{enumerate}

Exhibit 18: Mean and Standard Deviations for the ACE High Yield Index and the ACE BBB Index

\begin{center}
\begin{tabular}{lccc}
\hline
 & $\begin{array}{c}\text { ACE High Yield } \\ \text { Index } \\ (\%)\end{array}$ & $\begin{array}{c}\text { ACE BBB } \\ \text { Index (\%) }\end{array}$ & $\begin{array}{c}\text { Difference } \\ (\%)\end{array}$ \\
\hline
Mean return & 0.0157 & 0.0135 & -0.0021 \\
Standard deviation & 0.3157 & 0.3645 & 0.3622 \\
\hline
\end{tabular}
\end{center}

Using a 5\% level of significance, determine whether the mean of the differences is different from zero.

Solution

Step 1 State the hypotheses.

Step 2 Identify the appropriate test statistic.

Step 3 Specify the level of significance. $H_{0}: \mu_{d 0}=0$ versus $H_{a}: \mu_{d 0} \neq 0$

$t=\frac{\bar{d}-\mu_{d 0}}{s_{\bar{d}}}$

$5 \%$ Step 4 State the decision rule.

With $1,304-1=1,303$ degrees of freedom, the critical values are \textbackslash pm 1.962 . We reject the null hypothesis if the calculated $t$-statistic is less than -1.962 or greater than +1.962 .

Excel

Lower: T.INV(0.025,1303)

Upper: $\mathrm{T} . \mathrm{INV}(0.975,1303)$

$\mathbf{R} \operatorname{qt}(\mathrm{c}(.025, .975), 1303$

Python from scipy.stats import $\mathrm{t}$

Lower: t.ppf $(.025,1303)$

Upper: t.ppf(.975,1303)

Step 5 Calculate the test statistic.

$\bar{d}=-0.0021 \%$

$s_{\bar{d}}=\frac{s_{d}}{\sqrt{n}}=\frac{0.3622}{\sqrt{1,304}}=0.01003 \%$

$t=\frac{-0.00210-0}{0.01003}=-0.20937$

Step 6 Make a decision.

Fail to reject the null hypothesis because the calculated $t$-statistic falls within the bounds of the two critical values. There is insufficient evidence to indicate that the mean of the differences of returns is different from zero.

TESTING CONCERNING TESTS OF VARIANCES

identify the appropriate test statistic and interpret the results for a hypothesis test concerning (1) the variance of a normally distributed population and (2) the equality of the variances of two normally distributed populations based on two independent random samples

Often, we are interested in the volatility of returns or prices, and one approach to examining volatility is to evaluate variances. We examine two types of tests involving variance: tests concerning the value of a single population variance and tests concerning the difference between two population variances.

\section{Tests of a Single Variance}
Suppose there is a goal to keep the variance of a fund's returns below a specified target. In this case, we would want to compare the observed sample variance of the fund with the target. Performing a test of a population variance requires specifying the hypothesized value of the variance, $\sigma_{0}^{2}$. We can formulate hypotheses concerning whether the variance is equal to a specific value or whether it is greater than or less than a hypothesized value:

Two-sided alternative: $H_{0}: \sigma^{2}=\sigma_{0}^{2}$ versus $H_{a}: \sigma^{2} \neq \sigma_{0}^{2}$

One-sided alternative (right tail): $H_{0}: \sigma^{2} \leq \sigma_{0}^{2}$ versus $H_{a}: \sigma^{2}>\sigma_{0}^{2}$

One-sided alternative (left tail): $H_{0}: \sigma^{2} \geq \sigma_{0}^{2}$ versus $H_{a}: \sigma^{2}<\sigma_{0}^{2}$

In tests concerning the variance of a single normally distributed population, we make use of a chi-square test statistic, denoted $\chi^{2}$. The chi-square distribution, unlike the normal distribution and $t$-distribution, is asymmetrical. Like the $t$-distribution, the chi-square distribution is a family of distributions, with a different distribution for each possible value of degrees of freedom, $n-1$ ( $n$ is sample size). Unlike the $t$-distribution, the chi-square distribution is bounded below by zero; $\chi^{2}$ does not take on negative values.

If we have $n$ independent observations from a normally distributed population, the appropriate test statistic is

$$
\chi^{2}=\frac{(n-1) s^{2}}{\sigma_{0}^{2}}
$$

with $n-1$ degrees of freedom. The sample variance $\left(s^{2}\right)$ is in the numerator, and the hypothesized variance $\left(\sigma_{0}^{2}\right)$ is in the denominator.

In contrast to the $t$-test, for example, the chi-square test is sensitive to violations of its assumptions. If the sample is not random or if it does not come from a normally distributed population, inferences based on a chi-square test are likely to be faulty.

Since the chi-square distribution is asymmetric and bounded below by zero, we no longer have the convenient \textbackslash pm for critical values as we have with the $z$ - and the $t$-distributions, so we must either use a table of chi-square values or use software to generate the critical values. Consider a sample of 25 observations, so we have 24 degrees of freedom. We illustrate the rejection regions for the two- and one-sided tests at the 5\% significance level in Exhibit 19. Exhibit 19: Rejection Regions (Shaded) for the Chi-Square Distribution (df = 24) at $5 \%$ Significance

$$
\text { A. } H_{0}: \sigma^{2}=\sigma_{0}^{2} \text { versus } H_{a}: \sigma^{2} \neq \sigma_{0}^{2}
$$

Critical values are 12.40115 and 39.36408

\begin{center}
\includegraphics[max width=\textwidth]{2023_05_04_cff39ee44f77d6514e1bg-395(1)}
\end{center}

B. $H_{0}: \sigma^{2} \leq \sigma_{0}^{2}$ versus $H_{a}: \sigma^{2}>\sigma_{0}^{2}$

Critical value is 36.41503

\begin{center}
\includegraphics[max width=\textwidth]{2023_05_04_cff39ee44f77d6514e1bg-395(2)}
\end{center}

C. $H_{0}: \sigma^{2} \geq \sigma_{0}^{2}$ versus $H_{a}: \sigma^{2}<\sigma_{0}^{2}$

Critical value is 13.84843

\begin{center}
\includegraphics[max width=\textwidth]{2023_05_04_cff39ee44f77d6514e1bg-395}
\end{center}

\section{EXAMPLE 12}
\section{Risk and Return Characteristics of an Equity Mutual Fund}
\begin{enumerate}
  \item You continue with your analysis of Sendar Equity Fund, a midcap growth fund that has been in existence for only 24 months. During this period, Sendar Equity achieved a mean monthly return of $1.50 \%$ and a standard deviation of monthly returns of $3.60 \%$.

  \item Using a $5 \%$ level of significance, test whether the standard deviation of returns is different from $4 \%$.

  \item Using a $5 \%$ level of significance, test whether the standard deviation of returns is less than $4 \%$.

\end{enumerate}

\section{Solution to 1}
Step 1 State the hypotheses.

Step 2 Identify the appropriate test statistic.

Step 3 Specify the level of significance.

Step 4 State the decision rule.

Step 5 Calculate the test statistic.

Step 6 Make a decision $H_{0}: \sigma^{2}=16$ versus $H_{a}: \sigma^{2} \neq 16$

$X^{2}=\frac{(n-1) s^{2}}{\sigma_{0}^{2}}$

$5 \%$

With $24-1=23$ degrees of freedom, the critical values are 11.68855 and 38.07563.

We reject the null hypothesis if the calculated $\chi^{2}$ statistic is less than 11.68855 or greater than 38.07563 .

Excel

Lower: CHISQ.INV(0.025,23)

Upper: CHISQ.INV(0.975,23)

$\mathbf{R}$ qchisq(c(.025,.975),23)

Python from scipy.stats import chi2

Lower: chi2.ppf $(.025,23)$

Upper: chi2.ppf(.975,23)

$X^{2}=\frac{(24-1) 12.96}{16}=18.63000$

Fail to reject the null hypothesis because the calculated $\chi^{2}$ statistic falls within the bounds of the two critical values. There is insufficient evidence to indicate that the variance is different from $16 \%$ (or, equivalently, that the standard deviation is different from $4 \%$ ).

\section{Solution to 2:}
Step 1 State the hypotheses.

Step 2 Identify the appropriate test statistic.

Step 3 Specify the level of significance.

Step 4 State the decision rule. $H_{0}: \sigma^{2} \geq 16$ versus $H_{a}: \sigma^{2}<16$

$X^{2}=\frac{(n-1) s^{2}}{\sigma_{0}^{2}}$

\section{$5 \%$}
With $24-1=23$ degrees of freedom, the critical value is 13.09051.

We reject the null hypothesis if the calculated $\chi^{2}$ statistic is less than 13.09051. Excel CHISQ.INV $(0.05,23)$

$\mathbf{R}$ qchisq(.05,23)

Python from scipy.stats import chi 2

$$
\text { chi2.ppf }(.05,23)
$$

Step 5 Calculate the test statistic.

$X^{2}=\frac{(24-1) 12.96}{16}=18.63000$

Step 6 Make a decision.

Fail to reject the null hypothesis because the calculated $\chi^{2}$ statistic is greater than the critical value. There is insufficient evidence to indicate that the variance is less than $16 \%$ (or, equivalently, that the standard deviation is less than $4 \%$ ).

\section{Test Concerning the Equality of Two Variances (F-Test)}
There are many instances in which we want to compare the volatility of two samples, in which case we can test for the equality of two variances. Examples include comparisons of baskets of securities against indexes or benchmarks, as well as comparisons of volatility in different periods. Suppose we have a hypothesis about the relative values of the variances of two normally distributed populations with variances of $\sigma_{1}^{2}$ and $\sigma_{2}^{2}$, distinguishing the two populations as 1 or 2 . We can formulate the hypotheses as two sided or one sided:

Two-sided alternative:

$H_{0}: \sigma_{1}^{2}=\sigma_{2}^{2}$ versus $H_{a}: \sigma_{1}^{2} \neq \sigma_{2}^{2}$

or, equivalently,

$H_{0}: \frac{\sigma_{1}^{2}}{\sigma_{2}^{2}}=1$ versus $H_{a}: \frac{\sigma_{1}^{2}}{\sigma_{2}^{2}} \neq 1$

One-sided alternative (right side):

$H_{0}: \sigma_{1}^{2} \leq \sigma_{2}^{2}$ versus $H_{a}: \sigma_{1}^{2}>\sigma_{2}^{2}$

or, equivalently,

$H_{0}: \frac{\sigma_{1}^{2}}{\sigma_{2}^{2}} \leq 1$ versus $H_{a}: \frac{\sigma_{1}^{2}}{\sigma_{2}^{2}}>1$

One-sided alternative (left side):

$H_{0}: \sigma_{1}^{2} \geq \sigma_{2}^{2}$ versus $H_{a}: \sigma_{1}^{2}<\sigma_{2}^{2}$

or, equivalently,

$H_{0}: \frac{\sigma_{1}^{2}}{\sigma_{2}^{2}} \geq 1$ versus $H_{a}: \frac{\sigma_{1}^{2}}{\sigma_{2}^{2}}<1$

Given independent random samples from these populations, tests related to these hypotheses are based on an F-test, which is the ratio of sample variances. Tests concerning the difference between the variances of two populations make use of the $F$-distribution. Like the chi-square distribution, the $F$-distribution is a family of asymmetrical distributions bounded from below by zero. Each $F$-distribution is defined by two values of degrees of freedom, which we refer to as the numerator and denominator degrees of freedom. The $F$-test, like the chi-square test, is not robust to violations of its assumptions.

Suppose we have two samples, the first with $n_{1}$ observations and a sample variance $s_{1}^{2}$ and the second with $n_{2}$ observations and a sample variance $s_{2}^{2}$. The samples are random, independent of each other, and generated by normally distributed populations. A test concerning differences between the variances of the two populations is based on the ratio of sample variances, as follows:

$$
F=\frac{s_{1}^{2}}{s_{2}^{2}}
$$

with $\mathrm{df}_{1}=\left(n_{1}-1\right)$ numerator degrees of freedom and $\mathrm{df}_{2}=\left(n_{2}-1\right)$ denominator degrees of freedom. Note that $\mathrm{df}_{1}$ and $\mathrm{df}_{2}$ are the divisors used in calculating $s_{1}^{2}$ and $s_{2}^{2}$, respectively.

When we rely on tables to arrive at critical values, a convention is to use the larger of the two sample variances in the numerator in Equation 9; doing so reduces the number of $F$-tables needed. The key is to be consistent with how the alternative hypothesis is specified and the order of the sample sizes for the degrees of freedom.

Consider two samples, the first with 25 observations and the second with 40 observations. We show the rejection region and critical values in Exhibit 20 for twoand one-sided alternative hypotheses at the $5 \%$ significance level. Exhibit 20: Rejection Regions (Shaded) for the F-Distribution Based on Sample Sizes of 25 and 40 at $5 \%$ Significance

\section{A. $H_{0}: \sigma_{1}^{2}=\sigma_{2}^{2}$ versus $H_{a}: \sigma_{1}^{2} \neq \sigma_{2}^{2}$}
Critical values are 0.49587 and 2.15095

\begin{center}
\includegraphics[max width=\textwidth]{2023_05_04_cff39ee44f77d6514e1bg-399(2)}
\end{center}

Critical value is 1.89566

\begin{center}
\includegraphics[max width=\textwidth]{2023_05_04_cff39ee44f77d6514e1bg-399}
\end{center}

C. $H_{0}: \sigma_{1}^{2} \geq \sigma_{2}^{2}$ versus $H_{a}: \sigma_{1}^{2}<\sigma_{2}^{2}$

Critical value is 0.55551

\begin{center}
\includegraphics[max width=\textwidth]{2023_05_04_cff39ee44f77d6514e1bg-399(1)}
\end{center}

Consider Investments One and Two (from Exhibit 1), with standard deviations of returns of 1.4284 and 2.5914 , respectively, calculated over the 33-year period. If we want to know whether the variance of Investment One is different from that of Investment Two, we use the $F$-distributed test statistic. With 32 and 32 degrees of freedom, the critical values are 0.49389 and 2.02475 at the $5 \%$ significance level. The calculated $F$-statistic is

$$
F=\frac{2.5914^{2}}{1.4284^{2}}=3.29131
$$

Therefore, we reject the null hypothesis that the variances of these two investments are the same because the calculated $F$-statistic is outside of the critical values. We can conclude that one investment is riskier than the other.

\section{EXAMPLE 13}
\section{Volatility and Regulation}
You are investigating whether the population variance of returns on a stock market index changed after a change in market regulation. The first 418 weeks occurred before the regulation change, and the second 418 weeks occurred after the regulation change. You gather the data in Exhibit 21 for 418 weeks of returns both before and after the change in regulation. You have specified a 5\% level of significance.

Exhibit 21: Index Returns and Variances before and after the Market Regulation Change

\begin{center}
\begin{tabular}{lccc}
\hline
 & $\boldsymbol{n}$ & $\begin{array}{c}\text { Mean Weekly Return } \\ \text { (\%) }\end{array}$ & Variance of Returns \\
\hline
Before regulation change & 418 & 0.250 & 4.644 \\
After regulation change & 418 & 0.110 & 3.919 \\
\hline
\end{tabular}
\end{center}

\begin{enumerate}
  \item Test whether the variance of returns is different before the regulation change versus after the regulation change, using a $5 \%$ level of significance.
\end{enumerate}

\section{Solution to 1}
Step 1 State the hypotheses.

Step 2 Identify the appropriate test statistic.

Step 3 Specify the level of significance.

Step 4 State the decision rule.

Step 5 Calculate the test statistic.

Step 6 Make a decision. $H_{0}: \sigma_{\text {Before }}^{2}=\sigma_{\text {After }}^{2}$ versus $H_{a}: \sigma_{\text {Before }}^{2} \neq \sigma_{\text {After }}^{2}$

$F=\frac{s_{\text {Before }}^{2}}{s_{A f t e r}^{2}}$

$5 \%$

With $418-1=417$ and $418-1=417$ degrees of freedom, the critical values are 0.82512 and 1.21194 .

Reject the null if the calculated $F$-statistic is less than 0.82512 or greater than 1.21194.

Excel

Left side: F.INV(0.025,417,417)

Right side: F.INV $(0.975,417,417)$

R $\operatorname{qf}(\mathrm{c}(.025, .975), 417,417)$

Python from scipy.stats import $\mathrm{f}$

Left side: f.ppf(.025,417,417)

Right side: f.ppf(.975,417,417)

$F=\frac{4.644}{3.919}=1.18500$

Fail to reject the null hypothesis since the calculated $F$-statistic falls within the bounds of the two critical values. There is not sufficient evidence to indicate that the weekly variances of returns are different in the periods before and after the regulation change. 2. Test whether the variance of returns is greater before the regulation change versus after the regulation change, using a $5 \%$ level of significance.

\section{Solution to 2}
Step 1 State the hypotheses. $\quad H_{0}: \sigma_{\text {Before }}^{2} \leq \sigma_{\text {After }}^{2}$ versus $H_{a}: \sigma_{\text {Before }}^{2}>\sigma_{\text {After }}^{2}$

Step 2 Identify the appropriate test statistic.

$$
F=\frac{s_{\text {Before }}^{2}}{s_{A f t e r}^{2}}
$$

Step 3 Specify the level of significance.

$5 \%$

Step 4 State the decision rule.

With $418-1=417$ and $418-1=417$ degrees of freedom, the critical value is 1.17502 .

We reject the null hypothesis if the calculated $F$-statistic is greater than 1.17502.

Excel F.INV $(0.95,417,417)$

R $\mathrm{qf}(.95,417,417)$

Python from scipy.stats import $\mathrm{f}$

$$
\text { f.ppf(.95,417,417) }
$$

Step 5 Calculate the test statistic.

$F=\frac{4.644}{3.919}=1.18500$

Step 6 Make a decision.

Reject the null hypothesis since the calculated $F$-statistic is greater than 1.17502. There is sufficient evidence to indicate that the weekly variances of returns before the regulation change are greater than the variances after the regulation change.

\section{EXAMPLE 14}
The Volatility of Derivatives Expiration Days

\begin{enumerate}
  \item You are interested in investigating whether quadruple witching days-that is, the occurrence of stock option, index option, index futures, and single stock futures expirations on the same day-exhibit greater volatility than normal trading days. Exhibit 22 presents the daily standard deviation of returns for normal trading days and quadruple witching days during a four-year period.
\end{enumerate}

Exhibit 22: Standard Deviation of Returns: Normal Trading Days and Derivatives Expiration Days

\begin{center}
\begin{tabular}{llrl}
\hline
Period & Type of Day & $\boldsymbol{n}$ & Standard Deviation (\%) \\
\hline
1 & Normal trading days & 138 & 0.821 \\
2 & Quadruple witching days & 16 & 1.217 \\
\hline
\end{tabular}
\end{center}

Test to determine whether the variance of returns for quadruple witching days is greater than the variance for non-expiration, normal trading days. Use a $5 \%$ level of significance.

\section{Solution}
Step 1 State the hypotheses.

Step 2 Identify the appropriate test statistic.

Step 3 Specify the level of significance.

Step 4 State the decision rule.

Step 5 Calculate the test statistic.

Step 6 Make a decision. $H_{0}: \sigma_{\text {Period } 2}^{2} \leq \sigma_{\text {Period } 1}^{2}$ versus $H_{a}: \sigma_{\text {Period } 2}^{2}>\sigma_{\text {Period } 1}^{2}$

$F=\frac{s_{\text {Period } 2}^{2}}{s_{\text {Period } 1}^{2}}$

\section{$5 \%$}
With $16-1=15$ and $138-1=137$ degrees of freedom, the critical value is 1.73997.

We reject the null hypothesis if the calculated $F$-statistic is greater than 1.73997.

Excel F.INV(0.95,15,137)

$\mathbf{R}$ qf $(.95,15,137)$

Python from scipy.stats import $\mathrm{f}$

f.ppf $(.95,15,137)$

$F=\frac{1.48109}{0.67404}=2.19733$

Reject the null hypothesis since the calculated $F$-statistic is greater than 1.73997. There is sufficient evidence to indicate that the variance of returns for quadruple witching days is greater than the variance for normal trading days.

\section{PARAMETRIC VS. NONPARAMETRIC TESTS}
compare and contrast parametric and nonparametric tests, and describe situations where each is the more appropriate type of test

The hypothesis-testing procedures we have discussed up to this point have two characteristics in common. First, they are concerned with parameters, and second, their validity depends on a definite set of assumptions. Mean and variance, for example, are two parameters, or defining quantities, of a normal distribution. The tests also make specific assumptions-in particular, assumptions about the distribution of the population producing the sample. Any test or procedure with either of these two characteristics is a parametric test or procedure. In some cases, however, we are concerned about quantities other than parameters of distributions. In other cases, we may believe that the assumptions of parametric tests do not hold. In cases where we are examining quantities other than population parameters or where assumptions of the parameters are not satisfied, a nonparametric test or procedure can be useful.

A nonparametric test is a test that is not concerned with a parameter or a test that makes minimal assumptions about the population from which the sample comes. In Exhibit 23, we give examples of nonparametric alternatives to the parametric, $t$-distributed tests concerning means. Exhibit 23: Nonparametric Alternatives to Parametric Tests Concerning Means

\begin{center}
\begin{tabular}{|c|c|c|}
\hline
 & Parametric & Nonparametric \\
\hline
Tests concerning a single mean & $\begin{array}{l}t \text {-distributed test } \\ z \text {-distributed test }\end{array}$ & Wilcoxon signed-rank tes \\
\hline
$\begin{array}{l}\text { Tests concerning differences } \\ \text { between means }\end{array}$ & $t$-distributed test & $\begin{array}{l}\text { Mann-Whitney } U \text { test } \\ \text { (Wilcoxon rank sum test) }\end{array}$ \\
\hline
$\begin{array}{l}\text { Tests concerning mean differences } \\ \text { (paired comparisons tests) }\end{array}$ & $t$-distributed test & $\begin{array}{l}\text { Wilcoxon signed-rank tes } \\ \text { Sign test }\end{array}$ \\
\hline
\end{tabular}
\end{center}

\section{Uses of Nonparametric Tests}
We primarily use nonparametric procedures in four situations: (1) when the data we use do not meet distributional assumptions, (2) when there are outliers, (3) when the data are given in ranks or use an ordinal scale, or (4) when the hypotheses we are addressing do not concern a parameter.

The first situation occurs when the data available for analysis suggest that the distributional assumptions of the parametric test are not satisfied. For example, we may want to test a hypothesis concerning the mean of a population but believe that neither $t$-nor $z$-distributed tests are appropriate because the sample is small and may come from a markedly non-normally distributed population. In that case, we may use a nonparametric test. The nonparametric test will frequently involve the conversion of observations (or a function of observations) into ranks according to magnitude, and sometimes it will involve working with only "greater than" or "less than" relationships (using the + and - signs to denote those relationships). Characteristically, one must refer to specialized statistical tables to determine the rejection points of the test statistic, at least for small samples. Such tests, then, typically interpret the null hypothesis as a hypothesis about ranks or signs.

Second, whereas the underlying distribution of the population may be normal, there may be extreme values or outliers that influence the parametric statistics but not the nonparametric statistics. For example, we may want to use a nonparametric test of the median, in the case of outliers, instead of a test of the mean.

Third, we may have a sample in which observations are ranked. In those cases, we also use nonparametric tests because parametric tests generally require a stronger measurement scale than ranks. For example, if our data were the rankings of investment managers, we would use nonparametric procedures to test the hypotheses concerning those rankings.

A fourth situation in which we use nonparametric procedures occurs when our question does not concern a parameter. For example, if the question concerns whether a sample is random or not, we use the appropriate nonparametric test (a "runs test"). The nonparametric runs test is used to test whether stock price changes can be used to forecast future stock price changes-in other words, a test of the random-walk theory. Another type of question that nonparametric methods can address is whether a sample came from a population following a particular probability distribution.

\section{Nonparametric Inference: Summary}
Nonparametric statistical procedures extend the reach of inference because they make few assumptions, can be used on ranked data, and may address questions unrelated to parameters. Quite frequently, nonparametric tests are reported alongside parametric tests; the user can then assess how sensitive the statistical conclusion is to the assumptions underlying the parametric test. However, if the assumptions of the parametric test are met, the parametric test (where available) is generally preferred over the nonparametric test because the parametric test may have more power-that is, a greater ability to reject a false null hypothesis.

\section{EXAMPLE 15}
\section{The Use of Nonparametric Tests}
\begin{enumerate}
  \item A nonparametric test is most appropriate when the:
\end{enumerate}

A. data consist of ranked values.

B. validity of the test depends on many assumptions.

C. sample sizes are large but are drawn from a population that may be non-normal.

\section{Solution}
A is correct. When the samples consist of ranked values, parametric tests are not appropriate. In such cases, nonparametric tests are most appropriate.

\section{4}
\section{TESTS CONCERNING CORRELATION}
explain parametric and nonparametric tests of the hypothesis that the population correlation coefficient equals zero, and determine whether the hypothesis is rejected at a given level of significance

In many contexts in investments, we want to assess the strength of the linear relationship between two variables; that is, we want to evaluate the correlation between them. A significance test of a correlation coefficient allows us to assess whether the relationship between two random variables is the result of chance. If we decide that the relationship does not result from chance, then we are inclined to use this information in modeling or forecasting.

If the correlation between two variables is zero, we conclude that there is no linear relation between the two variables. We use a test of significance to assess whether the correlation is different from zero. After we estimate a correlation coefficient, we need to ask whether the estimated correlation is significantly different from zero.

A correlation may be positive (that is, the two variables tend to move in the same direction at the same time) or negative (that is, the two variables tend to move in different directions at the same time). The correlation coefficient is a number between -1 and +1 , where -1 denotes a perfect negative or inverse, straight-line relationship between the two variables; +1 denotes a perfect positive, straight-line relationship; and 0 represents the absence of any straight-line relationship (that is, no correlation).

The most common hypotheses concerning correlation occur when comparing the population correlation coefficient with zero because we are often asking whether there is a relationship, which implies a null of the correlation coefficient equal to zero (that is, no relationship). Hypotheses concerning the population correlation coefficient may be two or one sided, as we have seen in other tests. Let $\rho$ represent the population correlation coefficient. The possible hypotheses are as follows:

Two sided: $H_{0}: \rho=0$ versus $H_{a}: \rho \neq 0$

One sided (right side): $H_{0}: \rho \leq 0$ versus $H_{a}: \rho>0$

One sided (left side): $H_{0}: \rho \geq 0$ versus $H_{a}: \rho<0$

We use the sample correlation to test these hypotheses on the population correlation.

\section{Parametric Test of a Correlation}
The parametric pairwise correlation coefficient is often referred to as the Pearson correlation, the bivariate correlation, or simply the correlation. Our focus is on the testing of the correlation and not the actual calculation of this statistic, but it helps distinguish this correlation from the nonparametric correlation if we look at the formula for the sample correlation. Consider two variables, $X$ and $Y$. The sample correlation, $r_{X Y}$, is

$$
r_{X Y}=\frac{s_{X Y}}{s_{X} s_{Y}}
$$

where $s_{X Y}$ is the sample covariance between the $X$ and $Y$ variables, $s_{X}$ is the standard deviation of the $X$ variable, and $s_{Y}$ is the standard deviation of the $Y$ variable. We often drop the subscript to represent the correlation as simply $r$.

Therefore, you can see from this formula that each observation is compared with its respective variable mean and that, because of the covariance, it matters how much each observation differs from its respective variable mean. Note that the covariance drives the sign of the correlation.

If the two variables are normally distributed, we can test to determine whether the null hypothesis $\left(H_{0}: \rho=0\right)$ should be rejected using the sample correlation, $r$. The formula for the $t$-test is

$$
t=\frac{r \sqrt{n-2}}{\sqrt{1-r^{2}}}
$$

This test statistic is $t$-distributed with $n-2$ degrees of freedom. One practical observation concerning Equation 10 is that the magnitude of $r$ needed to reject the null hypothesis decreases as sample size $n$ increases, for two reasons. First, as $n$ increases, the number of degrees of freedom increases and the absolute value of the critical value of the $t$-statistic decreases. Second, the absolute value of the numerator increases with larger $n$, resulting in a larger magnitude of the calculated $t$-statistic. For example, with sample size $n=12, r=0.35$ results in a $t$-statistic of 1.182 , which is not different from zero at the 0.05 level $\left(t_{\alpha / 2}= \pm 2.228\right)$. With a sample size of $n=$ 32 , the same sample correlation, $r=0.35$, yields a $t$-statistic of 2.046 , which is just significant at the 0.05 level $\left(t_{\alpha / 2}= \pm 2.042\right)$.

Another way to make this point is that when sampling from the same population, a false null hypothesis is more likely to be rejected (that is, the power of the test increases) as we increase the sample size, all else equal, because a higher number of observations increases the numerator of the test statistic. We show this in Exhibit 24 for three different sample correlation coefficients, with the corresponding calculated $t$-statistics and significance at the $5 \%$ level for a two-sided alternative hypothesis. As the sample size increases, significance is more likely to be indicated, but the rate of achieving this significance depends on the sample correlation coefficient; the higher the sample correlation, the faster significance is achieved when increasing the sample size. As the sample sizes increase as ever-larger datasets are examined, the null hypothesis is almost always rejected and other tools of data analysis must be applied. Exhibit 24: Calculated Test Statistics for Different Sample Sizes and Sample

Correlations with a $5 \%$ Level of Significance

\begin{center}
\includegraphics[max width=\textwidth]{2023_05_04_cff39ee44f77d6514e1bg-406}
\end{center}

\section{EXAMPLE 16}
\section{Examining the Relationship between Returns on Investment One and Investment Two}
\begin{enumerate}
  \item An analyst is examining the annual returns for Investment One and Investment Two, as displayed in Exhibit 1. Although this time series plot provides some useful information, the analyst is most interested in quantifying how the returns of these two series are related, so she calculates the correlation coefficient, equal to 0.43051 , between these series.
\end{enumerate}

Is there a significant positive correlation between these two return series if she uses a $1 \%$ level of significance?

\section{Solution}
Step 1 State the hypotheses.

Step 2 Identify the appropriate test statistic.

Step 3 Specify the level of significance.

Step 4 State the decision rule

Step 5 Calculate the test statistic.

Step 6 Make a decision $H_{0}: \rho \leq 0$ versus $H_{a}: \rho>0$

$t=\frac{r \sqrt{n-2}}{\sqrt{1-r^{2}}}$

$1 \%$

With $33-2=31$ degrees of freedom and a one-sided test with a $1 \%$ level of significance, the critical value is 2.45282 .

We reject the null hypothesis if the calculated $t$-statistic is greater than 2.45282.

$t=\frac{0.43051 \sqrt{33-2}}{\sqrt{1-0.18534}}=2.65568$

Reject the null hypothesis since the calculated $t$-statistic is greater than 2.45282. There is sufficient evidence to reject the $H_{0}$ in favor of $H_{a}$, that the correlation between the annual returns of these two investments is positive.

\section{Tests Concerning Correlation: The Spearman Rank Correlation}
\section{Coefficient}
When we believe that the population under consideration meaningfully departs from normality, we can use a test based on the Spearman rank correlation coefficient, $r_{S}$. The Spearman rank correlation coefficient is essentially equivalent to the usual correlation coefficient but is calculated on the ranks of the two variables (say, $X$ and $Y$ ) within their respective samples. The calculation of $r_{S}$ requires the following steps:

\begin{enumerate}
  \item Rank the observations on $X$ from largest to smallest. Assign the number 1 to the observation with the largest value, the number 2 to the observation with second largest value, and so on. In case of ties, assign to each tied observation the average of the ranks that they jointly occupy. For example, if the third and fourth largest values are tied, we assign both observations the rank of 3.5 (the average of 3 and 4). Perform the same procedure for the observations on $Y$.

  \item Calculate the difference, $d_{i}$, between the ranks for each pair of observations on $X$ and $Y$, and then calculate $d_{i}^{2}$ (the squared difference in ranks).

  \item With $n$ as the sample size, the Spearman rank correlation is given by

\end{enumerate}

$r_{s}=1-\frac{6 \sum_{i=1}^{n} d_{i}^{2}}{n\left(n^{2}-1\right)}$

Suppose an analyst is examining the relationship between returns for two investment funds, $A$ and B, of similar risk over 35 years. She is concerned that the assumptions for the parametric correlation may not be met, so she decides to test Spearman rank correlations. Her hypotheses are $H_{0}: r_{S}=0$ and $H_{a}: r_{S} \neq 0$. She gathers the returns, ranks the returns for each fund, and calculates the difference in ranks and the squared differences. A partial table is provided in Exhibit 25.

Exhibit 25: Differences and Squared Differences in Ranks for Fund A and Fund B over 35 Years

\begin{center}
\begin{tabular}{lcccccc}
\hline
Year & Fund A & Fund B & Rank of A & Rank of B & $\boldsymbol{d}$ & $\boldsymbol{d}^{\mathbf{2}}$ \\
\hline
1 & 2.453 & 1.382 & 27 & 31 & -4 & 16 \\
2 & 3.017 & 3.110 & 24 & 24 & 0 & 0 \\
3 & 4.495 & 6.587 & 19 & 7 & 12 & 144 \\
4 & 3.627 & 3.300 & 23 & 23 & 0 & 0 \\
. &  &  &  &  &  &  \\
. &  &  &  &  &  &  \\
30 & 2.269 & 0.025 & 28 & 35 & -7 & 49 \\
31 & 6.354 & 4.428 & 10 & 19 & -9 & 81 \\
32 & 6.793 & 4.165 & 8 & 20 & -12 & 144 \\
33 & 7.300 & 7.623 & 5 & 5 & 0 & 0 \\
34 & 6.266 & 4.527 & 11 & 18 & -7 & 49 \\
35 & 1.257 & 4.704 & 34 & 16 & 18 & 324 \\
\hline
\end{tabular}
\end{center}

The Spearman rank correlation is:

$$
r_{s}=1-\frac{6 \sum_{i=1}^{n} d_{i}^{2}}{n\left(n^{2}-1\right)}=1-\frac{6(2,202)}{35(1,225-1)}=0.6916 .
$$

The test of hypothesis for the Spearman rank correlation depends on whether the sample is small or large $(n>30)$. For small samples, the researcher requires a specialized table of critical values, but for large samples, we can conduct a $t$-test using the test statistic in Equation 10, which is $t$-distributed with $n-2$ degrees of freedom.

In this example, for a two-tailed test with a $5 \%$ significance level, the critical values for $n-2=35-2=33$ degrees of freedom are \textbackslash pm 2.0345 . For the sample information in Exhibit 24, the calculated test statistic is

$$
t=\frac{0.6916 \sqrt{33}}{\sqrt{1-\left(0.6916^{2}\right)}}=5.5005 .
$$

Accordingly, we reject the null hypothesis $\left(H_{0}: r_{S}=0\right)$, concluding that there is sufficient evidence to indicate that the correlation between the returns of Fund $\mathrm{A}$ and Fund $B$ is different from zero.

\section{EXAMPLE 17}
\section{Testing the Exchange Rate Correlation}
\begin{enumerate}
  \item An analyst gathers exchange rate data for five currencies relative to the US dollar. Upon inspection of the distribution of these exchange rates, she observes a departure from normality, especially with negative skewness for four of the series and positive skewness for the fifth. Therefore, she decides to examine the relationships among these currencies using Spearman rank correlations. She calculates these correlations between the currencies over 180 days, which are shown in the correlogram in Exhibit 26. In this correlogram, the lower triangle reports the pairwise correlations and the upper triangle provides a visualization of the magnitude of the correlations, with larger circles indicating larger absolute value of the correlations and darker circles indicating correlations that are negative.
\end{enumerate}

Exhibit 26: Spearman Rank Correlations between Exchanges Rates Relative to the US Dollar

\begin{center}
\includegraphics[max width=\textwidth]{2023_05_04_cff39ee44f77d6514e1bg-408}
\end{center}

For any of these pairwise Spearman rank correlations, can we reject the null hypothesis of no correlation $\left(H_{0}: r_{S}=0\right.$ and $\left.H_{a}: r_{S} \neq 0\right)$ at the $5 \%$ level of significance?

\section{Solution}
The critical $t$-values for $2.5 \%$ in each tail of the distribution are \textbackslash pm 1.97338 . There are five exchange rates, so there are ${ }_{5} \mathrm{C}_{2}$, or 10 , unique correlation pairs. Therefore, we need to calculate $10 t$-statistics. For example, the correlation between EUR/USD and AUD/USD is 0.6079. The calculated $t$-statistic is $\frac{0.6079 \sqrt{180-2}}{\sqrt{1-0.6079^{2}}}=\frac{8.11040}{0.79401}=10.2144$. Repeating this $t$-statistic calculation for each pair of exchange rates yields the test statistics shown in Exhibit 27.

Exhibit 27: Calculated Test Statistics for Test of Spearman Rank Correlations

\begin{center}
\begin{tabular}{lcccc}
\hline
 & AUD/USD & CAD/USD & EUR/USD & GBP/USD \\
\hline
CAD/USD & 29.7409 &  &  &  \\
EUR/USD & 10.2144 & 9.1455 &  &  \\
GBP/USD & 12.4277 & 13.2513 & 7.4773 &  \\
JPY/USD & -2.6851 & -3.6726 & 5.2985 & -2.7887 \\
\hline
\end{tabular}
\end{center}

The analyst should reject all 10 null hypotheses, because the calculated $t$-statistics for all exchange rate pairs fall outside the bounds of the two critical values. She should conclude that all the exchange rate pair correlations are different from zero at the $5 \%$ level.

\section{TEST OF INDEPENDENCE USING CONTINGENCY TABLE DATA}
explain tests of independence based on contingency table data

When faced with categorical or discrete data, we cannot use the methods that we have discussed up to this point to test whether the classifications of such data are independent. Suppose we observe the following frequency table of 1,594 exchange-traded funds (ETFs) based on two classifications: size (that is, market capitalization) and investment type (value, growth, or blend), as shown in Exhibit 28. The classification of the investment type is discrete, so we cannot use correlation to assess the relationship between size and investment type. Exhibit 28: Size and Investment Type Classifications of 1,594 ETFs

\begin{center}
\begin{tabular}{lcccc}
\hline
 & \multicolumn{4}{c}{Size Based on Market Capitalization} \\
 & Small & Medium & Large & Total \\
\hline
Investment Type & 50 & 110 & 343 & 503 \\
\hline
Value & 42 & 122 & 202 & 366 \\
Growth & 56 & 149 & 520 & 725 \\
Blend & 148 & 381 & 1,065 & 1,594 \\
Total &  &  &  &  \\
\end{tabular}
\end{center}

This table is referred to as a contingency table or a two-way table (because there are two classifications, or classes-size and investment type).

If we want to test whether there is a relationship between the size and investment type, we can perform a test of independence using a nonparametric test statistic that is chi-square distributed:

$$
\chi^{2}=\sum_{i=1}^{m} \frac{\left(O_{i j}-E_{i j}\right)^{2}}{E_{i j}},
$$

where

$m=$ the number of cells in the table, which is the number of groups in the first class multiplied by the number of groups in the second class

$O_{i j}=$ the number of observations in each cell of row $i$ and column $j$ (i.e., observed frequency)

$E_{i j}=$ the expected number of observations in each cell of row $i$ and column $j$, assuming independence (i.e., expected frequency)

This test statistic has $(r-1)(c-1)$ degrees of freedom, where $r$ is the number of rows and $c$ is the number of columns.

In Exhibit 28, size class has three groups (small, medium, and large) and investment type class has three groups (value, growth, and blend), so $m$ is $9(=3 \times 3)$. The number of ETFs in each cell $\left(O_{i j}\right)$, the observed frequency, is given, so to calculate the chi-square test statistic, we need to estimate $E_{i j}$, the expected frequency, which is the number of ETFs we would expect to be in each cell if size and investment type are completely independent. The expected number of ETFs $\left(E_{i j}\right)$ is calculated using

$$
E_{i j}=\frac{(\text { Total row } i) \times(\text { Total column } j)}{\text { Overall total }} .
$$

Consider one combination of size and investment type, small-cap value:

$$
E_{i j}=\frac{503 \times 148}{1,594}=46.703
$$

We repeat this calculation for each combination of size and investment type (i.e., $m$ = 9 pairs) to arrive at the expected frequencies, shown in Panel A of Exhibit 29.

Next, we calculate $\frac{\left(O_{i j}-E_{i j}\right)^{2}}{E_{i j}}$, the squared difference between observed and expected frequencies scaled by expected frequency, for each cell as shown in Panel B of Exhibit 29. Finally, by summing the values of $\frac{\left(O_{i j}-E_{i j}\right)^{2}}{E_{i j}}$ for each of the $m$ cells, we calculate the chi-square statistic as 32.08025 . Exhibit 29: Inputs to Chi-Square Test Statistic Calculation for 1,594 ETFs

Assuming Independence of Size and Investment Type

A. Expected Frequency of ETFs by Size and Investment Type

\begin{center}
\begin{tabular}{lccc}
\hline
 & \multicolumn{2}{c}{Size Based on Market Capitalization} &  \\
\hline
Investment Type & Small & Medium & Large \\
\hline
Value & 46.703 & 120.228 & 336.070 \\
Growth & 33.982 & 87.482 & 244.536 \\
Blend & 67.315 & 173.290 & 484.395 \\
\hline
Total & 148.000 & 381.000 & $1,065.000$ \\
\hline
\end{tabular}
\end{center}

B. Scaled Squared Deviation for Each Combination of Size and Investment Type

\begin{center}
\begin{tabular}{lccc}
\hline
 & \multicolumn{2}{c}{Size Based on Market Capitalization} &  \\
\hline
Investment Type & Small & Medium & Large \\
\hline
Value & 0.233 & 0.870 & 0.143 \\
Growth & 1.892 & 13.620 & 7.399 \\
Blend & 1.902 & 3.405 & 2.617 \\
\hline
\end{tabular}
\end{center}

In our ETF example, we test the null hypothesis of independence between the two classes (i.e., no relationship between size and investment type) versus the alternative hypothesis of dependence (i.e., a relationship between size and investment type) using a 5\% level of significance, as shown in Exhibit 30. If, on the one hand, the observed values are equal to the expected values, the calculated test statistic would be zero. If, on the other hand, there are differences between the observed and expected values, these differences are squared, so the calculated chi-square statistic will be positive. Therefore, for the test of independence using a contingency table, there is only one rejection region, on the right side.

Exhibit 30: Test of Independence of Size and Investment Type for 1,594 ETFs

Step 1 State the hypotheses.

Step 2 Identify the appropriate test statistic.

Step 3 Specify the level of significance.

Step 4 State the decision rule.

Step 5 Calculate the test statistic. $H_{0}$ : ETF size and investment type are not related, so these classifications are independent;

$H_{a}:$ ETF size and investment type are related, so these classifications are not independent.

$\chi^{2}=\sum_{i=1}^{m} \frac{\left(O_{i j}-E_{i j}\right)^{2}}{E_{i j}}$

$5 \%$

With $(3-1) \times(3-1)=4$ degrees of freedom and a one-sided test with a $5 \%$ level of significance, the critical value is 9.4877 .

We reject the null hypothesis if the calculated $\chi^{2}$ statistic is greater than 9.4877.

$\begin{array}{ll}\text { Excel } & \text { CHISQ.INV }(0.95,4) \\ \mathbf{R} & \text { qchisq }(.95,4) \\ \text { Python } & \text { from scipy.stats import chi2 } \\ & \text { chi2.ppf }(.95,4)\end{array}$

$X^{2}=32.08025$ Step 6 Make a decision.

Reject the null hypothesis of independence because the calculated $\chi^{2}$ test statistic is greater than 9.4877 . There is sufficient evidence to conclude that ETF size and investment type are related (i.e., not independent).

We can visualize the contingency table in a graphic referred to as a mosaic. In a mosaic, a grid reflects the comparison between the observed and expected frequencies. Consider Exhibit 31, which represents the ETF contingency table.

\section{Exhibit 31: Mosaic of the ETF Contingency Table}
\begin{center}
\includegraphics[max width=\textwidth]{2023_05_04_cff39ee44f77d6514e1bg-412}
\end{center}

The width of the rectangles in Exhibit 31 reflect the proportion of ETFs that are small, medium, and large, whereas the height reflects the proportion that are value, growth, and blend. The darker shading indicates whether there are more observations than expected under the null hypothesis of independence, whereas the lighter shading indicates that there are fewer observations than expected, with "more" and "fewer" determined by reference to the standardized residual boxes. The standardized residual, also referred to as a Pearson residual, is

$$
\text { Standardized residual }=\frac{O_{i j}-E_{i j}}{\sqrt{E_{i j}}} .
$$

The interpretation for this ETF example is that there are more medium-size growth ETFs (standardized residual of 3.69) and fewer large-size growth ETFs (standardized residual of -2.72) than would be expected if size and investment type were independent.

\section{EXAMPLE 18}
\section{Using Contingency Tables to Test for Independence}
Consider the contingency table in Exhibit 32, which classifies 500 randomly selected companies on the basis of two environmental, social, and governance (ESG) rating dimensions: environmental rating and governance rating.

Exhibit 32: Classification of 500 Randomly Selected Companies

Based on Environmental and Governance Ratings

\section{Governance Rating}
\begin{center}
\begin{tabular}{lcccc}
\hline
Environmental Rating & Progressive & Average & Poor & Total \\
\hline
Progressive & 35 & 40 & 5 & 80 \\
Average & 80 & 130 & 50 & 260 \\
Poor & 40 & 60 & 60 & 160 \\
Total & 155 & 230 & 115 & 500 \\
\hline
\end{tabular}
\end{center}

\begin{enumerate}
  \item What are the expected frequencies for these two ESG rating dimensions if these categories are independent?
\end{enumerate}

\section{Solution to 1}
The expected frequencies based on independence of the governance rating and the environmental rating are shown in Panel A of Exhibit 33. For example, using Equation 12, the expected frequency for the combination of progressive governance and progressive environmental ratings is

$$
E_{i j}=\frac{155 \times 80}{500}=24.80
$$

Exhibit 33: Inputs to Chi-Square Test Statistic Calculation Assuming Independence of Environmental and Governance Ratings

A. Expected Frequencies of Environmental and Governance Ratings Assuming Independence

\begin{center}
\begin{tabular}{lccc}
\hline
 & \multicolumn{2}{c}{Governance Rating} &  \\
\hline
Environmental Rating & Progressive & Average & Poor \\
\hline
Progressive & 24.8 & 36.8 & 18.4 \\
Average & 80.6 & 119.6 & 59.8 \\
Poor & 49.6 & 73.6 & 36.8 \\
\hline
\end{tabular}
\end{center}

B. Scaled Squared Deviation for Each Combination of Environmental and Governance Ratings

\begin{center}
\begin{tabular}{lccc}
\hline
 & \multicolumn{2}{c}{Governance Rating} &  \\
\hline
Environmental Rating & Progressive & Average & Poor \\
\hline
Progressive & 4.195 & 0.278 & 9.759 \\
Average & 0.004 & 0.904 & 1.606 \\
Poor & 1.858 & 2.513 & 14.626 \\
\hline
\end{tabular}
\end{center}

\begin{enumerate}
  \setcounter{enumi}{1}
  \item Using a $5 \%$ level of significance, determine whether these two ESG rating dimensions are independent of one another.
\end{enumerate}

\section{Solution to 2}
Step 1 State the hypotheses.

Step 2 Identify the appropriate test statistic.

Step 3 Specify the level of significance.

Step 4 State the decision rule. $H_{0}$ : Governance and environmental ratings are not related, so these ratings are independent;

$H_{a}$ : Governance and environmental ratings are related, so these ratings are not independent.

$$
X^{2}=\sum_{i=1}^{m} \frac{\left(O_{i j}-E_{i j}\right)^{2}}{E_{i j}}
$$

$5 \%$

With $(3-1) \times(3-1)=4$ degrees of freedom and a one-sided test with a $5 \%$ level of significance, the critical value is 9.487729 .

We reject the null hypothesis if the calculated $\chi^{2}$ statistic is greater than 9.487729.

Excel CHISQ.INV $(0.95,4)$

$\mathbf{R}$ qchisq $(.95,4)$

Python from scipy.stats import chi2

chi2.ppf(.95,4)

$\chi^{2}=35.74415$

To calculate the test statistic, we first calculate the squared difference between observed and expected frequencies scaled by expected frequency for each cell, as shown in Panel B of Exhibit 33. Then, summing the values in each of the $m$ cells (see Equation 11), we calculate the chi-square statistic as 35.74415 .

Reject the null hypothesis because the calculated $\chi^{2}$ test statistic is greater than 9.487729. There is sufficient evidence to indicate that the environmental and governance ratings are related, so they are not independent.

\section{SUMMARY}
In this reading, we have presented the concepts and methods of statistical inference and hypothesis testing.

\begin{itemize}
  \item A hypothesis is a statement about one or more populations.

  \item The steps in testing a hypothesis are as follows: 1. State the hypotheses.

\end{itemize}

\begin{enumerate}
  \setcounter{enumi}{1}
  \item Identify the appropriate test statistic and its probability distribution.

  \item Specify the significance level.

  \item State the decision rule.

  \item Collect the data and calculate the test statistic.

  \item Make a decision.

\end{enumerate}

\begin{itemize}
  \item We state two hypotheses: The null hypothesis is the hypothesis to be tested; the alternative hypothesis is the hypothesis accepted if the null hypothesis is rejected.

  \item There are three ways to formulate hypotheses. Let $\theta$ indicate the population parameters:

\end{itemize}

\begin{enumerate}
  \item Two-sided alternative: $H_{0}: \theta=\theta_{0}$ versus $H_{a}: \theta \neq \theta_{0}$

  \item One-sided alternative (right side): $H_{0}: \theta \leq \theta_{0}$ versus $H_{\mathrm{a}}: \theta>\theta_{0}$

  \item One-sided alternative (left side): $H_{0}: \theta \geq \theta_{0}$ versus $H_{\mathrm{a}}: \theta<\theta_{0}$

\end{enumerate}

where $\theta_{0}$ is a hypothesized value of the population parameter and $\theta$ is the true value of the population parameter.

\begin{itemize}
  \item When we have a "suspected" or "hoped for" condition for which we want to find supportive evidence, we frequently set up that condition as the alternative hypothesis and use a one-sided test. However, the researcher may select a "not equal to" alternative hypothesis and conduct a two-sided test to emphasize a neutral attitude.

  \item A test statistic is a quantity, calculated using a sample, whose value is the basis for deciding whether to reject or not reject the null hypothesis. We compare the computed value of the test statistic to a critical value for the same test statistic to decide whether to reject or not reject the null hypothesis.

  \item In reaching a statistical decision, we can make two possible errors: We may reject a true null hypothesis (a Type I error, or false positive), or we may fail to reject a false null hypothesis (a Type II error, or false negative).

  \item The level of significance of a test is the probability of a Type I error that we accept in conducting a hypothesis test. The standard approach to hypothesis testing involves specifying only a level of significance (that is, the probability of a Type I error). The complement of the level of significance is the confidence level.

  \item The power of a test is the probability of correctly rejecting the null (rejecting the null when it is false). The complement of the power of the test is the probability of a Type II error.

  \item A decision rule consists of determining the critical values with which to compare the test statistic to decide whether to reject or not reject the null hypothesis. When we reject the null hypothesis, the result is said to be statistically significant.

  \item The $(1-\alpha)$ confidence interval represents the range of values of the test statistic for which the null hypothesis is not be rejected.

  \item The statistical decision consists of rejecting or not rejecting the null hypothesis. The economic decision takes into consideration all economic issues pertinent to the decision. - The $p$-value is the smallest level of significance at which the null hypothesis can be rejected. The smaller the $p$-value, the stronger the evidence against the null hypothesis and in favor of the alternative hypothesis. The $p$-value approach to hypothesis testing involves computing a $p$-value for the test statistic and allowing the user of the research to interpret the implications for the null hypothesis.

  \item For hypothesis tests concerning the population mean of a normally distributed population with unknown variance, the theoretically correct test statistic is the $t$-statistic.

  \item When we want to test whether the observed difference between two means is statistically significant, we must first decide whether the samples are independent or dependent (related). If the samples are independent, we conduct a test concerning differences between means. If the samples are dependent, we conduct a test of mean differences (paired comparisons test).

  \item When we conduct a test of the difference between two population means from normally distributed populations with unknown but equal variances, we use a $t$-test based on pooling the observations of the two samples to estimate the common but unknown variance. This test is based on an assumption of independent samples.

  \item In tests concerning two means based on two samples that are not independent, we often can arrange the data in paired observations and conduct a test of mean differences (a paired comparisons test). When the samples are from normally distributed populations with unknown variances, the appropriate test statistic is $t$-distributed.

  \item In tests concerning the variance of a single normally distributed population, the test statistic is chi-square with $n-1$ degrees of freedom, where $n$ is sample size.

  \item For tests concerning differences between the variances of two normally distributed populations based on two random, independent samples, the appropriate test statistic is based on an $F$-test (the ratio of the sample variances). The degrees of freedom for this $F$-test are $n_{1}-1$ and $n_{2}-1$, where $n_{1}$ corresponds to the number of observations in the calculation of the numerator and $n_{2}$ is the number of observations in the calculation of the denominator of the $F$-statistic.

  \item A parametric test is a hypothesis test concerning a population parameter or a hypothesis test based on specific distributional assumptions. In contrast, a nonparametric test either is not concerned with a parameter or makes minimal assumptions about the population from which the sample comes.

  \item A nonparametric test is primarily used when data do not meet distributional assumptions, when there are outliers, when data are given in ranks, or when the hypothesis we are addressing does not concern a parameter.

  \item In tests concerning correlation, we use a $t$-statistic to test whether a population correlation coefficient is different from zero. If we have $n$ observations for two variables, this test statistic has a $t$-distribution with $n-2$ degrees of freedom.

  \item The Spearman rank correlation coefficient is calculated on the ranks of two variables within their respective samples. - A chi-square distributed test statistic is used to test for independence of two categorical variables. This nonparametric test compares actual frequencies with those expected on the basis of independence. This test statistic has degrees of freedom of $(r-1)(c-2)$, where $r$ is the number of categories for the first variable and $c$ is the number of categories of the second variable.

\end{itemize}

\section{REFERENCES}
Benjamini, Y., Y. Hochberg. 1995. "Controlling the False Discovery Rate: A Practical and Powerful Approach to Multiple Testing." Journal of the Royal Statistical Society. Series B. Methodological, 57: 289-300.

\section{PRACTICE PROBLEMS}
\begin{enumerate}
  \item Which of the following statements about hypothesis testing is correct?
\end{enumerate}

A. The null hypothesis is the condition a researcher hopes to support.

B. The alternative hypothesis is the proposition considered true without conclusive evidence to the contrary.

C. The alternative hypothesis exhausts all potential parameter values not accounted for by the null hypothesis.

\begin{enumerate}
  \setcounter{enumi}{1}
  \item Willco is a manufacturer in a mature cyclical industry. During the most recent industry cycle, its net income averaged $\$ 30$ million per year with a standard deviation of $\$ 10$ million ( $n=6$ observations). Management claims that Willco's performance during the most recent cycle results from new approaches and that Willco's profitability will exceed the $\$ 24$ million per year observed in prior cycles.
\end{enumerate}

A. With $\mu$ as the population value of mean annual net income, formulate null and alternative hypotheses consistent with testing Willco management's claim.

B. Assuming that Willco's net income is at least approximately normally distributed, identify the appropriate test statistic and calculate the degrees of freedom.

C. Based on critical value of 2.015 , determine whether to reject the null hypothesis.

\begin{enumerate}
  \setcounter{enumi}{2}
  \item Which of the following statements is correct with respect to the null hypothesis?
\end{enumerate}

A. It can be stated as "not equal to" provided the alternative hypothesis is stated as "equal to."

B. Along with the alternative hypothesis, it considers all possible values of the population parameter.

C. In a two-tailed test, it is rejected when evidence supports equality between the hypothesized value and the population parameter.

\begin{enumerate}
  \setcounter{enumi}{3}
  \item Which of the following statements regarding a one-tailed hypothesis test is correct?
\end{enumerate}

A. The rejection region increases in size as the level of significance becomes smaller.

B. A one-tailed test more strongly reflects the beliefs of the researcher than a two-tailed test.

C. The absolute value of the rejection point is larger than that of a two-tailed test at the same level of significance.

\begin{enumerate}
  \setcounter{enumi}{4}
  \item A hypothesis test for a normally distributed population at a 0.05 significance level implies a:
\end{enumerate}

A. $95 \%$ probability of rejecting a true null hypothesis.

B. $95 \%$ probability of a Type I error for a two-tailed test. C. $5 \%$ critical value rejection region in a tail of the distribution for a one-tailed test.

\begin{enumerate}
  \setcounter{enumi}{5}
  \item The value of a test statistic is best described as the basis for deciding whether to:
A. reject the null hypothesis.
B. accept the null hypothesis.
C. reject the alternative hypothesis.

  \item Which of the following is a Type I error?
A. Rejecting a true null hypothesis
B. Rejecting a false null hypothesis
C. Failing to reject a false null hypothesis

  \item A Type II error is best described as:
A. rejecting a true null hypothesis.
B. failing to reject a false null hypothesis.
C. failing to reject a false alternative hypothesis.

  \item The level of significance of a hypothesis test is best used to:
A. calculate the test statistic.
B. define the test's rejection points.
C. specify the probability of a Type II error.

  \item All else equal, is specifying a smaller significance level in a hypothesis test likely to increase the probability of a:

\end{enumerate}

\begin{center}
\begin{tabular}{lccc}
 & Type I error? &  & Type II error? \\
\cline { 2 - 2 }
A. & No & No &  \\
B. & No & Yes &  \\
C. & Yes & No &  \\
\end{tabular}
\end{center}

\begin{enumerate}
  \setcounter{enumi}{10}
  \item The probability of correctly rejecting the null hypothesis is the:
A. $p$-value.
B. power of a test.
C. level of significance.

  \item The power of a hypothesis test is:
A. equivalent to the level of significance.
B. the probability of not making a Type II error.
C. unchanged by increasing a small sample size.

  \item For each of the following hypothesis tests concerning the population mean, $\mu$, state the conclusion regarding the test of the hypotheses.

\end{enumerate}

A. $H_{0}: \mu=10$ versus $H_{a}: \mu \neq 10$, with a calculated $t$-statistic of 2.05 and critical $t$-values of \textbackslash pm 1.984

B. $H_{0}: \mu \leq 10$ versus $H_{a}: \mu>10$, with a calculated $t$-statistic of 2.35 and a critical $t$-value of +1.679

C. $H_{0}: \mu=10$ versus $H_{a}: \mu \neq 10$, with a calculated $t$-statistic of 2.05 , a $p$-value of $4.6352 \%$, and a level of significance of $5 \%$.

D. $H_{0}: \mu \leq 10$ versus $H_{a}: \mu>10$, with a $2 \%$ level of significance and a calculated test statistic with a $p$-value of $3 \%$.

\begin{enumerate}
  \setcounter{enumi}{13}
  \item In the step "stating a decision rule" in testing a hypothesis, which of the following elements must be specified?
A. Critical value
B. Power of a test
C. Value of a test statistic

  \item When making a decision about investments involving a statistically significant result, the:

\end{enumerate}

A. economic result should be presumed to be meaningful.

B. statistical result should take priority over economic considerations.

C. economic logic for the future relevance of the result should be further explored.

\begin{enumerate}
  \setcounter{enumi}{15}
  \item An analyst tests the profitability of a trading strategy with the null hypothesis that the average abnormal return before trading costs equals zero. The calculated $t$-statistic is 2.802 , with critical values of \textbackslash pm 2.756 at significance level $\alpha=0.01$. After considering trading costs, the strategy's return is near zero. The results are most likely:
\end{enumerate}

A. statistically but not economically significant.

B. economically but not statistically significant.

C. neither statistically nor economically significant.

\begin{enumerate}
  \setcounter{enumi}{16}
  \item Which of the following statements is correct with respect to the $p$-value?
\end{enumerate}

A. It is a less precise measure of test evidence than rejection points.

B. It is the largest level of significance at which the null hypothesis is rejected.

C. It can be compared directly with the level of significance in reaching test conclusions.

\begin{enumerate}
  \setcounter{enumi}{17}
  \item Which of the following represents a correct statement about the $p$-value?
\end{enumerate}

A. The $p$-value offers less precise information than does the rejection points approach.

B. A larger $p$-value provides stronger evidence in support of the alternative hypothesis. C. A $p$-value less than the specified level of significance leads to rejection of the null hypothesis.

\begin{enumerate}
  \setcounter{enumi}{18}
  \item Which of the following statements on $p$-value is correct?
A. The $p$-value indicates the probability of making a Type II error.
B. The lower the $p$-value, the weaker the evidence for rejecting the $H_{0}$.
C. The $p$-value is the smallest level of significance at which $H_{0}$ can be rejected.

  \item The following table shows the significance level $(\alpha)$ and the $p$-value for two hypothesis tests.

\end{enumerate}

\begin{center}
\begin{tabular}{lcc}
\hline
 & $\boldsymbol{a}$ & $\boldsymbol{p}$-Value \\
\hline
Test 1 & 0.02 & 0.05 \\
Test 2 & 0.05 & 0.02 \\
\hline
\end{tabular}
\end{center}

In which test should we reject the null hypothesis?
A. Test 1 only
B. Test 2 only
C. Both Test 1 and Test 2

\begin{enumerate}
  \setcounter{enumi}{20}
  \item Identify the appropriate test statistic or statistics for conducting the following hypothesis tests. (Clearly identify the test statistic and, if applicable, the number of degrees of freedom. For example, "We conduct the test using an $x$-statistic with $y$ degrees of freedom.")
\end{enumerate}

A. $H_{0}: \mu=0$ versus $H_{a}: \mu \neq 0$, where $\mu$ is the mean of a normally distributed population with unknown variance. The test is based on a sample of 15 observations.

B. $H_{0}: \mu=5$ versus $H_{a}: \mu \neq 5$, where $\mu$ is the mean of a normally distributed population with unknown variance. The test is based on a sample of 40 observations.

C. $H_{0}: \mu \leq 0$ versus $H_{a}: \mu>0$, where $\mu$ is the mean of a normally distributed population with known variance $\sigma^{2}$. The sample size is 45 .

D. $H_{0}: \sigma^{2}=200$ versus $H_{a}: \sigma^{2} \neq 200$, where $\sigma^{2}$ is the variance of a normally distributed population. The sample size is 50 .

E. $H_{0}: \sigma_{1}^{2}=\sigma_{2}^{2}$ versus $H_{a}: \sigma_{1}^{2} \neq \sigma_{2}^{2}$, where $\sigma_{1}^{2}$ is the variance of one normally distributed population and $\sigma_{2}^{2}$ is the variance of a second normally distributed population. The test is based on two independent samples, with the first sample of size 30 and the second sample of size 40 .

F. $H_{0}: \mu_{1}-\mu_{2}=0$ versus $H_{a}: \mu_{1}-\mu_{2} \neq 0$, where the samples are drawn from normally distributed populations with unknown but assumed equal variances. The observations in the two samples (of size 25 and 30, respectively) are independent.

\begin{enumerate}
  \setcounter{enumi}{21}
  \item For each of the following hypothesis tests concerning the population mean, state the conclusion. A. $H_{0}: \sigma^{2}=0.10$ versus $H_{a}: \sigma^{2} \neq 0.10$, with a calculated chi-square test statistic of 45.8 and critical chi-square values of 42.950 and 86.830 .
\end{enumerate}

B. $H_{0}: \sigma^{2}=0.10$ versus $H_{a}: \sigma^{2} \neq 0.10$, with a $5 \%$ level of significance and a $p$-value for this calculated chi-square test statistic of $4.463 \%$.

C. $H_{0}: \sigma_{1}^{2}=\sigma_{2}^{2}$ versus $H_{a}: \sigma_{1}^{2} \neq \sigma_{2}^{2}$, with a calculated $F$-statistic of 2.3. With 40 and 30 degrees of freedom, the critical $F$-values are 0.498 and 1.943 .

D. $H_{0}: \sigma^{2} \leq 10$ versus $H_{a}: \mu \sigma^{2}>10$, with a calculated test statistic of 32 and a critical chi-square value of 26.296 .

\section{The following information relates to questions 23-24}
Performance in Forecasting Quarterly Earnings per Share

\begin{center}
\begin{tabular}{lccc}
\hline
 & $\begin{array}{c}\text { Number of } \\ \text { Forecasts }\end{array}$ & $\begin{array}{c}\text { Mean Forecast Error } \\ \text { (Predicted - Actual) }\end{array}$ & $\begin{array}{c}\text { Standard Deviation of } \\ \text { Forecast Errors }\end{array}$ \\
\hline
Analyst A & 10 & 0.05 & 0.10 \\
Analyst B & 15 & 0.02 & 0.09 \\
\hline
\end{tabular}
\end{center}

Critical $t$-values:

\begin{center}
\begin{tabular}{|c|c|c|}
\hline
\multirow[b]{2}{*}{Degrees of Freedom} & \multicolumn{2}{|c|}{Area in the Right-Side Rejection Area} \\
\hline
 & $p=0.05$ & $p=0.025$ \\
\hline
8 & 1.860 & 2.306 \\
\hline
9 & 1.833 & 2.262 \\
\hline
10 & 1.812 & 2.228 \\
\hline
11 & 1.796 & 2.201 \\
\hline
12 & 1.782 & 2.179 \\
\hline
13 & 1.771 & 2.160 \\
\hline
14 & 1.761 & 2.145 \\
\hline
15 & 1.753 & 2.131 \\
\hline
16 & 1.746 & 2.120 \\
\hline
17 & 1.740 & 2.110 \\
\hline
18 & 1.734 & 2.101 \\
\hline
19 & 1.729 & 2.093 \\
\hline
20 & 1.725 & 2.086 \\
\hline
21 & 1.721 & 2.080 \\
\hline
22 & 1.717 & 2.074 \\
\hline
23 & 1.714 & 2.069 \\
\hline
24 & 1.711 & 2.064 \\
\hline
25 & 1.708 & 2.060 \\
\hline
26 & 1.706 & 2.056 \\
\hline
27 & 1.703 & 2.052 \\
\hline
\end{tabular}
\end{center}

\begin{enumerate}
  \setcounter{enumi}{22}
  \item Investment analysts often use earnings per share (EPS) forecasts. One test of forecasting quality is the zero-mean test, which states that optimal forecasts should have a mean forecasting error of zero. The forecasting error is the difference between the predicted value of a variable and the actual value of the variable.
\end{enumerate}

You have collected data (shown in the previous table) for two analysts who cover two different industries: Analyst A covers the telecom industry; Analyst B covers automotive parts and suppliers.

A. With $\mu$ as the population mean forecasting error, formulate null and alternative hypotheses for a zero-mean test of forecasting quality.

B. For Analyst A, determine whether to reject the null at the 0.05 level of significance.

C. For Analyst B, determine whether to reject the null at the 0.05 level of significance.

\begin{enumerate}
  \setcounter{enumi}{23}
  \item Reviewing the EPS forecasting performance data for Analysts A and B, you want to investigate whether the larger average forecast errors of Analyst A relative to Analyst B are due to chance or to a higher underlying mean value for Analyst A. Assume that the forecast errors of both analysts are normally distributed and that the samples are independent.
\end{enumerate}

A. Formulate null and alternative hypotheses consistent with determining whether the population mean value of Analyst A's forecast errors $\left(\mu_{1}\right)$ are larger than Analyst B's $\left(\mu_{2}\right)$.

B. Identify the test statistic for conducting a test of the null hypothesis formulated in Part A.

C. Identify the rejection point or points for the hypotheses tested in Part A at the 0.05 level of significance.

D. Determine whether to reject the null hypothesis at the 0.05 level of significance.

\begin{enumerate}
  \setcounter{enumi}{24}
  \item An analyst is examining a large sample with an unknown population variance. Which of the following is the most appropriate test to test the hypothesis that the historical average return on an index is less than or equal to $6 \%$ ?
A. One-sided $t$-test
B. Two-sided $t$-test
C. One-sided chi-square test

  \item Which of the following tests of a hypothesis concerning the population mean is most appropriate?

\end{enumerate}

A. A $z$-test if the population variance is unknown and the sample is small

B. A $z$-test if the population is normally distributed with a known variance

C. A $t$-test if the population is non-normally distributed with unknown variance and a small sample

\begin{enumerate}
  \setcounter{enumi}{26}
  \item For a small sample from a normally distributed population with unknown variance, the most appropriate test statistic for the mean is the:
A. $z$-statistic.
B. $t$-statistic.
C. $X^{2}$ statistic.

  \item An investment consultant conducts two independent random samples of five-year performance data for US and European absolute return hedge funds. Noting a return advantage of $50 \mathrm{bps}$ for US managers, the consultant decides to test whether the two means are different from one another at a 0.05 level of significance. The two populations are assumed to be normally distributed with unknown but equal variances. Results of the hypothesis test are contained in the following tables.

\end{enumerate}

\begin{center}
\begin{tabular}{lccc}
\hline
 & Sample Size & $\begin{array}{c}\text { Mean Return } \\ (\%)\end{array}$ & Standard Deviation \\
\hline
US managers & 50 & 4.7 & 5.4 \\
European managers & 50 & 4.2 & 4.8 \\
\hline
Null and alternative hypotheses & $H_{0}: \mu_{U S}-\mu_{E}=0 ; H_{a}: \mu_{U S}-\mu_{E} \neq 0$ &  &  \\
Calculated test statistic & 0.4893 &  &  \\
Critical value rejection points & \textbackslash pm 1.984 &  &  \\
\hline
\end{tabular}
\end{center}

The mean return for US funds is $\mu_{U S}$ and $\mu_{E}$ is the mean return for European funds.

The results of the hypothesis test indicate that the:

A. null hypothesis is not rejected.

B. alternative hypothesis is statistically confirmed.

C. difference in mean returns is statistically different from zero.

\begin{enumerate}
  \setcounter{enumi}{28}
  \item A pooled estimator is used when testing a hypothesis concerning the:
\end{enumerate}

A. equality of the variances of two normally distributed populations.

B. difference between the means of two at least approximately normally distributed populations with unknown but assumed equal variances.

C. difference between the means of two at least approximately normally distributed populations with unknown and assumed unequal variances.

\begin{enumerate}
  \setcounter{enumi}{29}
  \item The following table gives data on the monthly returns on the S\&P 500 Index and small-cap stocks for a 40 -year period and provides statistics relating to their mean differences. Further, the entire sample period is split into two subperiods of 20 years each, and the return data for these subperiods is also given in the table.
\end{enumerate}

\begin{center}
\begin{tabular}{lccc}
\hline
 & $\begin{array}{c}\text { Small-Cap } \\ \text { S\&0 Return } \\ \text { (\%) }\end{array}$ & $\begin{array}{c}\text { Stock Return } \\ \text { (\%) }\end{array}$ & $\begin{array}{c}\text { Differences } \\ \text { (S\&P 500 - Small-Cap } \\ \text { Stock) }\end{array}$ \\
Measure & 1.0542 & 1.3117 & -0.258 \\
Entire sample period, 480 months & 5.9570 & 3.752 &  \\
Mean & 4.2185 &  &  \\
Standard deviation &  & 1.2741 & -0.640 \\
\end{tabular}
\end{center}

\begin{center}
\begin{tabular}{lccc}
\hline
 & $\begin{array}{c}\text { S\&P 500 Return } \\ \mathbf{( \% )}\end{array}$ & $\begin{array}{c}\text { Small-Cap } \\ \text { Stock Return } \\ \text { Measure }\end{array}$ & $\begin{array}{c}\text { Differences } \\ \text { (S\&P 500 - Small-Cap } \\ \text { Stock) }\end{array}$ \\
\hline
Standard deviation & 4.0807 & 6.5829 & 4.096 \\
Second subperiod, 240 months &  &  & 0.125 \\
Mean & 1.4739 & 1.3492 & 3.339 \\
Standard deviation & 4.3197 & 5.2709 &  \\
\hline
\end{tabular}
\end{center}

Use a significance level of 0.05 and assume that mean differences are approximately normally distributed.

A. Formulate null and alternative hypotheses consistent with testing whether any difference exists between the mean returns on the S\&P 500 and small-cap stocks.

B. Determine whether to reject the null hypothesis for the entire sample period if the critical values are \textbackslash pm 1.96 .

C. Determine whether to reject the null hypothesis for the first subperiod if the critical values are \textbackslash pm 1.96 .

D. Determine whether to reject the null hypothesis for the second subperiod if the critical values are \textbackslash pm 1.96 .

\begin{enumerate}
  \setcounter{enumi}{30}
  \item When evaluating mean differences between two dependent samples, the most appropriate test is a:
A. $z$-test.
B. chi-square test.
C. paired comparisons test.

  \item A chi-square test is most appropriate for tests concerning:

\end{enumerate}

A. a single variance.

B. differences between two population means with variances assumed to be equal.

C. differences between two population means with variances assumed to not be equal.

\begin{enumerate}
  \setcounter{enumi}{32}
  \item During a 10-year period, the standard deviation of annual returns on a portfolio you are analyzing was $15 \%$ a year. You want to see whether this record is sufficient evidence to support the conclusion that the portfolio's underlying variance of return was less than 400, the return variance of the portfolio's benchmark.
\end{enumerate}

A. Formulate null and alternative hypotheses consistent with your objective.

B. Identify the test statistic for conducting a test of the hypotheses in Part A, and calculate the degrees of freedom.

C. Determine whether the null hypothesis is rejected or not rejected at the 0.05 level of significance using a critical value of 3.325 .

\begin{enumerate}
  \setcounter{enumi}{33}
  \item You are investigating whether the population variance of returns on an index changed subsequent to a market disruption. You gather the following data for 120 months of returns before the disruption and for 120 months of returns after the disruption. You have specified a 0.05 level of significance.
\end{enumerate}

\begin{center}
\begin{tabular}{lccc}
\hline
Time Period & $\boldsymbol{n}$ & $\begin{array}{c}\text { Mean Monthly Return } \\ (\%)\end{array}$ & Variance of Returns \\
\hline
Before disruption & 120 & 1.416 & 22.367 \\
After disruption & 120 & 1.436 & 15.795 \\
\hline
\end{tabular}
\end{center}

A. Formulate null and alternative hypotheses consistent with the research goal.

B. Identify the test statistic for conducting a test of the hypotheses in Part A, and calculate the degrees of freedom.

C. Determine whether to reject the null hypothesis at the 0.05 level of significance if the critical values are 0.6969 and 1.4349 .

\begin{enumerate}
  \setcounter{enumi}{34}
  \item Which of the following should be used to test the difference between the variances of two normally distributed populations?
A. $t$-test
B. $F$-test
C. Paired comparisons test

  \item In which of the following situations would a nonparametric test of a hypothesis most likely be used?

\end{enumerate}

A. The sample data are ranked according to magnitude.

B. The sample data come from a normally distributed population.

C. The test validity depends on many assumptions about the nature of the population.

\begin{enumerate}
  \setcounter{enumi}{36}
  \item An analyst is examining the monthly returns for two funds over one year. Both funds' returns are non-normally distributed. To test whether the mean return of one fund is greater than the mean return of the other fund, the analyst can use:
\end{enumerate}

A. a parametric test only.

B. a nonparametric test only.

C. both parametric and nonparametric tests.

\begin{enumerate}
  \setcounter{enumi}{37}
  \item The following table shows the sample correlations between the monthly returns for four different mutual funds and the S\&P 500. The correlations are based on 36 monthly observations. The funds are as follows:
\end{enumerate}

Fund $1 \quad$ Large-cap fund

Fund 2 Mid-cap fund

Fund $3 \quad$ Large-cap value fund

Fund $4 \quad$ Emerging market fund

S\&P 500 US domestic stock index

\begin{center}
\begin{tabular}{lccccc}
\hline
 & Fund 1 & Fund 2 & Fund 3 & Fund 4 & S\&P 500 \\
\hline
Fund 1 & 1 &  &  &  &  \\
Fund 2 & 0.9231 & 1 &  &  &  \\
Fund 3 & 0.4771 & 0.4156 & 1 & 1 & 1 \\
Fund 4 & 0.7111 & 0.7238 & 0.3102 & 0.7515 &  \\
S\&P 500 & 0.8277 & 0.8223 & 0.5791 &  &  \\
\hline
\end{tabular}
\end{center}

Test the null hypothesis that each of these correlations, individually, is equal to zero against the alternative hypothesis that it is not equal to zero. Use a $5 \%$ significance level and critical $t$-values of \textbackslash pm 2.032 .

\begin{enumerate}
  \setcounter{enumi}{38}
  \item You are interested in whether excess risk-adjusted return (alpha) is correlated with mutual fund expense ratios for US large-cap growth funds. The following table presents the sample.
\end{enumerate}

\begin{center}
\begin{tabular}{ccc}
\hline
Mutual Fund & Alpha & Expense Ratio \\
\hline
1 & -0.52 & 1.34 \\
2 & -0.13 & 0.40 \\
3 & -0.50 & 1.90 \\
4 & -1.01 & 1.50 \\
5 & -0.26 & 1.35 \\
6 & -0.89 & 0.50 \\
7 & -0.42 & 1.00 \\
8 & -0.23 & 1.50 \\
9 & -0.60 & 1.45 \\
\hline
\end{tabular}
\end{center}

A. Formulate null and alternative hypotheses consistent with the verbal description of the research goal.

B. Identify and justify the test statistic for conducting a test of the hypotheses in Part A.

C. Determine whether to reject the null hypothesis at the 0.05 level of significance if the critical values are \textbackslash pm 2.306 .

\begin{enumerate}
  \setcounter{enumi}{39}
  \item Jill Batten is analyzing how the returns on the stock of Stellar Energy Corp. are related with the previous month's percentage change in the US Consumer Price Index for Energy (CPIENG). Based on 248 observations, she has computed the sample correlation between the Stellar and CPIENG variables to be -0.1452 . She also wants to determine whether the sample correlation is significantly different from zero. The critical value for the test statistic at the 0.05 level of significance is approximately 1.96. Batten should conclude that the statistical relationship between Stellar and CPIENG is:
\end{enumerate}

A. significant, because the calculated test statistic is outside the bounds of the critical values for the test statistic.

B. significant, because the calculated test statistic has a lower absolute value than the critical value for the test statistic.

C. insignificant, because the calculated test statistic is outside the bounds of the critical values for the test statistic. 41. An analyst group follows 250 firms and classifies them in two dimensions. First, they use dividend payment history and earnings forecasts to classify firms into one of three groups, with 1 indicating the dividend stars and 3 the dividend laggards. Second, they classify firms on the basis of financial leverage, using debt ratios, debt features, and corporate governance to classify the firms into three groups, with 1 indicating the least risky firms based on financial leverage and 3 indicating the riskiest. The classification of the 250 firms is as follows:

\begin{center}
\begin{tabular}{cccc}
\hline
 & \multicolumn{3}{c}{Dividend Group} \\
\cline { 2 - 4 }
Financial Leverage & $\mathbf{2}$ & $\mathbf{3}$ &  \\
\hline
1 & $\mathbf{1}$ & 40 & 40 \\
2 & 30 & 10 & 20 \\
3 & 10 & 50 & 10 \\
\hline
\end{tabular}
\end{center}

A. What are the null and alternative hypotheses to test whether the dividend and financial leverage groups are independent of one another?

B. What is the appropriate test statistic to use in this type of test?

C. If the critical value for the 0.05 level of significance is 9.4877 , what is your conclusion?

\begin{enumerate}
  \setcounter{enumi}{41}
  \item Which of the following statements is correct regarding the chi-square test of independence?
\end{enumerate}

A. The test has a one-sided rejection region.

B. The null hypothesis is that the two groups are dependent.

C. If there are two categories, each with three levels or groups, there are six degrees of freedom.

\section{SOLUTIONS}
\begin{enumerate}
  \item C is correct. Together, the null and alternative hypotheses account for all possible values of the parameter. Any possible values of the parameter not covered by the null must be covered by the alternative hypothesis (e.g., $H_{0}: \mu \leq 5$ versus $H_{a}: \mu>$ 5).

  \item 
\end{enumerate}

A. As stated in the text, we often set up the "hoped for" or "suspected" condition as the alternative hypothesis. Here, that condition is that the population value of Willco's mean annual net income exceeds $\$ 24$ million. Thus, we have $H_{0}: \mu \leq 24$ versus $H_{a}: \mu>24$.

B. Given that net income is normally distributed with unknown variance, the appropriate test statistic is $t=\frac{\bar{X}-\mu_{0}}{s / \sqrt{n}}=1.469694$ with $n-1=6-1=5$ degrees of freedom.

C. We reject the null if the calculated $t$-statistic is greater than 2.015. The calculated $t$-statistic is $t=\frac{30-24}{10 \sqrt{6}}=1.469694$. Because the calculated test statistic does not exceed 2.015, we fail to reject the null hypothesis. There is not sufficient evidence to indicate that the mean net income is greater than $\$ 24$ million.

\begin{enumerate}
  \setcounter{enumi}{2}
  \item A is correct. The null hypothesis and the alternative hypothesis are complements of one another and together are exhaustive; that is, the null and alternative hypotheses combined consider all the possible values of the population parameter.

  \item B is correct. One-tailed tests in which the alternative is "greater than" or "less than" represent the beliefs of the researcher more firmly than a "not equal to" alternative hypothesis.

  \item $\mathrm{C}$ is correct. For a one-tailed hypothesis test, there is a $5 \%$ rejection region in one tail of the distribution.

  \item A is correct. Calculated using a sample, a test statistic is a quantity whose value is the basis for deciding whether to reject the null hypothesis.

  \item A is correct. The definition of a Type I error is when a true null hypothesis is rejected.

  \item B is correct. A Type II error occurs when a false null hypothesis is not rejected.

  \item B is correct. The level of significance is used to establish the rejection points of the hypothesis test.

  \item B is correct. Specifying a smaller significance level decreases the probability of a Type I error (rejecting a true null hypothesis) but increases the probability of a Type II error (not rejecting a false null hypothesis). As the level of significance decreases, the null hypothesis is less frequently rejected.

  \item B is correct. The power of a test is the probability of rejecting the null hypothesis when it is false.

  \item B is correct. The power of a hypothesis test is the probability of correctly rejecting the null when it is false. Failing to reject the null when it is false is a Type II error. Thus, the power of a hypothesis test is the probability of not committing a Type II error.

  \item We make the decision either by comparing the calculated test statistic with the critical values or by comparing the $p$-value for the calculated test statistic with the level of significance.

\end{enumerate}

A. Reject the null hypothesis because the calculated test statistic is outside the bounds of the critical values.

B. The calculated $t$-statistic is in the rejection region that is defined by +1.679 , so we reject the null hypothesis.

C. The $p$-value corresponding to the calculated test statistic is less than the level of significance, so we reject the null hypothesis.

D. We fail to reject because the $p$-value for the calculated test statistic is greater than what is tolerated with a $2 \%$ level of significance.

\begin{enumerate}
  \setcounter{enumi}{13}
  \item $\mathrm{B}$ is correct. The critical value in a decision rule is the rejection point for the test. It is the point with which the test statistic is compared to determine whether to reject the null hypothesis, which is part of the fourth step in hypothesis testing.

  \item C is correct. When a statistically significant result is also economically meaningful, one should further explore the logic of why the result might work in the future.

  \item A is correct. The hypothesis is a two-tailed formulation. The $t$-statistic of 2.802 falls outside the critical rejection points of less than -2.756 and greater than 2.756. Therefore, the null hypothesis is rejected; the result is statistically significant. However, despite the statistical results, trying to profit on the strategy is not likely to be economically meaningful because the return is near zero after transaction costs.

  \item C is correct. When directly comparing the $p$-value with the level of significance, it can be used as an alternative to using rejection points to reach conclusions on hypothesis tests. If the $p$-value is smaller than the specified level of significance, the null hypothesis is rejected. Otherwise, the null hypothesis is not rejected.

  \item $\mathrm{C}$ is correct. The $p$-value is the smallest level of significance at which the null hypothesis can be rejected for a given value of the test statistic. The null hypothesis is rejected when the $p$-value is less than the specified significance level.

  \item $C$ is correct. The $p$-value is the smallest level of significance $(\alpha)$ at which the null hypothesis can be rejected.

  \item B is correct. The $p$-value is the smallest level of significance $(\alpha)$ at which the null hypothesis can be rejected. If the $p$-value is less than $\alpha$, the null is rejected. In Test 1 , the $p$-value exceeds the level of significance, whereas in Test 2 , the $p$-value is less than the level of significance.

  \item 
\end{enumerate}

A. The appropriate test statistic is a $t$-statistic, $t=\frac{\bar{X}-\mu_{0}}{\frac{s}{\sqrt{n}}}$, with $n-1=15-1=$ 14 degrees of freedom. A $t$-statistic is correct when the sample comes from an approximately normally distributed population with unknown variance. B. The appropriate test statistic is a $t$-statistic, $t=\frac{\bar{X}-\mu_{0}}{\frac{s}{\sqrt{n}}}$, with $40-1=39$ degrees of freedom. A $t$-statistic is theoretically correct when the sample comes from a normally distributed population with unknown variance.

C. The appropriate test statistic is a $z$-statistic, $z=\frac{\bar{X}-\mu_{0}}{\frac{\sigma}{\sqrt{n}}}$, because the sample comes from a normally distributed population with a known variance. D. The appropriate test statistic is chi-square, $x^{2}=\frac{s^{2}(n-1)}{\sigma_{0}^{2}}$, with $50-1=49$
degrees of freedom.

E. The appropriate test statistic is the $F$-statistic, $F=\sigma 12 / \sigma 22$, with 29 and 39 degrees of freedom.

F. The appropriate test statistic is a $t$-statistic using a pooled estimate of the population variance: $t=\frac{\left(\bar{X}_{1}-\bar{X}_{2}\right)-\left(\mu_{1}-\mu_{2}\right)}{\sqrt{\frac{s_{p}^{2}}{n_{1}}+\frac{s_{p}^{2}}{n_{2}}}}$, where $s_{p}^{2}=\frac{\left(n_{1}-1\right) s_{1}^{2}+\left(n_{2}-1\right) s_{2}^{2}}{n_{1}+n_{2}-2}$. The $t$-statistic has $25+30-2=53$ degrees of freedom. This statistic is appropriate because the populations are normally distributed with unknown variances; because the variances are assumed to be equal, the observations can be pooled to estimate the common variance. The requirement of independent samples for using this statistic has been met.

\begin{enumerate}
  \setcounter{enumi}{21}
  \item We make the decision either by comparing the calculated test statistic with the critical values or by comparing the $p$-value for the calculated test statistic with the level of significance.
\end{enumerate}

A. The calculated chi-square falls between the two critical values, so we fail to reject the null hypothesis.

B. The $p$-value for the calculated test statistic is less than the level of significance (the 5\%), so we reject the null hypothesis.

C. The calculated $F$-statistic falls outside the bounds of the critical $F$-values, so we reject the null hypothesis.

D. The calculated chi-square exceeds the critical value for this right-side test, so we reject the null hypothesis.

23.

A. $H_{0}: \mu=0$ versus $H_{a}: \mu \neq 0$.

B. The $t$-test is based on $t=\frac{\bar{X}-\mu_{0}}{s / \sqrt{n}}$ with $n-1=10-1=9$ degrees of freedom. At the 0.05 significance level, we reject the null if the calculated $t$-statistic is outside the bounds of \textbackslash pm 2.262 (from the table for 9 degrees of freedom and 0.025 in the right side of the distribution). For Analyst A, we have a calculated test statistic of $t=\frac{0.05-0}{0.10 \sqrt{10}}=1.58114$. We, therefore, fail to reject the null hypothesis at the 0.05 level.

C. For Analyst B, the $t$-test is based on $t$ with $15-1=14$ degrees of freedom. At the 0.05 significance level, we reject the null if the calculated $t$-statistic is outside the bounds of \textbackslash pm 2.145 (from the table for 14 degrees of freedom). The calculated test statistic is $t=\frac{0.02-0}{0.09 \sqrt{10}}=0.86066$. Because 0.86066 is within the range of \textbackslash pm 2.145 , we fail to reject the null at the 0.05 level.

24.

A. Stating the suspected condition as the alternative hypothesis, we have $H_{0}: \mu_{A}-\mu_{B} \leq 0$ versus $H_{a}: \mu_{A}-\mu_{B}>0$,

where

$\mu_{A}=$ the population mean value of Analyst A's forecast errors

$\mu_{B}=$ the population mean value of Analyst B's forecast errors

B. We have two normally distributed populations with unknown variances.

Based on the samples, it is reasonable to assume that the population variances are equal. The samples are assumed to be independent; this assumption is reasonable because the analysts cover different industries. The appropriate test statistic is $t$ using a pooled estimate of the common variance: $t=\frac{\left(\bar{X}_{1}-\bar{X}_{2}\right)-\left(\mu_{1}-\mu_{2}\right)}{\sqrt{\frac{s_{p}^{2}}{n_{1}}+\frac{s_{p}^{2}}{n_{2}}}}$, where $s_{p}^{2}=\frac{\left(n_{1}-1\right) s_{1}^{2}+\left(n_{2}-1\right) s_{2}^{2}}{n_{1}+n_{2}-2}$. The number of degrees of freedom is $n_{A}+n_{B}-2=10+15-2=23$.

C. For $\mathrm{df}=23$, according to the table, the rejection point for a one-sided (right side) test at the 0.05 significance level is 1.714 .

D. We first calculate the pooled estimate of variance:

E. $s_{p}^{2}=\frac{(10-1) 0.01+(15-1) 0.0081}{10+15-2}=0.0088435$.

We then calculate the $t$-distributed test statistic:

$t=\frac{(0.05-0.02)-0}{\sqrt{\frac{0.0088435}{10}+\frac{0.0088435}{15}}}=\frac{0.03}{0.0383916}=0.78142$.

Because $0.78142<1.714$, we fail to reject the null hypothesis. There is not sufficient evidence to indicate that the mean for Analyst A exceeds that for Analyst B.

\begin{enumerate}
  \setcounter{enumi}{24}
  \item A is correct. If the population sampled has unknown variance and the sample is large, a $z$-test may be used. Hypotheses involving "greater than" or "less than" postulations are one sided (one tailed). In this situation, the null and alternative hypotheses are stated as $H_{0}: \mu \leq 6 \%$ and $H_{a}: \mu>6 \%$, respectively. A one-tailed $t$-test is also acceptable in this case, and the rejection region is on the right side of the distribution.

  \item B is correct. The $z$-test is theoretically the correct test to use in those limited cases when testing the population mean of a normally distributed population with known variance.

  \item B is correct. A $t$-statistic is the most appropriate for hypothesis tests of the population mean when the variance is unknown and the sample is small but the population is normally distributed.

  \item A is correct. The calculated $t$-statistic value of 0.4893 falls within the bounds of the critical $t$-values of \textbackslash pm 1.984 . Thus, $H_{0}$ cannot be rejected; the result is not statistically significant at the 0.05 level.

  \item B is correct. The assumption that the variances are equal allows for the combining of both samples to obtain a pooled estimate of the common variance.

  \item 
\end{enumerate}

A. We test $H_{0}: \mu_{d}=0$ versus $H_{a}: \mu_{d} \neq 0$, where $\mu_{d}$ is the population mean difference. B. This is a paired comparisons $t$-test with $n-1=480-1=479$ degrees of freedom. At the 0.05 significance level, we reject the null hypothesis if the calculated $t$ is less than -1.96 or greater than 1.96 .

$t=\frac{\bar{d}-\mu_{d 0}}{s \bar{d}}=\frac{-0.258-0}{3.752 / \sqrt{480}}=\frac{-0.258}{0.171255}=-1.506529$, or -1.51

Because the calculate $t$-statistic is between \textbackslash pm 1.96 , we do not reject the null hypothesis that the mean difference between the returns on the S\&P 500 and small-cap stocks during the entire sample period was zero.

C. This $t$-test now has $n-1=240-1=239$ degrees of freedom. At the 0.05 significance level, we reject the null hypothesis if the calculated $t$ is less than -1.96 or greater than 1.96 .

$t=\frac{\bar{d}-\mu_{d 0}}{s \bar{d}}=\frac{-0.640-0}{4.096 / \sqrt{240}}=\frac{-0.640}{0.264396}=-2.420615$, or -2.42

Because $-2.42<-1.96$, we reject the null hypothesis at the 0.05 significance level. We conclude that during this subperiod, small-cap stocks significantly outperformed the S\&P 500 .

D. This $t$-test has $n-1=240-1=239$ degrees of freedom. At the 0.05 significance level, we reject the null hypothesis if the calculated $t$-statistic is less than -1.96 or greater than 1.96 . The calculated test statistic is

$t=\frac{\bar{d}-\mu_{d 0}}{s \bar{d}}=\frac{0.125-0}{3.339 / \sqrt{240}}=\frac{0.125}{0.215532}=0.579962$, or 0.58.

At the 0.05 significance level, because the calculated test statistic of 0.58 is between \textbackslash pm 1.96 , we fail to reject the null hypothesis for the second subperiod.

\begin{enumerate}
  \setcounter{enumi}{30}
  \item C is correct. A paired comparisons test is appropriate to test the mean differences of two samples believed to be dependent.

  \item A is correct. A chi-square test is used for tests concerning the variance of a single normally distributed population.

  \item 
\end{enumerate}

A. We have a "less than" alternative hypothesis, where $\sigma^{2}$ is the variance of return on the portfolio. The hypotheses are $H_{0}: \sigma^{2} \geq 400$ versus $H_{a}: \sigma^{2}<400$, where 400 is the hypothesized value of variance, $\sigma_{0}^{2}$. This means that the rejection region is on the left side of the distribution.

B. The test statistic is chi-square distributed with $10-1=9$ degrees of freedom: $\mathrm{X}^{2}=\frac{(n-1) s^{2}}{\sigma_{0}^{2}}$

C. The test statistic is calculated as

$\chi^{2}=\frac{(n-1) s^{2}}{\sigma_{0}^{2}}=\frac{9 \times 15^{2}}{400}=\frac{2,025}{400}=5.0625$, or 5.06.

Because 5.06 is not less than 3.325, we do not reject the null hypothesis; the calculated test statistic falls to the right of the critical value, where the critical value separates the left-side rejection region from the region where we fail to reject.

We can determine the critical value for this test using software: Excel [CHISQ.INV $(0.05,9)]$

$\mathbf{R}[\mathrm{qchisq}(.05,9)]$

Python [from scipy.stats import chi2 and chi2.ppf(.05,9)]

We can determine the $p$-value for the calculated test statistic of 17.0953 using software:

Excel [CHISQ.DIST(5.06,9,TRUE)]

$\mathbf{R}$ [pchisq(5.06,9,lower.tail=TRUE)]

Python [from scipy.stats import chi2 and chi2.cdf(5.06,9)]

34.

A. We have a "not equal to" alternative hypothesis:

$H_{0}: \sigma_{\text {Before }}^{2}=\sigma_{\text {After }}^{2}$ versus $H_{a}: \sigma_{\text {Before }}^{2} \neq \sigma_{\text {After }}^{2}$

B. To test a null hypothesis of the equality of two variances, we use an $F$-test:

$F=\frac{s_{1}^{2}}{s_{2}^{2}}$

$F=22.367 / 15.795=1.416$, with $120-1=119$ numerator and $120-1=119$ denominator degrees of freedom. Because this is a two-tailed test, we use critical values for the $0.05 / 2=0.025$ level. The calculated test statistic falls within the bounds of the critical values (that is, between 0.6969 and 1.4349), so we fail to reject the null hypothesis; there is not enough evidence to indicate that the variances are different before and after the disruption. Note that we could also have formed the $F$-statistic as $15.796 / 22.367=0.706$ and draw the same conclusion.

We could also use software to calculate the critical values:

Excel [F.INV $(0.025,119,119)$ and F.INV $(0.975,119,119)]$

$\mathbf{R}[\mathrm{qf}(\mathrm{c}(.025, .975), 119,119)]$

Python from scipy.stats import $\mathrm{f}$ and f.ppf

$[(.025,119,119)$ and

f.ppf(.975,119,119)]

Additionally, we could use software to calculate the $p$-value of the calculated test statistic, which is $5.896 \%$ and then compare it with the level of significance:

Excel [(1-F.DIST(22.367/15.796,119,119,TRUE))*2 or

F.DIST(15.796/22.367,119,119,TRUE)*2]

R $[(1-\operatorname{-pf}(22.367 / 15.796,119,119)))^{2}$ or

$\left.\operatorname{pf}(15.796 / 22.367,119,119)^{*} 2\right]$

Python from scipy.stats import $f$ and f.cdf

$\left[(15.796 / 22.367,119,119)^{*} 2\right.$ or

(1-f.cdf(22.367/15.796,119,119))*2]

\begin{enumerate}
  \setcounter{enumi}{34}
  \item B is correct. An $F$-test is used to conduct tests concerning the difference between the variances of two normally distributed populations with random independent samples.

  \item A is correct. A nonparametric test is used when the data are given in ranks.

  \item B is correct. There are only 12 (monthly) observations over the one year of the sample and thus the samples are small. Additionally, the funds' returns are non-normally distributed. Therefore, the samples do not meet the distributional assumptions for a parametric test. The Mann-Whitney U test (a nonparametric test) could be used to test the differences between population means.

  \item The hypotheses are $H_{0}: \rho=0$ and $H_{a}: \rho \neq 0$. The calculated test statistics are based on the formula $t=\frac{r \sqrt{n-2}}{\sqrt{1-r^{2}}}$. For example, the calculated $t$-statistic for the correlation of Fund 3 and Fund 4 is

\end{enumerate}

$t=\frac{r \sqrt{n-2}}{\sqrt{1-r^{2}}}=\frac{0.3102 \sqrt{36-2}}{\sqrt{1-0.3102^{2}}}=1.903$.

Repeating this calculation for the entire matrix of correlations gives the following:

\begin{center}
\begin{tabular}{lccccc}
\hline
 & \multicolumn{4}{c}{Calculated t-Statistics for Correlations} &  \\
\hline
 & Fund 1 & Fund 2 & Fund 3 & Fund 4 & S\&P 500 \\
\hline
Fund 1 &  &  &  &  &  \\
Fund 2 &  &  &  &  &  \\
Fund 3 & 13.997 &  &  &  &  \\
Fund 4 & 3.165 & 2.664 &  &  &  \\
S\&P 500 & 5.897 & 6.116 & 1.903 &  &  \\
\hline
\end{tabular}
\end{center}

With critical values of \textbackslash pm 2.032 , with the exception of the correlation between Fund 3 and Fund 4 returns, we reject the null hypothesis for these correlations. In other words, there is sufficient evidence to indicate that the correlations are different from zero, with the exception of the correlation of returns between Fund 3 and Fund 4.

We could use software to determine the critical values:

Excel [T.INV $(0.025,34)$ and T.INV $(0.975,34)]$

$\mathbf{R}[\mathrm{qt}(\mathrm{c}(.025, .975), 34)]$

Python [from scipy.stats import $t$ and t.ppf(.025,34) and t.ppf $(.975,34)]$

We could also use software to determine the $p$-value for the calculated test statistic to enable a comparison with the level of significance. For example, for $t=$ 2.664 , the $p$-value is 0.01172 :

Excel [(1-T.DIST(2.664,34,TRUE) $\left.)^{*} 2\right]$

$\mathbf{R}\left[(1-\operatorname{pt}(2.664,34))^{*} 2\right]$

Python [from scipy.stats import $t$ and (1-t.cdf(2.664,34))*2]

39.

A. We have a "not equal to" alternative hypothesis:

$H_{0}: \rho=0$ versus $H_{a}: \rho \neq 0$ B. Mutual fund expense ratios are bounded from above and below; in practice, there is at least a lower bound on alpha (as any return cannot be less than $-100 \%)$, and expense ratios cannot be negative. These variables may not be normally distributed, and the assumptions of a parametric test are not likely to be fulfilled. Thus, a nonparametric test appears to be appropriate.

We would use the nonparametric Spearman rank correlation coefficient to conduct the test: $r_{s}=1-\frac{6 \sum_{i=1}^{n} d_{i}^{2}}{n\left(n^{2}-1\right)}$ with the $t$-distributed test statistic of $t=\frac{r \sqrt{n-2}}{\sqrt{1-r^{2}}}$

C. The calculation of the Spearman rank correlation coefficient is given in the following table.

\begin{center}
\begin{tabular}{ccccccc}
\hline
Mutual & Alpha & $\begin{array}{c}\text { Expense } \\ \text { Ratio }\end{array}$ & $\begin{array}{c}\text { Rank } \\ \text { by }\end{array}$ & $\begin{array}{c}\text { Rank by } \\ \text { Expense } \\ \text { Ratio }\end{array}$ & $\begin{array}{c}\text { Difference } \\ \text { in Rank }\end{array}$ & $\begin{array}{c}\text { Difference } \\ \text { Squared }\end{array}$ \\
\hline
1 & -0.52 & 1.34 & 6 & 6 & 0 & 0 \\
2 & -0.13 & 0.40 & 1 & 9 & -8 & 64 \\
3 & -0.50 & 1.90 & 5 & 1 & 4 & 16 \\
4 & -1.01 & 1.50 & 9 & 2 & 7 & 49 \\
5 & -0.26 & 1.35 & 3 & 5 & -2 & 4 \\
6 & -0.89 & 0.50 & 8 & 8 & 0 & 0 \\
7 & -0.42 & 1.00 & 4 & 7 & -3 & 9 \\
\hline
 & -0.23 & 1.50 & 2 & 2 & 0 & 9 \\
\hline
\end{tabular}
\end{center}

$$
r_{s}=1-\frac{6(151)}{9(80)}=-0.25833
$$

The calculated test statistic, using the $t$-distributed test statistic $t=\frac{r \sqrt{n-2}}{\sqrt{1-r^{2}}}$ is $t=\frac{-0.25833 \sqrt{7}}{\sqrt{1-0.066736}}=\frac{-0.683486}{0.9332638}=-0.7075$. On the basis of this value falling within the range of \textbackslash pm 2.306 , we fail to reject the null hypothesis that the Spearman rank correlation coefficient is zero.

\begin{enumerate}
  \setcounter{enumi}{39}
  \item A is correct. The calculated test statistic is
\end{enumerate}

$t=\frac{r \sqrt{n-2}}{\sqrt{1-r^{2}}}$

$=\frac{-0.1452 \sqrt{248-2}}{\sqrt{1-(-0.1452)^{2}}}=-2.30177$.

Because the value of $t=-2.30177$ is outside the bounds of \textbackslash pm 1.96 , we reject the null hypothesis of no correlation and conclude that there is enough evidence to indicate that the correlation is different from zero.

41.

A. The hypotheses are as follows:

$H_{0}$ : Dividend and financial leverage ratings are not related, so these groupings are independent. $H_{a}$ : Dividend and financial leverage ratings are related, so these groupings are not independent.

B. The appropriate test statistic is $X^{2}=\sum_{i=1}^{m} \frac{\left(O_{i j}-E_{i j}\right)^{2}}{E_{i j}}$, where $O_{i j}$ represents the observed frequency for the $i$ and $j$ group and $E_{i j}$ represents the expected frequency for the $i$ and $j$ group if the groupings are independent.

C. The expected frequencies based on independence are as follows:

\begin{center}
\begin{tabular}{ccccc}
\hline
 & \multicolumn{3}{c}{Dividend Group} &  \\
\cline { 2 - 5 }
Financial Leverage Group & $\mathbf{1}$ & $\mathbf{2}$ & $\mathbf{3}$ & Sum \\
\hline
1 & 38.4 & 48 & 33.6 & 120 \\
2 & 19.2 & 24 & 16.8 & 60 \\
3 & 22.4 & 28 & 19.6 & 70 \\
Sum & 80 & 100 & 70 & 250 \\
\hline
\end{tabular}
\end{center}

The scaled squared deviations for each combination of financial leverage and dividend grouping are:

\begin{center}
\begin{tabular}{cccc}
\hline
 & \multicolumn{3}{c}{Dividend Group} \\
\cline { 2 - 4 }
Financial Leverage Group & $\mathbf{1}$ & $\mathbf{2}$ & $\mathbf{3}$ \\
\hline
1 & 0.06667 & 1.33333 & 1.21905 \\
2 & 6.07500 & 8.16667 & 0.60952 \\
3 & 6.86429 & 17.28571 & 4.70204 \\
\hline
\end{tabular}
\end{center}

The sum of these scaled squared deviations is the calculated chi-square statistic of 46.3223. Because this calculated value exceeds the critical value of 9.4877, we reject the null hypothesis that these groupings are independent.

\begin{enumerate}
  \setcounter{enumi}{41}
  \item A is correct. The test statistic comprises squared differences between the observed and expected values, so the test involves only one side, the right side. B is incorrect because the null hypothesis is that the groups are independent, and $\mathrm{C}$ is incorrect because with three levels of groups for the two categorical variables, there are four degrees of freedom.
\end{enumerate}

\section*{LEARNING MODULE }
\section{7}
\section{Introduction to Linear Regression}
Pamela Peterson Drake, PhD, CFA, is at James Madison University (USA).

\section{LEARNING OUTCOME}
\begin{center}
\begin{tabular}{c|l}
Mastery & The candidate should be able to: \\
\hline
$\square$ & $\begin{array}{l}\text { describe a simple linear regression model and the roles of the } \\ \text { dependent and independent variables in the model } \\ \text { describe the least squares criterion, how it is used to estimate } \\ \text { regression coefficients, and their interpretation } \\ \text { explain the assumptions underlying the simple linear regression } \\ \text { model, and describe how residuals and residual plots indicate if these } \\ \text { assumptions may have been violated } \\ \text { calculate and interpret the coefficient of determination and the } \\ \text { F-statistic in a simple linear regression } \\ \text { describe the use of analysis of variance (ANOVA) in regression } \\ \text { analysis, interpret ANOVA results, and calculate and interpret the } \\ \text { standard error of estimate in a simple linear regression } \\ \text { formulate a null and an alternative hypothesis about a population } \\ \text { value of a regression coefficient, and determine whether the null } \\ \text { hypothesis is rejected at a given level of significance } \\ \text { calculate and interpret the predicted value for the dependent } \\ \text { variable, and a prediction interval for it, given an estimated linear } \\ \text { regression model and a value for the independent variable } \\ \text { describe different functional forms of simple linear regressions }\end{array}$ \\
$\square$ &  \\
\end{tabular}
\end{center}$\square$

\section{SIMPLE LINEAR REGRESSION}
describe a simple linear regression model and the roles of the dependent and independent variables in the model Financial analysts often need to examine whether a variable is useful for explaining another variable. For example, the analyst may want to know whether earnings growth, or perhaps cash flow growth, helps explain the company's value in the marketplace.

Regression analysis is a tool for examining this type of issue.

Suppose an analyst is examining the return on assets (ROA) for an industry and observes the ROA for the six companies shown in Exhibit 1. The average of these ROAs is $12.5 \%$, but the range is from $4 \%$ to $20 \%$.

\section{Exhibit 1: Return on Assets of Selected Companies}
\begin{center}
\begin{tabular}{lc}
\hline
Company & ROA (\%) \\
\hline
A & 6 \\
B & 4 \\
C & 15 \\
D & 20 \\
E & 10 \\
F & 20 \\
\hline
\end{tabular}
\end{center}

In trying to understand why the ROAs differ among these companies, we could look at why the ROA of Company A differs from that of Company B, why the ROA of Company A differs from that of Company D, why the ROA of Company F differs from that of Company C, and so on, comparing each pair of ROAs. A way to make this a simpler exploration is to try to understand why each company's ROA differs from the mean ROA of $12.5 \%$. We look at the sum of squared deviations of the observations from the mean to capture variations in ROA from their mean. Let $Y$ represent the variable that we would like to explain, which in this case is the return on assets. Let $Y_{i}$ represent an observation of a company's ROA, and let $\bar{Y}$ represent the mean ROA for the sample of size $n$. We can describe the variation of the ROAs as

$$
\text { Variation of } Y=\sum_{i=1}^{n}\left(Y_{i}-\bar{Y}\right)^{2} \text {. }
$$

Our goal is to understand what drives these returns on assets or, in other words, what explains the variation of $Y$. The variation of $Y$ is often referred to as the sum of squares total (SST), or the total sum of squares.

We now ask whether it is possible to explain the variation of the ROA using another variable that also varies among the companies; note that if this other variable is constant or random, it would not serve to explain why the ROAs differ from one another. Suppose the analyst believes that the capital expenditures in the previous period, scaled by the prior period's beginning property, plant, and equipment, are a driver for the ROA variable. Let us represent this scaled capital expenditures variable as CAPEX, as we show in Exhibit 2.

Exhibit 2: Return on Assets and Scaled Capital

Expenditures

\begin{center}
\begin{tabular}{lcc}
\hline
Company & $\begin{array}{c}\text { ROA } \\ (\%)\end{array}$ & $\begin{array}{c}\text { CAPEX } \\ \text { (\%) }\end{array}$ \\
\hline
A & 6.0 & 0.7 \\
B & 4.0 & 0.4 \\
C & 15.0 & 5.0 \\
\end{tabular}
\end{center}

\begin{center}
\begin{tabular}{lcc}
\hline
Company & $\begin{array}{c}\text { ROA } \\ \text { (\%) }\end{array}$ & $\begin{array}{c}\text { CAPEX } \\ \text { (\%) }\end{array}$ \\
\hline
D & 20.0 & 10.0 \\
E & 10.0 & 8.0 \\
F & 20.0 & 12.5 \\
Arithmetic mean &  & 6.10 \\
\hline
\end{tabular}
\end{center}

The variation of $X$, in this case CAPEX, is calculated as

$$
\text { Variation of } X=\sum_{i=1}^{n}\left(X_{i}-\bar{X}\right)^{2} \text {. }
$$

We can see the relation between ROA and CAPEX in the scatter plot (or scattergram) in Exhibit 3, which represents the two variables in two dimensions. Typically, we present the variable whose variation we want to explain along the vertical axis and the variable whose variation we want to use to explain that variation along the horizontal axis. Each point in this scatter plot represents a paired observation that consists of CAPEX and ROA. From a casual visual inspection, there appears to be a positive relation between ROA and CAPEX: Companies with higher CAPEX tend to have a higher ROA.

\section{Exhibit 3: Scatter Plot of ROA and CAPEX}
\begin{center}
\includegraphics[max width=\textwidth]{2023_05_04_cff39ee44f77d6514e1bg-441}
\end{center}

In the ROA example, we use the capital expenditures to explain the returns on assets. We refer to the variable whose variation is being explained as the dependent variable, or the explained variable; it is typically denoted by $Y$. We refer to the variable whose variation is being used to explain the variation of the dependent variable as the independent variable, or the explanatory variable; it is typically denoted by $X$. Therefore, in our example, the ROA is the dependent variable $(Y)$ and CAPEX is the independent variable $(X)$.

A common method for relating the dependent and independent variables is through the estimation of a linear relationship, which implies describing the relation between the two variables as represented by a straight line. If we have only one independent variable, we refer to the method as simple linear regression (SLR); if we have more than one independent variable, we refer to the method as multiple regression. Linear regression allows us to test hypotheses about the relationship between two variables, by quantifying the strength of the relationship between the two variables, and to use one variable to make predictions about the other variable. Our focus is on linear regression with a single independent variable-that is, simple linear regression.

\section{EXAMPLE 1}
Identifying the Dependent and Independent Variables in a Regression

\begin{enumerate}
  \item An analyst is researching the relationship between corporate earnings growth and stock returns. Specifically, she is interested in whether earnings revisions affect stock price returns in the same period. She collects five years of monthly data on "Wall Street" EPS revisions for a sample of 100 companies and on their monthly stock price returns over the five-year period.
\end{enumerate}

What are the dependent and independent variables in her model?

\section{Solution}
The dependent variable is monthly stock price returns, and the independent variable is Wall Street EPS revisions, since in the analyst's model, the variation in monthly stock price returns is being explained by the variation in EPS revisions.

\section{ESTIMATING THE PARAMETERS OF A SIMPLE LINEAR REGRESSION}
describe the least squares criterion, how it is used to estimate regression coefficients, and their interpretation

\section{The Basics of Simple Linear Regression}
Regression analysis begins with the dependent variable, the variable whose variation you are seeking to explain. The independent variable is the variable whose variation you are using to explain changes in the dependent variable. For example, you might try to explain small-stock returns (the dependent variable) using returns to the S\&P 500 Index (the independent variable). Or you might try to explain a country's inflation rate (the dependent variable) as a function of growth in its money supply (the independent variable).

As the name implies, linear regression assumes a linear relationship between the dependent and the independent variables. The goal is to fit a line to the observations on $Y$ and $X$ to minimize the squared deviations from the line; this is the least squares criterion-hence, the name least squares regression. Because of its common use, linear regression is often referred to as ordinary least squares (OLS) regression.

Using notation, the linear relation between the dependent and independent variables is described as $Y_{i}=b_{0}+b_{1} X_{i}+\varepsilon_{i}, i=1, \ldots, n$.

Equation 3 is a model that does not require that every $(Y, X)$ pair for an observation fall on the regression line. This equation states that the dependent variable, $Y$, is equal to the intercept, $b_{0}$, plus a slope coefficient, $b_{1}$, multiplied by the independent variable, $X$, plus an error term, $\varepsilon$. The error term, or simply the error, represents the difference between the observed value of $Y$ and that expected from the true underlying population relation between $Y$ and $X$. We refer to the intercept, $b_{0}$, and the slope coefficient, $b_{1}$, as the regression coefficients. A way that we often describe this simple linear regression relation is that $Y$ is regressed on $X$.

Consider the ROA and CAPEX scatter diagram from Exhibit 3, which we elaborate on in Exhibit 4 by including the fitted regression line. This line represents the average relationship between ROA and CAPEX; not every observation falls on the line, but the line describes the mean relation between ROA and CAPEX.

\section{Exhibit 4: Fitted Regression Line of ROA and CAPEX}
\begin{center}
\includegraphics[max width=\textwidth]{2023_05_04_cff39ee44f77d6514e1bg-443}
\end{center}

\section{Estimating the Regression Line}
We cannot observe the population parameter values $b_{0}$ and $b_{1}$ in a regression model. Instead, we observe only $\hat{b}_{0}$ and $\hat{b}_{1}$, which are estimates (as indicated by the "hats" above the coefficients) of the population parameters based on the sample. Thus, predictions must be based on the parameters' estimated values, and testing is based on estimated values in relation to the hypothesized population values.

We estimate the regression line as the line that best fits the observations. In simple linear regression, the estimated intercept, $\hat{b}_{0}$ and slope, $\hat{b}_{1}$, are such that the sum of the squared vertical distances from the observations to the fitted line is minimized. The focus is on the sum of the squared differences between the observations on $Y_{i}$ and the corresponding estimated value, $\widehat{Y}_{i}$, on the regression line.

We represent the value of the dependent variable for the $i$ th observation that falls on the line as $\hat{Y}_{i}$, which is equal to $\hat{b}_{0}+\hat{b}_{1} X_{i}$. The $\hat{Y}_{i}$ is what the estimated value of the $Y$ variable would be for the $i$ th observation based on the mean relationship between $Y$ and $X$. The residual for the $i$ th observation, $e_{i}$, is how much the observed value of $Y_{i}$ differs from the $\hat{Y}_{i}$ estimated using the regression line: $e_{i}=Y_{i}-\hat{Y}_{i}$. Note the subtle difference between the error term and the residual: The error term refers to the true underlying population relationship, whereas the residual refers to the fitted linear relation based on the sample.

Fitting the line requires minimizing the sum of the squared residuals, the sum of squares error (SSE), also known as the residual sum of squares:

$$
\begin{aligned}
& \text { Sum of squares error }=\sum_{i=1}^{n}\left(Y_{i}-\hat{Y}_{i}\right)^{2} \\
& =\sum_{i=1}^{n}\left[Y_{i}-\left(\hat{b}_{0}+\hat{b}_{1} X_{i}\right)\right]^{2} \\
& =\sum_{i=1}^{n} e_{i}^{2} .
\end{aligned}
$$

Using least squares regression to estimate the values of the population parameters of $b_{0}$ and $b_{1}$, we can fit a line through the observations of $X$ and $Y$ that explains the value that $Y$ takes for any particular value of $X$.

As seen in Exhibit 5, the residuals are represented by the vertical distances from the fitted line (see the third and fifth observations, Companies C and E, respectively) and are, therefore, in the units of measurement represented by the dependent variable. The residual is in the same unit of measurement as the dependent variable: If the dependent variable is in euros, the error term is in euros, and if the dependent variable is in growth rates, the error term is in growth rates.

\section{Exhibit 5: Residuals of the Linear Regression}
\begin{center}
\includegraphics[max width=\textwidth]{2023_05_04_cff39ee44f77d6514e1bg-444}
\end{center}

How do we calculate the intercept $\left(\hat{b}_{0}\right)$ and the slope $\left(\hat{b}_{1}\right)$ for a given sample of $(Y, X)$ pairs of observations? The slope is the ratio of the covariance between $Y$ and $X$ to the variance of $X$, where $\bar{Y}$ is the mean of the $Y$ variable and $\bar{X}$ is the mean of $X$ variable:

$$
\hat{b}_{1}=\frac{\text { Covariance of } Y \text { and } X}{\text { Variance of } X}=\frac{\frac{\sum_{i=1}^{n}\left(Y_{i}-\bar{Y}\right)\left(X_{i}-\bar{X}\right)}{n-1}}{\frac{\sum_{i=1}^{n}\left(X_{i}-X\right)^{2}}{n-1}}
$$

Simplifying,

$$
\hat{b}_{1}=\frac{\sum_{i=1}^{n}\left(Y_{i}-\bar{Y}\right)\left(X_{i}-\bar{X}\right)}{\sum_{i=1}^{n}\left(X_{i}-\bar{X}\right)^{2}}
$$

Once we estimate the slope, we can then estimate the intercept using the mean of $Y$ and the mean of $X$ :

$$
\hat{b}_{0}=\bar{Y}-\hat{b}_{1} \bar{X} .
$$

We show the calculation of the slope and the intercept in Exhibit 6 .

\section{Exhibit 6: Estimating Slope and Intercept for the ROA Model}
\begin{center}
\begin{tabular}{lccccc}
\hline
Company & ROA $\left(Y_{\boldsymbol{i}}\right)$ & CAPEX $\left(X_{\boldsymbol{i}}\right)$ & $\left(Y_{i}-\bar{Y}\right)^{\mathbf{2}}$ & $\left(X_{i}-\bar{X}\right)^{2}$ & $\left(Y_{i}-\bar{Y}\right)\left(X_{i}-\bar{X}\right)$ \\
\hline
A & 6.0 & 0.7 & 42.25 & 29.16 & 35.10 \\
B & 4.0 & 0.4 & 72.25 & 32.49 & 48.45 \\
C & 15.0 & 5.0 & 6.25 & 1.21 & -2.75 \\
D & 20.0 & 10.0 & 56.25 & 15.21 & 29.25 \\
E & 10.0 & 8.0 & 6.25 & 3.61 & -4.75 \\
F & 20.0 & 12.5 & 56.25 & 40.96 & 48.00 \\
\cline { 2 - 6 }
Sum & 75.0 & 36.6 & 239.50 & 122.64 & 153.30 \\
Arithmetic mean & 12.5 & 6.100 &  &  &  \\
\hline
\end{tabular}
\end{center}

Slope coefficient: $\hat{b}_{1}=\frac{153.30}{122.64}=1.25$.

Intercept: $\hat{b}_{0}=12.5-(1.25 \times 6.10)=4.875$

ROA regression model: $\widehat{Y}_{i}=4.875+1.25 X_{i}+\varepsilon_{i}$.

Notice the similarity of the formula for the slope coefficient and that of the pairwise correlation. The sample correlation, $r$, is the ratio of the covariance to the product of the standard deviations:

$$
r=\frac{\text { Covariance of } Y \text { and } X}{\left(\begin{array}{l}
\text { Standard deviation } \\
\text { of } Y
\end{array}\right)\left(\begin{array}{l}
\text { Standard deviation } \\
\text { of } X
\end{array}\right)}
$$

The subtle difference between the slope and the correlation formulas is in the denominator: For the slope, this is the variance of the independent variable, but for the correlation, the denominator is the product of the standard deviations. For our ROA and CAPEX analysis,

$$
\text { Covariance of } Y \text { and } X: \operatorname{cov}_{X Y}=\frac{\sum_{i=1}^{n}\left(Y_{i}-\bar{Y}\right)\left(X_{i}-\bar{X}\right)}{n-1}=\frac{153.30}{5}=30.66 \text {. }
$$

Standard deviation of $Y$ and $X$ :

$$
\begin{aligned}
& S_{Y}=\sqrt{\frac{\sum_{i=1}^{n}\left(Y_{i}-\bar{Y}\right)^{2}}{n-1}}=\sqrt{\frac{239.50}{5}}=6.9210 ; \\
& S_{X}=\sqrt{\frac{\sum_{i=1}^{n}\left(X_{i}-\bar{X}\right)^{2}}{n-1}}=\sqrt{\frac{122.64}{5}}=4.9526 . \\
& r=\frac{30.66}{(6.9210)(4.9526)}=0.89458945 .
\end{aligned}
$$

Because the denominators of both the slope and the correlation are positive, the sign of the slope and the correlation are driven by the numerator: If the covariance is positive, both the slope and the correlation are positive, and if the covariance is negative, both the slope and the correlation are negative.

\section{HOW DO ANALYSTS PERFORM SIMPLE LINEAR REGRESSION?}
Typically, an analyst will use the data analysis functions on a spreadsheet, such as Microsoft Excel, or a statistical package in the $\mathrm{R}$ or Python programming languages to perform linear regression analysis. The following are some of the more common choices in practice.

Simple Linear Regression: Intercept and Slope

\begin{itemize}
  \item Excel: Use the INTERCEPT, SLOPE functions.

  \item $R$ : Use the $\operatorname{lm}$ function.

  \item Python: Use the sm.OLS function in the statsmodels package.

\end{itemize}

\section{Correlations}
\begin{itemize}
  \item Excel: Use the CORREL function.

  \item $R$ : Use the cor function in the stats library.

  \item Python: Use the corrcoef function in the numpy library.

\end{itemize}

Note that in $\mathrm{R}$ and Python, there are many choices for regression and correlation analysis.

\section{Interpreting the Regression Coefficients}
What is the meaning of the regression coefficients? The intercept is the value of the dependent variable if the value of the independent variable is zero. Importantly, this does not make sense in some contexts, especially if it is unrealistic that the independent variable would be zero. For example, if we have a model where money supply explains GDP growth, the intercept has no meaning because, practically speaking, zero money supply is not possible. If the independent variable were money supply growth, however, the intercept is meaningful. The slope is the change in the dependent variable for a one-unit change in the independent variable. If the slope is positive, then the change in the independent variable and that of the dependent variable will be in the same direction; if the slope is negative, the change in the independent variable and that of the dependent variable will be in opposite directions.

\section{INTERPRETING POSITIVE AND NEGATIVE SLOPES}
Suppose the dependent variable $(Y)$ is in millions of euros and the independent variable $(X)$ is in millions of US dollars.

If the slope is positive 1.2 , then

$\uparrow$ USD1 million $\rightarrow \uparrow$ EUR1.2 million

$\downarrow$ USD1 million $\rightarrow \downarrow$ EUR1.2 million

If the slope is negative 1.2 , then

$$
\begin{aligned}
& \uparrow \text { USD1 million } \rightarrow \downarrow \text { EUR1.2 million } \\
& \downarrow \text { USD1 million } \rightarrow \uparrow \text { EUR1.2 million }
\end{aligned}
$$

Using the ROA regression model from Exhibit 6, we would interpret the estimated coefficients as follows:

\begin{itemize}
  \item The return on assets for a company is $4.875 \%$ if the company makes no capital expenditures. - If CAPEX increases by one unit-say, from $4 \%$ to $5 \%-R O A$ increases by $1.25 \%$
\end{itemize}

Using the estimated regression coefficients, we can determine the values of the dependent variable if they follow the average relationship between the dependent and independent variables. A result of the mathematics of the least squares fitting of the regression line is that the expected value of the residual term is zero: $\mathrm{E}(\varepsilon)=0$.

We show the calculation of the predicted dependent variable and residual term for each observation in the ROA example in Exhibit 7. Note that the sum and average of $Y_{i}$ and $\widehat{Y}_{1}$ are the same, and the sum of the residuals is zero.

\section{Exhibit 7: Calculation of the Dependent Variable and Residuals for the ROA and CAPEX Model}
\begin{center}
\begin{tabular}{lcccc}
\hline
 & $\mathbf{( 1 )}$ & $\mathbf{( 2 )}$ & $(\mathbf{3})$ & $(\mathbf{4})$ \\
\hline
Company & ROA $\left(\boldsymbol{Y}_{\boldsymbol{i}}\right)$ & $\begin{array}{c}\text { CAPEX } \\ \left(\boldsymbol{X}_{\boldsymbol{i}}\right)\end{array}$ & $\begin{array}{c}\text { Predicted ROA } \\ \left(\hat{Y}_{i}\right)\end{array}$ & $\begin{array}{c}(\mathbf{1})-(\mathbf{3}) \\ \text { Residual }\left(\boldsymbol{e}_{\boldsymbol{i}}\right)\end{array}$ \\
\hline
A & 6.0 & 0.7 & 5.750 & 0.250 \\
B & 4.0 & 0.4 & 5.375 & -1.375 \\
C & 15.0 & 5.0 & 11.125 & 3.875 \\
D & 20.0 & 10.0 & 17.375 & 2.625 \\
E & 10.0 & 8.0 & 14.875 & -4.875 \\
F & 20.0 & 12.5 & 20.500 & -0.500 \\
\cline { 2 - 5 }
Sum & 75.0 & 36.6 & 75.000 & 0.000 \\
Average & 12.5 & 6.1 & 12.5 & 0.000 \\
\hline
\end{tabular}
\end{center}

For Company $C(i=3)$,

$\widehat{Y}_{i}=\hat{b}_{0}+\hat{b}_{1} X_{i}+\varepsilon_{i}=4.875+1.25 X_{i}+\varepsilon_{i}$

$\widehat{Y}_{i}=4.875+(1.25 \times 5.0)=4.875+6.25=11.125$

$Y_{i}-\hat{Y}_{i}=e_{i}=15.0-11.125=3.875$, the vertical distance in Exhibit 5 .

Whereas the sum of the residuals must equal zero by design, the focus of fitting the regression line in a simple linear regression is minimizing the sum of the squared residual terms.

\section{Cross-Sectional vs. Time-Series Regressions}
Regression analysis uses two principal types of data: cross sectional and time series. A cross-sectional regression involves many observations of $X$ and $Y$ for the same time period. These observations could come from different companies, asset classes, investment funds, countries, or other entities, depending on the regression model. For example, a cross-sectional model might use data from many companies to test whether predicted EPS growth explains differences in price-to-earnings ratios during a specific time period. Note that if we use cross-sectional observations in a regression, we usually denote the observations as $i=1,2, \ldots, n$.

Time-series data use many observations from different time periods for the same company, asset class, investment fund, country, or other entity, depending on the regression model. For example, a time-series model might use monthly data from many years to test whether a country's inflation rate determines its short-term interest rates. If we use time-series data in a regression, we usually denote the observations as $t=1,2, \ldots, T$. Note that in the sections that follow, we primarily use the notation $i=1,2, \ldots, n$, even for time series.

\section{EXAMPLE 2}
\section{Estimating a Simple Linear Regression Model}
An analyst is exploring the relationship between a company's net profit margin and research and development expenditures. He collects data for an industry and calculates the ratio of research and development expenditures to revenues (RDR) and the net profit margin (NPM) for eight companies. Specifically, he wants to explain the variation that he observes in the net profit margin by using the variation he observes in the companies' research and development spending. He reports the data in Exhibit 8.

\section{Exhibit 8: Observations on NPM and RDR for Eight}
Companies

\begin{center}
\begin{tabular}{lcc}
\hline
Company & $\begin{array}{c}\text { NPM } \\ \text { (\%) }\end{array}$ & $\begin{array}{c}\text { RDR } \\ \text { (\%) }\end{array}$ \\
\hline
1 & 4 & 8 \\
2 & 5 & 10 \\
3 & 10 & 6 \\
4 & 9 & 5 \\
5 & 5 & 7 \\
6 & 6 & 9 \\
7 & 12 & 5 \\
8 & 3 & 10 \\
\hline
\end{tabular}
\end{center}

\begin{enumerate}
  \item What is the slope coefficient for this simple linear regression model?
\end{enumerate}

\section{Solution to 1}
The slope coefficient for the regression model is -1.3 , and the details for the inputs to this calculation are in Exhibit 9.

\section{Exhibit 9: Details of Calculation of Slope of NPM Regressed on RDR}
\begin{center}
\begin{tabular}{lccccccc}
\hline
 & $\begin{array}{c}\text { NPM } \\ (\%)\end{array}$ & $\begin{array}{c}\text { RDR } \\ (\%)\end{array}$ & $Y_{i}-\bar{Y}$ &  &  &  \\
$\left(Y_{\boldsymbol{i}}\right)$ & $\left(X_{\boldsymbol{i}}\right)$ &  & $X_{i}-\bar{X}$ & $\left(Y_{i}-\bar{Y}\right)^{2}$ & $\left(X_{i}-\bar{X}\right)^{2}$ & $\left.X_{i}-\bar{X}\right)$ \\
\end{tabular}
\end{center}

Slope coefficient: $\hat{b}_{1}=\frac{-39}{30}=-1.3$.

\begin{enumerate}
  \setcounter{enumi}{1}
  \item What is the intercept for this regression model?
\end{enumerate}

\section{Solution to 2}
The intercept of the regression model is 16.5 :

Intercept: $\hat{b}_{0}=6.75-(-1.3 \times 7.5)=6.75+9.75=16.5$

\begin{enumerate}
  \setcounter{enumi}{2}
  \item How is this estimated linear regression model represented?
\end{enumerate}

\section{Solution to 3}
The regression model is represented by $\hat{Y}_{i}=16.5-1.3 X_{i}+\varepsilon_{i}$.

\begin{enumerate}
  \setcounter{enumi}{3}
  \item What is the pairwise correlation between NPM and RDR?
\end{enumerate}

\section{Solution to 4}
The pairwise correlation is -0.8421 :

$r=\frac{-39 / 7}{\sqrt{71.5 / 7} \sqrt{30 / 7}}=\frac{-5.5714}{(3.1960)(2.0702)}=-0.8421$.

\section{EXAMPLE 3}
\section{Interpreting Regression Coefficients}
An analyst has estimated a model that regresses a company's return on equity (ROE) against its growth opportunities (GO), defined as the company's three-year compounded annual growth rate in sales, over 20 years and produces the following estimated simple linear regression:

$\mathrm{ROE}_{i}=4+1.8 \mathrm{GO}_{i}+\varepsilon_{i}$.

Both variables are stated in percentages, so a GO observation of $5 \%$ is included as 5 .

\begin{enumerate}
  \item The predicted value of the company's ROE if its $\mathrm{GO}$ is $10 \%$ is closest to:
A. $1.8 \%$.
B. $15.8 \%$.
C. $22.0 \%$.
\end{enumerate}

\section{Solution to 1}
$\mathrm{C}$ is correct. The predicted value of $\mathrm{ROE}=4+(1.8 \times 10)=22$.

\begin{enumerate}
  \setcounter{enumi}{1}
  \item The change in ROE for a change in GO from $5 \%$ to $6 \%$ is closest to:
A. $1.8 \%$.
B. $4.0 \%$.
C. $5.8 \%$.
\end{enumerate}

\section{Solution to 2}
A is correct. The slope coefficient of 1.8 is the expected change in the dependent variable (ROE) for a one-unit change in the independent variable (GO). 3. The residual in the case of a GO of $8 \%$ and an observed ROE of $21 \%$ is closest to:
A. $-1.8 \%$.
B. $2.6 \%$.
C. $12.0 \%$.

\section{Solution to 3}
$\mathrm{B}$ is correct. The predicted value is $\mathrm{ROE}=4+(1.8 \times 8)=18.4$. The observed value of $\mathrm{ROE}$ is 21 , so the residual is $2.6=21.0-18.4$.

ASSUMPTIONS OF THE SIMPLE LINEAR REGRESSION MODEL

explain the assumptions underlying the simple linear regression model, and describe how residuals and residual plots indicate if these assumptions may have been violated

We have discussed how to interpret the coefficients in a simple linear regression model. Now we turn to the statistical assumptions underlying this model. Suppose that we have $n$ observations of both the dependent variable, $Y$, and the independent variable, $X$, and we want to estimate the simple linear regression of $Y$ regressed on $X$. We need to make the following four key assumptions to be able to draw valid conclusions from a simple linear regression model:

\begin{enumerate}
  \item Linearity: The relationship between the dependent variable, $Y$, and the independent variable, $X$, is linear.

  \item Homoskedasticity: The variance of the regression residuals is the same for all observations.

  \item Independence: The observations, pairs of $Y \mathrm{~s}$ and $X \mathrm{~s}$, are independent of one another. This implies the regression residuals are uncorrelated across observations.

  \item Normality: The regression residuals are normally distributed.

\end{enumerate}

Now we take a closer look at each of these assumptions and introduce the "best practice" of examining residual plots of regression results to identify potential violations of these key assumptions.

\section{Assumption 1: Linearity}
We are fitting a linear model, so we must assume that the true underlying relationship between the dependent and independent variables is linear. If the relationship between the independent and dependent variables is nonlinear in the parameters, estimating that relation with a simple linear regression model will produce invalid results: The model will be biased, because it will under- and overestimate the dependent variable at certain points. For example, $Y_{i}=b_{0} e^{b_{1} X_{i}}+\varepsilon_{i}$ is nonlinear in $b_{1}$, so we should not apply the linear regression model to it. Exhibit 10 shows an example of this exponential model, with a regression line indicated. You can see that this line does not fit this relationship well: For lower and higher values of $X$, the linear model underestimates the $Y$, whereas for the middle values, the linear model overestimates $Y$.

\section{Exhibit 10: Illustration of Nonlinear Relationship Estimated as a Linear}
 Relationship\begin{center}
\includegraphics[max width=\textwidth]{2023_05_04_cff39ee44f77d6514e1bg-451}
\end{center}

Another implication of this assumption is that the independent variable, $X$, must not be random; that is, it is non-stochastic. If the independent variable is random, there would be no linear relation between the dependent and independent variables. Although we may initially assume that the independent variable in the regression model is not random, that assumption may not always be true.

When we look at the residuals of a model, what we would like to see is that the residuals are random. The residuals should not exhibit a pattern when plotted against the independent variable. As we show in Exhibit 11, the residuals from the Exhibit 10 linear regression do not appear to be random but, rather, exhibit a relationship with the independent variable, $X$, falling for some range of $X$ and rising in another.

Exhibit 11: Illustration of Residuals in a Nonlinear Relationship Estimated as a Linear Relationship

\begin{center}
\includegraphics[max width=\textwidth]{2023_05_04_cff39ee44f77d6514e1bg-451(1)}
\end{center}

\section{Assumption 2: Homoskedasticity}
Assumption 2, that the variance of the residuals is the same for all observations, is known as the homoskedasticity assumption. In terms of notation, this assumption relates to the squared residuals:

$$
\mathrm{E}\left(\varepsilon_{i}^{2}\right)=\sigma_{\varepsilon}^{2}, i=1, \ldots, n .
$$

If the residuals are not homoscedastic, that is, if the variance of residuals differs across observations, then we refer to this as heteroskedasticity.

Suppose you are examining a time series of short-term interest rates as the dependent variable and inflation rates as the independent variable over 16 years. We may believe that short-term interest rates $(Y)$ and inflation rates $(X)$ should be related (that is, interest rates are higher with higher rates of inflation. If this time series spans many years, with different central bank actions that force short-term interest rates to be (artificially) low for the last eight years of the series, then it is likely that the residuals in this estimated model will appear to come from two different models. We will refer to the first eight years as Regime 1 (normal rates) and the second eight years as Regime 2 (low rates). If the model fits differently in the two regimes, the residuals and their variances will be different.

You can see this situation in Exhibit 12, which shows a scatter plot with an estimated regression line. The slope of the regression line over all 16 years is 1.1979.

Exhibit 12: Scatter Plot of Interest Rates $(Y)$ and Inflation Rates $(X)$

\begin{center}
\includegraphics[max width=\textwidth]{2023_05_04_cff39ee44f77d6514e1bg-452}
\end{center}

We plot the residuals of this model in Exhibit 13 against the years. In this plot, we indicate the distance that is two standard deviations from zero (the mean of the residuals) for the first eight years' residuals and then do the same for the second eight years. As you can see, the residuals appear different for the two regimes: the variation in the residuals for the first eight years is much smaller than the variation for the second eight years.

\section{Exhibit 13: Residual Plot for Interest Rates $(Y)$ vs. Inflation Rates $(X)$ Model}
\begin{center}
\includegraphics[max width=\textwidth]{2023_05_04_cff39ee44f77d6514e1bg-453}
\end{center}

Why does this happen? The model seems appropriate, but when we examine the residuals (Exhibit 13), an important step in assessing the model fit, we see that the model fits better in some years compared with others. The difference in variance of residuals between the two regimes is apparent from the much wider band around residuals for Regime 2 (the low-rate period). This indicates a clear violation of the homoskedasticity assumption.

If we estimate a regression line for each regime, we can see that the model for the two regimes is quite different, as we show in Exhibit 14. In the case of Regime 1 (normal rates), the slope is 1.0247, whereas in Regime 2 (low rates) the slope is -0.2805 . In sum, the clustering of residuals in two groups with much different variances clearly indicates the existence of distinct regimes for the relationship between short-term interest rates and the inflation rate.

\section{Exhibit 14: Fitted Regression Lines for the Two Regimes}
\begin{center}
\includegraphics[max width=\textwidth]{2023_05_04_cff39ee44f77d6514e1bg-454}
\end{center}

\section{Assumption 3: Independence}
We assume that the observations ( $Y$ and $X$ pairs) are uncorrelated with one another, meaning they are independent. If there is correlation between observations (that is, autocorrelation), they are not independent and the residuals will be correlated. The assumption that the residuals are uncorrelated across observations is also necessary for correctly estimating the variances of the estimated parameters of $b_{0}$ and $b_{1}$ (i.e., $\hat{b}_{0}$ and $\hat{b}_{1}$ ) that we use in hypothesis tests of the intercept and slope, respectively. It is important to examine whether the residuals exhibit a pattern, suggesting a violation of this assumption. Therefore, we need to visually and statistically examine the residuals for a regression model.

Consider the quarterly revenues of a company regressed over 40 quarters, as shown in Exhibit 15, with the regression line included. It is clear that these revenues display a seasonal pattern, an indicator of autocorrelation.

\section{Exhibit 15: Regression of Quarterly Revenues vs. Time (40 Quarters)}
\begin{center}
\includegraphics[max width=\textwidth]{2023_05_04_cff39ee44f77d6514e1bg-455(1)}
\end{center}

In Exhibit 16, we plot the residuals from this model and see that there is a pattern. These residuals are correlated, specifically jumping up in Quarter 4 and then falling back the subsequent quarter. In sum, the patterns in both Exhibits 15 and 16 indicate a violation of the assumption of independence.

\section{Exhibit 16: Residual Plot for Quarterly Revenues vs. Time Model}
\begin{center}
\includegraphics[max width=\textwidth]{2023_05_04_cff39ee44f77d6514e1bg-455}
\end{center}

\section{Assumption 4: Normality}
The assumption of normality requires that the residuals be normally distributed. This does not mean that the dependent and independent variables must be normally distributed; it only means that the residuals from the model are normally distributed. However, in estimating any model, it is good practice to understand the distribution of the dependent and independent variables to explore for outliers. An outlier in either or both variables can substantially influence the fitted line such that the estimated model will not fit well for most of the other observations.

With normally distributed residuals, we can test a particular hypothesis about a linear regression model. For large sample sizes, we may be able to drop the assumption of normality by appealing to the central limit theorem; asymptotic theory (which deals with large samples) shows that in many cases, the test statistics produced by standard regression programs are valid even if the model's residuals are not normally distributed.

\section{EXAMPLE 4}
\section{Assumptions of Simple Linear Regression}
\begin{enumerate}
  \item An analyst is investigating a company's revenues and estimates a simple linear time-series model by regressing revenues against time, where time -1 , $2, \ldots, 15$-is measured in years. She plots the company's observed revenues and the estimated regression line, as shown in Exhibit 17. She also plots the residuals from this regression model, as shown in Exhibit 18.
\end{enumerate}

\section{Exhibit 17: Revenues vs. Time Using Simple Linear Regression}
\begin{center}
\includegraphics[max width=\textwidth]{2023_05_04_cff39ee44f77d6514e1bg-456}
\end{center}

\section{Exhibit 18: Residual Plot for Revenues vs. Time}
\begin{center}
\includegraphics[max width=\textwidth]{2023_05_04_cff39ee44f77d6514e1bg-457}
\end{center}

Based on Exhibits 17 and 18, describe which assumption(s) of simple linear regression the analyst's model may be violating.

\section{Solution}
The correct model is not linear, as evident from the pattern of the revenues in Exhibit 17. In the earlier years (i.e., 1 and 2) and later years (i.e., 14 and 15), the linear model underestimates revenues, whereas for the middle years (i.e., 7-11), the linear model overestimates revenues. Moreover, the curved pattern of residuals in Exhibit 18 indicates potential heteroskedasticity (residuals have unequal variances), lack of independence of observations, and non-normality (a concern given the small sample size of $n=15$ ). In sum, the analyst should be concerned that her model violates all the assumptions governing simple linear regression (linearity, homoskedasticity, independence, and normality).

\section{ANALYSIS OF VARIANCE}
calculate and interpret the coefficient of determination and the $F$-statistic in a simple linear regression

describe the use of analysis of variance (ANOVA) in regression analysis, interpret ANOVA results, and calculate and interpret the standard error of estimate in a simple linear regression

The simple linear regression model sometimes describes the relationship between two variables quite well, but sometimes it does not. We must be able to distinguish between these two cases to use regression analysis effectively. Remember our goal is to explain the variation of the dependent variable. So, how well has this goal been achieved, given our choice of independent variable?

\section{Breaking down the Sum of Squares Total into Its Components}
We begin with the sum of squares total and then break it down into two parts: the sum of squares error and the sum of squares regression (SSR). The sum of squares regression is the sum of the squared differences between the predicted value of the dependent variable, $\hat{Y}_{i}$, based on the estimated regression line, and the mean of the dependent variable, $\bar{Y}$ :

$$
\sum_{i=1}^{n}\left(\hat{Y}_{i}-\bar{Y}\right)^{2}
$$

We have already defined the sum of squares total, which is the total variation in $Y$, and the sum of squares error, the unexplained variation in $Y$. Note that the sum of squares regression is the explained variation in $Y$. So, as illustrated in Exhibit 19, SST = SSR + SSE, meaning total variation in $Y$ equals explained variation in $Y$ plus unexplained variation in $Y$.

\section{Exhibit 19: Breakdown of Variation of Dependent Variable}
\begin{center}
\includegraphics[max width=\textwidth]{2023_05_04_cff39ee44f77d6514e1bg-458}
\end{center}

We show the breakdown of the sum of squares total formula for our ROA regression example in Exhibit 20. The total variation of ROA that we want to explain (SST) is 239.50. This number comprises the variation unexplained (SSE), 47.88, and the variation explained (SSR), 191.63. These sum of squares values are important inputs into measures of the fit of the regression line.

Exhibit 20: Breakdown of Sum of Squares Total for ROA Model

\begin{center}
\begin{tabular}{|c|c|c|c|c|c|c|}
\hline
Company & $\begin{array}{c}\mathrm{ROA} \\ \left(Y_{i}\right)\end{array}$ & $\begin{array}{c}\text { CAPEX } \\ \left(X_{i}\right)\end{array}$ & $\begin{array}{c}\text { Predicted } \\ \text { ROA } \\ (\hat{Y})\end{array}$ & $\begin{array}{c}\text { Variation } \\ \text { to } \mathrm{Be} \\ \text { Explained } \\ \left(Y_{i}-\bar{Y}\right)^{2}\end{array}$ & $\begin{array}{l}\text { Variation } \\ \text { Unexplained } \\ \left(Y_{i}-\hat{Y}_{i}\right)^{2}\end{array}$ & $\begin{array}{l}\text { Variation } \\ \text { Explainec } \\ \left(\hat{Y}_{i}-\bar{Y}\right)^{2}\end{array}$ \\
\hline
A & 6.0 & 0.7 & 5.750 & 42.25 & 0.063 & 45.563 \\
\hline
B & 4.0 & 0.4 & 5.375 & 72.25 & 1.891 & 50.766 \\
\hline
$\mathrm{C}$ & 15.0 & 5.0 & 11.125 & 6.25 & 15.016 & 1.891 \\
\hline
\end{tabular}
\end{center}

\begin{center}
\begin{tabular}{|c|c|c|c|c|c|c|}
\hline
Company & $\begin{array}{c}\mathrm{ROA} \\ \left(Y_{i}\right)\end{array}$ & $\begin{array}{c}\text { CAPEX } \\ \left(X_{i}\right)\end{array}$ & $\begin{array}{c}\text { Predicted } \\ \text { ROA } \\ (\widehat{Y})\end{array}$ & $\begin{array}{c}\text { Variation } \\ \text { to } \mathrm{Be} \\ \text { Explained } \\ \left(Y_{i}-\bar{Y}\right)^{2}\end{array}$ & $\begin{array}{c}\text { Variation } \\ \text { Unexplained } \\ \left(Y_{i}-\hat{Y}_{i}\right)^{2}\end{array}$ & $\begin{array}{l}\text { Variation } \\ \text { Explained } \\ \left(\hat{Y}_{i}-\bar{Y}\right)^{2}\end{array}$ \\
\hline
D & 20.0 & 10.0 & 17.375 & 56.25 & 6.891 & 23.766 \\
\hline
E & 10.0 & 8.0 & 14.875 & 6.25 & 23.766 & 5.641 \\
\hline
\multirow{2}{*}{$\mathrm{F}$} & 20.0 & 12.5 & 20.500 & 56.25 & 0.250 & 64.000 \\
\hline
 &  &  &  & 239.50 & 47.88 & 191.625 \\
\hline
Mean & 12.50 &  &  &  &  &  \\
\hline
\end{tabular}
\end{center}

Sum of squares total $=239.50$.

Sum of squares error $=47.88$

Sum of squares regression $=191.63$.

\section{Measures of Goodness of Fit}
There are several measures that we can use to evaluate goodness of fit-that is, how well the regression model fits the data. These include the coefficient of determination, the $F$-statistic for the test of fit, and the standard error of the regression.

The coefficient of determination, also referred to as the $R$-squared or $R^{2}$, is the percentage of the variation of the dependent variable that is explained by the independent variable:

Coefficient of determination $=\frac{\text { Sum of squares regression }}{\text { Sum of squares total }}$

Coefficient of determination $=\frac{\sum_{i=1}^{n}\left(\widehat{Y}_{i}-\bar{Y}\right)^{2}}{\sum_{i=1}^{n}\left(Y_{i}-\bar{Y}\right)^{2}}$.

By construction, the coefficient of determination ranges from $0 \%$ to $100 \%$. In our ROA example, the coefficient of determination is $191.625 \div 239.50$, or 0.8001 , so 80.01\% of the variation in ROA is explained by CAPEX. In a simple linear regression, the square of the pairwise correlation is equal to the coefficient of determination:

$$
r^{2}=\frac{\sum_{i=1}^{n}\left(\widehat{Y}_{i}-\bar{Y}\right)^{2}}{\sum_{i=1}^{n}\left(Y_{i}-\bar{Y}\right)^{2}}=R^{2}
$$

In our earlier ROA regression analysis, $r=0.8945$, so we now see that $r^{2}$ is indeed equal to the coefficient of determination $\left(R^{2}\right)$, since $(0.8945)^{2}=0.8001$.

Whereas the coefficient of determination-the portion of the variation of the dependent variable explained by the independent variable-is descriptive, it is not a statistical test. To see if our regression model is likely to be statistically meaningful, we will need to construct an $F$-distributed test statistic.

In general, we use an $F$-distributed test statistic to compare two variances. In regression analysis, we can use an $F$-distributed test statistic to test whether the slopes in a regression are equal to zero, with the slopes designated as $b_{i}$, against the alternative hypothesis that at least one slope is not equal to zero:

$H_{0}: b_{1}=b_{2}=b_{3}=\ldots=b_{k}=0$

$H_{a}:$ At least one $b_{k}$ is not equal to zero.

For simple linear regression, these hypotheses simplify to

$H_{0}: b_{1}=0$.

$H_{a}: b_{1} \neq 0$. The $F$-distributed test statistic is constructed by using the sum of squares regression and the sum of squares error, each adjusted for degrees of freedom; in other words, it is the ratio of two variances. We divide the sum of squares regression by the number of independent variables, represented by $k$. In the case of a simple linear regression, $\mathrm{k}=1$, so we arrive at the mean square regression (MSR), which is the same as the sum of squares regression:

$$
\text { MSR }=\frac{\text { Sum of squares regression }}{k}=\frac{\sum_{i=1}^{n}\left(\widehat{Y}_{i}-\bar{Y}\right)^{2}}{1} .
$$

So, for simple linear regression,

$$
\mathrm{MSR}=\sum_{i=1}^{n}\left(\widehat{Y}_{i}-\bar{Y}\right)^{2} .
$$

Next, we calculate the mean square error (MSE), which is the sum of squares error divided by the degrees of freedom, which are $n-k-1$. In simple linear regression, $n-k-1$ becomes $n-2$ :

$$
\begin{aligned}
\text { MSE } & =\frac{\text { Sum of squares error }}{n-k-1} . \\
\text { MSE } & =\frac{\sum_{i=1}^{n}\left(Y_{i}-\widehat{Y}_{i}\right)^{2}}{n-2} .
\end{aligned}
$$

Therefore, the $F$-distributed test statistic (MSR/MSE) is

$$
\begin{aligned}
& F=\frac{\frac{\text { Sum of squares regression }}{k}}{\frac{\text { Sum of squares error }}{n-k-1}}=\frac{\text { MSR }}{\text { MSE }} \\
& F=\frac{\frac{\sum_{i=1}^{n}\left(\hat{Y}_{i}-Y^{2}\right.}{1}}{\frac{\sum_{i=1}^{n}\left(Y_{i}-\hat{Y}_{i}\right)^{2}}{n-2}}
\end{aligned}
$$

which is distributed with 1 and $n-2$ degrees of freedom in simple linear regression. The $F$-statistic in regression analysis is one sided, with the rejection region on the right side, because we are interested in whether the variation in $Y$ explained (the numerator) is larger than the variation in $Y$ unexplained (the denominator).

\section{ANOVA and Standard Error of Estimate in Simple Linear Regression}
We often represent the sums of squares from a regression model in an analysis of variance (ANOVA) table, as shown in Exhibit 21, which presents the sums of squares, the degrees of freedom, the mean squares, and the $F$-statistic. Notice that the variance of the dependent variable is the ratio of the sum of squares total to $n-1$.

\section{Exhibit 21: Analysis of Variance Table for Simple Linear Regression}
\begin{center}
\includegraphics[max width=\textwidth]{2023_05_04_cff39ee44f77d6514e1bg-460}
\end{center}

\begin{center}
\begin{tabular}{lcc}
\hline
Source & Sum of Squares & $\begin{array}{c}\text { Degrees of } \\ \text { Freedom }\end{array}$ \\
\hline
Error & FSE $=\sum_{i=1}^{n}\left(Y_{i}-\widehat{Y}_{i}\right)^{2}$ & $n-2$ \\
Total & $\mathrm{SST}=\sum_{i=1}^{n}\left(Y_{i}-\bar{Y}\right)^{2}$ & $n-1$ \\
\hline
\end{tabular}
\end{center}

From the ANOVA table, we can also calculate the standard error of the estimate $\left(s_{e}\right)$, which is also known as the standard error of the regression or the root mean square error. The $s_{e}$ is a measure of the distance between the observed values of the dependent variable and those predicted from the estimated regression; the smaller the $s_{e}$, the better the fit of the model. The $s_{e}$, along with the coefficient of determination and the $F$-statistic, is a measure of the goodness of the fit of the estimated regression line. Unlike the coefficient of determination and the $F$-statistic, which are relative measures of fit, the standard error of the estimate is an absolute measure of the distance of the observed dependent variable from the regression line. Thus, the $s_{e}$ is an important statistic used to evaluate a regression model and is used in calculating prediction intervals and performing tests on the coefficients. The calculation of $s_{e}$ is straightforward once we have the ANOVA table because it is the square root of the MSE:

Standard error of the estimate $\left(s_{e}\right)=\sqrt{\mathrm{MSE}}=\sqrt{\frac{\sum_{i=1}^{n}\left(Y_{i}-\hat{Y}_{i}\right)^{2}}{n-2}}$.

We show the ANOVA table for our ROA regression example in Exhibit 22, using the information from Exhibit 20. For a 5\% level of significance, the critical $F$-value for the test of whether the model is a good fit (that is, whether the slope coefficient is different from zero) is 7.71. We can get this critical value in the following ways:

\begin{itemize}
  \item Excel $[\mathrm{F} . \mathrm{INV}(0.95,1,4)]$

  \item $R[\mathrm{qf}(.95,1,4)]$

  \item Python [from scipy.stats import $\mathrm{f}$ and f.ppf(.95,1,4)]

\end{itemize}

With a calculated $F$-statistic of 16.0104 and a critical $F$-value of 7.71 , we reject the null hypothesis and conclude that the slope of our simple linear regression model for ROA is different from zero.

\section{Exhibit 22: ANOVA Table for ROA Regression Model}
\begin{center}
\begin{tabular}{lcccc}
 &  &  &  &  \\
 & $\begin{array}{c}\text { Degrees } \\ \text { of }\end{array}$ &  &  &  \\
Source & Sum of Squares & Freedom & Mean Square & F-Statistic \\
\hline
Regression & 191.625 & 1 & 191.625 & 16.0104 \\
Error & 47.875 & 4 & 11.96875 &  \\
Total & 239.50 & 5 &  &  \\
\hline
\end{tabular}
\end{center}

The calculations to derive the ANOVA table and ultimately to test the goodness of fit of the regression model can be time consuming, especially for samples with many observations. However, statistical packages, such as SAS, SPSS Statistics, and Stata, as well as software, such as Excel, $\mathrm{R}$, and Python, produce the ANOVA table as part of the output for regression analysis.

\section{EXAMPLE 5}
\section{Using ANOVA Table Results to Evaluate a Simple Linear Regression}
Suppose you run a cross-sectional regression for 100 companies, where the dependent variable is the annual return on stock and the independent variable is the lagged percentage of institutional ownership (INST). The results of this simple linear regression estimation are shown in Exhibit 23. Evaluate the model by answering the questions below.

Exhibit 23: ANOVA Table for Annual Stock Return Regressed on Institutional Ownership

\begin{center}
\begin{tabular}{lccc}
\hline
Source & Sum of Squares & $\begin{array}{c}\text { Degrees of } \\ \text { Freedom }\end{array}$ & Mean Square \\
\hline
Regression & 576.1485 & 1 & 576.1485 \\
Error & $1,873.5615$ & 98 & 19.1180 \\
Total & $2,449.7100$ &  &  \\
\hline
\end{tabular}
\end{center}

\begin{enumerate}
  \item What is the coefficient of determination for this regression model?
\end{enumerate}

\section{Solution to 1}
The coefficient of determination is sum of squares regression/sum of squares total: $576.148 \div 2,449.71=0.2352$, or $23.52 \%$.

\begin{enumerate}
  \setcounter{enumi}{1}
  \item What is the standard error of the estimate for this regression model?
\end{enumerate}

\section{Solution to 2}
The standard error of the estimate is the square root of the mean square error: $\sqrt{19.1180}=4.3724$.

\begin{enumerate}
  \setcounter{enumi}{2}
  \item At a $5 \%$ level of significance, do we reject the null hypothesis of the slope coefficient equal to zero if the critical $F$-value is 3.938 ?
\end{enumerate}

\section{Solution to 3}
Using a six-step process for testing hypotheses, we get the following:

Step $1 \quad$ State the hypotheses.

Step $2 \quad$ Identify the appropriate test statistic.

Step 3 Specify the level of significance.

Step $4 \quad$ State the decision rule.

Step 5 Calculate the test statistic.

Step $6 \quad$ Make a decision. $H_{0}: b_{1}=0$ versus $H_{a}: b_{1} \neq 0$

$F=\frac{M S R}{\text { MSE }}$

with 1 and 98 degrees of freedom.

$\alpha=5 \%$ (one tail, right side).

Critical $F$-value $=3.938$.

Reject the null hypothesis if the calculated $F$-statistic is greater than 3.938 .

$F=\frac{576.1485}{19.1180}=30.1364$

Reject the null hypothesis because the calculated $F$-statistic is greater than the critical $F$-value. There is sufficient evidence to indicate that the slope coefficient is different from 0.0 . 4. Based on your answers to the preceding questions, evaluate this simple linear regression model.

\section{Solution to 4}
The coefficient of determination indicates that variation in the independent variable explains $23.52 \%$ of the variation in the dependent variable. Also, the $F$-statistic test confirms that the model's slope coefficient is different from 0 at the $5 \%$ level of significance. In sum, the model seems to fit the data reasonably well.

\section{HYPOTHESIS TESTING OF LINEAR REGRESSION COEFFICIENTS}
formulate a null and an alternative hypothesis about a population value of a regression coefficient, and determine whether the null hypothesis is rejected at a given level of significance

\section{Hypothesis Tests of the Slope Coefficient}
We can use the $F$-statistic to test for the significance of the slope coefficient (that is, whether it is significantly different from zero), but we also may want to perform other hypothesis tests for the slope coefficient-for example, testing whether the population slope is different from a specific value or whether the slope is positive. We can use a $t$-distributed test statistic to test such hypotheses about a regression coefficient.

Suppose we want to check a stock's valuation using the market model; we hypothesize that the stock has an average systematic risk (i.e., risk similar to that of the market), as represented by the coefficient on the market returns variable. Or we may want to test the hypothesis that economists' forecasts of the inflation rate are unbiased (that is, on average, not overestimating or underestimating actual inflation rates). In each case, does the evidence support the hypothesis? Such questions as these can be addressed with hypothesis tests on the regression slope. To test a hypothesis about a slope, we calculate the test statistic by subtracting the hypothesized population slope $\left(B_{1}\right)$ from the estimated slope coefficient $\left(\hat{b}_{1}\right)$ and then dividing this difference by the standard error of the slope coefficient, $s_{\hat{b}_{1}}$ :

$$
t=\frac{\hat{b}_{1}-B_{1}}{s_{\hat{b}_{1}}}
$$

This test statistic is $t$-distributed with $n-k-1$ or $n-2$ degrees of freedom because two parameters (an intercept and a slope) were estimated in the regression.

The standard error of the slope coefficient $(s \hat{b})$ for a simple linear regression is the ratio of the model's standard error of the estimate $\left(s_{e}\right)$ to the square root of the variation of the independent variable:

$$
s_{\hat{b}_{1}}=\frac{s_{e}}{\sqrt{\sum_{i=1}^{n}\left(X_{i}-\bar{X}\right)^{2}}}
$$

We compare the calculated $t$-statistic with the critical values to test hypotheses. Note that the greater the variability of the independent variable, the lower the standard error of the slope (Equation 16) and hence the greater the calculated $t$-statistic (Equation 15). If the calculated $t$-statistic is outside the bounds of the critical $t$-values, we reject the null hypothesis, but if the calculated $t$-statistic is within the bounds of the critical values, we fail to reject the null hypothesis. Similar to tests of the mean, the alternative hypothesis can be two sided or one sided.

Consider our previous simple linear regression example with ROA as the dependent variable and CAPEX as the independent variable. Suppose we want to test whether the slope coefficient of CAPEX is different from zero to confirm our intuition of a significant relationship between ROA and CAPEX. We can test the hypothesis concerning the slope using the six-step process, as we show in Exhibit 24. As a result of this test, we conclude that the slope is different from zero; that is, CAPEX is a significant explanatory variable of ROA.

\section{Exhibit 24: Test of the Slope for the Regression of ROA on CAPEX}
Step 1

Step 2 State the hypotheses.

Identify the appropriate test statistic.

Step 3 Specify the level of significance.

Step $4 \quad$ State the decision rule. $H_{0}: b_{1}=0$ versus $H_{a}: b_{1} \neq 0$

$t=\frac{\hat{b}-1 B_{1}}{s \hat{b}_{1}}$

with $6-2=4$ degrees of freedom.

$\alpha=5 \%$.

Critical $t$-values $= \pm 2.776$.

We can determine this from

Excel

Lower: T.INV(0.025,4)

Upper: T.INV $(0.975,4)$

R qt(c(.025,.975),4)

Python from scipy.stats import t

Lower: t.ppf(.025,4)

Upper: t.ppf(.975,4)

We reject the null hypothesis if the calculated $t$-statistic is less than -2.776 or greater than +2.776 .

The slope coefficient is 1.25 (Exhibit 6).

The mean square error is 11.96875 (Exhibit 22).

The variation of CAPEX is 122.640 (Exhibit 6).

$s_{e}=\sqrt{11.96875}=3.459588$.

$s_{s_{b}}=\frac{3.459588}{\sqrt{122.640}}=0.312398$.

$t=\frac{1.25-0}{0.312398}=4.00131$.

Reject the null hypothesis of a zero slope. There is sufficient evidence to indicate that the slope is different from zero.

A feature of simple linear regression is that the $t$-statistic used to test whether the slope coefficient is equal to zero and the $t$-statistic to test whether the pairwise correlation is zero (that is, $H_{0}: \rho=0$ versus $H_{a}: \rho \neq 0$ ) are the same value. Just as with a test of a slope, both two-sided and one-sided alternatives are possible for a test of a correlation-for example, $H_{0}: \rho \leq 0$ versus $H_{a}: \rho>0$. The test-statistic to test whether the correlation is equal to zero is

$$
t=\frac{r \sqrt{n-2}}{\sqrt{1-r^{2}}}
$$

In our example of ROA regressed on CAPEX, the correlation $(r)$ is 0.8945 . To test whether this correlation is different from zero, we perform a test of hypothesis, shown in Exhibit 25. As you can see, we draw a conclusion similar to that for our test of the slope, but it is phrased in terms of the correlation between ROA and CAPEX: There is a significant correlation between ROA and CAPEX.

\section{Exhibit 25: Test of the Correlation between ROA and CAPEX}
Step 1 State the hypotheses.

Step 2 Identify the appropriate test statistic.

Step 3 Specify the level of significance.

Step $4 \quad$ State the decision rule. $H_{0}: \rho=0$ versus $H_{a}: \rho \neq 0$

$$
t=\frac{r \sqrt{n-2}}{\sqrt{1-r^{2}}} .
$$

with $6-2=4$ degrees of freedom.

$\alpha=5 \%$.

Critical $t$-values $= \pm 2.776$.

Reject the null if the calculated $t$-statistic is less than -2.776 or greater than +2.776 .

$$
t=\frac{0.8945 \sqrt{4}}{\sqrt{1-0.8001}}=4.00131 .
$$

Reject the null hypothesis of no correlation. There is sufficient evidence to indicate that the correlation between ROA and CAPEX is different from zero.

Another interesting feature of simple linear regression is that the test-statistic used to test the fit of the model (that is, the $F$-distributed test statistic) is related to the calculated $t$-statistic used to test whether the slope coefficient is equal to zero: $t^{2}=F$; therefore, $4.00131^{2}=16.0104$.

What if instead we want to test whether there is a one-to-one relationship between ROA and CAPEX, implying a slope coefficient of 1.0. The hypotheses become $H_{0}: b_{1}$ $=1$ and $H_{a}: b_{1} \neq 1$. The calculated $t$-statistic is

$$
t=\frac{1.25-1}{0.312398}=0.80026 .
$$

This calculated test statistic falls within the bounds of the critical values, \textbackslash pm 2.776 , so we fail to reject the null hypothesis: There is not sufficient evidence to indicate that the slope is different from 1.0.

What if instead we want to test whether there is a positive slope or positive correlation, as our intuition suggests? In this case, all the steps are the same as in Exhibits 24 and 25 except the critical values because the tests are one sided. For a test of a positive slope or positive correlation, the critical value for a $5 \%$ level of significance is +2.132. We show the test of hypotheses for a positive slope and a positive correlation in Exhibit 26. Our conclusion is that there is sufficient evidence supporting both a positive slope and a positive correlation.

\section{Exhibit 26: One-Sided Tests for the Slope and Correlation}
Test of the Slope

Step 1

Step 2 State the hypotheses.

Identify the appropriate test statistic. $H_{0}: b_{1} \leq 0$ versus $H_{a}: b_{1}>0$

$t=\frac{\hat{b}-1 B_{1}}{s_{\hat{b}_{1}}}$

with $6-2=4$ degrees of freedom. Test of the Correlation

$H_{0}: \rho \leq 0$ versus $H_{a}: \rho>0$

$$
t=\frac{r \sqrt{n-2}}{\sqrt{1-r^{2}}} .
$$

with $6-2=4$ degrees of freedom.

\begin{center}
\begin{tabular}{|c|c|c|c|}
\hline
 &  & Test of the Slope & Test of the Correlation \\
\hline
Step 3 & $\begin{array}{l}\text { Specify the level of } \\ \text { significance. }\end{array}$ & $\alpha=5 \%$ & $\alpha=5 \%$ \\
\hline
Step 4 & State the decision rule. & $\begin{array}{l}\text { Critical } t \text {-value }=2.132 \text {. } \\ \text { Reject the null if the calculated } t \text {-statistic } \\ \text { is greater than } 2.132 \text {. }\end{array}$ & $\begin{array}{l}\text { Critical } t \text {-value }=2.132 \\ \text { Reject the null if the calculated } t \text {-statistic } i \\ \text { greater than } 2.132 .\end{array}$ \\
\hline
Step 5 & $\begin{array}{l}\text { Calculate the test } \\ \text { statistic. }\end{array}$ & $t=\frac{1.25-0}{0.312398}=4.00131$ & $t=\frac{0.8945 \sqrt{4}}{\sqrt{1-0.8001}}=4.00131$ \\
\hline
Step 6 & Make a decision. & $\begin{array}{l}\text { Reject the null hypothesis. There is suffi- } \\ \text { cient evidence to indicate that the slope } \\ \text { is greater than zero. }\end{array}$ & $\begin{array}{l}\text { Reject the null hypothesis. There is suffi- } \\ \text { cient evidence to indicate that the correla- } \\ \text { tion is greater than zero. }\end{array}$ \\
\hline
\end{tabular}
\end{center}

\section{Hypothesis Tests of the Intercept}
There are occasions when we want to test whether the population intercept is a specific value. As a reminder on how to interpret the intercept, consider the simple linear regression with a company's revenue growth rate as the dependent variable $(Y)$ and the GDP growth rate of its home country as the independent variable $(X)$. The intercept is the company's revenue growth rate if the GDP growth rate is $0 \%$.

The equation for the standard error of the intercept, $s_{\hat{b}}^{0}$, is

$$
s_{\hat{b}_{0}}=\sqrt{\frac{1}{n}+\frac{\bar{X}^{2}}{\sum_{i=1}^{n}\left(X_{i}-\bar{X}\right)^{2}}} .
$$

We can test whether the intercept is different from the hypothesized value, $B_{0}$, by comparing the estimated intercept $\left(\widehat{b}_{0}\right)$ with the hypothesized intercept and then dividing the difference by the standard error of the intercept:

$$
t_{\text {intercept }}=\frac{\hat{b}_{0}-B_{0}}{s_{b_{0}}}=\frac{\hat{b}_{0}-B_{0}}{\sqrt{\frac{1}{n}+\frac{\bar{X}^{2}}{\sum_{i=1}^{n}\left(X_{i}-\bar{X}\right)^{2}}}}
$$

In the ROA regression example, the intercept is $4.875 \%$. Suppose we want to test whether the intercept is greater than $3 \%$. The one-sided hypothesis test is shown in Exhibit 27. As you can see, we reject the null hypothesis. In other words, there is sufficient evidence that if there are no capital expenditures (CAPEX $=0$ ), ROA is greater than $3 \%$.

\section{Exhibit 27: Test of Hypothesis for Intercept for Regression of ROA on CAPEX}
Step $1 \quad$ State the hypotheses.

Step 2 Identify the appropriate test statistic.

Step $3 \quad$ Specify the level of significance.

Step $4 \quad$ State the decision rule.

Step 5 Calculate the test statistic. $H_{0}: b_{0} \leq 3 \%$ versus $H_{a}: b_{0}>3 \%$

$t_{\text {intercept }}=\frac{\hat{b}_{0}-B_{0}}{s_{\hat{b}}}$

with $6-2=4$ degrees of freedom.

$\alpha=5 \%$.

Critical $t$-value $=2.132$.

Reject the null if the calculated $t$-statistic is greater than 2.132 .

$$
t_{\text {intercept }}=\frac{4.875-3.0}{\sqrt{\frac{1}{6}+\frac{6.1^{2}}{122.64}}}=\frac{1.875}{0.68562}=2.73475
$$

\begin{center}
\includegraphics[max width=\textwidth]{2023_05_04_cff39ee44f77d6514e1bg-467}
\end{center}

\section{Hypothesis Tests of Slope When Independent Variable Is an Indicator Variable}
Suppose we want to examine whether a company's quarterly earnings announcements influence its monthly stock returns. In this case, we could use an indicator variable, or dummy variable, that takes on only the values 0 or 1 as the independent variable. Consider the case of a company's monthly stock returns over a 30-month period. A simple linear regression model for investigating this question would be monthly returns, RET, regressed on the indicator variable, EARN, that takes on a value of 0 if there is no earnings announcement that month and 1 if there is an earnings announcement:

$$
\operatorname{RET}_{i}=b_{0}+b_{1} \mathrm{EARN}_{i}+\varepsilon_{i}
$$

This regression setup allows us to test whether there are different returns for earnings-announcement months versus non-earnings-announcement months. The observations and regression results are shown graphically in Exhibit 28.

\section{Exhibit 28: Earnings Announcements, Dummy Variable, and Stock Returns}
\begin{center}
\includegraphics[max width=\textwidth]{2023_05_04_cff39ee44f77d6514e1bg-467(1)}
\end{center}

Clearly there are some months in which the returns are different from other months, and these correspond to months in which there was an earnings announcement. We estimate the simple linear regression model and perform hypothesis testing in the same manner as if the independent variable were a continuous variable. In a simple linear regression, the interpretation of the intercept is the predicted value of the dependent variable if the indicator variable is zero. Moreover, the slope, when the indicator variable is 1 , is the difference in the means if we grouped the observations by the indicator variable. The results of the regression are given in Panel A of Exhibit 29.

\section{Exhibit 29: Regression and Test of Differences Using an Indicator Variable}
\section{A. Regression Estimation Results}
\begin{center}
\begin{tabular}{lccc}
\hline
 & $\begin{array}{c}\text { Estimated } \\ \text { Coefficients }\end{array}$ & $\begin{array}{c}\text { Standard Error of } \\ \text { Coefficients }\end{array}$ & $\begin{array}{c}\text { Calculated Test } \\ \text { Statistic }\end{array}$ \\
\hline
Intercept & 0.5629 & 0.0560 & 10.0596 \\
EARN & 1.2098 & 0.1158 & 10.4435 \\
\hline
\end{tabular}
\end{center}

Degrees of freedom $=28$.

Critical $t$-values $=+2.0484(5 \%$ significance $)$.

\section{B. Test of Differences in Means}
\begin{center}
\begin{tabular}{|c|c|c|c|}
\hline
 & $\begin{array}{c}\text { RET for } \\ \text { Earnings-Announcement } \\ \text { Months }\end{array}$ & $\begin{array}{c}\text { RET for } \\ \text { Non-Earnings-Announcement } \\ \text { Months }\end{array}$ & $\begin{array}{c}\text { Difference in } \\ \text { Means }\end{array}$ \\
\hline
Mean & 1.7727 & 0.5629 & 1.2098 \\
\hline
Variance & 0.1052 & 0.0630 &  \\
\hline
Observations & 7 & 23 &  \\
\hline
$\begin{array}{l}\text { Pooled } \\ \text { variance }\end{array}$ &  &  & 0.07202 \\
\hline
$\begin{array}{l}\text { Calculated tes } \\ \text { statistic }\end{array}$ &  &  & 10.4435 \\
\hline
\end{tabular}
\end{center}

Degrees of freedom $=28$.

Critical $t$-values $=+2.0484(5 \%$ significance $)$.

We can see the following from Panel A of Exhibit 29:

\begin{itemize}
  \item The intercept (0.5629) is the mean of the returns for non-earnings-announcement months.

  \item The slope coefficient (1.2098) is the difference in means of returns between earnings-announcement and non-announcement months.

  \item We reject the null hypothesis that the slope coefficient on EARN is equal to zero. We also reject the null hypothesis that the intercept is zero. The reason is that in both cases, the calculated test statistic exceeds the critical $t$-value.

\end{itemize}

We could also test whether the mean monthly return is the same for both the non-earnings-announcement months and the earnings-announcement months by testing the following:

$$
H_{0}: \mu_{R E T e a r n i n g s}=\mu_{R E T N o n-e a r n i n g s} \text { and } H_{a}: \mu_{R E T e a r n i n g s} \neq \mu_{R E T N o n-e a r n i n g s}
$$

The results of this hypothesis test are gleaned from Panel B of Exhibit 29. As you can see, we reject the null hypothesis that there is no difference in the mean RET for the earnings-announcement and non-earnings-announcements months at the $5 \%$ level of significance, since the calculated test statistic (10.4435) exceeds the critical value $(2.0484)$

\section{Test of Hypotheses: Level of Significance and $p$-Values}
The choice of significance level in hypothesis testing is always a matter of judgment. Analysts often choose the 0.05 level of significance, which indicates a $5 \%$ chance of rejecting the null hypothesis when, in fact, it is true (a Type I error, or false positive). Of course, decreasing the level of significance from 0.05 to 0.01 decreases the probability of Type I error, but it also increases the probability of Type II error-failing to reject the null hypothesis when, in fact, it is false (that is, a false negative).

The $p$-value is the smallest level of significance at which the null hypothesis can be rejected. The smaller the $p$-value, the smaller the chance of making a Type I error (i.e., rejecting a true null hypothesis), so the greater the likelihood the regression model is valid. For example, if the $p$-value is 0.005 , we reject the null hypothesis that the true parameter is equal to zero at the $0.5 \%$ significance level ( $99.5 \%$ confidence). In most software packages, the $p$-values provided for regression coefficients are for a test of null hypothesis that the true parameter is equal to zero against the alternative that the parameter is not equal to zero.

In our ROA regression example, the calculated $t$-statistic for the test of whether the slope coefficient is zero is 4.00131 . The $p$-value corresponding to this test statistic is 0.008 , which means there is just a $0.8 \%$ chance of rejecting the null hypotheses when it is true. Comparing this $p$-value with the level of significance of $5 \%$ (and critical values of \textbackslash pm 2.776 ) leads us to easily reject the null hypothesis of $H_{0}: b_{1}=0$.

How do we determine the $p$-values? Since this is the area in the distribution outside the calculated test statistic, we need to resort to software tools. For the $p$-value corresponding to the $t=4.00131$ from the ROA regression example, we could use the following:

\begin{itemize}
  \item Excel 1-T.DIST(4.00131,4,TRUE) $)^{2} 2$

  \item $\mathbf{R}(1-\mathrm{pt}(4.00131,4))^{*} 2$

  \item Python from scipy.stats import t and (1 - t.cdf(4.00131,4))*2

\end{itemize}

\section{EXAMPLE 6}
\section{Hypothesis Testing of Simple Linear Regression Results}
An analyst is interested in interpreting the results of and performing tests of hypotheses for the market model estimation that regresses the daily return on ABC stock on the daily return on the fictitious Europe-Asia-Africa (EAA) Equity Index, his proxy for the stock market. He has generated the regression results presented in Exhibit 30.

Exhibit 30: Selected Results of Estimation of Market Model for ABC Stock

\begin{center}
\begin{tabular}{lc}
\hline
Standard error of the estimate $\left(s_{e}\right)$ & 1.26 \\
Standard deviation of ABC stock returns & 0.80 \\
Standard deviation of EAA Equity Index returns & 0.70 \\
Number of observations & 1,200 \\
Intercept & Coefficients \\
Slope of EAA Equity Index returns & 0.010 \\
\hline
\end{tabular}
\end{center}

\begin{enumerate}
  \item If the critical $t$-values are \textbackslash pm 1.96 (at the $5 \%$ significance level), is the slope coefficient different from zero?
\end{enumerate}

\section{Solution to 1}
First, we calculate the variation of the independent variable using the standard deviation of the independent variable:

$$
\sum_{i=1}^{n}\left(X_{i}-\bar{X}\right)^{2}=\frac{\sum_{i=1}^{n}\left(X_{i}-\bar{X}\right)^{2}}{n-1} \times(n-1)
$$

So,

$$
\sum_{i=1}^{n}\left(X_{i}-\bar{X}\right)^{2}=0.70^{2} \times 1,199=587.51 .
$$

Next, the standard error of the estimated slope coefficient is

$$
s_{\hat{b}_{1}}=\frac{s_{e}}{\sqrt{\sum_{i=1}^{n}\left(X_{i}-\bar{X}\right)^{2}}}=\frac{1.26}{\sqrt{587.51}}=0.051983,
$$

and the test statistic is

$$
t=\frac{\hat{b}_{1}-B_{1}}{s_{\hat{b}_{1}}}=\frac{0.982-0}{0.051983}=18.89079
$$

The calculated test statistic is outside the bounds of \textbackslash pm 1.96 , so we reject the null hypothesis of a slope coefficient equal to zero.

\begin{enumerate}
  \setcounter{enumi}{1}
  \item If the critical $t$-values are \textbackslash pm 1.96 (at the $5 \%$ significance level), is the slope coefficient different from 1.0?
\end{enumerate}

\section{Solution to 2}
The calculated test statistic for the test of whether the slope coefficient is equal to 1.0 is

$$
t=\frac{0.982-1}{0.051983}=-0.3463 \text {. }
$$

The calculated test statistic is within the bounds of \textbackslash pm 1.96 , so we fail to reject the null hypothesis of a slope coefficient equal to 1.0 , which is evidence that the true population slope may be 1.0.

\section{PREDICTION USING SIMPLE LINEAR REGRESSION AND PREDICTION INTERVALS}
calculate and interpret the predicted value for the dependent variable, and a prediction interval for it, given an estimated linear regression model and a value for the independent variable

Financial analysts often want to use regression results to make predictions about a dependent variable. For example, we might ask, "How fast will the sales of XYZ Corporation grow this year if real GDP grows by $4 \%$ ?" But we are not merely interested in making these forecasts; we also want to know how certain we can be about the forecasts' results. A forecasted value of the dependent variable, $\widehat{Y}_{f}$ is determined using the estimated intercept and slope, as well as the expected or forecasted independent variable, $X_{f}$ :

$$
\widehat{Y}_{f}=\hat{b}_{0}+\hat{b}_{1} X_{f}
$$

In our ROA regression model, if we forecast a company's CAPEX to be $6 \%$, the forecasted ROA based on our estimated equation is $12.375 \%$ :

$$
\widehat{Y}_{f}=4.875+(1.25 \times 6)=12.375
$$

However, we need to consider that the estimated regression line does not describe the relation between the dependent and independent variables perfectly; it is an average of the relation between the two variables. This is evident because the residuals are not all zero.

Therefore, an interval estimate of the forecast is needed to reflect this uncertainty. The estimated variance of the prediction error, $s_{f}^{2}$, of $Y$, given $X$, is

$$
s_{f}^{2}=s_{e}^{2}\left[1+\frac{1}{n}+\frac{\left(X_{f}-\bar{X}\right)^{2}}{(n-1) s_{X}^{2}}\right]=s_{e}^{2}\left[1+\frac{1}{n}+\frac{\left(X_{f}-\bar{X}\right)^{2}}{\sum_{i=1}^{n}\left(X_{i}-\bar{X}\right)^{2}}\right],
$$

and the standard error of the forecast is

$s_{f}=s_{e} \sqrt{1+\frac{1}{n}+\frac{\left(X_{f}-\bar{X}\right)^{2}}{\sum_{i=1}^{n}\left(X_{i}-\bar{X}\right)^{2}}}$

The standard error of the forecast depends on

\begin{itemize}
  \item the standard error of the estimate, $s_{e}$

  \item the number of observations, $n$;

  \item the forecasted value of the independent variable, $X_{f}$ used to predict the dependent variable and its deviation from the estimated mean, $\bar{X}$; and

  \item the variation of the independent variable.

\end{itemize}

We can see the following from the equation for the standard error of the forecast:

\begin{enumerate}
  \item The better the fit of the regression model, the smaller the standard error of the estimate $\left(s_{e}\right)$ and, therefore, the smaller standard error of the forecast.

  \item The larger the sample size $(n)$ in the regression estimation, the smaller the standard error of the forecast.

  \item The closer the forecasted independent variable $\left(X_{f}\right)$ is to the mean of the independent variable $(\bar{X})$ used in the regression estimation, the smaller the standard error of the forecast.

\end{enumerate}

Once we have this estimate of the standard error of the forecast, determining a prediction interval around the predicted value of the dependent variable $\left(\widehat{Y}_{f}\right)$ is very similar to estimating a confidence interval around an estimated parameter. The prediction interval is

$$
\widehat{Y}_{f} \pm t_{\text {critical for } \alpha / 2} s_{f}
$$

We outline the steps for developing the prediction interval in Exhibit 31. Exhibit 31: Creating a Prediction Interval around the Predicted Dependent

Variable

\begin{center}
\includegraphics[max width=\textwidth]{2023_05_04_cff39ee44f77d6514e1bg-472}
\end{center}

For our ROA regression model, given that the forecasted value of CAPEX is 6.0, the predicted value of $Y$ is 12.375 :

$$
\hat{Y}_{f}=4.875+1.25 X_{f}=4.875+(1.25 \times 6.0)=12.375 \text {. }
$$

Assuming a 5\% significance level ( $\alpha$ ), two sided, with $n-2$ degrees of freedom (so, df $=4$ ), the critical values for the prediction interval are \textbackslash pm 2.776 .

The standard error of the forecast is

$s_{f}=3.459588 \sqrt{1+\frac{1}{6}+\frac{(6-6.1)^{2}}{122.640}}=3.459588 \sqrt{1.166748}=3.736912$.

The $95 \%$ prediction interval then becomes

$12.375 \pm 2.776(3.736912)$

$12.375 \pm 10.3737$

$$
\left\{2.0013<\hat{Y}_{f}<22.7487\right\}
$$

For our ROA regression example, we can see how the standard error of the forecast $\left(s_{f}\right)$ changes as our forecasted value of the independent variable gets farther from the mean of the independent variable $\left(X_{f}-\bar{X}\right)$ in Exhibit 32. The mean of CAPEX is 6.1\%, and the band that represents one standard error of the forecast, above and below the forecast, is minimized at that point and increases as the independent variable gets farther from $\bar{X}$.

\section{Exhibit 32: ROA Forecasts and Standard Error of the Forecast}
\begin{center}
\includegraphics[max width=\textwidth]{2023_05_04_cff39ee44f77d6514e1bg-473}
\end{center}

\begin{itemize}
  \item 
  \begin{itemize}
    \item One Standard Error of the Forecast above the Forecast
  \end{itemize}
  \item One Standard Error of the Forecast below the Forecast

\end{itemize}

\section{EXAMPLE 7}
\section{Predicting Net Profit Margin Using R\&D Spending}
Suppose we want to forecast a company's net profit margin (NPM) based on its research and development expenditures scaled by revenues (RDR), using the model estimated in Example 2 and the details provided in Exhibit 8 . The regression model was estimated using data on eight companies as

$$
\widehat{Y}_{f}=16.5-1.3 X_{f}
$$

with a standard error of the estimate $\left(s_{e}\right)$ of 1.8618987 and variance of RDR, $\frac{\sum_{i=1}^{n}\left(X_{i}-\bar{X}\right)^{2}}{(n-1)}$, of 4.285714, as given .

\begin{enumerate}
  \item What is the predicted value of NPM if the forecasted value of RDR is 5 ?
\end{enumerate}

\section{Solution to 1}
The predicted value of NPM is $10: 16.5-(1.3 \times 5)=10$.

\begin{enumerate}
  \setcounter{enumi}{1}
  \item What is the standard error of the forecast $\left(s_{f}\right)$ if the forecasted value of RDR is $5 ?$
\end{enumerate}

\section{Solution to 2}
To derive the standard error of the forecast $\left(s_{f}\right)$, we first have to calculate the variation of RDR. Then, we have all the pieces to calculate $s_{f}$ :

$$
\begin{aligned}
& \sum_{i=1}^{n}\left(X_{i}-\bar{X}\right)^{2}=4.285714 \times 7=30 . \\
& s_{f}=1.8618987 \sqrt{1+\frac{1}{8}+\frac{(5-7.5)^{2}}{30}}=2.1499 .
\end{aligned}
$$

\begin{enumerate}
  \setcounter{enumi}{2}
  \item What is the $95 \%$ prediction interval for the predicted value of NPM using critical $t$-values $(\mathrm{df}=6)$ of \textbackslash pm 2.447 ?
\end{enumerate}

\section{Solution to 3}
The $95 \%$ prediction interval for the predicted value of NPM is

$$
\begin{aligned}
& \{10 \pm 2.447(2.1499)\} \\
& \left\{4.7392<\hat{Y}_{f}<15.2608\right\}
\end{aligned}
$$

\begin{enumerate}
  \setcounter{enumi}{3}
  \item What is the predicted value of NPM if the forecasted value of RDR is 15 ?
\end{enumerate}

\section{Solution to 4}
The predicted value of NPM is $-3: 16.5-(1.3 \times 15)=-3$.

\begin{enumerate}
  \setcounter{enumi}{4}
  \item What is the standard error of the forecast if the forecasted value of RDR is $15 ?$
\end{enumerate}

\section{Solution to 5}
To derive the standard error of the forecast, we first must calculate the variation of RDR. Then, we can calculate $s_{f}$

$$
\begin{aligned}
& \sum_{i=1}^{n}\left(X_{i}-\bar{X}\right)^{2}=4.285714 \times 7=30 . \\
& s_{f}=1.8618987 \sqrt{1+\frac{1}{8}+\frac{(15-7.5)^{2}}{30}}=3.2249 .
\end{aligned}
$$

\begin{enumerate}
  \setcounter{enumi}{5}
  \item What is the $95 \%$ prediction interval for the predicted value of NPM using critical $t$-values $(\mathrm{df}=6)$ of \textbackslash pm 2.447 ?
\end{enumerate}

\section{Solution to 6}
The $95 \%$ prediction interval for the predicted value of NPM is

$$
\begin{aligned}
& \{-3 \pm 2.447(3.2249)\} \\
& \left\{-10.8913<\hat{Y}_{f}<4.8913\right\}
\end{aligned}
$$

\section{FUNCTIONAL FORMS FOR SIMPLE LINEAR REGRESSION}
Not every set of independent and dependent variables has a linear relation. In fact, we often see non-linear relationships in economic and financial data. Consider the revenues of a company over time illustrated in Exhibit 33, with revenues as the dependent $(Y)$ variable and time as the independent $(X)$ variable. Revenues grow at a rate of $15 \%$ per year for several years, but then the growth rate eventually declines to just $5 \%$ per year. Estimating this relationship as a simple linear model would understate the dependent variable, revenues, for some ranges of the independent variable, time, and would overstate it for other ranges of the independent variable.

\section{Exhibit 33: Company Revenues over Time}
\begin{center}
\includegraphics[max width=\textwidth]{2023_05_04_cff39ee44f77d6514e1bg-475}
\end{center}

We can still use the simple linear regression model, but we need to modify either the dependent or the independent variables to make it work well. This is the case with many different financial or economic data that you might use as dependent and independent variables in your regression analysis.

There are several different functional forms that can be used to potentially transform the data to enable their use in linear regression. These transformations include using the $\log$ (i.e., natural logarithm) of the dependent variable, the log of the independent variable, the reciprocal of the independent variable, the square of the independent variable, or the differencing of the independent variable. We illustrate and discuss three often-used functional forms, each of which involves log transformation:

\begin{enumerate}
  \item the log-lin model, in which the dependent variable is logarithmic but the independent variable is linear;

  \item the lin-log model, in which the dependent variable is linear but the independent variable is logarithmic; and

  \item the log-log model, where both the dependent and independent variables are in logarithmic form.

\end{enumerate}

\section{The Log-Lin Model}
In the log-lin model, the dependent variable is in logarithmic form and the independent variable is not, as follows:

$$
\ln Y_{i}=b_{0}+b_{1} X_{i}
$$

The slope coefficient in this model is the relative change in the dependent variable for an absolute change in the independent variable. We can transform the $Y$ variable (revenues) in Exhibit 33 into its natural $\log (\ln )$ and then fit the regression line, as we show in Exhibit 34. From this chart, we see that the log-lin model is a better fitting model than the simple linear regression model.

\section{Exhibit 34: Log-Lin Model Applied to Company Revenues over Time}
\begin{center}
\includegraphics[max width=\textwidth]{2023_05_04_cff39ee44f77d6514e1bg-476}
\end{center}

It is important to note that in working with a log-lin model, you must take care when making a forecast. For example, suppose the estimated regression model is $\ln Y=-7$ $+2 X$. If $X$ is $2.5 \%$, then the forecasted value of $\ln Y$ is -2 . In this case, the predicted value of $Y$ is the antilog of -2 , or $e^{-2}=0.135335$. Another caution is that you cannot directly compare a log-lin model with a lin-lin model (that is, the regression of $Y$ on $X$ without any transformation) because the dependent variables are not in the same form - we would have to transform the $R^{2}$ and $F$-statistic to enable a comparison. However, looking at the residuals is helpful.

\section{The Lin-Log Model}
The lin-log model is similar to the log-lin model, but only the independent variable is in logarithmic form:

$$
Y_{i}=b_{0}+b_{1} \ln X_{i}
$$

The slope coefficient in this regression model provides the absolute change in the dependent variable for a relative change in the independent variable.

Suppose an analyst is examining the cross-sectional relationship between operating profit margin, the dependent variable $(Y)$, and unit sales, the independent variable $(X)$, and gathers data on a sample of 30 companies. The scatter plot and regression line for these observations are shown in Exhibit 35. Although the slope is different from zero at the $5 \%$ level (the calculated $t$-statistic on the slope is 5.8616 , compared with critical $t$-values of \textbackslash pm 2.048 ), given the $R^{2}$ of $55.10 \%$, the issue is whether we can get a better fit by using a different functional form.

\section{Exhibit 35: Relationship between Operating Profit Margin and Unit Sales}
\begin{center}
\includegraphics[max width=\textwidth]{2023_05_04_cff39ee44f77d6514e1bg-477}
\end{center}

If instead we use the natural $\log$ of the unit sales as the independent variable in our model, we get a very different picture, as shown in Exhibit 36 . The $R^{2}$ for the model of operating profit margin regressed on the natural $\log$ of unit sales jumps to $97.17 \%$. Since the dependent variable is the same in both the original and transformed models, we can compare the standard error of the estimate: 2.2528 with the original independent variable and a much lower 0.5629 with the transformed independent variable. Clearly the log-transformed explanatory variable has resulted in a better fitting model.

\section{Exhibit 36: Relationship Between Operating Profit Margin and Natural}
 Logarithm of Unit Sales\begin{center}
\includegraphics[max width=\textwidth]{2023_05_04_cff39ee44f77d6514e1bg-477(1)}
\end{center}

Operating Profit Margin ........ Lin-Log Regression Line

\section{The Log-Log Model}
The log-log model, in which both the dependent variable and the independent variable are linear in their logarithmic forms, is also referred to as the double-log model.

$$
\ln Y_{i}=b_{0}+b_{1} \ln X_{i}
$$

This model is useful in calculating elasticities because the slope coefficient is the relative change in the dependent variable for a relative change in the independent variable. Consider a cross-sectional model of company revenues (the $Y$ variable) regressed on advertising spending as a percentage of selling, general, and administrative expenses, ADVERT (the $X$ variable). As shown in Exhibit 37, a simple linear regression model results in a shallow regression line, with a coefficient of determination of just $20.89 \%$.

\section{Exhibit 37: Fitting a Linear Relation Between Revenues and Advertising}
Spending

\begin{center}
\includegraphics[max width=\textwidth]{2023_05_04_cff39ee44f77d6514e1bg-478}
\end{center}

However, if instead we use the natural logarithms of both the revenues and ADVERT, we get a much different picture of this relationship. As shown in Exhibit 38, the estimated regression line has a significant positive slope; the log-log model's $R^{2}$ increases by more than four times, from $20.89 \%$ to $84.91 \%$; and the $F$-statistic jumps from 7.39 to 157.52. So, using the log-log transformation dramatically improves the regression model fit relative to our data.

\section{Exhibit 38: Fitting a Log-Log Model of Revenues and Advertising Spending}
\begin{center}
\includegraphics[max width=\textwidth]{2023_05_04_cff39ee44f77d6514e1bg-479}
\end{center}

Natural Log of Revenues $\ldots . . .$. Log-Log Regression Line

\section{Selecting the Correct Functional Form}
The key to fitting the appropriate functional form of a simple linear regression is examining the goodness of fit measures-the coefficient of determination $\left(R^{2}\right)$, the $F$-statistic, and the standard error of the estimate $\left(s_{e}\right)$-as well as examining whether there are patterns in the residuals. In addition to fit statistics, most statistical packages provide plots of residuals as part of the regression output, which enables you to visually inspect the residuals. To reiterate an important point, what you want to see in these plots is random residuals.

As an example, consider the relationship between the monthly returns on DEF stock and the monthly returns of the EAA Equity Index, as depicted in Panel A of Exhibit 39, with the regression line indicated. Using the equation for this regression line, we calculate the residuals and plot them against the EAA Equity Index, as shown in Panel B of Exhibit 39. The residuals appear to be random, bearing no relation to the independent variable. The distribution of the residuals, shown in Panel $\mathrm{C}$ of Exhibit 39 , shows that the residuals are approximately normal. Using statistical software, we can investigate further by examining the distribution of the residuals, including using a normal probability plot or statistics to test for normality of the residuals. Exhibit 39: Monthly Returns on DEF Stock Regressed on Returns on the EAA Index

\section{A. Scatterplot of Returns on DEF Stock and Return on the EAA Index}
Return on DEF Stock (\%)

\begin{center}
\includegraphics[max width=\textwidth]{2023_05_04_cff39ee44f77d6514e1bg-480(1)}
\end{center}

\section{B. Scatterplot of Residuals and the Returns on the EAA Index}
Residual Return on DEF Stock (\%)

\begin{center}
\includegraphics[max width=\textwidth]{2023_05_04_cff39ee44f77d6514e1bg-480(2)}
\end{center}

\section{Histogram of Residuals}
Number of Observations

\begin{center}
\includegraphics[max width=\textwidth]{2023_05_04_cff39ee44f77d6514e1bg-480}
\end{center}

\begin{center}
\includegraphics[max width=\textwidth]{2023_05_04_cff39ee44f77d6514e1bg-480(3)}
\end{center}

Residual Range (\%)

\section{EXAMPLE 8}
\section{Comparing Functional Forms}
An analyst is investigating the relationship between the annual growth in consumer spending (CONS) in a country and the annual growth in the country's GDP (GGDP). The analyst estimates the following two models:

\begin{center}
\begin{tabular}{|c|c|c|}
\hline
 & Model 1 & Model 2 \\
\hline
 & $\begin{array}{l}\mathrm{GGDP}_{i}=b_{0}+b_{1} \mathrm{CONS}_{i} \\ +\varepsilon_{i}\end{array}$ & $\mathrm{GGDP}_{i}=b_{0}+b_{1} \ln \left(\mathrm{CONS}_{i}\right)+\varepsilon_{i}$ \\
\hline
Intercept & 1.040 & 1.006 \\
\hline
Slope & 0.669 & 1.994 \\
\hline
$R^{2}$ & 0.788 & 0.867 \\
\hline
$\begin{array}{l}\text { Standard error of } \\ \text { the estimate }\end{array}$ & 0.404 & 0.320 \\
\hline
F-statistic & 141.558 & 247.040 \\
\hline
\end{tabular}
\end{center}

\begin{enumerate}
  \item Identify the functional form used in these models.
\end{enumerate}

\section{Solution to 1}
Model 1 is the simple linear regression with no variable transformation, whereas Model 2 is a lin-log model with the natural log of the variable CONS as the independent variable.

\begin{enumerate}
  \setcounter{enumi}{1}
  \item Explain which model has better goodness-of-fit with the sample data.
\end{enumerate}

\section{Solution to 2}
The lin-log model, Model 2, fits the data better. Since the dependent variable is the same for the two models, we can compare the fit of the models using either the relative measures ( $R^{2}$ or $F$-statistic) or the absolute measure of fit, the standard error of the estimate. The standard error of the estimate is lower for Model 2, whereas the $R^{2}$ and $F$-statistic are higher for Model 2 compared with Model 1.

\section{SUMMARY}
\begin{itemize}
  \item The dependent variable in a linear regression is the variable whose variability the regression model tries to explain. The independent variable is the variable whose variation the researcher uses to explain the variation of the dependent variable.

  \item If there is one independent variable in a linear regression and there are $n$ observations of the dependent and independent variables, the regression model is $Y_{i}=b_{0}+b_{1} X_{i}+\varepsilon_{i}, i=1, \ldots, n$, where $Y_{i}$ is the dependent variable, $X_{i}$ is the independent variable, and $\varepsilon_{i}$ is the error term. In this model, the coefficients $b_{0}$ and $b_{1}$ are the population intercept and slope, respectively. - The intercept is the expected value of the dependent variable when the independent variable has a value of zero. The slope coefficient is the estimate of the population slope of the regression line and is the expected change in the dependent variable for a one-unit change in the independent variable.

  \item The assumptions of the classic simple linear regression model are as follows:

  \item Linearity: A linear relation exists between the dependent variable and the independent variable.

  \item Homoskedasticity: The variance of the error term is the same for all observations.

  \item Independence: The error term is uncorrelated across observations.

  \item Normality: The error term is normally distributed.

  \item The estimated parameters in a simple linear regression model minimize the sum of the squared errors.

  \item The coefficient of determination, or $R^{2}$, measures the percentage of the total variation in the dependent variable explained by the independent variable.

  \item To test the fit of the simple linear regression, we can calculate an $F$-distributed test statistic and test the hypotheses $H_{0}: b_{1}=0$ versus $H_{a}: b_{1} \neq$ 0 , with 1 and $n-2$ degrees of freedom.

  \item The standard error of the estimate is an absolute measure of the fit of the model calculated as the square root of the mean square error.

  \item We can evaluate a regression model by testing whether the population value of a regression coefficient is equal to a particular hypothesized value. We do this by calculating a $t$-distributed test statistic that compares the estimated parameter with the hypothesized parameter, dividing this difference by the standard error of the coefficient.

  \item An indicator (or dummy) variable takes on only the values 0 or 1 and can be used as the independent variable in a simple linear regression. In such a model, the interpretation of the intercept is the predicted value of the dependent variable if the indicator variable is 0 , and when the indicator variable is 1 , the slope is the difference in the means if we grouped the observations by the indicator variable.

  \item We calculate a prediction interval for a regression coefficient using the estimated coefficient, the standard error of the estimated coefficient, and the critical value for the $t$-distributed test statistic based on the level of significance and the appropriate degrees of freedom, which are $n-2$ for simple regression.

  \item We can make predictions for the dependent variable using an estimated linear regression by inserting the forecasted value of the independent variable into the estimated model.

  \item The standard error of the forecast is the product of the standard error of the estimate and a term that reflects the sample size of the regression, the variation of the independent variable, and the deviation between the forecasted value of the independent variable and the mean of the independent variable in the regression.

  \item The prediction interval for a particular forecasted value of the dependent variable is formed by using the forecasted value of the dependent variable and extending above and below this value a quantity that reflects the critical $t$-value corresponding to the degrees of freedom, the level of significance, and the standard error of the forecast. - If the relationship between the independent variable and the dependent variable is not linear, we can often transform one or both of these variables to convert this relation to a linear form, which then allows the use of simple linear regression.

\end{itemize}

\section{PRACTICE PROBLEMS}
\begin{enumerate}
  \item Homoskedasticity is best described as the situation in which the variance of the residuals of a regression is:
A. zero.
B. normally distributed.
C. constant across observations.

  \item Julie Moon is an energy analyst examining electricity, oil, and natural gas consumption in different regions over different seasons. She ran a simple regression explaining the variation in energy consumption as a function of temperature. The total variation of the dependent variable was 140.58 , and the explained variation was 60.16 . She had 60 monthly observations.

\end{enumerate}

A. Calculate the coefficient of determination.

B. Calculate the $F$-statistic to test the fit of the model.

C. Calculate the standard error of the estimate of the regression estimation.

D. Calculate the sample standard deviation of monthly energy consumption.

\begin{enumerate}
  \setcounter{enumi}{2}
  \item An economist collected the monthly returns for KDL's portfolio and a diversified stock index. The data collected are shown in the following table:
\end{enumerate}

\begin{center}
\begin{tabular}{lcc}
\hline
Month & Portfolio Return (\%) & Index Return (\%) \\
\hline
1 & 1.11 & -0.59 \\
2 & 72.10 & 64.90 \\
3 & 5.12 & 4.81 \\
4 & 1.01 & 1.68 \\
5 & -1.72 & -4.97 \\
6 & 4.06 & -2.06 \\
\hline
\end{tabular}
\end{center}

The economist calculated the correlation between the two returns and found it to be 0.996 . The regression results with the KDL return as the dependent variable and the index return as the independent variable are given as follows:

\section{Regression Statistics}
\begin{center}
\begin{tabular}{|c|c|c|c|c|c|}
\hline
$R^{2}$ &  & 0.9921 &  &  &  \\
\hline
Standard error &  & 2.8619 &  &  &  \\
\hline
Observations &  & 6 &  &  &  \\
\hline
Source & df & $\begin{array}{l}\text { Sum of } \\ \text { Squares }\end{array}$ & $\begin{array}{c}\text { Mean } \\ \text { Square }\end{array}$ & $\boldsymbol{F}$ & p-Value \\
\hline
Regression & 1 & $4,101.6205$ & $4,101.6205$ & 500.7921 & 0.0000 \\
\hline
Residual & 4 & 32.7611 & 8.1903 &  &  \\
\hline
Total & 5 & $4,134.3815$ &  &  &  \\
\hline
\end{tabular}
\end{center}

\begin{center}
\begin{tabular}{lcccc}
\hline
 & Coefficients & Standard Error & t-Statistic & $\boldsymbol{p}$-Value \\
\hline
Intercept & 2.2521 & 1.2739 & 1.7679 & 0.1518 \\
Index return $(\%)$ & 1.0690 & 0.0478 & 22.3784 & 0.0000 \\
\hline
\end{tabular}
\end{center}

When reviewing the results, Andrea Fusilier suspected that they were unreliable. She found that the returns for Month 2 should have been $7.21 \%$ and 6.49\%, instead of the large values shown in the first table. Correcting these values resulted in a revised correlation of 0.824 and the following revised regression results:

\section{Regression Statistics}
$R^{2}$

Standard error 2.0624

Observations 6

\begin{center}
\begin{tabular}{|c|c|c|c|c|c|}
\hline
Source & df & $\begin{array}{l}\text { Sum of } \\ \text { Squares }\end{array}$ & $\begin{array}{c}\text { Mean } \\ \text { Square }\end{array}$ & $\boldsymbol{F}$ & p-Value \\
\hline
Regression & 1 & 35.8950 & 35.8950 & 8.4391 & 0.044 \\
\hline
Residual & 4 & 17.0137 & 4.2534 &  &  \\
\hline
\multirow[t]{2}{*}{Total} & 5 & 52.91 &  &  &  \\
\hline
 & \multicolumn{2}{|c|}{Coefficients} & Standard Error & t-Statistic & $p$-Value \\
\hline
Intercept & \multicolumn{2}{|c|}{2.2421} & 0.8635 & 2.5966 & 0.060 \\
\hline
Slope & \multicolumn{2}{|c|}{0.6217} & 0.2143 & 2.9050 & 0.044 \\
\hline
\end{tabular}
\end{center}

Explain how the bad data affected the results.

\section{The following information relates to questions}
4-7

An analyst is examining the annual growth of the money supply for a country over the past 30 years. This country experienced a central bank policy shift 15 years ago, which altered the approach to the management of the money supply. The analyst estimated a model using the annual growth rate in the money supply regressed on the variable (SHIFT) that takes on a value of 0 before the policy shift and 1 after. She estimated the following:

\begin{center}
\begin{tabular}{lccc}
\hline
 & Coefficients & Standard Error & $\boldsymbol{t}$-Stat. \\
\hline
Intercept & 5.767264 & 0.445229 & 12.95348 \\
SHIFT & -5.13912 & 0.629649 & -8.16188 \\
\hline
\end{tabular}
\end{center}

Critical $t$-values, level of significance of 0.05 :

One-sided, left side: -1.701

One-sided, right side: +1.701

Two-sided: \textbackslash pm 2.048 4. The variable SHIFT is best described as:
A. an indicator variable.
B. a dependent variable.
C. a continuous variable.

\begin{enumerate}
  \setcounter{enumi}{4}
  \item The interpretation of the intercept is the mean of the annual growth rate of the money supply:
A. over the enter entire period.
B. after the shift in policy.
C. before the shift in policy.

  \item The interpretation of the slope is the:

\end{enumerate}

A. change in the annual growth rate of the money supply per year.

B. average annual growth rate of the money supply after the shift in policy.

C. difference in the average annual growth rate of the money supply from before to after the shift in policy.

\begin{enumerate}
  \setcounter{enumi}{6}
  \item Testing whether there is a change in the money supply growth after the shift in policy, using a 0.05 level of significance, we conclude that there is:
\end{enumerate}

A. sufficient evidence that the money supply growth changed.

B. not enough evidence that the money supply growth is different from zero.

C. not enough evidence to indicate that the money supply growth changed.

\begin{enumerate}
  \setcounter{enumi}{7}
  \item You are examining the results of a regression estimation that attempts to explain the unit sales growth of a business you are researching. The analysis of variance output for the regression is given in the following table. The regression was based on five observations $(n=5)$.
\end{enumerate}

\begin{center}
\begin{tabular}{lccccc}
\hline
Source & $\mathbf{d f}$ & $\begin{array}{c}\text { Sum of } \\ \text { Squares }\end{array}$ & $\begin{array}{c}\text { Mean } \\ \text { Square }\end{array}$ & $\boldsymbol{F}$ & $\boldsymbol{p}$-Value \\
\hline
Regression & 1 & 88.0 & 88.0 & 36.667 & 0.00904 \\
Residual & 3 & 7.2 & 2.4 &  &  \\
Total & 4 & 95.2 &  &  &  \\
\hline
\end{tabular}
\end{center}

A. Calculate the sample variance of the dependent variable using information in the table.

B. Calculate the coefficient of determination for this estimated model.

C. What hypothesis does the $F$-statistic test?

D. Is the $F$-test significant at the 0.05 significance level?

E. Calculate the standard error of the estimate.

\section{The following information relates to questions}
\section{9-12}
Kenneth McCoin, CFA, is a challenging interviewer. Last year, he handed each job applicant a sheet of paper with the information in the following table, and he then asked several questions about regression analysis. Some of McCoin's questions, along with a sample of the answers he received to each, are given below. McCoin told the applicants that the independent variable is the ratio of net income to sales for restaurants with a market cap of more than $\$ 100$ million and the dependent variable is the ratio of cash flow from operations to sales for those restaurants. Which of the choices provided is the best answer to each of McCoin's questions?

Regression Statistics

\begin{center}
\includegraphics[max width=\textwidth]{2023_05_04_cff39ee44f77d6514e1bg-487}
\end{center}

\begin{enumerate}
  \setcounter{enumi}{8}
  \item The coefficient of determination is closest to:
A. 0.7436 .
B. 0.8261 .
C. 0.8623 .

  \item The correlation between $X$ and $Y$ is closest to:
A. -0.7436 .
B. 0.7436 .
C. 0.8623 .

  \item If the ratio of net income to sales for a restaurant is $5 \%$, the predicted ratio of cash flow from operations (CFO) to sales is closest to:
A. -4.054
B. 0.524 .
C. 4.207 . 12. Is the relationship between the ratio of cash flow to operations and the ratio of net income to sales significant at the 0.05 level?

\end{enumerate}

A. No, because the $R^{2}$ is greater than 0.05

B. No, because the $p$-values of the intercept and slope are less than 0.05

C. Yes, because the $p$-values for $F$ and $t$ for the slope coefficient are less than 0.05

\section{The following information relates to questions}
\section{3-17}
Howard Golub, CFA, is preparing to write a research report on Stellar Energy Corp. common stock. One of the world's largest companies, Stellar is in the business of refining and marketing oil. As part of his analysis, Golub wants to evaluate the sensitivity of the stock's returns to various economic factors. For example, a client recently asked Golub whether the price of Stellar Energy Corp. stock has tended to rise following increases in retail energy prices. Golub believes the association between the two variables is negative, but he does not know the strength of the association.

Golub directs his assistant, Jill Batten, to study the relationships between (1) Stellar monthly common stock returns and the previous month's percentage change in the US Consumer Price Index for Energy (CPIENG) and (2) Stellar monthly common stock returns and the previous month's percentage change in the US Producer Price Index for Crude Energy Materials (PPICEM). Golub wants Batten to run both a correlation and a linear regression analysis. In response, Batten compiles the summary statistics shown in Exhibit 1 for 248 months. All the data are in decimal form, where 0.01 indicates a $1 \%$ return. Batten also runs a regression analysis using Stellar monthly returns as the dependent variable and the monthly change in CPIENG as the independent variable. Exhibit 2 displays the results of this regression model.

Exhibit 1: Descriptive Statistics

\begin{center}
\begin{tabular}{|c|c|c|c|}
\hline
 & \multirow{2}{*}{$\begin{array}{c}\text { Stellar Common } \\
\text { Stock Monthly Return }\end{array}$} & \multicolumn{2}{|c|}{Lagged Monthly Change} \\
\hline
 &  & CPIENG & PPICEM \\
\hline
Mean & 0.0123 & 0.0023 & 0.0042 \\
\hline
Standard deviation & 0.0717 & 0.0160 & 0.0534 \\
\hline
Covariance, Stellar vs. CPIENG & -0.00017 &  &  \\
\hline
Covariance, Stellar vs. PPICEM & -0.00048 &  &  \\
\hline
Covariance, CPIENG vs. PPICEM & 0.00044 &  &  \\
\hline
\end{tabular}
\end{center}

\begin{center}
\begin{tabular}{ll}
Stellar Common & Lagged Monthly Change \\
\cline { 2 - 2 }
Stock Monthly Return & CPIENG PPICEM \\
\cline { 2 - 2 }
\end{tabular}
\end{center}

\begin{center}
\begin{tabular}{ll}
\hline
Correlation, Stellar vs. CPIENG & -0.1452 \\
\hline
\end{tabular}
\end{center}

Exhibit 2: Regression Analysis with CPIENG

Regression Statistics

\begin{center}
\begin{tabular}{lccc}
\hline
$R^{2}$ &  &  &  \\
Standard error of the estimate & 0.0211 &  &  \\
Observations & 0.0710 &  &  \\
 &  & 248 &  \\
 & Coefficients & Standard Error & $\boldsymbol{t}$-Statistic \\
\hline
Intercept & 0.0138 & 0.0046 & 3.0275 \\
CPIENG (\%) & -0.6486 & 0.2818 & -2.3014 \\
\hline
\end{tabular}
\end{center}

Critical $t$-values

One-sided, left side: -1.651

One-sided, right side: +1.651

Two-sided: \textbackslash pm 1.967

\begin{enumerate}
  \setcounter{enumi}{12}
  \item Which of the following best describes Batten's regression?
A. Time-series regression
B. Cross-sectional regression
C. Time-series and cross-sectional regression

  \item Based on the regression, if the CPIENG decreases by $1.0 \%$, the expected return on Stellar common stock during the next period is closest to:
A. $0.0073(0.73 \%)$.
B. $0.0138(1.38 \%)$.
C. $0.0203(2.03 \%)$

  \item Based on Batten's regression model, the coefficient of determination indicates that:

\end{enumerate}

A. Stellar's returns explain $2.11 \%$ of the variability in CPIENG.

B. Stellar's returns explain $14.52 \%$ of the variability in CPIENG.

C. changes in CPIENG explain 2.11\% of the variability in Stellar's returns.

\begin{enumerate}
  \setcounter{enumi}{15}
  \item For Batten's regression model, 0.0710 is the standard deviation of:
A. the dependent variable.
B. the residuals from the regression.
C. the predicted dependent variable from the regression. 17. For the analysis run by Batten, which of the following is an incorrect conclusion from the regression output?
\end{enumerate}

A. The estimated intercept from Batten's regression is statistically different from zero at the 0.05 level of significance.

B. In the month after the CPIENG declines, Stellar's common stock is expected to exhibit a positive return.

C. Viewed in combination, the slope and intercept coefficients from Batten's regression are not statistically different from zero at the 0.05 level of significance.

\section{The following information relates to questions}
 $18-26$Anh Liu is an analyst researching whether a company's debt burden affects investors' decision to short the company's stock. She calculates the short interest ratio (the ratio of short interest to average daily share volume, expressed in days) for 50 companies as of the end of 2016 and compares this ratio with the companies' debt ratio (the ratio of total liabilities to total assets, expressed in decimal form). Liu provides a number of statistics in Exhibit 1. She also estimates a simple regression to investigate the effect of the debt ratio on a company's short interest ratio. The results of this simple regression, including the analysis of variance (ANOVA), are shown in Exhibit 2.

In addition to estimating a regression equation, Liu graphs the 50 observations using a scatter plot, with the short interest ratio on the vertical axis and the debt ratio on the horizontal axis.

Exhibit 1: Summary Statistics

\begin{center}
\begin{tabular}{|c|c|c|}
\hline
Statistic & $\begin{array}{c}\text { Debt Ratio } \\ \quad x_{i}\end{array}$ & $\begin{array}{l}\text { Short Interest Ratio } \\ \qquad Y_{i}\end{array}$ \\
\hline
Sum & 19.8550 & 192.3000 \\
\hline
$\begin{array}{l}\text { Sum of squared deviations } \\ \text { from the mean }\end{array}$ & $\sum_{i=1}^{n}\left(X_{i}-\bar{X}\right)^{2}=2.2225$ & $\sum_{i=1}^{n}\left(Y_{i}-\bar{Y}\right)^{2}=412.2042$ \\
\hline
$\begin{array}{l}\text { Sum of cross-products of devi- } \\ \text { ations from the mean }\end{array}$ & $\sum^{n}\left(X_{i}-\bar{X}\right)\left(Y_{i}-\bar{Y}\right)$ & 7) $=-9.2430$. \\
\hline
\end{tabular}
\end{center}

\section{Exhibit 2: Regression of the Short Interest Ratio on the Debt Ratio}
\begin{center}
\begin{tabular}{lccc}
\hline
ANOVA & $\begin{array}{c}\text { Degrees of } \\ \text { Freedom (df) }\end{array}$ & Sum of Squares & Mean Square \\
\hline
Regression & 1 & 38.4404 & 38.4404 \\
Residual & 48 & 373.7638 & 7.7867 \\
Total & 49 & 412.2042 &  \\
\end{tabular}
\end{center}

Regression Statistics

\begin{center}
\begin{tabular}{lccc}
\hline
ANOVA & $\begin{array}{c}\text { Degrees of } \\ \text { Freedom (df) }\end{array}$ & Sum of Squares & Mean Square \\
\hline
$R^{2}$ & 0.0933 &  &  \\
Standard error of & 2.7905 &  &  \\
estimate & 50 & Standard Error & 6.5322 \\
Observations & Coefficients & 0.8416 & -2.2219 \\
Intercept & 5.4975 & 1.8718 &  \\
Debt ratio (\%) & -4.1589 &  &  \\
\end{tabular}
\end{center}

Critical $t$-values for a 0.05 level of significance:

One-sided, left side: -1.677

One-sided, right side: +1.677

Two-sided: \textbackslash pm 2.011

Liu is considering three interpretations of these results for her report on the relationship between debt ratios and short interest ratios:

Interpretation 1 Companies' higher debt ratios cause lower short interest ratios.

Interpretation 2 Companies' higher short interest ratios cause higher debt ratios.

Interpretation 3 Companies with higher debt ratios tend to have lower short interest ratios.

She is especially interested in using her estimation results to predict the short interest ratio for MQD Corporation, which has a debt ratio of 0.40 .

\begin{enumerate}
  \setcounter{enumi}{17}
  \item Based on Exhibits 1 and 2, if Liu were to graph the 50 observations, the scatter plot summarizing this relation would be best described as:
A. horizontal.
B. upward sloping.
C. downward sloping.

  \item Based on Exhibit 1, the sample covariance is closest to:
A. -9.2430 .
B. -0.1886 .
C. 8.4123 .

  \item Based on Exhibits 1 and 2, the correlation between the debt ratio and the short interest ratio is closest to:
A. -0.3054 .
B. 0.0933 .
C. 0.3054 . 21. Which of the interpretations best describes Liu's findings?
A. Interpretation 1
B. Interpretation 2
C. Interpretation 3

  \item The dependent variable in Liu's regression analysis is the:
A. intercept.
B. debt ratio.
C. short interest ratio.

  \item Based on Exhibit 2, the degrees of freedom for the $t$-test of the slope coefficient in this regression are:
A. 48 .
B. 49 .
C. 50 .

  \item Which of the following should Liu conclude from the results shown in Exhibit 2?

\end{enumerate}

A. The average short interest ratio is 5.4975 .

B. The estimated slope coefficient is different from zero at the 0.05 level of significance.

C. The debt ratio explains $30.54 \%$ of the variation in the short interest ratio.

\begin{enumerate}
  \setcounter{enumi}{24}
  \item Based on Exhibit 2, the short interest ratio expected for MQD Corporation is closest to:
A. 3.8339 .
B. 5.4975 .
C. 6.2462 .

  \item Based on Liu's regression results in Exhibit 2, the $F$-statistic for testing whether the slope coefficient is equal to zero is closest to:
A. -2.2219 .
B. 3.5036 .
C. 4.9367 .

\end{enumerate}

The following information relates to questions 27-31

Elena Vasileva recently joined EnergyInvest as a junior portfolio analyst. Vasileva's supervisor asks her to evaluate a potential investment opportunity in Amtex, a multinational oil and gas corporation based in the United States. Vasileva's supervisor suggests using regression analysis to examine the relation between Amtex shares and returns on crude oil.

Vasileva notes the following assumptions of regression analysis:

Assumption 1 The error term is uncorrelated across observations.

Assumption 2 The variance of the error term is the same for all observations.

Assumption 3 The dependent variable is normally distributed.

Vasileva runs a regression of Amtex share returns on crude oil returns using the monthly data she collected. Selected data used in the regression are presented in Exhibit 1, and selected regression output is presented in Exhibit 2. She uses a 1\% level of significance in all her tests.

\section{Exhibit 1: Selected Data for Crude Oil Returns and Amtex Share Returns}
\begin{center}
\begin{tabular}{|c|c|c|c|c|c|c|}
\hline
 & $\begin{array}{l}\text { Oil Return } \\ \qquad\left(X_{i}\right)\end{array}$ & $\begin{array}{c}\text { Amtex Return } \\ \qquad\left(Y_{i}\right)\end{array}$ & $\begin{array}{l}\text { Cross-Product } \\ \left(X_{i}-\bar{X}\right) \quad\left(Y_{i}-\bar{Y}\right)\end{array}$ & $\begin{array}{c}\text { Predicted } \\ \text { Amtex Return } \\ \hat{Y}_{i}\end{array}$ & $\begin{array}{c}\text { Regression } \\ \text { Residual } \\ Y_{i}-\hat{Y}_{i}\end{array}$ & $\begin{array}{c}\text { Squared Residua } \\ \qquad\left(Y_{i}-\hat{Y}_{i}\right)^{2}\end{array}$ \\
\hline
Month 1 & -0.032000 & 0.033145 & -0.000388 & 0.002011 & -0.031134 & 0.000969 \\
\hline
$\otimes$ & $\otimes$ & $\otimes$ & $\otimes$ & $\otimes$ & $\otimes$ & $\otimes$ \\
\hline
Month 36 & 0.028636 & 0.062334 & 0.002663 & 0.016282 & -0.046053 & 0.002121 \\
\hline
Sum &  &  & 0.085598 &  &  & 0.071475 \\
\hline
Average & -0.018056 & 0.005293 &  &  &  &  \\
\hline
\end{tabular}
\end{center}

Exhibit 2: Selected Regression Output, Dependent

Variable: Amtex Share Return

\begin{center}
\begin{tabular}{lcc}
\hline
 & Coefficient & Standard Error \\
\hline
Intercept & 0.0095 & 0.0078 \\
Oil return & 0.2354 & 0.0760 \\
\hline
\end{tabular}
\end{center}

Critical $t$-values for a $1 \%$ level of significance:

One-sided, left side: -2.441

One-sided, right side $:+2.441$

Two-sided: \textbackslash pm 2.728

Vasileva expects the crude oil return next month, Month 37 , to be -0.01 . She computes the standard error of the forecast to be 0.0469 .

\begin{enumerate}
  \setcounter{enumi}{26}
  \item Which of Vasileva's assumptions regarding regression analysis is incorrect?
A. Assumption 1
B. Assumption 2
C. Assumption 3 28. Based on Exhibit 1, the standard error of the estimate is closest to:
A. 0.04456 .
B. 0.04585 .
C. 0.05018 .

  \item Based on Exhibit 2, Vasileva should reject the null hypothesis that:
A. the slope is less than or equal to 0.15 .
B. the intercept is less than or equal to zero.
C. crude oil returns do not explain Amtex share returns.

  \item Based on Exhibit 2 and Vasileva's prediction of the crude oil return for Month 37, the estimate of Amtex share return for Month 37 is closest to:
A. -0.0024 .
B. 0.0071 .
C. 0.0119 .

  \item Using information from Exhibit 2, the $99 \%$ prediction interval for Amtex share return for Month 37 is best described as:
A. $\widehat{Y}_{f} \pm 0.0053$.
B. $\widehat{Y}_{f} \pm 0.0469$.
C. $\widehat{Y}_{f} \pm 0.1279$.

\end{enumerate}

\section{The following information relates to questions 32-34}
Doug Abitbol is a portfolio manager for Polyi Investments, a hedge fund that trades in the United States. Abitbol manages the hedge fund with the help of Robert Olabudo, a junior portfolio manager.

Abitbol looks at economists' inflation forecasts and would like to examine the relationship between the US Consumer Price Index (US CPI) consensus forecast and the actual US CPI using regression analysis. Olabudo estimates regression coefficients to test whether the consensus forecast is unbiased. If the consensus forecasts are unbiased, the intercept should be 0.0 and the slope will be equal to 1.0. Regression results are presented in Exhibit 1. Additionally, Olabudo calculates the $95 \%$ prediction interval of the actual CPI using a US CPI consensus forecast of 2.8

\section{Exhibit 1: Regression Output: Estimating US CPI}
\section{Regression Statistics}
\begin{center}
\begin{tabular}{lccc}
\hline
Regression Statistics &  &  &  \\
\hline
Standard error of estimate & 0.0009 &  &  \\
Observations & 60 &  &  \\
 & Coefficients & Standard Error & $\boldsymbol{t}$-Statistic \\
 & 0.0001 & 0.0002 & 0.5000 \\
Intercept & 0.9830 & 0.0155 & 63.4194 \\
\hline
US CPI consensus forecast &  &  &  \\
\hline
\end{tabular}
\end{center}

Notes:

\begin{enumerate}
  \item The absolute value of the critical value for the $t$-statistic is 2.002 at the $5 \%$ level of significance.

  \item The standard deviation of the US CPI consensus forecast is $s_{x}=0.7539$.

  \item The mean of the US CPI consensus forecast is $\bar{X}=1.3350$.

\end{enumerate}

Finally, Abitbol and Olabudo discuss the forecast and forecast interval:

Observation 1 For a given confidence level, the forecast interval is the same no matter the US CPI consensus forecast.

Observation 2 A larger standard error of the estimate will result in a wider confidence interval.

\begin{enumerate}
  \setcounter{enumi}{31}
  \item Based on Exhibit 1, Olabudo should:
A. conclude that the inflation predictions are unbiased.
B. reject the null hypothesis that the slope coefficient equals one.
C. reject the null hypothesis that the intercept coefficient equals zero.

  \item Based on Exhibit 1, Olabudo should calculate a prediction interval for the actual US CPI closest to:
A. 2.7506 to 2.7544 .
B. 2.7521 to 2.7529 .
C. 2.7981 to 2.8019 .

  \item Which of Olabudo's observations of forecasting is correct?
A. Only Observation 1
B. Only Observation 2
C. Both Observation 1 and Observations 2

\end{enumerate}

\section{The following information relates to questions}
 35-38Espey Jones is examining the relation between the net profit margin (NPM) of companies, in percent, and their fixed asset turnover (FATO). He collected a sample of 35 companies for the most recent fiscal year and fit several different functional forms, settling on the following model:

$\operatorname{lnNPM}{ }_{i}=b_{0}+b_{1} \mathrm{FATO}_{i}$.

The results of this estimation are provided in Exhibit 1.

\section{Exhibit 1: Results of Regressing NPM on FATO}
\begin{center}
\begin{tabular}{lccccc}
\hline
Source & df & $\begin{array}{c}\text { Sum of } \\ \text { Squares }\end{array}$ & $\begin{array}{c}\text { Mean } \\ \text { Square }\end{array}$ & $\boldsymbol{F}$ & $\boldsymbol{p}$-Value \\
\hline
Regression & 1 & 102.9152 & 102.9152 & $1,486.7079$ & 0.0000 \\
Residual & 32 & 2.2152 & 0.0692 &  &  \\
Total & 33 & 105.1303 &  &  &  \\
\hline
 &  &  &  &  &  \\
\hline
 & Standard &  &  &  &  \\
\hline
Coefficients & Error & $\boldsymbol{t}$ - Statistic & $\boldsymbol{p}$-Value &  &  \\
\hline
Intercept & 0.5987 & 0.0561 & 10.6749 & 0.0000 &  \\
FATO & 0.2951 & 0.0077 & 38.5579 & 0.0000 &  \\
\hline
\end{tabular}
\end{center}

\begin{enumerate}
  \setcounter{enumi}{34}
  \item The coefficient of determination is closest to:
A. 0.0211 .
B. 0.9789 .
C. 0.9894 .

  \item The standard error of the estimate is closest to:
A. 0.2631 .
B. 1.7849 .
C. 38.5579 .

  \item At a 0.01 level of significance, Jones should conclude that:

\end{enumerate}

A. the mean net profit margin is $0.5987 \%$.

B. the variation of the fixed asset turnover explains the variation of the natural $\log$ of the net profit margin.

C. a change in the fixed asset turnover from 3 to 4 times is likely to result in a change in the net profit margin of $0.5987 \%$.

\begin{enumerate}
  \setcounter{enumi}{37}
  \item The predicted net profit margin for a company with a fixed asset turnover of 2 times is closest to:
A. $1.1889 \%$.
B. $1.8043 \%$.
C. $3.2835 \%$

  \item Using the same information as in Question 8, what would Accent's cost of goods sold be under the weighted average cost method?
A. $€ 120,000$.
B. $€ 122,000$.
C. $€ 124,000$.

\end{enumerate}

\section{SOLUTIONS}
\begin{enumerate}
  \item C is correct. Homoskedasticity is the situation in which the variance of the residuals is constant across the observations.

  \item 
\end{enumerate}

A. The coefficient of determination is 0.4279 :

$\frac{\text { Explained variation }}{\text { Total variation }}=\frac{60.16}{140.58}=0.4279$.

B. $F=\frac{60.16 / 1}{(140.58-60.16)(60-2)}=\frac{60.16}{1.3866}=43.3882$.

C. Begin with the sum of squares error of $140.58-60.16=80.42$. Then calculate the mean square error of $80.42 \div(60-2)=1.38655$. The standard error of the estimate is the square root of the mean square error: $s_{e}=\sqrt{1.38655}=$ 1.1775.

D. The sample variance of the dependent variable uses the total variation of the dependent variable and divides it by the number of observations less one:

$\sum_{i=1}^{n} \frac{\left(Y_{i}-\bar{Y}\right)^{2}}{n-1}=\frac{\text { Total variation }}{n-1}=\frac{140.58}{60-1}=2.3827$.

The sample standard deviation of the dependent variable is the square root of the variance, or $\sqrt{2.3827}=1.544$.

\begin{enumerate}
  \setcounter{enumi}{2}
  \item The Month 2 data point is an outlier, lying far away from the other data values. Because this outlier was caused by a data entry error, correcting the outlier improves the validity and reliability of the regression. In this case, revised $R^{2}$ is lower (from 0.9921 to 0.6784 ). The outliers created the illusion of a better fit from the higher $R^{2}$; the outliers altered the estimate of the slope. The standard error of the estimate is lower when the data error is corrected (from 2.8619 to 2.0624), as a result of the lower mean square error. However, at a 0.05 level of significance, both models fit well. The difference in the fit is illustrated in Exhibit 1.
\end{enumerate}

\section{Exhibit 1: The Fit of the Model with and without Data Errors}
A. Before the Data Errors Are Corrected

\begin{center}
\includegraphics[max width=\textwidth]{2023_05_04_cff39ee44f77d6514e1bg-499(1)}
\end{center}

\section{B. After the Data Errors Are Corrected}
Portfolio Return (\%)

\begin{center}
\includegraphics[max width=\textwidth]{2023_05_04_cff39ee44f77d6514e1bg-499}
\end{center}

\begin{enumerate}
  \setcounter{enumi}{3}
  \item A is correct. SHIFT is an indicator or dummy variable because it takes on only the values 0 and 1 .

  \item C is correct. In a simple regression with a single indicator variable, the intercept is the mean of the dependent variable when the indicator variable takes on a value of zero, which is before the shift in policy in this case.

  \item C is correct. Whereas the intercept is the average of the dependent variable when the indicator variable is zero (that is, before the shift in policy), the slope is the difference in the mean of the dependent variable from before to after the change in policy.

  \item A is correct. The null hypothesis of no difference in the annual growth rate is rejected at the 0.05 level: The calculated test statistic of -8.16188 is outside the bounds of \textbackslash pm 2.048 .

  \item 
\end{enumerate}

A. The sample variance of the dependent variable is the sum of squares total divided by its degrees of freedom $(n-1=5-1=4$, as given). Thus, the sample variance of the dependent variable is $95.2 \div 4=23.8$.

B. The coefficient of determination $=88.0 \div 95.2=0.92437$.

C. The $F$-statistic tests whether all the slope coefficients in a linear regression are equal to zero. D. The calculated value of the $F$-statistic is 36.667 , as shown in the table. The corresponding $p$-value is less than 0.05 , so you reject the null hypothesis of a slope equal to zero.

E. The standard error of the estimate is the square root of the mean square error: $s_{e}=\sqrt{2.4}=1.54919$.

\begin{enumerate}
  \setcounter{enumi}{8}
  \item A is correct. The coefficient of determination is the same as $R^{2}$, which is 0.7436 in the table.

  \item $\mathrm{C}$ is correct. Because the slope is positive, the correlation between $X$ and $Y$ is simply the square root of the coefficient of determination: $\sqrt{0.7436}=0.8623$.

  \item $\mathrm{C}$ is correct. To make a prediction using the regression model, multiply the slope coefficient by the forecast of the independent variable and add the result to the intercept. Expected value of CFO to sales $=0.077+(0.826 \times 5)=4.207$.

  \item $\mathrm{C}$ is correct. The $p$-value is the smallest level of significance at which the null hypotheses concerning the slope coefficient can be rejected. In this case, the $p$-value is less than 0.05 , and thus the regression of the ratio of cash flow from operations to sales on the ratio of net income to sales is significant at the $5 \%$ level.

  \item A is correct. The data are observations over time.

  \item $\mathrm{C}$ is correct. From the regression equation, Expected return $=0.0138+(-0.6486$ $x-0.01)=0.0138+0.006486=0.0203$, or $2.03 \%$.

  \item C is correct. $R^{2}$ is the coefficient of determination. In this case, it shows that $2.11 \%$ of the variability in Stellar's returns is explained by changes in CPIENG.

  \item B is correct. The standard error of the estimate is the standard deviation of the regression residuals.

  \item $\mathrm{C}$ is the correct response because it is a false statement. The slope and intercept are both statistically different from zero at the 0.05 level of significance.

  \item $\mathrm{C}$ is correct. The slope coefficient (shown in Exhibit 2) is negative. We could also determine this by looking at the cross-product (Exhibit 1 ), which is negative.

  \item B is correct. The sample covariance is calculated as $\frac{\sum_{i=1}^{n}\left(X_{i}-\bar{X}\right)\left(Y_{i}-\bar{Y}\right)}{n-1}=-9.2430 \div 49=-0.1886$.

  \item A is correct. In simple regression, the $R^{2}$ is the square of the pairwise correlation. Because the slope coefficient is negative, the correlation is the negative of the square root of 0.0933 , or -0.3054 .

  \item $\mathrm{C}$ is correct. Conclusions cannot be drawn regarding causation; they can be drawn only about association; therefore, Interpretations 1 and 2 are incorrect.

  \item $C$ is correct. Liu explains the variation of the short interest ratio using the variation of the debt ratio.

  \item A is correct. The degrees of freedom are the number of observations minus the number of parameters estimated, which equals 2 in this case (the intercept and the slope coefficient). The number of degrees of freedom is $50-2=48$.

  \item B is correct. The $t$-statistic is -2.2219 , which is outside the bounds created by the critical $t$-values of \textbackslash pm 2.011 for a two-tailed test with a $5 \%$ significance level. The value of 2.011 is the critical $t$-value for the $5 \%$ level of significance (2.5\% in one tail) for 48 degrees of freedom. A is incorrect because the mean of the short interest ratio is $192.3 \div 50=3.846$. $\mathrm{C}$ is incorrect because the debt ratio explains 9.33\% of the variation of the short interest ratio.

  \item A is correct. The predicted value of the short interest ratio $=5.4975+(-4.1589 \times$ $0.40)=5.4975-1.6636=3.8339$.

  \item $C$ is correct because $F=\frac{\text { Mean square regression }}{\text { Mean square error }}=\frac{38.4404}{7.7867}=4.9367$.

  \item C is correct. The assumptions of the linear regression model are that (1) the relationship between the dependent variable and the independent variable is linear in the parameters $b_{0}$ and $b_{1}$, (2) the residuals are independent of one another, (3) the variance of the error term is the same for all observations, and (4) the error term is normally distributed. Assumption 3 is incorrect because the dependent variable need not be normally distributed.

  \item B is correct. The standard error of the estimate for a linear regression model with one independent variable is calculated as the square root of the mean square error:

\end{enumerate}

$s_{e}=\sqrt{\frac{0.071475}{34}}=0.04585$.

\begin{enumerate}
  \setcounter{enumi}{28}
  \item C is correct. Crude oil returns explain the Amtex share returns if the slope coefficient is statistically different from zero. The slope coefficient is 0.2354 , and the calculated $t$-statistic is
\end{enumerate}

$t=\frac{0.2354-0.0000}{0.0760}=3.0974$

which is outside the bounds of the critical values of \textbackslash pm 2.728 .

Therefore, Vasileva should reject the null hypothesis that crude oil returns do not explain Amtex share returns, because the slope coefficient is statistically different from zero.

A is incorrect because the calculated $t$-statistic for testing the slope against 0.15 is $t=\frac{0.2354-0.1500}{0.0760}=1.1237$, which is less than the critical value of +2.441 .

$\mathrm{B}$ is incorrect because the calculated $t$-statistic is $t=\frac{0.0095-0.0000}{0.0078}=1.2179$, which is less than the critical value of +2.441 .

\begin{enumerate}
  \setcounter{enumi}{29}
  \item B is correct. The predicted value of the dependent variable, Amtex share return, given the value of the independent variable, crude oil return, -0.01 , is calculated as $\widehat{Y}=\hat{b}_{0}+\hat{b}_{1} X_{i}=0.0095+[0.2354 \times(-0.01)]=0.0071$.

  \item $C$ is correct. The predicted share return is $0.0095+[0.2354 \times(-0.01)]=0.0071$. The lower limit for the prediction interval is $0.0071-(2.728 \times 0.0469)=-0.1208$, and the upper limit for the prediction interval is $0.0071+(2.728 \times 0.0469)=$ 0.1350 .

\end{enumerate}

A is incorrect because the bounds of the interval should be based on the standard error of the forecast and the critical $t$-value, not on the mean of the dependent variable.

B is incorrect because bounds of the interval are based on the product of the standard error of the forecast and the critical $t$-value, not simply the standard error of the forecast.

\begin{enumerate}
  \setcounter{enumi}{31}
  \item A is correct. We fail to reject the null hypothesis of a slope equal to one, and we fail to reject the null hypothesis of an intercept equal to zero. The test of the slope equal to 1.0 is $t=\frac{0.9830-1.000}{0.0155}=-1.09677$. The test of the intercept equal to 0.0 is $t=\frac{0.0001-0.0000}{.00002}=0.5000$. Therefore, we conclude that the forecasts are unbiased.

  \item A is correct. The forecast interval for inflation is calculated in three steps:

\end{enumerate}

Step 1. Make the prediction given the US CPI forecast of 2.8:

$$
\begin{aligned}
& \overparen{Y}=b_{0}+b_{1} X \\
& =0.0001+(0.9830 \times 2.8) \\
& =2.7525 .
\end{aligned}
$$

Step 2. Compute the variance of the prediction error:

$$
\begin{aligned}
& s_{f}^{2}=s_{e}^{2}\left\{1+(1 / n)+\left[\left(X_{f}-\bar{X}\right)^{2}\right] /\left[(n-1) \times s_{x}^{2}\right]\right\} . \\
& s_{f}^{2}=0.0009^{2}\left\{1+(1 / 60)+\left[(2.8-1.3350)^{2}\right] /\left[(60-1) \times 0.7539^{2}\right]\right\} . \\
& s_{f}^{2}=0.00000088 . \\
& s_{f}=0.0009 .
\end{aligned}
$$

Step 3. Compute the prediction interval:

$\widehat{Y} \pm t_{c} \times s_{f}$

$2.7525 \pm(2.0 \times 0.0009)$

Lower bound: $2.7525-(2.0 \times 0.0009)=2.7506$.

Upper bound: $2.7525+(2.0 \times 0.0009)=2.7544$.

So, given the US CPI forecast of 2.8, the $95 \%$ prediction interval is 2.7506 to 2.7544 .

\begin{enumerate}
  \setcounter{enumi}{33}
  \item B is correct. The confidence level influences the width of the forecast interval through the critical $t$-value that is used to calculate the distance from the forecasted value: The larger the confidence level, the wider the interval. Therefore, Observation 1 is not correct.
\end{enumerate}

Observation 2 is correct. The greater the standard error of the estimate, the greater the standard error of the forecast.

\begin{enumerate}
  \setcounter{enumi}{34}
  \item B is correct. The coefficient of determination is $102.9152 \div 105.1303=0.9789$.

  \item A is correct. The standard error is the square root of the mean square error, or $\sqrt{0.0692}=0.2631$.

  \item B is correct. The $p$-value corresponding to the slope is less than 0.01 , so we reject the null hypothesis of a zero slope, concluding that the fixed asset turnover explains the natural $\log$ of the net profit margin.

  \item $\mathrm{C}$ is correct. The predicted natural $\log$ of the net profit margin is $0.5987+(2 \times$ $0.2951)=1.1889$. The predicted net profit margin is $e^{1.1889}=3.2835 \%$.

  \item $C$ is correct. Under the weighted average cost method:

\end{enumerate}

\begin{center}
\begin{tabular}{lcc}
October purchases & 10,000 units & $\$ 100,000$ \\
November purchases & 5,000 units & $\$ 55,000$ \\
\cline { 2 - 3 }
Total & 15,000 units & $\$ 155,000$ \\
\cline { 2 - 3 }
\end{tabular}
\end{center}

$\$ 155,000 / 15,000$ units $=\$ 10.3333$

$\$ 10.3333 \times 12,000$ units $=\$ 124,000$

\section{APPENDICES}
Appendix A

Appendix B

Appendix C

Appendix D

Appendix $\mathrm{E}$ Cumulative Probabilities for a Standard Normal Distribution

Table of the Student's $t$-Distribution (One-Tailed Probabilities)

Values of $X^{2}$ (Degrees of Freedom, Level of Significance)

Table of the $F$-Distribution

Critical Values for the Durbin-Watson Statistic $(\alpha=.05)$ Appendix $A$

Cumulative Probabilities for a Standard Normal Distribution

$P(Z \leq x)=N(x)$ for $x \geq 0$ or $P(Z \leq z)=N(z)$ for $z \geq 0$

\begin{center}
\begin{tabular}{|c|c|c|c|c|c|c|c|c|c|c|}
\hline
$x$ or $z$ & 0 & 0.01 & 0.02 & 0.03 & 0.04 & 0.05 & 0.06 & 0.07 & 0.08 & 0.09 \\
\hline
0.00 & 0.5000 & 0.5040 & 0.5080 & 0.5120 & 0.5160 & 0.5199 & 0.5239 & 0.5279 & 0.5319 & 0.535 \\
\hline
0.10 & 0.5398 & 0.5438 & 0.5478 & 0.5517 & 0.5557 & 0.5596 & 0.5636 & 0.5675 & 0.5714 & 0.575 \\
\hline
0.20 & 0.5793 & 0.5832 & 0.5871 & 0.5910 & 0.5948 & 0.5987 & 0.6026 & 0.6064 & 0.6103 & 0.614 \\
\hline
0.30 & 0.6179 & 0.6217 & 0.6255 & 0.6293 & 0.6331 & 0.6368 & 0.6406 & 0.6443 & 0.6480 & 0.6517 \\
\hline
0.40 & 0.6554 & 0.6591 & 0.6628 & 0.6664 & 0.6700 & 0.6736 & 0.6772 & 0.6808 & 0.6844 & 0.687 \\
\hline
0.50 & 0.6915 & 0.6950 & 0.6985 & 0.7019 & 0.7054 & 0.7088 & 0.7123 & 0.7157 & 0.7190 & 0.722 \\
\hline
0.60 & 0.7257 & 0.7291 & 0.7324 & 0.7357 & 0.7389 & 0.7422 & 0.7454 & 0.7486 & 0.7517 & 0.754 \\
\hline
0.70 & 0.7580 & 0.7611 & 0.7642 & 0.7673 & 0.7704 & 0.7734 & 0.7764 & 0.7794 & 0.7823 & 0.785 \\
\hline
0.80 & 0.7881 & 0.7910 & 0.7939 & 0.7967 & 0.7995 & 0.8023 & 0.8051 & 0.8078 & 0.8106 & 0.813 \\
\hline
0.90 & 0.8159 & 0.8186 & 0.8212 & 0.8238 & 0.8264 & 0.8289 & 0.8315 & 0.8340 & 0.8365 & 0.838 \\
\hline
1.00 & 0.8413 & 0.8438 & 0.8461 & 0.8485 & 0.8508 & 0.8531 & 0.8554 & 0.8577 & 0.8599 & 0.862 \\
\hline
1.10 & 0.8643 & 0.8665 & 0.8686 & 0.8708 & 0.8729 & 0.8749 & 0.8770 & 0.8790 & 0.8810 & 0.883 \\
\hline
1.20 & 0.8849 & 0.8869 & 0.8888 & 0.8907 & 0.8925 & 0.8944 & 0.8962 & 0.8980 & 0.8997 & 0.901 \\
\hline
1.30 & 0.9032 & 0.9049 & 0.9066 & 0.9082 & 0.9099 & 0.9115 & 0.9131 & 0.9147 & 0.9162 & 0.917 \\
\hline
1.40 & 0.9192 & 0.9207 & 0.9222 & 0.9236 & 0.9251 & 0.9265 & 0.9279 & 0.9292 & 0.9306 & 0.931 \\
\hline
1.50 & 0.9332 & 0.9345 & 0.9357 & 0.9370 & 0.9382 & 0.9394 & 0.9406 & 0.9418 & 0.9429 & 0.944 \\
\hline
1.60 & 0.9452 & 0.9463 & 0.9474 & 0.9484 & 0.9495 & 0.9505 & 0.9515 & 0.9525 & 0.9535 & 0.954 \\
\hline
1.70 & 0.9554 & 0.9564 & 0.9573 & 0.9582 & 0.9591 & 0.9599 & 0.9608 & 0.9616 & 0.9625 & 0.963 \\
\hline
1.80 & 0.9641 & 0.9649 & 0.9656 & 0.9664 & 0.9671 & 0.9678 & 0.9686 & 0.9693 & 0.9699 & 0.970 \\
\hline
1.90 & 0.9713 & 0.9719 & 0.9726 & 0.9732 & 0.9738 & 0.9744 & 0.9750 & 0.9756 & 0.9761 & $0.976 ?$ \\
\hline
2.00 & 0.9772 & 0.9778 & 0.9783 & 0.9788 & 0.9793 & 0.9798 & 0.9803 & 0.9808 & 0.9812 & 0.981 \\
\hline
2.10 & 0.9821 & 0.9826 & 0.9830 & 0.9834 & 0.9838 & 0.9842 & 0.9846 & 0.9850 & 0.9854 & 0.985 \\
\hline
2.20 & 0.9861 & 0.9864 & 0.9868 & 0.9871 & 0.9875 & 0.9878 & 0.9881 & 0.9884 & 0.9887 & 0.989 \\
\hline
2.30 & 0.9893 & 0.9896 & 0.9898 & 0.9901 & 0.9904 & 0.9906 & 0.9909 & 0.9911 & 0.9913 & 0.991 \\
\hline
2.40 & 0.9918 & 0.9920 & 0.9922 & 0.9925 & 0.9927 & 0.9929 & 0.9931 & 0.9932 & 0.9934 & 0.993 \\
\hline
2.50 & 0.9938 & 0.9940 & 0.9941 & 0.9943 & 0.9945 & 0.9946 & 0.9948 & 0.9949 & 0.9951 & 0.9952 \\
\hline
2.60 & 0.9953 & 0.9955 & 0.9956 & 0.9957 & 0.9959 & 0.9960 & 0.9961 & 0.9962 & 0.9963 & 0.996 \\
\hline
2.70 & 0.9965 & 0.9966 & 0.9967 & 0.9968 & 0.9969 & 0.9970 & 0.9971 & 0.9972 & 0.9973 & 0.997 \\
\hline
2.80 & 0.9974 & 0.9975 & 0.9976 & 0.9977 & 0.9977 & 0.9978 & 0.9979 & 0.9979 & 0.9980 & 0.998 \\
\hline
2.90 & 0.9981 & 0.9982 & 0.9982 & 0.9983 & 0.9984 & 0.9984 & 0.9985 & 0.9985 & 0.9986 & 0.998 \\
\hline
3.00 & 0.9987 & 0.9987 & 0.9987 & 0.9988 & 0.9988 & 0.9989 & 0.9989 & 0.9989 & 0.9990 & 0.999 \\
\hline
3.10 & 0.9990 & 0.9991 & 0.9991 & 0.9991 & 0.9992 & 0.9992 & 0.9992 & 0.9992 & 0.9993 & $0.999:$ \\
\hline
3.20 & 0.9993 & 0.9993 & 0.9994 & 0.9994 & 0.9994 & 0.9994 & 0.9994 & 0.9995 & 0.9995 & 0.999 \\
\hline
3.30 & 0.9995 & 0.9995 & 0.9995 & 0.9996 & 0.9996 & 0.9996 & 0.9996 & 0.9996 & 0.9996 & 0.9997 \\
\hline
3.40 & 0.9997 & 0.9997 & 0.9997 & 0.9997 & 0.9997 & 0.9997 & 0.9997 & 0.9997 & 0.9997 & 0.999 \\
\hline
3.50 & 0.9998 & 0.9998 & 0.9998 & 0.9998 & 0.9998 & 0.9998 & 0.9998 & 0.9998 & 0.9998 & 0.999 \\
\hline
3.60 & 0.9998 & 0.9998 & 0.9999 & 0.9999 & 0.9999 & 0.9999 & 0.9999 & 0.9999 & 0.9999 & 0.999 \\
\hline
3.70 & 0.9999 & 0.9999 & 0.9999 & 0.9999 & 0.9999 & 0.9999 & 0.9999 & 0.9999 & 0.9999 & 0.999 \\
\hline
3.80 & 0.9999 & 0.9999 & 0.9999 & 0.9999 & 0.9999 & 0.9999 & 0.9999 & 0.9999 & 0.9999 & 0.9995 \\
\hline
3.90 & 1.0000 & 1.0000 & 1.0000 & 1.0000 & 1.0000 & 1.0000 & 1.0000 & 1.0000 & 1.0000 & 1.000 \\
\hline
4.00 & 1.0000 & 1.0000 & 1.0000 & 1.0000 & 1.0000 & 1.0000 & 1.0000 & 1.0000 & 1.0000 & 1.0000 \\
\hline
\end{tabular}
\end{center}

For example, to find the $z$-value leaving 2.5 percent of the area/probability in the upper tail, find the element 0.9750 in the body of the table. Read 1.90 at the left end of the element's row and 0.06 at the top of the element's column, to give $1.90+0.06=1.96$. Table generated with Excel.

Quantitative Methods for Investment Analysis, Second Edition, by Richard A. DeFusco, CFA, Dennis W. McLeavey, CFA, Jerald E. Pinto, CFA, and David E. Runkle, CFA. Copyright @ 2004 by CFA Institute.

\section{Appendix A (continued)}
Cumulative Probabilities for a Standard Normal Distribution

$P(Z \leq x)=N(x)$ for $x \leq 0$ or $P(Z \leq z)=N(z)$ for $z \leq 0$

\begin{center}
\begin{tabular}{|c|c|c|c|c|c|c|c|c|c|c|}
\hline
$x$ or $z$ & 0 & 0.01 & 0.02 & 0.03 & 0.04 & 0.05 & 0.06 & 0.07 & 0.08 & 0.09 \\
\hline
0.0 & 0.5000 & 0.4960 & 0.4920 & 0.4880 & 0.4840 & 0.4801 & 0.4761 & 0.4721 & 0.4681 & 0.4641 \\
\hline
-0.10 & 0.4602 & 0.4562 & 0.4522 & 0.4483 & 0.4443 & 0.4404 & 0.4364 & 0.4325 & 0.4286 & 0.4247 \\
\hline
-0.20 & 0.4207 & 0.4168 & 0.4129 & 0.4090 & 0.4052 & 0.4013 & 0.3974 & 0.3936 & 0.3897 & 0.3859 \\
\hline
-0.30 & 0.3821 & 0.3783 & 0.3745 & 0.3707 & 0.3669 & 0.3632 & 0.3594 & 0.3557 & 0.3520 & 0.3483 \\
\hline
-0.40 & 0.3446 & 0.3409 & 0.3372 & 0.3336 & 0.3300 & 0.3264 & 0.3228 & 0.3192 & 0.3156 & 0.3121 \\
\hline
-0.50 & 0.3085 & 0.3050 & 0.3015 & 0.2981 & 0.2946 & 0.2912 & 0.2877 & 0.2843 & 0.2810 & 0.2776 \\
\hline
-0.60 & 0.2743 & 0.2709 & 0.2676 & 0.2643 & 0.2611 & 0.2578 & 0.2546 & 0.2514 & 0.2483 & 0.2451 \\
\hline
-0.70 & 0.2420 & 0.2389 & 0.2358 & 0.2327 & 0.2296 & 0.2266 & 0.2236 & 0.2206 & 0.2177 & 0.2148 \\
\hline
-0.80 & 0.2119 & 0.2090 & 0.2061 & 0.2033 & 0.2005 & 0.1977 & 0.1949 & 0.1922 & 0.1894 & 0.1867 \\
\hline
-0.90 & 0.1841 & 0.1814 & 0.1788 & 0.1762 & 0.1736 & 0.1711 & 0.1685 & 0.1660 & 0.1635 & 0.1611 \\
\hline
-1.00 & 0.1587 & 0.1562 & 0.1539 & 0.1515 & 0.1492 & 0.1469 & 0.1446 & 0.1423 & 0.1401 & 0.1379 \\
\hline
-1.10 & 0.1357 & 0.1335 & 0.1314 & 0.1292 & 0.1271 & 0.1251 & 0.1230 & 0.1210 & 0.1190 & 0.1170 \\
\hline
-1.20 & 0.1151 & 0.1131 & 0.1112 & 0.1093 & 0.1075 & 0.1056 & 0.1038 & 0.1020 & 0.1003 & 0.0985 \\
\hline
-1.30 & 0.0968 & 0.0951 & 0.0934 & 0.0918 & 0.0901 & 0.0885 & 0.0869 & 0.0853 & 0.0838 & 0.0823 \\
\hline
-1.40 & 0.0808 & 0.0793 & 0.0778 & 0.0764 & 0.0749 & 0.0735 & 0.0721 & 0.0708 & 0.0694 & 0.0681 \\
\hline
-1.50 & 0.0668 & 0.0655 & 0.0643 & 0.0630 & 0.0618 & 0.0606 & 0.0594 & 0.0582 & 0.0571 & 0.0559 \\
\hline
-1.60 & 0.0548 & 0.0537 & 0.0526 & 0.0516 & 0.0505 & 0.0495 & 0.0485 & 0.0475 & 0.0465 & 0.0455 \\
\hline
-1.70 & 0.0446 & 0.0436 & 0.0427 & 0.0418 & 0.0409 & 0.0401 & 0.0392 & 0.0384 & 0.0375 & 0.0367 \\
\hline
-1.80 & 0.0359 & 0.0351 & 0.0344 & 0.0336 & 0.0329 & 0.0322 & 0.0314 & 0.0307 & 0.0301 & 0.0294 \\
\hline
-1.90 & 0.0287 & 0.0281 & 0.0274 & 0.0268 & 0.0262 & 0.0256 & 0.0250 & 0.0244 & 0.0239 & 0.0233 \\
\hline
-2.00 & 0.0228 & 0.0222 & 0.0217 & 0.0212 & 0.0207 & 0.0202 & 0.0197 & 0.0192 & 0.0188 & 0.0183 \\
\hline
-2.10 & 0.0179 & 0.0174 & 0.0170 & 0.0166 & 0.0162 & 0.0158 & 0.0154 & 0.0150 & 0.0146 & 0.0143 \\
\hline
-2.20 & 0.0139 & 0.0136 & 0.0132 & 0.0129 & 0.0125 & 0.0122 & 0.0119 & 0.0116 & 0.0113 & 0.0110 \\
\hline
-2.30 & 0.0107 & 0.0104 & 0.0102 & 0.0099 & 0.0096 & 0.0094 & 0.0091 & 0.0089 & 0.0087 & 0.0084 \\
\hline
-2.40 & 0.0082 & 0.0080 & 0.0078 & 0.0075 & 0.0073 & 0.0071 & 0.0069 & 0.0068 & 0.0066 & 0.0064 \\
\hline
-2.50 & 0.0062 & 0.0060 & 0.0059 & 0.0057 & 0.0055 & 0.0054 & 0.0052 & 0.0051 & 0.0049 & 0.0048 \\
\hline
-2.60 & 0.0047 & 0.0045 & 0.0044 & 0.0043 & 0.0041 & 0.0040 & 0.0039 & 0.0038 & 0.0037 & 0.0036 \\
\hline
-2.70 & 0.0035 & 0.0034 & 0.0033 & 0.0032 & 0.0031 & 0.0030 & 0.0029 & 0.0028 & 0.0027 & 0.0026 \\
\hline
-2.80 & 0.0026 & 0.0025 & 0.0024 & 0.0023 & 0.0023 & 0.0022 & 0.0021 & 0.0021 & 0.0020 & 0.0019 \\
\hline
-2.90 & 0.0019 & 0.0018 & 0.0018 & 0.0017 & 0.0016 & 0.0016 & 0.0015 & 0.0015 & 0.0014 & 0.0014 \\
\hline
-3.00 & 0.0013 & 0.0013 & 0.0013 & 0.0012 & 0.0012 & 0.0011 & 0.0011 & 0.0011 & 0.0010 & 0.0010 \\
\hline
-3.10 & 0.0010 & 0.0009 & 0.0009 & 0.0009 & 0.0008 & 0.0008 & 0.0008 & 0.0008 & 0.0007 & 0.0007 \\
\hline
-3.20 & 0.0007 & 0.0007 & 0.0006 & 0.0006 & 0.0006 & 0.0006 & 0.0006 & 0.0005 & 0.0005 & 0.0005 \\
\hline
-3.30 & 0.0005 & 0.0005 & 0.0005 & 0.0004 & 0.0004 & 0.0004 & 0.0004 & 0.0004 & 0.0004 & 0.0003 \\
\hline
-3.40 & 0.0003 & 0.0003 & 0.0003 & 0.0003 & 0.0003 & 0.0003 & 0.0003 & 0.0003 & 0.0003 & 0.0002 \\
\hline
-3.50 & 0.0002 & 0.0002 & 0.0002 & 0.0002 & 0.0002 & 0.0002 & 0.0002 & 0.0002 & 0.0002 & 0.0002 \\
\hline
-3.60 & 0.0002 & 0.0002 & 0.0001 & 0.0001 & 0.0001 & 0.0001 & 0.0001 & 0.0001 & 0.0001 & 0.0001 \\
\hline
-3.70 & 0.0001 & 0.0001 & 0.0001 & 0.0001 & 0.0001 & 0.0001 & 0.0001 & 0.0001 & 0.0001 & 0.0001 \\
\hline
-3.80 & 0.0001 & 0.0001 & 0.0001 & 0.0001 & 0.0001 & 0.0001 & 0.0001 & 0.0001 & 0.0001 & 0.0001 \\
\hline
-3.90 & 0.0000 & 0.0000 & 0.0000 & 0.0000 & 0.0000 & 0.0000 & 0.0000 & 0.0000 & 0.0000 & 0.0000 \\
\hline
-4.00 & 0.0000 & 0.0000 & 0.0000 & 0.0000 & 0.0000 & 0.0000 & 0.0000 & 0.0000 & 0.0000 & 0.0000 \\
\hline
\end{tabular}
\end{center}

For example, to find the $z$-value leaving 2.5 percent of the area/probability in the lower tail, find the element 0.0250 in the body of the table. Read -1.90 at the left end of the element's row and 0.06 at the top of the element's column, to give $-1.90-0.06=-1.96$. Table generated with Excel. Appendix B

Table of the Student's $t$-Distribution (One-Tailed Probabilities)

\begin{center}
\includegraphics[max width=\textwidth]{2023_05_04_cff39ee44f77d6514e1bg-506}
\end{center}

Appendix C

Values of $\chi^{2}$ (Degrees of Freedom, Level of Significance)

\begin{center}
\begin{tabular}{|c|c|c|c|c|c|c|c|c|c|}
\hline
\multirow{2}{*}{$\begin{array}{c}\text { Degrees of } \\
\text { Freedom }\end{array}$} & \multicolumn{9}{|c|}{Probability in Right Tail} \\
\hline
 & 0.99 & 0.975 & 0.95 & 0.9 & 0.1 & 0.05 & 0.025 & 0.01 & 0.005 \\
\hline
1 & 0.000157 & 0.000982 & 0.003932 & 0.0158 & 2.706 & 3.841 & 5.024 & 6.635 & 7.879 \\
\hline
2 & 0.020100 & 0.050636 & 0.102586 & 0.2107 & 4.605 & 5.991 & 7.378 & 9.210 & 10.597 \\
\hline
3 & 0.1148 & 0.2158 & 0.3518 & 0.5844 & 6.251 & 7.815 & 9.348 & 11.345 & 12.838 \\
\hline
4 & 0.297 & 0.484 & 0.711 & 1.064 & 7.779 & 9.488 & 11.143 & 13.277 & 14.860 \\
\hline
5 & 0.554 & 0.831 & 1.145 & 1.610 & 9.236 & 11.070 & 12.832 & 15.086 & 16.750 \\
\hline
6 & 0.872 & 1.237 & 1.635 & 2.204 & 10.645 & 12.592 & 14.449 & 16.812 & 18.548 \\
\hline
7 & 1.239 & 1.690 & 2.167 & 2.833 & 12.017 & 14.067 & 16.013 & 18.475 & 20.278 \\
\hline
8 & 1.647 & 2.180 & 2.733 & 3.490 & 13.362 & 15.507 & 17.535 & 20.090 & 21.955 \\
\hline
9 & 2.088 & 2.700 & 3.325 & 4.168 & 14.684 & 16.919 & 19.023 & 21.666 & 23.589 \\
\hline
10 & 2.558 & 3.247 & 3.940 & 4.865 & 15.987 & 18.307 & 20.483 & 23.209 & 25.188 - - |- \\
\hline
11 & 3.053 & 3.816 & 4.575 & 5.578 & 17.275 & 19.675 & 21.920 & 24.725 & 26.757 \\
\hline
12 & 3.571 & 4.404 & 5.226 & 6.304 & 18.549 & 21.026 & 23.337 & 26.217 & 28.300 \\
\hline
13 & 4.107 & 5.009 & 5.892 & 7.041 & 19.812 & 22.362 & 24.736 & 27.688 & 29.819 \\
\hline
14 & 4.660 & 5.629 & 6.571 & 7.790 & 21.064 & 23.685 & 26.119 & 29.141 & 31.319 \\
\hline
15 & 5.229 & 6.262 & 7.261 & 8.547 & 22.307 & 24.996 & 27.488 & 30.578 & 32.801 \\
\hline
16 & 5.812 & 6.908 & 7.962 & 9.312 & 23.542 & 26.296 & 28.845 & 32.000 & 34.267 \\
\hline
17 & 6.408 & 7.564 & 8.672 & 10.085 & 24.769 & 27.587 & 30.191 & 33.409 & 35.718 \\
\hline
18 & 7.015 & 8.231 & 9.390 & 10.865 & 25.989 & 28.869 & 31.526 & 34.805 & 37.156 \\
\hline
19 & 7.633 & 8.907 & 10.117 & 11.651 & 27.204 & 30.144 & 32.852 & 36.191 & 38.582 \\
\hline
20 & 8.260 & 9.591 & 10.851 & 12.443 & 28.412 & 31.410 & 34.170 & 37.566 & 39.997 \\
\hline
21 & 8.897 & 10.283 & 11.591 & 13.240 & 29.615 & 32.671 & 35.479 & 38.932 & 41.401 \\
\hline
22 & 9.542 & 10.982 & 12.338 & 14.041 & 30.813 & 33.924 & 36.781 & 40.289 & 42.796 \\
\hline
23 & 10.196 & 11.689 & 13.091 & 14.848 & 32.007 & 35.172 & 38.076 & 41.638 & 44.181 \\
\hline
24 & 10.856 & 12.401 & 13.848 & 15.659 & 33.196 & 36.415 & 39.364 & 42.980 & 45.558 \\
\hline
25 & 11.524 & 13.120 & 14.611 & 16.473 & 34.382 & 37.652 & 40.646 & 44.314 & 46.928 \\
\hline
26 & 12.198 & 13.844 & 15.379 & 17.292 & 35.563 & 38.885 & 41.923 & 45.642 & 48.290 \\
\hline
27 & 12.878 & 14.573 & 16.151 & 18.114 & 36.741 & 40.113 & 43.195 & 46.963 & 49.645 \\
\hline
28 & 13.565 & 15.308 & 16.928 & 18.939 & 37.916 & 41.337 & 44.461 & 48.278 & 50.994 \\
\hline
29 & 14.256 & 16.047 & 17.708 & 19.768 & 39.087 & 42.557 & 45.722 & 49.588 & 52.335 \\
\hline
30 & 14.953 & 16.791 & 18.493 & 20.599 & 40.256 & 43.773 & 46.979 & 50.892 & 53.672 \\
\hline
50 & 29.707 & 32.357 & 34.764 & 37.689 & 63.167 & 67.505 & 71.420 & 76.154 & 79.490 \\
\hline
60 & 37.485 & 40.482 & 43.188 & 46.459 & 74.397 & 79.082 & 83.298 & 88.379 & 91.952 \\
\hline
80 & 53.540 & 57.153 & 60.391 & 64.278 & 96.578 & 101.879 & 106.629 & 112.329 & 116.321 \\
\hline
100 & 70.065 & 74.222 & 77.929 & 82.358 & 118.498 & 124.342 & 129.561 & 135.807 & 140.170 \\
\hline
\end{tabular}
\end{center}

To have a probability of 0.05 in the right tail when $\mathrm{df}=5$, the tabled value is $X^{2}(5,0.05)=11.070$.

Quantitative Methods for Investment Analysis, Second Edition, by Richard A. DeFusco, CFA, Dennis W. McLeavey, CFA, Jerald E. Pinto, CFA, and David E. Runkle, CFA. Copyright ๑ 2004 by CFA Institute. Appendix D

Table of the F-Distribution

\begin{center}
\includegraphics[max width=\textwidth]{2023_05_04_cff39ee44f77d6514e1bg-508}
\end{center}

Appendix D (continued)

Table of the F-Distribution

\begin{center}
\includegraphics[max width=\textwidth]{2023_05_04_cff39ee44f77d6514e1bg-509}
\end{center}

Appendix D (continued)

Table of the F-Distribution

\begin{center}
\includegraphics[max width=\textwidth]{2023_05_04_cff39ee44f77d6514e1bg-510}
\end{center}

Appendix D (continued)

Table of the F-Distribution

\begin{center}
\includegraphics[max width=\textwidth]{2023_05_04_cff39ee44f77d6514e1bg-511}
\end{center}

Appendix $\mathbf{E}$

Critical Values for the Durbin-Watson Statistic $(\alpha=.05)$

\begin{center}
\begin{tabular}{|c|c|c|c|c|c|c|c|c|c|c|}
\hline
\multirow[b]{2}{*}{$n$} & \multicolumn{2}{|c|}{$K=1$} & \multicolumn{2}{|c|}{$K=2$} & \multicolumn{2}{|c|}{$K=3$} & \multicolumn{2}{|c|}{$K=4$} & \multicolumn{2}{|c|}{$K=5$} \\
\hline
 & $d_{1}$ & $d_{u}$ & $d_{1}$ & $d_{u}$ & $d_{1}$ & $d_{u}$ & $d_{1}$ & $d_{u}$ & $d_{1}$ & $d_{u}$ \\
\hline
15 & 1.08 & 1.36 & 0.95 & 1.54 & 0.82 & 1.75 & 0.69 & 1.97 & 0.56 & 2.21 \\
\hline
16 & 1.10 & 1.37 & 0.98 & 1.54 & 0.86 & 1.73 & 0.74 & 1.93 & 0.62 & 2.15 \\
\hline
17 & 1.13 & 1.38 & 1.02 & 1.54 & 0.90 & 1.71 & 0.78 & 1.90 & 0.67 & 2.10 \\
\hline
18 & 1.16 & 1.39 & 1.05 & 1.53 & 0.93 & 1.69 & 0.82 & 1.87 & 0.71 & 2.06 \\
\hline
19 & 1.18 & 1.40 & 1.08 & 1.53 & 0.97 & 1.68 & 0.86 & 1.85 & 0.75 & 2.02 \\
\hline
20 & 1.20 & 1.41 & 1.10 & 1.54 & 1.00 & 1.68 & 0.90 & 1.83 & 0.79 & 1.99 \\
\hline
21 & 1.22 & 1.42 & 1.13 & 1.54 & 1.03 & 1.67 & 0.93 & 1.81 & 0.83 & 1.96 \\
\hline
22 & 1.24 & 1.43 & 1.15 & 1.54 & 1.05 & 1.66 & 0.96 & 1.80 & 0.86 & 1.94 \\
\hline
23 & 1.26 & 1.44 & 1.17 & 1.54 & 1.08 & 1.66 & 0.99 & 1.79 & 0.90 & 1.92 \\
\hline
24 & 1.27 & 1.45 & 1.19 & 1.55 & 1.10 & 1.66 & 1.01 & 1.78 & 0.93 & 1.90 \\
\hline
25 & 1.29 & 1.45 & 1.21 & 1.55 & 1.12 & 1.66 & 1.04 & 1.77 & 0.95 & 1.89 \\
\hline
26 & 1.30 & 1.46 & 1.22 & 1.55 & 1.14 & 1.65 & 1.06 & 1.76 & 0.98 & 1.88 \\
\hline
27 & 1.32 & 1.47 & 1.24 & 1.56 & 1.16 & 1.65 & 1.08 & 1.76 & 1.01 & 1.86 \\
\hline
28 & 1.33 & 1.48 & 1.26 & 1.56 & 1.18 & 1.65 & 1.10 & 1.75 & 1.03 & 1.85 \\
\hline
29 & 1.34 & 1.48 & 1.27 & 1.56 & 1.20 & 1.65 & 1.12 & 1.74 & 1.05 & 1.84 \\
\hline
30 & 1.35 & 1.49 & 1.28 & 1.57 & 1.21 & 1.65 & 1.14 & 1.74 & 1.07 & 1.83 \\
\hline
31 & 1.36 & 1.50 & 1.30 & 1.57 & 1.23 & 1.65 & 1.16 & 1.74 & 1.09 & 1.83 \\
\hline
32 & 1.37 & 1.50 & 1.31 & 1.57 & 1.24 & 1.65 & 1.18 & 1.73 & 1.11 & 1.82 \\
\hline
33 & 1.38 & 1.51 & 1.32 & 1.58 & 1.26 & 1.65 & 1.19 & 1.73 & 1.13 & 1.81 \\
\hline
34 & 1.39 & 1.51 & 1.33 & 1.58 & 1.27 & 1.65 & 1.21 & 1.73 & 1.15 & 1.81 \\
\hline
35 & 1.40 & 1.52 & 1.34 & 1.58 & 1.28 & 1.65 & 1.22 & 1.73 & 1.16 & 1.80 \\
\hline
36 & 1.41 & 1.52 & 1.35 & 1.59 & 1.29 & 1.65 & 1.24 & 1.73 & 1.18 & 1.80 \\
\hline
37 & 1.42 & 1.53 & 1.36 & 1.59 & 1.31 & 1.66 & 1.25 & 1.72 & 1.19 & 1.80 \\
\hline
38 & 1.43 & 1.54 & 1.37 & 1.59 & 1.32 & 1.66 & 1.26 & 1.72 & 1.21 & 1.79 \\
\hline
39 & 1.43 & 1.54 & 1.38 & 1.60 & 1.33 & 1.66 & 1.27 & 1.72 & 1.22 & 1.79 \\
\hline
40 & 1.44 & 1.54 & 1.39 & 1.60 & 1.34 & 1.66 & 1.29 & 1.72 & 1.23 & 1.79 \\
\hline
45 & 1.48 & 1.57 & 1.43 & 1.62 & 1.38 & 1.67 & 1.34 & 1.72 & 1.29 & 1.78 \\
\hline
50 & 1.50 & 1.59 & 1.46 & 1.63 & 1.42 & 1.67 & 1.38 & 1.72 & 1.34 & 1.77 \\
\hline
55 & 1.53 & 1.60 & 1.49 & 1.64 & 1.45 & 1.68 & 1.41 & 1.72 & 1.38 & 1.77 \\
\hline
60 & 1.55 & 1.62 & 1.51 & 1.65 & 1.48 & 1.69 & 1.44 & 1.73 & 1.41 & 1.77 \\
\hline
65 & 1.57 & 1.63 & 1.54 & 1.66 & 1.50 & 1.70 & 1.47 & 1.73 & 1.44 & 1.77 \\
\hline
70 & 1.58 & 1.64 & 1.55 & 1.67 & 1.52 & 1.70 & 1.49 & 1.74 & 1.46 & 1.77 \\
\hline
75 & 1.60 & 1.65 & 1.57 & 1.68 & 1.54 & 1.71 & 1.51 & 1.74 & 1.49 & 1.77 \\
\hline
80 & 1.61 & 1.66 & 1.59 & 1.69 & 1.56 & 1.72 & 1.53 & 1.74 & 1.51 & 1.77 \\
\hline
85 & 1.62 & 1.67 & 1.60 & 1.70 & 1.57 & 1.72 & 1.55 & 1.75 & 1.52 & 1.77 \\
\hline
90 & 1.63 & 1.68 & 1.61 & 1.70 & 1.59 & 1.73 & 1.57 & 1.75 & 1.54 & 1.78 \\
\hline
95 & 1.64 & 1.69 & 1.62 & 1.71 & 1.60 & 1.73 & 1.58 & 1.75 & 1.56 & 1.78 \\
\hline
100 & 1.65 & 1.69 & 1.63 & 1.72 & 1.61 & 1.74 & 1.59 & 1.76 & 1.57 & 1.78 \\
\hline
\end{tabular}
\end{center}

Note: $K=$ the number of slope parameters in the model.

Source: From J. Durbin and G. S. Watson, “Testing for Serial Correlation in Least Squares Regression, II. Biometrika 38 (1951): 159-178.


\end{document}