\documentclass[10pt]{article}
\usepackage[utf8]{inputenc}
\usepackage[T1]{fontenc}
\usepackage{amsmath}
\usepackage{amsfonts}
\usepackage{amssymb}
\usepackage[version=4]{mhchem}
\usepackage{stmaryrd}
\usepackage{hyperref}
\hypersetup{colorlinks=true, linkcolor=blue, filecolor=magenta, urlcolor=cyan,}
\urlstyle{same}
\usepackage{graphicx}
\usepackage[export]{adjustbox}
\graphicspath{ {./images/} }
\usepackage{multirow}

\title{FIXED INCOME, DERIVATIVES, ALTERNATIVE INVESTMENTS, PORTFOLIO MANAGEMENT }


\author{by Owen M. Concannon, CFA, Robert M. Conroy, DBA, CFA, Alistair\\
Byrne, PhD, CFA, and Vahan Janjigian, PhD, CFA.}
\date{}


\DeclareUnicodeCharacter{00D7}{$\times$}

\begin{document}
\maketitle
\section{CFA $^{\circledR}$ Program Curriculum $2023 \cdot$ LEVEL 1 • VOLUME 5}
@2022 by CFA Institute. All rights reserved. This copyright covers material written expressly for this volume by the editor/s as well as the compilation itself. It does not cover the individual selections herein that first appeared elsewhere. Permission to reprint these has been obtained by CFA Institute for this edition only. Further reproductions by any means, electronic or mechanical, including photocopying and recording, or by any information storage or retrieval systems, must be arranged with the individual copyright holders noted.

CFA $^{\circ}$, Chartered Financial Analyst ${ }^{\circ}$, AIMR-PPS $^{\circ}$, and GIPS ${ }^{\circ}$ are just a few of the trademarks owned by CFA Institute. To view a list of CFA Institute trademarks and the Guide for Use of CFA Institute Marks, please visit our website at \href{http://www.cfainstitute.org}{www.cfainstitute.org}.

This publication is designed to provide accurate and authoritative information in regard to the subject matter covered. It is sold with the understanding that the publisher is not engaged in rendering legal, accounting, or other professional service. If legal advice or other expert assistance is required, the services of a competent professional should be sought.

All trademarks, service marks, registered trademarks, and registered service marks are the property of their respective owners and are used herein for identification purposes only.

ISBN 978-1-953337-02-3 (paper)

ISBN 978-1-953337-27-6 (ebook)

2022

\section{CONTENTS}
How to Use the CFA Program Curriculum

Errata

Designing Your Personal Study Program

CFA Institute Learning Ecosystem (LES)

Feedback

Feedback

Fixed Income

Learning Module 1

Understanding Fixed-Income Risk and Return

Introduction 4

Sources of Return $\quad 4$

Macaulay and Modified Duration $\quad 11$

Macaulay, Modified, and Approximate Duration $\quad 11$

Approximate Modified and Macaulay Duration 16

Effective and Key Rate Duration $\quad 18$

Key Rate Duration $\quad 21$

Properties of Bond Duration $\quad 22$

Duration of a Bond Portfolio $\quad 28$

Money Duration and the Price Value of a Basis Point $\quad 30$

Bond Convexity $\quad 32$

Investment Horizon, Macaulay Duration and Interest Rate Risk $\quad 40$

$\begin{array}{lr}\text { Yield Volatility } & 40\end{array}$

Investment Horizon, Macaulay Duration, and Interest Rate Risk $\quad 42$

$\begin{array}{lr}\text { Credit and Liquidity Risk } & 46\end{array}$

Empirical Duration $\quad 48$

Summary 49

References $\quad 53$

Practice Problems $\quad 54$

$\begin{array}{ll}\text { Solutions } & 60\end{array}$

Learning Module $2 \quad$ Fundamentals of Credit Analysis $\quad 67$

$\begin{array}{ll}\text { Introduction } & 67\end{array}$

$\begin{array}{lr}\text { Credit Risk } & 68\end{array}$

Capital Structure, Seniority Ranking, and Recovery Rates $\quad 70$

$\begin{array}{ll}\text { Capital Structure } & 70\end{array}$

$\begin{array}{ll}\text { Seniority Ranking } & 71\end{array}$

Recovery Rates 72

Rating Agencies, Credit Ratings, and Their Role in the Debt Markets $\quad 76$

$\begin{array}{ll}\text { Credit Ratings } & 77\end{array}$

$\begin{array}{lr}\text { Issuer vs. Issue Ratings } & 78\end{array}$

$\begin{array}{ll}\text { ESG Ratings } & 79\end{array}$

$\begin{array}{lr}\text { Risks in Relying on Agency Ratings } & 80\end{array}$

Traditional Credit Analysis: Corporate Debt Securities 85

$\begin{array}{lr}\text { Credit Analysis vs. Equity Analysis: Similarities and Differences } & 85\end{array}$

The Four Cs of Credit Analysis: A Useful Framework 86 Credit Risk vs. Return: Yields and Spreads

Non-Sovereign Government Debt $\quad 123$

Summary 125

$\begin{array}{lr}\text { Practice Problems } & 130\end{array}$

$\begin{array}{ll}\text { Solutions } & 141\end{array}$

\section{Derivatives}
Learning Module 1

Derivative Instrument and Derivative Market Features

$\begin{array}{lr}\text { Introduction } & 149\end{array}$

$\begin{array}{lr}\text { Summary } & 149\end{array}$

$\begin{array}{lr}\text { Derivative Features } & 151\end{array}$

$\begin{array}{lr}\text { Definition and Features of a Derivative } & 151\end{array}$

Derivative Underlyings $\quad 154$

Equities 154

Fixed-Income Instruments $\quad 154$

$\begin{array}{lr}\text { Currencies } & 155\end{array}$

$\begin{array}{lr}\text { Commodities } & 155\end{array}$

$\begin{array}{lr}\text { Credit } & 155\end{array}$

Other 156

Investor Scenarios $\quad 156$

Derivative Markets $\quad 159$

$\begin{array}{lr}\text { Over-the-Counter (OTC) Derivative Markets } & 159\end{array}$

$\begin{array}{lr}\text { Exchange-Traded Derivative (ETD) Markets } & 160\end{array}$

\begin{center}
\begin{tabular}{ll}
Central Clearing & 161 \\
\hline
Extestor Scenariss & 162 \\
\hline
\end{tabular}
\end{center}

Investor Scenarios $\quad 162$

$\begin{array}{lr}\text { Practice Problems } & 165\end{array}$

$\begin{array}{ll}\text { Solutions } & 167\end{array}$

\section{Learning Module 2}
Forward Commitment and Contingent Claim Features and Instruments

$\begin{array}{lr}\text { Forwards } & 171\end{array}$

$\begin{array}{lr}\text { Futures } & 174\end{array}$

$\begin{array}{lr}\text { Swaps } & 180\end{array}$

Options 183

$\begin{array}{lr}\text { Scenario 1: } \text { Transact }\left(S_{T}>X\right) & 183\end{array}$

Scenario 2: Do Not Transact $\left(S_{T}<X\right) \quad 183$

$\begin{array}{lr}\text { Credit Derivatives } & 188\end{array}$

$\begin{array}{ll}\text { Forward commitments vs contingent claims } & 191\end{array}$

$\begin{array}{lr}\text { Practice Problems } & 195\end{array}$

Solutions 197 Learning Module 4

Learning Module 5

Learning Module 6

Learning Module 7

Learning Module 8 Summary 199

Derivative Benefits $\quad 201$

Derivative Risks $\quad 209$

Issuer Use of Derivatives $\quad 213$

Investor Use of Derivatives $\quad 216$

$\begin{array}{ll}\text { Practice Problems } & 219\end{array}$

Solutions $\quad 221$

Arbitrage, Replication, and the Cost of Carry in Pricing Derivatives 223

Introduction 223

Summary 223

Arbitrage $\quad 226$

Replication $\quad 229$

Costs and Benefits Associated with Owning the Underlying 234

Pricing and Valuation of Forward Contracts and for an Underlying with Varying Maturities $\quad 247$

$\begin{array}{lr}\text { Introduction } & 247\end{array}$

Summary $\quad 247$

Pricing and Valuation of Forward Commitments $\quad 250$

$\begin{array}{lr}\text { Pricing versus Valuation of Forward Contracts } & 250\end{array}$

Pricing and Valuation of Interest Rate Forward Contracts $\quad 261$

$\begin{array}{ll}\text { Interest Rate Forward Contracts } & 261\end{array}$

$\begin{array}{lr}\text { Practice Problems } & 273\end{array}$

$\begin{array}{ll}\text { Solutions } & 275\end{array}$

Pricing and Valuation of Futures Contracts $\quad 277$

$\begin{array}{ll}\text { Introduction } & 277\end{array}$

Summary $\quad 277$

Pricing of Futures Contracts at Inception $\quad 280$

MTM Valuation: Forwards versus Futures $\quad 282$

Interest Rate Futures versus Forward Contracts $\quad 284$

Forward and Futures Price Differences $\quad 288$

Interest Rate Forward and Futures Price Differences $\quad 289$

Effect of Central Clearing of OTC Derivatives $\quad 290$

Practice Problems 293

Solutions $\quad 295$

Pricing and Valuation of Interest Rates and Other Swaps 297

Introduction $\quad 297$

Summary $\quad 297$

$\begin{array}{lr}\text { Swaps vs. Forwards } & 300\end{array}$

$\begin{array}{lr}\text { Swap Values and Prices } & 307\end{array}$

Practice Problems $\quad 313$

$\begin{array}{ll}\text { Solutions } & 315\end{array}$

Pricing and Valuation of Options 317

$\begin{array}{lr}\text { Introduction } & 317\end{array}$

Summary 317

0 indicates an optional segment Option Value relative to the Underlying Spot Price $\quad 320$

$\begin{array}{ll}\text { Option Exercise Value } & 321\end{array}$

$\begin{array}{lr}\text { Option Moneyness } & 321\end{array}$

Option Time Value $\quad 323$

Arbitrage $\quad 325$

$\begin{array}{lr}\text { Replication } & 327\end{array}$

Factors Affecting Option Value $\quad 331$

$\begin{array}{ll}\text { Value of the Underlying } & 331\end{array}$

Exercise Price $\quad 332$

Time to Expiration $\quad 333$

Risk-Free Interest Rate $\quad 334$

Volatility of the Underlying 334

Income or Cost Related to Owning Underlying Asset 334

Practice Problems $\quad 337$

Solutions $\quad 339$

Learning Module $9 \quad$ Option Replication Using Put-Call Parity 341

$\begin{array}{lr}\text { Introduction } & 341\end{array}$

Summary 341

$\begin{array}{lr}\text { Put-Call Parity } & 343\end{array}$

$\begin{array}{ll}\text { Option Strategies Based on Put-Call Parity } & 348\end{array}$

$\begin{array}{ll}\text { Put-Call Forward Parity and Option Applications } & 352\end{array}$

Put-Call Forward Parity $\quad 352$

Option Put-Call Parity Applications: Firm Value 354

Learning Module 10

Valuing a Derivative Using a One-Period Binomial Model

Introduction

$\begin{array}{ll}\text { Summary } & 359\end{array}$

$\begin{array}{lr}\text { Binomial Valuation } & 361\end{array}$

The Binomial Model $\quad 362$

Pricing a European Call Option $\quad 363$

$\begin{array}{lr}\text { Risk Neutrality } & 371\end{array}$

$\begin{array}{lr}\text { Practice Problems } & 375\end{array}$

$\begin{array}{ll}\text { Solutions } & 377\end{array}$

\section{Alternative Investments}
Learning Module 1

Categories, Characteristics, and Compensation Structures of Alternative Investments

Advantages and Disadvantages of Direct Investing, Co-Investing, and Fund Investing

Due Diligence for Fund Investing, Direct Investing, and Co-Investing $\quad 388$

Investment and Compensation Structures 389 Learning Module 2

\section{Learning Module 3}
Private Capital, Real Estate, Infrastructure, Natural Resources, and Hedge Funds

$\begin{array}{ll}\text { Summary—Private Capital } & 429\end{array}$

$\begin{array}{lr}\text { Introduction and Overview of Private Capital } & 430\end{array}$

$\begin{array}{lr}\text { Description of Private Equity } & 430\end{array}$

Description of Private Debt 436

Risk and Return Characteristics of Private Capital 439

Diversification Benefits of Private Capital $\quad 441$

Real Estate $\quad 442$

Summary-Real Estate $\quad 442$

Introduction and Overview of Real Estate $\quad 443$

Description of Real Estate $\quad 445$

Forms of Real Estate Investing 447

Risk and Return Characteristics of Real Estate $\quad 451$

Diversification Benefits of Real Estate $\quad 454$

Infrastructure 456

Summary-Infrastructure 456

Introduction and Overview of Infrastructure $\quad 456$

Description of Infrastructure $\quad 458$

Risk and Return Characteristics of Infrastructure $\quad 460$

Diversification Benefits of Infrastructure $\quad 463$

$\begin{array}{ll}\text { Natural Resources } & 465\end{array}$

Summary—Natural Resources $\quad 465$

Introduction and Overview of Natural Resources $\quad 465$

Description of Commodities $\quad 466$

Description of Timberland and Farmland $\quad 469$

$\begin{array}{lc}\text { Risk and Return Characteristics of Natural Resources } & 471 \\ \text { Diversification Benefits of Natural Resources } & 473 \\ \text { Hedge Funds } & 478 \\ \text { Summary-Hedge Funds } & 478 \\ \text { Introduction and Overview of Hedge Funds } & 479 \\ \text { Description of Hedge Funds } & 480 \\ \text { Forms of Hedge Fund Investments } & 485 \\ \text { Risk and Return Characteristics of Hedge Funds } & 486 \\ \text { Diversification Benefits of Hedge Funds } & 487 \\ \text { Practice Problems } & 489 \\ \text { Solutions } & 493\end{array}$

\section{Portfolio Management}
\section{Learning Module 1}
Portfolio Management: An Overview 499

Introduction 499

$\begin{array}{ll}\text { Portfolio Perspective: Diversification and Risk Reduction } & 500\end{array}$

Historical Example of Portfolio Diversification: Avoiding Disaster 500

$\begin{array}{ll}\text { Portfolios: Reduce Risk } & 502\end{array}$

Portfolio Perspective: Risk-Return Trade-off, Downside Protection, Modern

Portfolio Theory

505

Historical Portfolio Example: Not Necessarily Downside Protection $\quad 505$

$\begin{array}{lr}\text { Portfolios: Modern Portfolio Theory } & 508\end{array}$

$\begin{array}{ll}\text { Steps in the Portfolio Management Process } & 509\end{array}$

$\begin{array}{lr}\text { Step One: The Planning Step } & 509\end{array}$

$\begin{array}{lr}\text { Step Two: The Execution Step } & 510\end{array}$

$\begin{array}{lr}\text { Step Three: The Feedback Step } & 512\end{array}$

$\begin{array}{lr}\text { Types of Investors } & 513\end{array}$

Individual Investors $\quad 513$

$\begin{array}{lr}\text { Institutional Investors } & 514\end{array}$

$\begin{array}{lr}\text { The Asset Management Industry } & 519\end{array}$

$\begin{array}{lr}\text { Active versus Passive Management } & 520\end{array}$

$\begin{array}{ll}\text { Traditional versus Alternative Asset Managers } & 521\end{array}$

$\begin{array}{lr}\text { Ownership Structure } & 521\end{array}$

Asset Management Industry Trends $\quad 521$

$\begin{array}{lr}\text { Pooled Interest - Mutual Funds } & 524\end{array}$

Mutual Funds $\quad 524$

$\begin{array}{lr}\text { Pooled Interest - Type of Mutual Funds } & 526\end{array}$

Money Market Funds $\quad 526$

Bond Mutual Funds $\quad 526$

$\begin{array}{lr}\text { Stock Mutual Funds } & 527\end{array}$

Hybrid/Balanced Funds $\quad 527$

Pooled Interest - Other Investment Products $\quad 528$

Exchange-Traded Funds $\quad 528$

$\begin{array}{lr}\text { Hedge Funds } & 529\end{array}$

$\begin{array}{lr}\text { Private Equity and Venture Capital Funds } & 529\end{array}$

Summary $\quad 530$

References $\quad 531$

Practice Problems 532 Solutions

Money-Weighted Return or Internal Rate of Return 540

Time-Weighted Rate of Return $\quad 545$

Annualized Return 549

Other Major Return Measures and their Applications 551 Gross and Net Return $\quad 551$

Pre-tax and After-tax Nominal Return $\quad 552$

Real Returns $\quad 552$

$\begin{array}{ll}\text { Leveraged Return } & 554\end{array}$

Historical Return and Risk 554

Historical Mean Return and Expected Return 554

Nominal Returns of Major US Asset Classes $\quad 556$

Real Returns of Major US Asset Classes $\quad 556$

Nominal and Real Returns of Asset Classes in Major Countries 558

$\begin{array}{ll}\text { Risk of Major Asset Classes } & 558\end{array}$

$\begin{array}{lr}\text { Risk-Return Trade-off } & 558\end{array}$

$\begin{array}{ll}\text { Other Investment Characteristics } & 559\end{array}$

$\begin{array}{lr}\text { Distributional Characteristics } & 559\end{array}$

$\begin{array}{ll}\text { Market Characteristics } & 561\end{array}$

Risk Aversion and Portfolio Selection $\quad 562$

The Concept of Risk Aversion $\quad 562$

Utility Theory and Indifference Curves 563

Indifference Curves $\quad 564$

Application of Utility Theory to Portfolio Selection 568

Portfolio Risk \& Portfolio of Two Risky Assets $\quad 571$

Portfolio of Two Risky Assets $\quad 571$

Portfolio of Many Risky Assets $\quad 580$

Importance of Correlation in a Portfolio of Many Assets $\quad 581$

The Power of Diversification $\quad 581$

Correlation and Risk Diversification $\quad 583$

Historical Risk and Correlation 583

Historical Correlation among Asset Classes $\quad 584$

Avenues for Diversification $\quad 584$

Efficient Frontier: Investment Opportunity Set \& Minimum Variance

Portfolios

Investment Opportunity Set $\quad 587$

Minimum-Variance Portfolios $\quad 588$

Efficient Frontier: A Risk-Free Asset and Many Risky Assets 590

Capital Allocation Line and Optimal Risky Portfolio $\quad 590$

$\begin{array}{ll}\text { The Two-Fund Separation Theorem } & 591\end{array}$

Efficient Frontier: Optimal Investor Portfolio 593

Investor Preferences and Optimal Portfolios $\quad 598$

$\begin{array}{ll}\text { Summary } & 598\end{array}$

$\begin{array}{lr}\text { Practice Problems } & 600\end{array}$ $\begin{array}{ll}\text { Solutions } & 670\end{array}$

\section{How to Use the CFA Program Curriculum}
The CFA Program exams measure your mastery of the core knowledge, skills, and abilities required to succeed as an investment professional. These core competencies are the basis for the Candidate Body of Knowledge $\left(\mathrm{CBOK}^{\mathrm{m}}\right)$ ). The CBOK consists of four components:

\begin{itemize}
  \item A broad outline that lists the major CFA Program topic areas (www. \href{http://cfainstitute.org/programs/cfa/curriculum/cbok}{cfainstitute.org/programs/cfa/curriculum/cbok})

  \item Topic area weights that indicate the relative exam weightings of the top-level topic areas (\href{http://www.cfainstitute.org/programs/cfa/curriculum}{www.cfainstitute.org/programs/cfa/curriculum})

  \item Learning outcome statements (LOS) that advise candidates about the specific knowledge, skills, and abilities they should acquire from curriculum content covering a topic area: LOS are provided in candidate study sessions and at the beginning of each block of related content and the specific lesson that covers them. We encourage you to review the information about the LOS on our website (\href{http://www.cfainstitute.org/programs/cfa/curriculum/}{www.cfainstitute.org/programs/cfa/curriculum/} study-sessions), including the descriptions of LOS "command words" on the candidate resources page at \href{http://www.cfainstitute.org}{www.cfainstitute.org}.

  \item The CFA Program curriculum that candidates receive upon exam registration

\end{itemize}

Therefore, the key to your success on the CFA exams is studying and understanding the CBOK. You can learn more about the CBOK on our website: www.cfainstitute. org/programs/cfa/curriculum/cbok.

The entire curriculum, including the practice questions, is the basis for all exam questions and is selected or developed specifically to teach the knowledge, skills, and abilities reflected in the CBOK.

\section{ERRATA}
The curriculum development process is rigorous and includes multiple rounds of reviews by content experts. Despite our efforts to produce a curriculum that is free of errors, there are instances where we must make corrections. Curriculum errata are periodically updated and posted by exam level and test date online on the Curriculum Errata webpage (\href{http://www.cfainstitute.org/en/programs/submit-errata}{www.cfainstitute.org/en/programs/submit-errata}). If you believe you have found an error in the curriculum, you can submit your concerns through our curriculum errata reporting process found at the bottom of the Curriculum Errata webpage.

\section{DESIGNING YOUR PERSONAL STUDY PROGRAM}
An orderly, systematic approach to exam preparation is critical. You should dedicate a consistent block of time every week to reading and studying. Review the LOS both before and after you study curriculum content to ensure that you have mastered the applicable content and can demonstrate the knowledge, skills, and abilities described by the LOS and the assigned reading. Use the LOS self-check to track your progress and highlight areas of weakness for later review.

Successful candidates report an average of more than 300 hours preparing for each exam. Your preparation time will vary based on your prior education and experience, and you will likely spend more time on some study sessions than on others.

\section{CFA INSTITUTE LEARNING ECOSYSTEM (LES)}
Your exam registration fee includes access to the CFA Program Learning Ecosystem (LES). This digital learning platform provides access, even offline, to all of the curriculum content and practice questions and is organized as a series of short online lessons with associated practice questions. This tool is your one-stop location for all study materials, including practice questions and mock exams, and the primary method by which CFA Institute delivers your curriculum experience. The LES offers candidates additional practice questions to test their knowledge, and some questions in the LES provide a unique interactive experience.

\section{FEEDBACK}
Please send any comments or feedback to \href{mailto:info@cfainstitute.org}{info@cfainstitute.org}, and we will review your suggestions carefully.

\section{Fixed Income}
\section{LEARNING MODULE
1}
\section{Understanding Fixed-Income Risk and Return}
by James F. Adams, PhD, CFA, and Donald J. Smith, PhD.

James F. Adams, PhD, CFA, is at New York University (USA). Donald J. Smith, PhD, is at Boston University Questrom School of Business (USA).

\section{LEARNING OUTCOME}
\begin{center}
\begin{tabular}{c|l}
Mastery & The candidate should be able to: \\
$\square$ & $\begin{array}{l}\text { calculate and interpret the sources of return from investing in a } \\ \text { fixed-rate bond } \\ \text { define, calculate, and interpret Macaulay, modified, and effective } \\ \text { durations } \\ \text { explain why effective duration is the most appropriate measure of } \\ \text { interest rate risk for bonds with embedded options } \\ \text { define key rate duration and describe the use of key rate durations } \\ \text { in measuring the sensitivity of bonds to changes in the shape of the } \\ \text { benchmark yield curve } \\ \text { explain how a bond's maturity, coupon, and yield level affect its } \\ \text { interest rate risk } \\ \text { calculate the duration of a portfolio and explain the limitations of } \\ \text { portfolio duration } \\ \text { calculate and interpret the money duration of a bond and price value } \\ \text { of a basis point (PVBP) } \\ \text { calculate and interpret approximate convexity and compare } \\ \text { approximate and effective convexity } \\ \text { calculate the percentage price change of a bond for a specified } \\ \text { change in yield, given the bond's approximate duration and convexity } \\ \text { describe how the term structure of yield volatility affects the interest } \\ \text { rate risk of a bond } \\ \text { describe the relationships among a bond's holding period return, its } \\ \text { duration, and the investment horizon } \\ \text { explain how changes in credit spread and liquidity affect } \\ \text { yield-to-maturity of a bond and how duration and convexity can be } \\ \text { used to estimate the price effect of the changes } \\ \text { describe the difference between empirical duration and analytical } \\ \text { duration }\end{array}$ \\
$\square$ &  \\
\end{tabular}
\end{center}\$\textbackslash square l \textbackslash begin\{aligned\} \& \textbackslash square \textbackslash 

\& \textbackslash square\textbackslash end\{aligned\}\$

\section{INTRODUCTION}
Successful analysts must develop a solid understanding of the risk and return characteristics of fixed-income investments. Beyond the vast global market for public and private fixed-rate bonds, many financial assets and liabilities with known future cash flows you will encounter throughout your career are evaluated using similar principles. This analysis starts with the yield-to-maturity, or internal rate of return on future cash flows, introduced in the fixed-income valuation reading. Fixed-rate bond returns are affected by many factors, the most important of which is the full receipt of all interest and principal payments on scheduled dates. Assuming no default, return is also affected by interest rate changes that affect coupon reinvestment and the bond price if it is sold prior to maturity. Price change measures may be derived from the mathematical relationship used to calculate a bond's price. Specifically, duration estimates the price change for a given change in interest rates, and convexity improves on the duration estimate by considering that the price and yield-to-maturity relationship of a fixed-rate bond is non-linear.

Sources of return on a fixed-rate bond investment include the receipt and reinvestment of coupon payments and either the redemption of principal if the bond is held to maturity or capital gains (or losses) if the bond is sold earlier. Fixed-income investors holding the same bond may have different interest rate risk exposures if their investment horizons differ.

We introduce bond duration and convexity, showing how these statistics are calculated and used as interest rate risk measures. Although procedures and formulas exist to calculate duration and convexity, these statistics can be approximated using basic bond-pricing techniques and a financial calculator. Commonly used versions of the statistics are covered, including Macaulay, modified, effective, and key rate durations, and we distinguish between risk measures based on changes in the bond's yield-to-maturity (i.e., yield duration and convexity) and on benchmark yield curve changes (i.e., curve duration and convexity).

We then return to the investment time horizon. When an investor has a short-term horizon, duration and convexity are used to estimate the change in the bond price. Note that yield volatility matters, because bonds with varying times-to-maturity have different degrees of yield volatility. When an investor has a long-term horizon, the interaction between coupon reinvestment risk and market price risk matters. The relationship among interest rate risk, bond duration, and the investment horizon is explored.

Finally, we discuss how duration and convexity may be extended to credit and liquidity risks and highlight how these factors can affect a bond's return and risk. In addition, we highlight the use of statistical methods and historical data to establish empirical as opposed to analytical duration estimates.

\section{SOURCES OF RETURN}
calculate and interpret the sources of return from investing in a fixed-rate bond

Fixed-rate bond investors have three sources of return: (1) receipt of promised coupon and principal payments on the scheduled dates, (2) reinvestment of coupon payments, and (3) potential capital gains or losses on the sale of the bond prior to maturity. In this section, it is assumed that the issuer makes the coupon and principal payments as scheduled. Here, the focus is primarily on how interest rate changes affect the reinvestment of coupon payments and a bond's market price if sold prior to maturity. Credit risk is considered later and is also the primary subject of a subsequent reading.

When a bond is purchased at a premium or a discount, it adds another aspect to the rate of return. Recall from the fixed-income valuation reading that a discount bond offers the investor a "deficient" coupon rate below the market discount rate. The amortization of this discount in each period brings the return in line with the market discount rate as the bond's carrying value is "pulled to par." For a premium bond, the coupon rate exceeds the market discount rate and the amortization of the premium adjusts the return to match the market discount rate. Through amortization, the bond's carrying value reaches par value at maturity.

A series of examples will demonstrate the effect of a change in interest rates on two investors' realized rate of returns. Interest rates are the rates at which coupon payments are reinvested and the market discount rates at the time of purchase and at the time of sale if the bond is not held to maturity. In Example 1 and Example 2, interest rates are unchanged. The two investors, however, have different time horizons for holding the bond. Example 3 and Example 4 show the impact of higher interest rates on the two investors' total return. Example 5 and Example 6 show the impact of lower interest rates. In each of the six examples, an investor initially buys a 10-year, $8 \%$ annual coupon payment bond at a price of 85.503075 per 100 of par value. The bond's yield-to-maturity is $10.40 \%$.

$$
\begin{aligned}
& 85.503075=\frac{8}{(1+r)^{1}}+\frac{8}{(1+r)^{2}}+\frac{8}{(1+r)^{3}}+\frac{8}{(1+r)^{4}}+\frac{8}{(1+r)^{5}}+ \\
& \frac{8}{(1+r)^{6}}+\frac{8}{(1+r)^{7}}+\frac{8}{(1+r)^{8}}+\frac{8}{(1+r)^{9}}+\frac{108}{(1+r)^{10}}, \\
& r=0.1040
\end{aligned}
$$

\section{EXAMPLE 1}
A "buy-and-hold" investor purchases a 10-year, $8 \%$ annual coupon payment bond at 85.503075 per 100 of par value and holds it until maturity. The investor receives the series of 10 coupon payments of 8 (per 100 of par value) for a total of 80, plus the redemption of principal (100) at maturity. In addition to collecting the coupon interest and the principal, the investor may reinvest the cash flows. If the coupon payments are reinvested at $10.40 \%$, the future value of the coupons on the bond's maturity date is 129.970678 per 100 of par value.

$$
\begin{aligned}
& {\left[8 \times(1.1040)^{9}\right]+\left[8 \times(1.1040)^{8}\right]+\left[8 \times(1.1040)^{7}\right]+\left[8 \times(1.1040)^{6}\right]+} \\
& {\left[8 \times(1.1040)^{5}\right]+\left[8 \times(1.1040)^{4}\right]+\left[8 \times(1.1040)^{3}\right]+\left[8 \times(1.1040)^{2}\right]+} \\
& {\left[8 \times(1.1040)^{1}\right]+8=129.970678}
\end{aligned}
$$

The first coupon payment of 8 is reinvested at $10.40 \%$ for nine years until maturity, the second is reinvested for eight years, and so forth. The future value of the annuity is obtained easily on a financial calculator, using 8 for the payment that is received at the end of each of the 10 periods. The amount in excess of the coupons, $49.970678(=129.970678-80)$, is the "interest-on-interest" gain from compounding.

The investor's total return is 229.970678 , the sum of the reinvested coupons (129.970678) and the redemption of principal at maturity (100). The realized rate of return is $10.40 \%$.

$$
85.503075=\frac{229.970678}{(1+r)^{10}}, r=0.1040
$$

Example 1 demonstrates that the yield-to-maturity at the time of purchase measures the investor's rate of return under three assumptions: (1) The investor holds the bond to maturity, (2) there is no default by the issuer, and (3) the coupon interest payments are reinvested at that same rate of interest.

Example 2 considers another investor who buys the 10 -year, $8 \%$ annual coupon payment bond and pays the same price. This investor, however, has a four-year investment horizon. Therefore, coupon interest is only reinvested for four years, and the bond is sold immediately after receiving the fourth coupon payment.

\section{EXAMPLE 2}
A second investor buys the 10 -year, $8 \%$ annual coupon payment bond and sells the bond after four years. Assuming that the coupon payments are reinvested at $10.40 \%$ for four years, the future value of the reinvested coupons is 37.347111 per 100 of par value.

$$
\left[8 \times(1.1040)^{3}\right]+\left[8 \times(1.1040)^{2}\right]+\left[8 \times(1.1040)^{1}\right]+8=37.347111
$$

The interest-on-interest gain from compounding is 5.347111 (= 37.347111 - 32). After four years, when the bond is sold, it has six years remaining until maturity. If the yield-to-maturity remains $10.40 \%$, the sale price of the bond is 89.668770.

$$
\begin{aligned}
& \frac{8}{(1.1040)^{1}}+\frac{8}{(1.1040)^{2}}+\frac{8}{(1.1040)^{3}}+\frac{8}{(1.1040)^{4}}+ \\
& \frac{8}{(1.1040)^{5}}+\frac{8}{(1.1040)^{6}}=89.668770
\end{aligned}
$$

The total return is $127.015881(=37.347111+89.668770)$, and the realized rate of return is $10.40 \%$.

$$
85.503075=\frac{127.015881}{(1+r)^{4}}, r=0.1040
$$

In Example 2, the investor's horizon yield is $10.40 \%$. A horizon yield is the internal rate of return between the total return (the sum of reinvested coupon payments and the sale price or redemption amount) and the purchase price of the bond. The horizon yield on a bond investment is the annualized holding-period rate of return.

Example 2 demonstrates that the realized horizon yield matches the original yield-to-maturity if: (1) coupon payments are reinvested at the same interest rate as the original yield-to-maturity, and (2) the bond is sold at a price on the constant-yield price trajectory, which implies that the investor does not have any capital gains or losses when the bond is sold.

Capital gains arise if a bond is sold at a price above its constant-yield price trajectory and capital losses occur if a bond is sold at a price below its constant-yield price trajectory. This trajectory is based on the yield-to-maturity when the bond is purchased. The trajectory is shown in Exhibit 1 for a 10-year, 8\% annual payment bond purchased at a price of 85.503075 per 100 of par value. Exhibit 1: Constant-Yield Price Trajectory for a 10-Year, 8\% Annual Payment Bond

\begin{center}
\includegraphics[max width=\textwidth]{2023_05_04_36535b8d80b32081d422g-019}
\end{center}

Note: Price is price per 100 of par value.

A point on the trajectory represents the carrying value of the bond at that time. The carrying value is the purchase price plus the amortized amount of the discount if the bond is purchased at a price below par value. If the bond is purchased at a price above par value, the carrying value is the purchase price minus the amortized amount of the premium.

The amortized amount for each year is the change in the price between two points on the trajectory. The initial price of the bond is 85.503075 per 100 of par value. Its price (the carrying value) after one year is 86.395394 , calculated using the original yield-to-maturity of $10.40 \%$. Therefore, the amortized amount for the first year is 0.892320 (= 86.395394 - 85.503075). The bond price in Example 2 increases from 85.503075 to 89.668770 , and that increase over the four years is movement along the constant-yield price trajectory. At the time the bond is sold, its carrying value is also 89.668770, so there is no capital gain or loss.

Example 3 and Example 4 demonstrate the impact on investors' realized horizon yields if interest rates go up by 100 basis points (bps). The market discount rate on the bond increases from $10.40 \%$ to $11.40 \%$. Coupon reinvestment rates go up by 100 bps as well.

\section{EXAMPLE 3}
The buy-and-hold investor purchases the 10 -year, $8 \%$ annual payment bond at 85.503075. After the bond is purchased and before the first coupon is received, interest rates go up to $11.40 \%$. The future value of the reinvested coupons at $11.40 \%$ for 10 years is 136.380195 per 100 of par value.

$$
\begin{aligned}
& {\left[8 \times(1.1140)^{9}\right]+\left[8 \times(1.1140)^{8}\right]+\left[8 \times(1.1140)^{7}\right]+\left[8 \times(1.1140)^{6}\right]+} \\
& {\left[8 \times(1.1140)^{5}\right]+\left[8 \times(1.1140)^{4}\right]+\left[8 \times(1.1140)^{3}\right]+\left[8 \times(1.1140)^{2}\right]+} \\
& {\left[8 \times(1.1140)^{1}\right]+8=136.380195}
\end{aligned}
$$

The total return is $236.380195(=136.380195+100)$. The investor's realized rate of return is $10.70 \%$.

$$
85.503075=\frac{236.380195}{(1+r)^{10}}, r=0.1070
$$

In Example 3, the buy-and-hold investor benefits from the higher coupon reinvestment rate. The realized horizon yield is $10.70 \%, 30 \mathrm{bps}$ higher than the outcome in Example 1, when interest rates are unchanged. There is no capital gain or loss because the bond is held until maturity. The carrying value at the maturity date is par value, the same as the redemption amount.

\section{EXAMPLE 4}
The second investor buys the 10 -year, $8 \%$ annual payment bond at 85.503075 and sells it in four years. After the bond is purchased, interest rates go up to $11.40 \%$. The future value of the reinvested coupons at $11.40 \%$ after four years is 37.899724 per 100 of par value.

$$
\left[8 \times(1.1140)^{3}\right]+\left[8 \times(1.1140)^{2}\right]+\left[8 \times(1.1140)^{1}\right]+8=37.899724
$$

The sale price of the bond after four years is 85.780408 .

$$
\begin{aligned}
& \frac{8}{(1.1140)^{1}}+\frac{8}{(1.1140)^{2}}+\frac{8}{(1.1140)^{3}}+\frac{8}{(1.1140)^{4}}+ \\
& \frac{8}{(1.1140)^{5}}+\frac{108}{(1.1140)^{6}}=85.780408
\end{aligned}
$$

The total return is $123.680132(=37.899724+85.780408)$, resulting in a realized four-year horizon yield of $9.67 \%$.

$$
85.503075=\frac{123.680132}{(1+r)^{4}}, r=0.0967
$$

In Example 4, the second investor has a lower realized rate of return compared with the investor in Example 2, in which interest rates are unchanged. The future value of reinvested coupon payments goes up by 0.552613 (= $37.899724-37.347111)$ per 100 of par value because of the higher interest rates. But there is a capital loss of 3.888362 (= $89.668770-85.780408)$ per 100 of par value. Notice that the capital loss is measured from the bond's carrying value, the point on the constant-yield price trajectory, and not from the original purchase price. The bond is now sold at a price below the constant-yield price trajectory. The reduction in the realized four-year horizon yield from $10.40 \%$ to $9.67 \%$ is a result of the capital loss being greater than the gain from reinvesting coupons at a higher rate, which reduces the investor's total return.

Example 5 and Example 6 complete the series of rate-of-return calculations for the two investors. Interest rates decline by $100 \mathrm{bps}$. The required yield on the bond falls from $10.40 \%$ to $9.40 \%$ after the purchase of the bond. The interest rates at which the coupon payments are reinvested fall as well.

\section{EXAMPLE 5}
The buy-and-hold investor purchases the 10-year bond at 85.503075 and holds the security until it matures. After the bond is purchased and before the first coupon is received, interest rates go down to $9.40 \%$. The future value of reinvesting the coupon payments at $9.40 \%$ for 10 years is 123.888356 per 100 of par value.

$$
\begin{aligned}
& {\left[8 \times(1.0940)^{9}\right]+\left[8 \times(1.0940)^{8}\right]+\left[8 \times(1.0940)^{7}\right]+\left[8 \times(1.0940)^{6}\right]+} \\
& {\left[8 \times(1.0940)^{5}\right]+\left[8 \times(1.0940)^{4}\right]+\left[8 \times(1.0940)^{3}\right]+\left[8 \times(1.0940)^{2}\right]+} \\
& {\left[8 \times(1.0940)^{1}\right]+8=123.888356}
\end{aligned}
$$

The total return is 223.888356 , the sum of the future value of reinvested coupons and the redemption of par value. The investor's realized rate of return is $10.10 \%$.

$$
85.503075=\frac{223.888365}{(1+r)^{10}}, r=0.1010
$$

In Example 5, the buy-and-hold investor suffers from the lower coupon reinvestment rates. The realized horizon yield is $10.10 \%, 30 \mathrm{bps}$ lower than the result in Example 1 , when interest rates are unchanged. There is no capital gain or loss because the bond is held until maturity. Example 1, Example 3, and Example 5 indicate that the interest rate risk for a buy-and-hold investor arises entirely from changes in coupon reinvestment rates.

\section{EXAMPLE 6}
The second investor buys the 10-year bond at 85.503075 and sells it in four years. After the bond is purchased, interest rates go down to $9.40 \%$. The future value of the reinvested coupons at $9.40 \%$ is 36.801397 per 100 of par value.

$$
\left[8 \times(1.0940)^{3}\right]+\left[8 \times(1.0940)^{2}\right]+\left[8 \times(1.0940)^{1}\right]+8=36.801397
$$

This reduction in future value is offset by the higher sale price of the bond, which is 93.793912 per 100 of par value.

$$
\begin{aligned}
& \frac{8}{(1.0940)^{1}}+\frac{8}{(1.0940)^{2}}+\frac{8}{(1.0940)^{3}}+\frac{8}{(1.0940)^{4}}+ \\
& \frac{8}{(1.0940)^{5}}+\frac{108}{(1.0940)^{6}}=93.793912
\end{aligned}
$$

The total return is $130.595309(=36.801397+93.793912)$, and the realized yield is $11.17 \%$.

$$
85.503075=\frac{130.595309}{(1+r)^{4}}, \quad r=0.1117
$$

The investor in Example 6 has a capital gain of 4.125142 (= 93.793912 - 89.668770). The capital gain is measured from the carrying value, the point on the constant-yield price trajectory. That gain offsets the reduction in the future value of reinvested coupons of 0.545714 (= $37.347111-36.801397)$. The total return is higher than that in Example 2, in which the interest rate remains at $10.40 \%$.

In these examples, interest income for the investor is the return associated with the passage of time. Therefore, interest income includes the receipt of coupon interest, the reinvestment of those cash flows, and the amortization of the discount from purchase at a price below par value (or the premium from purchase at a price above par value) to bring the return back in line with the market discount rate. A capital gain or loss is the return to the investor associated with the change in the value of the security. On the fixed-rate bond, a change in value arises from a change in the yield-to-maturity, which is the implied market discount rate. In practice, the way interest income and capital gains and losses are calculated and reported on financial statements depends on financial and tax accounting rules.

This series of examples illustrates an important point about fixed-rate bonds: The investment horizon is at the heart of understanding interest rate risk and return. There are two offsetting types of interest rate risk that affect the bond investor: coupon reinvestment risk and market price risk. The future value of reinvested coupon payments (and, in a portfolio, the principal on bonds that mature before the horizon date) increases when interest rates rise and decreases when rates fall. The sale price on a bond that matures after the horizon date (and thus needs to be sold) decreases when interest rates rise and increases when rates fall. Coupon reinvestment risk matters more when the investor has a long-term horizon relative to the time-to-maturity of the bond. For instance, a buy-and-hold investor only has coupon reinvestment risk. Market price risk matters more when the investor has a short-term horizon relative to the time-to-maturity. For example, an investor who sells the bond before the first coupon is received has only market price risk. Therefore, two investors holding the same bond (or bond portfolio) can have different exposures to interest rate risk if they have different investment horizons.

\section{EXAMPLE 7}
\begin{enumerate}
  \item An investor buys a four-year, $10 \%$ annual coupon payment bond priced to yield $5.00 \%$. The investor plans to sell the bond in two years once the second coupon payment is received. Calculate the purchase price for the bond and the horizon yield assuming that the coupon reinvestment rate after the bond purchase and the yield-to-maturity at the time of sale are (1) $3.00 \%$, (2) $5.00 \%$, and (3) $7.00 \%$.
\end{enumerate}

\section{Solution:}
The purchase price is 117.729753.

$\frac{10}{(1.0500)^{1}}+\frac{10}{(1.0500)^{2}}+\frac{10}{(1.0500)^{3}}+\frac{110}{(1.0500)^{4}}=117.729753$

\begin{enumerate}
  \item $3.00 \%$ : The future value of reinvested coupons is 20.300 .
\end{enumerate}

$(10 \times 1.0300)+10=20.300$

The sale price of the bond is 113.394288 .

$\frac{10}{(1.0300)^{1}}+\frac{110}{(1.0300)^{2}}=113.394288$

Total return: $20.300+113.394288=133.694288$.

If interest rates go down from $5.00 \%$ to $3.00 \%$, the realized rate of return over the two-year investment horizon is $6.5647 \%$, higher than the original yield-to-maturity of $5.00 \%$.

$117.729753=\frac{133.694288}{(1+r)^{2}}, \quad r=0.065647$

\begin{enumerate}
  \setcounter{enumi}{1}
  \item $5.00 \%$ : The future value of reinvested coupons is 20.500 .
\end{enumerate}

$(10 \times 1.0500)+10=20.500$

The sale price of the bond is 109.297052 .

$\frac{10}{(1.0500)^{1}}+\frac{110}{(1.0500)^{2}}=109.297052$

Total return: $20.500+109.297052=129.797052$.

If interest rates remain $5.00 \%$ for reinvested coupons and for the required yield on the bond, the realized rate of return over the two-year investment horizon is equal to the yield-to-maturity of $5.00 \%$.

$117.729753=\frac{129.797052}{(1+r)^{2}}, \quad r=0.050000$

\begin{enumerate}
  \setcounter{enumi}{2}
  \item $7.00 \%$ : The future value of reinvested coupons is 20.700 .
\end{enumerate}

$(10 \times 1.0700)+10=20.700$ The bond is sold at 105.424055 .

$\frac{10}{(1.0700)^{1}}+\frac{110}{(1.0700)^{2}}=105.424055$

Total return: $20.700+105.424055=126.124055$

$117.729753=\frac{126.124055}{(1+r)^{2}}, \quad r=0.035037$

If interest rates go up from $5.00 \%$ to $7.00 \%$, the realized rate of return over the two-year investment horizon is $3.5037 \%$, lower than the yield-to-maturity of $5.00 \%$.

\section{MACAULAY AND MODIFIED DURATION}
define, calculate, and interpret Macaulay, modified, and effective durations

This section covers two commonly used measures of interest rate risk: duration and convexity. It distinguishes between risk measures based on changes in a bond's own yield-to-maturity (yield duration and convexity) and those that affect the bond based on changes in a benchmark yield curve (curve duration and convexity).

\section{Macaulay, Modified, and Approximate Duration}
The duration of a bond measures the sensitivity of the bond's full price (including accrued interest) to changes in the bond's yield-to-maturity or, more generally, to changes in benchmark interest rates. Duration estimates changes in the bond price assuming that variables other than the yield-to-maturity or benchmark rates are held constant. Most importantly, the time-to-maturity is unchanged. Therefore, duration measures the instantaneous (or, at least, same-day) change in the bond price. The accrued interest is the same, so it is the flat price that goes up or down when the full price changes. Duration is a useful measure because it represents the approximate amount of time a bond would have to be held for the market discount rate at purchase to be realized if there is a single change in interest rate. If the bond is held for the duration period, an increase from reinvesting coupons is offset by a decrease in price if interest rates increase and a decrease from reinvesting coupons is offset by an increase in price if interest rates decrease.

There are several types of bond duration. In general, these can be divided into yield duration and curve duration. Yield duration is the sensitivity of the bond price with respect to the bond's own yield-to-maturity. Curve duration is the sensitivity of the bond price (or more generally, the market value of a financial asset or liability) with respect to a benchmark yield curve. The benchmark yield curve could be the government yield curve on coupon bonds, the spot curve, or the forward curve, but in practice, the government par curve is often used. Yield duration statistics used in fixed-income analysis include Macaulay duration, modified duration, money duration, and the price value of a basis point (PVBP). A curve duration statistic often used is effective duration. Effective duration is covered later in this reading.

Macaulay duration is named after Frederick Macaulay, the Canadian economist who first wrote about the statistic in 1938. Equation 1 is a general formula to calculate the Macaulay duration (MacDur) of a traditional fixed-rate bond.

$$
\begin{aligned}
& \text { MacDur }= \\
& {\left[\frac{\frac{(1-t / T) \times P M T}{(1+r)^{1-t / T}}+\frac{(2-t / T) \times P M T}{(1+r)^{2-t / T}}+\cdots+\frac{(N-t / T) \times(P M T+F V)}{(1+r)^{N-t / T}}}{\frac{P M T}{(1+r)^{1-t / T}}+\frac{P M T}{(1+r)^{2-t / T}}+\cdots+\frac{P M T+F V}{(1+r)^{N-t / T}}}\right]}
\end{aligned}
$$

where

$t=$ the number of days from the last coupon payment to the settlement date

$T=$ the number of days in the coupon period

$t / T=$ the fraction of the coupon period that has gone by since the last payment

$P M T=$ the coupon payment per period

$F V=$ the future value paid at maturity, or the par value of the bond

$r=$ the yield-to-maturity, or the market discount rate, per period

$N=$ the number of evenly spaced periods to maturity as of the beginning of the current period

The denominator in Equation 1 is the full price $\left(P V^{F u l l}\right)$ of the bond including accrued interest. It is the present value of the coupon interest and principal payments, with each cash flow discounted by the same market discount rate, $r$.

$$
P V^{F u l l}=\frac{P M T}{(1+r)^{1-t / T}}+\frac{P M T}{(1+r)^{2-t / T}}+\cdots+\frac{P M T+F V}{(1+r)^{N-t / T}}
$$

Equation 3 combines Equation 1 and Equation 2 to reveal an important aspect of the Macaulay duration: Macaulay duration is a weighted average of the time to receipt of the bond's promised payments, where the weights are the shares of the full price that correspond to each of the bond's promised future payments.

$$
\text { MacDur }=\left\{\begin{array}{c}
(1-t / T)\left[\frac{\frac{P M T}{(1+r)^{1-t / T}}}{P V^{F u l l}}\right]+(2-t / T)\left[\frac{\frac{P M T}{(1+r)^{2-t / T}}}{P V^{F u l l}}\right]+\cdots+ \\
(N-t / T)\left[\frac{\frac{P M T+F V}{(1+r)^{N-t / T}}}{P V^{F u l l}}\right]
\end{array}\right\}
$$

The times to receipt of cash flow measured in terms of time periods are $1-t / T, 2$ $-t / T, \ldots, N-t / T$. The weights are the present values of the cash flows divided by the full price. Therefore, Macaulay duration is measured in terms of time periods. A couple of examples will clarify this calculation.

Consider first the 10-year, $8 \%$ annual coupon payment bond used in Examples 1-6. The bond's yield-to-maturity is $10.40 \%$, and its price is 85.503075 per 100 of par value. This bond has 10 evenly spaced periods to maturity. Settlement is on a coupon payment date so that $t / T=0$. Exhibit 2 illustrates the calculation of the bond's Macaulay duration.

\section{Exhibit 2: Macaulay Duration of a 10-Year, $8 \%$ Annual Payment Bond}
\begin{center}
\begin{tabular}{ccccc}
\hline
Period & Cash Flow & Present Value & Weight & Period $\times$ Weight \\
\hline
1 & 8 & 7.246377 & 0.08475 & 0.0847 \\
2 & 8 & 6.563747 & 0.07677 & 0.1535 \\
3 & 8 & 5.945423 & 0.06953 & 0.2086 \\
4 & 8 & 5.385347 & 0.06298 & 0.2519 \\
\end{tabular}
\end{center}

\begin{center}
\begin{tabular}{ccccc}
\hline
Period & Cash Flow & Present Value & Weight & Period $\times$ Weight \\
\hline
5 & 8 & 4.878032 & 0.05705 & 0.2853 \\
6 & 8 & 4.418507 & 0.05168 & 0.3101 \\
7 & 8 & 4.002271 & 0.04681 & 0.3277 \\
8 & 8 & 3.625245 & 0.04240 & 0.3392 \\
9 & 8.283737 & 0.03840 & 0.3456 &  \\
10 & 108 & 40.154389 & 0.46963 & 4.6963 \\
\cline { 3 - 4 }
 &  & 85.503075 & 1.00000 & 7.0029 \\
\hline
\end{tabular}
\end{center}

The first two columns of Exhibit 2 show the number of periods to the receipt of the cash flow and the amount of the payment per 100 of par value. The third column is the present value of the cash flow. For example, the final payment is 108 (the last coupon payment plus the redemption of principal) and its present value is 40.154389 .

$$
\frac{108}{(1.1040)^{10}}=40.154389
$$

The sum of the present values is the full price of the bond. The fourth column is the weight, the share of total market value corresponding to each cash flow. The final payment of 108 per 100 of par value is $46.963 \%$ of the bond's market value.

$$
\frac{40.154389}{85.503075}=0.46963
$$

The sum of the weights is 1.00000 . The fifth column is the number of periods to the receipt of the cash flow (the first column) multiplied by the weight (the fourth column). The sum of that column is 7.0029 , which is the Macaulay duration of this 10-year, $8 \%$ annual coupon payment bond. This statistic is sometimes reported as 7.0029 years, although the time frame is not needed in most applications.

Now consider an example between coupon payment dates. A 6\% semiannual payment corporate bond that matures on 14 February 2027 is purchased for settlement on 11 April 2019. The coupon payments are 3 per 100 of par value, paid on 14 February and 14 August of each year. The yield-to-maturity is $6.00 \%$ quoted on a street-convention semiannual bond basis. The full price of this bond comprises the flat price plus accrued interest. The flat price for the bond is 99.990423 per 100 of par value. The accrued interest is calculated using the $30 / 360$ method to count days. This settlement date is 57 days into the 180-day semiannual period, so $t / T=57 / 180$. The accrued interest is $0.950000(=57 / 180 \times 3)$ per 100 of par value. The full price for the bond is $100.940423(=99.990423+0.950000)$. Exhibit 3 shows the calculation of the bond's Macaulay duration.

Exhibit 3: Macaulay Duration of an Eight-Year, $6 \%$ Semiannual Payment Bond Priced to Yield $6.00 \%$

\begin{center}
\begin{tabular}{cccccc}
\hline
Period & Time to Receipt & Cash Flow & Present Value & Weight & Time $\times$ Weight \\
\hline
1 & 0.6833 & 3 & 2.940012 & 0.02913 & 0.019903 \\
2 & 1.6833 & 3 & 2.854381 & 0.02828 & 0.047601 \\
3 & 2.6833 & 3 & 2.771244 & 0.02745 & 0.073669 \\
4 & 3.6833 & 3 & 2.690528 & 0.02665 & 0.098178 \\
5 & 4.6833 & 3 & 2.612163 & 0.02588 & 0.121197 \\
6 & 5.6833 & 3 & 2.536080 & 0.02512 & 0.142791 \\
7 & 6.6833 & 3 & 2.462214 & 0.02439 & 0.163025 \\
\end{tabular}
\end{center}

\begin{center}
\begin{tabular}{cccccc}
\hline
Period & Time to Receipt & Cash Flow & Present Value & Weight & Time $\times$ Weight \\
\hline
8 & 7.6833 & 3 & 2.390499 & 0.02368 & 0.181959 \\
9 & 8.6833 & 3 & 2.320873 & 0.02299 & 0.199652 \\
10 & 9.6833 & 3 & 2.253275 & 0.02232 & 0.216159 \\
11 & 10.6833 & 3 & 2.187645 & 0.02167 & 0.231536 \\
12 & 11.6833 & 3 & 2.123927 & 0.02104 & 0.245834 \\
13 & 12.6833 & 3 & 2.062065 & 0.02043 & 0.259102 \\
14 & 13.6833 & 3 & 2.002005 & 0.01983 & 0.271389 \\
15 & 14.6833 & 1.943694 & 0.01926 & 0.282740 &  \\
16 & 15.6833 & 103 & 64.789817 & 0.64186 & 10.066535 \\
 &  & 100.940423 & 1.00000 & 12.621268 &  \\
\hline
\end{tabular}
\end{center}

There are 16 semiannual periods to maturity between the last coupon payment date of 14 February 2019 and maturity on 14 February 2027. The time to receipt of cash flow in semiannual periods is in the second column: $0.6833=1-57 / 180,1.6833=2-57 / 180$, etc. The cash flow for each period is in the third column. The annual yield-to-maturity is $6.00 \%$, so the yield per semiannual period is $3.00 \%$. When that yield is used to get the present value of each cash flow, the full price of the bond is 100.940423, the sum of the fourth column. The weights, which are the shares of the full price corresponding to each cash flow, are in the fifth column. The Macaulay duration is the sum of the items in the sixth column, which is the weight multiplied by the time to receipt of each cash flow. The result, 12.621268 , is the Macaulay duration on an eight-year, $6 \%$ semiannual payment bond for settlement on 11 April 2019 measured in semiannual periods. Similar to coupon rates and yields-to-maturity, duration statistics invariably are annualized in practice. Therefore, the Macaulay duration typically is reported as 6.310634 years (= 12.621268/2). (Such precision for the duration statistic is not needed in practice. Typically, "6.31 years" is enough. The full precision is shown here to illustrate calculations.) Microsoft Excel users can obtain the Macaulay duration using the DURATION financial function-DURATION(DATE(2019,4,11),DATE(2027,2,14),0.06,0.06,2,0)and inputs that include the settlement date, maturity date, annual coupon rate as a decimal, annual yield-to-maturity as a decimal, periodicity, and day count code $(0$ for 30/360, 1 for actual/actual).

Another approach to calculating the Macaulay duration is to use a closed-form equation derived using calculus and algebra (see Smith 2014). Equation 4 is a general closed-form formula for determining the Macaulay duration of a fixed-rate bond, where $c$ is the coupon rate per period (PMT/FV).

$$
\text { MacDur }=\left\{\frac{1+r}{r}-\frac{1+r+[N \times(c-r)]}{c \times\left[(1+r)^{N}-1\right]+r}\right\}-(t / T)
$$

The Macaulay duration of the 10 -year, $8 \%$ annual payment bond is calculated by entering $r=0.1040, c=0.0800, N=10$, and $t / T=0$ into Equation 4.

MacDur $=\frac{1+0.1040}{0.1040}-\frac{1+0.1040+[10 \times(0.0800-0.1040)]}{0.0800 \times\left[(1+0.1040)^{10}-1\right]+0.1040}=7.0029$

Therefore, the weighted average time to receipt of the interest and principal payments that will result in realization of the initial market discount rate on this 10-year bond is 7.00 years.

The Macaulay duration of the 6\% semiannual payment bond maturing on 14 February 2027 is obtained by entering $r=0.0300, c=0.0300, N=16$, and $t / T=57 / 180$ into Equation 4.

$$
\begin{aligned}
& \text { MacDur }=\left[\frac{1+0.0300}{0.0300}-\frac{1+0.0300+[16 \times(0.0300-0.0300)]}{0.0300 \times\left[(1+0.0300)^{16}-1\right]+0.0300}\right]-(57 / 180) \\
= & 12.621268
\end{aligned}
$$

Equation 4 uses the yield-to-maturity per period, the coupon rate per period, the number of periods to maturity, and the fraction of the current period that has gone by. Its output is the Macaulay duration in terms of periods. It is converted to annual duration by dividing by the number of periods in the year.

The calculation of the modified duration (ModDur) statistic of a bond requires a simple adjustment to Macaulay duration. It is the Macaulay duration statistic divided by one plus the yield per period.

ModDur $=\frac{\text { MacDur }}{1+r}$

For example, the modified duration of the 10 -year, $8 \%$ annual payment bond is 6.3432.

ModDur $=\frac{7.0029}{1.1040}=6.3432$

The modified duration of the $6 \%$ semiannual payment bond maturing on 14 February 2027 is 12.253658 semiannual periods.

ModDur $=\frac{12.621268}{1.0300}=12.253658$

The annualized modified duration of the bond is $6.126829(=12.253658 / 2)$.

Microsoft Excel users can obtain the modified duration using the MDURATION financial function using the same inputs as for the Macaulay duration: MDURATION(DATE(2019,4,11),DATE(2027,2,14),0.06,0.06,2,0). Although modified duration might seem to be just a Macaulay duration with minor adjustments, it has an important application in risk measurement: Modified duration provides an estimate of the percentage price change for a bond given a change in its yield-to-maturity.

$\% \Delta P V^{\text {Full }} \approx-$ AnnModDur $\times \Delta$ Yield

The percentage price change refers to the full price, including accrued interest. The AnnModDur term in Equation 6 is the annual modified duration, and the $\Delta$ Yield term is the change in the annual yield-to-maturity. The $\approx$ sign indicates that this calculation is an estimation. The minus sign indicates that bond prices and yields-to-maturity move inversely.

If the annual yield on the 6\% semiannual payment bond that matures on 14 February 2027 jumps by 100 bps, from $6.00 \%$ to $7.00 \%$, the estimated loss in value for the bond is $6.1268 \%$.

$$
\% \Delta P V^{\text {Full }} \approx-6.126829 \times 0.0100=-0.061268
$$

If the yield-to-maturity were to drop by $100 \mathrm{bps}$ to $5.00 \%$, the estimated gain in value is also $6.1268 \%$.

$$
\% \Delta P V^{\text {Full }} \approx-6.126829 \times-0.0100=0.061268
$$

Modified duration provides a linear estimate of the percentage price change. In terms of absolute value, the change is the same for either an increase or a decrease in the yield-to-maturity. Recall that for a given coupon rate and time-to-maturity, the percentage price change is greater (in absolute value) when the market discount rate goes down than when it goes up. Later in this reading, a "convexity adjustment" to duration is introduced. It improves the accuracy of this estimate, especially when a large change in yield-to-maturity (such as $100 \mathrm{bps}$ ) is considered.

\section{APPROXIMATE MODIFIED AND MACAULAY DURATION}
 define, calculate, and interpret Macaulay, modified, and effective durationsThe modified duration statistic for a fixed-rate bond is easily obtained if the Macaulay duration is already known. An alternative approach is to approximate modified duration directly. Equation 7 is the approximation formula for annual modified duration.

$$
\text { ApproxModDur }=\frac{\left(P V_{-}\right)-\left(P V_{+}\right)}{2 \times(4 \text { Yield }) \times\left(P V_{0}\right)}
$$

The objective of the approximation is to estimate the slope of the line tangent to the price-yield curve. The slope of the tangent and the approximated slope are shown in Exhibit 4.

\section{Exhibit 4: Approximate Modified Duration}
Price

\begin{center}
\includegraphics[max width=\textwidth]{2023_05_04_36535b8d80b32081d422g-028}
\end{center}

$\Delta$ Yield $\quad \Delta$ Yield

To estimate the slope, the yield-to-maturity is changed up and down by the same amount-the $\Delta$ Yield. Then the bond prices given the new yields-to-maturity are calculated. The price when the yield is increased is denoted $P V_{+}$. The price when the yield-to-maturity is reduced is denoted $P V_{-}$. The original price is $P V_{0}$. These prices are the full prices, including accrued interest. The slope of the line based on $P V_{+}$and $P V_{-}$is the approximation for the slope of the line tangent to the price-yield curve. The following example illustrates the remarkable accuracy of this approximation. In fact, as $\Delta$ Yield approaches zero, the approximation approaches AnnModDur.

Consider the $6 \%$ semiannual coupon payment corporate bond maturing on 14 February 2027. For settlement on 11 April 2019, the full price $\left(P V_{0}\right)$ is 100.940423 given that the yield-to-maturity is $6.00 \%$.

$$
P V_{0}=\left[\frac{3}{(1.03)^{1}}+\frac{3}{(1.03)^{2}}+\cdots+\frac{103}{(1.03)^{16}}\right] \times(1.03)^{57 / 180}=100.940423
$$

Raise the annual yield-to-maturity by five bps, from $6.00 \%$ to $6.05 \%$. This increase corresponds to an increase in the yield-to-maturity per semiannual period of $2.5 \mathrm{bps}$, from $3.00 \%$ to $3.025 \%$ per period. The new full price $\left(P V_{+}\right)$is 100.631781 .

$$
\begin{aligned}
& P V_{+}=\left[\frac{3}{(1.03025)^{1}}+\frac{3}{(1.03025)^{2}}+\cdots+\frac{103}{(1.03025)^{16}}\right] \times(1.03025)^{57 / 180} \\
& =100.631781
\end{aligned}
$$

Lower the annual yield-to-maturity by five bps, from 6.00\% to $5.95 \%$. This decrease corresponds to a decrease in the yield-to-maturity per semiannual period of $2.5 \mathrm{bps}$, from $3.00 \%$ to $2.975 \%$ per period. The new full price $\left(P V_{-}\right)$is 101.250227.

$$
\begin{aligned}
& P V_{-}=\left[\frac{3}{(1.02975)^{1}}+\frac{3}{(1.02975)^{2}}+\cdots+\frac{103}{(1.02975)^{16}}\right] \times(1.02975)^{57 / 180} \\
& =101.250227
\end{aligned}
$$

Enter these results into Equation 7 for the $5 \mathrm{bp}$ change in the annual yield-to-maturity, or $\Delta$ Yield $=0.0005$

ApproxModDur $=\frac{101.250227-100.631781}{2 \times 0.0005 \times 100.940423}=6.126842$

The "exact" annual modified duration for this bond is 6.126829 and the "approximation" is 6.126842-virtually identical results. Therefore, although duration can be calculated using the approach in Exhibits 2 and 3-basing the calculation on the weighted average time to receipt of each cash flow-or using the closed-form formula as in Equation 4, it can also be estimated quite accurately using the basic bond-pricing equation and a financial calculator. The Macaulay duration can be approximated as well-the approximate modified duration multiplied by one plus the yield per period.

ApproxMacDur = ApproxModDur $\times(1+r)$

The approximation formulas produce results for annualized modified and Macaulay durations. The frequency of coupon payments and the periodicity of the yield-to-maturity are included in the bond price calculations.

\section{EXAMPLE 8}
\begin{enumerate}
  \item Assume that the $3.75 \%$ US Treasury bond that matures on 15 August 2041 is priced to yield $5.14 \%$ for settlement on 15 October 2020 . Coupons are paid semiannually on 15 February and 15 August. The yield-to-maturity is stated on a street-convention semiannual bond basis. This settlement date is 61 days into a 184-day coupon period, using the actual/actual day-count convention. Compute the approximate modified duration and the approximate Macaulay duration for this Treasury bond assuming a 5 bp change in the yield-to-maturity.
\end{enumerate}

\section{Solution:}
The yield-to-maturity per semiannual period is $0.0257(=0.0514 / 2)$. The coupon payment per period is $1.875(=3.75 / 2)$. At the beginning of the period, there are 21 years (42 semiannual periods) to maturity. The fraction of the period that has passed is $61 / 184$. The full price at that yield-to-maturity is 82.967530 per 100 of par value.

$$
P V_{0}=\left[\frac{1.875}{(1.0257)^{1}}+\frac{1.875}{(1.0257)^{2}}+\cdots+\frac{101.875}{(1.0257)^{42}}\right] \times(1.0257)^{61 / 184}=82.96753
$$

Raise the yield-to-maturity from $5.14 \%$ to $5.19 \%$-therefore, from $2.57 \%$ to $2.595 \%$ per semiannual period-and the price becomes 82.411395 per 100 of par value.

$$
\begin{aligned}
& P V_{+}=\left[\frac{1.875}{(1.02595)^{1}}+\frac{1.875}{(1.02595)^{2}}+\cdots+\frac{101.875}{(1.02595)^{42}}\right] \times(1.02595)^{61 / 184} \\
& =82.411395
\end{aligned}
$$

Lower the yield-to-maturity from $5.14 \%$ to $5.09 \%$-therefore, from $2.57 \%$ to $2.545 \%$ per semiannual period-and the price becomes 83.528661 per 100 of par value.

$$
\begin{aligned}
& P V_{-}=\left[\frac{1.875}{(1.02545)^{1}}+\frac{1.875}{(1.02545)^{2}}+\cdots+\frac{101.875}{(1.02545)^{42}}\right] \times(1.02545)^{61 / 184} \\
= & 83.528661
\end{aligned}
$$

The approximate annualized modified duration for the Treasury bond is 13.466.

ApproxModDur $=\frac{83.528661-82.411395}{2 \times 0.0005 \times 82.967530}=13.466$

The approximate annualized Macaulay duration is 13.812 .

$$
\text { ApproxMacDur }=13.466 \times 1.0257=13.812
$$

Therefore, from these statistics, the investor knows that the weighted average time to receipt of interest and principal payments is 13.812 years (the Macaulay duration) and that the estimated loss in the bond's market value is 13.466\% (the modified duration) if the market discount rate were to suddenly go up by $1 \%$ from $5.14 \%$ to $6.14 \%$.

\section{EFFECTIVE AND KEY RATE DURATION}
\begin{center}
\includegraphics[max width=\textwidth]{2023_05_04_36535b8d80b32081d422g-030}
\end{center}

Another approach to assess the interest rate risk of a bond is to estimate the percentage change in price given a change in a benchmark yield curve-for example, the government par curve. This estimate, which is very similar to the formula for approximate modified duration, is called the effective duration. The effective duration of a bond is the sensitivity of the bond's price to a change in a benchmark yield curve. The formula to calculate effective duration (EffDur) is Equation 9.

$$
\text { EffDur }=\frac{\left(P V_{-}\right)-\left(P V_{+}\right)}{2 \times(\Delta \mathrm{Curve}) \times\left(P V_{0}\right)}
$$

The difference between approximate modified duration and effective duration is in the denominator. Modified duration is a yield duration statistic in that it measures interest rate risk in terms of a change in the bond's own yield-to-maturity ( $\Delta$ Yield). Effective duration is a curve duration statistic in that it measures interest rate risk in terms of a parallel shift in the benchmark yield curve ( $\Delta$ Curve).

Effective duration is essential to the measurement of the interest rate risk of a complex bond, such as a bond that contains an embedded call option. The duration of a callable bond is not the sensitivity of the bond price to a change in the yield-to-worst (i.e., the lowest of the yield-to-maturity, yield-to-first-call, yield-to-second-call, and so forth). The problem is that future cash flows are uncertain because they are contingent on future interest rates. The issuer's decision to call the bond depends on the ability to refinance the debt at a lower cost of funds. In brief, a callable bond does not have a well-defined internal rate of return (yield-to-maturity). Therefore, yield duration statistics, such as modified and Macaulay durations, do not apply; effective duration is the appropriate duration measure.

The specific option-pricing models that are used to produce the inputs to effective duration for a callable bond are covered in later readings. However, as an example, suppose that the full price of a callable bond is 101.060489 per 100 of par value. The option-pricing model inputs include (1) the length of the call protection period, (2) the schedule of call prices and call dates, (3) an assumption about credit spreads over benchmark yields (which includes any liquidity spread as well), (4) an assumption about future interest rate volatility, and (5) the level of market interest rates (e.g., the government par curve). The analyst then holds the first four inputs constant and raises and lowers the fifth input. Suppose that when the government par curve is raised and lowered by $25 \mathrm{bps}$, the new full prices for the callable bond from the model are 99.050120 and 102.890738, respectively. Therefore, $P V_{0}=101.060489, P V_{+}$ $=99.050120, P V_{-}=102.890738$, and $\Delta$ Curve $=0.0025$. The effective duration for the callable bond is 7.6006 .

EffDur $=\frac{102.890738-99.050120}{2 \times 0.0025 \times 101.060489}=7.6006$

This curve duration measure indicates the bond's sensitivity to the benchmark yield curve-in particular, the government par curve-assuming no change in the credit spread. In practice, a callable bond issuer might be able to exercise the call option and obtain a lower cost of funds if (1) benchmark yields fall and the credit spread over the benchmark is unchanged or (2) benchmark yields are unchanged and the credit spread is reduced (e.g., because of an upgrade in the issuer's rating). A pricing model can be used to determine a "credit duration" statistic-that is, the sensitivity of the bond price to a change in the credit spread. On a traditional fixed-rate bond, modified duration estimates the percentage price change for a change in the benchmark yield and/or the credit spread. For bonds that do not have a well-defined internal rate of return because the future cash flows are not fixed-for instance, callable bonds and floating-rate notes-pricing models are used to produce different statistics for changes in benchmark interest rates and for changes in credit risk.

Another fixed-income security for which yield duration statistics, such as modified and Macaulay durations, are not relevant is a mortgage-backed bond. These securities arise from a residential (or commercial) loan portfolio securitization. The key point for measuring interest rate risk on a mortgage-backed bond is that the cash flows are contingent on homeowners' ability to refinance their debt at a lower rate. In effect, the homeowners have call options on their mortgage loans.

A practical consideration in using effective duration is in setting the change in the benchmark yield curve. With approximate modified duration, accuracy is improved by choosing a smaller yield-to-maturity change. But the pricing models for more-complex securities, such as callable and mortgage-backed bonds, include assumptions about the behavior of the corporate issuers, businesses, or homeowners. Rates typically need to change by a minimum amount to affect the decision to call a bond or refinance a mortgage loan because issuing new debt involves transaction costs. Therefore, estimates of interest rate risk using effective duration are not necessarily improved by choosing a smaller change in benchmark rates. Effective duration has become an important tool in the financial analysis of not only traditional bonds but also financial liabilities. Example 9 demonstrates such an application of effective duration.

\section{EXAMPLE 9}
\begin{enumerate}
  \item Defined-benefit pension schemes typically pay retirees a monthly amount based on their wage level at the time of retirement. The amount could be fixed in nominal terms or indexed to inflation. These programs are referred to as "defined-benefit pension plans" when US GAAP or IFRS accounting standards are used. In Australia, they are called "superannuation funds."
\end{enumerate}

A British defined-benefit pension scheme seeks to measure the sensitivity of its retirement obligations to market interest rate changes. The pension scheme manager hires an actuarial consultancy to model the present value of its liabilities under three interest rate scenarios: (1) a base rate of $5 \%$, (2) a $100 \mathrm{bp}$ increase in rates, up to $6 \%$, and (3) a $100 \mathrm{bp}$ drop in rates, down to $4 \%$.

The actuarial consultancy uses a complex valuation model that includes assumptions about employee retention, early retirement, wage growth, mortality, and longevity. The following chart shows the results of the analysis.

\begin{center}
\begin{tabular}{cc}
\hline
Interest Rate Assumption & Present Value of Liabilities \\
\hline
$4 \%$ & GBP973.5 million \\
$5 \%$ & GBP926.1 million \\
$6 \%$ & GBP871.8 million \\
\hline
\end{tabular}
\end{center}

Compute the effective duration of the pension scheme's liabilities.

\section{Solution:}
$P V_{0}=926.1, P V_{+}=871.8, P V_{-}=973.5$, and $\Delta$ Curve $=0.0100$. The effective duration of the pension scheme's liabilities is 5.49.

EffDur $=\frac{973.5-871.8}{2 \times 0.0100 \times 926.1}=5.49$

This effective duration statistic for the pension scheme's liabilities might be used in asset allocation decisions to decide the mix of equity, fixed income, and alternative assets.

Although effective duration is the most appropriate interest rate risk measure for bonds with embedded options, it also is useful with traditional bonds to supplement the information provided by the Macaulay and modified yield durations. Exhibit 5 displays the Bloomberg Yield and Spread (YAS) Analysis page for the 2.875\% US Treasury note that matures on 15 May 2028.

\section{Exhibit 5: Bloomberg YAS Page for the 2.875\% US Treasury Note}
\begin{center}
\includegraphics[max width=\textwidth]{2023_05_04_36535b8d80b32081d422g-033}
\end{center}

(c) 2019 Bloomberg L.P. All rights reserved. Reproduced with permission.

In Exhibit 5, the quoted (flat) asked price for the bond is 100-07, which is equal to 100 and 7/32nds per 100 of par value for settlement on 13 July 2018. Most bond prices are stated in decimals, but US Treasuries are usually quoted in fractions. As a deci$\mathrm{mal}$, the flat price is 100.21875 . The accrued interest uses the actual/actual day-count method. That settlement date is 59 days into a 184-day semiannual coupon payment period. The accrued interest is 0.4609375 per 100 of par value $(=59 / 184 \times 0.02875 / 2$ $\times 100$ ). The full price of the bond is 100.679688 . The yield-to-maturity of the bond is $2.849091 \%$, stated on a street-convention semiannual bond basis.

The modified duration for the bond is shown in Exhibit 5 to be 8.482, which is the conventional yield duration statistic. Its curve duration, however, is 8.510 , which is the price sensitivity with respect to changes in the US Treasury par curve. On Bloomberg, the effective duration is called the "OAS duration" because it is based on the option-pricing model that is also used to calculate the option-adjusted spread. The small difference arises because the government yield curve is not flat. When the par curve is shifted in the model, the government spot curve is also shifted, although not in the same "parallel" manner. Therefore, the change in the bond price is not the same as it would be if its own yield-to-maturity changed by the same amount as the change in the par curve. In general, the modified duration and effective duration on a traditional option-free bond are not identical. The difference narrows when the yield curve is flatter, the time-to-maturity is shorter, and the bond is priced closer to par value (so that the difference between the coupon rate and the yield-to-maturity is smaller). The modified duration and effective duration on an option-free bond are identical only in the rare circumstance of a flat yield curve.

\section{Key Rate Duration}
The effective duration for a sample callable bond was calculated previously as

$$
\text { EffDur }=\frac{102.890738-99.050120}{2 \times 0.0025 \times 101.060489}=7.6006
$$

This duration measure indicates the bond's sensitivity to the benchmark yield curve if all yields change by the same amount. "Key rate" duration provides further insight into a bond's sensitivity to non-parallel benchmark yield curve changes. A key rate duration (or partial duration) is a measure of a bond's sensitivity to a change in the benchmark yield at a specific maturity. Key rate durations define a security's price sensitivity over a set of maturities along the yield curve, with the sum of key rate durations being identical to the effective duration:

$$
\begin{aligned}
& \text { KeyRateDur }^{k}=-\frac{1}{P V} \times \frac{\Delta P V}{\Delta r^{k}} \\
& \sum_{k=1}^{n} \text { KeyRateDur }^{k}=\text { EffDur }
\end{aligned}
$$

where $r^{k}$ represents the $k$ th key rate. In contrast to effective duration, key rate durations help identify "shaping risk" for a bond - that is, a bond's sensitivity to changes in the shape of the benchmark yield curve (e.g., the yield curve becoming steeper or flatter).

The previous illustration of effective duration assumed a parallel shift of $25 \mathrm{bps}$ at all maturities. However, the analyst may want to know how the price of the callable bond is expected to change if short-term benchmark rates (say, for a current two-year Treasury note with modified duration of 1.9) rise by 25 bps but longer-maturity benchmark rates remain unchanged. This scenario would represent a flattening of the yield curve, given that the yield curve is upward sloping. Using key rate durations, the expected price change would be approximately equal to minus the key rate duration for the short-maturity segment $(-1.9)$ times the 0.0025 interest rate shift at that segment, or $-0.475 \%$ based on the following formula:

$$
\frac{\Delta P V}{P V}=- \text { KeyRateDur }^{k} \times \Delta r^{k}
$$

Of course, for parallel shifts in the benchmark yield curve, key rate durations will indicate the same interest rate sensitivity as effective duration.

\section{PROPERTIES OF BOND DURATION}
explain how a bond's maturity, coupon, and yield level affect its interest rate risk

The Macaulay and modified yield duration statistics for a traditional fixed-rate bond are functions of the input variables: the coupon rate or payment per period, the yield-to-maturity per period, the number of periods to maturity (as of the beginning of the period), and the fraction of the period that has gone by. The properties of bond duration are obtained by changing one of these variables while holding the others constant. Because duration is the basic measure of interest rate risk on a fixed-rate bond, these properties are important to understand.

The closed-form formula for Macaulay duration, presented as Equation 4 and again here, is useful in demonstrating the characteristics of the bond duration statistic.

$$
\text { MacDur }=\left\{\frac{1+r}{r}-\frac{1+r+[N \times(c-r)]}{c \times\left[(1+r)^{N}-1\right]+r}\right\}-(t / T)
$$

The same characteristics hold for modified duration. Consider first the fraction of the period that has gone by $(t / T)$. Macaulay and modified durations depend on the day-count basis used to obtain the yield-to-maturity. The duration of a bond that uses the actual/actual method to count days is slightly different from that of an otherwise comparable bond that uses the $30 / 360$ method. The key point is that for a constant yield-to-maturity $(r)$, the expression in braces is unchanged as time passes during the period. Therefore, the Macaulay duration decreases smoothly as $t$ goes from $t=0$ to $t=T$, which creates a "saw-tooth" pattern. This pattern for a typical fixed-rate bond is illustrated in Exhibit 6.

Exhibit 6: Macaulay Duration between Coupon Payments with a Constant Yield-to-Maturity

\begin{center}
\includegraphics[max width=\textwidth]{2023_05_04_36535b8d80b32081d422g-035}
\end{center}

As times passes during the coupon period (moving from right to left in the diagram), the Macaulay duration declines smoothly and then jumps upward after the coupon is paid.

The characteristics of bond duration related to changes in the coupon rate, the yield-to-maturity, and the time-to-maturity are illustrated in Exhibit 7.

\section{Exhibit 7: Properties of the Macaulay Yield Duration}
\begin{center}
\includegraphics[max width=\textwidth]{2023_05_04_36535b8d80b32081d422g-035(1)}
\end{center}

Time-to-Maturity

Exhibit 7 shows the graph for coupon payment dates when $t / T=0$, thus not displaying the saw-tooth pattern between coupon payments. The relationship between the Macaulay duration and the time-to-maturity for a zero-coupon bond is the 45-degree line: MacDur $=N$ when $c=0$ (and $t / T=0)$. Therefore, the Macaulay duration of a zero-coupon bond is its time-to-maturity.

A perpetuity or perpetual bond, which also is called a consol, is a bond that does not mature. There is no principal to redeem. The investor receives a fixed coupon payment forever unless the bond is callable. Non-callable perpetuities are rare, but they have an interesting Macaulay duration: MacDur $=(1+r) / r$ as $N$ approaches infinity. In effect, the second expression within the braces approaches zero as the number of periods to maturity increases because $N$ in the numerator is a coefficient but $N$ in the denominator is an exponent and the denominator increases faster than the numerator as $N$ grows larger.

Typical fixed-rate coupon bonds with a stated maturity date are portrayed in Exhibit 7 as the premium and discount bonds. The usual pattern is that longer times-to-maturity correspond to higher Macaulay duration statistics. This pattern always holds for bonds trading at par value or at a premium above par. In Equation 4 , the second expression within the braces is a positive number for premium and par bonds. The numerator is positive because the coupon rate $(c)$ is greater than or equal to the yield-to-maturity $(r)$, whereas the denominator is always positive. Therefore, the Macaulay duration is always less than $(1+r) / r$, and it approaches that threshold from below as the time-to-maturity increases.

The curious result displayed in Exhibit 7 is in the pattern for discount bonds. Generally, the Macaulay duration increases for a longer time-to-maturity. But at some point when the time-to-maturity is high enough, the Macaulay duration exceeds $(1+$ $r) / r$, reaches a maximum, and then approaches the threshold from above. In Equation 4 , such a pattern develops when the number of periods $(N)$ is large and the coupon rate $(c)$ is below the yield-to-maturity $(r)$. Then the numerator of the second expression within the braces can become negative. The implication is that on long-term discount bonds, the interest rate risk can actually be less than on a shorter-term bond, which explains why the word "generally" is needed in describing the maturity effect for the relationship between bond prices and yields-to-maturity. Generally, for the same coupon rate, a longer-term bond has a greater percentage price change than a shorter-term bond when their yields-to-maturity change by the same amount. The exception is when the longer-term bond has a lower duration statistic.

Coupon rates and yields-to-maturity are both inversely related to the Macaulay duration. In Exhibit 7, for the same time-to-maturity and yield-to-maturity, the Macaulay duration is higher for a zero-coupon bond than for a low-coupon bond trading at a discount. Also, the low-coupon bond trading at a discount has a higher duration than a high-coupon bond trading at a premium. Therefore, all else being equal, a lower-coupon bond has a higher duration and more interest rate risk than a higher-coupon bond. The same pattern holds for the yield-to-maturity. A higher yield-to-maturity reduces the weighted average of the time to receipt of cash flow. More weight is on the cash flows received in the near term, and less weight is on the cash flows received in the more-distant future periods if those cash flows are discounted at a higher rate.

In summary, the Macaulay and modified duration statistics for a fixed-rate bond depend primarily on the coupon rate, yield-to-maturity, and time-to-maturity. A higher coupon rate or a higher yield-to-maturity reduces the duration measures. A longer time-to-maturity usually leads to a higher duration. It always does so for a bond priced at a premium or at par value. But if the bond is priced at a discount, a longer time-to-maturity might lead to a lower duration. This situation only occurs if the coupon rate is low (but not zero) relative to the yield and the time-to-maturity is long.

\section{EXAMPLE 10}
A hedge fund specializes in investments in emerging market sovereign debt. The fund manager believes that the implied default probabilities are too high, which means that the bonds are viewed as "cheap" and the credit spreads are too high. The hedge fund plans to take a position on one of these available bonds.

Bond Time-to-Maturity Coupon Rate Price Yield-to-Maturity

(A) 10 years $\quad 10 \% \quad 58.075279 \quad 20 \%$

\begin{center}
\begin{tabular}{lcccc}
\hline
Bond & Time-to-Maturity & Coupon Rate & Price & Yield-to-Maturity \\
\hline
(B) & 20 years & $10 \%$ & 51.304203 & $20 \%$ \\
$(\mathrm{C})$ & 30 years & $10 \%$ & 50.210636 & $20 \%$ \\
\hline
\end{tabular}
\end{center}

The coupon payments are annual. The yields-to-maturity are effective annual rates. The prices are per 100 of par value.

\begin{enumerate}
  \item Compute the approximate modified duration of each of the three bonds using a $1 \mathrm{bp}$ change in the yield-to-maturity and keeping precision to six decimals (because approximate duration statistics are very sensitive to rounding).
\end{enumerate}

\section{Solution to 1:}
\section{Bond A:}
$P V_{0}=58.075279$

$P V_{+}=58.047598$

$\frac{10}{(1.2001)^{1}}+\frac{10}{(1.2001)^{2}}+\cdots+\frac{110}{(1.2001)^{10}}=58.047598$

$P_{-}=58.102981$

$\frac{10}{(1.1999)^{1}}+\frac{10}{(1.1999)^{2}}+\cdots+\frac{110}{(1.1999)^{10}}=58.102981$

The approximate modified duration of Bond $\mathrm{A}$ is 4.768 .

ApproxModDur $=\frac{58.102981-58.047598}{2 \times 0.0001 \times 58.075279}=4.768$

\section{Bond B:}
$P V_{0}=51.304203$

$P V_{+}=51.277694$

$\frac{10}{(1.2001)^{1}}+\frac{10}{(1.2001)^{2}}+\cdots+\frac{110}{(1.2001)^{20}}=51.277694$

$P V_{-}=51.330737$

$\frac{10}{(1.1999)^{1}}+\frac{10}{(1.1999)^{2}}+\cdots+\frac{110}{(1.1999)^{20}}=51.330737$

The approximate modified duration of Bond B is 5.169.

ApproxModDur $=\frac{51.330737-51.277694}{2 \times 0.0001 \times 51.304203}=5.169$

\section{Bond C:}
$P V_{0}=50.210636$

$P V_{+}=50.185228$

$\frac{10}{(1.2001)^{1}}+\frac{10}{(1.2001)^{2}}+\cdots+\frac{110}{(1.2001)^{30}}=50.185228$

$P V_{-}=50.236070$

$$
\frac{10}{(1.1999)^{1}}+\frac{10}{(1.1999)^{2}}+\cdots+\frac{110}{(1.1999)^{30}}=50.236070
$$

The approximate modified duration of Bond $C$ is 5.063.

$$
\text { ApproxModDur }=\frac{50.236070-50.185228}{2 \times 0.0001 \times 50.210636}=5.063
$$

\begin{enumerate}
  \setcounter{enumi}{1}
  \item Which of the three bonds is expected to have the highest percentage price increase if the yield-to-maturity on each decreases by the same amount-for instance, by 10 bps from $20 \%$ to $19.90 \%$ ?
\end{enumerate}

\section{Solution to 2:}
Despite the significant differences in times-to-maturity (10, 20, and 30 years), the approximate modified durations on the three bonds are fairly similar (4.768, 5.169, and 5.063). Because the yields-to-maturity are so high, the additional time to receipt of interest and principal payments on the 20and 30-year bonds have low weight. Nevertheless, Bond B, with 20 years to maturity, has the highest modified duration. If the yield-to-maturity on each is decreased by the same amount-for instance, by $10 \mathrm{bps}$, from $20 \%$ to $19.90 \%$-Bond B would be expected to have the highest percentage price increase because it has the highest modified duration. This example illustrates the relationship between the Macaulay duration and the time-to-maturity on discount bonds in Exhibit 7. The 20-year bond has a higher duration than the 30-year bond.

Callable bonds require the use of effective duration because Macaulay and modified yield duration statistics are not relevant. The yield-to-maturity for callable bonds is not well-defined because future cash flows are uncertain. Exhibit 8 illustrates the impact of the change in the benchmark yield curve ( $\Delta$ Curve) on the price of a callable bond price compared with that on a comparable non-callable bond. The two bonds have the same credit risk, coupon rate, payment frequency, and time-to-maturity. The vertical axis is the bond price. The horizontal axis is a benchmark yield-for instance, a point on the par curve for government bonds.

Exhibit 8: Interest Rate Risk Characteristics of a Callable Bond

\begin{center}
\includegraphics[max width=\textwidth]{2023_05_04_36535b8d80b32081d422g-038}
\end{center}

As shown in Exhibit 8, the price of the non-callable bond is always greater than that of the callable bond with otherwise identical features. The difference is the value of the embedded call option. Recall that the call option is an option to the issuer and not the holder of the bond. When interest rates are high compared with the coupon rate, the value of the call option is low. When rates are low, the value of the call option is much greater because the issuer is more likely to exercise the option to refinance the debt at a lower cost of funds. The investor bears the "call risk" because if the bond is called, the investor must reinvest the proceeds at a lower interest rate.

Exhibit 8 shows the inputs for calculating the effective duration of the callable bond. The entire benchmark curve is raised and lowered by the same amount, $\Delta$ Curve. The key point is that when benchmark yields are high, the effective durations of the callable and non-callable bonds are very similar. Although the exhibit does not illustrate it, the slopes of the lines tangent to the price-yield curve are about the same in such a situation. But when interest rates are low, the effective duration of the callable bond is lower than that of the otherwise comparable non-callable bond. That is because the callable bond price does not increase as much when benchmark yields fall. The slope of the line tangent to the price-yield curve would be flatter. The presence of the call option limits price appreciation. Therefore, an embedded call option reduces the effective duration of the bond, especially when interest rates are falling and the bond is more likely to be called. The lower effective duration can also be interpreted as a shorter expected life-the weighted average of time to receipt of cash flow is reduced.

Exhibit 9 considers another embedded option-a put option.

\section{Exhibit 9: Interest Rate Risk Characteristics of a Putable Bond}
Price

\begin{center}
\includegraphics[max width=\textwidth]{2023_05_04_36535b8d80b32081d422g-039}
\end{center}

A putable bond allows the investor to sell the bond back to the issuer prior to maturity, usually at par value, which protects the investor from higher benchmark yields or credit spreads that otherwise would drive the bond to a discounted price. Therefore, the price of a putable bond is always higher than that of an otherwise comparable non-putable bond. The price difference is the value of the embedded put option.

An embedded put option reduces the effective duration of the bond, especially when rates are rising. If interest rates are low compared with the coupon rate, the value of the put option is low and the impact of a change in the benchmark yield on the bond's price is very similar to the impact on the price of a non-putable bond. But when benchmark interest rates rise, the put option becomes more valuable to the investor. The ability to sell the bond at par value limits the price depreciation as rates rise. In summary, the presence of an embedded option reduces the sensitivity of the bond price to changes in the benchmark yield curve, assuming no change in credit risk.

\section{7}
\section{DURATION OF A BOND PORTFOLIO}
calculate the duration of a portfolio and explain the limitations of portfolio duration

Similar to equities, bonds are typically held in a portfolio. There are two ways to calculate the duration of a bond portfolio: (1) the weighted average of time to receipt of the aggregate cash flows, and (2) the weighted average of the individual bond durations that comprise the portfolio. The first method is the theoretically correct approach, but it is difficult to use in practice. The second method is commonly used by fixed-income portfolio managers, but it has its own limitations. The differences in these two methods to compute portfolio duration can be examined with a numerical example.

Suppose an investor holds the following portfolio of two zero-coupon bonds:

\begin{center}
\begin{tabular}{|c|c|c|c|c|c|c|c|c|}
\hline
Bond & Maturity & Price & Yield & $\begin{array}{c}\text { Macaulay } \\ \text { Duration }\end{array}$ & $\begin{array}{l}\text { Modified } \\ \text { Duration }\end{array}$ & Par Value & Market Value & Weigh \\
\hline
$(\mathrm{X})$ & 1 year & 98.00 & $2.0408 \%$ & 1 & 0.980 & $10,000,000$ & $9,800,000$ & 0.50 \\
\hline
$(Y)$ & 30 years & 9.80 & $8.0503 \%$ & 30 & 27.765 & $100,000,000$ & $9,800,000$ & 0.50 \\
\hline
\end{tabular}
\end{center}

The prices are per 100 of par value. The yields-to-maturity are effective annual rates. The total market value for the portfolio is $19,600,000$. The portfolio is evenly weighted in terms of market value between the two bonds.

The first approach views the portfolio as a series of aggregated cash flows. Its cash flow yield is $7.8611 \%$. A cash flow yield is the internal rate of return on a series of cash flows, usually used on a complex security such as a mortgage-backed bond (using projected cash flows based on a model of prepayments as a result of refinancing) or a portfolio of fixed-rate bonds. It is the solution for $r$ in the following equation.

$$
19,600,000=\frac{10,000,000}{(1+r)^{1}}+\frac{0}{(1+r)^{2}}+\cdots+\frac{0}{(1+r)^{29}}+\frac{100,000,000}{(1+r)^{30}}, \quad r=0.078611
$$

The Macaulay duration of the portfolio in this approach is the weighted average of the times to receipt of aggregated cash flow. The cash flow yield is used to obtain the weights. This calculation is similar to Equation 1, and the portfolio duration is 16.2825.

$$
\text { MacDur }=\left[\frac{\frac{1 \times 10,000,000}{(1.078611)^{1}}+\frac{30 \times 100,000,000}{(1.078611)^{30}}}{\frac{10,000,000}{(1.078611)^{1}}+\frac{100,000,000}{(1.078611)^{30}}}\right]=16.2825
$$

There are just two future cash flows in the portfolio-the redemption of principal on the two zero-coupon bonds. In more complex portfolios, a series of coupon and principal payments may occur on some dates, with an aggregated cash flow composed of coupon interest on some bonds and principal on those that mature.

The modified duration of the portfolio is the Macaulay duration divided by one plus the cash flow yield per period (here, the periodicity is 1).

$$
\text { ModDur }=\frac{16.2825}{1.078611}=15.0958
$$

The modified duration for the portfolio is 15.0958. That statistic indicates the percentage change in the market value given a change in the cash flow yield. If the cash flow yield increases or decreases by $100 \mathrm{bps}$, the market value of the portfolio is expected to decrease or increase by about $15.0958 \%$.

Although this approach is theoretically correct, it is difficult to use in practice. First, the cash flow yield is not commonly calculated for bond portfolios. Second, the amount and timing of future coupon and principal payments are uncertain if the portfolio contains callable or putable bonds or floating-rate notes. Third, interest rate risk is usually expressed as a change in benchmark interest rates, not as a change in the cash flow yield. Fourth, the change in the cash flow yield is not necessarily the same amount as the change in the yields-to-maturity on the individual bonds. For instance, if the yields-to-maturity on the two zero-coupon bonds in this portfolio both increase or decrease by $10 \mathrm{bps}$, the cash flow yield increases or decreases by only $9.52 \mathrm{bps}$.

In practice, the second approach to portfolio duration is commonly used. The Macaulay and modified durations for the portfolio are calculated as the weighted average of the statistics for the individual bonds. The shares of overall portfolio market value are the weights. This weighted average approximates the theoretically correct portfolio duration, which is obtained using the first approach. This approximation becomes more accurate when the differences in the yields-to-maturity on the bonds in the portfolio are smaller. When the yield curve is flat, the two approaches produce the same portfolio duration.

Given the equal "50/50" weights in this simple numerical example, this version of portfolio duration is easily computed.

Average Macaulay duration $=(1 \times 0.50)+(30 \times 0.50)=15.50$

Average modified duration $=(0.980 \times 0.50)+(27.765 \times 0.50)=14.3725$

Note that $0.980=1 / 1.020404$ and $27.765=30 / 1.080503$. An advantage of the second approach is that callable bonds, putable bonds, and floating-rate notes can be included in the weighted average using the effective durations for these securities.

The main advantage to the second approach is that it is easily used as a measure of interest rate risk. For instance, if the yields-to-maturity on the bonds in the portfolio increase by $100 \mathrm{bps}$, the estimated drop in the portfolio value is $14.3725 \%$. However, this advantage also indicates a limitation: This measure of portfolio duration implicitly assumes a parallel shift in the yield curve. A parallel yield curve shift implies that all rates change by the same amount in the same direction. In reality, interest rate changes frequently result in a steeper or flatter yield curve. Yield volatility is discussed later in this reading.

\section{EXAMPLE 11}
An investment fund owns the following portfolio of three fixed-rate government bonds:

\begin{center}
\begin{tabular}{lccc}
\hline
 & Bond A & Bond B & Bond C \\
\hline
Par value & EUR25,000,000 & EUR25,000,000 & EUR50,000,000 \\
Coupon rate & $9 \%$ & $11 \%$ & $8 \%$ \\
Time-to-maturity & 6 years & 8 years & 12 years \\
Yield-to-maturity & $9.10 \%$ & $9.38 \%$ & $9.62 \%$ \\
Market value & EUR24,886,343 & EUR27,243,887 & EUR44,306,787 \\
Macaulay duration & 4.761 & 5.633 & 7.652 \\
\hline
\end{tabular}
\end{center}

The total market value of the portfolio is EUR96,437,017. Each bond is on a coupon date so that there is no accrued interest. The market values are the full prices given the par value. Coupons are paid semiannually. The yields-to-maturity are stated on a semiannual bond basis, meaning an annual rate for a periodicity of 2. The Macaulay durations are annualized.

\begin{enumerate}
  \item Calculate the average (annual) modified duration for the portfolio using the shares of market value as the weights.
\end{enumerate}

\section{Solution to 1:}
The average (annual) modified duration for the portfolio is 6.0495 .

$$
\begin{aligned}
& \left(\frac{4.761}{1+\frac{0.0910}{2}} \times \frac{24,886,343}{96,437,017}\right)+\left(\frac{5.633}{1+\frac{0.0938}{2}} \times \frac{27,243,887}{96,437,017}\right)+ \\
& \left(\frac{7.652}{1+\frac{0.0962}{2}} \times \frac{44,306,787}{96,437,017}\right)=6.0495
\end{aligned}
$$

Note that the annual modified duration for each bond is the annual Macaulay duration, which is given, divided by one plus the yield-to-maturity per semiannual period.

\begin{enumerate}
  \setcounter{enumi}{1}
  \item Estimate the percentage loss in the portfolio's market value if the (annual) yield-to-maturity on each bond goes up by 20 bps.
\end{enumerate}

\section{Solution to 2:}
The estimated decline in market value if each yield rises by $20 \mathrm{bps}$ is $1.21 \%$ : $-6.0495 \times 0.0020=-0.0121$

\section{MONEY DURATION AND THE PRICE VALUE OF A BASIS POINT}
calculate and interpret the money duration of a bond and price value of a basis point (PVBP)

Modified duration is a measure of the percentage price change of a bond given a change in its yield-to-maturity. A related statistic is money duration. The money duration of a bond is a measure of the price change in units of the currency in which the bond is denominated. The money duration can be stated per 100 of par value or in terms of the actual position size of the bond in the portfolio. In the United States, money duration is commonly called "dollar duration."

Money duration (MoneyDur) is calculated as the annual modified duration times the full price $\left(P V^{F u l l}\right)$ of the bond, including accrued interest.

$$
\text { MoneyDur }=\text { AnnModDur } \times P V^{F u l l}
$$

The estimated change in the bond price in currency units is calculated using Equation 12, which is very similar to Equation 6 . The difference is that for a given change in the annual yield-to-maturity ( $\Delta$ Yield), modified duration estimates the percentage price change and money duration estimates the change in currency units.

$$
\Delta P V^{F u l l} \approx-\text { MoneyDur } \times \Delta \text { Yield }
$$

For a theoretical example of money duration, consider the $6 \%$ semiannual coupon payment bond that matures on 14 February 2027 and is priced to yield $6.00 \%$ for settlement on 11 April 2019. The full price of the bond is 100.940423 per 100 of par value, and the annual modified duration is 6.1268. Suppose that a Nairobi based life insurance company has a position in the bond for a par value of KES100,000,000. The market value of the investment is KES 100,940,423. The money duration of this bond is KES $618,441,784(=6.1268 \times \operatorname{KES} 100,940,423)$. Therefore, if the yield-to-maturity rises by 100 bps-from $6.00 \%$ to $7.00 \%$-the expected loss is approximately -KES $6,184,418$ (=-KES $618,441,784 \times 0.0100)$. On a percentage basis, that expected loss is approximately $6.1268 \%$. The "convexity adjustment" introduced in the next section makes these estimates more accurate.

Another version of money duration is the value of one basis point in price terms. The price value of a basis point (or PVBP) is an estimate of the change in the full bond price given a $1 \mathrm{bp}$ change in the yield-to-maturity. The PVBP can be calculated using a formula similar to that for the approximate modified duration. Equation 13 is the formula for the PVBP.

$$
\mathrm{PVBP}=\frac{\left(P V_{-}\right)-\left(P V_{+}\right)}{2}
$$

$P V_{-}$and $P V_{+}$are the full prices calculated by decreasing and increasing the yield-to-maturity by $1 \mathrm{bp}$. The PVBP is also called the "PV01," standing for the "price value of an 01 " or "present value of an 01 ," where "01" means 1 bp. In the United States, it is commonly called the "DV01," or the "dollar value of a 01." The PVBP is particularly useful for bonds where future cash flows are uncertain, such as callable bonds. A related statistic called a "basis point value" (BPV) is simply the money duration times $0.0001(1 \mathrm{bp})$.

For a numerical example of the PVBP calculation, consider the $2.875 \%$ semiannual coupon payment US Treasury note that matures on 15 May 2028. In Exhibit 5, the PVBP for the Treasury note is shown to be 0.08540 . Its yield-to-maturity is $2.849091 \%$, and the settlement date is 59 days into a 184-day period. To confirm this result, calculate the new prices by increasing and decreasing the yield-to-maturity. First, increase the yield by $1 \mathrm{bp}(0.01 \%)$, from $2.849091 \%$ to $2.859091 \%$, to solve for a $P V_{+}$of 100.594327 .

$$
\begin{aligned}
& P V_{+}=\left[\frac{1.4375}{\left(1+\frac{0.02859091}{2}\right)^{1}}+\cdots+\frac{101.4375}{\left(1+\frac{0.02859091}{2}\right)^{20}}\right] \times\left(1+\frac{0.02859091}{2}\right)^{59 / 184} \\
= & 100.594327
\end{aligned}
$$

Then, decrease the yield-to-maturity by $1 \mathrm{bp}$, from $2.849091 \%$ to $2.839091 \%$, to solve for a $P V_{-}$of 100.765123 .

$$
\begin{aligned}
& P V_{-}=\left[\frac{1.4375}{\left(1+\frac{0.02839091}{2}\right)^{1}}+\cdots+\frac{101.4375}{\left(1+\frac{0.02839091}{2}\right)^{20}}\right] \times\left(1+\frac{0.02839091}{2}\right)^{59 / 184} \\
= & 100.765123
\end{aligned}
$$

The PVBP is obtained by substituting these results into Equation 13.

PVBP $=\frac{100.765123-100.594327}{2}=0.08540$

Another money duration statistic reported on the Bloomberg YAS page is "risk." It is shown to be 8.540. Bloomberg's risk statistic is simply the PVBP (or PV01) times 100.

\section{EXAMPLE 12}
A life insurance company holds a USD10 million (par value) position in a 5.95\% Dominican Republic bond that matures on 25 January 2027. The bond is priced (flat) at 101.996 per 100 of par value to yield $5.6511 \%$ on a street-convention semiannual bond basis for settlement on 24 July 2018 . The total market value of the position, including accrued interest, is USD10,495,447, or 104.95447 per 100 of par value. The bond's (annual) Macaulay duration is 6.622 .

\begin{enumerate}
  \item Calculate the money duration per 100 in par value for the sovereign bond.
\end{enumerate}

\section{Solution to 1:}
The money duration is the annual modified duration times the full price of the bond per 100 of par value.

$$
\left(\frac{6.622}{1+\frac{0.056511}{2}}\right) \times \text { USD104.95447 }=\text { USD675.91 }
$$

\begin{enumerate}
  \setcounter{enumi}{1}
  \item Using the money duration, estimate the loss on the position for each $1 \mathrm{bp}$ increase in the yield-to-maturity for that settlement date.
\end{enumerate}

\section{Solution to 2:}
For each $1 \mathrm{bp}$ increase in the yield-to-maturity, the loss is estimated to be USD 0.067591 per 100 of par value: USD $675.91 \times 0.0001$ = USD 0.067591 .

Given a position size of USD 10 million in par value, the estimated loss per basis-point increase in the yield is USD 6,759.10. The money duration is per 100 of par value, so the position size of USD10 million is divided by USD 100.

USD0.067591 $\times \frac{\text { USD10,000,000 }}{\text { USD100 }}=$ USD6, 759.10

\section{BOND CONVEXITY}
calculate and interpret approximate convexity and compare approximate and effective convexity

calculate the percentage price change of a bond for a specified change in yield, given the bond's approximate duration and convexity

Modified duration measures the primary effect on a bond's percentage price change given a change in the yield-to-maturity. A secondary effect is measured by the convexity statistic, which is illustrated in Exhibit 10 for a traditional (option-free) fixed-rate bond.

\section{Exhibit 10: Convexity of a Traditional (Option-Free) Fixed-Rate Bond}
\begin{center}
\includegraphics[max width=\textwidth]{2023_05_04_36535b8d80b32081d422g-045}
\end{center}

The true relationship between the bond price and the yield-to-maturity is the curved (convex) line shown in Exhibit 10. This curved line shows the actual bond price given its market discount rate. Duration (in particular, money duration) estimates the change in the bond price along the straight line that is tangent to the curved line. For small yield-to-maturity changes, there is little difference between the lines. But for larger changes, the difference becomes significant.

The convexity statistic for the bond is used to improve the estimate of the percentage price change provided by modified duration alone. Equation 14 is the convexity-adjusted estimate of the percentage change in the bond's full price.

$$
\begin{aligned}
& \% \Delta P V^{\text {Full }} \approx \\
& (- \text { AnnModDur } \times \triangle \text { Yield })+\left[\frac{1}{2} \times \text { AnnConvexity } \times(\Delta \text { Yield })^{2}\right]
\end{aligned}
$$

The first bracketed expression, the "first-order" effect, is the same as Equation 6. The (annual) modified duration, AnnModDur, is multiplied by the change in the (annual) yield-to-maturity, $\Delta$ Yield. The second bracketed expression, the "second-order" effect, is the convexity adjustment. The convexity adjustment is the annual convexity statistic, AnnConvexity, times one-half, multiplied by the change in the yield-to-maturity squared. This additional term is a positive amount on a traditional (option-free) fixed-rate bond for either an increase or decrease in the yield. In Exhibit 10, this amount adds to the linear estimate provided by the duration alone, which brings the adjusted estimate very close to the actual price on the curved line. But it still is an estimate, so the $\approx$ sign is used.

Similar to the Macaulay and modified durations, the annual convexity statistic can be calculated in several ways. It can be calculated using tables, such as Exhibits 2 and 3. It also is possible to derive a closed-form equation for the convexity of a fixed-rate bond on and between coupon payment dates using calculus and algebra (see D. Smith, 2014). But like modified duration, convexity can be approximated with accuracy. Equation 15 is the formula for the approximate convexity statistic, ApproxCon.

$$
\text { ApproxCon }=\frac{\left(P V_{-}\right)+\left(P V_{+}\right)-\left[2 \times\left(P V_{0}\right)\right]}{\left(\Delta \text { Yeld }^{2} \times\left(P V_{0}\right)\right.}
$$

This equation uses the same inputs as Equation 7 for ApproxModDur. The new price when the yield-to-maturity is increased is $P V_{+}$. The new price when the yield is decreased by the same amount is $P V_{-}$. The original price is $P V_{0}$. These are the full prices, including accrued interest, for the bond. The accuracy of this approximation can be demonstrated with the special case of a zero-coupon bond. The absence of coupon payments simplifies the interest rate risk measures. The Macaulay duration of a zero-coupon bond is $N-t / T$ in terms of periods to maturity. The exact convexity statistic of a zero-coupon bond, also in terms of periods, is calculated with Equation 16.

Convexity (of a zero-coupon bond) $=\frac{[N-(t / T)] \times[N+1-(t / T)]}{(1+r)^{2}}$

$N$ is the number of periods to maturity as of the beginning of the current period, $t / T$ is the fraction of the period that has gone by, and $r$ is the yield-to-maturity per period.

For an example of this calculation, consider a long-term, zero-coupon US Treasury bond. The bond's Bloomberg YAS page is shown in Exhibit 11.

\section{Exhibit 11: Bloomberg YAS Page for the Zero-Coupon US Treasury Bond}
\begin{center}
\includegraphics[max width=\textwidth]{2023_05_04_36535b8d80b32081d422g-046}
\end{center}

(c) 2019 Bloomberg L.P. All rights reserved. Reproduced with permission.

The bond matures on 15 February 2048 and its asked price was 42.223649 per 100 of par value for settlement on 13 July 2018. Its yield-to-maturity was $2.935 \%$ stated on a street-convention semiannual bond basis. Even though it is a zero-coupon bond, its yield-to-maturity is based on the actual/actual day-count convention. That settlement date was 148 days into a 181-day period. The annual modified duration was 29.163.

For this bond, $N=60, t / T=148 / 181$, and $r=0.02935 / 2$. Entering these variables into Equation 16 produces a convexity of 3,459.45 in terms of semiannual periods.

$$
\frac{[60-(148 / 181)] \times[60+1-(148 / 181)]}{\left(1+\frac{0.02935}{2}\right)^{2}}=3,459.45
$$

As with the other statistics, convexity is annualized in practice and for use in the convexity adjustment in Equation 14. It is divided by the periodicity squared. The yield-to-maturity on this zero-coupon bond is stated on a semiannual bond basis, meaning a periodicity of 2 . Therefore, the annualized convexity statistic is 864.9.

$$
\frac{3,459.45}{4}=864.9
$$

For example, suppose that the yield-to-maturity is expected to fall by $10 \mathrm{bps}$, from $2.935 \%$ to $2.835 \%$. Given the (annual) modified duration of 29.163 and (annual) convexity of 864.9 , the expected percentage price gain is $2.9595 \%$.

$$
\begin{aligned}
& \% \Delta P V^{\text {Full }} \approx[-29.163 \times-0.0010]+\left[\frac{1}{2} \times 864.9 \times(-0.0010)^{2}\right] \\
& =0.029163+0.000432 \\
& =0.029595
\end{aligned}
$$

Modified duration alone (under)estimates the gain to be $2.9163 \%$. The convexity adjustment adds 4.32 bps.

The long-term, zero-coupon bond of Exhibit 11 demonstrates the difference between yield duration and convexity and curve duration and convexity, even on an option-free bond. Its modified duration is 29.163, whereas its effective duration is 29.530. Its yield convexity is reported on the Bloomberg page to be 8.649 , and its effective convexity is 8.814 . (Note that although Bloomberg scales the convexity statistics by dividing by 100 , either raw or scaled convexity figures are acceptable in practice.) In general, the differences are heightened when the benchmark yield curve is not flat, when the bond has a long time-to-maturity, and the bond is priced at a significant discount or premium.

To obtain the ApproxCon for this long-term, zero-coupon bond, calculate $P V_{0}$, $P V_{+}$, and $P V_{-}$for yields-to-maturity of $2.935 \%, 2.945 \%$, and $2.925 \%$, respectively. For this exercise, $\Delta$ Yield $=0.0001$

$$
\begin{aligned}
& P V_{0}=\frac{100}{\left(1+\frac{0.02935}{2}\right)^{60}} \times\left(1+\frac{0.02935}{2}\right)^{148 / 181}=42.223649 \\
& P V_{+}=\frac{100}{\left(1+\frac{0.02945}{2}\right)^{60}} \times\left(1+\frac{0.02945}{2}\right)^{148 / 181}=42.100694 \\
& P V_{-}=\frac{100}{\left(1+\frac{0.02925}{2}\right)^{60}} \times\left(1+\frac{0.02925}{2}\right)^{148 / 181}=42.346969
\end{aligned}
$$

Using these results, first calculate ApproxModDur using Equation 7 to confirm that these inputs are correct. In Exhibit 11, modified duration is stated to be 29.163.

$$
\text { ApproxModDur }=\frac{42.346969-42.100694}{2 \times 0.0001 \times 42.223649}=29.163
$$

Using Equation 15, ApproxCon is 864.9.

$$
\text { ApproxCon }=\frac{42.346969+42.100694-(2 \times 42.223649)}{(0.0001)^{2} \times 42.223649}=864.9
$$

This result, 864.9, is an approximation for annualized convexity. The number of periods in the year is included in the price calculations. This approximation in this example is the same as the "exact" result using the closed-form equation for the special case of the zero-coupon bond. Any small difference is not likely to be meaningful for practical applications

Because this is an individual zero-coupon bond, it is easy to calculate the new price if the yield-to-maturity does go down by $50 \mathrm{bps}$, to $2.435 \%$.

$$
\frac{100}{\left(1+\frac{0.02435}{2}\right)^{60}} \times\left(1+\frac{0.02435}{2}\right)^{148 / 181}=48.860850
$$

Therefore, the actual percentage price increase is $15.7192 \%$.

$\frac{48.860850-42.223649}{42.223649}=0.157192$

The convexity-adjusted estimate is $15.6626 \%$.

$$
\begin{aligned}
& \% \triangle P V^{\text {Full }} \approx(-29.163 \times-0.0050)+\left[\frac{1}{2} \times 864.9 \times(-0.0050)^{2}\right] \\
& =0.145815+0.010811 \\
& =0.156626
\end{aligned}
$$

\section{EXAMPLE 13}
A Dutch bank holds a large position in a zero-coupon Federal Republic of Germany government bond maturing on 11 April 2025. The yield-to-maturity is $-0.72 \%$ for settlement on 11 May 2020, stated as an effective annual rate on an Actual/Actual basis. That settlement date is 30 days into the 365-day year using this day count method.

\begin{enumerate}
  \item Calculate the full price of the bond per 100 of par value.
\end{enumerate}

\section{Solution to 1:}
There are five years from the beginning of the current period on 11 April 2020 to maturity on 11 April 2025.

The full price of the bond is 103.617526 per 100 of par value. Note that

$1+r=1+(-0.0072)=0.9928$

$$
P V_{0}=\left[\frac{100}{(0.9928)^{5}}\right] \times(0.9928)^{30 / 365}=103.617526
$$

\begin{enumerate}
  \setcounter{enumi}{1}
  \item Calculate the approximate modified duration and approximate convexity using a $1 \mathrm{bp}$ increase and decrease in the yield-to-maturity.
\end{enumerate}

\section{Solution to 2:}
$P V_{+}=103.566215$, and $P V_{-}=103.668868$.

$$
\begin{aligned}
& P V_{+}=\left[\frac{100}{(0.9929)^{5}}\right] \times(0.9929)^{30 / 365}=103.566215 \\
& P V_{-}=\left[\frac{100}{(0.9927)^{5}}\right] \times(0.9927)^{30 / 365}=103.668868
\end{aligned}
$$

The approximate modified duration is 4.9535 .

ApproxModDur $=\frac{103.668868-103.566215}{2 \times 0.0001 \times 103.617526}=4.9535$

The approximate convexity is 29.918 .

$$
\text { ApproxCon }=\frac{103.668868+103.566215-(2 \times 103.617526)}{(0.0001)^{2} \times 103.617526}=29.918
$$

\begin{enumerate}
  \setcounter{enumi}{2}
  \item Calculate the estimated convexity-adjusted percentage price change resulting from a $100 \mathrm{bp}$ increase in the yield-to-maturity.
\end{enumerate}

\section{Solution to 3:}
The convexity-adjusted percentage price drop resulting from a $100 \mathrm{bp}$ increase in the yield-to-maturity is estimated to be $4.80391 \%$. Modified duration alone estimates the percentage drop to be $4.9535 \%$. The convexity adjustment adds 14.96 bps.

$$
\begin{aligned}
& \% \triangle P V^{\text {Full }} \approx-(4.9535 \times 0.0100)+\left[\frac{1}{2} \times 29.918 \times(-0.0100)^{2}\right] \\
& =-0.049535+0.001496 \\
& =-0.0480391
\end{aligned}
$$

\begin{enumerate}
  \setcounter{enumi}{3}
  \item Compare the estimated percentage price change with the actual change, assuming the yield-to-maturity jumps $100 \mathrm{bps}$ to $0.28 \%$ on that settlement date.
\end{enumerate}

\section{Solution to 4:}
The new full price if the yield-to-maturity goes from $-0.72 \%$ to $0.28 \%$ on that settlement date is 98.634349 .

$$
\begin{aligned}
& P F^{\text {Full }}=\left[\frac{100}{(1.0028)^{5}}\right] \times(1.0028)^{30 / 365}=98.634349 \\
& \% \Delta P V^{\text {Full }}=\frac{98.634349-103.617526}{103.617526}=-0.04809203
\end{aligned}
$$

The actual percentage change in the bond price is $-4.809203 \%$. The convexity-adjusted estimate is $-4.80391 \%$, whereas the estimated change using modified duration alone is $-4.9535 \%$.

The money duration of a bond indicates the first-order effect on the full price of a bond in units of currency given a change in the yield-to-maturity. The money convexity statistic (MoneyCon) is the second-order effect. The money convexity of the bond is the annual convexity multiplied by the full price, such that

$$
\Delta P V^{\text {Full }} \approx-(\text { MoneyDur } \times \Delta \text { Yield })+\left[\frac{1}{2} \times \text { MoneyCon } \times(\Delta \text { Yield })^{2}\right]
$$

For a money convexity example, consider again the Nairobi-based life insurance company that has a KES100 million position in the $6.00 \%$ bond that matures on 14 February 2027. Previously, using the money duration alone, the estimated loss was KES6,184,418 if the yield-to-maturity increased by $100 \mathrm{bps}$. The money duration for the position is KES618,441,784. That estimation is improved by including the convexity adjustment. Given the approximate modified duration of 6.1268 for a 5 bp change in the yield-to-maturity $(\triangle$ Yield $=0.0005)$ and given that $P V_{0}=100.940423, P V_{+}=$ 100.631781, and $P V_{-}=101.250227$, we use Equation 15 to calculate the approximate convexity:

$$
\text { ApproxCon }=\frac{101.250227+100.631781-(2 \times 100.940423)}{(0.0005)^{2} \times 100.940423}=46.047
$$

The money convexity is 46.047 times the market value of the position, KES100,940,423. The convexity-adjusted loss given a $100 \mathrm{bp}$ jump in the yield-to-maturity is KES5,952,018:

$$
\begin{aligned}
& -[(6.1268 \times \operatorname{KES} 100,940,423) \times 0.0100]+ \\
& {\left[\frac{1}{2} \times(46.047 \times \operatorname{KES} 100,940,423) \times(0.0100)^{2}\right] } \\
= & -\operatorname{KES} 6,184,418+\mathrm{KES} 232,400 \\
= & - \text { KES5, } 952,018
\end{aligned}
$$

The factors that lead to greater convexity are the same as for duration. A fixed-rate bond with a longer time-to-maturity, a lower coupon rate, and a lower yield-to-maturity has greater convexity than a bond with a shorter time-to-maturity, a higher coupon rate, and a higher yield-to-maturity. Another factor is the dispersion of cash flows, meaning the degree to which payments are spread out over time. If two bonds have the same duration, the one that has the greater dispersion of cash flows has the greater convexity. The positive attributes of greater convexity for an investor are shown in Exhibit 12. Exhibit 12: The Positive Attributes of Greater Bond Convexity on a Traditional (Option-Free) Bond

Price

\begin{center}
\includegraphics[max width=\textwidth]{2023_05_04_36535b8d80b32081d422g-050}
\end{center}

The two bonds in Exhibit 12 are assumed to have the same price, yield-to-maturity, and modified duration. Therefore, they share the same line tangent to their priceyield curves. The benefit of greater convexity occurs when their yields-to-maturity change. For the same decrease in yield-to-maturity, the more convex bond appreciates more in price. And for the same increase in yield-to-maturity, the more convex bond depreciates less in price. The conclusion is that the more convex bond outperforms the less convex bond in both bull (rising price) and bear (falling price) markets. This conclusion assumes, however, that this positive attribute is not "priced into" the bond. To the extent that it is included, the more convex bond would have a higher price (and lower yield-to-maturity). That does not diminish the value of convexity. It only suggests that the investor has to pay for it. As economists say, "There is no such thing as a free lunch."

\section{EXAMPLE 14}
\begin{enumerate}
  \item The investment manager for a UK defined-benefit pension scheme is considering two bonds about to be issued by a large life insurance company. The first is a 30-year, $4 \%$ semiannual coupon payment bond. The second is a 100 -year, 4\% semiannual coupon payment "century" bond. Both bonds are expected to trade at par value at issuance.
\end{enumerate}

Calculate the approximate modified duration and approximate convexity for each bond using a $5 \mathrm{bp}$ increase and decrease in the annual yield-to-maturity. Retain accuracy to six decimals per 100 of par value.

\section{Solution:}
In the calculations, the yield per semiannual period goes up by 2.5 bps to $2.025 \%$ and down by 2.5 bps to $1.975 \%$. The 30 -year bond has an approximate modified duration of 17.381 and an approximate convexity of 420.80 .

$$
\begin{aligned}
& P V_{+}=\frac{2}{(1.02025)^{1}}+\cdots+\frac{102}{(1.02025)^{60}}=99.136214 \\
& P V_{-}=\frac{2}{(1.01975)^{1}}+\cdots+\frac{102}{(1.01975)^{60}}=100.874306
\end{aligned}
$$

ApproxModDur $=\frac{100.874306-99.136214}{2 \times 0.0005 \times 100}=17.381$

ApproxCon $=\frac{100.874306+99.136214-(2 \times 100)}{(0.0005)^{2} \times 100}=420.80$

The 100-year century bond has an approximate modified duration of 24.527 and an approximate convexity of $1,132.88$.

$$
\begin{aligned}
& P V_{+}=\frac{2}{(1.02025)^{1}}+\cdots+\frac{102}{(1.02025)^{200}}=98.787829 \\
& P V_{-}=\frac{2}{(1.01975)^{1}}+\cdots+\frac{102}{(1.01975)^{200}}=101.240493
\end{aligned}
$$

ApproxModDur $=\frac{101.240493-98.787829}{2 \times 0.0005 \times 100}=24.527$

ApproxCon $=\frac{101.240493+98.787829-(2 \times 100)}{(0.0005)^{2} \times 100}=1,132.88$

The century bond offers a higher modified duration, 24.527 compared with 17.381, and a much greater degree of convexity, 1,132.88 compared with 420.80

In the same manner that the primary, or first-order, effect of a shift in the benchmark yield curve is measured by effective duration, the secondary, or second-order, effect is measured by effective convexity. The effective convexity of a bond is a curve convexity statistic that measures the secondary effect of a change in a benchmark yield curve. A pricing model is used to determine the new prices when the benchmark curve is shifted upward $\left(P V_{+}\right)$and downward $\left(P V_{-}\right)$by the same amount ( $\Delta$ Curve). These changes are made holding other factors constant-for example, the credit spread. Then, Equation 18 is used to calculate the effective convexity (EffCon) given the initial price $\left(P V_{0}\right)$.

EffCon $=\frac{\left[\left(P V_{-}\right)+\left(P V_{+}\right)\right]-\left[2 \times\left(P V_{0}\right)\right]}{(\Delta \mathrm{Curve})^{2} \times\left(P V_{0}\right)}$

This equation is very similar to Equation 15, for approximate yield convexity. The difference is that in Equation 15, the denominator includes the change in the yield-to-maturity squared, $(\Delta \text { Yield })^{2}$. Here, the denominator includes the change in the benchmark yield curve squared, ( $\Delta$ Curve $)^{2}$.

Consider again the callable bond example in our initial discussion of effective duration. It is assumed that an option-pricing model is used to generate these callable bond prices: $P V_{0}=101.060489, P V_{+}=99.050120, P V_{-}=102.890738$, and $\Delta$ Curve $=$ 0.0025 . The effective duration for the callable bond is 7.6006 .

EffDur $=\frac{102.890738-99.050120}{2 \times 0.0025 \times 101.060489}=7.6006$

Using these inputs in Equation 18, the effective convexity is -285.17.

EffCon $=\frac{102.890738+99.050120-(2 \times 101.060489)}{(0.0025)^{2} \times 101.060489}=-285.17$

Negative convexity, which could be called "concavity" is an important feature of callable bonds. Putable bonds, on the other hand, always have positive convexity. As a second-order effect, effective convexity indicates the change in the first-order effect (i.e., effective duration) as the benchmark yield curve is changed. In Exhibit 8 , as the benchmark yield goes down, the slope of the line tangent to the curve for the non-callable bond steepens, which indicates positive convexity. But the slope of the line tangent to the callable bond flattens as the benchmark yield goes down. Technically, it reaches an inflection point, which is when the effective convexity shifts from positive to negative.

In summary, when the benchmark yield is high and the value of the embedded call option is low, the callable and the non-callable bonds experience very similar effects from interest rate changes. They both have positive convexity. But as the benchmark yield is reduced, the curves diverge. At some point, the callable bond moves into the range of negative convexity, which indicates that the embedded call option has more value to the issuer and is more likely to be exercised. This situation limits the potential price appreciation of the bond arising from lower interest rates, whether because of a lower benchmark yield or a lower credit spread.

Another way to understand why a callable bond can have negative convexity is to rearrange Equation 18.

$$
\text { EffCon }=\frac{\left[\left(P V_{-}\right)-\left(P V_{0}\right)\right]-\left[\left(P V_{0}\right)-\left(P V_{+}\right)\right]}{(\triangle \text { Curve })^{2} \times\left(P V_{0}\right)}
$$

In the numerator, the first bracketed expression is the increase in price when the benchmark yield curve is lowered. The second expression is the decrease in price when the benchmark yield curve is raised. On a non-callable bond, the increase is always larger than the decrease (in absolute value). This result is the "convexity effect" for the relationship between bond prices and yields-to-maturity. On a callable bond, the increase can be smaller than the decrease (in absolute value). That creates negative convexity, as illustrated in Exhibit 8.

INVESTMENT HORIZON, MACAULAY DURATION AND INTEREST RATE RISK

describe how the term structure of yield volatility affects the interest rate risk of a bond describe the relationships among a bond's holding period return, its duration, and the investment horizon

This section explores the effect of yield volatility on the investment horizon, and on the interaction between the investment horizon, market price risk, and coupon reinvestment risk.

\section{Yield Volatility}
An important aspect in understanding the interest rate risk and return characteristics of an investment in a fixed-rate bond is the time horizon. This section considers a short-term horizon. A primary concern for the investor is the change in the price of the bond given a sudden (i.e., same-day) change in its yield-to-maturity. The accrued interest does not change, so the impact of the change in the yield is on the flat price of the bond. Next, we consider a long-term horizon. The reinvestment of coupon interest then becomes a key factor in the investor's horizon yield.

Bond duration is the primary measure of risk arising from a change in the yield-to-maturity. Convexity is the secondary risk measure. In the discussion of the impact on the bond price, the phrase "for a given change in the yield-to-maturity" is used repeatedly. For instance, the given change in the yield-to-maturity could be $1 \mathrm{bp}$, $25 \mathrm{bps}$, or $100 \mathrm{bps}$. In comparing two bonds, it is assumed that the "given change" is the same for both securities. When the government bond par curve is shifted up or down by the same amount to calculate effective duration and effective convexity, the events are described as "parallel" yield curve shifts. Because yield curves are rarely (if ever) straight lines, this shift may also be described as a "shape-preserving" shift to the yield curve. The key assumption is that all yields-to-maturity under consideration rise or fall by the same amount across the curve.

Although the assumption of a parallel shift in the yield curve is common in fixed-income analysis, it is not always realistic. The shape of the yield curve changes based on factors affecting the supply and demand of shorter-term versus longer-term securities. In fact, the term structure of bond yields (also called the "term structure of interest rates") is typically upward sloping. However, the term structure of yield volatility may have a different shape depending on a number of factors. The term structure of yield volatility is the relationship between the volatility of bond yields-to-maturity and times-to-maturity.

For example, a central bank engaging in expansionary monetary policy might cause the yield curve to steepen by reducing short-term interest rates. But this policy might cause greater volatility in short-term bond yields-to-maturity than in longer-term bonds, resulting in a downward-sloping term structure of yield volatility. Longer-term bond yields are mostly determined by future inflation and economic growth expectations. Those expectations often tend to be less volatile.

The importance of yield volatility in measuring interest rate risk is that bond price changes are products of two factors: (1) the impact per basis-point change in the yield-to-maturity and (2) the number of basis points in the yield-to-maturity change. The first factor is duration or the combination of duration and convexity, and the second factor is the yield volatility. For example, consider a 5-year bond with a modified duration of 4.5 and a 30-year bond with a modified duration of 18.0. Clearly, for a given change in yield-to-maturity, the 30-year bond represents much more interest rate risk to an investor who has a short-term horizon. In fact, the 30-year bond appears to have four times the risk given the ratio of the modified durations. But that assumption neglects the possibility that the 30 -year bond might have half the yield volatility of the 5-year bond.

Equation 14, restated here, summarizes the two factors.

$\% \Delta P V^{\text {Full }} \approx(-$ AnnModDur $\times \Delta$ Yield $)+\left[\frac{1}{2} \times\right.$ AnnConvexity $\left.\times(\Delta \text { Yield })^{2}\right]$

The estimated percentage change in the bond price depends on the modified duration and convexity as well as on the yield-to-maturity change. Parallel shifts between two bond yields and along a benchmark yield curve are common assumptions in fixed-income analysis. However, an analyst must be aware that non-parallel shifts frequently occur in practice.

\section{EXAMPLE 15}
\begin{enumerate}
  \item A fixed-income analyst is asked to rank three bonds in terms of interest rate risk. Interest rate risk here means the potential price decrease on a percentage basis given a sudden change in financial market conditions. The increases in the yields-to-maturity represent the "worst case" for the scenario being considered.
\end{enumerate}

\begin{center}
\begin{tabular}{lccc}
\hline
Bond & Modified Duration & Convexity & $\Delta$ Yield \\
\hline
A & 3.72 & 12.1 & $25 \mathrm{bps}$ \\
B & 5.81 & 40.7 & $15 \mathrm{bps}$ \\
\end{tabular}
\end{center}

\begin{center}
\begin{tabular}{lccc}
\hline
Bond & Modified Duration & Convexity & $\Delta$ Yield \\
\hline
$\mathrm{C}$ & 12.39 & 158.0 & $10 \mathrm{bps}$ \\
\hline
\end{tabular}
\end{center}

The modified duration and convexity statistics are annualized. $\Delta$ Yield is the increase in the annual yield-to-maturity. Rank the bonds in terms of interest rate risk.

\section{Solution:}
Calculate the estimated percentage price change for each bond:

\section{Bond A:}
$(-3.72 \times 0.0025)+\left[\frac{1}{2} \times 12.1 \times(0.0025)^{2}\right]=-0.009262$

\section{Bond B:}
$(-5.81 \times 0.0015)+\left[\frac{1}{2} \times 40.7 \times(0.0015)^{2}\right]=-0.008669$

\section{Bond C:}
$$
(-12.39 \times 0.0010)+\left[\frac{1}{2} \times 158.0 \times(0.0010)^{2}\right]=-0.012311
$$

Based on these assumed changes in the yield-to-maturity and the modified duration and convexity risk measures, Bond $\mathrm{C}$ has the highest degree of interest rate risk (a potential loss of 1.2311\%), followed by Bond A (a potential loss of $0.9262 \%$ ) and Bond B (a potential loss of $0.8669 \%$ ).

\section{Investment Horizon, Macaulay Duration, and Interest Rate Risk}
Although short-term interest rate risk is a concern to some investors, other investors have a long-term horizon. Day-to-day changes in bond prices cause unrealized capital gains and losses. Those unrealized gains and losses might need to be accounted for in financial statements. This section considers a long-term investor concerned only with the total return over the investment horizon. Therefore, interest rate risk is important to this investor. The investor faces coupon reinvestment risk as well as market price risk if the bond needs to be sold prior to maturity.

Earlier, we discussed examples of interest rate risk using a 10-year, $8 \%$ annual coupon payment bond that is priced at 85.503075 per 100 of par value. The bond's yield-to-maturity is $10.40 \%$. A key result in Example 3 is that an investor with a 10-year time horizon is concerned only with coupon reinvestment risk. This situation assumes, of course, that the issuer makes all of the coupon and principal payments as scheduled. The buy-and-hold investor has a higher total return if interest rates rise (see Example 3) and a lower total return if rates fall (see Example 5). The investor in Examples 4 and 6 has a four-year horizon. This investor faces market price risk in addition to coupon reinvestment risk. In fact, the market price risk dominates because this investor has a higher total return if interest rates fall (see Example 6) and a lower return if rates rise (see Example 4).

Now, consider a third investor who has a seven-year time horizon. If interest rates remain at $10.40 \%$, the future value of reinvested coupon interest is 76.835787 per 100 of par value.

$$
\begin{aligned}
& {\left[8 \times(1.1040)^{6}\right]+\left[8 \times(1.1040)^{5}\right]+\left[8 \times(1.1040)^{4}\right]+\left[8 \times(1.1040)^{3}\right]+} \\
& {\left[8 \times(1.1040)^{2}\right]+\left[8 \times(1.1040)^{1}\right]+8=76.835787}
\end{aligned}
$$

The bond is sold for a price of 94.073336 , assuming that the bond stays on the constant-yield price trajectory and continues to be "pulled to par."

$$
\frac{8}{(1.1040)^{1}}+\frac{8}{(1.1040)^{2}}+\frac{108}{(1.1040)^{3}}=94.073336
$$

The total return is $170.909123(=76.835787+94.073336)$ per 100 of par value, and the horizon yield, as expected, is $10.40 \%$.

$$
85.503075=\frac{170.909123}{(1+r)^{7}}, \quad r=0.1040
$$

Following Examples 3 and 4, assume that the yield-to-maturity on the bond rises to $11.40 \%$. Also, coupon interest is now reinvested each year at $11.40 \%$. The future value of reinvested coupons becomes 79.235183 per 100 of par value.

$$
\begin{aligned}
& {\left[8 \times(1.1140)^{6}\right]+\left[8 \times(1.1140)^{5}\right]+\left[8 \times(1.1140)^{4}\right]+\left[8 \times(1.1140)^{3}\right]+} \\
& {\left[8 \times(1.1140)^{2}\right]+\left[8 \times(1.1140)^{1}\right]+8=79.235183}
\end{aligned}
$$

After receiving the seventh coupon payment, the bond is sold. There is a capital loss because the price, although much higher than at purchase, is below the constant-yield price trajectory.

$$
\frac{8}{(1.1140)^{1}}+\frac{8}{(1.1140)^{2}}+\frac{108}{(1.1140)^{3}}=91.748833
$$

The total return is $170.984016(=79.235183+91.748833)$ per 100 of par value and the holding-period rate of return is $10.407 \%$.

$$
85.503075=\frac{170.984016}{(1+r)^{7}}, \quad r=0.10407
$$

Following Examples 5 and 6, assume that the coupon reinvestment rates and the bond yield-to-maturity fall to $9.40 \%$. The future value of reinvested coupons is 74.512177 .

$$
\begin{aligned}
& {\left[8+(1.0940)^{6}\right]+\left[8+(1.0940)^{5}\right]+\left[8+(1.0940)^{4}\right]+\left[8+(1.0940)^{3}\right]+} \\
& {\left[8+(1.0940)^{2}\right]+\left[8+(1.0940)^{1}\right]+8=74.52177}
\end{aligned}
$$

The bond is sold at a capital gain because the price is above the constant-yield price trajectory.

$$
\frac{8}{(1.0940)^{1}}+\frac{8}{(1.0940)^{2}}+\frac{108}{(1.0940)^{3}}=96.481299
$$

The total return is $170.993476(=74.512177+96.481299)$ per 100 of par value, and the horizon yield is $10.408 \%$.

$$
85.503075=\frac{170.993476}{(1+r)^{7}}, \quad r=0.10408
$$

These results are summarized in the following table to reveal the remarkable outcome: The total returns and horizon yields are virtually the same. The investor with the 7-year horizon, unlike those having a 4- or 10-year horizon, achieves the same holding-period rate of return whether interest rates rise, fall, or remain the same. Note that the terms "horizon yield" and "holding-period rate of return" are used interchangeably in this reading. Sometimes "horizon yield" refers to yields on bonds that need to be sold at the end of the investor's holding period.

\begin{center}
\begin{tabular}{ccccc}
\hline
$\begin{array}{l}\text { Interest } \\ \text { Rate }\end{array}$ & $\begin{array}{c}\text { Future Value of } \\ \text { Reinvested Coupon }\end{array}$ & Sale Price & Total Return & Horizon Yield \\
\hline
$9.40 \%$ & 74.512177 & 96.481299 & 170.993476 & $10.408 \%$ \\
$10.40 \%$ & 76.835787 & 94.073336 & 170.909123 & $10.400 \%$ \\
\end{tabular}
\end{center}

\begin{center}
\begin{tabular}{lcccc}
\hline
$\begin{array}{l}\text { Interest } \\ \text { Rate }\end{array}$ & $\begin{array}{c}\text { Future Value of } \\ \text { Reinvested Coupon }\end{array}$ & Sale Price & Total Return & Horizon Yield \\
\hline
$11.40 \%$ & 79.235183 & 91.748833 & 170.984016 & $10.407 \%$ \\
\hline
\end{tabular}
\end{center}

This particular bond was chosen as an example to demonstrate an important property of Macaulay duration: For a particular assumption about yield volatility, Macaulay duration indicates the investment horizon for which coupon reinvestment risk and market price risk offset each other. In Exhibit 2, the Macaulay duration of this 10-year, $8 \%$ annual payment bond is calculated to be 7.0029 years. This is one of the applications for duration in which "years" is meaningful and in which Macaulay duration is used rather than modified duration. The particular assumption about yield volatility is that there is a one-time "parallel" shift in the yield curve that occurs before the next coupon payment date. Exhibit 13 illustrates this property of bond duration, assuming that the bond is initially priced at par value.

Exhibit 13: Interest Rate Risk, Macaulay Duration, and the Investment

Horizon

\section{A. Interest Rates Rise}
\begin{center}
\includegraphics[max width=\textwidth]{2023_05_04_36535b8d80b32081d422g-056(1)}
\end{center}

\section{B. Interest Rates Fall}
\begin{center}
\includegraphics[max width=\textwidth]{2023_05_04_36535b8d80b32081d422g-056}
\end{center}

As demonstrated in Panel A of Exhibit 13, when interest rates rise, duration measures the immediate drop in value. In particular, the money duration indicates the change in price. Then as time passes, the bond price is "pulled to par." The gain in the future value of reinvested coupons starts small but builds over time as more coupons are received. The curve indicates the additional future value of reinvested coupons because of the higher interest rate. At some point in the lifetime of the bond, those two effects offset each other and the gain on reinvested coupons is equal to the loss on the sale of the bond. That point in time is the Macaulay duration statistic. The same pattern is displayed in the Panel B when interest rates fall, which leads to a reduction in the bond yield and the coupon reinvestment rate. There is an immediate jump in the bond price, as measured by the money duration, but then the "pull to par" effect brings the price down as time passes. The impact from reinvesting at a lower rate starts small but then becomes more significant over time. The loss on reinvested coupons is with respect to the future value if interest rates had not fallen. Once again, the bond's Macaulay duration indicates the point in time when the two effects offset each other and the gain on the sale of the bond matches the loss on coupon reinvestment.

The earlier numerical example and Exhibit 13 allow for a statement of the general relationships among interest rate risk, the Macaulay duration, and the investment horizon.

\begin{enumerate}
  \item When the investment horizon is greater than the Macaulay duration of a bond, coupon reinvestment risk dominates market price risk. The investor's risk is to lower interest rates.

  \item When the investment horizon is equal to the Macaulay duration of a bond, coupon reinvestment risk offsets market price risk.

  \item When the investment horizon is less than the Macaulay duration of the bond, market price risk dominates coupon reinvestment risk. The investor's risk is to higher interest rates.

\end{enumerate}

In the numerical example, the Macaulay duration of the bond is 7.0 years. Statement 1 reflects the investor with the 10 -year horizon; Statement 2 , the investor with the 7-year horizon; and Statement 3, the investor with the 4-year horizon.

The difference between the Macaulay duration of a bond and the investment horizon is called the duration gap. The duration gap is a bond's Macaulay duration minus the investment horizon. The investor with the 10-year horizon has a negative duration gap and currently is at risk of lower rates. The investor with the 7-year horizon has a duration gap of zero and currently is hedged against interest rate risk. The investor with the 4-year horizon has a positive duration gap and currently is at risk of higher rates. The word "currently" is important because interest rate risk is connected to an immediate change in the bond's yield-to-maturity and the coupon reinvestment rates. As time passes, the investment horizon is reduced and the Macaulay duration of the bond also changes. Therefore, the duration gap changes as well.

\section{EXAMPLE 16}
An investor plans to retire in 10 years. As part of the retirement portfolio, the investor buys a newly issued, 12-year, $8 \%$ annual coupon payment bond. The bond is purchased at par value, so its yield-to-maturity is $8.00 \%$ stated as an effective annual rate.

\begin{enumerate}
  \item Calculate the approximate Macaulay duration for the bond, using a 1 bp increase and decrease in the yield-to-maturity and calculating the new prices per 100 of par value to six decimal places.
\end{enumerate}

\section{Solution to 1:}
The approximate modified duration of the bond is 7.5361. $P V_{0}=100, P V_{+}=$ 99.924678 , and $P V_{-}=100.075400$.

$$
\begin{aligned}
& P V_{+}=\frac{8}{(1.0801)^{1}}+\cdots+\frac{108}{(1.0801)^{12}}=99.924678 \\
& P V_{-}=\frac{8}{(1.0799)^{1}}+\cdots+\frac{108}{(1.0799)^{12}}=100.075400
\end{aligned}
$$

ApproxModDur $=\frac{100.075400-99.924678}{2 \times 0.0001 \times 100}=7.5361$

The approximate Macaulay duration is $8.1390(=7.5361 \times 1.08)$.

\begin{enumerate}
  \setcounter{enumi}{1}
  \item Calculate the duration gap at the time of purchase.
\end{enumerate}

\section{Solution to 2:}
Given an investment horizon of 10 years, the duration gap for this bond at purchase is negative: $8.1390-10=-1.8610$.

\begin{enumerate}
  \setcounter{enumi}{2}
  \item Does this bond at purchase entail the risk of higher or lower interest rates? Interest rate risk here means an immediate, one-time, parallel yield curve shift.
\end{enumerate}

\section{Solution to 3:}
A negative duration gap entails the risk of lower interest rates. To be precise, the risk is an immediate, one-time, parallel, downward yield curve shift because coupon reinvestment risk dominates market price risk. The loss from reinvesting coupons at a rate lower than $8 \%$ is larger than the gain from selling the bond at a price above the constant-yield price trajectory.

\section{CREDIT AND LIQUIDITY RISK}
explain how changes in credit spread and liquidity affect yield-to-maturity of a bond and how duration and convexity can be used to estimate the price effect of the changes

The focus of this reading is to demonstrate how bond duration and convexity estimate the bond price change, either in percentage terms or in units of currency, given an assumed yield-to-maturity change. This section addresses the source of the change in the yield-to-maturity. In general, the yield-to-maturity on a corporate bond is composed of a government benchmark yield and a spread over that benchmark. A change in the bond's yield-to-maturity can originate in either component or a combination of the two.

The key point is that for a traditional (option-free) fixed-rate bond, the same duration and convexity statistics apply for a change in the benchmark yield as for a change in the spread. The "building blocks" approach covered in an earlier reading shows that these yield-to-maturity changes can be broken down further. A change in the benchmark yield can arise from a change in either the expected inflation rate or the expected real rate of interest. A change in the spread can arise from a change in the credit risk of the issuer or in the liquidity of the bond. Therefore, for a fixed-rate bond, the "inflation duration," the "real rate duration," the "credit duration," and the "liquidity duration" are all the same number. The inflation duration would indicate the change in the bond price if expected inflation were to change by a certain amount. In the same manner, the real rate duration would indicate the bond price change if the real rate were to go up or down. The credit duration and liquidity duration would indicate the price sensitivity that would arise from changes in those building blocks in the yield-to-maturity. A bond with a modified duration of 5.00 and a convexity of 32.00 will appreciate in value by about $1.26 \%$ if its yield-to-maturity goes down by 25 bps: $(-5.00 \times-0.0025)+\left[1 / 2 \times 32.00 \times(-0.0025)^{2}\right]=+0.0126$, regardless of the source of the yield-to-maturity change.

Suppose that the yield-to-maturity on a corporate bond is $6.00 \%$. If the benchmark yield is $4.25 \%$, the spread is $1.75 \%$. An analyst believes that credit risk makes up $1.25 \%$ of the spread and liquidity risk, the remaining $0.50 \%$. Credit risk includes the probability of default as well as the recovery of assets if default does occur. A credit rating downgrade or an adverse change in the ratings outlook for a borrower reflects a higher risk of default. Liquidity risk refers to the transaction costs associated with selling a bond. In general, a bond with greater frequency of trading and a higher volume of trading provides fixed-income investors with more opportunity to purchase or sell the security and thus has less liquidity risk. In practice, there is a difference between the bid (or purchase) and the offer (or sale) price. This difference depends on the type of bond, the size of the transaction, and the time of execution, among other factors. For instance, government bonds often trade at just a few basis points between the purchase and sale prices. More thinly traded corporate bonds can have a much wider difference between the bid and offer prices.

The problem for a fixed-income analyst is that it is rare for the changes in the components of the overall yield-to-maturity to occur in isolation. In practice, the analyst is concerned with the interaction between changes in benchmark yields and spreads, between changes in expected inflation and the expected real rate, and between changes in credit and liquidity risk. For example, during a financial crisis, a "flight to quality" can cause government benchmark yields to fall as credit spreads widen. An unexpected credit downgrade on a corporate bond can result in greater credit as well as liquidity risk.

\section{EXAMPLE 17}
\begin{enumerate}
  \item The (flat) price on a fixed-rate corporate bond falls one day from 92.25 to 91.25 per 100 of par value because of poor earnings and an unexpected ratings downgrade of the issuer. The (annual) modified duration for the bond is 7.24. Which of the following is closest to the estimated change in the credit spread on the corporate bond, assuming benchmark yields are unchanged?
A. $15 \mathrm{bps}$
B. $100 \mathrm{bps}$
C. $108 \mathrm{bps}$
\end{enumerate}

\section{Solution:}
Given that the price falls from 92.25 to 91.25 , the percentage price decrease is $1.084 \%$.

$$
\frac{91.25-92.25}{92.25}=-0.01084
$$

Given an annual modified duration of 7.24 , the change in the yield-to-maturity is $14.97 \mathrm{bps}$.

$$
-0.01084 \approx-7.24 \times \Delta \text { Yield, } \Delta \text { Yield }=0.001497
$$

Therefore, the answer is A. The change in price reflects a credit spread increase on the bond of about $15 \mathrm{bps}$.

\section{2}
\section{EMPIRICAL DURATION}
describe the difference between empirical duration and analytical duration

The approach taken in this reading to estimate duration and convexity statistics using mathematical formulas is often referred to as analytical duration. These estimates of the impact of benchmark yield changes on bond prices implicitly assume that government bond yields and spreads are independent variables that are uncorrelated with one another. Analytical duration offers a reasonable approximation of the price-yield relationship in many situations, but fixed-income professionals often use historical data in statistical models that incorporate various factors affecting bond prices to calculate empirical duration estimates. These estimates calculated over time and in different interest rate environments inform the fixed-income portfolio decision-making process and will be addressed in detail in later readings.

For instance, in the "flight to quality" example cited earlier in which investors sell risky assets during market turmoil and purchase default-risk-free government bonds, we might expect analytical and empirical duration estimates to differ among bond types. For example, on the one hand, for a government bond with little or no credit risk, we would expect analytical and empirical duration to be similar because benchmark yield changes largely drive bond prices. On the other hand, the same macroeconomic factors driving government bond yields lower in a market stress scenario will cause high-yield bond credit spreads to widen because of an increase in expected default risk. Since credit spreads and benchmark yields are negatively correlated under this scenario, wider credit spreads will partially or fully offset the decline in government benchmark yields, resulting in lower empirical duration estimates than analytical duration estimates. Importantly, analysts must consider the correlation between benchmark yields and credit spreads when deciding whether to use empirical or analytical duration estimates.

\section{EXAMPLE 18}
\begin{enumerate}
  \item AFC Investment Ltd. is a fixed-income investment firm that actively manages a government bond fund and a corporate bond fund. Holdings of the government bond fund are mainly medium-term US Treasury securities but also include debt of highly rated developed-market sovereign issuers. About half of the corporate bond fund is invested in investment-grade issues, and the other half consists of high-yield issues, all with a mix of maturities and from a mix of North American, European, and Asian companies.
\end{enumerate}

Explain why empirical duration is likely to be a more accurate risk measure for AFC's corporate bond fund than for its government bond fund.

\section{Solution:}
The government bond fund includes debt securities of the US government and other highly rated developed-market sovereign issuers. Since benchmark yields are the primary driver of changes in overall bond yields in this fund, the results of analytical duration and empirical duration should be broadly similar.

The corporate bond fund includes a wide variety of debt securities with varying levels of credit quality and liquidity and, therefore, different credit and liquidity spreads. Interactions between benchmark yield changes and credit and liquidity spreads would tend to offset each other, particularly during stressed market conditions, making empirical duration significantly lower than analytical duration. As a result, empirical duration may be the more accurate risk measure for the corporate bond fund.

\section{SUMMARY}
This reading covers the risk and return characteristics of fixed-rate bonds. The focus is on the widely used measures of interest rate risk-duration and convexity. These statistics are used extensively in fixed-income analysis. The following are the main points made in the reading:

\begin{itemize}
  \item The three sources of return on a fixed-rate bond purchased at par value are: (1) receipt of the promised coupon and principal payments on the scheduled dates, (2) reinvestment of coupon payments, and (3) potential capital gains, as well as losses, on the sale of the bond prior to maturity.

  \item For a bond purchased at a discount or premium, the rate of return also includes the effect of the price being "pulled to par" as maturity nears, assuming no default.

  \item The total return is the future value of reinvested coupon interest payments and the sale price (or redemption of principal if the bond is held to maturity).

  \item The horizon yield (or holding period rate of return) is the internal rate of return between the total return and purchase price of the bond.

  \item Coupon reinvestment risk increases with a higher coupon rate and a longer reinvestment time period.

  \item Capital gains and losses are measured from the carrying value of the bond and not from the purchase price. The carrying value includes the amortization of the discount or premium if the bond is purchased at a price below or above par value. The carrying value is any point on the constant-yield price trajectory.

  \item Interest income on a bond is the return associated with the passage of time. Capital gains and losses are the returns associated with a change in the value of a bond as indicated by a change in the yield-to-maturity.

  \item The two types of interest rate risk on a fixed-rate bond are coupon reinvestment risk and market price risk. These risks offset each other to a certain extent. An investor gains from higher rates on reinvested coupons but loses if the bond is sold at a capital loss because the price is below the constant-yield price trajectory. An investor loses from lower rates on reinvested coupon but gains if the bond is sold at a capital gain because the price is above the constant-yield price trajectory.

  \item Market price risk dominates coupon reinvestment risk when the investor has a short-term horizon (relative to the time-to-maturity on the bond). - Coupon reinvestment risk dominates market price risk when the investor has a long-term horizon (relative to the time-to-maturity)-for instance, a buy-and-hold investor.

  \item Bond duration, in general, measures the sensitivity of the full price (including accrued interest) to a change in interest rates.

  \item Yield duration statistics measuring the sensitivity of a bond's full price to the bond's own yield-to-maturity include the Macaulay duration, modified duration, money duration, and price value of a basis point.

  \item Curve duration statistics measuring the sensitivity of a bond's full price to the benchmark yield curve are usually called "effective durations."

  \item Macaulay duration is the weighted average of the time to receipt of coupon interest and principal payments, in which the weights are the shares of the full price corresponding to each payment. This statistic is annualized by dividing by the periodicity (number of coupon payments or compounding periods in a year).

  \item Modified duration provides a linear estimate of the percentage price change for a bond given a change in its yield-to-maturity.

  \item Approximate modified duration approaches modified duration as the change in the yield-to-maturity approaches zero.

  \item Effective duration is very similar to approximate modified duration. The difference is that approximate modified duration is a yield duration statistic that measures interest rate risk in terms of a change in the bond's own yield-to-maturity, whereas effective duration is a curve duration statistic that measures interest rate risk assuming a parallel shift in the benchmark yield curve.

  \item Key rate duration is a measure of a bond's sensitivity to a change in the benchmark yield curve at specific maturity segments. Key rate durations can be used to measure a bond's sensitivity to changes in the shape of the yield curve.

  \item Bonds with an embedded option do not have a meaningful internal rate of return because future cash flows are contingent on interest rates. Therefore, effective duration is the appropriate interest rate risk measure, not modified duration.

  \item The effective duration of a traditional (option-free) fixed-rate bond is its sensitivity to the benchmark yield curve, which can differ from its sensitivity to its own yield-to-maturity. Therefore, modified duration and effective duration on a traditional (option-free) fixed-rate bond are not necessarily equal.

  \item During a coupon period, Macaulay and modified durations decline smoothly in a "saw-tooth" pattern, assuming the yield-to-maturity is constant. When the coupon payment is made, the durations jump upward.

  \item Macaulay and modified durations are inversely related to the coupon rate and the yield-to-maturity.

  \item Time-to-maturity and Macaulay and modified durations are usually positively related. They are always positively related on bonds priced at par or at a premium above par value. They are usually positively related on bonds priced at a discount below par value. The exception is on long-term, low-coupon bonds, on which it is possible to have a lower duration than on an otherwise comparable shorter-term bond. - The presence of an embedded call option reduces a bond's effective duration compared with that of an otherwise comparable non-callable bond. The reduction in the effective duration is greater when interest rates are low and the issuer is more likely to exercise the call option.

  \item The presence of an embedded put option reduces a bond's effective duration compared with that of an otherwise comparable non-putable bond. The reduction in the effective duration is greater when interest rates are high and the investor is more likely to exercise the put option.

  \item The duration of a bond portfolio can be calculated in two ways: (1) the weighted average of the time to receipt of aggregate cash flows and (2) the weighted average of the durations of individual bonds that compose the portfolio.

  \item The first method to calculate portfolio duration is based on the cash flow yield, which is the internal rate of return on the aggregate cash flows. It cannot be used for bonds with embedded options or for floating-rate notes.

  \item The second method is simpler to use and quite accurate when the yield curve is relatively flat. Its main limitation is that it assumes a parallel shift in the yield curve in that the yields on all bonds in the portfolio change by the same amount.

  \item Money duration is a measure of the price change in terms of units of the currency in which the bond is denominated.

  \item The price value of a basis point (PVBP) is an estimate of the change in the full price of a bond given a 1 bp change in the yield-to-maturity.

  \item Modified duration is the primary, or first-order, effect on a bond's percentage price change given a change in the yield-to-maturity. Convexity is the secondary, or second-order, effect. It indicates the change in the modified duration as the yield-to-maturity changes.

  \item Money convexity is convexity times the full price of the bond. Combined with money duration, money convexity estimates the change in the full price of a bond in units of currency given a change in the yield-to-maturity.

  \item Convexity is a positive attribute for a bond. Other things being equal, a more convex bond appreciates in price more than a less convex bond when yields fall and depreciates less when yields rise.

  \item Effective convexity is the second-order effect on a bond price given a change in the benchmark yield curve. It is similar to approximate convexity. The difference is that approximate convexity is based on a yield-to-maturity change and effective convexity is based on a benchmark yield curve change.

  \item Callable bonds have negative effective convexity when interest rates are low. The increase in price when the benchmark yield is reduced is less in absolute value than the decrease in price when the benchmark yield is raised.

  \item The change in a bond price is the product of: (1) the impact per basis-point change in the yield-to-maturity and (2) the number of basis points in the yield change. The first factor is estimated by duration and convexity. The second factor depends on yield volatility.

  \item The investment horizon is essential in measuring the interest rate risk on a fixed-rate bond.

  \item For a particular assumption about yield volatility, the Macaulay duration indicates the investment horizon for which coupon reinvestment risk and market price risk offset each other. The assumption is a one-time parallel shift to the yield curve in which the yield-to-maturity and coupon reinvestment rates change by the same amount in the same direction. - When the investment horizon is greater than the Macaulay duration of the bond, coupon reinvestment risk dominates price risk. The investor's risk is to lower interest rates. The duration gap is negative.

  \item When the investment horizon is equal to the Macaulay duration of the bond, coupon reinvestment risk offsets price risk. The duration gap is zero.

  \item When the investment horizon is less than the Macaulay duration of the bond, price risk dominates coupon reinvestment risk. The investor's risk is to higher interest rates. The duration gap is positive.

  \item Credit risk involves the probability of default and degree of recovery if default occurs, whereas liquidity risk refers to the transaction costs associated with selling a bond.

  \item For a traditional (option-free) fixed-rate bond, the same duration and convexity statistics apply if a change occurs in the benchmark yield or a change occurs in the spread. The change in the spread can result from a change in credit risk or liquidity risk.

  \item In practice, there often is interaction between changes in benchmark yields and in the spread over the benchmark.

  \item Empirical duration uses statistical methods and historical bond prices to derive the price-yield relationship for specific bonds or bond portfolios.

\end{itemize}

\section{REFERENCES}
Smith, Donald J. 2014. Bond Math: The Theory behind the Formulas. 2nd ed.Hoboken, NJ: John Wiley \& Sons.

\section{PRACTICE PROBLEMS}
\begin{enumerate}
  \item A "buy-and-hold" investor purchases a fixed-rate bond at a discount and holds the security until it matures. Which of the following sources of return is least likely to contribute to the investor's total return over the investment horizon, assuming all payments are made as scheduled?
\end{enumerate}

A. Capital gain

B. Principal payment

C. Reinvestment of coupon payments

\begin{enumerate}
  \setcounter{enumi}{1}
  \item Which of the following sources of return is most likely exposed to interest rate risk for an investor of a fixed-rate bond who holds the bond until maturity?
\end{enumerate}

A. Capital gain or loss

B. Redemption of principal

C. Reinvestment of coupon payments

\begin{enumerate}
  \setcounter{enumi}{2}
  \item An investor purchases a bond at a price above par value. Two years later, the investor sells the bond. The resulting capital gain or loss is measured by comparing the price at which the bond is sold to the:
\end{enumerate}

A. carrying value.

B. original purchase price.

C. original purchase price value plus the amortized amount of the premium.

\section{The following information relates to questions}
 4-6An investor purchases a nine-year, $7 \%$ annual coupon payment bond at a price equal to par value. After the bond is purchased and before the first coupon is received, interest rates increase to $8 \%$. The investor sells the bond after five years. Assume that interest rates remain unchanged at $8 \%$ over the five-year holding period.

\begin{enumerate}
  \setcounter{enumi}{3}
  \item Per 100 of par value, the future value of the reinvested coupon payments at the end of the holding period is closest to:
A. 35.00 .
B. 40.26 .
C. 41.07 .

  \item The capital gain/loss per 100 of par value resulting from the sale of the bond at the end of the five-year holding period is closest to a:

\end{enumerate}

A. loss of 8.45 .
B. loss of 3.31 .
C. gain of 2.75 .

\begin{enumerate}
  \setcounter{enumi}{5}
  \item Assuming that all coupons are reinvested over the holding period, the investor's five-year horizon yield is closest to:
A. $5.66 \%$.
B. $6.62 \%$.
C. $7.12 \%$.

  \item An investor buys a $6 \%$ annual payment bond with three years to maturity. The bond has a yield-to-maturity of $8 \%$ and is currently priced at 94.845806 per 100 of par. The bond's Macaulay duration is closest to:
A. 2.62 .
B. 2.78 .
C. 2.83

  \item An investor buys a three-year bond with a $5 \%$ coupon rate paid annually. The bond, with a yield-to-maturity of $3 \%$, is purchased at a price of 105.657223 per 100 of par value. Assuming a 5-basis point change in yield-to-maturity, the bond's approximate modified duration is closest to:
A. 2.78 .
B. 2.86 .
C. 5.56 .

  \item Which of the following statements about duration is correct? A bond's:
A. effective duration is a measure of yield duration.
B. modified duration is a measure of curve duration.
C. modified duration cannot be larger than its Macaulay duration (assuming a positive yield-to-maturity).

  \item The interest rate risk of a fixed-rate bond with an embedded call option is best measured by:
A. effective duration.
B. modified duration.
C. Macaulay duration.

  \item Which of the following is most appropriate for measuring a bond's sensitivity to shaping risk?
A. Key rate duration
B. Effective duration
C. Modified duration 12. A Canadian pension fund manager seeks to measure the sensitivity of her pension liabilities to market interest rate changes. The manager determines the present value of the liabilities under three interest rate scenarios: a base rate of $7 \%$, a 100 basis point increase in rates up to $8 \%$, and a 100 basis point drop in rates down to $6 \%$. The results of the manager's analysis are presented below:

\end{enumerate}

\begin{center}
\begin{tabular}{cc}
\hline
Interest Rate Assumption & Present Value of Liabilities \\
\hline
$6 \%$ & CAD510.1 million \\
$7 \%$ & CAD455.4 million \\
$8 \%$ & CAD373.6 million \\
\hline
\end{tabular}
\end{center}

The effective duration of the pension fund's liabilities is closest to:
A. 1.49 .
B. 14.99 .
C. 29.97 .

\begin{enumerate}
  \setcounter{enumi}{12}
  \item Which of the following statements about Macaulay duration is correct?
A. A bond's coupon rate and Macaulay duration are positively related.
B. A bond's Macaulay duration is inversely related to its yield-to-maturity.
C. The Macaulay duration of a zero-coupon bond is less than its time-to-maturity.

  \item Assuming no change in the credit risk of a bond, the presence of an embedded put option:
A. reduces the effective duration of the bond.
B. increases the effective duration of the bond.
C. does not change the effective duration of the bond.

  \item A bond portfolio consists of the following three fixed-rate bonds. Assume annual coupon payments and no accrued interest on the bonds. Prices are per 100 of par value.

\end{enumerate}

\begin{center}
\begin{tabular}{lcccccc}
\hline
Bond & Maturity & $\begin{array}{c}\text { Market } \\ \text { Value }\end{array}$ & Price & Coupon & Yield-to-Maturity & $\begin{array}{c}\text { Modified } \\ \text { Duration }\end{array}$ \\
\hline
A & 6 years & 170,000 & 85.0000 & $2.00 \%$ & $4.95 \%$ & 5.42 \\
B & 10 years & 120,000 & 80.0000 & $2.40 \%$ & $4.99 \%$ & 8.44 \\
C & 15 years & 100,000 & 100.0000 & $5.00 \%$ & $5.00 \%$ & 10.38 \\
\hline
\end{tabular}
\end{center}

The bond portfolio's modified duration is closest to:
A. 7.62 .
B. 8.08 .
C. 8.20 ,

\begin{enumerate}
  \setcounter{enumi}{15}
  \item A limitation of calculating a bond portfolio's duration as the weighted average of the yield durations of the individual bonds that compose the portfolio is that it:
A. assumes a parallel shift to the yield curve.
B. is less accurate when the yield curve is less steeply sloped.
C. is not applicable to portfolios that have bonds with embedded options.

  \item Using the information below, which bond has the greatest money duration per 100 of par value assuming annual coupon payments and no accrued interest?

\end{enumerate}

\begin{center}
\begin{tabular}{lccccc}
\hline
Bond & Time-to-Maturity & $\begin{array}{c}\text { Price Per 100 } \\ \text { of Par Value }\end{array}$ & $\begin{array}{c}\text { Coupon } \\ \text { Rate }\end{array}$ & Yield-to-Maturity & $\begin{array}{c}\text { Modified } \\ \text { Duration }\end{array}$ \\
\hline
A & 6 years & 85.00 & $2.00 \%$ & $4.95 \%$ & 5.42 \\
B & 10 years & 80.00 & $2.40 \%$ & $4.99 \%$ & 8.44 \\
C & 9 years & 85.78 & $3.00 \%$ & $5.00 \%$ & 7.54 \\
\hline
\end{tabular}
\end{center}

A. Bond A
B. Bond B
C. Bond C

\begin{enumerate}
  \setcounter{enumi}{17}
  \item A bond with exactly nine years remaining until maturity offers a $3 \%$ coupon rate with annual coupons. The bond, with a yield-to-maturity of $5 \%$, is priced at 85.784357 per 100 of par value. The estimated price value of a basis point for the bond is closest to:
A. 0.0086 .
B. 0.0648 .
C. 0.1295 .

  \item The "second-order" effect on a bond's percentage price change given a change in yield-to-maturity can be best described as:
A. duration.
B. convexity.
C. yield volatility.

  \item A bond is currently trading for 98.722 per 100 of par value. If the bond's yield-to-maturity (YTM) rises by 10 basis points, the bond's full price is expected to fall to 98.669. If the bond's YTM decreases by 10 basis points, the bond's full price is expected to increase to 98.782 . The bond's approximate convexity is closest to:
A. 0.071 .
B. 70.906
C. $1,144.628$.

  \item A bond has an annual modified duration of 7.020 and annual convexity of 65.180. If the bond's yield-to-maturity decreases by 25 basis points, the expected per- centage price change is closest to:
A. $1.73 \%$.
B. $1.76 \%$.
C. $1.78 \%$

  \item A bond has an annual modified duration of 7.140 and annual convexity of 66.200. The bond's yield-to-maturity is expected to increase by 50 basis points. The expected percentage price change is closest to:
A. $-3.40 \%$.
B. $-3.49 \%$.
C. $-3.57 \%$.

  \item Which of the following statements relating to yield volatility is most accurate? If the term structure of yield volatility is downward sloping, then:
A. short-term rates are higher than long-term rates.
B. long-term yields are more stable than short-term yields.
C. short-term bonds will always experience greater price fluctuation than long-term bonds.

  \item The holding period for a bond at which the coupon reinvestment risk offsets the market price risk is best approximated by:
A. duration gap.
B. modified duration.
C. Macaulay duration.

  \item When the investor's investment horizon is less than the Macaulay duration of the bond she owns:
A. the investor is hedged against interest rate risk.
B. reinvestment risk dominates, and the investor is at risk of lower rates.
C. market price risk dominates, and the investor is at risk of higher rates.

  \item An investor purchases an annual coupon bond with a $6 \%$ coupon rate and exactly 20 years remaining until maturity at a price equal to par value. The investor's investment horizon is eight years. The approximate modified duration of the bond is 11.470 years. The duration gap at the time of purchase is closest to:
A. -7.842 .
B. 3.470 .
C. 4.158 .

  \item A manufacturing company receives a ratings upgrade and the price increases on its fixed-rate bond. The price increase was most likely caused by a(n):

\end{enumerate}

A. decrease in the bond's credit spread. B. increase in the bond's liquidity spread.

C. increase of the bond's underlying benchmark rate.

\begin{enumerate}
  \setcounter{enumi}{27}
  \item Empirical duration is likely the best measure of the impact of yield changes on portfolio value, especially under stressed market conditions, for a portfolio consisting of:
\end{enumerate}

A. $100 \%$ sovereign bonds of several AAA rated euro area issuers.

B. $100 \%$ covered bonds of several AAA rated euro area corporate issuers.

C. $25 \%$ AAA rated sovereign bonds, $25 \%$ AAA rated corporate bonds, and $50 \%$ high-yield (i.e., speculative-grade) corporate bonds, all from various euro area sovereign and corporate issuers.

\section{SOLUTIONS}
\begin{enumerate}
  \item A is correct. A capital gain is least likely to contribute to the investor's total return. There is no capital gain (or loss) because the bond is held to maturity. The carrying value of the bond at maturity is par value, the same as the redemption amount. When a fixed-rate bond is held to its maturity, the investor receives the principal payment at maturity. This principal payment is a source of return for the investor. A fixed-rate bond pays periodic coupon payments, and the reinvestment of these coupon payments is a source of return for the investor. The investor's total return is the redemption of principal at maturity and the sum of the reinvested coupons.

  \item C is correct. Because the fixed-rate bond is held to maturity (a "buy-and-hold" investor), interest rate risk arises entirely from changes in coupon reinvestment rates. Higher interest rates increase income from reinvestment of coupon payments, and lower rates decrease income from coupon reinvestment. There will not be a capital gain or loss because the bond is held until maturity. The carrying value at the maturity date is par value, the same as the redemption amount. The redemption of principal does not expose the investor to interest rate risk. The risk to a bond's principal is credit risk.

  \item A is correct. Capital gains (losses) arise if a bond is sold at a price above (below) its constant-yield price trajectory. A point on the trajectory represents the carrying value of the bond at that time. That is, the capital gain/loss is measured from the bond's carrying value, the point on the constant-yield price trajectory, and not from the original purchase price. The carrying value is the original purchase price plus the amortized amount of the discount if the bond is purchased at a price below par value. If the bond is purchased at a price above par value, the carrying value is the original purchase price minus (not plus) the amortized amount of the premium. The amortized amount for each year is the change in the price between two points on the trajectory.

  \item $\mathrm{C}$ is correct. The future value of reinvested cash flows at $8 \%$ after five years is closest to 41.07 per 100 of par value.

\end{enumerate}

$$
\left[7 \times(1.08)^{4}\right]+\left[7 \times(1.08)^{3}\right]+\left[7 \times(1.08)^{2}\right]+\left[7 \times(1.08)^{1}\right]+7=41.0662
$$

The 6.07 difference between the sum of the coupon payments over the five-year holding period (35) and the future value of the reinvested coupons (41.07) represents the "interest-on-interest" gain from compounding.

\begin{enumerate}
  \setcounter{enumi}{4}
  \item B is correct. The capital loss is closest to 3.31 per 100 of par value. After five years, the bond has four years remaining until maturity and the sale price of the bond is 96.69 , calculated as
\end{enumerate}

$\frac{7}{(1.08)^{1}}+\frac{7}{(1.08)^{2}}+\frac{7}{(1.08)^{3}}+\frac{107}{(1.08)^{4}}=96.69$

The investor purchased the bond at a price equal to par value (100). Because the bond was purchased at a price equal to its par value, the carrying value is par value. Therefore, the investor experienced a capital loss of $96.69-100=-3.31$.

\begin{enumerate}
  \setcounter{enumi}{5}
  \item B is correct. The investor's five-year horizon yield is closest to $6.62 \%$. After five years, the sale price of the bond is 96.69 (from problem 5) and the future value of reinvested cash flows at $8 \%$ is 41.0662 (from problem 4) per 100 of par value. The total return is $137.76(=41.07+96.69)$, resulting in a realized five-year horizon yield of $6.62 \%$ : $100.00=\frac{137.76}{(1+r)^{5}}, \quad r=0.0662$

  \item $\mathrm{C}$ is correct. The bond's Macaulay duration is closest to 2.83. Macaulay duration (MacDur) is a weighted average of the times to the receipt of cash flow. The weights are the shares of the full price corresponding to each coupon and principal payment.

\end{enumerate}

\begin{center}
\begin{tabular}{lcccc}
\hline
Period & Cash Flow & Present Value & Weight & Period $\times$ Weight \\
\hline
1 & 6 & 5.555556 & 0.058575 & 0.058575 \\
2 & 6 & 5.144033 & 0.054236 & 0.108472 \\
3 & 106 & 84.146218 & 0.887190 & 2.661570 \\
\cline { 2 - 4 }
 & 94.845806 & 1.000000 & 2.828617 &  \\
\hline
\end{tabular}
\end{center}

Thus, the bond's Macaulay duration (MacDur) is 2.83 .

Alternatively, Macaulay duration can be calculated using the following closed-form formula:

MacDur $=\left\{\frac{1+r}{r}-\frac{1+r+[N \times(c-r)]}{c \times\left[(1+r)^{N}-1\right]+r}\right\}-(t / T)$

MacDur $=\left\{\frac{1.08}{0.08}-\frac{1.08+[3 \times(0.06-0.08)]}{0.06 \times\left[(1.08)^{3}-1\right]+0.08}\right\}-0$

MacDur $=13.50-10.67=2.83$

\begin{enumerate}
  \setcounter{enumi}{7}
  \item A is correct. The bond's approximate modified duration is closest to 2.78. Approximate modified duration is calculated as
\end{enumerate}

ApproxModDur $=\frac{\left(P V_{-}\right)-\left(P V_{+}\right)}{2 \times(4 \text { Yield }) \times\left(P V_{0}\right)}$

Lower yield-to-maturity by 5 bps to $2.95 \%$ :

$P V_{-}=\frac{5}{(1+0.0295)^{1}}+\frac{5}{(1+0.0295)^{2}}+\frac{5+100}{(1+0.0295)^{3}}=105.804232$

Increase yield-to-maturity by 5 bps to $3.05 \%$ :

$P V_{+}=\frac{5}{(1+0.0305)^{1}}+\frac{5}{(1+0.0305)^{2}}+\frac{5+100}{(1+0.0305)^{3}}=105.510494$

$P V_{0}=105.657223, \Delta$ Yield $=0.0005$

ApproxModDur $=\frac{105.804232-105.510494}{2 \times 0.0005 \times 105.657223}=2.78$

\begin{enumerate}
  \setcounter{enumi}{8}
  \item C is correct. A bond's modified duration cannot be larger than its Macaulay duration assuming a positive yield-to-maturity. The formula for modified duration is
\end{enumerate}

ModDur $=\frac{\text { MacDur }}{1+r}$

where $r$ is the bond's yield-to-maturity per period. Therefore, ModDur will typically be less than MacDur.

Effective duration is a measure of curve duration. Modified duration is a measure of yield duration.

\begin{enumerate}
  \setcounter{enumi}{9}
  \item A is correct. The interest rate risk of a fixed-rate bond with an embedded call option is best measured by effective duration. A callable bond's future cash flows are uncertain because they are contingent on future interest rates. The issuer's decision to call the bond depends on future interest rates. Therefore, the yield-to-maturity on a callable bond is not well defined. Only effective duration, which takes into consideration the value of the call option, is the appropriate interest rate risk measure. Yield durations like Macaulay and modified durations are not relevant for a callable bond because they assume no changes in cash flows when interest rates change.

  \item A is correct. Key rate duration is used to measure a bond's sensitivity to a shift at one or more maturity segments of the yield curve which result in a change to yield curve shape. Modified and effective duration measure a bond's sensitivity to parallel shifts in the entire curve.

  \item B is correct. The effective duration of the pension fund's liabilities is closest to 14.99. The effective duration is calculated as follows:

\end{enumerate}

EffDur $=\frac{\left(P V_{-}\right)-\left(P V_{+}\right)}{2 \times(\Delta \text { Curve }) \times\left(P V_{0}\right)}$

$P V_{0}=455.4, P V_{+}=373.6, P V_{-}=510.1$, and $\Delta \mathrm{Curve}=0.0100$

EffDur $=\frac{510.1-373.6}{2 \times 0.0100 \times 455.4}=14.99$

\begin{enumerate}
  \setcounter{enumi}{12}
  \item B is correct. A bond's yield-to-maturity is inversely related to its Macaulay duration: The higher the yield-to-maturity, the lower its Macaulay duration and the lower the interest rate risk. A higher yield-to-maturity decreases the weighted average of the times to the receipt of cash flow, and thus decreases the Macaulay duration.
\end{enumerate}

A bond's coupon rate is inversely related to its Macaulay duration: The lower the coupon, the greater the weight of the payment of principal at maturity. This results in a higher Macaulay duration. Zero-coupon bonds do not pay periodic coupon payments; therefore, the Macaulay duration of a zero-coupon bond is its time-to-maturity.

\begin{enumerate}
  \setcounter{enumi}{13}
  \item A is correct. The presence of an embedded put option reduces the effective duration of the bond, especially when rates are rising. If interest rates are low compared with the coupon rate, the value of the put option is low and the impact of the change in the benchmark yield on the bond's price is very similar to the impact on the price of a non-putable bond. But when benchmark interest rates rise, the put option becomes more valuable to the investor. The ability to sell the bond at par value limits the price depreciation as rates rise. The presence of an embedded put option reduces the sensitivity of the bond price to changes in the benchmark yield, assuming no change in credit risk.

  \item A is correct. The portfolio's modified duration is closest to 7.62. Portfolio duration is commonly estimated as the market-value-weighted average of the yield durations of the individual bonds that compose the portfolio.

\end{enumerate}

The total market value of the bond portfolio is $170,000+120,000+100,000=$ $390,000$.

The portfolio duration is $5.42 \times(170,000 / 390,000)+8.44 \times(120,000 / 390,000)+$ $10.38 \times(100,000 / 390,000)=7.62$.

\begin{enumerate}
  \setcounter{enumi}{15}
  \item A is correct. A limitation of calculating a bond portfolio's duration as the weighted average of the yield durations of the individual bonds is that this measure implicitly assumes a parallel shift to the yield curve (all rates change by the same amount in the same direction). In reality, interest rate changes frequently result in a steeper or flatter yield curve. This approximation of the "theoretically correct" portfolio duration is more accurate when the yield curve is flatter (less steeply sloped). An advantage of this approach is that it can be used with portfolios that include bonds with embedded options. Bonds with embedded options can be included in the weighted average using the effective durations for these securities.

  \item $B$ is correct. Bond $B$ has the greatest money duration per 100 of par value. Money duration (MoneyDur) is calculated as the annual modified duration (AnnModDur) times the full price $\left(P V^{F u l l}\right)$ of the bond including accrued interest. Bond $\mathrm{B}$ has the highest money duration per 100 of par value.

\end{enumerate}

MoneyDur $=$ AnnModDur $\times P V^{\text {Full }}$

MoneyDur of Bond $\mathrm{A}=5.42 \times 85.00=460.70$

MoneyDur of Bond $\mathrm{B}=8.44 \times 80.00=675.20$

MoneyDur of Bond $\mathrm{C}=7.54 \times 85.78=646.78$

\begin{enumerate}
  \setcounter{enumi}{17}
  \item B is correct. The PVBP is closest to 0.0648 . The formula for the price value of a basis point is
\end{enumerate}

$\operatorname{PVBP}=\frac{\left(P V_{-}\right)-\left(P V_{+}\right)}{2}$

where

$\mathrm{PVBP}=$ price value of a basis point

$P V_{-}=$full price calculated by lowering the yield-to-maturity by $1 \mathrm{bp}$

$P V_{+}=$full price calculated by raising the yield-to-maturity by $1 \mathrm{bp}$

Lowering the yield-to-maturity by $1 \mathrm{bp}$ to $4.99 \%$ results in a bond price of 85.849134:

$P V_{-}=\frac{3}{(1+0.0499)^{1}}+\cdots+\frac{3+100}{(1+0.0499)^{9}}=85.849134$

Increasing the yield-to-maturity by $1 \mathrm{bp}$ to $5.01 \%$ results in a bond price of 85.719638:

$P V_{+}=\frac{3}{(1+0.0501)^{1}}+\cdots+\frac{3+100}{(1+0.0501)^{9}}=85.719638$

PVBP $=\frac{85.849134-85.719638}{2}=0.06475$

Alternatively, the PVBP can be derived using modified duration:

ApproxModDur $=\frac{\left(P V_{-}\right)-\left(P V_{+}\right)}{2 \times(\text { Yield }) \times\left(P V_{0}\right)}$

ApproxModDur $=\frac{85.849134-85.719638}{2 \times 0.0001 \times 85.784357}=7.548$

$\operatorname{PVBP}=7.548 \times 85.784357 \times 0.0001=0.06475$

\begin{enumerate}
  \setcounter{enumi}{18}
  \item B is correct. Convexity measures the "second order" effect on a bond's percentage price change given a change in yield-to-maturity. Convexity adjusts the percentage price change estimate provided by modified duration to better approximate the true relationship between a bond's price and its yield-to-maturity which is a curved line (convex).
\end{enumerate}

Duration estimates the change in the bond's price along the straight line that is tangent to this curved line ("first order" effect). Yield volatility measures the magnitude of changes in the yields along the yield curve.

\begin{enumerate}
  \setcounter{enumi}{19}
  \item B is correct. The bond's approximate convexity is closest to 70.906. Approximate convexity (ApproxCon) is calculated using the following formula:
\end{enumerate}

ApproxCon $=\left[P V_{-}+P V_{+}-\left(2 \times P V_{0}\right)\right] /\left(\Delta\right.$ Yield $\left.^{2} \times P V_{0}\right)$

where

$P V=$ new price when the yield-to-maturity is decreased

$P V_{+}=$new price when the yield-to-maturity is increased

$P V_{0}=$ original price

$\Delta$ Yield $=$ change in yield-to-maturity

ApproxCon $=[98.782+98.669-(2 \times 98.722)] /\left(0.001^{2} \times 98.722\right)=70.906$

\begin{enumerate}
  \setcounter{enumi}{20}
  \item $\mathrm{C}$ is correct. The expected percentage price change is closest to $1.78 \%$. The convexity-adjusted percentage price change for a bond given a change in the yield-to-maturity is estimated by
\end{enumerate}

$\% \Delta P V^{\text {Full }} \approx[-$ AnnModDur $\times \Delta$ Yield $]+\left[0.5 \times\right.$ AnnConvexity $\left.\times(\Delta \text { Yield })^{2}\right]$

$\% \Delta P V^{\text {Full }} \approx[-7.020 \times(-0.0025)]+\left[0.5 \times 65.180 \times(-0.0025)^{2}\right]=0.017754$, or $1.78 \%$

\begin{enumerate}
  \setcounter{enumi}{21}
  \item B is correct. The expected percentage price change is closest to $-3.49 \%$. The convexity-adjusted percentage price change for a bond given a change in the yield-to-maturity is estimated by
\end{enumerate}

$$
\begin{aligned}
& \% \Delta P V^{\text {Full }} \approx[- \text { AnnModDur } \times \Delta \text { Yield }]+\left[0.5 \times \text { AnnConvexity } \times(\Delta \text { Yield })^{2}\right] \\
& \% \Delta P V^{\text {Full }} \approx[-7.140 \times 0.005]+\left[0.5 \times 66.200 \times(0.005)^{2}\right]=-0.034873, \text { or }-3.49 \%
\end{aligned}
$$

\begin{enumerate}
  \setcounter{enumi}{22}
  \item B is correct. If the term structure of yield volatility is downward-sloping, then short-term bond yields-to-maturity have greater volatility than for long-term bonds. Therefore, long-term yields are more stable than short-term yields. Higher volatility in short-term rates does not necessarily mean that the level of short-term rates is higher than long-term rates. With a downward-sloping term structure of yield volatility, short-term bonds will not always experience greater price fluctuation than long-term bonds. The estimated percentage change in a bond price depends on the modified duration and convexity as well as on the yield-to-maturity change.

  \item $\mathrm{C}$ is correct. When the holder of a bond experiences a one-time parallel shift in the yield curve, the Macaulay duration statistic identifies the number of years necessary to hold the bond so that the losses (or gains) from coupon reinvestment offset the gains (or losses) from market price changes. The duration gap is the difference between the Macaulay duration and the investment horizon. Modified duration approximates the percentage price change of a bond given a change in its yield-to-maturity.

  \item $\mathrm{C}$ is correct. The duration gap is equal to the bond's Macaulay duration minus the investment horizon. In this case, the duration gap is positive, and price risk dominates coupon reinvestment risk. The investor risk is to higher rates.

\end{enumerate}

The investor is hedged against interest rate risk if the duration gap is zero; that is, the investor's investment horizon is equal to the bond's Macaulay duration. The investor is at risk of lower rates only if the duration gap is negative; that is, the investor's investment horizon is greater than the bond's Macaulay duration. In this case, coupon reinvestment risk dominates market price risk.

\begin{enumerate}
  \setcounter{enumi}{25}
  \item $C$ is correct. The duration gap is closest to 4.158. The duration gap is a bond's Macaulay duration minus the investment horizon. The approximate Macaulay duration is the approximate modified duration times one plus the yield-to-maturity. It is $12.158(=11.470 \times 1.06)$.
\end{enumerate}

Given an investment horizon of eight years, the duration gap for this bond at purchase is positive: $12.158-8=4.158$. When the investment horizon is less than the Macaulay duration of the bond, the duration gap is positive, and price risk dominates coupon reinvestment risk.

\begin{enumerate}
  \setcounter{enumi}{26}
  \item A is correct. The price increase was most likely caused by a decrease in the bond's credit spread. The ratings upgrade most likely reflects a lower expected probability of default and/or a greater level of recovery of assets if default occurs. The decrease in credit risk results in a smaller credit spread. The increase in the bond price reflects a decrease in the yield-to-maturity due to a smaller credit spread. The change in the bond price was not due to a change in liquidity risk or an increase in the benchmark rate.

  \item C is correct. Empirical duration is the best measure-better than analytical duration-of the impact of yield changes on portfolio value, especially under stressed market conditions, for a portfolio consisting of a variety of different bonds from different issuers, such as the portfolio described in Answer C. In this portfolio, credit spread changes on the high-yield bonds may partly or fully offset yield changes on the AAA rated sovereign bonds and spread changes on the AAA rated corporate bonds; this interaction is best captured using empirical duration. The portfolios described in Answers A and B consist of the same types of bonds from similar issuers-sovereign bonds from similar-rated sovereign issuers $(\mathrm{A})$ and covered bonds from similar-rated corporate issuers (B)-so empirical and analytical durations should be roughly similar in each of these portfolios.

\end{enumerate}

\section*{LEARNING MODULE 2 }
\section{Fundamentals of Credit Analysis}
Christopher L. Gootkind, CFA, is at Loomis, Sayles \& Company, LP (USA).

\section{LEARNING OUTCOME}
\begin{center}
\begin{tabular}{c|l}
Mastery & The candidate should be able to: \\
\hline
$\square$ & $\begin{array}{l}\text { describe credit risk and credit-related risks affecting corporate bonds } \\ \text { describe default probability and loss severity as components of credit } \\ \text { risk } \\ \text { describe seniority rankings of corporate debt and explain the } \\ \text { potential violation of the priority of claims in a bankruptcy } \\ \text { proceeding } \\ \text { compare and contrast corporate issuer credit ratings and issue credit } \\ \text { ratings and describe the rating agency practice of "notching" } \\ \text { explain risks in relying on ratings from credit rating agencies } \\ \text { explain the four Cs (Capacity, Collateral, Covenants, and Character) } \\ \text { of traditional credit analysis } \\ \text { calculate and interpret financial ratios used in credit analysis } \\ \text { evaluate the credit quality of a corporate bond issuer and a bond of } \\ \text { that issuer, given key financial ratios of the issuer and the industry } \\ \text { describe macroeconomic, market, and issuer-specific factors that } \\ \text { influence the level and volatility of yield spreads } \\ \text { explain special considerations when evaluating the credit of } \\ \text { high-yield, sovereign, and non-sovereign government debt issuers } \\ \text { and issues }\end{array}$ \\
$\square$ &  \\
\end{tabular}
\end{center}

INTRODUCTION

describe credit risk and credit-related risks affecting corporate bonds describe default probability and loss severity as components of credit
risk

\section{Fundamentals of Credit Analysis}
With bonds outstanding worth many trillions of US dollars, the debt markets play a critical role in the global economy. Companies and governments raise capital in the debt market to fund current operations; buy equipment; build factories, roads, bridges, airports, and hospitals; acquire assets; and so on. By channeling savings into productive investments, the debt markets facilitate economic growth. Credit analysis has a crucial function in the debt capital markets-efficiently allocating capital by properly assessing credit risk, pricing it accordingly, and repricing it as risks change. How do fixed-income investors determine the riskiness of that debt, and how do they decide what they need to earn as compensation for that risk?

In the sections that follow, we cover basic principles of credit analysis, which may be broadly defined as the process by which credit risk is evaluated. Readers will be introduced to the definition of credit risk, the interpretation of credit ratings, the four Cs of traditional credit analysis, and key financial measures and ratios used in credit analysis. We explain, among other things, how to compare bond issuer creditworthiness within a given industry as well as across industries and how credit risk is priced in the bond market.

Our coverage focuses primarily on analysis of corporate debt; however, credit analysis of sovereign and nonsovereign, particularly municipal, government bonds will also be addressed. Structured finance, a segment of the debt markets that includes securities backed by such pools of assets as residential and commercial mortgages as well as other consumer loans, will not be covered here.

We first introduce the key components of credit risk—default probability and loss severity- along with such credit-related risks as spread risk, credit migration risk, and liquidity risk. We then discuss the relationship between credit risk and the capital structure of the firm before turning attention to the role of credit rating agencies. We also explore the process of analyzing the credit risk of corporations and examine the impact of credit spreads on risk and return. Finally, we look at special considerations applicable to the analysis of (i) high-yield (low-quality) corporate bonds and (ii) government bonds.

\section{CREDIT RISK}
Credit risk is the risk of loss resulting from the borrower (issuer of debt) failing to make full and timely payments of interest and/or principal. Credit risk has two components. The first is known as default risk, or default probability, which is the probability that a borrower defaults-that is, fails to meet its obligation to make full and timely payments of principal and interest according to the terms of the debt security. The second component is loss severity (also known as "loss given default") in the event of default-that is, the portion of a bond's value (including unpaid interest) an investor loses. A default can lead to losses of various magnitudes. In most instances, in the event of default, bondholders will recover some value, so there will not be a total loss on the investment. Thus, credit risk is reflected in the distribution of potential losses that may arise if the investor is not paid in full and on time. Although it is sometimes important to consider the entire distribution of potential losses and their respective probabilities-for instance, when losses have a disproportionate impact on the various tranches of a securitized pool of loans-it is often convenient to summarize the risk with a single default probability and loss severity and focus on the expected loss:

Expected loss $=$ Default probability $\times$ Loss severity given default.

The loss severity, and hence the expected loss, can be expressed as either a monetary amount (e.g., $€ 450,000$ ) or as a percentage of the principal amount (e.g., 45\%). The latter form of expression is generally more useful for analysis because it is independent of the amount of investment. Loss severity is often expressed as ( 1 - Recovery rate), where the recovery rate is the percentage of the principal amount recovered in the event of default.

Because default risk (default probability) is quite low for most high-quality debt issuers, bond investors tend to focus primarily on assessing this probability and devote less effort to assessing the potential loss severity arising from default. However, as an issuer's default risk rises, investors will focus more on what the recovery rate might be in the event of default. This issue will be discussed in more detail later. Important credit-related risks include the following:

\begin{itemize}
  \item Spread risk. Corporate bonds and other "credit-risky" debt instruments typically trade at a yield premium, or spread, to bonds that have been considered "default-risk free," such as US Treasury bonds or German government bonds. Yield spreads, expressed in basis points, widen based on factors specific to the issuer, such as a decline in creditworthiness, sometimes referred to as credit migration or downgrade risk, or factors associated with the market as a whole, such as an increase in market liquidity risk or a general aversion to risk during periods of financial distress.

  \item Credit migration risk or downgrade risk. This is the risk that a bond issuer's creditworthiness deteriorates, or migrates lower, leading investors to believe the risk of default is higher and thus causing the yield spreads on the issuer's bonds to widen and the price of its bonds to fall. The term "downgrade" refers to action by the major bond rating agencies, whose role will be covered in more detail later.

  \item Market liquidity risk. This is the risk that the price at which investors can actually transact-buying or selling-may differ from the price indicated in the market. To compensate investors for the risk that there may not be sufficient market liquidity for them to buy or sell bonds in the quantity they desire, the spread or yield premium on corporate bonds includes a market liquidity component, in addition to a credit risk component. Unlike stocks, which trade on exchanges, most markets bonds trade primarily over the counter through broker-dealers trading for their own accounts. Their ability and willingness to make markets, as reflected in the bid-ask spread, is an important determinant of market liquidity risk. The two main issuer-specific factors that affect market liquidity risk are (1) the size of the issuer (that is, the amount of publicly traded debt an issuer has outstanding) and (2) the credit quality of the issuer. In general, the less debt an issuer has outstanding, the less frequently its debt trades and thus the higher the market liquidity risk. And the lower the quality of the issuer, the higher the market liquidity risk.

\end{itemize}

During times of financial stress or crisis, such as in late 2008 , market liquidity can decline sharply, causing yield spreads on corporate bonds and other credit-risky debt to widen and their prices to drop.

\section{EXAMPLE 1}
\section{Defining Credit Risk}
\begin{enumerate}
  \item Which of the following best defines credit risk?
\end{enumerate}

A. The probability of default times the severity of loss given default

B. The loss of principal and interest payments in the event of bankruptcy C. The risk of not receiving full interest and principal payments on a timely basis

\section{Solution:}
$\mathrm{C}$ is correct. Credit risk is the risk that the borrower will not make full and timely payments.

\begin{enumerate}
  \setcounter{enumi}{1}
  \item Which of the following is the best measure of credit risk?
\end{enumerate}

A. The expected loss

B. The severity of loss

C. The probability of default

\section{Solution:}
A is correct. The expected loss captures both of the key components of credit risk: (the product of) the probability of default and the loss severity in the event of default. Neither component alone fully reflects the risk.

\begin{enumerate}
  \setcounter{enumi}{2}
  \item Which of the following is NOT credit or credit-related risk?
\end{enumerate}

A. Default risk

B. Interest rate risk

C. Downgrade or credit migration risk

\section{Solution:}
$\mathrm{B}$ is correct. Bond price changes due to general interest rate movements are not considered credit risk.

\section{CAPITAL STRUCTURE, SENIORITY RANKING, AND RECOVERY RATES}
describe seniority rankings of corporate debt and explain the potential violation of the priority of claims in a bankruptcy proceeding

The various debt obligations of a given borrower will not necessarily all have the same seniority ranking, or priority of payment. In this section, we will introduce the topic of an issuer's capital structure and discuss the various types of debt claims that may arise from that structure, as well as their ranking and how those rankings can influence recovery rates in the event of default. The term "creditors" is used throughout our coverage to mean holders of debt instruments, such as bonds and bank loans. Unless specifically stated, it does not include such obligations as trade credit, tax liens, or employment-related obligations.

\section{Capital Structure}
The composition and distribution across operating units of a company's debt and equity-including bank debt, bonds of all seniority rankings, preferred stock, and common equity-is referred to as its capital structure. Some companies and industries have straightforward capital structures, with all the debt equally ranked and issued by one main operating entity. Other companies and industries, due to their frequent acquisitions and divestitures (e.g., media companies or conglomerates) or high levels of regulation (e.g., banks and utilities), tend to have more complicated capital structures. Companies in these industries often have many different subsidiaries, or operating companies, that have their own debt outstanding and parent holding companies that also issue debt, with different levels or rankings of seniority. Similarly, the cross-border operations of multinational corporations tend to increase the complexity of their capital structures.

\section{Seniority Ranking}
Just as borrowers can issue debt with many different maturity dates and coupons, they can also have many different rankings in terms of seniority. The ranking refers to the priority of payment, with the most senior or highest-ranking debt having the first claim on the cash flows and assets of the issuer. This level of seniority can affect the value of an investor's claim in the event of default and restructuring. Broadly, there is secured debt and unsecured debt. Secured debt means the debtholder has a direct claim-a pledge from the issuer-on certain assets and their associated cash flows. Unsecured bondholders have only a general claim on an issuer's assets and cash flows. In the event of default, unsecured debtholders' claims rank below (i.e., get paid after) those of secured creditors under what's known as the priority of claims.

Exhibit 1: Seniority Ranking

\begin{center}
\includegraphics[max width=\textwidth]{2023_05_04_36535b8d80b32081d422g-083}
\end{center}

Within each category of debt are finer gradations of types and rankings. Within secured debt, there is first mortgage and first lien debt, which are the highest-ranked debt in terms of priority of repayment. First mortgage debt or loan refers to the pledge of a specific property (e.g., a power plant for a utility or a specific casino for a gaming company). First lien debt or loan refers to a pledge of certain assets that could include buildings but might also include property and equipment, licenses, patents, brands,

\section{Fundamentals of Credit Analysis}
and so on. There can also be second lien, or even third lien, secured debt, which, as the name implies, has a secured interest in the pledged assets but ranks below first lien debt in both collateral protection and priority of payment.

Within unsecured debt, there can also be finer gradations and seniority rankings. The highest-ranked unsecured debt is senior unsecured debt. It is the most common type of all corporate bonds outstanding. Other, lower-ranked debt includes subordinated debt and junior subordinated debt. Among the various creditor classes, these obligations have among the lowest priority of claims and frequently have little or no recovery in the event of default. That is, their loss severity can be as high as $100 \%$. (See Exhibit 1 for a sample seniority ranking.) For regulatory and capital purposes, banks in Europe and the United States have issued debt and debt-like securities that rank even lower than subordinated debt, typically referred to as hybrids or trust preferred, and are intended to provide a capital cushion in times of financial distress. Many of them did not work as intended during the global financial crisis that began in 2008, and most were phased out, potentially to be replaced by more effective instruments that automatically convert to equity in certain circumstances.

Companies issue-and investors buy-debt with different seniority rankings for many reasons. Issuers are interested in optimizing their cost of capital-finding the right mix of the various types of both debt and equity-for their industry and type of business. Issuers may offer secured debt because that is what the market (i.e., investors) may require-given a company's perceived riskiness or because secured debt is generally lower cost due to the reduced credit risk inherent in its higher priority of claims. Or, issuers may offer subordinated debt because (1) they believe it is less expensive than issuing equity, as debtholders require a lower rate of return due to their superior place in line in the event of default; (2) doing so prevents dilution to existing shareholders; (3) it is typically less restrictive than issuing senior debt; and (4) investors are willing to buy it because they believe the yield being offered is adequate compensation for the risk they perceive.

\section{EXAMPLE 2}
\section{Seniority Ranking}
\begin{enumerate}
  \item The Acme Company has senior unsecured bonds as well as both first and second lien debt in its capital structure. Which ranks higher with respect to priority of claims: senior unsecured bonds or second lien debt?
\end{enumerate}

\section{Solution:}
Second lien debt ranks higher than senior unsecured bonds because of its secured position.

\section{Recovery Rates}
All creditors at the same level of the capital structure are treated as one class; thus, a senior unsecured bondholder whose debt is due in 30 years has the same pro rata claim in bankruptcy as one whose debt matures in six months. This provision is referred to as bonds ranking pari passu ("on an equal footing") in right of payment.

Defaulted debt will often continue to be traded by investors and broker-dealers based on their assessment that either in liquidation of the bankrupt company's assets or in reorganization, the bonds will have some recovery value. In the case of reorganization or restructuring (whether through formal bankruptcy or on a voluntary basis), new debt, equity, cash, or some combination thereof could be issued in exchange for the original defaulted debt.

As discussed, recovery rates vary by seniority of ranking in a company's capital structure, under the priority of claims treatment in bankruptcy. Over many decades, there have been enough defaults to generate statistically meaningful historical data on recovery rates by seniority ranking. Exhibit 2 provides recovery rates by seniority ranking for North American non-financial companies. For example, as shown in Exhibit 2, investors on average recovered $46.9 \%$ of the value of senior secured debt that defaulted in 2016 but only $29.2 \%$ of the value of senior unsecured issues that defaulted that year.

Exhibit 2: Average Corporate Debt Recovery Rates Measured by Ultimate

Recoveries

\begin{center}
\begin{tabular}{lccccccc}
\hline
 & \multicolumn{3}{c}{Emergence Year*} & \multicolumn{3}{c}{Default Year} \\
\hline
Seniority Ranking & $\mathbf{2 0 1 7}$ & $\mathbf{2 0 1 6}$ & $\mathbf{1 9 8 7 - 2 0 1 7}$ & $\mathbf{2 0 1 7}$ & $\mathbf{2 0 1 6}$ & $\mathbf{1 9 8 7 - 2 0 1 7}$ \\
\hline
Bank loans & $81.3 \%$ & $72.6 \%$ & $80.4 \%$ & $80.2 \%$ & $78.3 \%$ & $80.4 \%$ \\
Senior secured bonds & $52.3 \%$ & $35.9 \%$ & $62.3 \%$ & $57.5 \%$ & $46.9 \%$ & $62.3 \%$ \\
Senior unsecured & $54.1 \%$ & $11.7 \%$ & $47.9 \%$ & $47.4 \%$ & $29.2 \%$ & $47.9 \%$ \\
bonds &  &  &  &  &  &  \\
Subordinated bonds & $4.5 \%$ & $6.6 \%$ & $28.0 \%$ & NA & $8.0 \%$ & $28.0 \%$ \\
\hline
\end{tabular}
\end{center}

$N A=$ not available

"Emergence year is typically the year the defaulted company emerges from bankruptcy. Default year data refer to the recovery rate of debt that defaulted in that year (i.e., 2016 and 2017) or range of years (i.e., 1987-2017). Data are for North American nonfinancial companies.

Source: Moody's Investors Service's Ultimate Recovery Database.

A few things are worth noting:

\begin{enumerate}
  \item Recovery rates can vary widely by industry. Companies that go bankrupt in industries that are in secular decline (e.g., newspaper publishing) will most likely have lower recovery rates than those that go bankrupt in industries merely suffering from a cyclical economic downturn.

  \item Recovery rates can also vary depending on when they occur in a credit cycle. Credit cycles describe the changing availability-and pricing-of credit. When the economy is strong or improving, the willingness of lenders to extend credit on favorable terms is high. Conversely, when the economy is weak or weakening, lenders pull back, or "tighten" credit, by making it less available and more expensive. As shown in Exhibit 3, at or near the bottom of a credit cycle-which is almost always closely linked with an economic cycle-recoveries will tend to be lower than at other times in the credit cycle.

\end{enumerate}

\section{Exhibit 3: Global Recovery Rates by Seniority Ranking, 1990-2017}
\begin{center}
\includegraphics[max width=\textwidth]{2023_05_04_36535b8d80b32081d422g-086}
\end{center}

Source: Based on data from Moody's Investors Service's Ultimate Recovery Database.

\begin{enumerate}
  \setcounter{enumi}{2}
  \item These recovery rates are averages. In fact, there can be large variability, both across industries, as noted, as well as across companies within a given industry. Factors might include composition and proportion of debt across an issuer's capital structure. An abundance of secured debt will lead to smaller recovery rates on lower-ranked debt.
\end{enumerate}

Understanding recovery rates is important because they are a key component of credit analysis and risk. Recall that the best measure of credit risk is expected lossthat is, probability of default times loss severity given default. And loss severity equals (1 - Recovery rate). Having an idea how much one can lose in the event of default is a critical factor in valuing credit, particularly lower-quality credit, as the default risk rises.

Priority of claims: Not always absolute. The priority of claims in bankruptcy-the idea that the highest-ranked creditors get paid out first, followed by the next level, and on down, like a waterfall-is well established and is often described as "absolute." In principle, in the event of bankruptcy or liquidation:

\begin{itemize}
  \item Creditors with a secured claim have the right to the value of that specific property before any other claim. If the value of the pledged property is less than the amount of the claim, then the difference becomes a senior unsecured claim.

  \item Unsecured creditors have a right to be paid in full before holders of equity interests (common and preferred shareholders) receive value on their interests.

  \item Senior unsecured creditors take priority over all subordinated creditors. A creditor is senior unsecured unless expressly subordinated.

\end{itemize}

In practice, however, creditors with lower seniority and even shareholders may receive some consideration without more senior creditors being paid in full. Why might this be the case? In bankruptcy, there are different classes of claimants, and all classes that are impaired (that is, receive less than full claim) get to vote to confirm the plan of reorganization. This vote is subject to the absolute priority of claims. Either by consent of the various parties or by the judge's order, however, absolute priority may not be strictly enforced in the final plan. There may be disputes over the value of various assets in the bankruptcy estate (e.g., what is a plant, or a patent portfolio, worth?) or the present value or timing of payouts. For example, what is the value of the new debt I'm receiving for my old debt of a reorganized company before it emerges from bankruptcy?

Resolution of these disputes takes time, and cases can drag on for months and years. In the meantime, during bankruptcy, substantial expenses are being incurred for legal and accounting fees, and the value of the company may be declining as key employees leave, customers go elsewhere, and so on. Thus, to avoid the time, expense, and uncertainty over disputed issues, such as the value of property in the estate and the legality of certain claims, the various claimants have an incentive to negotiate and compromise. This frequently leads to creditors with lower seniority and other claimants (e.g., even shareholders) receiving more consideration than they are legally entitled to.

It's worth noting that in the United States, the bias is toward reorganization and recovery of companies in bankruptcy, whereas in other jurisdictions, such as the United Kingdom, the bias is toward liquidation of companies in bankruptcy and maximizing value to the banks and other senior creditors. It's also worth noting that bankruptcy and bankruptcy laws are very complex and can vary greatly by country, so it is difficult to generalize about how creditors will fare. As shown in the earlier chart, there is huge variability in recovery rates for defaulted debt. Every case is different.

\section{EXAMPLE 3}
\section{Priority of Claims}
\begin{enumerate}
  \item Under which circumstance is a subordinated bondholder most likely to recover some value in a bankruptcy without a senior creditor getting paid in full? When:
\end{enumerate}

A. absolute priority rules are enforced.

B. the various classes of claimants agree to it.

C. the company is liquidated rather than reorganized.

\section{Solution:}
B is correct. All impaired classes get to vote on the reorganization plan.

Negotiation and compromise are often preferable to incurring huge legal and accounting fees in a protracted bankruptcy process that would otherwise reduce the value of the estate for all claimants. This process may allow junior creditors (e.g., subordinated bondholders) to recover some value even though more senior creditors do not get paid in full.

\begin{enumerate}
  \setcounter{enumi}{1}
  \item In the event of bankruptcy, claims at the same level of the capital structure are:
\end{enumerate}

A. on an equal footing, regardless of size, maturity, or time outstanding.

B. paid in the order of maturity from shortest to longest, regardless of size or time outstanding.

C. paid on a first-in, first-out (FIFO) basis so that the longest-standing claims are satisfied first, regardless of size or maturity.

\section{Solution:}
A is correct. All claims at the same level of the capital structure are pari passu (on an equal footing).

\section{RATING AGENCIES, CREDIT RATINGS, AND THEIR ROLE IN THE DEBT MARKETS}
compare and contrast corporate issuer credit ratings and issue credit ratings and describe the rating agency practice of "notching" explain risks in relying on ratings from credit rating agencies

The major credit rating agencies-Moody's Investors Service ("Moody's"), Standard \& Poor's ("S\&P"), and Fitch Ratings ("Fitch")\_- play a central, if somewhat controversial, role in the credit markets. For the vast majority of outstanding bonds, at least two of the agencies provide ratings: a symbol-based measure of the potential risk of default of a particular bond or issuer of debt. In the public and quasi-public bond markets, underwritten by investment banks as opposed to privately placed, issuers won't offer, and investors won't buy, bonds that do not carry ratings from Moody's, S\&P, or Fitch. This practice applies for all types of bonds: government or sovereign; entities with implicit or explicit guarantees from the government, such as Ginnie Mae in the United States and Pfandbriefe in Germany; supranational entities, such as the World Bank, which are owned by several governments; corporate; non-sovereign government; and mortgage- and asset-backed debt. How did the rating agencies attain such a dominant position in the credit markets? What are credit ratings, and what do they mean? How does the market use credit ratings? What are the risks of relying solely or excessively on credit ratings?

The history of the major rating agencies goes back more than 100 years. John Moody began publishing credit analysis and opinions on US railroads in 1909. S\&P published its first ratings in 1916. They have grown in size and prominence since then. Many bond investors like the fact that there are independent analysts who meet with the issuer and often have access to material, non-public information, such as financial projections that investors cannot receive, to aid in the analysis. What has also proven very attractive to investors is that credit ratings provide direct and easy comparability of the relative credit riskiness of all bond issuers, within and across industries and bond types, although there is some debate about ratings comparability across the types of bonds. For instance, investigations conducted after the 2008-2009 global financial crisis suggested that, for a given rating category, municipal bonds have experienced a lower historical incidence of default than corporate debt.

Several factors have led to the near universal use of credit ratings in the bond markets and the dominant role of the major credit rating agencies. These factors include the following:

\begin{itemize}
  \item Independent assessment of credit risk

  \item Ease of comparison across bond issuers, issues, and market segments

  \item Regulatory and statutory reliance and usage

  \item Issuer payment for ratings

  \item Huge growth of debt markets

  \item Development and expansion of bond portfolio management and the accompanying bond indexes

\end{itemize}

However, in the aftermath of the global financial crisis of 2008-2009, when the rating agencies were blamed for contributing to the crisis with their overly optimistic ratings on securities backed by subprime mortgages, attempts were made to reduce the role and dominant positions of the major credit rating agencies. New rules, regulations, and legislation were passed to require the agencies to be more transparent, reduce conflicts of interest, and stimulate more competition. The "issuer pay" model allows the distribution of ratings to a broad universe of investors and undoubtedly facilitated widespread reliance on ratings. Challenging the dominance of Moody's, S\&P, and Fitch, additional credit rating agencies have emerged. Some credit rating agencies that are well-established in their home markets but are not so well known globally, such as Dominion Bond Rating Service (DBRS) in Canada and Japan Credit Rating Agency (JCR) in Japan, have tried to raise their profiles. The market dominance of the biggest credit rating agencies, however, remains largely intact.

\section{Credit Ratings}
The three major global credit rating agencies-Moody's, S\&P, and Fitch-use similar, symbol-based ratings that are basically an assessment of a bond issue's risk of default. Exhibit 4 shows their long-term ratings ranked from highest to lowest. Ratings on short-term debt, although available, are not shown here.

\section{Exhibit 4: Long-Term Rating Matrix: Investment Grade vs. Non-Investment}
 Grade\begin{center}
\begin{tabular}{|c|c|c|c|c|}
\hline
 &  & Moody's & S\&P & Fitch \\
\hline
\multirow{10}{*}{Investment Grade} & \multirow{4}{*}{$\begin{array}{l}\text { High-Quality } \\
\text { Grade }\end{array}$} & Aaa & AAA & AAA \\
\hline
 &  & Aa1 & $\mathrm{AA}+$ & $\mathrm{AA}+$ \\
\hline
 &  & $\mathrm{Aa} 2$ & AA & AA \\
\hline
 &  & $\mathrm{Aa} 3$ & $\mathrm{AA}-$ & AA- \\
\hline
 & \multirow{3}{*}{$\begin{array}{l}\text { Upper-Medium } \\
\text { Grade }\end{array}$} & A1 & $\mathrm{A}+$ & $\mathrm{A}+$ \\
\hline
 &  & A2 & A & A \\
\hline
 &  & A3 & A- & A- \\
\hline
 & \multirow{3}{*}{$\begin{array}{l}\text { Low-Medium } \\
\text { Grade }\end{array}$} & Baa1 & $\mathrm{BBB}+$ & $\mathrm{BBB}+$ \\
\hline
 &  & Baa2 & BBB & BBB \\
\hline
 &  & Baa3 & $\mathrm{BBB}-$ & $\mathrm{BBB}-$ \\
\hline
\multirow{12}{*}{$\begin{array}{l}\text { Non-Investment Grade } \\
\text { ("Junk" or "High Yield") }\end{array}$} & \multirow{11}{*}{$\begin{array}{c}\text { Low Grade or } \\
\text { Speculative } \\
\text { Grade }\end{array}$} & $\mathrm{Ba} 1$ & $\mathrm{BB}+$ & $\mathrm{BB}+$ \\
\hline
 &  & $\mathrm{Ba} 2$ & BB & BB \\
\hline
 &  & $\mathrm{Ba} 3$ & $\mathrm{BB}-$ & $\mathrm{BB}-$ \\
\hline
 &  & B1 & $\mathrm{B}+$ & $\mathrm{B}+$ \\
\hline
 &  & B2 & B & B \\
\hline
 &  & B3 & B- & B- \\
\hline
 &  & Caa1 & CCC+ & $\mathrm{CCC}+$ \\
\hline
 &  & Caa2 & CCC & CCC \\
\hline
 &  & Caa3 & CCC- & $\mathrm{CCC}-$ \\
\hline
 &  & $\mathrm{Ca}$ & CC & CC \\
\hline
 &  & C & $\mathrm{C}$ & $\mathrm{C}$ \\
\hline
 & Default & $\mathrm{C}$ & D & D \\
\hline
\end{tabular}
\end{center}

Bonds rated triple- $\mathrm{A}$ (Aaa or $\mathrm{AAA}$ ) are said to be "of the highest quality, subject to the lowest level of credit risk" (Moody's Investors Service, see "Rating Symbols and Definitions", \href{https://www.moodys.com/Pages/amr002002.aspx}{https://www.moodys.com/Pages/amr002002.aspx}) and thus have

\section{Fundamentals of Credit Analysis}
extremely low probabilities of default. Double-A (Aa or AA) rated bonds are referred to as "high-quality grade" and are also regarded as having very low default risk. Bonds rated single-A are referred to as "upper-medium grade." Baa (Moody's) or BBB (S\&P and Fitch) are called "low-medium grade." Bonds rated Baa3/BBB- or higher are called "investment grade." Bonds rated Ba1 or lower by Moody's and BB+ or lower by S\&P and Fitch, respectively, have speculative credit characteristics and increasingly higher default risk. As a group, these bonds are referred to in a variety of ways: "low grade," "speculative grade," "non-investment grade," "below investment grade," "high yield," and, in an attempt to reflect the extreme level of risk, some observers refer to these bonds as "junk bonds." The D rating is reserved for securities that are already in default in S\&P's and Fitch's scales. For Moody's, bonds rated C are likely, but not necessarily, in default. Generally, issuers of bonds rated investment grade are more consistently able to access the debt markets and can borrow at lower interest rates than those rated below investment grade.

In addition, rating agencies will typically provide outlooks on their respective ratings-positive, stable, or negative-and may provide other indicators on the potential direction of their ratings under certain circumstances, such as "On Review for Downgrade" or "On CreditWatch for an Upgrade." It should also be noted that, in support of the ratings they publish, the rating agencies also provide extensive written commentary and financial analysis on the obligors they rate, as well as summary industry statistics.

\section{Issuer vs. Issue Ratings}
Rating agencies will typically provide both issuer and issue ratings, particularly as they relate to corporate debt. Terminology used to distinguish between issuer and issue ratings includes corporate family rating (CFR) and corporate credit rating (CCR) or issuer credit rating and issue credit rating. An issuer credit rating is meant to address an obligor's overall creditworthiness-its ability and willingness to make timely payments of interest and principal on its debt. The issuer credit rating usually applies to its senior unsecured debt.

Issue ratings refer to specific financial obligations of an issuer and take into consideration such factors as ranking in the capital structure (e.g., secured or subordinated). Although cross-default provisions, whereby events of default (such as non-payment of interest) on one bond trigger default on all outstanding debt, imply the same default probability for all issues, specific issues may be assigned different credit ratings-higher or lower-due to a rating adjustment methodology known as notching.

\section{Notching}
For the rating agencies, likelihood of default-default risk-is the primary factor in assigning their ratings. However, there are secondary factors as well. These factors include the priority of payment in the event of a default (e.g., secured versus senior unsecured versus subordinated) as well as potential loss severity in the event of default. Another factor considered by rating agencies is structural subordination, which can arise when a corporation with a holding company structure has debt at both its parent holding company and operating subsidiaries. Debt at the operating subsidiaries will get serviced by the cash flow and assets of the subsidiaries before funds can be passed ("upstreamed") to the holding company to service debt at that level.

Recognizing these different payment priorities, and thus the potential for higher (or lower) loss severity in the event of default, the rating agencies have adopted a notching process whereby their credit ratings on issues can be moved up or down from the issuer rating, which is usually the rating applied to its senior unsecured debt. As a general rule, the higher the senior unsecured rating, the smaller the notching adjustment. The reason behind this is that the higher the rating, the lower the perceived risk of default; so, the need to "notch" the rating to capture the potential difference in loss severity is greatly reduced. For lower-rated credits, however, the risk of default is greater and thus the potential difference in loss from a lower (or higher) priority ranking is a bigger consideration in assessing an issue's credit riskiness. Thus, the rating agencies will typically apply larger rating adjustments. For example, S\&P applies the following notching guidelines:

A key principle is that investment-grade ratings focus more on timeliness, while non-investment-grade ratings give additional weight to recovery. For example, subordinated debt can be rated up to two notches below a non-investment-grade corporate credit rating, but one notch at most if the corporate credit rating is investment grade. Conversely, ... the 'AAA' rating category need not be notched at all, while at the 'CCC' level the gaps may widen.

The rationale for this convention is straightforward: as default risk increases, the concern over what can be recovered takes on greater relevance and, therefore, greater rating significance. Accordingly, the ultimate recovery aspect of ratings is given more weight as one moves down the rating spectrum.(Standard \& Poor's, "Rating the Issue," in Corporate Ratings Criteria 2008 [New York: Standard \& Poor's, 2008]: 65)

Exhibit 5 is an example of S\&P's notching criteria, as applied to Infor Software Parent LLC, Inc. (Infor). Infor is a US-based global software and services company whose corporate credit rating from $S \& P$ is $B-$. Note how the company's senior secured bonds are rated $B$, whereas its senior unsecured bonds are rated two notches lower at CCC + and its holding company debt is rated even one notch lower at CCC.

\section{Exhibit 5: Infor S\&P Ratings Detail (as of December 2018)}
\begin{center}
\begin{tabular}{ll}
\hline
Corporate credit rating & B-/Stable \\
Senior secured (3 issues) & $\mathrm{B}$ \\
Senior unsecured (2 issues) & CCC + \\
Holding company debt (1 issue) & $\mathrm{CCC}$ \\
\hline
\end{tabular}
\end{center}

Source: Standard \& Poor's Financial Services, LLC.

\section{ESG Ratings}
ESG investing has received a great deal of attention and acceptance as a viable investment strategy, even as a source of alpha. ESG investing involves asset selection and investment decision making based on disciplined evaluation of one or all of these considerations:

\begin{itemize}
  \item Environmental themes, such as investing in companies that are responding to consumer demand for sustainable practices, the reduction of carbon emissions, and other environmental agendas.

  \item Social themes, such as investing in companies committed to a diverse and inclusive workplace, minimizing salary gap, and community involvement programs.

  \item Governance themes, such as investing in companies committed to diverse board composition and strong oversight.

\end{itemize}

\section{Fundamentals of Credit Analysis}
MSCI Inc., a global provider of market indexes and analytic tools, has launched a set of ratings that aim to measure a company's attitudes, practices, and advances related to ESG. Their rules-based methodology aims to identify and track leaders and laggards in the space. Companies are evaluated according to their exposure to ESG risks and how well they manage those risks relative to peers. The MSCI ESG rating scale has seven tiers and spans AAA to CCC. From best to worst, leaders receive AAA or AA; the next echelon constitutes an average rating of A, BBB, or BB; while laggards receive B or CCC. MSCI has also added countries, mutual funds, and ETFs to their ESG rating system. Traditional rating agencies, such as Moody's and S\&P, are evaluating ESG for borrowers under these criteria too.

ESG investing allows investors to align their investment objectives with long-term trends considered important to the overall sustainability of a business. Some evidence suggests that funds that value 'ethics', by construction, also outperform key market indexes.

\section{Risks in Relying on Agency Ratings}
The dominant position of the rating agencies in the global debt markets, and the near-universal use of their credit ratings on debt securities, suggests that investors believe they do a good job assessing credit risk. In fact, with a few exceptions (e.g., too high ratings on US subprime mortgage-backed securities issued in the mid-2000s, which turned out to be much riskier than expected), their ratings have proved quite accurate as a relative measure of default risk. For example, Exhibit 6 shows historical $\mathrm{S} \& \mathrm{P}$ one-year global corporate default rates by rating category for the 20 -year period from 1998 to 2017. It measures the percentage of issues that defaulted in a given calendar year based on how they were rated at the beginning of the year.

Exhibit 6: Global Corporate Annual Default Rates by Rating Category (\%)

\begin{center}
\begin{tabular}{lccccccc}
\hline
 & AAA & AA & A & BBB & BB & B & CCC/C \\
\hline
$\mathbf{1 9 9 8}$ & 0.00 & 0.00 & 0.00 & 0.41 & 0.82 & 4.63 & 42.86 \\
$\mathbf{1 9 9 9}$ & 0.00 & 0.17 & 0.18 & 0.20 & 0.95 & 7.29 & 33.33 \\
$\mathbf{2 0 0 0}$ & 0.00 & 0.00 & 0.27 & 0.37 & 1.16 & 7.70 & 35.96 \\
$\mathbf{2 0 0 1}$ & 0.00 & 0.00 & 0.27 & 0.34 & 2.96 & 11.53 & 45.45 \\
$\mathbf{2 0 0 2}$ & 0.00 & 0.00 & 0.00 & 1.01 & 2.89 & 8.21 & 44.44 \\
$\mathbf{2 0 0 3}$ & 0.00 & 0.00 & 0.00 & 0.23 & 0.58 & 4.07 & 32.73 \\
$\mathbf{2 0 0 4}$ & 0.00 & 0.00 & 0.08 & 0.00 & 0.44 & 1.45 & 16.18 \\
$\mathbf{2 0 0 5}$ & 0.00 & 0.00 & 0.00 & 0.07 & 0.31 & 1.74 & 9.09 \\
$\mathbf{2 0 0 6}$ & 0.00 & 0.00 & 0.00 & 0.00 & 0.30 & 0.82 & 13.33 \\
$\mathbf{2 0 0 7}$ & 0.00 & 0.00 & 0.00 & 0.00 & 0.20 & 0.25 & 15.24 \\
$\mathbf{2 0 0 8}$ & 0.00 & 0.38 & 0.39 & 0.49 & 0.81 & 4.09 & 27.27 \\
$\mathbf{2 0 0 9}$ & 0.00 & 0.00 & 0.22 & 0.55 & 0.75 & 10.94 & 49.46 \\
$\mathbf{2 0 1 0}$ & 0.00 & 0.00 & 0.00 & 0.00 & 0.58 & 0.86 & 22.62 \\
$\mathbf{2 0 1 1}$ & 0.00 & 0.00 & 0.00 & 0.07 & 0.00 & 1.67 & 16.30 \\
$\mathbf{2 0 1 2}$ & 0.00 & 0.00 & 0.00 & 0.00 & 0.03 & 1.57 & 27.52 \\
$\mathbf{2 0 1 3}$ & 0.00 & 0.00 & 0.00 & 0.00 & 0.10 & 1.64 & 24.50 \\
$\mathbf{2 0 1 4}$ & 0.00 & 0.00 & 0.00 & 0.00 & 0.00 & 0.78 & 17.42 \\
$\mathbf{2 0 1 6}$ & 0.00 & 0.00 & 0.00 & 0.00 & 0.16 & 2.40 & 26.51 \\
 & 0.00 & 0.00 & 0.00 & 0.00 & 0.47 & 3.70 & 33.17 \\
\end{tabular}
\end{center}

\begin{center}
\begin{tabular}{lcccccccc}
\hline
 & AAA & AA & A & BBB & BB & B & CCC/C \\
\hline
$\mathbf{2 0 1 7}$ & 0.00 & 0.00 & 0.00 & 0.00 & 0.08 & 0.98 & 26.23 \\
Mean & 0.00 & 0.03 & 0.07 & 0.19 & 0.69 & 3.82 & 27.98 \\
Max & 0.00 & 0.38 & 0.39 & 1.01 & 2.96 & 11.53 & 45.45 \\
Min & 0.00 & 0.00 & 0.00 & 0.00 & 0.00 & 0.25 & 9.09 \\
\hline
\end{tabular}
\end{center}

Source: Based on data from Standard \& Poor's Financial Services, LLC.

As Exhibit 6 shows, the highest-rated bonds have extremely low default rates. With very few exceptions, the lower the rating, the higher the annual rate of default, with bonds rated CCC and lower experiencing the highest default rates by far.

Relying on credit rating agency ratings, however, also has limitations and risks, including the following:

\begin{itemize}
  \item Credit ratings can change over time. Over a long time period (e.g., many years), credit ratings can migrate-move up or down-significantly from what they were at the time of bond issuance. Using Standard \& Poor's data, Exhibit 7 shows the average three-year migration (or "transition") by rating from 1981 to 2017 . Note that the higher the credit rating, the greater the ratings stability. Even for AAA rated credits, however, only about $65 \%$ of the time did ratings remain in that rating category over a three-year period. (Of course, AAA rated credits can have their ratings move in only one direction-down.) A very small fraction of AAA rated credits became non-investment grade or defaulted within three years. For single-B rated credits, only $41 \%$ of the time did ratings remain in that rating category over three-year periods. This observation about how credit ratings can change over time isn't meant to be a criticism of the rating agencies. It is meant to demonstrate that creditworthiness can and does change-up or down-and that bond investors should not assume an issuer's credit rating will remain the same from time of purchase through the entire holding period.
\end{itemize}

Exhibit 7: Average Three-Year Global Corporate Transition Rates, 1981-2017 (\%)

\begin{center}
\begin{tabular}{|c|c|c|c|c|c|c|c|c|c|}
\hline
From/To & AAA & AA & A & BBB & BB & B & CCC/C & D & NR \\
\hline
AAA & 65.48 & 22.09 & 2.35 & 0.32 & 0.19 & 0.08 & 0.11 & 0.13 & 9.24 \\
\hline
AA & 1.21 & 66.14 & 18.53 & 2.06 & 0.35 & 0.22 & 0.03 & 0.12 & 11.33 \\
\hline
A & 0.06 & 4.07 & 68.85 & 11.72 & 1.30 & 0.44 & 0.09 & 0.25 & 13.21 \\
\hline
BBB & 0.02 & 0.28 & 8.42 & 64.66 & 7.11 & 1.64 & 0.30 & 0.87 & 16.70 \\
\hline
BB & 0.01 & 0.06 & 0.51 & 11.08 & 47.04 & 11.58 & 1.25 & 3.96 & 24.51 \\
\hline
B & 0.00 & 0.03 & 0.21 & 0.78 & 10.23 & 41.46 & 4.67 & 12.57 & 30.05 \\
\hline
$\mathrm{CCC} / \mathrm{C}$ & 0.00 & 0.00 & 0.14 & 0.61 & 1.63 & 16.86 & 10.54 & 40.65 & 29.57 \\
\hline
\end{tabular}
\end{center}

Notes: $\mathrm{D}$ = default. NR = not rated-that is, certain corporate issuers were no longer rated by S\&P.

This could occur for a variety of reasons, including issuers paying off their debt and no longer needing ratings.

Source: S\&P Global Ratings, “2017 Annual Global Corporate Default Study and Rating Transitions" (5 April 2018): 53. - Credit ratings tend to lag the market's pricing of credit risk. Bond prices and credit spreads frequently move more quickly because of changes in perceived creditworthiness than rating agencies change their ratings (or even outlooks) up or down. Bond prices and relative valuations can move every day, whereas bond ratings, appropriately, don't change that often. Even over long time periods, however, credit ratings can badly lag changes in bond prices. Exhibit 8 shows the price and Moody's rating of a bond from US automaker Ford Motor Company before, during, and after the global financial crisis in 2008. Note how the bond's price moved down sharply well before Moody's downgraded its credit rating-multiple times-and also how the bond's price began to recover-and kept recovering-well before Moody's upgraded its credit rating on Ford debt.

Exhibit 8: Historical Example: Ford Motor Company Senior Unsecured Debt: Price vs. Moody's Rating 2005-2011

\begin{center}
\includegraphics[max width=\textwidth]{2023_05_04_36535b8d80b32081d422g-094}
\end{center}

Sources: Data based on Bloomberg Finance LP and Moody's Investors Service, Inc.

Moreover, particularly for certain speculative-grade credits, two bonds with similar ratings may trade at very different valuations. This is partly a result of the fact that credit ratings primarily try to assess the risk of default, whereas for low-quality credits, the market begins focusing more on expected loss (default probability times loss severity). So, bonds from two separate issuers with comparable (high) risk of default but different recovery rates may have similar ratings but trade at significantly different dollar prices.

Thus, bond investors who wait for rating agencies to change their ratings before making buy and sell decisions in their portfolios may be at risk of underperforming other investors who make portfolio decisions in advance of-or not solely based on-rating agency changes. - Rating agencies may make mistakes. The mis-rating of billions of dollars of subprime-backed mortgage securities is one example. Other historical examples include the mis-ratings of US companies Enron and WorldCom and European issuer Parmalat. Like many investors, the rating agencies did not understand the accounting fraud being committed in those companies.

\begin{itemize}
  \item Some risks are difficult to capture in credit ratings. Examples include litigation risk, such as that which can affect tobacco companies, or environmental and business risks faced by chemical companies and utility power plants. This would also include the impact from natural disasters. Leveraged transactions, such as debt-financed acquisitions and large stock buybacks (share repurchases), are often difficult to anticipate and thus to capture in credit ratings.
\end{itemize}

As described, there are risks in relying on credit rating agency ratings when investing in bonds. Thus, while the credit rating agencies will almost certainly continue to play a significant role in the bond markets, it is important for investors to perform their own credit analyses and draw their own conclusions regarding the credit risk of a given debt issue or issuer.

\section{EXAMPLE 4}
\section{Credit Ratings}
\begin{enumerate}
  \item Using the S\&P rating scale, investment-grade bonds carry which of the following ratings?
A. AAA to EEE
B. $\mathrm{BBB}-$ to $\mathrm{CCC}$
C. AAA to $\mathrm{BBB}-$
\end{enumerate}

\section{Solution:}
$\mathrm{C}$ is correct.

\begin{enumerate}
  \setcounter{enumi}{1}
  \item Using both Moody's and S\&P ratings, which of the following pairs of ratings is considered high yield, also known as "below investment grade," "speculative grade," or "junk"?
A. Baa1/BBB-
B. $\mathrm{B} 3 / \mathrm{CCC}+$
C. $\mathrm{Baa} / \mathrm{BB}+$
\end{enumerate}

Solution:

$B$ is correct. Note that issuers with such ratings as Baa3/BB+ (answer C) are called "crossovers" because one rating is investment grade (the Moody's rating of Baa3) and the other is high yield (the S\&P rating of $B B+$ ).

\begin{enumerate}
  \setcounter{enumi}{2}
  \item What is the difference between an issuer rating and an issue rating?
\end{enumerate}

A. The issuer rating applies to all of an issuer's bonds, whereas the issue rating considers a bond's seniority ranking.

B. The issuer rating is an assessment of an issuer's overall creditworthiness, whereas the issue rating is always higher than the issuer rating. C. The issuer rating is an assessment of an issuer's overall creditworthiness, typically reflected as the senior unsecured rating, whereas the issue rating considers a bond's seniority ranking (e.g., secured or subordinated).

\section{Solution:}
$\mathrm{C}$ is correct.

\begin{enumerate}
  \setcounter{enumi}{3}
  \item Based on the practice of notching by the rating agencies, a subordinated bond from a company with an issuer rating of BB would likely carry what rating?
A. $\mathrm{B}+$
B. $\mathrm{BB}$
C. $\mathrm{BBB}-$
\end{enumerate}

\section{Solution:}
A is correct. The subordinated bond would have its rating notched lower than the company's BB rating, probably by two notches, reflecting the higher weight given to loss severity for below-investment-grade credits.

\begin{enumerate}
  \setcounter{enumi}{4}
  \item The fixed-income portfolio manager you work with asked you why a bond from an issuer you cover didn't rise in price when it was upgraded by Fitch from B+ to BB. Which of the following is the most likely explanation?
\end{enumerate}

A. Bond prices never react to rating changes.

B. The bond doesn't trade often, so the price hasn't adjusted to the rating change yet.

C. The market was expecting the rating change, and so it was already "priced in" to the bond.

\section{Solution:}
$\mathrm{C}$ is correct. The market was anticipating the rating upgrade and had already priced it in. Bond prices often do react to rating changes, particularly multi-notch ones. Even if bonds don't trade, their prices adjust based on dealer quotations given to bond pricing services.

\begin{enumerate}
  \setcounter{enumi}{5}
  \item Amalgamated Corp. and Widget Corp. each have bonds outstanding with similar coupons and maturity dates. Both bonds are rated B2, B-, and B by Moody's, S\&P, and Fitch, respectively. The bonds, however, trade at very different prices-the Amalgamated bond trades at $€ 89$, whereas the Widget bond trades at $€ 62$. What is the most likely explanation of the price (and yield) difference?
\end{enumerate}

A. Widget's credit ratings are lagging the market's assessment of the company's credit deterioration.

B. The bonds have similar risks of default (as reflected in the ratings), but the market believes the Amalgamated bond has a higher expected loss in the event of default.

C. The bonds have similar risks of default (as reflected in the ratings), but the market believes the Widget bond has a higher expected recovery rate in the event of default.

\section*{Solution: }
A is correct. Widget's credit ratings are probably lagging behind the market's assessment of its deteriorating creditworthiness. Answers B and C both state the situation backwards. If the market believed that the Amalgamated bond had a higher expected loss given default, then that bond would be trading at a lower, not a higher, price. Similarly, if the market believed that the Widget bond had a higher expected recovery rate in the event of default, then that bond would be trading at a higher, not a lower, price.

\section{TRADITIONAL CREDIT ANALYSIS: CORPORATE DEBT SECURITIES}
The goal of credit analysis is to assess an issuer's ability to satisfy its debt obligations, including bonds and other indebtedness, such as bank loans. These debt obligations are contracts, the terms of which specify the interest rate to be paid, the frequency and timing of payments, the maturity date, and the covenants that describe the permissible and required actions of the borrower. Because corporate bonds are contracts, enforceable by law, credit analysts generally assume an issuer's willingness to pay and concentrate instead on assessing its ability to pay. Thus, the main focus in credit analysis is to understand a company's ability to generate cash flow over the term of its debt obligations. In so doing, analysts must assess both the credit quality of the company and the fundamentals of the industry in which the company operates. Traditional credit analysis considers the sources, predictability, and sustainability of cash generated by a company to service its debt obligations. This section will focus on corporate credit analysis; in particular, it will emphasize non-financial companies. Financial institutions have very different business models and funding profiles from industrial and utility companies.

\section{Credit Analysis vs. Equity Analysis: Similarities and Differences}
The description of credit analysis suggests credit and equity analyses should be very similar, and in many ways, they are. There are motivational differences, however, between equity and fixed-income investors that are an important aspect of credit analysis. Strictly speaking, management works for the shareholders of a company. Its primary objective is to maximize the value of the company for its owners. In contrast, management's legal duty to its creditors-including bondholders-is to meet the terms of the governing contracts. Growth in the value of a corporation from rising profits and cash flow accrues to the shareholders, while the best outcome for bondholders is to receive full, timely payment of interest and repayment of principal when due. Conversely, shareholders are more exposed to the decline in value if a company's earnings and cash flow decline because bondholders have a prior claim on cash flow and assets. But if a company's earnings and cash flow decline to the extent that it can no longer make its debt payments, then bondholders are at risk of loss as well.

In summary, in exchange for a prior claim on cash flow and assets, bondholders do not share in the growth in value of a company (except to the extent that its creditworthiness improves) but have downside risk in the event of default. In contrast, shareholders have theoretically unlimited upside opportunity, but in the event of default, their investment is typically wiped out before the bondholders suffer a loss. This is very similar to the type of payoff patterns seen in financial options. In fact, in recent years, credit risk models have been developed based on the insights of option pricing theory. Although it is beyond the scope of this present introduction to the subject, it is an expanding area of interest to both institutional investors and rating agencies.

Thus, although the analysis is similar in many respects for both equity and credit, equity analysts are interested in the strategies and investments that will increase a company's value and grow earnings per share. They then compare that earnings and growth potential with that of other companies in a given industry. Credit analysts will look more at the downside risk by measuring and assessing the sustainability of a company's cash flow relative to its debt levels and interest expense. Importantly for credit analysts, the balance sheet will show the composition of an issuer's debt-the overall amount, how much is coming due and when, and the distribution by seniority ranking. In general, equity analysts will focus more on income and cash flow statements, whereas credit analysts tend to focus more on the balance sheet and cash flow statements.

\section{The Four Cs of Credit Analysis: A Useful Framework}
Traditionally, many analysts evaluated creditworthiness based on what is often called the "four Cs of credit analysis":

\begin{itemize}
  \item Capacity

  \item Collateral

  \item Covenants

  \item Character

\end{itemize}

Capacity refers to the ability of the borrower to make its debt payments on time; this is the focus of this section. Collateral refers to the quality and value of the assets supporting the issuer's indebtedness. Covenants are the terms and conditions of lending agreements that the issuer must comply with. Character refers to the quality of management. Each of these will now be covered in greater detail. Please note that the list of Cs is a convenient way to summarize the important aspects of the analysis; it is not a checklist to be applied mechanically, nor is it always exhaustive.

\section{Capacity}
Capacity is the ability of a borrower to service its debt. To determine that, credit analysis, in a process similar to equity analysis, starts with industry analysis and then turns to examination of the specific issuer (company analysis).

\section{Industry structure.}
The Porter framework (Michael E. Porter, "The Five Competitive Forces That Shape Strategy," Harvard Business Review 86 [1; 2008]: 78-93) considers the effects of five competitive forces on an industry:

\begin{enumerate}
  \item Threat of entry. Threat of entry depends on the extent of barriers to entry and the expected response from incumbents to new entrants. Industries with high entry barriers tend to be more profitable and have lower credit risk than industries with low entry barriers because incumbents do not need to hold down prices or take other steps to deter new entrants. High entry barriers can take many forms, including high capital investment, such as in aerospace; large, established distribution systems, such as in auto dealerships; patent protection, such as in technology or pharmaceutical industries; or a high degree of regulation, such as in utilities.

  \item Bargaining power of suppliers. An industry that relies on just a few suppliers tends to be less profitable and to have greater credit risk than an industry that has multiple suppliers. Industries and companies with just a few suppliers have limited negotiating power to keep the suppliers from raising prices, whereas industries that have many suppliers can play them off against each other to keep prices in check.

  \item Bargaining power of customers. Industries that rely heavily on just a few main customers have greater credit risk because the negotiating power lies with the buyers.

  \item Threat of substitutes. Industries (and companies) that offer products and services that provide great value to their customers, and for which there are not good or cost-competitive substitutes, typically have strong pricing power, generate substantial cash flows, and represent less credit risk than other industries or companies. Certain (patent-protected) drugs are an example. Over time, however, disruptive technologies and inventions can increase substitution risk. For example, years ago, airplanes began displacing many trains and steamships. Newspapers were considered to have a nearly unassailable market position until television and then the internet became substitutes for how people received news and information. Over time, recorded music has shifted from records to tapes, to compact discs, to $\mathrm{mp} 3 \mathrm{~s}$ and other forms of digital media.

  \item Rivalry among existing competitors. Industries with strong rivalrybecause of numerous competitors, slow industry growth, or high barriers to exit-tend to have less cash flow predictability and, therefore, higher credit risk than industries with less competition. Regulation can affect the extent of rivalry and competition. For example, regulated utilities typically have a monopoly position in a given market, which results in relatively stable and predictable cash flows.

\end{enumerate}

It is important to consider how companies in an industry generate revenues and earn profits. Is it an industry with high fixed costs and capital investment or one with modest fixed costs? These structures generate revenues and earn profits in very different ways. Two examples of industries with high fixed costs, also referred to as "having high operating leverage," are airlines and hotels. Many of their operating costs are fixed-running a hotel, flying a plane-so they cannot easily cut costs. If an insufficient number of people stay at a hotel or fly in a plane, fixed operating costs may not be covered and losses may result. With higher occupancy of a hotel or plane, revenues are higher, and it is more likely that fixed costs will be covered and profits earned.

\section{Industry fundamentals.}
After understanding an industry's structure, the next step is to assess its fundamentals, including its sensitivity to macroeconomic factors, its growth prospects, its profitability, and its business need-or lack thereof-for high credit quality. Judgments about these can be made by looking at the following:

\begin{itemize}
  \item Cyclical or non-cyclical. This is a crucial assessment because industries that are cyclical-that is, have greater sensitivity to broader economic performance-have more volatile revenues, margins, and cash flows and
\end{itemize}

\section{Fundamentals of Credit Analysis}
thus are inherently riskier than non-cyclical industries. Food producers, retailers, and health care companies are typically considered non-cyclical, whereas auto and steel companies can be very cyclical. Companies in cyclical industries should carry lower levels of debt relative to their ability to generate cash flow over an economic cycle than companies in less-cyclical or non-cyclical industries.

\begin{itemize}
  \item Growth prospects. Although growth is typically a greater focus for equity analysts than for credit analysts, bond investors have an interest in growth as well. Industries that have little or no growth tend to consolidate via mergers and acquisitions. Depending upon how these are financed (e.g., using stock or debt) and the economic benefits (or lack thereof) of the merger, they may or may not be favorable to corporate bond investors. Weaker competitors in slow-growth industries may begin to struggle financially, adversely affecting their creditworthiness.

  \item Published industry statistics. Analysts can get an understanding of an industry's fundamentals and performance by researching statistics that are published by and available from a number of different sources, including the rating agencies, investment banks, industry publications, and frequently, government agencies.

\end{itemize}

\section{Company fundamentals.}
Following analysis of an industry's structure and fundamentals, the next step is to assess the fundamentals of the company: the corporate borrower. Analysts should examine the following:

\begin{itemize}
  \item Competitive position

  \item Track record/operating history

  \item Management's strategy and execution

  \item Ratios and ratio analysis

  \item When assessing the company fundamentals, analysts should also explore how the fundamentals are impacted by environmental, social, and governance (ESG) factors.

\end{itemize}

\section{Competitive position.}
Based on their knowledge of the industry structure and fundamentals, analysts assess a company's competitive position within the industry. What is its market share? How has it changed over time: Is it increasing, decreasing, holding steady? Is it well above (or below) its peers? How does it compare with respect to cost structure? How might it change its competitive position? What sort of financing might that require?

\section{Track record/Operating history.}
How has the company performed over time? It's useful to go back several years and analyze the company's financial performance, perhaps during times of both economic growth and contraction. What are the trends in revenues, profit margins, and cash flow? Capital expenditures represent what percentage of revenues? What are the trends on the balance sheet-use of debt versus equity? Was this track record developed under the current management team? If not, when did the current management team take over? Management's strategy and execution.

What is management's strategy for the company-to compete and to grow? Does it make sense, and is it plausible? How risky is it, and how differentiated is it from its industry peers? Is it venturing into unrelated businesses? Does the analyst have confidence in management's ability to execute? What is management's track record, both at this company and at previous ones?

Credit analysts also want to know and understand how management's strategy will affect its balance sheet. Does management plan to manage the balance sheet prudently, in a manner that doesn't adversely affect bondholders? Analysts can learn about management's strategy from reading comments, discussion, and analysis that are included with financial statements filed with appropriate regulators, listening to conference calls about earnings or other big announcements (e.g., acquisitions), going to company websites to find earnings releases and copies of slides of presentations at various industry conferences, visiting and speaking with the company, and so on.

\section{EXAMPLE 5}
\section{Industry and Company Analysis}
\begin{enumerate}
  \item Given a hotel company, a chemical company, and a food retail company, which is most likely to be able to support a high debt load over an economic cycle?
\end{enumerate}

A. The hotel company, because people need a place to stay when they travel.

B. The chemical company, because chemicals are a key input to many products.

C. The food retail company, because such products as food are typically resistant to recessions.

\section{Solution:}
$\mathrm{C}$ is correct. Food retail companies are considered non-cyclical, whereas hotel and chemical companies are more cyclical and thus more vulnerable to economic downturns.

\begin{enumerate}
  \setcounter{enumi}{1}
  \item Heavily regulated monopoly companies, such as utilities, often carry high debt loads. Which of the following statements about such companies is most accurate?
\end{enumerate}

A. Regulators require them to carry high debt loads.

B. They generate strong and stable cash flows, enabling them to support high levels of debt.

C. They are not very profitable and need to borrow heavily to maintain their plant and equipment.

\section{Solution:}
B is correct. Because such monopolies' financial returns are generally dictated by the regulators, they generate consistent cash flows and are thus able to support high debt levels.

\begin{enumerate}
  \setcounter{enumi}{2}
  \item XYZ Corp. manufactures a commodity product in a highly competitive industry in which no company has significant market share and where there are low barriers to entry. Which of the following best describes XYZ's ability to take on substantial debt?
\end{enumerate}

A. Its ability is very limited because companies in industries with those characteristics generally cannot support high debt loads.

B. Its ability is high because companies in industries with those characteristics generally have high margins and cash flows that can support significant debt.

C. We don't have enough information to answer the question.

Solution:

A is correct. Companies in industries with those characteristics typically have low margins and limited cash flow and thus cannot support high debt levels.

\section{Ratios and ratio analysis.}
To provide context to the analysis and understanding of a company's fundamentalsbased on the industry in which it operates, its competitive position, its strategy and execution-a number of financial measures derived from the company's principal financial statements are examined. Credit analysts calculate a number of ratios to assess the financial health of a company, identify trends over time, and compare companies across an industry to get a sense of relative creditworthiness. Note that typical values of these ratios vary widely from one industry to another because of different industry characteristics previously identified: competitive structure, economic cyclicality, regulation, and so on.

We will categorize the key credit analysis measures into three different groups:

\begin{itemize}
  \item Profitability and cash flow

  \item Leverage

  \item Coverage

\end{itemize}

\section{Profitability and cash flow measures.}
It is from profitability and cash flow generation that companies can service their debt. Credit analysts typically look at operating profit margins and operating income to get a sense of a company's underlying profitability and see how it varies over time. Operating income is defined as operating revenues minus operating expenses and is commonly referred to as "earnings before interest and taxes" (EBIT). Credit analysts focus on EBIT because it is useful to determine a company's performance prior to costs arising from its capital structure (i.e., how much debt it carries versus equity). And "before taxes" is used because interest expense is paid before income taxes are calculated.

Several measures of cash flow are used in credit analysis; some are more conservative than others because they make certain adjustments for cash that gets used in managing and maintaining the business or in making payments to shareholders. The cash flow measures and leverage and coverage ratios discussed next are non-IFRS in the sense that they do not have official IFRS definitions; the concepts, names, and definitions given should be viewed as one usage among several possible, in most cases.

\begin{itemize}
  \item Earnings before interest, taxes, depreciation, and amortization
\end{itemize}

(EBITDA). EBITDA is a commonly used measure of cash flow that takes operating income and adds back depreciation and amortization expense because those are non-cash items. This is a somewhat crude measure of cash flow because it excludes certain cash-related expenses of running a business, such as capital expenditures and changes in (non-cash) working capital. Thus, despite its popularity as a cash flow measure, analysts look at other measures in addition to EBITDA.

\begin{itemize}
  \item Funds from operations (FFO). Standard \& Poor's defines funds from operations as net income from continuing operations plus depreciation, amortization, deferred income taxes, and other non-cash items.

  \item Free cash flow before dividends (FCF before dividends). This measures excess cash flow generated by the company (excluding non-recurring items) before payments to shareholders or that could be used to pay down debt or pay dividends. It can be calculated as net income (excluding non-recurring items) plus depreciation and amortization minus increase (plus decrease) in non-cash working capital minus capital expenditures. This is, depending upon the treatment of dividends and interest in the cash flow statement, approximated by the cash flow from operating activities minus capital expenditures. Companies that have negative free cash flow before payments to shareholders will be consuming cash they have or will need to rely on additional financing-from banks, bond investors, or equity investors. This obviously represents higher credit risk.

  \item Free cash flow after dividends (FCF after dividends). This measure just takes free cash flow before dividends and subtracts dividend payments. If this number is positive, it represents cash that could be used to pay down debt or build up cash on the balance sheet. Either action may be viewed as deleveraging, which is favorable from a credit risk standpoint. Some credit analysts will calculate net debt by subtracting balance sheet cash from total debt, although they shouldn't assume the cash will be used to pay down debt. Actual debt paid down from free cash flow is a better indicator of deleveraging. Some analysts will also deduct stock buybacks to get the "truest" measure of free cash flow that can be used to de-lever on either a gross or net debt basis; however, others view stock buybacks (share repurchases) as more discretionary and as having less certain timing than dividends, and thus treat those two types of shareholder payments differently when calculating free cash flow.

\end{itemize}

\section{Leverage ratios.}
A few measures of leverage are used by credit analysts. The most common are the debt/capital, debt/EBITDA, and measures of funds or cash flows/debt ratios. Note that many analysts adjust a company's reported debt levels for debt-like liabilities, such as underfunded pensions and other retiree benefits, as well as operating leases. When adjusting for leases, analysts will typically add back the imputed interest or rent expense to various cash flow measures.

\begin{itemize}
  \item Debt/capital. Capital is calculated as total debt plus shareholders equity. This ratio shows the percentage of a company's capital base that is financed with debt. A lower percentage of debt indicates lower credit risk. This traditional ratio is generally used for investment-grade corporate issuers. Where goodwill or other intangible assets are significant (and subject to obsolescence, depletion, or impairment), it is often informative to also compute the debt to capital ratio after assuming a write-down of the after-tax value of such assets.

  \item Debt/EBITDA. This ratio is a common leverage measure. Analysts use it on a "snapshot" basis, as well as to look at trends over time and at projections and to compare companies in a given industry. Rating agencies often use it as a trigger for rating actions, and banks reference it in loan covenants. A

\end{itemize}

\section{Fundamentals of Credit Analysis}
higher ratio indicates more leverage and thus higher credit risk. Note that this ratio can be very volatile for companies with high cash flow variability, such as those in cyclical industries and with high operating leverage (fixed costs).

\begin{itemize}
  \item FFO/debt. Credit rating agencies often use this leverage ratio. They publish key median and average ratios, such as this one, by rating category so analysts can get a sense of why an issuer is assigned a certain credit rating, as well as where that rating may migrate based on changes to such key ratios as this one. A higher ratio indicates greater ability to pay debt by funds from operations.

  \item FCF after dividends/debt. A higher ratio indicates that a greater amount of debt can be paid off from free cash flow after dividend payments.

\end{itemize}

\section{Coverage ratios.}
Coverage ratios measure an issuer's ability to meet-to "cover"-its interest payments. The two most common are the EBITDA/interest expense and EBIT/interest expense ratios.

\begin{itemize}
  \item EBITDA/interest expense. This measurement of interest coverage is a bit more liberal than the one that uses EBIT because it does not subtract out the impact of (non-cash) depreciation and amortization expense. A higher ratio indicates higher credit quality.

  \item EBIT/interest expense. Because EBIT does not include depreciation and amortization, it is considered a more conservative measure of interest coverage. This ratio is now used less frequently than EBITDA/interest expense.

\end{itemize}

Exhibit 9 is an example of key average credit ratios by rating category for industrial companies over the 12-month period 3Q2017-3Q2018, as calculated by Bloomberg Barclays Indices, using public company data. Note only a few AAA-rated corporations remain, so the small sample size can skew the average ratios of a few key credit metrics. That said, it should be clear that, overall, higher-rated issuers have stronger credit metrics.

\section{Exhibit 9: Industrial Comparative Ratio Analysis}
\begin{center}
\begin{tabular}{|c|c|c|c|c|c|c|c|c|}
\hline
$\begin{array}{l}\text { Credit } \\ \text { Rating }\end{array}$ & $\begin{array}{l}\text { EBITDA } \\ \text { Margin } \\ (\%)\end{array}$ & $\begin{array}{c}\text { Return on } \\ \text { Capital } \\ \text { (\%) }\end{array}$ & $\begin{array}{c}\text { EBIT } \\ \text { Interest } \\ \text { Coverage } \\ \text { (x) }\end{array}$ & $\begin{array}{c}\text { EBITDA } \\ \text { Interest } \\ \text { Coverage } \\ \text { (x) }\end{array}$ & $\begin{array}{c}\text { FFO/Debt } \\ \text { (\%) }\end{array}$ & $\begin{array}{l}\text { Free Operations } \\ \text { Cash Flow/Debt } \\ \text { (\%) }\end{array}$ & $\begin{array}{l}\text { Debt/ } \\ \text { EBITDA } \\ \text { (x) }\end{array}$ & $\begin{array}{l}\text { Debt/Debt } \\ \text { plus Equity } \\ \text { (\%) }\end{array}$ \\
\hline
\multicolumn{9}{|l|}{Aаa} \\
\hline
US & 66.4 & 6.5 & 4.2 & 21.3 & 51.9 & 43.5 & -0.2 & 43.3 \\
\hline
\multicolumn{9}{|l|}{Aa} \\
\hline
US & 21.9 & 10.8 & 15.4 & 45.0 & 109.9 & 58.1 & 1.2 & 50.6 \\
\hline
\multicolumn{9}{|l|}{A} \\
\hline
US & 26.0 & 13.5 & 13.3 & 18.9 & 49.1 & 31.8 & 1.8 & 51.2 \\
\hline
\multicolumn{9}{|l|}{Baa} \\
\hline
US & 23.9 & 11.5 & 7.2 & NA & 40.7 & 20.3 & 3.9 & 49.4 \\
\hline
\multicolumn{9}{|l|}{Ba} \\
\hline
US & 21.7 & 3.5 & 4.6 & NA & 27.7 & 11.0 & 4.1 & 64.0 \\
\hline
B &  &  &  &  &  &  &  &  \\
\hline
\end{tabular}
\end{center}

\begin{center}
\begin{tabular}{|c|c|c|c|c|c|c|c|c|}
\hline
$\begin{array}{l}\text { Credit } \\ \text { Rating }\end{array}$ & $\begin{array}{c}\text { EBITDA } \\ \text { Margin } \\ \text { (\%) }\end{array}$ & $\begin{array}{c}\text { Return on } \\ \text { Capital } \\ \text { (\%) }\end{array}$ & $\begin{array}{c}\text { EBIT } \\ \text { Interest } \\ \text { Coverage } \\ \text { (x) }\end{array}$ & $\begin{array}{c}\text { EBITDA } \\ \text { Interest } \\ \text { Coverage } \\ \text { (x) }\end{array}$ & $\begin{array}{c}\text { FFO/Debt } \\ \text { (\%) }\end{array}$ & $\begin{array}{c}\text { Free Operations } \\ \text { Cash Flow/Debt } \\ (\%)\end{array}$ & $\begin{array}{c}\text { Debt/ } \\ \text { EBITDA } \\ (x)\end{array}$ & $\begin{array}{l}\text { Debt/Debt } \\ \text { plus Equity } \\ \text { (\%) }\end{array}$ \\
\hline
$U S$ & 21.2 & 3.7 & 2.5 & NA & 20.3 & 1.8 & 5.2 & 69.3 \\
\hline
\multicolumn{9}{|l|}{Caa} \\
\hline
$U S$ & 16.0 & 0.2 & -0.6 & 1.3 & 10.0 & -6.6 & 9.3 & 95.3 \\
\hline
\end{tabular}
\end{center}

Note: As of 19 December 2018.

Source: Bloomberg Barclays Indices.

Comments on issuer liquidity.

An issuer's access to liquidity is also an important consideration in credit analysis. Companies with high liquidity represent lower credit risk than those with weak liquidity, other factors being equal. The global financial crisis of 2008-2009 showed that access to liquidity via the debt and equity markets should not be taken for granted, particularly for companies that do not have strong balance sheets or steady operating cash flow.

When assessing an issuer's liquidity, credit analysts tend to look at the following:

\begin{itemize}
  \item Cash on the balance sheet. Cash holdings provide the greatest assurance of having sufficient liquidity to make promised payments.

  \item Net working capital. The big US automakers used to have enormous negative working capital, despite having high levels of cash on the balance sheet. This proved disastrous when the global financial crisis hit in 2008 and the economy contracted sharply. Auto sales-and thus revenues-fell, the auto companies cut production, and working capital consumed billions of dollars in cash as accounts payable came due when the companies most needed liquidity.

  \item Operating cash flow. Analysts will project this figure out a few years and consider the risk that it may be lower than expected.

  \item Committed bank lines. Committed but untapped lines of credit provide contingent liquidity in the event that the company is unable to tap other, potentially cheaper, financing in the public debt markets.

  \item Debt coming due and committed capital expenditures in the next one to two years. Analysts will compare the sources of liquidity with the amount of debt coming due as well as with committed capital expenditures to ensure that companies can repay their debt and still invest in the business if the capital markets are somehow not available.

\end{itemize}

As will be discussed in more detail in the section on special considerations for high-yield credits, issuer liquidity is a bigger consideration for high-yield companies than for investment-grade companies.

\section{EXAMPLE 6}
Mallinckrodt PLC (Mallinckrodt) is an Ireland-incorporated specialty pharmaceutical company. As a credit analyst, you have been asked to assess its creditworthiness-on its own, compared to a competitor in its overall industry, and compared with a similarly rated company in a different industry. Using the financial statements provided in Exhibit 10 through Exhibit 15 for the three years ending 31 December 2015, 2016, and 2017, address the following:

\section{Exhibit 10: Mallinckrodt PLC Financial Statements}
Consolidated Statements of Operations

(Dollars in millions, except per share amounts)

Net revenues

\begin{center}
\begin{tabular}{ccc}
$\begin{array}{c}\text { Year End } \\ \text { 30 Sept.* }\end{array}$ & \multicolumn{2}{c}{Years Ended 31 Dec.} \\
\hline
2015 & 2016 & 2017 \\
\hline
$2,923.1$ & $3,399.5$ & $3,221.6$ \\
\hline
\end{tabular}
\end{center}

Operating expenses:

Cost of sales

Research and development

Selling, general, and administrative expenses

Restructuring charges, net

Non-restructuring impairment charges

Gain on divestiture and license

$\begin{array}{ccc}1,300.2 & 1,549.6 & 1,565.3 \\ 203.3 & 267.0 & 277.3 \\ 1,023.8 & 1,070.3 & 920.9\end{array}$

Total operating expenses

$\begin{array}{lll}45.0 & 33.0 & 31.2\end{array}$

$\begin{array}{lll}- & 231.2 & 63.7\end{array}$

(3.0) $\quad-\quad$ (56.9)

$2,569.3 \quad 3,151.1 \quad 2,801.5$

Operating income

Other (expense) income:

Interest expense

(255.6) $\quad(378.1)$

$(369.1)$

Interest income

Other income (expense), net

Total other (expense) income, net

$\begin{array}{ccc}1.0 & 1.6 & 4.6 \\ 8.1 & (3.5) & 6.0 \\ (\mathbf{2 4 6 . 5 )} & \mathbf{( 3 8 0 . 0 )} & \mathbf{( 3 5 8 . 5 )}\end{array}$

Income before income taxes and

107.3

(131.6)

61.6 non-controlling interest

Provision (Benefit) for income taxes

(129.3)

$(340.0)$

$(1,709.6)$

Net income

Income from discontinued operations, net of income taxes

Net income attributable to common

\section{Exhibit 11: Mallinckrodt PLC Financial Statements}
Consolidated Balance Sheets

\begin{center}
\begin{tabular}{|c|c|c|c|}
\hline
\multirow[b]{2}{*}{(Dollars in millions)} & \multirow{2}{*}{$\begin{array}{c}\begin{array}{c}\text { Year End } \\
30 \text { Sept.* }\end{array} \\
2015\end{array}$} & \multicolumn{2}{|c|}{Years Ended 31 Dec.} \\
\hline
 &  & 2016 & 2017 \\
\hline
\multicolumn{4}{|l|}{ASSETS} \\
\hline
\multicolumn{4}{|l|}{Current assets:} \\
\hline
Cash and cash equivalents & 365.9 & 342.0 & $1,260.9$ \\
\hline
Accounts receivable & 489.6 & 431.0 & 445.8 \\
\hline
Inventories & 262.1 & 350.7 & 340.4 \\
\hline
Deferred income taxes & 139.2 & - & - \\
\hline
Prepaid expenses and other current assets & 194.4 & 131.9 & 84.1 \\
\hline
Notes receivable & - & - & 154.0 \\
\hline
Current assets held for sale & 394.9 & 310.9 & - \\
\hline
Total current assets & $1,846.1$ & $1,566.5$ & $2,285.2$ \\
\hline
Property, plant, and equipment, net & 793.0 & 881.5 & 966.8 \\
\hline
Goodwill & $3,649.4$ & $3,498.1$ & $3,482.7$ \\
\hline
Intangible assets, net & $9,666.3$ & $9,000.5$ & $8,375.0$ \\
\hline
Other assets & 225.7 & 259.7 & 171.2 \\
\hline
Long-term assets held for sale & 223.6 & - & - \\
\hline
Total assets & $16,404.1$ & $15,206.3$ & $15,280.9$ \\
\hline
\end{tabular}
\end{center}

\section{LIABILITIES AND EQUITY}
Current liabilities:

Current maturities of long-term debt

Accounts payable

Accrued payroll and payroll-related costs

Accrued interest

Income taxes payable

Accrued and other current liabilities

Current liabilities held for sale

Total current liabilities

Long-term debt

Pension and post-retirement benefits

Environmental liabilities

Deferred income taxes

Other income tax liabilities

Other liabilities

Long-term liabilities held for sale

Total liabilities

$\begin{array}{ccc}22.0 & 271.2 & 313.7 \\ 116.8 & 112.1 & 113.3 \\ 95.0 & 76.1 & 98.5 \\ 80.2 & 68.7 & 57.0 \\ - & 101.7 & 15.8 \\ 486.1 & 557.1 & 452.1 \\ 129.3 & 120.3 & - \\ \mathbf{9 2 9 . 4} & \mathbf{1 , 3 0 7 . 2} & \mathbf{1 , 0 5 0 . 4} \\ & & \\ 6,474.3 & 5,880.8 & 6,420.9 \\ 114.2 & 136.4 & 67.1 \\ 73.3 & 73.0 & 73.2 \\ 3,117.5 & 2,398.1 & 689.0 \\ 121.3 & 70.4 & 94.1 \\ 209.0 & 356.1 & 364.2 \\ 53.9 & - & - \\ \mathbf{1 1 , 0 9 2 . 9} & \mathbf{1 0 , 2 2 2 . 0} & \mathbf{8 , 7 5 8 . 9}\end{array}$

Shareholders' equity:

\section{Consolidated Balance Sheets}
\begin{center}
\begin{tabular}{lccc}
\hline
 & Year End &  &  \\
 & $\mathbf{3 0}$ Sept.* & Years Ended 31 Dec. &  \\
\hline
(Dollars in millions) & $\mathbf{2 0 1 5}$ & $\mathbf{2 0 1 6}$ & $\mathbf{2 0 1 7}$ \\
\hline
Ordinary shares & 23.5 & 23.6 & 18.4 \\
Ordinary shares held in treasury at cost & $(109.7)$ & $(919.8)$ & $(1,564.7)$ \\
Additional paid-in capital & $5,357.6$ & $5,424.0$ & $5,492.6$ \\
Retained earnings & 38.9 & 529.0 & $2,588.6$ \\
Accumulated other comprehensive income & 0.9 & $(72.5)$ & $(12.9)$ \\
Total shareholders' equity & $5,311.2$ & $4,984.3$ & $6,522.0$ \\
Total liabilities and shareholders' equity & $16,404.1$ & $15,206.3$ & $15,280.9$ \\
\hline
\end{tabular}
\end{center}

"Mallinckrodt changed their fiscal year end from 30 September to 31 December.

Source: Company filings, Loomis, Sayles \& Company

Exhibit 12: Mallinckrodt PLC Financial Statements

\section{Consolidated Statements of Cash Flow}
\begin{center}
\begin{tabular}{|c|c|c|c|}
\hline
\multirow[b]{2}{*}{(Dollars in millions)} & \multirow{2}{*}{$\begin{array}{c}\text { Year End } \\
30 \text { Sept.* } \\
2015\end{array}$} & \multicolumn{2}{|c|}{Years Ended 31 Dec} \\
\hline
 &  & 2016 & 2017 \\
\hline
\multicolumn{4}{|l|}{Cash Flows from Operating Activities:} \\
\hline
Net income (loss) & 324.7 & 279.4 & $2,134.4$ \\
\hline
Depreciation and amortization & 672.5 & 831.7 & 808.3 \\
\hline
Share-based compensation & 117.0 & 45.4 & 59.2 \\
\hline
Deferred income taxes & $(191.6)$ & $(528.3)$ & -1744.1 \\
\hline
Non-cash impairment charges & - & 231.2 & 63.7 \\
\hline
Inventory provisions & - & 8.5 & 34.1 \\
\hline
$\begin{array}{l}\text { Gain on disposal of discontinued } \\ \text { operations }\end{array}$ & - & 1.7 & -418.1 \\
\hline
Other non-cash items & $(25.5)$ & 45.5 & -21.4 \\
\hline
Change in working capital & 33.4 & 153.7 & -188.8 \\
\hline
Net cash from operating activities & 930.5 & $1,068.8$ & 727.3 \\
\hline
\multicolumn{4}{|l|}{Cash Flows from Investing Activities:} \\
\hline
Capital expenditures & $(148.0)$ & $(199.1)$ & -186.1 \\
\hline
$\begin{array}{l}\text { Acquisitions and intangibles, net of cash } \\ \text { acquired }\end{array}$ & $(2,154.7)$ & $(247.2)$ & -76.3 \\
\hline
Proceeds from divestitures, net of cash & - & 3.0 & 576.9 \\
\hline
Other & 3.0 & $(4.9)$ & 3.9 \\
\hline
Net cash from investing activities & $(2,299.7)$ & $(448.2)$ & 318.4 \\
\hline
\multicolumn{4}{|l|}{Cash Flows from Financing Activities:} \\
\hline
Issuance of external debt & $3,010.0$ & 226.3 & 1,465 \\
\hline
\end{tabular}
\end{center}

\section{Consolidated Statements of Cash Flow}
\begin{center}
\begin{tabular}{|c|c|c|c|}
\hline
\multirow[b]{2}{*}{(Dollars in millions)} & \multirow{2}{*}{$\begin{array}{c}\text { Year End } \\
30 \text { Sept.* } \\
2015\end{array}$} & \multicolumn{2}{|c|}{Years Ended 31 Dec.} \\
\hline
 &  & 2016 & 2017 \\
\hline
$\begin{array}{l}\text { Repayment of external debt and capital } \\ \text { leases }\end{array}$ & $(1,848.4)$ & $(525.7)$ & -917.2 \\
\hline
Debt financing costs & $(39.9)$ & - & -12.7 \\
\hline
Proceeds from exercise of share options & 34.4 & 10.8 & 4.1 \\
\hline
Repurchase of shares & $(92.2)$ & $(536.3)$ & -651.7 \\
\hline
Other & $(28.1)$ & $(21.8)$ & -17.7 \\
\hline
Net cash from financing activities & $1,035.8$ & $(846.7)$ & -130.2 \\
\hline
Effect of currency rate changes on cash & $(11.6)$ & $(1.2)$ & 2.5 \\
\hline
$\begin{array}{l}\text { Net increase (decrease) in cash and cash } \\ \text { equivalents }\end{array}$ & $(345.0)$ & $(227.3)$ & 918 \\
\hline
$\begin{array}{l}\text { Cash and cash equivalents at beginning of } \\ \text { period }\end{array}$ & 777.6 & 588.4 & 361.1 \\
\hline
Cash and cash equivalents at end of period & 432.6 & 361.1 & $1,279.1$ \\
\hline
\end{tabular}
\end{center}

"Mallinckrodt changed their fiscal year end from 30 September to 31 December.

Source: Company filings, Loomis, Sayles \& Company.

Exhibit 13: Mallinckrodt PLC Credit Ratios

\begin{center}
\begin{tabular}{lccc}
\hline
 & $\mathbf{2 0 1 5}$ & $\mathbf{2 0 1 6}$ & $\mathbf{2 0 1 7}$ \\
\hline
Operating margin & $12.1 \%$ & $7.3 \%$ & $13.0 \%$ \\
Debt/EBITDA & $6.3 \mathrm{x}$ & $5.7 \mathrm{x}$ & $5.5 \mathrm{x}$ \\
EBITDA/Interest & $4.0 \mathrm{x}$ & $2.9 \mathrm{x}$ & $3.3 \mathrm{x}$ \\
FCF/Debt & $12.0 \%$ & $14.1 \%$ & $8.0 \%$ \\
Debt/Capital & $55.0 \%$ & $55.2 \%$ & $50.8 \%$ \\
\hline
\end{tabular}
\end{center}

Source: Company filings, Loomis, Sayles \& Company.

Exhibit 14: Johnson \& Johnson's Credit Ratios

\begin{center}
\begin{tabular}{lccc}
\hline
 & $\mathbf{2 0 1 5}$ & $\mathbf{2 0 1 6}$ & $\mathbf{2 0 1 7}$ \\
\hline
Operating profit margin & $26.2 \%$ & $29.5 \%$ & $25.8 \%$ \\
Debt/EBITDA & $0.9 \mathrm{x}$ & $1.1 \mathrm{x}$ & $1.4 \mathrm{x}$ \\
EBITDA/Interest & $40.1 \mathrm{x}$ & $34.4 \mathrm{x}$ & $27.1 \mathrm{x}$ \\
FCF after dividends/Debt & $81.1 \%$ & $57.3 \%$ & $51.4 \%$ \\
Debt/Capital & $22.0 \%$ & $28.1 \%$ & $36.5 \%$ \\
\hline
\end{tabular}
\end{center}

Source: Company filings, Loomis, Sayles \& Company. Exhibit 15: ArcelorMittal Credit Ratios

\begin{center}
\begin{tabular}{lccc}
\hline
 & $\mathbf{2 0 1 5}$ & $\mathbf{2 0 1 6}$ & $\mathbf{2 0 1 7}$ \\
\hline
Operating profit margin & $0.3 \%$ & $5.5 \%$ & $7.7 \%$ \\
Debt/EBITDA & $5.8 \mathrm{x}$ & $2.3 \mathrm{x}$ & $1.6 \mathrm{x}$ \\
EBITDA/Interest & $2.5 \mathrm{x}$ & $5.0 \mathrm{x}$ & $9.2 \mathrm{x}$ \\
FCF after dividends/Debt & $-2.8 \%$ & $1.9 \%$ & $13.5 \%$ \\
Debt/Capital & $41.8 \%$ & $29.7 \%$ & $24.0 \%$ \\
\hline
\end{tabular}
\end{center}

Source: Company filings, Loomis, Sayles \& Company.

\begin{enumerate}
  \item Calculate Mallinckrodt's operating profit margin, EBITDA, and free cash flow after dividends. Comment on what these measures indicate about Mallinckrodt's profitability and cash flow.
\end{enumerate}

Operating profit margin (\%) = Operating income/Revenue

2015: $353.8 / 2,923.1=0.121$ or $12.1 \%$

2016: $248.4 / 3,399.5=0.073$ or $7.3 \%$

2017: $420.1 / 3,221.6=0.130$ or $13.0 \%$

EBITDA $=$ Operating income + Depreciation and Amortization

2015: $353.8+672.5=1,026.3$

2016: $248.4+831.7=1,080.1$

$2017: 420.1+808.3=1,228.4$

FCF after dividends $=$ Cash flow from operations - Capital expenditures Dividends

2015: $930.5-148-0=782.5$

2016: $1,068.8-199.1-0=869.7$

2017: $727.3-186.1-0=541.2$

Operating profit margin decreased from 2015 to 2016 but increased from 2016 to 2017. Conversely, FCF after dividends increased from 2015 to 2016 but decreased from 2016 to 2017. EBITDA increased from 2015 to 2017. From 2015 to 2016, sales increased by $16.3 \%$ and operating expenses increased by $22.6 \%$. As a result, operating profit margin decreased even though EBITDA and FCF increased. However, from 2016 to 2017, sales decreased by $5.2 \%$ and operating expenses decreased by $11.1 \%$. As a result, operating profit margin and EBITDA increased, while FCF after dividends decreased.

\begin{enumerate}
  \setcounter{enumi}{1}
  \item Determine Mallinckrodt's leverage ratios: debt/EBITDA, debt/capital, free cash flow after dividends/debt. Comment on what these leverage ratios indicate about Mallinckrodt's creditworthiness.
\end{enumerate}

Debt/EBITDA

Total debt $=$ Short-term debt and Current portion of long-term debt + Long-term debt 2015: Debt: $22.0+6,474.3=6,496.3$

Debt/EBITDA: $6,496.3 / 1,026.3=6.3 x$

2016: Debt: $271.2+5,880.8=6,152.0$

Debt/EBITDA: 6,152.0/1,080.1 $=5.7 \mathrm{x}$

2017: Debt: $313.7+6,420.9=6,734.6$

Debt/EBITDA: 6,734.6/1,228. $4=5.5 x$

Debt/Capital (\%)

Capital $=$ Debt + Equity

2015: Capital: $6,496.3+5,311.2=11,807.5$

Debt/Capital: $6,496.3 / 11,807.5=55.0 \%$

2016: Capital: $6,152.0+4,984.3=11,136.3$

Debt/Capital: $6,152.0 / 11,136.3=55.2 \%$

2017: Capital: 6,734.6 + 6,522.0 = 13,256.6

Debt/Capital: 6,734.6/13,256.6 = 50.8\%

FCF after dividends/Debt (\%)

$2015: 782.5 / 6,496.3=12.0 \%$

$2016: 869.7 / 6,152=14.1 \%$

$2017: 541.2 / 6,734.6=8.0 \%$

Although debt/EBITDA and debt/capital improved between 2015 and 2017, the "FCF after dividends/Debt" deteriorated significantly as cash flow from operations declined as a result of the loss taken on disposal of discontinued operations. Given that the loss is most likely a non-recurring event, Mallinckrodt's creditworthiness likely improved over the 2015 to 2017 period.

\begin{enumerate}
  \setcounter{enumi}{2}
  \item Calculate Mallinckrodt's interest coverage using both EBIT and EBITDA. Comment on what these coverage ratios indicate about Mallinckrodt's creditworthiness.
\end{enumerate}

EBIT/Interest expense

2015: $353.8 / 255.6=1.4 \mathrm{x}$

2016: $248.4 / 378.1=0.7 x$

2017: $420.1 / 369.1=1.1 x$

EBITDA/Interest expense

2015: $1,026.3 / 255.6=4.0 x$

2016: $1,080.1 / 378.1=2.9 x$ $2017: 1,228.4 / 369.1=3.3 \mathrm{x}$

Based on these coverage ratios, Mallinckrodt's creditworthiness declined from 2015 to 2016 and then showed modest improvement in 2017. The 2017 coverage ratios are still weaker than the 2015 coverage ratios, indicating that growth in EBIT and EBITDA are not keeping pace with the rising interest expense.

\begin{enumerate}
  \setcounter{enumi}{3}
  \item Using the credit ratios provided in Exhibit 14 on Johnson \& Johnson, compare the creditworthiness of Mallinckrodt relative to Johnson \& Johnson. Johnson \& Johnson (J\&J) has a higher operating profit margin, better leverage ratios-lower debt/EBITDA, higher FCF after dividends/debt over the three years, lower debt/capital, and better interest coverage as measured by EBITDA/interest. Collectively, those ratios suggest J\&J has higher credit quality than Mallinckrodt.

  \item Compare the Exhibit 15 credit ratios of Luxembourg-based ArcelorMittal, one of the world's largest global steelmakers, with those of Mallinckrodt. Comment on the volatility of the credit ratios of the two companies. Which company looks to be more cyclical? What industry factors might explain some of the differences? In comparing the creditworthiness of these two companies, what other factors might be considered to offset greater volatility of credit ratios?

\end{enumerate}

Mallinckrodt has both a higher and a less volatile operating profit margin than ArcelorMittal (Arcelor). However, while Mallinckrodt's leverage ratios have been deteriorating, Arcelor's have been improving. Based on the volatility of its cash flow and operating profit margin, Arcelor appears to be a much more cyclical credit. However, with its meaningfully lower debt levels, one could expect Arcelor to have a higher credit rating.

A steelmaker likely has a significant amount of long-term assets financed by debt. It is a highly competitive industry with little ability to distinguish products from other competitors. To mitigate the impact of its more volatile credit ratios, Arcelor might maintain high levels of liquidity. Its size and global diversity may also be a "plus." Given its size, it may be able to negotiate favorable supplier and customer contracts and keep costs down through economies of scale.

\section{Collateral}
Collateral, or asset value, analysis is typically emphasized more with lower credit quality companies. As discussed earlier, credit analysts focus primarily on probability of default, which is mostly about an issuer's ability to generate sufficient cash flow to support its debt payments, as well as its ability to refinance maturing debt. Only when the default probability rises to a sufficient level do analysts typically consider asset or collateral value in the context of loss severity in the event of default.

Analysts do think about the value and quality of a company's assets; however, these are difficult to observe directly. Factors to consider include the nature and amount of intangible assets on the balance sheet. Some assets, such as patents, are clearly valuable and can be sold if necessary to cover liabilities. Goodwill, on the other hand, is not considered a high-quality asset. In fact, sustained weak financial performance most likely implies that a company's goodwill will be written down, reinforcing its poor quality. Another factor to consider is the amount of depreciation an issuer takes relative to its capital expenditures: Low capital expenditures relative to depreciation expense could imply that management is insufficiently investing in its business, which will lead to lower-quality assets, potentially reduced future operating cash flow, and higher loss severity in the event of default.

A market-based signal that credit analysts use to impute the quality of a publicly traded company's assets, and its ability to support its debt, is equity market capitalization. For instance, a company whose stock trades below book value may have lower-quality assets than is suggested by the amount reported on the balance sheet.

As economies become more service- and knowledge-based and those types of companies issue debt, it's important to understand that these issuers rely more on human and intellectual capital than on "hard assets." In generating profits and cash flow, these companies are not as asset intensive. One example would be software companies. Another example would be investment management firms. Human- and intellectual-capital-based companies may generate a lot of cash flow, but their collateral value is questionable unless there are patents and other types of intellectual property and "intangible capital" that may not appear directly on the balance sheet but could be valuable in the event of financial distress or default.

Regardless of the nature of the business, the key point of collateral analysis is to assess the value of the assets relative to the issuer's level-and seniority ranking-of debt.

\section{Covenants}
Covenants are meant to protect creditors while also giving management sufficient flexibility to operate its business on behalf of and for the benefit of the shareholders. They are integral to credit agreements, whether they are bonds or bank loans, and they spell out what the issuer's management is (1) obligated to do and (2) limited in doing. The former are called "affirmative covenants," whereas the latter are called "negative" or "restrictive covenants." Obligations would include such duties as making interest and principal payments and filing audited financial statements on a timely basis. Covenants might also require a company to redeem debt in the event of the company being acquired, the change of control covenant, or to keep the ratio of debt to EBITDA below some prescribed amount. The limitations might include a cap on the amount of cash that can be paid out to shareholders relative to earnings, or perhaps on the amount of additional secured debt that can be issued. Covenant violations are a breach of contract and can be considered default events unless they are cured in a short time or a waiver is granted.

For corporate bonds, covenants are described in the bond prospectus, the document that is part of a new bond issue. The prospectus describes the terms of the bond issue, as well as supporting financial statements, to help investors perform their analyses and make investment decisions as to whether or not to submit orders to buy the new bonds. Actually, the trust deed or bond indenture is the governing legal credit agreement and is typically incorporated by reference in the prospectus.

Covenants are an important but underappreciated part of credit analysis. Strong covenants protect bond investors from the possibility of management taking actions that would hurt an issuer's creditworthiness. For example, without appropriate covenants, management might pay large dividends, undertake stock buybacks well in excess of free cash flow, sell the company in a leveraged buyout, or take on a lot of secured debt that structurally subordinates unsecured bondholders. All of these actions would enrich shareholders at the expense of bondholders. Recall that management works for the shareholders and that bonds are contracts, with management's only real obligation to creditors being to uphold the terms of the contract. The inclusion of covenants in the contract is intended to protect bondholders.

The bond-buying investor base is very large and diverse, particularly for investment-grade debt. It includes institutional investors, such as insurance companies, investment management firms, pension funds, mutual funds, hedge funds, sovereign wealth funds, and so on. Although there are some very large institutional investors, the buyer base is fragmented and does not-and legally cannot-act as a syndicate. Thus, bondholders are generally not able to negotiate strong covenants on most new bond issues. At the same time, issuers expect that borrowing from capital markets will carry lighter convenants than borrowing from banks. During weak economic or market conditions, however, covenants on new bond issues tend to be stronger because investors require additional incentive to lend money while seeking more protection. A few organized institutional investor groups are focused on strengthening covenants: the Credit Roundtable (\href{http://thecreditroundtable.org}{thecreditroundtable.org}) in the United States and the European Model Covenant Initiative in the United Kingdom.

Covenant language is often very technical and written in "legalese," so it can be helpful to have an in-house person with a legal background to review and interpret the specific covenant terms and wording. One might also use a third-party service specializing in covenant analysis, such as Covenant Review (\href{http://www.covenantreview.com}{www.covenantreview.com}).

We will go into more detail on specific covenants in the section on special considerations for high-yield bonds.

\section{Character}
The character of a corporate borrower can be difficult to observe. The analysis of character as a factor in credit analysis dates to when loans were made to companies owned by individuals. Most corporate bond issuers are now publicly owned by shareholders or privately owned by pools of capital, such as private equity firms. Management often has little ownership in a corporation, so analysis and assessment of character is different than it would be for owner-managed firms. Credit analysts can make judgments about management's character in the following ways:

\begin{itemize}
  \item An assessment of the soundness of management's strategy.

  \item Management's track record in executing past strategies, particularly if they led to bankruptcy or restructuring. A company run by executives whose prior positions/ventures resulted in significant distress might still be able to borrow in the debt markets, but it would likely have to borrow on a secured basis and/or pay a higher interest rate.

  \item Use of aggressive accounting policies and/or tax strategies. Examples might include using a significant amount of off-balance-sheet financing, capitalizing versus immediately expensing items, recognizing revenue prematurely, and/or frequently changing auditors. These are potential warning flags to other behaviors or actions that may adversely impact an issuer's creditworthiness.

  \item Any history of fraud or malfeasance-a major warning flag to credit analysts.

  \item Previous poor treatment of bondholders-for example, management actions that resulted in major credit rating downgrades. These actions might include a debt-financed acquisition, a large special dividend to shareholders, or a major debt-financed stock buyback program.

\end{itemize}

\section{EXAMPLE 7}
\section{The Four Cs}
\begin{enumerate}
  \item Which of the following would not be a bond covenant?
\end{enumerate}

A. The issuer must file financial statements with the bond trustee on a timely basis. B. The company can buy back as much stock as it likes.

C. If the company offers security to any creditors, it must offer security to this bond issue.

\section{Solution:}
B is correct. Covenants describe what the borrower is (1) obligated to do or (2) limited in doing. It's the absence of covenants that would permit a company to buy back as much stock as it likes. A requirement that the company offer security to this bond issue if it offers security to other creditors (answer C) is referred to as a "negative pledge."

\begin{enumerate}
  \setcounter{enumi}{1}
  \item Why should credit analysts be concerned if a company's stock trades below book value?
\end{enumerate}

A. It means the company is probably going bankrupt.

B. It means the company will probably incur lots of debt to buy back its undervalued stock.

C. It's a signal that the company's asset value on its balance sheet may be impaired and have to be written down, suggesting less collateral protection for creditors.

\section{Solution:}
$\mathrm{C}$ is correct.

\begin{enumerate}
  \setcounter{enumi}{2}
  \item If management is of questionable character, how can investors incorporate this assessment into their credit analysis and investment decisions?
\end{enumerate}

A. They can choose not to invest based on the increased credit risk.

B. They can insist on getting collateral (security) and/or demand a higher return.

C. They can choose not to invest or insist on additional security and/or higher return.

\section{Solution:}
$\mathrm{C}$ is correct. Investors can always say no if they are not comfortable with the credit risk presented by a bond or issuer. They may also decide to lend to a borrower with questionable character only on a secured basis and/or demand a higher return for the perceived higher risk.

\section{CREDIT RISK VS. RETURN: YIELDS AND SPREADS}
describe macroeconomic, market, and issuer-specific factors that influence the level and volatility of yield spreads

The material in this section applies to all bonds subject to credit risk. For simplicity, in what follows all such bonds are sometimes referred to as "corporate" bonds.

\section{Fundamentals of Credit Analysis}
As in other types of investing, taking more risk in credit offers higher potential return but with more volatility and less certainty of earning that return. Using credit ratings as a proxy for risk, Exhibit 16 shows the composite yield-to-maturity for bonds of all maturities within each rating category in the US and European bond markets according to Bloomberg Barclays, one of the largest providers of fixed-income market indexes. For non-investment-grade bonds, the yield-to-call (YTC) or yield-to-worst (YTW) is reported as many contain optionality.

\section{Exhibit 16: Corporate Yields by Rating Category (\%)}
\begin{center}
\begin{tabular}{|c|c|c|c|c|c|c|c|c|}
\hline
\multirow{2}{*}{$\begin{array}{l}\text { Bloomberg Barclays } \\
\text { Indices }\end{array}$} & \multicolumn{4}{|c|}{Investment Grade} & \multicolumn{4}{|c|}{Non-Investment Grade} \\
\hline
 & AAA & AA & $\mathbf{A}$ & BBB & BB & B & CCC & CC-D \\
\hline
US & 3.63 & 3.52 & 3.86 & 4.35 & 5.14 & 6.23 & 8.87 & 19.51 \\
\hline
Pan European* & 1.25 & 0.76 & 1.18 & 1.67 & 2.92 & 5.63 & 8.78 & 54.95 \\
\hline
\end{tabular}
\end{center}

"Pan European yields may be "artificially low" due to the ECB's extraordinary corporate bond Quantitative Easing (QE) Program.

Note: Data as of 30 September 2018.

Source: Bloomberg Barclays Indices.

Note that the lower the credit quality, the higher the quoted yield. The realized yield, or return, will almost always be different because of changes in interest rates, re-investment of coupons, holding period changes, and the credit-related risks discussed earlier. In addition to the absolute returns, the volatility of returns will also vary by rating. Trailing 12 -month returns by credit rating category and the volatility (standard deviation) of those returns are shown in Exhibit 17. Exhibit 17: US Credit Trailing 12-Month Returns by Rating Category, 31

December 1996-30 September 2018

\begin{center}
\includegraphics[max width=\textwidth]{2023_05_04_36535b8d80b32081d422g-117}
\end{center}

Sources: Bloomberg Barclays Indices and Loomis, Sayles \& Company.

As shown in the exhibit, the higher the credit risk, the greater the return potential and the higher the volatility of that return. This pattern is consistent with other types of investing that involve risk and return (although average returns on single-B rated bonds appear anomalous in this example).

For extremely liquid bonds that are deemed to have virtually no default risk (e.g., German government bonds, or Bunds), the yield for a given maturity is a function of real interest rates plus an expected inflation rate premium. The yield on corporate bonds will include an additional risk premium that provides the investor with compensation for the credit and liquidity risks and possibly the tax impact of holding a specific bond. Changes in any of these components will alter the yield, price, and return on the bond. In general, however, it is not possible to directly observe the market's assessment of the components separately-analysts can only observe the total yield spread.

Spreads on all corporate bonds can be affected by a number of factors, with lower-quality issuers typically experiencing greater spread volatility. These factors, which are frequently linked, include the following:

\begin{itemize}
  \item Credit cycle. As the credit cycle improves, credit spreads will narrow. Conversely, a deteriorating credit cycle will cause credit spreads to widen. Spreads are tightest at or near the top of the credit cycle, when financial markets believe risk is low; they are widest at or near the bottom of the credit cycle, when financial markets believe risk is high.
\end{itemize}

\section{Fundamentals of Credit Analysis}
\begin{itemize}
  \item Broader economic conditions. Not surprisingly, weakening economic conditions will push investors to desire a greater risk premium and drive overall credit spreads wider. Conversely, a strengthening economy will cause credit spreads to narrow because investors anticipate credit measures will improve due to rising corporate cash flow, thus reducing the risk of default. In a steady, low-volatility environment, credit spreads will typically also narrow as investors tend to "reach for yield."

  \item Funding availability in financial sector. Bonds trade primarily over the counter, so investors need broker-dealers to commit capital for market-making purposes. Episodes of financial and regulatory stress have the potential to greatly reduce the total capital available for making markets (to facilitate trading) and the willingness to buy/sell credit-risky bonds. Future regulatory reform may well lead to persistent or even permanent reductions in broker-provided capital. Funding stresses would naturally translate into wider spreads.

  \item General market supply and demand. In periods of heavy new issue supply, credit spreads will widen if there is insufficient demand. In periods of high demand for bonds, spreads will move tighter.

  \item Financial performance of the issuer. Corporate bond spreads will be impacted by earnings releases, news, and other developments associated with the issuer. Earlier we explained the "four Cs of credit analysis": capacity, collateral, covenants, and character. Announcement and disclosures by the issuer will impact investors' view of the issuer's financial performance and ability to service and repay its debt. Good news increases the attractiveness of buying and holding bonds issued by the corporation, which raises bond prices and narrows spreads; bad news will have the opposite effect.

\end{itemize}

A number of these factors played a role during the global financial crisis of 2008-2009, causing spreads to widen dramatically, as shown in Exhibit 18, before narrowing sharply as governments intervened and markets stabilized. This is shown in two panels-one for investment grade, another for high yield-because of the much greater spread volatility in high-yield bonds, particularly CCC rated credits. This spread volatility is reflected in the different spread ranges on the $y$-axes. OAS is option-adjusted spread, which incorporates the value of the embedded call option in certain corporate bonds that issuers have the right to exercise before maturity.

\section{Exhibit 18: US Investment-Grade and High-Yield Corporate Spreads}
A. Investment-Grade Corporate Spreads

\begin{center}
\includegraphics[max width=\textwidth]{2023_05_04_36535b8d80b32081d422g-119(1)}
\end{center}

B. High-Yield Corporate Spreads

OAS (bps)

\begin{center}
\includegraphics[max width=\textwidth]{2023_05_04_36535b8d80b32081d422g-119}
\end{center}

Sources: Bloomberg Barclays Indices and Loomis, Sayles \& Company.

\section{EXAMPLE 8}
\section{Yield Spreads}
\begin{enumerate}
  \item Which bonds are likely to exhibit the greatest spread volatility?
A. Bonds from issuers rated AA
B. Bonds from issuers rated $B B$
C. Bonds from issuers rated A
\end{enumerate}

\section{Solution:}
B is correct. Lower-quality bonds exhibit greater spread volatility than higher-quality bonds. All of the factors that affect spreads-the credit cycle, economic conditions, financial performance, market-making capacity, and supply/demand conditions-will tend to have a greater impact on the pricing of lower-quality credits.

\begin{enumerate}
  \setcounter{enumi}{1}
  \item If investors become increasingly worried about the economy-say, as shown by declining stock prices-what is the most likely impact on credit spreads?
\end{enumerate}

A. There will be no change to credit spreads. They aren't affected by equity markets.

B. Narrower spreads will occur. Investors will move out of equities into debt securities.

C. Wider spreads will occur. Investors are concerned about weaker creditworthiness.

Solution:

$\mathrm{C}$ is correct. Investors will require higher yields as compensation for the credit deterioration-including losses - that is likely to occur in a weakening economy.

\section{Credit Risk vs. Return: The Price Impact of Spread Changes}
We have discussed how yield spreads on credit-risky debt obligations, such as corporate bonds, can fluctuate based on a number of factors, including changes in the market's view of issuer-specific or idiosyncratic risk. The next question to consider is how these spread changes affect the price of and return on these bonds.

Although bond investors do concern themselves with default risks, recall that the probability of default for higher-quality bonds is typically very low: For investment-grade bonds, annual defaults are nearly always well below 1\% (recall Exhibit 6). On the other hand, default rates can be very high for lower-quality issuers, although they can vary widely depending upon the credit cycle, among other things. What most investors in investment-grade debt focus on more than default risk is spread risk-that is, the effect on prices and returns from changes in spreads.

The price impact from spread changes is driven by two main factors: (1) the modified duration (price sensitivity with respect to changes in interest rates) of the bond and (2) the magnitude of the spread change. The effect on return to the bondholder depends on the holding period used for calculating the return.

The simplest example is that of a small, instantaneous change in the yield spread. In this case, the price impact (i.e., the percentage change in price, including accrued interest) can be approximated by

$$
\% \Delta \mathrm{PV}^{\text {Full }} \approx-\text { AnnModDur } \times \Delta \text { Spread }
$$

where AnnModDur is the annual modified duration. The negative sign in this equation reflects the fact that because bond prices and yields move in opposite directions, lower spreads have a positive impact on bond prices and thus returns, whereas higher spreads have a negative impact on bond returns. Note that if the spread change is expressed in basis points, then the price impact will also be in basis points, whereas if the spread change is expressed as a decimal, the price impact will also be expressed as a decimal. For larger spread changes (and thus larger yield changes), the impact of convexity needs to be incorporated into the approximation:

$\% \Delta \mathrm{PV}^{\text {Full }} \approx-($ AnnModDur $\times \Delta$ Spread $)+1 / 2$ AnnConvexity $\times(\Delta \text { Spread })^{2}$.

In this case, one must be careful to ensure that convexity (denoted by AnnConvexity) is appropriately scaled to be consistent with the way the spread change is expressed. In general, for bonds without embedded options, one can scale convexity so that it has the same order of magnitude as the duration squared and then express the spread change as a decimal. For example, for a bond with duration of 5.0 and reported convexity of 0.235 , one would re-scale convexity to 23.5 before applying the formula. For a $1 \%$ (i.e., $100 \mathrm{bps}$ ) increase in spread, the result would be

$$
\% \Delta \mathrm{PV}^{\text {Full }}=(-5.0 \times 0.01)+1 / 2 \times 23.5 \times(0.01)^{2}=-0.048825 \text { or }-4.8825 \% .
$$

The price impact of instantaneous spread changes is illustrated in Exhibit 19 using two bonds from British Telecom, the UK telecommunications company. The bonds, denominated in British pounds, are priced to provide a certain spread over British government bonds (gilts) of a similar maturity. From the starting spread, in increments of 25 bps and for both wider and narrower spreads, the new price and actual return for each spread change are calculated. In addition, the exhibit shows the approximate returns with and without the convexity term. As can be seen, the approximation using only duration is reasonably accurate for small spread changes; for larger changes, however, the convexity term generally provides a meaningful improvement.

\section{Exhibit 19: Impact of Duration on Price for a Given Change in Spread}
Issuer: British Telecom, 5.75\%, 12/07/2028

Price: $\pounds 122.978 \quad$ Modified Duration: $7.838 \quad$ Spread to Gilt Curve: 150.7 b.p.

Accrued interest: $0.958 \quad$ Convexity: $77.2 \quad$ YTM (conv): 3.16

\begin{center}
\begin{tabular}{|c|c|c|c|c|c|c|c|c|c|}
\hline
\multirow[b]{2}{*}{Spread $\Delta$ (b.p.)} & \multicolumn{9}{|c|}{Scenarios} \\
\hline
 & -100 & -75 & -50 & -25 & 0 & 25 & 50 & 75 & 100 \\
\hline
Spread (b.p.) & 50.7 & 75.7 & 100.7 & 125.7 & 150.7 & 175.7 & 200.7 & 225.7 & 250.7 \\
\hline
$\operatorname{Price}(\pounds)$ & 131.62 & 129.12 & 126.68 & 124.29 & 122.98 & 119.69 & 117.47 & 115.30 & 113.18 \\
\hline
Price + Accrued $(\pounds)$ & 132.58 & 130.08 & 127.64 & 125.25 & 123.94 & 120.65 & 118.43 & 116.26 & 114.14 \\
\hline
$\operatorname{Price} \Delta(\pounds)$ & 8.64 & 6.14 & 3.70 & 1.31 & 0.00 & -3.29 & -5.51 & -7.68 & -9.80 \\
\hline
\multicolumn{10}{|l|}{Return (\%)} \\
\hline
Actual & $6.97 \%$ & $4.96 \%$ & $2.99 \%$ & $1.06 \%$ & $0.00 \%$ & $-2.65 \%$ & $-4.44 \%$ & $-6.20 \%$ & $-7.91 \%$ \\
\hline
Approx: Dur only & $7.84 \%$ & $5.88 \%$ & $3.92 \%$ & $1.96 \%$ & $0.00 \%$ & $-1.96 \%$ & $-3.92 \%$ & $-5.88 \%$ & $-7.84 \%$ \\
\hline
Approx: Dur \& Cvx & $8.22 \%$ & $6.10 \%$ & $4.02 \%$ & $1.98 \%$ & $0.00 \%$ & $-1.94 \%$ & $-3.82 \%$ & $-5.66 \%$ & $-7.45 \%$ \\
\hline
\end{tabular}
\end{center}

Issuer: British Telecom, 3.625\%, 21/11/2047

Price: $\pounds 94.244 \quad$ Modified Duration: $17.144 \quad$ Spread to Gilt Curve: 210.8 b.p.

Accrued interest: 2.185

Convexity: 408.4

YTM (conv): 4.11

\begin{center}
\begin{tabular}{|c|c|c|c|c|c|c|c|c|c|}
\hline
\multirow[b]{2}{*}{Spread $\Delta$ (b.p.)} & \multicolumn{9}{|c|}{Scenarios} \\
\hline
 & -100 & -75 & -50 & -25 & 0 & 25 & 50 & 75 & 100 \\
\hline
Spread (b.p.) & 110.8 & 135.8 & 160.8 & 185.8 & 210.8 & 235.8 & 260.8 & 285.8 & 310.8 \\
\hline
Price $(\pounds)$ & 111.28 & 106.38 & 101.77 & 97.41 & 93.24 & 89.41 & 85.75 & 82.30 & 79.04 \\
\hline
Price $+\operatorname{Accrued~}(\mathfrak{E})$ & 113.47 & 108.57 & 103.96 & 99.60 & 95.43 & 91.60 & 87.94 & 84.48 & 81.22 \\
\hline
\end{tabular}
\end{center}

\begin{center}
\begin{tabular}{|c|c|c|c|c|c|c|c|c|c|}
\hline
\multirow[b]{2}{*}{$\operatorname{Price} \Delta(\pounds)$} & \multicolumn{9}{|c|}{Scenarios} \\
\hline
 & 18.04 & 13.14 & 8.53 & 4.17 & 0.00 & -3.83 & -7.49 & -10.95 & -14.21 \\
\hline
\multicolumn{10}{|l|}{Return (\%)} \\
\hline
Actual & $18.90 \%$ & $13.77 \%$ & $8.93 \%$ & $4.37 \%$ & $0.00 \%$ & $-4.02 \%$ & $-7.85 \%$ & $-11.47 \%$ & $-14.89 \%$ \\
\hline
Approx: Dur only & $17.14 \%$ & $12.86 \%$ & $8.57 \%$ & $4.29 \%$ & $0.00 \%$ & $-4.29 \%$ & $-8.57 \%$ & $-12.86 \%$ & $-17.14 \%$ \\
\hline
Approx: Dur \& Cvx & $19.19 \%$ & $14.01 \%$ & $9.08 \%$ & $4.41 \%$ & $0.00 \%$ & $-4.16 \%$ & $-8.06 \%$ & $-11.71 \%$ & $-15.10 \%$ \\
\hline
\end{tabular}
\end{center}

Note: Settle date is 13 December 2018.

Source: Bloomberg Finance, L.P.

Note that the price change for a given spread change is higher for the longer-duration bond-in this case, the 2047 maturity British Telecom bond-than for the shorter-duration, 2028 maturity British Telecom bond. Longer-duration corporate bonds are referred to as having "higher spread sensitivity"; that is, their prices, and thus returns, are more volatile with respect to changes in spread. It is essentially the same concept as duration for any bond: The longer the duration of a bond, the greater the price volatility for a given change in interest rates/yields.

In addition, investors want to be compensated for the fact that the further time one is from a bond's maturity (i.e., the longer the bond), the greater the uncertainty about an issuer's future creditworthiness. Based on credit analysis, an investor might be confident that an issuer's risk of default is relatively low in the near term; however, looking many years into the future, the investor's uncertainty grows because of factors that are increasingly difficult, if not impossible, to forecast (e.g., poor management strategy or execution, technological obsolescence, natural or man-made disasters, corporate leveraging events). This increase in credit risk over time can be seen in Exhibit 20. Note that in this Standard \& Poor's study, one-year default rates for the 2017 issuance pool are 0\% for all rating categories of BB or higher. The three-year default rates for bonds issued in 2015 are materially higher, and the observed defaults include bonds originally rated up to BBB (i.e., low investment grade). The 5-year default rates for bonds issued in 2013 are higher than the 3-year default rates, and the defaults also include bonds initially rated as high as BBB. In addition to the risk of default rising over time, the data also show quite conclusively that the lower the credit rating, the higher the risk of default. Finally, note the very high risk of default for bonds rated CCC or lower over all time horizons. This is consistent with Exhibit 7 presented earlier, which showed significant three-year ratings variability ("migration"), with much of the migration to lower credit ratings (i.e., higher risk of default).

Exhibit 20: Default Rate by Rating Category (\%)

(Non-financials)

\begin{center}
\begin{tabular}{lccc}
\hline
Credit Rating & $\begin{array}{c}\text { 1 Year } \\ \text { (2017 pool) }\end{array}$ & $\begin{array}{c}\mathbf{3} \text { Year } \\ \text { (2015 pool) }\end{array}$ & $\begin{array}{c}\mathbf{1 0} \text { Year } \\ \text { (2013 pool) }\end{array}$ \\
\hline
AAA & 0.00 & 0.00 & 0.00 \\
AA & 0.00 & 0.00 & 0.00 \\
A & 0.00 & 0.00 & 0.00 \\
BBB & 0.00 & 0.08 & 0.27 \\
BB & 0.10 & 2.46 & 3.33 \\
B & 0.95 & 10.11 & 12.90 \\
\end{tabular}
\end{center}

\begin{center}
\begin{tabular}{cccc}
\hline
Credit Rating & $\begin{array}{c}\mathbf{1} \text { Year } \\ \text { (2017 pool) }\end{array}$ & $\begin{array}{c}\mathbf{3} \text { Year } \\ \mathbf{( 2 0 1 5} \text { pool) }\end{array}$ & $\begin{array}{c}\mathbf{1 0} \text { Year } \\ \mathbf{( 2 0 1 3} \text { pool) }\end{array}$ \\
\hline
$\mathrm{CCC} / \mathrm{C}$ & 27.15 & 41.43 & 44.70 \\
\hline
\end{tabular}
\end{center}

Source: Based on data from S\&P Global Ratings, "2017 Annual Global Corporate Default Study and Rating Transitions" (5 April 2018).

\section{EXAMPLE 9}
\section{Price Impact}
\begin{enumerate}
  \item Calculate the price impact on a 10 -year corporate bond with a $4.75 \%$ coupon priced at 100 , with an instantaneous $50 \mathrm{bps}$ widening in spread due to the issuer's announcement that it was adding substantial debt to finance an acquisition (which resulted in a two-notch downgrade by the rating agencies). The bond has a modified duration of 7.9, and its convexity is 74.9.
\end{enumerate}

\section{Solution:}
The impact from the $50 \mathrm{bps}$ spread widening is:

Price impact $\approx-($ AnnModDur $\times \Delta$ Spread $)+1 / 2$ AnnConvexity $\times(\Delta \text { Spread })^{2}$

$$
=-(0.0050 \times 7.9)+(0.5 \times 74.9) \times(0.0050)^{2}
$$

$=-0.0386$, or $-3.86 \%$.

Because yields and bond prices move in opposite directions, the wider spread caused the bond price to fall. Using a bond-pricing calculator, the exact return is $-3.85 \%$, so this approximation was very accurate.

In summary, spread changes can have a significant impact on the price and performance of credit-risky bonds over a given holding period, and the higher the modified duration of the bond(s), the greater the price impact from changes in spread. Wider spreads hurt bond performance, whereas narrower spreads help bond performance. For bond investors who actively manage their portfolios (i.e., don't just buy bonds and hold them to maturity), forecasting spread changes and expected credit losses on both individual bonds and their broader portfolios is an important strategy for enhancing investment performance.

\section{HIGH-YIELD, SOVEREIGN, AND NON-SOVEREIGN CREDIT ANALYSIS}
explain special considerations when evaluating the credit of high-yield, sovereign, and non-sovereign government debt issuers and issues

Thus far, we have focused primarily on basic principles of credit analysis and investing with emphasis on higher-quality, investment-grade corporate bonds. Although many of these principles are applicable to other credit-risky segments of the bond market, some differences in credit analysis need to be considered. This section focuses on special considerations in evaluating the credit of debt issuers from the following three market segments: high-yield corporate bonds, sovereign bonds, and non-sovereign government bonds.

\section{High Yield}
Recall that high-yield, or non-investment-grade, corporate bonds are those rated below Baa3/BBB- by the major rating agencies. These bonds are sometimes referred to as "junk bonds" because of the higher risk inherent in their weak balance sheets and/or poor or less-proven business prospects.

Companies are rated below investment grade for many reasons, including the following:

\begin{itemize}
  \item Highly leveraged capital structure

  \item Weak or limited operating history

  \item Limited or negative free cash flow

  \item Highly cyclical business

  \item Poor management

  \item Risky financial policies

  \item Lack of scale and/or competitive advantages

  \item Large off-balance-sheet liabilities

  \item Declining industry (e.g., newspaper publishing)

\end{itemize}

Companies with weak balance sheets and/or business profiles have lower margin for error and greater risk of default relative to higher-quality investment-grade names. And the higher risk of default means more attention must be paid to recovery analysis (or loss severity, in the event of default). Consequently, high-yield analysis typically is more in-depth than investment-grade analysis and thus has special considerations. This includes the following:

\begin{itemize}
  \item Greater focus on issuer liquidity and cash flow

  \item Detailed financial projections

  \item Detailed understanding and analysis of the debt structure

  \item Understanding of an issuer's corporate structure

  \item Covenants

  \item Equity-like approach to high-yield analysis

\end{itemize}

\section{Liquidity}
Liquidity-that is, having cash and/or the ability to generate or raise cash-is important to all issuers. It is absolutely critical for high-yield companies. Investment-grade companies typically have substantial cash on their balance sheets, generate a lot of cash from operations relative to their debt (or else they wouldn't be investment grade!), and/or are presumed to have alternate sources of liquidity, such as bank lines and commercial paper. For these reasons, investment-grade companies can more easily roll over (refinance) maturing debt. On the other hand, high-yield companies may not have those options available. For example, there is no high-yield commercial paper market, and bank credit facilities often carry tighter restrictions for high-yield companies. Both bad company-specific news and difficult financial market conditions can lead to high-yield companies being unable to access the debt markets. And although the vast majority of investment-grade corporate debt issuers have publicly traded equity and can thus use that equity as a financing option, many high-yield companies are privately held and thus don't have access to public equity markets.

Thus, issuer liquidity is a key focus in high-yield analysis. Sources of liquidity, from strongest to weakest, are the following:

\begin{enumerate}
  \item Cash on the balance sheet

  \item Working capital

  \item Operating cash flow

  \item Bank credit facilities

  \item Equity issuance

  \item Asset sales

\end{enumerate}

Cash on the balance sheet is easy to see and self-evident as a source for repaying debt, although a portion of it may be 'trapped' overseas for certain tax, business, accounting, or regulatory reasons and thus not easily accessible. As mentioned earlier, working capital can be a large source or use of liquidity, depending on its amount, its use in a company's cash-conversion cycle, and its role in a company's operations. Operating cash flow is a ready source of liquidity as sales turn to receivables, which turn to cash over a fairly short time period. Bank lines, or credit facilities, can be an important source of liquidity, though there may be some covenants relating to the use of the bank lines that are crucial to know and will be covered a little later. Equity issuance may not be a reliable source of liquidity because an issuer is private or because of poor market conditions if a company does have publicly traded equity. Asset sales are the least reliable source of liquidity because both the potential value and the actual time of closing can be highly uncertain.

The amount of these liquidity sources should be compared with the amount and timing of upcoming debt maturities. A large amount of debt coming due in the next 6-12 months alongside low sources of liquidity will be a warning flag for bond investors and could push an issuer into default because investors may choose not to buy new bonds intended to pay off the existing debt. Insufficient liquidity-that is, running out of cash or no longer having access to external financing to refinance or pay off existing debt-is the principal reason issuers default. Although liquidity is important for industrial companies, it is an absolute necessity for financial firms, as seen in the case of Lehman Brothers and other troubled firms during the global financial crisis of 2008. Financial institutions are highly levered and often highly dependent on funding longer-term assets with short-term liabilities.

\section{Financial Projections}
Because high-yield companies have less room for error, it's important to forecast, or project, future earnings and cash flow out several years, perhaps including several scenarios, to assess whether the issuer's credit profile is stable, improving, or declining and thus whether it needs other sources of liquidity or is at risk of default. Ongoing capital expenditures and working capital changes should be incorporated as well. Special emphasis should be given to realistic "stress" scenarios that could expose a borrower's vulnerabilities.

\section{Fundamentals of Credit Analysis}
\section{Debt Structure}
High-yield companies tend to have many layers of debt in their capital structures, with varying levels of seniority and, therefore, different potential recovery rates in the event of default. (Recall the historical table of default recovery rates based on seniority in Exhibit 2.) A high-yield issuer will often have at least some of the following types of obligations in its debt structure:

\begin{itemize}
  \item (Secured) Bank debt

  \item Second lien debt

  \item Senior unsecured debt

  \item Subordinated debt, which may include convertible bonds

  \item Preferred stock

\end{itemize}

The lower the ranking in the debt structure, the lower the credit rating and the lower the expected recovery in the event of default. In exchange for these associated higher risks, investors will normally demand higher yields.

As discussed earlier, a standard leverage calculation used by credit analysts is debt/ EBITDA and is quoted as a multiple (e.g., "5.2x levered"). For an issuer with several layers of debt with different expected recovery rates, high-yield analysts should calculate leverage at each level of the debt structure. Example 10 shows calculations of gross leverage, as measured by debt/EBITDA, at each level of the debt structure and net leverage for the entire debt structure. Gross leverage calculations do not adjust debt for cash on hand. Net leverage adjusts debt by subtracting cash from total debt.

\section{EXAMPLE 10}
\section{Debt Structure and Leverage}
\begin{enumerate}
  \item Hexion Inc. is a specialty chemical company. It has a complicated, high-yield debt structure, consisting of first lien debt (loans and bonds), secured bonds, second lien bonds, and senior unsecured debt, due to a series of mergers as well as a leveraged buyout in 2005. Exhibit 21 is a simplified depiction of the company's debt structure, as well as some key credit-related statistics.
\end{enumerate}

Exhibit 21: Hexion Inc. Debt and Leverage Structure as of Year-End 2017

\section{Financial Information (\$ millions)}
Cash

Total debt $\$ 3,668$

Net debt $\quad \$ 3,553$

Interest expense $\quad \$ 329$

EBITDA $\quad \$ 365$

Debt Structure (\$ millions)

First lien debt (loans and bonds) Debt Structure (\$ millions)

TOTAL DEBT

$\$ 3,669$

Source: Company filings, Loomis, Sayles \& Company

Using the information provided, address the following:

\begin{enumerate}
  \item Calculate gross leverage, as measured by debt/EBITDA, through each level of debt, including total debt.

  \item Calculate the net leverage, as measured by (Debt - Cash)/EBITDA, for the total debt structure.

  \item Why might Hexion have so much secured debt relative to unsecured debt (both senior and subordinated)? (Note: This question draws on concepts from earlier sections.)

\end{enumerate}

\section{Solutions to 1 and 2:}
\begin{center}
\begin{tabular}{ccc}
\hline
 & $\begin{array}{c}\text { Gross } \\ \text { Leverage } \\ \text { (Debt/EBITDA) }\end{array}$ & $\begin{array}{c}\text { Net Leverage } \\ \text { (Debt - Cash)/ } \\ \text { EBITDA }\end{array}$ \\
\hline
\end{tabular}
\end{center}

\section{Secured debt leverage}
(First lien + Secured debt)/EBITDA

$(2,607+225) / 365$

$7.8 \mathrm{x}$

\section{Second lien leverage}
(First lien + Secured debt + Second lien debt)/EBITDA

$(2,607+225+574) / 365$

$9.3 \mathrm{x}$

Total leverage (includes unsecured)

(Total debt/EBITDA)

$3,669 / 365$

$10.1 \mathrm{x}$

Net leverage (leverage net of cash through entire debt structure)

(Total debt - Cash)/EBITDA

$9.7 \mathrm{x}$

\section{Solution to 3:}
Hexion might have that much secured debt because (1) it was less expensive than issuing additional unsecured debt on which investors would have demanded a higher yield and/or (2) given the riskiness of the business (chemicals are a cyclical business), the high leverage of the business model, and the riskiness of the balance sheet (lots of debt from a leveraged buyout), investors would only be willing to lend the company money on a secured basis.

High-yield companies that have a lot of secured debt (typically bank debt) relative to unsecured debt are said to have a "top-heavy" capital structure. With this structure, there is less capacity to take on more bank debt in the event of financial stress. Along with the often more stringent covenants associated with bank debt and its generally shorter maturity compared with other types of debt, this means that these issuers are more susceptible to default, as well as to lower recovery for the various less secured creditors.

\section{Corporate Structure}
Many debt-issuing corporations, including high-yield companies, utilize a holding company structure with a parent and several operating subsidiaries. Knowing where an issuer's debt resides (parent versus subsidiaries) and how cash can move from subsidiary to parent ("upstream") and vice versa ("downstream") are critical to the analysis of high-yield issuers.

In a holding company structure, the parent owns stock in its subsidiaries. Typically, the parent doesn't generate much of its own earnings or cash flow but instead receives dividends from its subsidiaries. The subsidiaries' dividends are generally paid out of earnings after they satisfy all their other obligations, such as debt payments. To the extent that their earnings and cash flow are weak, subsidiaries may be limited in their ability to pay dividends to the parent. Moreover, subsidiaries that carry a lot of their own debt may have restrictions or limitations on how much cash they can provide to the parent via dividends or in another way, such as through an intercompany loan. These restrictions and limitations on cash moving between parent and subsidiaries can have a major impact on their respective abilities to meet their debt obligations. The parent's reliance on cash flow from its subsidiaries means the parent's debt is structurally subordinated to the subsidiaries' debt and thus will usually have a lower recovery rating in default.

For companies with very complex holding companies, there may also be one or more intermediate holding companies, each carrying their own debt, and in some cases, they may not own $100 \%$ of the subsidiaries' stock. Often times a default in one subsidiary may not trigger a cross-default. This structure is sometimes seen in high-yield companies that have been put together through many mergers and acquisitions or that were part of a leveraged buyout.

Exhibit 22 shows the capital structure of Infor, Inc. (Infor), a high-yield software and services company highlighted earlier as an example of the credit rating agency notching process. Infor's capital structure consists of a parent company that has debt-in this case, convertible senior notes-as well as a subsidiary with multiple layers of outstanding debt by seniority.

\section{Exhibit 22: Infor's Capital Structure}
\begin{center}
\includegraphics[max width=\textwidth]{2023_05_04_36535b8d80b32081d422g-129}
\end{center}

"Conversion from euro to US dollar.

Source: Company filing, Loomis, Sayles \& Company.

Thus, high-yield investors should analyze and understand an issuer's corporate structure, including the distribution of debt between the parent and its subsidiaries. Leverage ratios should be calculated at each of the debt-issuing entities, as well as on a consolidated basis.

Also important is that although the debt of an operating subsidiary may be "closer to" and better secured by particular assets of the subsidiary, the credit quality of a parent company might still be higher. The parent company could, while being less directly secured by any particular assets, still benefit from the diversity and availability of all the cash flows in the consolidated system. In short, credit quality is not simply an automatic analysis of debt provisions and liens.

\section{Covenant Analysis}
As discussed earlier, analysis of covenants is very important for all bonds. It is especially important for high-yield credits because of their reduced margin of safety. Key covenants for high-yield issuers may include the following:

\begin{itemize}
  \item Change of control put

  \item Restricted payments

  \item Limitations on liens and additional indebtedness

  \item Restricted versus unrestricted subsidiaries

\end{itemize}

Under the change of control put, in the event of an acquisition (a "change of control"), bondholders have the right to require the issuer to buy back their debt (a "put option"), often at par or at some small premium to par value. This covenant is intended to protect creditors from being exposed to a weaker, more indebted borrower as a result of acquisition. For investment-grade issuers, this covenant typically has a two-pronged test: acquisition of the borrower and a consequent downgrade to a high-yield rating.

The restricted payments covenant is meant to protect creditors by limiting how much cash can be paid out to shareholders over time. The restricted payments "basket" is typically sized relative to an issuer's cash flow and debt outstanding-or is being raised-and is an amount that can grow with retained earnings or cash flow, giving management more flexibility to make payouts.

The limitations on liens covenant is meant to put limits on how much secured debt an issuer can have. This covenant is important to unsecured creditors who are structurally subordinated to secured creditors; the higher the amount of debt that is layered ahead of them, the less they stand to recover in the event of default.

With regard to restricted versus unrestricted subsidiaries, issuers may classify certain of their subsidiaries as restricted and others as unrestricted as it pertains to offering guarantees for their holding company debt. These subsidiary guarantees can be very useful to holding company creditors because they put their debt on equal standing (pari passu) with debt at the subsidiaries instead of with structurally subordinated debt. Restricted subsidiaries should be thought of as those that are designated to help service parent-level debt, typically through guarantees. They tend to be an issuer's larger subsidiaries and have significant assets, such as plants and other facilities, and/or cash flow. There may be tax or legal (e.g., country of domicile) reasons why certain subsidiaries are restricted while others are not. Analysts should carefully read the definitions of restricted versus unrestricted subsidiaries in the indenture because sometimes the language is so loosely written that the company can reclassify subsidiaries from one type to another with a simple vote by a board of directors or trustees.

For high-yield investors, it is also important to know what covenants are in an issuer's bank credit agreements. These agreements are typically filed with the securities commission in the country where the loan document was drafted. Bank covenants can be more restrictive than bond covenants and may include so-called maintenance covenants, such as leverage tests, whereby the ratio of, say, debt/EBITDA may not exceed "x" times. In the event a covenant is breached, the bank is likely to block further loans under the agreement until the covenant is cured. If not cured, the bank may accelerate full payment of the facility, triggering a default.

\section{Equity-like approach to high-yield analysis}
High-yield bonds are sometimes thought of as a "hybrid" between higher-quality bonds, such as investment-grade corporate debt, and equity securities. Their more volatile price and spread movements are less influenced by interest rate changes than are higher-quality bonds, and they show greater correlation with movements in equity markets. Indeed, as shown in Exhibit 23, historical returns on high-yield bonds and the standard deviation of those returns fall somewhere between investment-grade bonds and equities. Exhibit 23: US Trailing 12-Month Returns by Asset Class, 31 December 1988-30 September 2018

\begin{center}
\includegraphics[max width=\textwidth]{2023_05_04_36535b8d80b32081d422g-131}
\end{center}

Sources: Bloomberg Barclays Indices; Haver Analytics; and Loomis, Sayles \& Company.

Consequently, an equity market-like approach to analyzing a high-yield issuer can be useful. One approach is to calculate an issuer's enterprise value. Enterprise value (EV) is usually calculated by adding equity market capitalization and total debt and then subtracting excess cash. Enterprise value is a measure of what a business is worth (before any takeover premium) because an acquirer of the company would have to either pay off or assume the debt and it would receive the acquired company's cash.

Bond investors like using EV because it shows the amount of equity "cushion" beneath the debt. It can also give a sense of (1) how much more leverage management might attempt to put on a company in an effort to increase equity returns or (2) how likely-and how expensive-a credit-damaging leveraged buyout might be. Similar to how stock investors look at equity multiples, bond investors may calculate and compare EV/EBITDA and debt/EBITDA across several issuers as part of their analysis. Narrow differences between EV/EBITDA and debt/EBITDA for a given issuer indicate a small equity cushion and, therefore, potentially higher risk for bond investors.

\section{Sovereign Debt}
Governments around the world issue debt to help finance their general operations, including such current expenses as wages for government employees, and investments in such long-term assets as infrastructure and education. Government bonds in developed countries have traditionally been viewed as the default risk-free rate off of which all other credits are priced. Fiscal challenges in developed countries exacerbated by the 2008 global financial crisis and the 2011-2012 eurozone crisis, however, have called into question the notion of a "risk-free rate," even for some of the highest-quality government borrowers. As their capital markets have developed, an increasing number of sovereign governments have been able to issue debt in foreign markets (generally denominated in a currency other than that of the sovereign government, often the US dollar or euro) as well as debt in the domestic market (issued in the sovereign government's own currency). Generally, sovereign governments with

\section{Fundamentals of Credit Analysis}
weak currencies can access foreign debt markets only by issuing bonds in foreign currencies that are viewed to be safer stores of value. Debt issued in the domestic market is somewhat easier to service because the debt is typically denominated in the country's own currency, subject to its own laws, and money can be printed to service the sovereign government's domestic debt. Twenty years ago, many emerging market countries could only issue debt in foreign markets because a domestic market did not exist. Today, many are able to issue debt domestically and have successfully built yield curves of domestic bonds across the maturity spectrum. All sovereign governments are best able to service foreign and domestic debt if they run "twin surpluses"-that is, a government budget surplus as well as a current account surplus.

Despite ongoing financial globalization and the development of domestic bond markets, sovereign government defaults occur. Defaults are often precipitated by such events as war, political upheaval, major currency devaluation, a sharp deterioration in trade, or dramatic price declines in a country's key commodity exports. Default risks for some developed countries escalated after 2009 as government revenues dropped precipitously following the global financial crisis of 2008, expenditures surged, and financial markets focused on the long-term sustainability of public finances, given aging populations and rising social security needs. Some of the weaker and more highly indebted members of the eurozone became unable to access the debt markets at economical rates and had to seek loans from the International Monetary Fund (IMF) and the European Union. These weaker governments had previously been able to borrow at much lower rates because of their membership in the European Union and adoption of the euro. Intra-eurozone yield spreads widened and countries were shut out of markets, however, as the global financial crisis exacted a high toll on their public finances and, in some cases, their banking systems, which became contingent liabilities for the sovereigns. In Ireland, the government guaranteed most bank liabilities, undermining the country's own fiscal metrics.

Like corporate analysis, sovereign credit analysis is based on a combination of qualitative and quantitative factors. Ultimately, the two key issues for sovereign analysis are: (1) a government's ability to pay and (2) its willingness to pay. Willingness to pay is important because of the principle of sovereign immunity, where investors are generally unable to force a sovereign to pay its debts. Sovereign immunity prevents governments from being sued. To date, global initiatives aimed at creating a mechanism for orderly sovereign restructurings and defaults have not found traction.

To illustrate the most important considerations in sovereign credit analysis, we present a basic framework for evaluating sovereign credit and assigning sovereign debt ratings (this outline was developed from the detailed exposition of Standard \& Poor's methodology given in S\&P Global Ratings, "Sovereign Rating Methodology" [18 December 2017]). The framework highlights the specific characteristics analysts should expect in a high-quality sovereign credit. Some of these are self-explanatory (e.g., absence of corruption). For others, a brief rationale and/or range of values is included to clarify interpretation. Most, but not all, of these items are included in rating agency Standard \& Poor's methodology.

\section{Institutional and economic profile}
\begin{itemize}
  \item Institutional assessment

  \item Institutions' ability to deliver sound public finances and balanced economic growth.

  \item Effectiveness and predictability of policymaking institutions.

  \item Track record of managing previous political, economic, and/or financial crises. - Ability and willingness to implement reforms to address fiscal challenges.

  \item Transparent and accountable institutions with low perceived level of corruption.

  \item Independence of statistical offices and media.

  \item Checks and balances between institutions.

  \item Unbiased enforcement of contracts and respect for rule of law and property rights.

  \item Debt repayment culture.

  \item Potential external and domestic security risks.

  \item Economic assessment

  \item Income per capita: More prosperous countries generally have a broader and deeper tax base with which to support debt.

  \item Trend growth prospects: Creditworthiness is supported by sustainable and durable trend growth across business cycles.

  \item Diversity and stability of growth: Sovereigns exposed to economic concentration are more vulnerable. A narrow economy tends to show higher volatility in growth and can impair a government's balance sheet.

\end{itemize}

\section{Flexibility and performance profile}
\begin{itemize}
  \item External assessment

  \item Status of currency: Sovereigns that control a reserve currency or a very actively traded currency are able to use their own currency in many international transactions and are less vulnerable to adverse shifts in global investor portfolios.

  \item External liquidity: Countries with a substantial supply of foreign currency (foreign exchange reserves plus current account receipts) relative to projected funding needs in foreign currency (current account payments plus debt maturities) are less vulnerable to interruption of external liquidity.

  \item External debt: Countries with low foreign debt relative to current account receipts are better able to service their foreign debt. This is similar to a coverage ratio for a corporation.

  \item Fiscal assessment

  \item Fiscal performance and flexibility: Trend change in net general government debt as a percentage of GDP. Stable or declining debt as a percentage of GDP indicates a strong credit; a rising ratio can prove unsustainable and is, therefore, a sign of diminishing creditworthiness.

  \item Long-term fiscal trends: Perceived willingness and ability to increase revenue or cut expenditure to ensure debt service.

  \item Debt burden and structure: Net general government debt of less than $30 \%$ is good; more than $100 \%$ is poor. General government interest expense as a percentage of revenue: Less than $5 \%$ is good; greater than $15 \%$ is poor.

  \item Ability to access funding, manage both the amortization profile and contingent liabilities arising from the financial sector, public enterprises, and guarantees.

\end{itemize}

\section{Fundamentals of Credit Analysis}
\begin{itemize}
  \item Monetary assessment

  \item Ability to use monetary policy tailored to domestic economic objectives (e.g., growth) to address imbalances or shocks.

  \item Exchange rate regime: Sovereigns with a reserve currency have the most flexibility. A freely floating currency allows maximum effectiveness for monetary policy. A fixed-rate regime limits effectiveness and flexibility. A hard peg, such as a currency board or monetary union, affords no independent monetary policy.

  \item Credibility of monetary policy: Measured by track record of low and stable inflation. Credible monetary policy is supported by an operationally and legally independent central bank with a clear mandate. The central bank's ability to be a lender of last resort to the financial system also enhances stability.

  \item Confidence in the central bank provides a foundation for confidence in the currency as a store of value and for the central bank to effectively manage policy through a crisis.

  \item Most effective policy transmission occurs in systems with sound banking systems and well-developed domestic capital markets, including active money market and corporate bond markets, such that policymakers credibly enact policy relying on market-based policy tools (e.g., open market operations) over administrative policy tools (e.g., reserve requirements).

\end{itemize}

In light of a sovereign government's various powers-taxation, regulation, monetary policy, and ultimately, the sovereign's ability to "print money" to repay debt-within its own economy, it is virtually always at least as good a credit in its domestic currency as it is in foreign currency. Thus, credit rating agencies often distinguish between domestic and foreign bonds, with domestic bond ratings sometimes one notch higher. Of course, if a sovereign government were to rely heavily on printing money to repay debt, it would fuel high inflation or hyperinflation and increase default risk on domestic debt as well.

\section{EXAMPLE 11}
\section{Sovereign Debt}
\begin{enumerate}
  \item Exhibit 24 shows several key sovereign statistics for Portugal.
\end{enumerate}

\section{Exhibit 24: Key Sovereign Statistics for Portugal}
\begin{center}
\begin{tabular}{lcccccccc}
\hline
$\boldsymbol{\epsilon}$ (billions), except where noted & $\mathbf{2 0 0 6}$ & $\mathbf{2 0 0 8}$ & $\mathbf{2 0 1 0}$ & $\mathbf{2 0 1 2}$ & $\mathbf{2 0 1 4}$ & $\mathbf{2 0 1 5}$ & $\mathbf{2 0 1 6}$ & $\mathbf{2 0 1 7}$ \\
\hline
Nominal GDP & 160.3 & 171.2 & 172.6 & 168.4 & 173.1 & 179.8 & 186.5 & 194.6 \\
Population (millions) & 10.6 & 10.6 & 10.6 & 10.5 & 10.4 & 10.3 & 10.3 & 10.3 \\
Unemployment (\%) & 8.6 & 8.5 & 12 & 15.6 & 13.9 & 12.4 & 11.1 & 8.9 \\
Exports as share GDP (\%) & 22.2 & 22.6 & 21.3 & 26.8 & 27.8 & 27.6 & 26.8 & 28.3 \\
Current account as share GDP (\%) & -10.7 & -12.6 & -10 & -2.1 & 0.2 & 0.3 & 0.7 & 0.6 \\
 &  &  &  &  &  &  &  &  \\
Government revenues & 64.8 & 70.7 & 71.8 & 72.2 & 77.2 & 78.8 & 79.9 & 83.1 \\
Government expenditures & 71.4 & 77.1 & 88.7 & 81.7 & 89.6 & 86.7 & 83.5 & 88.9 \\
Budget balance (surplus/deficit) & -6.5 & -6.4 & -16.9 & -9.5 & -12.4 & -7.9 & -3.6 & -5.8 \\
\end{tabular}
\end{center}

\begin{center}
\begin{tabular}{|c|c|c|c|c|c|c|c|c|}
\hline
$\epsilon$ (billions), except where noted & 2006 & 2008 & 2010 & 2012 & 2014 & 2015 & 2016 & 2017 \\
\hline
Government interest payments & 4.2 & 5.3 & 5.2 & 8.2 & 8.5 & 8.2 & 7.8 & 7.4 \\
\hline
Primary balance (surplus/deficit) & -2.2 & -1.1 & -11.7 & -1.3 & -3.9 & 0.3 & 4.2 & 1.6 \\
\hline
Government debt & 102.4 & 123.1 & 161.3 & 212.6 & 226 & 231.5 & 241 & 242.8 \\
\hline
Interest rate on new debt (\%) & 3.9 & 4.5 & 5.4 & 3.4 & 3.8 & 2.4 & 3.2 & 3.1 \\
\hline
\end{tabular}
\end{center}

Sources: Haver Analytics, Eurostat, and Instituto Nacional de Estatística (Portugal).

\begin{enumerate}
  \item Calculate the government debt/GDP for Portugal for the years 2014-2017 as well as for the years 2006, 2008, 2010, and 2012.

  \item Calculate GDP/capita for the same periods.

  \item Based on those calculations, as well as other data from Exhibit 24, what can you say about Portugal's credit trend?

\end{enumerate}

\section{Solutions to 1 and 2:}
\begin{center}
\begin{tabular}{lccccccccc}
\hline
 & $\mathbf{2 0 0 6}$ & $\mathbf{2 0 0 8}$ & $\mathbf{2 0 1 0}$ & $\mathbf{2 0 1 2}$ & $\mathbf{2 0 1 4}$ & $\mathbf{2 0 1 5}$ & $\mathbf{2 0 1 6}$ & $\mathbf{2 0 1 7}$ \\
\hline
Gross government debt/GDP & $64 \%$ & $72 \%$ & $93 \%$ & $126 \%$ & $131 \%$ & $129 \%$ & $129 \%$ & $125 \%$ \\
GDP/capita & 15,123 & 16,151 & 16,283 & 16,038 & 16,644 & 17,456 & 18,107 & 18,893 \\
\hline
\end{tabular}
\end{center}

\section{Solution to 3:}
The credit trend is stabilizing. Government debt/GDP is declining ever so slightly after peaking in 2014. The government is running a modest budget deficit with a primary balance that is in surplus for the past three years. Portugal is running a very small current account surplus, reducing its reliance on external funding, and has increased its exports as a share of GDP. Unemployment, while still fairly high, has fallen in the past several years. Interest payments on government debt have started to decline, both as a percentage of GDP and in absolute terms. The interest rate on new government debt has stabilized, perhaps benefitting from the European Central Bank's quantitative easing policies. Taken together, there are strong indications that the Portuguese government's credit situation has stabilized and may be expected to improve further if current trends are sustained.

\section{Non-Sovereign Government Debt}
Sovereigns are the largest issuers of government debt, but non-sovereign-sometimes called sub-sovereign or local-governments and the quasi-government entities that are created by governments issue bonds as well. The non-sovereign or local governments include governments of states, provinces, regions, and cities. For example, the City of Tokyo (Tokyo Metropolitan Government) has debt outstanding, as does the Lombardy region in Italy, the City of Buenos Aires in Argentina, and the State of California in the United States. Local government bonds may be referred to as municipal bonds.

However, when people talk about municipal bonds, they are usually referring to US municipal bonds, which represent one of the largest bond markets. As of year-end 2017, the US municipal bond market was approximately $\$ 3.9$ trillion in size, roughly 9\% of the total US bond market (Securities Industry and Financial Markets Association [SIFMA], "Outstanding U.S. Bond Market Data," as of 2Q 2018). The US municipal bond market is composed of both tax-exempt and, to a lesser extent, taxable bonds issued by state and city governments and their agencies. Municipal borrowers may also issue bonds on behalf of private entities, such as non-profit colleges or hospitals. Historically, for any given rating category, these bonds have much lower default rates than corporate bonds with the same ratings. For example, according to Moody's Investors Service (“US Municipal Bond Defaults and Recoveries, 1970-2017"), the 10-year average cumulative default rate from 1970 through 2017 was 0.17\% for municipal bonds, compared with a $10.24 \% 10$-year average cumulative default rate for all corporate debt.

The majority of local government bonds, including municipal bonds, are either general obligation bonds or revenue bonds. General obligation (GO) bonds are unsecured bonds issued with the full faith and credit of the issuing non-sovereign government. These bonds are supported by the taxing authority of the issuer. Revenue bonds are issued for specific project financing (e.g., financing for a new sewer system, a toll road, bridge, hospital, or sports arena).

The credit analysis of $\mathrm{GO}$ bonds has some similarities to sovereign debt analysis (e.g., the ability to levy and collect taxes and fees to help service debt) but also some differences. For example, almost without exception, US municipalities must balance their operating budgets (i.e., exclusive of long-term capital projects) annually. Non-sovereign governments are unable to use monetary policy the way many sovereigns can.

The economic analysis of non-sovereign government GO bonds, including US municipal bonds, focuses on employment, per capita income (and changes in it over time), per capita debt (and changes in it over time), the tax base (depth, breadth, diversification, stability, etc.), demographics, and net population growth, as well as an analysis of whether the area represented by the non-sovereign government has the infrastructure and location to attract and support new jobs. Analysis should look at the volatility and variability of revenues during times of both economic strength and weakness. An overreliance on one or two types of tax revenue-particularly a volatile one, such as capital gains taxes or sales taxes-can signal increased credit risk. Pensions and other post-retirement obligations may not show up directly on the non-sovereign government's balance sheet, and many of these entities have underfunded pensions that need to be addressed. Adding the unfunded pension and post-retirement obligations to the debt reveals a more realistic picture of the issuer's debt and longer-term obligations. The relative ease or difficulty in managing the annual budgeting process and the government's ability to operate consistently within its budget are also important credit analysis considerations.

Disclosure by non-sovereign governments varies widely, with some of the smaller issuers providing limited financial information. Reporting requirements are inconsistent, so the financial reports may not be available for six months or more after the closing of a reporting period.

Exhibit 25 compares several key debt statistics from two of the larger states in the United States: Illinois and Texas. Illinois has the lowest credit ratings of any of the states, whereas Texas has one of the highest. Note the higher debt burden (and lower ranking) across several measures: Total debt, Debt/Capita, Debt/Personal income, and debt as a percentage of state GDP. When including net pension liabilities of government employees and retirees, the debt burdens are even greater, especially in the case of Illinois. What is not shown here is that Illinois also has a higher tax burden and greater difficulty balancing its budget on an annual basis than Texas does.

\section{Exhibit 25: Municipal Debt Comparison: Illinois vs. Texas}
\begin{center}
\begin{tabular}{|c|c|c|}
\hline
 & Illinois & Texas \\
\hline
S\&P & $\mathrm{BBB}-$ & AAA \\
\hline
Fitch & $\mathrm{BBB}$ & AAA \\
\hline
Unemployment rate $(\%)^{*}$ & 4.20 & 3.70 \\
\hline
Median household income $(\$)^{* *}$ & $\$ 61,229$ & $\$ 57,051$ \\
\hline
\multicolumn{3}{|l|}{Debt burden, net (\$/rank) ${ }^{* * *}$} \\
\hline
Total (millions) & $37,374(5)$ & $11,603(13)$ \\
\hline
Per capita & $2,919(6)$ & $410(42)$ \\
\hline
As a percentage of 2016 personal income & $5.60(5)$ & $0.90(42)$ \\
\hline
As a percentage of 2016 GDP & $4.70(6)$ & $0.73(42)$ \\
\hline
\multicolumn{3}{|l|}{ANPL, net $(\$ / \text { rank })^{* * * *}$} \\
\hline
Total (millions) & $250,136(1)$ & $140,253(3)$ \\
\hline
Per capita & $19,539(1)$ & $4,955(19)$ \\
\hline
As a percentage of 2017 personal income & $37.00(1)$ & $10.60(19)$ \\
\hline
As a percentage of 2017 GDP & $30.50(1)$ & $8.30(20)$ \\
\hline
\end{tabular}
\end{center}

\begin{itemize}
  \item Source: Bureau of Labor Statistics, data as of October 2018.
\end{itemize}

** Source: US Census Bureau, data as of 2017.

*** Source: Moody's Investors Service, Inc., debt data as of 2017.

**** Source: Moody's Investors Service, Inc., adjusted net pension liability data as of 2017.

Revenue bonds, which are issued to finance a specific project, have a higher degree of risk than GO bonds because they are dependent on a single source of revenue. The analysis of these bonds is a combination of an analysis of the project and the finances around the particular project. The project analysis focuses on the need and projected utilization of the project, as well as on the economic base supporting the project. The financial analysis has some similarities to the analysis of a corporate bond in that it is focused on operating results, cash flow, liquidity, capital structure, and the ability to service and repay the debt. A key credit measure for revenue-backed non-sovereign government bonds is the debt-service-coverage (DSC) ratio, which measures how much revenue is available to cover debt payments (principal and interest) after operating expenses. Many revenue bonds have a minimum DSC ratio covenant; the higher the DSC ratio, the stronger the creditworthiness.

\section{SUMMARY}
We introduced basic principles of credit analysis. We described the importance of the credit markets and credit and credit-related risks. We discussed the role and importance of credit ratings and the methodology associated with assigning ratings, as well as the risks of relying on credit ratings. We covered the key components of credit analysis and the financial measure used to help assess creditworthiness.

\section{Fundamentals of Credit Analysis}
We also discussed risk versus return when investing in credit and how spread changes affect holding period returns. In addition, we addressed the special considerations to take into account when doing credit analysis of high-yield companies, sovereign borrowers, and non-sovereign government bonds.

\begin{itemize}
  \item Credit risk is the risk of loss resulting from the borrower failing to make full and timely payments of interest and/or principal.

  \item The key components of credit risk are risk of default and loss severity in the event of default. The product of the two is expected loss. Investors in higher-quality bonds tend not to focus on loss severity because default risk for those securities is low.

  \item Loss severity equals ( 1 - Recovery rate).

  \item Credit-related risks include downgrade risk (also called credit migration risk) and market liquidity risk. Either of these can cause yield spreads-yield premiums - to rise and bond prices to fall.

  \item Downgrade risk refers to a decline in an issuer's creditworthiness.

\end{itemize}

Downgrades will cause its bonds to trade with wider yield spreads and thus lower prices.

\begin{itemize}
  \item Market liquidity risk refers to a widening of the bid-ask spread on an issuer's bonds. Lower-quality bonds tend to have greater market liquidity risk than higher-quality bonds, and during times of market or financial stress, market liquidity risk rises.

  \item The composition of an issuer's debt and equity is referred to as its "capital structure." Debt ranks ahead of all types of equity with respect to priority of payment, and within the debt component of the capital structure, there can be varying levels of seniority.

  \item With respect to priority of claims, secured debt ranks ahead of unsecured debt, and within unsecured debt, senior debt ranks ahead of subordinated debt. In the typical case, all of an issuer's bonds have the same probability of default due to cross-default provisions in most indentures. Higher priority of claim implies higher recovery rate-lower loss severity-in the event of default.

  \item For issuers with more complex corporate structures-for example, a parent holding company that has operating subsidiaries-debt at the holding company is structurally subordinated to the subsidiary debt, although the possibility of more diverse assets and earnings streams from other sources could still result in the parent having higher effective credit quality than a particular subsidiary.

  \item Recovery rates can vary greatly by issuer and industry. They are influenced by the composition of an issuer's capital structure, where in the economic and credit cycle the default occurred, and what the market's view of the future prospects are for the issuer and its industry.

  \item The priority of claims in bankruptcy is not always absolute. It can be influenced by several factors, including some leeway accorded to bankruptcy judges, government involvement, or a desire on the part of the more senior creditors to settle with the more junior creditors and allow the issuer to emerge from bankruptcy as a going concern, rather than risking smaller and delayed recovery in the event of a liquidation of the borrower.

  \item Credit rating agencies, such as Moody's, Standard \& Poor's, and Fitch, play a central role in the credit markets. Nearly every bond issued in the broad debt markets carries credit ratings, which are opinions about a bond issue's creditworthiness. Credit ratings enable investors to compare the credit risk of debt issues and issuers within a given industry, across industries, and across geographic markets.

  \item Bonds rated Aaa to Baa3 by Moody's and AAA to BBB- by Standard \& Poor's (S\&P) and/or Fitch (higher to lower) are referred to as "investment grade." Bonds rated lower than that-Ba1 or lower by Moody's and $\mathrm{BB}+$ or lower by S\&P and/or Fitch-are referred to as "below investment grade" or "speculative grade." Below-investment-grade bonds are also called "high-yield" or "junk" bonds.

  \item The rating agencies rate both issuers and issues. Issuer ratings are meant to address an issuer's overall creditworthiness-its risk of default. Ratings for issues incorporate such factors as their rankings in the capital structure.

  \item The rating agencies will notch issue ratings up or down to account for such factors as capital structure ranking for secured or subordinated bonds, reflecting different recovery rates in the event of default. Ratings may also be notched due to structural subordination.

  \item Rating agencies incorporate ESG factors into their ratings of firms. Some have launched a set of ratings that aim to measure a company's attitudes, practices, and advances related to ESG. They identify and track leaders and laggards in the space. Companies are evaluated according to their exposure to ESG risks and how well they manage those risks relative to peers.

  \item There are risks in relying too much on credit agency ratings.

\end{itemize}

Creditworthiness may change over time, and initial/current ratings do not necessarily reflect the creditworthiness of an issuer or bond over an investor's holding period. Valuations often adjust before ratings change, and the notching process may not adequately reflect the price decline of a bond that is lower ranked in the capital structure. Because ratings primarily reflect the probability of default but not necessarily the severity of loss given default, bonds with the same rating may have significantly different expected losses (default probability times loss severity). And like analysts, credit rating agencies may have difficulty forecasting certain credit-negative outcomes, such as adverse litigation and leveraging corporate transactions, and such low probability/high severity events as earthquakes and hurricanes.

\begin{itemize}
  \item The role of corporate credit analysis is to assess the company's ability to make timely payments of interest and to repay principal at maturity.

  \item Credit analysis is similar to equity analysis. It is important to understand, however, that bonds are contracts and that management's duty to bondholders and other creditors is limited to the terms of the contract. In contrast, management's duty to shareholders is to act in their best interest by trying to maximize the value of the company-perhaps even at the expense of bondholders at times.

  \item Credit analysts tend to focus more on the downside risk given the asymmetry of risk/return, whereas equity analysts focus more on upside opportunity from earnings growth, and so on.

  \item The " $4 \mathrm{Cs}$ " of credit-capacity, collateral, covenants, and character-provide a useful framework for evaluating credit risk.

  \item Credit analysis focuses on an issuer's ability to generate cash flow. The analysis starts with an industry assessment-structure and fundamentals-and continues with an analysis of an issuer's competitive position, management strategy, and track record.

\end{itemize}

\section{Fundamentals of Credit Analysis}
\begin{itemize}
  \item Credit measures are used to calculate an issuer's creditworthiness, as well as to compare its credit quality with peer companies. Key credit ratios focus on leverage and interest coverage and use such measures as EBITDA, free cash flow, funds from operations, interest expense, and balance sheet debt.

  \item An issuer's ability to access liquidity is also an important consideration in credit analysis.

  \item The higher the credit risk, the greater the offered/required yield and potential return demanded by investors. Over time, bonds with more credit risk offer higher returns but with greater volatility of return than bonds with lower credit risk.

  \item The yield on a credit-risky bond comprises the yield on a default risk-free bond with a comparable maturity plus a yield premium, or "spread," that comprises a credit spread and a liquidity premium. That spread is intended to compensate investors for credit risk-risk of default and loss severity in the event of default-and the credit-related risks that can cause spreads to widen and prices to decline-downgrade or credit migration risk and market liquidity risk.

\end{itemize}

Yield spread $=$ Liquidity premium + Credit spread

\begin{itemize}
  \item In times of financial market stress, the liquidity premium can increase sharply, causing spreads to widen on all credit-risky bonds, with lower-quality issuers most affected. In times of credit improvement or stability, however, credit spreads can narrow sharply as well, providing attractive investment returns.

  \item The impact of spread changes on holding period returns for credit-risky bonds is a product of two primary factors: the basis point spread change and the sensitivity of price to yield as reflected by (end-of-period) modified duration and convexity. Spread narrowing enhances holding period returns, whereas spread widening has a negative impact on holding period returns. Longer-duration bonds have greater price and return sensitivity to changes in spread than shorter-duration bonds.

\end{itemize}

Price impact $\approx-($ AnnModDur $\times \Delta$ Spread $)+1 / 2$ AnnConvexity $\times(\Delta \text { Spread })^{2}$.

\begin{itemize}
  \item For high-yield bonds, with their greater risk of default, more emphasis should be placed on an issuer's sources of liquidity and its debt structure and corporate structure. Credit risk can vary greatly across an issuer's debt structure depending on the seniority ranking. Many high-yield companies have complex capital structures, resulting in different levels of credit risk depending on where the debt resides.

  \item Covenant analysis is especially important for high-yield bonds. Key covenants include payment restrictions, limitation on liens, change of control, coverage maintenance tests (often limited to bank loans), and any guarantees from restricted subsidiaries. Covenant language can be very technical and legalistic, so it may help to seek legal or expert assistance.

  \item An equity-like approach to high-yield analysis can be helpful. Calculating and comparing enterprise value with EBITDA and debt/EBITDA can show a level of equity "cushion" or support beneath an issuer's debt.

  \item Sovereign credit analysis includes assessing both an issuer's ability and willingness to pay its debt obligations. Willingness to pay is important because, due to sovereign immunity, a sovereign government cannot be forced to pay its debts. - In assessing sovereign credit risk, a helpful framework is to focus on five broad areas: (1) institutional effectiveness and political risks, (2) economic structure and growth prospects, (3) external liquidity and international investment position, (4) fiscal performance, flexibility, and debt burden, and (5) monetary flexibility.

  \item Among the characteristics of a high-quality sovereign credit are the absence of corruption and/or challenges to political framework; governmental checks and balances; respect for rule of law and property rights; commitment to honor debts; high per capita income with stable, broad-based growth prospects; control of a reserve or actively traded currency; currency flexibility; low foreign debt and foreign financing needs relative to receipts in foreign currencies; stable or declining ratio of debt to GDP; low debt service as a percentage of revenue; low ratio of net debt to GDP; operationally independent central bank; track record of low and stable inflation; and a well-developed banking system and active money market.

  \item Non-sovereign or local government bonds, including municipal bonds, are typically either general obligation bonds or revenue bonds.

  \item General obligation (GO) bonds are backed by the taxing authority of the issuing non-sovereign government. The credit analysis of GO bonds has some similarities to sovereign analysis-debt burden per capita versus income per capita, tax burden, demographics, and economic diversity. Underfunded and "off-balance-sheet" liabilities, such as pensions for public employees and retirees, are debt-like in nature.

  \item Revenue-backed bonds support specific projects, such as toll roads, bridges, airports, and other infrastructure. The creditworthiness comes from the revenues generated by usage fees and tolls levied.

\end{itemize}

\section{PRACTICE PROBLEMS}
\begin{enumerate}
  \item The risk that a bond's creditworthiness declines is best described by:
\end{enumerate}

A. credit migration risk.

B. market liquidity risk.

C. spread widening risk.

\begin{enumerate}
  \setcounter{enumi}{1}
  \item Stedsmart Ltd and Fignermo Ltd are alike with respect to financial and operating characteristics, except that Stedsmart Ltd has less publicly traded debt outstanding than Fignermo Ltd. Stedsmart Ltd is most likely to have:
A. no market liquidity risk.
B. lower market liquidity risk.
C. higher market liquidity risk.

  \item Credit risk of a corporate bond is best described as the:

\end{enumerate}

A. risk that an issuer's creditworthiness deteriorates.

B. probability that the issuer fails to make full and timely payments.

C. risk of loss resulting from the issuer failing to make full and timely payments.

\begin{enumerate}
  \setcounter{enumi}{3}
  \item The risk that the price at which investors can actually transact differs from the quoted price in the market is called:
A. spread risk.
B. credit migration risk.
C. market liquidity risk.

  \item Loss severity is best described as the:

\end{enumerate}

A. default probability multiplied by the loss given default.

B. portion of a bond's value recovered by bondholders in the event of default.

C. portion of a bond's value, including unpaid interest, an investor loses in the event of default.

\begin{enumerate}
  \setcounter{enumi}{5}
  \item The two components of credit risk are default probability and:
A. spread risk.
B. loss severity.
C. market liquidity risk.

  \item For a high-quality debt issuer with a large amount of publicly traded debt, bond investors tend to devote most effort to assessing the issuer's:

\end{enumerate}

A. default risk. B. loss severity.

C. market liquidity risk.

\begin{enumerate}
  \setcounter{enumi}{7}
  \item The expected loss for a given debt instrument is estimated as the product of default probability and:
A. $(1+$ Recovery rate)
B. (1 - Recovery rate).
C. $1 /(1+$ Recovery rate).

  \item In the event of default, the recovery rate of which of the following bonds would most likely be the highest?
A. First mortgage debt
B. Senior unsecured debt
C. Junior subordinate debt

  \item During bankruptcy proceedings of a firm, the priority of claims was not strictly adhered to. Which of the following is the least likely explanation for this outcome?

\end{enumerate}

A. Senior creditors compromised.

B. The value of secured assets was less than the amount of the claims.

C. A judge's order resulted in actual claims not adhering to strict priority of claims.

\begin{enumerate}
  \setcounter{enumi}{10}
  \item The priority of claims for senior subordinated debt is:
A. lower than for senior unsecured debt.
B. the same as for senior unsecured debt.
C. higher than for senior unsecured debt.

  \item A senior unsecured credit instrument holds a higher priority of claims than one ranked as:
A. mortgage debt.
B. second lien loan.
C. senior subordinated.

  \item In a bankruptcy proceeding, when the absolute priority of claims is enforced:

\end{enumerate}

A. senior subordinated creditors rank above second lien holders.

B. preferred equity shareholders rank above unsecured creditors.

C. creditors with a secured claim have the first right to the value of that specific property.

\begin{enumerate}
  \setcounter{enumi}{13}
  \item In the event of default, which of the following is most likely to have the highest recovery rate?
\end{enumerate}

A. Second lien

B. Senior unsecured

C. Senior subordinated

\begin{enumerate}
  \setcounter{enumi}{14}
  \item Which of the following corporate debt instruments has the highest seniority ranking?
\end{enumerate}

A. Second lien

B. Senior unsecured

C. Senior subordinated

\begin{enumerate}
  \setcounter{enumi}{15}
  \item A fixed-income analyst is least likely to conduct an independent analysis of credit risk because credit rating agencies:
\end{enumerate}

A. may at times mis-rate issues.

B. often lag the market in pricing credit risk.

C. cannot foresee future debt-financed acquisitions.

\begin{enumerate}
  \setcounter{enumi}{16}
  \item The process of moving credit ratings of different issues up or down from the issuer rating in response to different payment priorities is best described as:
A. notching.
B. structural subordination.
C. cross-default provisions.

  \item The factor considered by rating agencies when a corporation has debt at both its parent holding company and operating subsidiaries is best referred to as:
A. credit migration risk.
B. corporate family rating.
C. structural subordination.

  \item Which type of security is most likely to have the same rating as the issuer?
A. Preferred stock
B. Senior secured bond
C. Senior unsecured bond

  \item An issuer credit rating usually applies to a company's:
A. secured debt.
B. subordinated debt.
C. senior unsecured debt.

  \item The rating agency process whereby the credit ratings on issues are moved up or down from the issuer rating best describes:
A. notching.
B. pari passu ranking.
C. cross-default provisions.

  \item The notching adjustment for corporate bonds rated Aa2/AA is most likely:
A. larger than the notching adjustment for corporate bonds rated $\mathrm{B} 2 / \mathrm{B}$.
B. the same as the notching adjustment for corporate bonds rated B2/B.
C. smaller than the notching adjustment for corporate bonds rated $B 2 / B$.

  \item Which of the following statements about credit ratings is most accurate?
A. Credit ratings can migrate over time.
B. Changes in bond credit ratings precede changes in bond prices.
C. Credit ratings are focused on expected loss rather than risk of default.

  \item If goodwill makes up a large percentage of a company's total assets, this most likely indicates that:
A. the company has low free cash flow before dividends.
B. there is a low likelihood that the market price of the company's common stock is below book value.
C. a large percentage of the company's assets are not of high quality.

  \item In order to analyze the collateral of a company, a credit analyst should assess the:
A. cash flows of the company.
B. soundness of management's strategy.
C. value of the company's assets in relation to the level of debt.

  \item In order to determine the capacity of a company, it would be most appropriate to analyze the:
A. company's strategy.
B. growth prospects of the industry.
C. aggressiveness of the company's accounting policies.

  \item A credit analyst is evaluating the credit worthiness of three companies: a construction company, a travel and tourism company, and a beverage company. Both the construction and travel and tourism companies are cyclical, whereas the beverage company is non-cyclical. The construction company has the highest debt level of the three companies. The highest credit risk is most likely exhibited by the:
A. construction company.
B. beverage company. C. travel and tourism company.

  \item Based on the information provided in Exhibit 1, the EBITDA interest coverage ratio of Adidas AG is closest to:
A. $16.02 x$.
B. $23.34 \mathrm{x}$.
C. $37.98 \mathrm{x}$.

\end{enumerate}

Exhibit 1: Adidas AG Excerpt from Consolidated Income Statement in a given year ( $\epsilon$ in millions)

\begin{center}
\begin{tabular}{lc}
\hline
Gross profit & 12,293 \\
Royalty and commission income & 154 \\
Other operating income & 56 \\
Other operating expenses & 9,843 \\
Operating profit & 2,660 \\
Interest income & 64 \\
Interest expense & 166 \\
Income before taxes & 2,558 \\
Income taxes & 640 \\
Net income & 1,918 \\
\hline
\end{tabular}
\end{center}

Additional information:

Depreciation and amortization: $€ 1,214$ million

Source: Adidas AG Annual Financial Statements, December 2019.

\begin{enumerate}
  \setcounter{enumi}{28}
  \item The following information is from the annual report of Adidas AG for December 2019:
\end{enumerate}

\begin{itemize}
  \item Depreciation and amortization: $€ 1,214$ million

  \item Total assets: $€ 20,640$ million

  \item Total debt: $€ 4,364$ million

  \item Shareholders' equity: $€ 7,058$ million

\end{itemize}

The debt/capital of Adidas AG is closest to:
A. $21.14 \%$.
B. $38.21 \%$.
C. $61.83 \%$.

\begin{enumerate}
  \setcounter{enumi}{29}
  \item Funds from operations (FFO) of Pay Handle Ltd (a fictitious company) increased in 20X1. In 20X1, the total debt of the company remained unchanged while additional common shares were issued. Pay Handle Ltd's ability to service its debt in 20X1, as compared to 20X0, most likely:
\end{enumerate}

A. improved.

B. worsened. C. remained the same.

\begin{enumerate}
  \setcounter{enumi}{30}
  \item Based on the information in Exhibit 2, GZ Group's (a hypothetical company) credit risk is most likely:
\end{enumerate}

A. lower than the industry.

B. higher than the industry.

C. the same as the industry.

Exhibit 2: European Food, Beverage, and Tobacco Industry and GZ Group Selected Financial Ratios for $20 \times 0$

\begin{center}
\begin{tabular}{lccccc}
\hline
 & $\begin{array}{c}\text { Total } \\ \text { Debt/Total } \\ \text { Capital } \\ (\%)\end{array}$ & $\begin{array}{c}\text { FFO/Total } \\ \text { Debt } \\ (\%)\end{array}$ & $\begin{array}{c}\text { Return on } \\ \text { Capital } \\ (\%)\end{array}$ & $\begin{array}{c}\text { Total Debt/ } \\ \text { EBITDA } \\ (\mathbf{x})\end{array}$ & $\begin{array}{c}\text { EBITDA } \\ \text { Interest } \\ \text { Coverage } \\ (\mathbf{x})\end{array}$ \\
\hline
GZ Group & 47.1 & 77.5 & 19.6 & 1.2 & 17.7 \\
Industry median & $\mathbf{4 2 . 4}$ & $\mathbf{2 3 . 6}$ & $\mathbf{6 . 5 5}$ & $\mathbf{2 . 8 5}$ & $\mathbf{6 . 4 5}$ \\
\hline
\end{tabular}
\end{center}

\begin{enumerate}
  \setcounter{enumi}{31}
  \item Based on the information in Exhibit 3, the credit rating of DCM Group (a hypothetical company in the European food \& beverage sector) is most likely:
\end{enumerate}

A. lower than $\mathrm{AB}$ plc.

B. higher than $\mathrm{AB}$ plc.

C. the same as AB plc.

Exhibit 3: DCM Group and AB plc Selected Financial Ratios for $20 \times 0$

\begin{center}
\begin{tabular}{lccccc}
\hline
 & $\begin{array}{c}\text { Total } \\ \text { Debt/Total } \\ \text { Capital } \\ (\%)\end{array}$ & $\begin{array}{c}\text { FFO/Total } \\ \text { Debt } \\ (\%)\end{array}$ & $\begin{array}{c}\text { Return on } \\ \text { Capital } \\ (\%)\end{array}$ & $\begin{array}{c}\text { Total } \\ \text { Debt/EBITDA } \\ (\mathbf{x})\end{array}$ & $\begin{array}{c}\text { EBITDA } \\ \text { Interest } \\ \text { Coverage } \\ (\mathbf{x})\end{array}$ \\
\hline
Company & 0.2 & 84.3 & 0.1 & 1.0 & 13.9 \\
AB plc & 42.9 & 22.9 & 8.2 & 3.2 & 3.2 \\
DCM Group & $\mathbf{4 2 . 4}$ & $\mathbf{2 3 . 6}$ & $\mathbf{6 . 5 5}$ & $\mathbf{2 . 8 5}$ & $\mathbf{6 . 4 5}$ \\
$\begin{array}{l}\text { European Food, } \\ \text { Beverage, and } \\ \text { Tobacco median }\end{array}$ &  &  &  &  &  \\
\hline
\end{tabular}
\end{center}

\begin{enumerate}
  \setcounter{enumi}{32}
  \item Which industry characteristic most likely has a positive effect on a company's ability to service debt?
\end{enumerate}

A. Low barriers to entry in the industry

B. High number of suppliers to the industry

C. Broadly dispersed market share among large number of companies in the industry 34. When determining the capacity of a borrower to service debt, a credit analyst should begin with an examination of:
A. industry structure.
B. industry fundamentals.
C. company fundamentals.

\begin{enumerate}
  \setcounter{enumi}{34}
  \item Which of the following accounting issues should mostly likely be considered a character warning flag in credit analysis?
\end{enumerate}

A. Expensing items immediately

B. Changing auditors infrequently

C. Significant off-balance-sheet financing

\begin{enumerate}
  \setcounter{enumi}{35}
  \item In credit analysis, capacity is best described as the:
\end{enumerate}

A. quality of management.

B. ability of the borrower to make its debt payments on time.

C. quality and value of the assets supporting an issuer's indebtedness.

\begin{enumerate}
  \setcounter{enumi}{36}
  \item Among the four Cs of credit analysis, the recognition of revenue prematurely most likely reflects a company's:
\end{enumerate}

A. character.

B. covenants.

C. collateral.

The following information relates to questions 38-39

Exhibit 1: Industrial Comparative Ratio Analysis, Year 20XX

\begin{center}
\begin{tabular}{lcccccc}
\hline
 & $\begin{array}{c}\text { EBITDA } \\ \text { Margin } \\ (\%)\end{array}$ & $\begin{array}{c}\text { Return on } \\ \text { Capital } \\ (\%)\end{array}$ & $\begin{array}{c}\text { EBIT/ } \\ \text { Interest } \\ \text { Expense } \\ (\times)\end{array}$ & $\begin{array}{c}\text { EBITDA/ } \\ \text { Interest } \\ \text { Expense } \\ (\times)\end{array}$ & $\begin{array}{c}\text { Debt/ } \\ \text { EBITDA } \\ (\times)\end{array}$ & $\begin{array}{c}\text { Debt/ } \\ \text { Capital } \\ (\%)\end{array}$ \\
\hline
Company A & 25.1 & 25.0 & 15.9 & 19.6 & 1.6 & 35.2 \\
Company B & 29.6 & 36.3 & 58.2 & 62.4 & 0.5 & 15.9 \\
Company C & 21.8 & 16.6 & 8.9 & 12.4 & 2.5 & 46.3 \\
\hline
\end{tabular}
\end{center}

\begin{enumerate}
  \setcounter{enumi}{37}
  \item Based on only the leverage ratios in Exhibit 1, the company with the highest credit risk is:
\end{enumerate}

A. Company A.
B. Company B.
C. Company C.

\begin{enumerate}
  \setcounter{enumi}{38}
  \item Based on only the coverage ratios in Exhibit 4, the company with the highest credit quality is:
A. Company A.
B. Company B.
C. Company C.
\end{enumerate}

\section{The following information relates to questions}
 40-41\section{Exhibit 1: Consolidated Income Statement ( $\_$millions)}
\begin{center}
\begin{tabular}{|c|c|c|}
\hline
 & Company X & Company $Y$ \\
\hline
Net revenues & 50.7 & 83.7 \\
\hline
Operating expenses & 49.6 & 70.4 \\
\hline
Operating income & 1.1 & 13.3 \\
\hline
Interest income & 0.0 & 0.0 \\
\hline
Interest expense & 0.6 & 0.8 \\
\hline
Income before income taxes & 0.5 & 12.5 \\
\hline
Provision for income taxes & -0.2 & -3.5 \\
\hline
Net income & 0.3 & 9.0 \\
\hline
\end{tabular}
\end{center}

\section{Exhibit 2: Consolidated Balance Sheets ( \textbackslash pm millions)}
\begin{center}
\begin{tabular}{lcc}
\hline
 & Company $\mathbf{X}$ & Company $\mathbf{Y}$ \\
\hline
ASSETS &  &  \\
Current assets & 10.3 & 21.9 \\
Property, plant, and equipment, net & 3.5 & 20.1 \\
Goodwill & 8.3 & 85.0 \\
Other assets & 0.9 & 5.1 \\
\hline
Total assets & 23.0 & 132.1 \\
\hline
\end{tabular}
\end{center}

LIABILITIES AND SHAREHOLDERS' EQUITY

Current liabilities

\begin{center}
\begin{tabular}{|c|c|c|}
\hline
 & Company X & Company Y \\
\hline
Accounts payable and accrued expenses & 8.4 & 16.2 \\
\hline
Short-term debt & 0.5 & 8.7 \\
\hline
Total current liabilities & 8.9 & 24.9 \\
\hline
Long-term debt & 11.7 & 21.1 \\
\hline
Other non-current liabilities & 1.1 & 22.1 \\
\hline
Total liabilities & 21.7 & 68.1 \\
\hline
Total shareholders' equity & 1.3 & 64.0 \\
\hline
Total liabilities and shareholders' equity & 23.0 & 132.1 \\
\hline
\end{tabular}
\end{center}

\section{Exhibit 3: Consolidated Statements of Cash Flow ( $\in$ millions)}
\begin{center}
\begin{tabular}{lcc}
\hline
 & Company $\mathbf{X}$ & Company $\mathbf{Y}$ \\
\hline
CASH FLOWS FROM OPERATING &  &  \\
ACTIVITIES &  & 9.0 \\
Net income & 0.3 & 3.8 \\
Depreciation & 1.0 & 1.6 \\
Goodwill impairment & 2.0 & -0.4 \\
Changes in working capital & 0.0 & 14.0 \\
\end{tabular}
\end{center}

\section{CASH FLOWS FROM INVESTING ACTIVITIES}
Additions to property and equipment

\begin{center}
\begin{tabular}{cc}
-1.0 & -4.0 \\
-0.1 & 0.0 \\
0.2 & 2.9 \\
0.3 & 0.0 \\
\hline
-0.6 & -1.1 \\
\hline
\end{tabular}
\end{center}

Net cash used in investing activities

CASH FLOWS FROM FINANCING

ACTIVITIES

Repurchase of common stock

\begin{center}
\begin{tabular}{ccc}
-1.5 & -4.0 \\
-0.3 & -6.1 \\
0.0 & -3.4 \\
3.9 & 3.9 \\
-3.4 & -2.5 \\
\hline
-1.3 & -12.1 \\
\hline
\end{tabular}
\end{center}

NET INCREASE IN CASH AND CASH 40. Based on Exhibits 1-3, in comparison to Company X, Company Y has a higher:
A. debt/capital.
B. debt/EBITDA.
C. free cash flow after dividends/debt.

\begin{enumerate}
  \setcounter{enumi}{40}
  \item Based on Exhibits 5-7, in comparison to Company Y, Company X has greater:
A. leverage.
B. interest coverage.
C. operating profit margin.

  \item Holding all other factors constant, the most likely effect of low demand and heavy new issue supply on bond yield spreads is that yield spreads will:
A. widen.
B. tighten.
C. not be affected.

  \item Credit yield spreads most likely widen in response to:
A. high demand for bonds.
B. weak performance of equities.
C. strengthening economic conditions.

  \item The factor that most likely results in corporate credit spreads widening is:
A. an improving credit cycle.
B. weakening economic conditions.
C. a period of high demand for bonds.

  \item Credit spreads are most likely to widen:
A. in a strengthening economy.
B. as the credit cycle improves.
C. in periods of heavy new issue supply and low borrower demand.

  \item Which of the following factors in credit analysis is more important for general obligation non-sovereign government debt than for sovereign debt?
A. Per capita income
B. Power to levy and collect taxes
C. Requirement to balance an operating budget

  \item In contrast to high-yield credit analysis, investment-grade analysis is more likely to rely on:

\end{enumerate}

A. spread risk. B. anassessment of bank credit facilities.

C. matching of liquidity sources to upcoming debt maturities.

\begin{enumerate}
  \setcounter{enumi}{47}
  \item Which of the following factors would best justify a decision to avoid investing in a country's sovereign debt?
\end{enumerate}

A. Freely floating currency

B. A population that is not growing

C. Suitable checks and balances in policymaking

\section{SOLUTIONS}
\begin{enumerate}
  \item A is correct. Credit migration risk or downgrade risk refers to the risk that a bond issuer's creditworthiness may deteriorate or migrate lower. The result is that investors view the risk of default to be higher, causing the spread on the issuer's bonds to widen.

  \item C is correct. Market liquidity risk refers to the risk that the price at which investors transact may be different from the price indicated in the market. Market liquidity risk is increased by (1) less debt outstanding and/or (2) a lower issue credit rating. Because Stedsmart Ltd is comparable to Fignermo Ltd except for less publicly traded debt outstanding, it should have higher market liquidity risk.

  \item $\mathrm{C}$ is correct. Credit risk is the risk of loss resulting from the borrower failing to make full and timely payments of interest and/or principal.

  \item C is correct. Market liquidity risk is the risk that the price at which investors can actually transact-buying or selling-may differ from the price indicated in the market.

  \item C is correct. Loss severity is the portion of a bond's value (including unpaid interest) an investor loses in the event of default.

  \item B is correct. The two components of credit risk are default probability and loss severity. In the event of default, loss severity is the portion of a bond's value (including unpaid interest) an investor loses. A and C are incorrect because spread and market liquidity risk are credit-related risks, not components of credit risk.

  \item A is correct. Credit risk has two components: default risk and loss severity. Because default risk is quite low for most high-quality debt issuers, bond investors tend to focus more on this likelihood and less on the potential loss severity.

  \item B is correct. The expected loss for a given debt instrument is the default probability multiplied by the loss severity given default. The loss severity is often expressed as ( 1 - Recovery rate).

  \item A is correct. First mortgage debt is senior secured debt and has the highest priority of claims. First mortgage debt also has the highest expected recovery rate. First mortgage debt refers to the pledge of specific property. Neither senior unsecured nor junior subordinate debt has any claims on specific assets.

  \item B is correct. Whether or not secured assets are sufficient for the claims against them does not influence priority of claims. Any deficiency between pledged assets and the claims against them becomes senior unsecured debt and still adheres to the guidelines of priority of claims.

  \item A is correct. Senior subordinated debt is ranked lower than senior unsecured debt and thus has a lower priority of payment.

  \item $C$ is correct. The highest-ranked unsecured debt is senior unsecured debt. Lower-ranked debt includes senior subordinated debt. A and B are incorrect because mortgage debt and second lien loans are secured and higher ranked.

  \item $\mathrm{C}$ is correct. According to the absolute priority of claims, in the event of bankruptcy, creditors with a secured claim have the right to the value of that specific property before any other claim.

  \item A is correct. A second lien has a secured interest in the pledged assets. Second lien debt ranks higher in priority of payment than senior unsecured and senior subordinated debt and thus would most likely have a higher recovery rate.

  \item A is correct. Second lien debt is secured debt, which is senior to unsecured debt and to subordinated debt.

  \item $\mathrm{C}$ is correct. Both analysts and rating agencies have difficulty foreseeing future debt-financed acquisitions.

  \item A is correct. Notching is the process for moving ratings up or down relative to the issuer rating when rating agencies consider secondary factors, such as priority of claims in the event of a default and the potential loss severity.

  \item $\mathrm{C}$ is correct. Structural subordination can arise when a corporation with a holding company structure has debt at both its parent holding company and operating subsidiaries. Debt at the operating subsidiaries is serviced by the cash flow and assets of the subsidiaries before funds are passed to the parent holding company.

  \item $C$ is correct. The issuer credit rating usually applies to its senior unsecured debt.

  \item C is correct. An issuer credit rating usually applies to its senior unsecured debt.

  \item A is correct. Recognizing different payment priorities, and thus the potential for higher (or lower) loss severity in the event of default, the rating agencies have adopted a notching process whereby their credit ratings on issues can be moved up or down from the issuer rating (senior unsecured).

  \item $C$ is correct. As a general rule, the higher the senior unsecured rating, the smaller the notching adjustment. Thus, for corporate bonds rated Aa2/AA, the rating agencies will typically apply smaller rating adjustments, or notches, to the related issue.

  \item A is correct. Credit migration is the risk that a bond issuer's creditworthiness deteriorates, or migrates lower. Over time, credit ratings can migrate significantly from what they were at the time a bond was issued. An investor should not assume that an issuer's credit rating will remain the same from the time of purchase through the entire holding period.

  \item C is correct. Goodwill is viewed as a lower quality asset compared with tangible assets that can be sold and more easily converted into cash.

  \item $\mathrm{C}$ is correct. The value of assets in relation to the level of debt is important to assess the collateral of the company-that is, the quality and value of the assets that support the debt levels of the company.

  \item B is correct. The growth prospects of the industry provide the analyst insight regarding the capacity of the company.

  \item A is correct. The construction company is both highly leveraged, which increases credit risk, and in a highly cyclical industry, which results in more volatile earnings.

  \item $\mathrm{B}$ is correct. The interest expense is $€ 166$ million and EBITDA = Operating profit + Depreciation and amortization $=€ 2,660+1,214$ million $=€ 3,874$ million. EBITDA interest coverage $=$ EBITDA/Interest expense $=3,874 / 166=23.34$ times.

  \item B is correct. Total debt is $€ 4,364$ million with Total capital $=$ Total debt + Shareholders' equity $=€ 4,364+7,058=€ 11,422$ million. The Debt $/$ Capital $=$ $4,364 / 11,422=38.21 \%$.

  \item A is correct. If the debt of the company remained unchanged but FFO increased, more cash was available to service debt compared to the previous year. Additionally, debt/capital improved, which implies that the ability of Pay Handle Ltd to service their debt also improved.

  \item A is correct. Based on four of the five credit ratios, GZ Group's credit quality is superior to that of the industry.

  \item A is correct. DCM Group has more financial leverage and less interest coverage than $\mathrm{AB}$ plc, which implies greater credit risk.

  \item B is correct. An industry with a high number of suppliers reduces the suppliers' negotiating power, thus helping companies control expenses and aiding in the servicing of debt.

  \item A is correct. Credit analysis starts with industry structure-for example, by looking at the major forces of competition, followed by an analysis of industry fundamentals-and then turns to examination of the specific issuer.

  \item C is correct. Credit analysts can make judgments about management's character by evaluating the use of aggressive accounting policies, such as using a significant amount of off-balance-sheet financing. This activity is a potential warning flag for other behaviors such as using a significant amount of off-balance-sheet financing.

  \item B is correct. Capacity refers to the ability of a borrower to service its debt. Capacity is determined through credit analysis of an issuer's industry and of the specific issuer.

  \item A is correct. Credit analysts can make judgments about management's character in a number of ways, including by observing its use of aggressive accounting policies and/or tax strategies. An example of this aggressiveness is recognizing revenue prematurely.

  \item C is correct. Debt/capital and debt/EBITDA are used to assess a company's leverage. Higher leverage ratios indicate more leverage and thus higher credit risk. Company C's debt/capital (46.3\%) and debt/EBITDA (2.5×) are higher than those for Companies A and B.

  \item B is correct. The EBITDA/interest expense and EBIT/interest expense are coverage ratios. Coverage ratios measure an issuer's ability to meet its interest payments. A higher ratio indicates better credit quality. Company B's EBITDA/ interest expense $(62.4 \times)$ and EBIT/interest expense $(58.2 \times)$ are higher than those for Companies A and C.

  \item C is correct because Company $Y$ has a higher ratio of free cash flow after dividends to debt than Company $X$, not lower, as shown in the following table.

\end{enumerate}

Free cash flow after dividends as a $\%$ of debt $=\frac{\text { FCF after dividends }}{\text { Debt }}$

\section{Company X}
Company $Y$

\begin{center}
\begin{tabular}{|c|c|c|}
\hline
 & Company $X$ & Company Y \\
\hline
\multicolumn{3}{|l|}{Less} \\
\hline
Net capital expenditures & -0.8 & -1.1 \\
\hline
Dividends & -0.3 & -6.1 \\
\hline
Free cash flow after dividends & $\pounds 2.2$ & $\pounds 6.8$ \\
\hline
Debt & $\pounds 12.2$ & $\pounds 29.8$ \\
\hline
Free cash flow after dividends as a $\%$ of debt & $(2.2 / 12.2) \times 100$ & $(6.8 / 29.8) \times 100$ \\
\hline
Free cash flow after dividends as a $\%$ of debt & $18.0 \%$ & $22.8 \%$ \\
\hline
\end{tabular}
\end{center}

A is incorrect. Company $\mathrm{Y}$ has a lower debt/capital than Company $\mathrm{X}$, as shown in the following table.

Debt divided by Capital (\%) $=\frac{\text { Debt }}{(\text { Debt }+ \text { Equity })}$

\begin{center}
\begin{tabular}{|c|c|c|}
\hline
 & Company X & Company Y \\
\hline
Debt & $\pounds 12.2$ & $\pounds 29.8$ \\
\hline
\multicolumn{3}{|l|}{Capital} \\
\hline
Debt & 12.2 & 29.8 \\
\hline
+ Equity & 1.3 & 64.0 \\
\hline
Capital & $\pounds 13.5$ & $\pounds 93.8$ \\
\hline
Debt/Capital (\%) & $(12.2 / 13.5) \times 100$ & $(29.8 / 93.8) \times 100$ \\
\hline
Debt/Capital (\%) & $90.4 \%$ & $31.8 \%$ \\
\hline
\end{tabular}
\end{center}

B is incorrect because Company $\mathrm{Y}$ has a lower debt/EBITDA than Company $\mathrm{Y}$, not higher, as shown in the following table.

\begin{center}
\begin{tabular}{lcc}
\hline
 & Company $\mathbf{X}$ & Company $\mathbf{Y}$ \\
\hline
Operating income & $\pounds 1.1$ & $\pounds 13.3$ \\
EBIT & $\pounds 1.1$ &  \\
plus &  & $\pounds 13.3$ \\
Depreciation & 1.0 &  \\
Amortization & 0.0 & 3.8 \\
EBITDA & $\pounds 2.1$ & 0.0 \\
\cline { 2 - 3 }
Debt & $\pounds 12.2$ & $\pounds 17.1$ \\
 &  & $\pounds 29.8$ \\
Debt/EBITDA & $12.2 / 2.1$ &  \\
 &  & $29.8 / 17.1$ \\
Debt/EBITDA & 5.81 & 1.74 \\
\hline
\end{tabular}
\end{center}

\begin{enumerate}
  \setcounter{enumi}{40}
  \item A is correct. Compared with Company Y, based on both their debt/capital and their ratios of free cash flow after dividends to debt, which are measures of leverage commonly used in credit analysis, Company $\mathrm{X}$ is more highly leveraged, as shown in the following table. Debt divided by Capital $(\%)=\frac{\text { Debt }}{(\text { Debt }+ \text { Equity })}$
\end{enumerate}

\begin{center}
\begin{tabular}{lcc}
\hline
 & Company X & Company Y \\
Debt & $\pounds 2.2$ & $\pounds 29.8$ \\
 &  &  \\
Capital & 2.2 & 29.8 \\
Debt & 4.3 & 64.0 \\
\cline { 2 - 3 }
+quity & $\pounds 6.5$ & $\pounds 93.8$ \\
Capital &  &  \\
 & $(12.2 / 13.5) \times 100$ & $(29.8 / 93.8) \times 100$ \\
Debt/Capital (\%) & $90.4 \%$ & $31.8 \%$ \\
Debt/Capital (\%) &  &  \\
\hline
\end{tabular}
\end{center}

Free cash flow after dividends as a $\%$ of debt $=\frac{\text { FCF after dividends }}{\text { Debt }}$

\begin{center}
\begin{tabular}{|c|c|c|}
\hline
 & Company $X$ & Company Y \\
\hline
Cash flow from operations & $\pounds 3.3$ & $\pounds 14.0$ \\
\hline
\multicolumn{3}{|l|}{Less} \\
\hline
Net capital expenditures & -0.8 & -1.1 \\
\hline
Dividends & -0.3 & -6.1 \\
\hline
Free cash flow after dividends & $\pounds 2.2$ & $\pounds 6.8$ \\
\hline
Debt & $\pounds 12.2$ & $\pounds 29.8$ \\
\hline
Free cash flow after dividends as a $\%$ of debt & $(2.2 / 12.2) \times 100$ & $(6.8 / 29.8) \times 100$ \\
\hline
Free cash flow after dividends as a $\%$ of debt & $18.0 \%$ & $22.8 \%$ \\
\hline
\end{tabular}
\end{center}

\begin{enumerate}
  \setcounter{enumi}{41}
  \item A is correct. Low demand implies wider yield spreads, and heavy supply widens spreads even further.

  \item B is correct. In weak financial markets, including weak markets for equities, credit spreads will widen.

  \item B is correct. Weakening economic conditions will push investors to desire a greater risk premium and drive overall credit spreads wider.

  \item $C$ is correct. In periods of heavy new issue supply, credit spreads will widen if demand is insufficient.

  \item $C$ is correct. Non-sovereign governments typically must balance their operating budgets and lack the discretion to use monetary policy as many sovereigns can.

  \item A is correct. Most investors in investment-grade debt focus on spread risk-that is, the effect of changes in spreads on prices and returns-while in high-yield analysis, the focus on default risk is relatively greater.

  \item B is correct. Among the most important considerations in sovereign credit analysis is growth and age distribution of population. A relatively young and growing population contributes to growth in GDP and an expanding tax base and relies less on social services, pensions, and health care relative to an older population.

\end{enumerate}

\section{Derivatives}
\section{LEARNING MODULE 1}
\section{Derivative Instrument and Derivative Market Features}
\section{LEARNING OUTCOME}
\begin{center}
\begin{tabular}{c|l}
Mastery & The candidate should be able to: \\
\hline
$\square$ & $\begin{array}{l}\text { define a derivative and describe basic features of a derivative } \\ \text { instrument } \\ \text { describe the basic features of derivative markets, and contrast } \\ \text { over-the-counter and exchange-traded derivative markets }\end{array}$ \\
$\square$ &  \\
\end{tabular}
\end{center}

\section{INTRODUCTION}
Earlier lessons described markets for financial assets related to equities, fixed income, currencies, and commodities. These markets are known as cash markets or spot markets in which specific assets are exchanged at current prices referred to as cash prices or spot prices. Derivatives involve the future exchange of cash flows whose value is derived from or based on an underlying value. The following lessons define and describe features of derivative instruments and derivative markets.

\section{Summary}
\begin{itemize}
  \item A derivative is a financial contract that derives its value from the performance of an underlying asset, which may represent a firm commitment or a contingent claim.

  \item Derivative markets expand the set of opportunities available to market participants beyond the cash market to create or modify exposure to an underlying.

  \item The most common derivative underlyings include equities, fixed income and interest rates, currencies, commodities, and credit.

  \item Over-the-counter (OTC) derivative markets involve the initiation of customized, flexible contracts between derivatives end users and financial intermediaries.

  \item Exchange-traded derivatives (ETDs) are standardized contracts traded on an organized exchange, which requires collateral on deposit to protect against counterparty default. - For derivatives that are centrally cleared, a central counterparty (CCP) assumes the counterparty credit risk of the derivative counterparties and provides clearing and settlement services.

\end{itemize}

\section{LEARNING MODULE PRE-TEST}
\begin{enumerate}
  \item Describe two reasons why an investor may use a derivative instrument. Reason 1. An investor may use a derivative to benefit from an expected decline in the value of an underlying.
\end{enumerate}

Reason 2. An investor may use a derivative to create a relatively large exposure to an underlying with a relatively small cash outlay.

\begin{enumerate}
  \setcounter{enumi}{1}
  \item Determine the correct answer to complete the following sentences: Market participants use derivative agreements to exchange cash flows in the future based on an . A type of derivative in which one of the counterparties determines whether and when the trade will settle is known as a
\end{enumerate}

Market participants use derivative agreements to exchange cash flows in the future based on an underlying. A type of derivative in which one of the counterparties determines whether and when the trade will settle is known as a contingent claim.

\begin{enumerate}
  \setcounter{enumi}{2}
  \item Describe two uses of a derivative with a credit underlying.
\end{enumerate}

Use 1. Financial institutions and other lenders may purchase a credit default swap to offset credit risks arising from their lending or trade activities.

Use 2. Investors may purchase or sell a CDS to diversify their portfolios by increasing or decreasing credit risk without buying or selling bonds.

\begin{enumerate}
  \setcounter{enumi}{3}
  \item Contrast the features of exchange-traded derivative (ETD) and over the counter (OTC) derivative contracts.
\end{enumerate}

ETD contracts are formal and standardized, with terms set by an exchange, while the terms of OTC contracts can be customized to match a desired risk exposure profile.

\begin{enumerate}
  \setcounter{enumi}{4}
  \item Explain why the OTC derivative market is preferable for an end user seeking to hedge a specific underlying transaction.
\end{enumerate}

End users may hedge a specific existing or anticipated underlying exposure based upon customized terms using an OTC derivative contract.

\begin{enumerate}
  \setcounter{enumi}{5}
  \item Identify one benefit of central clearing for OTC derivative markets.
\end{enumerate}

Central clearing centralizes and standardizes the management of credit risk, clearing, and settlement of transactions between financial intermediaries through a central clearing counterparty (CCP) in order to reduce systemic risk.

\section{DERIVATIVE FEATURES}
define a derivative and describe basic features of a derivative
instrument

\section{Definition and Features of a Derivative}
A derivative is a financial instrument that derives its value from the performance of an underlying asset. The asset in a derivative is called the underlying. The underlying may not be an individual asset but rather a group of standardized assets or variables, such as interest rates or a credit index.

Market participants use derivative agreements to exchange cash flows in the future based on an underlying value. For example, Exhibit 1 shows the one-time future exchange of publicly traded shares of stock at a fixed price in a derivative known as a forward contract.

\section{Exhibit 1: Forward Contract}
\begin{center}
\includegraphics[max width=\textwidth]{2023_05_04_36535b8d80b32081d422g-163}
\end{center}

A derivative does not directly pass through the returns of the underlying but transforms the performance of the underlying. In Exhibit 1, AMY Investments agrees today $(t=0)$ to deliver 1,000 shares of Airbus (AIR) at a fixed price of $€ 30$ per share on a future date $(t=T)$, which in our example is in six months. The forward contract allows AMY to transfer the price risk of underlying AIR shares to a second party, or a counterparty, by entering into this derivative contract. If the spot price of $\operatorname{AIR}\left(S_{T}\right)$ is $€ 25$ per share at time $T$ in six months, AMY will either receive $€ 30,000$ from its counterparty, a financial intermediary, for 1,000 AIR shares now worth just $€ 25,000$, or simply settle with the intermediary the $€ 5,000$ difference in cash. Derivative transactions usually involve at least one financial intermediary as a counterparty. As we will see later, counterparty credit risk, or the likelihood that a counterparty is unable to meet its financial obligations under the contract, is an important consideration for these instruments.

\section{Derivative Instrument and Derivative Market Features}
A derivative contract is a legal agreement between counterparties with a specific maturity, or length of time until the closing of the transaction, or settlement. The buyer of a derivative enters a contract whose value changes in a way similar to a long position in the underlying, and the seller has exposure similar to a short position. The contract size (sometimes referred to as notional principal or amount) is agreed upon at the outset and may remain constant or change over time.

Exhibit 1 is an example of a stand-alone derivative, a distinct derivative contract, such as a derivative on a stock or bond. An embedded derivative is a derivative within an underlying, such as a callable, puttable, or convertible bond. Exhibit 2 provides a sample term sheet that includes key features of AMY Investment's stand-alone forward contract with a financial intermediary.

\section{Exhibit 2: Sample Forward Contract Term Sheet}
\section{Contract Type:}
Firm commitment or contingent right to exchange future cash flows

Maturity:

Final date upon which payment or settlement occurs

\section{Counterparties:}
Legal entities entering the derivative contract

Underlying:

Reference asset or variable used as source for contract value

\section{Contract Size}
Amount(s) used for calculation to price and value the derivative

Underlying Price:

Pre-agreed price for commitment or contingent claim settlement

\section{Contract Details}
Forward Transaction Term Sheet

Start Date: $\quad$ [Spot start]

Maturity Date: $\quad$ [Six months from Start Date]

Forward [Financial Intermediary]

Purchaser:

Forward Seller: AMY Investments

Forward $\quad 1,000$ shares of Airbus (AIR) com-

Delivery: $\quad$ mon stock traded on the Frankfurt Stock Exchange

Forward Price: $€ 30$ per share

Business Days: $\quad$ Frankfurt

Documentation: ISDA Agreement and credit terms acceptable to both parties

The derivative between AMY and the financial intermediary is a firm commitment, in which a pre-determined amount is agreed to be exchanged at settlement. Firm commitments include forward contracts, futures contracts, and swaps involving a periodic exchange of cash flows. Another type of derivative is a contingent claim, in which one of the counterparties determines whether and when the trade will settle. An option is the primary contingent claim.

Derivative markets expand the set of opportunities available to market participants to create or modify exposure to an underlying in several ways:

\begin{itemize}
  \item Investors can sell short to benefit from an expected decline in the value of the underlying.

  \item Investors may use derivatives as a tool for portfolio diversification.

  \item Issuers may offset the financial market exposure associated with a commercial transaction. - Market participants may create large exposures to an underlying with a relatively small cash outlay.

  \item Derivatives typically have lower transaction costs and are often more liquid than underlying spot market transactions.

\end{itemize}

Issuers and investors use derivatives to increase or decrease financial market exposures. For example, use of a derivative to offset or neutralize existing or anticipated exposure to an underlying is referred to as hedging, with the derivative itself commonly described as a hedge of the underlying transaction.

\section{KNOWLEDGE CHECK}
\section{Derivative Features}
\begin{enumerate}
  \item Identify one reason why an issuer may use a derivative instrument.
\end{enumerate}

\section{Solution}
An issuer may use a derivative to offset the financial market exposure associated with a commercial transaction. An issuer may also use a derivative to offset or neutralize existing or anticipated exposure to an underlying.

\begin{enumerate}
  \setcounter{enumi}{1}
  \item Identify which example corresponds to each of the following stand-alone or embedded derivative contract types:
A. Firm commitment
B. Contingent claim
C. Neither a firm commitment nor a contingent claim exchange-traded fund (ETF)

  \item Callable bond

  \item Fixed-price natural gas delivery contract

  \item Purchase of a FTSE 100 Index

\end{enumerate}

\section{Solution}
\begin{enumerate}
  \item B is correct. A callable bond is an example of an embedded derivative within an underlying, which is a contingent claim.

  \item A is correct. A fixed-price gas delivery contract is an example of a contract, which is a firm commitment with natural gas as the underlying.

  \item C is correct. A FTSE 100 Index exchange-traded fund (ETF) is neither a firm commitment nor a contingent claim but rather an example of a cash or spot market transaction.

  \item Determine the correct answers to fill in the blanks: Equities are an example of a derivative , and a is a legal entity entering a derivative contract.

\end{enumerate}

\section{Solution}
Equities are an example of a derivative underlying, and a counterparty is a legal entity entering a derivative contract.

\begin{enumerate}
  \setcounter{enumi}{3}
  \item Describe the use of a derivative for hedging purposes.
\end{enumerate}

\section{Solution}
Use of a derivative for hedging purposes involves offsetting or neutralizing an existing or anticipated exposure to an underlying, referred to as hedging. 5. Explain the settlement of a.) a contingent claim and b.) a futures contract.

\section{Solution}
An option is an example of a contingent claim. On the maturity date, settlement may or may not result in a payoff related to the exercise price, depending on the relative performance of the underlying.

A futures contract is a firm commitment. This contract results in a settlement payment on the maturity date equal to the difference between the current market price and a pre-agreed futures price.

\section{DERIVATIVE UNDERLYINGS}
 define a derivative and describe basic features of a derivativeinstrument

Derivatives are typically grouped by the underlying from which their value is derived. A derivative contract may reference more than one underlying. The most common derivative underlyings include equities, fixed income and interest rates, currencies, commodities, and credit.

\section{Equities}
Equity derivatives usually reference an individual stock, a group of stocks, or a stock index, such as the FTSE 100. Options are the most common derivatives on individual stocks. Index derivatives are commonly traded as options, forwards, futures, and swaps.

Index swaps, or equity swaps, allow the investor to pay the return on one stock index and receive the return on another index or interest rate. An investment manager can use index swaps to increase or reduce exposure to an equity market or sector without trading the individual shares. These swaps are widely used in top-down asset allocation strategies. Finally, options, futures, and swaps are available based upon the realized volatility of equity index prices over a certain period. These contracts allow market participants to manage the risk, or dispersion, of price changes separately from the direction of equity price changes.

Options on individual stocks are purchased and sold by investors and frequently used by issuers as compensation for their executives and employees. Stock options are granted to provide incentives to work toward stronger corporate performance in the expectation of higher stock prices. Stock options can result in companies paying lower cash compensation. Companies may also issue warrants, which are options granted to employees or sold to the public that allow holders to purchase shares at a fixed price in the future directly from the issuer.

\section{Fixed-Income Instruments}
Bonds are a widely used underlying, and related derivatives include options, forwards, futures, and swaps. Government issuers, such as the US Treasury or Japanese Ministry of Finance, usually have many bond issues outstanding. A single standardized futures contract associated with such bonds therefore often specifies parameters that allow more than one bond issue to be delivered to settle the contract. An interest rate is not an asset but rather a fixed-income underlying used in many interest rate derivatives, such as forwards, futures, and options. Interest rate swaps are a type of firm commitment frequently used by market participants to convert from fixed to floating interest rate exposure over a certain period. For example, an investment manager can use interest rate swaps to increase or reduce portfolio duration without trading bonds. An issuer, on the other hand, might use an interest rate swap to alter the interest rate exposure profile of its liabilities.

A market reference rate (MRR) is the most common interest rate underlying used in interest rate swaps. These rates typically match those of loans or other short-term obligations. Survey-based Libor rates used as reference rates in the past have been replaced by rates based on a daily average of observed market transaction rates. For example, the Secured Overnight Financing Rate (SOFR) is an overnight cash borrowing rate collateralized by US Treasuries. Other MRRs include the euro short-term rate $(€ S T R)$ and the Sterling Overnight Index Average (SONIA).

\section{Currencies}
Market participants frequently use derivatives to hedge the exposure of commercial and financial transactions that arise due to foreign exchange risk. For example, exporters often enter into forward contracts to sell foreign currency and purchase domestic currency under terms matching those of a delivery contract for goods or services in a foreign country. Alternatively, an investor might sell futures on a particular currency while retaining a securities portfolio denominated in that currency to benefit from a temporary decline in the value of that currency. Options, forwards, futures, and swaps based upon sovereign bonds and exchange rates are used to manage currency risk.

\section{Commodities}
Cash or spot markets for soft and hard commodities involve the physical delivery of the underlying upon settlement. Soft commodities are agricultural products, such as cattle and corn, and hard commodities are natural resources, such as crude oil and metals. Commodity derivatives are widely used to manage either the price risk of an individual commodity or a commodity index separate from physical delivery. For example, an airline, shipping, or freight company might purchase oil futures as a hedge against rising operating expenses due to higher fuel costs. An investor might purchase a commodity index futures contract to increase exposure to commodity prices without taking physical delivery of the underlying.

\section{Credit}
Credit derivative contracts are based upon the default risk of a single issuer or a group of issuers in an index. Credit default swaps (CDS) allow an investor to manage the risk of loss from borrower default separately from the bond market. CDS contracts trade on a spread that represents the likelihood of default. For example, an investor might buy or sell a CDS contract on a high-yield index to change its portfolio exposure to high-yield credit without buying or selling the underlying bonds. Alternatively, a bank may purchase a CDS contract to offset existing credit exposure to an issuer's potential default.

\section{Derivative Instrument and Derivative Market Features}
\section{Other}
Other derivative underlyings include weather, cryptocurrencies, and longevity, all of which can influence the financial performance of various market participants. For example, longevity risk is important to insurance companies and defined benefit pension plans that face exposure to increased life expectancy. Derivatives based upon these underlyings are less common and more difficult to price. Exhibit 3 provides a summary of common underlyings.

\section{Exhibit 3: Common Derivative Underlyings}
\begin{center}
\begin{tabular}{|c|c|c|}
\hline
Asset Class & Examples & Sample Uses \\
\hline
Equities & $\begin{array}{l}\text { Individual stocks } \\ \text { Equity indexes } \\ \text { Equity price volatility }\end{array}$ & $\begin{array}{l}\text { Change exposure profile (Investors) } \\ \text { Employee compensation (Issuers) }\end{array}$ \\
\hline
Interest Rates & $\begin{array}{l}\text { Sovereign bonds } \\ \text { (domestic) } \\ \text { Market reference rates }\end{array}$ & $\begin{array}{l}\text { Change duration exposure (Investors) } \\ \text { Alter debt exposure profile (Issuers) }\end{array}$ \\
\hline
$\begin{array}{l}\text { Foreign } \\ \text { Exchange }\end{array}$ & $\begin{array}{l}\text { Sovereign bonds (for- } \\ \text { eign) } \\ \text { Market exchange rates }\end{array}$ & $\begin{array}{l}\text { Manage global portfolio risks (Investors) } \\ \text { Manage global trade risks (Issuers) }\end{array}$ \\
\hline
Commodities & $\begin{array}{l}\text { Soft and hard commod- } \\ \text { ities } \\ \text { Commodity indexes }\end{array}$ & $\begin{array}{l}\text { Manage operating risks (Consumers/ } \\ \text { Producers) } \\ \text { Portfolio diversification (Investors) }\end{array}$ \\
\hline
Credit & $\begin{array}{l}\text { Individual refence } \\ \text { entities } \\ \text { Credit indexes }\end{array}$ & $\begin{array}{l}\text { Portfolio diversification (Investors) } \\ \text { Manage credit risk (Financial Intermediaries }\end{array}$ \\
\hline
Other & $\begin{array}{l}\text { Weather } \\ \text { Cryptocurrencies } \\ \text { Longevity }\end{array}$ & $\begin{array}{l}\text { Manage operating risks (Issuers) } \\ \text { Manage portfolio risks (Investors) }\end{array}$ \\
\hline
\end{tabular}
\end{center}

\section{RARE EARTH FUTURES AND THE LME LITHIUM CONTRACT}
Derivative underlyings continue to adapt to the growing importance of environmental, social, and governance (ESG) factors affecting commercial and financial markets. For example, as the automotive industry shifts from internal combustion engine technology to electric vehicle (EV) production due to environmental concerns, demand for rare earth metals, such as lithium, as inputs into the EV battery production process are of increasing importance.

In response to growing demand from commodity producers and end users as well as investors, the London Metal Exchange (LME) introduced a lithium futures contract in 2021. The LME lithium contract is cash settled in USD against a weekly published spot price for battery-grade lithium hydroxide monohydrate deliverable in China, Japan, and Korea based upon a lot size of one metric ton per contract.

\section{Investor Scenarios}
The following scenarios consider the specific goals of two parties and review the most appropriate derivative contract for each.

\section{Scenario 1: Hightest Capital}
Hightest Capital is a US-based investment fund with a well-diversified domestic equity portfolio. Hightest's senior portfolio manager believes that health care stocks will significantly outperform the overall index over the next six months. Ace Limited is a financial intermediary and member of the Chicago Board Options Exchange (CBOE).

Hightest purchases an option based upon a standardized contract on the S\&P 500 Health Care Select Sector Index (SIXV) with Ace as the financial intermediary and the spot SIXV price as the underlying. SIXV is comprised of approximately 60 health care equities included in the S\&P 500 Index. The contract is a contingent claim, which grants Hightest the right to purchase SIXV at a $5 \%$ premium to the current market price (spot SIXV $\times 1.05)$ in six months.

\section{Scenario 2: Esterr Inc.}
Esterr Inc. is a Toronto-based public company with a CAD250 million floating-rate term loan. The loan has a remaining maturity of three and a half years and is priced at three-month MRR (which is CORRA, or the Canadian Overnight Reference Rate Average) plus $150 \mathrm{bps}$. Esterr's treasurer is concerned about higher Canadian interest rates over the remaining life of the loan and would like to fix Esterr's interest expense.

Esterr enters into a CAD250 million interest rate swap contract with a financial intermediary with MRR as the underlying. Under the swap, Esterr agrees to pay a fixed interest rate and receive three-month MRR on a notional principal of CAD250 million for three and a half years based upon payment dates that match the term loan. The swap contract is a firm commitment.

\section{KNOWLEDGE CHECK}
\section{Derivative Underlyings}
\begin{enumerate}
  \item Describe how and why an underlying may be used in employee compensation.
\end{enumerate}

\section{Solution}
Derivatives with an equity underlying, in particular the stock of a particular issuer, may be included in the compensation of that company's employees. Stock options encourage employees to work to increase the equity value of the company since a rise in the share price will cause their options to appreciate in value.

\begin{enumerate}
  \setcounter{enumi}{1}
  \item Explain how a UK-based importer of goods from the euro zone might use a derivative with a currency underlying to mitigate risk.
\end{enumerate}

\section{Solution}
A UK-based importer of goods from the euro zone will likely pay EUR for goods that she intends to sell for GBP. To address this currency mismatch, she may consider entering a firm commitment to purchase EUR in exchange for GBP at a pre-determined price in the future based upon terms matching the import contract to offset risk to changes in the underlying spot exchange rate (i.e., GBP depreciation against EUR). 3. Identify A, B, and C in the following diagram, as in Exhibit 1, for the interest rate swap in Scenario 2 for Esterr Inc.

\begin{center}
\includegraphics[max width=\textwidth]{2023_05_04_36535b8d80b32081d422g-170(1)}
\end{center}

\section{Solution}
\begin{center}
\includegraphics[max width=\textwidth]{2023_05_04_36535b8d80b32081d422g-170}
\end{center}

\begin{enumerate}
  \setcounter{enumi}{4}
  \item Identify and describe the derivative features for the Esterr Inc. interest rate swap using the following term sheet, as in Exhibit 2.
\end{enumerate}

\section{Interest Rate Swap Term Sheet}
\begin{center}
\begin{tabular}{ll}
\hline
Start Date: & [Spot start] \\
Maturity Date: & [Three years and six months from Start Date] \\
Notional Principal: & CAD250,000,000 \\
Fixed-Rate Payer: & Esterr Inc. \\
Fixed Rate: & $2.05 \%$ on a semiannual, Act/365 basis \\
Floating-Rate Payer: & [Financial Intermediary] \\
Floating Rate: & Three-month Canadian Overnight Repo Rate Average \\
 & (CORRA) as published each Business Day by the Bank \\
 & of Canada \\
Payment Dates: & Semiannual exchange on a net basis \\
Business Days: & Toronto \\
Documentation: & ISDA Agreement and credit terms to match Esterr Inc. \\
 & Term Loan \\
\hline
\end{tabular}
\end{center}

A. Underlying:

B. Counterparties: and

C. Contract size:

\section{Contract type:}
\section{Solution}
A. Underlying: Interest rate (Canadian market reference rate, CORRA)

B. Counterparties: Esterr Inc. and Financial Intermediary

C. Contract size: CAD250,000,000

D. Contract type: Firm commitment (interest rate swap)

\begin{enumerate}
  \setcounter{enumi}{5}
  \item Identify which example corresponds to each derivative underlying type.
A. Soft commodities
  \item Aluminum futures
B. Hard commodities
  \item SOFR futures
C. Neither soft nor hard commodities
  \item Soybean options
\end{enumerate}

\section{Solution}
\begin{enumerate}
  \item B is correct. Aluminum futures are an example of a metals contract, which is a derivative with a hard commodity underlying.

  \item $C$ is correct. SOFR futures are an example of an interest rate contract, not a commodity-based derivative contract.

  \item A is correct. Soybean options are an example of a derivative contract with an agricultural, or soft, commodity underlying.

\end{enumerate}

\section{DERIVATIVE MARKETS}
describe the basic features of derivative markets, and contrast

over-the-counter and exchange-traded derivative markets

Derivatives usage was historically dominated by exchange-traded futures markets in soft and hard commodities. Derivatives were expanded to over-the-counter (OTC) financial derivatives in interest rates and currencies in the 1980s, then credit derivatives in the 1990s.

\section{Over-the-Counter (OTC) Derivative Markets}
OTC markets can be formal organizations, such as NASDAQ, or informal networks of parties that buy from and sell to one another, as in the US fixed-income markets. OTC derivative markets involve contracts entered between derivatives end users and dealers, or financial intermediaries, such as commercial banks or investment banks. OTC dealers, known as market makers, typically enter into offsetting bilateral transactions with one another to transfer risk to other parties. The terms of OTC contracts can be customized to match a desired risk exposure profile. This flexibility is important to end users seeking to hedge a specific existing or anticipated underlying exposure based upon non-standard terms. The structure of the OTC derivative markets is shown in Exhibit 4.

\section{Exhibit 4: Over-the-Counter Derivative Markets}
\begin{center}
\includegraphics[max width=\textwidth]{2023_05_04_36535b8d80b32081d422g-172}
\end{center}

\section{Exchange-Traded Derivative (ETD) Markets}
An exchange-traded derivative (ETD) includes futures, options, and other financial contracts available on exchanges, such as the National Stock Exchange (NSE) in India or the Brasil, Bolsa, Balcão (B3) exchange in Brazil. ETD contracts are more formal and standardized, which facilitates a more liquid and transparent market. Terms and conditions-such as the size of each contract, type, quality, and location of underlying for commodities and maturity date-are set by the exchange. Exhibit 5 shows the key terms of the London Metals Exchange (LME) lithium futures contract described earlier.

\section{Exhibit 5: LME Lithium Futures Contract Specifications}
\begin{center}
\begin{tabular}{ll}
\hline
Contract Maturities: & Monthly [from 1 month to 15 months] \\
Contract Size: & One metric ton \\
Delivery Type: & Cash settled \\
Price Quotation: & USD per metric ton \\
Final Maturity: & Last LME business day of contract month \\
Daily Settlement: & LME Trading Operations calculates daily settlement values \\
 & based on its published procedures \\
Final Settlement: & Based on the reported arithmetic monthly average of \\
 & Fastmarkets' lithium hydroxide monohydrate $56.5 \%$ LiOH. \\
 & H2O min, battery grade, spot price cif China, Japan, and \\
 & Korea, USD/kg price, which is available from Fastmarkets \\
from 16.30 London time on the last trading day &  \\
\end{tabular}
\end{center}

Exchange memberships are held by market makers (or dealers) that stand ready to buy at one price and sell at a higher price. With standard terms and an active market, they are often able to buy and sell simultaneously, earning a small bid-offer spread. When dealers cannot find a counterparty, risk takers (sometimes referred to as speculators) are often willing to take on exposure to changes in the underlying price.

Standardization also leads to an efficient clearing and settlement process. Clearing is the exchange's process of verifying the execution of a transaction, exchange of payments, and recording the participants. Settlement involves the payment of final amounts and/or delivery of securities or physical commodities between the counterparties based upon exchange rules. Derivative exchanges require collateral on deposit upon inception and during the life of a trade in order to minimize counterparty credit risk. This deposit is paid by each counterparty via a financial intermediary to the exchange, which then provides a guarantee against counterparty default. Finally, ETD markets have transparency, which means that full information on all transactions is disclosed to exchanges and national regulators.

OTC and ETD markets differ in several ways. OTC derivatives offer greater flexibility and customization than ETD. However, OTC instruments have less transparency, usually involve more counterparty risk, and may be less liquid. ETD contracts are more standardized, have lower trading and transaction costs, and may be more liquid than those in OTC markets, but their greater transparency and reduced flexibility may be a disadvantage to some market participants. The structure of the ETD markets is shown in Exhibit 6.

\section{Exhibit 6: Exchange-Traded Derivative Markets}
\begin{center}
\includegraphics[max width=\textwidth]{2023_05_04_36535b8d80b32081d422g-173}
\end{center}

\section{Central Clearing}
Following the 2008 global financial crisis, global regulatory authorities instituted a central clearing mandate for most OTC derivatives. This mandate requires that a central counterparty (CCP) assume the credit risk between derivative counterparties, one of which is typically a financial intermediary. CCPs provide clearing and settlement for most derivative contracts. Issuers and investors are able to maintain the flexibility and customization available in the OTC markets when facing a financial intermediary, while the management of credit risk, clearing, and settlement of transactions between financial intermediaries occurs in a way similar to ETD markets. This arrangement seeks to benefit from the transparency, standardization, and risk reduction features of ETD markets. However, the systemic credit risk transfer from financial intermediaries to CCPs also leads to centralization and concentration of risks. Proper safeguards must be in place to avoid excessive risk being held in CCPs.

Exhibit 7 shows the central clearing process for interest rate swaps which also applies to other swaps and derivative instruments. Under central clearing, a derivatives trade is executed in Step 1 on a swap execution facility (SEF), a swap trading platform accessed by multiple dealers. The original SEF transaction details are shared with

\section{Derivative Instrument and Derivative Market Features}
the CCP in Step 2, and the CCP replaces the existing trade in Step 3. This novation process substitutes the initial SEF contract with identical trades facing the CCP. The CCP serves as counterparty for both financial intermediaries, eliminating bilateral counterparty credit risk and providing clearing and settlement services.

\section{Exhibit 7: Central Clearing for Interest Rate Swaps}
Step 1: Trade executed on an SEF

\begin{center}
\includegraphics[max width=\textwidth]{2023_05_04_36535b8d80b32081d422g-174(2)}
\end{center}

Step 2: SEF trade information submitted to CCP

\begin{center}
\includegraphics[max width=\textwidth]{2023_05_04_36535b8d80b32081d422g-174(1)}
\end{center}

Step 3: CCP replaces (novates) existing trade, acting as new counterparty to both financial intermediaries

\begin{center}
\includegraphics[max width=\textwidth]{2023_05_04_36535b8d80b32081d422g-174}
\end{center}

\section{Investor Scenarios}
In this section, we assess the most appropriate derivative markets for the scenarios presented in Section 2.7.

\section{Scenario 1. Hightest Capital.}
Hightest's index option contract would most likely be traded on the ETD derivative market. The trade has a standard size, exercise price, and maturity date.

\section{Scenario 2. Esterr Inc.}
Esterr's interest rate swap is likely to be traded in the OTC market. The swap contract terms are tailored to match the payment dates and remaining maturity of Esterr's term loan. Esterr's counterparty will be a financial intermediary that executes the offsetting hedge on an SEF and then novates the original SEF trade to face a CCP, which serves as the credit risk intermediary between dealers.

\section{KNOWLEDGE CHECK}
\section{Derivative Markets}
\begin{enumerate}
  \item Describe the risk transfer process in OTC derivative markets.
\end{enumerate}

\section{Solution}
OTC dealers, known as market makers, typically enter into offsetting transactions with one another to transfer the risk of derivative contracts entered with end users.

\begin{enumerate}
  \setcounter{enumi}{1}
  \item Identify which of the following derivative markets corresponds to the following characteristics.
A. ETD
  \item Standardized contracts
B. OTC
  \item Includes market makers
C. Both ETD and OTC
  \item Greater confidentiality
\end{enumerate}

\section{Solution}
\begin{enumerate}
  \item A-ETD markets use standardized contracts.

  \item C-Both ETD and OTC markets use market makers.

  \item B-OTC markets have greater privacy.

  \item Determine the correct answers to fill in the blanks: involves the payment of final amounts and/or delivery of securities or physical commodities, while is the process of verifying the execution of a transaction, exchange of payments, and recording the participants.

\end{enumerate}

\section{Solution}
Settlement involves the payment of final amounts and/or delivery of securities or physical commodities, while clearing is the process of verifying the execution of a transaction, exchange of payments, and recording the participants.

\begin{enumerate}
  \setcounter{enumi}{3}
  \item Identify one potential risk concern about the central clearing of derivatives.
\end{enumerate}

\section{Solution}
The central clearing mandate transfers the systemic risk of derivatives transactions from the counterparties, typically financial intermediaries, to the CCPs. One concern is the centralization and concentration of risks in CCPs. Careful oversight must occur to ensure that these risks are properly managed. 5. Describe the steps for clearing a credit default swap.

\section{Solution}
The counterparties are financial intermediaries that first execute the trade on an SEF (swap execution facility). Then, trade details are shared with a CCP; the novation process substitutes the original contract with another where the CCP steps into the trade and acts as the new counterparty for each original party. The CCP clears and settles the trade.

\section{PRACTICE PROBLEMS}
Montau AG is a German capital goods producer that manufactures its products domestically and delivers its products to clients globally. Montau's global sales manager shares the following draft commercial contract with his Treasury team:

\section{Montau AG Commercial Export Contract}
\begin{center}
\begin{tabular}{ll}
\hline
Contract Date: & [Today] \\
Goods Seller: & Montau AG, Frankfurt, Germany \\
Goods Buyer: & Jeon Inc., Seoul, Korea \\
Description of Goods: & A-Series Laser Cutting Machine \\
Quantity: & One \\
Delivery Terms: & Freight on Board (FOB), Busan Korea with all shipping, tax \\
 & and delivery costs payable by Goods Buyer \\
[75 Days from Contract Date]
 &  \\
Delivery Date: & $100 \%$ of Contract Price payable by Goods Buyer to Good \\
Payment Terms: & Seller on Delivery Date \\
 & KRW650,000,000 \\
Contract Price: &  \\
\end{tabular}
\end{center}

Montau AG's Treasury manager is tasked with addressing the financial risk of this prospective transaction.

\begin{enumerate}
  \item Which of the following statements best describes why Montau AG should consider a derivative rather than a spot market transaction to manage the financial risk of this commercial contract?
\end{enumerate}

A. Montau AG is selling a machine at a contract price in KRW and incurs costs based in EUR.

B. Montau AG faces a 75-day timing difference between the commercial contract date and the delivery date when Montau AG is paid for the machine in KRW.

C. Montau AG is unable to sell KRW today in order to offset the contract price of machinery delivered to Jeon Inc.

\begin{enumerate}
  \setcounter{enumi}{1}
  \item Which of the following types of derivative and underlyings are best suited to hedge Montau's financial risk under the commercial transaction?
\end{enumerate}

A. Montau AG should consider a firm commitment derivative with currency as an underlying, specifically the sale of KRW at a fixed EUR price.

B. Montau AG should consider a contingent claim derivative with the price of the machine as its underlying, specifically an A-series laser cutting machine.

C. Montau AG should consider a contingent claim derivative with currency as an underlying, specifically the sale of EUR at a fixed KRW price. 3. Identify A, B, and $\mathrm{C}$ in the correct order in the following diagram, as in Exhibit 1, for the derivative to hedge Montau's financial risk under the commercial transaction.

\section{Exhibit 1}
Montau AG

\begin{center}
\includegraphics[max width=\textwidth]{2023_05_04_36535b8d80b32081d422g-178}
\end{center}

A

A. A: Financial intermediary, B: KRW650,000,000, C: Fixed EUR amount

B. A: Jeon Inc., B: KRW650,000,000, C: Fixed EUR amount

C. A: Financial intermediary, B: Fixed EUR amount, C: KRW650,000,000.

\begin{enumerate}
  \setcounter{enumi}{3}
  \item Which of the following statements about the most appropriate derivative market to hedge Montau AG's financial risk under the commercial contract is most accurate?
\end{enumerate}

A. The OTC market is most appropriate for Montau, as it is able to customize the contract to match its desired risk exposure profile.

B. The ETD market is most appropriate for Montau, as it offers a standardized and transparent contract to match its desired risk exposure profile.

C. Both the ETD and OTC markets are appropriate for Montau AG to hedge its financial risk under the transaction, so it should choose the market with the best price.

\section{SOLUTIONS}
\begin{enumerate}
  \item B is correct. A 75-day timing difference exists between the commercial contract date and the delivery date when Montau AG is paid for the machine in KRW. A is true but does not explain why the use of a derivative is preferable to a spot market transaction. If as in C Montau were to sell the KRW it receives and buy EUR in a spot market transaction on the delivery date, it would be exposed to unfavorable changes in the KRW/EUR exchange rate over the 75-day period. A derivative contract in which the underlying KRW/EUR forward rate is agreed today and exchanged on the delivery date allows Montau to hedge or offset the EUR value of the future KRW payment. The derivative is therefore a more suitable contract to address the financial risk of the commercial transaction than a spot market sale of KRW.

  \item A is correct. The derivative best suited to hedge Montau's financial risk is a firm commitment derivative in which a pre-determined amount is exchanged at settlement. The derivative underlying should be currencies, specifically the sale of KRW at a fixed EUR price in the future to offset or hedge the financial risk of the commercial contract. The machine price referenced under B is not considered an underlying, and C hedges the opposite of Montau's underlying exposure.

  \item $\mathrm{C}$ is correct as per the following diagram:

\end{enumerate}

\section{Exhibit 1}
Montau AG

\begin{center}
\includegraphics[max width=\textwidth]{2023_05_04_36535b8d80b32081d422g-179}
\end{center}

\begin{enumerate}
  \setcounter{enumi}{3}
  \item A is correct. The OTC market is most appropriate for Montau, as OTC contracts may be customized to match Montau's desired risk exposure profile. This is important to end users seeking to hedge a specific underlying exposure based upon non-standard terms. Montau would be unlikely to find an ETD contract under B that matches the exact size and maturity date of its desired hedge, which also makes $\mathrm{C}$ incorrect.
\end{enumerate}

\section{LEARNING MODULE 2}
\section{Forward Commitment and Contingent Claim Features and Instruments}
\section{LEARNING OUTCOME}
\begin{center}
\begin{tabular}{c|l}
Mastery & The candidate should be able to: \\
$\square$ & $\begin{array}{l}\text { define forward contracts, futures contracts, swaps, options (calls and } \\ \text { puts), and credit derivatives and compare their basic characteristics } \\ \text { determine the value at expiration and profit from a long or a short } \\ \text { position in a call or put option } \\ \text { contrast forward commitments with contingent claims }\end{array}$ \\
$\square$ &  \\
\end{tabular}
\end{center}

\section{INTRODUCTION}
An earlier lesson established a derivative as a financial instrument that derives its performance from an underlying asset, index, or other financial variable, such as equity price volatility. Primary derivative types include a firm commitment in which a predetermined amount is agreed to be exchanged between counterparties at settlement and a contingent claim in which one of the counterparties determines whether and when the trade will settle. The following lessons define and compare the basic features of forward commitments and contingent claims and explain how to calculate their values at maturity.

\section{Summary}
\begin{itemize}
  \item Forwards, futures, and swaps represent firm commitments, or derivative contracts that require counterparties to exchange an underlying in the future based on an agreed-on price.

  \item Forwards are a flexible over-the-counter (OTC) derivative instrument, while futures are standardized and traded on an exchange with a daily settlement of contract gains and losses.

  \item Swap contracts are a firm commitment to exchange a series of cash flows in the future. Interest rate swaps are the most common type and involve the exchange of fixed interest payments for floating interest payments. - Option contracts are contingent claims in which one of the counterparties determines whether and when a trade will settle. The option buyer pays a premium to the seller for the right to transact the underlying in the future at a pre-agreed exercise price.

  \item Credit default swap (CDS) contracts allow an investor to manage the risk of loss from issuer default separately from a cash bond.

  \item Market participants often create similar exposures to an underlying using firm commitments and contingent claims, although these derivative instrument types involve different payoff and profit profiles.

\end{itemize}

\section{LEARNING MODULE PRE-TEST}
\begin{enumerate}
  \item Describe two differences between a forward and a futures contract.
\end{enumerate}

\section{Solution:}
Difference 1. A futures contract is an exchange-traded derivative (ETD) with standardized terms set by the exchange, while a forward contract is an over-the-counter (OTC) derivative that can be customized to match a desired risk exposure profile.

Difference 2. A futures contract requires the daily settlement of gains and losses using futures margin accounts based on the mark to market. Credit terms of a forward contract are flexible and agreed between counterparties.

\begin{enumerate}
  \setcounter{enumi}{1}
  \item Identify which example fits each of the following firm commitments:
\end{enumerate}

A. Futures contract purchaser

B. Forward contract seller

C. Fixed-rate payer on an interest rate swap 1. Agrees to make a single exchange in the future at a pre-agreed price under an OTC contract

\begin{enumerate}
  \setcounter{enumi}{1}
  \item Agrees to a single exchange in the future based on standardized terms set by an exchange

  \item Agrees to a series of exchanges of interest fixed for floating interest payments

\end{enumerate}

\section{Solution:}
\begin{enumerate}
  \item B is correct. A forward contract seller agrees to make a single exchange in the future at a pre-agreed price under an OTC contract.

  \item A is correct. A futures contract purchaser agrees to a single exchange in the future based on standardized terms set by an exchange.

  \item C is correct. A fixed-rate payer on an interest rate swap agrees to a series of exchanges of fixed for floating interest payments.

  \item Identify which example fits each of the following contingent claims:
A. Put option purchaser

  \item Seeks to gain from an increase in the underlying price

\end{enumerate}

B. Call option purchaser

\begin{enumerate}
  \setcounter{enumi}{1}
  \item Allows the option to expire at maturity of the underlying price is above the exercise price
\end{enumerate}

C. Both a put option purchaser and a 3. Pays an option premium to the option call option purchaser seller when the contract is agreed on

\section{Solution:}
\begin{enumerate}
  \item B is correct. A call option purchaser seeks to gain from an increase in the underlying price.

  \item A is correct. A put option purchaser will allow an option to expire at maturity without exercise if the underlying price is above the exercise price.

  \item $C$ is correct. Both a put option purchaser and a call option purchaser will pay a premium to the option seller when the option contract is executed.

  \item Describe the contingent claim feature of a credit default swap.

\end{enumerate}

\section{Solution:}
A credit default swap (CDS) contract involves a payment from the credit protection seller to the credit protection buyer if the reference entity faces an event of default. This payment is said to be contingent on the default of the underlying reference entity.

\begin{enumerate}
  \setcounter{enumi}{4}
  \item Determine the correct answers to fill in the blanks: A call option buyer will exercise the option at maturity if the price exceeds the price but earns a profit on the transaction only if the gain upon exercise exceeds the
\end{enumerate}

\section{Solution:}
A call option buyer will exercise the option at maturity if the spot price exceeds the exercise price but earns a profit on the transaction only if the gain upon exercise exceeds the premium.

\begin{enumerate}
  \setcounter{enumi}{5}
  \item Describe the point at which a long forward position and a short put option with an exercise price, $X$, equal to the forward price, $F_{0}(T)$ have the same profit.
\end{enumerate}

\section{Solution:}
A long forward and short put position has the same profit when the put expires unexercised, and the forward buyer has a gain of $S_{T}-F_{0}(T)=p_{0}$. A long forward position gains from a rise in the underlying price with a payoff/ profit of $\left[S_{T}-F_{0}(T)\right]$, and a short put position has a profit of $\Pi=-\max (0, X$ $\left.-S_{T}\right)+p_{0}$ when the exercise price is equal to the forward price: $X=F_{0}(T)$.

\section{FORWARDS}
Forwards, futures, and swaps are the most common derivative contracts which represent a firm commitment. This firm commitment is an obligation of both counterparties to perform under the terms of the derivative contract. Key common features of this type of derivative include the following:

\begin{itemize}
  \item A specific contract size

  \item A specific underlying

  \item One or more exchanges of cash flows or underlying on a specific future date or dates

  \item Exchange(s) based on a pre-agreed price Despite their similarities, forwards, futures, and swaps each have different features, which are the subject of this lesson.

\end{itemize}

A forward contract is an over-the-counter (OTC) derivative in which two counterparties agree that one counterparty, the buyer, will purchase an underlying from the other counterparty, the seller, in the future at a pre-agreed fixed price. As noted earlier, OTC derivatives offer greater flexibility and customization than exchange-traded derivatives (ETD), but also usually involve more counterparty risk. Forward contracts are advantageous for derivative end users seeking to hedge an existing or forecasted underlying exposure based on specific terms. For example, an importer may enter a forward contract to buy the foreign currency needed to satisfy the commercial terms of a future goods delivery contract. Forward contracts are more flexible as to the size, underlying details, maturity, and/or credit terms than a similar ETD. A forward contract buyer has a long position and will therefore benefit from price appreciation of the underlying over the life of the contract.

To gain a better understanding of forwards, we must examine their payoff profile. Assume a forward contract is agreed at time $t=0$ and matures at time $T$. At time $t=$ 0 , the counterparties do not exchange a payment upfront but, rather, agree on delivery of the underlying at time $T$ for a forward price of $F_{0}(T)$. The subscript refers to the date on which the underlying price in the future is set $(t=0)$, and the $T$ in parentheses refers to the date of exchange $(t=T)$. The spot price of the underlying at time $T$ is $S_{T}$. Exhibit 1 shows the payoff from the forward buyer's perspective, which is a long forward position. Note that the payoff equals the profit, as no upfront payment is made.

\section{Exhibit 1: Long Forward (Forward Buyer) Payoff Profile}
\begin{center}
\includegraphics[max width=\textwidth]{2023_05_04_36535b8d80b32081d422g-184}
\end{center}

The symmetric payoff profile shown in Exhibit 1 is a common feature of firm commitments. Since the derivative price is a linear function of the underlying, firm commitments are also referred to as linear derivatives. At time $T$, the transaction is settled based on the difference between the forward price, $F_{0}(T)$, and the underlying price of $S_{T}$, or $\left[S_{T}-F_{0}(T)\right]$ from the buyer's perspective. That is, the buyer realizes a gain if she is able to take delivery of the underlying at a market value, $S_{T}$, that exceeds the pre-agreed price, $F_{0}(T)$. If the forward price exceeds the current market value $\left[F_{0}(T)>S_{T}\right]$, the buyer realizes a loss and must either take delivery of an asset at a loss of $\left[F_{0}(T)-S_{T}\right]$ or pay the seller this amount in cash. Forward contracts usually involve a single exchange in the future, as in Example 1.

\section{EXAMPLE 1}
\section{Forward Gold Purchase}
An investor, Procam Investments, enters a cash-settled forward contract with a financial intermediary to buy 100 ounces of gold at a forward price, $F_{0}(\mathrm{~T})$, of $\$ 1,792.13$ per ounce in three months.

\begin{enumerate}
  \item Today's spot gold price $\left(S_{0}\right)$ is $\$ 1,770$ per ounce.

  \item At contract maturity, the gold price $\left(S_{T}\right)$ is $\$ 1,780.50$ per ounce.

  \item The payoff, $S_{T}-F_{0}(T)$, is $-\$ 11.63=\$ 1,780.50-\$ 1,792.13$ per ounce.

  \item Procam (the buyer) must pay the financial intermediary (the seller) $\$ 1,163(=100 \times \$ 11.63)$ to settle the forward contract at maturity.

\end{enumerate}

\begin{center}
\includegraphics[max width=\textwidth]{2023_05_04_36535b8d80b32081d422g-185}
\end{center}

The contract may specify either the actual delivery of the underlying or a cash settlement. The settlement amount is equal to $\left[S_{T}-F_{0}(T)\right]$ from a buyer's perspective and $-\left[S_{T}-F_{0}(T)\right]=\left[F_{0}(T)-S_{T}\right]$ from a seller's perspective. Note that a buyer would have to pay $S_{0}$ at $t=0$ and realize a return of $\left(S_{T}-S_{0}\right)$ at time $T$ in order to create a similar exposure to the long forward position in the cash market.

\section{KNOWLEDGE CHECK}
\section{Forward Contracts}
\begin{enumerate}
  \item Describe a scenario in which a forward contract has cash settlement of zero at maturity and neither counterparty has defaulted.
\end{enumerate}

\section{Solution:}
A forward contract will have a cash settlement of zero at maturity if $S_{T}=$ $F_{0}(T)$ or the payoff from the buyer's perspective is $\left[S_{T}-F_{0}(T)\right]=0$. This is often referred to as the breakeven point for the forward contract for both buyer and seller in the absence of transaction costs and is visually represented by the $x$-axis intercept of the profit line in Exhibit 1 .

\begin{enumerate}
  \setcounter{enumi}{1}
  \item Determine the correct answers to fill in the blanks: An oil producer enters a derivative contract with an investor to sell 1,000 barrels of oil in two months at a forward price of $\$ 64$ per barrel. If the spot oil price at maturity is $\$ 58.50$ per barrel, the investor realizes a at maturity equal to
\end{enumerate}

\section{Solution:}
The oil forward price, $F_{0}(T)$, under the contract equals $\$ 64$ per barrel. At contract maturity, the spot oil price $\left(S_{T}\right)$ is $\$ 58.50$ per barrel.

\begin{itemize}
  \item Investor payoff per barrel: $\left[S_{T}-F_{0}(T)\right]=\$ 58.50-\$ 64.00=-\$ 5.50$ per barrel.

  \item Total amount the investor pays the oil producer to settle the forward contract for 1,000 barrels at maturity:

\end{itemize}

$1,000 \times \$ 5.50=\$ 5,500$.

An oil producer enters a derivative contract with an investor to sell 1,000 barrels of oil in two months at a forward price of $\$ 64$ per barrel. If the spot oil price at maturity is $\$ 58.50$ per barrel, the investor realizes a loss at maturity equal to $\$ 5,500$.

\begin{enumerate}
  \setcounter{enumi}{2}
  \item Identify the most likely forward contract participants that correspond to the following statements:
A. forward contract purchaser
  \item Seeks to benefit from underlying price depreciation
B. Forward contract seller
  \item Realizes a gain if the initial spot price of the underlying, $S_{0}$, exceelds the for- ward price of $F_{0}(T)$
C. Neither a forward contract purchaser
  \item Receives a positive payoff at maturity nor a seller if the spot price, $S_{T}$, exceeds the forward price of $F_{0}(T)$
\end{enumerate}

\section{Solution:}
\begin{enumerate}
  \item The correct answer is B. A forward seller pays $\left[F_{0}(T)-S_{T}\right]$ to the forward contract buyer at maturity and therefore benefits as the underlying spot price $S_{T}$ declines over time.

  \item The correct answer is $C$. Neither the buyer nor the seller of a forward contract realizes a gain if the initial spot price, $S_{0}$, exceeds the forward price of $F_{0}(T)$, as settlement is based on the future spot price, $S_{T}$.

  \item The correct answer is $\mathrm{A}$. The forward contract buyer realizes a gain at maturity if $S_{T}>F_{0}(T)$.

\end{enumerate}

\section{FUTURES}
Futures contracts are forward contracts with standardized sizes, dates, and underlyings that trade on futures exchanges. Futures markets offer both greater liquidity and protection against loss by default by combining contract uniformity with an organized market with rules, regulations, and a central clearing facility. The futures contract buyer creates a long exposure to the underlying by agreeing to purchase the underlying at a later date at a pre-agreed price. The seller makes the opposite commitment, creating a short exposure to the underlying by agreeing to sell the underlying asset in the future at an agreed-on price. This agreed-on price is called the futures price, $f_{0}(T)$. The frequency of futures contract maturities, contract sizes, and other details are established by the exchange based on buyer and seller interest.

The most important feature of futures contracts is the daily settlement of gains and losses and the associated credit guarantee provided by the exchange through its clearinghouse. At the end of each day, the clearinghouse engages in a practice called mark to market (MTM), also known as the daily settlement. The clearinghouse determines an average of the final futures trading price of the day and designates that price as the end-of-day settlement price. All contracts are then said to be marked to the end-of-day settlement price.

As with forward contracts, no cash is exchanged when a futures contract is initiated by a buyer or seller. However, each counterparty must deposit a required minimum sum (or initial margin) into a futures margin account held at the exchange that is used by the clearinghouse to settle the daily mark to market. Futures contracts must be executed with specialized financial intermediaries that clear and settle payments at the exchange on behalf of counterparties, as shown in Example 2.

\section{EXAMPLE 2}
\section{Purchase of a Gold Futures Contract}
\begin{center}
\includegraphics[max width=\textwidth]{2023_05_04_36535b8d80b32081d422g-187}
\end{center}

As in Example 1, Procam Investments enters a cash-settled contract to buy 100 ounces of gold at a price of $\$ 1,792.13$ per ounce in three months. Instead of the forward in Example 1, Procam purchases a futures contract $\left[f_{0}(T)=\$ 1,792.13\right]$ on the exchange via a financial intermediary. London Metals Exchange rules require an initial cash margin of $\$ 4,950$ per gold contract (100 ounces) sold or purchased:

\begin{itemize}
  \item Procam deposits $\$ 4,950$ in required initial margin with the exchange.

  \item Today's spot gold price $\left(S_{0}\right)$ is $\$ 1,770$ per ounce, and the opening gold futures price, $f_{0}(T)$, is $\$ 1,792.13$ per ounce.

  \item At today's close, the gold futures price, $f_{1}(T)$, settles at $\$ 1,797.13$ per ounce.

  \item Procam realizes a $\$ 500 \mathrm{MTM}$ gain, or $\$ 5$ per ounce $\times 100$ ounces. It receives a $\$ 500$ futures margin account deposit from the clearinghouse.

  \item Procam's futures margin account has an ending balance for the day of $\$ 5,450$, or $\$ 4,950$ initial margin plus the $\$ 500$ MTM gain.

\end{itemize}

Each futures contract specifies a maintenance margin, or minimum balance set below the initial margin, that each contract buyer and seller must hold in the futures margin account from trade initiation until final settlement at maturity. The clearinghouse moves funds daily between the buyer and seller margin accounts, crediting the accounts of those with mark to market gains and charging those with mark-to-market losses.

For example, London Metals Exchange rules require a maintenance margin of $\$ 4,500$ per 100 ounce gold contract sold or purchased. Now consider the seller of a futures contract with a position that offsets that of Procam in Example 2.

\begin{itemize}
  \item Seller deposits $\$ 4,950$ in required initial margin with the exchange.

  \item Today's opening gold futures price, $f_{0}(T)$, is $\$ 1,792.13$ per ounce.

  \item At the close, the gold futures price, $f_{1}(T)$, settles at $\$ 1,797.13$ per ounce.

  \item Seller realizes a $\$ 500$ MTM loss, or $\$ 5$ per ounce $\times 100$ ounces; $\$ 500$ is deducted from its futures margin account by the clearinghouse.

  \item Seller's futures margin account has an ending balance of $\$ 4,450$, or $\$ 4,950$ initial margin less the $\$ 500$ MTM loss.

  \item Seller's margin account is $\$ 50$ below the required maintenance margin $(\$ 4,450-\$ 4,500)=-\$ 50$

\end{itemize}

The seller receives a margin call, or request to immediately deposit funds to return the account balance to the initial margin. The seller must deposit $\$ 500$ in order to bring the margin account back to the $\$ 4,950$ initial margin. The amount required to replenish the futures margin account is sometimes referred to as variation margin. If a counterparty fails to meet the margin call, it must close out the contract as soon as possible and cover any additional losses. If the counterparty cannot meet its obligations, the clearinghouse provides a guarantee that it will cover the loss itself by maintaining an insurance fund. Exhibit 2 shows the futures margining and settlement process, where $f_{0}(T)$ is the futures price at inception and $f_{t}(T)$ represents the futures price on day $t$.

\section{Exhibit 2: Futures Margin and Settlement Process}
\begin{center}
\includegraphics[max width=\textwidth]{2023_05_04_36535b8d80b32081d422g-188(4)}
\end{center}

\begin{center}
\includegraphics[max width=\textwidth]{2023_05_04_36535b8d80b32081d422g-188(1)}
\end{center}

\begin{center}
\includegraphics[max width=\textwidth]{2023_05_04_36535b8d80b32081d422g-188(5)}
\end{center}

Financial intermediary/ Exchange
\includegraphics[max width=\textwidth, center]{2023_05_04_36535b8d80b32081d422g-188}

\begin{center}
\includegraphics[max width=\textwidth]{2023_05_04_36535b8d80b32081d422g-188(6)}
\end{center}

\includegraphics[max width=\textwidth, center]{2023_05_04_36535b8d80b32081d422g-188(2)}
\includegraphics[max width=\textwidth, center]{2023_05_04_36535b8d80b32081d422g-188(3)}

Financia intermediary/ Exchange

Exchanges reserve the right to impose more strict requirements than standard futures margin account rules to limit potential losses from counterparty default. For example, for large positions or a significant increase in price volatility of the underlying, an exchange may increase required margins and/or make margin calls on an intraday basis. Some futures contracts also limit daily price changes. These rules, called price limits, establish a band relative to the previous day's settlement price within which all trades must occur. If market participants wish to trade at a price outside these bands, trading stops until two parties agree on a trade at a price within the prescribed range. In other cases, exchanges use what is called a circuit breaker to pause intraday trading for a brief period if a price limit is reached.

Similar to forward contracts, final settlement at maturity for futures contracts is based on the difference between the futures price, $f_{0}(T)$, and the underlying price of $S_{T}$, or $\left[S_{T}-f_{0}(T)\right]$ from the buyer's perspective. Because Procam has agreed to purchase gold now (at $t=T$ ) valued at $\$ 1,780.50$ per ounce at a price of $\$ 1,792.13$ per ounce, it owes $\$ 11.63$ per ounce, or $\$ 1,163[(\$ 1,780.50-\$ 1,792.13) \times 100$ oz.] under both the forward from Example 1 and this futures contract. The net payoff profile shown in Exhibit 1 is the same for a futures contract as for a forward assuming they have the same maturity date, with the difference being the timing of the cash flows due to the daily futures contract mark-to-market settlement, as shown in Example 3.

\section{EXAMPLE 3}
\section{Final Settlement of a Gold Futures Contract}
\begin{center}
\includegraphics[max width=\textwidth]{2023_05_04_36535b8d80b32081d422g-189}
\end{center}

For purposes of exposition, we compress the three months of gold futures price changes from Example 2 into six days of trading in the following spreadsheet:

\begin{center}
\begin{tabular}{lcl}
 & \multicolumn{1}{c}{Procam's Futures Margin Account} &  \\
\hline
Gold Contract & 100 & ounces \\
\# of Contracts & 1 &  \\
Initial Futures Price $f_{0}(T)$ & $\$ 1,792.13$ & per ounce \\
Initial Position Value & $\$ 179,213$ &  \\
Initial Margin & $\$ 4,950$ &  \\
Maintenance Margin & $\$ 4,500$ &  \\
\hline
\end{tabular}
\end{center}

\begin{center}
\begin{tabular}{lccccc}
\hline
Day & $\begin{array}{c}\text { Futures } \\ \text { Price }\end{array}$ & $\begin{array}{c}\text { Day Gain } \\ \text { (Loss) }\end{array}$ & $\begin{array}{c}\text { Total Gain } \\ \text { (Loss) }\end{array}$ & $\begin{array}{c}\text { Margin } \\ \text { Balance }\end{array}$ & Margin Call \\
\hline
$T-6$ & $\$ 1,792.13$ &  &  & $\$ 4,950$ &  \\
$T-5$ & $\$ 1,797.13$ & $\$ 500$ & $\$ 500$ & $\$ 5,450$ & - \\
$T-4$ & $\$ 1,786.25$ & $(\$ 1,088)$ & $(\$ 588)$ & $\$ 4,362$ & $\$ 588$ \\
$T-3$ & $\$ 1,782.19$ & $(\$ 406)$ & $(\$ 994)$ & $\$ 4,544$ & - \\
$T-2$ & $\$ 1,777.45$ & $(\$ 474)$ & $(\$ 1,468)$ & $\$ 4,070$ & $\$ 880$ \\
$T-1$ & $\$ 1,779.50$ & $\$ 205$ & $(\$ 1,263)$ & $\$ 5,155$ & - \\
$T$ & $\$ 1,780.50$ & $\$ 100$ & $(\$ 1,163)$ & $\$ 5,255$ & $\$ 1,468$ \\
\hline
\end{tabular}
\end{center}

\begin{center}
\begin{tabular}{lc}
\multicolumn{2}{c}{Procam's Results from the Futures Contract} \\
\hline
Gold Contract & 100 \\
\# of Contracts & 1 \\
Initial Futures Price $f_{0}(T)$ & $\$ 1,792.13$ \\
Initial Position Value & $\$ 179,213$ \\
Total Gain (Loss) & $(\$ 1,163)$ \\
Final Futures Price $F_{T}(T)$ & $\$ 1,780.50$ \\
Final Position Value & $\$ 178,050$ \\
Sum of Margin Calls & $\$ 1,468$ \\
Beginning less Ending Margin Balance & $(\$ 305)$ \\
Total Payments to Margin Account & $\$ 1,163$ \\
\hline
\end{tabular}
\end{center}

As the gold futures price, $f_{0}(t)$, falls and the margin balance drops below the $\$ 4,500$ maintenance margin minimum over time, Procam must immediately replenish its balance to the $\$ 4,950$ initial margin each time this occurs:

\begin{itemize}
  \item On Day $T-4$, gold futures fall $\$ 10.88$ per ounce $(\$ 1,797.13$ $\$ 1,786.25)$. Procam's balance falls $\$ 1,088(100 \times \$ 10.88)$ to $\$ 4,362$ $(\$ 5,450-\$ 1,088)$. Procam faces a margin call of $\$ 588(\$ 4,950$ $-\$ 4,362)$.

  \item On Day $T-2$, gold futures fall $\$ 4.74$ per ounce $(\$ 1,782.19$ $\$ 1,777.45)$. Procam's balance falls $\$ 474(100 \times \$ 4.74)$ to $\$ 4,070(\$ 4,544$ - \$474). Procam faces a margin call of $\$ 880(\$ 4,950-\$ 4,070)$.

\end{itemize}

On the final trading day, Procam has paid a total of $\$ 1,468(\$ 588+\$ 880)$ in margin calls and its futures margin account balance is $\$ 5,155$, or $\$ 205$ in excess of the $\$ 4,950$ initial margin. Procam has a cumulative MTM loss of $\$ 1,263$ $(\$ 1,468-\$ 205)$ at the start of the last trading day.

\begin{itemize}
  \item The prior day's gold futures settlement price, $f_{T-1}(T)$, is $\$ 1,779.50$.

  \item Gold futures rise $\$ 1$ per ounce on the final trading day to settle at $f_{T}(T)=\$ 1,780.50$ per ounce, the same as for the forward in Example 1 .

  \item The daily change in Procam's margin account is an increase of $\$ 100$ ( $\$ 1$ per ounce $\times 100$ ounces), bringing the margin account to $\$ 5,255$.

  \item Procam's futures margin balance of $\$ 5,255$ is returned at settlement for a net return of $\$ 305(\$ 5,255-\$ 4,950)$ in margin.

  \item Procam receives a net return of $\$ 305$ in margin at settlement for a cumulative loss upon settlement of $\$ 1,163(\$ 305-\$ 1,468)$.

\end{itemize}

Both the forward and futures contracts involve a $\$ 1,163$ settlement loss, but the forward is fully settled at maturity while the futures contract is settled based on the daily MTM. The time value of money principle suggests that these forward and futures settlements are not equivalent amounts of money, but the differences are small for shorter maturities and low interest rates. Also note that under the forward contract in Example 1, the financial intermediary bears counterparty risk to Procam for the forward settlement. In practice, financial intermediaries often use collateral arrangements similar to futures margining for forwards or other derivatives to reduce counterparty credit risk.

At maturity, the number of outstanding contracts, or open interest, is settled via cash or physical delivery. A counterparty may instead choose to enter an offsetting futures contract before expiration to close out a position; for example, a futures contract buyer may simply sell the open contract. The clearinghouse marks the contract to the current price relative to the previous settlement price and closes out the participant's position.

Futures contracts specify whether physical delivery of an underlying or cash settlement occurs at expiration. For example, a commodity futures contract with physical delivery obligates the seller to deliver an underlying asset of a specific type, amount, and quality to a designated location. The buyer must accept and pay for delivery, which ensures that the futures price converges with the spot price at expiration.

\section{KNOWLEDGE CHECK}
\section{Futures Contracts}
\begin{enumerate}
  \item Determine the correct answers to fill in the blanks: If a futures contract buyer's margin account falls below the , or minimum balance that each contract buyer and seller must hold in the account from trade initiation until final settlement, the buyer must immediately deposit funds to return the account balance to the
\end{enumerate}

\section{Solution:}
If a futures contract buyer's margin account falls below the maintenance margin, or minimum balance that each contract buyer and seller must hold in the account from trade initiation until final settlement, the buyer must immediately deposit funds to return the account balance to the initial margin.

\begin{enumerate}
  \setcounter{enumi}{1}
  \item Describe the mark-to-market process for a futures contract.
\end{enumerate}

\section{Solution:}
The exchange clearinghouse determines an average of the final futures prices of the day and designates that price as the end-of-day settlement price. The daily settlement of gains and losses takes place via each counterparty's futures margin account.

\begin{enumerate}
  \setcounter{enumi}{2}
  \item Identify these futures contract participants that correspond to the following statements:
\end{enumerate}

A Futures contract purchaser

B. Futures contract seller

C. Both a futures contract purchaser and a seller 1. Must make a margin deposit at contract initiation and maintain a minimum balance until maturity

\begin{enumerate}
  \setcounter{enumi}{1}
  \item Receives a margin account deposit if the futures price increases on any trading day

  \item Receives a positive payoff if the spot price $S_{T}$ is below the futures price, $f_{0}(T)$, at maturity

\end{enumerate}

\section{Solution:}
\begin{enumerate}
  \item The correct answer is $C$. Both futures contract buyers and sellers must deposit margin and maintain a minimum margin balance (maintenance margin) over the life of a contract.

  \item The correct answer is $\mathrm{A}$. The futures contract buyer realizes a mark-to-market gain on any trading day when the futures price increases and receives a corresponding margin account deposit. 3. The correct answer is B. The futures contract seller realizes a positive payoff at maturity if $S_{T}<f_{0}(T)$.

\end{enumerate}

\section{SWAPS}
A swap is a firm commitment under which two counterparties exchange a series of cash flows in the future. One set of cash flows is typically variable, or floating, and determined by a market reference rate that resets each period. The other cash flow stream is usually fixed or may vary based on a different underlying asset or rate. In this case, we refer to the counterparty paying the variable cash flows as the floating-rate payer (or fixed-rate receiver) and the counterparty paying fixed cash flows as the fixed-rate payer (floating-rate receiver), as shown in Exhibit 3.

Exhibit 3: Swap Mechanics

\begin{center}
\includegraphics[max width=\textwidth]{2023_05_04_36535b8d80b32081d422g-192}
\end{center}

Swaps and forwards have similar features, such as a start date, a maturity date, and an underlying that are negotiated between counterparties and specified in a contract. Interest rate swaps in which a fixed rate is exchanged for a floating rate are the most common swap contract. For each period in the future, the market reference rate (MRR) paid by the floating-rate payer resets, while the fixed rate (referred to as the swap rate) is constant, as shown in Exhibit 4.

\section{Exhibit 4: Swap as a Series of Forward Exchanges}
\begin{center}
\includegraphics[max width=\textwidth]{2023_05_04_36535b8d80b32081d422g-192(1)}
\end{center}

Counterparties usually exchange a net payment on fixed- and floating-rate payments on the swap as in Example 4.

\section{EXAMPLE 4}
\section{Fyleton Investments Swap}
Fyleton Investments has entered a five-year, receive-fixed GBP200 million interest rate swap with a financial intermediary to increase the duration of its fixed-income portfolio. Under terms of the swap, Fyleton has agreed to receive a semiannual GBP fixed rate of $2.25 \%$ and pay six-month MRR.

\begin{center}
\includegraphics[max width=\textwidth]{2023_05_04_36535b8d80b32081d422g-193}
\end{center}

Calculate the first swap cash flow exchange if six-month MRR is set at 1.95\%.

\begin{itemize}
  \item The financial intermediary owes Fyleton a fixed cash flow payment of GBP2,250,000 (= GBP200 million $\times 0.0225 / 2)$.

  \item Fyleton owes the financial intermediary a floating cash flow payment of GBP1,950,000 (= GBP 200 million $\times 0.0195 / 2$ ).

  \item The fixed and floating payments are netted against one another, and the net result is that the financial intermediary pays Fyleton GBP300,000 (= GBP2,250,000 - GBP1,950,000).

\end{itemize}

The notional principal is usually not exchanged but, rather, is used for fixed and floating interest payment calculations, as in Example 4. The example demonstrates how an investment manager might use an interest rate swap to change portfolio duration without trading bonds. Issuers often use swaps to alter the exposure profile of a liability, such as a term loan.

As with futures and forward contracts, no money is exchanged when a swap contract is initiated. The value of a swap at inception is therefore effectively zero, ignoring transaction costs. In an earlier fixed-income lesson, it was shown that implied forward rates may be derived from spot rates. Forward MRRs may be used to determine the expected future cash flows for the floating leg of an interest rate swap. The swap rate for the fixed leg payments is determined by solving for a constant fixed yield that equates the present value of the fixed and floating leg payments.

As market conditions change and time passes, the mark-to-market value of a swap will deviate from zero, resulting in a positive MTM to one counterparty and an offsetting negative MTM to the other. Swap credit terms are privately negotiated between counterparties in an over-the-counter agreement and may range from uncollateralized exposure, where each counterparty bears the full default risk of the other, to terms similar to futures margining for one or both counterparties. An event of counterparty default usually triggers swap termination and MTM settlement as for any other debt claim. Centrally cleared swaps between financial intermediaries and a central counterparty (CCP) include margin provisions similar to futures in order to standardize and reduce counterparty risk.

\section{KNOWLEDGE CHECK}
\section{Swap Contracts}
\begin{enumerate}
  \item Describe a similarity of and a difference between forward and swap contracts.
\end{enumerate}

\section{Solution:}
Similarities: Both forwards and swaps represent firm commitments with an initial value of zero where cash flows are exchanged in the future at a pre-agreed price.

Difference: Forwards usually involve one future exchange of cash flows, while a swap contract involves more than one exchange of future cash flows.

\begin{enumerate}
  \setcounter{enumi}{1}
  \item Determine the correct answers to fill in the blanks: Under a swap contract, we refer to the counterparty paying the variable cash flows as the and the counterparty paying fixed cash flows as the
\end{enumerate}

\section{Solution:}
Under a swap contract, we refer to the counterparty paying the variable cash flows as the floating-rate payer (or fixed-rate receiver) and the counterparty paying fixed cash flows as the fixed-rate payer (or floating-rate receiver).

\begin{enumerate}
  \setcounter{enumi}{2}
  \item Identify the interest rate swap participants that correspond to the following statements:
\end{enumerate}

A. Fixed-rate payer 1. Makes a payment each interest period based on a market reference rate

B. Floating-rate payer

\begin{enumerate}
  \setcounter{enumi}{1}
  \item May face a positive or a negative mark to market over the life of an interest rate swap contract
\end{enumerate}

C. Both a fixed-rate payer and a

\begin{enumerate}
  \setcounter{enumi}{2}
  \item Receives a net payment on the swap floating-rate payer for any interest period for which the market reference rate exceeds the fixed rate
\end{enumerate}

\section{Solution:}
\begin{enumerate}
  \item The correct answer is B. A floating-rate payer on a swap makes a payment each period based on a market reference rate.

  \item The correct answer is $C$. Both a fixed-rate payer and a floating-rate payer may face a positive MTM or negative MTM on a swap contract.

  \item The correct answer is A. A fixed-rate payer (also known as the floating-rate receiver) receives a net payment if the market reference rate exceeds the fixed rate for a given period.

\end{enumerate}

\section{OPTIONS}
\begin{center}
\includegraphics[max width=\textwidth]{2023_05_04_36535b8d80b32081d422g-195}
\end{center}

Contingent claims are a type of ETD or OTC derivative contract in which one of the counterparties has the right to determine whether a trade will settle based on the underlying value. Option contracts are the most common contingent claim. Similar to forwards and futures, options are derivative contracts between a buyer and a seller that specify an underlying, contract size, a pre-agreed execution price, and a maturity date. The option buyer has the right but not the obligation to transact the trade, and the option seller has the obligation to fulfill the transaction as chosen by the option buyer. As a consequence, the payoff to an option buyer is always zero or positive. It can never be negative.

Assume an option buyer pays a premium $\left(c_{0}\right)$ of $\$ 5$ at $t=0$ for the right-but not the obligation-to buy stock $S$ at time $T$ at a pre-agreed price $(X)$ of $\$ 30$. The option buyer's decision at maturity depends on the stock price at maturity $\left(S_{T}\right)$, as shown in the following two scenarios:

\section{Scenario 1: Transact $\left(S_{T}>X\right)$}
\begin{itemize}
  \item If $S_{T}=\$ 40$, the option buyer chooses to exercise the option and buy the stock for $X=\$ 30$.

  \item The option buyer gains $\$ 10\left[\left(S_{T}-X\right)=\$ 40-\$ 30\right]$ on the transaction.

  \item The option buyer realizes a $\$ 5$ profit $\left[\left(S_{T}-X\right)-c_{0}=(\$ 40-\$ 30)-\$ 5\right]$.

\end{itemize}

\section{Scenario 2: Do Not Transact $\left(S_{T}<X\right)$}
\begin{itemize}
  \item If $S_{T}=\$ 25$, the option buyer chooses not to exercise the option and buy the stock for $X=\$ 30$.

  \item The option buyer realizes a $\$ 5$ loss $\left(c_{0}=\$ 5\right)$.

\end{itemize}

Exhibit 5 shows the option mechanics for the case where the option buyer pays for the right to purchase an underlying stock $S$ at a pre-agreed execution price of $X$ in the future $(t=T)$.

\section{Exhibit 5: Option Mechanics}
\begin{center}
\includegraphics[max width=\textwidth]{2023_05_04_36535b8d80b32081d422g-196}
\end{center}

The decision to transact the underlying is referred to as an exercise, and the pre-agreed execution price is called the exercise price (or strike price). Option buyers may transact the underlying in the future at their sole discretion at the exercise price, a pre-agreed future spot price that may be above, at, or below the forward price, as shown in an earlier lesson. This right to exercise in the future has a value that is paid upfront to the option seller in the form of an option premium.

Option contract terms, such as the right to buy or sell, exercise price, maturity, and size, may either be agreed on between the counterparties in an over-the-counter transaction or executed on an exchange based on standardized terms. This lesson focuses on European options, or options that may be exercised only at maturity, although other option styles exist. American options, for example, may be exercised at any time from contract inception until maturity. Note the labels "European" and "American" refer not to where these options are used but, rather, to the difference in when they can be exercised.

Two primary option types exist-namely, (1) the right to buy an underlying known as a call option and (2) the right to sell the underlying, or a put option. An option buyer will exercise a call or put option only if it returns a positive payoff. If not exercised, the option expires worthless, and the option buyer's loss equals the premium paid.

One factor in an option's value prior to maturity $(t<T)$ is the option's exercise value at time $t$, which is referred to as an option's intrinsic value.

We can say a call option is in-the-money at time $t$ if the spot price, $S_{t}$, exceeds $X$, with an intrinsic value equal to $\left(S_{t}-X\right)$. Both out-of-the-money options (where $S_{t}$ $<X)$ and at-the-money options $\left(S_{t}=X\right)$ have zero intrinsic value, so their price, $c_{t}$, consists solely of time value.

Call option buyers will gain from a rise in the price of the underlying. Exhibit 6 shows the payoff and profit at maturity for a call option buyer.

\section{Exhibit 6: Long Call Payoff Profile}
\begin{center}
\includegraphics[max width=\textwidth]{2023_05_04_36535b8d80b32081d422g-197}
\end{center}

The option buyer pays a call option premium, $c_{0}$, at time $t=0$ to the option seller and has the right to purchase the underlying, $S_{T}$, at an exercise price of $X$ at time $t=T$. The exercise payoff $\left(S_{T}-X\right)$ is positive if $S_{T}>X$ and zero if $S_{T} \leq X$. The call option value at maturity, $c_{T}$, may be expressed as follows:

$$
c_{T}=\max \left(0, S_{T}-X\right) \text {. }
$$

The call option buyer's profit equals the payoff minus the call premium, $c_{0}$ (ignoring the time value of money in this lesson):

$$
\Pi=\max \left(0, S_{T}-X\right)-c_{0} \text {. }
$$

This asymmetric payoff profile is a common feature of contingent claims, which are sometimes referred to as non-linear derivatives.

\section{EXAMPLE 5}
\section{Hightest Capital-Call Option Purchase}
Hightest Capital purchases a call option on the S\&P 500 Health Care Select Sector Index (SIXV). This six-month exchange-traded option contract has a size of 100 index units and an exercise price of $\$ 1,240$ per unit versus the initial SIXV spot price of $\$ 1,180.95$. The option premium paid upfront is $\$ 24.85$ per unit, or $\$ 2,485(=\$ 24.85 \times 100)$. As the option nears maturity, a Hightest analyst is asked to determine the expected option payoff and profit per unit at maturity under different scenarios for the SIXV spot price on the exercise date. She compiles the following table:

$c_{0}=$ Call option premium $=\$ 24.85$ per unit.

= Exercise price at time $T=\$ 1,240$ per unit.

\begin{center}
\begin{tabular}{|c|c|c|c|}
\hline
$\begin{array}{c}\text { SIXV Spot Price } \\ \left(S_{T}\right)\end{array}$ & $\begin{array}{l}\text { Exercise } \\ \text { Price }(X)\end{array}$ & $\begin{array}{c}\text { Payoff } \\ \max \left(0, S_{T}-X\right)\end{array}$ & $\begin{array}{c}\text { Profit } \\ \max \left(0, S_{T}-X\right)-c_{0}\end{array}$ \\
\hline
$\$ 1,280$ & $\$ 1,240$ & $\$ 40$ & $\$ 15.15$ \\
\hline
$\$ 1.260$ & $\$ 1,240$ & $\$ 20$ & $\$ 105$ \\
\hline
\end{tabular}
\end{center}

$S_{T}=$ Spot price per unit at time $T$.

\begin{center}
\begin{tabular}{cccc}
\hline
$\begin{array}{c}\text { SIXV Spot Price } \\ \left(\boldsymbol{S}_{\boldsymbol{T}}\right)\end{array}$ & $\begin{array}{c}\text { Exercise } \\ \text { Price }(X)\end{array}$ & $\begin{array}{c}\text { Payoff } \\ \max \left(\mathbf{0}, \boldsymbol{S}_{\boldsymbol{T}}-\boldsymbol{X}\right)\end{array}$ & $\begin{array}{c}\text { Profit } \\ \max \left(\mathbf{0}, \boldsymbol{S}_{\boldsymbol{T}}-\boldsymbol{X}\right)-\boldsymbol{C}_{\mathbf{0}}\end{array}$ \\
\hline
$\$ 1,240$ & $\$ 1,240$ & $\$ 0$ & $-\$ 24.85$ \\
$\$ 1,220$ & $\$ 1,240$ & $\$ 0$ & $-\$ 24.85$ \\
\hline
\end{tabular}
\end{center}

Example 5 raises another question regarding an option's value prior to maturity $(t<T)$. The longer the time to option maturity, the more likely it is that a favorable change in the underlying price will increase both the likelihood and profitability of exercise. This time value of an option is always positive and declines to zero as an option reaches maturity.

In contrast to the call option buyer with unlimited upside potential and a loss limited to the premium paid, the call option seller receives a maximum of the premium and faces unlimited downside risk as the underlying appreciates above the exercise price. The short call payoff profile in Exhibit 7 is a mirror image of Exhibit 6

\section{Exhibit 7: Short Call Payoff Profile}
\begin{center}
\includegraphics[max width=\textwidth]{2023_05_04_36535b8d80b32081d422g-198}
\end{center}

The option buyer and seller payoff profiles demonstrate the one-sided nature of counterparty credit risk for contingent claims. That is, the option seller has no credit exposure to the option buyer once the premium is paid. However, the option buyer faces the counterparty credit risk of the option seller equal to the option payoff at maturity.

Put option buyers benefit from a lower underlying price by selling the underlying at a pre-agreed exercise price. A put option buyer exercises when the underlying price is below the exercise price $\left(S_{T}<X\right)$, as shown in Exhibit 8 .

\section{Exhibit 8: Long Put Payoff Profile}
\begin{center}
\includegraphics[max width=\textwidth]{2023_05_04_36535b8d80b32081d422g-199}
\end{center}

The put option buyer pays a premium of $p_{0}$ at inception to the option seller and will exercise the option only if $\left(X-S_{T}\right)>0$. As in Equations 1 and 2, we may show the long put option payoff and profit, $\Pi$, as follows:

$$
\begin{aligned}
& p_{T}=\max \left(0, X-S_{T}\right) . \\
& \Pi=\max \left(0, X-S_{T}\right)-p_{0} .
\end{aligned}
$$

The payoff or profit from a put option seller's perspective is the opposite of the put option buyer's gain or loss for a given underlying price at expiration. As in the case of the call option seller, the put option seller has a maximum gain equal to the premium. However, although a call option seller faces unlimited potential loss as the underlying appreciates beyond the exercise price, the put option seller's loss is usually limited because the underlying price cannot fall below zero. The short put option payoff and profit are as follows:

$-p_{T}=-\max \left(0, X-S_{T}\right)$

$$
\Pi=-\max \left(0, X-S_{T}\right)+p_{0} \text {. }
$$

\section{KNOWLEDGE CHECK}
\section{Option Contracts}
\begin{enumerate}
  \item Calculate the SIXV spot price at maturity from Example 5 at which Hightest Capital will reach a breakeven point and earn zero profit.
\end{enumerate}

\section{Solution:}
The call purchaser has a profit equal to $\max \left(0, S_{T}-X\right)-c_{0}$. In the case of Hightest in Example 5, the SIXV exercise price is $\$ 1,240$ and the initial call premium is $\$ 24.85$. The breakeven or zero profit point is therefore equal to $\$ 1,240+\$ 24.85$, or $\$ 1,264.85$. 2. A put option seller receives a $\$ 5$ premium for a put option sold on an underlying with an exercise price of $\$ 30$. What is the option seller's maximum profit under the contract? What is the maximum loss under the contract?

\section{Solution:}
A put option seller receives a $\$ 5$ premium $\left(p_{0}\right)$ for a put option sold on an underlying with an exercise price $(X)$ of $\$ 30$. The put option seller's profit is $\Pi=-\max \left(0, X-S_{T}\right)+p_{0}$. If the option is unexercised, $-\max \left(0, X-S_{T}\right)=0$ and the put seller earns $p_{0}=\$ 5$. If the option is exercised and $S_{T}=0$, then $\Pi$ $=-\max (0,30-0)+5=-\$ 25$. Therefore, the option seller's maximum profit under the contract is $\$ 5$ and the maximum loss under the contract is $\$ 25$.

\begin{enumerate}
  \setcounter{enumi}{2}
  \item Identify the option contract participants that correspond to the following statements:
\end{enumerate}

A. Put option seller 1. Has no counterparty credit risk to the option buyer once the upfront premium has been paid

B. Call option seller

\begin{enumerate}
  \setcounter{enumi}{1}
  \item Earns a profit equal to the premium if the underlying price at maturity is less than the exercise price
\end{enumerate}

C. Both a put option seller and a call 3. Earns a profit equal to the premium if option seller the underlying price at maturity exceeds the exercise price

\section{Solution:}
\begin{enumerate}
  \item The correct answer is C. An option seller has no counterparty credit risk to the option buyer once the upfront premium has been paid.

  \item The correct answer is B. A call option seller earns a profit equal to the premium if the underlying price at maturity is less than the exercise price. 3. The correct answer is A. A put option seller earns a profit equal to the premium if the underlying price at maturity is greater than the exercise price.

\end{enumerate}

\section{CREDIT DERIVATIVES}
Credit derivative contracts are based on a credit underlying, or the default risk of a single debt issuer or a group of debt issuers in an index. The most common credit derivative contract is a credit default swap. CDS contracts allow an investor to manage the risk of loss from issuer default separately from a cash bond. CDS contracts trade based on a credit spread (CDS credit spread) similar to that of a cash bond. Credit spreads depend on the probability of default (POD) and the loss given default (LGD), as shown in an earlier lesson. A higher credit spread (or higher likelihood of issuer financial distress) corresponds to a lower cash bond price, and vice versa.

Despite their name, CDS contracts are contingent claims that share some features of firm commitments. Unlike the call and put options discussed earlier, both the timing of exercise and payment upon exercise under a CDS contract vary depending on the underlying issuer(s). As in the case of a standard interest rate swap, a CDS contract priced at a par spread has a zero net present value, and the notional amount is not exchanged but, rather, serves as a basis for spread and settlement calculations. In a CDS contract, the credit protection buyer pays the credit protection seller to assume the risk of loss from the default of an underlying third-party issuer. If an issuer credit event occurs-usually defined as bankruptcy, failure to pay an obligation, or an involuntary debt restructuring-the credit protection seller pays the credit protection buyer to settle the contract. This contingent payment equals the issuer loss given default for the CDS contract notional amount. Exhibit 9 shows the periodic cash flows under a CDS contract.

\section{Exhibit 9: Periodic Payments under a Credit Default Swap}
\begin{center}
\includegraphics[max width=\textwidth]{2023_05_04_36535b8d80b32081d422g-201}
\end{center}

The underlying may be a corporate or sovereign issuer, an index of issuers, or a special purpose entity with a portfolio of loans, mortgages, or bonds.

A buyer can use a CDS contract as a hedge of existing credit exposure to the underlying issuer. The credit protection afforded by a CDS is similar to insurance for a buyer with an existing fixed-income exposure to the third-party issuer. The buyer may suffer a loss in value on its fixed-income exposure from the credit event but will receive a payment from the CDS contract that will offset that loss.

A credit protection buyer without the corresponding fixed-income exposure who buys a CDS is seeking to gain from higher credit spreads (which correspond to lower cash bond prices) for an underlying issuer and is therefore short credit risk.

The credit protection seller receives a periodic fixed spread payment in exchange for assuming the contingent risk of paying the credit protection buyer to offset the loss under a credit event. The contract structure is similar to insurance, with the periodic premium over the life of the contract agreed to upfront and with the timing and size of the loss under the credit event being unknown. The seller's position is therefore similar to that of a long risk position in the issuer's underlying bond.

For example, Exhibit 10 shows the CDS contract for an underlying issuer that experiences credit migration $(t=1)$ followed by a credit event $(t=2)$.

\section{Exhibit 10: CDS Contract with Credit Migration and Credit Event}
\begin{center}
\includegraphics[max width=\textwidth]{2023_05_04_36535b8d80b32081d422g-202}
\end{center}

The protection buyer agrees to pay a fixed spread of $100 \mathrm{bps}$ p.a. at $t=0$ for the contract term. As the issuer's CDS spread widens to $250 \mathrm{bps}$ p.a. at $t=1$, the buyer gains on the CDS contract due to the low fixed spread paid while the seller loses due to the low fixed spread received relative to the current higher CDS market spread. As for any fixed-income instrument, the effective duration of the remaining contract may be used to approximate the MTM change. An issuer credit event at $t=2$ causes the contract to terminate, and the seller must make a payment to the buyer equal to the percentage of loss (LGD) multiplied by the CDS contract notional.

\section{KNOWLEDGE CHECK}
\section{Credit Derivatives}
\begin{enumerate}
  \item Determine the correct answer to fill in the blanks: The contingent payment under a credit default swap equals the for the CDS notional amount specified in the contract.
\end{enumerate}

\section{Solution:}
The contingent payment under a credit default swap equals the loss given default for the CDS notional amount specified in the contract.

\begin{enumerate}
  \setcounter{enumi}{1}
  \item Describe how a credit protection seller's position is similar to that of an underlying cash bond investment.
\end{enumerate}

\section{Solution:}
A credit protection seller receives a periodic CDS spread payment in exchange for the contingent risk of payment to the buyer under an issuer cred- it event. A cash bond investor receives a periodic coupon that incorporates an issuer's credit spread in exchange for a potential loss if the issuer defaults. Under the CDS contract and the cash bond, this potential payment or loss equals the LGD. The credit protection seller's position is therefore similar to that of a long risk position in the issuer's underlying bond.

\begin{enumerate}
  \setcounter{enumi}{2}
  \item Identify the CDS contract participants that correspond to the following statements:
\end{enumerate}

A. Credit protection buyer

B. Credit protection seller

C. Both a credit protection buyer and a credit protection seller 1. Seeks to gain from higher issuer credit spreads

\begin{enumerate}
  \setcounter{enumi}{1}
  \item Enters into a derivative contract that transfers the risk of loss from a credit event of an underlying third-party issuer

  \item Faces an MTM gain on the CDS contract if the CDS spread of the underlying issuer falls

\end{enumerate}

\section{Solution:}
\begin{enumerate}
  \item The correct answer is A. A credit protection buyer seeks to gain from higher issuer credit spreads.

  \item The correct answer is $\mathrm{C}$. Both the credit protection buyer and credit protection seller enter into a derivative contract that transfers the risk of loss from a credit event of an underlying third-party issuer.

  \item The correct answer is B. A credit protection seller faces an MTM gain on the CDS contract if the CDS spread of the underlying issuer falls. The decline in the issuer's CDS spread versus the original fixed spread on the CDS contract means that the protection seller is receiving an above-market spread. This above-market CDS spread more than compensates the seller for the new, lower level of credit risk and results in an MTM gain.

\end{enumerate}

\section{FORWARD COMMITMENTS VS CONTINGENT CLAIMS}
A firm commitment requires both counterparties to perform under a derivative contract, while an option buyer can decide whether to perform under the contract at maturity depending on the underlying price relative to the exercise price. Market participants often create similar exposures to an underlying using these different derivative instrument types. For example, both a long forward position and a long call option position will gain from an increase in the underlying price. Exhibit 11 contrasts the payoff and profit of these two derivative contracts, where the exercise price, $X$, equals the forward price, $F_{0}(T)$.

\section{Exhibit 11: Long Forward and Long Call Option Payoff Profiles}
\begin{center}
\includegraphics[max width=\textwidth]{2023_05_04_36535b8d80b32081d422g-204}
\end{center}

As shown earlier, both the long forward and the call option payoffs increase as $S_{T}$ rises. In the case of a forward, this linear relationship is equal to $\left[S_{T}-F_{0}(T)\right]$, with the payoff equal to profit because no cash is exchanged at inception. For the buyer of a call option with an exercise price of $F_{0}(T)$, Equation 2 changes to

$$
\Pi=\max \left[0, S_{T}-F_{0}(T)\right]-c_{0} .
$$

Setting the forward payoff/profit $\left[S_{T}-F_{0}(T)\right]$ equal to the call option profit, $\Pi$, gives us the following relative profit profile between the forward and option:

\begin{itemize}
  \item $S_{T}-F_{0}(T)>-c_{0}$ Forward profit exceeds option profit

  \item $S_{T}-F_{0}(T)=-c_{0}$ Forward profit equals call option profit

  \item $S_{T}-F_{0}(T)<-c_{0}$ Option profit exceeds forward profit

\end{itemize}

The side-by-side comparison in Exhibit 12 between the forward and call option profit diagrams shows the long call option's similarity to a long position in the underlying with downside protection in exchange for paying a premium.

Another contingent claim that benefits from a rise in the underlying price is the sold put option. While the long call option and long forward payoffs both rise when the underlying price is above the exercise price, the put option seller's profit is limited to the upfront premium. Exhibit 12 contrasts the short put payoff and profit with a long forward if the exercise price, $X$, equals $F_{0}(T)$.

\section{Exhibit 12: Long Forward and Short Put Option Payoff Profile}
\begin{center}
\includegraphics[max width=\textwidth]{2023_05_04_36535b8d80b32081d422g-205}
\end{center}

We can compare the long forward payoff/profit of $\left[S_{T}-F_{0}(T)\right]$ to a modified version of Equation 6:

$$
\Pi=-\max \left[0, F_{0}(T)-S_{T}\right]+p_{0} .
$$

Setting the forward profit $\left[S_{T}-F_{0}(T)\right]$ equal to the put option profit, $\Pi$, gives us the following relative profit profile between the forward and option:

\begin{itemize}
  \item $S_{T}-F_{0}(T)>p_{0}$ Forward profit exceeds option profit

  \item $S_{T}-F_{0}(T)=p_{0}$ Forward profit equals option profit)

  \item $S_{T}-F_{0}(T)<p_{0}$ Option profit exceeds forward profit

\end{itemize}

The side-by-side comparison in Exhibit 13 of the forward and sold put option profit diagrams shows the sold put option's similarity to a long position in the underlying, with gains from price appreciation forgone in exchange for receiving a premium. The apparent symmetry between long call and short put positions and the long forward position will be examined in greater detail in a later lesson.

\section{KNOWLEDGE CHECK}
\section{Firm Commitments and Contingent Claims}
\begin{enumerate}
  \item Determine the correct answers to fill in the blanks: A\_\_\_ forward position, a \_\_\_ call option position, and a put option position will gain from a decrease in the underlying price.
\end{enumerate}

\section{Solution:}
A short forward position, a short call option position, and a long put option position will gain from a decrease in the underlying price.

\begin{enumerate}
  \setcounter{enumi}{1}
  \item Identify the derivative positions that correspond to the following profit profiles at maturity assuming that the exercise price $(X)$ equals $F_{0}(T)$ : A. Short forward position
\end{enumerate}

B. Long put position

C. Short call position 1.

\begin{center}
\includegraphics[max width=\textwidth]{2023_05_04_36535b8d80b32081d422g-206}
\end{center}

2.

\begin{center}
\includegraphics[max width=\textwidth]{2023_05_04_36535b8d80b32081d422g-206(2)}
\end{center}

3.

\begin{center}
\includegraphics[max width=\textwidth]{2023_05_04_36535b8d80b32081d422g-206(1)}
\end{center}

\section{Solution:}
\begin{enumerate}
  \item The correct answer is B.

  \item The correct answer is $A$.

  \item The correct answer is $C$.

  \item Describe the point at which a short forward and a long put with an exercise price $(X)$ equal to the forward price, $F_{0}(T)$ have the same profit.

\end{enumerate}

\section{Solution:}
The short forward and long put positions will have the same profit when $F_{0}(T)-S_{T}=-p_{0}$. A short forward position gains from a decline in the underlying price with a payoff/profit of $\left[F_{0}(T)-S_{T}\right]$, and a long put position has a profit of $\Pi=\max \left[0, F_{0}(T)-S_{T}\right]-p_{0}$, from Equation 4 , with an exercise price equal to the forward price, $X=F_{0}(T)$.

\section{PRACTICE PROBLEMS}
Biomian Limited is a Mumbai-based biotech company with common stock and listed futures and options on the National Stock Exchange (NSE). The Viswan Family Office (VFO) currently owns 10,000 Biomian common shares. VFO would like to reduce its long Biomian position and diversify its equity market exposure but will delay a cash sale of shares for tax reasons for six months.

\begin{enumerate}
  \item Which of the following derivative contracts available to VFO's chief investment officer is best suited to reduce exposure to a decline in Biomian's stock price in the next six months?
\end{enumerate}

A. A short put position on Biomian stock that expires in six months

B. A long call position on Biomian stock that expires in six months

C. A short futures position in Biomian stock that settles in six months

\begin{enumerate}
  \setcounter{enumi}{1}
  \item VFO's market strategist believes that Biomian's share price will rise over the next six months but would like to protect against a decline in Biomian's share price over the period. Which of the following positions is best suited for VFO to manage its existing Biomian exposure based on this view?
\end{enumerate}

A. A long put position on Biomian stock that expires in six months

B. A short call position on Biomian stock that expires in six months

C. A long futures position in Biomian stock that settles in six months

\begin{enumerate}
  \setcounter{enumi}{2}
  \item Assume that Biomian shares rise over the next six months. Which of the following statements about VFO's derivative strategies under this scenario is most accurate?
\end{enumerate}

A. A forward sale of Biomian shares in six months would be more profitable than purchasing the right to sell Biomian shares in six months.

B. Purchasing the right to sell Biomian shares in six months would be more profitable than a forward sale of Biomian shares in six months.

C. We do not have enough information to determine whether a forward sale or the right to sell Biomian shares will be more profitable in six months.

\begin{enumerate}
  \setcounter{enumi}{3}
  \item VFO's market strategist is considering a six-month call option strategy on the NIFTY 50 benchmark Indian stock market index to increase broad market equity exposure. The NIFTY 50 price today is INR15,200, and the strategist observes that a call option with a INR16,000 exercise price $(X)$ is trading at a premium of INR1,500. Which of the following represents the payoff and profit of this strategy just prior to maturity if the NIFTY 50 is trading at INR16,500?
\end{enumerate}

A. Payoff is INR500; profit is -INR1,000. B. Payoff is INR1,300; profit is INR800.

C. Payoff is INR1,300; profit is INR500.

\section{SOLUTIONS}
\begin{enumerate}
  \item C is correct. VFO may consider either a short futures position in (or a forward sale of) Biomian shares in six months to achieve its objective. This firm commitment allows VFO to offset its long position with a short position in six months at a pre-agreed price. The futures contract is an exchange-traded derivative with standardized terms set by the exchange and requires initial margin and daily settlement. Answers A and B are contingent claims that can both potentially increase, not decrease, VFO's exposure to Biomian stock in six months.

  \item A is correct. VFO should purchase a six-month put option on Biomian shares to manage its exposure based on the market strategist's view. This contingent claim grants VFO the right but not the obligation to sell Biomian shares at a pre-agreed exercise price in exchange for a premium. A put option buyer exercises the option at maturity when the underlying price is below the exercise price. This allows VFO to continue to benefit from a rise in Biomian's share price over the next six months with a limited downside. Neither B nor C provides VFO with downside protection if Biomian stock declines in six months.

  \item $\mathrm{C}$ is correct. Under a forward sale of Biomian shares, the profit is $\left[F_{0}(T)-S_{T}\right]$. If the shares rise significantly over the next six months-that is, $S_{T}>F_{0}(T)$-then VFO's loss on the derivative is the difference between the Biomian forward price, $F_{0}(T)$, and the spot price, $S_{T}$. Under the long put option on Biomian shares, VFO's profit is $\max \left(0, X-S_{T}\right)-p_{0}$. If Biomian shares rise significantly over the next six months (i.e., $S_{T}>X$ ), then the option expires worthless and VFO's loss is limited to the put premium paid, $p_{0}$. If $\left[F_{0}(T)-S_{T}\right]>-p_{0}$, then VFO's loss would be greater under the firm commitment than under the contingent claim.

  \item A is correct. The profit is equal to $\Pi=\max \left(0, S_{T}-X\right)-c_{0}$, and the payoff is equal to $\max \left(0, S_{T}-X\right)$. The exercise price is INR16,000, and the spot price just prior to maturity is INR16,500, so $\Pi=-1,000[=(16,500-16,000)-1,500]$, and the payoff is equal to INR500 [= $(16,500-16,000)]$. LEARNING MODULE

\end{enumerate}

\begin{center}
\includegraphics[max width=\textwidth]{2023_05_04_36535b8d80b32081d422g-211}
\end{center}

\section{Derivative Benefits, Risks, and Issuer and Investor Uses}
\section{LEARNING OUTCOME}
\begin{center}
\begin{tabular}{c|l}
Mastery & The candidate should be able to: \\
\hline
$\square$ & describe benefits and risks of derivative instruments \\
$\square$ & compare the use of derivatives among issuers and investors \\
\end{tabular}
\end{center}

\section{INTRODUCTION}
Earlier lessons described how derivatives expand the set of opportunities available to market participants to create or modify exposure or to hedge the price of an underlying. This learning module describes the benefits and risks of using derivatives and compares their use among issuers and investors.

\section{Summary}
\begin{itemize}
  \item Derivatives allow market participants to allocate, manage, or trade exposure without exchanging an underlying in the cash market.

  \item Derivatives also offer greater operational and market efficiency than cash markets and allow users to create exposures unavailable in cash markets.

  \item Derivative instruments can involve risks such as a high degree of implicit leverage and less transparency in some cases than cash instruments, as well as basis, liquidity, and counterparty credit risks. Excessive risk taking in the past by market participants through the use of derivatives has contributed to market destabilization and systemic risk.

  \item Issuers typically use derivative instruments to offset or hedge market-based underlying exposures that impact their assets, liabilities, and earnings.

  \item Issuers usually seek hedge accounting treatment for derivatives to minimize income statement and cash flow volatility.

  \item Investors use derivatives to modify investment portfolio cash flows, replicate investment strategy returns in cash markets, and/or create exposures unavailable to cash market participants.

\end{itemize}

\section{LEARNING MODULE PRE-TEST}
\begin{enumerate}
  \item Identify and describe two operational advantages of a futures market transaction over a cash transaction in an underlying.
\end{enumerate}

\section{Solution}
1.) Futures margin requirements are low relative to the cost of an equivalent cash market purchase in an underlying.

2.) Investors may more easily take a short position in an underlying in the futures market than in the cash market.

\begin{enumerate}
  \setcounter{enumi}{1}
  \item Identify which derivative risk fits each of the following statements:
\end{enumerate}

A. Basis risk

B. Liquidity risk

C. Counterparty credit risk 1. Potential divergence between the cash flow timing of a derivative versus an underlying or hedged transaction

\begin{enumerate}
  \setcounter{enumi}{1}
  \item Potential divergence between the expected value of a derivative versus an underlying or hedged transaction

  \item Potential for a derivatives contract participant to fail to meet their obligations under an agreement

\end{enumerate}

\section{Solution}
\begin{enumerate}
  \item B is correct. Liquidity risk is the potential divergence between the cash flow timing of a derivative versus an underlying or hedged transaction.

  \item A is correct. Basis risk is the potential divergence between the expected value of a derivative versus an underlying or hedged transaction.

  \item $C$ is correct. Counterparty credit risk is the potential for a derivatives contract participant to fail to meet their obligations under an agreement.

  \item Identify which benefit of derivatives use fits each of the following examples:

\end{enumerate}

A. Price discovery function 1. Equity market participants monitor index futures prior to the market open for an indication of the direction of cash market prices in early trading.

B. Operational advantages $\quad$ 2. An issuer may wish to lock in its future debt costs in advance of the maturity of an outstanding debt issuance.

C. Ability to allocate, trans3. Futures contracts in physical commodities eliminate the need to directly transport, insure, and store a physical asset in order to take a position in its underlying price.

\section{Solution}
\begin{enumerate}
  \item A is correct. Equity market participants monitoring index futures prior to the market open for an indication of the direction of cash market prices in early trading is an example of the derivatives price discovery function. 2. $\mathrm{C}$ is correct. An issuer locking in its future debt costs in advance of the maturity of an outstanding debt issuance is an example of the ability to allocate, transfer, and manage risk.

  \item B is correct. Futures contracts in physical commodities eliminating the need to directly transport, insure, and store a physical asset in order to take a position in its underlying price is an example of the operational advantages of a derivative.

  \item Describe the market efficiency function of derivative markets.

\end{enumerate}

\section{Solution}
The lower transaction costs and greater operational efficiency of derivative markets also leads to greater market efficiency. When prices deviate from fundamental values, derivative markets offer less costly ways to exploit the mispricing.

\begin{enumerate}
  \setcounter{enumi}{4}
  \item Determine the correct answers to fill in the blanks: Cash flow hedges are derivatives designated as absorbing the cash flow of a rate asset or liability such as foreign exchange, interest rates, or commodities.
\end{enumerate}

\section{Solution}
Cash flow hedges are derivatives designated as absorbing the variable cash flow of a floating-rate asset or liability such as foreign exchange, interest rates, or commodities.

\begin{enumerate}
  \setcounter{enumi}{5}
  \item Determine the correct answers to fill in the blanks: The purpose of investor use of derivatives within a fund is usually to the return of the fund and/or to offset or \_\_\_ the fund's value against adverse movements in underlyings such as exchange rates, interest rates, and securities markets.
\end{enumerate}

\section{Solution}
The purpose of investor use of derivatives within a fund is usually to increase the return of the fund and/or to offset or hedge the fund's value against adverse movements in underlyings such as exchange rates, interest rates, and securities markets.

\section{DERIVATIVE BENEFITS}
describe benefits and risks of derivative instruments

Earlier lessons demonstrated how market participants use derivative instruments as an alternative to cash markets to hedge or offset commercial risk as well as create or modify exposure to the price of an underlying. We now take a more detailed look at these and other benefits of the use of derivatives, while also considering several risks unique to derivative instruments.

Derivative instruments provide users the opportunity to allocate, transfer, and/or manage risk without trading an underlying. Cash or spot market prices for financial instruments and commercial goods and services are a critical source of information

\section{Derivative Benefits, Risks, and Issuer and Investor Uses}
for the decision to buy or sell. However, in many instances, issuers and investors face a timing difference between an economic decision and the ability to transact in a cash market.

For example, issuers face the following timing differences when making operational and financing decisions:

\begin{itemize}
  \item A manufacturer may need to order commodity inputs for its production process in advance of receiving finished-goods orders.

  \item A retailer may await a shipment of goods priced in a foreign currency before selling domestic currency to make payment.

  \item An issuer may wish to lock in its future debt costs in advance of the maturity of an outstanding debt issuance.

\end{itemize}

Investors may face similar timing issues when making portfolio decisions that are separate from cash market transactions, as in the following cases:

\begin{itemize}
  \item An investor may seek to capitalize on a market view but lack the necessary cash on hand to transact in the spot market.

  \item In anticipation of a future stock dividend, debt coupon, or principal repayment, an investor may decide today how it will reinvest the proceeds in the future.

\end{itemize}

The ability to buy or sell a derivative instrument today at a pre-agreed price at a future date can bridge the timing gap between an economic decision and the ability to transact in underlying price risk under these scenarios. The use of forward commitments or contingent claims to allocate or transfer risk across time and among market participants able and willing to accept those exposures is a consistent theme in derivative markets. Example 1 builds on an earlier illustration of how an issuer may benefit from the use of a derivative associated with a commercial contract.

\section{EXAMPLE 1}
\section{Foreign Exchange Risk Transfer of an Export Contract}
Recall Montau AG, the German capital goods producer introduced earlier, which signs a commercial contract with Jeon, Inc., a Korean manufacturer, to deliver a laser cutting machine at a price of KRW650,000,000 in 75 days. Montau has fixed domestic currency (EUR) costs and therefore faces a timing mismatch between EUR costs incurred and EUR revenue realized upon the delivery of the machine and sale of KRW received in exchange for EUR in the spot FX market.

Describe Montau's currency exposure and how an FX forward contract may be used to mitigate its FX price risk.

Montau will receive KRW in 75 days, which it must sell to cover its EUR costs. This exposes the firm to KRW/EUR exchange rate changes for 75 days. Specifically, if the KRW depreciates versus the EUR, Montau will be able to purchase fewer EUR than expected with the KRW proceeds, resulting in a loss due to the FX timing mismatch. Montau's exposure profile due to the mismatch is as follows:

\begin{center}
\includegraphics[max width=\textwidth]{2023_05_04_36535b8d80b32081d422g-215(1)}
\end{center}

In order to mitigate its export contract exposure, Montau enters an FX forward to sell KRW and purchase EUR at a fixed price $\left[\mathrm{F}_{0}(\mathrm{~T})\right]$ in 75 days to eliminate the KRW/EUR exchange rate mismatch arising from the export contract. The FX forward payoff and profit profile is as follows:

\begin{center}
\includegraphics[max width=\textwidth]{2023_05_04_36535b8d80b32081d422g-215}
\end{center}

The FX forward payoff and profit profile offsets Montau's export contract exposure as a hedge, as shown in the following combined diagram:

\begin{center}
\includegraphics[max width=\textwidth]{2023_05_04_36535b8d80b32081d422g-215(2)}
\end{center}

The ability to trade and/or manage risk using derivatives extends to the creation of exposure profiles, which are unavailable in cash markets. The following example combines a long cash position with a sold derivative to increase an investor's expected return, based on a specific market view.

\section{EXAMPLE 2}
\section{Covered Call Option Strategy}
South China Sprintwyck Investments (SCSI) has a Chinese equity portfolio that has outperformed in the first half of the year due to an overweight position in health care industry shares. SCSI holds a long position in the Shenzhen China Securities Index (CSI) 300 Health Care Index (CSI 300) traded on the China Financial Futures Exchange (CFFEX). SCSI's CIO expects volatility in the CSI 300 to decline and the CSI 300 price at year-end to be at or slightly above the

\section{Derivative Benefits, Risks, and Issuer and Investor Uses}
current spot price. Rather than sell the CSI 300 position today in the cash market, she decides to sell a CSI 300 call option at an exercise price $5 \%$ above the current spot market price for the remaining six months of the year.

Describe the difference in SCSI's CSI 300 payoff profile between 1.) the long cash position and 2.) the sold call option plus long cash position (referred to as a covered call strategy) at the end of the year.

\begin{enumerate}
  \item SCSI's long cash position will rise or fall in value as the CSI 300 spot price changes until the end of the year.
\end{enumerate}

\begin{center}
\includegraphics[max width=\textwidth]{2023_05_04_36535b8d80b32081d422g-216(2)}
\end{center}

The difference between 1.) and 2.) is the sale of a CSI 300 call option. SCSI sells the call and receives an upfront premium. SCSI assumes unlimited downside risk as the CSI 300 price rises above the exercise price:

\begin{center}
\includegraphics[max width=\textwidth]{2023_05_04_36535b8d80b32081d422g-216}
\end{center}

\begin{enumerate}
  \setcounter{enumi}{1}
  \item The combination of the long CSI 300 cash position and the short CSI 300 call option results in the profit profile represented by the solid line below:
\end{enumerate}

\begin{center}
\includegraphics[max width=\textwidth]{2023_05_04_36535b8d80b32081d422g-216(1)}
\end{center}

The covered call strategy payoff may be described as follows:

\begin{itemize}
  \item CSI 300 appreciates by less than 5\%: The call option expires unexercised. SCSI's profit equals the long CSI 300 position plus the call premium.

  \item CSI 300 appreciates by more than 5\%: The call option is exercised. SCSI must pay the option buyer an amount equal to the gain on its long CSI 300 position above the exercise price. The option payoff is said to be "covered" by the long cash position. SCSI retains the call premium. Derivative instrument prices serve a price discovery function beyond the underlying cash or spot market. For example, futures prices are often seen as revealing information about the direction of cash markets in the future, although they cannot be strictly considered an unbiased forecast of future spot prices.

\end{itemize}

Market participants often use futures prices to gauge the direction of cash markets in the future in the following ways:

\begin{itemize}
  \item Equity market participants frequently monitor equity index futures prices prior to the stock market open for an indication of the direction of cash market prices in early trading.

  \item Analysts often use interest rate futures markets to extract investor expectations of a central bank benchmark interest rate increase or decrease at a future meeting.

  \item Commodity futures prices are a gauge of supply and demand dynamics between producers, consumers, and investors across maturities.

\end{itemize}

Example 3 provides a case of supply and demand effects on futures prices.

\section{EXAMPLE 3}
\section{Negative Oil Futures Prices in 2020}
In April 2020, the West Texas Intermediate (WTI) crude oil futures price fell below zero for the first time ever. The New York Mercantile Exchange (NYMEX) WTI crude oil futures contract has an underlying of 1,000 barrels of crude oil delivered to Cushing, Oklahoma, where energy companies store nearly 80 million barrels of oil. Widespread lockdowns in the wake of the COVID-19 pandemic caused demand to plummet, while producers could not cut oil production quickly enough in anticipation of the severe decline. Oil inventory in Cushing skyrocketed as a result.

\section{NYMEX Oil Futures Price (USD per contract), January-June 2020}
\begin{center}
\includegraphics[max width=\textwidth]{2023_05_04_36535b8d80b32081d422g-217}
\end{center}

Source: Bloomberg.

As the May futures contract approached expiration in late April 2020, investors assuming the financial risk of oil prices were surprised as oil refiners avoided physical delivery of oil upon contract settlement due to a sharp rise in storage costs. This forced the May futures price sharply lower, to a closing price of-USD37.63 on 20 April 2020. Oil producers and refiners used this futures pricing information to adjust supply more quickly and reduce pressure on storage facilities.

\section{Derivative Benefits, Risks, and Issuer and Investor Uses}
The price discovery function of derivatives extends to contingent claims. As we will see in a later lesson, option prices reflect several characteristics of the underlying, including the expected price risk of the underlying, known as implied volatility. The implied volatility of an option may be derived from its current price and other option features and provides a measure of the general level of uncertainty in the price of the underlying.

Derivative transactions offer a number of operational advantages to cash or spot markets in many instances.

\begin{enumerate}
  \item Transaction Costs: Commodity derivatives eliminate the need to transport, insure, and store a physical asset in order to take a position in its underlying price.

  \item Increased Liquidity: Derivative markets typically have greater liquidity as a result of the reduced capital required to trade derivatives versus an equivalent cash position in an underlying.

  \item Upfront Cash Requirements: Initial futures margin requirements and option premiums are low relative to the cost of a cash market purchase.

  \item Short Positions: In the absence of a liquid derivative market, taking a short position involves the costly process of locating a cash owner willing to lend the underlying for a period sufficient to facilitate a short sale.

\end{enumerate}

Example 4 contrasts the cash outlay of a spot trade with a futures contract based on transaction details familiar from an earlier lesson.

\section{EXAMPLE 4}
\section{Purchase of Spot Gold versus a Gold Futures Contract}
In an earlier example, Procam Investments enters a futures contract to buy 100 ounces of gold at a futures price $\left[f_{0}(T)\right]$ of USD1,792.13 per ounce that expires in three months and must post USD4,950 in initial margin as required by the exchange. The spot gold price $\left(\mathrm{S}_{0}\right)$ at the time Procam enters the futures contract is USD1,770 per ounce. Assume that Procam is able to borrow funds from a financial intermediary at a rate of $5 \%$ per year. Contrast the expected opportunity cost of the initial margin for the three-month futures contract with a cash purchase of 100 ounces of gold for the same three-month period.

\begin{center}
\includegraphics[max width=\textwidth]{2023_05_04_36535b8d80b32081d422g-219}
\end{center}

\begin{enumerate}
  \item Procam borrows the USD4,950 initial margin from a financial intermediary at $5 \%$ for three months.
\end{enumerate}

Opportunity (interest) cost of the initial margin:

$=(\mathrm{USD} 4,950 \times .05) / 4=\mathrm{USD} 61.88$

\begin{enumerate}
  \setcounter{enumi}{1}
  \item Procam borrows USD177,000 (USD $1,770 \times 100$ ounces) for the spot gold purchase from a financial intermediary at $5 \%$ for three months.
\end{enumerate}

Opportunity (interest) cost of the cash gold purchase price:

$=($ USD177, $000 \times .05) / 4=$ USD2 212.50

Procam gains exposure to the underlying price of 100 ounces of gold in the futures market at a fraction of the spot market cost. This comparison ignores three points about the futures contract and spot gold transactions:

\begin{enumerate}
  \item Procam's margin requirements change as the futures price fluctuates between $\mathrm{t}=0$ and $\mathrm{t}=\mathrm{T}$. Specifically, Procam faces margin calls if a lower futures price causes its margin account balance to fall below the maintenance margin. However, as the underlying price is usually bounded by zero, it is unlikely that Procam's borrowing cost for margin will exceed that of the spot cash purchase.

  \item Procam's spot gold purchase involves physical gold. It must therefore cover the cost of delivery, storage, and insurance for three months.

  \item The spot gold purchase takes place at a price of $\mathrm{S}_{0}=$ USD1,770 per ounce, while the pre-agreed futures price $\left[\mathrm{f}_{0}(\mathrm{~T})\right]=$ USD1,792.13 per ounce. At time $\mathrm{T}$ in three months, both the futures contract and the spot purchase result in a long position for Procam at a price of $S_{\mathrm{T}}$. The expected cost and return for these two transactions should therefore be equal, but the prices agreed at $\mathrm{t}=0$ are not. In a later lesson, we will explore the relationship between the spot gold price $\left(\mathrm{S}_{0}\right)$ and the gold futures price $\left[\mathrm{f}_{0}(\mathrm{~T})\right]$ and the effect of both borrowing and storage costs.

\end{enumerate}

The operational efficiency of derivative markets also leads to greater market efficiency. When prices deviate from fundamental values, derivative markets offer less costly ways to exploit the mispricing. As noted earlier, less capital is required,

\section{Derivative Benefits, Risks, and Issuer and Investor Uses}
transaction costs are lower, and short selling is easier. As a result of these market features, fundamental value is often reflected in derivative markets before it is restored in the underlying cash market. The existence of derivative markets therefore often causes financial markets in general to function more effectively. Exhibit 1 summarizes derivative market benefits for market participants.

\section{Exhibit 1: Benefits of Derivative Instruments}
Purpose

Description

\begin{center}
\begin{tabular}{|c|c|}
\hline
$\begin{array}{l}\text { Risk Allocation, Transfer, } \\ \text { and Management }\end{array}$ & $\begin{array}{l}\text { Allocate, trade, and/or manage underlying exposure without } \\ \text { trading the underlying } \\ \text { Create exposures unavailable in cash markets }\end{array}$ \\
\hline
Information Discovery & $\begin{array}{l}\text { Deliver expected price in the future as well as expected risk of } \\ \text { underlying }\end{array}$ \\
\hline
Operational Advantages & $\begin{array}{l}\text { Reduced cash outlay, lower transaction costs versus the under- } \\ \text { lying, increased liquidity and ability to "short" }\end{array}$ \\
\hline
Market Efficiency & Less costly to exploit arbitrage opportunities or mispricing \\
\hline
\end{tabular}
\end{center}

KNOWLEDGE CHECK

\section{Derivative Benefits}
\begin{enumerate}
  \item Describe a scenario in which an issuer faces a timing difference between an economic decision and an ability to transact in cash markets to manage price risk.
\end{enumerate}

\section{Solution}
An issuer may need to 1.) order commodity inputs for its production process in advance of receiving finished-goods orders, 2.) await a shipment of goods in a foreign currency before selling domestic currency to make payment, or 3.) lock in its future debt costs in advance of the maturity of an outstanding debt issuance.

\begin{enumerate}
  \setcounter{enumi}{1}
  \item Determine the correct answers to fill in the blanks: Derivative markets typically have greater than the underlying spot markets, a result of the reduced required to trade derivatives versus an equivalent cash position in an underlying.
\end{enumerate}

\section{Solution}
Derivative markets typically have greater liquidity than the underlying spot markets, a result of the reduced capital required to trade derivatives versus an equivalent cash position in an underlying.

\begin{enumerate}
  \setcounter{enumi}{2}
  \item Identify the proper derivative market benefits that correspond to the following statements:
\end{enumerate}

A. Price discovery function

B. Operational advantages $\quad$ 2. The ability to buy or sell a derivative today eliminates the timing mismatch between an economic decision and the ability to transact. C. Risk transfer

\begin{enumerate}
  \setcounter{enumi}{2}
  \item Investors track an equity index futures price to gauge sentiment before the market opens.
\end{enumerate}

\section{Solution}
\begin{enumerate}
  \item The correct answer is B. The low level of futures margin requirements versus the cost of a cash market purchase is an example of the operational advantage of using derivatives.

  \item The correct answer is $C$. The ability to buy or sell a derivative contract today eliminates the timing mismatch between an economic decision and the ability to transact, which is an example of the risk transfer function of derivatives.

  \item The correct answer is A. Investors use derivatives in a price discovery function when tracking an equity index futures price to gauge sentiment before the market opens.

\end{enumerate}

\section{DERIVATIVE RISKS}
$\square \quad$ describe benefits and risks of derivative instruments

While derivatives offer benefits such as the ability to efficiently hedge, allocate, and/ or transfer risk as well as greater operational and market efficiency, the greater complexity of derivative instruments and positions also gives rise to greater potential risks associated with their use.

The greater operational efficiency of derivative strategies that limit an investor's initial cash outlay translates to a high degree of implicit leverage versus a similar cash market position. To illustrate this effect, we return to an earlier example to measure and compare the leverage between a futures contract and a cash market purchase with and without the use of borrowing.

\section{EXAMPLE 5}
\section{Implicit Leverage of Spot versus Futures Purchases}
In Example 4, we compared a spot purchase and a three-month futures contract purchase of 100 ounces of gold by Procam Investments.

\begin{itemize}
  \item Procam's spot market purchase at $\mathrm{S}_{0}=$ USD1,770 per ounce results in a purchase price of USD177,000 (USD1,770 $\times 100$ ) paid in cash.

  \item Procam's three-month futures contract purchase at $f_{0}(T)=$ USD1,792.13 per ounce results in a purchase price of USD179,213 (USD1,792.13 $\times 100$ ). This purchase requires a USD4,950 initial margin deposit at the exchange.

\end{itemize}

In three months, the spot gold price $\left(\mathrm{S}_{\mathrm{T}}\right)$ is USD1,780.50 per ounce, and both the long cash position and the long futures contract position are valued at USD178,050 (USD1,780.50 $\times 100$ ). We divide the value change by the initial cash outlay for each transaction to compare implicit leverages:

\begin{itemize}
  \item Spot value change: 0.593\% gain [(USD178,050 - USD177,000)/ USD177,000].
\end{itemize}

\section{Derivative Benefits, Risks, and Issuer and Investor Uses}
\begin{itemize}
  \item Futures value change: 23.5\% loss [(USD178,050 - USD179,213)/ USD4,950].
\end{itemize}

Note the large order-of-magnitude difference between the cash and futures transactions. As mentioned earlier, we will explore the relationship between borrowing costs, gold storage costs (ignored in this example), the spot gold price $\left(\mathrm{S}_{0}\right)$, and the gold futures price $\left[\mathrm{f}_{0}(\mathrm{~T})\right]$ in a later lesson.

Implicit leverage is further magnified in an extreme case where only borrowed funds are used to enter the cash and futures transactions, as in Example 4:

\begin{itemize}
  \item Spot value change versus borrowing cost: $47.5 \%$ gain [(USD178,050 USD177,000)/ USD2,212.50].

  \item Futures value change versus borrowing cost: $1,879 \%$ loss [(USD178,050 - USD179,213)/ USD61.88].

\end{itemize}

This example emphasizes a very important point about derivatives: their inherent leverage magnifies the realized returns and risks and contributes to the severity of derivative-related losses.

Procam gains exposure to USD177,000 in underlying gold price risk with just USD61.88, implying a leverage ratio of 2,860. Procam benefits from this high leverage as gold prices rise, but its losses are rapidly magnified as gold prices decline. For example, as the futures price falls from USD179,213 to USD178,050, the modest $-0.649 \%$ gold price change translates to a loss of USD1,163. Procam's actual loss on this transaction is both the interest payment (USD61.88) and the loss on the trade (USD1,163), for a total loss of USD1,224.88, nearly 20 times (USD1,224.88/ USD61.88) the cost of financing.

The same principle holds for non-linear derivatives such as sold options, where an option seller may face unlimited downside risk as underlying price changes make it more favorable for an option buyer to exercise for a gain far in excess of the premium paid.

Leverage in derivatives creates significant exposures for the counterparties involved. These risks are mitigated through a combination of trading and exposure risk management, daily marking to market, the use of collateral arrangements, transaction and exposure limits, and centralized counterparties.

Derivatives offer the flexibility to create exposures beyond cash markets, which can add significant portfolio complexity and involve risks that are not well understood by stakeholders. This risk increases when a combination of derivatives and/or embedded derivatives is involved. For example, structured notes are a broad category of securities that incorporate the features of debt instruments and one or more embedded derivatives designed to achieve a particular issuer or investor objective. For instance, structured notes designed to create a derivative-based payoff profile for individual investors may involve greater cost, lower liquidity, and less transparency than an equivalent stand-alone derivative instrument.

Derivative users hedging commercial or financial exposure usually assume that a derivative will be highly effective in offsetting the price risk of an underlying. However, in some instances, the expected value of a derivative differs unexpectedly from that of the underlying, in what is known as basis risk. Basis risk may arise if a derivative instrument references a price or index that is similar to, but does not exactly match, an underlying exposure such as a different market reference rate or an issuer CDS spread versus that of an actual bond. Basis risk is affected by supply and demand dynamics in derivative markets, among other factors.

A related risk that can arise for both hedgers and risk takers is liquidity risk, or a divergence in the cash flow timing of a derivative versus that of an underlying transaction. The daily settlement of gains and losses in the futures market can give rise to liquidity risk. If an investor or issuer using a futures contract to hedge an underlying transaction is unable to meet a margin call due to a lack of funds, the counterparty's position is closed out and the investor or issuer must cover any losses on the derivative trade.

Counterparty credit risk is of critical importance to derivative market participants. Unlike loan and bond markets, where credit exposures are predictably based on notional outstanding plus accrued interest, daily swings in the price of an underlying, among other factors affecting derivative prices, require more frequent exposure monitoring and management. Counterparty credit exposure varies by the derivative type and market in which a derivative is transacted. Exhibit 2 contrasts the counterparty credit risk of a contingent claim with that of a forward commitment, using the example of a purchased call option and a long forward position.

\section{Exhibit 2: Counterparty Credit Risk of an Option versus a Forward}
\begin{center}
\includegraphics[max width=\textwidth]{2023_05_04_36535b8d80b32081d422g-223}
\end{center}

As for market type, the daily settlement of MTM gains and losses, which characterizes exchange-traded derivatives, substantially reduces counterparty credit risk. Exchanges reserve the right to increase margin requirements or require intraday margining for highly volatile or concentrated positions. In over-the-counter (OTC) markets, credit terms privately arranged between counterparties vary from uncollateralized exposure to terms similar to futures margining for one or both counterparties using collateral.

Broader derivatives use among market participants has increased the focus of financial market supervisory authorities on the potential market-wide impact, or systemic risk, associated with these instruments. Regulators continue to specifically focus on the impact of financial innovation and financial conditions to ensure financial stability as they monitor risk taking and leverage among derivative market participants. Market reforms such as the central clearing mandate for swaps between financial intermediaries and a central counterparty (CCP), outlined earlier, include margin provisions similar to futures in order to standardize and reduce counterparty credit risk.

Exhibit 3 summarizes the key risks associated with derivative instruments and markets.

\section{Derivative Benefits, Risks, and Issuer and Investor Uses}
Exhibit 3: Risks of Derivative Instruments

\begin{center}
\begin{tabular}{|c|c|}
\hline
Risk & Description \\
\hline
$\begin{array}{l}\text { Greater Potential for } \\ \text { Speculative Use }\end{array}$ & $\begin{array}{l}\text { High degree of implicit leverage for some derivative strate- } \\ \text { gies may increase the likelihood of financial distress. }\end{array}$ \\
\hline
Lack of Transparency & $\begin{array}{l}\text { Derivatives add portfolio complexity and may cre- } \\ \text { ate an exposure profile that is not well understood by } \\ \text { stakeholders. }\end{array}$ \\
\hline
Basis Risk & $\begin{array}{l}\text { Potential divergence between the expected value of a } \\ \text { derivative instrument versus an underlying or hedged } \\ \text { transaction }\end{array}$ \\
\hline
Liquidity Risk & $\begin{array}{l}\text { Potential divergence between the cash flow timing of } \\ \text { a derivative instrument versus an underlying or hedged } \\ \text { transaction }\end{array}$ \\
\hline
Counterparty Credit Risk & $\begin{array}{l}\text { Derivative instruments often give rise to counterparty } \\ \text { credit exposure, resulting from differences in the current } \\ \text { price versus the expected future settlement price. }\end{array}$ \\
\hline
$\begin{array}{l}\text { Destabilization and Systemic } \\ \text { Risk }\end{array}$ & $\begin{array}{l}\text { Excessive risk taking and use of leverage in derivative } \\ \text { markets may contribute to market stress, as in the } 2008 \\ \text { financial crisis. }\end{array}$ \\
\hline
\end{tabular}
\end{center}

\section{KNOWLEDGE CHECK}
\section{Derivative Risks}
\begin{enumerate}
  \item Determine the correct answers to fill in the blanks: The operational efficiency of derivative strategies that limit an investor's initial cash outlay translates to a degree of implicit leverage versus a similar cash market position.
\end{enumerate}

\section{Solution}
The greater operational efficiency of derivative strategies that limit an investor's initial cash outlay translates to a high degree of implicit leverage versus a similar cash market position.

\begin{enumerate}
  \setcounter{enumi}{1}
  \item Describe the counterparty credit risk faced by the seller of a call option.
\end{enumerate}

\section{Solution}
The seller of a call option receives an upfront premium in exchange for the right to purchase the underlying at the exercise price at maturity. Once the seller of a call option receives the premium from the option buyer, it has no further counterparty credit risk to the option buyer.

\begin{enumerate}
  \setcounter{enumi}{2}
  \item Match these derivative market risks to the following statements:
A. Liquidity risk
  \item The risk that excessive risk taking and use of leverage in derivative markets contribute to market stress
B. Basis risk
  \item The risk of a divergence in the cash flow timing of a deriv- ative versus that of an underlying transaction
C. Systemic risk
  \item The risk that the expected value of a derivative differs unexpectedly from that of the underlying
\end{enumerate}

\section{Solution}
\begin{enumerate}
  \item The correct answer is C. Systemic risk involves excessive risk taking and use of leverage in derivative markets that contribute to market stress.

  \item The correct answer is A. Liquidity risk is the divergence in the cash flow timing of a derivative versus that of an underlying transaction.

  \item The correct answer is B. Basis risk involves the risk that the expected value of a derivative differs unexpectedly from that of an underlying.

\end{enumerate}

\section{ISSUER USE OF DERIVATIVES}
compare the use of derivatives among issuers and investors

Issuers, investors, and financial intermediaries use derivative instruments to increase, decrease, or modify exposure to an underlying to meet their financial objectives. Financial analysts must gain a deeper understanding of the various uses of derivatives among market participants in order to interpret and replicate the wide range of strategies encountered in practice.

Non-financial corporate issuers often face risks to their assets, liabilities, and earnings as a result of changes in the price of an underlying. For example, a corporate issuer that either produces or uses a traded commodity in its operations will face asset volatility due to underlying inventory or input price changes. As corporate issuers access borrowed funds and ownership capital in financial markets, their cost of liabilities will fluctuate based on market reference rates and other variables. The impact of foreign exchange volatility on corporate issuer earnings is illustrated by extending an earlier example.

\section{EXAMPLE 6}
\section{Foreign Exchange Risk and Earnings Volatility}
Recall from Example 1 that Montau AG will deliver a laser cutting machine for KRW650,000,000 in 75 days and has hedged its FX exposure by agreeing to sell the KRW it will receive and purchase EUR upon delivery in an over-the-counter FX forward with a financial intermediary.

Montau agrees to a KRW/EUR forward exchange rate $\left[\mathrm{F}_{0}(\mathrm{~T})\right]$ of 1,350 (i.e., 1,350 KRW = 1 EUR), at which it will sell $650,000,000 \mathrm{KRW}$ and receive 481,481 EUR $(650,000,000 / 1,350)$ in 75 days. The production manager estimates the machine's cost to be $€ 430,000$. Montau's Treasury manager compiles the following profit margin scenarios at different $K R W / E U R$ spot rates $\left(\mathrm{S}_{\mathrm{T}}\right)$ :

\begin{center}
\begin{tabular}{lcccc}
\hline
$\begin{array}{l}\text { Spot KRW/EUR } \\ \left(\mathbf{s}_{\mathrm{T}}\right)\end{array}$ & $\begin{array}{c}\text { Unhedged } \\ \text { EUR } \\ \text { Proceeds }\end{array}$ & $\begin{array}{c}\text { Unhedged } \\ \text { Profit } \\ \text { Margin }\end{array}$ & $\begin{array}{c}\text { Hedged EUR } \\ \text { Proceeds }\end{array}$ & $\begin{array}{c}\text { Hedged Profit } \\ \text { Margin }\end{array}$ \\
\hline
1,525 & $€ 426,230$ & $-1 \%$ & $€ 481,481$ & $11 \%$ \\
1,400 & $€ 464,286$ & $7 \%$ & $€ 481,481$ & $11 \%$ \\
1,280 & $€ 507,813$ & $15 \%$ & $€ 481,481$ & $11 \%$ \\
\end{tabular}
\end{center}

\section{Derivative Benefits, Risks, and Issuer and Investor Uses}
\begin{center}
\begin{tabular}{lcccc}
\hline
$\begin{array}{l}\text { Spot KRW/EUR } \\ \left(\mathbf{s}_{\mathrm{T}}\right)\end{array}$ & $\begin{array}{c}\text { Unhedged } \\ \text { EUR } \\ \text { Proceeds }\end{array}$ & $\begin{array}{c}\text { Unhedged } \\ \text { Profit } \\ \text { Margin }\end{array}$ & $\begin{array}{c}\text { Hedged EUR } \\ \text { Proceeds }\end{array}$ & $\begin{array}{c}\text { Hedged Profit } \\ \text { Margin }\end{array}$ \\
\hline
1,225 & $€ 530,612$ & $19 \%$ & $€ 481,481$ & $11 \%$ \\
\hline
\end{tabular}
\end{center}

CFA

LI\_DER\_LM\_3

Example 6

\begin{center}
\includegraphics[max width=\textwidth]{2023_05_04_36535b8d80b32081d422g-226}
\end{center}

If Montau does not hedge its KRW/EUR exposure, its profit margin will fluctuate based on the KRW/EUR spot rate when the machine is delivered. A weaker KRW versus EUR results in a lower profit margin, while KRW appreciation results in a higher profit margin.

Example 6 illustrates one of the most common derivative strategies used by corporate issuers. The FX forward payoff offsets Montau's income statement and cash flow volatility due to currency changes.

Derivatives accounting has evolved from off-balance-sheet treatment to the reporting of these instruments on the balance sheet at their fair market value. This change aligns the recognition of derivative gains and losses with their designated risk management purpose, increasing transparency and disclosure of derivatives use. Many corporate issuers also establish risk management policies governing the objectives, guidelines, risk limits, and internal approval processes associated with derivatives use.

Derivatives accounting standards specify that any derivative purchased or sold must be marked to market through the income statement via earnings unless it is embedded in an asset or liability or qualifies for hedge accounting. Hedge accounting allows an issuer to offset a hedging instrument (usually a derivative) against a hedged transaction or balance sheet item to reduce financial statement volatility. Categorizing derivatives by hedge designation type sheds further light on both their intent and their expected financial statement impact, to the benefit of financial analysts and stakeholders.

For example, derivatives designated as absorbing the variable cash flow of a floating-rate asset or liability such as foreign exchange, interest rates, or commodities are referred to as cash flow hedges. Cash flow hedges may be either forward commitments or contingent claims. For instance, the FX forward in Example 6 is a cash flow hedge that offsets the variability of Montau's functional currency (EUR) proceeds from its commercial transaction in KRW. A swap to a fixed rate for a floating-rate debt liability is another example of a cash flow hedge.

A fair value hedge designation applies when a derivative is deemed to offset the fluctuation in fair value of an asset or liability. For example, an issuer might convert a fixed-rate bond issuance to a floating-rate obligation by entering into an interest rate swap to receive a fixed rate and pay a market reference rate through the bond's maturity. Alternatively, a commodities producer might sell its inventory forward in anticipation of lower future prices.

Net investment hedges occur when either a foreign currency bond or a derivative such as an FX swap or forward is used to offset the exchange rate risk of the equity of a foreign operation. Exhibit 4 summarizes these hedge designations. Exhibit 4: Hedge Accounting Designation Types

\begin{center}
\begin{tabular}{|c|c|c|}
\hline
Hedge Type & Description & Examples \\
\hline
Cash Flow & $\begin{array}{l}\text { Absorbs variable cash flow of } \\ \text { floating-rate asset or liability } \\ \text { (forecasted transaction) }\end{array}$ & $\begin{array}{l}\text { Interest rate swap to a fixed rate } \\ \text { for floating-rate debt } \\ \text { FX forward to hedge forecasted } \\ \text { sales }\end{array}$ \\
\hline
Fair Value & $\begin{array}{l}\text { Offsets fluctuation in fair value of } \\ \text { an asset or liability }\end{array}$ & $\begin{array}{l}\text { Interest rate swap to a floating } \\ \text { rate for fixed-rate debt } \\ \text { Commodity future to hedge } \\ \text { inventory }\end{array}$ \\
\hline
Net Investment & $\begin{array}{l}\text { Designated as offsetting the FX } \\ \text { risk of the equity of a foreign } \\ \text { operation }\end{array}$ & $\begin{array}{l}\text { Currency swap } \\ \text { Currency forward }\end{array}$ \\
\hline
\end{tabular}
\end{center}

Hedge accounting treatment for derivatives is highly desirable for corporate issuers, as it allows them to recognize derivative gains and losses at the same time as the associated underlying hedged transaction. Derivative mark-to-market changes are held within an equity account (Other Comprehensive Income) and released at the same time the underlying hedged transaction is recognized in earnings.

In order to qualify for this treatment, the dates, notional amounts, and other contract features of a derivative must closely match those of the underlying transaction. For this reason, issuers are far more likely to use OTC markets to create customized hedges to meet their specific needs. For example, Montau AG would face earnings volatility from the derivative position if it were to use a standardized two-month foreign exchange futures contract rather than the 75-day over-the-counter FX forward contract with a financial intermediary, as in Example 6.

\section{KNOWLEDGE CHECK}
\section{Issuer Use of Derivatives}
\begin{enumerate}
  \item Describe hedge accounting treatment.
\end{enumerate}

\section{Solution}
Hedge accounting allows an issuer to offset a hedging instrument (usually a derivative) against a hedged transaction or balance sheet item to reduce financial statement volatility.

\begin{enumerate}
  \setcounter{enumi}{1}
  \item Match these hedge designation types to the following statements:
A. Cash flow hedge 1. A derivative used to offset the fluctuation in fair value of an asset or liability
B. Fair value hedge $\quad$ 2. A derivative designated as absorbing the variable cash flow of a floating-rate asset or liability
C. Net investment 3. A derivative designated as offsetting the foreign exchange hedge risk of the equity of a foreign operation
\end{enumerate}

\section{Solution}
\begin{enumerate}
  \item The correct answer is B. A fair value hedge is a derivative used to offset the fluctuation in fair value of an asset or liability.
\end{enumerate}

\section{Derivative Benefits, Risks, and Issuer and Investor Uses}
\begin{enumerate}
  \setcounter{enumi}{1}
  \item The correct answer is A. A cash flow hedge is a derivative designated as absorbing the variable cash flow of a floating-rate asset or liability.

  \item The correct answer is $C$. A net investment hedge is a derivative designated as offsetting the foreign exchange risk of the equity of a foreign operation.

  \item Describe an example of a fair value hedge an issuer might use.

\end{enumerate}

\section{Solution}
An issuer might convert a fixed-rate bond issuance to a floating-rate obligation by entering into an interest rate swap to receive a fixed rate and pay a market reference rate through the bond's maturity. Alternatively, a commodities producer might sell its inventory forward in anticipation of lower future prices.

\section{INVESTOR USE OF DERIVATIVES}
compare the use of derivatives among issuers and investors

Issuers predominantly use derivatives to offset or hedge market-based underlying exposures incidental to their commercial operations and financing activities. In contrast, investors use derivatives to replicate a cash market strategy, hedge a fund's value against adverse movements in underlyings, or modify or add exposures using derivatives, which in some cases are unavailable in cash markets.

The greater liquidity and reduced capital required to trade derivatives may lead an investor to replicate a desired position using a derivative rather than cash. Alternatively, derivative hedges enable investors to isolate certain underlying exposures in the investment process while retaining a position in others. One example is the use of FX hedges when investing overseas in order to minimize the volatility of return due to currency fluctuations. Finally, the flexibility to take short positions or to increase or otherwise modify exposure using derivatives beyond cash alternatives is an attractive feature for portfolio managers targeting excess returns by using a variety of strategies.

An investment fund's prospectus typically specifies which derivative instruments may be used within a fund and for which purpose.

Several examples in earlier lessons illustrated the use of derivatives from an investor perspective using both forward commitments and contingent claims.

\section{Forward Commitments}
\begin{itemize}
  \item Recall an earlier example in which Procam Investments purchased a three-month gold forward or futures contract. The derivatives contract increased Procam's exposure to the underlying price of gold in the future with no initial cash outlay and no requirement to take immediate delivery of the physical asset, as in the case of a cash purchase.

  \item In the example of Fyleton Investments, the fund entered into a GBP interest rate swap to increase the duration of its assets with no initial cash outlay.

\end{itemize}

\section{Contingent Claims}
\begin{itemize}
  \item One example had Hightest Capital purchasing a call option to benefit from an expected increase in a health care stock index price above the exercise in exchange for an upfront premium.

  \item The SCSI long equity and short call (or covered call) example created an exposure profile under which SCSI realized a higher return than in its original long cash position if the underlying index price at the end of the year was stable to slightly higher, in accordance with the CIO's view.

\end{itemize}

In these and other instances throughout the curriculum, we find investors to be less focused than issuers on hedge accounting treatment, as an investment fund's derivative position is typically marked to market each day and included in the daily net asset value (NAV) of the portfolio or fund. This also explains why investors tend to transact more frequently in standardized and highly liquid exchange-traded derivative markets than do issuers.

\section{KNOWLEDGE CHECK}
\section{Investor Use of Derivatives}
\begin{enumerate}
  \item Determine the correct answer to complete the following sentence: An investment fund's typically specifies which derivative instruments may be used within a fund and for which purpose.
\end{enumerate}

\section{Solution}
An investment fund's prospectus typically specifies which derivative instruments may be used within a fund and for which purpose.

\begin{enumerate}
  \setcounter{enumi}{1}
  \item Describe two purposes of investor derivatives use within a fund.
\end{enumerate}

\section{Solution}
The purpose of investor derivatives use within a fund is usually to increase the return of the fund and/or to offset or hedge the fund's value against adverse movements in underlyings such as exchange rates, interest rates, and securities markets.

\begin{enumerate}
  \setcounter{enumi}{2}
  \item Match these derivative market participants to the following statements:
\end{enumerate}

$\begin{array}{ll}\text { A. Investors } & \begin{array}{l}\text { 1. They use derivatives to offset or hedge market-based } \\ \text { underlying exposures incidental to their commercial opera- } \\ \text { tions and financing activities. }\end{array} \\ \text { B. Both issuers and } & \begin{array}{l}\text { 2. They tend to transact more frequently in exchange-traded } \\ \text { investors }\end{array} \\ \text { C. Issuers } & \begin{array}{l}\text { 3. They use derivatives to change their exposure to an } \\ \text { underlying asset price without transacting in the cash } \\ \text { market. }\end{array}\end{array}$

\section{Solution}
\begin{enumerate}
  \item The correct answer is $C$. Issuers use derivatives to offset or hedge market-based underlying exposures incidental to their commercial operations and financing activities. 2. The correct answer is A. Investors tend to transact more frequently in exchange-traded derivative markets.

  \item The correct answer is B. Both issuers and investors use derivatives to change their exposure to an underlying asset price without transacting in the cash market.

\end{enumerate}

\section{PRACTICE PROBLEMS}
Consider the following structured note offered by Baywhite Financial:

\section{Baywhite Financial LLC 80\% Principal Protected Structured Note}
Description:

Start Date:

Maturity Date:

Issuance Price:

Face Value:

Payment at Maturity:

Partial Principal

Protection Percentage:

Additional Amount: The Baywhite Financial LLC 80\% Principal Protected Structured Note ("the Note") is linked to the performance of the S\&P 500 Health Care Select Sector Index (SIXV).

Baywhite Financial LLC

[Today]

[Six months from Start Date]

102\% of Face Value

Sold in a minimum denomination of USD1,000 and multiple units thereof

At maturity, you will receive a cash payment, for each USD1,000 principal amount note, of USD800 plus the Additional Amount, which may be zero.

80\% Principal Protection (20\% Principal at Risk)

At maturity, you will receive the greater of $100 \%$ of the returns on the S\&P 500 Health Care Select Sector Index (SIXV) in excess of $5 \%$ above the current spot price of the SIXV or zero.

As a financial analyst for a wealth management advisory firm, you have been tasked with comparing the features of the Baywhite Financial LLC Structured Note with those of a similar exchange-traded, stand-alone derivative instrument alternative in order to make a recommendation to the firm's clients.

\begin{enumerate}
  \item Which of the following statements best describes the derivative instrument that is embedded in the Baywhite Financial LLC Structured Note?
\end{enumerate}

A. The Structured Note has an embedded long futures contract with the S\&P 500 Health Care Select Sector Index (SIXV) as an underlying.

B. The Structured Note has an embedded long call option contract with the S\&P 500 Health Care Select Sector Index (SIXV) as an underlying.

C. The Structured Note has an embedded short put option contract with the S\&P 500 Health Care Select Sector Index (SIXV) as an underlying.

\begin{enumerate}
  \setcounter{enumi}{1}
  \item Which of the following statements best contrasts the credit risk of the Baywhite Financial LLC Structured Note with the counterparty credit risk of an investor entering into the embedded exchange-traded derivative on a stand-alone basis?
\end{enumerate}

A. An investor in the Baywhite Structured Note assumes the credit risk of Baywhite Financial LLC for $20 \%$ of the note's face value, as the remaining $80 \%$ is principal protected. An investor entering into the SIXV derivative on a stand-alone basis assumes the counterparty credit risk of a financial intermediary.

\section{Derivative Benefits, Risks, and Issuer and Investor Uses}
B. An investor in the Baywhite Structured Note assumes the credit risk of Baywhite Financial LLC for $80 \%$ of the note's face value, as the remaining $20 \%$ is associated with the embedded derivative. An investor entering into the SIXV derivative on a stand-alone basis assumes the counterparty credit risk of a financial intermediary.

C. An investor in the Baywhite Structured Note assumes the credit risk of Baywhite Financial LLC for $100 \%$ of the note's face value, while an investor entering into the SIXV derivative on a stand-alone basis assumes the counterparty credit risk of an exchange and its clearinghouse.

\begin{enumerate}
  \setcounter{enumi}{2}
  \item Which of the following statements most accurately describes the liquidity of the Baywhite Structured Note versus that of the embedded exchange-traded derivative?
\end{enumerate}

A. The Baywhite Structured Note is likely to be more liquid than the stand-alone SIXV call option, as the Note has $80 \%$ principal protection while an investor in the stand-alone derivative may lose the entire option premium if it expires worthless at maturity.

B. The Baywhite Structured Note is likely to be more liquid than the stand-alone SIXV call option, as the Note is priced at a stated $2 \%$ premium above par while an investor in the stand-alone derivative faces the lack of transparency as well as basis, liquidity, and counterparty credit risks associated with derivative transactions.

C. Structured notes such as the Baywhite Financial LLC Structured Note often involve greater cost, lower liquidity, and less transparency than an equivalent stand-alone derivative instrument, while the exchange-traded SIXV derivative contract is standardized and trades in a liquid, transparent market.

\begin{enumerate}
  \setcounter{enumi}{3}
  \item Which of the following statements best describes how an investor should evaluate the terms of the Baywhite Financial LLC Structured Note as compared with the stand-alone derivative price in order to make a recommendation?
\end{enumerate}

A. The Baywhite Financial LLC Structured Note issuance price of $2 \%$ above par value should be compared with the upfront premium for a six-month SIXV call option with an exercise price at $5 \%$ above the current SIXV spot price.

B. The Baywhite Financial LLC Structured Note $20 \%$ Principal at Risk should be compared with the upfront premium for a six-month SIXV call option with an exercise price at $5 \%$ above the current SIXV spot price.

C. The Baywhite Financial LLC Structured Note issuance price of $2 \%$ above par value plus the $20 \%$ Principal at Risk should be compared with the upfront premium for a six-month SIXV call option with an exercise price at $5 \%$ above the current SIXV spot price.

\section{SOLUTIONS}
\begin{enumerate}
  \item B is correct. The Structured Note is linked to the performance of the S\&P 500 Health Care Select Sector Index (SIXV). Note that the SIXV derivative is similar to that in the earlier SCSI CSI 300 example. The "Additional Amount" paid at maturity is equal to the greater of $100 \%$ of the returns on the S\&P 500 Health Care Select Sector Index (SIXV) in excess of $5 \%$ above the current spot price of the SIXV or zero. This payoff profile [Max $\left(0, \mathrm{~S}_{\mathrm{T}}-\mathrm{X}\right)$ ] is identical to that of a purchased six-month SIXV call option with an exercise price $(X)$ at $5 \%$ above today's SIXV spot price.

  \item C is correct. The investor assumes the credit risk of Baywhite Financial LLC for the full value of the structured note as the structured note issuer. Under the purchased exchange-traded SIXV call option, the investor faces the risk of the exchange and its clearinghouse, which provides a guarantee of contract settlement backed by the exchange insurance fund.

  \item C is correct. The Structured Note is likely to be far less liquid than the stand-alone SIXV call option, which is traded on a derivatives exchange. Recall from an earlier lesson that exchange-traded contracts are more formal and standardized, which facilitates a more liquid and transparent market. Note also that the Baywhite Financial LLC Structured Note is issued at 102\% of face value, suggesting that an investor will likely forgo this premium if selling the note prior to maturity.

  \item C is correct. The $20 \%$ Principal at Risk, or USD200 of Face Value for each USD1,000 (the minimum denomination), combined with the 2\% (or USD20) issue premium, should be compared with the upfront premium for a six-month SIXV call option with an exercise price at $5 \%$ above the current SIXV spot price. The comparison should also consider the additional credit risk and liquidity risk of the Structured Note versus the exchange-traded option.

\end{enumerate}

\section{LEARNING MODULE
4}
\section{Arbitrage, Replication, and the Cost of Carry in Pricing Derivatives}
\section{LEARNING OUTCOME}
\begin{center}
\begin{tabular}{c|l}
Mastery & The candidate should be able to: \\
\hline
$\square$ & $\begin{array}{l}\text { explain how the concepts of arbitrage and replication are used in } \\ \text { pricing derivatives } \\ \text { explain the difference between the spot and expected future price } \\ \text { of an underlying and the cost of carry associated with holding the } \\ \text { underlying asset }\end{array}$ \\
\hline
$\square$ &  \\
\hline
\end{tabular}
\end{center}

\section{INTRODUCTION}
Earlier derivative lessons established the features of derivative instruments and markets and addressed both the benefits and risks associated with their use. Forward commitments and contingent claims were distinguished by their different payoff profiles and other characteristics. We now turn our attention to the pricing and valuation of these instruments. As a first step, we explore how the price of a forward commitment is related to the spot price of an underlying asset in a way that does not allow for arbitrage opportunities. Specifically, the strategy of replication shows that identical payoffs to a forward commitment can be achieved from spot market transactions combined with borrowing or lending at the risk-free rate. Finally, the second lesson demonstrates how costs or benefits associated with owning an underlying asset affect the forward commitment price.

\section{Summary}
\begin{itemize}
  \item Forward commitments are an alternative means of taking a long or short position in an underlying asset. A link between forward prices and spot prices exists to prevent investors from taking advantage of arbitrage opportunities across cash and derivative instruments.

  \item A forward commitment may be replicated with a long or short spot position in the underlying asset and borrowing or lending at a risk-free rate. Investors can recreate a variety of positions by using appropriate combinations of spot, forward, and risk-free positions.

\end{itemize}

\section{Arbitrage, Replication, and the Cost of Carry in Pricing Derivatives}
\begin{itemize}
  \item The risk-free rate provides a fundamental link between spot and forward prices for underlying assets with no additional costs or benefits of ownership.

  \item The cost of carry is the net of the costs and benefits related to owning an underlying asset for a specific period and must be factored into the difference between the spot price and a forward price of a specific underlying asset.

  \item The cost of carry may include costs, such as storage and insurance for physical commodities, or benefits of ownership, such as dividends for stocks and interest for bonds. Foreign exchange represents a special case in which the cost of carry is the interest rate differential between two currencies.

  \item Forward prices may be greater than or less than the underlying spot price, depending on the specific cost of carry associated with owning the underlying asset.

\end{itemize}

\section{LEARNING MODULE PRE-TEST}
\begin{enumerate}
  \item Determine the correct answers to fill in the blanks: Identical assets or assets with identical cash flows must have the same \_\_. Assets with a known future price must have a price, which equals the future price discounted at the rate.
\end{enumerate}

\section{Solution:}
Identical assets or assets with identical cash flows must have the same price. Assets with a known future price must have a spot price, which equals the future price discounted at the risk-free rate.

\begin{enumerate}
  \setcounter{enumi}{1}
  \item Describe and justify a derivative replication strategy that combines cash markets and forward commitments under which an investor can generate a risk-free rate of return for an asset with no costs or benefits.
\end{enumerate}

\section{Solution:}
An investor can generate a risk-free rate of return by entering a long cash position and simultaneously entering a short forward commitment on the same underlying asset. The payoff on the long position rises as the underlying price increases (and falls as the price of the underlying decreases). The short forward commitment has the opposite payoff function (i.e., the payoff falls as the underlying asset price increases). The gains and losses from the two positions offset one another, with the net gain equal to the difference between the forward price and the spot price. This should equal the risk-free rate under the no-arbitrage condition with no other underlying asset costs or benefits. Investors may generate a similar risk-free return by combining a short cash position with a long forward commitment. The proceeds from the short sale are invested at the risk-free rate, and the forward purchase is used to settle the short cash position.

\begin{enumerate}
  \setcounter{enumi}{2}
  \item Calculate the arbitrage profit available to an investor who is able to buy an asset for a spot price of GBP50 at $t=0$ and simultaneously sells a six-month forward commitment on the same asset at a forward price of GBP52.50. The risk-free rate of interest is $4 \%$, and the asset has no additional costs or benefits.
\end{enumerate}

\section{Solution:}
The investor borrows at $4 \%$ for six months to buy the asset today for GBP50. After six months, the investor pays the lender $S_{0}(1+r)^{T}$ or GBP50.99 (= GBP50(1.04) $\left.{ }^{0.5}\right)$ in principal and interest and delivers the asset to satisfy the forward commitment to sell at GBP52.50. The investor's arbitrage profit is GBP1.51 (= GBP52.50 - GBP50.99).

\begin{enumerate}
  \setcounter{enumi}{3}
  \item Describe the relationship between the forward commitment price relative to the spot price and the costs and benefits associated with owning an underlying asset.
\end{enumerate}

\section{Solution:}
The forward commitment price must incorporate the net effect of all costs and benefits associated with owning the underlying asset, including the opportunity cost or risk-free interest rate; additional costs, such as storage and insurance for commodities; and benefits, such as stock dividends or bond coupons. Greater opportunity and other costs of owning an underlying asset will increase the forward commitment price, while greater benefits of asset ownership will reduce the forward commitment price relative to the spot price.

\begin{enumerate}
  \setcounter{enumi}{4}
  \item Calculate the six-month forward price for a stock index with a spot price of JPY1,250, a dividend yield of $1 \%$, and a risk-free rate of $0.5 \%$, and interpret the relationship between the spot and forward index price.
\end{enumerate}

\section{Solution:}
Use the following equation to solve for $F_{0}(T)$ in six months:

$$
F_{0}(T)=S_{0} e^{(r-i) T}
$$

Given a spot stock index price, $S_{0}$, if JPY $1,250, r=0.5 \%, i=1 \%$, and $T=0.5$ years, we can solve for $F_{0}(T)$ as JPY1,246.88 (= JPY1,250e $\left.e^{(0.005-0.01) \times 0.5}\right)$. The forward index price, $F_{0}(T)$, is below the spot index price, $S_{0}$, since the dividend yield (a benefit to owning the index) is greater than the risk-free rate (the opportunity cost).

\begin{enumerate}
  \setcounter{enumi}{5}
  \item Define the convenience yield associated with a commodity and describe its impact on the relationship between spot and forward prices.
\end{enumerate}

\section{Solution:}
The convenience yield is a non-cash benefit associated with owning an underlying physical commodity that arises under certain economic conditions. As in the case of any other benefit associated with owning the physical asset, a higher convenience yield will cause the forward price to decline relative to the spot price.

\section{ARBITRAGE}
explain how the concepts of arbitrage and replication are used in pricing derivatives

An earlier lesson on market efficiency established that market prices should not allow for the possibility of riskless profit or arbitrage in the absence of transaction costs. In its simplest form, an arbitrage opportunity arises if the "law of one price" does not hold, or an identical asset trades at the same time at different prices in different places.

In the case of a derivative contract whose value is derived from future cash flows associated with the price of an underlying asset, arbitrage opportunities arise either if two assets with identical future cash flows trade at different prices or if an asset with a known future price does not trade at the present value of its future price determined using an appropriate discount rate.

The first case of assets with identical future cash flows trading at different prices is illustrated in Exhibit 1.

Exhibit 1: Assets with Identical Future Cash Flows Trade at

Different Prices

\begin{center}
\includegraphics[max width=\textwidth]{2023_05_04_36535b8d80b32081d422g-238}
\end{center}

For example, assume the two assets are zero-coupon bonds with identical features and the same issuer. Both bonds mature on the same future date with a payoff of par and have the same risk of default between now and the maturity date.

\begin{enumerate}
  \item Bond A has a price of EUR99 at time $t=0\left(S_{0}{ }^{A}=\right.$ EUR99).

  \item Bond B has a price of EUR99.15 at time $t=0\left(S_{0}{ }^{B}=\right.$ EUR99.15).

  \item Both bonds have an expected future price of EUR100 $\left(S_{T}{ }^{A}=S_{T}{ }^{B}=\right.$ EUR100).

\end{enumerate}

This scenario represents an arbitrage opportunity for an investor.

\begin{itemize}
  \item At time $t=0$, the investor can:

  \item Sell Bond B short to receive proceeds of EUR99.15 and purchase Bond A for EUR99;

  \item Realize a net cash inflow at $t=0$ of EUR0.15.

  \item At time $t=T$, when both bonds mature, the investor:

  \item Receives EUR100 for Bond A and uses this to buy Bond B for EUR100 to cover the short position.

  \item The offsetting cash flows at time $T$ leave the investor with a riskless profit of the EUR0.15 price difference between Bonds A and B at time 0 . Other investors taking note of this discrepancy will also seek to earn a riskless profit at time $t=0$ by selling Bond $\mathrm{B}$, driving its price down, and buying Bond $\mathrm{A}$, driving its price up, until the prices converge. The arbitrage opportunity disappears once the bonds have the same price $\left(S_{0}{ }^{A}=S_{0}{ }^{B}\right)$.

\end{itemize}

A second type of derivative-related arbitrage opportunity arises when an asset with a known future price does not trade at the present value (PV) of its future price. An earlier time-value-of-money lesson distinguished between discrete and continuous compounding in calculating present versus future value. The future value of a single cash flow based on a discrete number of uniform periods follows the general formula in Equation 1:

$$
\mathrm{FV}_{N}=\mathrm{PV}(1+r)^{N}
$$

where $r$ is the stated interest rate per period and $N$ is the number of compounding periods. Continuous compounding is the case in which the length of the uniform periods approaches zero, so the number of periods per year approaches infinity and is calculated using the natural logarithm, as shown in Equation 2:

$$
\mathrm{FV}_{T}=\mathrm{PV} e^{r T} \text {. }
$$

While derivatives may be priced with either approach, for purposes of this and later lessons, we will use the discrete compounding method for individual underlying assets. However, for underlying assets that represent a portfolio, such as an equity, fixed-income, commodity, or credit index, or where the underlying involves foreign exchange where interest rates are denominated in two currencies, continuous compounding will be the preferred method. Ignoring additional costs or benefits associated with asset ownership, the appropriate discount rate, $r$, is the risk-free rate, as demonstrated in Example 1.

\section{EXAMPLE 1}
\section{Spot vs. Discounted Known Future Price of Gold}
In an earlier lesson, Procam Investments entered into a contract to buy 100 ounces of gold at an agreed-upon price of USD1,792.13 per ounce in three months. In this example, Procam does the opposite trade, given a discrepancy between spot and discounted known future gold prices. Assume that today's spot gold price $\left(S_{0}\right)$ is USD1,770 per ounce, and the annualized risk-free interest rate $(r)$ is $2 \%$. For purposes of this example, we assume that Procam can borrow at the risk-free rate and gold may be stored at no cost. Under these conditions, we demonstrate how Procam can generate a riskless profit:

\begin{itemize}
  \item At time $t=0$ :

  \item Procam borrows USD177,000 at 2.0\% interest for three months and purchases 100 ounces of gold at today's spot price.

  \item Procam enters into a forward contract today to sell 100 ounces of gold at a price of USD1,792.13 per ounce in three months.

\end{itemize}

\begin{center}
\includegraphics[max width=\textwidth]{2023_05_04_36535b8d80b32081d422g-239}
\end{center}

\begin{itemize}
  \item At time $t=T$ (in three months):
\end{itemize}

\section{Arbitrage, Replication, and the Cost of Carry in Pricing Derivatives}
\begin{itemize}
  \item Procam delivers 100 ounces of gold under the forward contract and receives USD179,213 (= $100 \times$ USD1,792.13).

  \item Procam repays the loan principal with interest:

\end{itemize}

USD177,878.44 $=$ USD177,000 $(1.02)^{0.25}$.

\begin{itemize}
  \item Procam's riskless profit at time $T$ is equal to the difference between the forward sale proceeds and the loan principal and interest: USD1,334.56 = USD179,213 - USD177,878.44.
\end{itemize}

In Example 1, the spot price of gold, $S_{0}$, is below the present value of the known future price of gold in three months' time $\left(S_{0}<S_{T}(1+r)^{-T}\right.$, since USD1,770 < USD1,783.28). Procam earns a riskless profit of USD13.35 per ounce by borrowing and purchasing gold in the cash market and simultaneously selling gold in the forward market at an agreed-upon price. We would expect that as other investors recognize and pursue this opportunity, the spot price will increase (and the forward price will fall) until the spot price is equal to the discounted value of the known future price $\left(S_{0}=S_{T}(1+r)^{-T}\right)$ to eliminate this arbitrage opportunity.

The two key arbitrage concepts used to price derivatives for an underlying with no additional cash flows may be summarized as follows:

\begin{itemize}
  \item Identical assets or assets with identical cash flows traded at the same time must have the same price $\left(S_{0}{ }^{A}=S_{0}{ }^{B}\right)$.

  \item Assets with a known future price must have a spot price that equals the future price discounted at the risk-free rate $\left(S_{0}=S_{T}(1+r)^{-T}\right)$.

\end{itemize}

These arbitrage conditions establish the relationship between spot prices, forward commitment prices, and the risk-free rate shown in Exhibit 2.

Exhibit 2: Spot Prices, Forward Commitment Prices, and the Risk-Free Rate

\begin{center}
\includegraphics[max width=\textwidth]{2023_05_04_36535b8d80b32081d422g-240}
\end{center}

Exhibit 2 shows the case of $r>0$ for an asset with no additional income or costs under discrete compounding. Note that for a given time $\mathrm{T}$, the forward price will be higher relative to the spot price with a higher risk-free rate $r$. Also, for a given risk-free rate $\mathrm{r}$, as $\mathrm{T}$ increases, the forward price will increase relative to the spot price. It is important to note that the relevant risk-free rate for most market participants is the repo rate, introduced in an earlier lesson, where borrowed funds are collateralized by highly liquid securities.

Recall from an earlier lesson that a forward commitment has a symmetric payoff profile. That is, at time $T$, the transaction is settled on the basis of the difference between the forward price $F_{0}(T)$ and the underlying price of $S_{T}$, or $F_{0}(T)-S_{T}$ from the seller's perspective, as in Exhibit 3.

\section{Exhibit 3: Forward Commitment Seller Payoff Profile}
\begin{center}
\includegraphics[max width=\textwidth]{2023_05_04_36535b8d80b32081d422g-241}
\end{center}

The forward seller realizes a gain if the seller is able to deliver the underlying at a market value $\left(S_{T}\right)$ below the pre-agreed price, $F_{0}(T)$. In Example 1, Procam borrows to buy the underlying at a spot price below the present value of $S_{T}$ to lock in a riskless gain. So far, we have taken the forward price, $F_{0}(T)$, as given. In what follows, we demonstrate how no-arbitrage conditions may be used to establish the relationship between spot and forward prices by replicating or recreating an exact offsetting position for a forward commitment.

\section{REPLICATION}
Replication is a strategy in which a derivative's cash flow stream may be recreated using a combination of long or short positions in an underlying asset and borrowing or lending cash. In contrast to the earlier arbitrage examples, replication is typically used to mirror or offset a derivative position when the law of one price holds and no riskless arbitrage profit opportunities exist. For example, Exhibit 4 compares a long forward commitment to the alternative of borrowing funds at the risk-free rate, $r$, to buy the underlying asset at today's spot price, $S_{0}$.

\section{Exhibit 4: Forward Commitment Replication}
Borrow $S_{0}$ and purchase at spot $\left(S_{0}\right)$

Borrow USD1,783.28 $\left(S_{0}\right)$ today at $t=0$

Buy at $S_{0}=$ USD1,783.28

$C_{0}=(U S D 1,783.28-U S D 1,783.28)=0$

Enter long forward contact $\left(F_{0}(T)\right)$

Agree to buy at USD1,792.13 $\left(F_{0}(T)\right)$ at $t=T$

$\mathrm{CF}_{0}=0$

\begin{center}
\includegraphics[max width=\textwidth]{2023_05_04_36535b8d80b32081d422g-242}
\end{center}

0 Time $\quad$ T Settle contract $\left(S_{T}-F_{0}(T)\right)$ at $t=T$

$\mathrm{CF}_{T}=\left(S_{T}-\mathbf{\$ 1 , 7 9 2 . 1 3 )}\right.$

\section{EXAMPLE 2}
\section{Replication of a Forward Commitment}
In Example 1, we reintroduced Procam Investments, which borrowed and purchased gold in the spot market and simultaneously sold gold in the forward market to earn a riskless profit. Here we change the assumption about today's spot gold price $\left(S_{0}\right)$. Specifically, the spot gold price has risen to USD1,783.28 $\left(=\right.$ USD1,792.13(1.02) $\left.{ }^{-0.25}\right)$ to eliminate the earlier arbitrage opportunity when the spot price was USD1,770. Assume again that the risk-free interest rate $(r)$ is $2 \%$ and gold can be stored at no cost.

\begin{enumerate}
  \item Long forward commitment (agree to buy at $F_{0}(T)$ at $t=T$ )
\end{enumerate}

\begin{itemize}
  \item Procam enters a forward commitment to buy 100 ounces of gold in three months at a forward price, $F_{0}(T)$, of USD1,792.13 per ounce.

  \item If $S_{T}=$ USD1,900 per ounce,

  \item $\quad$ Profit $=$ USD10,787 $=100 \times(U S D 1,900-U S D 1,792.13)$.

  \item If $S_{T}=\mathrm{USD} 1,700$ per ounce,

  \item $\quad$ Profit $=-\operatorname{USD}(9,213=100 \times(\mathrm{USD} 1,700-U S D 1,792.13)$.

\end{itemize}

\begin{enumerate}
  \setcounter{enumi}{1}
  \item Borrow and purchase (borrow $S_{0}$ and buy asset at $S_{0}$ at $t=0$ ):
\end{enumerate}

\begin{itemize}
  \item Procam borrows USD178,328 at $2 \%$ and buys 100 ounces of gold at today's spot price, $S_{0}$.

  \item Procam sells the gold in three months at spot price $S_{T}$.

  \item Procam repays the loan principal and interest: USD179,213 = USD178,328(1.02) 0.25

  \item If $S_{T}=\mathrm{USD} 1,900$,

  \item $\quad$ Profit $=$ USD10,787 $=100 \times(U S D 1,900-U S D 1,792.13)$.

  \item If $S_{T}=\mathrm{USD} 1,700$

  \item $\quad$ Profit $=-\mathrm{USD} 9,213=100 \times(\mathrm{USD} 1,700-\mathrm{USD} 1,792.13)$. Note that the same profits are observed under Scenarios 1 and 2, which are equal to $S_{T}-F_{0}(T)$. If we set the forward price, $F_{0}(T)$, equal to the future value of the spot rate using the risk-free rate $\left(F_{0}(T)=S_{0}(1+r)^{T}\right)$, the no-arbitrage condition demonstrates that Procam generates the same cash flow at time $T$ regardless of the direction of gold prices whether the company

\end{itemize}

\begin{enumerate}
  \item enters into a long forward commitment settled at time $T$ or

  \item borrows at the risk-free rate, buys the underlying asset, and holds it until time $T$.

\end{enumerate}

A separate but related form of replication pairing a long asset with a short forward is shown in Exhibit 5. The top half shows the profit/loss profile of the long asset, and the bottom half shows that of the short forward position. The long asset position produces a profit (loss) when $S_{T}>S_{0}\left(S_{T}<S_{0}\right)$. The short forward position produces a profit (loss) when $S_{T}<F_{0}\left(S_{T}>F_{0}\right)$.

\section{Exhibit 5: Payoffs for Long Asset and Short Forward Positions}
\begin{center}
\includegraphics[max width=\textwidth]{2023_05_04_36535b8d80b32081d422g-243}
\end{center}

In contrast to Example 2, these positions appear to offset one another. The following example evaluates the return generated by these combined positions.

\section{EXAMPLE 3}
\section{Risk-Free Trade Replication: Long Asset, Short Forward}
Procam Investments buys 100 ounces of gold at today's spot price $\left(S_{0}\right)$ of USD1,783.28 and simultaneously enters a forward commitment to sell gold at the forward price, $F_{0}(T)$, of USD1,792.13. Again, we assume that gold can be stored at no cost, and here we solve for Procam's rate of return.

\begin{itemize}
  \item At $t=0$, Procam's cash flow is $-S_{0}=-$ USD178,328.

  \item At $t=T$, Procam's cash flow is $F_{0}(T)=$ USD179,213.

  \item Solve for the rate of return on Procam's strategy, as follows:

  \item USD179,213 = USD178,328 $(1+r)^{0.25}$.

  \item $r=2.0 \%$, which is equal to the risk-free rate.

\end{itemize}

This example demonstrates that Procam can hedge its long gold cash position with a short derivative (i.e., selling the forward contract) and earn the risk-free rate of return as long as the no-arbitrage condition $\left(F_{0}(T)=S_{0}(1+r)^{T}\right)$ holds. The combined return is shown in Exhibit 6 . Note that if Procam borrows at the risk-free rate to purchase the underlying asset at $S_{0}$, it will earn zero.

\section{Exhibit 6: Combined Long Asset and Short Forward Profit}
\begin{center}
\includegraphics[max width=\textwidth]{2023_05_04_36535b8d80b32081d422g-244}
\end{center}

Because the forward price is assumed to be greater than the spot price in this example and $r$ is positive, the risk-free profit is a positive amount and is the same (i.e., risk free) regardless of the price of the underlying (i.e., $S_{T}$ ). As we will see in the next lesson, there may be benefits or costs to owning an underlying asset that cause the risk-free return in Exhibit 6 to differ from $F_{0}(T)-S_{0}$ and possibly even be negative.

\section{KNOWLEDGE CHECK}
\section{Arbitrage and Replication}
\begin{enumerate}
  \item Determine the correct answers to fill in the blanks: If the law of one price does not hold, a(n) asset trades at the same time at prices in different places.
\end{enumerate}

\section{Solution:}
If the law of one price does not hold, a(n) identical asset trades at the same time at different prices in different places.

\begin{enumerate}
  \setcounter{enumi}{1}
  \item An investor observes that the spot price, $\mathrm{S}_{0}$, of an underlying asset with no additional costs or benefits exceeds its known future price discounted at the risk-free rate, $S_{T}(1+r)^{-T}$. Describe and justify an arbitrage strategy that generates a riskless profit for the investor.
\end{enumerate}

\section{Solution:}
Since the spot price of the underlying asset exceeds the known future price discounted at the risk-free rate $\left(S_{0}>F_{0}(T)(1+r)^{-T}\right)$, at $t=0$, the investor:

\begin{itemize}
  \item Sells the underlying asset short in the spot market at $S_{0}$

  \item Simultaneously enters a long forward contract at $F_{0}(T)$

  \item Lends $S_{0}$ at the risk-free rate $r$ to receive $S_{0}(1+r)^{T}$ at time $T$.

\end{itemize}

At time $t=T$, the investor:

\begin{itemize}
  \item Settles the long forward position and receives $S_{\mathrm{T}}-F_{0}(T)$

  \item Offsets the short underlying asset position at $S_{T}$, and

  \item Retains $S_{0}(1+r)^{T}-F_{0}(T)$ as a riskless profit regardless of the underlying spot price at time $T$.

\end{itemize}

\begin{enumerate}
  \setcounter{enumi}{2}
  \item Formulate a replication strategy for a three-month short forward commitment for 1,000 shares of a non-dividend-paying stock.
\end{enumerate}

\section{Solution:}
The replication strategy for a three-month short forward commitment on a non-dividend-paying stock involves the short sale of 1,000 shares of stock at $t=0$ and investment of proceeds at the risk-free rate, $r$. At time $t=T$, the short sale is covered at $S_{T}$, and under the no-arbitrage condition of $F_{0}(T)=$ $S_{0}(1+r)^{T}$, the return is equal to $F_{0}(T)-S_{T}$ for both the short forward and the replication strategy.

\begin{enumerate}
  \setcounter{enumi}{3}
  \item Calculate the arbitrage profit if a spot asset with no additional costs or benefits trades at a spot price of 100 , the three-month forward price for the underlying asset is 102 , and the risk-free rate is $5 \%$.
\end{enumerate}

\section{Solution:}
The forward price, $F_{0}(T)=S_{0}(1+r)^{T}$, at which no-arbitrage opportunities would exist is $101.23\left(=100(1.05)^{0.25}\right)$. With an observed forward price of 102, the arbitrage opportunity would be to sell the forward contract and buy the underlying, borrowing at the risk-free rate to fund the purchase. The arbitrage profit is the difference between the observed forward price and the no-arbitrage forward price of 0.77 (= $102-101.23)$. 5. Describe the relationship between the spot and forward price for an underlying asset if the risk-free rate is negative.

\section{Solution:}
The relationship between the spot and forward rate for an asset with no additional costs or benefits of ownership other than the opportunity cost (risk-free rate) is equal to $F_{0}(T)=S_{0}(1+r)^{T}$. In a case where $r<0,(1+r)^{T}$ $<1$ and therefore the forward price, $F_{0}(T)$, is below the spot price, $S_{0}$, if the risk-free rate is negative.

\section{4}
\section{COSTS AND BENEFITS ASSOCIATED WITH OWNING THE UNDERLYING}
explain the difference between the spot and expected future price of an underlying and the cost of carry associated with holding the underlying asset

In the prior lesson, replication was used to illustrate the basic relationship between entering into a spot transaction versus a forward commitment. The linkage between the spot price of an asset with no associated cash flows and a forward commitment on the same asset was shown to be the risk-free rate of interest, $r$. In this lesson, we discuss cost of carry as the net of the costs and benefits related to owning an underlying asset for a specific period.

In the forward commitment example from the prior lesson, where no costs or benefits were associated with the underlying asset, the following relationship between the spot and forward prices was established:

$$
F_{0}(T)=S_{0}(1+r)^{T}
$$

This relationship is shown under continuous compounding in Equation 4:

$$
F_{0}(T)=S_{0} e^{r T}
$$

The risk-free rate, $r$, denotes the opportunity cost of holding ("carrying") the asset, whether or not the long investor borrows to finance the asset. This opportunity cost is present for all asset classes discussed below.

Equations 3 and 4 represent the special case of an underlying asset with no additional associated cash flows. However, many assets have additional costs or benefits of ownership that must be reflected in the forward commitment price in order to prevent riskless arbitrage opportunities from arising between underlying spot and derivative prices. Exhibit 7 demonstrates the effect of costs and benefits (usually dividend or interest income) on the spot price, forward commitment price, and risk-free rate relationships. Exhibit 7: Spot Prices, Forward Commitment Prices, and the Risk-Free Rate with Underlying Asset Costs and Benefits

\begin{center}
\includegraphics[max width=\textwidth]{2023_05_04_36535b8d80b32081d422g-247}
\end{center}

If an underlying asset owner incurs costs in addition to the opportunity cost, she should expect to be compensated for these added costs through a higher forward price, $F_{0}{ }^{+}(T)$. Income or other benefits accrue to the underlying asset owner and therefore should reduce the forward price to $F_{0}^{-}(T)$.

For underlying assets with ownership benefits or income $(I)$ or costs $(C)$ expressed as a known amount in present value terms at $t=0$-shown as $\mathrm{PV}_{0}()$-the relationship between spot and forward prices in discrete compounding terms can be shown as

$$
F_{0}(T)=\left[S_{0}-\mathrm{PV}_{0}(I)+\mathrm{PV}_{0}(C)\right](1+r)^{T}
$$

In other instances, the additional costs or benefits are expressed as a rate of return over the life of the contract. For income $(i)$ and cost $(c)$ expressed as rates of return, the relationship between spot and forward prices under continuous compounding is

$$
F_{0}(T)=S_{0} e^{(r+c-i) T}
$$

Whether expressed as a known amount in present value terms or as a rate of return, the forward price must incorporate the net effect of all costs and benefits associated with owning the underlying asset, including the following.

\begin{itemize}
  \item Opportunity cost (risk-free interest rate, $r$ ): A positive risk-free rate causes a forward price to be greater than the underlying spot price, all else equal, and the higher the risk-free rate, the greater the positive difference between the two. This opportunity cost applies to any asset.

  \item Other costs of ownership $(\boldsymbol{C}, \boldsymbol{c})$ : Owners of some underlying assets, such as physical commodities, must incur storage, transportation, insurance, and/or spoilage costs. An owner entering a contract for future delivery will expect to be compensated for these costs, resulting in a forward price that is therefore greater than the underlying spot price, all else equal.

  \item Benefits of ownership ( $I, i)$ : Alternatively, the owners of some underlying assets enjoy cash flow or other benefits associated with owning the underlying asset as opposed to a derivative on the asset. A counterparty entering a derivative contract for future delivery of an underlying asset forgoes these benefits and will therefore reduce the forward price by this amount. Stock dividends or bond coupons are examples of cash flow benefits.

\end{itemize}

Exhibit 8 illustrates the relationship between forward and spot prices in the presence of costs and benefits. For example, when the opportunity cost and other costs of ownership exceed the benefits, the forward price will be above the underlying spot asset price. Exhibit 8: Forward vs. Spot Price

Cost vs. Benefit

Opportunity and Other Cost $>$ Benefit

Opportunity and Other Cost $<$ Benefit

Opportunity and Other Cost $=$ Benefit Forward vs. Spot Price

$$
\begin{aligned}
& F_{0}(T)>S_{0} \\
& F_{0}(T)<S_{0} \\
& F_{0}(T)=S_{0}
\end{aligned}
$$

Each underlying asset type has different costs and benefits, which may vary over time and across markets. For example, owners of some individual equity securities receive the benefit of regular stock dividends, as in Example 4, while others do not. For equity indexes, the benefit is usually expressed as a rate of return, as shown in Example 5.

\section{EXAMPLE 4}
\section{Hightest Equity Forward with Dividend}
Assume Hightest Capital agrees to deliver 1,000 Unilever (UL) shares at an agreed-upon price to a financial intermediary in six months under a forward contract. Assume that UL has a spot price $\left(S_{0}\right)$ of EUR50 and pays no dividend $(I=0)$, and assume a risk-free rate $(r)$ of $5 \%$. We may use Equation 3 to solve for the forward price, $F_{0}(T)$, in six months:

$$
F_{0}(T)=S_{0}(1+r)^{T} \text {. }
$$

EUR51.23 $=\operatorname{EUR50}(1.05)^{0.5}$.

Now assume instead that Unilever pays a quarterly dividend of EUR0.30, which occurs in exactly three months and again at time $T$, with other details unchanged. Use Equation 5 with $\mathrm{PV}_{0}(C)=0$ to solve for $F_{0}(T)$ in six months:

$$
F_{0}(T)=\left[S_{0}-\mathrm{PV}_{0}(I)\right](1+r)^{T}
$$

First, solve for the present value of the dividend per share, $\mathrm{PV}_{0}(I)$, as follows:

$$
\operatorname{PV}_{0}(I)=\operatorname{EUR} 0.30(1.05)^{-0.25}+\operatorname{EUR} 0.30(1.05)^{-0.5}
$$

EUR0.5892 $=0.2964+0.2928$

Substitute $\mathrm{PV}_{0}(I)=$ EUR0.5892 into Equation 5 to solve for $F_{0}(T)$ :

$$
\begin{aligned}
& F_{0}(T)=\left(\text { EUR50 }- \text { EUR0.5892) }(1.05)^{0.5}\right. \\
& =\text { EUR50.6310. }
\end{aligned}
$$

Hightest's forward contract for 1,000 UL shares would therefore be priced at EUR50,631.00 $(=1,000 \times$ EUR50.6310). The opportunity cost of borrowing at the risk-free rate is EUR1.23 per share for six months and equals $F_{0}(T)-S_{0}$ when Unilever pays no dividend. A EUR0.30 quarterly dividend reduces the difference between the forward and spot price to approximately EUR0.63.

\section{EXAMPLE 5}
\section{Stock Index Futures with Dividend Yield}
The Viswan Family Office (VFO) would like to enter into a three-month forward commitment contract to purchase the NIFTY 50 benchmark Indian stock market index traded on the National Stock Exchange. The spot NIFTY 50 index price is INR15,200, the index dividend yield is $2.2 \%$, and the Indian rupee risk-free rate is $4 \%$. Use Equation 6 (with $c=0$ ) to solve for the forward price:

$$
\begin{aligned}
& F_{0}(T)=S_{0} e^{(r+c-i) T} \\
& =15,200 e^{(0.04-0.022) 0.25} \\
& =\operatorname{INR} 15,268.55 .
\end{aligned}
$$

Foreign exchange requires some adjustments to establish the no-arbitrage condition between spot and forward prices, as shown in the prior lesson. First, it is important to distinguish the "price" of the underlying asset, which for equities, fixed income, or commodities refers to units of currency for each asset-for example, one share of stock or the principal amount for a bond. In contrast, the foreign exchange rate is expressed as a spot rate $\left(S_{0, f l d}\right)$ specifying the number of units of a price currency (here denoted as $f$ or foreign currency) in the numerator per single unit of a base currency (here shown as $d$ or domestic currency) in the denominator. For example, for a USD/ EUR spot rate $\left(S_{0, f l d}\right)$ of 1.20 , the US dollar is the price currency $(f)$, and the euro is the base currency $(d)$, with USD1.20 equal to EUR1.

An FX (foreign exchange) forward contract involves the sale of one currency and purchase of the other on a future date at a forward price $\left(F_{0, f l d}\right)$ agreed on at inception. A long FX forward position involves the purchase of the base currency and the sale of the price currency. For example, a long USD/EUR FX forward position is the sale of US dollars and purchase of euros at a forward rate.

In the prior replication example, the derivative cash flow stream was recreated by combining a long or short position in an underlying asset and borrowing or lending cash. Here the foreign and domestic currency each has an opportunity cost-namely, the foreign risk-free rate $\left(r_{f}\right)$ and the domestic risk-free rate $\left(r_{d}\right)$, respectively.

\section{EXAMPLE 6}
\section{AUD/USD Foreign Exchange Forward Replication}
In order to replicate one currency's return in terms of the other for a given spot price today of $S_{0, f l d}$, we may solve for a forward rate $F_{0, f l d}(T)$ using the earlier arbitrage concept that assets with a known future price must have a spot price $\left(S_{0, f l d}\right)$ equal to the future price discounted at the risk-free rate.

Assume the current AUD/USD spot price is 1.3335. The Australian dollar is the price currency or foreign currency, and the US dollar is the base or domestic currency (AUD1.3335 = USD1). The six-month Australian dollar risk-free rate is $0.05 \%$, and the six-month US dollar risk-free rate is $0.20 \%$.

\begin{itemize}
  \item At time $t=0$
\end{itemize}

\begin{enumerate}
  \item Borrow USD1,000 at the $0.20 \%$ US dollar risk-free rate for six months.

  \item Purchase AUD1,333.50 at the AUD/USD spot rate $\left(S_{0, f / d}=1.3335\right)$.

  \item Lend the AUD1,333.50 received at the $0.05 \%$ Australian dollar risk-free rate for six months. - At time $t=T$ in six months:

  \item Receive Australian dollar loan proceeds of $1,333.83$ (= $\left.1,333.50 e^{(0.0005 \times 0.5)}\right)$.

  \item Exchange Australian dollar proceeds for US dollars at $S_{T, f / d}$ to repay the US dollar loan.

  \item Repay the US dollar loan (with interest) of $1,001\left(=1,000 e^{(0.002 \times}\right.$ $\left.0^{0.5)}\right)$.

\end{enumerate}

If the exchange in Step 5 at time $T$ is at a spot price, $S_{T, f \mid d}$, at which the Australian dollar loan proceeds (Step 4) exactly offset the US dollar loan (Step 6), no riskless arbitrage opportunity exists. Solve for $S_{T, f l d}$ by dividing the Australian dollars in Step 4 by the US dollars in Step 6:

$$
S_{T, f l d}=1.3325(=\mathrm{AUD} 1,333.83 / \mathrm{USD} 1,001)
$$

The following diagram summarizes the cash flows at time $t=0$ and time $T$ :

\begin{center}
\includegraphics[max width=\textwidth]{2023_05_04_36535b8d80b32081d422g-250(1)}
\end{center}

For the no-arbitrage condition to hold between the FX spot and forward price, the amount of Australian dollars necessary to purchase USD1 at time $T\left(F_{0, f / d}(T)\right)$ must have a spot price $\left(S_{0, f / d}\right)$ equal to the discounted future price. The relevant discount rate here involves the difference between the foreign and domestic risk-free rates, as shown in the following modified version of Equation 4:

\begin{center}
\includegraphics[max width=\textwidth]{2023_05_04_36535b8d80b32081d422g-250}
\end{center}

Solve for $F_{0, A U D / U S D}(T)$ in Example 6 as follows:

$$
F_{0, f l d}(T)=1.3325\left(=1.3335 e^{(-0.0015 \times 0.5)}\right) \text {. }
$$

From Equation 7, we see that it is the risk-free interest rate differential $\left(r_{f}-r_{d}\right)$, rather than the absolute level of interest rates, that determines the spot versus forward FX price relationship. For example, in Example 6, the Australian dollar risk-free rate is $0.15 \%$ below the US dollar rate $\left(r_{f}-r_{d}<0\right)$. Borrowing at the higher US dollar rate and lending at the lower Australian dollar rate results in a no-arbitrage forward price at which fewer Australian dollars are required to purchase USD1 in the future, so the Australian dollar is said to trade at a premium in the forward market versus the US dollar.

\section{Exhibit 9: FX Forward vs. Spot Price Relationship}
\begin{center}
\begin{tabular}{|c|c|c|c|}
\hline
$\begin{array}{l}\text { Interest Rate } \\ \text { Differential }\end{array}$ & $\begin{array}{c}\text { Forward vs. Spot } \\ \text { Price }\end{array}$ & $\begin{array}{l}\text { Foreign Currency } \\ \text { Forward }\end{array}$ & $\begin{array}{c}\text { FX Forward Premium/ } \\ \text { Discount }\end{array}$ \\
\hline
$\left(r_{f}-r_{d}\right)>0$ & $F_{0, f l d}(T)>S_{0, f l d}$ & Discount & Premium \\
\hline
$\left(r_{f}-r_{d}\right)<0$ & $F_{0, f / d}(T)<S_{0, f l d}$ & Premium & Discount \\
\hline
$\left(r_{f}-r_{d}\right)=0$ & $F_{0, f l d}(T)=S_{0, f l d}$ & $\begin{array}{c}\text { Neither a premium nor } \\ \text { a discount }\end{array}$ & $\begin{array}{c}\text { Neither a premium nor } \\ \text { a discount }\end{array}$ \\
\hline
\end{tabular}
\end{center}

We now examine this FX spot versus forward relationship in the case of the Montau AG example from an earlier lesson.

\section{EXAMPLE 7}
\section{Montau AG's FX Forward Rate}
An earlier lesson introduced Montau AG, a German capital goods producer. Montau signs a commercial contract with Jeon, Inc., a South Korean manufacturer, to deliver a laser cutting machine at a price of KRW650 million in 75 days. Montau faces a timing mismatch between domestic euro costs incurred and euro revenue realized upon the delivery of the machine and sale of South Korean won received in exchange for euros in the spot FX market. Montau enters a long KRW/EUR FX forward contract. That is, Montau agrees to sell South Korean won and purchase euros at a fixed price, $F_{0}(T)$, in 75 days to eliminate the KRW/EUR exchange rate mismatch arising from the export contract.

In this version of the Montau AG example, we use today's spot exchange rate and the domestic $\left(r_{d}\right)$ and foreign $\left(r_{f}\right)$ risk-free rates to solve for the KRW/EUR forward rate. As seen in Equation 7 , the difference between spot and forward FX rates involves the difference in risk-free rates. In the KRW/EUR case, the South Korean won is the price currency or foreign currency and $r_{f}$ is therefore the South Korean won interest rate. The euro is the base or domestic currency. We may therefore rewrite Equation 7 as follows:

$$
F_{0, K R W / E U R}(T)=S_{0, K R W / E U R} e^{(r} K R W^{-r} E U R{ }^{) T}
$$

Assume a spot KRW/EUR rate $\left(S_{0}\right)$ of 1,300 (that is, KRW1,300= EUR1), a South Korean won risk-free rate of $0.75 \%$, and a euro risk-free rate of $-0.25 \%\left(r_{d}\right)$. Calculate the KRW/EUR forward rate in 75 days consistent with no arbitrage.

$$
F_{0, K R W / E U R}(T)=1,300 \times e^{(0.0075+0.0025) \times(75 / 365)} \text {. }
$$

Solving for the KRW/EUR forward rate gives us $F_{0}(T)=1,302.67$. Notice that the $1 \%$ difference between South Korean won and euro interest rates leads to a forward price that is above the spot price $\left(F_{0, f l d}(T)>S_{0, f l d}\right)$. That is, in six months' time, more South Korean won will be required to purchase EUR1, and the South Korean won is said to trade at a forward discount versus the euro.

To demonstrate the no-arbitrage condition between the forward and spot rates, assume that Montau converts the KRW650 million into euros at the 1,300 KRW/ EUR spot rate to receive EUR500,000 (= KRW650,000,000/1,300) and invests this for 75 days at a continuously compounded $r_{f}=-0.25 \%$ :

EUR499, 743.22 $=\operatorname{EUR500,000~} \times e^{[-0.0025 \times(75 / 365)]}$

Assume that Jeon Inc. invests the KRW650 million at the South Korean won risk-free rate $\left(r_{d}=0.75 \%\right)$ for 75 days to receive

\section{Arbitrage, Replication, and the Cost of Carry in Pricing Derivatives}
\begin{center}
\includegraphics[max width=\textwidth]{2023_05_04_36535b8d80b32081d422g-252}
\end{center}

An arbitrage-free forward commitment price should therefore allow Montau to convert KRW651,002,484.59 into EUR499,743.22 after 75 days.

$$
F_{0}(T)=651,002,484.59 / 499,743.22=1,302.67
$$

This confirms that the two prices (spot and forward exchange rates) are consistent with the risk-free interest rate differential in the two different currencies.

In contrast to securities or cash stored electronically, commodities usually involve known costs associated with the storage, insurance, transportation, and potential spoilage (in the case of soft commodities) of these physical assets. A non-cash benefit of holding a physical commodity versus a derivative is known as a convenience yield. In physical goods markets, economic conditions may arise that cause market participants to prefer to own the physical commodity. As a simple example, if crude oil inventories are very low, refineries may bid up the spot oil price so that forward prices do not fully reflect storage costs and interest rates. The following example illustrates the impact of these carry costs on the relationship between spot and forward commitment prices for a commodity, as well as the possibility of a convenience yield.

\section{EXAMPLE 8}
\section{Procam's Gold Forward Contract with Storage Costs}
Recall from earlier examples that Procam borrowed and purchased gold in the spot market and simultaneously sold gold in the forward market for three months to earn a riskless profit. Under the assumption of a $2 \%$ risk-free rate, we demonstrated that the spot gold price $\left(S_{0}\right)$ would need to rise to USD1,783.28 $(=$ USD1,792.13(1.02) $\left.{ }^{-0.25}\right)$ in order to eliminate the earlier arbitrage opportunity for a given USD1,792.13 forward price where gold may be stored at no cost.

Given the spot price of USD1,783.28, how would the forward gold price change to satisfy the no-arbitrage condition if we were to introduce a USD2 per ounce cost of gold storage and insurance payable at the end of the contract?

The forward commitment price for a commodity with known storage cost amounts may be determined using Equation 5 , where $\mathrm{PV}_{0}(I)=0$ :

$$
F_{0}(T)=\left[S_{0}+\mathrm{PV}_{0}(C)\right](1+r)^{T}
$$

First, solve for the present value of the storage cost per ounce $\mathrm{PV}_{0}(C)$ as follows:

$$
\begin{aligned}
& \mathrm{PV}_{0}(C)=\mathrm{USD} 2(1.02)^{-0.25} \\
& =\mathrm{USD} 1.99
\end{aligned}
$$

Substitute $\mathrm{PV}_{0}(C)=$ USD1.99 into Equation 5 to solve for $F_{0}(T)$ :

$$
\begin{aligned}
& F_{0}(T)=(\mathrm{USD} 1,783.28+\mathrm{USD} 1.99)(1.02)^{0.25} \\
& =\text { USD1,794.13. }
\end{aligned}
$$

Note that the addition of storage and insurance costs increases the difference between the spot and forward price. Finally, note that a forward price, $F_{0}(T)$, significantly below the no-arbitrage price may indicate the presence of a convenience yield.

The additional costs and benefits of underlying asset ownership are summarized in Exhibit 10. Exhibit 10: Cost of Carry for Underlying Assets

\begin{center}
\includegraphics[max width=\textwidth]{2023_05_04_36535b8d80b32081d422g-253}
\end{center}

Interest rates and credit have a term structure-that is, different prices or rates for different maturities. These forward contracts are addressed in a later lesson.

\section{KNOWLEDGE CHECK}
\section{Cost of Carry}
\begin{enumerate}
  \item Describe the relationship between the spot and forward price for an underlying asset whose benefits exceed the opportunity and other costs of ownership.
\end{enumerate}

\section{Solution:}
If the benefits for an owner of an underlying asset exceed the opportunity and other costs of owning the underlying asset, the spot price will be greater than the forward price.

\begin{enumerate}
  \setcounter{enumi}{1}
  \item Identify which example corresponds to each of the following relationships between the spot and forward rate:

  \item $F_{0}(T)>S_{0}$

  \item $F_{0}(T)<S_{0}$

  \item Not enough information to determine the relationship between $F_{0}(T)$ and $S_{0}$ A. A fixed-coupon bond priced at par whose coupon is above the risk-free rate B. A foreign currency forward where the domestic risk-free rate is greater than the foreign risk-free rate

\end{enumerate}

C. A commodity with a convenience yield as well as storage and insurance costs

\section{Solution:}
\begin{enumerate}
  \item B is correct. The FX forward rate is greater than the spot rate if the domestic risk-free rate is greater than the foreign risk-free rate.
\end{enumerate}

\section{Arbitrage, Replication, and the Cost of Carry in Pricing Derivatives}
\begin{enumerate}
  \setcounter{enumi}{1}
  \item A is correct. A fixed-coupon bond priced at par has an income that exceeds the opportunity cost of the risk-free rate, so $\mathrm{F}_{0}(\mathrm{~T})<\mathrm{S}_{0}$.

  \item $\mathrm{C}$ is correct. We do not have enough information to fully evaluate the benefit (convenience yield) versus the cost (risk-free rate, storage, and insurance) of holding the physical commodity asset in this example.

  \item Determine the correct answers to fill in the blanks: A positive risk-free rate causes a forward price to be than the underlying spot price, all else equal, and the higher the risk-free rate, the the difference between the two.

\end{enumerate}

Solution:

A positive risk-free rate causes a forward price to be greater than the underlying spot price, all else equal, and the higher the risk-free rate, the greater the difference between the two.

\begin{enumerate}
  \setcounter{enumi}{3}
  \item An analyst observes that the current spot MXN/USD exchange rate is 19.50, the Mexican peso six-month risk free rate is $4 \%$, and the six-month US dollar risk free rate is $0.25 \%$. Describe the relationship between the MXN/USD spot and six-month forward rate, and justify your answer.
\end{enumerate}

\section{Solution:}
The relationship between the MXN/USD FX spot and forward price depends on the risk-free interest rate differential between the Mexican peso and the US dollar $\left(r_{f}-r_{\mathrm{d}}>0\right.$, since $\left.3.75 \%=4.0 \%-0.25 \%\right)$. Since the Mexican peso rate is $3.75 \%$ above the US dollar rate, we would expect that borrowing at a higher Mexican peso rate and lending at a lower US dollar rate would result in a no-arbitrage forward price at which more Mexican pesos are required to purchase USD1 in the future, so the Mexican peso is said to trade at a discount in the forward market versus the US dollar $\left(F_{0, M X N / U S-}\right.$ $\left.D^{(T)}>S_{0, M X N / U S D}\right)$.

We can show this using Equation 7 for the case of MXN/USD as follows:

$F_{0, M X N / U S D}(T)=S_{0, M X N / U S D} e^{\left(r^{r}\right.}$ MXN $^{-r} U S D{ }^{2}$.

Based on the current $S_{0, f l d}$ of 19.50, $r_{f}=4 \%, r_{d}=0.25 \%$, and $T$ of 0.5 , we may solve for the no-arbitrage forward price, $F_{0, f / d}$, as

$$
\begin{aligned}
& F_{0, M X N / U S D}(T)=19.50 e^{[(0.04-0.0025) \times 0.5]} \\
& =19.8691 .
\end{aligned}
$$

\begin{enumerate}
  \setcounter{enumi}{4}
  \item Assume that new, stricter environmental regulations associated with oil storage and insurance cause the cost of these services to increase sharply. Describe the anticipated effect of these increased costs on the relationship between oil spot and forward commitment prices.
\end{enumerate}

\section{Solution:}
Owners of physical commodities who must incur storage and insurance costs over time expect to be compensated by higher forward commitment prices for future delivery of the underlying assets. Assuming other factors are constant, higher storage and insurance costs therefore lead to higher forward commitment prices and a greater difference between spot and forward prices. As shown in Equation 5, an increase in $C$ increases $F_{0}(T)$ for a given $S_{0}$ :

$$
F_{0}(T)=\left[S_{0}-\mathrm{PV}_{0}(I)+\mathrm{PV}_{0}(C)\right](1+r)^{T} \text {. }
$$

\section{KNOWLEDGE CHECK}
Rook Point Investors LLC is a firm based in Auckland, New Zealand, that employs trading strategies to exploit mispricing opportunities across markets and geographies. A newly hired financial analyst at Rook Point has been asked to address the following questions.

\begin{enumerate}
  \item Identify which of the following activities corresponds to which replication strategy.

  \item Sell the asset short for $S_{0}$ at $t=0$, and lend proceeds of the asset sale at the risk-free rate, $r$. At time $t=T$, buy back the asset at the spot price, $S_{T}$.

  \item Purchase an asset at today's spot

\end{enumerate}

B. Long forward replication price $\left(S_{0}\right)$, and simultaneously enter into a forward commitment to sell the asset at the forward price, $F_{0}(T)$.

3 . Borrow at the risk-free rate, $r$, and

C. Short forward replication A. Risk-free trade replication buy the underlying asset at today's spot price $\left(S_{0}\right)$. At time $T$, sell the asset at the spot price $\left(S_{T}\right)$. Repay the loan principal and interest $\left(S_{0}(1+r)\right.$ $T$ ) at time $T$.

\begin{enumerate}
  \item $C$ is correct. The short sale of an asset at time $t=0$ creates a position that replicates a short forward, as the return is equal to $F_{0}(T)-S_{T}$ for both a short forward and the replication strategy.

  \item A is correct. This long cash, short derivative position earns the risk-free rate $r$ as long as the no-arbitrage condition $\left(F_{0}(T)=S_{0}(1+r)^{T}\right)$ holds. If the investor instead borrows at the risk-free rate, $r$, to purchase the underlying asset at $S_{0}$, the return is zero.

  \item B is correct. This strategy returns $S_{T}-F_{0}(T)$, as in the case of a long forward commitment, with $F_{0}(T)=S_{0}(1+r)^{T}$ under the no-arbitrage condition.

  \item The analyst observes a current USD/EUR spot exchange rate of 1.192 (that is, USD1.192 = EUR1), a US dollar risk-free rate of $0.5 \%$, and a euro risk-free rate of $-0.25 \%$. Which of the following statements describes the action Rook Point can take today to earn a riskless profit if the one-year USD/EUR forward rate is observed to be 1.201 ?

\end{enumerate}

A. No arbitrage opportunity exists, because the observed one-year USD/ EUR FX forward rate equals the no-arbitrage rate capturing the net effect of the domestic versus foreign risk-free rate for one year.

B. Since the no-arbitrage one-year USD/EUR forward rate is 1.195 , Rook Point should borrow in euros and buy US dollars today, simultaneously selling US dollars against euros one year forward.

C. Since the no-arbitrage one-year USD/EUR forward rate is 1.195, Rook Point should borrow in US dollars and buy euros today, simultaneously buying US dollars and selling euros one year forward.

A is correct. Recall from Equation 7 that the spot versus forward relationship for a foreign exchange may be shown as follows:

$$
\left.F_{0, f l d}(T)=S_{0, f l d} e^{(r-r)}{ }^{(r}\right) T
$$

In the specific case of USD/EUR, the equation may be rewritten as $F_{0, U S D / E U R}(T)=S_{0, U S D / E U R} e^{(r}{ }_{U S D}{ }_{E U}{ }_{E U R}{ }^{T}$.

If $S_{0}=1.192, r_{f}=0.5 \%$, and $r_{d}=-0.25 \%$, the no-arbitrage forward price in one year equals $1.201\left(=1.192 e^{(0.005+0.0025)}\right)$ and no riskless profit opportunity exists.

\section{KNOWLEDGE CHECK}
Rook Point has asked the analyst to review the platinum market for investment opportunities. The analyst incorporates platinum storage and handling costs of USD1 per ounce payable at the end of each quarter and observes a risk-free rate of $1.5 \%$. The following table represents today's platinum spot and futures prices (per ounce of platinum):

\begin{center}
\begin{tabular}{cccc}
\hline
 & \multicolumn{2}{c}{Observed Futures Prices} &  \\
\hline
Spot & 3-month futures & 6-month futures & One-year futures \\
\hline
USD1,097 & USD1,102.09 & USD1,107.50 & USD1,115.00 \\
\hline
\end{tabular}
\end{center}

\begin{enumerate}
  \item Which of the following activities maximizes Rook Point's riskless profit in the platinum market?
\end{enumerate}

A. Sell platinum short in the spot market, and simultaneously buy the one-year platinum futures contract.

B. Borrow to purchase and store platinum for six months, and simultaneously sell the one-year platinum futures contract.

C. Sell platinum short in the spot market, and simultaneously buy the three-month futures contract.

A is correct. First, solve for the no-arbitrage price for three-month, six-month, and one-year platinum futures prices using Equation 5 with $\mathrm{PV}_{0}(I)=0$ :

$F_{0}(T)=\left[S_{0}+\mathrm{PV}_{0}(C)\right](1+r)^{T}$

with $S_{0}=$ USD1,097, $r=1.5 \%$ and $C=$ USD1 to obtain the following: $F_{0}(3 \mathrm{~m})=\left[\mathrm{USD} 1,097+(1.015)^{-0.25}\right](1.015)^{0.25}$.

USD1,102.09 $=(\mathrm{USD} 1,097+0.9963)(1.015)^{0.25}$.

$F_{0}(6 \mathrm{~m})=\left[\mathrm{USD} 1,097+(1.015)^{-0.25}+(1.015)^{-0.5}\right](1.015)^{0.5}$.

USD1,107.20 $=($ USD1,097 $+0.9963+0.9926)(1.015)^{0.5}$.

$F_{0}(1 \mathrm{y})=\left[\mathrm{USD} 1,097+(1.015)^{-0.25}+(1.015)^{-0.5}+(1.015)^{-0.75}+(1.015)^{-1}\right]$ $(1.015)$

USD1,117.48 $=($ USD1, $097+0.9963+0.9926+0.9889+0.9852)(1.015)$.

\begin{center}
\begin{tabular}{cccc}
\hline
 &  & No-Arbitrage Futures Prices &  \\
\hline
Spot & 3-month futures & 6-month futures & One-year futures \\
\hline
USD1,097 & USD1,102.09 & USD1,107.20 & USD1,117.48 \\
\hline
\end{tabular}
\end{center}

Note that the current three-month futures price is equal to the no-arbitrage price, while the six-month price is USD0.30 greater than the no-arbitrage price (= USD1,107.50 - USD1,107.20). The one-year price is USD2.48 below the no-arbitrage price (= USD1,115.00 - USD1,117.48). The one-year contract represents the opportunity to earn the maximum riskless profit by buying the one-year forward commitment and selling the asset short today in the spot market.

\begin{enumerate}
  \setcounter{enumi}{1}
  \item Rook Point is evaluating Japanese equities with a focus on the Nikkei 225 index. An analyst observes a spot index price of JPY31,000, a six-month Japanese yen risk-free rate of $-0.1 \%$, and an index dividend yield of $1.4 \%$. Which of the following statements most accurately describes both the arbitrage opportunity and the action Rook Point should take today to earn a riskless profit if the current six-month index futures price is JPY31,233?
\end{enumerate}

A. No arbitrage opportunity exists, because the Nikkei 225 six-month forward commitment price incorporates the net effect of all costs and benefits associated with owning the underlying asset for six months.

B. The current Nikkei 225 six-month forward commitment price is above the no-arbitrage price, so Rook Point should sell the futures contract and sell the index short today, investing the proceeds at the risk-free rate.

C. The current Nikkei 225 six-month forward commitment price is above the no-arbitrage price, so Rook Point should borrow at the risk-free rate to purchase the spot index at the spot price and simultaneously sell the futures contract.

$\mathrm{C}$ is correct. Based on the current $S_{0}$ of JPY31,000, $r=-0.1 \%, i=1.4 \%$, and $T$ of 0.5 , we may solve for the no-arbitrage forward commitment price in six months as JPY30,768.37 (= JPY31,000e $\left.e^{[(-0.001-0.014) \times 0.5]}\right)$. The observed six-month futures price exceeds the no-arbitrage price by JPY464.63 $(=$ JPY31,233 - JPY30,768.37). In order to capitalize on this price discrepancy, Rook Point should borrow at the risk-free rate, $r$, today to purchase the index at the spot price and simultaneously sell the futures contract, using the long cash position to settle the futures at time $T$.

\begin{enumerate}
  \setcounter{enumi}{2}
  \item In reviewing spot and forward commitment price relationships in commodity markets, the analyst finds that copper futures prices are below spot prices although the risk-free interest rate is positive. Which of the following statements provides the most likely explanation for this relationship?
\end{enumerate}

A. The cost of storage and other carry costs are greater than the risk-free rate, causing the spot price to rise above the futures price.

B. The cost of storage and other carry costs are below the risk-free rate, causing the spot price to rise above the futures price.

C. The convenience yield benefit is greater than the risk-free interest rate plus storage and other costs.

$\mathrm{C}$ is correct. The copper market is likely exhibiting the effects of the convenience yield benefit. If we assume that storage costs are positive (along with interest rates), the cost-of-carry component that would cause the spot price to be above the futures price is the convenience yield. Specifically, the convenience yield benefit would need to be greater than the combination of interest and storage costs.

\section{LEARNING MODULE}
\begin{center}
\includegraphics[max width=\textwidth]{2023_05_04_36535b8d80b32081d422g-259}
\end{center}

\section{Pricing and Valuation of Forward Contracts and for an Underlying with Varying Maturities}
\section{LEARNING OUTCOME}
\begin{center}
\begin{tabular}{c|l}
Mastery & The candidate should be able to: \\
\hline
$\square$ & $\begin{array}{l}\text { Explain how the value and price of a forward contract are } \\ \text { determined at initiation, during the life of the contract, and at } \\ \text { expiration } \\ \text { Explain how forward rates are determined for an underlying with a } \\ \text { term structure and describe their uses }\end{array}$ \\
\hline
\end{tabular}
\end{center}

\section{INTRODUCTION}
Earlier lessons introduced forward commitment features, payoff profiles, and concepts used in pricing these derivative instruments. In particular, the relationship between spot and forward commitment prices was established as the opportunity cost of owning the underlying asset (represented by the risk-free rate) as well as any additional cost or benefit associated with holding the underlying asset. This price relationship both prevents arbitrage and allows a forward commitment to be replicated using spot market transactions and risk-free borrowing or lending.

In the first lesson, we explore the pricing and valuation of forward commitments on a mark-to-market basis from inception through maturity. This analysis is essential for issuers, investors, and financial intermediaries alike to assess the value of any asset or liability portfolio that includes these instruments. The second lesson addresses forward pricing for the special case of underlying assets with different maturities such as interest rates, credit spreads, and volatility. The prices of these forward commitments across the so-called term structure are an important building block for pricing swaps and related instruments in later lessons.

\section{Summary}
\begin{itemize}
  \item A forward commitment price agreed upon at contract inception remains fixed and establishes the basis on which the underlying asset (or cash) will be exchanged in the future versus the spot price at maturity. - For an underlying asset that does not generate cash flows, the value of a long forward commitment prior to expiration equals the current spot price of the underlying asset minus the present value of the forward price discounted at the risk-free rate. The reverse is true for a short forward commitment. Foreign exchange represents a special case in which the spot versus forward price is a function of the difference between risk-free rates across currencies.

  \item For an underlying asset with additional costs and benefits, the forward contract mark-to-market (MTM) value is adjusted by the sum of the present values of all additional cash flows through maturity.

  \item Underlying assets with a term structure, such as interest rates, have different rates or prices for different times-to-maturity. These zero or spot and forward rates are derived from coupon bonds and market reference rates and establish the building blocks of interest rate derivatives pricing.

  \item Implied forward rates represent a breakeven reinvestment rate linking short-dated and long-dated zero-coupon bonds over a specific period.

  \item A forward rate agreement (FRA) is a contract in which counterparties agree to apply a specific interest rate to a future period.

\end{itemize}

\section{LEARNING MODULE PRE-TEST}
\begin{enumerate}
  \item Match the following situations with their corresponding forward contract valuation for an asset with no additional costs or benefits.

  \item At time $t=0$, the spot price of the underlying asset rises instantaneously and other market parameters remain unchanged.

  \item At time $t$, the present value of the forward price discounted at the risk-free rate $(r)$ equals the current spot price $\left(S_{t}\right)$.

  \item At time $T$, the forward contract price, $F_{0}(T)$, is greater than the current spot price, $S_{T}$ A. The forward contract buyer has an MTM gain.

\end{enumerate}

B. The forward contract seller has an MTM gain.

C. The MTM value of the forward contract is zero.

\section{Solution:}
\begin{enumerate}
  \item A is correct. In order to satisfy the no-arbitrage condition, the original spot price, $S_{0}$ at $t=0$, must equal the present value of the forward price discounted at the risk-free rate, $r$. An immediate increase in the spot price to $S_{0}{ }^{+}>S_{0}$ results in an MTM gain for the forward buyer.

  \item C is correct. At any time $t$, the MTM value, $V_{t}(T)$, is equal to the difference between the current spot price, $S_{t}$, and the present value of the forward price discounted at the risk-free rate, $r$, or $F_{0}(T)(1+r)^{-(T-t)}$. When $S_{t}=F_{0}(T)$ $(1+r)^{-(T-t)}$, then $V_{t}(T)=0$.

  \item B is correct. The MTM value to the forward contract seller upon settlement at time $T$ is equal to the settlement value of $F_{0}(T)-S_{T}$.

  \item Identify the following statement as true or false and justify your answer: An increase in the risk-free rate, $r$, following the inception of a forward contract will cause the present value of the forward price, $F_{0}(T)$, to fall, increasing the forward contract's MTM value to the forward seller if other parameters remain unchanged.

\end{enumerate}

\section{Solution:}
False. The mark-to-market value from the forward seller's perspective is equal to $V_{t}(T)$ in the following equation:

$$
V_{t}(T)=F_{0}(T)(1+r)^{-(T-t)}-S_{t}
$$

While it is true that an increase in the risk-free rate, $r$, following the inception of a forward contract will cause the present value of the forward price, $F_{0}(T)$, to fall, this result will reduce the MTM value from the contract seller's perspective.

Hightest Capital has entered into a three-month JPY/USD forward contract where it has agreed to purchase USD1,000,000 in exchange for JPY111,277,800. At time $t=0$ when the contract is initiated, the JPY/USD spot exchange rate is 111 - that is, JPY111 = USD1.

\begin{enumerate}
  \item Describe the circumstances under which Hightest can realize an immediate MTM gain once the contract is initiated where the JPY/USD spot price remains unchanged, and justify your answer.
\end{enumerate}

\section{Solution:}
If Hightest realizes an immediate MTM gain after contract inception with an unchanged FX spot price, it implies that the interest rate differential between JPY and USD has increased.

The long FX forward MTM from Hightest Capital's perspective is equal to the spot price minus the present value of the forward price discounted at the interest rate differential between the foreign currency and the domestic currency:

$$
\left.V_{0}(T)=S_{0, f l d}-F_{0, f l d}(T) \mathrm{e}^{-\left(r_{f}-r\right.} d\right) T
$$

Note that JPY is the price, or foreign, currency and USD is the base, or domestic, currency, so we can rewrite the equation as:

$$
V_{0}(T)=S_{0, J P Y / U S D}-F_{0, J P Y / U S D}(T) \mathrm{e}^{-\left(r_{J P Y}{ }^{-}{ }_{U S D}\right) T}
$$

A higher interest rate differential $\left(r_{I P Y}-r_{U S D}\right)$ increases the discount rate for the forward rate $F_{0, J P Y / U S D}(T)$, so $\left.S_{0, J P Y / U S D}>F_{0, J P Y / U S D}(\mathrm{~T}) \mathrm{e}^{-\left(r_{J P Y}-r\right.} U S D\right) T$, resulting in an MTM gain for Hightest Capital.

\begin{enumerate}
  \setcounter{enumi}{1}
  \item Calculate the two-year zero rate for a $3 \%$ annual coupon bond priced at 99 per 100 face value if a one-year annual coupon bond from the same issuer has a yield-to-maturity of $2.50 \%$.
\end{enumerate}

\section{Solution:}
The yield-to-maturity and the zero rate for a bond with a single cash flow at maturity in one period are identical, so the one-year zero rate, $z_{1}$, equals $2.50 \%$. Solve for the two-year zero rate $\left(z_{2}\right)$ in the following equation:

$$
99=3 / 1.025+103 /\left(1+z_{2}\right)^{2}
$$

Solve for $z_{2}$ to get $3.5422 \%$. 3. Determine the correct answers to complete the following sentence: Similar to the swap, fixed versus floating payments on an FRA occur on a \_\_ basis and the notional is not but is used solely for interest calculations.

\section{Solution:}
Similar to the swap, fixed versus floating payments on an FRA occur on a net basis and the notional is not exchanged but is used solely for interest calculations.

\begin{enumerate}
  \setcounter{enumi}{3}
  \item A financial intermediary enters into an FRA contract as the fixed-rate payer of the three-month market reference rate (MRR) in one year's time. Describe the likely interest rate scenario under which the financial intermediary will face a loss on this contract on the settlement date.
\end{enumerate}

\section{Solution:}
In a forward rate agreement, the fixed-rate payer agrees to pay the forward rate agreed upon at inception for a future interest period and receive the MRR for the same interest period on a given notional amount. If the forward rate is greater than the MRR, the financial intermediary must make a net payment to the other counterparty equal to the difference between the MRR and the fixed rate on the notional and therefore faces a loss on the transaction.

\section{2}
\section{PRICING AND VALUATION OF FORWARD COMMITMENTS}
Explain how the value and price of a forward contract are determined at initiation, during the life of the contract, and at expiration

\section{Pricing versus Valuation of Forward Contracts}
When counterparties enter into forward, futures, or swap contracts with one another, these contracts have an initial value of zero (ignoring trading and transaction costs as well as counterparty credit exposure). While forward commitments require no cash outlay at inception, their price incorporates the opportunity cost of a long cash position as measured by the risk-free rate. The forward price or forward rate established at inception remains fixed and determines the basis on which the underlying asset (or cash) will be exchanged in the future versus the spot price at maturity.

As time passes and/or the underlying asset spot price and other parameters change, the value of a forward contract changes. This mark-to-market value of a contract reflects the change in the underlying price and other factors that would result in a gain or loss to a counterparty if the forward contract were to be settled immediately. The MTM gain of the forward seller will equal the MTM loss of the forward buyer and vice versa. Recall that a key difference between exchange-traded futures and over-the-counter forwards is that the futures clearinghouse settles these MTM changes in cash on a daily basis, while forward contract settlement typically occurs at maturity.

\section{Pricing and Valuation of Forward Contracts at Initiation}
The prior learning module established that a forward contract agreed at time $t=0$ occurs at a forward price, $F_{0}(T)$, that satisfies no-arbitrage conditions for the underlying spot price $\left(S_{0}\right)$, the risk-free rate of return $(r)$, and any additional costs or benefits associated with underlying asset ownership until the forward contract matures at time T. In an earlier example, AMY Investments agreed to purchase 1,000 Airbus (AIR) shares trading at the spot price $\left(S_{T}\right)$ at maturity at an agreed upon forward price, $F_{0}(T)$, of EUR30 per share, as shown in Exhibit 1.

\section{Exhibit 1: Forward Contract Value at Initiation}
\begin{center}
\includegraphics[max width=\textwidth]{2023_05_04_36535b8d80b32081d422g-263}
\end{center}

If we assume $S_{0}=$ EUR29.70, $r=1.00 \%$ and $T$ is one year, $F_{0}(T)$ of EUR30 satisfies the no-arbitrage condition at $t=0$ (ignoring transaction costs). The forward contract is neither an asset nor a liability to AMY Investments (the buyer) or the financial intermediary (the seller) and therefore has a value of zero to both parties:

$$
V_{0}(T)=0
$$

\section{Pricing and Valuation of Forward Contracts at Maturity}
Recall from an earlier lesson that a forward commitment has a symmetric payoff profile. That is, at time $T$ a forward contract is settled based on the difference between the forward price, $F_{0}(T)$, and the underlying spot price, $S_{T}$, or $S_{T}-F_{0}(T)$, from the buyer's perspective, as shown in Exhibit 2. Exhibit 2: Forward Contract Value at Maturity
\includegraphics[max width=\textwidth, center]{2023_05_04_36535b8d80b32081d422g-264}

\begin{center}
\begin{tabular}{lcc}
\hline
Outcome & $\boldsymbol{V}_{\boldsymbol{T}}(\boldsymbol{T})$ (long position) & $\boldsymbol{v}_{\boldsymbol{T}}(\boldsymbol{T})$ (short position) \\
\hline
$S_{T}>F_{0}(T)$ & $S_{T}-F_{0}(T)>0$ & $F_{0}(T)-S_{T}<0$ \\
$S_{T}<F_{0}(T)$ & $S_{T}-F_{0}(T)<0$ & $F_{0}(T)-S_{T}>0$ \\
$S_{T}=F_{0}(T)$ & $S_{T}-F_{0}(T)=0$ & $F_{0}(T)-S_{T}=0$ \\
\hline
\end{tabular}
\end{center}

AMY must pay the agreed upon price of $F_{0}(T)$ in exchange for the underlying asset at the current spot price of $S_{T}$. Since the contract settles at maturity, its value at maturity is equal to the settlement amount from each counterparty's perspective. For example, at time $t=T$, the value of the forward contract at maturity, $V_{T}(T)$, from the perspective of AMY (the forward buyer) equals:

$$
V_{T}(T)=S_{T}-F_{0}(T)
$$

\section{EXAMPLE 1}
\section{Value of Biomian Forward Positions at Maturity}
The Viswan Family Office (VFO) currently owns 10,000 common non-dividend-paying shares of Biomian Limited, a Mumbai-based biotech company, at a spot price of INR 295 per share. VFO agrees to sell forward 1,000 shares of Biomian stock to a financial intermediary for INR300.84 per share in six months. Calculate the contract value at maturity, $V_{T}(T)$, from both the buyer's and the seller's perspective if the spot price at maturity $\left(S_{T}\right)$ is:

(1) $S_{T}=$ INR 287

(2) $S_{T}=\operatorname{INR} 312$

\section{Solution:}
At contract inception, VFO enters into a forward contract with a financial intermediary to sell Biomian for $F_{0}(T)=$ INR 300.84. The forward contract value at initiation for both VFO and the financial intermediary, $V_{0}(T)$, is zero. (1) $S_{T}=$ INR287 and $F_{0}(T)=$ INR300.84. The contract value per share at maturity equals its settlement value from the perspective of both the financial intermediary (buyer) and VFO (seller), as follows:

\begin{itemize}
  \item Buyer (long forward position): $V_{T}(T)=S_{T}-F_{0}(T)$
\end{itemize}

$V_{T}(T)=-$ INR13.84 $=287-300.84$

\begin{itemize}
  \item Viswan Family Office (short forward position): $V_{T}(T)=F_{0}(T)-S_{T}$
\end{itemize}

$V_{T}(T)=$ INR13.84 $=300.84-287$

(2) $S_{T}=$ INR312 and $F_{0}(T)=$ INR300.84. The contract value per share at maturity equals its settlement value from the perspective of both the financial intermediary (buyer) and VFO (seller), as follows:

\begin{itemize}
  \item Buyer (long forward position): $V_{T}(T)=S_{T}-F_{0}(T)$
\end{itemize}

$V_{T}(T)=$ INR11.16 $=312-300.84$

\begin{itemize}
  \item Viswan Family Office (short forward position): $V_{T}(T)=F_{0}(T)-S_{T}$
\end{itemize}

$V_{T}(T)=-$ INR11.16 $=300.84-312$

\section{Pricing and Valuation of Forward Contracts during the Life of the Contract}
Once a forward contract is initiated between two counterparties, the passage of time and changes in the underlying asset's spot price, among other factors, will cause the forward contract value to change. The mark-to-market value of a contract at any point in time from inception to maturity, $V_{t}(T)$, reflects the relationship between the current spot price at time $t\left(S_{t}\right)$ and the present value of the forward price at time $t$ discounted at the current risk-free rate. Exhibit 3 shows this relationship for the Airbus equity forward example from AMY Investments' perspective (the long forward position).

\section{Exhibit 3: Forward Contract Value at Inception, over Time, and at Maturity}
\begin{center}
\includegraphics[max width=\textwidth]{2023_05_04_36535b8d80b32081d422g-265}
\end{center}

Recall that under no-arbitrage conditions, the forward price of an underlying asset with no additional cost or benefit of ownership equals the future value of the spot price at the risk-free rate, $r$ :

$$
F_{0}(\mathrm{~T})=S_{0}(1+r)^{T}
$$

At any time $t$ over the life of the contract where $t<T$, we can show the present value of the forward price, $F_{0}(T)$, at time $t$ as follows:

$$
\mathrm{PV}_{t} \text { of } F_{0}(T)=F_{0}(T)(1+r)^{-(T-t)}
$$

If $S_{t}$ is the spot price of the underlying asset at time $t$, Equation 5 shows the forward contract MTM value at time $t, V_{t}(T)$, from the long forward position's perspective:

$$
V_{t}(T)=S_{t}-F_{0}(T)(1+r)^{-(T-t)}
$$

Exhibit 4 shows the relationship between the spot price, the forward price, and the present value of the forward price, $F_{0}(T)(1+r)^{-(T-t)}$, over time as represented by the dashed line. The slope of the dashed line is equal to $r$.

\section{Exhibit 4: Present Value of the Forward Price over Time}
\begin{center}
\includegraphics[max width=\textwidth]{2023_05_04_36535b8d80b32081d422g-266}
\end{center}

We explore these relationships, as well as the MTM impact of an instantaneous change in the spot price $\left(S_{0}\right)$, in the following example.

\section{EXAMPLE 2}
\section{Implied Risk-Free Rate and Biomian Forward Contract MTM}
As in Example 1, VFO enters into a six-month forward contract with a financial intermediary to sell Biomian shares for $F_{0}(\mathrm{~T})=$ INR300.84 per share. The current spot price is INR295 per share.

\begin{enumerate}
  \item Calculate the risk-free rate implied by the spot and forward prices.

  \item Calculate the forward contract MTM from VFO's perspective if Biomian's share price rises instantaneously to INR325 at contract inception $(t=0)$.

\end{enumerate}

\section{Solution:}
\begin{enumerate}
  \item We can use Equation 3-with $S_{0}=295, F_{0}(T)=300.84$, and $T=0.5-$ to solve for the risk-free rate, $r$ :
\end{enumerate}

$F_{0}(T)=S_{0}(1+r)^{T}$

$300.84=295(1+r)^{0.5}$

$r=0.04$ or $4 \%$

\begin{enumerate}
  \setcounter{enumi}{1}
  \item As the forward contract seller, we must rearrange Equation 5 to solve for the contract MTM value from VFO's perspective as follows:
\end{enumerate}

$$
V_{t}(T)=F_{0}(T)(1+r)^{-(T-t)}-S_{t}
$$

Note that if $t=0, F_{0}(T)(1+r)^{-(T-t)}$ simplifies to $F_{0}(T)(1+r)^{-T}=S_{0}$, so the contract value from VFO's perspective can be shown simply as:

$$
\begin{aligned}
& V_{t}(T)=S_{0}-S_{\mathrm{t}} \\
& -\mathrm{INR} 30=\text { INR295 - INR325 }
\end{aligned}
$$

If we consider the new higher Biomian price as $S_{0}{ }^{+}=$INR325, we can show this MTM change in the following diagram:

\begin{center}
\includegraphics[max width=\textwidth]{2023_05_04_36535b8d80b32081d422g-267}
\end{center}

The combination of spot price changes over the life of a forward contract and the passage of time causes the MTM value of a forward contract to fluctuate over time, representing a gain or loss to contract participants so long as $S_{t} \neq F_{0}(T)(1+r)^{-(T-t)}$. Exhibit 5 shows this relationship, assuming a constant risk-free rate over the life of a contract for long and short forward contract positions.

\section{Exhibit 5: Forward Contract MTM}
\begin{center}
\includegraphics[max width=\textwidth]{2023_05_04_36535b8d80b32081d422g-267(1)}
\end{center}

Example 3 illustrates these combined effects for VFO's Biomian forward contract and examines the effect of a change in the risk-free rate.

\section{EXAMPLE 3}
\section{Biomian Forward Contract MTM over Time and Changes in Risk-Free Rate}
As in earlier examples, VFO enters into a six-month forward contract with a financial intermediary to sell Biomian shares at $F_{0}(\mathrm{~T})=$ INR300.84 per share. The spot price at $t=0$ is INR295 per share and the risk-free rate is $4 \%$.

\begin{enumerate}
  \item Calculate the forward contract MTM from VFO's perspective in three months $(t=0.25)$ if Biomian's spot price $\left(S_{t}\right)$ falls to INR285 per share.

  \item Show the forward contract MTM from VFO's perspective in Question (1) if the risk-free rate doubles from $4 \%$ to $8 \%$, and interpret the results.

\end{enumerate}

\section{Solution:}
(1) From VFO's perspective as the forward contract seller, rearrange Equation 5 to solve for the contract MTM value as follows:

$V_{t}(T)=F_{0}(T)(1+r)^{-(T-t)}-S_{t}$

Solve for $V_{t}(T)$ where $F_{0}(T)=300.84, r=0.04, \mathrm{~T}=0.5, t=0.25$, and $S_{t}=285$ :

$V_{t}(T)=300.84(1.04)^{-(0.25)}-285$

$V_{t}(T)=$ INR12.90 MTM gain

(2) Solve for $V_{t}(T)$ using the same equation and inputs as in Question (1) except that $r=0.08$ :

$V_{t}(T)=F_{0}(T)(1+r)^{-(T-t)}-S_{t}$

Solve for $V_{t}(T)$ where $F_{0}(T)=300.84, r=0.08, T=0.5, t=0.25$, and $S_{t}=285$ :

$V_{t}(T)=300.84(1.08)^{-(0.25)}-285$

$V_{t}(T)=$ INR10.11 MTM gain

VFO's MTM gain on the contract has declined by INR2.79 (12.90 - 10.11). The higher risk-free rate increases the opportunity cost of a cash position and lowers the present value of the forward price, reducing VFO's MTM gain, $V_{t}(T)$, as represented by the present value of $F_{0}(T)-S_{t}$.

\section{Pricing and Valuation of Forward Contracts with Additional Costs or Benefits}
The pricing and valuation of forward contracts in this lesson have assumed that there are no cash flows associated with the underlying asset. We showed that the contract MTM value at any time $t$ equals the difference between the current spot price and the present value of the forward price discounted at the risk-free rate. Now we turn our attention to how the cost of carry, or the net of all costs and benefits related to owning an underlying asset or index, affects the valuation of a forward contract. Recall the relationship between spot and forward prices for underlying assets with ownership benefits or income $(I)$ or costs $(C)$, which is expressed as a present value at time $t=$ 0 in discrete compounding terms:

$$
F_{0}(T)=\left(S_{0}-\mathrm{PV}_{0}(I)+\mathrm{PV}_{0}(C)\right)(1+r)^{T}
$$

Equation 6 incorporates the cost of carry and satisfies the no-arbitrage condition at $t$ $=0$ (ignoring transaction costs). A forward contract at a price, $F_{0}(T)$, that incorporates these known costs and benefits is neither an asset nor a liability to the buyer or seller at inception: $V_{0}(T)=0$. Also, since the forward price, $F_{0}(T)$, incorporates the cost of carry, the value at maturity, $V_{T}(T)$, is equal to the difference between the spot price $\left(S_{T}\right)$ at maturity and the original forward price, $F_{0}(T)$. At any time $t$ over the life of the contract, the MTM value of the forward contract, $V_{t}(T)$, will depend on the difference between the current spot price adjusted for any remaining costs or benefits from time $t$ through maturity, $S_{t}-\mathrm{PV}_{t}(I)+\mathrm{PV}_{t}(C)$, and the present value of the forward price, $\mathrm{PV}_{t} F_{0}(T)=F_{0}(T)(1+r)^{-(T-t)}$. This result is shown in Equation 7 from the long forward counterparty's perspective:

$$
V_{t}(T)=\left(S_{t}-\mathrm{PV}_{t}(I)+\mathrm{PV}_{t}(C)\right)-F_{0}(T)(1+r)^{-(T-t)}
$$

\section{EXAMPLE 4}
\section{Hightest Equity Forward Valuation}
In an earlier example, Hightest Capital agreed to deliver 1,000 Unilever (UL) shares to a financial intermediary in six months under a forward contract at a price of EUR 50,631.10, or EUR 50.6311 per share. Unilever pays a quarterly dividend of EUR 0.30 three months after contract inception and at time $T$, and the risk-free rate $(r)$ is $5 \%$. Calculate the forward contract breakeven price, $S_{t}$, where $V_{t}(T)=\mathrm{MTM}=0$ four months after contract inception if the risk-free rate, $r$, remains unchanged at $5 \%$.

Use Equation 7 with $\mathrm{PV}_{t}(C)=0$ to solve for $V_{t}(T)=0$ four months after contract inception:

$V_{t}(T)=\left(S_{t}-\mathrm{PV}_{t}(I)\right)-F_{0}(T)(1+r)^{-(T-t)}$

First, solve for the present value of the dividend per share, $\mathrm{PV}_{t}(I)$, given that the second dividend will be paid in two months:

PV $_{t}(I)=$ EUR0.30(1.05) $)^{-0.167}$

$=$ EUR0.2976

\begin{center}
\includegraphics[max width=\textwidth]{2023_05_04_36535b8d80b32081d422g-269}
\end{center}

Substitute $\operatorname{PV}_{t}(I)=$ EUR0.2976 into Equation 5 to solve for $V_{t}(T)=0$ :

$0=\left(S_{t}-0.2976\right)-50.6311(1.05)^{-0.167}$

$S_{t}=$ EUR50.2202 + EUR0.2976

$=$ EUR50.5178

Hightest Capital's breakeven spot rate, $S_{t}$ (i.e., where $V_{t}(T)=0$ ), four months after inception of the forward contract is therefore EUR50.5178 per share.

Forward commitments on underlying indexes such as equity indexes or commodity indexes are predominantly exchange-traded index futures contracts. An earlier lesson used the rate of return over the life of the contract under continuous compounding to establish the relationship between spot and forward pricing for these instruments. The valuation of index futures and other futures contracts will be addressed in a later lesson. Pricing and Valuation of Forward Contracts

Recall that for a foreign exchange forward, both spot $\left(S_{0, f / d}\right)$ and forward $\left(F_{0, f / d}(T)\right)$ prices are expressed in terms of units of a price currency ( $f$ or foreign currency) per single unit of a base currency ( $d$ or domestic currency). The spot price versus forward price relationship reflects the difference between the foreign risk-free rate, $r_{f}$, and the domestic risk-free rate, $r_{d}$, as shown in Equation 8:

$$
\left.F_{0, \mathrm{f} / \mathrm{d}}(\mathrm{T})=S_{0, f l d} \mathrm{e}_{f}^{\left(r_{f}-r\right.} d{ }^{\prime}\right) T
$$

At trade inception at $t=0$, the currency with the lower risk-free rate for the forward period is said to trade at a forward premium-that is, fewer units of currency are required to purchase one unit of the other-while the currency with the higher risk-free rate trades at a forward discount.

At any given time $t$, the MTM value of the FX forward is the difference between the current spot FX price $\left(S_{t, f l d}\right)$ and the present value of the forward price discounted by the current difference in risk-free rates $\left(r_{f}-r_{d}\right)$ for the remaining period through maturity, as shown in Equation 9:

$$
V_{t}(T)=S_{t, f l d}-F_{0, f l d}(T) \mathrm{e}^{-\left(r_{f}^{-r} d\right.} d(T-t)
$$

Changes in the interest rate differential $\left(r_{f}-r_{d}\right)$ represent a change in the relative opportunity cost between currencies. As described in an earlier lesson, a price change of one currency in terms of another is referred to as appreciation or depreciation. For example, if fewer USD are required to purchase one EUR, then the USD/EUR exchange rate falls and the USD is said to appreciate against the EUR.

A greater interest rate differential-that is, an increase in $\left(r_{f}-r_{d}\right)$-causes the price, or foreign, currency to depreciate on a forward basis and the base, or domestic, currency to appreciate. The following example illustrates the effect of an interest rate change on the FX forward contract MTM value.

\section{EXAMPLE 5}
\section{Rook Point Investors LLC FX Forward MTM}
Rook Point Investors LLC has entered into a long one-year USD/EUR forward contract. That is, it has agreed to purchase EUR1,000,000 in exchange for USD1,201,000 in one year. At time $t=0$ when the contract is initiated, the USD/ EUR spot exchange rate is 1.192 (i.e., USD1.192 = EUR1), the one-year USD risk-free rate is $0.50 \%$, and the one-year EUR risk-free rate is $-0.25 \%$.

Describe the MTM impact on the FX forward contract from Rook Point's perspective if the one-year USD risk-free rate instantaneously rises by $0.25 \%$ once the contract is initiated, with other details unchanged.

The instantaneous rise in the USD risk-free rate by $0.25 \%$ (from $0.5 \%$ to $0.75 \%$ ) increases the difference between the foreign and domestic risk-free rates $\left(r_{f}-r_{d}\right)$, increasing the discount rate used to calculate the present value of the forward rate, $F_{0, f l d}(T)$. Since $S_{0, f l d}>F_{0, f l d}(T) \mathrm{e}^{-(r} f_{f}^{-r} d(T-t), V_{0}(T)>0$ from Rook Point's perspective and it realizes an MTM gain.

We can solve for the MTM value at time $t=0$ by first rewriting Equation 9: $V_{t}(T)=S_{0, U S D / E U R}-F_{0, \text { USD/EUR }}(T) \mathrm{e}^{-(r}$ USD $\left.{ }^{-r} E U R{ }^{\prime}\right) T$

Solve for $V_{t}(T)$ from Rook Point's perspective, with $S_{0, f / d}=1.192$ USD/EUR, $F_{0, f l d}=1.201 \mathrm{USD} / \mathrm{EUR}, r_{f}=0.75 \%, r_{d}=-0.25 \%$, and $T=1$.

$=1.192-1.201 \mathrm{e}^{-(0.0075+0.0025)}$

$=0.00295 \mathrm{USD} / \mathrm{EUR}$

Note that this positive exchange rate difference of 0.00295 USD/EUR can be shown in Equation 9 if we instead substitute the USD needed to purchase EUR1,000,000 to arrive at a USD contract value:

$$
\begin{aligned}
& =\text { USD1,192,000 }- \text { USD1,201,000e- } \mathrm{e}^{-(0.0075+0.0025)} \\
& =\text { USD2,950.15 MTM gain }
\end{aligned}
$$

In this and other examples involving interest rates, we have assumed constant risk-free rates both over time and across maturities. In the next lesson, we will explore how spot and forward prices change for variables, with different prices across maturities.

\section{KNOWLEDGE CHECK}
\section{Pricing versus Valuation of Forward Contracts}
\begin{enumerate}
  \item Identify which MTM situation corresponds to which forward price versus spot price relationship for an underlying asset with no additional costs or benefits.

  \item At time $t=0$ once the forward price is agreed, the spot price of the underlying asset immediately falls and other market parameters remain unchanged.

  \item At time $T$, the forward contract price, $F_{0}(T)$, equals the current spot price, $S_{T}$.

  \item At time $T$, the forward contract price, $F_{0}(T)$, is below the current spot price, $S_{T}$. A. The forward contract seller has an MTM loss.

\end{enumerate}

B. The forward contract seller has an MTM gain.

C. The MTM value of the forward contract is zero.

\section{Solution:}
\begin{enumerate}
  \item B is correct. To satisfy the no-arbitrage condition, the original spot price, $S_{0}$ at $t=0$, must equal the present value of the forward price discounted at the risk-free rate, $r$. Therefore, an immediate fall in the spot price to $S_{0}{ }^{-}<S_{0}$ results in an MTM gain for the forward contract seller.

  \item C is correct. At time $T$, the MTM value, $V_{t}(T)$, is equal to contract settlement, or the difference between $F_{0}(T)$ and $S_{T}$, which in this case is zero. 3 . A is correct. The MTM value to the forward seller upon settlement at time $T$ equals $F_{0}(T)-S_{T}$, so the seller has a loss if $F_{0}(T)<S_{T}$.

  \item Identify the following statement as true or false and justify your answer: At time $t$, a forward contract with no additional cash flows has a value equal to the difference between the current spot price and the present value of the forward price. Therefore, the MTM value of the forward contract from the seller's perspective is $S_{t}$ - PV of $F_{0}(T)$.

\end{enumerate}

\section{Solution:}
This statement is false. Although the forward contract MTM equals the difference between the current spot price and the present value of the forward price, the MTM value of the forward contract from the seller's perspective is PV of $F_{0}(T)-S_{t}$

\begin{enumerate}
  \setcounter{enumi}{2}
  \item Assume that Hightest Capital enters into a six-month forward contract to purchase Unilever (UL) shares at EUR51.23. UL shares pay no dividend, and the risk-free rate across all maturities is $5 \%$. Calculate the forward contract MTM value from Hightest's perspective in three months' time if the current UL spot rate, $S_{t}$, is EUR50.50 and the risk-free rate does not change.
\end{enumerate}

\section{Solution:}
Use Equation 5 to solve for the forward contract value in three months from Hightest's perspective:

$V_{t}(T)=S_{t}-F_{0}(T)(1+r)^{-(T-t)}$

If $S_{t}=$ EUR50.50, $F_{0}(T)=$ EUR51.23, $r=5 \%$, and $T-t=0.25$ :

$V_{t}(T)=$ EUR50.50 - (EUR51.23)(1.05) ${ }^{-0.25}$

$=-$ EUR0.1089

Hightest therefore has an MTM loss of EUR0.11 on the forward contract in three months' time.

\begin{enumerate}
  \setcounter{enumi}{3}
  \item Match the following statements with their corresponding MTM situation for an underlying asset with additional costs and benefits.

  \item At time $t$, the present value of the benefits of owning the underlying asset is greater than the present value of the costs.

  \item At time $t$, the present value of the benefits of owning the underlying asset is equal to the present value of the costs.

  \item At time $t$, the present value of the benefits of owning the underlying asset is less than the present value of the costs. A. The forward contract has an MTM value equal to the difference between the current spot price and the present value of the forward price.

\end{enumerate}

B. The forward contract has an MTM value greater than the difference between the current spot price and the present value of the forward price.

C. The forward contract has an MTM value less than the difference between the current spot price and the present value of the forward price.

\section{Solution:}
Recall the forward contract MTM value, $V_{t}(T)$, for underlying assets with additional costs and benefits from the long forward counterparty's perspective in Equation 7:

$V_{t}(T)=\left(S_{t}-\mathrm{PV}_{t}(I)+\mathrm{PV}_{t}(C)\right)-F_{0}(T)(1+r)^{-(T-t)}$

\begin{enumerate}
  \item $C$ is correct. A higher present value of the benefits of owning the underlying asset versus the present value of the costs will reduce the MTM below the difference between the current spot price and the present value of the forward price. For Equation 7, if $\mathrm{PV}_{t}(C)-\mathrm{PV}_{t}(I)<0$, then:
\end{enumerate}

$V_{t}(T)<S_{t}-F_{0}(T)(1+r)^{-(T-t)}$

\begin{enumerate}
  \setcounter{enumi}{1}
  \item A is correct. If the present values of the remaining benefits and costs of owning the underlying asset offset each other, then the forward contract MTM equals the difference between the current spot price and the present value of the forward price. For Equation 7 , if $\operatorname{PV}_{t}(C)-\mathrm{PV}_{t}(I)=0$, then: $V_{t}(T)=S_{t}-F_{0}(\mathrm{~T})(1+r)^{-(T-t)}$

  \item B is correct. A lower present value of the benefits of owning the underlying asset versus the present value of the costs will increase the MTM beyond the difference between the current spot price and the present value of the forward price. For Equation 7 , if $\mathrm{PV}_{t}(C)-\mathrm{PV}_{t}(I)>0$, then:

\end{enumerate}

$V_{t}(T)>S_{t}-F_{0}(T)(1+r)^{-(T-t)}$

\begin{enumerate}
  \setcounter{enumi}{4}
  \item Rook Point Investors LLC has entered into a short six-month FX forward contract. That is, Rook Point agrees to sell South African rand (ZAR) and buy EUR at a forward ZAR/EUR price of 17.2506 in six months. The ZAR/ EUR spot price is 16.909 , the six-month South African risk-free rate is 3.5\%, and the six-month EUR risk-free rate is $-0.5 \%$. Describe the FX forward MTM impact from Rook Point's perspective of an immediate appreciation in ZAR/EUR to 16.5 if other parameters are unchanged.
\end{enumerate}

\section{Solution:}
An immediate appreciation in the ZAR/EUR spot price after contract inception will result in an MTM gain from Rook Point's perspective as the forward seller. If the FX forward contract were settled immediately, Rook Point would have to deliver fewer ZAR than previously agreed in order to purchase a fixed EUR amount.

The FX forward MTM from Rook Point's perspective equals the present value of the forward price discounted at the interest rate differential between the foreign currency and the domestic currency minus the spot price: $\left.V_{0}(T)=F_{0, f l d}(T) \mathrm{e}^{-(r} f_{f}^{-r} d\right) T-S_{0, f l d}$

Note that ZAR is the price, or foreign, currency and EUR is the base, or domestic, currency, so we can rewrite the equation as:

$V_{0}(T)=F_{0, Z A R / E U R}(T) \mathrm{e}^{-\left(r Z_{A R}{ }^{-r} E_{E R R}\right)^{T}-S_{0, Z A R / E U R}}$

If the ZAR price $\left(S_{0, Z A R / E U R}\right)$ appreciates from 16.909 to 16.5, we can show that Rook Point would have a 0.4090 gain (i.e., deliver fewer EUR for the same amount of ZAR), as follows:

$V_{t}(T)=17.2506 \mathrm{e}^{-(0.035--0.005) \times(0.5)}-16.5$

$=16.909-16.5$

$=0.4090$

\section{PRICING AND VALUATION OF INTEREST RATE FORWARD CONTRACTS}
$\square \quad \begin{aligned} & \text { Explain how forward rates are determined for an underlying with a } \\ & \text { term structure and describe their uses }\end{aligned}$

\section{Interest Rate Forward Contracts}
The relationship between spot and forward prices of underlying assets uses a constant risk-free interest rate as the opportunity cost of owning the underlying asset. Unlike equities and commodities, addressed earlier, interest rates are characterized by a term structure-that is, different interest rates exist for different times-to-maturity. While this lesson focuses on interest rates, similar principles apply to other underlying variables with a term structure, including credit spreads and implied volatility as well as foreign exchange, where two interest rate term structures are involved.

\section{Spot Rates and Discount Factors}
The relationships between spot and forward interest rates, established in earlier fixed-income lessons, are key building blocks for interest rate derivatives pricing. These building blocks are usually based on a government benchmark or market reference rate. While most fixed-income instruments have coupon cash flows prior to maturity, interest rate derivatives pricing and valuation are based on the price and yield of single cash flows on a future date. The transformation of these cash flows from one form into another is a first step in the process, as in the following example.

\section{EXAMPLE 6}
\section{Zero Rate}
Assume we observe three most recently issued annual fixed-coupon government bonds, with coupons and prices as follows:

\begin{center}
\begin{tabular}{lcc}
\hline
Years to Maturity & Annual Coupon & PV (per 100 FV) \\
\hline
1 & $1.50 \%$ & 99.125 \\
2 & $2.50 \%$ & 98.275 \\
3 & $3.25 \%$ & 98.000 \\
\hline
\end{tabular}
\end{center}

Note that each bond trades at a discount $(P V<F V)$. As a first step, we solve for each bond's yield-to-maturity (YTM) using the Excel RATE function (=RATE(nper, pmt, pv,[fv],[type])). For example, for the three-year bond, we use nper $=3, p m t=3.25, p v=-98, f v=100$, and type $=0$ (indicating payment at the end of the period) to solve for the three-year bond YTM of 3.9703\%.

\begin{center}
\begin{tabular}{lccc}
$\begin{array}{l}\text { Years to } \\ \text { Maturity }\end{array}$ & Annual Coupon & PV (per 100 FV) & YTM \\
\hline
1 & $1.50 \%$ & 99.125 & $2.3960 \%$ \\
2 & $2.50 \%$ & 98.275 & $3.4068 \%$ \\
3 & $3.25 \%$ & 98.000 & $3.9703 \%$ \\
\hline
\end{tabular}
\end{center}

We can use these government yields-to-maturity to solve for a sequence of yields-to-maturity on zero-coupon bonds, or zero rates $\left(z_{1}, \ldots z_{\mathrm{N}}\right)$, where $z_{\mathrm{i}}$ is the zero rate for period $i$.

Starting with the one-year bond, which consists of a single cash flow at maturity, we can solve for the one-year zero rate $\left(z_{1}\right)$ as follows:

One-year:

$99.125=\frac{101.5}{\left(1+z_{1}\right)^{1}} ; z_{1}=2.3960 \%$

The yield-to-maturity rate and the zero rate for a bond with a single cash flow at maturity in one period are identical. Since all cash flows at time $t=1$ are discounted at $z_{1}$, we can substitute $z_{1}$ into the two-year fixed-coupon bond calculation to solve for the zero-coupon rate at the end of the second period $\left(z_{2}\right)$ :

Two-year:

$98.275=\frac{2.5}{1.02396}+\frac{102.5}{\left(1+z_{2}\right)^{2}} ; z_{2}=3.4197 \%$

We then substitute both $z_{1}$ and $z_{2}$ into the three-year bond equation to solve for the zero-coupon cash flow at the end of year three $\left(z_{3}\right)$ :

Three-year:

$98.00=\frac{3.25}{(1.02396)}+\frac{3.25}{(1.034197)^{2}}+\frac{103.25}{\left(1+z_{3}\right)^{3}} ; z_{3}=4.0005 \%$

The zero rates are summarized in the following diagram:

\begin{center}
\includegraphics[max width=\textwidth]{2023_05_04_36535b8d80b32081d422g-275}
\end{center}

This process of deriving zero or spot rates from coupon bonds using forward substitution, as shown in Example 6, is sometimes referred to as bootstrapping. The price equivalent of a zero rate is the present value of a currency unit on a future date, known as a discount factor. The discount factor for period $i\left(\mathrm{DF}_{i}\right)$ is:

$$
D F_{i}=\frac{1}{\left(1+z_{i}\right)^{i}}
$$

For each of the zero rates shown in Example 6, an equivalent discount factor can be derived as follows:

One-year: $\mathrm{DF}_{1}=1 /\left(1+z_{1}\right) ; \mathrm{DF}_{1}=0.976601$

Two-year: $\mathrm{DF}_{2}=1 /\left(1+z_{2}\right)^{2} ; \mathrm{DF}_{2}=0.934961$

Three-year: $\mathrm{DF}_{3}=1 /\left(1+z_{3}\right)^{3} ; \mathrm{DF}_{3}=0.888982$

A discount factor may also be interpreted as the price of a zero-coupon cash flow or bond. We can use the discount factor for any period to demonstrate the same no-arbitrage condition from an earlier lesson-namely, that an asset with a known future price must trade at the present value of its future price as determined by using an appropriate discount rate.

For example, assume a two-year GBP risk-free zero rate $\left(z_{2}\right)$ of $3.42 \%$ and a two-year zero-coupon bond with a face value of GBP100 trading at a price of GBP92.45. The no-arbitrage price of the zero-coupon bond is equal to GBP93.4955: GBP100/(1.0342) ${ }^{2}$. In order to earn a riskless arbitrage profit, we can take the following steps, as shown in Exhibit 6.

\section{Exhibit 6: Asset with Known Future Price Does Not Trade at Its Present Value}
Borrow $S_{0}$ at $z$ and Buy $S_{0^{\prime}}$ Hold to maturity $(T) \quad$ Use proceeds at maturity to repay $S_{0}\left(1+z_{T}\right)^{T}$

Borrow $\pounds 92.45$ at 3.42\% and Buy two-year zero $\left(S_{0}\right)$ Receive $\pounds 100$ at $T\left(S_{T}\right)$ and Repay $S_{0}\left(1+z_{T}\right)^{\top}=\pounds 98.88$

\begin{center}
\includegraphics[max width=\textwidth]{2023_05_04_36535b8d80b32081d422g-276}
\end{center}

At time $t=0$ :

\begin{itemize}
  \item Borrow GBP92.45 at the risk-free rate of 3.42\%.

  \item Purchase the two-year zero-coupon note for GBP92.45.

\end{itemize}

At time $t=T$ (in two years):

\begin{itemize}
  \item Receive GBP100 on the zero-coupon bond maturity date.

  \item Repay loan of GBP98.88-GBP92.45 $\times(1.0342)^{2}$-and earn a riskless arbitrage profit of GBP1.12.

\end{itemize}

\section{Forward Rates}
As in the case of other underlying assets, the forward market for interest rates involves a delivery date beyond the usual cash market settlement date. Given the term structures of different interest rates for different maturities, an interest rate forward contract specifies both the length of the forward period and the tenor of the underlying rate.

For example, a two-year forward contract that references a three-year underlying interest rate (which starts at the end of year two and matures at the end of year five) is referred to as a "2y3y" forward rate, which we will denote as $F_{2,3}$. Short-term market reference rates (MRRs) usually reference a forward rate in months-for instance, $F_{3 \mathrm{~m}, 6 \mathrm{~m}}$ references a six-month MRR that begins in three months and matures nine months from today.

The breakeven reinvestment rate linking a short-dated and a long-dated zero-coupon bond is an implied forward rate (IFR). That is, the implied forward rate is the interest rate for a period in the future at which an investor earns the same return from:

\begin{enumerate}
  \item investing for a period from today until the forward start date and rolling over the proceeds at the implied forward rate, or

  \item investing today through the final maturity of the forward rate.

\end{enumerate}

The fact that these strategies have equal returns establishes the no-arbitrage condition for the implied forward rate. The following example demonstrates how spot rates can be used to derive forward rates.

\section{EXAMPLE 7}
\section{Implied Forward Rate $\left(I F R_{1,1}\right)$}
We return to the earlier example of three most recently issued annual fixed-coupon government bonds, with coupons and prices as well as yields-to-maturity and zero or spot rates as follows:

\begin{center}
\begin{tabular}{lccc}
\hline
Years to Maturity & Annual Coupon & PV (per 100 FV) & YTM \\
\hline
1 & $1.50 \%$ & 99.125 & $2.3960 \%$ \\
2 & $2.50 \%$ & 98.275 & $3.4068 \%$ \\
\end{tabular}
\end{center}

\begin{center}
\begin{tabular}{lccc}
\hline
Years to Maturity & Annual Coupon & PV (per 100 FV) & YTM \\
\hline
3 & $3.25 \%$ & 98.000 & $3.9703 \%$ \\
\hline
 &  &  &  \\
\hline
Years to Maturity & Zero Rate &  &  \\
\hline
1 & $2.3960 \%$ &  &  \\
2 & $3.4197 \%$ &  &  \\
3 & $4.0005 \%$ &  &  \\
\hline
\end{tabular}
\end{center}

Using the one-year and two-year zero rates from the prior example, an investor faces the following investment choices over a two-year period:

\begin{enumerate}
  \item Invest USD 100 for one year today at the zero rate $\left(z_{1}\right)$ of $2.396 \%$ and reinvest the proceeds of USD102.40 at the one-year rate in one year's time, or the "1y1y" implied forward rate $\left(I F R_{1,1}\right)$.

  \item Invest USD 100 for two years at the two-year zero rate $\left(z_{2}\right)$ of $3.4197 \%$ to receive USD100 $\left(1+z_{2}\right)^{2}$, or USD106.96.
\includegraphics[max width=\textwidth, center]{2023_05_04_36535b8d80b32081d422g-277}

\end{enumerate}

In order to arrive at the same return for investment choices 1.) and 2.), we set them equal to each other and solve for $I F R_{1,1}$ as follows:

$\mathrm{USD} 100 \times\left(1+z_{1}\right) \times\left(1+\operatorname{IFR}_{1,1}\right)=\operatorname{USD} 100 \times\left(1+z_{2}\right)^{2}$

$\operatorname{USD} 100 \times(1.02396) \times\left(1+I F R_{1,1}\right)=\operatorname{USD} 100 \times(1.034197)^{2}$

$I F R_{1,1}=4.4536 \%$

We can demonstrate that the IFR $_{1,1}$ of $4.4536 \%$ creates identical returns for the first and second strategies by calculating the return on the first strategy as follows:

$\mathrm{USD} 100 \times(1.02396) \times(1.044536)=\mathrm{USD} 106.96$

In general, assume a shorter-term bond matures in $A$ periods and a longer-term bond matures in $B$ periods. The yields-to-maturity per period on these bonds are denoted as $z_{A}$ and $z_{B}$. The first bond is an $A$-period zero-coupon bond trading in the cash market. The second is a $B$-period zero-coupon cash market bond. The implied forward rate between period $A$ and period $B$ is denoted as $I F R_{A, B-A}$. It is a forward rate on a bond that starts in period $A$ and ends in period $B$. Its tenor is $B-A$ periods.

Equation 11 is a general formula for the relationship between the two spot rates $\left(z_{\mathrm{A}}, z_{\mathrm{B}}\right)$ and the implied forward rate $\left(I F R_{A, B-A}\right)$ :

$\left(1+z_{A}\right)^{A} \times\left(1+I F R_{A, B-A}\right)^{B-A}=\left(1+z_{B}\right)^{B}$

Exhibit 7 shows the possible implied forward rates over three periods.

\section{Exhibit 7: Implied Forward Rate Example}
\begin{center}
\includegraphics[max width=\textwidth]{2023_05_04_36535b8d80b32081d422g-278}
\end{center}

Using Equation 11, we can solve for the one-period rate in two periods (IFR $\mathrm{IF}_{2,1}$ ) as follows:

$$
\begin{aligned}
& \left(1+z_{2}\right)^{2} \times\left(1+I F R_{2,1}\right)^{1}=\left(1+z_{3}\right)^{3} \\
& (1.034197)^{2} \times\left(1+I F R_{2,1}\right)=(1.040005)^{3} \\
& I F R_{2,1}=5.1719 \%
\end{aligned}
$$

A series of forward rates can be used to construct a forward curve of rates with the same time frame that are implied by cash market transactions or may be observed in the interest rate derivatives market. Exhibit 8 summarizes the relationship between spot or zero rates and the forward curve for one-year rates from the earlier government bond example.

\section{Exhibit 8: Interest Rate Spot or Zero Curve and Forward Curve}
\begin{center}
\includegraphics[max width=\textwidth]{2023_05_04_36535b8d80b32081d422g-279}
\end{center}

While our examples so far have focused on government benchmark rates, we now turn our attention to short-term market reference rates. Assume, for example, that today we observe a current three-month MRR of $1.25 \%$ and a current six-month MRR of $1.75 \%$. How can we solve for the three-month implied forward MRR in three months' time $\left(I F R_{3 \mathrm{~m}, 3 \mathrm{~m}}\right)$ ?

In order to apply Equation 11 in this case, we must first ensure that the time frames, or periodicities, of the interest rates are the same. Here, the six-month rate has two periods per year and the three-month rate has four. Recall from an earlier fixed-income lesson that we can convert an annual percentage rate for $m$ periods per year, denoted as $A P R_{m}$, into an annual percentage rate for $n$ periods per year, $A P R_{n}$, as follows:

$$
\left(1+\frac{A P R_{m}}{m}\right)^{m}=\left(1+\frac{A P R_{n}}{n}\right)^{n} .
$$

First, we must convert the $1.75 \%$ semiannual MRR into a quarterly rate:

$$
\begin{aligned}
& \left(1+A P R_{4} / 4\right)^{4}=(1+0.0175 / 2)^{2} \\
& A P R_{4}=1.74619 \%
\end{aligned}
$$

We can use this to solve Equation 11 for $I F R_{3 m, 3 m}$ :

$$
\begin{aligned}
& \left(1+z_{3 m} / 4\right) \times\left(1+I F R_{3 m, 3 m} / 4\right)=\left(1+z_{6 \mathrm{~m}} / 4\right)^{2} \\
& (1+0.0125 / 4) \times\left(1+I F R_{3 \mathrm{~m}, 3 \mathrm{~m}} / 4\right)=(1+0.0174619 / 4)^{2} \\
& I F R_{3 m, 3 m}=2.24299 \%
\end{aligned}
$$

Our result can be confirmed by showing that CNY100,000,000 invested at:

\begin{enumerate}
  \item $1.25 \%$ for three months and reinvested at $2.24299 \%$ for the following three months, or

  \item $1.75 \%$ for six months will both return CNY100,875,000 in six months:

  \item CNY100,875,000 (= CNY100,000,000 $\times(1+0.0125 / 4) \times(1+0.0224299 / 4))$

  \item CNY100,875,000 $(=$ CNY100,000,000 $\times(1+0.0175 / 2))$

\end{enumerate}

As we will see in the next section, this breakeven reinvestment rate between zero rates of different maturities establishes a no-arbitrage price for a one-period interest rate forward contract.

\section{Forward Rate Agreements (FRAs)}
An OTC derivatives contract in which counterparties agree to apply a specific interest rate to a future period is a forward rate agreement (FRA). The underlying is a hypothetical deposit of a notional amount in the future at a market reference rate that is fixed at contract inception $(t=0)$. The FRA buyer, or long position, agrees to pay the deposit interest based on the agreed upon fixed rate and receives deposit interest based on a market reference rate that begins in $A$ periods and ends in $B$ periods (with a tenor of $B-A$ periods) and is determined on or just before the forward settlement date at time $t=A$.

Exhibit 9 shows FRA mechanics at $t=0$ and settlement at time $A$.

\section{Exhibit 9: Forward Rate Agreement (FRA) Mechanics}
\begin{center}
\includegraphics[max width=\textwidth]{2023_05_04_36535b8d80b32081d422g-280}
\end{center}

If Exhibit 9 looks familiar, it is because an FRA is a one-period version of the interest rate swap introduced as a series of exchanges in an earlier lesson. Similar to the swap, fixed versus floating payments on an FRA occur on a net basis and the notional is not exchanged but is used solely for interest calculations. The implied forward rates shown in the previous section represent the FRA fixed rate for a given period where no riskless profit opportunities exist. This no-arbitrage interest rate ensures that, similar to other forward contracts, the forward rate agreement has a value of zero $\left(V_{0}(T)=0\right)$ to both parties at inception. FRA settlement and other details are best demonstrated by extending our earlier implied forward rate example.

\section{EXAMPLE 8}
\section{CNY Forward Rate Agreement}
In a prior example, we solved for a three-month implied forward MRR in three months' time ( $I F R_{3 m, 3 m}$ ) of $2.24299 \%$. Yangzi Bank enters into an agreement to pay the three-month rate in three months' time and receive MRR as in Exhibit 9. Yangzi Bank uses the FRA to offset or hedge an underlying liability in three months on which it will owe MRR.

Consider the following FRA term sheet:

\section{Yangzi Bank CNY Forward Rate Agreement Term Sheet}
\begin{center}
\begin{tabular}{ll}
\hline
Start Date: & [Today] \\
Maturity Date: & [Three months from Start Date] \\
Notional Principal: & CNY 100,000,000 \\
Fixed-Rate Payer: & Yangzi Bank \\
Fixed Rate: & $2.24299 \%$ on a quarterly actual/360 basis \\
Floating-Rate Payer: & [Financial Intermediary] \\
Floating Rate: & Three-month CNY MRR on a quarterly actual/360 basis \\
Payment Date: & Maturity Date \\
Business Days: & Shanghai \\
Documentation: & ISDA Agreement and credit terms acceptable to both \\
 & parties \\
\end{tabular}
\end{center}

Similar to forward agreements for other underlying assets, the FRA settlement amount is a function of the difference between $\mathrm{F}_{0}(T)$ (here, a forward interest rate, $I F R_{A, B-A}$ ) and $S_{T}$ (the market reference rate $M R R_{B-A}$, or the MRR for $B$ $A$ periods, which ends at time $B$ ). In our example, $M R R_{3 m}$ is the three-month market reference rate, which sets in three months' time. FRAs are usually cash settled at the beginning of the period during which the reference rate applies. We calculate the net payment amount from the perspective of Yangzi Bank (the FRA buyer or fixed-rate payer) at the end of the interest period as follows:

Net Payment $=\left(M R R_{B-A}-I F R_{A, B-A}\right) \times$ Notional Principal $\times$ Period

If $M R R_{B-A}$ in three months' time sets at $2.15 \%$ and we assume a 90 -day interest period, we calculate the net payment at the end of the period as follows:

Net Payment $=\left(M R R_{B-A}-I F R_{A, B-A}\right) \times$ Notional Principal $\times$ Period

$(2.15 \%-2.24299 \%) \times$ CNY100,000,000 $\times 90 / 360$

$=-\mathrm{CNY23,247.50}$

This is the net payment amount at the end of six months. However, the FRA settles at the beginning rather than the end of the interest period, which is three months in our example. We must therefore calculate the present value of the settlement amount to the beginning of the period during which the reference rate applies, using $M R R_{B-A}$ as the discount rate:

Cash Settlement (PV): -CNY23,123.21 =-CNY23,247.50/(1+0.0215/4) Pricing and Valuation of Forward Contracts

\begin{center}
\includegraphics[max width=\textwidth]{2023_05_04_36535b8d80b32081d422g-282}
\end{center}

In this case, since $M R R_{B-A}$ sets below the fixed rate, as the fixed-rate payer and floating-rate $\left(M R R_{B-A}\right)$ receiver, Yangzi Bank must make a net settlement payment to the financial intermediary. Because Yangzi Bank has used the FRA to offset or hedge an underlying liability on which it owes MRR, the net settlement payment it makes on the FRA is offset by a lower MRR payment on its liability.

Forward rate agreements are almost exclusively used by financial intermediaries to manage rate-sensitive assets or liabilities on their balance sheets. The forward rate agreement and its single-period swap equivalent that settles at the end of an interest period form the basic building blocks for interest rate swaps, which are more frequently used by issuers and investors to manage interest rate risk.

\section{KNOWLEDGE CHECK}
\section{Forward Rate Agreements}
\begin{enumerate}
  \item Determine the correct answers to complete the following sentences: The yield-to-maturity rate and the zero rate for a bond with a single cash flow at maturity in one period are The price equivalent of a zero rate is the present value of a currency unit on a future date, known as a
\end{enumerate}

\section{Solution:}
The yield-to-maturity rate and the zero rate for a bond with a single cash flow at maturity in one period are identical. The price equivalent of a zero rate is the present value of a currency unit on a future date, known as a discount factor.

\begin{enumerate}
  \setcounter{enumi}{1}
  \item An analyst observes three- and four-year government benchmark zero-coupon bonds priced at 93 and 90 per 100 face value, respectively.
\end{enumerate}

\begin{center}
\includegraphics[max width=\textwidth]{2023_05_04_36535b8d80b32081d422g-282(1)}
\end{center}

\section{Solution:}
First, use the three-year $\left(D F_{3}\right)$ and four-year $\left(\mathrm{DF}_{4}\right)$ discount factors provided per unit of currency to derive three-year $\left(z_{3}\right)$ and four-year $\left(z_{4}\right)$ zero rates, which can be calculated as follows: $D F_{i}=\frac{1}{\left(1+z_{i}\right)^{i}}$

Therefore:

$0.93=1 /\left(1+z_{3}\right)^{3} ; z_{3}=2.4485 \%$

$0.90=1 /\left(1+z_{4}\right)^{4} ; z_{4}=2.6690 \%$

Solve for IFR $_{3,1}$ as follows:

$\left(1+z_{3}\right)^{3} \times\left(1+I F R_{3,1}\right)=\left(1+z_{4}\right)^{4}$

$(1.024485)^{3} \times\left(1+I F R_{3,1}\right)=(1.02669)^{4}$

$I F R_{3,1}=3.3333 \%$

\begin{enumerate}
  \setcounter{enumi}{2}
  \item Match the following descriptions with their corresponding interest rate derivative building block.
A. Forward rate agreement
  \item The breakeven reinvestment rate linking a short-dated and long-dated zero-coupon bond
B. Implied forward rate
  \item The present value of a currency unit on a future date
C. Discount factor
  \item A derivative in which counterparties agree to apply a specific interest rate to a future time period
\end{enumerate}

\section{Solution:}
\begin{enumerate}
  \item B is correct. The breakeven reinvestment rate linking a short-dated and a long-dated zero-coupon bond is the implied forward rate.

  \item $C$ is correct. The price equivalent of a zero rate is the present value of a currency unit on a future date, known as a discount factor.

  \item A is correct. A forward rate agreement is a derivative in which counterparties agree to apply a specific interest rate to a future period.

  \item A trader observes one-year and two-year zero-coupon bonds that yield $4 \%$ and $5 \%$, respectively, and would like to protect herself against a rise in one-year rates a year from now. Explain the position she should take in the FRA contract to achieve this objective and the forward interest rate she should expect on the contract.

\end{enumerate}

\section{Solution:}
A forward rate agreement involves an underlying hypothetical future deposit at a market reference rate fixed at contract inception $(t=0)$. To protect against higher rates, the trader should enter into a fixed-rate payer FRA in order to realize a gain if one-year spot rates one year from now exceed the current forward rate. A long FRA position, or fixed-rate payer, agrees to pay interest based on the fixed rate and receives interest based on a variable market reference rate determined at settlement. The FRA is priced based on the implied forward rate, or breakeven reinvestment rate between a shorter and a longer zero rate. We can solve for $I F R_{1,1}$ as follows:

$\left(1+z_{1}\right) \times\left(1+I F R_{1,1}\right)=\left(1+z_{2}\right)^{2}$

$(1.04) \times\left(1+I F R_{1,1}\right)=(1.05)^{2}$

$I F R_{1,1}=6.0096 \%$

\begin{enumerate}
  \setcounter{enumi}{4}
  \item A counterparty agrees to be the FRA fixed-rate receiver on a one-month AUD MRR in three months' time based on a AUD150,000,000 notional amount. If $I F R_{3 m, 1 m}$ at contract inception is $0.50 \%$ and one-month AUD MRR sets at $0.35 \%$ for settlement of the contract, calculate the settlement amount and interpret the results.
\end{enumerate}

\section{Solution:}
A short FRA position, or fixed-rate receiver, agrees to pay interest based on a market reference rate determined at settlement and receives interest based on a pre-agreed fixed rate, the implied forward rate $\left(I F R_{3 m, 1 m}\right)$. Since AUD $M R R_{B-A}$ at settlement is below the pre-agreed fixed rate, the FRA fixed-rate receiver realizes a gain and receives a net payment based on the following calculation:

Net Payment $=\left(I F R_{A, B-A}-M R R_{B-A}\right) \times$ Notional Principal $\times$ Period $=(0.50 \%-0.35 \%) \times$ AUD150,000,000 $\times(1 / 12)$

$=$ AUD18,750 at the end of the period

Solve for the present value of settlement given the contract is settled at the beginning of the period, using $M R R_{B-A}$ as the discount rate:

Cash Settlement $(\mathrm{PV})=$ AUD18,750.00/(1+0.35\%/12)

$=$ AUD18,744.53

\section{PRACTICE PROBLEMS}
Baywhite Financial is a broker-dealer and wealth management firm that helps its clients manage their portfolios using stand-alone derivative strategies. A new Baywhite analyst is asked to evaluate the following client situations.

\begin{enumerate}
  \item Match the following definitions with their corresponding forward pricing or valuation component.

  \item Equal to the difference between the current spot price (adjusted by remaining costs and benefits through maturity) and the present value of the forward price

  \item Future value of the underlying asset spot price $\left(S_{0}\right)$ compounded at the risk-free rate incorporating the present value of the costs and benefits of asset ownership

  \item Under no-arbitrage conditions for a given underlying spot price, $\mathrm{S}_{0}$, adjusted by the costs and benefits, risk-free rate $(r)$, and forward price, $F_{0}(T)$, this should be equal to zero (ignoring transaction costs).

  \item A Baywhite client currently owns 5,000 common non-dividend-paying shares of Vivivyu Inc. (VIVU), a digital media company, at a spot price of USD173 per share. The client enters into a forward commitment to sell half of its VIVU position in six months at a price of USD175.58. Which of the following market events is most likely to result in the greatest gain in the VIVU forward contract MTM value from the client's perspective?

\end{enumerate}

A. An increase in the risk-free rate

B. An immediate decline in the VIVU spot price following contract inception

C. A steady rise in the spot price of VIVU stock over time

\begin{enumerate}
  \setcounter{enumi}{2}
  \item A Baywhite client has entered into a long six-month MXN/USD FX forward contract - that is, an agreement to sell MXN and buy USD. The MXN/USD spot exchange rate at inception is 19.8248 (MXN19.8248 = USD1), the six-month MXN risk-free rate is $4.25 \%$, and the six-month USD risk-free rate is $0.5 \%$.
\end{enumerate}

Baywhite's market strategist predicts that the Mexican central bank (Banco de Mexico) will surprise the market with a $50 \mathrm{bp}$ short-term rate cut at its upcoming meeting. Which of the following statements best describes how the client's existing FX forward contract will be impacted if this prediction is realized and other parameters remain unchanged?

A. The lower interest rate differential between MXN and USD will cause the MXN/USD contract forward rate to be adjusted downward.

B. The client will realize an MTM gain on the FX forward contract due to the decline in the MXN versus USD interest rate differential. C. The lower interest rate differential between MXN and USD will cause the client to realize an MTM loss on the MXN/USD forward contract.

\begin{enumerate}
  \setcounter{enumi}{3}
  \item A client seeking advice on her fixed-income portfolio observes the price and yield-to-maturity of one-year $\left(r_{1}\right)$ and two-year $\left(r_{2}\right)$ annual coupon government benchmark bonds currently available in the market. Which of the following statements best describes how the analyst can determine a breakeven reinvestment rate in one year's time to help decide whether to invest now for one or two years?
\end{enumerate}

A. As the two-year rate involves intermediate cash flows, divide the square root of $\left(1+r_{2}\right)$ by $\left(1+r_{1}\right)$ and subtract 1 to arrive at a breakeven reinvestment rate for one year in one year's time.

B. Since the first year's returns are compounded in the second year, set $\left(1+r_{1}\right)$ multiplied by 1 plus the breakeven reinvestment rate equal to $\left(1+r_{2}\right)^{2}$ and solve for the breakeven reinvestment rate.

C. Since the breakeven reinvestment involves a zero-coupon cash flow, first substitute the one-year rate $\left(r_{1}\right)$ into the two-year bond price equation to solve for the two-year spot or zero rate $\left(z_{2}\right)$, then set $\left(1+r_{1}\right) \times(1+$ breakeven reinvestment rate $)=\left(1+z_{2}\right)^{2}$ and solve for the breakeven reinvestment rate.

\begin{enumerate}
  \setcounter{enumi}{4}
  \item Baywhite Financial seeks to gain a competitive advantage by making margin loans at fixed rates for up to 60 days to its investor clients. Since Baywhite borrows at a variable one-month market reference rate to finance these client loans, the firm enters into one-month FRA contracts on one-month MRR to hedge the interest rate exposure of its margin loan book. Which of the following statements best describes Baywhite's interest rate exposure and the FRA position it should take to hedge that exposure?
\end{enumerate}

A. Baywhite faces exposure to a rise in one-month MRR over the next 30 days, so it should enter into the FRA as a fixed-rate payer in order to benefit from a rise in one-month MRR above the FRA rate and offset its higher borrowing cost.

B. Baywhite faces exposure to a rise in one-month MRR over the next 30 days, so it should enter into the FRA as a fixed-rate receiver in order to benefit from a rise in one-month MRR above the FRA rate and offset its higher borrowing cost.

C. Baywhite faces exposure to a decline in one-month MRR over the next 30 days, so it should enter into the FRA as a fixed-rate receiver in order to benefit from a rise in one-month MRR above the FRA rate and offset its higher borrowing cost.

\section{SOLUTIONS}
\begin{enumerate}
  \item Solution to 1:

  \item B is correct. The forward contract MTM value between inception and maturity, $V_{t}(T)$, is equal to the difference between the current spot price (adjusted by costs and benefits through maturity) and the present value of the forward price.

  \item $\mathrm{C}$ is correct. The forward price, $F_{0}(T)$, is the future value of the underlying asset spot price $\left(S_{0}\right)$ compounded at the risk-free rate incorporating the present value of the costs and benefits of asset ownership.

  \item A is correct. Under no-arbitrage conditions for a given underlying spot price, $S_{0}$, adjusted by the costs and benefits, risk-free rate $(r)$, and forward price, $F_{0}(T)$, the forward contract MTM value at inception, $V_{0}(T)$, should be equal to zero (ignoring transaction costs).

  \item B is correct. The original VIVU spot price $\left(S_{0}\right)$ at $t=0$ must equal the present value of the forward price discounted at the risk-free rate, so an immediate fall in the spot price to $S_{0}{ }^{-}<S_{0}$ results in an MTM gain for the forward contract seller. A is not correct, since a higher risk-free rate will reduce the contract MTM from the client's perspective by reducing the PV of $F_{0}(T)$, while $C$ will also reduce the forward contract MTM from the seller's perspective.

  \item $\mathrm{C}$ is correct. A decline in the interest rate differential between MXN and USD will cause the client to realize an MTM loss on the MXN/USD forward contract, while B states that this decline will result in an MTM gain. A is incorrect as the forward price, $F_{0}(T)$, is not adjusted during the contract life.

\end{enumerate}

Specifically, the original MXN/USD forward exchange rate at inception is equal to $20.20\left(=19.8248 \mathrm{e}^{(.0425-0.005) \times 0.5}\right)$. If the $\mathrm{MXN}$ rate were to decline by 50 bps immediately after the contract is agreed, a new MXN/USD forward contract would be at a forward exchange rate of $20.15\left(=19.8248 \mathrm{e}^{(.0375-0.005) \times 0.5}\right)$. The MXN would weaken or depreciate against the USD. Since the MXN seller has locked in a forward sale at the original 20.20 versus the new 20.15 rate, the seller's MTM loss is equal to 0.05 , or MXN50,000 per MXN1,000,000 $(=0.05 \times$ $1,000,000)$ notional amount.

\begin{enumerate}
  \setcounter{enumi}{3}
  \item C is correct. The one-year annual rate equals the one-year zero rate, as it involves a single cash flow at maturity $\left(z_{1}=r_{1}\right)$. Since the breakeven reinvestment rate involves a single cash flow at maturity, substitute the one-year rate $\left(r_{1}\right)$ into the two-year bond price equation to solve for $z_{2}$, then set $\left(1+r_{1}\right) \times(1+$ breakeven reinvestment rate $)=\left(1+z_{2}\right)^{2}$ and solve for the breakeven reinvestment rate $\left(I F R_{1,1}\right)$.

  \item A is correct. As Baywhite faces exposure to a rise in one-month MRR over the next 30 days, it should enter into the FRA as a fixed-rate payer in order to benefit from a rise in one-month MRR above the FRA rate and offset its higher borrowing cost. Both $\mathrm{B}$ and $\mathrm{C}$ are incorrect, as the fixed-rate receiver in an FRA does not benefit but rather must make a higher payment upon settlement if MRR rises.

\end{enumerate}

\section{LEARNING MODULE
6}
\section{Pricing and Valuation of Futures Contracts}
\section{LEARNING OUTCOME}
\begin{center}
\begin{tabular}{c|l}
Mastery & The candidate should be able to: \\
\hline
$\square$ & compare the value and price of forward and futures contracts \\
$\square$ & explain why forward and futures prices differ \\
\end{tabular}
\end{center}

\section{INTRODUCTION}
Many of the pricing and valuation principles associated with forward commitments are common to both forward and futures contracts. For example, previous lessons demonstrated that forward commitments have a price that prevents market participants from earning riskless profit through arbitrage. It was also shown that long and short forward commitments may be replicated using a combination of long or short cash positions and borrowing or lending at the risk-free rate. Finally, both forward and futures pricing and valuation incorporate the cost of carry, or the benefits and costs of owning an underlying asset over the life of a derivative contract.

We now turn our attention to futures contracts. We discuss what distinguishes them from other forward commitments and how they are used by issuers and investors. We expand upon the daily settlement of futures contract gains and losses introduced earlier and explain the differences between forwards and futures. We also address and distinguish the interest rate futures market and its role in interest rate derivative contracts.

\section{Summary}
\begin{itemize}
  \item Futures are standardized, exchange-traded derivatives (ETDs) with zero initial value and a futures price $f_{0}(T)$ established at inception. The futures price, $f_{0}(T)$, equals the spot price compounded at the risk-free rate as in the case of a forward contract.

  \item The primary difference between forward and futures valuation is the daily settlement of futures gains and losses via a margin account. Daily settlement resets the futures contract value to zero at the current futures price $f_{t}(T)$. This process continues until contract maturity and the futures price converge to the spot price, $S_{T}$. - The cumulative realized mark-to-market (MTM) gain or loss on a futures contract is approximately the same as for a comparable forward contract.

  \item Daily settlement and margin requirements give rise to different cash flow patterns between futures and forwards, resulting in a pricing difference between the two contract types. The difference depends on both interest rate volatility and the correlation between interest rates and futures prices.

  \item The futures price for short-term interest rate futures is given by (100 yield), where yield is expressed in percentage terms. There is a price difference between interest rate futures and forward rate agreements (FRAs) due to convexity bias.

  \item The emergence of derivatives central clearing has introduced futures-like margining requirements for over-the-counter (OTC) derivatives, such as forwards. This arrangement has reduced the difference in the cash flow impact of ETDs and OTC derivatives and the price difference in futures versus forwards.

\end{itemize}

\section{LEARNING MODULE PRE-TEST}
\begin{enumerate}
  \item Describe the relationship between the spot price and futures price for an underlying asset with no additional cost or benefit of ownership.
\end{enumerate}

\section{Solution:}
As in the case of a forward contract, the futures price is the spot price compounded at the risk-free rate over the life of the contract.

\begin{enumerate}
  \setcounter{enumi}{1}
  \item Calculate the one-year futures price of a stock with a spot price $\left(S_{0}\right)$ of $€ 125$ and an annual dividend of $€ 2.50$ paid at maturity if the risk-free rate is $1 \%$ and interpret the result.
\end{enumerate}

\section{Solution:}
The no arbitrage futures price for an underlying asset with known benefits, such as a dividend, may be determined using the following equation:

$f_{0}(T)=\left[S_{0}-\mathrm{PV}_{0}(l)\right](1+r)^{T}$.

First, solve for the present value of the dividend $\mathrm{PV}_{0}(I)$ as follows:

$€ 2.48=(€ 2.50 / 1.01)$

Substitute $\mathrm{PV}_{0}(I)$ into the original equation to solve for $f_{0}(T)$ :

$f_{0}(T)=€ 123.75=(€ 125-€ 2.48)(1.01)$.

The stock spot price, $S_{0}$, is greater than the futures price, $f_{0}(T)$, as the dividend benefit is greater than the effect of compounding at the risk-free rate. Stated differently, the dividend yield of $2 \%(2.50 / 125)$ exceeds the $1 \%$ risk-free rate. 3. Identify the following statement as true or false and justify your answer: The futures price, $f_{0}(T)$, is constant until the contract matures, whereas the forward price, $F_{0}(T)$, fluctuates daily based upon market changes.

\section{Solution:}
The statement is false. The forward price, $F_{0}(T)$, is constant until the contract matures, whereas the futures price, $f_{0}(T)$, fluctuates daily based upon market changes.

\begin{enumerate}
  \setcounter{enumi}{3}
  \item Identify which of the following situations leads to which relationship between forward and futures prices for forward commitment contracts with otherwise identical terms.

  \item Futures prices are positively correlated with interest rates, and interest rates change over the contract period.

  \item Futures prices are negatively correlated with interest rates, and interest rates change over the contract period.

  \item Interest rates are constant over the forward commitment contract period. A. Forward prices are above futures prices: $F_{0}(T)>f_{0}(T)$.

\end{enumerate}

B. Forward and futures prices are the same: $F_{0}(T)=f_{0}(T)$.

C. Futures prices are above forward prices: $f_{0}(T)>F_{0}(T)$.

\section{Solution:}
\begin{enumerate}
  \item C is correct. If futures prices are positively correlated with interest rates, then higher prices lead to futures profits reinvested at rising rates, and lower prices lead to losses that may be financed at lower rates.

  \item A is correct. If futures prices are negatively correlated with interest rates, then higher prices lead to futures profits reinvested at lower rates, and lower prices lead to losses that must be financed at higher rates.

  \item B is correct. If interest rates are constant over the forward commitment contract period, then forward and futures prices are the same.

  \item Explain how the convexity bias can give rise to a difference in pricing of futures versus forward contracts.

\end{enumerate}

\section{Solution:}
Interest rate futures contracts have a linear payoff profile, which is a fixed amount for a given basis point change regardless of the magnitude of interest rate changes or final maturity of the underlying deposit. Interest rate forwards, on the other hand, have a price/yield relationship that varies in a non-linear manner. This difference increases with both the size of interest rate changes and the final maturity of the underlying contract.

\begin{enumerate}
  \setcounter{enumi}{5}
  \item Identify the following statement as true or false and justify your answer: Central clearing of OTC derivatives has reduced the difference in the cash flow impact of futures and forward contracts.
\end{enumerate}

\section{Solution:}
The statement is true. Under a central clearing framework for OTC derivatives, financial intermediaries who serve as counterparties are required to post daily margin or eligible collateral to the central counterparty (CCP) in a process very similar to futures margining. Dealers, therefore, often impose similar requirements on derivatives end users.

\section{PRICING OF FUTURES CONTRACTS AT INCEPTION}
compare the value and price of forward and futures contracts

When a forward commitment is initiated, no cash is exchanged and the contract is neither an asset nor a liability to the buyer or seller. The value of both a forward contract and a futures contract at initiation is zero:

$$
V_{0}(T)=0
$$

An underlying asset with no cost or benefit has a futures price $f_{0}(T)$ at $t=0$ of:

$$
f_{0}(T)=S_{0}(1+r)^{T}
$$

where $r$ is the risk-free rate and $T$ is the time to maturity. As in the case of a forward contract, the futures price is the spot price compounded at the risk-free rate over the life of the contract. This is shown in Exhibit 1, where the slope of the line is equal to the risk-free rate, $r$.

\section{Exhibit 1: Futures Price at Initiation}
\begin{center}
\includegraphics[max width=\textwidth]{2023_05_04_36535b8d80b32081d422g-292}
\end{center}

As for forwards, we use discrete compounding as in Equation 2 for futures on individual underlying assets. However, for underlying assets that are comprised of a portfolio-such as an equity, fixed-income, commodity, or credit index-or where the underlying involves foreign exchange with interest rates denominated in two currencies, continuous compounding is the preferred method, as shown in Equation 3:

$$
f_{0}(T)=S_{0} e^{r T}
$$

\section{EXAMPLE 1}
\section{Procam Investments - Gold Futures Contract}
As shown in a previous lesson, Procam Investments purchases a 100-ounce gold futures contract. The current spot price is $\$ 1,770.00$ per ounce, the risk-free rate is $2.0 \%$, and we assume gold may be stored at no cost. Calculate the no arbitrage futures price, $f_{0}(T)$, for settlement in 91 days $(T=91 / 365$ or 0.24932$)$.

\section{Solution}
Using Equation 2: $f_{0}(T)=S_{0}(1+r)^{T}$ $\$ 1,778.76$ per ounce $=\$ 1,770.00 \times(1.02)^{0.24932}$

Contract price $=\$ 177,876.04(=100 \times \$ 1,778.76)$

The futures price, $f_{0}(T)$, is identical to the forward price from a previous lesson. As in the case of a forward, for underlying assets with ownership benefits or income $(I)$ or costs $(C)$ expressed as a known amount in present value terms at time $t$ $=0$, the spot versus futures price relationship using discrete compounding is shown in Equation 4 and Exhibit 2:

$$
f_{0}(T)=\left[S_{0}-\mathrm{PV}_{0}(I)+\mathrm{PV}_{0}(C)\right](1+\mathrm{r})^{\mathrm{T}} .
$$

\section{Exhibit 2: Futures Prices with Underlying Asset Costs and Benefits}
\begin{center}
\includegraphics[max width=\textwidth]{2023_05_04_36535b8d80b32081d422g-293}
\end{center}

\section{EXAMPLE 2}
\section{Procam's Gold Futures Contract with Storage Costs}
Procam purchased a gold futures contract in Example 1 at $f_{0}(T)$ of $\$ 177,876.04$ (or $\$ 1,778.76$ per ounce) with $S_{0}$ equal to $\$ 1,770$ per ounce. How would $f_{0}(T)$ change to satisfy the no arbitrage condition if a $\$ 2$ per ounce cost of gold storage and insurance were payable at the end of the contract?

The futures price for a commodity with known storage cost amounts may be determined using Equation 4 , where $\mathrm{PV}_{0}(I)=0$ :

$$
f_{0}(T)=\left[S_{0}+\mathrm{PV}_{0}(C)\right](1+r)^{T} .
$$

First, solve for the present value of the storage cost per ounce, $\mathrm{PV}_{0}(C)$, as follows:

$$
\mathrm{PV}_{0}(C)=\$ 1.99=\left[\$ 2(1.02)^{-0.24982}\right] \text {. }
$$

Substitute $\mathrm{PV}_{0}(C)=\$ 1.99$ into Equation 4 to solve for $f_{0}(T)$ :

$$
\begin{aligned}
& f_{0}(T)=(\$ 1,770.00+\$ 1.99)(1.02)^{-0.24982} \\
& =\$ 1,780.78 \text { per ounce }
\end{aligned}
$$

The addition of storage and insurance costs increases the difference between the spot price and futures price. Finally, as in the case of forwards, a futures price, $f_{0}(T)$, significantly below the no arbitrage price including cash costs and benefits may indicate the presence of a convenience yield.

\section{MTM VALUATION: FORWARDS VERSUS FUTURES}
Examples 1 and 2 show the similarities between forward and futures prices at contract inception. Over time, different forward and futures contract features lead to different MTM values for contracts with the same underlying assets and otherwise identical details. Example 3 shows how the daily settlement of gains and losses causes this difference to arise.

\section{EXAMPLE 3}
\section{Procam Forward versus Futures Pricing and Valuation}
We extend the earlier example to compare forward and futures pricing and valuation. In both cases, Procam Investments enters a cash-settled forward commitment to buy 100 ounces of gold at a price $\left(f_{0}[T]=F_{0}[T]\right)$ of $\$ 1,778.76$ per ounce in 91 days, with a risk-free rate of $2 \%$ and no gold storage cost.

\section{Forward Contract}
The forward price is $F_{0}(T)=\$ 1,778.76$ per ounce. No cash is exchanged or deposited at inception, and the contract value at inception, $V_{0}(T)$, is zero.

\begin{center}
\includegraphics[max width=\textwidth]{2023_05_04_36535b8d80b32081d422g-294}
\end{center}

Over time, the forward price, $F_{0}(T)$, does not change, and the MTM at any time, $\left[V_{t}(T)\right]$, equals the difference between the current spot price, $S_{t}$, and the present value of the forward price, $\mathrm{PV}_{t}$ of $F_{0}(T)$, shown from Procam's (the forward buyer's) perspective:

$$
V_{t}(T)=S_{t}-F_{0}(T)(1+r)^{-(T-t)}
$$

For example, say 71 days have elapsed and 20 days remain to maturity, $T-t=$ $20 / 365$ or 0.0548 . If the gold spot price, $\left(S_{t}\right)$, has fallen by $\$ 50$ since inception to $\$ 1,720$ per ounce, solve for $V_{t}(T)$ as follows:

$$
\begin{aligned}
& V_{t}(T)=\$ 1,720-\$ 1,778.76(1.02)^{-0.0548} \\
& =-\$ 56.83 \text { per ounce, or a } \$ 5,683 \text { MTM loss }(=-\$ 56.83 \times 100 \text { ounces }) .
\end{aligned}
$$

Under terms of the forward contract, no settlement of the MTM amount occurs until maturity. This process of resetting the contract value to zero each day makes it very unlikely that the futures contract would reach a similar MTM value.

\section{Futures Contract}
The futures price, $f_{0}(T)$, is $\$ 1,778.76$ per ounce. As per futures exchange daily settlement rules, the contract buyer and seller must post an initial cash margin of $\$ 4,950$ per gold contract (100 ounces) and maintain a maintenance margin of $\$ 4,500$ per contract. If a margin balance falls below $\$ 4,500$, a counterparty receives a margin call and must immediately replenish its account to the initial $\$ 4,950$.

\begin{center}
\includegraphics[max width=\textwidth]{2023_05_04_36535b8d80b32081d422g-295}
\end{center}

Consider the first day of trading, where the spot gold price, $\left(S_{0}\right)$, is $\$ 1,770$ per ounce and the opening gold futures price, $f_{0}(T)$, is $\$ 1,778.76$ per ounce.

\begin{itemize}
  \item Assume that the gold futures price, $f_{1}(T)$, falls by $\$ 5$ on the first trading day to $\$ 1,773.76$ and the spot price, $S_{1}$, ends the day at a no arbitrage equivalent of $\$ 1,765.12\left(=\$ 1,773.76(1.02)^{-90 / 365}\right)$.

  \item Procam realizes a $\$ 500$ MTM loss (= $\$ 5$ per ounce $\times 100$ ounces) deducted from its margin account, leaving Procam with $\$ 4,450$.

  \item The MTM value of Procam's futures contract resets to zero at the futures closing price of $\$ 1,773.76$ per ounce.

  \item Since Procam's margin account balance has fallen below the $\$ 4,500$ maintenance level, it must deposit $\$ 500$ to return the balance to the $\$ 4,950$ initial margin.

\end{itemize}

\section{Futures versus Forward Price and Value over Time}
Using the same details, we compare the futures and forward price and value over two trading days. Assume that day two trading opens at day one's closing spot and futures prices. The following table shows the comparison:

\section{Beginning of Day 2 Trading}
\begin{center}
\begin{tabular}{lcccc}
\hline
Contract Type & $\begin{array}{c}\text { Contract } \\ \text { Price }\end{array}$ & $\begin{array}{c}\text { Contract } \\ \text { MTM }\end{array}$ & $\begin{array}{c}\text { Realized } \\ \text { MTM }\end{array}$ & $\begin{array}{c}\text { Margin } \\ \text { Deposit }\end{array}$ \\
\hline
Forward & $F_{0}(T)=\$ 177,876$ & $-\$ 498$ & $\$ 0$ & $\$ 0$ \\
Futures & $f_{1}(T)=\$ 177,376$ & $\$ 0$ & $-\$ 500$ & $\$ 4,950$ \\
\hline
\end{tabular}
\end{center}

The forward MTM contract value, $V_{t}(T)$, equals the difference between the current spot price, $S_{1}=\$ 1,765.12$, and the present value of the original forward price, $\mathrm{PV}_{t}\left[F_{0}(T)\right]$, here with 90 days remaining to maturity, $T-\mathrm{t}=90 / 365$ or 0.24657 :

$$
\begin{aligned}
& V_{t}(T)=\$ 1,765.12-\$ 1,778.76(1.02)^{-0.24657} \\
& =\$ 4.98 \text { per ounce. }
\end{aligned}
$$

Extending our example to the beginning of day three at the prior day's close, assume a $\$ 4$ per ounce fall in the gold futures price on day two, $f_{2}(T)$, to $\$ 1,769.76$ and a no arbitrage equivalent spot price decline, $S_{2}$, to $\$ 1,761.24$ (= $\left.\$ 1,769.76[1.02]^{-89 / 365}\right)$. The following table shows the summary on day three:

\section{Beginning of Day 3 Trading}
\begin{center}
\begin{tabular}{lcccc}
\hline
Contract Type & $\begin{array}{c}\text { Contract } \\ \text { Price }\end{array}$ & $\begin{array}{c}\text { Contract } \\ \text { MTM }\end{array}$ & $\begin{array}{c}\text { Realized } \\ \text { MTM }\end{array}$ & $\begin{array}{c}\text { Margin } \\ \text { Deposit }\end{array}$ \\
\hline
Forward & $F_{0}(T)=\$ 177,876$ & $-\$ 895$ & $\$ 0$ & $\$ 0$ \\
Futures & $f_{2}(T)=\$ 176,976$ & $\$ 0$ & $-\$ 400$ & $\$ 4,550$ \\
\hline
\end{tabular}
\end{center}

The forward MTM contract value, $V_{t}(T)$, equals the difference between the current spot price, $S_{2}=\$ 1,761.24$, and the present value of the original forward price, $\mathrm{PV}_{t}\left[F_{0}(T)\right]$, here with 89 days remaining to maturity, $T-t=89 / 365$ or 0.24384 :

$$
\begin{aligned}
& V t(T)=\$ 1,761.24-\$ 1,778.76(1.02)^{-0.24384} \\
& =\$ 8.95 \text { per ounce. }
\end{aligned}
$$

Example 3 demonstrates the key differences in the price and value of forward and futures contracts over time. The forward contract price, $F_{0}(T)$, remains fixed until the contract matures. Forward contract MTM value changes are captured by the difference between the current spot price, $S_{t}$, and the present value of the forward price, $\mathrm{PV}_{t}\left[F_{0}(T)\right]$. This forward contract MTM is not settled until maturity, giving rise to counterparty credit risk over time since no cash is exchanged from inception of the contract to its maturity or expiration. Futures contract prices fluctuate daily based upon market changes. The daily settlement mechanism resets the futures MTM to zero, and variation margin is exchanged to settle the difference, reducing counterparty credit risk. The cumulative realized MTM gain or loss on a futures contract is approximately the same as for a comparable forward contract. We will explore these differences in the next lesson after first turning our attention to forward and futures contracts on market reference rates.

\begin{center}
\includegraphics[max width=\textwidth]{2023_05_04_36535b8d80b32081d422g-296}
\end{center}

\section{INTEREST RATE FUTURES VERSUS FORWARD CONTRACTS}
In an earlier lesson on interest rate forward contracts, zero rates derived from coupon bonds were used to derive a future investment breakeven rate (or implied forward rate). The implied forward rate was shown to equal the no arbitrage fixed rate on a forward rate agreement (FRA) under which counterparties exchange a fixed-for-floating cash flow at a time in the future.

Futures markets on short-term interest rates offer market participants a highly liquid, standardized alternative to FRAs. Interest rate futures contracts are available for monthly or quarterly market reference rates for successive periods out to final contract maturities of up to ten years in the future. Although the underlying variable is the market reference rate (MRR) on a hypothetical deposit on a future date as for forward rate agreements, interest rate futures trade on a price basis as per the following general formula:

$$
f_{A, B-A}=100-\left(100 \times \mathrm{MRR}_{A, B-A}\right),
$$

where $f_{A, B-A}$ represents the futures price for a market reference rate for $B-A$ periods that begins in $A$ periods $\left(\mathrm{MRR}_{A, B-A}\right)$, as shown in Exhibit 3.

\section{Exhibit 3: Interest Rate Futures Contract Mechanics}
\begin{center}
\includegraphics[max width=\textwidth]{2023_05_04_36535b8d80b32081d422g-297}
\end{center}

For example, we may solve for the implied three-month MRR rate in three months' time (where $A=3 \mathrm{~m}, B=6 \mathrm{~m}, B-A=3 \mathrm{~m}$ ) if an interest rate futures contract is trading at a price of 98.25 using Equation 5:

$$
\begin{aligned}
& f_{3 m, 3 m}: 98.25=100-\left(100 \times \operatorname{MRR}_{A, B-A}\right) \\
& \operatorname{MRR}_{3 m, 3 m}=1.75 \% .
\end{aligned}
$$

This (100 - yield) price convention results in an inverse price/yield relationship but is not the same as the price of a zero-coupon bond at the contract rate. A long futures position involves earning or receiving MRR in $A$ periods, whereas a short position involves paying MRR in $A$ periods. The interest rate exposure profile for long and short futures contracts are as follows:

\begin{itemize}
  \item Long futures contract (lender): Gains as prices rise, future MRR falls

  \item Short futures contract (borrower): Gains as prices fall, future MRR rises

\end{itemize}

In an earlier lesson, Yangzi Bank enters into an FRA as a fixed-rate payer to hedge a liability on which it owes MRR in the future, realizing a gain on the FRA contract as rates rise. Note that this would be equivalent to taking a short position on a CNY MRR futures contract if one were available. Exhibit 4 summarizes the relationship between futures and FRAs.

\section{Exhibit 4: Interest Rate Futures versus FRAs}
\begin{center}
\begin{tabular}{lcc}
\hline
Contract Type & Gains from Rising MRR & Gains from Falling MRR \\
\hline
Interest rate futures & Short futures contract & Long futures contract \\
Forward rate & $\begin{array}{c}\text { Long FRA: FRA fixed-rate payer } \\ \text { (FRA floating-rate receiver) }\end{array}$ & $\begin{array}{c}\text { Short FRA: FRA floating-rate } \\ \text { payer (FRA fixed-rate receiver) }\end{array}$ \\
\hline
\end{tabular}
\end{center}

Interest rate futures daily settlement occurs based on price changes, which translate into futures contract basis point value (BPV) as follows: Futures Contract BPV $=$ Notional Principal $\times 0.01 \% \times$ Period .

For example, assuming a $\$ 1,000,000$ notional for three-month MRR of $2.21 \%$ for one quarter (or $90 / 360$ days), the underlying deposit contract value would be:

$\$ 1,005,525=\$ 1,000,000 \times[1+(2.21 \% / 4)]$.

Consider how a one basis point $(0.01 \%)$ change in MRR affects contract value:

1 bp increase $(2.22 \%): \$ 1,005,550=\$ 1,000,000 \times[1+(2.22 \% / 4)]$.

1 bp decrease $(2.20 \%): \$ 1,005,500=\$ 1,000,000 \times[1+(2.20 \% / 4)]$.

Both the increase and decrease in MRR by one basis point change the contract BPV by $\$ 25$. Short-term interest rate futures are characterized by a fixed linear relationship between price and yield changes. The following example illustrates their use in practice.

\section{EXAMPLE 4}
\section{Interest Rate Futures - Baywhite Margin Loan Book}
In an earlier example, Baywhite Financial offered 60-day margin loans at fixed rates to its clients and borrowed at a variable one-month MRR to finance the loans. Describe Baywhite's residual interest rate exposure and how it may use interest rate futures as a hedge.
\includegraphics[max width=\textwidth, center]{2023_05_04_36535b8d80b32081d422g-298}

The diagram shows that Baywhite faces the risk of higher MRR in one month's time $\left(M R R_{1,1}\right)$, which would reduce the return on its fixed margin loans. In the prior example, Baywhite entered an FRA where it agreed to pay fixed one-month MRR and receive floating. If Baywhite were to use an interest rate futures contract instead, it would sell a futures contract on one-month MRR. The futures contract BPV for a $\$ 50,000,000$ notional amount is:

Contract BPV $=\$ 416.67(=\$ 50,000,000 \times 0.01 \% \times[1 / 12])$

If Baywhite sells $f_{1,1}$ for $\$ 98.75$ (or $\mathrm{MRR}_{1,1}=1.25 \%$ ) and settles at maturity at a price of $\$ 97.75\left(\mathrm{MRR}_{1,1}=2.25 \%\right)$, it would expect to have a cumulative gain on the contract through maturity equal to $\$ 41,667$ (= Contract BPV $\times 100 \mathrm{bps}$ ).

\section{KNOWLEDGE CHECK}
\section{Valuation and Pricing of Futures Contracts}
\begin{enumerate}
  \item Identify which of the following features corresponds to which type of forward commitment contract. 1. The daily change in contract price is used to A. Forward contract determine and settle the MTM.

  \item Inclusion of storage and insurance costs $\quad$ B. Futures contract increases the difference between the spot and forward commitment contract price.

  \item The contract price established at inception

\end{enumerate}

C. Both a forward and a futures remains unchanged over time. contract

\section{Solution:}
\begin{enumerate}
  \item B is correct. The futures contract price change at the close of each trading day is used to determine the daily MTM settlement via the margin account. 2. $\mathrm{C}$ is correct. Storage and insurance costs increase the forward commitment price for both forward and futures contracts.
\end{enumerate}

3 . A is correct. The forward contract price, $F_{0}(T)$, established at $t=0$ remains unchanged and is used to calculate the MTM settlement at maturity.

\begin{enumerate}
  \setcounter{enumi}{1}
  \item Calculate the correct answer to fill in the blank and justify your response: An investor entered a short oil futures contract position three months ago on 1,000 barrels at an initial price of $\$ 69.00$ per barrel. The constant risk-free rate is $0.50 \%$. Daily oil spot and futures prices for the final 10 days of trading are shown in the following table. The change in the investor's futures contract value on day $T-5$ is closest to
\end{enumerate}

\begin{center}
\begin{tabular}{lll}
\hline
Day & Crude Oil Spot Price (\$) & Crude Oil Futures Price (\$) \\
\hline
T-10 & 69.62 & 68.69 \\
T-9 & 69.01 & 68.11 \\
T-8 & 66.88 & 66.15 \\
T-7 & 65.18 & 64.77 \\
T-6 & 66.72 & 66.02 \\
T-5 & 68.59 & 68.01 \\
T-4 & 68.80 & 68.08 \\
T-3 & 68.93 & 68.32 \\
T-2 & 69.43 & 69.15 \\
T-1 & 69.36 & 69.18 \\
T & 70.03 & 70.03 \\
\hline
\end{tabular}
\end{center}

\section{Solution:}
The answer is $\$(1,990)$. The MTM of the investor's futures position is the intraday futures price change per barrel multiplied by 1,000 barrels:

$$
V_{T-5}(T)=-\left[f_{T-5}(T)-f_{T-6}(T)\right] \times 1,000=-(68.01-66.02) \times 1,000=
$$\$\$

-\$ 1,990

\$\$

Note the negative sign refers to the investor's short futures position. The investor realizes a loss as the futures price rises due to the short position.

\begin{enumerate}
  \setcounter{enumi}{2}
  \item Determine the correct answers to complete the following sentences: The daily settlement mechanism resets the futures MTM to , and margin is exchanged to settle the difference. The realized MTM gain or loss on a futures contract is approximately the same as for a comparable forward contract.
\end{enumerate}

\section{Solution:}
The daily settlement mechanism resets the futures MTM to zero, and margin is exchanged to settle the difference. The cumulative realized MTM gain or loss on a futures contract is approximately the same as for a comparable forward contract.

\begin{enumerate}
  \setcounter{enumi}{3}
  \item Identify the following statement as true or false and justify your answer: An FRA fixed-rate receiver (floating-rate payer) position is equivalent to a long interest rate futures contract on MRR, as both positions realize a gain as MRR falls below the initial fixed rate.
\end{enumerate}

\section{Solution:}
The statement is true. An FRA fixed-rate receiver (floating-rate payer) position realizes a gain as MRR falls as the counterparty receives the fixed MRR and owes the floating MRR in the future. A long futures contract price is based on (100 - yield), which rises as yield-to-maturity (MRR) falls.

\begin{enumerate}
  \setcounter{enumi}{4}
  \item From May 2020 to January 2021, the three-month SONIA (Sterling Overnight Index Average) futures contract expiring in June 2021 traded at a price above 100. Describe the interest rate scenario implied by this futures price and justify your response.
\end{enumerate}

\section{Solution:}
The future interest rate scenario implied by the futures price above 100 is a negative SONIA interest rate in June 2021. For example, if we consider the futures price for three-month SONIA one year forward as of June 2020 from Equation 5:

$f_{1 y, 3 m}=100-\left(100 \times \mathrm{MRR}_{1 y, 3 m}\right)$.

If $f_{1 y, 3 m}>100$, this implies that $\mathrm{MRR}_{1 y, 3 m}<0$.

\section{FORWARD AND FUTURES PRICE DIFFERENCES}
explain why forward and futures prices differ

Despite their similar symmetric payoff profile at maturity, differences exist between forward and futures valuation and pricing because of different cash flow profiles over the life of a futures versus a forward contract with otherwise similar characteristics. The distinguishing features of a futures contract are the posting of initial margin, daily mark-to-market, and settlement of gains and losses.

These features limit the MTM value of a futures contract to the daily gain or loss since the previous day's settlement. When that value is paid out in the daily settlement via the margin account, the futures price resets to the current settlement price and the MTM value goes to zero. Forward contracts, on the other hand, involve privately negotiated credit terms (which sometimes involve cash or securities collateral) and do not require daily MTM cash settlement. Forward contract settlement occurs at maturity in a one-time cash settlement of the cumulative change in contract value. The different patterns of cash flows for forwards and futures can lead to a difference in the pricing of forwards versus futures. Forward and futures prices are identical under certain conditions, namely:

\begin{itemize}
  \item if interest rates are constant, or

  \item if futures prices and interest rates are uncorrelated.

\end{itemize}

On the other hand, violations of these assumptions can give rise to differences in pricing between these two contracts. For example, if futures prices are positively correlated with interest rates, long futures contracts are more attractive than long forward positions for the same underlying and maturity. The reason is because rising prices lead to futures profits that are reinvested in periods of rising interest rates, and falling prices lead to losses that occur in periods of falling interest rates. The price differential will also vary with the volatility of interest rates.

A negative correlation between futures prices and interest rates leads to the opposite interpretation, with long forward positions being more desirable than long futures positions. In general, the more desirable contract will tend to have the higher price.

\section{INTEREST RATE FORWARD AND FUTURES PRICE DIFFERENCES}
The short maturity of most futures contracts and the ability of most market participants to borrow near risk-free rates for these maturities typically results in little to no distinction between futures and forward prices. An exception to this is the so-called convexity bias, which arises given the difference in price changes for interest rate futures versus forward contracts, as illustrated in the following example.

\section{EXAMPLE 5}
\section{Interest Rate Forwards versus Futures}
Let us return to an example from the prior lesson with an interest rate futures contract of $\$ 1,000,000$ notional for three-month MRR of $2.21 \%$ for one quarter (or $90 / 360$ days). Recall that the underlying deposit contract value was:

$$
\$ 1,005,525=\$ 1,000,000 \times[1+(2.21 \% / 4)]
$$

The contract BPV was shown to be $\$ 25(=\$ 1,000,000 \times 0.01 \% \times[1 / 4])$.

Consider in contrast a $\$ 1,000,000$ notional FRA on three-month MRR in three months' time with an identical 2.21\% rate. The net payment on the FRA is based upon the difference between MRR and the implied forward rate (IFR):

Net Payment $=\left(\mathrm{MRR}_{B-A}-\operatorname{IFR}_{A, B-A}\right) \times$ Notional Principal $\times$ Period.

For example, if the observed MRR in three months is $2.22 \%(+0.01 \%)$, the net payment at maturity would be $\$ 25(=\$ 1,000,000 \times 0.01 \% \times[1 / 4])$. However, the settlement of an FRA is based upon the present value of the final cash flow discounted at MRR, so:

Cash Settlement (PV): $\$ 24.86=\$ 25 /(1+0.0222 / 4)$

If we increase the magnitude of the MRR change at settlement and compare these changes between a long interest rate futures position and a short receive-fixed (pay floating) FRA contract, we arrive at the following result:

\begin{center}
\begin{tabular}{ccc}
\hline
MRR $_{\mathbf{3 m}, \mathbf{3 m}}$ & $\begin{array}{c}\text { Short FRA Cash } \\ \text { Settlement (PV) }\end{array}$ & Long Futures Settlement \\
\hline
$2.01 \%$ & $\$ 497.50$ & $\$ 500$ \\
$2.11 \%$ & $\$ 248.69$ & $\$ 250$ \\
$2.21 \%$ & $\$ 0$ & $\$ 0$ \\
$2.31 \%$ & $(\$ 248.56)$ & $(\$ 250)$ \\
$2.41 \%$ & $(\$ 497.01)$ & $(\$ 500)$ \\
\hline
\end{tabular}
\end{center}

Although the settlement values differ due to different conventions across these instruments, note that while the futures contract has a fixed linear payoff profile for a given basis point change, the FRA settlement does not.

In the FRA contract in Example 5, we see that the percentage price change is greater (in absolute value) when MRR falls than when it rises. Although the difference here is very small due to the short forward period, note that this non-linear relationship is the convexity property, which characterizes fixed-income instruments from earlier lessons, as shown in Exhibit 5.

\section{Exhibit 5: Convexity Bias}
\begin{center}
\includegraphics[max width=\textwidth]{2023_05_04_36535b8d80b32081d422g-302}
\end{center}

The discounting feature of the FRA, which is not present in the futures contract, leads to a convexity bias that is greater for longer discounting periods. You will recall from an earlier lesson that a discount factor is the price equivalent of a zero rate and is the present value of a currency unit on a future date, which may also be interpreted as the price of a zero-coupon cash flow or bond. We will show later how this discount factor is used to price interest rate swaps and other derivatives.

\section{EFFECT OF CENTRAL CLEARING OF OTC DERIVATIVES}
In periods of market and/or counterparty financial stress, large price movements combined with a derivative counterparty's inability to meet a margin call may force the closeout of a futures transaction prior to maturity. An OTC forward contract with more flexible credit terms, however, may remain outstanding.

The advent of derivatives central clearing, introduced in an earlier lesson, has created futures-like margining requirements for OTC derivative dealers who buy and sell forwards to derivatives end users. Dealers who are required to post cash or highly liquid securities to a central counterparty often impose similar requirements on derivatives end users. These dealer margin requirements reduce the difference in the cash flow impact of exchange-traded and OTC derivatives. This arrangement between dealers and their counterparties, shown in Exhibit 6, has been added to the original central clearing diagram from an earlier lesson.

\section{Exhibit 6: Margin Requirements for Centrally Cleared OTC Derivatives}
\begin{center}
\includegraphics[max width=\textwidth]{2023_05_04_36535b8d80b32081d422g-303}
\end{center}

Investors who actively use both exchange-traded futures or OTC forwards must therefore maintain sufficient cash or eligible collateral to fulfill margin or collateral requirements. Market participants must also consider the financing, transaction, and administrative costs of maintaining these positions when using derivatives in a portfolio.

\section{KNOWLEDGE CHECK}
\section{Forward and Futures Prices}
\begin{enumerate}
  \item Identify the following statement as true or false and justify your answer: If futures prices are positively correlated with interest rates, long futures contracts are more attractive than long forward positions for the same underlying and maturity.
\end{enumerate}

\section{Solution:}
The statement is true. If futures contract prices rise as interest rates rise, a long futures contract holder can reinvest futures contract profits at higher interest rates.

\begin{enumerate}
  \setcounter{enumi}{1}
  \item An investor seeks to hedge its three-month MRR exposure on a $€ 25,000,000$ liability in two months and observes an implied forward rate today (IF$\mathrm{R}_{2 m, 3 m}$ ) of $2.95 \%$. Calculate the settlement amounts if the investor enters a long pay-fixed (receive floating) FRA and a short futures contract, and compare and interpret the results if $\mathrm{MRR}_{2 m, 3 m}$ settles at $3.25 \%$.
\end{enumerate}

\section{Solution:}
Solve for the pay-fixed FRA Cash Settlement (PV) value as follows:

Net Payment $=\left(\mathrm{MRR}_{B-A}-\mathrm{IFR}_{A, B-A}\right) \times$ Notional Principal $\times$ Period

$=\pounds 18,750(=[3.25 \%-2.95 \%] \times \pounds 25,000,000 \times[1 / 4])$.

The present value based upon $\mathrm{MRR}_{2 m, 5 m}$ of $3.25 \%$ is:

$=\pounds 18,598.88(=\pounds 18,750 /[1+0.0325 / 4])$.

For the futures contract, contract BPV is equal to: Contract BPV $=\pounds 625(=\pounds 25,000,000 \times 0.01 \% \times[1 / 4])$

For a 30-basis point increase in MRR (=3.25\% - 2.95\%), the short futures contract will realize a price appreciation of $€ 18,750(=\pounds 625 \times 30)$. Both contracts result in a gain from the investor's perspective as MRR rises. However, the futures settlement is larger due to the discounting of the FRA final payment to the settlement date.

\begin{enumerate}
  \setcounter{enumi}{2}
  \item Explain why short futures contracts are more attractive than short forward positions if futures prices are negatively correlated with interest rates for positions with the same underlying and maturity.
\end{enumerate}

\section{Solution:}
The reason that short futures contracts are more attractive than short forward positions if futures prices are negatively correlated with interest rates is because falling prices lead to futures profits that are reinvested in periods of rising interest rates, and rising prices lead to losses that occur in periods of falling interest rates.

\begin{enumerate}
  \setcounter{enumi}{3}
  \item Identify the following statement as true or false and justify your answer:
\end{enumerate}

The convexity bias between interest rate futures and interest rate forwards causes the percentage price change to be greater (in absolute value) when MRR rises than when it falls for a forward than for a futures contract.

\section{Solution:}
The statement is false. The convexity bias between interest rate futures and interest rate forwards causes the percentage price change to be greater (in absolute value) when MRR falls than when it rises for a forward contract, as opposed to a futures contract.

\begin{enumerate}
  \setcounter{enumi}{4}
  \item Explain how central clearing of derivatives reduces the difference in futures and forward prices for the same underlying and maturity periods.
\end{enumerate}

\section{Solution:}
The central clearing of derivatives has created futures-like margining requirements for OTC derivative dealers who buy and sell forwards to derivatives end users. Dealers who are required to post cash or highly liquid securities to a central counterparty often impose similar requirements on derivatives end users. This arrangement between derivative dealers and their counterparties will reduce the difference in the cash flow impact of ETD and OTC derivatives. Hence, any price difference between ETD futures and OTC forwards will be reduced.

\section{PRACTICE PROBLEMS}
Ace Limited is a financial intermediary active in both futures and forward markets. You have been hired as an investment consultant and asked to review Ace's activities and answer the following questions.

\begin{enumerate}
  \item Ace serves as a futures commission merchant to assist several of its commodity trading adviser (CTA) clients to clear and settle their futures margin positions with the futures exchange. Ace is reviewing the copper futures market for a CTA client considering a long copper futures position for the first time. Details of the copper futures market are as follows:
\end{enumerate}

\section{CME Copper Futures Contract Specifications}
\begin{center}
\begin{tabular}{ll}
\hline
Contract Maturities: & Monthly [from 1 month to 15 months] \\
Contract Size: & 25,000 pounds \\
Delivery Type: & Cash settled \\
Price Quotation: & $\$$ per pound \\
Initial Margin: & $\$ 10,000$ per contract \\
Maintenance Margin: & $\$ 6,000$ per contract \\
Final Maturity: & Last CME business day of contract month \\
Daily Settlement: & CME Trading Operations calculates daily settlement values \\
 & based on its published procedures \\
\end{tabular}
\end{center}

Today's copper spot price is $\$ 4.25$ per pound, and the constant risk-free rate is $1.875 \%$. Each contract has a $\$ 10$ storage cost payable at the end of the month. Which of the following statements best characterizes the margin exposure profile of Ace's CTA client if it enters a one-month copper futures contract?

A. The CTA will be expected to post $\$ 10,000$ initial margin and would receive a margin call if the copper futures price were to immediately fall below $\$ 4.10$ per pound or below a price of $\$ 102,425$ per contract.

B. The CTA would be expected to post $\$ 10,000$ in initial margin and would receive a margin call at any time over the life of the contract if the copper futures price were to immediately fall below $\$ 3.86$ per pound or below a price of $\$ 96,425$ per contract.

C. The CTA will be expected to post $\$ 10,000$ initial margin, but we cannot determine the exact futures price at which a margin call will occur as the futures MTM is settled each day and the contract value resets to zero.

\begin{enumerate}
  \setcounter{enumi}{1}
  \item One of Ace's investor clients has entered a long six-month forward transaction with Ace on 100 shares of Xenaliya (XLYA), a non-dividend-paying technology stock. The stock's spot price per share, $S_{0}$, is $€ 85$, and the risk-free rate is a constant $1 \%$ for all maturities. Ace has hedged the client transaction with a long six-month XLYA futures contract at a price $f_{0}(T)$ of $€ 85.42$ and posted initial margin of $€ 1,000$. Three months after the forward and futures contracts are initiated, XYLA announces a strategic partnership with a major global technology firm, and its spot share price jumps $€ 15$ on the day's trading to close at $€ 123$.
\end{enumerate}

Which of the following statements best characterizes the impact of the day's trading on the MTM value of the forward versus the futures contract?

A. Ace's client realizes an MTM gain of approximately $€ 1,500(=€ 15 \times 100)$ on its margin account, which Ace must deposit at the end of the day to cover its margin call.

B. Ace's client benefits from an MTM unrealized gain on its forward contract with Ace, and Ace has a corresponding MTM gain of approximately $€ 1,500$ $(=€ 15 \times 100)$ deposited in its margin account by the exchange.

C. Because Ace has entered a hedge of its client's long forward position on XLYA by executing a futures contract with otherwise identical terms, the two contract MTM values exactly offset one another and no cash is exchanged on either transaction.

\begin{enumerate}
  \setcounter{enumi}{2}
  \item Identify which of the following corresponds to which description.

  \item Long interest rate futures position

  \item Pay fixed (receive floating) FRA contract

  \item Receive fixed (pay floating) FRA contract A. Results in a gain when MRR settles above the initial forward commitment rate at maturity

\end{enumerate}

B. Results in a loss when MRR settles above the initial forward commitment rate at maturity

C. Has a forward commitment price that will increase as short-term interest rates fall

\begin{enumerate}
  \setcounter{enumi}{3}
  \item Ace's investor clients usually use OTC forward transactions that Ace must clear with a central counterparty. Which of the following statements related to the impact on Ace from clearing these positions is most accurate?
\end{enumerate}

A. If Ace's counterparties enter long forward contracts whose prices are positively correlated with interest rates, Ace will have to post more collateral to central counterparties than for otherwise similar futures contracts, since rising prices will lead to counterparty MTM gains reinvested at higher rates.

B. If Ace's counterparties enter short forward contracts whose prices are negatively correlated with interest rates, Ace will have to post less collateral to central counterparties than for otherwise similar futures contracts, since falling prices will lead to counterparty MTM gains reinvested at higher rates.

C. Since Ace is required to post collateral (cash or highly liquid securities) to the central counterparty to clear its client forward transactions, Ace will face similar margining requirements to those of standardized exchange-traded futures markets.

\section{SOLUTIONS}
\begin{enumerate}
  \item A is correct. The CTA will face a margin call if the copper contract price falls by more than $\$ 4,000$, or $\$ 0.16(=\$ 4,000 / 25,000)$ per pound. We may solve for the price at which the CTA receives a margin call by first solving for the initial futures contract price, $f_{0}(T)$, at contract inception as follows:
\end{enumerate}

$f_{0}(T)=\left[S_{0}+\mathrm{PV}_{0}(C)\right](1+r)^{T}$.

Solve for $\mathrm{PV}_{0}(C)$ per pound as follows:

$\mathrm{PV}_{0}(C)=\$ 9.98\left(=\$ 10\left[1.01875^{-(1 / 12)}\right]\right)$.

Substitute $\mathrm{PV}_{0}(C)=\$ 9.98$ into Equation 4 to solve for $f_{0}(T)$ :

$f_{0}(T)=[(\$ 4.25 \times 25,000)+\$ 9.98](1.01875-(1 / 12))$

$f_{0}(T)=\$ 106,425$ per contract $(\approx \$ 4.257$ per pound $)$.

So, $\$ 106,425-\$ 4,000=\$ 102,425$ per contract, and $\$ 4.257-\$ 0.16=\$ 4.10$ per pound.

$B$ is incorrect as it assumes there is no maintenance margin, and while $\mathrm{C}$ may be true under some circumstances, the change in $A$ is immediate (occurs at trade inception).

\begin{enumerate}
  \setcounter{enumi}{1}
  \item B is correct. The long investor client forward position with Ace benefits from an MTM gain on its forward contract with Ace, but no cash is exchanged until maturity. Ace receives a deposit in its futures margin account equal to the daily MTM futures contract gain, which if spot and futures prices change by approximately the same amount will be equal to $€ 1,500(€ 15 \times 100)$.

  \item 
  \begin{enumerate}
    \item C is correct. The futures contract price changes daily based upon a (yield 100) quoting convention, so its price will increase as yields fall and vice versa. The fixed rate on an FRA does not change for the life of the contract.
  \end{enumerate}
  \item A is correct. An FRA fixed-rate payer (floating-rate receiver) will realize a gain on the contract upon settlement (equal to the present value of the difference between the fixed rate and MRR multiplied by the contract notional over the specified interest period) if MRR settles above the initial fixed rate on the contract.

  \item B is correct. If the MRR settles above the initial forward commitment rate at maturity, the FRA fixed-rate payer has an MTM loss on the contract.

  \item $\mathrm{C}$ is correct. Mandatory central clearing requirements impose margin requirements on financial intermediaries similar to those of standardized exchange-traded futures markets, who often pass these costs and/or requirements on to their clients. Answers A and B are incorrect, as the MTM gains on the forward contracts are not realized until maturity.

\end{enumerate}

\section{LEARNING MODULE}
\begin{center}
\includegraphics[max width=\textwidth]{2023_05_04_36535b8d80b32081d422g-309}
\end{center}

\section{Pricing and Valuation of Interest Rates and Other Swaps}
\section{LEARNING OUTCOME}
\begin{center}
\begin{tabular}{c|l}
Mastery & The candidate should be able to: \\
\hline
$\square$ & $\begin{array}{l}\text { describe how swap contracts are similar to but different from a series } \\ \text { of forward contracts } \\ \text { contrast the value and price of swaps }\end{array}$ \\
$\square$ &  \\
\end{tabular}
\end{center}

\section{INTRODUCTION}
Swap contracts were introduced earlier as a firm commitment to exchange a series of cash flows in the future, with interest rate swaps where fixed cash flows are exchanged for floating payments being the most common type. Subsequent lessons addressed the pricing and valuation of forward and futures contracts across the term structure, which form the building blocks for swap contracts.

In this lesson, we will explore how swap contracts are related to these other forward commitment types. While financial intermediaries often use forward rate agreements or short-term interest rate futures contracts to manage interest rate exposure, issuers and investors usually prefer swap contracts, since they better match rate-sensitive assets and liabilities with periodic cash flows, such as fixed-coupon bonds, variable-rate loans, or known future commitments. It is important for these market participants not only to be able to match expected future cash flows using swaps but also to ensure that their change in value is consistent with existing or desired underlying exposures. The following lessons compare swap contracts with forward contracts and contrast the value and price of swaps.

\section{Summary}
\begin{itemize}
  \item A swap contract is an agreement between two counterparties to exchange a series of future cash flows, whereas a forward contract is a single exchange of value at a later date. - Interest rate swaps are similar to forwards in that both contracts are firm commitments with symmetric payoff profiles and no cash is exchanged at inception, but they differ in that the fixed swap rate is constant, whereas a series of forward contracts has different forward rates at each maturity.

  \item A swap is priced by solving for the par swap rate, a fixed rate that sets the present value of all future expected floating cash flows equal to the present value of all future fixed cash flows.

  \item The value of a swap at inception is zero (ignoring transaction and counterparty credit costs). On any settlement date, the value of a swap equals the current settlement value plus the present value of all remaining future swap settlements.

  \item A swap contract's value changes as time passes and interest rates change. For example, a rise in expected forward rates increases the present value of floating payments, causing a mark-to-market (MTM) gain for the fixed-rate payer (floating-rate receiver) and an MTM loss for the fixed-rate receiver (floating-rate payer).

\end{itemize}

\section{LEARNING MODULE PRE-TEST}
\begin{enumerate}
  \item Identify which of the following characteristics matches which forward commitment contract.

  \item Involves periodic settlements based on the difference between a fixed rate established for each period and market reference rate (MRR)

  \item Has a symmetric payoff profile and a value of zero to both counterparties at inception

  \item Involves periodic settlements based on the difference between a constant

\end{enumerate}

fixed rate and the MRR

\section{Solution:}
\begin{enumerate}
  \item B is correct. A series of FRAs involves periodic settlements based on the difference between a fixed rate established for each period and the MRR.

  \item A is correct. Both an interest rate swap and a series of forward rate agreements have a symmetric payoff profile and a value of zero to both counterparties at inception.

  \item $C$ is correct. An interest rate swap involves periodic settlements based on the difference between a constant fixed rate and the MRR.

  \item Determine the correct answers to fill in the blanks: An FRA has a single settlement that occurs at the of an interest period, while a standard swap has periodic settlements that occur at the of each respective period.

\end{enumerate}

\section{Solution:}
An FRA has a single settlement that occurs at the beginning of an interest period, while a standard swap has periodic settlements that occur at the end of each respective period. 3. Determine the correct answers to fill in the blanks: A fixed-income portfolio manager seeking to gain from falling interest rates may consider entering a -fixed, -floating interest rate swap rather than purchas-

ing bonds.

Solution:

A fixed-income portfolio manager seeking to gain from falling interest rates may consider entering a receive-fixed, pay-floating interest rate swap rather than purchasing bonds.

\begin{enumerate}
  \setcounter{enumi}{3}
  \item Determine the correct answers to fill in the blanks: The value of a swap on any settlement date equals the value of all remaining settlement value plus the present
\end{enumerate}

\section{Solution:}
 swap settlements.The value of a swap on any settlement date equals the current settlement value plus the present value of all remaining future swap settlements.

\begin{enumerate}
  \setcounter{enumi}{4}
  \item An investor enters into a 10-year pay-fixed EUR100 million swap at a rate of $1.12 \%$ versus six-month EUR MRR. Assume six-month EUR MRR sets today at $0.25 \%$, calculate the periodic settlement value of the swap from the investor's perspective in six months' time, and interpret your answer.
\end{enumerate}

\section{Solution:}
From the investor's (fixed-rate payer's) perspective, the periodic settlement value of the swap is equal to

$$
\begin{aligned}
& \text { Periodic settlement value }=\left(\text { MRR }-s_{N}\right) \times \text { Notional amount } \times \text { Period } \\
& =- \text { EUR435,000 }=(0.25 \%-1.12 \%) \times \text { EUR } 100 \text { million } \times 0.5 \text { years. }
\end{aligned}
$$

Since EUR MRR has set below the fixed swap rate, the fixed-rate payer must make a net payment to the fixed-rate receiver at the end of the interest period.

\begin{enumerate}
  \setcounter{enumi}{5}
  \item Describe the relationship between future floating-rate and fixed-rate payments on a swap where the fixed-rate receiver has a positive MTM value on the position on a settlement date.
\end{enumerate}

\section{Solution:}
The value of a swap on a settlement date after inception of the contract equals the current settlement value plus the present value of all remaining future net swap settlements, or

$\Sigma$ PV(Fixed payments received) $-\Sigma$ PV(Floating payments paid)

from the perspective of the fixed-rate receiver. In order for the fixed-rate receiver to have a positive MTM value on the swap, the present value of fixed-rate payments received would have to exceed the present value of expected floating-rate payments made, or

$\Sigma \mathrm{PV}$ (Fixed payments received) $>\Sigma \mathrm{PV}$ (Floating payments paid).

\section{SWAPS VS. FORWARDS}
describe how swap contracts are similar to but different from a series of forward contracts

A swap contract is an agreement between two parties to exchange a series of future cash flows, while a forward contract is an agreement for a single exchange of value at a later date. Although this lesson focuses on interest rate swaps, similar principles apply to other underlying variables where a series of cash flows are exchanged on a future date.

An earlier lesson showed how implied forward rates may be derived from spot rates. An implied forward rate for a given period in the future is equivalent to the forward rate agreement (FRA) fixed rate for that same period for which no riskless profit opportunities exist. The single cash flow of an FRA is similar to a single-period swap, as shown in Exhibit 1.

\section{Exhibit 1: Swap and FRA Payoff Profile}
\begin{center}
\includegraphics[max width=\textwidth]{2023_05_04_36535b8d80b32081d422g-312}
\end{center}

In each case, the net difference between a fixed rate agreed on at inception and an MRR set in the future is used as the basis for determining cash settlement on a given notional principal over a specific time period. For example, a fixed-rate payer on a swap or FRA will realize a gain if the MRR sets at a rate higher than the agreed-on fixed rate and will receive a net payment from the floating-rate payer. However, as we saw in an earlier lesson, the FRA has a single settlement, which occurs at the beginning of an interest period, while a standard swap has periodic settlements, which occur at the end of each respective period.

Other similarities between interest rate forwards and swaps include the symmetric payoff profile and the fact that no cash flow is exchanged upfront. Both interest rate forward and swap contracts involve counterparty credit exposure.

Since interest rates are characterized by a term structure, different FRA fixed rates usually exist for different times to maturity. In contrast, a standard interest rate swap has a constant fixed rate over its life, which includes multiple periods. This relationship is shown visually in Exhibit 2 and numerically in Example 1, which extends an earlier spot and forward rate example. Exhibit 2: Series of FRAs vs. Standard Interest Rate Swap

Series of Forward Rate Agreements (at Different Fixed Rates)

\begin{center}
\includegraphics[max width=\textwidth]{2023_05_04_36535b8d80b32081d422g-313}
\end{center}

Standard Interest Rate Swap (At A Constant Fixed Rate)

\begin{center}
\includegraphics[max width=\textwidth]{2023_05_04_36535b8d80b32081d422g-313(1)}
\end{center}

\section{EXAMPLE 1}
\section{Swaps as a Combination of Forwards}
Recall from an earlier lesson that three recently issued annual fixed-coupon government bonds had the following coupons, prices, yields-to-maturity, and zero (or spot) rates:

\begin{center}
\begin{tabular}{lcccc}
\hline
$\begin{array}{l}\text { Years to } \\ \text { Maturity }\end{array}$ & Annual Coupon & PV (per 100 FV) & YTM & Zero Rates \\
\hline
1 & $1.50 \%$ & 99.125 & $2.3960 \%$ & $2.3960 \%$ \\
2 & $2.50 \%$ & 98.275 & $3.4068 \%$ & $3.4197 \%$ \\
3 & $3.25 \%$ & 98.000 & $3.9703 \%$ & $4.0005 \%$ \\
\hline
\end{tabular}
\end{center}

We solve for the implied forward rate ( $\operatorname{IFR}_{A, B-A}$ ), the break-even reinvestment rate for a period starting in the future $($ at $t=A)$ between short-dated $\left(z_{A}\right)$ and longer-dated $\left(z_{B}\right)$ zero rate using the following formula:

$$
\left(1+z_{A}\right)^{A} \times\left(1+\operatorname{IFR}_{A, B-A}\right)^{B-A}=\left(1+z_{B}\right)^{B} .
$$

The respective spot rate at time $t=0\left(\operatorname{IFR}_{0,1}\right)$ and the implied forward rates in one year $\left(\operatorname{IFR}_{1,1}\right)$ and in two years (IFR $\left.{ }_{2,1}\right)$ are as follows:

$$
\begin{aligned}
& \operatorname{IFR}_{0,1}=2.3960 \%=(1.023960)-1 \\
& \mathrm{IFR}_{1,1}=4.4536 \%=(1.034197)^{2} /(1.02396)-1 . \\
& \text { IFR }_{2,1}=5.1719 \%=(1.040005)^{3} /(1.034197)^{2}-1 .
\end{aligned}
$$

One way to create a forward commitment for multiple periods with an initial value of zero would be to use a series of forward rate agreements, exchanging cash flows based on the respective implied forward rate (i.e., the FRA fixed rate) for each period. However, the fixed rate would vary for each period based on the term structure of interest rates. Instead, a standard interest rate swap is characterized by a constant fixed rate over multiple periods. These rates are shown for both alternatives below.

\begin{center}
\includegraphics[max width=\textwidth]{2023_05_04_36535b8d80b32081d422g-314}
\end{center}

The method used to solve for a swap rate was introduced in an earlier fixed-income lesson. The par rate (PMT) was shown to be the fixed rate at which a fixed-coupon bond has a price equal to par (or 100) using a sequence of zero rates $\left(z_{i}\right.$ for period $i$ ) as market discount factors, as follows:

$$
100=\frac{P M T}{\left(1+z_{1}\right)^{1}}+\frac{P M T}{\left(1+z_{2}\right)^{2}}+\cdots+\frac{P M T+100}{\left(1+z_{N}\right)^{N}} .
$$

A standard interest rate swap represents an exchange of fixed payments (with no final principal payment) at a constant rate for a series of floating-rate cash flows expected to equal the respective implied forward rates at time $t=0$. We must therefore modify the par bond rate calculation in Equation 2 to solve for PMT as a par swap rate.

\section{Exhibit 3: Par Swap Rate, Spot, and Forward Curve}
\begin{center}
\includegraphics[max width=\textwidth]{2023_05_04_36535b8d80b32081d422g-315(1)}
\end{center}

The par swap rate is the fixed rate that equates the present value of all future expected floating cash flows to the present value of fixed cash flows.

$\Sigma \mathrm{PV}($ Floating payments $)=\Sigma \mathrm{PV}($ Fixed payments), or

$$
\sum_{i=1}^{N} \frac{\mathrm{IFR}}{\left(1+z_{i}\right)^{i}}=\sum_{i=1}^{N} \frac{s_{i}}{\left(1+z_{i}\right)^{i}} .
$$

In our three-period example, we use the implied forward rates, IFR, to solve for $s_{3}$, or the three-year swap rate:

$$
\frac{\text { IFR }_{0,1}}{\left(1+z_{1}\right)}+\frac{\text { IFR }_{1,1}}{\left(1+z_{2}\right)^{2}}+\frac{\text { IFR }_{2,1}}{\left(1+z_{3}\right)^{3}}=\frac{s_{3}}{\left(1+z_{1}\right)}+\frac{s_{3}}{\left(1+z_{2}\right)^{2}}+\frac{s_{3}}{\left(1+z_{3}\right)^{3}} .
$$

Substitute each of these building blocks into Equation 3 to solve for $s_{3}$ :

\begin{center}
\includegraphics[max width=\textwidth]{2023_05_04_36535b8d80b32081d422g-315}
\end{center}

$0.111017=2.800545 \times s_{3}$.

$s_{3}=3.9641 \%$.

The three-year swap rate of $3.9641 \%$ may be interpreted as a multiperiod breakeven rate at which an investor would be indifferent to

\begin{itemize}
  \item paying the fixed swap rate and receiving the respective forward rates or

  \item receiving the fixed swap rate and paying the respective forward rates.

\end{itemize}

For this reason, we may think of the fixed swap rate as an internal rate of return on the implied forward rates through the maturity of the swap as of $t=0$. Exhibit 3 also demonstrates that while the combined value of the equivalent forward contracts is zero at time $t=0$, some individual forward exchanges may have a positive present value and some will have a negative present value. For example, a fixed-rate receiver (floating-rate payer) on the swap in Example 1 would expect the following cash flow in one year's time:

\begin{itemize}
  \item Receive the fixed swap rate $\left(s_{3}\right)$ of $3.9641 \%$.

  \item Pay the initial floating rate (IFR $\left._{0,1}\right)$ of $2.396 \%$. - Receive a net payment of $1.5681 \%$ (= 3.9641\% - $2.396 \%)$ on the notional for the period.

\end{itemize}

Derivative end users, such as issuers and investors, tend to use swaps more often than individual interest rate forward contracts for a number of reasons. For example, the ability to match cash flows of underlying assets or liabilities allows issuers to transform their exposure profile as in Example 2.

\section{EXAMPLE 2}
\section{Esterr Inc. Swap to Fixed}
Recall from an earlier lesson that Esterr Inc. has a CAD250 million floating-rate term loan at three-month Canadian MRR plus 150 basis points with three and a half years to maturity paid quarterly. Esterr enters into a CAD250 million interest rate swap contract to pay a fixed quarterly swap rate of $2.05 \%$ and receive a three-month floating MRR on a notional principal of CAD250 million based on payment dates that match the term loan. The combined loan and swap exposure may be shown as follows:

\begin{center}
\includegraphics[max width=\textwidth]{2023_05_04_36535b8d80b32081d422g-316}
\end{center}

Under the swap arrangement, Esterr can fix its interest expense for the term loan and avoids the administrative burden of multiple forward contracts at different forward rates. Consider the following cash flow scenarios on the interest rate swap and term loan (assuming each interest period is 0.25 year):

\section{Scenario 1:}
CAD MRR sets at 3.75\%. As the fixed-rate payer, Esterr

\begin{itemize}
  \item pays the fixed swap rate of $2.05 \%$ and receives CAD MRR of $3.75 \%$,

  \item receives a net swap payment of $1.70 \%$ for the quarter, and

  \item makes a term loan payment of $5.25 \%(=3.75 \%+1.50 \%)$.

\end{itemize}

Esterr's interest expense is $3.55 \%$ (= $5.25 \%$ paid $-1.70 \%$ received).

\section{Scenario 2:}
CAD MRR sets at $1.25 \%$. As the fixed-rate payer, Esterr

\begin{itemize}
  \item pays the fixed swap rate of $2.05 \%$ and receives CAD MRR of $1.25 \%$,

  \item pays a net swap payment of $0.80 \%$ for the quarter, and - makes a term loan payment of $2.75 \%(=1.25 \%+1.50 \%)$.

\end{itemize}

Esterr's interest expense is $3.55 \%$ (= $2.75 \%$ paid $+0.80 \%$ paid).

Regardless of where CAD MRR sets each period, Esterr's interest expense for each quarterly interest period is approximately CAD2,218,750 $(=3.55 \%$, [equal to the $2.05 \%$ swap rate $+1.50 \%$ loan spread] $\times$ CAD250 million $\times 0.25$ year).

As a fixed-income instrument with periodic fixed cash flows through maturity, an interest rate swap should be expected to have risk and return features similar to those of a fixed-coupon bond of a similar maturity. This feature makes interest rate swaps a more attractive alternative to forward rate agreements for investors in managing fixed-income exposures as well. Consider the following comparison of a cash bond position to paying or receiving a fixed swap rate:

\section{Exhibit 4: Using Swaps to Manage Fixed-Income Exposure}
\begin{center}
\includegraphics[max width=\textwidth]{2023_05_04_36535b8d80b32081d422g-317}
\end{center}

Floating-Rate

Payer (Long Bond Position)

\begin{center}
\begin{tabular}{lccc}
\hline
 &  &  &  \\
Instrument & Position & $\begin{array}{c}\text { Higher interest } \\ \text { rates }\end{array}$ & Lower interest rates \\
\hline
Cash bond & Long fixed bond & Loss &  \\
Interest rate swap & Receive fixed &  & Gain \\
Cash bond & Pay floating & Loss & Loss \\
 & Short fixed bond & Gain & Loss \\
Interest rate swap & Long floating-rate note &  &  \\
 & Pay fixed & Gain &  \\
\hline
\end{tabular}
\end{center}

While the next lesson will explore the value and price of interest rate swaps in greater detail, Exhibit 4 demonstrates the similarity between a long (short) bond position and a receive (pay) fixed interest rate swap. Given their greater liquidity than and similar benchmark exposure profile to individual bond positions, active fixed-income portfolio managers often use swaps rather than underlying securities to adjust their interest rate exposure. For example, if interest rates are expected to fall, a portfolio manager may choose to receive fixed on a swap rather than purchase underlying bonds to realize a gain in a lower rate environment.

As mentioned earlier, FRAs are primarily used by financial intermediaries to manage their rate-sensitive positions on a period-by-period basis. Issuers and investors typically opt for the greater efficiency of interest rate swaps to manage their interest rate exposures. As we will see in later lessons, the greater liquidity of interest rate swaps has also led to their use both as a bond pricing benchmark and an underlying variable for other derivative instruments.

\section{KNOWLEDGE CHECK}
\section{Interest Rate Swaps vs. Forward Contracts}
\begin{enumerate}
  \item Determine the correct answers to fill in the blanks: A fixed-rate payer on a swap or FRA will realize a(n) if the MRR sets at a rate higher than the agreed-on fixed rate and will a net payment the floating-rate payer.
\end{enumerate}

\section{Solution:}
A fixed-rate payer on a swap or FRA will realize a gain if the MRR sets at a rate higher than the agreed-on fixed rate and will receive a net payment from the floating-rate payer.

\begin{enumerate}
  \setcounter{enumi}{1}
  \item Identify which of the following characteristics matches which forward commitment contract.

  \item The price of this contract is the implied forward rate, or the breakeven reinvestment rate, for a period starting in the future.

  \item Involves counterparty credit risk

  \item Has a constant fixed rate for which the present value of future fixed versus floating cash flow exchanges is equal to zero A. Interest rate swap

\end{enumerate}

B. Forward rate agreement

C. Both an interest rate swap and an interest rate forward contract

\section{Solution:}
\begin{enumerate}
  \item B is correct. The breakeven reinvestment rate, or implied forward rate, is the no-arbitrage FRA fixed rate.

  \item $C$ is correct. Both an interest rate swap and a forward rate agreement involve counterparty credit risk.

  \item A is correct. An interest rate swap has a constant fixed rate for which the present value of fixed versus floating cash flow exchanges equals zero.

  \item Identify which of the following benefits of using swaps over forwards are most applicable to which derivative end users.

  \item Swaps allow these end users to match the periodic cash flows of a specific balance sheet liability to transform their interest rate exposure profile.

  \item Swaps enable these end users to actively adjust their interest rate exposure profile without buying or selling underlying securities A. Both issuers and investors

\end{enumerate}

B. Issuers 3. Swaps involving a series of cash

C. Investors

flows enable these end users to avoid

the administrative burden of entering

into and managing multiple forward

contracts.

\section{Solution:}
\begin{enumerate}
  \item B is correct. Swaps allow issuers to match the periodic cash flows of a specific balance sheet liability to transform their interest rate exposure profile, as shown in Example 2.

  \item $C$ is correct. Swaps enable investors to actively adjust their interest rate exposure profile without buying or selling underlying securities.

  \item A is correct. Both issuers and investors benefit from a reduced administrative burden of entering one interest rate swap for a series of cash flows rather than multiple individual forward contracts.

  \item Identify the following statement as true or false, and justify your response: The market reference rate (MRR) is the internal rate of return on the implied forward rates over the life of an interest rate swap.

\end{enumerate}

\section{Solution:}
False. The fixed swap rate is the internal rate of return on the implied forward rates over the life of an interest rate swap.

\begin{enumerate}
  \setcounter{enumi}{4}
  \item Explain how an active fixed-income portfolio manager might use an interest rate swap rather than underlying bonds to realize a gain in a lower-interest rate environment, and justify your response.
\end{enumerate}

\section{Solution:}
A manager may choose to receive fixed, pay floating on an interest rate swap, with the fixed cash flow stream being similar to owning a fixed-coupon bond. If interest rates are expected to fall, the manager will realize an MTM gain in a lower-rate environment.

\section{SWAP VALUES AND PRICES}
contrast the value and price of swaps

In the prior lesson, we showed a swap price (or par swap rate) to be a periodic fixed rate that equates the present value of all future expected floating cash flows to the present value of fixed cash flows. The swap rate $\left(s_{N}\right.$ for $N$ periods) is equivalent to a forward rate, $F_{0}(T)$, that satisfies the no-arbitrage condition. Similar to other forwards, the initial contract value (ignoring transaction and counterparty credit costs) is zero $\left(V_{0}(T)=0\right)$.

In contrast to other forward commitments, which involve a single settlement at maturity, a swap contract involves a series of periodic settlements with a final settlement at contract maturity. Recall from an earlier lesson that the value of a forward contract at maturity from the long forward (or forward buyer's) perspective is $V_{T}(T)$ $=S_{T}-F_{0}(T)$, where $S_{T}$ is the spot price at maturity and $F_{0}(T)$ is the forward price. For a swap with periodic exchanges, the current MRR is the "spot" price and the fixed swap rate, $s_{N}$, is the forward price. Restating this result for the fixed-rate payer on a swap, the periodic settlement value is

Periodic settlement value $=\left(\mathrm{MRR}-s_{N}\right) \times$ Notional amount $\times$ Period.

The value of a swap on any settlement date equals the current settlement value in Equation 4 plus the present value of all remaining future swap settlements. Although swap market conventions vary, for purposes of this lesson we will assume the MRR sets at the beginning of each interest period, with the same periodicity and day count convention as the swap rate. The fixed versus floating difference is exchanged at the end of each period.

\section{EXAMPLE 3}
\section{Esterr Inc. Swap Value and Price}
Esterr entered a 3.5-year CAD250 million interest rate swap contract under which it pays a fixed quarterly swap rate of $2.05 \%$ and receives three-month CAD MRR. The fixed swap price of $2.05 \%$ paid by Esterr remains constant over the life of the contract. While the first three-month CAD MRR is known at $t=$ 0 , the remaining 13 MRR settings are unknown but are expected to equal the respective implied forward rates (IFRs) for each period through maturity. A prior lesson showed how IFRs are derived from zero rates, which were then used to solve for the fixed swap rate by setting the present value of fixed payments equal to the present value of floating payments.

The value of Esterr's swap contract will change as time passes and interest rates change. We first consider the passage of time with no change to expected interest rates. That is, the MRR sets each period based on the implied forward rates at trade inception. Assume the following implied forward rates apply to Esterr's swap at $t=0$ :

\begin{center}
\begin{tabular}{lcl}
\hline
Period & IFR & Rate \\
\hline
1 & IFR $_{0,3 m}$ & $0.50 \%$ \\
2 & IFR $_{3 m, 3 m}$ & $0.74 \%$ \\
3 & IFR $_{6 m, 3 m}$ & $0.98 \%$ \\
4 & IFR $_{9 m, 3 m}$ & $1.22 \%$ \\
5 & IFR $_{12 m, 3 m}$ & $1.46 \%$ \\
6 & IFR $_{15 m, 3 m}$ & $1.70 \%$ \\
7 & IFR $_{18 m, 3 m}$ & $1.94 \%$ \\
8 & IFR $_{21 m, 3 m}$ & $2.18 \%$ \\
9 & IFR $_{24 m, 3 m}$ & $2.43 \%$ \\
10 & IFR $_{27 m, 3 m}$ & $2.67 \%$ \\
11 & IFR $_{30 m, 3 m}$ & $2.91 \%$ \\
12 & IFR $_{33 m, 3 m}$ & $3.15 \%$ \\
13 & IFR $_{36 m, 3 m}$ & $3.39 \%$ \\
14 & IFR $_{39 m, 3 m}$ & $3.78 \%$ \\
\hline
\end{tabular}
\end{center}

Based on these forward rates, Esterr expects to make a net swap payment each quarter through the seventh period-since $\left(\mathrm{MRR}-s_{N}\right)=(1.94 \%-2.05 \%)$ $<0$-and receive a net quarterly swap payment starting in the eighth period, where $\left(\mathrm{MRR}-s_{N}\right)=(2.18 \%-2.05 \%)>0$. Using Equation 1 , consider the periodic settlement values for the first two periods from Esterr's perspective as the fixed-rate payer:

\begin{itemize}
  \item Period 1: - CAD968,750 $=(0.5 \%-2.05 \%) \times$ CAD250 million $\times 0.25$

  \item Period 2: - CAD818,750 $=(0.74 \%-2.05 \%) \times$ CAD250 million $\times 0.25$.

\end{itemize}

What is the swap MTM value from Esterr's perspective immediately after the second periodic settlement if forward rates remain unchanged? Note that the swap price (the fixed swap rate of $2.05 \%$ ) was set at inception to equate the present value of fixed versus floating payments.

\begin{center}
\includegraphics[max width=\textwidth]{2023_05_04_36535b8d80b32081d422g-321(1)}
\end{center}

As we move forward in time with no change to interest rate expectations, the present value of remaining floating payments rises above the present value of fixed payments at $2.05 \%$, as Esterr has made a net settlement payment in the first two periods.

\begin{center}
\includegraphics[max width=\textwidth]{2023_05_04_36535b8d80b32081d422g-321}
\end{center}

As the diagram shows, as time progresses, Esterr realizes an MTM gain on the swap, since

$\Sigma$ PV(Floating payments received) $>\Sigma$ PV(Fixed payments paid).

Note that this result depends on the relative level of IFRs and shape of the forward curve, which in this case is upward sloping.

If we instead consider interest rate changes only, from Esterr's perspective as the fixed-rate payer (and floating-rate receiver), we can show the conditions under which Esterr has a swap MTM gain or loss:

\begin{itemize}
  \item Esterr realizes an MTM gain on the swap as the fixed-rate payer if $\Sigma \mathrm{PV}($ Floating payments received $)>\Sigma \mathrm{PV}($ Fixed payments paid $)$

  \item Esterr realizes an MTM loss on the swap as the fixed-rate payer if

\end{itemize}

$\Sigma \mathrm{PV}($ Floating payments received $)<\Sigma \mathrm{PV}($ Fixed payments paid).

A rise in the expected forward rates after inception will increase the present value of floating payments, while the fixed swap rate will remain the same. We show the effect of an immediate change in interest rates following trade inception in the following diagram:

\begin{center}
\includegraphics[max width=\textwidth]{2023_05_04_36535b8d80b32081d422g-322}
\end{center}

The new forward curve in this diagram is composed of higher IFRs. If we were to now solve for a fixed swap rate using these higher IFRs by setting floating and fixed cash flows equal to one another, the new swap rate would be above the original swap rate. However, since Esterr has locked in future fixed payments at the lower original swap rate while receiving higher expected future MRRs, the swap has a positive MTM value to Esterr.

Another interpretation of an interest rate swap is that the fixed-rate payer (floating-rate receiver) is long a floating-rate note (FRN) priced at the MRR and short a fixed-rate bond with a coupon equal to the fixed swap rate. Note that the combination of long and short bond positions causes both the purchase and sale prices of the two bonds at inception and the return of principal at maturity cancel one another out, and only the fixed versus variable coupon payments remain. The following example shows how the change in an interest rate swap's value is similar to that of a fixed-income security using an earlier investor swap example.

\section{EXAMPLE 4}
\section{Fyleton Investments}
Fyleton Investments entered a five-year, receive-fixed GBP200 million interest rate swap in an example from an earlier lesson to increase the duration of its fixed-income portfolio. Assume in this case that Fyleton receives a fixed swap rate of $2.38 \%$ on a semiannual basis versus six-month GBP MRR. Further assume that initial six-month GBP MRR sets at $0.71 \%$ and, as in the case of Esterr, the MRR forward curve is upward sloping. How will the value of Fyleton's swap change as time passes and interest rates change?

First, consider the passage of time with no rate changes. The first-period swap settlement value (in six months) from Fyleton's perspective as the fixed-rate receiver is

GBP $1,670,000=(2.38 \%-0.71 \%) \times \mathrm{GBP} 200$ million $\times 0.5$. What is the MTM value from Fyleton's perspective immediately following the first settlement if implied forward rates remain the same as at trade inception? As the fixed-rate receiver, Fyleton will realize an MTM loss on the swap, since

$\Sigma$ PV(Floating payments paid) $>\Sigma$ PV(Fixed payments received).

Second, consider a change in forward rates. A decline in expected forward rates immediately following trade inception will reduce the present value of floating payments, while the fixed swap rate will remain the same. Fyleton will realize an MTM gain as the fixed-rate receiver, since

$\Sigma \mathrm{PV}$ (Fixed payments received) $>\Sigma \mathrm{PV}$ (Floating payments paid).

This MTM gain is shown by the different size of the shaded areas under the original swap rate using the new forward curve in the following diagram.

\begin{center}
\includegraphics[max width=\textwidth]{2023_05_04_36535b8d80b32081d422g-323}
\end{center}

Note the similarity between the receive-fixed swap exposure profile and that of a long cash fixed-rate bond position. In the first instance, we would expect a fixed-rate par bond to be priced at a discount as time passes if rates rise as per an upward-sloping forward curve, while an FRN priced at the MRR would remain at par. In the second instance, a decline in implied forward rates (or downward shift in the forward curve) would cause a fixed-rate par bond to be priced at a premium, while the FRN price would not change. We will explore how term structure and yield curve changes affect swap values and bond prices in greater detail later in the curriculum.

\section{KNOWLEDGE CHECK}
\section{Swap Prices and Values}
\begin{enumerate}
  \item Determine the correct answers to fill in the blanks: A rise in the expected forward rates after inception will the present value of floating payments, causing a fixed-rate receiver to realize a(n) in MTM value on the swap contract.
\end{enumerate}

\section{Solution:}
A rise in the expected forward rates after inception will increase the present value of floating payments, causing a fixed-rate receiver to realize a decline in MTM value on the swap contract.

\begin{enumerate}
  \setcounter{enumi}{1}
  \item Identify the following statement as true or false, and justify your response: The fixed rate on an interest rate swap is priced such that the present value of the floating payments (based on respective implied forward rates for each period) is equal to the present value of the fixed payments at $t=0$.
\end{enumerate}

\section{Solution:}
This statement is true. The fixed rate on an interest rate swap may be solved by setting the present value of floating payments equal to the present value of fixed payments. We use zero rates to derive implied forward rates, or breakeven reinvestment rates, between a shorter and a longer zero rate. These IFRs represent the respective floating rates that are expected to apply for each future period at time $t=0$.

\begin{enumerate}
  \setcounter{enumi}{2}
  \item Identify which of the following statements is associated with which position in an interest rate swap contract.

  \item Establishes a set of certain net future cash flows on a swap contract at inception

  \item Realizes an MTM gain on a swap contract if the expected future floating-rate payments increase

  \item An investor may increase portfolio duration by entering this position in a fixed-rate receiver B. Fixed-rate receiver (Floating-rate payer)

\end{enumerate}

C. Neither a fixed-rate payer nor a

A. Fixed-rate payer (Floating-rate receiver)

\includegraphics[max width=\textwidth, center]{2023_05_04_36535b8d80b32081d422g-324}
swap contract

\section{Solution:}
\begin{enumerate}
  \item C is correct. A swap contract establishes a set of certain fixed future cash flows that are exchanged for a set of expected (uncertain) floating future cash flows. Therefore, neither a fixed-rate payer nor a fixed-rate receiver knows the net future cash flows of a swap at inception.

  \item A is correct. A fixed-rate payer realizes an MTM gain on a swap contract if the expected future floating-rate payments increase.

  \item B is correct. A fixed-rate receiver may increase portfolio duration by entering this position on a swap, because receiving fixed is similar to a long bond position.

  \item Determine the correct answers to fill in the blanks: The value of a swap on any settlement date equals the value of all remaining swap settlements.

\end{enumerate}

\section{Solution:}
The value of a swap on any settlement date equals the current settlement value plus the present value of all remaining future swap settlements.

\begin{enumerate}
  \setcounter{enumi}{4}
  \item Describe how an investor may use an interest rate swap to reduce the duration of a fixed-income portfolio, and justify your response.
\end{enumerate}

\section{Solution:}
A pay-fixed swap is similar to a short bond position that reduces duration, because the fixed-rate payer (floating-rate receiver) is effectively long a floating-rate note priced at the $M R R$ and short a fixed-rate bond with a coupon equal to the fixed swap rate.

\section{PRACTICE PROBLEMS}
Ace Limited is a financial intermediary that is active in forward and swap markets with its issuer and investor clients. You have been asked to consult on a number of client situations to determine the best course of action.

\begin{enumerate}
  \item Identify which of the following statements is associated with which Ace counterparty swap position.

  \item Ace's counterparty with this swap position will realize an MTM gain if implied forward rates rise.

  \item Ace's counterparty with this swap position will make a net payment if the initial market reference rate sets above the fixed swap rate.

  \item Ace's counterparty with this position will face an initial swap contract value (ignoring transaction and counterparty credit costs) of zero. A. A receive-fixed interest rate swap

\end{enumerate}

B. A pay-fixed interest rate swap

C. Both a receive-fixed and a pay-fixed interest rate swap

\begin{enumerate}
  \setcounter{enumi}{1}
  \item Ace's client is an asset manager with a significant portion of its fixed-rate bond investment portfolio maturing soon. Ace intends to reinvest the proceeds in five-year bond maturities. Which of the following describes the best course of action in the derivatives market for Ace's client to address its bond reinvestment risk?
\end{enumerate}

A. Ace's client should consider receiving fixed on a cash-settled five-year forward-starting swap that starts and settles in three months in order to best address its bond reinvestment risk.

B. Ace's client should consider paying fixed on a cash-settled five-year forward-starting swap starting in three months in order to best address its bond reinvestment risk.

C. Ace's client should consider entering a series of forward rate agreements (FRAs) from today until five years from now under which it pays a fixed rate and receives a floating rate each period ending in five years to address its bond reinvestment risk.

\begin{enumerate}
  \setcounter{enumi}{2}
  \item Ace enters a 10-year GBP interest rate swap with a client in which Ace receives an initial six-month GBP MRR of 1.75\% and pays a fixed GBP swap rate of 3.10\% for the first semiannual period. Which of the following statements best describes the value of the swap from Ace's perspective three months after the inception of the trade?
\end{enumerate}

A. Ace has an MTM loss on the swap, because it owes a net settlement payment to its counterparty equal to $1.35 \%$ multiplied by the notional and period.

B. Ace has an MTM gain on the swap, because once it makes the first known net payment to its counterparty, the remainder of the future net fixed versus floating cash flows must have a positive present value from Ace's perspective. C. While the present value of fixed and future cash flows was set to zero by solving for the swap rate at inception, we do not have enough information to determine whether the swap currently has a positive or negative value from Ace's perspective following inception.

\begin{enumerate}
  \setcounter{enumi}{3}
  \item At time $t=0$, Ace observes the following zero rates over three periods:
\end{enumerate}

\begin{center}
\begin{tabular}{ll}
\hline
Periods & Zero Rates \\
\hline
1 & $2.2727 \%$ \\
2 & $3.0323 \%$ \\
3 & $3.6355 \%$ \\
\hline
\end{tabular}
\end{center}

Which of the following best describes how Ace arrives at a three-period par swap rate $\left(s_{3}\right)$ ?

A. Since the par swap rate represents the fixed rate at which the present value of fixed and future cash flows equal one another, we discount each zero rate back to the present using zero rates and solve for $s_{3}$ to get $2.961 \%$.

B. Since the par swap rate represents the fixed rate at which the present value of fixed and future cash flows equal one another, we first solve for the implied forward rate per period using zero rates, then discount each implied forward rate back to the present using zero rates, and solve for $s_{3}$ to get $3.605 \%$.

C. Since the par swap rate represents the fixed rate at which the present value of fixed and future cash flows equal one another, we first solve for the implied forward rate per period using zero rates, then discount each zero rate back to the present using implied forward rates, and solve for $s_{3}$ to get $3.009 \%$.

\begin{enumerate}
  \setcounter{enumi}{4}
  \item Ace's issuer client has swapped its outstanding fixed-rate debt to floating to match asset portfolio cash flows that generate an MRR-based return. Which of the following statements best describes how Ace's MTM credit exposure to the issuer changes if interest rates rise immediately following trade inception?
\end{enumerate}

A. Since the client receives fixed and pays floating swap, it faces an MTM loss on the transaction as rates rise, increasing Ace's MTM exposure to the client.

B. Since the client receives fixed and pays floating swap, it faces an MTM gain on the transaction as rates rise, decreasing Ace's MTM exposure to the client.

C. Since the swap's value is equal to the current settlement plus future expected settlement amounts, we do not have enough information to determine whether Ace's MTM exposure to the client increases or decreases.

\section{SOLUTIONS}
\begin{enumerate}
  \item 
  \begin{enumerate}
    \item B is correct. A pay-fixed swap counterparty will realize an MTM gain if implied forward rates rise.
  \end{enumerate}
  \item A is correct. A receive-fixed swap counterparty will make a net payment if the initial market reference rate sets above the fixed swap rate.

  \item $C$ is correct. Both a receive-fixed and a pay-fixed swap counterparty will face an initial swap contract value (ignoring transaction and counterparty credit costs) of zero.

  \item A is correct. Ace's client should consider receiving fixed on a five-year swap. A receive-fixed swap has a risk and return profile similar to that of a long fixed-rate bond position. Ace's client would therefore expect to have a similar MTM gain or loss on the swap position as if it had purchased a five-year bond at inception.

  \item C is correct. At time $t=0$, the present value of fixed and future cash flows was set to zero by solving for the swap rate at inception. Although the current settlement value is known, we cannot determine whether the swap has a positive or negative value from Ace's perspective three months later without further informationspecifically, the current level of future forward rates.

  \item B is correct. Since the expected floating cash flows on the swap are the implied forward rates, we first use zero rates to solve for IFRs using Equation 1:

\end{enumerate}

$\left(1+z_{A}\right)^{A} \times\left(1+\operatorname{IFR}_{A, B-A}\right)^{B-A}=\left(1+z_{B}\right)^{B}$.

We may solve for these rates as $\operatorname{IFR}_{0,1}=2.2727 \%, \operatorname{IFR}_{1,1}=3.7975 \%$, and $\operatorname{IFR}_{2,1}=$ $4.8525 \%$. We then substitute the respective IFRs discounted by zero rates into the following equation to solve for $s_{3}$ :

$\frac{I F R_{0,1}}{\left(1+z_{1}\right)}+\frac{I F R_{1,1}}{\left(1+z_{2}\right)^{2}}+\frac{I F R_{2,1}}{\left(1+z_{3}\right)^{3}}=\frac{s_{3}}{\left(1+z_{1}\right)}+\frac{s_{3}}{\left(1+z_{2}\right)^{2}}+\frac{s_{3}}{\left(1+z_{3}\right)^{3}}$.

Solving for the left-hand side of the equation, we get

$0.10159=\frac{2.2727 \%}{1.022727}+\frac{3.7975 \%}{(1.030323)^{2}}+\frac{4.8525 \%}{(1.036355)^{3}}$

$=\frac{2.2727 \%}{1.022727}+\frac{3.7975 \%}{(1.030323)^{2}}+\frac{4.8525 \%}{(1.036355)^{3}}$

Solving for the right-hand side, we get

$2.81819 s_{3}=\left[\frac{1}{1.022727}+\frac{1}{(1.030323)^{2}}+\frac{1}{(1.036355)^{3}}\right] \times s_{3}$.

$s_{3}=3.605 \%=0.10159 \div 2.81819$.

A is incorrect, because it discounts zero rates, not IFRs, back to the present using zero rates, while $C$ incorrectly discounts zero rates by the respective IFRs.

\begin{enumerate}
  \setcounter{enumi}{4}
  \item A is correct. Since the client receives fixed and pays floating swap, in a rising-rate environment, $\Sigma \mathrm{PV}$ (Floating payments) $>\Sigma P$ PV(Fixed payments), and it will therefore owe more in future floating-rate settlements than it will receive in fixed-rate settlements, resulting in an MTM loss for the client and an increase in Ace's MTM exposure.
\end{enumerate}

\section{LEARNING MODULE 8}
\section{Pricing and Valuation of Options}
\section{LEARNING OUTCOME}
\begin{center}
\begin{tabular}{c|l}
Mastery & The candidate should be able to: \\
\hline
$\square$ & $\begin{array}{l}\text { Explain the exercise value, moneyness, and time value of an option } \\ \text { Contrast the use of arbitrage and replication concepts in pricing } \\ \text { forward commitments and contingent claims } \\ \text { Identify the factors that determine the value of an option and } \\ \text { describe how each factor affects the value of an option }\end{array}$ \\
$\square$ &  \\
\end{tabular}
\end{center}

\section{INTRODUCTION}
Option contracts are contingent claims in which one of the counterparties determines whether and when a trade will settle. Unlike a forward commitment with a value of zero to both counterparties at inception, an option buyer pays a premium to the seller for the right to transact the underlying in the future at a pre-agreed price. The contingent nature of options affects their price as well as their value over time.

In the first lesson, we explore three features unique to contingent claims related to an option's value versus the spot price of the underlying: the exercise, or intrinsic, value; the relationship between an option's spot price and its exercise price, referred to as "moneyness"; and the time value. We then turn to how the arbitrage and replication concepts introduced earlier for forward commitments differ when applied to contingent claims with an asymmetric payoff profile. Finally, we identify and describe factors that determine the value of an option. These lessons focus on European options, which can be exercised only at expiration.

\section{Summary}
\begin{itemize}
  \item An option's value comprises its exercise value and its time value. The exercise value is the option's value if it were immediately exercisable, while the time value captures the possibility that the passage of time and the variability of the underlying price will increase the profitability of exercise at maturity. - Option moneyness expresses the relationship between the underlying price and the exercise price. A put or call option is "at the money" when the underlying price equals the exercise price. An option is more likely to be exercised if it is "in the money" - with an underlying price above (for a call) or below (for a put) the exercise price-and less likely to be exercised if it is "out of the money."

  \item Due to their asymmetric payoff profile, options are characterized by no-arbitrage price bounds. The lower bound is a function of the present value of the exercise price and the underlying price, while the upper bound is the underlying price for a call and the exercise price for a put.

  \item As in the case of forward commitments, the replication of option contracts uses a combination of long (for a call) or short (for a put) positions in an underlying asset and borrowing or lending cash. The replicating transaction for an option is based on a proportion of the underlying, which is closely associated with the moneyness of the option.

  \item The underlying price, the exercise price, the time to maturity, the risk-free rate, the volatility of the underlying price, and any income or cost associated with owning the underlying asset are key factors in determining the value of an option.

  \item Changes in the volatility of the underlying price and the time to expiration will usually have the same directional effect on put and call option values. Changes to the exercise price, the risk-free rate, and any income or cost associated with owning the underlying asset have the opposite effect on call options versus put options.

\end{itemize}

\section{LEARNING MODULE PRE-TEST}
\begin{enumerate}
  \item Determine the correct answers to complete the following sentence: The lower bound of a call price is the underlying's price the present value of its exercise price or zero, whichever is
\end{enumerate}

\section{Solution:}
The lower bound of a call price is the underlying's price minus the present value of its exercise price or zero, whichever is greater.

\begin{enumerate}
  \setcounter{enumi}{1}
  \item Match the following statements about replication strategies with their associated derivative instrument(s):

  \item At time $t=0$, lend at the risk-free rate and sell the underlying at $S_{0}$.

  \item The replication strategy is executed at inception and is settled at maturity with no adjustment over time.

  \item At time $t=T$, sell the underlying at $S_{T}$ and repay the loan of $X$. A. Neither a call option nor a put option replication strategy

\end{enumerate}

B. A put option replication strategy

C. A call option replication strategy if exercised

\section{Solution:}
\begin{enumerate}
  \item B is correct. At time $t=0$, a put option replication strategy involves lending at the risk-free rate and selling the underlying at $S_{0}$. 2. A is correct. As both call and put options involve a non-linear payoff profile, their replication strategy requires adjustment over time as the likelihood of exercise changes.

  \item $\mathrm{C}$ is correct. A call option replication strategy if exercised involves repaying the loan of $X$ and selling the underlying at $S_{T}$ at time $t=T$.

  \item A European call option with three months remaining to maturity on an underlying stock with no additional cash flows has an exercise price $(X)$ of GBP 50, a risk-free rate of $2 \%$, and a current underlying price $\left(S_{t}\right)$ of GBP 57.50. If the current call option price is GBP 10, calculate the exercise value and the time value and interpret the results.

\end{enumerate}

\section{Solution:}
An option's value comprises its exercise value plus its time value. The exercise value of a call option is $\operatorname{Max}\left(0, S_{t}-\mathrm{PV}(X)\right)$ and is calculated as follows:

Call Option Exercise Value $=\operatorname{Max}\left(0, S_{t}-X(1+r)^{-(T-t)}\right)$

$\operatorname{Max}\left(0\right.$, GBP $\left.57.50-\operatorname{GBP} 50(1.02)^{-0.25}\right)$

$=$ GBP 7.75

The exercise value is positive, as the current underlying price exceeds the present value of the exercise price. The time value is the difference between the option price and the exercise value, representing the possibility that the option payoff at maturity will exceed the current exercise value due to a favorable price change:

Call Option Time Value $=c_{t}-\operatorname{Max}\left(0, S_{t}-X(1+r)^{-(T-t)}\right)$

$=$ GBP $2.25(=$ GBP $10-$ GBP 7.75)

The time value is always positive and declines to zero at maturity $(t=T)$.

\begin{enumerate}
  \setcounter{enumi}{3}
  \item Match the following underlying price and exercise price relationships with their associated put option:

  \item $S_{T}=100, X=100$

\end{enumerate}

A. An at-the-money put option

\begin{enumerate}
  \setcounter{enumi}{1}
  \item $S_{T}=110, X=100$
\end{enumerate}

B. An in-the-money put option

\begin{enumerate}
  \setcounter{enumi}{2}
  \item $S_{T}=90, X=100$
\end{enumerate}

C. An out-of-the-money put option

\section{Solution:}
Put options are in the money when $S_{T}<X$, at the money when $S_{T}=X$, and out of the money when $S_{T}>X$. Therefore:

\begin{enumerate}
  \item A is correct. Since $S_{T}=X=100$, this is an at-the-money put option.

  \item C is correct. Since $S_{T}>X$, this is an out-of-the-money put option.

  \item B is correct. Since $S_{T}<X$, this is an in-the-money put option.

  \item Match the following changes in a factor affecting option value (holding other factors constant) with their corresponding option value change:

  \item A higher exercise price $(X)$

  \item A higher underlying price $\left(S_{T}\right)$ A. Decreases the value of both a call option and a put option

\end{enumerate}

B. Decreases the value of a call option 3. A decline in the volatility of the under- C. Decreases the value of a put option lying price

\section{Solution:}
\begin{enumerate}
  \item B is correct. A higher exercise price decreases the value of a call option; for a given underlying price at maturity $\left(S_{T}\right)$, the call option settlement value of $\operatorname{Max}\left(0, S_{T}-X\right)$ will decrease for a higher $X$.

  \item $\mathrm{C}$ is correct. A higher underlying price $\left(S_{T}\right)$ will decrease the value of a put option. Since a put option is the right to sell an underlying, the put option settlement value of $\operatorname{Max}\left(0, X-S_{T}\right)$ will fall as $S_{T}$ rises.

  \item A is correct. A decline in the volatility of the underlying price will decrease the value of both a call option and a put option. Lower price variability of the underlying will reduce the probability of a higher positive exercise value for a call or a put option without affecting the downside case where the option expires unexercised.

  \item Explain the effect of a decrease in the income on an underlying asset on the value of put and call options and justify your answer.

\end{enumerate}

\section{Solution:}
Income or other, non-cash benefits (such as convenience yield) accrue to the owner of an underlying asset but not to the owner of a derivative, whose value is based on the underlying. This income effect holds for both forward commitments and contingent claims. A decline in income on an underlying asset increases the value of a call option and decreases the value of a put.

\section{OPTION VALUE RELATIVE TO THE UNDERLYING SPOT PRICE}
Explain the exercise value, moneyness, and time value of an option

As shown in earlier lessons, the non-linear or asymmetric payoff profile of an option causes us to approach these derivative instruments differently than for a forward commitment. When evaluating these derivatives, whose value depends critically on whether the spot price crosses an exercise threshold at maturity, buyers and sellers frequently rely on three measures-exercise value, moneyness, and time value-to gauge an option's value over the life of the contract. Recall from an earlier lesson that American options can be exercised at any time, while European options can be exercised only at maturity. This lesson focuses solely on European options with no additional cost or benefit of owning the underlying asset.

\section{OPTION EXERCISE VALUE}
An option buyer will exercise a call or put option at maturity only if it returns a positive payoff-that is:

\begin{itemize}
  \item $\left(S_{T}-X\right)>0$ for a call

  \item $\left(X-S_{T}\right)>0$ for a put

\end{itemize}

If not exercised, the option expires worthless and the option buyer's loss equals the premium paid.

At any time before maturity $(t<T)$, buyers and sellers often gauge an option's value by comparing the underlying spot price $\left(S_{t}\right)$ with the exercise price $(X)$ to determine the option's exercise value at time $t$. This is the option contract's value if the option were exercisable at time $t$. The exercise value for a call and a put option at time $\mathrm{t}$ incorporating the time value of money is the difference between the spot price $\left(S_{t}\right)$ and the present value of the exercise price $(\mathrm{PV}(X))$, as follows:

Call Option Exercise Value: $\operatorname{Max}\left(0, S_{t}-X(1+r)^{-(T-t)}\right)$

Put Option Exercise Value: $\operatorname{Max}\left(0, X(1+r)^{-(T-t)}-S_{t}\right)$

If we assume an exercise price, $X$, equal to the forward price, $F_{0}(T)$, the exercise value of a call option is the same as the value of a long forward commitment at time $t\left(V_{t}(T)\right)$, shown earlier to equal $S_{t}-\operatorname{PV}\left(F_{0}(T)\right)$, provided that the spot price is greater than the present value of $X$. That is, for a call option where $F_{0}(T)=X$ :

If $S_{t}>\mathrm{PV}(X): S_{t}-\operatorname{PV}\left(F_{0}(T)\right)=\operatorname{Max}\left(0, S_{t}-\mathrm{PV}(X)\right)$

Note that this comparison ignores the upfront call option premium paid by the option buyer $\left(c_{0}\right.$ at time $\left.t=0\right)$.

\section{EXAMPLE 1}
\section{Put Option Exercise Value}
Consider the earlier case of a one-year put option with an exercise price $(X)$ of EUR 1,000, an initial price of EUR 990 , and a risk-free rate of $1 \%$. What is the exercise value of the option in six months if the spot price $\left(S_{t}\right)$ equals EUR 950? Use Equation 2 to solve for $\operatorname{Max}\left(0, \mathrm{PV}(X)-S_{t}\right)$ :

$$
\begin{aligned}
& X(1+r)^{-(T-t)}-S_{t} \\
& =\text { EUR } 45.04\left(=\operatorname{EUR} 1,000(1.01)^{-0.5}-\right.\text { EUR 950) }
\end{aligned}
$$

\section{OPTION MONEYNESS}
An option's exercise value at any time $t$ was shown to be its current payoff. The relationship between the option's total value and its exercise price expresses the option's moneyness. Examples of in-the-money (ITM) options include a call option whose underlying spot price is above the exercise price $(X)$ and a put option with an underlying spot price below the exercise price. When the underlying price is equal to the exercise price, the put or call option is said to be at the money (ATM). When the underlying price is below (above) the exercise price for a call (put) option, the option is less likely to be exercised and is said to be out of the money (OTM).

Also, the degree to which an option is in or out of the money affects the sensitivity of an option's price to underlying price changes. For example, a so-called deep-in-the-money option, or one that is highly likely to be exercised, usually demonstrates a nearly one-to-one correspondence between option price and underlying price changes. A deep-out-of-the-money option, which is very unlikely to be exercised, demonstrates far less option price sensitivity for a given underlying price change. In contrast, relatively small price changes in the underlying for an at-the-money option often determine whether the option will be exercised. Moneyness is often used to compare options on the same underlying but with different exercise prices and/or times to maturity. Exhibit 1 shows the moneyness of a call option at maturity and summarizes these relationships.

\section{Exhibit 1: Call Option Moneyness at Maturity}
\begin{center}
\includegraphics[max width=\textwidth]{2023_05_04_36535b8d80b32081d422g-334}
\end{center}

\section{EXAMPLE 2}
\section{Put Option Moneyness}
Recall in Example 1 that at time $t$, a put option with six months remaining to maturity had an exercise price $(X)$ of EUR 1,000 and an underlying spot price $\left(S_{t}\right)$ of EUR 950 . Describe the moneyness of the put option at time $t$.

Given that the underlying spot price is below the exercise price, the put option is in the money by EUR 50 .

\section{OPTION TIME VALUE}
While the exercise value of an option reflects its current payoff, an additional component of an option's value is derived from its remaining time to maturity. Although European options can be exercised only at maturity, they can be purchased or sold prior to maturity at a price $\left(c_{t}\right.$ or $p_{t}$, for a call or put, respectively) that reflects the option's future expected payoff. A longer time until expiration usually means a higher potential dispersion of the future underlying price for a given level of volatility. Similarly, an increase in volatility at a specific underlying price for a given time to expiration increases option value for the same reason. We will explore these factors further in a later lesson.

The time value of an option is equal to the difference between the current option price and the option's current payoff (or exercise value):

Call Option Time Value: $\operatorname{Max}\left(0, S_{t}-X(1+r)^{-(T-t)}\right)$

or: $c_{t}=\operatorname{Max}\left(0, S_{t}-X(1+r)^{-(T-t)}\right)+$ Time Value

Put Option Time Value: $p_{t}-\operatorname{Max}\left(0, X(1+r)^{-(T-t)}-S_{t}\right)$

or: $p_{t}=\operatorname{Max}\left(0, X(1+r)^{-(T-t)}-S_{t}\right)+$ Time Value

That is, the current option price is equal to the sum of its exercise value and time value. As Exhibit 2 shows, the time value of an option is always positive but declines to zero at maturity, a process referred to as time value decay.

\section{Exhibit 2: Exercise Value and Time Value of a Call Option}
\begin{center}
\includegraphics[max width=\textwidth]{2023_05_04_36535b8d80b32081d422g-335}
\end{center}

$C_{t}=$ Exercise value + Time value t $_{t}$

\section{EXAMPLE 3}
\section{Put Option Time Value}
Example 1 showed that a one-year put option with an exercise price $(X)$ of EUR 1,000 had an exercise value of EUR 45.04 with six months remaining to maturity when the spot price $\left(S_{t}\right)$ was EUR 950. If we observe a current put option price $\left(p_{t}\right)$ of EUR 50, what is the time value of the put option?

Use Equation 4 to solve for $p_{t}-\operatorname{Max}\left(0, \operatorname{PV}(X)-S_{t}\right)$ :

$$
p_{t}-\max \left(0, X(1+r)^{-(T-t)}-S_{t}\right)
$$

$=$ EUR $4.96(=$ EUR $50-$ EUR 45.04)

\section{KNOWLEDGE CHECK}
\section{Option Value relative to Underlying Spot Price}
\begin{enumerate}
  \item Describe the similarities and differences between the exercise value of a long put option position with an exercise price $(X)$ equal to the forward price, $F_{0}(T)$, and a short forward position payoff at maturity with the same underlying details.
\end{enumerate}

\section{Solution:}
The exercise value of a put option with an exercise price of $X=F_{0}(T)$ at maturity $\left(\operatorname{Max}\left(0, X-S_{T}\right)\right)$ is the same as the payoff of a short forward commitment at maturity $\left(F_{0}(T)-S_{T}\right)$ as long as the exercise price is greater than the underlying price. That is, for $F_{0}(T)=X$ :

$$
\text { If } X>S_{T}: F_{0}(T)-S_{T}=\operatorname{Max}\left(0, X-S_{T}\right)
$$

These payoff profiles will differ if $S_{T}>X$, since the short forward payoff at maturity will require a payment from the seller to the buyer, while the put option owner will allow the option to expire unexercised. Note that this payoff comparison ignores the upfront option premium paid by the put option buyer $\left(p_{0}\right)$ at time $t=0$.

\begin{enumerate}
  \setcounter{enumi}{1}
  \item A European put option with three months remaining to maturity on an underlying stock with no additional cash flows has an exercise price $(X)$ of GBP 50, a risk-free rate of $2 \%$, and a current underlying price $\left(S_{t}\right)$ of GBP 55 . If the current put option price is GBP 5 , calculate the exercise value and the time value and interpret the results.
\end{enumerate}

\section{Solution:}
An option's value comprises its exercise value plus its time value. The exercise value of a put option is $\operatorname{Max}\left(0, \mathrm{PV}(X)-S_{t}\right)$ and can be calculated as follows:

Put Option Exercise Value $=\operatorname{Max}\left(0, X(1+r)^{-(T-t)}-S_{t}\right)$

$\operatorname{Max}\left(0\right.$, GBP $\left.50(1.02)^{-0.25}-\operatorname{GBP} 55\right)$ $=0$

The exercise value is zero, as the current underlying price exceeds the present value of the exercise price. The time value is the difference between the option price and the exercise value and represents the possibility that the likelihood and profitability of exercise at maturity may increase due to a favorable price change:

Put Option Time Value $=p_{t}-\operatorname{Max}\left(0, X(1+r)^{-(T-t)}-S_{t}\right)$

$=$ GBP $5(=$ GBP $5-$ GBP 0$)$

The time value is always positive and declines to zero at maturity $(t=T)$. Since the put option is out of the money, its value consists solely of time value.

\begin{enumerate}
  \setcounter{enumi}{2}
  \item Match the following underlying price and exercise price relationships with their associated call option:

  \item $S_{T}=60, X=50$

  \item $S_{T}=50, X=50$

  \item $S_{T}=40, X=50$

\end{enumerate}

\section{Solution:}
Call options are in the money when $S_{T}>X$, at the money when $S_{T}=X$, and out of the money when $S_{T}<X$. Therefore:

\begin{enumerate}
  \item B is correct. Since $S_{T}>X$, this is an in-the-money call option.

  \item A is correct. Since $S_{T}=X=50$, this is an at-the-money call option.

  \item C is correct. Since $S_{T}<X$, this is an out-of-the-money call option.

  \item Describe how the moneyness of an option affects how the option's value will change for a given change in the price of the underlying.

\end{enumerate}

\section{Solution:}
An increase in the moneyness of an option will increase the sensitivity of its value to changes in the underlying price. For example, an option that is very likely to be exercised will have a nearly one-to-one change in option value for a given change in the underlying price, while an option that is unlikely to be exercised will have a relatively small change in value for a given change in the underlying price.

\section{ARBITRAGE
Contrast the use of arbitrage and replication concepts in pricing forward commitments and contingent claims}
An earlier lesson showed that riskless arbitrage opportunities arise if the "law of one price" does not hold-that is, an identical asset trades at different prices in different places at the same time. In the case of a derivative, we establish no-arbitrage conditions based on the payoff profile at maturity. As shown earlier, forward commitments with a symmetric payoff profile settle based on the difference between the forward price, $F_{0}(T)$, at contract inception $(t=0)$ and the underlying price, $S_{T}$, at contract maturity $(t=T)$, or $\left(S_{T}-F_{0}(T)\right)$ for a long forward position. For an underlying with no additional cash flows, it was shown that assets with a known future price must have a spot price equal to the future price discounted at the risk-free rate, $r: S_{0}=S_{T}(1+r)^{-T}$. For purposes of this lesson, we ignore any costs or benefits of owning the underlying other than the opportunity cost (or the risk-free rate, $r$ ).

Contingent claims are characterized by asymmetric payoff profiles, introduced earlier. In the case of European options representing the right to purchase an underlying (or call option, $c$ ) or the right to sell an underlying (or put option, $p$ ) at a given exercise price $(X)$ at maturity, the payoff profiles at maturity for an option buyer were shown to be:

$$
\begin{aligned}
& c_{T}=\operatorname{Max}\left(0, S_{T}-X\right) \\
& p_{T}=\operatorname{Max}\left(0, X-S_{T}\right)
\end{aligned}
$$

\section{Exhibit 3: European Option Exercise at Expiration}
\begin{center}
\includegraphics[max width=\textwidth]{2023_05_04_36535b8d80b32081d422g-338}
\end{center}

Recall that, unlike forward commitments with an initial price of zero, the option buyer pays the seller a premium $\left(c_{0}\right.$ for a call and $p_{0}$ for a put at time $\left.t=0\right)$, so the option buyer's profit at maturity was introduced earlier as follows (ignoring the time value of money):

$$
\begin{aligned}
& \Pi=\operatorname{Max}\left(0, S_{T}-X\right)-c_{0} \\
& \Pi=\operatorname{Max}\left(0, X-S_{T}\right)-p_{0}
\end{aligned}
$$

Equations 7 and 8 show the key distinction between forward commitments and contingent claims for purposes of arbitrage. The forward buyer enters into the contract with no cash paid upfront and has an unlimited gain or loss (bounded by zero for an underlying such as a stock that cannot have a negative price) at maturity as the underlying price rises or falls. The option buyer, in contrast, will exercise an option at maturity only if it is in the money. This conditional nature of option payoff profiles leads us to establish upper and lower no-arbitrage price bounds at any time $t$.

A call option buyer will exercise only if the spot price $\left(S_{T}\right)$ exceeds the exercise price $(X)$ at maturity. The lower bound of a call price is therefore the underlying's price minus the present value of its exercise price or zero, whichever is greater. In other words, an option which trades below its exercise value violates the no-arbitrage condition. The call buyer will not pay more for the right to purchase an underlying than the price of that underlying, which is the upper bound:

$$
\begin{aligned}
& \operatorname{Max}\left(0, S_{t}-X(1+r)^{-(T-t)}\right)<c_{t} \leq S_{t} \\
& c_{t, \text { Lower bound }}=\operatorname{Max}\left(0, S_{t}-X(1+r)^{-(T-t)}\right) \\
& c_{t, \text { Upper bound }=S_{t}}
\end{aligned}
$$

A put option buyer will exercise only if the spot price, $S_{T}$, is below $X$ at maturity. The exercise price, $X$, therefore represents the upper bound on the put value. The lower bound is the present value of the exercise price minus the spot price or zero, whichever is greater:

$$
\begin{aligned}
& \operatorname{Max}\left(0, X(1+r)^{-(T-t)}-S_{t}\right)<p_{t} \leq X \\
& p_{t, \text { Lower bound }}=\operatorname{Max}\left(0, X(1+r)^{-(T-t)}-S_{t}\right) \\
& p_{t, \text { Upper bound }}=X
\end{aligned}
$$

\section{EXAMPLE 4}
\section{Call Option Upper and Lower Bounds}
Consider a one-year call option with an exercise price, $X$, of EUR 1,000. The underlying asset, $S_{0}$, trades at EUR 990 at time $t=0$ and the risk-free rate, $r$, is $1 \%$. What are the no-arbitrage upper and lower bounds in six months' time if the underlying asset price, $S_{t}$, equals EUR 1,050?

As the option buyer will exercise only if $S_{T}>X$ at $t=T$, the lower bound is equal to $S_{t}-\mathrm{PV}(X)$ or zero, whichever is greater:

$$
\begin{aligned}
c_{t, \text { Lower bound }} & =\operatorname{Max}\left(0, S_{t}-X(1+r)^{-(T-t)}\right) \\
\mathrm{c}_{\mathrm{t}, \text { Lower bound }} & =\operatorname{Max}\left(0, \text { EUR } 1,050-\operatorname{EUR} 1,000(1.01)^{-0.5}\right) \\
\mathrm{c}_{\mathrm{t}, \text { Lower bound }} & =\operatorname{Max}(0, \text { EUR 54.96) }
\end{aligned}
$$

The call buyer will not pay more than $S_{t}$ for the right to purchase the underlying:

$$
\begin{aligned}
& c_{t, \text { Upper bound }}=S_{t} \\
& \mathrm{c}_{\mathrm{t}, \text { Upper bound }}=\text { EUR } 1,050
\end{aligned}
$$

\section{REPLICATION}
In an earlier lesson, we used replication to create forward commitment cash flows, utilizing a combination of long or short positions in an underlying asset and borrowing or lending cash. The ability of market participants to use replication strategies ensures that the law of one price holds and no riskless arbitrage profit opportunities exist. Recall from an earlier lesson that a call option is similar to a long forward position in that it increases in value as the underlying price rises but differs in that the call option settles only if there is a gain upon exercise. This apparent symmetry for positive outcomes is shown in Exhibit 4, where the exercise price, $X$, is equal to the forward price, $F_{0}(T)$.

\section{Exhibit 4: Call Option versus Long Forward Position}
\begin{center}
\includegraphics[max width=\textwidth]{2023_05_04_36535b8d80b32081d422g-340}
\end{center}

In order to replicate the call option at contract inception $(t=0)$, we also must borrow at the risk-free rate, $r$, and use the proceeds to purchase the underlying at a price of $S_{0}$. At option expiration $(t=T)$, unlike in the case of the forward commitment, there are two possible replication outcomes:

\begin{itemize}
  \item Exercise $\left(S_{T}>X\right)$ : Sell the underlying for $S_{T}$ and use the proceeds to repay $X$.

  \item No exercise $\left(S_{T}<X\right)$ : No settlement is required.

\end{itemize}

If exercise were certain, we would borrow $X(1+r)^{-T}$ at inception, as in the case of the forward. However, since it is uncertain, we instead borrow a proportion of $X(1+$ $r)^{-T}$ based on the likelihood of exercise at time $T$, which is closely associated with the moneyness of an option. The non-linear payoff profile of an option requires that the replicating transaction be adjusted as this likelihood changes, while the replicating trades for a forward commitment remain constant. Option replication will be addressed in greater detail in later lessons.

We now turn our attention to put option replication. It was shown earlier that a short put position with an exercise price equal to the forward price $\left(X=F_{0}(T)\right)$ mirrors the outcomes for a long forward position when the underlying price at maturity is below the forward price $\left(S_{T}<X=F_{0}(T)\right)$. The sold put decreases in value as the underlying price falls, but it settles only if there is a gain upon exercise. This relationship is shown in Exhibit 5, where the exercise price $(X)$ is equal to the forward price, $F_{0}(T)$.

\section{Exhibit 5: Put Option versus Short Forward Position}
\begin{center}
\includegraphics[max width=\textwidth]{2023_05_04_36535b8d80b32081d422g-341}
\end{center}

In order to replicate the put option at contract inception $(t=0)$, we must sell the underlying short at a price of $S_{0}$ and lend the proceeds at the risk-free rate, $r$. At option expiration $(t=T)$, unlike in the case of the forward commitment, there are two possible replication outcomes:

\begin{itemize}
  \item Exercise $\left(S_{T}<X\right)$ : Purchase the underlying for $S_{T}$ from the proceeds of the risk-free loan.

  \item No exercise $\left(S_{T}>X\right)$ : No settlement is required.

\end{itemize}

If exercise were certain, we would lend $X(1+r)^{-T}$ at inception, as in the case of the forward. However, as in the case of the call option, the asymmetric payoff profile requires adjustment over time based on the likelihood of exercise.

In the next lesson, we will turn our attention to the factors that drive the likelihood of option exercise prior to maturity.

\section{KNOWLEDGE CHECK}
\section{Arbitrage and Replication}
\begin{enumerate}
  \item Determine the correct answers to complete the following sentence: The lower bound of a call price is the underlying's price the present value of its price or zero, whichever is greater.
\end{enumerate}

\section{Solution:}
The lower bound of a call price is the underlying's price minus the present value of its exercise price or zero, whichever is greater.

\begin{enumerate}
  \setcounter{enumi}{1}
  \item A six-month European call option on an underlying stock with no additional cash flows has an exercise price $(X)$ of GBP 50, an initial underlying price $\left(S_{0}\right)$ of GBP 49.75 , and a risk-free rate of $2 \%$. Calculate the lower bound of the call price in three months' time if $S_{t}=$ GBP 65 .
\end{enumerate}

\section{Solution:}
As the call option buyer will exercise only if $S_{T}>X$ at maturity, the lower bound is equal to $S_{t}-\mathrm{PV}(X)$ or zero in three months' time, whichever is greater:

$c_{t, \text { Lower bound }}=\operatorname{Max}\left(0, S_{t}-X(1+r)^{-(T-t)}\right)$

Given $S_{t}=$ GBP 65, $X=$ GBP 50, $r=2 \%$, and $(T-t)=0.25$, we can solve for the call option's lower bound as follows:

$c_{t, \text { Lower bound }}=\operatorname{Max}\left(0, \operatorname{GBP} 65-\operatorname{GBP} 50(1.02)^{-(0.25)}\right)$

$=\mathrm{GBP} 15.25$

\begin{enumerate}
  \setcounter{enumi}{2}
  \item Match the following statements about replication strategies with their associated derivative instrument(s):
\end{enumerate}

1 . At time $t=0$, borrow at the risk-free rate and purchase the underlying at $S_{0}$.

\begin{enumerate}
  \setcounter{enumi}{1}
  \item The strategy requires adjustment over time as the likelihood of exercise changes.
\end{enumerate}

3 . At time $t=T$, receive the loan repayment and purchase the underlying at $S_{T}$.

\begin{abstract}
A. Both a call option and a put option replication strategy
B. A call option replication strategy
C. A put option replication strategy if exercised
\end{abstract}

\section{Solution:}
\begin{enumerate}
  \item B is correct. At time $t=0$, a call option replication strategy involves borrowing at the risk-free rate and purchasing the underlying.

  \item A is correct. As both call and put options have a non-linear payoff profile, the replication strategy requires adjustment over time as the likelihood of exercise changes.

  \item C is correct. A put option replication strategy if exercised involves receiving the loan repayment and purchasing the underlying at $S_{T}$ at time $t=T$. 4. Explain the key difference between the changes in the replication of a contingent claim versus a forward commitment over the life of the contract.

\end{enumerate}

\section{Solution:}
An option has an asymmetric payoff profile, since the option buyer will exercise only if the exercise value of the option is positive. Since the likelihood of option exercise changes over time, transactions replicating an option contract must be adjusted over time. A forward commitment has a symmetric payoff profile that will settle with certainty at a future date, and therefore the replicating transactions do not require adjustment over the contract life.

\section{FACTORS AFFECTING OPTION VALUE}
$$
\begin{aligned}
& \text { Identify the factors that determine the value of an option and } \\
& \text { describe how each factor affects the value of an option }
\end{aligned}
$$

We now turn our attention to identifying and describing factors that affect an option's value. Several of these factors are common to both forward commitments and options, and a number are unique to contingent claims. Factors that determine the value of an option include the value of the underlying, the exercise price, the time to maturity, the risk-free rate, the volatility of the underlying price, and any income or cost associated with owning the underlying asset.

\section{Value of the Underlying}
Changes in the value of the underlying will have the same directional effect on the right to transact the underlying under an option contract as on the obligation to transact under a forward commitment. For example, a call option and a long forward position will both appreciate if the spot price of the underlying rises, while a put option and a short forward position will both appreciate if the spot price of the underlying falls, as shown in Exhibit 6.

\section{Exhibit 6: Call and Put Option Value versus Underlying Value}
\begin{center}
\includegraphics[max width=\textwidth]{2023_05_04_36535b8d80b32081d422g-344}
\end{center}

A key distinction between forward commitments and contingent claims is the magnitude of a derivative's price change for a given change in the underlying value. For a forward commitment, the derivative's value is a linear (one-for-one) function of the underlying price, while for an option it is a non-linear relationship that depends on the likelihood that an option buyer will exercise in the future. The more likely it is that the option buyer will exercise (e.g., when the option is in the money), the more sensitive the option's value to the underlying price. This likelihood depends on the relationship between the value of the underlying and the exercise price, a feature unique to option contracts that we turn to next.

\section{Exercise Price}
The exercise price is the threshold that determines whether an option buyer chooses to transact at contract maturity. For a call option representing the right to buy the underlying, the exercise price represents a lower bound on the option's exercise value at maturity, leading to a higher option value for a lower exercise price. As call options settle based on $\operatorname{Max}\left(0, S_{T}-X\right)$ at expiry, a lower $X$ will increase both the likelihood of exercise and the settlement value if exercised.

The exercise price of a put option, in contrast, is an upper bound on its exercise value at maturity. A higher exercise price at which a put option buyer has the right to sell the underlying will therefore increase the value of the option. Exhibit 7 shows these differing effects of exercise price changes for both call and put options.

\section{Exhibit 7: Call and Put Option Value versus Exercise Price}
\begin{center}
\includegraphics[max width=\textwidth]{2023_05_04_36535b8d80b32081d422g-345}
\end{center}

\section{Time to Expiration}
Recall from a prior lesson that the time value, or difference between an option's price and its exercise value, represents the likelihood that favorable changes to the underlying price will increase the profitability of exercise. Unlike a long forward position where the buyer is equally likely to experience an increase or a decrease in the underlying price over a longer period, the one-sided (or asymmetric) payoff profile of an option allows the buyer to ignore any outcomes where the option will expire unexercised.

For a call option, a longer time to expiration will increase the option's value in all cases. The price appreciation potential of an underlying is essentially unlimited and increases over longer periods, while the downside is limited to the loss of the premium.

A put option representing the right to sell the underlying also usually benefits from the passage of time. In the case of a put option, a longer time to expiration offers greater potential for price depreciation below the exercise price, while the loss is limited to the premium if the underlying price rises. However, as put option buyers await the sale of the underlying in order to receive $\left(X-S_{T}\right)$ upon exercise, a longer time to expiry reduces the present value of the payoff. While less common, in some cases a longer time to expiration will lower a put's value, especially for deep-in-the-money puts with a longer time to expiration and a higher risk-free rate of interest.

\section{Risk-Free Interest Rate}
An earlier lesson showed the risk-free rate to be the opportunity cost of holding an asset, which extends to the no-arbitrage valuation of derivatives. For example, an option's exercise value was shown to be equal to the difference between the spot price and the present value of the exercise price:

Call Option Exercise Value: Max $\left(0,\left(S_{t}-\mathrm{PV}(X)\right)\right)$

Put Option Exercise Value: $\operatorname{Max}\left(0,\left(\mathrm{PV}(X)-S_{t}\right)\right)$

A higher risk-free rate therefore lowers the present value of the exercise price provided an option is in the money. A higher risk-free rate will increase the exercise value of a call option and decrease the exercise value of a put option. Note that the risk-free rate does not directly affect the time value of an option.

\section{Volatility of the Underlying}
Volatility is a measure of the expected dispersion of an underlying asset's future price movements. Higher price volatility of the underlying increases the likelihood of a higher positive exercise value without affecting the downside case in which the option expires unexercised, as shown in Exhibit 8. Note that this effect will be the same for both call options and put options.

For example, as volatility rises, a wider range of possible underlying prices increases an option's time value and the likelihood that it will end up in the money. Lower volatility will reduce the time value of both put and call options.

\section{Exhibit 8: Volatility and Option Value}
\begin{center}
\includegraphics[max width=\textwidth]{2023_05_04_36535b8d80b32081d422g-346}
\end{center}

\section{Income or Cost Related to Owning Underlying Asset}
In a prior lesson, the benefits and costs of owning an underlying asset over time were shown to affect the relationship of spot versus forward prices. For example, income or other, non-cash benefits (such as convenience yield) accrue to the owner of an underlying asset but not to the owner of a derivative. This effect holds for both forward commitments and contingent claims. Income or other benefits of ownership decrease the value of a call and increase the value of a put. Carry costs (such as storage and insurance for commodities) have the opposite effect, increasing the value of a call option and decreasing the value of a put. A summary of the factors that affect an option's value is provided in Exhibit 9, with the sign on the option value referring to the impact (positive or negative) of an increase in the factor.

\section{Exhibit 9: Factors That Affect the Value of an Option}
\begin{center}
\begin{tabular}{|c|c|c|}
\hline
Factor & Call Value & Put Value \\
\hline
Value of the underlying & + & - \\
\hline
Exercise price & - & + \\
\hline
Time to expiration & + & $+/-$ \\
\hline
Risk-free interest rate & + & - \\
\hline
Volatility of the underlying & + & + \\
\hline
Income/cost related to owning the underlying & $-1+$ & $+/-$ \\
\hline
\end{tabular}
\end{center}

\section{KNOWLEDGE CHECK}
\section{Factors That Affect Option Value}
\begin{enumerate}
  \item Determine the correct answers to complete the following sentence: For a call option representing the right to buy the underlying, the exercise price represents a bound on an option's exercise value at maturity, leading to a option value for a lower exercise price.
\end{enumerate}

\section{Solution:}
For a call option representing the right to buy the underlying, the exercise price represents a lower bound on an option's exercise value at maturity, leading to a higher option value for a lower exercise price.

\begin{enumerate}
  \setcounter{enumi}{1}
  \item Match the following changes in factors affecting option value (holding other factors constant) with their corresponding option value change:

  \item A lower exercise price $(X)$

  \item A lower underlying price $\left(S_{T}\right)$

  \item A rise in the volatility of the underlying price

\end{enumerate}

\begin{abstract}
A. Increases the value of both a call option and a put option
B. Increases the value of a call option
C. Increases the value of a put option
\end{abstract}

\section{Solution:}
\begin{enumerate}
  \item B is correct. A lower exercise price increases the value of a call option, since for a given underlying price at maturity of $S_{T}$, the call option settlement value of $\operatorname{Max}\left(0, S_{T}-X\right)$ will increase for a lower $X$.

  \item $C$ is correct. A lower underlying price $\left(S_{T}\right)$ will increase the value of a put option. Since a put option is the right to sell an underlying at the exercise price $(X)$, the put option settlement value of $\operatorname{Max}\left(0, X-S_{T}\right)$ will increase as $S_{T}$ declines. 3. A is correct. A rise in the volatility of the underlying price will increase the value of both a call option and a put option. Greater price variability of the underlying will increase the probability of a higher positive exercise value for a call or a put option without affecting the downside case where the option expires unexercised.

  \item Determine the correct answers to complete the following sentences: A key distinction between forward commitments and contingent claims is the of a derivative's price change for a given change in the underly-

\end{enumerate}

ing value. For a forward commitment, the derivative value is a function of the underlying price, while for an option it is a relationship.

\section{Solution:}
A key distinction between forward commitments and contingent claims is the magnitude of a derivative's price change for a given change in the underlying value. For a forward commitment, the derivative value is a linear function of the underlying price, while for an option it is a non-linear relationship.

\begin{enumerate}
  \setcounter{enumi}{3}
  \item Explain the effect of a decrease in the risk-free rate of interest on the value of put and call options and justify your answer.
\end{enumerate}

\section{Solution:}
A decrease in the risk-free rate increases the present value of an option's exercise price $(\mathrm{PV}(X))$. The exercise value for an in-the-money call option will fall, since $\left(S_{t}-P V(X)\right)$ is lower for a higher $P V(X)$, while the exercise value for an in-the-money put option will rise, since $\left(\operatorname{PV}(X)-S_{t}\right)$ rises for a higher $\operatorname{PV}(X)$. Note that the time value of an option is not directly affected by the risk-free rate.

\section{PRACTICE PROBLEMS}
The Viswan Family Office (VFO) owns non-dividend-paying shares of Biomian Limited that are currently priced $\left(S_{0}\right)$ at INR 295 per share. VFO's CIO is considering an offer to sell shares at a forward price $\left(F_{0}(T)\right)$ of INR 300.84 per share in six months based on a risk-free rate of $4 \%$. You have been asked to advise on the purchase of a put option or the sale of a call option with an exercise price $(X)$ equal to the forward price $\left(F_{0}(T)\right)$ as alternatives to a forward share sale.

\begin{enumerate}
  \item VFO is considering the purchase of the put option to hedge against a decline in Biomian's share price. Which of the following statements best characterizes the trade-off between the put and the forward based on no-arbitrage pricing?
\end{enumerate}

A. The gain on the forward sale will equal the purchased put option's profit at maturity provided the put option ends up in the money at maturity.

B. The loss on the forward sale will exceed the loss on the purchased put at maturity if Biomian's share price exceeds the forward price by more than the initial put premium paid.

C. We do not have enough information to answer this question, since we do not know the time value of the option at maturity.

\begin{enumerate}
  \setcounter{enumi}{1}
  \item In evaluating the purchased put strategy (with $X=F_{0}(T)$ ), the CIO has asked you to consider selling the put in three months' time if its price appreciates over that period. Which of the following best characterizes the no-arbitrage put price at that time?
\end{enumerate}

A. As VFO will exercise only if the spot price is below the exercise price, the lower bound of the put price is the greater of zero and the present value of the spot price minus the exercise price.

B. As VFO will exercise only if the spot price is below the exercise price, the upper bound of the put price equals the present value of the exercise price minus the spot price.

C. The put price can be no greater than the forward price and no less than the greater of zero and the present value of the exercise price minus the spot price.

\begin{enumerate}
  \setcounter{enumi}{2}
  \item VFO is considering a sold call strategy to generate income from the sale of a call. In your scenario analysis of the sold call option alternative, VFO has asked you to value the call option in three months' time if Biomian's spot price is INR 325 per share. Given an estimated call price of INR 46.41 at that time, which of the following correctly reflects the relationship between the call's exercise value and its time value?
\end{enumerate}

A. The call's exercise value is INR 24.16 , and its time value is INR 22.25 .

B. The call's exercise value is INR 27.10, and its time value is INR 19.31 .

C. The call's exercise value is INR 20.99, and its time value is INR 25.42.

\begin{enumerate}
  \setcounter{enumi}{3}
  \item In comparing the sold call and purchased put strategies at the forward price, VFO's CIO is concerned about how an immediate increase in the volatility of the underlying Biomian shares might affect option value. Which of the following statements about this volatility change and its effect on strategy is most accurate?
\end{enumerate}

A. An increase in the volatility of the underlying shares has the same effect on call and put option values, so this change should not affect VFO's strategy decision.

B. Since changes in the volatility of the underlying shares have the opposite effect on put versus call options, this change will increase the attractiveness of the put strategy versus the call strategy.

C. An increase in the volatility of the underlying shares will increase both the cost of the purchased put strategy and the premium received on the sold call strategy, so this change will increase the attractiveness of the call strategy versus the put strategy.

\begin{enumerate}
  \setcounter{enumi}{4}
  \item In comparing the Biomian purchased put and sold call strategies, which of the following statements is most correct about how the call and put values are affected by changes in factors other than volatility?
\end{enumerate}

A. Changes in the time to expiration and the risk-free rate have a similar directional effect on the put and call strategies, while changes in the exercise price tend to have the opposite effect.

B. Changes in the time to expiration tend to have a similar directional effect on the put and call strategies, while changes in the exercise price and the risk-free rate tend to have the opposite effect.

C. Changes in the risk-free rate have a similar directional effect on the put and call strategies, while changes in the exercise price and the time to expiration tend to have the opposite effect.

\section{SOLUTIONS}
\begin{enumerate}
  \item The correct answer is $B$. The loss on the forward sale will be greater than the loss on the purchased put at maturity if Biomian's share price exceeds the forward price by more than the initial put premium. VFO's downside return is limited to the put premium paid, while the forward sale has unlimited downside as Biomian shares appreciate. A is incorrect as it does not take the put premium paid into account, while $\mathrm{C}$ is incorrect as the time value of an option is equal to zero at maturity.

  \item The correct answer is $C$. The put exercise price, $X$ (equal to $F_{0}(T)$ in this case), represents the upper bound on the put value, while the lower bound is the greater of the present value of the exercise price minus the spot price and zero:

\end{enumerate}

$\operatorname{Max}\left(0, X(1+r)^{-(T-t)}-S_{t}\right)<p_{t} \leq X$

A is incorrect, as the lower bound of the put price is the greater of zero and the present value of the exercise price minus the spot price, not the present value of the spot price minus the exercise price. B is incorrect, as the lower, not the upper, bound of the put price equals the present value of the exercise price minus the spot price.

\begin{enumerate}
  \setcounter{enumi}{2}
  \item The correct answer is $B$. The option price is equal to the sum of the exercise value and the time value. A call option's exercise value is equal to the greater of zero and the spot price minus the present value of the exercise price:
\end{enumerate}

$$
\begin{aligned}
& \operatorname{Max}\left(0, S_{t}-X(1+r)^{-(T-t)}\right) \\
& =\operatorname{Max}\left(0, \operatorname{INR} 325-\operatorname{INR} 300.84(1.04)^{-0.25}\right) \\
& =\operatorname{INR} 27.10
\end{aligned}
$$

The time value is equal to the call price minus the exercise value, or INR $19.31(=$ INR46.41 - INR27.10). A is incorrect as it takes the spot price minus the exercise price as the exercise value, while $\mathrm{C}$ calculates the exercise value as the present value of the spot price minus the exercise price.

\begin{enumerate}
  \setcounter{enumi}{3}
  \item The correct answer is C. An increase in the volatility of the underlying share price will increase both the upfront premium received on the sold call option and the premium paid on the purchased put option. Therefore, since the purchased put strategy involves an increased upfront payment made by VFO and the sold call strategy involves an increased premium received, the volatility increase will increase the attractiveness of the call strategy versus the put strategy.

  \item The correct answer is B. Changes in the time to expiration tend to have a similar directional effect on the put and call strategies (the only exception being deep-in-the-money put options in some cases), while changes in the exercise price and the risk-free rate tend to have the opposite effect.

\end{enumerate}

\section{LEARNING MODULE}
\begin{center}
\includegraphics[max width=\textwidth]{2023_05_04_36535b8d80b32081d422g-353}
\end{center}

\section{Option Replication Using Put-Call Parity}
\section{LEARNING OUTCOME}
\begin{center}
\begin{tabular}{c|l}
Mastery & The candidate should be able to: \\
\hline
$\square$ & explain put-call parity for European options \\
$\square$ & explain put-call forward parity for European options \\
\end{tabular}
\end{center}

\section{INTRODUCTION}
Previous lessons examined the payoff and profit profiles of call options and put options, the upper and lower bounds of an option's value, and the factors impacting option values. In doing so, we contrasted the asymmetry of one-sided option payoffs with the linear or symmetric payoff of forwards and underlying assets.

We now extend this analysis further to show that there are ways to combine options to have an equivalent payoff to that of the underlying and a risk-free asset as well as a forward commitment. In the first lesson, we demonstrate that the value of a European call may be used to derive the value of a European put option with the same underlying details, and vice versa, under a no-arbitrage condition referred to as put-call parity. In the second lesson, we show how this may be extended to forward commitments and how the put-call parity relationship may be applied to option and other investment strategies. We will focus on European options on underlying assets with no income or benefit.

\section{Summary}
\begin{itemize}
  \item Put-call parity establishes a relationship that allows the price of a call option to be derived from the price of a put option with the same underlying details and vice versa.

  \item Put-call parity holds for European options with the same exercise price and expiration date, representing a no-arbitrage relationship between put option, call option, underlying asset, and risk-free asset prices.

  \item If put-call parity does not hold, then riskless arbitrage profit opportunities may be available to investors. - Put-call forward parity extends the put-call parity relationship to forward contracts given the equivalence of an underlying asset position and a long forward contract plus a risk-free bond.

  \item Under put-call forward parity, we may demonstrate that a purchased put option and a sold call option are equivalent to a long risk-free bond and short forward position, and a sold put and purchased call are equivalent to a long forward and short risk-free bond.

  \item Put-call parity may be applied beyond option-based strategies in financefor example, to demonstrate that equity holders have a position equivalent to a purchased call option on the value of the firm with unlimited upside, while debtholders have a sold put option position on firm value with limited upside.

\end{itemize}

\section{LEARNING MODULE PRE-TEST}
\begin{enumerate}
  \item Identify the following statement as true or false, and justify your answer: Put-call parity demonstrates that a long underlying position is equivalent to a purchased put option, a sold call option, and a risk-free bond.
\end{enumerate}

\section{Solution:}
The statement is false. Put-call parity demonstrates that a long underlying position is equivalent to a sold put option, a purchased call option, and a long risk-free bond. This is shown in the following equation:

Put-call parity: $S_{0}+p_{0}=c_{0}+X(1+r)^{-T}$ implies $S_{0}=-p_{0}+c_{0}+X(1+r)^{-T}$.

\begin{enumerate}
  \setcounter{enumi}{1}
  \item Identify which of the following positions has the same no-arbitrage value as which portfolio under put-call parity:

  \item Long call option $\left(c_{0}\right)$

  \item Short risk-free bond $\left(-X(1+r)^{-T}\right)$

  \item Short put option $\left(-p_{0}\right)$ A. Long underlying, short risk-free bond, and short call option

\end{enumerate}

B. Long underlying, long put option, and short risk-free bond

C. Short underlying, long call option, and short put option

\section{Solution:}
Recall that the put-call parity relationship may be expressed as

$$
S_{0}+p_{0}=c_{0}+X(1+r)^{-T} \text {. }
$$

\begin{enumerate}
  \item B is correct. A long call option position is the no-arbitrage equivalent of a long underlying position, a long put option, and a short risk-free bond position.

  \item C is correct. A short risk-free bond position is equivalent to a short underlying position, a long call option, and a short put option.

  \item A is correct. A short put option is equivalent to a long underlying position, a short risk-free bond, and a short call option.

  \item Consider a one-year put option with an exercise price $(X)$ of EUR1,100. The underlying asset, $S_{0}$, trades at EUR990 at time $t=0$, and the risk-free rate, $r$, is $1 \%$. Calculate the no-arbitrage put price $\left(p_{0}\right)$ if a one-year call option with the same exercise price is currently trading at EUR159.

\end{enumerate}

\section{Solution:}
The put-call parity relationship is given as

$$
S_{0}+p_{0}=c_{0}+X(1+r)^{-T}
$$

Substitute $S_{0}=$ EUR990, $c_{0}=$ EUR159, and $X(1+r)^{-T}=$ EUR1,089.11 $(=$

EUR1,100/1.01), and solve for $p_{0}$ :

$p_{0}=\mathrm{EUR} 159+\mathrm{EUR} 1,089.11-\mathrm{EUR} 990$, or $p_{0}=$ EUR258.11.

\begin{enumerate}
  \setcounter{enumi}{3}
  \item Identify the following as true or false, and justify your answer: Put-call forward parity links option, forward, and cash underlying prices given the equivalence of a cash underlying position with a forward purchase and risk-free bond position.
\end{enumerate}

\section{Solution:}
The statement is true. Since a cash underlying position has the same value at maturity as a forward purchase and risk-free bond position, the cash underlying position may be replaced by the forward purchase and risk-free bond portfolio to solve for put-call forward parity.

\begin{enumerate}
  \setcounter{enumi}{4}
  \item Determine the correct answers to complete the following sentence: A synthetic protective put position is a combination of a forward , a risk-free bond, and $\mathrm{a}(\mathrm{n})$ put on the underlying.
\end{enumerate}

\section{Solution:}
A synthetic protective put position is a combination of a forward purchase, a risk-free bond, and a purchased put on the underlying.

\begin{enumerate}
  \setcounter{enumi}{5}
  \item Determine the correct answer to complete the following sentence: When considering shareholder claims in option terms, the shareholder payoff resembles a(n) option on firm value.
\end{enumerate}

\section{Solution:}
When considering shareholder claims in option terms, the shareholder payoff resembles a call option on firm value.

\section{PUT-CALL PARITY}
explain put-call parity for European options

A prior lesson contrasted no-arbitrage pricing conditions and the replication of cash flows for forward commitments and contingent claims. Forwards have zero initial value and their certain payoff, which is replicated at inception by borrowing to purchase the underlying or selling the underlying and lending the sale proceeds. Option buyers pay an upfront premium, and their contingent payoff profiles lead us to establish upper and lower no-arbitrage price bounds. Option replication is similar to that of a forward but involves borrowing or lending to buy or sell a proportion of the underlying, which is adjusted as the moneyness of an option changes. We now extend this analysis using a combination of positions.

In this section, we show how combining cash and derivative instruments into a portfolio in a certain way enables us to price and value these positions without directly modeling them using no-arbitrage conditions. Consider an investor whose goal is to benefit from upward movements in the value of an underlying but who wants to protect her investment from downward movements in the underlying's value. Consider the following two portfolios, shown in Exhibit 1:

\begin{enumerate}
  \item At $t=0$, an investor purchases a call option $\left(c_{0}\right)$ on an underlying with an exercise price of $X$ and a risk-free bond today that pays $X$ at $t=T$. The cost of this strategy is $c_{0}+X(1+r)^{-T}$, where we assume the option expires at time $T$.

  \item At $t=0$, an investor purchases an underlying unit $\left(S_{0}\right)$ and a put option on the underlying $\left(p_{0}\right)$ with an exercise price of $X$ at $t=T$. The cost of this strategy is $p_{0}+S_{0}$.

\end{enumerate}

Exhibit 1 shows the payoff of the individual components of these two portfolios.

\section{Exhibit 1: Payoffs at Time T for Two Portfolios}
\begin{center}
\includegraphics[max width=\textwidth]{2023_05_04_36535b8d80b32081d422g-356}
\end{center}

At first glance, these portfolios appear to offer the investor a similar opportunity to benefit from underlying asset appreciation without exposure to an underlying price decline below the exercise price. In the first case (Portfolio 1), the investor buys a call option with a positive payoff if the underlying asset price rises above the exercise price $\left(S_{T}>X\right)$ and invests cash in a risk-free bond. Since the risk-free asset pays $X$ at time $T$, the investor pays $c_{0}+$ $X(1+r)^{-T}$ at time $t=0$. This combination of a purchased call and a risk-free bond is known as a fiduciary call and is shown in Exhibit 2.

\section{Exhibit 2: Portfolio 1 (Fiduciary Call) Payoff at Time $T$}
Long call option

\begin{center}
\includegraphics[max width=\textwidth]{2023_05_04_36535b8d80b32081d422g-357}
\end{center}

In the second instance (Portfolio 2) in Exhibit 1, the investor pays $S_{0}+p_{0}$ at inception and is hedged if the underlying price falls below $X$. This strategy of holding an underlying asset and purchasing a put on the same asset is sometimes called a protective put. The payoff for Portfolio 2 at time $T$ is shown in Exhibit 3 .

\section{Exhibit 3: Portfolio 2 (Protective Put) Payoff at Time T}
\begin{center}
\includegraphics[max width=\textwidth]{2023_05_04_36535b8d80b32081d422g-358}
\end{center}

Despite their differences, these two portfolios have identical payoff profiles, as evidenced by Exhibit 2 and Exhibit 3. Recall that no-arbitrage conditions require two assets with identical future cash flows to trade at the same price, ignoring transaction costs. In Exhibit 4, we evaluate this in the case of the two portfolios by comparing their cash flows under all possible scenarios at time $t=T$.

\section{Exhibit 4: Protective Put vs. Fiduciary Call at Expiration}
\begin{center}
\begin{tabular}{|c|c|c|c|}
\hline
Portfolio Position & $\begin{array}{l}\text { Put Exercised }\left(S_{T}\right. \\ <X)\end{array}$ & $\begin{array}{c}\text { No Exercise } \\ \left(S_{T}=X\right)\end{array}$ & $\begin{array}{c}\text { Call Exercised } \\ \qquad\left(S_{T}>X\right)\end{array}$ \\
\hline
\multicolumn{4}{|l|}{Protective Put:} \\
\hline
Underlying Asset & $S_{T}$ & $S_{T}$ & $S_{T}$ \\
\hline
Put Option & $X-S_{T}$ & 0 & 0 \\
\hline
Total: & $X$ & $S_{T}(=X)$ & $S_{T}$ \\
\hline
\multicolumn{4}{|l|}{Fiduciary Call:} \\
\hline
Call Option & 0 & 0 & $S_{T}-X$ \\
\hline
Risk-Free Asset & $X$ & $X$ & $X$ \\
\hline
Total: & $X$ & $X\left(=S_{T}\right)$ & $S_{T}$ \\
\hline
\end{tabular}
\end{center}

These two portfolios have cash flows that are identical at time $T$ under each scenario, and the prices of these portfolios must be equal at $t=0$ to satisfy the no-arbitrage condition in a relationship commonly referred to as put-call parity:

$$
S_{0}+p_{0}=c_{0}+X(1+r)^{-T} \text {. }
$$

In other words, under put-call parity, at $t=0$ the price of the long underlying asset plus the long put must equal the price of the long call plus the risk-free asset.

\section{EXAMPLE 1}
\section{Biomian Put-Call Parity}
In an earlier example, the Viswan Family Office (VFO) held non-dividend-paying Biomian shares currently priced $\left(S_{0}\right)$ at INR295 per share. VFO is considering the purchase of a six-month put on Biomian shares at an exercise price, $X$, of INR265. If VFO's chief investment officer observes a traded six-month call option price of INR59 per share for the same INR265 exercise price, what should he expect to pay for the put per share if the relevant risk-free rate is $4 \%$ ?

From Equation 1, the put-call parity relationship was shown as

$$
S_{0}+p_{0}=c_{0}+X(1+r)^{-T} .
$$

We can solve for the risk-free bond price as INR259.85 (= INR265(1.04) $\left.)^{-0.5}\right)$ and substitute into Equation 1:

$\operatorname{INR} 295+p_{0}=\operatorname{INR} 59+$ INR259.85

$p_{0}=\mathrm{INR} 23.85$.

VFO should expect to pay a six-month put option premium of $p_{0}=$ INR23.85.

Arbitrage profit opportunities arise if these portfolios trade at different prices. As in the case of individual assets, an investor able to borrow and lend at the risk-free rate can earn a riskless profit if either portfolio is mispriced, as shown in the following example.

\section{EXAMPLE 2}
\section{VFO Put-Call Parity Arbitrage Opportunity}
As in Example 1, the Viswan Family Office holds non-dividend-paying Biomian shares at a price $\left(S_{0}\right)$ of INR295 per share. VFO is considering the purchase of a put for which it expects to pay INR23.85 but which is instead currently priced at INR30. Assuming a risk-free rate of $4 \%$, identify the arbitrage opportunity and the steps VFO might take to earn a riskless profit.

From Equation 1, the put-call parity relationship is

$$
S_{0}+p_{0}=c_{0}+X(1+r)^{-T}
$$

Substituting the values from Example 1 and the current put price into Equation 1 gives us the following result:

INR295 + INR30 > INR59 + INR259.85,

$$
\text { so, } S_{0}+p_{0}>c_{0}+X(1+r)^{-T} \text {. }
$$

By selling the put and the shares and purchasing the call and the risk-free asset

\begin{center}
\begin{tabular}{|c|c|c|c|c|}
\hline
$\begin{array}{l}\text { Arbitrage } \\ \text { Position }\end{array}$ & $\begin{array}{c}\text { Cash Flow } \\ \text { at } t=0\end{array}$ & $\begin{array}{l}\text { Put } \\ \text { Exercised }\left(S_{T}\right. \\ <X)\end{array}$ & $\begin{array}{c}\text { No Exercise } \\ \qquad\left(S_{T}=X\right)\end{array}$ & $\begin{array}{l}\text { Call Exercised } \\ \quad\left(S_{T}>X\right)\end{array}$ \\
\hline
\multicolumn{5}{|l|}{Protective Put:} \\
\hline
$\begin{array}{l}\text { Sell Underlying } \\ \text { Asset }\end{array}$ & $S_{0}$ & $-S_{T}$ & $-S_{T}$ & $-S_{T}$ \\
\hline
Sell Put Option & $p_{0}$ & $-\left(X-S_{T}\right)$ & 0 & 0 \\
\hline
\end{tabular}
\end{center}

at $t=0$, VFO has a positive cash flow of INR6.15 (= INR295 + INR30 - INR59 - INR259.85), or $S_{0}+p_{0}-c_{0}-X(1+r)^{-T}$, as shown in the following diagram.

\begin{center}
\begin{tabular}{|c|c|c|c|c|}
\hline
$\begin{array}{l}\text { Arbitrage } \\ \text { Position }\end{array}$ & $\begin{array}{l}\text { Cash Flow } \\ \text { at } t=0\end{array}$ & $\begin{array}{l}\text { Put } \\ \text { Exercised }\left(S_{T}\right. \\ <X)\end{array}$ & $\begin{array}{l}\text { No Exercise } \\ \qquad\left(S_{T}=X\right)\end{array}$ & $\begin{array}{l}\text { Call Exercised } \\ \qquad\left(S_{T}>X\right)\end{array}$ \\
\hline
Total: & $S_{0}+p_{0}$ & $-X$ & $-S_{T}(=X)$ & $-S_{T}$ \\
\hline
\multicolumn{5}{|l|}{Fiduciary Call:} \\
\hline
Buy Call Option & $-c_{0}$ & 0 & 0 & $\left(S_{T}-X\right)$ \\
\hline
Buy Risk-Free Asset & $-X(1+r)^{-T}$ & $X$ & $X$ & X \\
\hline
Total: & $\begin{array}{c}-c_{0}-X(1+ \\ r)^{-T}\end{array}$ & $X$ & $X\left(=S_{T}\right)$ & $S_{T}$ \\
\hline
Overall Portfolio: & $\begin{array}{l}S_{0}+p_{0}-c_{0} \\ -X(1+r)^{-T}\end{array}$ & 0 & 0 & 0 \\
\hline
\end{tabular}
\end{center}

Note that the combined cash flows of the two portfolios are equal to zero under each scenario at time $t=T$, leaving VFO with an arbitrage profit of INR6.15.

\section{OPTION STRATEGIES BASED ON PUT-CALL PARITY}
The put-call parity relationship established between call option, put option, underlying asset, and risk-free asset pricing in the previous section provides the foundation for thinking about replication and pricing of individual derivative positions, cash positions, and option-based strategies.

For example, if we rearrange Equation 1 to solve for the put option premium, $p_{0}$, we find that it may be derived from a combination of a long call option $\left(c_{0}\right)$, a long risk-free bond $\left(X(1+r)^{-T}\right)$, and a short position in the underlying $\left(-S_{0}\right)$, as shown in Equation 2 and Exhibit 5.

$$
p_{0}=c_{0}+X(1+r)^{-T}-S_{0} \text {. }
$$

Note that Equation 2 is both a statement of what the price of the put option should be and also sets out a replicating portfolio for that put option using a call option, the underlying, and a risk-free bond, as shown in Exhibit 5.

\section{Exhibit 5: Put Option as a Long Call, Long Bond, and Short Underlying}
\begin{center}
\includegraphics[max width=\textwidth]{2023_05_04_36535b8d80b32081d422g-361}
\end{center}

Short underlying
\includegraphics[max width=\textwidth, center]{2023_05_04_36535b8d80b32081d422g-361(1)}

What is more, we may also demonstrate that the asymmetric payoff profiles of put and call options effectively offset one another in a no-arbitrage condition when combined to solve for the price of the underlying, as follows:

$$
S_{0}=c_{0}-p_{0}+X(1+r)^{-T} .
$$

Exhibit 6 summarizes the equivalence of these individual positions in terms of replicating positions using other cash and derivative positions.

\section{Exhibit 6: Replication of Individual Positions under Put-Call Parity}
\begin{center}
\begin{tabular}{|c|c|c|c|c|}
\hline
Position & $\begin{array}{l}\text { Underlying } \\ \qquad\left(S_{0}\right)\end{array}$ & $\begin{array}{l}\text { Risk-Free Bond } \\ \qquad\left(X(1+r)^{-T}\right)\end{array}$ & $\begin{array}{c}\text { Call Option } \\ \qquad\left(c_{0}\right)\end{array}$ & Put Option $\left(p_{0}\right)$ \\
\hline
Underlying $\left(S_{0}\right)$ & - & Long & Long & Short \\
\hline
$\begin{array}{l}\text { Risk-free bond (X(1 } \\ \left.+r)^{-T}\right)\end{array}$ & Long & - & Short & Long \\
\hline
Call option $\left(c_{0}\right)$ & Long & Short & - & Long \\
\hline
Put option & Short & Long & Long & - \\
\hline
\end{tabular}
\end{center}

These building blocks are used not only to generate riskless profits in the case of mispricing but also to create other option-based strategies, such as in the following example.

\section{EXAMPLE 3}
\section{VFO Covered Call Strategy}
Recall from an earlier example that the Viswan Family Office holds non-dividend-paying Biomian shares currently priced $\left(S_{0}\right)$ at INR295 per share. Since VFO's chief investment officer believes Biomian's share price will appreciate over the long term but remain relatively unchanged for the next six months, he would like to sell a six-month call option at a INR325 exercise price $(X)$ to generate short-term income in what is known as a covered call strategy.

Using put-call parity, how can he replicate this position using a risk-free bond (the risk-free rate is $4 \%$ ) and a put option, and what is the expected call option premium if the put option has a price of INR56?

The covered call strategy consists of a long position in the underlying and a short call option, or $\left(S_{0}-c_{0}\right)$ at inception. Recall from Equation 1 that put-call parity is shown as

$$
S_{0}+p_{0}=c_{0}+X(1+r)^{-T} \text {. }
$$

We may rearrange these terms to solve for $\left(S_{0}-c_{0}\right)$ :

$$
S_{0}-c_{0}=X(1+r)^{-T}-p_{0} \text {. }
$$

The covered call position is therefore equivalent to a long risk-free bond and a short put option, as shown in the following diagram.
\includegraphics[max width=\textwidth, center]{2023_05_04_36535b8d80b32081d422g-362}

We may solve for the no-arbitrage call price, $c_{0}$, using put-call parity by substituting terms into the following equation:

$S_{0}-c_{0}=X(1+r)^{-T}-p_{0}$.

INR295 $-\mathrm{c}_{0}=\operatorname{INR} 325(1.04)^{-0.5}-\operatorname{INR} 56$.

$c_{0}=\mathrm{INR} 32.31$

\section{KNOWLEDGE CHECK}
\section{Put-Call Parity}
\begin{enumerate}
  \item Determine the correct answers to complete the following sentence: If two portfolios have cash flows that are at time $T$ under each scenario, then the no-arbitrage prices of those portfolios must one another at $t=0$ in a relationship commonly referred to as put-call parity.
\end{enumerate}

\section{Solution:}
If two portfolios have cash flows that are identical at time $T$ under each scenario, then the no-arbitrage prices of those portfolios must equal one another at $t=0$ in a relationship commonly referred to as put-call parity.

\begin{enumerate}
  \setcounter{enumi}{1}
  \item Identify which of the following positions has the same no-arbitrage value as which portfolio under put-call parity:

  \item Short call option $\left(-c_{0}\right)$

  \item Long risk-free bond $\left(X(1+r)^{-T}\right)$

  \item Long put option $\left(p_{0}\right)$ A. Short underlying, long risk-free bond, and long call option

\end{enumerate}

B. Short underlying, short put option, and long risk-free bond

C. Long underlying, short call option, and long put option

\section{Solution:}
Recall that the put-call parity relationship may be expressed as

$S_{0}+p_{0}=c_{0}+X(1+r)^{-T}$.

\begin{enumerate}
  \item B is correct. A short call option position is the no-arbitrage equivalent of a short underlying position, a short put option, and a long risk-free bond position.

  \item $\mathrm{C}$ is correct. A long risk-free bond position is equivalent to a long underlying position, a short call option, and a long put option.

  \item A is correct. A long put option is equivalent to a short underlying position, a long risk-free bond, and a long call option.

  \item Determine the correct answer to complete the following sentence: The combination of a call and a risk-free bond position is known as a fiduciary call.

\end{enumerate}

\section{Solution:}
The combination of a purchased call and a risk-free bond position is known as a fiduciary call.

\begin{enumerate}
  \setcounter{enumi}{3}
  \item Identify the following statement as true or false, and justify your answer: The covered call strategy consists of a long position in the underlying and a long call option, or $\left(S_{0}+c_{0}\right)$ at inception.
\end{enumerate}

\section{Solution:}
The statement is false. The covered call strategy consists of a long position in the underlying and a short call option, or $\left(S_{0}-c_{0}\right)$ at inception.

\section*{PUT-CALL FORWARD PARITY AND OPTION APPLICATIONS }
Forward commitment replication was shown in earlier lessons to involve borrowing at the risk-free rate to purchase the underlying in the case of a long position or selling the underlying short and lending the proceeds at the risk-free rate for a short position. We also showed that a long underlying position combined with a short forward resulted in a risk-free return. We now incorporate these forward commitment building blocks into the put-call parity relationship shown in the prior lesson.

\section{PUT-CALL FORWARD PARITY}
In an earlier lesson, we learned that a long underlying asset position can be replicated by entering a long forward contract and purchasing the risk-free asset. Combining the synthetic asset with the put-call parity relationship-so, substituting the present value of $F_{0}(T)$ for $S_{0}$ in Equation 1-we have what is referred to as put-call forward parity:

$$
F_{0}(T)(1+r)^{-T}+p_{0}=c_{0}+X(1+r)^{-T} \text {. }
$$

We can demonstrate put-call forward parity by comparing a synthetic protective put position to the protective put and fiduciary call positions. Consider a modification of the portfolios used earlier to demonstrate put-call parity, as follows:

\begin{enumerate}
  \item At $t=0$, an investor purchases a forward contract and a risk-free bond with a face value equal to the forward price, $F_{0}(T)$, and a put option on the underlying $\left(p_{0}\right)$ with an exercise price of $X$ at $t=T$.

  \item At $t=0$, an investor purchases a call option $\left(c_{0}\right)$ on the same underlying with an exercise price of $X$ and a risk-free bond that pays $X$ at $t=T$.

\end{enumerate}

The first portfolio has replaced the cash underlying position with a synthetic underlying position using a forward purchase and a risk-free bond. The combination of the synthetic underlying position and a purchased put on the underlying is known as a synthetic protective put. Exhibit 7 demonstrates the equivalence of the protective and synthetic protective puts by comparing their cash flows at both $t=0$ and contract maturity, $T$.

\section{Exhibit 7: Protective Put vs. Synthetic Protective Put}
\begin{center}
\begin{tabular}{lccc}
\hline
Position & $\begin{array}{c}\text { Cash Flow } \\ \text { at } \boldsymbol{t}=\mathbf{0}\end{array}$ & $\begin{array}{c}\text { Put Exercised } \\ \left(\mathbf{S}_{\boldsymbol{T}}<\mathbf{X}\right)\end{array}$ & $\begin{array}{c}\text { No Exercise } \\ \left(\mathbf{S}_{\boldsymbol{T}} \geq \boldsymbol{X}\right)\end{array}$ \\
\hline
Protective Put: & $p_{0}$ & $X-S_{T}$ & 0 \\
Purchased Put $\left(p_{0}\right)$ & $S_{0}$ & $S_{T}$ & $S_{T}$ \\
Cash Underlying $\left(S_{0}\right)$ & $p_{0}+S_{0}$ & $X$ & $S_{T}$ \\
Total: &  &  &  \\
Synthetic Protective Put: & $p_{0}$ & $X-S_{T}$ & 0 \\
\end{tabular}
\end{center}

\begin{center}
\begin{tabular}{cccc}
\hline
Position & $\begin{array}{c}\text { Cash Flow } \\ \text { at } \boldsymbol{t}=\mathbf{0}\end{array}$ & $\begin{array}{c}\text { Put Exercised } \\ \left(\mathbf{S}_{\boldsymbol{T}}<\mathbf{X}\right)\end{array}$ & $\begin{array}{c}\text { No Exercise } \\ \left(\mathbf{S}_{\boldsymbol{T}} \geq \mathbf{X}\right)\end{array}$ \\
\hline
$\begin{array}{ccc}\text { Forward Purchase }\end{array}$ & 0 & $\mathrm{~S}_{\mathrm{T}}-\mathrm{F}_{0}(\mathrm{~T})$ & $\mathrm{S}_{\mathrm{T}}-\mathrm{F}_{0}(\mathrm{~T})$ \\
Risk-Free Bond & $F_{0}(T)$ & $F_{0}(T)$ &  \\
$F_{0}(T)(1+r)^{-T}$ & $F_{0}(T)(1+r)^{-T}$ & $X$ & $S_{T}$ \\
Total: & $p_{0}+F_{0}(T)(1+r)^{-T}$ & $X$ &  \\
$\left(=p_{0}+S_{0}\right)$ &  &  &  \\
\hline
\end{tabular}
\end{center}

Exhibit 8 compares the future cash flows of the synthetic protective put with the fiduciary call under all possible scenarios at expiration.

\section{Exhibit 8: Synthetic Protective Put vs. Fiduciary Call}
\begin{center}
\begin{tabular}{|c|c|c|c|}
\hline
Portfolio Position & $\begin{array}{l}\text { Put Exercised }\left(S_{T}\right. \\ \qquad<X)\end{array}$ & $\begin{array}{l}\text { No Exercise } \\ \left(S_{T}=X\right)\end{array}$ & $\begin{array}{l}\text { Call Exercised } \\ \qquad\left(S_{T}>X\right)\end{array}$ \\
\hline
\multicolumn{4}{|l|}{Synthetic Protective Put:} \\
\hline
Purchased Put $\left(p_{0}\right)$ & $X-S_{T}$ & 0 & 0 \\
\hline
Forward Purchase & $S_{T}-F_{0}(T)$ & $S_{T}-F_{0}(T)$ & $S_{T}-F_{0}(T)$ \\
\hline
$\begin{array}{l}\text { Risk-Free Bond } F_{0}(T)(1+ \\ r)^{-T}\end{array}$ & $F_{0}(T)$ & $F_{0}(T)$ & $F_{0}(T)$ \\
\hline
Total: & $X$ & $S_{T}(=X)$ & $S_{T}$ \\
\hline
\multicolumn{4}{|l|}{Fiduciary Call:} \\
\hline
Purchased Call $\left(c_{0}\right)$ & 0 & 0 & $S_{T}-X$ \\
\hline
Risk-Free Asset & $X$ & $X$ & $X$ \\
\hline
Total: & $X$ & $X\left(=S_{T}\right)$ & $S_{T}$ \\
\hline
\end{tabular}
\end{center}

It follows that the cost of the fiduciary call must equal the cost of the synthetic protective put, thereby demonstrating the put-call forward parity relationship:

$$
F_{0}(T)(1+r)^{-T}+p_{0}=c_{0}+X(1+r)^{-T} .
$$

If we rearrange these terms, we can demonstrate that a long put and a short call are equivalent to a long risk-free bond and short forward position:

$$
p_{0}-c_{0}=\left[X-F_{0}(T)\right](1+r)^{-T}
$$

Consider the earlier put-call parity example using a synthetic underlying position, as in the following example.

\section{EXAMPLE 4}
\section{VFO Put-Call Forward Parity}
Consider the Viswan Family Office example using a long forward and a risk-free bond, rather than a cash underlying position as in the prior example. Biomian shares trade at a price $\left(S_{0}\right)$ of INR295 per share. VFO is considering the purchase of a six-month put on Biomian shares at an exercise price $(X)$ of INR265. If VFO's chief investment officer observes a traded six-month call option price of INR59 per share for the same INR265 exercise price, what should he expect to pay for the put option per share if the relevant risk-free rate is $4 \%$ ?

From Equation 5, the put-call forward parity relationship is

$$
p_{0}-c_{0}=\left[X-F_{0}(T)\right](1+r)^{-T}
$$

Substituting terms and solving for $F_{0}(T)=\operatorname{INR} 300.84\left(=\operatorname{INR} 295(1.04)^{0.5}\right)$,

$p_{0}-\operatorname{INR} 59=(\operatorname{INR} 265-\operatorname{INR} 300.84)(1.04)^{-0.5}$.

$p_{0}=\mathrm{INR} 23.86$

VFO should expect to pay a six-month put option premium of $p_{0}=$ INR23.86.

\section{OPTION PUT-CALL PARITY APPLICATIONS: FIRM VALUE}
The insights established by the put-call parity relationship go well beyond option trading strategies, extending to modeling the value of a firm to describe the interests and financial claims of capital providers-namely, the owners of a firm's equity and the owners of its debt.

Assume that at time $t=0$, a firm with a market value of $V_{0}$ has access to borrowed capital in the form of zero-coupon debt with a face value of $D$. The market value of the firm's assets, $V_{0}$, is equal to the present value of its outstanding debt obligation, $\operatorname{PV}(D)$, and equity, $E_{0}: V_{0}=E_{0}+\operatorname{PV}(D)$.

When the debt matures at $T$, the firm's debt and assets are distributed between shareholders and debtholders with two possible outcomes depending on the firm's value at time $T\left(V_{T}\right)$ :

\begin{enumerate}
  \item Solvency $\left(V_{T}>D\right)$ : If the value of the firm $\left(V_{T}\right)$ exceeds the face value of the debt, or $V_{T}>D$, at time T, we say the firm is solvent and able to return capital to both its shareholders and debtholders.
\end{enumerate}

\begin{itemize}
  \item Debtholders receive $D$ and are repaid in full.

  \item Shareholders receive the residual: $E_{T}=V_{T}-D$.

\end{itemize}

\begin{enumerate}
  \setcounter{enumi}{1}
  \item Insolvency $\left(V_{T}<D\right)$ : If the value of the firm $\left(V_{T}\right)$ is below the face value of the debt, or $V_{T}<D$, at time $T$, we say the firm is insolvent. In the event of insolvency, shareholders receive nothing and debtholders are owed more than the value of the firm's assets. Debtholders therefore receive $V_{T}$ to settle their debt claim of $D$ at time $T$.
\end{enumerate}

\begin{itemize}
  \item Debtholders have a priority claim on assets and receive $V_{T}<D$.

  \item Shareholders receive the residual, $E_{T}=0$.

\end{itemize}

Unlike the risk-free bond shown in the prior put-call parity lesson, the firm has risky debt, because the bondholders receive $D$ only in the case of solvency (when $V_{T}$ > D). Debtholders therefore demand a premium similar to a put option premium from shareholders in order to assume the risk of insolvency.

Shareholders retain unlimited upside potential (if the firm remains solvent and can pay off its debt) and limited downside potential (if the firm becomes insolvent). In contrast, debtholder upside is limited to receiving debt repayment in the event of solvency, and principal and interest are at risk in the downside event of insolvency.

Next, we examine the respective payoff profiles at maturity more closely. At $t=T$,

\begin{itemize}
  \item shareholder payoff is $\max \left(0, V_{T}-D\right)$ and

  \item debtholder payoff is $\min \left(V_{T}, D\right)$. Consider these payoff profiles in terms of options:

  \item Shareholders hold a long position in the underlying firm's assets $\left(V_{T}\right)$ and have purchased a put option on firm value $\left(V_{T}\right)$ with an exercise price of $D$; that is, $\max \left(0, D-V_{T}\right)$.

  \item Debtholders hold a long position in a risk-free bond $(D)$ and have sold a put option to shareholders on firm value $\left(V_{T}\right)$ with an exercise price of $D$

\end{itemize}

The payoff profiles for shareholders versus debtholders is shown in Exhibit 9.

\section{Exhibit 9: Shareholder and Debtholder Payoff at Time $T$}
\section{Construction of Shareholder and Debtholder Payoffs}
\begin{center}
\includegraphics[max width=\textwidth]{2023_05_04_36535b8d80b32081d422g-367}
\end{center}

Firm Value Distribution Between Shareholders and Debtholders

\begin{center}
\includegraphics[max width=\textwidth]{2023_05_04_36535b8d80b32081d422g-367(1)}
\end{center}

Note that the shareholder's combination of a purchased put option and a long position in the firm's assets is equivalent to a call option on the firm's assets. The risky debt held by the debtholders is a combination of the risk-free debt, $D$, and the put option sold to shareholders.

Revisiting the put-call parity relationship, $S_{0}+p_{0}=c_{0}+\mathrm{PV}(X)$, from Equation 1 , we may substitute the value of the firm at time $0\left(V_{0}\right)$ for the underlying asset $\left(S_{0}\right)$, substitute the debt $(D)$ for the risk-free bond $(X)$, and solve for $V_{0}$ as follows:

$$
\begin{aligned}
& V_{0}+p_{0}=c_{0}+\mathrm{PV}(D) . \\
& V_{0}=c_{0}+\mathrm{PV}(D)-p_{0} .
\end{aligned}
$$

Equation 6 captures the value of the firm's assets at $t=0$ from a shareholder and debtholder perspective. The shareholder has a position with a payoff similar to that of a call option on firm value $\left(c_{0}\right)$. The debtholder has a position of $\operatorname{PV}(D)-p_{0}$, or risk-free debt of the firm plus a sold put option on firm value. The put option $\left(p_{0}\right)$ may be interpreted as the credit spread on the firm's debt, or the premium above the risk-free rate the firm must pay to debtholders to assume insolvency risk. This put option increases in value to shareholders as the likelihood of insolvency increases. From a debtholder's perspective, the more valuable the sold put, the more credit risk is present in the firm's debt.

\section{KNOWLEDGE CHECK}
\section{Put-Call Forward Parity and Option Applications}
\begin{enumerate}
  \item Identify the following statement as true or false, and justify your answer: A key link between put-call parity and put-call forward parity is that the cash underlying position is replaced by a synthetic underlying position using a forward purchase and a risk-free bond.
\end{enumerate}

\section{Solution:}
The statement is true. The replacement of the cash underlying position by a synthetic underlying position using a forward purchase and a risk-free bond links call option, put option, and forward prices under put-call forward parity.

\begin{enumerate}
  \setcounter{enumi}{1}
  \item Identify which of the following positions has the same no-arbitrage value as which portfolio under put-call forward parity:

  \item Long call option $\left(c_{0}\right)$

  \item Long risk-free bond $\left(X(1+r)^{-T}\right)$

  \item Long put option $\left(p_{0}\right)$ A. Forward sale, long risk-free bond, and long call option

\end{enumerate}

B. Forward purchase, long put option, and short risk-free bond

C. Long forward purchase, short call option, and long put option

\section{Solution:}
Recall that the put-call forward parity relationship may be expressed as

$$
F_{0}(T)(1+r)^{-T}+p_{0}=c_{0}+X(1+r)^{-T} \text {. }
$$

\begin{enumerate}
  \item B is correct. A long call option position is the no-arbitrage equivalent of a forward purchase, a long put option, and a short risk-free bond position.

  \item $\mathrm{C}$ is correct. A long risk-free bond position is equivalent to a long forward purchase, a short call option, and a long put option. 3. A is correct. A long put option is equivalent to a forward sale, a long risk-free bond, and a long call option.

  \item Describe the cash flows at the time of option expiration for a synthetic protective put.

\end{enumerate}

\section{Solution:}
A synthetic protective put is the combination of a synthetic underlying position (using a forward purchase and a long risk-free bond position equal to the exercise price) and a purchased put.

At the time of option expiration, time $T$, there are two possible scenarios:

Put option exercised $\left(X>S_{T}\right)$ : If the put option is exercised, an investor receives $\left(X-S_{T}\right)$. The risk-free asset returns $\mathrm{F}_{0}(\mathrm{~T})$, and the forward purchase returns $S_{T}-F_{0}(T)$

The total return equals $\left(X-S_{T}\right)+\left[S_{T}-F_{0}(T)\right]+F_{0}(T)=X$

Put option unexercised $\left(X \leq S_{T}\right)$ : If the put option is unexercised, it will expire worthless. The risk-free asset returns $F_{0}(T)$, and the forward purchase returns $S_{T}-F_{0}(T)$

The total return equals $\left[S_{T}-F_{0}(T)\right]+\mathrm{F}_{0}(T)=S_{T}$

\begin{enumerate}
  \setcounter{enumi}{3}
  \item Describe how a debtholder's position may be considered similar to the sale of a put option on firm value.
\end{enumerate}

\section{Solution:}
If the value of the firm $\left(V_{T}\right)$ is below the face value of its debt outstanding, or $V_{T}<D$ at time $T$, we say the firm is insolvent and debtholders receive less than the face value $(D)$ to settle their debt claim. Stated differently, a debtholder's payoff is $\min \left(D, V_{T}\right)=D-\max \left(0, D-V_{T}\right)$ and equals the debt face value $(D)$ minus a put option on firm value $\left(V_{T}\right)$ with an exercise price of $D$, which represents a sold put on firm value.

\section{LEARNING MODULE
10}
\section{Valuing a Derivative Using a One-Period Binomial Model}
\section{LEARNING OUTCOME}
\begin{center}
\begin{tabular}{c|l}
Mastery & The candidate should be able to: \\
\hline
$\square$ & explain how to value a derivative using a one-period binomial model \\
$\square$ & describe the concept of risk neutrality in derivatives pricing \\
\end{tabular}
\end{center}

\section{INTRODUCTION}
Earlier lessons explained how the principle of no arbitrage and replication can be used to value and price derivatives. The put-call parity relationship linked put option, call option, underlying asset, and risk-free asset prices. This relationship was extended to forward contracts given the equivalence of an underlying asset position and a long forward contract plus a risk-free bond.

Forward commitments can be priced without making assumptions about the underlying asset's price in the future. However, the pricing of options and other contingent claims requires a model for the evolution of the underlying asset's future price. The first lesson introduces the widely-used binomial model to value European put and call options. A simple one-period version is introduced, which may be extended to multiple periods and used to value more complex contingent claims. In the second lesson, we demonstrate the use of risk-neutral probabilities in derivatives pricing.

\section{Summary}
\begin{itemize}
  \item The one-period binomial model values contingent claims, such as options, and assumes the underlying asset will either increase by $R^{u}$ (up gross return) or decrease by $R^{d}$ (down gross return) over a single period that corresponds to the expiration of the derivative contract.

  \item The binomial model combines an option with the underlying asset to create a risk-free portfolio where the proportion of the option to the underlying security is determined by a hedge ratio.

  \item The hedged portfolio must earn the prevailing risk-free rate of return; otherwise, riskless arbitrage profit opportunities would be available. - Valuing a derivative via risk-free hedging is equivalent to computing the discounted expected payoff of the option using risk-neutral probabilities rather than actual probabilities.

  \item Neither the actual (real-world) probabilities of underlying price increases or decreases nor the expected return of the underlying are required to price an option.

  \item The one-period binomial model can be extended to multiple periods as well to value more complex contingent claims.

\end{itemize}

\section{LEARNING MODULE PRE-TEST}
\begin{enumerate}
  \item Describe how contingent claims, such as options, can be priced.
\end{enumerate}

\section{Solution:}
Unlike forward commitments, contingent claims, such as options, require that we model the future price behavior of the underlying asset because unlike forwards and futures, options have asymmetric payoffs. By modeling the future price behavior, the option and its underlying asset can be combined into a risk-free portfolio. The cost of this portfolio, where the proportion of the option and the underlying asset is set by a hedge ratio, determines the no-arbitrage price of the option.

\begin{enumerate}
  \setcounter{enumi}{1}
  \item Determine the correct answers to fill in the blanks: The law of can be used to value a derivative security since there is a one-to-one relationship between the derivative and its at maturity.
\end{enumerate}

\section{Solution:}
The law of one price can be used to value a derivative security since there is a one-to-one relationship between the derivative and its underlying asset at maturity.

\begin{enumerate}
  \setcounter{enumi}{2}
  \item When using a one-period binomial model to price a call option, an increase in the actual probability of an upward move in the underlying asset will result in the call option price:
\end{enumerate}

A. decreasing.

B. staying the same.

C. increasing.

\section{Solution:}
The correct answer is B. The call option price will stay the same. The actual (real-world) probabilities of an up or a down price movement in a binomial model do not influence the (no-arbitrage) price of an option.

\begin{enumerate}
  \setcounter{enumi}{3}
  \item Identify which of the various factor changes has which effect on the no-arbitrage price of a put option based on the one-period binomial model:

  \item The probability of an upward price movement, $q$, increases.

  \item The spread between the up and down factor, $R^{u}-R^{d}$, increases.

\end{enumerate}

\begin{abstract}
A. Put option price remains the
\end{abstract}

B. Put option price increases 3. The risk-neutral probability of price, $\quad$ C. Put option price decreases

$\pi$, increases.

\section{Solution:}
\begin{enumerate}
  \item The correct answer is A. The probability of an upward price movement, $q$, has no impact on value in the one-period binomial option pricing model. Thus, this change would not have any impact on the price of a put option, and the price of the put option would remain the same.

  \item The correct answer is B. The spread between the up and down factor, $R^{u}$ $-R^{d}$, increases the range of potential prices, which increases the likelihood that the option ends up in the money. Thus, this change would increase the price of a put option.

  \item The correct answer is $C$. The risk-neutral probability of price, $\pi$, captures the probability of the price of the underlying increasing. As $\pi$ increases, the likelihood of the put option ending up in the money decreases.

  \item A one-period binomial model assumes that the price of the underlying asset can change from $\$ 16.00$ today to either $\$ 20.00$ or $\$ 12.00$ at the end of the period. If the risk-free rate of return over the period is $5 \%$, what is the risk-neutral probability of a price increase?

\end{enumerate}

\section{Solution:}
An increase from $\$ 16.00$ to $\$ 20.00$ or a decrease from $\$ 16.00$ to $\$ 12.00$ corresponds to:

$$
R^{u}=\$ 20.00 / \$ 16.00=1.25 \text { and } R^{d}=\$ 12.00 / \$ 16.00=0.75
$$

Using the risk-neutral probability $(\pi)$ of a price increase:

$$
\begin{aligned}
& \pi=\left(1+r-R^{d}\right) /\left(R^{u}-R^{d}\right) \\
& =(1+0.05-0.75) /(1.25-0.75)=0.3 / 0.5=0.60
\end{aligned}
$$

\begin{enumerate}
  \setcounter{enumi}{5}
  \item Describe how a lower risk-free rate will affect the value of a put option.
\end{enumerate}

\section{Solution:}
Lower interest rates increase the value of a put option. A lower risk-free rate will reduce the risk-neutral probability of a price increase, $\pi$, and increase the present value of the expected option payoff. Since the value of a put option is inversely related to the price of the underlying asset, a decreased probability of an upward price move will increase the expected payoff from the put. Consequently, both effects will increase the put option value as the risk-free rate falls.

\section{BINOMIAL VALUATION}
$\square \quad$ explain how to value a derivative using a one-period binomial model The law of one price states that if the payoffs from any two assets (or portfolio of assets) at a given future time are identical in all possible scenarios, then the value of these two assets must also be identical today. Forward commitments offer symmetric payoffs at a predetermined price in the future, the value of which are independent of the future price behavior of the underlying asset.

The asymmetric payoff profile of options and other contingent claims makes valuation of these instruments more challenging. Assumptions about future prices are an important component in option valuation given the different payoffs under different scenarios whose likelihood changes over time. Option valuation therefore requires the specification of a model for the future (random) price behavior of the option's underlying asset.

The binomial model is a common tool used to determine the no-arbitrage value of an option. The simplicity of this model makes it attractive, as we only need to make an assumption about the magnitude of the potential upward and downward price changes of the underlying asset in a future time period.

\section{THE BINOMIAL MODEL}
The binomial model builds on a simple idea: Over a given period of time, the asset's price will either go up (u) to $S_{1}{ }^{u}>S_{0}$ or go down $(d)$ to $S_{1}{ }^{d}<S_{0}$. We do not need to know the future price in advance, because it is determined by the outcome of a random variable. The movement from $S_{0}$ to either $S_{1}{ }^{u}$ or $S_{1}{ }^{d}$ can be interpreted as the outcome of a Bernoulli trial.

Let us denote $q$ as the probability of an upward price movement and $1-q$ as the probability of a downward price movement. With only two possible outcomes-the price either goes up or down-the sum of probabilities must equal 1 . We will also find it useful to define the gross return from an up or a down price move as:

$$
\begin{aligned}
& R^{u}=S_{1}{ }^{u} / S_{0}>1 \\
& R^{d}=S_{1}{ }^{d} / S_{0}<1
\end{aligned}
$$

At first glance, it would appear that knowing $q$ is crucial in determining the value of an option on any underlying asset. However, knowing $q$ is not required; only specifying the values of $S_{1}{ }^{u}$ and $S_{1}{ }^{d}$ is needed. The difference between $S_{1}{ }^{u}$ and $S_{1}{ }^{d}$ measures the "spread" of possible future price outcomes. Specifying $S_{1}{ }^{u}$ and $S_{1}{ }^{d}$ (or $R^{u} S_{0}$ and $R^{d} S_{0}$ ) determines the volatility of the underlying asset, an important factor in valuing options as shown earlier. Simply stated, the size of the up and down price movements should match the underlying asset volatility, as shown in Exhibit 1.

\section{Exhibit 1: Price Movement for the Underlying Asset}
\begin{center}
\includegraphics[max width=\textwidth]{2023_05_04_36535b8d80b32081d422g-375}
\end{center}

Binomial models may be extended to multiple periods where the underlying asset price can move up or down in each period. This extension creates a binomial tree, a powerful way to model more realistic price dynamics. A simple one-period binomial model is sufficient to introduce the pricing methodology and the required steps in the option pricing procedure.

\section{PRICING A EUROPEAN CALL OPTION}
Consider a one-year European call option with an exercise price $(X)$ of $€ 100$. The underlying spot price $\left(S_{0}\right)$ is $€ 80$. The one-period binomial model corresponds to the option's time to expiration of one year.

The binomial model specifies the possible values of the underlying asset in one year, where the option value is a known function of the value of the underlying asset. Further, we assume that $S_{1}{ }^{d}<X<S_{1}$ u (i.e., the exercise price of the option $[X=€ 100]$ is between the value of the underlying in the two scenarios)-for example, setting $S_{1} d$ $=€ 60$ and $S_{1}{ }^{\mathrm{u}}=€ 110$. Then, $R^{u}=1.375$ and $R^{d}=0.75$. The value of the call option is as follows:

\begin{itemize}
  \item At $t=0$, the call option value is $c_{0 .}$

  \item At $t=1$, the call option value is either $c_{1}^{u}$ (if the underlying price rises to $\left.S_{1}^{\mathrm{u}}\right)$ or $c_{1}^{d}$ (if the underlying price falls to $S_{1}{ }^{d}$ ).

  \item Up move to $S_{1}$ : Call option ends up in-the-money

\end{itemize}

$c_{1}^{u}=\operatorname{Max}\left(0, S_{1}^{u}-X\right)=\operatorname{Max}(0, € 110-€ 100)=€ 10$

\begin{itemize}
  \item Down move to $S_{1}{ }^{d}$ :Call option expires out-of-the-money
\end{itemize}

$c_{1}^{d}=\operatorname{Max}\left(0, S_{1}^{d}-X\right)=\operatorname{Max}(0, € 60-€ 100)=0$

Exhibit 2 summarizes the one-period binomial model and the value of the underlying asset and call option.

\section{Exhibit 2: One-Period Binomial Option Pricing}
\begin{center}
\includegraphics[max width=\textwidth]{2023_05_04_36535b8d80b32081d422g-376}
\end{center}

The only unknown value is $c_{0}$, which may be determined using replication and no-arbitrage pricing. That is, the value of both the option and its underlying asset in each future scenario may be used to construct a risk-free portfolio. For example, assume that at $t=0$ we sell the call option at a price of $c_{0}$ and purchase $h$ units of the underlying asset. Denoting the value of the portfolio as $V$, we have the following:

$$
\begin{aligned}
& V_{0}=h S_{0}-c_{0} \\
& V_{1}{ }^{u}=h S_{1}{ }^{u}-c_{1}{ }^{u}=h \times R^{u} \times S_{0}-\operatorname{Max}\left(0, S_{1}{ }^{u}-X\right) \\
& V_{1}{ }^{d}=h S_{1}{ }^{d}-c_{1}{ }^{d}=h \times R^{d} \times S_{0}-\operatorname{Max}\left(0, S_{1}{ }^{d}-X\right)
\end{aligned}
$$

$V_{0}$ represents the initial portfolio investment in the portfolio, and $V_{1}{ }^{u}$ and $V_{1}{ }^{d}$ represent the portfolio value if the underlying price moves up or down, respectively. We must choose $h$ so that $V_{1}{ }^{u}=V_{1}{ }^{d}$ (i.e., the portfolio value is the same in either scenario). The value of the combination, $V_{1}$, will not change if the underlying asset price changes. For portfolio $V_{0}$, the impacts of the changes are as follows:

\begin{itemize}
  \item Up move from $S_{0}$ to $S_{1}{ }^{u}$ by $R^{u}$ :

  \item The asset value changes from $€ 80$ to $€ 110$ by 1.375 , or $37.5 \%$.

  \item The call option ends up in-the-money:

\end{itemize}

$c_{1}^{u}=\operatorname{Max}\left(0, S_{1}^{u}-X\right)=\operatorname{Max}(0, € 110-€ 100)=€ 10$.

\begin{itemize}
  \item Total portfolio value: $V_{1}{ }^{u}=h S_{1}{ }^{u}-c_{1}{ }^{u}=€ 110 \times h-€ 10$.

  \item Down move from $S_{0}$ to $S_{1}{ }^{d}$ by $R^{d}$ :

  \item The asset value changes from $€ 80$ to $€ 60$ by 0.75 , or $-25 \%$.

  \item The call option expires out-of-the-money:

\end{itemize}

$c_{1}^{d}=\operatorname{Max}\left(0, S_{1}^{d}-X\right)=\operatorname{Max}(0, € 60-€ 100)=0$.

\begin{itemize}
  \item Total portfolio value: $V_{1}{ }^{d}=h S_{1}{ }^{d}-c_{1}{ }^{d}=€ 60 \times h$.
\end{itemize}

Since we create two portfolios at time $t=0$ with identical payoffs at option expiry at time $t=1$, we must solve for $h$, the ratio between the underlying asset, $S_{0}$, and the call option, $c_{0}$, such that $V_{1}{ }^{u}=V_{1}{ }^{d}$, or $h S_{1}{ }^{u}-c_{1}{ }^{u}=h S_{1}{ }^{d}-c_{1}{ }^{d}$. Solving for $h^{*}$ yields:

$$
h^{*}=\frac{c_{1}^{u}-c_{1}^{d}}{s_{1}^{u}-s_{1}^{d}},
$$

where all quantities on the right-hand side are known at $t=0$. Equation 4 gives us the hedge ratio of the option, or the proportion of the underlying that will offset the risk associated with an option. In our sold call option example,

$$
h^{*}=\frac{c_{1}^{u}-c_{1}^{d}}{s_{1}^{u}-s_{1}^{d}}=\frac{€ 10-€ 0}{€ 110-€ 60}=\frac{10}{50}=0.20
$$

For each call option unit sold, we buy 0.2 units of the underlying asset (or for each underlying asset unit, we must sell 5 call options to equate the portfolio values at $t=$ 1). Consider the two scenarios as follows:

\begin{itemize}
  \item Up move from $S_{0}$ to $S_{1}^{u}$ by $R^{u}$ :

  \item Total portfolio value: $V_{1}^{u}=h S_{1}^{u}-c_{1}^{u}$

\end{itemize}

$=€ 110 \times 0.20-€ 10=€ 22-10=€ 12$.

\begin{itemize}
  \item Down move from $S_{0}$ to $S_{1}^{d}$ by $R^{d}$ :

  \item Total portfolio value $V_{1}^{d}=h S_{1}^{d}-c_{1}{ }^{d}=€ 60 \times 0.20=€ 12$.

\end{itemize}

The portfolio values are the same, $V_{1}^{u}=V_{1}^{d}$, which has two implications:

\begin{enumerate}
  \item We can use either portfolio to value the derivative, and

  \item The return $V_{1}^{u} / V_{0}=V_{1}^{d} / V_{0}$ must equal one plus the risk-free rate.

\end{enumerate}

Exhibit 3 summarizes these results.

\section{Exhibit 3: Value of the Hedged Portfolio}
\begin{center}
\includegraphics[max width=\textwidth]{2023_05_04_36535b8d80b32081d422g-377}
\end{center}

This hedging approach is not specific to call options and can be used for any derivative contract whose value is entirely determined by the underlying asset's value at $t=$ 1. Using the example of a call option, Exhibit 4 summarizes the construction of the risk-free portfolio consisting of the underlying asset, $S$, and the sold call option, $c$. Exhibit 4: Binomial Model - Riskless Hedge Portfolio

Combined portfolio: Underlying, $S$, and derivative, $c$

\begin{center}
\includegraphics[max width=\textwidth]{2023_05_04_36535b8d80b32081d422g-378}
\end{center}

Riskless portfolio: $h$ of the underlying, $S$, and short derivative position, $c$

\begin{center}
\includegraphics[max width=\textwidth]{2023_05_04_36535b8d80b32081d422g-378(1)}
\end{center}

To prevent arbitrage, the portfolio value at $t=1$,

$V_{1}=h \times R^{u} S_{0}-c_{1}^{u}=h \times R^{d} S_{0}-c_{1}^{d}$,

should be discounted at the risk-free rate so that:

$h S_{0}-c_{0}=\frac{V_{1}}{1+r}$

Based on no-arbitrage pricing and the certain portfolio payoff, $V_{1}$, the value of the call option may be shown as:

$$
c_{0}=h \times S_{0}-V_{1}(1+r)^{-1} .
$$

Substituting the information from the previous example, we already established that $V_{1}=€ 12$. For an annual risk-free rate of $5 \%$ :

$$
\begin{aligned}
& c_{0}=h \times S_{0}-V_{1}(1+\mathrm{r})^{-1} \\
& =0.2 \times € 80-€ 12(1.05)^{-1}=€ 16-€ 11.43=€ 4.57 .
\end{aligned}
$$

The call option price $c_{0}$ is therefore $€ 4.57$. We may confirm an investor in the hedge portfolio would earn the risk-free rate $(r)$ of $5 \%$ by comparing the initial portfolio value $V_{0}$ with the portfolio value after one period of $V_{1}\left(\right.$ recalling that $\left.V_{1}^{u}=V_{1}^{d}\right)$ equal to $€ 12$ :

$$
\begin{aligned}
& V_{0}=h \times S_{0}-c_{0} \\
& V_{0}=€ 11.43=€ 16-€ 4.57 \\
& V_{1}=V_{0}(1+r) \\
& V_{1}=€ 12=€ 11.43
\end{aligned}
$$

\section{EXAMPLE 1}
\section{Hightest Capital}
Hightest Capital believes that a particular non-dividend-paying stock is currently trading at $\$ 50$ and is considering the sale of a one-year European call option at an exercise price of $\$ 55$. Answer the following questions:

\begin{enumerate}
  \item If the stock price is expected to either go up or down by $20 \%$ over the next year, what price should Hightest expect to receive for the sold call option? Assume a risk-free rate of $5 \%$.

  \item How would the call option price change if the stock price were expected to either go up or down by $40 \%$ over the next year?

  \item If Hightest had a more optimistic outlook on the future stock price (i.e., they estimated a higher probability of the option ending up in-the-money), how would the expected call option price change?

  \item What would be the price of a one-year put option at an exercise price of $\$ 55$ if the stock price were expected to change by $20 \% ? 40 \%$ ?

\end{enumerate}

\section{Solution to 1:}
Denote the initial price of the underlying stock as $S_{0}=\$ 50$ and the exercise price of the call option as $X=\$ 55$. If the stock price moves up or down by $20 \%$, then:

$$
\begin{aligned}
& S_{1}^{u}=R^{u} S_{0}=1.2 \times \$ 50=\$ 60 . \\
& S_{1}^{d}=R^{d} S_{0}=0.8 \times \$ 50=\$ 40 .
\end{aligned}
$$

The price will either move up to $\$ 60$ or down to $\$ 40$. Given that the payoff of a call option at expiry is $\operatorname{Max}\left(0, S_{T}-X\right)$ :

$$
\begin{aligned}
& c_{1}^{u}=\operatorname{Max}\left(0, S_{1}^{u}-X\right)=\operatorname{Max}(0, \$ 60-\$ 55)=\$ 5 \\
& c_{1}^{d}=\operatorname{Max}\left(0, S_{1}^{d}-X\right)=\operatorname{Max}(0, \$ 40-\$ 55)=\$ 0 .
\end{aligned}
$$

The call option value is $\$ 5$ if the underlying stock increases in price by $20 \%$ and zero if it decreases by $20 \%$. The hedge ratio of the option is:

$$
h^{*}=\frac{\left(c_{1}^{u}-c_{1}^{d}\right)}{\left(S_{1}^{u}-S_{1}^{d}\right)}=\frac{\$ 5.00-\$ 0}{\$ 60.00-\$ 40.00}=\frac{\$ 5.00}{\$ 20.00}=0.25
$$

To create a risk-free portfolio, we can sell a call option and purchase 0.25 units of the underlying asset today. At maturity, the hedged portfolio value is:

$$
\begin{aligned}
& V_{1}=V_{1}^{u}=h^{*} S_{1}^{u}-c_{1}^{u}=0.25 \times \$ 60.00-\$ 5.00=\$ 10.00 . \\
& V_{1}=V_{1}^{d}=h^{*} S_{1}^{d}-c_{1}^{d}=0.25 \times \$ 40.00-\$ 0.00=\$ 10.00 .
\end{aligned}
$$

We can use either $V_{1}{ }^{u}$ or $V_{1}{ }^{d}$ to compute the certain value, $V_{1}$, and the present value of the hedged position today is:

$$
V_{0}=V_{1}(1+r)^{-1}=\$ 10.00 \times 1.05^{-1}=\$ 9.52 .
$$

The call option value is:

$$
c_{0}=h^{*} S_{0}-V_{0}=0.25 \times \$ 50.00-\$ 9.52=\$ 12.50-\$ 9.52=\$ 2.98 .
$$

The no-arbitrage price of the call option should be $\$ 2.98$. If not, then investors would be able to construct a synthetic risk-free asset using the option and its underlying asset with a higher return than the risk-free rate.

Note that the hedge ratio ( 0.25 units in this example) is positive in this case since the derivative is sold and the underlying is purchased. A negative hedge ratio implies that both the derivative and the underlying are purchased or sold to create the hedge. Also, while the hedge ratio is usually expressed as a fraction of an underlying unit, option contracts are usually traded in larger size, allowing a round number of underlying assets to be purchased or sold as a hedge. It is important that the ratio (4:1 in this case) of options to underlying units is maintained in the portfolio.

\section{Solution to 2:}
If the stock price changes by $40 \%$, then the call option payoff at expiry is:

$$
\begin{aligned}
& c_{1}^{u}=\operatorname{Max}\left(0, S_{1}^{u}-X\right)=\operatorname{Max}(0, \$ 70-\$ 55)=\$ 15 . \\
& c_{1}^{d}=\operatorname{Max}\left(0, S_{1}^{d}-X\right)=\operatorname{Max}(0, \$ 30-\$ 55)=0 .
\end{aligned}
$$

The hedge ratio of the option is:

$$
h^{*}=\frac{\left(c_{1}^{u}-c_{1}^{d}\right)}{\left(S_{1}^{u}-S_{1}^{d}\right)}=\frac{\$ 15-\$ 0}{\$ 70-\$ 30}=\frac{\$ 15}{\$ 40}=0.375 .
$$

At maturity, the perfectly hedged portfolio is worth:

$$
\begin{aligned}
& V_{1}=V_{1}^{u}=h^{*} S_{1}^{u}-c_{1}^{u}=0.375 \times \$ 70-\$ 15=\$ 11.25 . \\
& V_{1}=V_{1}^{d}=h^{*} S_{1}^{d}-c_{1}^{d}=0.375 \times \$ 30-\$ 0=\$ 11.25 .
\end{aligned}
$$

The present value of the hedged position at $t=0$ is:

$$
V_{0}=V_{1}(1+\mathrm{r})^{-1}=\$ 11.25 \times 1.05^{-1}=\$ 10.71 .
$$

Then, the call option price at $t=0$ is:

$$
c_{0}=h^{*} S_{0}-V_{0}=0.375 \times \$ 50-\$ 10.71=\$ 18.75-\$ 10.71=\$ 8.04 .
$$

A higher expected price change indicates higher volatility. An increase in the range of future price changes increases the value of the call option.

\section{Solution to 3:}
Since the actual probabilities of an up or a down move in the underlying asset do not affect the no-arbitrage value of the option, the option price that Hightest may charge should not change. Hightest can offset the risk of selling the call by purchasing $h^{*}$ units of the underlying asset, so any directional views on the stock price do not affect the hedge position.

\section{Solution to 4:}
Using put-call parity, $c_{0}-p_{0}=S_{0}-X(1+r)^{-T}$, and rearranging terms to solve for the put price, when the stock price changes by $20 \%$ and the call option price is $\$ 2.98$, the result can be calculated as:

$$
p_{0}=c_{0}-S_{0}+X(1+r)^{-T}=\$ 2.98-\$ 50+\$ 55(1+0.05)^{-1}=\$ 5.36 .
$$

When the underlying stock price changes by $40 \%$ and the call option price is $\$ 8.04$, the put option price is:

$$
p_{0}=c_{0}-S_{0}+X(1+r)^{-T}=\$ 8.04-\$ 50+\$ 55(1+0.05)^{-1}=\$ 10.42
$$

\section{KNOWLEDGE CHECK}
\section{Binomial Valuation of Options}
\begin{enumerate}
  \item Determine the correct answers to fill in the blanks: To price a contingent claim, such as an option, a model for the of the underlying asset is needed due to the nature of the contract's payoff.
\end{enumerate}

\section{Solution:}
To price a contingent claim, such as an option, a model for the future price behavior of the underlying asset is needed due to the asymmetric nature of the contract's payoff.

\begin{enumerate}
  \setcounter{enumi}{1}
  \item Describe the main difference between pricing a contingent claim and pricing a forward commitment.
\end{enumerate}

\section{Solution:}
The symmetric nature of a forward commitment's payoff (i.e., the obligation to transact at maturity) allows the commitment to be perfectly replicated without the need to model the future price behavior of the underlying asset. However, the asymmetric nature of a contingent claim's payoff (i.e., the right but not the obligation to transact at maturity) does require the future price behavior to be modeled.

\begin{enumerate}
  \setcounter{enumi}{2}
  \item If a one-period binomial model is used to price an at-the-money put option, which of the following statements is most accurate? The option will be:
\end{enumerate}

A. in-the-money if the price moves up.

B. out-of-the-money if the price moves up.

C. at-the-money if the price moves up or down.

\section{Solution:}
The correct answer is B; the option will be out-of-the-money if the price moves up. An at-the-money put option has an exercise price equal to the underlying asset price. Therefore, a price decrease will result in the put option moving in the money, and a price increase will result in the put option moving out of the money.

\begin{enumerate}
  \setcounter{enumi}{3}
  \item Explain how increasing the up gross return, $R^{u}$, and/or decreasing the down gross return, $R^{d}$, in a one-period binomial model would influence the price of a call and a put option.
\end{enumerate}

\section{Solution:}
In a one-period binomial model, the volatility of the underlying asset is represented by the spread between the up gross return, $R^{u}$, and the down gross return, $R^{d}$. Therefore, if either the up gross return increases or the down gross return decreases (or both), the price of the underlying asset at matu- rity will be more volatile. If all else remains equal, then the price of both call and put options will increase when the underlying asset is expected to have a higher volatility over the life of the option.

\begin{enumerate}
  \setcounter{enumi}{4}
  \item A put option on a non-dividend-paying stock has an exercise price, $X$, of $\pounds 21$ and six months left to maturity. The current stock price, $S_{0}$, is $\pounds 20$, and an investor believes that the stock's price in six months' time will be either $10 \%$ higher or $10 \%$ lower.
\end{enumerate}

a. Describe how the investor can construct a perfectly hedged portfolio using the put option and its underlying stock.

b. What will the value of the hedged portfolio be in the scenario that the stock price rises and the scenario that the stock price falls (assume a risk-free rate is $4 \%)$ ?

c. What is the no-arbitrage price of the put option?

\section{Solution:}
a. Denote the initial price of the underlying stock as $S_{0}=\pounds 20.00$ and the exercise price of the put option as $X=\pounds 21$.00. If the stock price moves up by $10 \%$, then:

$$
S_{1}^{u}=R^{u} S_{0}=1.1 \times \pounds 20.00=\pounds 22.00 .
$$

If the stock price moves down by $10 \%$, then

$$
S_{1}^{d}=R^{d} S_{0}=0.9 \times \pounds 20.00=\pounds 18.00 .
$$

Given that the payoff of a put option at expiry is $\operatorname{Max}\left(0, X-S_{T}\right)$ :

$$
\begin{aligned}
& p_{1}^{u}=\operatorname{Max}\left(0, X-S_{1}^{u}\right)=\operatorname{Max}(0, \pounds 21.00-\pounds 22.00)=\pounds 0 . \\
& p_{1}^{d}=\operatorname{Max}\left(0, X-S_{1}^{d}\right)=\operatorname{Max}(0, \pounds 21.00-\pounds 18.00)=\pounds 3.00 .
\end{aligned}
$$

The put option will be worth $\pounds 3.00$ if the underlying stock decreases in price by $10 \%$ and worthless if it increases by $10 \%$. The hedge ratio of the option is:

$$
h^{*}=\frac{\left(p_{1}^{u}-p_{1}^{d}\right)}{\left(S_{1}^{u}-S_{1}^{d}\right)}=\frac{\pounds 0-\pounds 3}{\pounds 22-\pounds 18}=\frac{\pounds 3}{\pounds 4}=-0.75 .
$$

So, to create a risk-free portfolio, we can buy the put option and buy 0.75 units of the underlying asset.

b. At maturity, the value of the perfectly hedged portfolio is:

$$
\begin{aligned}
& V_{1}=V_{1}^{u}=h^{*} S_{1}^{u}+p_{1}^{u}=0.75 \times \pounds 22+\pounds 0=\pounds 16.50 . \\
& V_{1}=V_{1}^{d}=h^{*} S_{1}^{d}+p_{1}^{d}=0.75 \times \pounds 18+\pounds 3.00=\pounds 16.50 .
\end{aligned}
$$

We can use either $V_{1}^{u}$ or $V_{1}{ }^{d}$ to compute the certain value, $V_{1}$, and the present value of the hedged position today is:

$$
V_{0}=V_{1}(1+r)^{-1}=\pounds 16.50 \times 1.04^{-0.5}=\pounds 16.18 \text {. }
$$

c. The price of the put option is:

$$
p_{0}=V_{0}-h^{*} S_{0}-=\pounds 16.18-0.75 \times \pounds 20=\pounds 16.18-\pounds 15=\pounds 1.18 .
$$

The no-arbitrage price of the put option should be $\pounds 1.18$. 6. Determine the correct answers to fill in the blanks: When the applies, the rate of return on all (real or synthetic)

risk-free assets should equal the

\section{Solution:}
When the law of one price applies, the rate of return on all (real or synthetic) risk-free assets should equal the risk-free rate.

\section{RISK NEUTRALITY}
describe the concept of risk neutrality in derivatives pricing

As shown in the prior lesson, an option's value is not affected by actual (real-world) probabilities of underlying price increases or decreases. This realization contributed to the growing use of options as it became easier to agree on a given option's value. As we will see in this lesson, only the expected volatility-that is, gross returns $R^{u}$ and $R^{d}$ introduced earlier-and not the expected return are required to price an option.

We may generalize the relationship between an option's value and that of the hedge portfolios from Equation 5 of the prior lesson using the example of a call option $c_{0}$ as follows:

$$
c_{0}=\frac{\left(\pi c_{1}^{u}+(1-\pi) c_{1}^{d}\right)}{(1+r)^{T}}
$$

The value of the call option today, $c_{0}$, is computed as the expected value of the option at expiration, $c_{1}{ }^{u}$ and $c_{1}{ }^{d}$, discounted at the risk-free rate, $r$. This shows the derivative's value as similar to any other asset in that it equals the present value of expected future cash flows. In this case, these cash flows are weighted by assumed probabilities that are consistent with risk-neutral returns on the underlying.

The risk-neutral probability $(\pi)$ is the computed probability used in binomial option pricing by which the discounted weighted sum of expected values of the underlying, $S_{1}{ }^{u}=R^{u} S_{0}$ and $S_{1}{ }^{d}=R^{d} S_{0}$, equal the current option price. Specifically, this probability is computed using the risk-free rate and assumed up gross return and down gross return of the underlying as in Equation 7.

$$
\pi=\frac{1+r-R^{d}}{R^{u}-R^{d}}
$$

More specifically, $\pi$, is the risk-neutral probability of an increase in the underlying price to $S_{1}{ }^{u}=R^{u} S_{0}$, and $(1-\pi)$ is that of a decrease, $S_{1}{ }^{d}=R^{d} S_{0}$.

Substituting the details from our earlier example, where $R^{u} S_{0}=€ 110, R^{d} S_{0}=€ 60$, $S_{0}=€ 80$, and $X=€ 100, R^{u}=1.375$ and $R^{d}=0.75$ and assuming an annual risk-free rate of $5 \%$ :

$$
\begin{aligned}
& \pi=\frac{1+r-R^{d}}{R^{u}-R^{d}}=\frac{1+0.05-0.75}{1.375-0.75}=0.48 . \\
& c_{0}=\frac{\left(\pi c_{1}^{u}+(1+\pi) c_{1}^{d}\right)}{(1+r)^{T}} \\
& =[0.48 \times \operatorname{Max}(0, € 110-€ 100)+0.52 \times \operatorname{Max}(0, € 60-€ 100)] / 1.05 \\
& =€ 4.80 / 1.05=€ 4.57, \text { which matches our earlier result from the prior lesson. }
\end{aligned}
$$

In Equation 7, the risk-neutral probabilities are determined solely by the up and down gross returns, $R^{u}$ and $R^{d}$, representing underlying asset volatility and the risk-free rate used to calculate the present value of future cash flows. This no-arbitrage derivative value established separately from investor views on risk is referred to as risk-neutral pricing.

The use of risk-neutral pricing goes well beyond the simple one-period binomial tree and may be applied to any model that uses future underlying asset price movements, as we will see later in the curriculum.

\section{EXAMPLE 2}
\section{Hightest Capital (revisited)}
Revisiting Example 1 and the European call option sold by Hightest Capital, we can now explore the option price using risk-neutral pricing. Answer the following questions:

\begin{enumerate}
  \item What is the risk-neutral probability of an up move and a down move in the one-period binomial model described in Example 1?

  \item Demonstrate how this risk-neutral probability can be used to arrive at the no-arbitrage price of the call option.

  \item What would be the value of a European put option on the same stock with the same exercise price and expiration date?

  \item Confirm that the call option price computed in question 2 and the put option price computed in question 3 both satisfy the put-call parity relationship.

\end{enumerate}

\section{Solution to 1:}
The risk-neutral probability of an up move, denoted $\pi$, is:

$$
\pi=\frac{1+r-R^{d}}{R^{u}-R^{d}}=\frac{1+0.05-0.8}{1.2-0.8}=0.625 .
$$

The risk-neutral probability of a down move is therefore:

$$
1-\pi=1-0.625=0.375 .
$$

\section{Solution to 2:}
The call option price today is given by the (risk-neutral) expected value of the option payoff at maturity, discounted at the risk-free rate, $r$. In Equation 6:

$$
\begin{aligned}
& c_{0}=\frac{\left(\pi c_{1}^{u}+(1-\pi) c_{1}^{d}\right)}{(1+r)^{T}} \\
& =[0.625 \times \operatorname{Max}(0, \$ 60-\$ 55)+0.375 \times \operatorname{Max}(0, \$ 40-\$ 55)] / 1.05 \\
& =\frac{\$ 3.125}{1.05}=\$ 2.98
\end{aligned}
$$

which matches the value in Example 1.

\section{Solution to 3:}
Since Equation 6 is valid for any European option, the put option value with an exercise price of $\$ 55$ is:

$$
p_{0}=\frac{\left(\pi p_{1}^{u}+(1-\pi) p_{1}^{d}\right)}{(1+r)^{T}}
$$

$$
\begin{aligned}
& =[0.625 \times \operatorname{Max}(0, \$ 55-\$ 60)+0.375 \times \operatorname{Max}(0, \$ 55-\$ 40)] / 1.05 \\
& =\$ 5.625 / 1.05=\$ 5.36
\end{aligned}
$$

Note that we have used the same risk-neutral probability $\pi$ as in Questions 1 and 2 , since these values are a function of the binomial model for the underlying asset for a given asset volatility and risk-free rate, not the specific option being priced.

\section{Solution to 4:}
European call and put options with the same exercise price and maturity date must satisfy put-call parity as defined by:

$$
S_{0}+p_{0}=c_{0}+X(1+r)^{-T}=\$ 50+\$ 5.36=\$ 2.98+\$ 55 \times 1.05^{-1}=\$ 55.36,
$$

which confirms the relationship.

\section{KNOWLEDGE CHECK}
\section{Hedging and Risk Neutrality}
\begin{enumerate}
  \item Which of the following factors influences the value of an option price when using a binomial model?
A. The risk-free rate of return
B. The level of investors' risk aversion
C. The probability of an upward price move
\end{enumerate}

\section{Solution:}
The correct answer is A, the risk-free rate of return. The value of an option is determined by its risk-neutral expectation discounted at the risk-free rate. In a one-period binomial model, the risk-neutral probabilities are determined only by the risk-free rate over the life of the option and the underlying asset's volatility (as measured by the up and down gross returns, $R^{u}$ and $R^{d}$ ). Because of the ability to construct a perfect hedge of the option using the underlying asset, an option's price is independent of investors' risk aversion and the probability of the underlying price moving up (or down).

\begin{enumerate}
  \setcounter{enumi}{1}
  \item If the underlying asset price in a one-period binomial model can increase by $15 \%$ or decrease by $10 \%$ over the period and the prevailing risk-free rate over the period is $4 \%$, what is the risk-neutral probability of an asset pricing decrease?
\end{enumerate}

\section{Solution:}
If the underlying asset price in a one-period binomial model can increase by $15 \%$, then the up gross return is $R^{u}=1.15$. Similarly, if the price can decrease by $10 \%$, then the down gross return is $R^{d}=0.9$.

The risk-neutral probability of an upward price move is:

$\pi=\frac{1+r-R^{d}}{R^{u}-R^{d}}=\frac{1+0.04-0.9}{1.15-0.9}=0.56$

The risk-neutral probability of a downward price move is $1-0.56=0.44$. 3. Which of the following best describes the risk-neutral pricing interpretation of the one-period binomial option pricing formula?

A. The real-world expected payoff discounted at the risk-free rate

B. The risk-neutral expected payoff discounted at the risk-free rate

C. The risk-neutral expected payoff discounted at a risk-adjusted rate

\section{Solution:}
The correct answer is B, the risk-neutral expected payoff discounted at the risk-free rate. The risk-neutral pricing interpretation of the option pricing formula states that the value of an option today is its risk-neutral expected value at maturity, discounted at the risk-free rate.

\begin{enumerate}
  \setcounter{enumi}{3}
  \item Determine the correct answers to fill in the blanks: If a call option is trading at a higher price than that implied from the binomial model, investors can earn a return in excess of the risk-free rate by at the risk-free rate, the call, and the underlying.
\end{enumerate}

\section{Solution:}
If a call option is trading at a higher price than that implied from the binomial model, investors can earn a return in excess of the risk-free rate by borrowing at the risk-free rate, selling the call, and buying the underlying. A synthetic risk-free asset can be created by this strategy that earns a return higher than the risk-free rate. Selling the over-priced call will provide a higher cash inflow than is required to generate the risk-free rate of return.

\begin{enumerate}
  \setcounter{enumi}{4}
  \item A stock's price is currently $¥ 8,000$. At the end of one month when its options expire, the stock price is either up by $5 \%$ or down by $15 \%$. If the risk-free rate is $-0.20 \%$ for the period, what is the value of a put option with a strike price of $¥ 7,950$ ?
A. $¥ 333.67$
B. $¥ 299.60$
C. $¥ 236.93$
\end{enumerate}

\section{Solution:}
The correct answer is B, ¥299.60. Using risk-neutral pricing, we can determine the risk-neutral probability as:

$$
\pi=\frac{1+r-R^{d}}{R^{u}-R^{d}}=\frac{1-0.002-0.85}{1.05-0.85}=0.74
$$

The risk-neutral probability of a down move is therefore $1-\pi=1-0.74=$ 0.26 . The value of a put option with an exercise price of $¥ 7,950$ is:

$$
\begin{aligned}
& p_{0}=\frac{\left(\pi p_{1}^{u}+(1-\pi) p_{1}^{d}\right)}{(1+r)^{T}} \\
& =[0.74 \times \operatorname{Max}(0, ¥ 7,950-¥ 8,400)+0.26 \times \operatorname{Max}(0, ¥ 7,950-¥ 6,800)] /(1 \\
& -0.002) \\
& =[0.26 \times ¥ 1,150] / 0.998=¥ 299.60 .
\end{aligned}
$$

\section{PRACTICE PROBLEMS}
Privatbank Kleinert KGaA, a private wealth manager in Munich, has a number of clients with large holdings in the German fintech firm SparCoin AG. Kleinert's analyst is concerned about a drop in SparCoin's share price in the next year and is recommending to clients that they consider purchasing a one-year put with an exercise price of $€ 100$. SparCoin's spot price $\left(S_{0}\right)$ is $€ 105.25$, and it pays no dividends. The risk-free rate is $0.37 \%$.

\begin{enumerate}
  \item Kleinert's analyst estimates a $50-50$ chance that the price of SparCoin will either increase by $12 \%$ or decline by $10 \%$ at the put option's expiration date. Which of the following statements best describes the no-arbitrage option price implied by this assumption?
\end{enumerate}

A. Since there is a $50 \%$ chance that the stock will fall to $€ 94.73$, there is a $50-50$ chance of a $€ 5.27$ payout upon exercise and the no-arbitrage put is therefore worth $€ 2.64(=€ 5.27 / 2)$.

B. Since there is a $50 \%$ chance that the stock will fall to $€ 94.73$, there is a $50-50$ chance of a $€ 5.27$ payout upon exercise and given the risk-neutral probability of 0.47 , the no-arbitrage put price is $€ 2.48(=€ 5.27 \times 0.47)$.

C. Since there is a $50 \%$ chance that the stock will fall to $€ 94.73$ and the risk-neutral probability is 0.47 , the no-arbitrage put price is $€ 2.78(=€ 5.27 \times$ $\{[1-0.47] / 1.0037\})$.

\begin{enumerate}
  \setcounter{enumi}{1}
  \item If Kleinert's clients observe that the one-year put option with a $€ 100$ exercise price is trading at $€ 2.50$, which of the following statements best describes how Kleinert's clients could take advantage of this to earn a risk-free return greater than $0.37 \%$ over the year.
\end{enumerate}

A. Kleinert should purchase the put option and also purchase approximately 0.23 shares per option to match the hedge ratio.

B. Kleinert should purchase the put option and purchase $50 \%$ of the underlying shares given the 50-50 chance the stock will fall and the put option exercised.

C. Kleinert should purchase the put option and purchase $47 \%$ of the underlying shares to match the risk-neutral probability of put exercise.

\begin{enumerate}
  \setcounter{enumi}{2}
  \item If risk-free investments yielded a higher return over the next year, which of the following statements best describes how this would affect the no-arbitrage value of the put option on SparCoin shares?
\end{enumerate}

A. An increase in the risk-free rate will have no effect on SparCoin's put option price, as it is solely a function of the probability and degree of share price increase or decrease upon option expiration.

B. An increase in the risk-free rate will increase the value of the put option, as it will increase the risk-neutral probability of a price decline.

C. An increase in the risk-free rate will decrease the value of the put option, as it will both increase the risk-neutral probability of a price increase $\pi$ and decrease the present value of the expected option payoff. 4. If the expected percentage increase and decrease in SparCoin's share price were to double, which of the following is the closest estimate of the one-year put option price with an exercise price of $€ 100$ ?

A. The one-year put option price will rise to $€ 7.90$.

B. The one-year put option price will rise to $€ 8.50$.

C. The one-year put option price will rise to $€ 7.40$.

\section{SOLUTIONS}
\begin{enumerate}
  \item The correct answer is C. A $12 \%$ increase in the stock price gives:
\end{enumerate}

$S_{1}^{u}=R^{u} S_{0}=1.12 \times € 105.25=€ 117.88$.

The put option will expire unexercised:

$p_{1}^{u}=\operatorname{Max}\left(0, X-S_{1}^{u}\right)=\operatorname{Max}(0, € 100-€ 117.88)=€ 0$.

Alternatively, a $10 \%$ price decrease gives:

$S_{1}^{d}=R^{d} S_{0}=0.9 \times € 105.25=€ 94.73$.

The put option will pay off:

$p_{1}^{d}=\operatorname{Max}\left(1, X-S_{1}^{d}\right)=\operatorname{Max}(0, € 100-€ 94.73)=€ 5.27$.

To price this option, the risk-neutral pricing formula gives the risk-neutral probability $\pi$ as:

$\pi=(1+0.0037-0.9) /(1.12-0.9)=0.47$.

The no-arbitrage price is:

$p_{0}=\frac{\left(\pi \times p_{1}^{u}+(1-\pi) p_{1}^{d}\right)}{(1+r)}$

$p_{0}=(0.47 \times € 0+0.53 \times € 5.27) /(1+0.0037)=€ 2.79 / 1.0037=€ 2.78$.

\begin{enumerate}
  \setcounter{enumi}{1}
  \item The correct answer is A. If the put option can be purchased for less than the no-arbitrage price, then a potential arbitrage opportunity is available. In this case, Kleinert's clients should purchase the underpriced put option and buy $h^{*}$ units of SparCoin's stock. The hedge ratio, $h^{*}$, is calculated as:
\end{enumerate}

$h^{*}=\frac{\left(p_{1}^{u}-p_{1}^{d}\right)}{\left(S_{1}^{u}-S_{1}^{d}\right)}=\frac{€ 0-€ 5.27}{€ 117.88-€ 94.73}=\frac{-€ 5.27}{€ 23.15}=-0.2276$.

Note that the negative hedge ratio implies that both the put option and underlying are purchased or sold to create a hedge. This initial purchase of the put option and stock will cost:

$€ 2.50+0.2276 \times € 105.25=€ 26.45$

Should the stock price decrease, the value of this portfolio will be:

$V_{1}=V_{1}^{d}=h^{*} S_{1}^{d}+p_{1}^{d}=0.2276 \times € 94.73+€ 5.27=€ 26.83$.

The strategy generates a risk-free return of $(€ 26.83-€ 26.45) / € 26.45=1.44 \%$, which is greater than the $0.37 \%$ return on other available risk-free investments.

\begin{enumerate}
  \setcounter{enumi}{2}
  \item The correct answer is C. Rising interest rates reduce the value of a put option. Increasing the risk-free rate will increase the risk-neutral probability of a price increase $\pi$ and decrease the present value of the expected option payoff. Since the value of a put option is inversely related to the price of the underlying asset, an increased probability of an upward price move will reduce the expected payoff from the put. Consequently, both of these effects will reduce the put option value as the return on risk-free investments increases.

  \item The correct answer is B. A $24 \%$ (previously 12\%) increase in the stock price gives: $S_{1}^{u}=R^{u} S_{0}=1.24 \times € 105.25=€ 130.51$.

\end{enumerate}

The put option will expire unexercised:

$p_{1}^{u}=\operatorname{Max}\left(0, X-S_{1}^{u}\right)=\operatorname{Max}(0, € 100.00-€ 130.51)=€ 0$.

Alternatively, a 20\% (previously 10\%) decrease gives:

$S_{1}^{d}=R^{d} S_{0}=0.8 \times € 105.25=€ 84.20$.

The put option will pay off:

$p_{1}^{d}=\operatorname{Max}\left(0, X-S_{1}^{d}\right)=\operatorname{Max}(0, € 100-€ 84.20)=€ 15.80$.

To price this option, the risk-neutral pricing formula gives the risk-neutral probability as:

$\pi=(1+0.0037-0.8) /(1.24-0.8)=0.46$.

The no-arbitrage price is:

$p_{0}=\frac{\left(\pi \times p_{1}^{u}+(1-\pi) p_{1}^{d}\right)}{(1+r)}=\frac{(0.46 \times € 0+0.54 \times € 15.80)}{(1+0.0037)}=\frac{€ 8.53}{1.0037}=€ 8.50$.

\section{Alternative Investments}
\section{LEARNING MODULE
1}
\section{Categories, Characteristics, and Compensation Structures of Alternative Investments}
\section{LEARNING OUTCOME}
\begin{center}
\begin{tabular}{c|l}
Mastery & The candidate should be able to: \\
\hline
$\square$ & $\begin{array}{l}\text { describe types and categories of alternative investments } \\ \text { describe characteristics of direct investment, co-investment, and } \\ \text { fund investment methods for alternative investments } \\ \text { describe investment and compensation structures commonly used in } \\ \text { alternative investments }\end{array}$ \\
$\square$ &  \\
\end{tabular}
\end{center}

\section{INTRODUCTION}
$\square \quad$ describe types and categories of alternative investments

"Alternative investments" is a label for a disparate group of investments to distinguish them from "traditional investments" -investments in long-only, publicly traded investments in stocks, bonds, and cash. We can think of three major categories of alternative investments based on how they differ from "traditional investments":

\begin{enumerate}
  \item Private Capital

  \item Real Assets

  \item Hedge Funds

\end{enumerate}

One of the key features of alternative investments is that investing in them requires special skills and/or information. Investing in alternative assets can require handling illiquidity, transacting on private markets, operating sophisticated investment strategies, or risk-return profiles that are very different from those of traditional long-only investments.

\section{Why Investors Consider Alternative Investments}
Alternative investments offer a variety of advantages:

\begin{itemize}
  \item Broader diversification through accessing a larger universe of investments and/or because of their lower correlation with traditional asset classes

  \item Opportunities for enhanced returns by improving the portfolio's risk-return profile

  \item Potentially increased return through higher yields, particularly compared with traditional investments in low-interest rate periods.

\end{itemize}

The 2019 annual report for the Yale University endowment provides one institutional investor's reasoning for investing in alternatives:

The heavy [75.2\%] allocation to nontraditional asset classes stems from the diversifying power they provide to the portfolio as a whole. Alternative assets, by their very nature, tend to be less efficiently priced than traditional marketable securities, providing an opportunity to exploit market inefficiencies through active management. Today's portfolio has significantly higher expected returns and lower volatility than the 1989 portfolio. ${ }^{1}$

This quote neatly illustrates the expected characteristics of alternative investments: diversifying power, higher expected returns, and illiquid and potentially less efficient markets. The quote also highlights the importance of having the willingness and the ability to take a long-term perspective. Endowments, pension funds, sovereign wealth funds, and even family offices allocate increasing portions of their portfolios to alternative investments seeking to benefit from diversification and return opportunities.

Alternative investments are not free of risk, of course, and their returns may be correlated with those of other investments, especially in periods of financial crisis. Over a long historical period, the average correlation of returns from alternative investments with those of traditional investments may be low, but in any particular period, the correlation can differ from the average. During periods of economic crisis, correlations among many assets (both alternative and traditional) can increase dramatically.

Alternative investments often have many of the following characteristics:

\begin{itemize}
  \item Narrow specialization of the investment managers

  \item Relatively low correlation of returns with those of traditional investments

  \item Less regulation and less transparency than traditional investments

\end{itemize}

As a result of these characteristics, alternative investments often exhibit the following:

\begin{itemize}
  \item Limited reliable historical risk and return data

  \item Unique legal and tax considerations

  \item Higher fees, often including performance or incentive fees

  \item Concentrated portfolios

  \item Restrictions on redemptions (i.e., "lockups" and "gates")

\end{itemize}

\section{KNOWLEDGE CHECK}
\begin{enumerate}
  \item Alternative investments focus exclusively on the private markets.
\end{enumerate}

A. True B. False

\section{Solution:}
False. Although many alternative investments are focused on private markets, there are alternative investments, such as hedge funds, that focus on the public markets.

\section{Categories of Alternative Investments}
Considering the variety of alternative investments, it is not surprising that no consensus exists on a definitive list of groups or categories. There is even considerable debate around categories versus subcategories. For the purpose of this reading, we divide Alternative Investments into three categories and several subcategories as follows:

\begin{enumerate}
  \item Private Capital:
\end{enumerate}

\begin{itemize}
  \item Private Equity (PE)

  \item Private Debt

\end{itemize}

\begin{enumerate}
  \setcounter{enumi}{1}
  \item Real Assets:
\end{enumerate}

\begin{itemize}
  \item Real Estate

  \item Infrastructure

  \item Natural Resources

  \item Commodities

  \item Agricultural land and Timberland

\end{itemize}

\section{Hedge Funds}
The following provides an overview of the various categories listed above. Each category is described in detail later in this reading.

\section{Private Capital}
\begin{itemize}
  \item Private equity. Private equity funds generally invest in companies, whether startups or established firms, that are not listed on a public exchange, or they invest in public companies with the intent to take them private. The majority of private equity activity, by value, involves leveraged buyouts of established profitable and cash-generating companies with solid customer bases, proven products, and high-quality management.

  \item Venture capital funds, a specialized form of private equity that typically involves investing in or providing financing to startup or early-stage companies with high growth potential, represent a small portion of the private equity market by value.

  \item Private debt. Private debt largely encompasses debt provided to private entities. Forms of private debt include

  \item direct lending (private loans with no intermediary),

  \item mezzanine loans (private subordinated debt),

  \item venture debt (private loans to startup or early-stage companies that may have little or negative cash flow), and - distressed debt (debt extended to companies that are "distressed" because of such issues as bankruptcy or other complications with meeting debt obligations).

\end{itemize}

\section{Real Assets}
\begin{itemize}
  \item Real estate. Real estate investments are made in buildings or land, either directly or indirectly. They include private commercial real estate equity (e.g., ownership of an office building) and private commercial real estate debt (e.g., directly issued loans or mortgages on commercial property). Securitization has broadened the definition of real estate investing to include public real estate equity (e.g., real estate investment trusts, or REITs) and public real estate debt (e.g., mortgage-backed securities).

  \item Infrastructure. Infrastructure assets are capital-intensive, long-lived real assets, such as airports, roads, dams, and schools, that are intended for public use and provide essential services. An increasingly common approach to infrastructure investing is a public-private partnership (PPP) approach, in which governments and private investors each have a stake. Infrastructure investments provide exposure to asset cash flows, but the asset itself generally returns to public authority ownership. Infrastructure can be thought of as "Real Estate for the Public," with cash flows from landing rights, road tolls, etcn.

\end{itemize}

\section{- Natural Resources}
\begin{itemize}
  \item Commodities. Commodity investments may take place in physical commodity products, such as grains, metals, and crude oil, either through owning physical assets, using derivative products, or investing in businesses engaged in the exploration and production of physical commodities.

  \item Agricultural land (or farmland). Agricultural land involves the cultivation of livestock or plants, and agricultural land investing covers various strategies, including the purchase of farmland in order to lease it back to farmers or to receive a stream of income from the growth, harvest, and sale of crops (e.g., corn, cotton, wheat) or livestock (e.g., cattle).

  \item Timberland. Investing in timberland generally involves investing capital in natural forests or managed tree plantations in order to earn a return when the trees are harvested. Timberland involves a longer investment cycle than that of agriculture. Timberland investors often rely on various drivers, such as biological growth, to increase the value of the trees so the wood can be sold at favorable prices in the future.

  \item Other. Other "real asset" investments may include tangible assets, such as fine wine, art, antique furniture and automobiles, stamps, coins, and other collectibles, and intangible assets, such as patents and litigation actions.

\end{itemize}

"Digital assets" are an emerging investment opportunity. Some include these assets in the "other" category. However, since 2015, the Commodity Futures Trading Commission (CFTC) has defined digital assets as a "digital commodity" and regulates them accordingly.

\section{HEDGE FUNDS}
\begin{itemize}
  \item Hedge funds are private investment vehicles that manage portfolios of securities and/or derivative positions using a variety of strategies. They may involve long and short positions and may be highly leveraged. Some hedge funds try to deliver investment performance that is independent of broader market performance. Although hedge funds may be invested entirely in traditional assets, these vehicles are considered alternative because of their specialized approach.
\end{itemize}

INVESTMENT METHODS

describe characteristics of direct investment, co-investment, and fund investment methods for alternative investments

\section{Summary-Investment Methods for Alternative Investments}
\begin{itemize}
  \item Investors can access alternative investments in three ways:

  \item Fund investment (such as a in a PE fund)

  \item Direct investment into a company or project (such as infrastructure or real estate)

  \item Co-investment into a portfolio company of a fund

  \item Different approaches to due diligence are necessary for alternative investments depending on the investment method (direct, co-investing, or fund investing).

  \item Operational, financial, counterparty, and liquidity risks may be key considerations for those investing in alternative investments. These risks can be analyzed during the due diligence process.

\end{itemize}

\section{Methods of Investing in Alternative Investments}
Institutional investors typically begin investing in alternative investments via funds. Then, as they gain experience, they may begin to invest via co-investing and direct investing. The largest and most sophisticated direct investors (such as some sovereign wealth funds) compete with fund managers for access to the best investment opportunities. Exhibit 1 shows an illustration of the three methods of investing in alternative investments:

\section{Exhibit 1: Three Methods of Investing in Alternative Assets}
\begin{center}
\includegraphics[max width=\textwidth]{2023_05_04_36535b8d80b32081d422g-398}
\end{center}

Investors with limited resources and/or experience generally enter into alternative investments through fund investing, where the investor contributes capital to a fund and the fund identifies, selects, and makes investments on the investor's behalf. For the fund's services, the investor is charged a fee based on the amount of the assets being managed, and a performance fee is applied if the fund manager delivers superior results. In Exhibit 1, the investor invests in the "alternative investments fund." The fund itself invests in three investments: Investments 1, 2, and 3. Fund investing can be viewed as an indirect method of investing in alternative assets. Fund investors have little or no leeway in the sense that their investment decisions are limited to either investing in the fund or not. Fund investors typically have neither the sophistication nor the experience to invest directly on their own. Furthermore, fund investors are typically unable to affect the fund's underlying investments. Note that fund investing is available for all major alternative investment types, including hedge funds, private capital, real estate, infrastructure, and natural resources.

Once investors have some experience investing in funds, prior to investing directly themselves, many investors gain direct investing experience via co-investing, where the investor invests in assets indirectly through the fund but also possesses rights (known as co-investment rights) to invest directly in the same assets. Through co-investing, an investor is able to make an investment alongside a fund when the fund identifies deals; the investor is not limited to participating in the deal solely by investing in the fund. Exhibit 1 illustrates the co-investing method: The investor invests in one deal (labeled "Investment 3") indirectly via fund investing while investing an additional amount directly via a co-investment.

The direct method of investing in alternative assets is typically reserved for larger and more sophisticated investors. Direct investing occurs when an investor makes a direct investment in an asset (labeled "Investment A") without the use of an intermediary. Direct investors have great flexibility and control when it comes to choosing their investments, selecting their preferred methods of financing, and planning their approach.

In private equity, this may mean the investor purchases a direct stake in a private company without the use of a special vehicle, such as a fund. Direct investment usually applies to private capital and real estate. Sizable investors, such as major pensions and sovereign wealth funds, however, may also invest directly in infrastructure and natural resources.

\section{KNOWLEDGE CHECK}
\begin{enumerate}
  \item In co-investing, the investor is able to invest both directly and indirectly in the same assets.
\end{enumerate}

A. True

B. False

\section{Solution}
True. In co-investing, the investor is able to invest both directly and indirectly in the same assets.

\begin{enumerate}
  \setcounter{enumi}{1}
  \item The largest and most sophisticated direct investors compete with fund managers for access to the best investment opportunities.
\end{enumerate}

A. True

B. False

\section{Solution}
True. The largest and most sophisticated direct investors compete with fund managers for access to the best investment opportunities.

\section{Advantages and Disadvantages of Direct Investing, Co-Investing, and Fund Investing}
The direct investing, co-investing, and fund investing approaches naturally have distinct advantages and disadvantages.

Exhibit 2 provides a summary of the advantages and disadvantages of the three methods of investing in alternative investments.

Exhibit 2: Summary of Advantages and Disadvantages of Fund Investing,

Co-Investing, and Direct Investing

Advantages

Disadvantages

Fund

investing - Lower level of investor involvement as the fund managers provide investment services and expertise

\begin{itemize}
  \item Access to alternative investments without possessing a high degree of investment expertise

  \item Lower minimum capital requirements

\end{itemize}

Co-investing - Investors can learn from the fund's process to become better at direct investing

\begin{itemize}
  \item Reduced management fees

  \item Allows more active management of the portfolio compared with fund investing and allows for a deeper relationship with the manager - Costly management and performance fees

  \item Investor must conduct thorough due diligence when selecting the right fund because of the wide dispersion of fund manager returns

  \item Investors less able to exit the investment as funds typically have lock-ups and other restrictions

  \item Reduced control over the investment selection process compared with direct investing

  \item May be subject to adverse selection bias

  \item Requires more active involvement compared with fund investing, which can be challenging

\end{itemize}

\begin{center}
\begin{tabular}{lll}
\hline
 & \multicolumn{1}{c}{Advantages} & \multicolumn{1}{c}{Disadvantages} \\
\hline
$\begin{array}{lll}\text { Direct } & \text { - Avoids paying ongoing manage- } \\ \text { investing } & \text { - Requires higher internal invest- } \\ & \text { - Greatest amount of flexibility for an external manager } & \text { - Less access to a fund's ready } \\ \text { the investor } & \text { diversification benefits or the fund } \\ & \text { mighest level of control over how } & \text { - Requires more complex due dili- } \\ \text { the asset is managed } & \text { gence because of the absence of a } \\ & \text { fund manager } \\ & \text { Higher minimum capital } \\ & \text { requirements }\end{array}$ &  &  \\
 &  &  \\
\end{tabular}
\end{center}

\section{KNOWLEDGE CHECK}
\begin{enumerate}
  \item Identify the advantages of direct investing for an investor.
\end{enumerate}

\section{Solution}
An advantage of direct investing is that the investor can avoid paying ongoing management fees. In addition, a direct investor has greater flexibility and control when it comes to choosing his investments, selecting his preferred methods of financing, and planning his approach.

\begin{enumerate}
  \setcounter{enumi}{1}
  \item In direct investing, an investor puts capital in an asset or business:
A. using a special purpose vehicle, such as a fund.
B. using a separate business entity, such as a joint venture.
C. without the use of an intermediary.
\end{enumerate}

\section{Solution}
$C$ is correct. Direct investing occurs when an investor makes a direct investment in an asset without the use of an intermediary.

\begin{enumerate}
  \setcounter{enumi}{2}
  \item Identify advantages of fund investing for an investor.
\end{enumerate}

\section{Solution}
There are several advantages of fund investing for an investor, including potential diversification benefits, lower minimum capital requirements, and access to alternative investment exposure without a high degree of investment expertise.

\section{Due Diligence for Fund Investing, Direct Investing, and Co-Investing}
One basic question an investor must consider when investing in alternative investments is whether to rely on the expertise of the fund manager or to undertake the investment selection process themself via direct investing. A detailed discussion on due diligence is beyond the scope of this reading (especially for direct investing). However, this section provides a brief overview of the due diligence concept. For direct investing, the investor has control over the choice of which company to invest in; however, the due diligence process of making an investment in a private company requires considerable expertise. The focus of the due diligence is the company itself, at a very detailed level.

Fund investing offers diversified portfolios, but these benefits come with additional fees. The fund itself provides due diligence on the underlying investments; however, the investor is responsible for conducting due diligence on the fund manager when choosing among funds to invest in.

With co-investing, the investor is currently investing in a fund but is given an opportunity by the fund (co-investment rights) to make an additional investment in a portfolio company (as shown in Exhibit 1). In this case, the investor will conduct direct due diligence on the portfolio company with the support of the general partner.

\section{KNOWLEDGE CHECK}
\begin{enumerate}
  \item An investor with relatively limited investment and due diligence experience will most likely invest in alternative assets using which method?
\end{enumerate}

A. Fund investing

B. Co-investing

C. Direct investing

\section{Solution}
A is correct. An investor with limited investment and due diligence experience will most likely choose fund investing to benefit from the fund manager's expertise.

\section{INVESTMENT AND COMPENSATION STRUCTURES}
describe investment and compensation structures commonly used in alternative investments

\section{Summary-Investment and Compensation Structures (GP/LP, Performance Fees, Hurdle Rates, High-Water Marks, Lock-Ups, Waterfall Calculations)}
\begin{itemize}
  \item Many alternative investments, such as hedge and private equity funds, use a partnership structure with a general partner that manages the business and limited partners (investors) who own fractional interests in the partnership.

  \item The general partner typically receives a management fee based on assets under management or committed capital (the former is common to hedge funds, and the latter is common to private equity funds) and an incentive fee based on realized profits.

  \item The fee structure affects the returns to investors (limited partners), with a waterfall representing the distribution method under which allocations are made to LPs and GPs. Waterfalls can be on a whole-of-fund basis (European) or deal-by-deal basis (American). This section explores the investment partnership and compensation structures used in alternative investments. We examine the contractual provisions typically included in partnership agreements and why they exist to protect investors. An overview of the most common investment clauses is provided.

\end{itemize}

\section{Partnership Structures}
Exhibit 3 shows the GP/LP partnership structure as it relates to the three methods (fund, co-invest, direct) for investing in alternative investments.

\section{Exhibit 3: Partnership Structures}
\begin{center}
\includegraphics[max width=\textwidth]{2023_05_04_36535b8d80b32081d422g-402}
\end{center}

In the world of alternative investments, partnership structures are common. In limited partnerships, the fund manager is the general partner (GP) and investors are the limited partners (LP). LPs, who are generally accredited investors (owing to legal restrictions on the fund), are expected to understand and be able to assume the risks associated with the investments, which are less regulated than offerings to the general public. The GP runs the business and theoretically bears unlimited liability for anything that goes wrong. The GP may also run multiple funds at a time.

Limited partners are outside investors who own a fractional interest in the partnership based on the amount of their initial investment and the terms set out in the partnership documentation. LPs commit to future investments, and the upfront cash outflow can be a small portion of their total commitment to the fund. Funds set up as limited partnerships typically have a limit on the number of LPs allowed to invest in the fund. LPs play passive roles and are not involved with the management of the fund (although co-investment rights allow for the LPs to make additional direct investments in the portfolio companies); the operations and decisions of the fund are controlled solely by the GP. Exhibit 4 illustrates the limited partnership structure for a hypothetical USD500 million investment fund.

\section{Exhibit 4: Example of a Limited Partnership Structure}
\begin{center}
\includegraphics[max width=\textwidth]{2023_05_04_36535b8d80b32081d422g-403}
\end{center}

Between the GP and the LPs there exists a principal/agent problem. The GP has specialized knowledge and skills to make/manage complex investments. The LPs are keen on gaining strong investment returns from their investment in the fund in partnership with the GP. This principal/agent problem is managed through compensation structures and limited partnership agreements (LPAs).

The partnership between the GP and LPs is governed by a limited partnership agreement, a legal document that outlines the rules of the partnership and establishes the framework that ultimately guides the fund's operations throughout its life. LPAs vary in length and complexity and may be dense with provisions and clauses, some of which are discussed later in this section. In addition to LPAs, side letters may also be negotiated. Side letters are agreements created between the GP and a certain number of LPs that exist outside the LPA. Some examples of clauses or details that may be included in a side letter include the following:

\begin{itemize}
  \item Potential additional reporting due to an LP's unique circumstances, such as regulatory or tax requirements

  \item First right of refusal and other similar clauses to outline potential treatment (regarding fees, co-investment rights, secondary sales, and, potentially, other matters) in comparison to other LPs

  \item Notice requirements in the event of litigation, insolvency, and related matters

  \item Most favored nation clauses, such as agreeing that if similar LPs pay lower fees, they will be offered to the LP

\end{itemize}

Certain structures are commonly adopted for specific alternative investments. For example, infrastructure investors frequently enter into public-private partnerships (PPP), which are agreements between the public sector and the private sector to finance, build, and operate public infrastructure, such as hospitals and toll roads. In real estate fund investing, investors may be classified as unitholders, and joint ventures are a partnership structure common in real estate direct investing.

\section{KNOWLEDGE CHECK}
\begin{enumerate}
  \item Investments in limited partnerships are regulated than offerings to the general public.
\end{enumerate}

\section{Solution}
Investments in limited partnerships are less regulated than offerings to the general public.

\begin{enumerate}
  \setcounter{enumi}{1}
  \item By drawing an arrow between the two, match which common structure is typical for each type of alternative investment:
\end{enumerate}

\begin{center}
\begin{tabular}{ll}
\hline
Types of Alternative Investment & Common Structure \\
\hline
1. Infrastructure & A. Joint venture \\
2. Real Estate fund investing & B. Unitholder \\
3. Real estate direct investing & C. Public-private partnership \\
\hline
\end{tabular}
\end{center}

\section{Solution}
Option 1 (Infrastructure) matches with C (Public-private partnership).

Option 2 (Real estate fund investing) matches with B (Unitholder). Option 3 (Real estate direct investing) matches with A (Joint venture).

\section{Compensation Structures}
Compensation structures, including management fees and performance fees, are in place to reward GPs for enhanced performance and to motivate investment professionals to work hard and stay involved with the fund for many years to come.

Funds are generally structured with a management fee typically ranging from $1 \%$ to $2 \%$ of assets under management (e.g., for hedge funds) or committed capital (e.g., for private equity funds), which is how much money in total that LPs have committed to the fund's future investments. Additionally, a performance fee (also referred to as an incentive fee or carried interest) is applied based on excess returns.

Private equity funds raise committed capital and draw down on those commitments, generally over three to five years, when they have a specific investment to make. Note that the management fee is typically based on committed capital, not invested capital; the committed-capital basis for management fees is an important distinction from hedge funds, whose management fees are based on assets under management (AUM). Having committed capital as the basis for management fee calculations reduces the incentive for GPs to deploy the committed capital as quickly as possible to grow their fee base. This allows the GPs to be selective about deploying capital into investment opportunities.

The partnership agreement usually specifies that the performance fee is earned only after the fund achieves a return known as a hurdle rate. The hurdle rate is a minimum rate of return, typically $8 \%$, that the GP must exceed in order to earn the performance fee. GPs typically receive $20 \%$ of the total profit of the private equity fund either net of any hard hurdle rate, in which case the GP earns fees on annual returns in excess of the hurdle rate, or net of the soft hurdle rate, in which case the fee is calculated on the entire annual gross return as long as the set hurdle is exceeded. Hurdle rates are less common for hedge funds but do appear from time to time.

Exhibit 5 illustrates a simple example of how performance fees are calculated using a typical " 2 and 20" compensation structure, where management fees are calculated on $2 \%$ of AUM or committed capital and performance fees are calculated on $20 \%$ of profits. An investment that was purchased in Year 1 for $\$ 10$ million and sold in Year 4 for $\$ 20$ million represents a return of $26 \%$ on an internal rate of return (IRR) basis. The LPs receive $80 \%$ of profits $(80 \% \times \$ 10$ million profit $=\$ 8$ million), and the GP receives $20 \%$ of profits $(20 \% \times \$ 10$ million profit $=\$ 2$ million). Note that this example assumes the hurdle rate has been exceeded and a catch-up clause exists in the partnership agreement; these concepts will be discussed in the next section.

\section{Exhibit 5: Simple Performance Fee Calculation}
\begin{center}
\includegraphics[max width=\textwidth]{2023_05_04_36535b8d80b32081d422g-405}
\end{center}

In addition to both management and performance fees charged to investors, leveraged buyout firms in private equity may charge consulting fees and monitoring fees to the underlying companies.

Hedge fund managers generally accrue an incentive fee on their quarterly performance, but they typically crystalize (realize) their incentive fee once annually. If they make money in the first quarter but then lose the same amount in the second, the investor recoups the $\mathrm{Q} 1$ incentive fee and pays only at the end of the year an incentive fee based on total annual returns net of management fees and other expenses.

\section{KNOWLEDGE CHECK}
\begin{enumerate}
  \item A fee is typically added to a management fee for a fund governed by a limited partnership agreement.
\end{enumerate}

\section{Solution}
A performance (incentive or carried interest) fee is typically added to a management fee for a fund governed by a limited partnership agreement.

\section{Common Investment Clauses, Provisions, and Contingencies}
For most alternative investment funds, particularly hedge funds and private equity funds, the GP does not earn a performance fee until the LPs have received their initial investment and the total return generated on the investment has exceeded a specified hurdle rate. In our example, the LPs receive $80 \%$ of the total profit of the fund plus the amount of their initial investment.

A catch-up clause may be included in the partnership agreement. Essentially, for a GP who earns a $20 \%$ performance fee, a catch-up clause allows the GP to receive $100 \%$ of the distributions above the hurdle rate until $20 \%$ of the profits generated is received, and then every excess dollar is split 80/20 between the LPs and GP. To illustrate, assume that the GP has earned an $18 \%$ IRR on an investment, the hurdle rate is $8 \%$, and the partnership agreement includes a catch-up clause. In this case, the LPs would receive the entirety of the first $8 \%$ profit, the GP would receive the entirety of the next $2 \%$ profit-because $2 \%$ out of $10 \%$ amounts to $20 \%$ of the profits accounted for so far, as the catch-up clause stipulates-and the remaining $8 \%$ would be split $80 / 20$ between the LPs and the GP. Effectively, the LPs earned $14.4 \%(18 \% \times 80 \%)$ and the GP earned 3.6\% $(18 \% \times 20 \%)$. Imagine the same scenario, absent a catch-up clause: The LPs would still receive the entirety of the first $8 \%$ profit; however, the remaining $10 \%$ would be split 80/20 between the LPs and GP, reducing the GP's return to $2.0 \%$ $[(18 \%-8 \%) \times 20 \%]$. These calculations are presented graphically in Exhibit 6 .

\section{Exhibit 6: Simple Catch-Up Clause Illustration}
\begin{center}
\includegraphics[max width=\textwidth]{2023_05_04_36535b8d80b32081d422g-406}
\end{center}

In hedge funds, fee calculations also take into account a high-water mark, which reflects the highest value used to calculate an incentive fee. A high-water mark is the highest value of the fund investment ever achieved at a performance fee crystallization date, net of fees, by the individual LP. A high-water mark clause states that a hedge fund manager must recuperate declines in value from the high-water mark before performance fees can be charged on newly generated profits. The use of high-water marks protects clients from paying twice for the same performance.

With all alternative investments, investor high-water marks generally carry over into new calendar years, although with hedge funds, an investor will no longer be able to claw back any incentive fees paid for a given calendar year if portfolio losses are subsequently incurred in a later calendar year. Given the generally more illiquid and longer-term nature of their holdings, private equity and real estate investments are more likely to contain clawback clauses for the entire life of the portfolio. Given that different clients invest at different times, it is possible that not all clients will be at their respective high-water marks at the same time; a client who invests on a dip will enjoy the fund's recovery and pay an incentive fee, whereas a client who invests at the top will need to earn back what was lost before being obliged to pay the incentive fee.

Needless to say, individual capital account fund accounting can become quite complicated when tracking such investment timing differences.

\section{KNOWLEDGE CHECK}
\begin{enumerate}
  \item The minimum rate of return that a GP must exceed in order to earn an incentive or performance fee is called the:
\end{enumerate}

A. high-water mark.

B. hurdle rate.

C. performance threshold.

\section{Solution}
$\mathrm{B}$ is correct. A high-water mark is the highest value used to calculate an incentive fee. "Performance threshold" is not a term that is generally used in the industry.

In alternative investments, a waterfall represents the distribution method that defines the order in which allocations are made to LPs and GPs. GPs usually receive a disproportionately larger share of the total profits relative to their initial investment, which incentivizes them to maximize profitability. There are two types of waterfalls: deal-by-deal (or American) waterfalls and whole-of-fund (or European) waterfalls. Deal-by-deal waterfalls are more advantageous to the GP because performance fees are collected on a per-deal basis, allowing the GP to get paid before LPs receive both their initial investment and their preferred rate of return (i.e., the hurdle rate) on the entire fund. In whole-of-fund waterfalls, all distributions go to the LPs as deals are exited and the GP does not participate in any profits until the LPs receive their initial investment and the hurdle rate has been met. In contrast to deal-by-deal waterfalls, whole-of-fund waterfalls occur at the aggregate fund level and are more advantageous to the LPs.

A clawback provision reflects the right of LPs to reclaim part of the GP's performance fee. Along either waterfall path, if a GP ever accrues (or actually pays itself) an incentive fee on gains that are not yet fully realized and then subsequently gives back these gains, an investor is typically able to claw back prior incentive fee accruals and payments. Clawback provisions are usually activated when a GP exits successful deals early on but incurs losses on deals later in the fund's life.

Exhibit 7 and Exhibit 8 illustrate deal-by-deal waterfalls and whole-of-fund waterfalls, respectively. Exhibit 7: Deal-by-Deal (American) Waterfall Example-with Clawback

Provision

\begin{center}
\begin{tabular}{|c|c|c|c|c|c|c|c|}
\hline
\multirow{2}{*}{Investment No.} & \multicolumn{2}{|c|}{Year} & \multicolumn{2}{|c|}{Amount (\$mm)} & \multicolumn{3}{|c|}{Profit} \\
\hline
 & Invested & Sold & Invested & Sold & $\$ \mathrm{~mm}$ & $\%$ & GP at $20 \%$ \\
\hline
1 & 1 & 4 & $\$ 10$ & $\$ 20$ & $\$ 10$ & $26.0 \%$ & $\$ 2$ \\
\hline
2 & 2 & 5 & $\$ 20$ & $\$ 35$ & $\$ 15$ & $20.5 \%$ & $\$ 3$ \\
\hline
3 & 2 & 7 & $\$ 40$ & $\$ 80$ & $\$ 40$ & $14.9 \%$ & $\$ 8$ \\
\hline
4 & 3 & 7 & $\$ 20$ & $\$ 20$ & - & - & - \\
\hline
5 & 3 & 8 & $\$ 35$ & $\$ 25$ & $(\$ 10)$ & neg & $(\$ 2)$ \\
\hline
6 & 4 & 9 & $\$ 25$ & $\$ 20$ & (\$5) & neg & $(\$ 1)$ \\
\hline
7 & 5 & 9 & $\$ 30$ & - & $(\$ 30)$ & neg & $(\$ 6)$ \\
\hline
8 & 5 & 10 & $\$ 20$ & - & (\$20) & neg & $(\$ 4)$ \\
\hline
Total & 1 & 10 & $\$ 200$ & $\$ 200$ & - & - & - \\
\hline
\end{tabular}
\end{center}

Exhibit 8: Whole-of-Fund (European) Waterfall Example

\begin{center}
\begin{tabular}{|c|c|c|c|c|c|c|c|}
\hline
\multirow{2}{*}{Investment No.} & \multicolumn{2}{|c|}{Year} & \multicolumn{2}{|c|}{Amount (\$mm)} & \multicolumn{3}{|c|}{Profit} \\
\hline
 & Invested & Sold & Invested & Sold & $\$ \mathrm{~mm}$ & $\%$ & GP at $20 \%$ \\
\hline
1 & 1 & 4 & $\$ 10$ & $\$ 20$ & $\$ 10$ & $26.0 \%$ & - \\
\hline
2 & 2 & 5 & $\$ 20$ & $\$ 35$ & $\$ 15$ & $20.5 \%$ & - \\
\hline
3 & 2 & 7 & $\$ 40$ & $\$ 80$ & $\$ 40$ & $14.9 \%$ & - \\
\hline
4 & 3 & 7 & $\$ 20$ & $\$ 20$ & - & - & - \\
\hline
5 & 3 & 8 & $\$ 35$ & $\$ 25$ & (\$10) & neg & - \\
\hline
6 & 4 & 9 & $\$ 25$ & $\$ 20$ & $(\$ 5)$ & neg & - \\
\hline
7 & 5 & 9 & $\$ 30$ & - & $(\$ 30)$ & neg & - \\
\hline
8 & 5 & 10 & $\$ 20$ & - & $(\$ 20)$ & neg & - \\
\hline
Total & 1 & 10 & $\$ 200$ & $\$ 200$ & - & - & - \\
\hline
\end{tabular}
\end{center}

\section{PRACTICE PROBLEMS}
\begin{enumerate}
  \item Which of the following is least likely to be considered an alternative investment?
A. Real estate
B. Commodities
C. Long-only equity funds

  \item An investor is seeking an investment that can take long and short positions, may use multi-strategies, and historically exhibits low correlation with a traditional investment portfolio. The investor's goals will be best satisfied with an investment in:
A. real estate.
B. a hedge fund.
C. a private equity fund.

  \item Relative to traditional investments, alternative investments are least likely to be characterized by:
A. high levels of transparency.
B. limited historical return data.
C. significant restrictions on redemptions.

  \item Alternative investment funds are typically managed:
A. actively.
B. to generate positive beta return.
C. assuming that markets are efficient.

  \item Compared with traditional investments, alternative investments are more likely to have:
A. greater use of leverage.
B. long-only positions in liquid assets.
C. more transparent and reliable risk and return data.

  \item The potential benefits of allocating a portion of a portfolio to alternative investments include:
A. ease of manager selection.
B. improvement in the portfolio's risk-return relationship.
C. accessible and reliable measures of risk and return.

  \item From the perspective of the investor, the most active approach to investing in alternative investments is:

\end{enumerate}

A. co-investing. B. fund investing.

C. direct investing.

\begin{enumerate}
  \setcounter{enumi}{7}
  \item In comparison to other alternative investment approaches, co-investing is most likely:
\end{enumerate}

A. more expensive.

B. subject to adverse selection bias.

C. the most flexible approach for the investor.

\begin{enumerate}
  \setcounter{enumi}{8}
  \item Relative to co-investing, direct investing due diligence is most likely:
\end{enumerate}

A. harder to control.

B. more independent.

C. equally thorough.

\begin{enumerate}
  \setcounter{enumi}{9}
  \item The investment method that typically requires the most thorough due diligence from an investor is:
A. fund investing.
B. co-investing.
C. direct investing.

  \item An alternative investment fund's hurdle rate is a:

\end{enumerate}

A. rate unrelated to a catch-up clause.

B. tool to protect clients from paying twice for the same performance.

C. minimum rate of return the GP must exceed in order to earn a performance fee.

\begin{enumerate}
  \setcounter{enumi}{11}
  \item An investor in a private equity fund is concerned that the general partner can receive incentive fees in excess of the agreed-on incentive fees by making distributions over time based on profits earned rather than making distributions only at exit from investments of the fund. Which of the following is most likely to protect the investor from the general partner receiving excess fees?
\end{enumerate}

A. high hurdle rate

B. clawback provision

C. lower capital commitment

\begin{enumerate}
  \setcounter{enumi}{12}
  \item Until the committed capital is fully drawn down and invested, the management fee for a private equity fund is based on:
A. invested capital.
B. committed capital.
C. assets under management.

  \item The distribution method by which profits generated by a fund are allocated be- tween LPs and the GP is called:
A. a waterfall.
B. an $80 / 20$ split.
C. a fair division.

  \item An American waterfall distributes performance fees on $\mathrm{a}(\mathrm{n})$ basis and is more advantageous to the
A. deal-by-deal; LPs
B. aggregate fund; LPs
C. deal-by-deal; GP

\end{enumerate}

\section{SOLUTIONS}
\begin{enumerate}
  \item C is correct. Long-only equity funds are typically considered traditional investments, and real estate and commodities are typically classified as alternative investments.

  \item B is correct. Hedge funds may take long and short positions, use a variety of strategies, and generally have a low correlation with traditional investments.

  \item A is correct. Alternative investments are characterized as typically having low levels of transparency.

  \item A is correct. There are many approaches to managing alternative investment funds, but typically these funds are actively managed.

  \item A is correct. Investing in alternative investments is often pursued through such special vehicles as hedge funds and private equity funds, which have flexibility to use leverage. Alternative investments include investments in such assets as real estate, which is an illiquid asset, and investments in such special vehicles as private equity and hedge funds, which may make investments in illiquid assets and take short positions. Obtaining information on strategies used and identifying reliable measures of risk and return are challenges of investing in alternatives.

  \item B is correct. Adding alternative investments to a portfolio may provide diversification benefits because of these investments' less-than-perfect correlation with other assets in the portfolio. As a result, allocating a portion of one's funds to alternatives could potentially result in an improved risk-return relationship. Challenges to allocating a portion of a portfolio to alternative investments include obtaining reliable measures of risk and return and selecting portfolio managers for the alternative investments.

  \item $\mathrm{C}$ is correct. From the perspective of the investor, direct investing is the most active approach to investing because of the absence of fund managers and the services and expertise they generally provide.

\end{enumerate}

A is incorrect because co-investing includes fund investing, which requires less due diligence compared with direct investing.

$\mathrm{B}$ is incorrect because fund investing in alternative assets demands less participation from the investor compared with the direct and co-investing approaches. An investor depends on the fund manager to identify, select, and manage the fund's investments.

\begin{enumerate}
  \setcounter{enumi}{7}
  \item B is correct. Co-investing may be subject to adverse selection bias. For example, the fund manager may make less attractive investment opportunities available to the co-investor while allocating its own capital to more appealing deals.
\end{enumerate}

A is incorrect because co-investing is likely not more expensive than fund investing since co-investors can co-invest an additional amount alongside the fund directly in a fund investment without paying management fees on the capital that has been directly invested.

$\mathrm{C}$ is incorrect because direct investing, not co-investing, provides the greatest amount of flexibility for the investor.

\begin{enumerate}
  \setcounter{enumi}{8}
  \item B is correct. Direct investing due diligence may be more independent than that of co-investing because the direct investing team is typically introduced to opportu- nities by third parties rather than fund managers, as is customary in co-investing. A is incorrect because the direct investing team has more control over the due diligence process compared with co-investing.
\end{enumerate}

$\mathrm{C}$ is incorrect because due diligence for direct investing requires the investor to conduct a thorough investigation into the important aspects of a target asset or business, whereas in co-investing, fund managers typically provide investors with access to a data room so they can view the due diligence completed by the fund managers.

\begin{enumerate}
  \setcounter{enumi}{9}
  \item $\mathrm{C}$ is correct. Due diligence in direct investing will typically be more thorough and more rigid from an investor's perspective because of the absence of a fund manager that would otherwise conduct a large portion of the necessary due diligence.

  \item $\mathrm{C}$ is correct. An alternative investment fund's hurdle rate is a minimum rate of return the GP must exceed in order to earn a performance fee.

\end{enumerate}

A is incorrect because if a catch-up clause is included in the partnership agreement, the catch-up clause permits distributions in relation to the hurdle rate. $B$ is incorrect because it is a high-water mark (not a hurdle rate) that protects clients from paying twice for the same performance.

\begin{enumerate}
  \setcounter{enumi}{11}
  \item B is correct. A clawback provision requires the general partner in a private equity fund to return any funds distributed (to the general partner) as incentive fees until the limited partners have received their initial investments and the contracted portion of the total profits. A high hurdle rate will result in distributions occurring only after the fund achieves a specified return. A high hurdle rate decreases the likelihood of, but does not prevent, excess distributions. Management fees, not incentive fees, are based on committed capital.

  \item B is correct. Until the committed capital is fully drawn down and invested, the management fee for a private equity fund is based on committed capital, not invested capital.

  \item A is correct. Although profits are typically split 80/20 between LPs and the GP, the distribution method of profits is not called an "80/20 split." "Fair division" is not a real term that exists in the industry.

  \item $\mathrm{C}$ is correct. American waterfalls, also known as deal-by-deal waterfalls, pay performance fees after every deal is completed and are more advantageous to the GP because they get paid sooner (compared with European, or whole-of-fund, waterfalls).

\end{enumerate}

\section{LEARNING MODULE 2}
\section{Performance Calculation and Appraisal of Alternative Investments}
\section{LEARNING OUTCOME}
\begin{center}
\begin{tabular}{c|l}
Mastery & The candidate should be able to: \\
\hline
$\square$ & describe issues in performance appraisal of alternative investments \\
$\square$ & $\begin{array}{l}\text { calculate and interpret returns of alternative investments both before } \\ \text { and after fees }\end{array}$ \\
\end{tabular}
\end{center}

\section{SUMMARY—ISSUES IN PERFORMANCE APPRAISAL}
describe issues in performance appraisal of alternative investments

\begin{itemize}
  \item Conducting performance appraisal on alternative investments can be challenging because these investments are often characterized by asymmetric risk-return profiles, limited portfolio transparency, illiquidity, product complexity, and complex fee structures.

  \item Traditional risk and return measures (such as mean return, standard deviation of returns, and beta) may provide an inadequate picture of alternative investments' risk and return characteristics. Moreover, these measures may be unreliable or not representative of specific investments.

  \item A variety of ratios can be calculated in order to review the performance of alternative investments, including the Sharpe ratio, Sortino ratio, Calmar ratio, and MAR ratio. The IRR (internal rate of return) and MOIC calculations are often used to evaluate private equity investments, and the cap rate is often used to evaluate real estate investments.

  \item Redemption rules, lockup periods, and timing differences in reporting can bring special challenges to performance appraisal of alternative investments.

\end{itemize}

\section{Performance Calculation and Appraisal of Alternative Investments}
\section{Overview of Performance Appraisal for Alternative Investments}
Investors frequently look to alternative investments for diversification and a chance to earn relatively high returns on a risk-adjusted basis. Investors also value low correlation and a more risk-neutral source of alpha.

Evaluating an alternative investment can be a subtle, qualitative exercise-one that depends on the initial objectives of the investor-as opposed to a purely quantitative, one-size-fits-all exercise. Much of the nuance revolves around not only the total net return created by an alternative investment but also the path and volatility (drawdown risk) required to create the total return and how an alternative investment fits into and benefits a larger portfolio of assets-in other words, its portfolio-level correlation benefit.

The attraction to alternative investments is often focused on their expected returns, but investors often neglect to consider the atypical risks they present-risks we can examine on both a standalone and portfolio basis:

\begin{itemize}
  \item Limited transparency

  \item Low portfolio liquidity

  \item High leverage and use of derivatives

  \item High product complexity

  \item Mark-to-market issues, especially for specialized products

  \item Limited redemption availability

  \item Difficulty in manager selection and diversification

  \item High fees, which can have a non-trivial impact on performance

\end{itemize}

\section{Common Approaches to Performance Appraisal and Application Challenges}
Common approaches to performance appraisal of alternative investments include such ratios as the Sharpe ratio, Sortino ratio, MAR ratio, and Calmar ratio. A summary of the various ratios is shown in Exhibit 9 .

Exhibit 1: Common Approaches to Performance Appraisal in Alternative Investments

\begin{center}
\begin{tabular}{|c|c|c|c|c|c|}
\hline
 & Description & Formula & Pros & Cons & Comments \\
\hline
Sharpe & $\begin{array}{l}\text { Risk-adjusted } \\ \text { return of a port- } \\ \text { folio (using stan- } \\ \text { dard deviation as } \\ \text { a risk measure) }\end{array}$ & $\begin{array}{c}\left(R_{p}-R_{f}\right) / \text { St. dev. of } \\ \text { portfolio }\end{array}$ & $\begin{array}{l}\text { Easy to } \\ \text { calculate }\end{array}$ & $\begin{array}{l}\text { Penalizes } \\ \text { upside } \\ \text { and } \\ \text { downside } \\ \text { volatility } \\ \text { equally }\end{array}$ & $\begin{array}{l}\text { Assumes a } \\ \text { normal distri- } \\ \text { bution, more } \\ \text { appropriate for } \\ \text { low-volatility } \\ \text { portfolios }\end{array}$ \\
\hline
Sortino & $\begin{array}{l}\text { Risk-adjusted } \\ \text { return of a } \\ \text { portfolio (using } \\ \text { downside devi- } \\ \text { ation as a risk }\end{array}$ & $\begin{array}{c}\left(R_{p}-R_{f}\right) / \text { Downside } \\ \text { deviation of portfolio }\end{array}$ & $\begin{array}{l}\text { Focused } \\ \text { on } \\ \text { downside } \\ \text { deviation }\end{array}$ & $\begin{array}{l}\text { More dif- } \\ \text { ficult to } \\ \text { calculate }\end{array}$ & $\begin{array}{l}\text { More appro- } \\ \text { priate for } \\ \text { high-volatility } \\ \text { portfolios }\end{array}$ \\
\hline
\end{tabular}
\end{center}

\begin{center}
\begin{tabular}{|c|c|c|c|c|c|}
\hline
 & Description & Formula & Pros & Cons & Comments \\
\hline
\multirow[t]{2}{*}{MAR} & $\begin{array}{l}\text { Risk-adjusted } \\ \text { return of a } \\ \text { portfolio: since } \\ \text { inception (using } \\ \text { maximum draw- } \\ \text { down as risk } \\ \text { measure) "gain/ } \\ \text { pain" }\end{array}$ & $\begin{array}{l}\text { (Fund average com- } \\ \text { pounded annual rate of } \\ \text { return since inception)/ } \\ \text { Maximum drawdown } \\ \text { since inception }\end{array}$ & $\begin{array}{c}\text { Shows } \\ \text { entire } \\ \text { history }\end{array}$ & $\begin{array}{l}\text { Recent } \\ \text { perfor- } \\ \text { mance } \\ \text { might } \\ \text { be more } \\ \text { relevant }\end{array}$ & $\begin{array}{l}\text { More appropri- } \\ \text { ate for hedge } \\ \text { funds and com- } \\ \text { modity trading } \\ \text { advisers (CTAs) }\end{array}$ \\
\hline
 & $\begin{array}{l}\text { Risk-adjusted } \\ \text { return of a port- } \\ \text { folio for a period } \\ \text { (using maximum } \\ \text { drawdown as } \\ \text { risk measure) } \\ \text { "gain/pain" }\end{array}$ & $\begin{array}{l}\text { (Fund average com- } \\ \text { pounded annual rate } \\ \text { of return for a period)/ } \\ \text { Maximum drawdown } \\ \quad \text { during period }\end{array}$ & $\begin{array}{l}\text { Focused } \\ \text { on recent } \\ \text { perfor- } \\ \text { mance }\end{array}$ & $\begin{array}{l}\text { Might } \\ \text { hide past } \\ \text { issues }\end{array}$ & $\begin{array}{l}\text { More appropri- } \\ \text { ate for hedge } \\ \text { funds and } \\ \text { CTAs, typically } \\ \text { uses a 36-month } \\ \text { period }\end{array}$ \\
\hline
\end{tabular}
\end{center}

These ratios are explained in more detail in the following subsections.

\section{Sharpe Ratio}
The Sharpe ratio is probably the first basic intuitive metric that some people use to evaluate an alternative investment. It is often prominently displayed in marketing materials. The single biggest flaw, however, in a dependence on the Sharpe ratio is probably the underlying assumption of normally distributed returns.

Return profiles of alternative investments tend to be asymmetric and skewed due to the risks discussed earlier, such as leverage, illiquidity, mark-to-market smoothing, and poor portfolio transparency, making the Sharpe ratio a less-than-ideal performance measure for alternative investments. For non-normal return distributions with significant skewness (fat tails in one direction or the other) and kurtosis (a measure of whether the data are heavy tailed or light tailed relative to a normal distribution), volatility is not a perfect measure of dispersion. Exhibit 10 provides a reminder of how skewness and kurtosis differ from a normal distribution.

\section{Exhibit 2: Distributions in Alternative Investments}
A. Skewness

Positively Skewed Distribution

\begin{center}
\includegraphics[max width=\textwidth]{2023_05_04_36535b8d80b32081d422g-417(2)}
\end{center}

Negatively Skewed Distribution

\begin{center}
\includegraphics[max width=\textwidth]{2023_05_04_36535b8d80b32081d422g-417(1)}
\end{center}

B. Kurtosis

\begin{center}
\includegraphics[max width=\textwidth]{2023_05_04_36535b8d80b32081d422g-417}
\end{center}

Although still widely cited, the Sharpe ratio may not, therefore, be a good risk-adjusted performance measurement to rely on.

\section{Performance Calculation and Appraisal of Alternative Investments}
\section{Sortino Ratio}
The Sortino ratio is a more useful measure than the Sharpe ratio because it focuses on the performance relative to downside volatility, thus better capturing the impact of skewness and kurtosis.

\section{Calmar Ratio and MAR Ratio}
The Calmar and MAR ratios are also better alternatives to the Sharpe ratio when looking at the performance of alternative investments. These two measures compare performance to drawdown. A drawdown is the percentage peak-to-trough (high-to-low) reduction in net asset value (NAV). A maximum drawdown (MDD) is the maximum observed loss from a peak to a trough of a portfolio, before a new peak is attained. The Calmar ratio is a comparison of the average annual compounded return to this maximum drawdown risk. The higher (lower) the Calmar ratio, the better (worse) an alternative asset performed on a risk-adjusted basis over a specified period of time. The Calmar ratio is typically calculated using the prior three years of performance, and it thus adjusts over time. Variations of the ratio exist: The MAR ratio uses a full investment history and the maximum drawdown. Both ratios help address the left-tailed return profile that sometimes characterizes alternative assets.

\section{KNOWLEDGE CHECK}
\begin{enumerate}
  \item The higher the Calmar ratio, the (better/worse) an alternative asset performed on a risk-adjusted basis over a specific period of time.
\end{enumerate}

\section{Solution}
The higher the Calmar ratio, the better an alternative asset performed on a risk-adjusted basis over a specific period of time.

\begin{enumerate}
  \setcounter{enumi}{1}
  \item True or false: The Sharpe ratio measures the amount of risk per unit of return.
\end{enumerate}

Explain your selection.

A. True

B. False

\section{Solution}
False. The Sharpe and Sortino ratios are risk-adjusted performance measures. They are measures of return per unit of risk.

\begin{enumerate}
  \setcounter{enumi}{2}
  \item True or false: The Sharpe and Sortino ratios share the same denominator.
\end{enumerate}

Explain your selection.
A. True
B. False

\section{Solution}
False. The Sharpe and Sortino ratios share the same numerator-average annualized return net of the risk-free rate. Their denominators are different. The denominator for the Sharpe ratio is standard deviation of returns, and the denominator for the Sortino ratio is downside deviation of returns-a semi-deviation measure of volatility only during periods of loss for an alternative investment.

\section{Private Equity and Real Estate Performance Evaluation}
Some types of alternative investments generally involve large initial capital outlays with capital inflows occurring much later in the investment cycle. Exhibit 11 shows this type of cash flow pattern and the net cash position over the life of the investment. Private equity investments generally involve an initial capital commitment, but actual capital flows often lag that commitment because capital "calls" are staggered over substantive periods of time. Private equity returns are frequently described in terms of the J-curve effect.

\section{Exhibit 3: The "J-Curve" Effect}
\begin{center}
\includegraphics[max width=\textwidth]{2023_05_04_36535b8d80b32081d422g-419}
\end{center}

Source:\href{https://corporatefinanceinstitute.com/resources/knowledge/economics/j-curve/}{https://corporatefinanceinstitute.com/resources/knowledge/economics/j-curve/}.

Private equity includes a substantive initial capital commitment promise, followed by high initial fee drag (calculated on total committed capital, not the capital actually called), followed by the identification of longer-term growth or turnaround opportunities, and an eventual positive expected return when the staggered returns of the fund are realized. The component investments mature and are sold at various times, and the partnership finally closes - but usually only after six to eight years. The line representing the return changes from downward sloping to positive and then to energetically upward sloping later in the investment's life.

The real estate pathway is similar to the J-curve effect. It starts with initial property purchases, followed at times by substantive cash outlays for improvements, followed by instances of accounting depreciation that can influence after-tax performance, typically followed by the receipt of rents and then an eventual property sale (often at a long-term tax-advantaged tax rate).

The measurement of success in both instances depends far more on the timing and magnitude of cash flows in and out of the investments, and these are often hard to standardize and anticipate. Given the long time horizon, the application of different tax treatments can have a non-trivial impact on after-tax investment returns.

\section{Performance Calculation and Appraisal of Alternative Investments}
As a general rule, the best way to start evaluating such investments is with the IRR of the respective cash flows into an investment and the timing thereof, versus the magnitude and the timing of the cash flows returned by the investment (inclusive of $\operatorname{tax}$ benefits).

In an independent, fixed-life private equity fund, the decisions to raise money, take money in the form of capital calls, and distribute proceeds are all at the discretion of the private equity manager. Timing of cash flows is an important part of the investment decision process. The private equity manager should thus be rewarded or penalized for the results of those timing decisions, and the calculation of an IRR is key for doing so.

Although the determination of an IRR involves certain assumptions about a financing rate to use for outgoing cash flows (typically a weighted average cost of capital) and a reinvestment rate to use for incoming cash flows (which must be assumed and may or may not actually be earned), the IRR is the key metric used to assess longer-term alternative investments in the private equity and real estate worlds.

Because of this complexity, a shortcut methodology often used by both private equity and real estate managers involves simply citing a multiple of invested capital (MOIC), or money multiple, on total invested capital (which is paid-in capital less management fees and fund expenses). Here, one simply measures the total value of all realized investments and residual asset values (assets that may still be awaiting their ultimate sale) relative to an initial total investment. MOIC is calculated as follows:

MOIC = (Realized value of investment + Unrealized value of investment $) /($ Total amount of invested capital).

\section{KNOWLEDGE CHECK: MOIC CALCULATION}
Himitsu, a Private Equity Firm, makes an initial investment of JPY3.8 billion into ZZZ company in year zero. Eight years later, it sells its stake in ZZZ for JPY8.5 billion. Additional capital investments are made in Year 2 and in Year 3 for JPY1.2 billion and JPY200 million, respectively.

\begin{enumerate}
  \item Calculate the MOIC.
\end{enumerate}

\section{Solution}
$\mathrm{MOIC}=8.5 /(3.8+1.2+0.2)=1.63 \times$

\begin{center}
\begin{tabular}{|c|c|c|}
\hline
 & Amount & Year \\
\hline
Invested Capital & $(3,800,000)$ & 0 \\
\hline
............... & ............. & 1 \\
\hline
Additional Capital & $(1,200,000)$ & 2 \\
\hline
Additional Capital & $(200,000)$ & 3 \\
\hline
$\ldots \ldots \ldots \ldots \ldots \ldots$. & ............ & $\ldots$ \\
\hline
Liquidity Event & $8,500,000$ & 8 \\
\hline
MOIC & \multicolumn{2}{|c|}{$1.63 \times$} \\
\hline
IRR & \multicolumn{2}{|c|}{$20 \%$} \\
\hline
\end{tabular}
\end{center}

Although the MOIC ignores the timing of cash flows, it is easier to calculate, and it is intuitively easier to understand when someone says she received two or three times her initial investment. But how long it takes to realize this value does matter. A $2 \times$ return on one's initial investment would be phenomenal if the return were collected over two years but far less compelling if it took 15 years to realize.

In general, because private equity and real estate investments involve longer holding periods, there is less emphasis on evaluating them in terms of shorter-term portfolio correlation benefits. After a private equity fund has fully drawn in its monetary commitments, interim accounting values for a private equity partnership become less critical for a period of time because no incoming or outgoing cash flows may immediately hinge on such valuations. During this "middle period" in the life of a private equity fund, accounting values may not always be particularly reflective of the future potential realizations (and hence the expected returns) of the fund. It is not that the value of the investments is not actually rising and falling in the face of economic influences; rather, accounting conventions simply leave longer-lived investments marked at their initial cost for some time or make only modest adjustments to carrying value until clearer impairments or realization events take place.

Although most private equity managers are conservative in their interim valuations (awaiting actual realization events), a lagged mark-to-market process can at times offer a false sense of success, diminishing short-term investment worry but subsequently delivering the occasional back-end-loaded investment disappointment. Private equity interim valuations and real estate appraisals can certainly be inaccurate or skewed at times despite the best efforts of auditors to present a fair valuation.

The lagging impact of shorter-term economic events on the interim accounting valuations of these strategies makes them appear more resilient and less correlated than they really are. A more realistic picture may emerge when premature portfolio liquidations are forced on managers. The lack of transparency around such investments and the slowness to mark them to market can be incorrectly construed by investors as an overall lack of volatility.

Along this imperfect path and in an effort to benchmark how longer-term investments may be faring, private equity and real estate managers are generally judged by where they fall in terms of a quartile ranking, which depicts their performance against a cohort of peer investment vehicles constructed with similar investment attributes and funded around the same time, or what is often referred to as the same vintage year.

Real estate managers are also often judged by the cap rate being earned on their properties, which is simply the net operating income divided by the market value of the property. Traditionally, the price originally paid for the property was used instead of market value. However, this approach can be misleading because it ignores the current value of properties that may have been purchased many years earlier. Although the cap rate can be useful for a quick comparison of real estate investments, it should not be used as the sole indicator.

The large latitude around carried valuations for both private equity and real estate strategies makes any application of shorter-term risk metrics highly inappropriate.

\section{KNOWLEDGE CHECK}
\begin{enumerate}
  \item A shortcut methodology involving measuring the total value of all distributions and residual asset values relative to an initial total investment often cited by private equity and real estate managers is the:
\end{enumerate}

A. MAR ratio.

B. capital loss ratio. C. multiple of invested capital.

\section{Solution}
$\mathrm{C}$ is correct. Private equity and real estate managers often cite a multiple of invested capital ratio, where one simply measures the total value of all distributions and residual asset values (assets that may still be awaiting their ultimate sale) relative to the initial total investment.

\section{Hedge Funds: Leverage, Illiquidity, and Redemption Terms}
\section{Leverage}
Hedge funds may use leverage to obtain higher returns on their investments. Leverage has the effect of magnifying gains and losses because it allows for taking a larger position relative to the capital committed. Hedge funds leverage their portfolios by using derivatives or borrowing capital from prime brokers, negotiating with them to establish margin requirements, interest, and fees in advance of trading. The hedge fund deposits cash or other collateral into a margin account with the prime broker, and the prime broker essentially lends the hedge fund the shares, bonds, or derivatives to make additional investments. The margin account represents the hedge fund's net equity in its positions. The minimum margin required depends on the riskiness of the investment portfolio and the creditworthiness of the hedge fund.

Leverage is a large part of the reason that some hedge funds either earn larger-than-normal returns or suffer significant losses. If the margin account or the hedge fund's equity in a position declines below a certain level, the lender initiates a margin call and requests that the hedge fund put up more collateral. An inability to meet margin calls can have the effect of magnifying or locking in losses because the hedge fund may have to liquidate (close) the losing position. This liquidation can lead to further losses if the order size is sufficiently large to move the security's market price before the fund can sufficiently eliminate the position. Under normal conditions, the application of leverage may be necessary for yielding meaningful returns from given quantitative, arbitrage, or relative value strategies. But with added leverage comes increased risk.

The application of leverage to a strategy may not be revealed by studying the Sharpe ratio, the Sortino ratio, or any another financial performance metric. Instead, it may underpin the strategy undetected. Understanding the impact of leverage on a portfolio is more about how that track record was created. Analysts evaluating the alternative asset space should, therefore, scrutinize the returns of any alternative asset that relies on high leverage.

\section{Illiquidity and Potential Redemption Pressures}
A second qualitative issue for many hedge funds is the manner in which portfolios are marked to market. This issue is less important for long-short managers trading only publicly traded equities, but it still can cause problems when trading less liquid securities.

Consider the long-short equity manager involved in thinly traded small-capitalization stocks. Perhaps the manager was able at one point to source a block of stock from a retiring firm founder in order to establish the long exposure. An outside fund administrator uses a daily quoted price to value these shares, but the manager knows there is little chance of actually liquidating all the shares at that price. This problem will be worse for a convertible bond manager, a credit-oriented manager, or a structured product manager who faces wider bid-offer spreads and deals in securities that are particularly illiquid or that trade only "by appointment" (in other words, securities that trade so infrequently an appointment is almost necessary).

Proper valuations are important for calculating performance and meeting potential redemptions without incurring undue transaction costs for liquidating exposures. The frequency with which alternative assets are valued and how they are valued vary among funds. Hedge funds are generally initially valued by the manager internally, and these valuations are confirmed by an outside administrator on a daily or perhaps weekly basis. Performance is reported to investors by the administrator on a monthly or quarterly basis. The valuation may use market values or estimated values of the underlying positions. When market prices or quotes are used for valuation, funds may differ in which price or quote they use (bid price, ask price, average quote, or median quote). A more conservative and accurate approach is to use bid prices for long positions and ask prices for short positions because these are more realistic prices at which the positions could be closed. However, some managers use a simplifying approach whereby they take the average of the bid and the ask; this approach is not as accurate and could be misleading.

In some instances, the underlying positions may be in highly illiquid or even non-traded investments, and since such securities may have no reliable market values, it becomes necessary to estimate values. The following is a methodology that involves the categorization of investments into three buckets: Level 1, 2, and 3 asset pricing.

\begin{itemize}
  \item Level 1 assets have an exchange-traded, publicly traded price available that is mandated to be used for valuation purposes.

  \item Level 2 asset values use outside quotes from brokers when publicly traded (Level 1) prices are not available.

  \item Level 3 asset values are computed using only internal models when outsider broker (Level 2) quotes are not available or not reliable.

\end{itemize}

For Level 3 asset pricing, no matter the model used by a manager in such circumstances, it should be independently tested, benchmarked, and calibrated to industry-accepted standards to ensure a consistency of approach. Because of the potential for conflicts of interest when applying estimates of value, hedge funds must develop procedures for in-house valuation, communicate these procedures to clients, and adhere to them.

Notwithstanding best practice, the very nature of assets that can be valued only on a "mark-to-model" basis can and should be a concern for the alternative asset investor. A model may reflect an imperfect theoretical valuation and not a true liquidation value. The illiquid nature of these assets means that estimates, rather than observable transaction prices, may well have factored into any valuation. As a result, returns may be smoothed or overstated and the volatility of returns, understated. As a generalized statement, any investment vehicle that is heavily involved with Level 3-priced assets deserves increased scrutiny and due diligence.

\section{EXAMPLE 1}
\section{Hedge Fund Valuation}
\begin{enumerate}
  \item A hedge fund with a market-neutral strategy restricts its investment universe to domestic publicly traded equity securities that are actively traded on an exchange or between over-the-counter brokers. In calculating net
\end{enumerate}

\section{Performance Calculation and Appraisal of Alternative Investments}
asset value, the fund is most likely to use which of the following methods to value underlying positions?

A. Exchange last-trade pricing

B. Average quotes adjusted for liquidity

C. Bid price for shorts and ask price for longs

\section{Solution}
A is correct. The fund is most likely to use exchange-traded last-trade pricing (Level 1 pricing) or quotes based on the average of bids or the average of asks for OTC stocks. The securities are actively traded, so no liquidity adjustment is required. If the fund uses bid-ask prices, it will use ask prices for shorts and bid prices for longs; these are the prices at which the positions are closed.

Another factor that can lock in or magnify losses for hedge funds is investor redemptions. Redemptions frequently occur when a hedge fund is performing poorly. Redemptions may require the hedge fund manager to liquidate some positions and potentially receive particularly disadvantageous prices when forced to do so by redemption pressures, while also incurring transaction costs.

Funds sometimes charge redemption fees (typically payable to the remaining investors) to discourage redemption and to offset the transaction costs for remaining investors. Notice periodsprovide an opportunity for the hedge fund manager to liquidate a position in an orderly fashion without magnifying the losses. Lockup periods-time periods when investors cannot withdraw their capital-provide the hedge fund manager the required time to implement and potentially realize a strategy's expected results. If the hedge fund receives a drawdown request shortly after a new investment, the lockup period forces the investors who made the request to stay in the fund for a period of time rather than be allowed to immediately withdraw. In addition, funds sometimes impose a gate, which limits or restricts redemptions for a period of time. Investors should be aware of their liquidity needs before investing in a fund with restrictive provisions.

A hedge fund's ability to demand a long lockup period while raising a significant amount of investment capital depends a great deal on the reputation of either the firm or the hedge fund manager. Funds of hedge funds may offer more redemption flexibility than is afforded to direct investors in hedge funds because of special redemption arrangements with the underlying hedge fund managers, the maintenance of added cash reserves, access to temporary bridge-loan financing, or the simple avoidance of less liquid hedge fund strategies.

Ideally, redemption terms should be designed to match the expected liquidity of the assets being invested in, but even with careful planning, an initial drawdown can turn into something far more serious when it involves illiquid and obscure assets. These left-tailed loss events are not easily modeled for hedge funds.

\section{EXAMPLE 2}
\section{Effect of Redemption}
\begin{enumerate}
  \item A European credit hedge fund has a very short redemption notice periodone week-because the fund's managers believe it invests in highly liquid asset classes and is market neutral. The fund has a small number of holdings that represent a significant portion of the outstanding issue of each holding. The fund's lockup period has expired. Unfortunately, in one particular month, because of the downgrades of two large holdings, the hedge fund has a drawdown (decline in NAV) of more than $10 \%$. The declines in value of the two holdings result in margin calls from their prime broker, and the drawdown results in requests to redeem $50 \%$ of total partnership interests. The combined requests are most likely to:
\end{enumerate}

A. force the hedge fund to liquidate or unwind $50 \%$ of its positions in an orderly fashion throughout the week.

B. have little effect on the prices received when liquidating the positions because the hedge fund has a week before the partnership interests are redeemed

C. result in a forced liquidation, which is likely to further drive down prices and result in ongoing pressures on the hedge fund as it tries to convince nervous investors to remain in the fund.

\section{Solution}
$\mathrm{C}$ is correct. One week may not be enough time to unwind such a large portion of the fund's positions in an orderly fashion that does not also further drive down prices. A downgrade is not likely to have a temporary effect, so even if other non-losing positions are liquidated to meet the redemption requests, it is unlikely that the two large holdings will return to previous or higher values in short order. Also, the hedge fund may have a week to satisfy the requests for redemptions, but the margin call must be met immediately. Overall, sudden redemptions at the fund level can have a cascading negative impact on a fund.

The previous discussion applies mostly to liquid alternative asset strategies, principally hedge funds. In the world of private equity and real estate alternative assets, other methodologies are used to measure relative performance. Here, we find more issues with lagged and smoothed pricing.

It is important to note that although the ratios we have considered are among the best performance and risk tools available, they can still lead to inappropriate conclusions, as shown in the following example.

\section{EXAMPLE 3}
Steamboat Structured Products LLC is a manager that specializes in the purchase and sale of residential mortgage-backed securities (RMBSs), sometimes hedged with other put option and short equity index exposures. For the most part, its strategy is geared to earn an attractive monthly mortgage payment on as low of a loan-to-value ratio of a secured property as possible. Steamboat managers are generally very good at sourcing such investments.

However, Steamboat's managers discover that in the wake of the 2008 financial crisis, many small pieces of "odd-lot" RMBS paper are being sold by financial institutions across the United States, sometimes with face values of just USD200,000-USD400,000. This paper is often offered at a $10 \%$ or $15 \%$ discount to the price at which a more sizable and significant "round-lot" block of USD1 million-USD2 million of the same type of paper might trade.

Steamboat managers know that their outside administrator will place only a single round-lot valuation on each security identifier (often referred to as a CUSIP) in their portfolio, and thus they can't resist starting to buy as many of these odd-lot offerings as possible. When they do, their administrator does indeed immediately mark them higher, which, as Steamboat continues this practice month after month, creates a lovely track record of constant "trading profits" on top of their natural coupon income. To the greatest extent possible, Steamboat

\section{Performance Calculation and Appraisal of Alternative Investments}
hopes to create odd-lot "matchers" to eventually aggregate its exposures into larger tradable blocks. But what happens instead is that the firm ends up with 2,000 small-position line items in its book and a glorious track record.

This is a story of an illiquid asset class that is naturally hard to trade and mark. Steamboat's Sharpe ratio, Sortino ratio, and Calmar ratio all look stronger than they really should be, mostly because of accounting conventions. If Steamboat were ever forced to sell its portfolio, the odd-lot discount earned as trading revenue would largely disappear and be replaced by the true liquidation values that Steamboat would find in the market from others for its accumulated odd-lot portfolio.

Performance ratio analysis must, therefore, be discounted when dealing with illiquid securities as described here. Although not necessarily fraudulent, standard accounting practices can be purposefully gamed.

\section{CALCULATING FEES AND RETURNS}
calculate and interpret returns of alternative investments both before and after fees

\section{Summary-Calculating Fees and Returns of Alternative Investments}
\begin{itemize}
  \item When comparing the performance of alternative investments versus an index, the analyst must be aware that indexes for alternative investments may be subject to a variety of biases, including survivorship and backfill biases.

  \item Alternative investment managers generally charge an incentive performance fee as well as a management fee based on assets under management (AUM) or committed capital. However, analysts need to be aware of any custom fee arrangements in place that will affect the calculation of fees and performance. These can include such arrangements as fee discounts based on custom liquidity terms or significant asset size; special share classes, such as "founders' shares"; and a departure from the typical management fee + performance fee structure in favor of "either/or" fees.

\end{itemize}

\section{Custom Fee Arrangements}
Although "2 and 20" and "1 and 10" are commonly quoted fee structures for hedge funds and funds of funds, respectively, many variations exist.

\begin{enumerate}
  \item Fees based on liquidity terms and asset size: Hedge funds may charge different rates depending on the liquidity terms that an investor is willing to accept (longer lockups are generally associated with lower fees), and hedge fund managers may discount their fees for larger investors or for placement agents who introduced these investors. Different investors in the same fund may well face different fee structures. Hedge fund managers negotiate terms, including fees and notice and lockup periods, with individual investors via side letters, which are special amendments to a standard offering memorandum's terms and conditions. For perspective, management fees for large limited partners (LPs) could range from $0.5 \%$ to $1.5 \%$, with incentive fees reduced to $10 \%-15 \%$, depending on the mandate. Such reductions can be meaningful in terms of net realized returns. However, some smaller hedge funds with strong performance (and capacity constraints) are able to maintain higher fees and may even turn down business from larger investors rather than agree to a lower fee.

  \item Founders' shares: As a way to entice early participation in start-up and emerging hedge funds, managers have increasingly offered incentives known as founders' class shares. Founders' shares entitle investors to a lower fee structure, such as 1.5 and 10 rather than 2 and 20 , and are typically available to be applied only to the first USD100 million in assets, although cutoff thresholds vary. Another option is to reduce the fees for early founders' share investors once the fund achieves a critical mass of assets or performance targets. Both paths act as an incentive to spur investors to make faster investment commitments than might otherwise be the case.

  \item Either/or fees: As a pushback against high hedge fund fees, institutional investors, such as the Teacher Retirement System of Texas, requested a new fee model that some hedge fund managers have started to accept in return for significant investors. Managers agree either to charge a 1\% management fee (simply to cover expenses during down years) or to receive a $30 \%$ incentive fee above a mutually agreed-on annual hurdle (to incentivize and reward managers during up years), whichever is greater. If a manager were to go without profits for a year or two, the $1 \%$ management fee effectively becomes an advance against an eventual $30 \%$ incentive-fee year (thereby reducing that future-year incentive fee by the prior years' advanced management fees. Although very different in structure from the traditional 2 and 20, such novel fee structures designed to reward performance and the delivery of true alpha above a benchmark are likely to become even more in demand in the institutional hedge fund industry. Hedge funds will likely also endeavor to charge high-net-worth investors (with smaller commitment sizes) more traditional fees.

\end{enumerate}

The following examples demonstrate fee structures and their effect on the resulting returns to investors.

\section{EXAMPLE 4}
\section{Incentive Fees Relative to Waterfall Types}
\begin{enumerate}
  \item A PE fund invests USD15 million in a nascent luxury yacht manufacturer and USD17 million in a new casino venture. The yacht manufacturer generates a USD9 million profit when the company is acquired by a larger competitor, but the casino venture turns out to be a flop when its state licensing is eventually denied and it generates a USD10 million loss. If the manager's carried interest incentive fee is $20 \%$ of the profits, what would this incentive be with a European-style waterfall whole-of-fund approach, and what would it be if the incentive is paid on an American-style waterfall deal-by-deal basis (assuming no clawback applies)?
\end{enumerate}

\section{Solution}
In aggregate, the fund lost money (+USD9 million - USD10 million = -USD1 million), so with a European-style whole-of-fund waterfall and

\section{Performance Calculation and Appraisal of Alternative Investments}
assuming the time period for the gain and the loss are the same, there is no incentive fee. With an American-style waterfall, the general partner (GP) could still earn 20\% $\times$ USD9 million = USD1.8 million on the yacht company, thereby further compounding the loss to the ultimate investor to USD2.8 million net of fees.

\begin{center}
\begin{tabular}{lccc}
\hline
 & Aggregate & Yacht Company & Casino Venture \\
\hline
Investment & USD32 $\mathrm{m}$ & USD15 m & USD17 m \\
Profit/Loss & - USD1 $\mathrm{m}$ & USD9 $\mathrm{m}$ & $-U S D 10 \mathrm{~m}$ \\
Incentive by Deal &  & $20 \% \times$ USD9 $\mathrm{m}=$ & USD0 m \\
USD1.8 m &  &  &  \\
Total Incentive & USD0 $\mathrm{m}$ & \multicolumn{2}{c}{USD1.8 m} \\
\hline
\end{tabular}
\end{center}

If the gain and loss in this example transpired sequentially over different years, perhaps with the yacht company gain occurring first and then the Casino venture loss later on, the outcome for a European-style waterfall would typically result in an initial accrued incentive fee for the yacht manufacturer gain, but if there is a clawback provision in place, then there would be a clawback of that fee for the investor in the subsequent year when the casino venture loss is eventually realized, still resulting in no overall incentive fee. Waterfall language and clawback provisions on fees are very important to study and understand in offering memorandums, and these terms can vary widely.

\section{EXAMPLE 5}
\section{Fee and Return Calculations}
AWJ Capital is a hedge fund with USD100 million of initial investment capital. It charges a $2 \%$ management fee based on year-end AUM and a $20 \%$ incentive fee. In its first year, AWJ Capital has a 30\% return. Assume management fees are calculated using end-of-period valuation.

\begin{enumerate}
  \item What are the fees earned by AWJ if the incentive and management fees are calculated independently? What is an investor's effective return given this fee structure?
\end{enumerate}

\section{Solution}
AWJ fees:

USD130 million $\times 2 \%=$ USD2. 6 million management fee .

(USD130 - USD100) million $\times 20 \%=$ USD6 million incentive fee .

Total fees to AWJ Capital = USD8.6 million .

Investor return: (USD130 - USD100 - USD8.6) million/USD100 million = $21.40 \%$.

\begin{enumerate}
  \setcounter{enumi}{1}
  \item What are the fees earned by AWJ assuming that the incentive fee is calculated from the return net of the management fee? What is an investor's net return given this fee structure?
\end{enumerate}

\section{Solution}
USD130 million $\times 2 \%=\operatorname{USD} 2.6$ million management fee

(USD130 - USD100 - USD2.6) million $\times 20 \%=$ USD5.48 million incentive fee.

Total fees to AWJ Capital = USD8.08 million

Investor return: (USD130 - USD100 - USD8.08) million/USD100 million $=$ $21.92 \%$

\begin{enumerate}
  \setcounter{enumi}{2}
  \item If the fee structure specifies a hurdle rate of $5 \%$ and the incentive fee is based on returns in excess of the hurdle rate, what are the fees earned by AWJ assuming the performance fee is calculated net of the management fee? What is an investor's net return given this fee structure?
\end{enumerate}

\section{Solution}
$\mathrm{USD} 130$ million $\times 2 \%=\mathrm{USD} 2.6$ million management fee

(USD130 - USD100 - USD5 - USD2.6) million $\times 20 \%=$ USD4.48 million incentive fee.

Total fees to AWJ Capital = USD7.08 million.

Investor return: (USD130 - USD100 - USD7.08) million/USD100 million $=$ $22.92 \%$

\begin{enumerate}
  \setcounter{enumi}{3}
  \item In the second year, the fund value declines to USD110 million. The fee structure is as specified for Question 1 but also includes the use of a high-water mark (computed net of fees). What are the fees earned by AWJ in the second year? What is an investor's net return for the second year given this fee structure?
\end{enumerate}

\section{Solution}
USD110 million $\times 2 \%=$ USD2.2 million management fee

No incentive fee because the fund has declined in value.

Total fees to AWJ Capital = USD2.2 million.

Investor return: (USD110 - USD2.2 - USD121.4) million/USD121.4 million = $-11.20 \%$

The beginning capital position in the second year for the investors is (USD130 - USD8.6) million = USD121.4 million. The ending capital position at the end of the second year is (USD110 - USD2.2) million = USD107.8 million.

\begin{enumerate}
  \setcounter{enumi}{4}
  \item In the third year, the fund value increases to USD128 million. The fee structure is as specified in Questions 1 and 4 . What are the fees earned by AWJ in the third year? What is an investor's net return for the third year given this fee structure?
\end{enumerate}

\section{Solution}
USD128 million $\times 2 \%=\mathrm{USD} 2.56$ million management fee .

\section{Performance Calculation and Appraisal of Alternative Investments}
(USD128 - USD121.4) million $\times 20 \%=\mathrm{USD} 1.32$ million incentive fee.

The USD121.4 million represents the high-water mark established at the end of Year 1 .

Total fees to AWJ Capital = USD3.88 million.

Investor return: (USD128 - USD3.88 - USD107.8) million/USD107.8 million $=15.14 \%$

The ending capital position at the end of Year 3 is USD124.12 million. This amount is the new high-water mark.

As the previous example illustrates, the return to an LP investor in a fund may differ significantly from the quoted return for the fund as a whole, which generally reflects the return that a "Day 1 " investor who made no capital movements would have earned. Hedge fund databases and indexes generally report performance net of fees. If fee structures vary, however, the actual net-of-fees returns earned by various investors often may vary from the one included in a given database or index.

The multi-layered fee structure of funds of hedge funds has the effect of further diluting returns to the investor, but as discussed, this disadvantage may be balanced by positive features, such as access to a diversified portfolio and to hedge funds that may otherwise be closed to direct investments, as well as expertise in due diligence in hedge fund selection. There is thus both added cost and added value.

Generally, over time many funds of funds have earned a reputation for being "fast" money because their managers tend to be the first to redeem their investment when a hedge fund performs poorly. They may also have negotiated more favorable redemption terms-a shorter lockup or notice period, for example. With the overall compression of hedge fund returns and issues of excessive fee layering, many fund-of-funds managers have been pressured to drop their incentive fees and simply charge a flat management fee.

\section{KNOWLEDGE CHECK}
\begin{enumerate}
  \item Fund offering documents typically offer terms that include a whereby incentive fees will accrue and subsequently be paid only on new earnings above and beyond the recoupment of any prior losses.
\end{enumerate}

\section{Solution}
Fund offering documents typically offer terms that include a high-water mark, whereby incentive fees will accrue and subsequently be paid only on new earnings above and beyond the recoupment of any prior losses.

\begin{enumerate}
  \setcounter{enumi}{1}
  \item True or false: Advantages of funds-of-hedge funds include due diligence in selecting individual hedge funds, access to hedge funds that may be closed to direct investments, and dilution of returns to the investor.
\end{enumerate}

Explain your selection.

A. True

B. False

\section{Solution}
False. Although these three attributes are indeed true of funds of hedge funds, the dilution of returns to the investor is a disadvantage for the investor, not an advantage. The "due diligence" and "access" attributes are advantages.

\section{EXAMPLE 6}
\section{Comparison of Returns: Investment Directly into a Hedge Fund or through a Fund of Hedge Funds}
An investor is contemplating investing EUR100 million in either the ABC Hedge Fund (ABC HF) or the XYZ Fund of Funds (XYZ FOF). XYZ FOF has a "1 and 10" fee structure and invests 10\% of its AUM in ABC HF. ABC HF has a standard "2 and 20" fee structure with no hurdle rate. Management fees are calculated on an annual basis on AUM at the beginning of the year. For simplicity, assume that management fees and incentive fees are calculated independently. ABC HF has a $20 \%$ return for the year before management and incentive fees.

\begin{enumerate}
  \item Calculate the return to the investor from investing directly in $\mathrm{ABC} \mathrm{HF}$.
\end{enumerate}

\section{Solution}
ABC HF has a profit before fees on a EUR100 million investment of EUR20 million (= EUR100 million $\times 20 \%)$. The management fee is EUR2 million (= EUR100 million $\times 2 \%)$, and the incentive fee is EUR4 million (= EUR20 million $\times 20 \%)$. The return to the investor is $14 \%$ [= $(20-2-4) / 100]$.

\begin{enumerate}
  \setcounter{enumi}{1}
  \item Calculate the return to the investor from investing in XYZ FOF. Assume that the other investments in the XYZ FOF portfolio generate the same return before management fees as those of ABC HF and that XYZ FOF has the same fee structure as ABC HF.
\end{enumerate}

\section{Solution}
XYZ FOF earns a 14\% return or EUR14 million profit after fees on EUR100 million invested with hedge funds. XYZ FOF charges the investor a management fee of EUR1 million (= EUR100 million $\times 1 \%)$ and an incentive fee of EUR1.4 million (= EUR14 million $\times 10 \%)$. The return to the investor is $11.6 \%$ $[=(14-1-1.4) / 100]$.

\begin{enumerate}
  \setcounter{enumi}{2}
  \item Why would the investor choose to invest in a fund of funds instead of a hedge fund given the effect of the "double fee" demonstrated in the answers to Questions 1 and 2?
\end{enumerate}

\section{Solution}
This scenario assumes that returns are the same for all underlying hedge funds. In practice, this result will not likely occur, and XYZ FOF may provide due diligence expertise and potentially valuable diversification.

\section{Alignment of Interests and Survivorship Bias}
The alternative asset business is attractive to portfolio managers because of how significant the fees can be if the fund performs well and AUM are significant. But as discussed previously, high fees destroy value and reduce the attractiveness of alternative investing. If a hedge fund manager can stay in business for just four to five years with acceptable returns, incentive fee allocations to the general partner can be substantive.

\section{Performance Calculation and Appraisal of Alternative Investments}
In comparison, the commitment of a private equity or real estate manager to stay in business for 6-10 years, especially with a structure that (1) first attempts to return all capital to investors, then (2) attempts in many instances to pay a preferred minimum return to investor, and then (3)-only after (1) and (2) occur-allows a manager to realize his own incentive fees is arguably a more aligned overall incentive structure than that of hedge funds. However, the overall time commitment for the investor is correspondingly of a much longer duration. The investor runs a larger risk of being locked into an investment that might be a disappointment.

One study suggests that more than a quarter of all hedge funds fail within the first three years because of performance problems that result in investor defections and the ultimate failure to generate sufficient revenue to cover the fund's operating costs. This is one reason survivorship bias is a major problem with hedge fund indexes. Survivorship bias results from including only current investment funds in a database. As such, the record excludes the returns of funds no longer available in the marketplace (i.e., funds that were liquidated).

Backfill bias is another problem: Certain surviving hedge funds may be added to databases and various hedge fund indexes only after they are initially successful and start to report their returns. Because of survivorship and backfill biases, hedge fund indexes may not reflect actual hedge fund performance but, rather, reflect only the performance of hedge funds that performed well and thus "survived."

\section{EXAMPLE 7}
\section{(1) Clawbacks Due to Return Timing Differences}
\begin{enumerate}
  \item The Granite Rock Fund makes investments in leveraged-buyout Company A and start-up Company B, each for USD10 million. One year later, the leveraged-buyout company returns a USD14 million profit, and two years later, the start-up company turns into a complete bust, deemed to be worth zero.
\end{enumerate}

If the GP's carried interest incentive fee is $20 \%$ of aggregate profits and there is a clawback provision, how much carried interest will the GP initially accrue and ultimately receive?

\section{Solution}
From leveraged-buyout Company A, the GP would initially accrue a $20 \%$ of USD14 million profit at the end of the first year, equal to USD2.8 million. Typically, this amount would be held in escrow for the benefit of the GP but not actually paid.

But then the GP loses USD10 million (from start-up Company B) of the initial USD14 million gain, so the aggregate whole-of-fund gain at the end of the second year would be only USD4 million; this amount times $20 \%$ would result in only an USD800,000 incentive fee. The general partner would then have to return USD2 million of the previously accrued incentive fees to the capital accounts of LP investors because of the clawback provision.

\section{(2) Soft and Hard Hurdles}
\begin{enumerate}
  \item A real estate investment fund has a USD100 million initial drawdown structure in its first year and fully draws this capital to purchase a property. The fund has a soft hurdle preferred return to investors of $8 \%$ per annum and an $80 \% / 20 \%$ carried interest incentive split thereafter. At the end of Year 2, the property is sold for a total of USD160 million.
\end{enumerate}

What are the correct distributions to the LPs and to the GP? And how would these have been different if the real estate investment fund had a hard hurdle of $8 \%$ per annum instead of a soft hurdle?

\section{Solution}
One needs to construct a waterfall of cash flows.

First, the LPs would be due their USD100 million initial investment.

Then, they would be due USD16 million (8\% preferred return on initial capital for two years).

The soft hurdle has been met, and the GP is ultimately due $20 \%$ of USD60 million, or USD12 million, which would be paid to the GP next as a catch-up to the achieved hurdle return.

The residual amount would be USD160 million - USD100 million - USD16 million - USD12 million = USD32 million. This amount would then be split 80\% to the LPs and $20 \%$ to the GP, or USD25.6 million and USD6.4 million, respectively.

So, the total payout with a soft annual hurdle of 8\% of the USD160 million would end up with the following waterfall:

\begin{center}
\begin{tabular}{lcc}
\hline
 & LP & GP \\
\hline
Return of Capital & USD100 m &  \\
$8 \%$ Preferred per Annum & USD16 m &  \\
GP Catch-Up 20\% &  & USD12 m \\
$80 \% / 20 \%$ Split & USD25.6 m & USD6.4 $\mathrm{m}$ \\
Total Payout & USD141.6 m & USD18.4 m \\
\hline
\end{tabular}
\end{center}

If the fund instead had a hard hurdle rate, only the amount above the USD100 return of capital and USD16 million preferred return would be subject to the $20 \%$ carried interest incentive to the GP: $20 \% \times$ USD44 million = USD8.8 million, quite a bit less than the carried interest payment with the soft hurdle. The LPs would be due the balance of USD35.2 million (USD44 million - USD8.8 million incentive). This would result in the following total payout:

\begin{center}
\begin{tabular}{lcc}
\hline
 & LP & GP \\
\hline
Return of Capital & USD100 m &  \\
$8 \%$ Preferred per Annum & USD16 m &  \\
$80 \% / 20 \%$ Split above Hurdle & USD35.2 m & USD8.8 $\mathrm{m}$ \\
Total Payout & USD151.2 m & USD8.8 $\mathrm{m}$ \\
\hline
 &  &  \\
\hline
\end{tabular}
\end{center}

\section{PRACTICE PROBLEMS}
\begin{enumerate}
  \item The Sharpe ratio is a less-than-ideal performance measure for alternative investments because:
\end{enumerate}

A. it uses a semi-deviation measure of volatility.

B. returns of alternative assets are not normally distributed.

C. alternative assets exhibit low correlation with traditional asset classes.

\begin{enumerate}
  \setcounter{enumi}{1}
  \item Which of the following statements regarding private equity performance calculations is true?
\end{enumerate}

A. The money multiple calculation relies on the amount and timing of cash flows.

B. The IRR calculation involves the assumption of two rates.

C. Because private equity funds have low volatility, accounting conventions allow them to use a lagged mark-to-market process.

\begin{enumerate}
  \setcounter{enumi}{2}
  \item Which of the following statements is not true of mark-to-model valuations?
\end{enumerate}

A. Return volatility may be understated.

B. Returns may be smooth and overstated.

C. A calibrated model will produce a reliable liquidation value.

\begin{enumerate}
  \setcounter{enumi}{3}
  \item An analyst wanting to assess the downside risk of an alternative investment is least likely to use the investment's:
\end{enumerate}

A. Sortino ratio.

B. value at risk (VaR).

C. standard deviation of returns.

\begin{enumerate}
  \setcounter{enumi}{4}
  \item The following performance data are provided for an alternative investment.
\end{enumerate}

Average Annual Compounded Return

\begin{center}
\begin{tabular}{cccc}
\hline
1 Year & 3 Years & 5 Years & Since Inception \\
$5.3 \%$ & $6.2 \%$ & $4.7 \%$ & $4.4 \%$ \\
\hline
\end{tabular}
\end{center}

Assume the maximum drawdown risk is steady at $10.2 \%$ over each time period. Assume the average drawdown risk is steady at $6.8 \%$ over each time period.

Using the data provided, calculate the Calmar ratio the way it is typically calculated. The Calmar ratio is the closest to:
A. 0.46 .
B. 0.61 .
C. 0.65 .

\begin{enumerate}
  \setcounter{enumi}{5}
  \item United Capital is a hedge fund with USD250 million of initial capital. United charges a $2 \%$ management fee based on assets under management at year end and a $20 \%$ incentive fee based on returns in excess of an $8 \%$ hurdle rate. In its first year, United appreciates 16\%. Assume management fees are calculated using end-of-period valuation. The investor's net return assuming the performance fee is calculated net of the management fee is closest to:
A. $11.58 \%$
B. $12.54 \%$.
C. $12.80 \%$.

  \item Capricorn Fund of Funds invests GBP100 million in each of Alpha Hedge Fund and ABC Hedge Fund. Capricorn Fund of Funds has a "1 and 10" fee structure. Management fees and incentive fees are calculated independently at the end of each year. After one year, net of their respective management and incentive fees, Capricorn's investment in Alpha is valued at GBP80 million and Capricorn's investment in ABC is valued at GBP140 million. The annual return to an investor in Capricorn Fund of Funds, net of fees assessed at the fund-of-funds level, is closest to:
A. $7.9 \%$.
B. $8.0 \%$.
C. $8.1 \%$.

  \item The following information applies to Rotunda Advisers, a hedge fund:

\end{enumerate}

\begin{itemize}
  \item USD288 million in AUM as of prior year end

  \item $2 \%$ management fee (based on year-end AUM)

  \item $20 \%$ incentive fee calculated:

  \item net of management fee

  \item using a $5 \%$ soft hurdle rate

  \item using a high-water mark (high-water mark is USD357 million)

  \item Current-year fund gross return is $25 \%$.

\end{itemize}

The total fee earned by Rotunda in the current year is closest to:
A. USD7.20 million.
B. USD20.16 million.
C. USD21.60 million.

\begin{enumerate}
  \setcounter{enumi}{8}
  \item A hedge fund has the following fee structure:
\end{enumerate}

$\begin{array}{ll}\text { Annual management fee based on year-end AUM } & 2 \%\end{array}$

$\begin{array}{ll}\text { Incentive fee } & 20 \%\end{array}$

Hurdle rate before incentive fee collection starts $\quad 4 \%$

$\begin{array}{ll}\text { Current high-water mark } & \text { USD610 million }\end{array}$

The fund has a value of USD583.1 million at the beginning of the year. After one year, it has a value of USD642 million before fees. The net percentage return to an investor for this year is closest to:

A. $6.72 \%$. B. $6.80 \%$.

C. $7.64 \%$.

\section{SOLUTIONS}
\begin{enumerate}
  \item B is correct. The Sharpe ratio assumes normally distributed returns. However, alternative assets tend to have non-normal return distributions with significant skewness (fat tails in one direction or the other) and kurtosis (sharper peak than a normal distribution has, with fatter tails). Therefore, the Sharpe ratio may not be a good risk-adjusted performance measure to rely on for alternative investments.
\end{enumerate}

A is incorrect because the Sharpe ratio does not use a semi-deviation measure of volatility; it uses standard deviation. The Sortino ratio uses a semi-deviation measure of volatility. Further, the use of semi-deviation instead of standard deviation actually makes the Sortino ratio a more attractive measure of alternative asset performance than the Sharpe ratio.

$\mathrm{C}$ is incorrect because correlation does not enter into the calculation of the Sharpe ratio. However, it is true that alternative assets can have low correlations with other asset classes.

\begin{enumerate}
  \setcounter{enumi}{1}
  \item B is correct. The determination of an IRR involves certain assumptions about a financing rate to use for outgoing cash flows (typically a weighted average cost of capital) and a reinvestment rate assumption to make on incoming cash flows (which must be assumed and may or may not actually be earned).
\end{enumerate}

A is incorrect because the money multiple calculation completely ignores the timing of cash flows.

$\mathrm{C}$ is incorrect because it is somewhat of a reversal of cause and effect: Private equity (PE) funds can appear to have low volatility because of the lag in their mark-to-market process. It's not that PE investments don't actually rise and fall behind the scenes with economic influences, but accounting conventions may simply leave longer-lived investments marked at their initial cost for some time or with only modest adjustments to such carrying value until known impairments or realization events begin to transpire. Also, because PE funds are not easily marked to market, their returns can appear somewhat smoothed, making them appear more resilient and less correlated with other assets than they really are. The slowness to re-mark them can unfortunately be confused by investors as an overall lack of volatility.

\begin{enumerate}
  \setcounter{enumi}{2}
  \item $\mathrm{C}$ is correct. It is not true that a calibrated model will produce a reliable liquidation value in a mark-to-model valuation. The need to use a model for valuation arises when an asset is so illiquid that there are not reliable market values available. A model may reflect only an imperfect theoretical valuation, not a true liquidation value, should liquidation become necessary. The illiquid nature of alternative assets means that estimates, rather than observable transaction prices, may have been used for valuation purposes.

  \item $\mathrm{C}$ is correct. Downside risk measures focus on the left side of the return distribution curve, where losses occur. The standard deviation of returns assumes that returns are normally distributed. Many alternative investments do not exhibit close-to-normal distributions of returns, which is a crucial assumption for the validity of a standard deviation as a comprehensive risk measure. Assuming normal probability distributions when calculating these measures will lead to an underestimation of downside risk for a negatively skewed distribution. Both the Sortino ratio and the VaR measure are measures of downside risk.

  \item B is correct. The Calmar ratio is typically calculated using the prior three years of performance and is a comparison of the average annual compounded return to its maximum drawdown risk. For this particular investment, the Calmar ratio is calculated as follows:

\end{enumerate}

$6.2 \%$ (average compounded return over the past three years) $/ 10.2 \%$ (maximum drawdown $)=0.60784 \approx 0.61$.

\begin{enumerate}
  \setcounter{enumi}{5}
  \item B is correct. The net investor return is $12.54 \%$, calculated as follows:
\end{enumerate}

End-of-year capital $=$ USD250 million $\times 1.16=$ USD290 million .

Management fee $=$ USD290 million $\times 2 \%=$ USD5.8 million.

Hurdle amount $=8 \%$ of USD250 million $=$ USD20 million.

Incentive fee $=(U S D 290-$ USD250 - USD20 - USD5.8) million $\times 20 \%=$ USD2.84 million.

Total fees to United Capital $=($ USD5.8 + USD2.84) million $=$ USD8.64 million.

Investor net return: (USD290 - USD250 - USD8.64)/USD250 = 12.54\%.

\begin{enumerate}
  \setcounter{enumi}{6}
  \item A is correct because the net investor return is $7.9 \%$, calculated as follows:
\end{enumerate}

First, note that " 1 and 10 " refers to a $1 \%$ management fee and a $10 \%$ incentive fee.

End-of-year capital $=$ GBP140 million + GBP80 million $=$ GBP220 million .

Management fee $=$ GBP220 million $\times 1 \%=$ GBP2.2 million.

Incentive fee $=($ GBP220 - GBP200 $)$ million $\times 10 \%=$ GBP2 million.

Total fees to Capricorn $=(G B P 2.2+$ GBP2 $)$ million $=$ GBP4.2 million.

Investor net return: (GBP220 - GBP200 - GBP4.2)/GBP200 = 7.9\%.

\begin{enumerate}
  \setcounter{enumi}{7}
  \item A is correct. Although the gross return of Rotunda results in a USD360 million gross NAV, the deduction of the USD7.2 million incentive fee brings NAV to USD352.8 million, which is below the prior high-water mark. Rotunda earns a management fee of USD7.20 million but does not earn an incentive fee because the year-end fund value net of management fee does not exceed the prior high-water mark of USD357 million. Since Rotunda is still also below the prior-year high-water mark, the hurdle rate of return is also basically irrelevant in this fee calculation.
\end{enumerate}

The specifics of this calculation are as follows:

End-of-year AUM = Prior year-end AUM $\times(1+$ Fund return $)=$ USD288 million $\times 1.25=$ USD360 million.

USD360 million $\times 2 \%=$ USD7.20 million management fee.

USD360 million - USD7.2 million = USD352.8 million AUM net of management fee.

The year-end AUM net of fees do not exceed the USD357 million high-water mark. Therefore, no incentive fee is earned.

\begin{enumerate}
  \setcounter{enumi}{8}
  \item C is correct. The management fee for the year is
\end{enumerate}

USD $642 \times 0.02=\mathrm{USD} 12.84$ million

Because the ending gross value of the fund of USD642 million exceeds the high-water mark of USD610 million, the hedge fund can collect an incentive fee on gains above this high-water mark but net of the hurdle rate of return. The incentive fee calculation becomes $\{\mathrm{USD} 642-[\mathrm{USD} 610 \times(1+0.04)]\} \times 0.20=\mathrm{USD} 1.52$ million.

The net return to the investor for the year is

[(USD642 - USD12.84 - USD1.52)/USD583.1] - $1 \approx 0.07638 \approx 7.64 \%$ LEARNING MODULE

\begin{center}
\includegraphics[max width=\textwidth]{2023_05_04_36535b8d80b32081d422g-441}
\end{center}

\section{Private Capital, Real Estate, Infrastructure, Natural Resources, and Hedge Funds}
\section{LEARNING OUTCOME}
\begin{center}
\begin{tabular}{c|l}
Mastery & The candidate should be able to: \\
\hline
$\square$ & explain investment characteristics of private equity \\
$\square$ & explain investment characteristics of private debt \\
$\square$ & explain investment characteristics of real estate \\
$\square$ & explain investment characteristics of infrastructure \\
$\square$ & explain investment characteristics of natural resources \\
$\square$ & explain investment characteristics of hedge funds \\
\end{tabular}
\end{center}

\section{SUMMARY - PRIVATE CAPITAL}
explain investment characteristics of private equity explain investment characteristics of private debt

\begin{itemize}
  \item Private Capital is a broad term for funding provided to companies sourced from neither the public equity nor debt markets. Capital provided in the form of equity investments is called private equity, whereas capital provided as a loan or other form of debt is called private debt.

  \item Private equity refers to investment in privately owned companies or in public companies intended to be taken private. Key private equity investment strategies include leveraged buyouts (e.g., MBOs and MBIs) and venture capital. Primary exit strategies include trade sale, IPO, and recapitalization.

  \item Private debt refers to various forms of debt provided by investors to private entities. Key private debt strategies include direct lending, mezzanine debt, and venture debt. Private debt also includes specialized strategies, such as CLOs, unitranche debt, real estate debt, and infrastructure debt.

\end{itemize}

\section{Introduction and Overview of Private Capital}
Private capital is the broad term for funding provided to companies that is sourced neither from the public markets, such as from the sale of equities, bonds, and other securities on exchanges, nor from traditional institutional providers, such as a government or bank. Capital raised from sources other than public markets and traditional institutions and in the form of an equity investment is called private equity (PE). Similarly sourced capital extended to companies through a loan or other form of debt is referred to as private debt. Private capital relates to the entire capital structure, comprising private equity and private debt.

Private capital largely consists of private investment funds and entities that invest in the equity or debt securities of privately held companies, real estate, or other assets. Many private investment firms have private equity and private debt arms; however, these teams typically refrain from investing in the same assets or businesses to avoid overexposure to a single investment and to avoid the conflict of interest that arises from being invested in both the equity and debt of an issuer. Private investment firms, even those with private debt arms, are typically referred to as "private equity firms." Although private equity is the largest component of private capital, "private equity" as a comprehensive generic term is inaccurate because other forms of private alternative finance have grown considerably in size and popularity.

\section{Description of Private Equity}
Private equity refers to investment in privately owned companies or in public companies with the intent to take them private. As business conditions and the availability of financing change, private equity firms may change focus. A firm may manage many private equity funds, each composed of several investments, and the companies owned are often called portfolio companies because they will be part of a private equity fund portfolio. Leveraged buyouts, venture capital, and growth capital represent the primary private equity strategies.

Private equity activity has grown over time. Exhibit 1 shows global private equity fundraising from 2010 to 2020 (by sub-asset class). Detailed information on private equity activity is not always readily available.

\section{Exhibit 1: Global Private Equity: Fundraising (USD billions)}
\begin{center}
\includegraphics[max width=\textwidth]{2023_05_04_36535b8d80b32081d422g-443}
\end{center}

Source: Exhibit from “McKinsey's Private Markets Annual Review," April 2021, McKinsey \& Company, \href{http://www.mckinsey.com}{www.mckinsey.com}. Copyright @ 2021 McKinsey \& Company. All rights reserved. Reprinted by permission. McKinsey continues to update this article annually here: \href{http://www.mckinsey.com/industries/}{www.mckinsey.com/industries/} private-equity-and-principal-investors/our-insights/mckinseys-private-markets-annual-review.

\section{Categories of Private Equity}
\section{Leveraged Buyouts}
Leveraged buyouts (LBOs), or highly leveraged transactions, arise when private equity firms establish buyout funds (or LBO funds) to acquire public companies or established private companies, with a significant percentage of the purchase price financed through debt. The target company's assets typically serve as collateral for the debt, and the target company's cash flows are expected to be sufficient to service the debt. The debt becomes part of the target company's capital structure after the buyout occurs. After the transaction, the target company becomes or remains a privately owned company. LBOs are sometimes called "going-private" transactions because after the acquisition of a publicly traded company, the target company's equity is substantially no longer publicly traded. (When the target company is private, it is not a going-private transaction.) The LBO may also be of a specific type. In management buyouts (MBOs), the current management team participates in the acquisition, and in management buy-ins (MBIs), the current management team is replaced with the acquiring team involved in managing the company. LBO managers seek to add value by improving company operations, boosting revenue, and ultimately increasing profits and cash flows. Cash-flow growth, in order of contribution, comes from organic revenue growth, cost reductions and restructuring, acquisitions, and then all other sources. The financial returns in this category, however, depend greatly on the use of leverage. If debt financing is unavailable or costly, LBOs are less likely to take place.

\section{Venture Capital}
Venture capital (VC) entails investing in or providing financing to private companies with high growth potential. Typically these are start-ups or young companies, but venture capital can be injected at various stages, ranging from concept creation for a company to the point of a company's IPO (initial public offering) launch or its acquisition by another company. The required investment return varies with the company's stage of development. Investors in early-stage companies demand higher expected returns relative to later-stage investors because the earlier the stage of development, the higher the risk.

\section{Private Capital, Real Estate, Infrastructure, Natural Resources, and Hedge Funds}
Venture capitalists, like all private equity managers, are active investors directly involved with their portfolio companies. VC funds typically invest in companies and receive an equity interest but may also provide financing in the form of debt (commonly, convertible debt).

Formative-stage financing is for a company still being formed. Its steps are as follows:

a. Pre-seed capital, or angel investing, is capital provided at the idea stage. Funds may be used to develop a business plan and to assess market potential. The amount of financing here is typically small and sourced from individuals, often friends and family, rather than from VC funds.

b. Seed-stage financing, or seed capital, generally supports product development and marketing efforts, including market research. This is the first stage at which $\mathrm{VC}$ funds usually invest.

c. Early-stage financing (early-stage $\mathrm{VC}$ ), or start-up stage financing, goes to companies moving toward operation but prior to commercial production or sales; early-stage financing may be injected to initiate both operation and commercial production.

Later-stage financing (expansion VC) comes after commercial production and sales have begun but before an IPO. Funds may be used to support initial growth, a major expansion (such as a physical plant upgrade), product improvements, or a major marketing campaign.

Mezzanine-stage financing (mezzanine venture capital) prepares a company to go public. It represents the bridge financing needed to fund a private firm until it can execute an IPO or be sold. The term "mezzanine-stage financing" is used because the financing is infused between private and public company status, focused on the timing rather than mechanism of the financing.

Formative-stage financing is generally carried out by providing ordinary or convertible preferred share sales to the investor or investors, likely the VC fund, while management retains control of the company. Later-stage financing generally involves management selling control of the company to the $\mathrm{VC}$ investor; financing is provided through equity and debt, although the fund may also use convertible bonds or convertible preferred shares. The VC fund offers debt financing for reasons of recovery and the control of assets in a bankruptcy situation, not to generate income. Simply put, debt financing affords the $\mathrm{VC}$ fund more protection than equity does.

When investing, a venture capitalist must be confident the portfolio company's management team is competent and armed with a solid business plan showing strong prospects for growth. Because these companies are immature businesses without years of operational and financial performance history, estimating company valuations from future prospects is highly uncertain. Accuracy here is more elusive than in LBO investing, which targets mature, underperforming public companies.

Certainty around valuation increases as the portfolio company matures and moves into later-stage financing, but even then, LBO investments enjoy more certainty. Exhibit 2 takes us through the growth stages of a company and the types of financing it may receive at each stage.

\section{Exhibit 2: The Private Equity Stage Continuum}
\begin{center}
\includegraphics[max width=\textwidth]{2023_05_04_36535b8d80b32081d422g-445}
\end{center}

Source: Private Equity Primer, “What Is Private Equity?” (12 February 2016). \href{http://www.pe-primer.com/}{www.pe-primer.com/} dealsherpapress/2016/2/12/bbcati52gcwdzrl34pts0x2tl783do (accessed 18 June 2020).

Other Private Equity Strategies

Among several other specialties, some private equity firms specialize in growth capital, also known as growth equity or minority equity investing. Growth capital generally refers to minority equity investments, whereby the firm takes a less-than-controlling interest in more mature companies looking for capital to expand or restructure operations, enter new markets, or finance major acquisitions. Many times, minority equity investing is initiated and sought by the management of the investee company. Their motive is to realize earnings from selling a portion of its shares before the company can go public but still retain control and participation in the success of the company. Although this scenario occurs most commonly with private companies, publicly quoted companies can seek private equity capital through PIPEs (private investments in public equities). Other private equity strategies secure returns by investing in companies in specific industries.

\section{KNOWLEDGE CHECK}
\begin{enumerate}
  \item Identify two of the three stages of the private equity continuum and the company growth stage each finances.
\end{enumerate}

\section{Solution}
The stages of the entire PE continuum and their associated growth stage financing are as follows:

\begin{enumerate}
  \item Venture capital focuses on start-up and seed-stage businesses.

  \item Growth equity focuses on more established businesses.

  \item Buyouts/LBOs focus on mature businesses.

\end{enumerate}

\section{Exit Strategies}
Private equity firms seek to improve new or underperforming businesses and then exit them at higher valuations, buying and holding companies for an average of five years. Holding time, however, can range from less than six months to more than 10 years. Before deciding on an exit strategy, private equity managers assess the dynamics of the industry in which the portfolio company competes, the overall economic cycle, interest rates, and company performance. Three common exit strategies are trade sales, IPOs, and SPACs.

\begin{itemize}
  \item Trade sale. This refers to the sale of a company to a strategic buyer, such as a competitor. A trade sale can be conducted by auction or by private negotiation.
\end{itemize}

Advantages of a trade sale include

a. an immediate cash exit for the private equity fund,

b. potential willingness of strategic buyers to pay more from anticipating synergies with their own business,

c. fast and simple execution,

d. improved process in contrast to an IPO, with lower transaction costs, lower levels of disclosure, and higher confidentiality because private equity firms generally deal with only one other party.

Disadvantages of trade sales include

a. possible opposition by management, who may wish to avoid ownership by a competitor for job security reasons,

b. a limited universe of potential trade buyers, and

c. lower attractiveness to portfolio company employees than for an IPO (which would monetize the shares and potentially attain a higher sale price).

\begin{itemize}
  \item Initial public offering (IPO). In an initial public offering, the portfolio company sells its shares, including some or all of those held by the private equity firm, to public investors.
\end{itemize}

Advantages of an IPO exit include

a. potential for the highest price,

b. management approval (because management will be retained),

c. publicity for the private equity firm, and

d. future upside potential (because the private equity firm may remain a large shareholder).

Disadvantages of an IPO exit include

a. high transaction costs to pay investment banks and lawyers,

b. long lead times,

c. the risk of stock market volatility (including short-term focus of some investors),

d. onerous disclosure requirements,

e. a potential lockup period (mandating the private equity firm to retain an equity position for a specified period after the IPO), and f. suitability of IPOs usually only for larger companies with attractive growth profiles.

\begin{itemize}
  \item Special purpose acquisition company (SPAC). The special purpose acquisition company technique starts as a shell company via an IPO through which sponsors raise a blind pool of cash aimed for merger or acquisition with private firms. Transactions must occur within a set period, typically 24 months, or the structure automatically unwinds, with funds returned to investors. Companies suitable for an IPO would be appropriate SPAC candidates, but the two strategies have different valuation methods with a single counterparty setting SPAC terms.
\end{itemize}

Advantages of a SPAC exit include

a. extended time for public disclosure on company prospects to build investor interest,

b. fixed valuation with lower volatility of share pricing,

c. flexibility of transaction structure to best suit the company's context, and

d. association with potentially high-profile and seasoned sponsors and their extensive investor network.

Disadvantages include

a. potential higher costs of capital from sponsor dilution, warrants, and other fees,

b. spread between the announced and true equity value because of the dilution,

c. deal and capital risk due to potential redemptions, and

d. significant stockholder overhang and churn over the number of months after the merger.

In addition, other exit strategies include recapitalization, secondary sales, and write-off/liquidation.

\begin{itemize}
  \item Recapitalization. Recapitalization via private equity describes the steps a firm takes to increase or introduce leverage to its portfolio company and pay itself a dividend out of the new capital structure. A recapitalization is not a true exit strategy, because the private equity firm typically maintains control; however, it does allow the private equity investor to extract money from the company to pay its investors and improve internal rate of return (IRR). A recapitalization may lead to a later exit.

  \item Secondary sales. This approach represents a sale of the company to another private equity firm or another group of investors. With the considerable amount of funds raised by global PE, there has been an increase in the proportion of secondary sales exits.

\end{itemize}

-Write-off/liquidation. A write-off occurs when a transaction has not gone well. The private equity firm revises the value of its investment downward or liquidates the portfolio company before moving on to other projects.

\section{Private Capital, Real Estate, Infrastructure, Natural Resources, and Hedge Funds}
The exit strategies we have discussed may be pursued individually or in concert or may be used for a partial exit strategy. For example, private equity funds may sell a portion of a portfolio company to a competitor via a trade sale and then complete a secondary sale to another private equity firm for the remaining portion. Company shares may also be distributed to fund investors, although such a move is unusual.

\section{KNOWLEDGE CHECK}
\begin{enumerate}
  \item Identify three exit strategies commonly used by private equity firms.
\end{enumerate}

\section{Solution}
Exit strategies commonly used by private equity firms include the following:

\begin{enumerate}
  \item Trade sale

  \item IPO

  \item SPAC

  \item Recapitalization

  \item Secondary sales

  \item Write-off/liquidation

\end{enumerate}

\section{Forms of Private Equity}
As covered in earlier, investors can invest in private equity through investing in $\mathrm{PE}$ funds of funds, investing in a PE fund, co-investing, or direct investing.

Indirect private equity involves greater separation between the investor and the ultimate operating company investment. It can be focused through a single private equity fund or through a fund-of-fund vehicle with stakes in various other private equity funds. Related management fees will be involved both at the individual fund level and with the sponsor of the fund of funds, creating greater costs for the investor.

With a direct private equity investment, the investment is made in a single, specific asset. But there can be a collective aspect to this type of investment when conducted as a co-investment. In that case, the investor will participate alongside a lead sponsor who sources, structures, and executes the transaction.

\section{Description of Private Debt}
Private debt primarily refers to the various forms of debt provided by investors to private entities. In the past decade, the expansion of the private debt market has been largely driven by private lending funds filling the gap between borrowing demand and reduced lending supply from traditional lenders in the face of tightened regulations following the 2008 financial crisis. We can organize the primary methods of private debt investing into four categories: direct lending, mezzanine loans, venture debt, and distressed debt. The broad array of debt strategies offers not only diversification benefits but also exposure to other investment spheres, such as real estate and infrastructure.

\section{Categories of Private Debt}
\section{Direct Lending}
Private debt investors get involved in direct lending by providing capital directly to borrowers and subsequently receiving interest, the original principal, and possibly other payments in exchange for their investment. As with typical bank loans, payments are usually received on a fixed schedule. The debt itself typically is senior and secured and has covenants in place to protect the lender/investor. It is provided by a small number of investors to private and, sometimes, public entities and differs from traditional debt instruments, such as bonds, which can be issued to many participants and be publicly traded.

Direct lending primarily involves private debt firms (or private equity firms with private debt arms) establishing funds with money raised from investors desiring higher-yielding debt. Fund managers then seek financing opportunities, such as providing a loan to a mid-market corporation or extending debt to another private equity fund that is seeking funds for acquisitions. In general, private debt funds provide debt, at higher interest rates, to entities needing capital but lacking good alternatives to traditional bank lenders, who themselves may be uninterested in or unable to transact with these borrowers. As with private equity, private debt fund managers conduct thorough due diligence before investing.

In direct lending, many firms may also provide debt in the form of a leveraged loan, a loan that is itself levered. Private debt firms that invest in leveraged loans first borrow money to finance the debt and then extend it to another borrower. By using leverage, a private debt firm can enhance the return on its loan portfolio.

\section{Mezzanine Debt}
Mezzanine debt refers to private credit subordinated to senior secured debt but senior to equity in the borrower's capital structure. Mezzanine debt is a pool of additional capital available to borrowers beyond senior secured debt, often used to finance LBOs, recapitalizations, corporate acquisitions, and similar transactions. Because of its typically junior ranking and its usually unsecured status, mezzanine debt is riskier than senior secured debt. To compensate investors for this heightened risk, investors commonly demand higher interest rates and may require options for equity participation. Mezzanine debt often comes with additional features, such as warrants or conversion rights. These provide equity participation to lenders/investors, conveying the option to convert their debt into equity or purchasing the equity of the underlying borrower under certain circumstances.

\section{Venture Debt}
Venture debt is private debt funding that provides venture capital backing to start-up or early-stage companies that may be generating little or negative cash flow. Entrepreneurs may seek venture debt, often in the form of a line of credit or term loan, to obtain additional financing without further diluting shareholder ownership. Venture debt can complement existing equity financing, allowing current shareholders to maintain ownership and control for a longer period of time. Similar to mezzanine debt, venture debt may carry additional features that compensate the investor/lender for the increased risk of default or, in the case of start-up and early-stage companies, a lack of substantial assets for pledging as debt collateral. One such feature could grant the lender rights to purchase equity in the borrowing company under certain circumstances.

\section{Distressed Debt}
Involvement in distressed debt typically entails buying the debt of mature companies in financial difficulty. These companies may be in bankruptcy, have defaulted on debt, or seem likely to default on debt. Some investors identify companies with a temporary cash-flow problem but a good business plan to help the company survive and ultimately flourish. These investors buy the company's debt, expecting both the company and its debt to increase in value. Turnaround investors buy debt aiming to be more active in distressed company management and direction, seeking to restructure and revive them.

\section{Other Private Debt Strategies}
Private debt firms may have specialties other than the aforementioned strategies. One could be investing in or issuing collateralized loan obligations (CLOs). These are leveraged structured vehicles collateralized by a portfolio of loans covering a diverse

\section{Private Capital, Real Estate, Infrastructure, Natural Resources, and Hedge Funds}
range of tranches, issuers, and industries. A CLO manager extends several loans to corporations, usually to firms involved in LBOs, corporate acquisitions, or similar transactions. It next pools these loans and then divides that pool into various tranches of debt and equity of differing seniority and security. The CLO manager sells each tranche to different investors according to their risk profile; the most senior portion of the CLO will be the least risky, and the most junior portion of the CLO (i.e., equity) will be the riskiest.

Another type of debt that could be directly extended to borrowers is unitranche debt. Unitranche debt consists of a hybrid or blended loan structure combining different tranches of secured and unsecured debt into a single loan with a single, blended interest rate. Since unitranche debt is a blend of secured and unsecured debt, its interest rate will generally fall in between the interest rates often demanded on secured and unsecured debt. The unitranche loan will usually be structured between senior and subordinated debt in priority ranking.

Some private debt firms invest in real estate debt or infrastructure debt. Real estate debt refers to loans and other forms of debt provided for real estate financing, where a specified real estate asset or property serves as collateral. Infrastructure debt encompasses the many forms of debt used to finance the construction, operation, and maintenance of infrastructure assets.

Private debt firms may also provide specialty loans, extended to niche borrowers in specific situations. For example, in litigation finance, a specialist funding company provides debt to clients, usually plaintiffs in litigation, for their legal fees and expenses in exchange for a share of judgements.

\section{KNOWLEDGE CHECK}
\begin{enumerate}
  \item Which of the following is not considered a strategy in private debt investing?
\end{enumerate}

A. Direct lending

B. Recapitalization

C. Mezzanine debt

\section{Solution}
$\mathrm{B}$ is correct. Recapitalization is when a private equity firm increases leverage or introduces it to the company and pays itself a dividend.

\begin{enumerate}
  \setcounter{enumi}{1}
  \item Which of the following forms of debt are likely to have additional features, such as warrants or conversion rights?
\end{enumerate}

A. Mezzanine debt

B. Venture debt

C. Both $\mathrm{A}$ and $\mathrm{B}$

\section{Solution}
$\mathrm{C}$ is correct. Both mezzanine debt and venture debt are likely to have additional features, such as warrants and conversion rights, to compensate debt holders for increased risk, including the risk of default.

\section{Forms of Private Debt}
Analogous to private equity investment, an investor wanting to include private debt in a portfolio has various alternatives along a comparable direct versus indirect distinction. In direct private debt investment, the investor makes a loan to a specific operating company. In the indirect approach, the investor takes an intermediated path, purchasing an interest in a fund that pools contributions typically on behalf of multiple participants to buy into the debt from a set of operating companies. For both approaches, in exchange for the debt, the investors receive interest payments and the return of principal after a designated term. The debt is typically secured and has various protections/covenants in place.

With a direct private debt investment, the investment is made in a single, specific asset. But again, there can be a collective aspect to this type of investment when conducted as a co-investment.

\section{Risk and Return Characteristics of Private Capital}
Investments in private capital vary in terms of risk and return because of various factors, including but not limited to ranking in an entity's capital structure. Typically, private equity, as the riskiest alternative, offers the highest returns, with private debt returns declining on a continuum down to its safest, most secured form of infrastructure debt. Exhibit 3 presents private equity and private debt categories by their risk and return levels. (Mirroring the risk-return pathway for traditional equity and debt investing, note the trade-off as investors select between junior and senior debt and between equity and debt.)

\section{Exhibit 3: Private Capital Risk and Return Levels by Category}
\begin{center}
\includegraphics[max width=\textwidth]{2023_05_04_36535b8d80b32081d422g-451}
\end{center}

Source: Based on a graph from Leon Sinclair, "The Rise of Private Debt," IHS Markit (7 August 2017).

Discussion of the distinct risk/return characteristics of private equity and private debt follows.

\section{Risk/Return of Private Equity}
The higher-return opportunities that private equity funds may provide relative to traditional investments are due to their ability to invest in private companies, their influence on portfolio companies' management and operations, and their use of leverage. Investing in private equity, including venture capital, is riskier than investing in common stocks and requires a higher return for accepting the higher risk, including illiquidity and leverage risks.

\section{Private Capital, Real Estate, Infrastructure, Natural Resources, and Hedge Funds}
Exhibit 4 shows the mean annual returns for the Cambridge Associates US Private Equity Index, the NASDAQ index, and the S\&P 500 Index for a variety of periods ending 31 December 2019. Private equity returns are based on IRR, making it difficult to identify reliable benchmarks for comparing returns to investments. Some investors may use the public market equivalent (PME) to match the timing of cash flows with the public market. Others may use an index of publicly traded private equity firms as a benchmark for private equity returns.

\section{Exhibit 4: Comparison of Mean Annual Returns for US Private Equity and US}
 Stocks\begin{center}
\begin{tabular}{lccccc}
\hline
Index & $\mathbf{1}$ Year & $\mathbf{5}$ Years & $\mathbf{1 0}$ Years & $\mathbf{2 0}$ Years & $\mathbf{2 5}$ Years \\
\hline
US private equity & 13.95 & 12.00 & 14.35 & 11.08 & 13.24 \\
NASDAQ & 35.23 & 13.63 & 14.74 & 4.03 & 10.43 \\
S\&P 500 & 31.49 & 11.70 & 13.56 & 6.06 & 10.22 \\
\hline
\end{tabular}
\end{center}

Source: Cambridge Associates, "US Private Equity Index and Selected Benchmark Statistics" (31 December 2019).

Published private equity indexes may be an unreliable measure of performance. Measuring historical private equity performance is challenging; as with hedge funds, private equity return indexes rely on self-reporting and are subject to survivorship, backfill, and other biases. This typically leads to an overstatement of returns. Moreover, prior to 2009 , in the absence of a liquidity event, private equity firms did not necessarily mark their investments to market. Failure to mark to market will understate measures of volatility and correlations with other investments. Thus, data adjustments are required to more reliably measure the benefits of private equity investing. Investors should require a higher return for accepting a higher risk, including illiquidity and leverage risks.

\section{Risk/Return of Private Debt}
Private debt investments may provide higher-yielding opportunities to fixed-income investors seeking increased returns relative to traditional bonds. Private debt funds may generate higher returns by taking opportunistic positions based on market inefficiencies. Private lending funds filled the financing gap left by traditional lenders following the 2008 financial crisis. Investors in private debt could realize higher returns from the illiquidity premium, which is the excess return investors require to compensate for a lack of liquidity, and may benefit from increased portfolio diversification.

The potential for higher returns is connected to higher levels of risk. Private debt investments vary in risk and return, with senior private debt providing a steadier yield and moderate risk and mezzanine private debt carrying higher growth potential, equity upside, and higher risk. Overall, investing in private debt is riskier than investing in traditional bonds. Investors should be aware of these risks, including illiquidity and heightened default risk when loans are extended to riskier entities or borrowers in riskier situations.

Exhibit 5 shows annualized returns and standard deviations for private debt funds from 2004 to 2016. The data show private debt funds appearing to offer an attractive risk-return trade-off. Exhibit 5: Historical Annualized Returns and Standard Deviation of Returns for Private Debt Funds, 2004-2016

\begin{center}
\begin{tabular}{lcc}
\hline
Private Debt Fund & Annualized Return & Standard Deviation \\
\hline
Mezzanine & $7.4 \%$ & $2.5 \%$ \\
Distressed & $7.9 \%$ & $6.7 \%$ \\
Direct lending only: &  &  \\
$\quad$ All direct lending & $9.2 \%$ & $2.0 \%$ \\
Direct lending (excl. & $11.1 \%$ & $3.4 \%$ \\
mezzanine) &  &  \\
\hline
\end{tabular}
\end{center}

Source: S. Munday, W. Hu, T. True, and J. Zhang, "Performance of Private Credit Funds: A First Look," Institute for Private Capital (7 May 2018, pp. 32-33).

\section{KNOWLEDGE CHECK}
\begin{enumerate}
  \item The type of private debt expected to have the greatest excess return potential is
A. Unitranche
B. Mezzanine
C. Infrastructure
\end{enumerate}

\section{Solution}
B is correct. As a junior form of subordinated debt, mezzanine private debt offers higher growth potential, equity upside, and higher risk, with the comparatively highest returns. Infrastructure debt is senior and poses the lowest risk. Unitranche debt is less risky than subordinated debt but riskier than infrastructure debt and is a blend of secured and unsecured debt; its interest rate generally falls in between the interest rates often demanded on secured and unsecured debt, and the loan itself is usually structured between senior and subordinated debt.

\section{Diversification Benefits of Private Capital}
Investments in private capital funds can add a moderate diversification benefit to a portfolio of publicly traded stocks and bonds. Correlations with public market indexes vary from 0.47 to 0.75 , as shown in Exhibit 6. And if investors identify skillful fund managers, benefits from excess returns are possible, given the additional leverage, market, and liquidity risks. Exhibit 6: Private Capital's Average Correlations with Public Market Indexes, 2013-2018

\begin{center}
\includegraphics[max width=\textwidth]{2023_05_04_36535b8d80b32081d422g-454}
\end{center}

Source: Y. Liu and O. Ruban, “Did Private Capital Deliver?” MSCI (30 January 2020). \href{http://www.msci.com/}{www.msci.com/} www/blog-posts/did-private-capital-deliver-/01697376382 (accessed 20 July 2020).

Private equity investments may offer vintage diversification because capital is not deployed all at one time but is invested over several years. Private debt investments, which offer more options than bonds and public forms of traditional fixed income, can also serve diversification goals. Ultimately, investors should prudently avoid concentration risk and aim to diversify across different managers, industries, and geographies.

\section{REAL ESTATE}
explain investment characteristics of real estate

\section{Summary-Real Estate}
\begin{itemize}
  \item Real estate includes two major sectors: residential and commercial. Residential real estate is the largest sector, totaling $75 \%$ of the global market. Commercial real estate primarily includes office buildings, shopping centers, and warehouses. Real estate property has some unique features, including heterogeneity (no two properties are identical) and fixed location.

  \item Real estate investments can be direct or indirect, in the public market (e.g., REITs) or private transactions, and in equity or debt.

\end{itemize}

\section{Introduction and Overview of Real Estate}
Individuals and institutions buy real property for their own use, as an investment, or for both reasons. The residential, or housing, market is made up of individual single-family detached homes and multi-family attached units, which share at least one wall with another unit (e.g., condominiums, townhouses). Commercial real estate includes primarily office buildings, retail shopping centers, and warehouses. In contrast to the owner-occupied market, rental properties are leased to tenants. A lease is a contract that conveys the use of the property from the owner (the landlord or lessor) to the tenant (the lessee) for a predetermined period in exchange for compensation (rent). When residential real estate, be it single unit or multi-unit, is owned with the intention to let, lease, or rent the property for income, it is classified as commercial (i.e., income-producing) real estate.

Real estate investing is typically classified as either direct or indirect ownership (equity investing) in real estate property, such as land and buildings. However, it also includes lending (debt investing) against real estate property. The key reasons for investing in real estate include the following potential benefits:

\begin{itemize}
  \item Competitive long-term total returns driven by both income generation and capital appreciation

  \item Multiple-year leases with fixed rents for some property types potentially providing stable income over many economic cycles

  \item Historically low correlations with other asset classes

  \item Inflation hedge if leases provide regular contractual rent step-ups or can be frequently marked to market

\end{itemize}

A title or deed represents real estate property ownership covering building and land-use rights along with air, mineral, and surface rights. Titles can be purchased, leased, sold, mortgaged, or transferred together or separately, in whole or in part. A title search is a crucial part of buyer and lender due diligence, ensuring the seller/ borrower owns the property without any liens or other claims against the asset, such as from other owners, lenders, or investors or from the government for unpaid taxes.

Residential real estate is by far the largest market sector by value and size. Savills World Research estimated in July 2018 that residential real estate accounted for more than $75 \%$ of global real estate values. Although the average value of a home is less than the average value of an office building, the aggregate space required to house people is much larger than that needed to accommodate office use and retail shopping.

The size of the professionally managed, institutional-quality global real estate market increased to USD9.6 trillion at the end of 2019, as shown in Exhibit 7, from USD6.8 trillion in 2013, yielding a compound average annual growth rate of 5.8\%.

Exhibit 7: Size of Professionally Managed Global Real Estate Market, 2019 (USD billions)

\begin{center}
\begin{tabular}{lcc}
\hline
Country or Region & Size & $\%$ of Total \\
\hline
United States & 3,418 & $35.8 \%$ \\
Japan & 881 & $9.2 \%$ \\
United Kingdom & 745 & $7.8 \%$ \\
China & 592 & $6.2 \%$ \\
Germany & 580 & $6.1 \%$ \\
France & 441 & $4.6 \%$ \\
\end{tabular}
\end{center}

\begin{center}
\begin{tabular}{lcc}
\hline
Country or Region & Size & $\%$ of Total \\
\hline
Hong Kong SAR & 378 & $4.0 \%$ \\
Canada & 361 & $3.8 \%$ \\
Australia & 306 & $3.2 \%$ \\
Rest of world & 1,848 & $19.3 \%$ \\
Total & 9,553 & $100 \%$ \\
\hline
\end{tabular}
\end{center}

Source: MSCI Real Estate.

Real estate is different from other asset classes due to its large required capital investment; its illiquidity; its diversity, as no two properties are identical in terms of location, tenant credit mix, lease term, age, and market demographics; and its necessarily fixed location. All these have important investment implications.

Furthermore, price discovery in the private market is opaque. Historical prices may not reflect market conditions. Transaction costs are high. Transaction activity may be limited in certain markets. It may be difficult for small investors to establish a diversified portfolio of wholly owned properties. Private market indexes are not investable, and property typically requires professional operational management.

\section{KNOWLEDGE CHECK}
\begin{enumerate}
  \item What are two potential benefits that attract investors to investing in real estate?
\end{enumerate}

\section{Solution}
There are four stated potential benefits that attract investors to investing in real estate:

\begin{enumerate}
  \item Competitive long-term total returns driven by both income generation and capital appreciation

  \item Multiple-year leases with fixed rents for some property types potentially providing stable income over many economic cycles

  \item Historically low correlations with other asset classes

  \item Inflation hedge if leases provide regular contractual rent step-ups or can be frequently marked to market

  \item True or false: The largest sector of the real estate market by value and size is commercial real estate.

\end{enumerate}

Explain your selection.

A. True

B. False

\section{Solution}
False. Commercial real estate is not the largest sector of the real estate market.

Residential real estate is by far the largest sector of the real estate market by value and size. The residential debt market greatly exceeds commercial property debt because of the larger total value of residential properties combined with property owners' greater ability to use leverage - up to $80 \%$ of the property's value or more in some cases. In addition, home mortgages are subsidized in some markets, including government guarantees.

\section{Description of Real Estate}
\section{Categories of Real Estate}
In this reading, residential properties are defined narrowly to include only owner-occupied, single residences (often referred to as single-family residential property). Residential properties owned with the intent to let, lease, or rent them are classified as commercial. Commercial properties also include office, retail, industrial and warehouse, and hospitality (e.g., hotel and motel) properties. Commercial properties may also have mixed uses. Commercial properties generate returns from income (e.g., rent) and capital appreciation. Several factors will affect opportunities for capital appreciation, including development strategies, market conditions, and property-specific features.

\section{Residential Property}
For many individuals and families, real estate investment takes the form of direct equity investment (i.e., ownership) in a residence with the intent to occupy it. In other words, a home is purchased. Most purchasers cannot pay $100 \%$ cash up front and must borrow funds to buy. Most lenders require an equity contribution of at least $10 \%-20 \%$ of the property purchase price in countries with well-developed mortgage markets. Any appreciation (depreciation) in the value of the home increases (decreases) the owner's equity in the home and is magnified by mortgage leverage. In countries without well-developed mortgage markets, homebuyers must save for a much longer period and may seek family assistance to finance the purchase.

Home loans may be held on the originator's balance sheet or securitized and offered to the financial markets. Securitization provides indirect debt investment opportunities in residential property to other investors via securitized debt products, such as residential mortgage-backed securities (RMBS).

\section{Commercial Real Estate}
Commercial property has traditionally been considered an appropriate direct investment, whether through equity or debt, for institutional funds and high-net-worth individuals with long time horizons and limited liquidity needs. This perception is primarily the result of the complexity, size, and relative illiquidity of the investments. Direct equity investing (i.e., ownership) is further complicated by the need of commercial property for active day-to-day management. The success of an equity investment in real estate is a function of factors including how well the property is managed, general economic and specific real estate market conditions, and the extent and terms of any debt financing.

To provide direct debt financing, the lender (investor) will conduct financial analysis to determine borrower creditworthiness, to ensure the property can generate sufficient cash flow for debt service, to estimate property value, and to evaluate economic conditions. The property value estimate is critical because the size of the loan relative to the property value - the loan-to-value ratio-determines the amount of risk held by the lender versus the borrower (equityholder). The borrower's equity in the property reflects borrower commitment to the success of the project, providing a cushion to the lender because the property is generally the sole collateral for the loan.

\section{REIT Investing}
Real estate investment trusts (REITs)are tax-advantaged entities (companies or trusts) that own, operate, and-to a limited extent-develop income-producing real estate property. REITs are known as pass-through businesses, meaning that REITs are not taxed at the corporate level. REITs are covered in further detail in Section 7.3.5 As of 2019, REITs were listed on stock exchanges in 39 countries, and their combined market capitalization exceeded USD1.6 trillion. ${ }^{2}$ Mortgage REITs, which invest primarily in mortgages, are similar to fixed-income investments. Equity REITs, which invest primarily in commercial or residential properties and use leverage, are similar to direct equity investments in leveraged real estate.

The business strategy for equity REITs is simple: Maximize property occupancy rates and rents to maximize income and dividends. Equity REITs, like other public companies, must report earnings per share based on net income as defined by generally accepted accounting principles (GAAP) or International Financial Reporting Standards (IFRS). Many report non-traditional measures, such as net asset value or variations of gross cash flow, to better estimate future dividends, because non-cash depreciation expenses can be high for asset-intensive businesses.

\section{Mortgage-Backed Securities}
MBS structuring is based on the asset-backed securitization model of transforming illiquid assets (mortgages) into liquid securities and transferring risk from asset owners (banks, finance companies) to investors. An MBS issuer/originator forms a special purpose vehicle (SPV) to buy mortgages from lenders and other mortgage owners and use them to create a diversified mortgage pool. The MBS issuer assigns the incoming stream of mortgage interest income and principal payments to individual security tranches sold to investors. Each tranche is assigned a priority distribution ranking.

Exhibit 8 illustrates the basic process for creating commercial mortgage-backed securities (CMBS). On the right-hand side of the exhibit, the ranking of losses indicates the priority of claims against the real estate property. Risk-averse investors, primarily insurance companies, prefer the lowest-risk tranches, which first receive interest and principal. Investors who choose the senior notes with the highest credit rating expect to earn a low return consistent with the senior tranche's low risk profile. Investors seeking the highest returns, carrying the highest risk, invest in the lowest-rated, most junior securities, last to receive interest and principal distributions. If mortgage defaults and losses are high, the lowest-ranked tranches bear the cost of the shortfall. The most junior tranche is referred to as the first-loss tranche.

\section{Exhibit 8: CMBS Structure}
\begin{center}
\includegraphics[max width=\textwidth]{2023_05_04_36535b8d80b32081d422g-458}
\end{center}

MBS valuations are influenced by the underlying borrowers' behavior. When interest rates decline, fixed-income security prices usually rise, but borrowers are also likely to accelerate refinancing their loans, resulting in the faster amortization of each MBS tranche and leaving MBS investors to reinvest their principal at the lower rates. Conversely, when rates rise, not only do fixed-income instrument prices decline, but property owner prepayments can also slow, lengthening the duration of most MBS tranches and contributing to further price weakness.

\section{KNOWLEDGE CHECK}
\begin{enumerate}
  \item When interest rates rise, the duration of most MBS tranches will
\end{enumerate}

\section{Solution}
When interest rates rise, the duration of most MBS tranches will lengthen.

\section{Forms of Real Estate Investing}
Real estate investing takes a variety of forms best illustrated with the four-quadrant model that arranges the quadrants by type (debt or equity) and source (private or public markets) of capital. Investors can choose to own all or part of a property's equity, either unlevered (100\% equity) or with a levered approach relying on debt and equity to finance the purchase. Alternatively, investors can gain exposure to real estate through debt ownership, either as a lender/originator or as a purchaser of debt instruments. Loans secured by real estate are called mortgages. Investors can access real estate debt by lending, buying mortgages, or purchasing mortgage-backed securities with the property ultimately serving as collateral.

Banks and insurance companies are among the largest originators and owners of mortgage debt. The residential debt market greatly exceeds commercial property debt because the total value of residential properties is larger and property owners have a greater ability to use leverage-securing loans for up to $80 \%$ of the property's value or more in some cases. In addition, home mortgages are subsidized in some markets and enjoy government guarantees.

Exhibit 9 presents examples of the basic forms of real estate in each of the quadrants: public or private structures, debt or equity ownership. Within the basic forms, there can be many variations.

\section{Exhibit 9: Basic Forms of Real Estate Investments and Examples}
Debt

Private

\begin{itemize}
  \item Mortgages

  \item Construction lending

  \item Mezzanine debt Equity

  \item Direct ownership of real estate: sole ownership, joint ventures, separate accounts, or real estate LPs

  \item Indirect ownership via real estate funds

  \item Private REITs

\end{itemize}

\begin{center}
\includegraphics[max width=\textwidth]{2023_05_04_36535b8d80b32081d422g-460}
\end{center}

\section{Direct Real Estate Investing}
Direct private investing involves purchasing a property and originating debt for one's own account. Ownership can be free and clear, whereby the property title is transferred to the owner(s) unencumbered by any financing liens, such as from outstanding mortgages. Owners can also borrow from mortgage lenders to fund the equity acquisition. Debt investors can also use leverage. Initial purchase costs associated with direct ownership may include legal expenses, survey costs, engineering/environmental studies, and valuation (appraisal) fees. In addition, ongoing maintenance and refurbishment charges are also incurred. Additional debt closing costs are incurred when owners take out loans to fund their investments.

There are distinct benefits to owning real estate directly:

\begin{itemize}
  \item Control: Only the owner decides when to buy or sell, when and how much to spend on capital projects (subject to lease terms and regulatory requirements), whom to select as tenants based on credit quality preference and tenant mix, and what types of lease terms to offer.

  \item Tax benefits: Property investors in many developed markets can use non-cash property depreciation expenses and interest expense, with some limitations, to reduce taxable income and lower their income tax bills. In fact, real estate investors in a country that permits accelerated depreciation and interest expense deductions can reduce taxable income below zero in the early years of asset ownership, and losses can be carried forward to offset future income. Thus, a property investment can be cash-flow positive while generating accounting losses and deferring tax payments.

\end{itemize}

Major disadvantages to investing directly include

\begin{itemize}
  \item extensive time required to manage the property,

  \item the need for local real estate market expertise,

  \item large capital requirements, and

  \item additional risks from portfolio concentration for smaller investors.

\end{itemize}

The owner may choose to handle all aspects of investing in and operating the property, including property selection, asset management, property management, leasing, and administration. To address the complexity of owning and managing commercial real estate, investors hire advisers to perform any number of functions, including identifying investments, negotiating terms, performing due diligence, conducting operations, asset management, and eventual disposal. Asset management focuses on maximizing property returns by deciding when and how much to invest in the property and when to sell the property. Many investors prefer to hire advisers or managers to manage the investors' direct real estate investment in what is called a separate account. This contains only the single investor's equity or perhaps a nominal stake from the adviser to help align interests. A separate account allows the investor to control the timing and value of acquisitions and dispositions and perhaps to make operating decisions as well.

Sometimes investors will form joint ventures with other investors to access real estate. Joint ventures are especially common when one party can uniquely contribute something of value, such as land, capital, development expertise, debt due diligence, or entrepreneurial talent. Joint ventures can be structured as general partnerships or, more commonly, as limited liability companies.

\section{KNOWLEDGE CHECK}
\begin{enumerate}
  \item True or false: Compared to other forms of real estate investing, an advantage of direct real estate investing is that there is no property management responsibility.
\end{enumerate}

Explain your selection.

A. True

B. False

Solution

False. Direct real estate investing does require property management responsibility.

\section{Indirect Real Estate Investing}
Indirect investing provides access to the underlying real estate assets through a variety of pooled investment vehicles. They can be public or private, such as limited partnerships, mutual funds, corporate shares, REITs, and exchange-traded funds (ETFs). Equity REITs own real estate equity, mortgage REITs own real estate mortgages and MBS, and hybrid REITs own both. Institutional investors and some high-net-worth investors can invest in MBS. Intermediaries facilitate the raising and pooling of capital and the creation of investable structures.

\section{Mortgages}
Mortgages represent passive investments in which the lender expects to receive a predefined stream of payments throughout the finite life of the mortgage. Mortgages may require full amortization or partial amortization with balloon payments due at maturity or may be interest only (IO). Borrowers can often choose among fixed-rate, floating-rate, and adjustable-rate options. Some of the floating- and variable-rate mortgages may have limitations, or caps, on how much the rate can change over a given period. Residential mortgages in developed markets usually have 15-, 25-, or 30 -year maturities. Commercial property loans may be similarly long for long-term property owners. Other owners can take out 5-, 7-, or 10-year loans to correspond with their anticipated holding period. Selling the property before the mortgage is due may result in prepayment penalties for the commercial borrower. If the borrower defaults on the loan, the lender may seek to take possession of the property. Investments may take place in the form of "whole" loans based on specific properties, typically by way of direct investment through private markets, or through participation in a pool of mortgage loans, typically by way of indirect investment in real estate through publicly traded securities, such as MBS.

\section{Private Capital, Real Estate, Infrastructure, Natural Resources, and Hedge Funds}
REITs and partnerships carry fees and administrative overhead for managing the assets embedded in their valuations. Fee structures for private REITs can be similar to those for private equity funds, with investment management fees based on either committed capital or invested capital. Funds also charge performance-based fees.

\section{Private Fund Investing}
Most private real estate funds are structured as infinite-life open-end funds. Like mutual funds, they allow investors to contribute or redeem capital throughout the life of the fund. Open-end funds generally offer exposure to core real estate, characterized by well-leased, high-quality institutional real estate in the best markets. Investors expect core real estate to deliver stable returns, primarily from income.

In addition to investing in core real estate, core-plus strategies will also accept slightly higher risks derived from non-core markets and sectors or properties with slightly more leasing risk. Non-core properties include large sectors with different risk profiles, such as hotels and nursing homes. Assets in secondary and tertiary markets and such niches as student housing, self-storage, and data centers are also considered non-core.

Investors seeking higher returns may also accept development, redevelopment, repositioning, and leasing risk. Finite-life closed-end funds are more commonly used for alpha- and beta-generating value-add and opportunistic investment styles. Value-add investments may require modest redevelopment or upgrades, the leasing of vacant space, or repositioning the underlying properties to earn a higher return than core properties. Opportunistic investing accepts the much higher risks of development, major redevelopment, repurposing of assets, taking on large vacancies, and speculating on significant improvement in market conditions. Closed-end fund managers will sell assets to realize the value created by management and lock in the investment's IRR. There are exceptions to the fund style/life/structure relationships-that is, core/infinite life/open end or opportunistic/finite life/closed end. On occasion, you will see core, closed-end, finite-life funds or open-end, infinite-life, value-add funds.

\section{REITs}
Real estate investment trusts are the preferred investment vehicles for owning income-producing real estate of both private and public investors. The main appeal of the REIT structure is the elimination of double corporate taxation: Corporations pay taxes on income, and then the dividend distributions of after-tax earnings are subsequently taxed at the shareholder's personal tax rate. REITs can avoid corporate income taxation by distributing dividends equal to $90 \%-100 \%$ of taxable net rental income.

REIT and REIT-like structures are similar worldwide. Whereas REITs are popular in the United States, other countries have structures separate from corporations for REITs that still permit dividend deductibility from the entities' income taxes.

Publicly traded REITs address many of the disadvantages related to private real estate investing. Listed REITs provide investors with much greater liquidity, lower trading costs, and better transparency. Management is employed within the REIT in most countries rather than brought in as a separate organization on contract, providing better alignment of interests with investors. Importantly, to modify real estate holdings, a REIT investor need only buy or sell REIT shares on listed markets instead of buying or selling real estate directly. The REIT is not forced to sell the company's underlying real estate like open-end funds experiencing mass redemptions.

\section{KNOWLEDGE CHECK}
\begin{enumerate}
  \item Fill in the blanks: Compared with private real estate investing, publicly traded real estate investment trusts provide significantly more liquidity, lower , and better
\end{enumerate}

Choice of words: transparency, trading costs, variety, happiness

\section{Solution}
Compared with private real estate investing, publicly traded real estate investment trusts provide significantly more liquidity, lower trading costs, and better transparency.

\section{Risk and Return Characteristics of Real Estate}
\section{Real Estate Indexes}
There are a variety of indexes globally designed to measure total and component real estate returns for listed securities and non-listed investment vehicles. Listed REIT indexes are straightforward in that, like other listed equity indexes, security pricing is reported by the major exchanges. Total returns are calculated assuming dividends are reinvested in the index. Importantly, listed REIT indexes are investable when their constituent stocks are freely traded. The more frequently the shares trade, the more reliable the index. Indexes containing small-cap, closely held, and low-float real estate stocks may not be as representative of investor returns. REIT indexes in general are not necessarily representative of the entire real estate universe because of different geographic concentrations and property types (residential, commercial, etc.) and the asset quality of the publicly traded REIT company portfolios. The National Association of Real Estate Investment Trusts (Nareit) in the United States, the European Public Real Estate Association (EPRA), and the Asian Public Real Estate Association (APREA) publish listed real estate company indexes for their respective regions and contribute to global indexes. Most of the leading for-profit index providers also offer a variety of REIT and real estate indexes.

As shown in Exhibit 10, REITs are the predominant structure among listed real estate companies in the United States. Outside the United States, there is an equal balance of REITs and non-REITs, as represented by the FTSE EPRA Nareit index series. Some other countries in which REITs make up most of the index include Japan, the United Kingdom, Australia, and Singapore. In contrast, non-REIT property companies make up most of the real estate indexes in Hong Kong SAR, Germany, and Sweden. Exhibit 10: REITs Make Up 50\% of the Property Index outside the United States

\begin{center}
\includegraphics[max width=\textwidth]{2023_05_04_36535b8d80b32081d422g-464}
\end{center}

REIT Index Market Cap Non-REIT Index Market Cap

Exhibit 11 displays global and regional listed REIT returns. The table shows some disparity among regional returns and supports the importance of country-specific, regional, and local knowledge.

\section{Exhibit 11: Historical Returns of Global REITs (through 31 December 2019)}
\section{FTSE EPRA Nareit Developed Index Series, USD}
\begin{center}
\begin{tabular}{lcccc}
\hline
Period & Global & North America & Asia Pacific & Europe \\
\hline
3 years & $9.31 \%$ & $7.76 \%$ & $10.24 \%$ & $13.05 \%$ \\
5 years & $6.53 \%$ & $6.63 \%$ & $5.69 \%$ & $7.40 \%$ \\
10 years & $9.25 \%$ & $11.39 \%$ & $6.58 \%$ & $8.66 \%$ \\
20 years & $9.23 \%$ & $11.06 \%$ & $7.05 \%$ & $9.38 \%$ \\
25 years & $8.92 \%$ & $11.13 \%$ & $7.00 \%$ & $8.88 \%$ \\
3-year volatility &  &  &  &  \\
10-year volatility & $8.24 \%$ & $10.17 \%$ & $9.05 \%$ & $10.51 \%$ \\
Dividend yield (Dec. 19) & $11.81 \%$ & $13.05 \%$ & $13.22 \%$ & $15.16 \%$ \\
\hline
\end{tabular}
\end{center}

Sources: FTSE, EPRA, NAREIT.

A variety of indexes measure private investment performance, and they may report on private fund performance, underlying property values, or property operating performance. REIT indexes are investable because they track listed REITs, but the same instruments are not available for private indexes. Real estate values are reported by property owners, advisers, and fund managers, and many private investors do not report results, giving rise to selection bias. The private fund and property indexes vary by property selection, valuation methodology, and longevity. In many cases, private index property values are based on third-party property appraisal and, therefore, offer a measure of independence from owner/manager estimates. But appraisals are backward looking, subject to the biases of the appraisers, and likely to offer a smoothed-out picture of actual market volatility.

Industry associations, practitioners, and academics have developed repeat sales indexes that are transaction based rather than appraisal based. When repeat sales take place, the changing property prices are measured and used to construct the index. These indexes also suffer from a sample selection bias because the properties sold in each period vary and may be unrepresentative of the larger market. Also, the properties sold are not a random sample and may be biased toward those that have increased or decreased in value, depending on economic conditions. The higher the number of sales, the more reliable and relevant the index.

\section{Real Estate Investment Risks}
Real estate investments, like any investment, may underperform expectations. Property values vary across macro (national and global economic) and micro (local) conditions and with interest rate levels. There are risks around the fund management team's ability to select, finance, and manage real properties and to account for changes in government regulation. Management of the underlying properties themselves includes handling rentals and leases, controlling expenses, directing maintenance and improvements, and ultimately disposing of the property. Expenses may increase because of circumstances beyond management's control or may not be covered by insurance. Returns to both debt and equity investors in real estate depend to a large extent on the ability of the owners or their agents to successfully operate the underlying properties.

Investments in distressed properties and in property development are subject to greater risks than investments in properties in sound financial condition or with stable operations. Property development is subject to special risks, including regulatory issues, construction delays, and cost overruns. Environmental regulation is one regulatory hurdle, as is the failure to receive zoning, occupancy, and other approvals and permits. Because the life cycle of such projects can be very lengthy, economic conditions may change. All these moving pieces may conspire to increase construction time or delay successful leases, which increases construction costs and reduces the level of rents relative to initial expectations. And there is also financing risk. Long-term financing with acceptable terms simply might be unavailable, forcing real estate acquisitions and developments to be financed with lines of credit or other forms of temporary funding. Financing problems with one property may delay or limit further development of the owner's other projects.

It is important to recognize that most direct investors, private funds, and public companies use leverage to increase returns for their investors. Leverage magnifies the effects of both gains and losses for equity investors and increases the risks the real estate owner or investor will be left with insufficient funds to make interest payments or repay the debt at maturity. In that case, the lender or debt investor will receive less than the entire interest and principal balances. As the loan-to-value ratio increases and interest coverage weakens, the probability of default increases.

\section{KNOWLEDGE CHECK}
\begin{enumerate}
  \item Adding real estate to a portfolio has been demonstrated to increase portfolio and reduce portfolio
\end{enumerate}

\section{Solution}
Adding real estate to a portfolio has been demonstrated to increase portfolio diversification and reduce portfolio risk.

\section{Diversification Benefits of Real Estate}
Many investors find the real estate sector attractive for providing high, steady current income. Although broad stock market gains are derived mostly from long-term capital appreciation, more than half of the returns from private core and listed real estate are derived from income. Throughout a market cycle, income is a more consistent source of return than capital appreciation, and hence, it is lower risk. That consistency stems from the underlying medium- to long-term property leases: the longer the lease, the more predictable the income. And the better the credit quality of a property's tenants, the more reliable the rents.

Many investors view REITs as alternative investments in the broader real estate category, and others see REITs as a fixed-income/equity hybrid, adding exposure simply for the incremental yield pickup over both high-grade bonds and equities. Others take a different focus: Some institutional investors analyze REIT investments through their public equity department, rather than engaging their real estate or alternative asset teams.

Real estate's moderate correlation with other asset classes adds to its attractiveness. Exhibit 12 lists the correlation of REITs with equity and debt securities during the 20-year period ended March 2020. The market capitalization of most REITs is on the smaller side, with the largest among them being notable exceptions.

\section{Exhibit 12: REIT Allocation Can Improve Portfolio Diversification}
\begin{center}
\begin{tabular}{lc}
Index & $\begin{array}{c}\text { Correlation with the FTSE Nareit All Equity } \\ \text { REITs Index }\end{array}$ \\
\hline
S\&P 500 & 0.57 \\
Russell 2000 Index & 0.64 \\
ICE BofAML Corp./Gov't. & 0.19 \\
ICE BofAML MBS & 0.09 \\
Bloomberg Barclays & 0.61 \\
US Agg./Credit/Corp. High Yield &  \\
\hline
\end{tabular}
\end{center}

Sources: Nareit, using data from FactSet and Intercontinental Exchange (ICE).

There are periods when equity REIT correlations with other securities are elevated, and they are at their highest during steep market downturns, such as the 2007-08 financial crisis. These correlations remained high during the post-crisis recovery, which lifted the value of most asset classes.

Exhibit 13 contains a summary of the performance and correlations of major asset classes and shows that European investors would benefit from real estate allocations.

Exhibit 13: Total Returns by Asset Class, January 1999-May 2019

\begin{center}
\begin{tabular}{lcccc}
\hline
 & $\begin{array}{c}\text { Total Return } \\ \text { (CAGR) }\end{array}$ & $\begin{array}{c}\text { Standard } \\ \text { Deviation }\end{array}$ & $\begin{array}{c}\text { Sharpe } \\ \text { Ratio }\end{array}$ & $\begin{array}{c}\text { Correlation with } \\ \text { Listed Real Estate } \\ \text { (Annual Returns) }\end{array}$ \\
\hline
Small-cap equities & $8.2 \%$ & $22.9 \%$ & 0.47 & 0.60 \\
Large-cap equities & 2.2 & 21.3 & 0.21 & 0.55 \\
Government bonds & 3.8 & 3.9 & 1.02 & 0.15 \\
\end{tabular}
\end{center}

\begin{center}
\begin{tabular}{|c|c|c|c|c|}
\hline
 & $\begin{array}{l}\text { Total Return } \\ \text { (CAGR) }\end{array}$ & $\begin{array}{l}\text { Standard } \\ \text { Deviation }\end{array}$ & $\begin{array}{c}\text { Sharpe } \\ \text { Ratio }\end{array}$ & $\begin{array}{l}\text { Correlation with } \\ \text { Listed Real Estate } \\ \text { (Annual Returns) }\end{array}$ \\
\hline
$\begin{array}{l}\text { Investment-grade cor- } \\ \text { porate bonds }\end{array}$ & 4.3 & 5.5 & 0.84 & 0.41 \\
\hline
$\begin{array}{l}\text { High-yield corporate } \\ \text { bonds }\end{array}$ & 7.1 & 18.0 & 0.45 & 0.44 \\
\hline
$\begin{array}{l}\text { Diversified } \\ \text { commodities }\end{array}$ & 1.8 & 18.9 & 0.19 & 0.14 \\
\hline
Listed real estate & 8.4 & 22.7 & 0.49 & 1.00 \\
\hline
\end{tabular}
\end{center}

Source: Data sources are as follows: small-cap equities, S\&P Europe SmallCap index; large-cap equities, STOXX Europe 50 Index; government bonds, Bloomberg Barclays Pan-European Aggregate Government A Index; investment-grade corporate bonds, Bloomberg Barclays Pan-European Aggregate Corporate Index; high-yield corporate bonds, Bloomberg Barclays Pan-European High Yield Total Return Index; diversified commodities, Bloomberg Commodity Index; listed real estate, FTSE EPRA Nareit Developed Europe Index.

Whether listed real estate behaves like stocks or private real estate is a matter of ongoing debate. We know that listed REITs are priced continuously whereas private real estate is appraised perhaps once a year, and there are other indications that their correlation is weak. Listed equity investors will discount future cash flows, and appraisers place heavy emphasis on current market conditions and recent trends.

CEM Benchmarking, based in Toronto, has performed several studies looking at the role of listed and unlisted real estate in the portfolio of defined contribution pension funds. CEM Benchmarking has concluded that real estate improves the risk-return profile of funds diversified across equities, fixed income, and alternatives. CEM analyzed the performance of each asset class over 12 years, from 2005 to 2016, across EUR2 trillion of assets under management (AUM) at the end of 2016. The data covered 36\% of European pension assets and provided a comprehensive view about how real estate performs relative to the pension funds' other assets. Exhibit 14 presents the asset correlations for Dutch and UK pension funds, which are among the pension fund markets with the deepest real estate concentrations.

\section{Exhibit 14: Correlations between Aggregate Asset Class Net Returns for}
 Dutch and UK Pension Funds\begin{center}
\begin{tabular}{|c|c|c|c|c|c|c|c|}
\hline
$\begin{array}{l}\text { Correlation } \\ \text { Matrix }\end{array}$ & $\begin{array}{l}\text { Public } \\ \text { Equity }\end{array}$ & $\begin{array}{l}\text { Private } \\ \text { Equity }\end{array}$ & $\begin{array}{l}\text { Fixed } \\ \text { Income }\end{array}$ & $\begin{array}{l}\text { Hedge } \\ \text { Funds }\end{array}$ & $\begin{array}{l}\text { Listed } \\ \text { Real } \\ \text { Estate }\end{array}$ & $\begin{array}{c}\text { Unlisted } \\ \text { Real } \\ \text { Estate }\end{array}$ & $\begin{array}{c}\text { Unlisted } \\ \text { Infrastructure }\end{array}$ \\
\hline
Public equity & 1.00 &  &  &  &  &  &  \\
\hline
Private equity & 0.93 & 1.00 &  &  &  &  &  \\
\hline
Fixed income & 0.12 & 0.06 & 1.00 &  &  &  &  \\
\hline
Hedge funds & 0.91 & 0.77 & 0.16 & 1.00 &  &  &  \\
\hline
Listed real estate & 0.86 & 0.83 & 0.28 & 0.81 & 1.00 &  &  \\
\hline
$\begin{array}{l}\text { Unlisted real } \\ \text { estate }\end{array}$ & 0.76 & 0.86 & 0.37 & 0.64 & 0.88 & 1.00 &  \\
\hline
$\begin{array}{l}\text { Unlisted } \\ \text { infrastructure }\end{array}$ & 0.73 & 0.67 & 0.56 & 0.62 & 0.66 & 0.67 & 1.00 \\
\hline
\end{tabular}
\end{center}

Notes: The period spans 2005-2016. Unlisted real estate and private equity are adjusted for lagged and smoothed reporting. Actual, as-reported correlations would appear higher and standard deviations, lower.

CEM Benchmarking conducted another study addressing some of the differences between listed and unlisted real estate performance by comparing the returns, volatility, correlations, and Sharpe ratios of various private real estate investing styles. The study covered more than 200 US public and private defined benefit pension funds over 20 years through 2017. Unlisted real estate's correlation with listed real estate was 0.91, supporting the view that listed REITs behave more like stocks in the short term and more like real estate in the long term. The outperformance of listed real estate was primarily due to fees and carry charged by external managers in the unlisted markets.

INFRASTRUCTURE

explain investment characteristics of infrastructure

\section{Summary-Infrastructure}
\begin{itemize}
  \item The assets underlying infrastructure investments are real, capital intensive, and long lived. These assets are intended for public use, and they provide essential services. Examples include airports, health care facilities, and power plants. Funding is often done on a public-private partnership basis.

  \item Social infrastructure assets are directed toward human activities and include such assets as educational, health care, social housing, and correctional facilities, with the focus on providing, operating, and maintaining the asset infrastructure.

  \item Infrastructure investments may also be categorized by the underlying asset's stage of development. Investing in infrastructure assets that are to be constructed is generally referred to as greenfield investment. Investing in existing infrastructure assets may be referred to as brownfield investment.

\end{itemize}

\section{Introduction and Overview of Infrastructure}
The assets underlying infrastructure investments are real, capital intensive, and long-lived. They are intended for public use and provide essential services-for instance, airports, health care facilities, and power plants. Most infrastructure assets are financed, owned, and operated by governments. A 2017 World Bank/PPIAF study of infrastructure spending in developing countries reported that $83 \%$ came from public investment,${ }^{3}$ but more and more infrastructure is being financed privately, with the increasing use of public-private partnerships (PPPs) by local, regional, and national governments. PPPs are typically defined as a long-term contractual relationship between the public and private sectors for the purpose of having the private sector deliver a project or service traditionally provided by the public sector.

Allocations to infrastructure investments have increased due to increasing interest by investors (demand-side growth) but also because governments have provided more investment opportunities (supply-side growth) as they expand the financing of infrastructure assets and privatization of services. By the end of 2019, according to Preqin, global infrastructure funds had raised more than USD98 billion for infrastructure projects. The year-over-year increase of approximately $15 \%$ and up from USD30 billion in 2012, with assets under management of USD582 billion (as of the end of June 2019), ${ }^{4}$ reflects what is commonly considered a growth market.

Infrastructure investors may intend to lease the assets back to the government, to sell newly constructed assets to the government, or to hold and operate the assets until they reach operational maturity or perhaps for even longer. Exhibit 15 shows the characteristics of a typical infrastructure investment.

\section{Exhibit 15: Typical Infrastructure Investment Characteristics}
\begin{center}
\includegraphics[max width=\textwidth]{2023_05_04_36535b8d80b32081d422g-469}
\end{center}

From an investment perspective, if assets are being held and operated, the relatively inelastic demand for the assets and services is advantageous; regulation and the high costs of the assets create high barriers to entry, giving the service provider a strong competitive position. (Regulation, in contrast, deters adverse monopolistic behavior from the public perspective.) Maintenance, asset replacement, and operating costs are factored into the pricing of the services.

Investors expect these assets to generate stable long-term cash flows that adjust for economic growth and inflation. Well-defined contractual structures allocating the risk and responsibilities for asset delivery, service provision, and legal and financial obligations contribute to such stability.

These structural aspects allow for relatively high levels of leverage, particularly important for funding given the typically high up-front capital investments needed. Leverage enhances investor returns, and with debt commonly issued on a non-recourse basis, the exposure of investors to this liability is capped.

An increasingly important theme of infrastructure investing is the incorporation of environmental, social, and governance (ESG) criteria, whether through application of technology such as renewable energy sources, consideration of project impact on the surrounding environment, or transformation of the business relationships and governance approach. Key infrastructure stakeholders are increasingly supporting this theme: the government/public procuring authority, by incorporating environmental and social impact assessments in the planning and permit processes; financiers, by applying, for instance, the Equator Principles as a condition for advancing loan facilities; and investors, who consciously use ESG criteria to choose investments.

\section{Private Capital, Real Estate, Infrastructure, Natural Resources, and Hedge Funds}
The Equator Principles represent a risk management and due diligence framework adopted by financial institutions globally to take account of environmental and social risk in projects.

\section{Description of Infrastructure}
\section{Categories of Infrastructure Investments}
Infrastructure investments are frequently categorized on the basis of the underlying assets. The broadest categorization organizes investments into economic and social infrastructure assets.

Economic infrastructure assets support economic activity and include three broad types of assets: transportation assets, information and communication technology (ICT) assets, and utility and energy assets.

\begin{itemize}
  \item Transportation assets include roads, bridges, tunnels, airports, seaports, and heavy and light/urban railway systems. Income will usually be linked to demand based on traffic, airport, and seaport charges, tolls, and rail fares and hence is deemed to carry market risk.

  \item ICT assets include infrastructure that stores, broadcasts, and transmits information or data, such as telecommunication towers and data centers.

  \item Utility and energy assets generate power and produce potable water; they transmit, store, and distribute gas, water, and electricity; and they treat solid waste.

\end{itemize}

Utility assets increasingly encompass environmentally sustainable development, with a greater focus on renewable technologies, including solar, wind, and waste-to-energy power generation. Other energy assets may encompass downstream oil and gas infrastructure, such as pipelines and liquefied natural gas (LNG) terminals, and also natural resource assets, such as mining assets. The income earned from utility assets may also carry demand risk as buyers' energy and natural resources needs fluctuate. Alternatively, utilities can institute "take-or-pay" arrangements locking buyers into minimum purchases whether supply is needed or not. Buyers usually have recourse if the utility falls short on performance, delivering supplies that are late or of inferior quality.

Social infrastructure assets are directed toward human activities and include such assets as educational, health care, social housing, and correctional facilities, with the focus on providing, operating, and maintaining the asset infrastructure. The relevant services administered through those facilities, be they medical or schooling, are usually provided separately by the public authority or by a private service provider contracted by the public authority. In some countries, this model has been extended to other public infrastructure, such as courthouses and government and municipal buildings. Income from social infrastructure is typically derived from a type of lease payment that depends on assuring availability (often referred to as availability payments) and on managing and maintaining the asset according to predefined standards. For instance, an availability payment may be reduced or voided if a hospital operating room is unavailable because of an electromechanical fault.

Infrastructure investments may also be categorized by the underlying asset's stage of development. Investing in infrastructure assets that are to be constructed is generally referred to as greenfield investment. The intent may be to lease or sell the assets to the government after construction or to hold and operate the assets. If they are held, that can be over the long term or for a shorter period until operational maturity with subsequent sale to new investors, thus ensuring capital appreciation to reflect the construction and commissioning risk. Greenfield investors typically invest alongside strategic investors or developers who specialize in developing the underlying assets.

Investing in existing infrastructure assets may be referred to as brownfield investment. Perhaps the assets are owned by a government aiming to privatize them, lease them out, or sell and lease them back, or they may be owned by greenfield investors seeking to realize investment value through a sale. Typically, some of the assets' financial and operating history is available. Brownfield investors may include strategic investors who specialize in the operation of the underlying assets, but particularly with privatizations, there will be financial investors involved who focus on the long-term stable returns.

Finally, infrastructure may be categorized by location. Infrastructure investments are available globally, and the geographic location of the underlying assets will inform the political and macroeconomic risks, particularly in light of the government's relationship to the assets.

Risks and expected returns will differ on the basis of the underlying asset's category, stage of development, and geographic location. How the investment is actualized-its form-also affects the risks and expected returns.

\section{KNOWLEDGE CHECK}
\begin{enumerate}
  \item Describe one of the three ways by which an infrastructure investment is frequently categorized.
\end{enumerate}

\section{Solution}
Infrastructure assets are frequently categorized on the basis of (1) underlying assets, (2) the underlying asset's state of development, and (3) the geographic location of the underlying assets.

The first category comprises economic assets supporting economic activity, such as transportation and energy/utility assets. It also includes social infrastructure assets, which enable public services directed toward human activities, such as education, health care, and housing.

The second category refers to the relevant stages of the infrastructure asset's life cycle. These include a greenfield investment for a newly created asset and a brownfield investment for an established asset.

The third category highlights the specific place (local, regional, national) associated with the government entity directly involved with the assets.

\section{Forms of Infrastructure Investments}
As is the case with real estate investments, infrastructure investments take a variety of forms. The choice of the vehicle affects the investor's liquidity, cash flow, and income streams. Investing directly in the underlying assets provides control and the opportunity to capture full value. It entails a large investment, however, and results in concentration and liquidity risks while the assets are managed and operated. Hence, direct investment often happens in consortia, with strategic partners better placed to manage certain risks (e.g., operational) or with other financial/institutional investors to limit individual concentration risk.

Most investors invest indirectly in infrastructure projects. Indirect investment vehicles include infrastructure funds (similar in structure to private equity funds and either closed end or open end), infrastructure ETFs, and company shares. Different asset class strategies, including debt and mezzanine investment, are offered, but the

\section{Private Capital, Real Estate, Infrastructure, Natural Resources, and Hedge Funds}
vast majority of investors are focused on equity instruments. According to IJInvestor, equity is preferred by $77 \%$ of fund investors, with $10 \%$ looking for mixed strategies and 9\% looking for debt and mezzanine strategies. ${ }^{5}$

Investors concerned about liquidity and diversification may choose publicly traded infrastructure securities or master limited partnerships. Publicly traded infrastructure securities have the benefits of liquidity, reasonable fees, transparent governance, market prices, and possible diversification among underlying assets. An investor should be aware, however, that publicly traded infrastructure securities represent a small segment of infrastructure investment and tend to be clustered in certain asset categories. Master Limited Partnerships (MLPs) trade on exchanges and are pass-through entities, similar to REITs. They also share with REITs applicable regulations that vary among countries, and income is passed through to the investors for taxation; MLPs generally distribute most free cash flow to their investors. Typically, the GP manages the partnership, receives a fee, and holds a small partnership interest, with LPs owning the remaining majority partnership interest.

\section{KNOWLEDGE CHECK}
\begin{enumerate}
  \item True or false: Infrastructure investments can be made both directly and indirectly.
\end{enumerate}

Explain your selection:
A. True
B. False

\section{Solution}
True. Infrastructure investments can be made both directly and indirectly. Direct investments often happen in consortia, with strategic partners or with other financial/institutional investors. However, most investors invest indirectly using infrastructure funds, ETFs, and company shares.

\section{Risk and Return Characteristics of Infrastructure}
The lowest-risk infrastructure investments have more stable cash flows and higher dividend payout ratios. They typically enjoy fewer growth opportunities, however, and have lower expected returns. For example, an MLP with an investment in a brownfield asset leased back to a government represents a low-risk infrastructure investment, as does an investment in assets with a history of steady cash flows, such as certain established toll roads. An investment in a fund building a new power plant without any operating history (a greenfield investment) is riskier. Exhibit 27 provides a breakdown of the typical risk-return profile of various infrastructure investments.

\section{Exhibit 16: Typical Infrastructure Investment Risk-Return Profile}
\begin{center}
\includegraphics[max width=\textwidth]{2023_05_04_36535b8d80b32081d422g-473}
\end{center}

There is a risk that infrastructure revenues diverge meaningfully from expectations, leverage creates financing risk, and operational risk and construction risk are always present. As for regulatory risk, because projects involve essential services, governments usually regulate the investments with strictures on the sale of the underlying assets, on operations and service quality, and on prices and profit margins. Global infrastructure investing introduces additional uncertainty, such as currency risk, political risk, and profit-repatriation risk. Hence, the fund manager must ensure these risks are appropriately managed and mitigated through insurance, financial instruments, and other available mechanisms. Exhibit 28 shows risk management considerations for various risks of infrastructure investments.

\section{Exhibit 17: Typical Infrastructure Risk Management}
\begin{center}
\includegraphics[max width=\textwidth]{2023_05_04_36535b8d80b32081d422g-474}
\end{center}

Returns, of course, depend on such factors as investment type. Exhibit 18 provides an illustrative range of returns for a private infrastructure fund according to different risk profiles, noting that returns may vary at an individual investment/asset level depending on its specific characteristics.

\section{Exhibit 18: Private Infrastructure Fund Illustrative Target Returns}
\begin{center}
\begin{tabular}{|c|c|c|}
\hline
Higher-Risk Profile & Medium-Risk Profile & Lower-Risk Profile \\
\hline
$\begin{array}{l}\text { Greenfield projects without } \\ \text { guarantees of demand upon } \\ \text { completion-e.g., variable elec- } \\ \text { tricity prices, uncertain traffic } \\ \text { on roads and through ports }\end{array}$ & $\begin{array}{l}\text { Mostly brownfield assets } \\ \text { (with some capital expen- } \\ \text { diture requirements) and } \\ \text { some greenfield assets (with } \\ \text { limited construction and } \\ \text { demand risk) }\end{array}$ & $\begin{array}{l}\text { Brownfield assets with miti- } \\ \text { gated risks-e.g., fully con- } \\ \text { structed with contracted/ } \\ \text { regulated revenues }\end{array}$ \\
\hline
$\begin{array}{l}\text { Located in OECD countries } \\ \text { and emerging markets }\end{array}$ & $\begin{array}{l}\text { Located primarily in OECD } \\ \text { countries }\end{array}$ & $\begin{array}{l}\text { Located in the most stable } \\ \text { OECD countries }\end{array}$ \\
\hline
$\begin{array}{l}\text { High weighting to capital } \\ \text { appreciation }\end{array}$ & $\begin{array}{l}\text { Mix of yield and capital } \\ \text { appreciation }\end{array}$ & $\begin{array}{l}\text { High weighting to current } \\ \text { yield }\end{array}$ \\
\hline
$\begin{array}{l}\text { Target equity returns of } \\ 14 \%+\end{array}$ & $\begin{array}{l}\text { Target equity returns of } \\ 10 \%-12 \%\end{array}$ & $\begin{array}{l}\text { Target equity returns of } \\ 6 \%-8 \%\end{array}$ \\
\hline
\end{tabular}
\end{center}

Infrastructure Opportunity Today," Cambridge Associates Research Note (April 2017).

Most infrastructure funds gravitate toward the medium- and lower-risk profiles. Rolling one-year-horizon returns have been around $10 \%$ annually since $2016 .{ }^{6}$

In summary, infrastructure investments generally provide investors with stable, long-term capital growth and cash distributions, according to the risks assumed, and these risks tend to be relatively well defined. Preqin maintains a return series of private funds investing in infrastructure deals. Standard \& Poor's, FTSE Group, and other firms also publish indexes of publicly traded infrastructure companies.

\section{KNOWLEDGE CHECK}
\begin{enumerate}
  \item Which of the following infrastructure investment characteristics is most likely valuable to investors aiming to sell newly constructed assets to the government?
\end{enumerate}

A. "Strategically important"

B. "Monopolistic and regulated"

C. "Significant capital investment"

\section{Solution}
A is correct. An infrastructure investment that is characterized as "strategically important" is the characteristic that will most likely increase the demand of the public buyer to effectively provide essential services to its citizens and therefore be the most valuable to investors. B would be more advantageous for investors holding and operating an asset and charging fees to the buyer, with the inelastic demand supporting pricing and regulations increasing the barriers to entry that improve the competitive position. $\mathrm{C}$ is more of a challenge than a benefit to the investor, requiring greater funding capability and potentially higher financial risks.

\section{Diversification Benefits of Infrastructure}
Investors expect infrastructure assets to generate stable long-term cash flows that adjust for economic growth and inflation. Investors may also expect capital appreciation, depending on the type and timing of their investment.

Investing in infrastructure may add an income stream, increase portfolio diversification by adding an asset class with typically low correlation with existing investments, and offer some protection against inflation. Low exposure to short-term GDP growth issues may also be a factor.

As an example of an investor seeking potential portfolio diversification benefits from infrastructure investments, Exhibit 19 provides an illustrative example of a hypothetical Australian pension fund looking at such opportunities. Exhibit 30 shows the low correlation of infrastructure-and in particular, unlisted Australian infrastructure-with other asset classes over the 10-year period through June 2019. Exhibit 19: Quarterly Return Correlations for Selected Asset Classes, 2009-2019

\begin{center}
\begin{tabular}{|c|c|c|c|c|c|c|}
\hline
 & $\begin{array}{c}\text { Global } \\ \text { Treasury } \\ \text { Bonds }\end{array}$ & $\begin{array}{c}\text { Global } \\ \text { Equities }\end{array}$ & $\begin{array}{c}\text { Australian } \\ \text { Equities }\end{array}$ & $\begin{array}{l}\text { Global } \\ \text { REITs }\end{array}$ & $\begin{array}{c}\text { Listed } \\ \text { Infrastructure }\end{array}$ & $\begin{array}{c}\text { Unlisted } \\ \text { Australian } \\ \text { Infrastructure }\end{array}$ \\
\hline
$\begin{array}{l}\text { Global Treasury } \\ \text { bonds }\end{array}$ & 1.00 &  &  &  &  &  \\
\hline
Global equities & -0.49 & 1.00 &  &  &  &  \\
\hline
$\begin{array}{l}\text { Australian } \\ \text { equities }\end{array}$ & -0.36 & 0.89 & 1.00 &  &  &  \\
\hline
Global REITs & 0.18 & 0.64 & 0.70 & 1.00 &  &  \\
\hline
$\begin{array}{l}\text { Listed } \\ \text { infrastructure }\end{array}$ & -0.12 & 0.81 & 0.75 & 0.68 & 1.00 &  \\
\hline
$\begin{array}{l}\text { Unlisted } \\ \text { Australian } \\ \text { infrastructure }\end{array}$ & -0.08 & 0.06 & -0.05 & -0.21 & 0.13 & 1.00 \\
\hline
\end{tabular}
\end{center}

Notes: All data are for the period 31 March 2009-30 June 2019, except for the FTSE EPRA/Nareit Developed Rental Net index (global REITs), for which the data series begins in May 2009.

\begin{itemize}
  \item Global Treasury bonds: Bloomberg Barclays Global Treasury GDP Index

  \item Global equities: MSCI World Net Accumulation Index

  \item Australian equities: S\&P/ASX 200

  \item Global REITs: FTSE EPRA/Nareit Developed Rental Net index

  \item Listed infrastructure: Dow Jones Brookfield Global Infrastructure Net Accumulation Index

  \item Unlisted infrastructure: MSCI Australian Unlisted Infrastructure Index

\end{itemize}

Source: John Julian, “The Pros of Infrastructure Investment in a Lower-for-Longer Environment," AMP Capital (October 2019). \href{http://www.ampcapital.com/au/en/insights-hub/articles/2019/october/the-pros-o}{www.ampcapital.com/au/en/insights-hub/articles/2019/october/the-pros-o} f-infrastructure-investment-in-a-lower-for-longer-environment.

Because these investments are placed in long-lived assets, infrastructure may better match the longer-term liability structure of certain investors, such as pension funds, superannuation schemes, and life insurance companies. It also suits the longer-term horizon of sovereign wealth funds, which tend to make the largest allocations to this asset class-around $5 \%-6 \%$ of total AUM, according to Preqin. ${ }^{7}$

\section{KNOWLEDGE CHECK}
\begin{enumerate}
  \item To highlight typical risk management approaches on infrastructure projects, draw connecting lines to match a particular risk type with its appropriate mitigating mechanism.
\end{enumerate}

Infrastructure Risk Type

\begin{enumerate}
  \item Construction

  \item Operational

  \item Demand/volume

  \item Currency

  \item Political Offsetting Mechanism
A. Incentivized maintenance contract
B. Insurance against debt default
C. PPP agreement
D. Take-or-pay arrangement
E. Fixed-price date-certain contract

\end{enumerate}

\section{Solution}
Option 1 (Construction) matches with $E$ (Fixed-price date-certain contract).

Option 2 (Operational) matches with A (Incentivized maintenance contract). Option 3 (Demand/volume) matches with D (Take-or-pay arrangement). Option 4 (Currency) matches with C (PPP agreement). Option 5 (Political) matches with B (Insurance against debt default).

\section{NATURAL RESOURCES}
explain investment characteristics of natural resources

\section{Summary-Natural Resources}
\begin{itemize}
  \item Natural resources include commodities (hard and soft), agricultural land (farmland), and timberland.

  \item Commodity investments may involve investing in actual physical commodities or in producers of commodities. More typically, they are made using commodity derivatives (futures or swaps), possibly via a CTA (see hedge funds).

  \item Returns to commodity investing are based on changes in price and do not include an income stream, such as dividends, interest, or rent (apart from income earned on the collateral). However, timberland offers an income stream based on the sale of trees, wood, and other products. In effect both a factory and a warehouse, timberland is a sustainable investment mitigating climate-related risks.

  \item Farmland, like timberland, has an income component related to harvest quantities and agricultural commodity prices. However, farmland lacks the production flexibility of timberland, because farm products must be harvested when ripe.

\end{itemize}

\section{Introduction and Overview of Natural Resources}
We can define natural resources as a unique asset category comprising commodities and raw land used for farming and timber. Some investors also include real estate assets and infrastructure in this category to form a "real assets" asset class, which can include inflation-protected securities and miscellaneous investments designed to protect against an increase in consumer prices. Others include timberland and farmland in their real estate portfolios. Regardless, the particular investment characteristics of natural resources, the specialized knowledge natural resources require, and their increased inclusion in portfolios today justify a separate examination of the sector.

Commodities are broadly familiar to both professional investors and the general public. Crude oil, soybeans, copper, and gold are all commodities. Commodities are considered either "hard" (those mined, such as copper, or extracted, such as oil) or "soft" (those grown over a period of time, such as livestock, grains, and cash crops, such as coffee). Timberland investment involves ownership of raw land and the harvesting of its trees for lumber, thus generating an income stream and the potential for capital gain; it has formed part of large institutional portfolios for decades. Farmland as an investment is a more recent phenomenon, with only a few dedicated funds involved. With population growth, weather, and water management becoming more topical, however, investors may turn to these land assets to address sustainability concerns in portfolios.

It is important to note that natural resources involve investing in physical land and products coming from that land-petroleum, metals, grains-not from the companies producing them. Until around the year 1999, the only commonly held investment vehicles related to this asset class were financial instruments, such as stocks and bonds. Today, many investment opportunities, such as ETFs, limited partnerships, REITS, swaps, and futures, are available.

This section covers both commodities and land (farmland and timberland), with the broader real estate category already covered.

\section{Description of Commodities}
\section{Categories of Commodities}
Commodities are traditionally physical products that can be standardized on quality, location, and delivery for the purpose of investing. Returns on commodity investments are primarily based on changes in price rather than on income from interest, dividends, or rent. Holding commodities (i.e., the physical products) incurs transportation and storage costs. Therefore, trading in physical commodities is primarily limited to a smaller group of entities in the physical supply chain. Most commodity investors trade commodity derivatives, not physical commodities. The underlying asset of the commodity derivative may be a single commodity or a commodity index.

Prices of commodity derivatives are, to a significant extent, a function of underlying commodity prices and thus are affected by the physical supply chain and general supply-demand dynamics. Commodity derivative index price volatility is highly correlated with the underlying physical goods. This makes sense because supply chain participants use futures to hedge forward purchases and sales of physical commodities. Imagine how uncertain food prices would be if soybean farmers could not hedge their crop risk. Investors and speculators trade commodity derivatives for profit based largely on changes or expected changes in underlying prices. Such non-hedging investors include retail and institutional investors, hedge funds, proprietary desks within financial institutions, and trading desks operating in the physical supply chain. (Commodity producers and consumers can both hedge and speculate on commodity prices.)

Commodity sectors include precious and base (i.e., industrial) metals, energy products, and agricultural products. Exhibit 20 offers examples of each type. The relative importance, amount, and price of individual commodities evolve with society's preferences and needs. Increasing industrialization of emerging markets has driven strong global demand for commodities. These markets need increasing amounts of oil, steel, and other materials to support manufacturing, infrastructure development, and the consumption demands of their populations. Emerging technologies, such as advanced cell phones and electric vehicles, create demand for new materials and destroy demand for old resources. Markets for specific commodities evolve over time. Exhibit 20: Examples of Commodities

\begin{center}
\begin{tabular}{ll}
\hline
Sector & Sample Commodities \\
\hline
Energy & Oil, natural gas, electricity, coal \\
Base metals & Copper, aluminum, zinc, lead, tin, nickel \\
Precious metals & Gold, silver, platinum \\
Agriculture & Grains, livestock, coffee \\
Other & Carbon credits, freight, forest products \\
\hline
\end{tabular}
\end{center}

Commodities may be further classified by physical location and grade or quality. For example, there are many grades and delivery locations for crude oil and wheat. Commodity derivative contracts thus specify quantity, quality, maturity date, and delivery location.

\section{KNOWLEDGE CHECK}
\begin{enumerate}
  \item Which of the following can be considered an investment in the commodities asset class?
A. A set of rare antique coins, some of which are made of silver and gold
B. A lease on an oil tanker shipping oil between Saudi Arabia and China
C. Ownership of a battery factory that uses industrial metals
D. A metric tonne of coffee packed in jute bags in a warehouse in Brazil
\end{enumerate}

\section{Solution}
D is correct, assuming the investor owns the commodity directly, not via a derivative contract. A describes an investment in collectables (whose value has more to do with the rarity of the collectible rather than the actual value of the silver or gold). B describes a financial contract-effectively, a lease or bond. $C$ is also a financial contract that pays on the use of commodities but depends on many other factors (technology, marketing, rent, and employees) for returns.

To be transparent, investable, and replicable, commodity indexes typically set their prices based on futures contracts rather than underlying commodities. Commodity index performance thus can differ from that of related physical commodities. Different commodity indexes are composed of different commodities and index weights. As a result, exposures to specific commodities and commodity sectors vary; however, the low correlation between commodities and other asset classes (e.g., stocks and bonds) means improved portfolio diversification is possible regardless of the index chosen.

Futures are the basis for most commodity investment. Each contract is for future delivery or receipt of a set amount of the commodity in question-for example, 1,000 barrels of oil or 10 metric tonnes of cocoa. The futures price can be formalized in the following form:

Futures price $\approx$ Spot price $(1+r)+$ Storage costs - Convenience yield,

where $r$ is the period's short-term risk-free interest rate. The contract's collateral should also be included for a total return calculation. The storage and interest costs together are sometimes referred to as the "cost of carry" or the "carry." Buyers of futures contracts have no immediate access to the commodity and, therefore, cannot

\section{Private Capital, Real Estate, Infrastructure, Natural Resources, and Hedge Funds}
benefit from it. So the futures price is adjusted for the loss of convenience, a value that varies. For example, the convenience yield from possessing heating oil in Finland is higher during the winter than in the summer.

Futures prices may be different from spot prices depending on the convenience yield. When futures prices are higher than the spot price, the commodity forward curve is upward sloping and the prices are referred to as being in contango. Contango generally occurs when there is little or no convenience yield. When futures prices are lower than the spot price, the commodity forward curve is downward sloping and the prices are referred to as being in backwardation. Backwardation occurs when the convenience yield is high. As a rule of thumb, a contango scenario generally lowers the return of the long-only investor, and a backwardation scenario enhances it.

The pricing of derivatives and the theories of commodity pricing are covered in other readings in the CFA Program curriculum.

\section{Digital Commodities}
Since 2015, the Commodity Futures Trading Commission (CFTC) has defined cryptocurrency as a "digital commodity," and the CFTC considers "digital assets" and "digital asset derivatives" as commodities and regulates them accordingly.

What is a "digital asset"? Digital assets continue to evolve. They vary in terms of design and application. A single, widely accepted definition of "digital asset" has yet to emerge. However, digital assets can be thought of as anything that can be stored and transmitted electronically and has associated ownership or use rights.

\section{Key Features of Digital Assets}
\begin{itemize}
  \item A digital asset can represent physical or virtual assets, a value, or a use right/service (e.g., computer storage space).

  \item Digital assets may take many forms (such as digital tokens and virtual currencies) and may utilize various underlying technologies, including distributed ledger technology (DLT).

  \item Digital assets have a varied set of features and applications that touch a range of regulatory domains.

  \item Digital assets are created and maintained with software (code) and exist as data on a network. The software and network together enable digital asset transactions.

  \item Depending on its design, function, and use, a digital asset may be characterized differently, including as a commodity, swap, or other derivative.

\end{itemize}

\section{Digital Assets vs. Virtual Currencies}
Digital asset is a broader term that encompasses additional applications, including ownership, transaction tracking, identity management, and smart contracts. A digital asset may express characteristics of a commodity or commodity derivative

Digital token refers to a digital asset that requires another blockchain network to operate and may serve a variety of functions beyond virtual currency-for example, utility tokens.

Virtual currencies-sometimes called "coins," "native tokens," or "intrinsic tokens"continue to be an important part of the digital asset landscape, which includes markets the CFTC oversees. A common feature of virtual currencies is their intended purpose to serve as a medium of exchange (source: \href{http://www.cftc.gov/digitalassets/index.htm}{www.cftc.gov/digitalassets/index.htm}).

\section{Forms of Commodity Investments}
The majority of commodity investing is implemented through derivatives. Physical commodities often generate opaque prices, tax obligations, and costs arising from storage, brokerage, and transportation, all increasing the attractiveness of standardized futures traded on transparent exchanges. Commodity derivatives include futures, forwards, options, and swaps, traded on exchanges or over the counter (OTC).

Futures contracts are obligations to buy or sell a specific amount of a given commodity at a fixed price, location, and date in the future. Futures contracts are exchange traded, are marked to market daily, and may or may not be settled on delivery or receipt of the physical commodity at the end of the contract. This delivery obligation became dramatically important during stressful periods. For example, during the global financial crisis in 2008 and in the 2020 COVID-19 pandemic, as oil demand collapsed, oil producers could not find buyers and global storage filled suddenly. Even commodity-related ETFs were affected, forcing some to close and impose large losses on investors.

For futures contracts, counterparty risk is managed by the exchange and clearing broker. Commodity exposure can be achieved through means other than direct investment in commodities or commodity derivatives, including the following:

\begin{itemize}
  \item Exchange-traded products (ETPs, either funds or notes) may be suitable for investors restricted to equity shares or seeking simplified trading through a standard brokerage account. ETPs may invest in commodities or commodity futures. For example, the SPDR Gold Shares ETF attempts to track the price of physical gold by holding bullion in vaults. It owned just under USD49 billion in gold bullion as of March 2020. ETPs may use leverage and be long or short (also known as "inverse"). Similar to mutual funds or unit trusts, ETPs charge fees included in their expense ratios.

  \item Investing with Commodity Trading Advisors (CTAs) are another way to gain commodity exposure. However, to gain pure commodity exposure, one would need to find a fund focused solely on commodities because modern CTAs often invest in a variety of futures, including commodities, equities, fixed income, and foreign exchange. A commodity-focused CTA might concentrate on a specific commodity (such as grains) or be broadly diversified across commodities. More details on CTAs and managed futures are covered in Section 10, "Hedge Funds."

  \item Funds specializing in specific commodity sectors exist. For example, private energy partnerships, which are similar to private equity funds, enable institutional exposure to the energy sector. Management fees range from 1\% to $3 \%$ of committed capital, with a lockup periods of 10 years and extensions of 1- and 2-year periods. Publicly available energy mutual funds and unit trusts typically focus on the oil and gas sector, often acting as fixed-income investments to pay dividends from rents or capital gains. They may focus on upstream (drilling), midstream (refineries), or downstream (chemicals). Their management fees compare with those of other public equity managers and range from $0.4 \%$ to $1 \%$.

\end{itemize}

\section{Description of Timberland and Farmland}
\section{Categories of Timberland and Farmland}
This section discusses land owned or leased for the returns it generates from crops and timber. Given that these resources consume carbon as part of the plant life cycle, their value comes not just from the harvest but also from the offset to human

\section{Private Capital, Real Estate, Infrastructure, Natural Resources, and Hedge Funds}
activity. Water rights are also part of the direct and implied value of these properties; conservation easements may create value by supporting traditions and nature conservation. Demand for arable land may rise as interest in investments that adhere to ESG considerations grows.

Sustained interest in these investments stems from their global nature, the income from crops, inflation protection from holding land, and their degree of insulation from financial market volatility. US farmland, for example, enjoyed positive returns both during periods when US GDP declined significantly (1973-1975 and 2007-2009) and when the United States experienced higher-than-normal inflation (1915-1920, 1940-1951, and 1967-1981). Timberland has been part of institutional and ultra-high-net-worth portfolios for decades, typically trading in larger units of land. Farmland can be found in much smaller sizes-perhaps tens of or a few hundred acres. In fact, many farms are still family owned-for example, 98\% of US farmland as of 2019 (according to \href{http://www.ers.usda.gov}{www.ers.usda.gov}). However, family farms are increasingly leased out to commercial farms with the family retaining ownership. One of the main challenges of these investments is their long market cycle, particularly in new-growth forest and crops that are picked, such as fruit.

Timberland's income stream is based on the sale of trees, wood, and other timber products and has been not highly correlated with other asset classes. Timberland serves as both a factory and a warehouse. Timber (trees) can be grown and easily stored by simply not harvesting it. This characteristic offers the flexibility of harvesting when timber prices are higher and delaying harvests when prices are down. The three primary return drivers are biological growth, changes in spot and futures prices of lumber (cut wood), and changes in the price of the underlying land.

Farmland is often perceived as an inflation hedge. Similar to timberland, the returns include an income component related to harvest quantities and agricultural commodity prices. Farmland consists mainly of row crops that are planted and harvested (i.e., more than one round of planting and harvesting can occur in a year depending on the crop and the climate) and permanent crops that grow on trees (e.g., nuts) or vines (e.g., grapes). Unlike timberland, farm products must be harvested when ripe, with little flexibility in production. Therefore, commodity futures contracts can be combined with farmland holdings to generate an overall hedged return. A farm is inherently "long" its crop and, therefore, will sell futures for delivery at the time of the harvest. Farmland may also be used as pastureland for livestock. Like timberland, farmland has three primary return drivers: harvest quantities, commodity prices (e.g., the price of corn), and land price changes.

\section{KNOWLEDGE CHECK}
\begin{enumerate}
  \item Alexandra is considering buying a tract of farmland for long-term capital appreciation and current income. Which of the following factors plays a role in evaluating the attractiveness of the investment?
\end{enumerate}

A. The land's rights to preferential water access

B. The land's proximity to a large, privately held nature preserve

C. The land's soil chemical composition allowing the growth of a variety of crops

D. All of the above

\section{Solution}
The answer is D. A is important to ensure that crops can be grown with a reasonable chance of success. B allows for the land to be rented to the stewards of the nature preserve and to lay fallow instead of growing food for human consumption. Nature preserve foundations are interested in the land around them because (1) runoff and pesticide use affect the plants and animals in the preserve, (2) migration of animals in the preserve may incur onto the farmland, and (3) foundations often look to add to their holdings, allowing for a future sale of the land. $\mathrm{C}$ allows for the possibility of being able to react to market preferences and more stable productivity and thus more reliable income.

\begin{enumerate}
  \setcounter{enumi}{1}
  \item Large institutional investors consider timberland investments because:
\end{enumerate}

A. small parcel sizes permit fine tuning of their holdings across geography and wood types.

B. The optionality around harvesting gives investors the choice between cutting trees for lumber and current income or letting them grow another year for future gain.

C. The short return history allows for many alpha opportunities by knowledgeable active managers.

D. Clear-cutting trees and modifying nature is appreciated by ESG investors, which are becoming a larger portion of the investment universe.

\section{Solution}
B is the correct answer. Large institutional investors consider timberland investments because optionality around harvesting gives investors the choice between cutting trees for lumber and current income or letting them grow another year for future gain. A is incorrect because the parcel sizes are generally large, especially compared with farmland. $\mathrm{C}$ is incorrect because the return history is relatively long, not short. D states the opposite reason why ESG investors may be interested in timberland investments-for the opportunity to create "conservation zones."

\section{Forms of Timberland and Farmland Investments}
The primary investment vehicles for timberland and farmland are investment funds, whether offered on the public markets, such as real estate investment trusts in the United States, or administered privately through limited partnerships. Larger investors may consider direct investments for particular assets with appeal. For example, Middle Eastern sovereign wealth funds (SWFs) have made investments in farmland in Africa and Southeast Asia.

Owning physical farmland opens the door to a wide variety of foodstuffs-spices, nuts, fruits, and vegetables-a much broader array than the corn, soy, and wheat typically offered by futures investment. There is, however, limited price transparency or information to guide investment decisions without the assistance of sector specialists. The illiquidity of direct farm and timberland investments is also limiting.

\section{Risk and Return Characteristics of Natural Resources}
Commodities, farmland, and timberland have different return drivers and cycles. Commodities are priced on a second-by-second basis on public exchanges, whereas land generally has an infrequent pricing mechanism and may include estimates as opposed to actual transactions. Keeping these market structure differences in mind will help investors consider their relative benefits and challenges.

\section{Private Capital, Real Estate, Infrastructure, Natural Resources, and Hedge Funds}
\section{Risk/Return of Commodities}
Investing in commodities is motivated by portfolio diversification opportunities. In Exhibit 32, we present the historical returns of commodities. Commodity investments, especially when combined with leverage, exhibit high volatility.

Exhibit 21: Historical Returns of Commodities, 1990-Q1 2020 (quarterly data)

\begin{center}
\begin{tabular}{lccc}
\hline
 & Global Stocks & Global Bonds & Commodities \\
\hline
Annualized return & $4.8 \%$ & $5.3 \%$ & $-1.7 \%$ \\
Annualized standard & $16.2 \%$ & $6.0 \%$ & $23.8 \%$ \\
deviation &  &  &  \\
Worst calendar year & $-43.5 \%$ & $-5.2 \%$ & $-46.5 \%$ \\
 & $(2008)$ & $(1999)$ & $(2008)$ \\
Best calendar year & $31.6 \%$ & $19.7 \%$ & $49.7 \%$ \\
 & $(2003)$ & $(1995)$ & $(2000)$ \\
\hline
\end{tabular}
\end{center}

Sources: Global stocks = MSCI ACWI Index; global bonds = Bloomberg Barclays Global Aggregate Index; commodities = S\&P GSCI Total Return.

Commodity prices = function (supply, demand, cost of production, cost of storage, value to users, global economic conditions).

Supplies of physical commodities are determined by production and inventory levels and secondarily by the actions of non-hedging investors. Demand for commodities is determined by the needs of end users and secondarily by the actions of non-hedging investors. Investor actions can both dampen and stimulate commodity price movements, at least in the short term.

Producers cannot alter commodity supply levels quickly because extended lead times are often needed to affect production levels. For example, agricultural output may be altered by planting more crops and changing farming techniques, but at least one growing cycle must pass before there are results. And at least one factor beyond the producer's control-the weather-will significantly affect output. Building the necessary infrastructure for increased oil and mining production may take many years, involving both developing the mine itself and the necessary transportation and smelting components. For commodities, suppliers' inability to quickly respond to changes in demand may result in supply too low in times of economic growth and too high when the economy slows. And despite advancing technology, the cost of new supply may grow over time.

Overall demand levels are influenced by global manufacturing dynamics and economic growth. When demand levels and investors' orders to buy and sell during a given period change quickly, the resulting mismatch of supply and demand may lead to price volatility.

\section{Risk/Return of Timberland and Farmland}
Turning to land, Exhibit 22 provides a comparison of returns on US timber and farmland. The National Council of Real Estate Investment Fiduciaries (NCREIF) constructs a variety of appraisal-based indexes for property, timber, and farmland. Over the 1990-Q1 2020 period, farmland had the highest annualized return and timber had the highest standard deviation. Exhibit 22: Historical Returns of US Real Estate Indexes, 1990-Q1 2020 (quarterly data)

\begin{center}
\begin{tabular}{lcc}
\hline
 & \multicolumn{2}{c}{NCREIF Data} \\
\hline
Annualized return & Timber & Farmland \\
Annualized standard deviation & $9.1 \%$ & $11.0 \%$ \\
Worst calendar year & $7.0 \%$ & $5.9 \%$ \\
 & $-5.2 \%$ & $2.0 \%$ \\
Best calendar year & $(2001)$ & $(2001)$ \\
 & $37.3 \%$ & $33.9 \%$ \\
 & $(1992)$ & $(2005)$ \\
\hline
\end{tabular}
\end{center}

Although the data in Exhibit 22 make farmland appear to be a very attractive investment, it has definite risks. Liquidity is very low and the risk of negative cash flow high because fixed costs are relatively high (land requires care and crops need fertilizer, seed, and so on), and revenue is highly variable based on weather. The risks of timber and farmland are similar to those of other real estate investments in raw land, but we highlight weather as a unique and more exogenous risk for these assets compared to traditional commercial and residential real estate properties. Drought and flooding can dramatically decrease the harvest yields for crops and thus the expected income stream. The second primary risk is the international competitive landscape. Although real estate is often considered a local investment, productive land generates commodities that are globally traded and consumed. For example, there have been interruptions in world trade, and growing agricultural competition has resulted in declining grain prices. Timber and farmland investments should consider the international context as a major risk factor.

\section{KNOWLEDGE CHECK}
\begin{enumerate}
  \item Describe a unique risk of an investment in farmland that distinguishes it from a real estate investment in raw land.
\end{enumerate}

\section{Solution}
There are two significant risks that distinguish an investment in farmland from a real estate investment in raw land:

\begin{enumerate}
  \item Weather is a more unique and exogenous risk for farmland, with drought or flooding dramatically decreasing many crop yields and thus the expected income stream.

  \item Productive land generates globally traded and consumed commodities. This international competitive landscape can result in interruptions in world trade, growing foreign agricultural competition, and resulting declines in crop prices.

\end{enumerate}

\section{Diversification Benefits of Natural Resources}
Although they often entail higher transaction costs and higher informational hurdles, alternative assets such as natural resources generally offer diversification as a major benefit. Knowing the source of that diversification helps managers to better set investor expectations.

\section{Diversification Benefits of Commodities}
Beyond its profit potential, commodity investing can have appeal for two key reasons:

\begin{enumerate}
  \item Commodities are effective for portfolio diversification, with commodity returns historically having a low correlation with other investment returns.

  \item Commodities are effective hedges against inflation, with commodity prices historically positively correlated with inflation.

\end{enumerate}

Institutional investors, particularly endowments, foundations, and increasingly corporate and public pension funds and sovereign wealth funds, are increasing allocations to commodities and commodity derivatives.

The portfolio diversification argument is based on the observation that commodities historically have behaved differently from stocks and bonds during the business cycle. Exhibit 23 shows the quarterly correlation between selected commodity, global equity, and global bond indexes. From 1990 through Q1 2020, commodities had low correlations with global stocks and global bonds of 0.37 and 0.05 , respectively. The correlations of stocks, bonds, and commodities are expected to be positive because all the assets have some exposure to the global business cycle. The selected commodity index, the S\&P GSCI (Goldman Sachs Commodity Index), is heavily weighted toward the energy sector, with each underlying commodity possibly exhibiting unique behavior.

Exhibit 23: Historical Commodity Return Correlations, 1990-Q1 2020

(quarterly data)

\begin{center}
\begin{tabular}{lcccc}
\hline
 & Global Stocks & Global Bonds & Commodities & US CPI \\
\hline
Global stocks & 1 & 0.09 & 0.37 & 0.21 \\
Global bonds &  & 1 & 0.05 & -0.03 \\
Commodities &  &  & 1 & 0.64 \\
\hline
\end{tabular}
\end{center}

Sources: Global stocks = MSCI ACWI Index; global bonds = Bloomberg Barclays Global Aggregate Index; commodities = S\&P GSCI Total Return.

The argument for commodities as a hedge against inflation derives from some commodity prices being components of inflation calculations. Commodities, especially energy and food, affect consumers' cost of living. The positive correlation between quarterly commodity price changes and quarterly changes in the US CPI of 0.64 supports this assertion. In contrast, the quarterly return correlations between the US CPI and global stocks and global bonds are close to zero. The volatility of commodity prices, especially energy and food, is much higher than that of reported consumer inflation. Consumer inflation is computed from many products, including housing, whose prices change more slowly than commodity prices, and inflation calculations use statistical smoothing techniques and behavioral assumptions. However, even in the recent period of low inflation (or even negative inflation in many countries), investing in commodities outperformed global stocks and bonds on average during the same calendar year if inflation was relatively high (i.e., US CPI greater than $2 \%$; see Exhibit 24). Exhibit 24: Historical Asset Class Returns When US CPI Was >2\% vs. <2\%, 1990-2019 (quarterly data)

\begin{center}
\begin{tabular}{lccc}
\hline
 & Global Stocks & Global Bonds & Commodities \\
\hline
Higher inflation & $+10.9 \%$ & $+7.1 \%$ & $+15.6 \%$ \\
Lower inflation & $+0.3 \%$ & $+2.7 \%$ & $-20.0 \%$ \\
\hline
\end{tabular}
\end{center}

Sources: Global stocks = MSCI ACWI Index; global bonds = Bloomberg Barclays Global Aggregate Index; commodities = S\&P GSCI Total Return.

In addition, Exhibit 24 affirms the sensitivity of commodities to inflation as measured by its relatively high correlation.

However, in looking at the weight of food and energy in inflation calculations, they are a relatively small portion. A review of the US CPI calculations indicates that their approximate weight of the commodity price impacts is $13 \%$. Furthermore, this number is overstated because these price changes as measured by CPI include all the other costs to consumers, such as gasoline taxes, supermarket rent and labor costs, and brand marketing. Therefore, even though there is a direct impact from changes in commodity costs on consumer inflation, it is marginal for the purposes of this discussion.

Exhibit 25 outlines a number of common assets in various regimes of inflation. This highlights the diversification opportunities that can exist between commodities and traditional asset classes.

\section{Exhibit 25: Asset Returns in Various Inflation Regimes}
\begin{center}
\includegraphics[max width=\textwidth]{2023_05_04_36535b8d80b32081d422g-487}
\end{center}

Equities Fixed Income REITs Commodities

Source: Wellington Management Company LLP.

\section{Diversification Benefits of Timberland and Farmland}
Investors may also review qualitative aspects of diversification. For example, to adhere to the ESG principles of responsible and sustainable investing, they may add timberland and farmland to portfolios. The following case study shows how investing in timberland can help mitigate climate change.

\section{TIMBERLAND CASE STUDY: CAMPBELL GLOBAL}
\section{Investing Responsibly in Timberland Assets: A Climate-Conscious Case Study}
Campbell Global (CG) is a global investment manager focused on forest and natural resources investments.

\begin{center}
\includegraphics[max width=\textwidth]{2023_05_04_36535b8d80b32081d422g-488}
\end{center}

Source: Campbell Global, LLC.

In addition to their economic value, forests serve as vast carbon sinks, with trees removing atmospheric $\mathrm{CO}_{2}$ and providing carbon storage. In one year, a single Douglas fir tree, a common commercial timber species in the US Pacific Northwest, stores the $\mathrm{CO}_{2}$ equivalent of driving 400 miles in a standard automobile. ${ }^{8}$ Globally, the Earth's forests are estimated to absorb as much as $30 \%$ of human-induced $\mathrm{CO}_{2}$ emissions. ${ }^{9}$

Sustainably harvested wood products and materials also store atmospheric $\mathrm{CO}_{2}$ long after removal from a forest, with one cubic meter of wood capturing nearly a metric ton of $\mathrm{CO}_{2}{ }^{10}$ In addition to carbon sequestration, forests also provide benefits of clean water and wildlife habitat, recreational opportunities, and a source of living-wage jobs in rural communities. These attributes are in agreement with the UN's Sustainable Development Goals and contribute to advancing the UN's mission for sustainable future globally. With these effects, there is increasing awareness that well-managed forests are a critical component of any global climate change strategy.

CG uses scenario analyses to identify climate-related risks beginning at a broad country-level scale, narrowing down to a specific property, and then testing the impact of various risks to the site's present and future suitability. Factors analyzed to gauge climate risks include precipitation patterns, temperature fluctuations, severity of weather events, presence of pests or disease, and annual average growth rates for commercial tree species. While many climate-related risks in forestry are mitigated through active management, during this iterative process CG analyzes both the potential positive and negative impacts associated with these risks to assess potential changes in net asset value. The following table illustrates climate risks evaluated, their impact on the forest, and the ability to mitigate the risks.

\begin{center}
\begin{tabular}{|c|c|c|}
\hline
Climate Risk & Implication & CG Mitigants \\
\hline
$\begin{array}{l}\text { Change in } \\ \text { temperature }\end{array}$ & $\begin{array}{l}\text { Increased fire } \\ \text { danger }\end{array}$ & $\begin{array}{l}\text { Property-specific fire plans; re-evaluate } \\ \text { target regions/country for investment }\end{array}$ \\
\hline
$\begin{array}{l}\text { Change in } \\ \text { precipitation } \\ \text { patterns }\end{array}$ & $\begin{array}{l}\text { Changes in tree } \\ \text { species' range; } \\ \text { increased drought } \\ \text { and related fire risk }\end{array}$ & $\begin{array}{l}\text { Vegetation suitability modeling and } \\ \text { genetic tree improvement; re-evaluate } \\ \text { target regions/country for investment }\end{array}$ \\
\hline
$\begin{array}{l}\text { Frequency of } \\ \text { extreme weather } \\ \text { events }\end{array}$ & $\begin{array}{l}\text { Loss of standing } \\ \text { timber from wind } \\ \text { events }\end{array}$ & $\begin{array}{l}\text { Re-evaluate target regions; } \\ \text { property-specific response plans; geo- } \\ \text { graphically diverse portfolio construction }\end{array}$ \\
\hline
\end{tabular}
\end{center}

\begin{center}
\begin{tabular}{lll}
\hline
Climate Risk & \multicolumn{1}{c}{Implication} & \multicolumn{1}{c}{CG Mitigants} \\
\hline
$\begin{array}{ll}\text { Presence of pests } \\ \text { or disease }\end{array}$ & $\begin{array}{l}\text { Early onset and } \\ \text { increased frequency } \\ \text { of individual tree } \\ \text { mortality }\end{array}$ & $\begin{array}{l}\text { Immediate treatment, which may include } \\ \text { removal of affected trees to prevent } \\ \text { further spread of pests or disease in the } \\ \text { forest }\end{array}$ \\
Increased or &  &  \\
Change in & $\begin{array}{l}\text { Effects will vary by region, may influence } \\ \text { growth }\end{array}$ & $\begin{array}{l}\text { planting stock decisions; re-evaluate } \\ \text { forest growth model assumptions }\end{array}$ \\
\hline
\end{tabular}
\end{center}

Climate change opportunities and challenges highlighted in the CG investment process include the following:

\begin{itemize}
  \item Identifying afforestation opportunities that mitigate climate change by sequestering $\mathrm{CO}_{2}$ emissions from the atmosphere in trees and soil, while offering many important co-benefits for communities, biodiversity, and soil and water quality.

  \item Protecting existing carbon stocks by minimizing impacts to carbon stored on the forest floor through tailored forest management practices.

  \item Enhancing forest carbon sequestration by replanting areas as soon as possible so the new forest will quickly begin removing $\mathrm{CO}_{2}$ from the atmosphere.

\end{itemize}

The ability to quantify, evaluate, and report the year-over-year changes in the carbon footprint of a forest can influence the impacts an organization has on the environment, leading to increased transparency and more-informed business decisions. Incorporating climate change factors in a firm's investment process not only mitigates climate-related risks; it also promotes and enhances the natural solutions forests provide. Understanding and measuring the comprehensive carbon stores of forests may lead to business decisions that improve carbon sequestration, critical for addressing climate change.

Exhibit 26 shows the quarterly correlation between timber, farmland, and selected global equity and global bond indexes. For the 1990-2020 period, timber and farmland exhibited low correlation with traditional assets; the correlation of timber with global stocks and global bonds was 0.04 and 0.07 , while the correlation of farmland with global stocks and global bonds was 0.15 and -0.04 , respectively.

Exhibit 26: Historical Commodity Return Correlations, 1990-Q1 2020 (quarterly basis)

\begin{center}
\begin{tabular}{lcc}
\hline
NCREIF Data & Timber & Farmland \\
\hline
Global stocks & 0.04 & 0.15 \\
Global bonds & 0.07 & -0.04 \\
\hline
\end{tabular}
\end{center}

Sources: Global stocks = MSCI ACWI Index; global bonds = Bloomberg Barclays Global Aggregate Index.

The United States has experienced three periods of elevated inflation as measured by the US Consumer Price Index since the early 1900s, as shown in Exhibit 38. Farmland provided investors with meaningful inflation protection in each of the time periods.

Exhibit 38 Annualized US Inflation Rates and Agricultural Commodity Returns during Inflationary Periods

\begin{center}
\begin{tabular}{lccc}
\hline
 & $\mathbf{1 9 1 5 - 1 9 2 0}$ & $\mathbf{1 9 4 0 - 1 9 5 1}$ & $\mathbf{1 9 6 7 - 1 9 8 1}$ \\
\hline
Inflation Rate & $13.5 \%$ & $5.9 \%$ & $7.6 \%$ \\
Ag. Commodity & $9.9 \%$ & $7.9 \%$ & $12.0 \%$ \\
Returns &  &  &  \\
\hline
\end{tabular}
\end{center}

Similar to the challenges of comparing buildings in real estate, it is difficult to reliably or consistently compare a plot of prime farmland with a more marginal neighbor or parcels with diverging water rights. As with other alternative benchmarks (e.g., hedge fund indexes), productive land indexes may offer a good overall understanding but limited specific value.

\section{KNOWLEDGE CHECK}
\begin{enumerate}
  \item True or false: Stocks and bonds of companies involved in the production of natural resources are the most appropriate instruments for retail investors wanting investment exposure to the sector.
\end{enumerate}

Explain your selection.

A. True

B. False

\section{Solution}
False. Stocks and bonds of companies involved in the production of natural resources are not the most appropriate instruments for retail investors wanting investment exposure to the sector.

Up to about 1999, the only commonly available investment vehicles related to this asset class were indeed financial instruments (stocks and bonds). Rather than investing in the physical land and the products that come from it, investors concentrated on the companies that produced natural resources. Currently, however, the wide variety of direct investments via ETFs, limited partnerships, REITS, swaps, and futures allows for almost everyone to participate in these assets directly.

\section{HEDGE FUNDS}
explain investment characteristics of hedge funds

\section{Summary-Hedge Funds}
\begin{itemize}
  \item Hedge funds are typically classified by strategy. One such classification includes four broad categories of strategies: equity hedge (e.g., market neutral), event driven (e.g., merger arbitrage), relative value (e.g., convertible bond arbitrage), and macro and CTA strategies (e.g., commodity trading advisers). - Funds of hedge funds are funds that create a diversified portfolio of hedge funds. These vehicles are attractive to small investors without the resources to select individual hedge funds and build a portfolio of them.
\end{itemize}

\section{Introduction and Overview of Hedge Funds}
Hedge funds originally started as equity investment vehicles in which offsetting short and long positions protected the overall portfolio against major stock market moves, but hedge funds are no longer restricted to equities. Many hedge funds trade sovereign and corporate debt, commodities, futures contracts, options, and other derivatives. Even real estate investments are found in hedge fund structures. Nor do all hedge funds maintain short positions or use leverage. Instead, many simply exploit niche areas of expertise in a sophisticated manner; hedging and leverage may or may not be involved.

Fee structures have evolved to include management fees in addition to incentive fees, which can extend from a flat percentage rate to include hurdle rates and other mechanisms.

Investors in modern hedge funds are subject to extended holding periods (known as lockup periods) and subsequent notice periods before an investment redemption is possible. Compared, for example, to many mutual funds allowing easier redemption and guaranteed liquidity on one day's notice, such mandatory periods combine with a lighter regulatory compliance burden (in certain areas) to allow hedge funds more flexibility. With reduced operating constraints, hedge funds may invest in less liquid and unnoticed opportunities, the true valuation of which may at times be opaque.

Restrictions on redemptions by investors are typically imposed. Investors may be required to keep their money in the hedge fund for a lockup period before they are allowed to make withdrawals or redeem shares. Investors may also be required to give notice, typically 30-90 days, of their intent to redeem. A redemption fee may be charged, typically payable to the fund itself (rather than the manager). This is to protect remaining investors in the fund, particularly in circumstances where the redemption takes place during the lockup period. This characteristic is called a soft lockup, and it offers a path (albeit an expensive one) to redeem early. With more flexibility in portfolio construction, hedge funds enjoy leeway to invest in situations in which time may be needed to generate an expected return and that thus are unsuitable for a mutual fund offering daily liquidity.

Hedge funds are generally deemed riskier from an oversight (fraud risk) point of view, but some hedge funds take less absolute market risk in their portfolio construction than is taken by registered products available to retail investors. A hedge fund's true market risk and its distinction between regulatory risk and illiquidity risk can thus often be confused.

A contemporary hedge fund generally has the following characteristics:

\begin{itemize}
  \item Portfolio management: Creatively selects investments involved in one or more asset classes (equities, credit, fixed income, commodities, futures, foreign exchange, and sometimes even hard assets, such as real estate). These sometimes trade in different geographic regions. Leverage is often used. Generally takes both long and short positions (when possible), and quite often uses derivatives to express a view or establish a hedge. Often enjoys light investment restrictions detailed in its private placement offering memorandum.

  \item Goal: Generate high returns, either in an absolute sense or on a risk-adjusted basis relative to its portfolio-level volatility. Gauging performance relative to a traditional index benchmark can be difficult, given that low correlation with traditional asset investing is frequently a selling point.

\end{itemize}

\section{KNOWLEDGE CHECK}
\begin{enumerate}
  \item The goal of hedge fund redemption restrictions is typically to hedge fund manager flexibility.
\end{enumerate}

\section{Solution}
The goal of hedge fund redemption restrictions is typically to increase hedge fund manager flexibility.

\section{Description of Hedge Funds}
\section{Categories of Hedge Fund Investments}
Hedge funds are typically classified by strategy, although categorizations vary. Many classifications focus on the most common strategies, but others focus on the basis of different criteria, such as the underlying assets in which the funds are invested. Classifications change over time as new strategies arise, often based on new products and market opportunities. Hedge fund categorization is important to allow investors to review aggregate performance data, select strategies with which to build a portfolio of funds, and select or construct appropriate performance benchmarks. HFR classifies hedge funds under four broad strategy categories, as follows:

\begin{itemize}
  \item Equity hedge strategies

  \item Event-driven strategies

  \item Relative value strategies

  \item Macro strategies

\end{itemize}

Exhibit 27 shows the approximate percentage of hedge fund AUM by strategy, according to HFR, for 1990, 2010, and 2019. Based on this AUM measure, event-driven and relative value strategies have grown in popularity during the last 30 years, while macro and equity hedge funds have declined.

\section{Exhibit 27: Percentage of AUM by Strategy}
\begin{center}
\includegraphics[max width=\textwidth]{2023_05_04_36535b8d80b32081d422g-492}
\end{center}

Source: HFR data, as of March 2019.

\section{Equity Hedge Strategies}
Equity hedge strategies can be considered the original hedge fund category. They focus on public equity markets and take long and short positions in equity and equity derivative securities. Most equity hedge strategies use a "bottom-up" security-specific approach-company-level analysis, followed by overall industry analysis, followed by overall market analysis-with relatively balanced long and short exposures. A contrasting "top-down" approach entails global macro analysis, followed by sector/ regional analysis, followed by individual company analysis or any market-timing approach. Some strategies use individual equities only, whereas others use index ETF securities for portfolio-balancing purposes. Examples of equity hedge strategies include the following:

\begin{itemize}
  \item Market neutral. These strategies use quantitative (technical) and fundamental analysis to identify under- and overvalued equity securities. The hedge fund takes long positions in securities identified as undervalued and short positions in overvalued securities. The hedge fund tries to maintain a net position that is neutral with respect to market risk and other risk factors (size, industry, momentum, value, etc.). Ideally, the portfolio has an overall beta of approximately zero. The intent is to profit from the movements of individual securities while remaining largely indifferent to market risk. To achieve a meaningful return, market-neutral portfolios may require the application of leverage. They are generally stable, low-return portfolios but are exposed to occasional spurts of volatility, when sudden leverage reduction may be required.

  \item Fundamental L/S growth. These strategies use fundamental analysis to identify companies expected to exhibit high growth and capital appreciation. The hedge fund takes long positions in these companies while shorting companies with business models being disintermediated or under downward pressure-where continued revenue growth is deemed unlikely. Portfolios tend to end up long biased.

  \item Fundamental value. These strategies use fundamental analysis to identify companies deemed undervalued and unloved for any number of corporate-performance or sector-driven reasons but with an identified path to corporate rejuvenation. The hedge fund takes long positions in these companies and sometimes hedges the portfolio by shorting index ETFs or more growth-oriented companies deemed overvalued. Portfolios also tend to be long biased. They have a positive factor bias to value as well as to small-cap factors since many underappreciated value situations tend to be in the small-cap sector.

  \item Short biased. These strategies use quantitative (technical) and fundamental analysis to focus the bulk of their portfolio on shorting overvalued equity securities (against limited or no long-side exposures). Managers are often more forensic in their fundamental analysis and sometimes are activist in trying to expose previously unrecognized accounting or business flaws and improve their portfolio's chance for profits. These funds vary in adjusting their short exposure over time. Managers tend to be contrarian, and the funds can be useful additions to larger portfolios in terms of weathering periods of market stress. Short-biased managers, however, have had a difficult time overall posting meaningful long-term returns during the past 30 years of generally positive market conditions.

  \item Sector specific. These strategies exploit manager expertise in a particular sector and use quantitative (technical) and fundamental analysis to identify opportunities. Technology, biotech/life sciences, and financial services

\end{itemize}

\section{Private Capital, Real Estate, Infrastructure, Natural Resources, and Hedge Funds}
are common areas of focus. Given increased availability of company news, added sector-specialist expertise may be a prerequisite for truly differentiated security selection. The more complex a sector (as in biotechnology, with its binary outcomes on drug trials) or the more opaqueness in accounting practices (as with financial or insurance stocks), the more value sector-specific managers bring.

\section{KNOWLEDGE CHECK}
\begin{enumerate}
  \item Identify three types of equity hedge strategies.
\end{enumerate}

\section{Solution}
Types of equity hedge strategies include market neutral, fundamental longshort growth, fundamental value, short biased, and sector specific.

\section{Event-Driven Strategies}
Event-driven strategies seek to profit from defined catalyst events, typically involving changes in corporate structure, such as an acquisition or restructuring. They are considered to be based on bottom-up security-specific analysis, as opposed to top-down analysis. Investments may include long and short positions in common and preferred stocks, debt securities, and options. Further subdivisions of this category by HFR include the following:

\begin{itemize}
  \item Merger arbitrage. Generally, these strategies involve going long (buying) the stock of the company being acquired at a discount to its announced takeover price and going short (selling) the stock of the acquiring company when the merger or acquisition is announced. The manager may expect to profit once the initial deal spread narrows to the closing value of the transaction after it is fully consummated. The spread exists because there is always some uncertainty over closure of the deal given legal and regulatory hurdles. Equally, the acquirer may step away due to a sudden change in market conditions damaging perceived merger attractiveness. Shorting the acquirer is also a way to express the risk of merger overpayment or of an increased post-merger debt load. The primary risk in merger arbitrage is that the announced combination fails to occur and the deal spread re-widens to pre-merger levels before the hedge fund can unwind its position. Since the expected risk and return on a merger arbitrage strategy typically starts off being quite modest, leverage is regularly used to amplify merger spreads into acceptable total return targets. Unfortunately, this application of leverage also increases losses should actual circumstances move against the investment.

  \item Distressed/restructuring. These strategies focus on securities of companies either in or perceived to be near bankruptcy. In one approach, hedge funds simply purchase fixed-income securities trading at a significant discount to par but still senior enough to be deemed "money good" (backed by enough corporate assets to be fully repayable at par or at least at a significant premium to the available bond purchase price) in a bankruptcy situation. Alternatively, a fund may purchase the so-called fulcrum debt instrument expected to convert into new equity upon restructuring or bankruptcy.

  \item Special situations. These strategies focus on opportunities to buy equity of companies engaged in restructuring activities other than mergers, acquisitions, or bankruptcy. These activities include security issuance or repurchase, special capital distributions, rescue finance, and asset sales/spin-offs. - Activist. The term "activist" is short for "activist shareholder." Here, managers secure sufficient equity holdings to allow them to influence corporate policies or direction. They thus try to create business catalysts, moving the investment towards a desired outcome. For example, the activist hedge fund may advocate for divestitures, restructuring, capital distributions to shareholders, or changes in management and company strategy affecting their equity holdings. Such hedge funds are distinct from private equity because they operate primarily in the public equity market.

\end{itemize}

Event-driven strategies tend to be long biased, with merger arbitrage having the least bias. It nonetheless still tends to suffer when market conditions weaken because of increased risk that mergers will fail.

\section{KNOWLEDGE CHECK}
\begin{enumerate}
  \item Explain the goal of event-driven strategies.
\end{enumerate}

\section{Solution}
Event-driven strategies seek to profit from defined catalyst events, typically those that involve changes in corporate structure, such as an acquisition or restructuring.

\section{Relative Value Strategies}
Relative value funds seek to profit from a pricing discrepancy between related securities based on an unusual short-term relationship. The expectation is that the discrepancy will be resolved over time. Examples of relative value strategies include the following:

\begin{itemize}
  \item Convertible bond arbitrage. This conceptually market-neutral investment strategy seeks to exploit a perceived mispricing between a convertible bond and its component parts -namely, the underlying bond and the embedded stock option-relative to the pricing of a reference equity into which the bond may someday convert. The strategy typically involves buying convertible debt securities and simultaneously selling a certain amount of the same issuer's common stock. Residual bankruptcy risks can be further hedged using equity put options or credit default swap derivative hedges on the credit of the issuer.

  \item Fixed income (general). These strategies focus on the relative value within the fixed-income markets, with an emphasis on sovereign debt and sometimes the relative pricing of investment-grade corporate debt. Strategies may incorporate long-short trades between two different issuers, between corporate and government issuers, between different parts of the same issuer's capital structure, or between different parts of an issuer's yield curve. Currency dynamics and government yield-curve considerations may also come into play.

  \item Fixed income (asset backed, mortgage backed, and high yield). These strategies focus on the relative value of various higher-yielding securities, including asset-backed securities (ABS), mortgage-backed securities (MBS), and high-yield loans and bonds. The hedge fund seeks both to earn an attractive highly secured coupon and to exploit relative security mispricings. At times, some of these securities may be further parsed into structured note products with unique return attributes relative to interest rate movements and general credit spread changes. Opaque mark-to-market pricing and illiquidity issues are significant considerations. "Mark-to-market" refers to the current expected fair market value for which a given security

\end{itemize}

\section{Private Capital, Real Estate, Infrastructure, Natural Resources, and Hedge Funds}
would likely be available for purchase or sale if traded in current market conditions. For structured products, absent hard broker/dealer quotes, this is often a "mark-to-model" type of calculation that is less reliable than an actual bid-offer spread between active market participants.

\begin{itemize}
  \item Volatility. These strategies typically use options to go long or short market volatility, either in a specific asset class or across asset classes. For example, a short-volatility strategy involves selling options to earn the premiums and benefit from calm markets, but it can experience significant losses during unexpected periods of market stress. A long-volatility strategy tends to suffer the cost of small premiums in anticipation of larger market moves where positions may benefit from both directionality and the ability to rebalance option-based exposure through various methods.

  \item Multi-strategy. These strategies trade relative value within and across asset classes or instruments. Rather than focusing on one type of trade (e.g., convertible arbitrage), a single basis for a trade (e.g., volatility), or a particular asset class (e.g., fixed income), this strategy instead looks for any available investment opportunities, often with different pods of managers executing unique market approaches. The goal of a multi-strategy manager is to initially deploy (and later redeploy) capital efficiently and quickly across various strategy areas as conditions change.

\end{itemize}

\section{KNOWLEDGE CHECK}
\begin{enumerate}
  \item True or false: Event-driven funds seek to profit from a pricing discrepancy, an unusual short-term relationship, between related securities.
\end{enumerate}

Explain your reasoning.

A. True

B. False

\section{Solution}
False. Relative value funds seek to profit from a pricing discrepancy, an unusual short-term relationship, between related securities.

\section{Macro and CTA Strategies}
Macro strategies emphasize a top-down approach to identify economic trends. Trades are made on the basis of expected movements in economic variables. Generally, these funds trade opportunistically in fixed-income, equity, currency, and commodity markets. Macro hedge funds use long and short positions to profit from a view on overall market direction because it is influenced by major economic trends and events. Because these funds generally benefit most from periods of higher volatility, the active moves by national authorities, such as central banks, to smooth out economic shocks likely shrink their investment sphere.

Managed futures funds are actively managed funds making diversified directional investments primarily in the futures markets on the basis of technical and fundamental strategies. Managed futures funds are also known as commodity trading advisers (CTAs) because they historically focused on commodity futures. However, CTAs may include investments in a variety of futures, including commodities, equities, fixed income, and foreign exchange. CTAs generally use models that measure trends and momentum over different time horizons. CTA investments can be useful for portfolio diversification, particularly in times of strong trending market conditions and especially during periods of acute market stress when other fundamental strategies may be expected to perform poorly. However, mean-reverting markets, which may cause false momentum breakout signals, can lead to uncomfortable and occasionally extended drawdown periods before strong trends emerge for the CTA. To the extent that many CTAs have migrated to trade more and more financial products (such as stock index futures and bond futures), the reliability of CTA diversification benefits has diminished.

\section{Forms of Hedge Fund Investments}
A common structural characteristic of a hedge fund is that it is set up as a private investment partnership or offshore fund. Under certain legal restrictions (which vary by jurisdiction), the offering can be open only to a limited number of investors willing and able to make a substantive initial investment.

An investor can choose to select and invest in an individual hedge fund. But selecting the appropriate individual hedge fund manager can require extensive due diligence to understand the fund's investment strategy and to accurately measure returns over time. In addition, minimum investments can be substantial, with lockup periods and redemption restrictions, as described below, that can be imposed before the investor again can access to the invested capital. So if the timing proves poor relative to the success of a particular strategy, the relative performance of the total portfolio can suffer. As a result, some investors may opt to invest in a fund of hedge funds. Funds of hedge funds are funds that hold a portfolio of hedge funds. They create a diversified portfolio of hedge funds and are aimed at smaller investors or those who do not have the resources, time, or expertise to choose among hedge fund managers. Also, the managers of funds of hedge funds (commonly shortened to "funds of funds") are required to have some expertise in conducting due diligence on hedge funds and may be able to negotiate better redemption or fee terms than individual investors can. Funds of funds may diversify across fund strategies, investment regions, and management styles.

Funds of funds may charge an added $1 \%$ management fee and an added $10 \%$ incentive fee ( 1 and 10) on top of the fees charged by the underlying hedge funds held in the fund of funds. At both the hedge fund level and the fund-of-funds level, the incentive fee is typically calculated on profits net of management fees.

Fee structures are important to scrutinize here. Fee layering can sorely reduce the end investor's initial gross return of an investment in a hedge fund or fund of funds. Each hedge fund into which a fund of funds invests is structured to receive a management fee plus an incentive fee. The result may be the investor is paying fees more than once for management of the same assets. Finally, liquidity can be of additional concern for investors in funds of funds in times of crisis, when fund redemptions can hurt performance.

\section{KNOWLEDGE CHECK}
\begin{enumerate}
  \item Define funds of hedge funds, and identify their primary purpose.
\end{enumerate}

\section{Solution}
Funds of hedge funds are funds that hold a portfolio of hedge funds. Their primary purpose is to add value in manager selection and due diligence and create a diversified portfolio of hedge funds accessible to small investors.

\section{Private Capital, Real Estate, Infrastructure, Natural Resources, and Hedge Funds}
\begin{enumerate}
  \setcounter{enumi}{1}
  \item Identify two potential concerns for an investor using a hedge fund of funds.
\end{enumerate}

\section{Solution}
One potential concern for an investor using a hedge fund of funds is fee layering, which can result in the investor paying fees more than once for the management of the same assets. Another potential concern is liquidity in times of crisis, when fund redemptions can hurt performance.

According to Eurekahedge, by 2018 the average hedge fund fee level had compressed to $1.3 \%$ in management fees and $15.5 \%$ in incentive fees-considerably less than the traditional 2 and 20 structure but still quite high relative to traditional mutual funds. Fund-of-funds fee levels similarly have compressed; funds of funds can be found charging, for example, either a $1 \%$ flat management fee or a $50 \mathrm{bp}$ management fee and only a $5 \%$ incentive fee.

Realization of the value-destruction potential of fee layers combined with increasingly compressed hedge fund returns (for various reasons) made funds of funds far less popular in 2020 than in the 2000-05 period. Still, a fund of funds may offer compensating advantages, such as access for smaller investors, diversified hedge fund portfolios, better redemption terms, and due diligence expertise.

\section{Risk and Return Characteristics of Hedge Funds}
Exhibit 28 compares the returns, risk, and performance measures for the HFRI Fund of Funds Composite Index, the MSCI ACWI Index, and the Bloomberg Barclays Global Aggregate Index. The HFRI Fund of Funds Composite Index is an equally weighted performance index of funds of hedge funds included in the HFR Database. Hedge fund indexes suffer from issues related to self-reporting and survivorship bias. However, the HFRI Fund of Funds Composite Index reflects the actual performance of portfolios of hedge funds. The measures shown here may reflect a lower reported return because of the added layer of fees, but they likely represent a fairer, more conservative, and more accurate estimate of average hedge fund performance than HFR's composite index of individual funds. The returns are likely just mildly biased toward equity long-short funds since these are always a substantial portion of funds of funds' allocation mix.

As shown in Exhibit 28, over the 25-year period between 1990 and 2014, hedge funds enjoyed higher returns than either stocks or bonds and a standard deviation almost identical to that of bonds. On the basis of the Sortino ratio, hedge funds do not appear as attractive as bonds if returns are adjusted for downside deviation, as reflected in the relative Sortino measures shown in Exhibit 28. The worst drawdown, or period of largest cumulative negative returns for hedge funds and global equities, began in 2007 (when each peaked) and ended in 2009.

Notwithstanding this period of stress, when accounting for hedge fund returns' modest overall correlation with global stock returns (0.56) and negligible correlation with global bond returns (0.07) over this 25-year period, hedge funds have certainly offered added value to institutional investors as a portfolio diversifier.

Exhibit 28: Historical Risk-Return Characteristics of Hedge Funds and Other Investments, 1990-2014

\begin{center}
\begin{tabular}{lccc}
\hline
 & FoF & Global Stocks & Global Bonds \\
\hline
Annualized return & $7.2 \%$ & $6.9 \%$ & $6.3 \%$ \\
Annualized volatility & $6.0 \%$ & $16.5 \%$ & $5.8 \%$ \\
Sharpe ratio & 0.63 & 0.21 & 0.49 \\
\end{tabular}
\end{center}

\begin{center}
\begin{tabular}{lccc}
\hline
 & FoF & Global Stocks & Global Bonds \\
\hline
Sortino ratio & 0.74 & 0.43 & 1.09 \\
Percentage of positive months & $69.3 \%$ & $61.3 \%$ & $62.7 \%$ \\
Best month & $6.8 \%$ & $11.9 \%$ & $6.2 \%$ \\
Worst month & $-7.5 \%$ & $-19.8 \%$ & $-3.8 \%$ \\
Worst drawdown & $-22.2 \%$ & $-54.6 \%$ & $-10.1 \%$ \\
FoF correlation (avg. monthly) &  & 0.56 & 0.07 \\
\hline
\end{tabular}
\end{center}

Sources: Fund-of-funds (FoF) data are from the HFRI Fund of Funds Composite Index; global stock data are from the MSCI ACWI Index; global bond data are from the Bloomberg Barclays Global Aggregate Index.

Notably, as shown in Exhibit 29, for the subsequent five-year period between 2015 and 2019 , the absolute return of funds of funds relative in particular to global equities has declined, while their performance correlation with equity markets actually increased. This trend has made hedge fund allocations arguably less useful and also somewhat less popular. But some allocators have continued to find value in maintaining or actually increasing their allocation to a mix of hedge funds as a bond market substitute in their overall portfolio building.

Exhibit 29: Historical Risk-Return Characteristics of Hedge Funds and Other Investments, 2015-2019

\begin{center}
\begin{tabular}{lccc}
\hline
 & FoF & Global Stocks & Global Bonds \\
\hline
Annualized return & $2.48 \%$ & $10.41 \%$ & $2.42 \%$ \\
Annualized volatility & $1.08 \%$ & $11.42 \%$ & $1.33 \%$ \\
FoF correlation (avg. monthly) &  & 0.86 & 0.03 \\
\hline
\end{tabular}
\end{center}

Sources: FoF data are from the HFRI Fund of Funds Composite Index; global stock data are from the MSCI ACWI Index; global bond data are from the Bloomberg Barclays Global Aggregate Index (USD Unhedged). The risk-free rate used in the Sharpe ratio calculation is from the average of one-month T-bill rates over the period.

\section{Diversification Benefits of Hedge Funds}
The diversification benefits of hedge funds first came to prominence with the dot-com bubble unwinding in 2000-2002, when they generally performed well compared with traditional long-only investment products. Starting with funds of funds, institutional investors increased their hedge fund exposure and then expanded it further through direct allocations following the 2008 financial crash. Seeking better diversification and risk mitigation, despite high hedge fund fees, the investors sought absolute and uncorrelated risk-adjusted returns rather than outsized upside performance. The institutionalization of the hedge fund industry changed its very nature: Most hedge funds failed to keep pace with the positive equity and bond market advances of 2009-2019, but they maintained a place in institutional asset allocations because of their risk diversification properties.

Exhibit 42 shows return relationships between hedge funds and some major asset categories from 2011 to 2020 . While hedge fund risk diversification benefits can merit investigation, experience also shows that their prospective advantage can vary over time. Investors need to be thorough in conducting due diligence when selecting a hedge fund manager. Exhibit 30: Historical Hedge Fund Return Correlations, January 2011December 2020 (monthly data)

\begin{center}
\includegraphics[max width=\textwidth]{2023_05_04_36535b8d80b32081d422g-500}
\end{center}

Sources: Hedge fund data are from the Dow Jones Credit Suisse Hedge Fund Index. Investment-grade bond data are from the Bloomberg Barclays US Aggregate Bond Index.

\section{PRACTICE PROBLEMS}
\begin{enumerate}
  \item A collateralized loan obligation specialist is most likely to:
A. sell its debt at a single interest rate.
B. cater to niche borrowers in specific situations.
C. rely on diverse risk profiles to complete deals.

  \item Private capital is:

\end{enumerate}

A. accurately described by the generic term "private equity."

B. a source of diversification benefits from both debt and equity.

C. predisposed to invest in both the debt and equity of a client's firm.

\begin{enumerate}
  \setcounter{enumi}{2}
  \item The first stage of financing at which a venture capital fund most likely invests is the:
A. seed stage.
B. mezzanine stage.
C. angel investing stage.

  \item A private equity fund desiring to realize an immediate and complete cash exit from a portfolio company is most likely to pursue:
A. an IPO.
B. a trade sale.
C. a recapitalization.

  \item Angel investing capital is typically provided in which stage of financing?
A. Later stage
B. Formative stage
C. Mezzanine stage

  \item Private equity funds are most likely to use:
A. merger arbitrage strategies.
B. leveraged buyouts.
C. market-neutral strategies.

  \item The majority of real estate property may be classified as either:
A. debt or equity.
B. commercial or residential.
C. direct ownership or indirect ownership. 8. Which of the following relates to a benefit when owning real estate directly?
A. Taxes
B. Capital requirements
C. Portfolio concentration

  \item Which of the following statements regarding mortgage-backed securities is true?

\end{enumerate}

A. Insurance companies prefer the first-loss tranche.

B. When interest rates rise, prepayments will likely accelerate.

C. When interest rates fall, the low-risk senior tranche will amortize more quickly.

\begin{enumerate}
  \setcounter{enumi}{9}
  \item Which of the following statements regarding REITs is true?
\end{enumerate}

A. According to GAAP, equity REITs are exempt from reporting earnings per share.

B. Though equity REIT correlations with other asset classes are typically moderate, they are highest during steep market downturns.

C. The REIT corporation pays taxes on income, and the REIT shareholder pays taxes on the REIT's dividend distribution of after-tax earnings.

\begin{enumerate}
  \setcounter{enumi}{10}
  \item What is the most significant drawback of a repeat sales index to measure returns to real estate?
\end{enumerate}

A. Sample selection bias

B. Understatement of volatility

C. Reliance on subjective appraisals

\begin{enumerate}
  \setcounter{enumi}{11}
  \item As the loan-to-value ratio increases for a real estate investment, risk most likely increases for:
\end{enumerate}

A. debt investors only.

B. equity investors only.

C. both debt and equity investors.

\begin{enumerate}
  \setcounter{enumi}{12}
  \item Compared with direct investment in infrastructure, publicly traded infrastructure securities are characterized by:
A. higher concentration risk.
B. more transparent governance.
C. greater control over the infrastructure assets.

  \item Which of the following forms of infrastructure investment is the most liquid?
A. An unlisted infrastructure mutual fund
B. A direct investment in a greenfield project
C. An exchange-traded MLP 15. An investor chooses to invest in a brownfield, rather than a greenfield, infrastructure project. The investor is most likely motivated by:
A. growth opportunities.
B. predictable cash flows.
C. higher expected returns.

  \item The privatization of an existing hospital is best described as:
A. a greenfield investment.
B. a brownfield investment.
C. an economic infrastructure investment.

  \item Risks in infrastructure investing are most likely greatest when the project involves:
A. construction of infrastructure assets.
B. investment in existing infrastructure assets.
C. investing in assets that will be leased back to a government.

  \item A primary risk to investing in timber is most likely its:
A. high correlation with other asset classes.
B. dependence on an international competitive context.
C. return volatility compounded by financial market exposure.

  \item A significant characteristic of farmland distinguishing it from timberland is its:
A. commodity price-driven returns.
B. less flexibility in timing of harvest.
C. value as an offset to other human activities.

  \item Which of the following statements about commodity investing is invalid?
A. Few commodity investors trade actual physical commodities.
B. Commodity producers and consumers both hedge and speculate.
C. Commodity indexes are based on the price of physical commodities.

  \item An investor seeks a current income stream as a component of total return and desires an investment that historically has low correlation with other asset classes. The investment most likely to achieve the investor's goals is:
A. timberland.
B. collectibles.
C. commodities.

  \item If a commodity's forward curve is upward sloping and there is little or no conve- nience yield, the market is said to be in:
A. backwardation.
B. contango.
C. equilibrium.

  \item Which approach is most commonly used by equity hedge strategies?
A. Top down
B. Bottom up
C. Market timing

  \item An investor may prefer a single hedge fund to a fund of funds if she seeks:
A. due diligence expertise.
B. better redemption terms.
C. a less complex fee structure.

  \item Hedge funds are similar to private equity funds in that both:
A. are typically structured as partnerships.
B. assess management fees based on assets under management.
C. do not earn an incentive fee until the initial investment is repaid.

  \item Both event-driven and macro hedge fund strategies use:
A. long-short positions.
B. a top-down approach.
C. long-term market cycles.

  \item Hedge fund losses are most likely to be magnified by a:
A. margin call.
B. lockup period.
C. redemption notice period.

  \item An equity hedge fund following a fundamental growth strategy uses fundamental analysis to identify companies that are most likely to:
A. be undervalued.
B. be either undervalued or overvalued.
C. experience high growth and capital appreciation.

\end{enumerate}

\section{SOLUTIONS}
\begin{enumerate}
  \item C is correct. A CLO manager will extend several loans to corporations (usually to firms involved in LBOs, corporate acquisitions, or other similar types of transactions), pool these loans, and then divide that pool into various tranches of debt and equity that range in seniority and security. The CLO manager will then sell each tranche to different investors according to their risk profiles; the most senior portion of the CLO will be the least risky, and the most junior portion of the CLO (i.e., equity) will be the riskiest.
\end{enumerate}

A is incorrect because with the different CLO tranches having distinct risks varying with their seniority and security, they will be priced over a range of interest rates. In contrast, unitranche debt combines different tranches of secured and unsecured debt into a single loan with a single, blended interest rate.

$B$ is incorrect because debt extended to niche borrowers in specific situations is more commonly offered through specialty loans. For example, in litigation finance, a specialist funding company provides debt to a client to finance the borrower's legal fees and expenses in exchange for a portion of any case winnings.

\begin{enumerate}
  \setcounter{enumi}{1}
  \item B is correct. Investments in private capital funds can add diversity to a portfolio composed of publicly traded stocks and bonds because they have less-than-perfect correlation with those investments. There is also the potential to offer further diversification within the private capital asset class. For example, private equity investments may also offer vintage diversification since capital is not deployed at a single point in time but is invested over several years. Private debt provides investors with the opportunity to diversify the fixed-income portion of their portfolios since private debt investments offer more choices than bonds and other public forms of traditional fixed income.
\end{enumerate}

A is incorrect because although private equity is considered by many to be the largest component of private capital, using "private equity" as a generic term could be less accurate and possibly misleading since other private forms of alternative finance have grown considerably in size and popularity.

$\mathrm{C}$ is incorrect because although many private investment firms have private equity and private debt arms, these teams typically will not invest in the same assets or businesses to avoid overexposure to a single investment.

\begin{enumerate}
  \setcounter{enumi}{2}
  \item A is correct. The seed stage supports market research and product development and is generally the first stage at which venture capital funds invest. The seed stage follows the angel investing stage. In the angel investing stage, funds are typically provided by individuals (often friends or family), rather than a venture capital fund, to assess an idea's potential and to transform the idea into a plan. Mezzanine-stage financing is provided by venture capital funds to prepare the portfolio company for its IPO.

  \item B is correct. Private equity funds can realize an immediate cash exit in a trade sale. Using this strategy, the portfolio company is typically sold to a strategic buyer.

  \item B is correct. Formative-stage financing occurs when the company is still in the process of being formed and encompasses several financing steps. Angel investing capital is typically raised in this early stage of financing.

  \item B is correct. The majority of private equity activity involves leveraged buyouts. Merger arbitrage and market neutral are strategies used by hedge funds.

  \item B is correct. The majority of real estate property may be classified as either com- mercial or residential.

  \item A is correct. When owning real estate directly, there is a benefit related to taxes. The owner can use property non-cash depreciation expenses to reduce taxable income and lower the current income tax bill. In fact, accelerated depreciation and interest expense can reduce taxable income below zero in the early years of asset ownership, and losses can be carried forward to offset future income. Thus, a property investment can be cash-flow positive while generating accounting losses and deferring tax payments. If the tax losses do not reverse during the life of the asset, depreciation-recapture taxes can be triggered when the property is sold.

\end{enumerate}

$B$ is incorrect because the large capital requirement is a major disadvantage of investing directly in real estate.

$\mathrm{C}$ is incorrect because a disadvantage for smaller investors who own real estate directly is that they bear the risk of portfolio concentration.

\begin{enumerate}
  \setcounter{enumi}{8}
  \item $C$ is correct. When interest rates decline, borrowers are likely to refinance their loans at a faster pace than before, resulting in faster amortization of each MBS tranche, including the senior tranche, which is the lowest-risk tranche.
\end{enumerate}

A is incorrect because risk-averse investors, primarily insurance companies, prefer the lowest-risk tranches, which are the first to receive interest and principal. The junior-most tranche is referred to as the first-loss tranche. It is the highest-risk tranche and is the last to receive interest and principal distributions. B is incorrect because when interest rates rise, prepayments will likely slow down, lengthening the duration of most MBS tranches. Prepayments will likely increase when interest rates decline, because borrowers are likely to refinance their loans at a faster pace.

\begin{enumerate}
  \setcounter{enumi}{9}
  \item B is correct. Real estate investments, including REITs, provide important portfolio benefits due to moderate correlation with other asset classes. However, there are periods when equity REIT correlations with other securities are high, and their correlations are highest during steep market downturns.
\end{enumerate}

A is incorrect because equity REITs, like other public companies, must report earnings per share based on net income as defined by GAAP or IFRS.

$\mathrm{C}$ is incorrect because REITs can avoid this double taxation. A REIT can avoid corporate income taxation by distributing dividends equal to $90 \%-100 \%$ of its taxable net rental income. This ability to avoid double taxation is the main appeal of the REIT structure.

\begin{enumerate}
  \setcounter{enumi}{10}
  \item A is correct. A repeat sales index uses the changes in price of repeat sales properties to construct the index. Sample selection bias is a significant drawback because the properties that sell in each period vary and may not be representative of the overall market the index is meant to cover. The properties that transact are not a random sample and may be biased toward properties that changed in value. Understated volatility and reliance on subjective appraisals by experts are drawbacks of an appraisal index.

  \item $\mathrm{C}$ is correct. The higher the loan-to-value ratio, the higher leverage is for a real estate investment, which increases the risk to both debt and equity investors.

  \item B is correct. Publicly traded infrastructure securities, which include shares of companies, exchange-traded funds, and listed funds that invest in infrastructure, provide the benefits of transparent governance, liquidity, reasonable fees, market prices, and the ability to diversify among underlying assets. Direct investment in infrastructure involves a large capital investment in any single project, resulting in high concentration risks. Direct investment in infrastructure provides control over the assets and the opportunity to capture the assets' full value.

  \item C is correct. Publicly traded infrastructure securities, such as exchange-traded MLPs, provide the benefit of liquidity.

  \item B is correct. A brownfield investment is an investment in an existing infrastructure asset, which is more likely to have a history of steady cash flows compared with that of a greenfield investment. Growth opportunities and returns are expected to be lower for brownfield investments, which are less risky than greenfield investments.

  \item $B$ is correct. Investing in an existing infrastructure asset with the intent to privatize, lease, or sell and lease back the asset is referred to as a brownfield investment. An economic infrastructure asset supports economic activity and includes such assets as transportation and utility assets. Hospitals are social infrastructure assets, which are focused on human activities.

  \item A is correct. Infrastructure projects involving construction have more risk than investments in existing assets with a demonstrated cash flow or investments in assets that are expected to generate regular cash flows.

  \item B is correct. A primary risk of timber is the international competitive landscape. Timber is a globally sold and consumed commodity subject to world trade interruptions. So the international context can be considered one of its major risk factors.

\end{enumerate}

A is incorrect because timberland offers an income stream based on the sale of trees, wood, and other timber products that has not been highly correlated with other asset classes.

$\mathrm{C}$ is incorrect because investors are interested in timber due to its global nature (everyone requires shelter), the current income generated from the sale of the product, inflation protection from holding the land, and its safe haven characteristics (it offers some insulation from financial market volatility).

\begin{enumerate}
  \setcounter{enumi}{18}
  \item B is correct. Unlike timberland products, farm products must be harvested when ripe, so there is little flexibility in the timing of harvest. In contrast, timber (trees) can be grown and easily "stored" by simply not harvesting. This feature offers the flexibility of harvesting more trees when timber prices are up and delaying harvests when prices are down.
\end{enumerate}

A is incorrect because just as a primary return driver for timberland is change in either the spot or futures price of the commodity (lumber from cut wood), farmland's returns are driven by agricultural commodity prices, with commodity futures contracts potentially combined with farmland holdings to generate an overall hedged return.

$\mathrm{C}$ is incorrect because both farmland and timberland are owned or leased for the benefit of the bounty each generates in the form of crops and timber. And since these resources consume carbon as part of the plant life cycle, the considered value comes not just from the harvest but also from the offset to other human activities.

\begin{enumerate}
  \setcounter{enumi}{19}
  \item $\mathrm{C}$ is correct. In order to be transparent, investable, and replicable, commodity indexes typically use the price of futures contracts on the commodities included in the index rather than the prices of the physical commodities themselves.
\end{enumerate}

A is incorrect because trading in physical commodities is primarily limited to a smaller group of entities that are part of the physical supply chain. Thus, most commodity investors do not trade actual physical commodities but, rather, trade commodity derivatives. $B$ is incorrect because although supply chain participants use futures to hedge their forward purchases and sales of the physical commodities, those commodity producers and consumers nonetheless both hedge and speculate on commodity prices.

\begin{enumerate}
  \setcounter{enumi}{20}
  \item A is correct. Timberland offers an income stream based on the sale of timber products as a component of total return and has historically generated returns not highly correlated with other asset classes.

  \item B is correct. Contango is a condition in the futures markets in which the spot price is lower than the futures price, the forward curve is upward sloping, and there is little or no convenience yield. Backwardation is the opposite condition in the futures markets, where the spot price exceeds the futures price, the forward curve is downward sloping, and the convenience yield is high. Equilibrium is an economic term where supply is equal to demand.

  \item B is correct. Most equity hedge strategies use a bottom-up strategy. A is incorrect because most equity hedge strategies use a bottom-up (not top-down) strategy.

\end{enumerate}

$\mathrm{C}$ is incorrect because most equity hedge strategies use a bottom-up (not market-timing) strategy.

\begin{enumerate}
  \setcounter{enumi}{23}
  \item C is correct. Hedge funds of funds have multi-layered fee structures, whereas the fee structure for a single hedge fund is less complex. Funds of funds presumably have some expertise in conducting due diligence on hedge funds and may be able to negotiate more favorable redemption terms than an individual investor in a single hedge fund could.

  \item A is correct. Private equity funds and hedge funds are typically structured as partnerships where investors are limited partners and the fund is the general partner. The management fee for private equity funds is based on committed capital, whereas for hedge funds, the management fees are based on assets under management. For most private equity funds, the general partner does not earn an incentive fee until the limited partners have received their initial investment back.

  \item A is correct. Long-short positions are used by both types of hedge funds to potentially profit from anticipated market or security moves. Event-driven strategies use a bottom-up approach and seek to profit from a catalyst event typically involving a corporate action, such as an acquisition or a restructuring. Macro strategies seek to profit from expected movements in evolving economic variables.

  \item A is correct. Margin calls can magnify losses. To meet the margin call, the hedge fund manager may be forced to liquidate a losing position in a security, which, depending on the position size, could exert further price pressure on the security resulting in further losses. Restrictions on redemptions, such as lockup and notice periods, may allow the manager to close positions in a more orderly manner and minimize forced-sale liquidations of losing positions.

  \item $\mathrm{C}$ is correct. Fundamental growth strategies take long positions in companies identified, using fundamental analysis, to have high growth and capital appreciation. Fundamental value strategies use fundamental analysis to identify undervalued companies. Market-neutral strategies use quantitative and fundamental analysis to identify under- and overvalued companies.

\end{enumerate}

\section{Portfolio Management}
\section*{LEARNING MODULE 
 1 }
\section{Portfolio Management: An Overview}
Owen M. Concannon, CFA, is at Neuberger Berman (USA). Robert M. Conroy, DBA, CFA, is at the Darden School of Business, University of Virginia (USA). Alistair Byrne, PhD, CFA, is at State Street Global Advisors (United Kingdom). Vahan Janjigian, PhD, CFA, is at Greenwich Wealth Management, LLC (USA).

\section{LEARNING OUTCOME}
\begin{center}
\begin{tabular}{c|l}
Mastery & The candidate should be able to: \\
\hline
$\square$ & describe the portfolio approach to investing \\
$\square$ & $\begin{array}{l}\text { describe the steps in the portfolio management process } \\ \text { describe types of investors and distinctive characteristics and needs } \\ \text { of each } \\ \text { describe defined contribution and defined benefit pension plans } \\ \text { describe aspects of the asset management industry } \\ \text { describe mutual funds and compare them with other pooled } \\ \text { investment products }\end{array}$ \\
\end{tabular}
\end{center}

\section{INTRODUCTION}
This reading provides an overview of portfolio management and the asset management industry, including types of investors and investment plans and products. A portfolio approach is important to investors in achieving their financial objectives. We outline the steps in the portfolio management process in managing a client's investment portfolio. We next compare the financial needs of different types of investors: individual and institutional investors. We then describe both defined contribution and defined benefit pension plans. The asset management ${ }^{1}$ industry, which serves as a critical link between providers and seekers of investment capital around the world, is broadly discussed. Finally, we describe mutual funds and other types of pooled investment products offered by asset managers.

1 Note that both "investment management" and "asset management" are commonly used throughout the CFA Program curriculum. The terms are often used interchangeably in practice.

\section{PORTFOLIO PERSPECTIVE: DIVERSIFICATION AND RISK REDUCTION}
describe the portfolio approach to investing

One of the biggest challenges faced by individuals and institutions is to decide how to invest for future needs. For individuals, the goal might be to fund retirement needs. For such institutions as insurance companies, the goal is to fund future liabilities in the form of insurance claims, whereas endowments seek to provide income to meet the ongoing needs of such institutions as universities. Regardless of the ultimate goal, all face the same set of challenges that extend beyond just the choice of what asset classes to invest in. They ultimately center on formulating basic principles that determine how to think about investing. One important question is: Should we invest in individual securities, evaluating each in isolation, or should we take a portfolio approach? By "portfolio approach," we mean evaluating individual securities in relation to their contribution to the investment characteristics of the whole portfolio. In the following section, we illustrate a number of reasons why a diversified portfolio perspective is important.

\section{Historical Example of Portfolio Diversification: Avoiding Disaster}
Portfolio diversification helps investors avoid disastrous investment outcomes. This benefit is most convincingly illustrated by examining what may happen when individuals have not diversified.

We are usually not able to observe how individuals manage their personal investments. However, in the case of US $401(\mathrm{k})$ individual retirement portfolios, ${ }^{2}$ it is possible to see the results of individuals' investment decisions. When we examine their retirement portfolios, we find that some individual participants make sub-optimal investment decisions.

During the 1990s, Enron Corporation was one of the most admired corporations in the United States. A position in Enron shares returned over 27 percent per year from 1990 to September 2000, compared to 13 percent for the S\&P 500 Index for the same time period.

2 In the United States, 401(k) plans are employer-sponsored individual retirement savings plans. They allow individuals to save a portion of their current income and defer taxation until the time when the savings and earnings are withdrawn. In some cases, the sponsoring firm will also make matching contributions in the form of cash or shares. Individuals within certain limits have control of the invested funds and consequently can express their preferences as to which assets to invest in. Exhibit 1: Value of US\$1 Invested from January 1990 to September 2000

Enron vs. S\&P 500 Composite Index (01/01/1990 = US\$1.00)

\begin{center}
\includegraphics[max width=\textwidth]{2023_05_04_36535b8d80b32081d422g-513}
\end{center}

Source: Thomson Reuters Datastream.

During this time period, thousands of Enron employees participated in the company's 401(k) retirement plan. The plan allowed employees to set aside some of their earnings in a tax-deferred account. Enron participated by matching the employees' contributions. Enron made the match by depositing required amounts in the form of Enron shares. Enron restricted the sale of its contributed shares until an employee turned 50 years old. In January 2001, the employees' 401(k) retirement accounts were valued at over US $\$ 2$ billion, of which US $\$ 1.3$ billion (or 62 percent) was in Enron shares. Although Enron restricted the sale of shares it contributed, less than US $\$ 150$ million of the total of US\$1.3 billion in shares had this restriction. The implication was that Enron employees continued to hold large amounts of Enron shares even though they were free to sell them and invest the proceeds in other assets.

A typical individual was Roger Bruce, ${ }^{3}$ a 67 -year-old Enron retiree who held all of his US 2 million in retirement funds in Enron shares. Between January 2001 and January 2002, Enron's share price fell from about US\$90 per share to zero.

3 Singletary (2001). Exhibit 2: Value of US\$1 Invested from January 1990 to January 2002 Enron vs. S\&P 500 Composite Index

$(1 / 1 / 1990=$ US\$1.00)

\begin{center}
\includegraphics[max width=\textwidth]{2023_05_04_36535b8d80b32081d422g-514}
\end{center}

Employees and retirees who had invested all or most of their retirement savings in Enron shares, just like Mr. Bruce, experienced financial ruin. The hard lesson that the Enron employees learned from this experience was to "not put all your eggs in one basket." Unfortunately, the typical Enron employee did have most of his or her eggs in one basket. Most employees' wages and financial assets were dependent on Enron's continued viability; hence, any financial distress on Enron would have a material impact on an employee's financial health. The bankruptcy of Enron resulted in the closing of its operations, the dismissal of thousands of employees, and its shares becoming worthless. Hence, the failure of Enron was disastrous to the typical Enron employee.

Enron employees were not the only ones to be victims of over-investment in a single company's shares. In the defined contribution retirement plans at Owens Corning, Northern Telecom, Corning, and ADC Telecommunications, employees all held more than 25 percent of their assets in the company's shares during a time (March 2000 to December 2001) in which the share prices in these companies fell by almost 90 percent. The good news in this story is that the employees participating in employer-matched $401(\mathrm{k})$ plans since 2001 have significantly reduced their holdings of their employers' shares.

Thus, by taking a diversified portfolio approach, investors can spread away some of the risk. Rational investors are concerned about the risk-return trade-off of their investments. The portfolio approach provides investors with a way to reduce the risk associated with their wealth without necessarily decreasing their expected rate of return.

\section{Portfolios: Reduce Risk}
In addition to avoiding a potential disaster associated with over investing in a single security, portfolios also generally offer equivalent expected returns with lower overall volatility of returns - as represented by a measure such as standard deviation. Consider

4 This expression, which most likely originated in England in the 1700 s, has a timeless sense of wisdom. this simple example: Suppose you wish to make an investment in companies listed on the Hong Kong Stock Exchange (HKSE) and you start with a sample of five companies. ${ }^{5}$ The cumulative returns for 16 fiscal quarters are shown in Exhibit 3.

Exhibit 3: Cumulative Wealth Index of Sample of Shares Listed on HKSE (initial amount= US\$1.00)

\begin{center}
\includegraphics[max width=\textwidth]{2023_05_04_36535b8d80b32081d422g-515}
\end{center}

Source: Thomson Reuters Datastream.

The individual quarterly returns for each of the five shares are shown in Exhibit 4. The annualized means and annualized standard deviations for each are also shown. 6

Exhibit 4: Quarterly Returns (in Percent) for Sample of HKSE Listed Shares over 16 Fiscal Quarters

\begin{center}
\begin{tabular}{|c|c|c|c|c|c|c|}
\hline
 & $\begin{array}{l}\text { Yue Yuen } \\ \text { Industrial }\end{array}$ & $\begin{array}{c}\text { Cathay } \\ \text { Pacific } \\ \text { Airways }\end{array}$ & $\begin{array}{l}\text { Hutchison } \\ \text { Whampoa }\end{array}$ & Li \& Fung & COSCO Pacific & $\begin{array}{c}\text { Equally } \\ \text { Weighted } \\ \text { Portfolio }\end{array}$ \\
\hline
Q1 & $-11.1 \%$ & $-2.3 \%$ & $0.6 \%$ & $-13.2 \%$ & $-1.1 \%$ & $-5.4 \%$ \\
\hline
Q2 & -0.5 & -5.4 & 10.8 & 1.7 & 21.0 & 5.5 \\
\hline
Q3 & 5.7 & 6.8 & 19.1 & 13.8 & 15.5 & 12.2 \\
\hline
$\mathrm{Q} 4$ & 5.3 & 4.6 & -2.1 & 16.9 & 12.4 & 7.4 \\
\hline
Q5 & 17.2 & 2.4 & 12.6 & 14.5 & -7.9 & 7.8 \\
\hline
Q6 & -17.6 & -10.4 & -0.9 & 4.4 & -16.7 & -8.2 \\
\hline
Q7 & 12.6 & 7.4 & 4.2 & -10.9 & 15.4 & 5.7 \\
\hline
Q8 & 7.5 & -0.4 & -3.6 & 29.2 & 21.9 & 10.9 \\
\hline
Q9 & -7.9 & 1.3 & -5.1 & -2.0 & -1.6 & -3.1 \\
\hline
$\mathrm{Q} 10$ & 8.2 & 27.5 & 0.1 & 26.0 & -10.1 & 10.3 \\
\hline
Q11 & 18.3 & 24.3 & 16.5 & 22.8 & 25.7 & 21.5 \\
\hline
O12 & 0.1 & -2.6 & -6.7 & -0.4 & 0.3 & -1.8 \\
\hline
\end{tabular}
\end{center}

5 A sample of five companies from a similar industry group was arbitrarily selected for illustration purposes 6 Mean quarterly returns are annualized by multiplying the quarterly mean by 4. Quarterly standard deviations are annualized by taking the quarterly standard deviation and multiplying it by 2.

\begin{center}
\begin{tabular}{|c|c|c|c|c|c|c|}
\hline
 & $\begin{array}{l}\text { Yue Yuen } \\ \text { Industrial }\end{array}$ & $\begin{array}{c}\text { Cathay } \\ \text { Pacific } \\ \text { Airways }\end{array}$ & $\begin{array}{l}\text { Hutchison } \\ \text { Whampoa }\end{array}$ & Li \& Fung & COSCO Pacific & $\begin{array}{c}\text { Equally } \\ \text { Weighted } \\ \text { Portfolio }\end{array}$ \\
\hline
Q13 & -6.2 & -4.2 & 16.7 & 11.9 & 11.1 & 5.8 \\
\hline
Q14 & -8.0 & 17.9 & -1.8 & 12.4 & 8.4 & 5.8 \\
\hline
Q15 & 3.5 & -20.1 & -8.5 & -20.3 & -31.5 & -15.4 \\
\hline
Q16 & 2.1 & -11.8 & -2.6 & 24.2 & -6.1 & 1.2 \\
\hline
Mean annual return & $7.3 \%$ & $8.7 \%$ & $12.3 \%$ & $32.8 \%$ & $14.2 \%$ & $15.1 \%$ \\
\hline
$\begin{array}{l}\text { Annual standard } \\ \text { deviation }\end{array}$ & $20.2 \%$ & $25.4 \%$ & $18.1 \%$ & $29.5 \%$ & $31.3 \%$ & $17.9 \%$ \\
\hline
Diversification ratio &  &  &  &  &  & $71.9 \%$ \\
\hline
\end{tabular}
\end{center}

Source: Thomson Reuters Datastream.

Suppose you want to invest in one of these five securities next year. There is a wide variety of risk-return trade-offs for the five shares selected. If you believe that the future will replicate the past, then choosing Li \& Fung would be a good choice. For the prior four years, Li \& Fung provided the best trade-off between return and risk. In other words, it provided the most return per unit of risk. However, if there is no reason to believe that the future will replicate the past, it is more likely that the risk and return on the one security selected will be more like selecting one randomly. When we randomly selected one security each quarter, we found an average annualized return of 15.1 percent and an average annualized standard deviation of 24.9 percent, which would now become your expected return and standard deviation, respectively.

Alternatively, you could invest in an equally weighted portfolio of the five shares, which means that you would invest the same dollar amount in each security for each quarter. The quarterly returns on the equally weighted portfolio are just the average of the returns of the individual shares. As reported in Exhibit 4, the equally weighted portfolio has an average return of 15.1 percent and a standard deviation of 17.9 percent. As expected, the equally weighted portfolio's return is the same as the return on the randomly selected security. However, the same does not hold true for the portfolio standard deviation. That is, the standard deviation of an equally weighted portfolio is not simply the average of the standard deviations of the individual shares. In a more advanced reading we will demonstrate in greater mathematical detail how such a portfolio offers a lower standard deviation of return than the average of its individual components due to the correlations or interactions between the individual securities.

Because the mean return is the same, a simple measure of the value of diversification is calculated as the ratio of the standard deviation of the equally weighted portfolio to the standard deviation of the randomly selected security. This ratio may be referred to as the diversification ratio. In this case, the equally weighted portfolio's standard deviation is approximately 72 percent of the average standard deviation of the 5 stocks (24.9\%). The diversification ratio of the portfolio's standard deviation to the individual asset's standard deviation measures the risk reduction benefits of a simple portfolio construction method, equal weighting. Even though the companies were chosen from a similar industry grouping, we see significant risk reduction. An even greater portfolio effect (i.e., lower diversification ratio) could have been realized if we had chosen companies from completely different industries.

This example illustrates one of the critical ideas about portfolios: Portfolios affect risk more than returns. In the prior section portfolios helped avoid the effects of downside risk associated with investing in a single company's shares. In this section we extended the notion of risk reduction through portfolios to illustrate why individuals and institutions should hold portfolios.

\textbackslash section\{PORTFOLIO PERSPECTIVE: RISK-RETURN TRADE-OFF, DOWNSIDE PROTECTION, MODERN PORTFOLIO THEORY

$\square \quad$ describe the portfolio approach to investing

In the previous section we compared an equally weighted portfolio to the selection of a single security. In this section we examine additional combinations of the same set of shares and observe the trade-offs between portfolio volatility of returns and expected return (for short, their risk-return trade-offs). If we select the portfolios with the best combination of risk and return (taking historical statistics as our expectations for the future), we produce the set of portfolios shown in Exhibit 5 .

\section{Exhibit 5: Optimal Portfolios for Sample of HKSE Listed Shares}
Expected Return

\begin{center}
\includegraphics[max width=\textwidth]{2023_05_04_36535b8d80b32081d422g-517}
\end{center}

Source: Thomson Reuters Datastream

In addition to illustrating that the diversified portfolio approach reduces risk, Exhibit 5 also shows that the composition of the portfolio matters. For example, an equally weighted portfolio (20 percent of the portfolio in each security) of the five shares has an expected return of 15.1 percent and a standard deviation of 17.9 percent. Alternatively, a portfolio with 25 percent in Yue Yuen Industrial (Holdings), 3 percent in Cathay Pacific, 52 percent in Hutchison Whampoa, 20 percent in Li \& Fung, and 0 percent in COSCO Pacific produces a portfolio with an expected return of 15.1 percent and a standard deviation of 15.6 percent. Compared to a simple equally weighted portfolio, this provides an improved trade-off between risk and return because a lower level of risk was achieved for the same level of return.

\section{Historical Portfolio Example: Not Necessarily Downside Protection}
A major reason that portfolios can effectively reduce risk is that combining securities whose returns do not move together provides diversification. Sometimes a subset of assets will go up in value at the same time that another will go down in value. The fact that these may offset each other creates the potential diversification benefit we attribute to portfolios. However, an important issue is that the co-movement or correlation pattern of the securities' returns in the portfolio can change in a manner unfavorable to the investor. We use historical return data from a set of global indexes to show the impact of changing co-movement patterns.

When we examine the returns of a set of global equity indexes over the last 15 years, we observe a reduction in the diversification benefit due to a change in the pattern of co-movements of returns. Exhibit 6 and Exhibit 7 show the cumulative returns for a set of five global indexes ${ }^{7}$ for two different time periods. Comparing the first time period, from Q4 1993 through Q3 2000 (as shown in Exhibit 6), with the last time period, from Q1 2006 through Q1 2009 (as shown in Exhibit 7), we show that the degree to which these global equity indexes moved together increased over time.

\section{Exhibit 6: Returns to Global Equity Indexes Q4 1993-Q3 2000}
\begin{center}
\includegraphics[max width=\textwidth]{2023_05_04_36535b8d80b32081d422g-518}
\end{center}

Source: Thomson Reuters Datastream.

7 The S\&P 500, Hang Seng, and Nikkei 500 are broad-based composite equity indexes designed to measure the performance of equities in the United States, Hong Kong SAR, and Japan. MSCI stands for Morgan Stanley Capital International. EAFE refers to developed markets in Europe, Australasia, and the Far East. AC indicates all countries, and EM is emerging markets. All index returns are in US dollars.

\section{Exhibit 7: Returns to Global Equity Indexes Q1 2006-Q1 2009}
\begin{center}
\includegraphics[max width=\textwidth]{2023_05_04_36535b8d80b32081d422g-519}
\end{center}

Source: Thomson Reuters Datastream

The latter part of the second time period, from Q4 2007 to Q1 2009, was a period of dramatic declines in global share prices. Exhibit 8 shows the mean annual returns and standard deviation of returns for this time period.

Exhibit 8: Returns to Global Equity Indexes

Q4 1993-Q3 2000

Q1 2006-Q1 2009

Q4 2007-Q1 2009

\begin{center}
\begin{tabular}{|c|c|c|c|c|c|c|}
\hline
Global Index & Mean & $\begin{array}{c}\text { Stand. } \\ \text { Dev. }\end{array}$ & Mean & $\begin{array}{c}\text { Stand. } \\ \text { Dev. }\end{array}$ & Mean & Stand. Dev. \\
\hline
S\&P 500 & $20.5 \%$ & $13.9 \%$ & $-6.3 \%$ & $21.1 \%$ & $-40.6 \%$ & $23.6 \%$ \\
\hline
MSCI EAFE US\$ & 10.9 & 14.2 & -3.5 & 29.4 & -48.0 & 35.9 \\
\hline
Hang Seng & 20.4 & 35.0 & 5.1 & 34.2 & -53.8 & 34.0 \\
\hline
Nikkei 500 & 3.3 & 18.0 & -13.8 & 27.6 & -48.0 & 30.0 \\
\hline
MSCI AC EAFE + EM US\$ & 7.6 & 13.2 & -4.9 & 30.9 & -52.0 & 37.5 \\
\hline
Randomly selected index & $12.6 \%$ & $18.9 \%$ & $-4.7 \%$ & $28.6 \%$ & $-48.5 \%$ & $32.2 \%$ \\
\hline
Equally weighted portfolio & $12.6 \%$ & $14.2 \%$ & $-4.7 \%$ & $27.4 \%$ & $-48.5 \%$ & $32.0 \%$ \\
\hline
Diversification ratio &  & $75.1 \%$ &  & $95.8 \%$ &  & $99.4 \%$ \\
\hline
\end{tabular}
\end{center}

Source: Thomson Reuters Datastream.

During the period Q4 2007 through Q1 2009, the average return for the equally weighted portfolio, including dividends, was -48.5 percent. Other than reducing the risk of earning the return of the worst performing market, the diversification benefits were small. Exhibit 9 shows the cumulative quarterly returns of each of the five indexes over this time period. All of the indexes declined in unison. The lesson is that although portfolio diversification generally does reduce risk, it does not necessarily provide the same level of risk reduction during times of severe market turmoil as it does when the economy and markets are operating 'normally'. In fact, if the economy or markets fail totally (which has happened numerous times around the world), then diversification is a false promise. In the face of a worldwide contagion, diversification was ineffective, as illustrated at the end of 2008.

\section{Exhibit 9: Return to Global Equity Indexes Q4 2007-Q1 2009}
Cumulative Returns (Q4 2007 = US $\$ 1.00$ )

\begin{center}
\includegraphics[max width=\textwidth]{2023_05_04_36535b8d80b32081d422g-520}
\end{center}

Source: Thomson Reuters Datastream.

\section{KNOWLEDGE CHECK}
Portfolios are most likely to provide:
A. risk reduction.
B. risk elimination.
C. downside protection.

\section{Solution:}
A is correct. Combining assets into a portfolio should reduce the portfolio's volatility. However, the portfolio approach does not necessarily provide downside protection or eliminate all risk.

\section{Portfolios: Modern Portfolio Theory}
The concept of diversification has been around for a long time and has a great deal of intuitive appeal. However, the actual theory underlying this basic concept and its application to investments only emerged in 1952 with the publication of Harry Markowitz's classic article on portfolio selection. ${ }^{8}$ The article provided the foundation for what is now known as modern portfolio theory (MPT). The main conclusion of MPT is that investors should not only hold portfolios but should also focus on how individual securities in the portfolios are related to one another. In addition to the diversification benefits of portfolios to investors, the work of William Sharpe (1964), John Lintner (1965), and Jack Treynor (1961) demonstrated the role that portfolios play in determining the appropriate individual asset risk premium (i.e., the return in excess of the risk-free return expected by investors as compensation for the asset's risk). According to capital market theory, the priced risk of an individual security is affected by holding it in a well-diversified portfolio. The early research provided the insight that an asset's risk should be measured in relation to the remaining systematic or non-diversifiable risk, which should be the only risk that affects the asset's price. This view of risk is the basis of the capital asset pricing model, or CAPM, which is discussed in greater detail in other readings. Although MPT has limitations, the concepts and intuitions illustrated in the theory continue to be the foundation of knowledge for portfolio managers.

\section{STEPS IN THE PORTFOLIO MANAGEMENT PROCESS}
describe the steps in the portfolio management process

In the previous section we discussed a portfolio approach to investing. When establishing and managing a client's investment portfolio, certain critical steps are followed in the process. We describe these steps in this section.

\begin{itemize}
  \item The Planning Step

  \item Understanding the client's needs

  \item Preparation of an investment policy statement (IPS)

  \item The Execution Step

  \item Asset allocation

  \item Security analysis

  \item Portfolio construction

  \item The Feedback Step

  \item Portfolio monitoring and rebalancing

  \item Performance measurement and reporting

\end{itemize}

\section{Step One: The Planning Step}
The first step in the investment process is to understand the client's needs (objectives and constraints) and develop an investment policy statement (IPS). A portfolio manager is unlikely to achieve appropriate results for a client without a prior understanding of the client's needs. The IPS is a written planning document that describes the client's investment objectives and the constraints that apply to the client's portfolio. The IPS may state a benchmark-such as a particular rate of return or the performance of a particular market index-that can be used in the feedback stage to assess the performance of the investments and whether objectives have been met. The IPS should be reviewed and updated regularly (for example, either every three years or when a major change in a client's objectives, constraints, or circumstances occurs).

\section{Step Two: The Execution Step}
The next step is for the portfolio manager to construct a suitable portfolio based on the IPS of the client. The portfolio execution step consists of first deciding on a target asset allocation, which determines the weighting of asset classes to be included in the portfolio. This step is followed by the analysis, selection, and purchase of individual investment securities.

\section{Asset Allocation}
The next step in the process is to assess the risk and return characteristics of the available investments. The analyst forms economic and capital market expectations that can be used to form a proposed allocation of asset classes suitable for the client. Decisions that need to be made in the asset allocation of the portfolio include the distribution between equities, fixed-income securities, and cash; sub-asset classes, such as corporate and government bonds; and geographical weightings within asset classes. Alternative assets-such as real estate, commodities, hedge funds, and private equity-may also be included.

Economists and market strategists may set the top down view on economic conditions and broad market trends. The returns on various asset classes are likely to be affected by economic conditions; for example, equities may do well when economic growth has been unexpectedly strong whereas bonds may do poorly if inflation increases. The economists and strategists will attempt to forecast these conditions.

Top down-A top-down analysis begins with consideration of macroeconomic conditions. Based on the current and forecasted economic environment, analysts evaluate markets and industries with the purpose of investing in those that are expected to perform well. Finally, specific companies within these industries are considered for investment.

Bottom up-Rather than emphasizing economic cycles or industry analysis, a bottom-up analysis focuses on company-specific circumstances, such as management quality and business prospects. It is less concerned with broad economic trends than is the case for top-down analysis, but instead focuses on company specifics.

\section{Security Analysis}
The top-down view can be combined with the bottom-up insights of security analysts who are responsible for identifying attractive investments in particular market sectors. They will use their detailed knowledge of the companies and industries they cover to assess the expected level and risk of the cash flows that each security will produce. This knowledge allows the analysts to assign a valuation to the security and identify preferred investments.

\section{Portfolio Construction}
The portfolio manager will then construct the portfolio, taking account of the target asset allocation, security analysis, and the client's requirements as set out in the IPS. A key objective will be to achieve the benefits of diversification (i.e., to avoid putting all the eggs in one basket). Decisions need to be taken on asset class weightings, sector weightings within an asset class, and the selection and weighting of individual securities or assets. The relative importance of these decisions on portfolio performance depends at least in part on the investment strategy selected; for example, consider an investor that actively adjusts asset sector weights in relation to forecasts of sector performance and one who does not. Although all decisions have an effect on portfolio performance, the asset allocation decision is commonly viewed as having the greatest impact.

Exhibit 10 shows the broad portfolio weights of the endowment funds of Yale University and the University of Virginia as of June 2017. As you can see, the portfolios have a heavy emphasis on such alternative assets as hedge funds, private equity, and real estate-Yale University particularly so.

\section{Exhibit 10: Endowment Portfolio Weights, June 2017}
\begin{center}
\begin{tabular}{lcc}
\hline
Asset Class & $\begin{array}{c}\text { Yale University } \\ \text { Endowment }\end{array}$ & $\begin{array}{c}\text { University of Virginia } \\ \text { Endowment }\end{array}$ \\
\hline
Public equity & $19.1 \%$ & $26.7 \%$ \\
Fixed income & 4.6 & 9.1 \\
Private equity & 14.2 & 15.7 \\
Real assets (e.g., real estate) & 18.7 & 12.1 \\
Absolute return (e.g., hedge funds) & 25.1 & 19.6 \\
Cash & 1.2 & 2.3 \\
Other & 17.2 & 14.5 \\
Portfolio value & US $\$ 27.2 \mathrm{bn}$ & US $\$ 8.6 \mathrm{bn}$ \\
\hline
\end{tabular}
\end{center}

Sources: "2017 Yale Endowment Annual Report" (p. 2): \href{http://www.yale.edu/investments/Yale_Endowment_17}{www.yale.edu/investments/Yale\_Endowment\_17}. pdf; "University of Virginia Investment Management Company Annual Report 2017" (p. 26): http:// \href{http://uvm-web.eservices.virginia.edu/public/reports/FinancialStatements_2017.pdf}{uvm-web.eservices.virginia.edu/public/reports/FinancialStatements\_2017.pdf}.

Risk management is an important part of the portfolio construction process. The client's risk tolerance will be set out in the IPS, and the portfolio manager must make sure the portfolio is consistent with it. As noted above, the manager will take a diversified portfolio perspective: What is important is not the risk of any single investment, but rather how all the investments perform as a portfolio.

The endowments shown above are relatively risk tolerant investors. Contrast the asset allocation of the endowment funds with the portfolio mix of the insurance companies shown in Exhibit 11. You will notice that the majority of the insurance assets are invested in fixed-income investments, typically of high quality. Note that the Yale University portfolio has less than 5 percent invested in fixed income, with the remainder invested in such growth assets as equity, real estate, and hedge funds. This allocation is in sharp contrast to the Massachusetts Mutual Life Insurance Company (MassMutual) portfolio, which is 80 percent invested in bonds, mortgages, loans, and cash-reflecting the differing risk tolerance and constraints (life insurers face regulatory constraints on their investments).

\section{Exhibit 11: MassMutual Portfolio, December $2017^{9}$}
Asset Classes Portfolio \%

Bonds $56 \%$

9 Asset class definitions: Bonds-Debt instruments of corporations and governments as well as various types of mortgage- and asset-backed securities; Preferred and Common Shares-Investments in preferred and common equities; Mortgages-Mortgage loans secured by various types of commercial property as well as residential mortgage whole loan pools; Real Estate-Investments in real estate; Policy Loans-Loans

\begin{center}
\begin{tabular}{lc}
\hline
Asset Classes & Portfolio \% \\
\hline
Preferred and common shares & 9 \\
Mortgages & 14 \\
Real estate & 1 \\
Policy loans & 8 \\
Partnerships & 5 \\
Other assets & 5 \\
Cash & 2 \\
\hline
\end{tabular}
\end{center}

Source: "MassMutual Financial Group 2017 Annual Report" (p. 8): \href{http://www.massmutual.com/mmfg/docs/}{www.massmutual.com/mmfg/docs/} annual\_report/index.html.

The portfolio construction phase also involves trading. Once the portfolio manager has decided which securities to buy and in what amounts, the securities must be purchased. In many investment firms, the portfolio manager will pass the trades to a buy-side trader-a colleague who specializes in securities trading-who will contact a stockbroker or dealer to have the trades executed.

\section{Step Three: The Feedback Step}
Finally, the feedback step assists the portfolio manager in rebalancing the portfolio due to a change in, for example, market conditions or the circumstances of the client.

\section{Portfolio Monitoring and Rebalancing}
Once the portfolio has been constructed, it needs to be monitored and reviewed and the composition revised as the security analysis changes because of changes in security prices and changes in fundamental factors. When security and asset weightings have drifted from the intended levels as a result of market movements, some rebalancing may be required. The portfolio may also need to be revised if it becomes apparent that the client's needs or circumstances have changed.

\section{Performance Evaluation and Reporting}
Finally, the performance of the portfolio must be evaluated, which will include assessing whether the client's objectives have been met. For example, the investor will wish to know whether the return requirement has been achieved and how the portfolio has performed relative to any benchmark that has been set. Analysis of performance may suggest that the client's objectives need to be reviewed and perhaps changes made to the IPS. As we will discuss in the next section, there are numerous investment products that clients can use to meet their investment needs. Many of these products are diversified portfolios that an investor can purchase.

by policyholders that are secured by insurance and annuity contracts; Partnerships-Investments in partnerships and limited liability companies; Cash-Cash, short-term investments, receivables for securities, and derivatives. Cash equivalents have short maturities (less than one year) or are highly liquid and able to be readily sold.

\section{TYPES OF INVESTORS}
describe types of investors and distinctive characteristics and needs of each

describe defined contribution and defined benefit pension plans

The portfolio management process described in the previous section may apply to different types of investment clients. Such clients are broadly divided among individual (or retail) and institutional investors. Each of these segments has distinctive characteristics and needs, as discussed in the following sub-sections.

\section{Individual Investors}
Individual investors have a variety of motives for investing and constructing portfolios. Short-term goals can include providing for children's education, saving for a major purchase (such as a vehicle or a house), or starting a business. The retirement goalinvesting to provide for an income in retirement-is a major part of the investment planning of most individuals. Many employees of public and private companies invest for retirement through defined contribution pension plans (DC plans). DC plans are retirement plans in the employee's name usually funded by both the employee and the employer. Examples include 401(k) plans in the United States, group personal pension schemes in the United Kingdom, and superannuation plans in Australia. With DC plans, individuals will invest part of their wages while working, expecting to draw on the accumulated funds to provide income during retirement or to transfer some of their wealth to their heirs. The key to a DC plan is that the employee accepts the investment and inflation risk and is responsible for ensuring that there are enough assets in the plan to meet their needs upon retirement.

Some individuals will be investing for growth and will therefore seek assets that have the potential for capital gains. Others, such as retirees, may need to draw an income from their assets and may therefore choose to invest more in fixed-income and dividend-paying shares. The investment needs of individuals will depend in part on their broader financial circumstances, such as their employment prospects and whether or not they own their own residence. They may also need to consider such issues as building up a cash reserve and the purchase of appropriate insurance policies before undertaking longer-term investments.

Asset managers serving individual investors typically distribute their products directly to investors or through intermediaries such as financial advisers and/or retirement plan providers. The distribution network for individual investors varies globally. In the United States, financial advisers are independent or employed by national or regional broker-dealers, banks, and trust companies. Additionally, many asset managers distribute investment strategies to investors through major online brokerage and custodial firms.

In Europe, retail investment product distribution is fragmented and, in turn, varies by country/region. In continental Europe, for example, distribution is primarily driven through financial advisers affiliated with retail and private banks. In the United Kingdom, products are sold through independent advisers as well as through advisers representing a bank or insurance group. Retail distribution in Switzerland and in the Nordic countries is driven mainly through large regional and private banks. In contrast to the United States and Europe, in many Asian markets retail distribution is dominated by large regional retail banks and global banks with private banking divisions. Globally, many wealth management firms and asset managers target high-net-worth investors. These clients often require more customized investment solutions alongside tax and estate planning services.

\section{Institutional Investors}
Institutional investors primarily include defined benefit pension plans, endowments and foundations, banks, insurance companies, investment companies, and sovereign wealth funds. Each of these has unique goals, asset allocation preferences, and investment strategy needs.

\section{Defined Benefit Pension Plans}
Pension plans are typically categorized as either defined contribution (DC) or defined benefit (DB). We previously described DC plans, which relate to individual investors. Defined benefit pension plans (DB plans) are company-sponsored plans that offer employees a predefined benefit on retirement. The future benefit is defined because the DB plan requires the plan sponsor to specify the obligation stated in terms of the retirement income benefits owed to participants. Generally, employers are responsible for the contributions made to a DB plan and bear the risk associated with adequately funding the benefits offered to employees. Plans are committed to paying pensions to members, and the assets of these plans are there to fund those payments. Plan managers need to ensure that sufficient assets will be available to pay pension benefits as they come due. The plan may have an indefinitely long time horizon if new plan members are being admitted or a finite time horizon if the plan has been closed to new members. In some cases, the plan managers attempt to match the fund's assets to its liabilities by, for example, investing in bonds that will produce cash flows corresponding to expected future pension payments. There may be many different investment philosophies for pension plans, depending on funded status and other variables.

An ongoing trend is that plan sponsors increasingly favor DC plans over DB plans because DC plans typically have lower costs/risk to the company. As a result, DB plans have been losing market share of pension assets to DC plans. Nevertheless, DB plans, both public and private, remain sizable sources of investment funds for asset managers. As Exhibit 12 shows, global pension assets totaled more than US $\$ 41$ trillion by the end of 2017. The United States, United Kingdom, and Japan represent the three largest pension markets in the world, comprising more than $76 \%$ of global pension assets.

\section{Exhibit 12: Global Pension Assets (as of year-end 2017)}
\begin{center}
\begin{tabular}{lc}
\hline
Country/Region & Total Assets (US\$ billions) \\
\hline
United States & 25,411 \\
United Kingdom & 3,111 \\
Japan & 3,054 \\
Australia & 1,924 \\
Canada & 1,769 \\
Netherlands & 1,598 \\
Switzerland & 906 \\
South Korea & 725 \\
Germany & 472 \\
Brazil & 269 \\
South Africa & 258 \\
\end{tabular}
\end{center}

\begin{center}
\begin{tabular}{lc}
\hline
Country/Region & Total Assets (US\$ billions) \\
\hline
Finland & 233 \\
Malaysia & 227 \\
Chile & 205 \\
Mexico & 177 \\
Italy & 184 \\
France & 167 \\
Chinese mainland & 177 \\
Hong Kong SAR & 164 \\
Ireland & 157 \\
India & 120 \\
Spain & 44 \\
Total & $\mathbf{4 1 , 3 5 5}$ \\
\hline
\end{tabular}
\end{center}

Note: Column does not sum precisely because of rounding.

Source: Willis Towers Watson.

By geography, the United States and Australia have a higher proportion of pension assets in DC plans, whereas Canada, Japan, the Netherlands, and the United Kingdom remain weighted toward DB plans (see Exhibit 13).

\section{Exhibit 13: Pension Plan Type by Geography}
\begin{center}
\includegraphics[max width=\textwidth]{2023_05_04_36535b8d80b32081d422g-527}
\end{center}

$\square \mathrm{DB} \square \mathrm{DC}$

Notes: "P7" represents the combination of the seven countries listed. No data were available for Switzerland for this study.

Sources: Willis Towers Watson and secondary sources.

\section{Endowments and Foundations}
Endowments are funds of non-profit institutions that help the institutions provide designated services. In contrast, foundations are grant-making entities. Endowments and foundations collectively represent an estimated US\$1.6 trillion in assets in the United States, which is the primary market for endowments and foundations. Endowments and foundations typically allocate a sizable portion of their assets in alternative investments (Exhibit 14). This large allocation to alternative investments primarily reflects the typically long time horizon of endowments and foundations, as well the popularity of endowment-specific asset allocation models developed by Yale University's endowment managers David Swensen and Dean Takahashi.

\section{Exhibit 14: Asset Allocations for US College and University Endowments and Affiliated Foundations (as of 30 June 2017, dollar weighted)}
\begin{center}
\begin{tabular}{lc}
\hline
Asset Class & Percentage Allocation \\
\hline
Domestic equity & 15 \\
Fixed income & 7 \\
Foreign equity & 20 \\
Alternatives & 54 \\
Cash & 4 \\
\hline
\end{tabular}
\end{center}

Source: National Association of College and University Budget Officers and Commonfund Institute.

A typical investment objective of an endowment or a foundation is to maintain the real (inflation-adjusted) capital value of the fund while generating income to fund the objectives of the institution. Most foundations and endowments are established with the intent of having perpetual lives. Exhibit 15 describes the Yale University endowment's approach to balancing short-term spending needs with ensuring that future generations also benefit from the endowment, and it also shows the Wellcome Trust's approach. The investment approach undertaken considers the objectives and constraints of the institution (for example, no tobacco investments for a medical endowment).

\section{Exhibit 15: Spending Rules}
The following examples of spending rules are excerpts from the Yale University endowment (in the United States) and from the Wellcome Trust (in the United Kingdom).

\section{Yale University Endowment}
The spending rule is at the heart of fiscal discipline for an endowed institution. Spending policies define an institution's compromise between the conflicting goals of providing substantial support for current operations and preserving purchasing power of Endowment assets. The spending rule must be clearly defined and consistently applied for the concept of budget balance to have meaning.

The Endowment spending policy, which allocates Endowment earnings to operations, balances the competing objectives of providing a stable flow of income to the operating budget and protecting the real value of the Endowment over time. The spending policy manages the trade-of between these two objectives by combining a long-term spending rate target with a smoothing rule, which adjusts spending in any given year gradually in response to changes in Endowment market value. The target spending rate approved by the Yale Corporation currently stands at $5.25 \%$. According to the smoothing rule, Endowment spending in a given year sums to $80 \%$ of the previous year's spending and $20 \%$ of the targeted long-term spending rate applied to the fiscal year-end market value two years prior.

Source: 2017 Yale Endowment Annual Report (p.18) [\href{http://investments.yale}{http://investments.yale}. edu/endowment-update/]

\section{Wellcome Trust}
Our overall investment objective is to generate $4.5 \%$ percent real return over the long term.

This is to provide for real increases in annual expenditure while reserving the Trust's capital base to balance the needs of current and future beneficiaries.

We use this absolute return strategy because it aligns asset allocation with funding requirements and provides a competitive framework in which to judge individual investments.

Source: Wellcome Trust website (\href{https://wellcome.ac.uk/about-us/investments}{https://wellcome.ac.uk/about-us/investments})

\section{Banks}
Banks are financial intermediaries that accept deposits and lend money. Banks often have excess reserves that are invested in relatively conservative and very short-duration fixed-income investments, with a goal of earning an excess return above interest obligations due to depositors. Liquidity is a paramount concern for banks that stand ready to meet depositor requests for withdrawals. Many large banks have asset management divisions that offer retail and institutional products to their clients.

\section{Insurance Companies}
Insurance companies receive premiums for the policies they write, and they need to invest these premiums in a manner that will allow them to pay claims.

Insurance companies can be segmented into two broad types: life insurers and property and casualty (P\&C) insurers. Insurance premiums from policyholders comprise an insurance company's general account. To pay claims to policyholders, regulatory guidelines maintain that an insurance company's general account is typically invested conservatively in a diverse allocation of fixed-income securities. General account portfolio allocations differ among life, P\&C, and other specialty insurers (e.g., reinsurance) because of both the varying duration of liabilities and the unique liquidity considerations across insurance type. ${ }^{10}$ In contrast to the general account, an insurer's surplus account is the difference between its assets and liabilities. An insurer's surplus account typically targets a higher return than the general account and thus often invests in less-conservative asset classes, such as public and private equities, real estate, infrastructure, and hedge funds.

Many insurance companies have in-house portfolio management teams responsible for managing general account assets. Some insurance companies offer portfolio management services and products in addition to their insurance offerings. An increasing trend among insurers (particularly in the United States) is outsourcing some of the

10 For example, life insurers tend to invest in longer-term assets (e.g., 30-year government and corporate bonds) relative to $\mathrm{P} \& \mathrm{C}$ insurers because of the longer-term nature of their liabilities. portfolio management responsibilities-primarily sophisticated alternative asset classes-to unaffiliated asset managers. Several insurers manage investments for third-party clients, often through separately branded subsidiaries.

\section{Sovereign Wealth Funds}
Sovereign wealth funds (SWFs) are state-owned investment funds or entities that invest in financial or real assets. SWFs do not typically manage specific liability obligations, such as pensions, and have varying investment horizons and objectives based on funding the government's goals (for example, budget stabilization or future development projects). SWF assets more than doubled from 2007 to March 2018, totaling more than US $\$ 7.6$ trillion. ${ }^{11}$ Exhibit 16 lists the 10 largest SWFs in the world. The largest SWFs tend to be concentrated in Asia and in natural resource-rich places.

Exhibit 16: Largest Sovereign Wealth Funds (as of August 2018, in US\$ billions)

\begin{center}
\begin{tabular}{lcc}
\hline
Place & Sovereign Wealth Fund (Inception Year) & Assets \\
\hline
Norway & Government Pension Fund-Global (1990) & 1,058 \\
Chinese Mainland & China Investment Corporation (2007) & 941 \\
UAE - Abu Dhabi & Abu Dhabi Investment Authority (1976) & 683 \\
Kuwait & Kuwait Investment Authority (1953) & 592 \\
Hong Kong SAR & Hong Kong Monetary Authority Investment & 523 \\
Saudi Arabia & Portfolio (1993) &  \\
Chinese Mainland & SAMA Foreign Holdings (1952) & 516 \\
Singapore & SAFE Investment Company (1997) & 441 \\
Singapore & Government of Singapore Investment & 390 \\
Saudi Arabia & Authority (1981) &  \\
Total SWF Assets under & Temasek Holdings (1974) & 375 \\
Management & Public Investment Fund (2008) & 360 \\
\hline
\end{tabular}
\end{center}

Source: SWF Institute (\href{http://www.swfinstitute.org}{www.swfinstitute.org}).

Investment needs vary across client groups. With some groups of clients, generalizations are possible. In other groups, needs vary by client. Exhibit 17 summarizes needs within each group.

\section{Exhibit 17: Summary of Investment Needs by Client Type}
\begin{center}
\begin{tabular}{|c|c|c|c|c|}
\hline
Client & Time Horizon & Risk Tolerance & Income Needs & Liquidity Needs \\
\hline
Individual investors & Varies by individual & Varies by individual & Varies by individual & Varies by individual \\
\hline
$\begin{array}{l}\text { Defined benefit pension } \\ \text { plans }\end{array}$ & Typically long term & Typically quite high & $\begin{array}{l}\text { High for mature funds; } \\ \text { low for growing funds }\end{array}$ & $\begin{array}{l}\text { Varies by maturity o } \\ \text { the plan }\end{array}$ \\
\hline
$\begin{array}{l}\text { Endowments and } \\ \text { foundations }\end{array}$ & Very long term & Typically high & $\begin{array}{l}\text { To meet spending } \\ \text { commitments }\end{array}$ & Typically quite low \\
\hline
\end{tabular}
\end{center}

11 SWFI, "Sovereign Wealth Fund Rankings" (\href{https://www.swfinstitute.org/sovereign-wealth-fund-rankings/;}{https://www.swfinstitute.org/sovereign-wealth-fund-rankings/;} retrieved October 2018).

\begin{center}
\begin{tabular}{|c|c|c|c|c|}
\hline
Client & Time Horizon & Risk Tolerance & Income Needs & Liquidity Needs \\
\hline
Banks & Short term & Quite low & $\begin{array}{l}\text { To pay interest on } \\ \text { deposits and opera- } \\ \text { tional expenses }\end{array}$ & $\begin{array}{l}\text { High to meet repayment } \\ \text { of deposits }\end{array}$ \\
\hline
Insurance companies & $\begin{array}{l}\text { Short term for property } \\ \text { and casualty; long } \\ \text { term for life insurance } \\ \text { companies }\end{array}$ & Typically quite low & Typically low & High to meet claims \\
\hline
Investment companies & Varies by fund & Varies by fund & Varies by fund & $\begin{array}{l}\text { High to meet } \\ \text { redemptions }\end{array}$ \\
\hline
Sovereign wealth funds & Varies by fund & Varies by fund & Varies by fund & Varies by fund \\
\hline
\end{tabular}
\end{center}

\section{THE ASSET MANAGEMENT INDUSTRY}
describe aspects of the asset management industry

The portfolio management process and investor types are broad components of the asset management industry, which is an integral component of the global financial services sector. At the end of 2017, the industry managed more than US $\$ 79$ trillion of assets owned by a broad range of institutional and individual investors (Exhibit 18). ${ }^{12}$ Although nearly $80 \%$ of the world's professionally managed assets are in North America and Europe, the fastest-growing markets are in Asia and Latin America.

Exhibit 18: Global Assets under Management (AUM) by Region (year-end 2017)

Market Size

(US\$ trillions) Market Share (\%)

\begin{center}
\begin{tabular}{lcc}
\hline
North America & 37.4 & $47 \%$ \\
Europe & 22.2 & 28 \\
Japan and Australia & 6.2 & 8 \\
Chinese mainland & 4.2 & 5 \\
Asia (excluding Japan, Australia, and Chinese & 3.5 & 2 \\
mainland) &  & 2 \\
Latin America & 1.8 & 1.4 \\
Middle East and Africa & 79.2 & $100 \%$ \\
Total Global AUM &  &  \\
\hline
\end{tabular}
\end{center}

Notes: Total Global AUM in this exhibit represents assets professionally managed in exchange for a fee. The total of US $\$ 79.2$ trillion includes certain offshore assets that are not represented in the specific regional categories above.

Source: Boston Consulting Group.

12 \href{http://image-src.bcg.com/Images/BCG-Seizing-the-Analytics-Advantage-June-2018-R-3_tcm9-194512}{http://image-src.bcg.com/Images/BCG-Seizing-the-Analytics-Advantage-June-2018-R-3\_tcm9-194512}. pdf (accessed on 6 September 2018). The asset management industry is highly competitive. The universe of firms in the industry is broad, ranging from "pure-play" independent asset managers to diversified commercial banks, insurance companies, and brokerages that offer asset management services in addition to their core business activities. Given the increasingly global nature of the industry, many asset managers have investment research and distribution offices around the world. An asset manager is commonly referred to as a buy-side firm given that it uses (buys) the services of sell-side firms. A sell-side firm is a broker/dealer that sells securities and provides independent investment research and recommendations to their clients (i.e., buy-side firms).

Asset managers offer a broad range of strategies. Specialist asset managers may focus on a specific asset class (e.g., emerging market equities) or style (e.g., quantitative investing), while "full service" managers typically offer a wide variety of asset classes and styles. Another type of asset manager firm is a "multi-boutique," in which a holding company owns several asset management firms that typically have specialized investment strategies. The multi-boutique structure allows individual asset management firms to retain their own unique investment cultures-and often equity ownership stakes-while also benefiting from the centralized, shared services of the holding company (e.g., technology, sales and marketing, operations, and legal services).

\section{Active versus Passive Management}
Asset managers may offer either active or passive management. As of year-end 2017, active management considerably exceeded passive management in terms of global assets under management and industry revenue (Exhibit 19), although passive management has demonstrated significant growth.

Exhibit 19: Global Asset Management Industry Assets and Revenue (as of year-end 2017)

\begin{center}
\begin{tabular}{lcccc}
\hline
 & $\begin{array}{c}\text { Assets } \\ \text { (US\$ } \\ \text { trillions) }\end{array}$ & $\begin{array}{c}\text { Revenue } \\ \text { (US\$ billions) }\end{array}$ & $\begin{array}{c}\text { Market Share } \\ \text { by Assets } \\ (\%)\end{array}$ & $\begin{array}{c}\text { Market Share } \\ \text { by Revenue } \\ (\%)\end{array}$ \\
\hline
Category & $\mathbf{6 4}$ & $\mathbf{2 5 8}$ & $\mathbf{8 0 \%}$ & $\mathbf{9 4 \%}$ \\
Actively Managed & 12 & 117 & 15 & 43 \\
Alternatives & 15 & 55 & 19 & 20 \\
Active Specialties & 11 & 27 & 14 & 10 \\
Multi Asset Class & 26 & 59 & 33 & 21 \\
Core & $\mathbf{1 6}$ & $\mathbf{1 7}$ & $\mathbf{2 0 \%}$ & $\mathbf{6 \%}$ \\
Passively Managed & $\mathbf{8 0}$ & $\mathbf{2 7 5}$ & $\mathbf{1 0 0 \%}$ & $\mathbf{1 0 0 \%}$ \\
\hline
Total &  &  &  &  \\
\hline
\end{tabular}
\end{center}

Note: Some columns may not sum precisely because of rounding.

Source: Boston Consulting Group.

Through fundamental research, quantitative research, or a combination of both, active asset managers generally attempt to outperform either predetermined performance benchmarks, such as the S\&P 500, or, for multi-asset class portfolios, a combination of benchmarks. In contrast to active managers, passive managers attempt to replicate the returns of a market index. Despite the rise of passive management in asset share, its share of industry revenue remains small given the low management fees relative to active management. As Exhibit 19 illustrates, passive management represents a fifth of global assets under management but only $6 \%$ of industry revenue. Asset managers are increasingly offering other strategies beyond traditional market-cap-weighted exposures. Some of these other strategies, commonly known as smart beta, are based on such factors as size, value, momentum, or dividend characteristics. Smart beta involves the use of simple, transparent, rules-based strategies as a basis for investment decisions. Typically, smart beta strategies feature somewhat higher management fees and higher portfolio turnover relative to passive market-cap weighted strategies.

\section{Traditional versus Alternative Asset Managers}
Asset managers are typically categorized as either "traditional" or "alternative." Traditional managers generally focus on long-only equity, fixed-income, and multi-asset investment strategies, generating most of their revenues from asset-based management fees. Alternative asset managers, however, focus on hedge fund, private equity, and venture capital strategies, among others, while generating revenue from both management and performance fees (or "carried interest"). As Exhibit 19 demonstrates, alternative asset managers have a relatively low proportion of total global assets under management but generate a disproportionately high total of industry revenue.

Increasingly, the line between traditional and alternative managers has blurred. Many traditional managers have introduced higher-margin alternative products to clients. Concurrently, alternative managers seeking to reduce the revenue volatility associated with performance fees have increasingly offered retail versions of their institutional alternative strategies (typically referred to as "liquid alternatives") as well as long-only investment strategies. These liquid alternatives are often offered through highly regulated pooled investment products (e.g., mutual funds) and typically feature less leverage, no performance fees, and more liquid holdings than typical alternative products.

\section{Ownership Structure}
The ownership structure of an asset manager can play an important role in retaining and incentivizing key personnel. Portfolio managers who have personal capital invested in their firms or investment strategies are often viewed favorably by potential investors because of perceived alignment of management and client interests.

The majority of asset management firms are privately owned, typically by individuals who either established their firms or play key roles in their firms' management. Privately owned firms are typically structured as limited liability companies or limited partnerships.

While less common than privately owned managers, publicly traded asset managers have substantial assets under management. A prevalent ownership form in the industry is represented by asset management divisions of large, diversified financial services companies that offer asset management alongside insurance and banking services.

\section{Asset Management Industry Trends}
The asset management industry is evolving and continues to be shaped by socio-economic trends, shifting investor demands, advances in technology, and the expansion of global capital markets. Three key trends that we discuss in this section include the growth of passive investing, "big data" in the investment process, and the emergence of robo-advisers in the wealth management industry.

\section{Growth of Passive Investing}
As we saw in Exhibit 19, passively managed assets comprised nearly a fifth of global assets under management at the end of 2017. Management of passive assets is concentrated among a reasonably small group of asset managers and tends to be concentrated in equity strategies. As shown in Exhibit 20, the top three managers account for $70 \%$ of industry's assets. One key catalyst supporting the growth of passive investing is low cost for investors-management fees for index (or other passive) funds are often a fraction of those for active strategies. Another catalyst is the challenge that many active asset managers face in generating ex ante alpha, especially in somewhat more-efficiently priced markets, such as large-cap US equities.

Exhibit 20: Top Five ETP Managers Globally (as of 30 July 2017)

\begin{center}
\begin{tabular}{lcc}
\hline
ETP Provider & Assets (US\$ billions) & Market Share (\%) \\
\hline
iShares & 1,583 & 37 \\
Vanguard & 803 & 19 \\
State Street Global Advisors & 596 & 14 \\
PowerShares & 132 & 3 \\
Nomura & 100 & 2 \\
\hline
\end{tabular}
\end{center}

Source: ETFGI.

\section{Use of "Big Data" in the Investment Process}
The prevalence of new data is extraordinary: In 2013, IBM estimated that $90 \%$ of the world's entire universe of data was created in the previous two years. The digitization of data and an exponential increase in computing power and data storage capacity have expanded additional information sources for asset managers. Massive amounts of data containing information of potential value to investors are created and captured daily. These data include both structured data-such as order book data and security returns-and data lacking recognizable structure, which is generated by a vast number of activities on the internet and elsewhere (e.g., compiled search information). The term "big data" is used to refer to these massively large datasets and their analysis.

Asset managers are using advanced statistical and machine-learning techniques to help process and analyze these new sources of data. Such techniques are used in both fundamentally driven and quantitatively driven investment processes. For example, computers are used to "read" earnings and economic data releases much faster than humans can and react with short-term trading strategies.

Third-party research vendors are supplying a vast range of relevant new data for asset managers, such as data used for time-series and predictive models. Among the most popular new sources of data are social media data and imagery and sensor data.

\begin{itemize}
  \item Social media data. Real-time media and content outlets, such as Twitter and Facebook, provide meaningful market and company-specific announcements for investors and asset managers. In addition, the aggregation and analysis of social media users can aid key market sentiment indicators (e.g., short-term directional market movements) and indicate potential specific user trends related to products and services.

  \item Imagery and sensor data. Satellite imagery and geolocation devices provide vast real-time data to investment professionals. As the cost of launching and maintaining satellites has decreased, more satellites have been launched to track sensors and imagery that are relevant to economic considerations (e.g., weather conditions, cargo ship traffic patterns) and company-specific considerations (e.g., retailer parking capacity/usage, tracking of retail customers).

\end{itemize}

The challenge for asset managers is to discover data with predictive potential and to do so faster than fellow market participants. Many market participants are participating in an "information arms race" that has required substantial investments in specialized human capital (e.g., programmers, data scientists), technology, and information technology infrastructure to effectively convert various forms of structured and unstructured data into alpha-generating portfolio and security-level decisions.

\section{Robo-Advisers: An Expanding Wealth Management Channel}
Robo-advisers represent technology solutions that use automation and investment algorithms to provide several wealth management services-notably, investment planning, asset allocation, tax loss harvesting, and investment strategy selection. Investment and advice services provided by robo-advisers typically reflect an investor's general investment goals and risk tolerance preferences (often obtained from an investor questionnaire). Robo-adviser platforms range from exclusively digital investment advice platforms to hybrid offerings that offer both digital investment advice and the services of a human financial adviser.

At the end of 2017, robo-advisers managed an estimated US $\$ 180$ billion in assets, ${ }^{13}$ and market participants expect that number to grow considerably over time. This expected rapid growth in robo-advisory assets is based on several industry trends:

\begin{itemize}
  \item Growing demand from "mass affluent" and younger investors:
\end{itemize}

Traditional investment advice has often underserved younger and "mass affluent" investors with lower relative levels of investable assets. Given the efficiencies of robo-advisers and the scalability of technology, customized but standardized investment advice now can be offered to a wide range and size of investors.

\begin{itemize}
  \item Lower fees: The cost of digital investment advice provided by robo-advisers is often a fraction of traditional investment advice channels because of scalability. For example, in the United States, a typical financial adviser may charge a $1 \%$ annual advisory fee ${ }^{14}$ based on a client's assets, while robo-adviser fees typically average $0.20 \%$ annually. ${ }^{15}$ Additionally, robo-advisers often rely on lower fee underlying portfolio investment options, such as index funds or ETFs, when constructing portfolios for clients.

  \item New entrants: Reflecting low barriers to entry, large wealth management firms have introduced robo-adviser solutions to service certain customer segments and appeal to a new generation of investors. In addition to these large wealth managers, other less-traditional entrants, such as insurance companies and asset managers, are developing solutions to cross-sell into their existing clients. Many market observers expect that non-financial firms (large technology leaders) will also become key players in the robo-adviser industry as they look to monetize their access to user data.

\end{itemize}

13 S\&P Global Market Intelligence.

14 \href{http://www.riainabox.com/blog/2016-ria-industry-study-average-investment-advisory-fee-is-0-99-percent}{http://www.riainabox.com/blog/2016-ria-industry-study-average-investment-advisory-fee-is-0-99-percent}.

15 Deloitte, "Robo-Advisors Capitalizing on a Growing Opportunity" (\href{https://www2.deloitte.com/content/}{https://www2.deloitte.com/content/} dam/Deloitte/us/Documents/strategy/us-cons-robo-advisors.pdf).

\section{POOLED INTEREST - MUTUAL FUNDS}
$$
\begin{aligned}
& \text { describe mutual funds and compare them with other pooled } \\
& \text { investment products }
\end{aligned}
$$

In the asset management industry, a challenge faced by all investors is to and find the right set of investment products to meet their needs. There is a diverse set of investment products available to investors, ranging from a simple brokerage account in which the individual creates her own portfolio by assembling individual securities, to large institutions that employ individual portfolio managers to meet clients' investment management needs. Among the major investment products offered by asset managers are mutual funds and other pooled investment products, such as separately managed accounts, exchange-traded funds, hedge funds, and private equity/venture capital funds.

\section{Mutual Funds}
Rather than assemble a portfolio on their own, individual investors and institutions can turn over the selection and management of their investment portfolio to a third party. One way of doing this is through a mutual fund. This type of fund is a comingled investment pool in which investors in the fund each have a pro-rata claim on the income and value of the fund. The value of a mutual fund is referred to as the "net asset value." It is computed daily based on the closing price of the securities in the portfolio.

Mutual funds represent a primary investment product of individual investors globally. According to the International Investment Funds Association, worldwide regulated open-end fund assets totaled US $\$ 50$ trillion as of the first quarter of 2018. Exhibit 21 shows the growth of global open-end funds over the past five years by region. Mutual funds provide several advantages, including low investment minimums, diversified portfolios, daily liquidity, and standardized performance and tax reporting.

\section{Exhibit 21: Worldwide Regulated Open-End Funds: Total Net Assets (as of year-end, in US\$ trillions)}
\begin{center}
\begin{tabular}{lccccccc}
\hline
 & $\mathbf{2 0 1 1}$ & $\mathbf{2 0 1 2}$ & $\mathbf{2 0 1 3}$ & $\mathbf{2 0 1 4}$ & $\mathbf{2 0 1 5}$ & $\mathbf{2 0 1 6}$ & $\mathbf{Q 1 ~ 2 0 1 8}$ \\
\hline
World & $\mathbf{2 7 . 9}$ & $\mathbf{3 1 . 9}$ & $\mathbf{3 6 . 3}$ & $\mathbf{3 8 . 0}$ & $\mathbf{3 8 . 2}$ & $\mathbf{4 0 . 4}$ & $\mathbf{5 0 . 0}$ \\
Americas & 14.6 & 16.5 & 18.9 & 20.0 & 19.6 & 21.1 & 24.9 \\
Europe & 10.3 & 11.9 & 13.6 & 13.8 & 13.7 & 14.1 & 18.1 \\
Asia and Pacific & 2.9 & 3.3 & 3.7 & 4.1 & 4.7 & 5.0 & 6.8 \\
Africa & 0.1 & 0.1 & 0.1 & 0.1 & 0.1 & 0.1 & 0.2 \\
\hline
\end{tabular}
\end{center}

Notes: Components may not add to the total because of rounding. Regulated open-end funds include mutual funds, exchange-traded funds (ETFs), and institutional funds.

Source: International Investment Funds Association (IIFA).

Mutual funds are one of the most important investment vehicles for individuals and institutions. The best way to understand how a mutual fund works is to consider a simple example. Suppose that an investment firm wishes to start a mutual fund with a target amount of US $\$ 10$ million. It is able to reach this goal through investments from five individuals and two institutions. The investment of each is as follows:

\begin{center}
\begin{tabular}{|c|c|c|c|}
\hline
Investor & $\begin{array}{c}\text { Amount } \\ \text { Invested (US\$) }\end{array}$ & Percent of Total & Number of Shares \\
\hline
\multicolumn{4}{|c|}{Individuals} \\
\hline
A & $\$ 1.0$ million & $10 \%$ & 10,000 \\
\hline
B & 1.0 & 10 & 10,000 \\
\hline
C & 0.5 & 5 & 5,000 \\
\hline
D & 2.0 & 20 & 20,000 \\
\hline
E & 0.5 & 5 & 5,000 \\
\hline
\multicolumn{4}{|c|}{Institutions} \\
\hline
$\mathrm{X}$ & 2.0 & 20 & 20,000 \\
\hline
Y & 3.0 & 30 & 30,000 \\
\hline
Totals & \$10.0 million & $100 \%$ & 100,000 \\
\hline
\end{tabular}
\end{center}

Based on the US $\$ 10$ million value (net asset value), the investment firm sets a total of 100,000 shares at an initial value of US\$100 per share (US $\$ 10$ million/100,000 = US\$100). The investment firm will appoint a portfolio manager to be responsible for the investment of the US\$10 million. Going forward, the total value of the fund or net asset value will depend on the value of the assets in the portfolio.

The fund can be set up as an open-end fund or a closed-end fund. If it is an open-end fund, it will accept new investment money and issue additional shares at a value equal to the net asset value of the fund at the time of investment. For example, assume that at a later date the net asset value of the fund increases to US\$12 million and the new net asset value per share is US\$120. A new investor, F, wishes to invest US\$0.96 million in the fund. If the total value of the assets in the fund is now US\$12 million or US\$120 per share, in order to accommodate the new investment the fund would create 8,000 (US\$0.96 million/US\$120) new shares. After this investment, the net asset value of the fund would be US\$12.96 million and there would be a total of 108,000 shares.

Funds can also be withdrawn at the net asset value per share. Suppose on the same day Investor $\mathrm{E}$ wishes to withdraw all her shares in the mutual fund. To accommodate this withdrawal, the fund will have to liquidate US $\$ 0.6$ million in assets to retire 5,000 shares at a net asset value of US $\$ 120$ per share (US $\$ 0.6$ million/US $\$ 120$ ). The combination of the inflow and outflow on the same day would be as follows:

\begin{center}
\begin{tabular}{lcc}
\hline
Type & Investment (US\$) & Shares \\
\hline
Inflow (Investor F buys) & $\$ 960,000$ & 8,000 \\
Outflow (Investor E sells) & $-\$ 600,000$ & $-5,000$ \\
\cline { 2 - 3 }
Net & $\$ 360,000$ & 3,000 \\
\hline
\end{tabular}
\end{center}

The net of the inflows and outflows on that day would be US $\$ 360,000$ of new funds to be invested and 3,000 new shares created. However, the number of shares held and the value of the shares of all remaining investors, except Investor E, would remain the same.

An alternative to setting the fund up as an open-end fund would be to create a closed-end fund in which no new investment money is accepted into the fund. New investors invest by buying existing shares, and investors in the fund liquidate by selling their shares to other investors. Hence, the number of outstanding shares does not change. One consequence of this fixed share base is that, unlike open-end funds in which new shares are created and sold at the current net asset value per share, closed-end funds can sell for a premium or discount to net asset value depending on the demand for the shares. There are advantages and disadvantages to each type of fund. The open-end fund structure makes it easy to grow in size but creates pressure on the portfolio manager to manage the cash inflows and outflows. One consequence of this structure is the need to liquidate assets that the portfolio manager might not want to sell at the time to meet redemptions. Conversely, the inflows require finding new assets in which to invest. As such, open-end funds tend not to be fully invested but rather keep some cash for redemptions not covered by new investments. Closed-end funds do not have these problems, but they do have a limited ability to grow. Of the total net asset value of all US mutual funds at the end of 2017 (US\$19 trillion), only approximately 1 percent were in the form of closed-end funds.

In addition to open-end or closed-end funds, mutual funds can be classified as load or no-load funds. The primary difference between the two is whether the investor pays a sales charge (a "load") to purchase, hold, or redeem shares in the fund. In the case of the no-load fund, there is no fee for investing in the fund or for redemption but there is an annual fee based on a percentage of the fund's net asset value. Load funds are funds in which, in addition to the annual fee, a percentage fee is charged to invest in the fund and/or for redemptions from the fund. In addition, load funds are usually sold through retail brokers who receive part of the upfront fee. Overall, the number and importance of load funds has declined over time.

\begin{center}
\includegraphics[max width=\textwidth]{2023_05_04_36535b8d80b32081d422g-538}
\end{center}

\section{POOLED INTEREST - TYPE OF MUTUAL FUNDS}
$$
\mid \begin{aligned}
& \text { describe mutual funds and compare them with other pooled } \\
& \text { investment products }
\end{aligned}
$$

The following section introduces the major types of mutual funds differentiated by the asset type that they invest in: money market funds, bond mutual funds, stock mutual funds, and hybrid or balanced funds.

\section{Money Market Funds}
Money market funds are mutual funds that invest in short-term money market instruments such as treasury bills, certificates of deposit, and commercial paper. They aim to provide security of principal, high levels of liquidity, and returns in line with money market rates. Many funds operate on a constant net asset value (CNAV) basis where the share price is maintained at $\$ 1$ (or local currency equivalent). Others operate on a variable net asset value (VNAV) basis where the unit price can vary. In the United States, there are two basic types of money market funds: taxable and tax-free. Taxable money market funds invest in high-quality, short-term corporate debt and federal government debt. Tax-free money market funds invest in short-term state and local government debt. Although money market funds have been a substitute for bank savings accounts since the early 1980s, they are not insured in the same way as bank deposits.

\section{Bond Mutual Funds}
A bond mutual fund is an investment fund consisting of a portfolio of individual bonds and, occasionally, preferred shares. The net asset value of the fund is the sum of the value of each bond in the portfolio divided by the number of shares. Investors in the mutual fund hold shares, which account for their pro-rata share or interest in the portfolio. The major difference between a bond mutual fund and a money market fund is the maturity of the underlying assets. In a money market fund the maturity is as short as overnight and rarely longer than 90 days. A bond mutual fund, however, holds bonds with maturities as short as one year and as long as 30 years (or more). Exhibit 22 illustrates the general categories of bond mutual funds. ${ }^{16}$

\section{Exhibit 22: Bond Mutual Funds}
\begin{center}
\begin{tabular}{ll}
\hline
Type of Bond Mutual Fund & Securities Held \\
\hline
Global & $\begin{array}{l}\text { Domestic and non-domestic government, corporate, } \\ \text { and securitized debt }\end{array}$ \\
Government & $\begin{array}{l}\text { Government bonds and other government-affiliated } \\ \text { bonds }\end{array}$ \\
Corporate & Below investment-grade corporate debt debt \\
High yield & Inflation-protected government debt \\
Inflation protected & National tax-free bonds (e.g., US municipal bonds) \\
National tax-free bonds &  \\
\end{tabular}
\end{center}

\section{Stock Mutual Funds}
Historically, the largest types of mutual funds based on market value of assets under management are stock (equity) funds.

There are two types of stock mutual funds. The first is an actively managed fund in which the portfolio manager seeks outstanding performance through the selection of the appropriate stocks to be included in the portfolio. Passive management is followed by index funds that are very different from actively managed funds. Their goal is to match or track the performance of different indexes. The first index fund was introduced in 1976 by the Vanguard Group.

There are several major differences between actively managed funds and index funds. First, management fees for actively managed funds are higher than for index funds. The higher fees for actively managed funds reflect its goal to outperform an index, whereas the index fund simply aims to match the return on the index. Higher fees are required to pay for the research conducted to actively select securities. A second difference is that the level of trading in an actively managed fund is much higher than in an index fund, which depending on the jurisdiction, has tax implications. Mutual funds are often required to distribute all income and capital gains realized in the portfolio, so the actively managed fund tends to have more opportunity to realize capital gains. This results in higher taxes relative to an index fund, which uses a buy-and-hold strategy. Consequently, there is less buying and selling in an index fund and less likelihood of realizing capital gains distributions.

\section{Hybrid/Balanced Funds}
Hybrid or balanced funds are mutual funds that invest in both bonds and stocks. These types of funds represent a small fraction of the total investment in US mutual funds but are more common in Europe. These types of funds, however, have gained popularity with the growth of lifecycle funds. Lifecycle or Target Date funds manage the asset mix based on a desired retirement date. For example, if an investor is 40 years

16 In the United States, judicial rulings on federal powers of taxation have created a distinction between (federally) taxable and (federally) tax-exempt bonds and a parallel distinction for US bond mutual funds. old in 2019 and planned to retire at the age of 67, he could invest in a mutual fund with a target date of 2046 and the fund would manage the appropriate asset mix over the next 27 years. In 2019 it might be 90 percent invested in shares and 10 percent in bonds. As time passes, however, the fund would gradually change the mix of shares and bonds to reflect the appropriate mix given the time to retirement.

\section{POOLED INTEREST - OTHER INVESTMENT PRODUCTS}
describe mutual funds and compare them with other pooled investment products

A fund management service for institutions or individual investors with substantial assets is the separately managed account (SMA), which is also commonly referred to as a "managed account," "wrap account," or "individually managed account."

SMAs are managed exclusively for the benefit of a single individual or institution. Unlike a mutual fund, the assets of an SMA are owned directly by the individual or institution. The main disadvantage of an SMA is that the required minimum investment is usually much higher than with a mutual fund.

Large institutional investors are generally the dominant users of SMAs. SMAs enable asset managers to implement an investment strategy that matches an investor's specific objectives, portfolio constraints, and tax considerations, where applicable. For example, a public pension plan investing in an asset manager's large value equity strategy might have a socially responsible investment preference. In this case, the plan sponsor may wish to exclude certain industries, such as tobacco and defense, while also including additional companies that are deemed favorable according to other environmental, social, and governance (ESG) considerations.

\section{Exchange-Traded Funds}
Exchange-traded funds (ETFs) are investment funds that trade on exchanges (similar to individual stocks) and are generally structured as open-end funds. ETFs represent one of the fastest-growing investment products in the asset management industry. According to BlackRock, global ETF assets increased from US\$428 billion in 2005 to US\$4.9 trillion as of June 2018. Long-term investors-both institutional and retail-use ETFs in building a diversified asset allocation. While ETFs are structured similarly to open-end mutual funds, some key differences exist between the two products. One difference relates to transaction price. Because they are traded on exchanges, ETFs can be transacted (and are priced) intraday. That is, ETF investors buy the shares from other investors just as if they were buying or selling shares of stock. ETF investors can also short shares or purchase the shares on margin. In contrast, mutual funds typically can be purchased or sold only once a day, and short sales or purchasing shares on margin is not allowed. Mutual fund investors buy the fund shares directly from the fund, and all investments are settled at the net asset value. In practice, the market price of the ETF is likely to be close to the net asset value of the underlying investments.

Other key differences between ETFs and mutual funds relate to transaction costs and treatment of dividends and the minimum investment amount. Dividends on ETFs are paid out to the shareholders whereas mutual funds usually reinvest the dividends. Finally, the minimum required investment in ETFs is usually smaller than that of mutual funds.

\section{Hedge Funds}
Hedge funds are private investment vehicles that typically use leverage, derivatives, and long and short investment strategies. The origin of hedge funds can be traced back as far as 1949 to a fund managed by A.W. Jones \& Co. It offered a strategy of a non-correlated offset to the "long-only" position typical of most portfolios. Since then, the hedge fund industry has grown considerably, with global hedge fund assets totaling US $\$ 3.3$ trillion as of May 2017.

Hedge fund investment strategies are diverse and can range from specific niche strategies (e.g., long-short financial services) to global multi-strategy approaches. Consequently, hedge funds are often used by investors for portfolio diversification purposes. In general, hedge funds share a few distinguishing characteristics:

\begin{itemize}
  \item Short selling: Many hedge funds implement short positions directly or synthetically using such derivatives as options, futures, and credit default swaps.

  \item Absolute return seeking: Hedge funds often seek positive returns in all market environments.

  \item Leverage: Many hedge funds use financial leverage (bank borrowing) or implicit leverage (using derivatives). The use and amount of leverage are dependent on the investment strategy being implemented.

  \item Low correlation: Some hedge funds have historically exhibited low return correlations with traditional equity and/or fixed-income asset classes.

  \item Fee structures: Hedge funds typically charge two distinct fees: a traditional asset-based management fee (AUM fee) and an incentive (or performance) fee in which the hedge fund earns a portion of the fund's realized capital gains. ${ }^{17}$ Hedge funds have traditionally charged management fees of $2 \%$ and incentive fees of up to $20 \%$, although there has been downward pressure on those fees amid increased competition and the availability of competing products.

\end{itemize}

Hedge funds are not readily available to all investors. They typically require a high minimum investment and often have restricted liquidity by allowing only periodic (e.g., quarterly) withdrawals or having a long fixed-term commitment.

\section{Private Equity and Venture Capital Funds}
Private equity funds and venture capital funds are alternative funds that seek to buy, optimize, and ultimately sell portfolio companies to generate profits. As of December 2017, assets under management in the private equity industry totaled US $\$ 3.1$ trillion, a historical high point. ${ }^{18}$ Most private equity and venture capital funds have a lifespan of approximately 7-10 years (but usually subject to contractual extensions). Unlike most traditional asset managers that trade in public securities, private equity and venture firms often take a "hands-on" approach to their portfolio companies through a combination of financial engineering (e.g., realizing expense synergies, changing capital structures), installment of executive management and board members, and significant contributions to the development of a target company's business strategy.

17 Performance fees are often subject to high-water mark provisions, which preclude a manager from earning a performance fee unless the value of a fund at the end of a predefined measurement period is higher than the value of the fund at the beginning of the measurement period. The unpredictability of future performance leads to uncertainty in performance fee revenue, which is regarded as less reliable than revenue derived from management fees.

18 \href{https://www.pionline.com/article/20180724/ONLINE/180729930/preqin-private-equity-aum-gro}{https://www.pionline.com/article/20180724/ONLINE/180729930/preqin-private-equity-aum-gro} ws-20-in-2017-to-record-306-trillion\# (accessed 13 November 2018) The final investment stage, often referred to as the "exit" or "harvesting" stage, occurs when a private equity or venture capital fund divests its portfolio companies through a merger with another company, the acquisition by another company, or an initial public offering (IPO).

As with most alternative funds, the majority of private equity and venture capital funds are structured as limited partnerships. These limited partnership agreements exist between the fund manager, called the general partner (GP), and the fund's investors, called limited partners (LPs). The funds generate revenue through several types of fees:

\begin{itemize}
  \item Management fees: Fees are based on committed capital (or sometimes net asset value or invested capital) and typically range from 1-3\% annually. Sometimes these fees step down several years into the investment period of a fund.

  \item Transaction fees: Fees are paid by portfolio companies to the fund for various corporate and structuring services. Typically, a percentage of the transaction fee is shared with the LPs by offsetting the management fee.

  \item Carried interest: Carried interest is the GP's share of profits (typically 20\%) on sales of portfolio companies. Most GPs do not earn the incentive fee until LPs have recovered their initial investment.

  \item Investment income. Investment income includes profits generated on capital contributed to the fund by the GP.

\end{itemize}

\section{SUMMARY}
\begin{itemize}
  \item A portfolio approach to investing could be preferable to simply investing in individual securities.

  \item The problem with focusing on individual securities is that this approach may lead to the investor "putting all her eggs in one basket."

  \item Portfolios provide important diversification benefits, allowing risk to be reduced without necessarily affecting or compromising return.

  \item Understanding the needs of your client and preparing an investment policy statement represent the first steps of the portfolio management process. Those steps are followed by asset allocation, security analysis, portfolio construction, portfolio monitoring and rebalancing, and performance measurement and reporting.

  \item Types of investors include individual and institutional investors. Institutional investors include defined benefit pension plans, endowments and foundations, banks, insurance companies, and sovereign wealth funds.

  \item The asset management industry is an integral component of the global financial services sector. Asset managers offer either active management, passive management, or both. Asset managers are typically categorized as traditional or alternative, although the line between traditional and alternative has blurred.

  \item Three key trends in the asset management industry include the growth of passive investing, "big data" in the investment process, and robo-advisers in the wealth management industry.

  \item Investors use different types of investment products in their portfolios. These include mutual funds, separately managed accounts, exchange-traded funds, hedge funds, and private equity and venture capital funds.

\end{itemize}

\section{REFERENCES}
Lintner, John. 1965. “The Valuation of Risk Assets and the Selection of Risky Investments in Stock Portfolios and Capital Budgets." Review of Economics and Statistics, vol. 47, no. 1 (February):13-37. 10.2307/1924119

Markowitz, Harry M. 1952. "Portfolio Selection." Journal of Finance, vol. 7, no. 1 (March):77-91. $10.2307 / 2975974$

Sharpe, William F. 1964. "Capital Asset Prices: A Theory of Market Equilibrium under Conditions of Risk." Journal of Finance, vol. 19, no. 3 (September):425-442. 10.2307/2977928

Singletary, Michelle. 2001. "Cautionary Tale of an Enron Employee Who Went for Broke." \href{http://Seattlepi.com}{Seattlepi.com} (10 December): \href{http://www.seattlepi.com/money/49894_singletary10.shtml}{http://www.seattlepi.com/money/49894\_singletary10.shtml}.

Treynor, J. L. 1961. “Toward a Theory of Market Value of Risky Assets." Unpublished manuscript.

\section{PRACTICE PROBLEMS}
\begin{enumerate}
  \item Investors should use a portfolio approach to:
A. reduce risk.
B. monitor risk.
C. eliminate risk.

  \item Which of the following is the best reason for an investor to be concerned with the composition of a portfolio?
A. Risk reduction.
B. Downside risk protection.
C. Avoidance of investment disasters.

  \item With respect to the formation of portfolios, which of the following statements is most accurate?
A. Portfolios affect risk less than returns.
B. Portfolios affect risk more than returns.
C. Portfolios affect risk and returns equally.

  \item With respect to the portfolio management process, the asset allocation is determined in the:
A. planning step.
B. feedback step.
C. execution step.

  \item The planning step of the portfolio management process is least likely to include an assessment of the client's
A. securities.
B. constraints.
C. risk tolerance.

  \item With respect to the portfolio management process, the rebalancing of a portfolio's composition is most likely to occur in the:
A. planning step.
B. feedback step.
C. execution step. 7. An analyst gathers the following information for the asset allocations of three portfolios:

\end{enumerate}

\begin{center}
\begin{tabular}{cccc}
\hline
Portfolio & Fixed Income (\%) & Equity (\%) & Alternative Assets (\%) \\
\hline
1 & 25 & 60 & 15 \\
2 & 60 & 25 & 15 \\
3 & 15 & 60 & 25 \\
\hline
\end{tabular}
\end{center}

Which of the portfolios is most likely appropriate for a client who has a high degree of risk tolerance?
A. Portfolio 1 .
B. Portfolio 2 .
C. Portfolio 3 .

\begin{enumerate}
  \setcounter{enumi}{7}
  \item Which of the following institutions will on average have the greatest need for liquidity?
A. Banks.
B. Investment companies.
C. Non-life insurance companies.

  \item Which of the following institutional investors will most likely have the longest time horizon?
A. Defined benefit plan.
B. University endowment.
C. Life insurance company.

  \item A defined benefit plan with a large number of retirees is likely to have a high need for:
A. income.
B. liquidity.
C. insurance.

  \item Which of the following institutional investors is most likely to manage investments in mutual funds?
A. Insurance companies.
B. Investment companies.
C. University endowments.

  \item Which of the following investment products is most likely to trade at their net asset value per share?
A. Exchange traded funds.
B. Open-end mutual funds. C. Closed-end mutual funds.

  \item Which of the following financial products is least likely to have a capital gain distribution?
A. Exchange traded funds.
B. Open-end mutual funds.
C. Closed-end mutual funds.

  \item Which of the following forms of pooled investments is subject to the least amount of regulation?
A. Hedge funds.
B. Exchange traded funds.
C. Closed-end mutual funds.

  \item Which of the following pooled investments is most likely characterized by a few large investments?

\end{enumerate}

A. Hedge funds.

B. Buyout funds.

C. Venture capital funds.

\section{SOLUTIONS}
\begin{enumerate}
  \item A is correct. Combining assets into a portfolio should reduce the portfolio's volatility. Specifically, "individuals and institutions should hold portfolios to reduce risk." As illustrated in the reading, however, risk reduction may not be as great during a period of dramatic economic change.

  \item A is correct. Combining assets into a portfolio should reduce the portfolio's volatility. The portfolio approach does not necessarily provide downside protection or guarantee that the portfolio always will avoid losses.

  \item B is correct. As illustrated in the reading, portfolios reduce risk more than they increase returns.

  \item C is correct. The client's objectives and constraints are established in the investment policy statement and are used to determine the client's target asset allocation, which occurs in the execution step of the portfolio management process.

  \item A is correct. Securities are analyzed in the execution step. In the planning step, a client's objectives and constraints are used to develop the investment policy statement.

  \item B is correct. Portfolio monitoring and rebalancing occurs in the feedback step of the portfolio management process.

  \item C is correct. Portfolio 3 has the same equity exposure as Portfolio 1 and has a higher exposure to alternative assets, which have greater volatility (as discussed in the section of the reading comparing the endowments from Yale University and the University of Virginia).

  \item A is correct. The excess reserves invested by banks need to be relatively liquid. Although investment companies and non-life insurance companies have high liquidity needs, the liquidity need for banks is on average the greatest.

  \item B is correct. Most foundations and endowments are established with the intent of having perpetual lives. Although defined benefit plans and life insurance companies have portfolios with a long time horizon, they are not perpetual.

  \item A is correct. Income is necessary to meet the cash flow obligation to retirees. Although defined benefit plans have a need for income, the need for liquidity typically is quite low. A retiree may need life insurance; however, a defined benefit plan does not need insurance.

  \item B is correct. Investment companies manage investments in mutual funds. Although endowments and insurance companies may own mutual funds, they do not issue or redeem shares of mutual funds.

  \item B is correct. Open-end funds trade at their net asset value per share, whereas closed-end funds and exchange traded funds can trade at a premium or a discount.

  \item A is correct. Exchange traded funds do not have capital gain distributions. If an investor sells shares of an ETF (or open-end mutual fund or closed-end mutual fund), the investor may have a capital gain or loss on the shares sold; however, the gain (or loss) from the sale is not a distribution.

  \item A is correct. Hedge funds are currently exempt from the reporting requirements of a typical public investment company.

  \item B is correct. Buyout funds or private equity firms make only a few large investments in private companies with the intent of selling the restructured companies in three to five years. Venture capital funds also have a short time horizon; however, these funds consist of many small investments in companies with the expectation that only a few will have a large payoff (and that most will fail).

\end{enumerate}

\section{LEARNING MODULE 2}
\section{Portfolio Risk and Return: Part I}
by Vijay Singal, PhD, CFA.

Vijay Singal, PhD, CFA, is at Virginia Tech (USA).

\section{LEARNING OUTCOME}
\begin{center}
\begin{tabular}{c|l}
Mastery & The candidate should be able to: \\
\hline
$\square$ & $\begin{array}{l}\text { calculate and interpret major return measures and describe their } \\ \text { appropriate uses } \\ \text { compare the money-weighted and time-weighted rates of return and } \\ \text { evaluate the performance of portfolios based on these measures } \\ \text { describe characteristics of the major asset classes that investors } \\ \text { consider in forming portfolios } \\ \text { explain risk aversion and its implications for portfolio selection } \\ \square\end{array} \square$ \\
$\square$ & $\begin{array}{l}\text { explain the selection of an optimal portfolio, given an investor's } \\ \text { utility (or risk aversion) and the capital allocation line } \\ \text { calculate and interpret the mean, variance, and covariance (or } \\ \text { correlation) of asset returns based on historical data } \\ \text { calculate and interpret portfolio standard deviation } \\ \text { describe the effect on a portfolio's risk of investing in assets that are } \\ \text { less than perfectly correlated } \\ \text { describe and interpret the minimum-variance and efficient frontiers } \\ \text { of risky assets and the global minimum-variance portfolio }\end{array}$ \\
$\square$ &  \\
\end{tabular}
\end{center}

\section{INTRODUCTION}
Construction of an optimal portfolio is an important objective for an investor. In this reading, we will explore the process of examining the risk and return characteristics of individual assets, creating all possible portfolios, selecting the most efficient portfolios, and ultimately choosing the optimal portfolio tailored to the individual in question.

During the process of constructing the optimal portfolio, several factors and investment characteristics are considered. The most important of those factors are risk and return of the individual assets under consideration. Correlations among individual assets along with risk and return are important determinants of portfolio risk. Creating a portfolio for an investor requires an understanding of the risk profile of the investor. Although we will not discuss the process of determining risk aversion for individuals or institutional investors, it is necessary to obtain such information for making an informed decision. In this reading, we will explain the broad types of investors and how their risk-return preferences can be formalized to select the optimal portfolio from among the infinite portfolios contained in the investment opportunity set.

The reading is organized as follows: Sections $2-8$ discuss the investment characteristics of assets. In particular, we show the various types of returns and risks, their computation and their applicability to the selection of appropriate assets for inclusion in a portfolio. Sections 9-11 discuss risk aversion and how indifference curves, which incorporate individual preferences, can be constructed. The indifference curves are then applied to the selection of an optimal portfolio using two risky assets. Sections 12-14 provides an understanding and computation of portfolio risk. The role of correlation and diversification of portfolio risk are examined in detail. Sections 15-17 begins with the risky assets available to investors and constructs a large number of risky portfolios. It illustrates the process of narrowing the choices to an efficient set of risky portfolios before identifying the optimal risky portfolio. The risky portfolio is combined with investor risk preferences to generate the investor's optimal portfolio. A summary concludes this reading.

\section{INVESTMENT CHARACTERISTICS OF ASSETS: RETURN}
calculate and interpret major return measures and describe their appropriate uses

Financial assets are frequently defined in terms of their risk and return characteristics. Comparison along these two dimensions simplifies the process of building a portfolio from among the multitude of available assets. In this section, we will compute, evaluate, and compare various measures of return and risk.

\section{Return}
Financial assets normally generate two types of return for investors. First, they may provide periodic income through cash dividends or interest payments. Second, the price of a financial asset can increase or decrease, leading to a capital gain or loss.

Some financial assets provide return through only one of these mechanisms. For example, investors in non-dividend-paying stocks obtain their return from price movement only. Similarly, you could also own or have a claim to assets that only generate periodic income. For example, defined benefit pension plans and retirement annuities make income payments as long as you live.

In the following section, we consider the computation and application of various types of returns.

\section{Holding Period Return}
Returns can be measured over a single period or over multiple periods. Single period returns are straightforward because there is only one way to calculate them. Multiple period returns, however, can be calculated in various ways and it is important to be aware of these differences to avoid confusion.

A holding period return is the return earned from holding an asset for a single specified period of time. The period may be 1 day, 1 week, 1 month, 5 years, or any specified period. If the asset (bond, stock, etc.) is bought now, time $(t=0)$, at a price of 100 and sold later, say at time $(t=1)$, at a price of 105 with no dividends or other income, then the holding period return is 5 percent [(105 - 100)/100)]. If the asset also pays an income of 2 units at time $(t=1)$, then the total return is 7 percent. This return can be generalized and shown as a mathematical expression in which $P$ is the price and $I$ is the income:

$$
R=\frac{\left(P_{1}-P_{0}\right)+I_{1}}{P_{0}}
$$

The subscript indicates the time of the price or income, $(t=0)$, is the beginning of the period and $(t=1)$ is the end of the period. The following two observations are important.

\begin{itemize}
  \item We computed a capital gain of 5 percent and a dividend yield of 2 percent in the above example. For ease of illustration, we assumed that the dividend is paid at time $t=1$. If the dividend was received any time before $t=1$, our holding period return would have been higher because we would have earned a return by reinvesting the dividend for the remainder of the period.

  \item Return can be expressed in decimals (0.07), fractions (7/100), or as a percent (7\%). They are all equivalent.

\end{itemize}

The holding period return can be computed for a period longer than one year. For example, you may need to compute a three-year holding period return from three annual returns. In that case, the holding period return is computed by compounding the three annual returns: $R=\left[\left(1+R_{1}\right) \times\left(1+R_{2}\right) \times\left(1+R_{3}\right)\right]-1$, where $R_{1}, R_{2}$, and $R_{3}$ are the three annual returns.

\section{Arithmetic or Mean Return}
When assets have returns for multiple holding periods, it is necessary to aggregate those returns into one overall return for ease of comparison and understanding. Most holding period returns are reported as daily, monthly, or annual returns. When comparing returns across assets, it is important that the returns are computed using a common time period.

There are different methods for aggregating returns across several holding periods. The remainder of this section presents various ways of computing average returns and discusses their applicability.

The simplest way to compute the return is to take a simple arithmetic average of all holding period returns. Thus, three annual returns of -50 percent, 35 percent, and 27 percent will give us an average of 4 percent per year $=\left(\frac{-50 \%+35 \%+27 \%}{3}\right)$. The arithmetic average return is easy to compute and has known statistical properties, such as standard deviation. We can calculate its standard deviation to determine how dispersed the observations are around the mean or if the mean return is statistically different from zero.

In general, the arithmetic or mean return is denoted by $\bar{R}_{i}$ and given by the following equation for asset $i$, where $R_{i t}$ is the return in period $t$ and $T$ is the total number of periods:

$$
\bar{R}_{i}=\frac{R_{i 1}+R_{i 2}+\ldots+R_{i, T-1}+R_{i T}}{T}=\frac{1}{T} \sum_{t=1}^{T} R_{i t}
$$

\section{Geometric Mean Return}
The arithmetic mean return assumes that the amount invested at the beginning of each period is the same. In an investment portfolio, however, even if there are no cash flows into or out of the portfolio, the base amount changes each year. (The previous year's earnings must be added to the beginning value of the subsequent year's investmentthese earnings will be "compounded" by the returns earned in that subsequent year.) We can use the geometric mean return to account for the compounding of returns.

A geometric mean return provides a more accurate representation of the growth in portfolio value over a given time period than does an arithmetic mean return. In general, the geometric mean return is denoted by $\bar{R}_{G i}$ and given by the following equation for asset $i$ :

$$
\bar{R}_{G i}=\sqrt[T]{\left(1+R_{i 1}\right) \times\left(1+R_{i 2}\right) \times \ldots \times\left(1+R_{i, T-1}\right) \times\left(1+R_{i T}\right)}-1
$$

where $R_{i t}$ is the return in period $t$ and $T$ is the total number of periods.

In the example in Section 2, we calculated the arithmetic mean to be 4 percent. Exhibit 1 shows the actual return for each year and the balance at the end of each year using actual returns. Beginning with an initial investment of $€ 1.0000$, we will have a balance of $€ 0.8573$ at the end of the three-year period as shown in the third column. Note that we compounded the returns because, unless otherwise stated, we earn a return on the balance as of the end of the prior year. That is, we will receive a return of 35 percent in the second year on the balance at the end of the first year, which is only $€ 0.5000$, not the initial balance of $€ 1.0000$. Let us compare the balance at the end of the three-year period computed using geometric returns with the balance we would calculate using the 4 percent annual arithmetic mean return from our earlier example. The ending value using the arithmetic mean return is $€ 1.1249\left(=1.0000 \times 1.04^{3}\right)$. This is much larger than the actual balance of $€ 0.8573$. In general, the arithmetic return is biased upward unless each of the underlying holding period returns are equal. The bias in arithmetic mean returns is particularly severe if holding period returns are a mix of both positive and negative returns, as in the example.

\section{Exhibit 1}
\begin{center}
\begin{tabular}{lcccc}
\hline
 & $\begin{array}{c}\text { Actual Return } \\ \text { for the Year } \\ (\%)\end{array}$ & $\begin{array}{c}\text { Year-End } \\ \text { Amount }\end{array}$ & $\begin{array}{c}\text { Year-End Amount } \\ \text { Using Arithmetic } \\ \text { Return of 4\% }\end{array}$ & $\begin{array}{c}\text { Year-End Amount } \\ \text { Using Geometric } \\ \text { Return of -5\% }\end{array}$ \\
\hline
Year 0 &  & $€ 1.0000$ & $€ 1.0000$ & $€ 1.0000$ \\
Year 1 & -50 & 0.5000 & 1.0400 & 0.9500 \\
Year 2 & 35 & 0.6750 & 1.0816 & 0.9025 \\
Year 3 & 27 & 0.8573 & 1.1249 & 0.8574 \\
\hline
\end{tabular}
\end{center}

MONEY-WEIGHTED RETURN OR INTERNAL RATE OF RETURN

compare the money-weighted and time-weighted rates of return and evaluate the performance of portfolios based on these measures

The arithmetic and geometric return computations do not account for the cash flows into and out of a portfolio. If the investor had invested $€ 10,000$ in the first year, $€ 1,000$ in the second year, and $€ 1,000$ in the third year, then the return of -50 percent in the first year significantly hurts her. On the other hand, if she had invested only $€ 100$ in the first year, the effect of the -50 percent return is drastically reduced. The money-weighted return accounts for the money invested and provides the investor with information on the return she earns on her actual investment. The money-weighted return and its calculation are similar to the internal rate of return and the yield to maturity. Just like the internal rate of return, amounts invested are cash outflows from the investor's perspective and amounts returned or withdrawn by the investor, or the money that remains at the end of an investment cycle, is a cash inflow for the investor.

The money-weighted return can be illustrated most effectively with an example. In this example, we use the returns from the previous example. Assume that the investor invests $€ 100$ in a mutual fund at the beginning of the first year, adds another $€ 950$ at the beginning of the second year, and withdraws $€ 350$ at the end of the second year. The cash flows are shown in Exhibit 2.

\section{Exhibit 2}
\begin{center}
\begin{tabular}{lccc}
\hline
Year & $\mathbf{1}$ & $\mathbf{2}$ & $\mathbf{3}$ \\
\hline
Balance from previous year & $€ 0$ & $€ 50$ & $€ 1,000$ \\
New investment by the investor (cash inflow & 100 & 950 & 0 \\
for the mutual fund) at the start of the year &  &  & 1,000 \\
Net balance at the beginning of year & 100 & $35 \%$ & $27 \%$ \\
Investment return for the year & $-50 \%$ & 350 & 270 \\
Investment gain (loss) & -50 & -350 & 0 \\
Withdrawal by the investor (cash outflow for & 0 &  & $€ 1,000$ \\
the mutual fund) at the end of the year &  &  &  \\
Balance at the end of year &  & 50 &  \\
\end{tabular}
\end{center}

The internal rate of return is the discount rate at which the sum of present values of these cash flows will equal zero. In general, the equation may be expressed as follows, where $T$ is the number of periods, $C F_{t}$ is the cash flow at time $t$, and IRR is the internal rate of return or the money-weighted rate of return:

$$
\sum_{t=0}^{T} \frac{C F_{t}}{(1+\mathrm{IRR})^{t}}=0
$$

A cash flow can be positive or negative; a positive cash flow is an inflow where money flows to the investor, whereas a negative cash flow is an outflow where money flows away from the investor. We can compute the internal rate of return by using the above equation. The flows are expressed as follows, where each cash inflow or outflow occurs at the end of each year. Thus, $\mathrm{CF}_{0}$ refers to the cash flow at the end of Year 0 or beginning of Year 1 , and $\mathrm{CF}_{3}$ refers to the cash flow at end of Year 3 or beginning of Year 4. Because cash flows are being discounted to the present-that is, end of Year 0 or beginning of Year 1 -the period of discounting $\mathrm{CF}_{0}$ is zero.

$$
\begin{aligned}
& \mathrm{CF}_{0}=-100 \\
& \mathrm{CF}_{1}=-950 \\
& \mathrm{CF}_{2}=+350 \\
& \mathrm{CF}_{3}=+1,270 \\
& \frac{\mathrm{CF}_{0}}{(1+\mathrm{IRR})^{0}}+\frac{\mathrm{CF}_{1}}{(1+\mathrm{IRR})^{1}}+\frac{\mathrm{CF}_{2}}{(1+\mathrm{IRR})^{2}}+\frac{\mathrm{CF}_{3}}{(1+\mathrm{IRR})^{3}} \\
&= \frac{-100}{1}+\frac{-950}{(1+\mathrm{IRR})^{1}}+\frac{+350}{(1+\mathrm{IRR})^{2}}+\frac{+1270}{(1+\mathrm{IRR})^{3}}=0 \\
& \mathrm{IRR}=26.11 \%
\end{aligned}
$$

IRR $=26.11 \%$ is the internal rate of return, or the money-weighted rate of return, which tells the investor what she earned on the actual euros invested for the entire period. This return is much greater than the arithmetic and geometric mean returns because only a small amount was invested when the mutual fund's return was -50 percent.

Next, we'll illustrate calculating the money-weighted return for a dividend paying stock. Consider an investment that covers a two-year horizon. At time $t=0$, an investor buys one share at $\$ 200$. At time $t=1$, he purchases an additional share at $\$ 225$. At the end of Year 2, $t=2$, he sells both shares for $\$ 235$ each. During both years, the stock pays a per-share dividend of $\$ 5$. The $t=1$ dividend is not reinvested. Exhibit 3 shows the total cash inflows and outflows.

\section{Exhibit 3: Cash Flows}
\begin{center}
\begin{tabular}{ll}
\hline
Time & Outflows \\
\hline
0 & $\$ 200$ to purchase the first share \\
1 & $\$ 225$ to purchase the second share \\
\hline
Time & Inflows \\
\hline
1 & $\$ 5$ dividend received from first share $($ and not reinvested) \\
210 dividend ( $\$ 5$ per share $\times 2$ shares) received &  \\
2470 received from selling two shares at $\$ 235$ per share &  \\
2 &  \\
\end{tabular}
\end{center}

To solve for the money-weighted return, we use either a financial calculator that allows us to enter cash flows or a spreadsheet with an IRR function. The first step is to group net cash flows by time. For this example, we have $-\$ 200$ for the $t=0$ net cash flow, $-\$ 220=-\$ 225+\$ 5$ for the $t=1$ net cash flow, and $\$ 480$ for the $t=2$ net cash flow. After entering these cash flows, we use the spreadsheet's or calculator's IRR function to find that the money-weighted rate of return is 9.39 percent.

$$
\begin{aligned}
& \mathrm{CF}_{0}=-200 \\
& \mathrm{CF}_{1}=-220 \\
& \mathrm{CF}_{2}=+480 \\
& \frac{\mathrm{CF}_{0}}{(1+\mathrm{IRR})^{0}}+\frac{\mathrm{CF}_{1}}{(1+\mathrm{IRR})^{1}}+\frac{\mathrm{CF}_{2}}{\left(1+\mathrm{IRR}^{2}\right.} \\
& =\frac{-200}{1}+\frac{-220}{(1+\mathrm{IRR})^{1}}+\frac{480}{(1+\mathrm{IRR})^{2}}=0 \\
& \mathrm{IRR}=9.39 \%
\end{aligned}
$$

Now we take a closer look at what has happened to the portfolio during each of the two years. In the first year, the portfolio generated a one-period holding period return of $(\$ 5+\$ 225-\$ 200) / \$ 200=15$ percent. At the beginning of the second year, the amount invested is $\$ 450$, calculated as $\$ 225$ (per share price of stock) $\times 2$ shares, because the $\$ 5$ dividend was spent rather than reinvested. At the end of the second year, the proceeds from the liquidation of the portfolio are $\$ 470$ (as detailed in Exhibit 3) plus $\$ 10$ in dividends (as also detailed in Exhibit 3). So in the second year the portfolio produced a holding period return of $(\$ 10+\$ 470-\$ 450) / \$ 450=6.67$ percent. The mean holding period return was $(15 \%+6.67 \%) / 2=10.84$ percent. The money-weighted rate of return, which we calculated as 9.39 percent, puts a greater weight on the second year's relatively poor performance (6.67 percent) than the first year's relatively good performance ( 15 percent), as more money was invested in the second year than in the first. That is the sense in which returns in this method of calculating performance are "money weighted." Although the money-weighted return is an accurate measure of what the investor actually earned on the money invested, it is limited in its applicability to other situations. For example, it does not allow for return comparison between different individuals or different investment opportunities. Two investors in the same mutual fund or with the same portfolio of underlying investments may have different money-weighted returns because they invested different amounts in different years.

\section{EXAMPLE 1}
\section{Computation of Returns}
Ulli Lohrmann and his wife, Suzanne Lohrmann, are planning for retirement and want to compare the past performance of a few mutual funds they are considering for investment. They believe that a comparison over a five-year period would be appropriate. They are given the following information about the Rhein Valley Superior Fund that they are considering.

\begin{center}
\begin{tabular}{|c|c|c|}
\hline
Year & $\begin{array}{l}\text { Assets Under Management at the Beginning } \\ \text { of Year }(\epsilon)\end{array}$ & Net Return (\%) \\
\hline
1 & 30 million & 15 \\
\hline
2 & 45 million & -5 \\
\hline
3 & 20 million & 10 \\
\hline
4 & 25 million & 15 \\
\hline
5 & 35 million & 3 \\
\hline
\end{tabular}
\end{center}

The Lohrmanns are interested in aggregating this information for ease of comparison with other funds.

\begin{enumerate}
  \item Compute the holding period return for the five-year period.
\end{enumerate}

\section{Solution}
The holding period return is $R=\left(1+R_{1}\right)\left(1+R_{2}\right)\left(1+R_{3}\right)\left(1+R_{4}\right)\left(1+R_{5}\right)$ $-1=(1.15)(0.95)(1.10)(1.15)(1.03)-1=0.4235=42.35 \%$ for the five-year period.

\begin{enumerate}
  \setcounter{enumi}{1}
  \item Compute the arithmetic mean annual return.
\end{enumerate}

\section{Solution}
The arithmetic mean annual return can be computed as an arithmetic mean of the returns given by this equation:

$$
\bar{R}_{i}=\frac{15 \%-5 \%+10 \%+15 \%+3 \%}{5}=7.60 \%
$$

\begin{enumerate}
  \setcounter{enumi}{2}
  \item Compute the geometric mean annual return. How does it compare with the arithmetic mean annual return?
\end{enumerate}

\section{Solution}
The geometric mean annual return can be computed using this equation:

$$
\begin{aligned}
& \bar{R}_{G i}=\sqrt[T]{\left(1+R_{i 1}\right) \times\left(1+R_{i 2}\right) \times \ldots \times\left(1+R_{i, T-1}\right) \times\left(1+R_{i T}\right)}-1 \\
= & \sqrt[5]{1.15 \times 0.95 \times 1.10 \times 1.15 \times 1.03}-1 \\
= & \sqrt[5]{1.4235}-1=0.0732=7.32 \%
\end{aligned}
$$

Thus, the geometric mean annual return is 7.32 percent, slightly less than the arithmetic mean return.

\begin{enumerate}
  \setcounter{enumi}{3}
  \item The Lohrmanns want to earn a minimum annual return of 5 percent. Is the money-weighted annual return greater than 5 percent?
\end{enumerate}

\section{Solution}
To calculate the money-weighted rate of return, tabulate the annual returns and investment amounts to determine the cash flows, as shown in Exhibit 4. All amounts are in millions of euros.

\section{Exhibit 4}
\begin{center}
\begin{tabular}{lccccc}
\hline
Year & $\mathbf{1}$ & $\mathbf{2}$ & $\mathbf{3}$ & $\mathbf{4}$ & $\mathbf{5}$ \\
\hline
Balance from previous year & 0 & 34.50 & 42.75 & 22.00 & 28.75 \\
New investment by the & 30.00 & 10.50 & 0 & 3.00 & 6.25 \\
investor (cash inflow for the &  &  &  &  &  \\
Rhein fund) &  &  &  &  &  \\
\end{tabular}
\end{center}

$\mathrm{CF}_{0}=-30.00, \mathrm{CF}_{1}=-10.50, \mathrm{CF}_{2}=+22.75, \mathrm{CF}_{3}=-3.00, \mathrm{CF}_{4}=-6.25, \mathrm{CF}_{5}=+36.05$.

For clarification, it may be appropriate to explain the notation for cash flows. Each cash inflow or outflow occurs at the end of each year. Thus, $\mathrm{CF}_{0}$ refers to the cash flow at the end of Year 0 or beginning of Year 1 , and $\mathrm{CF}_{5}$ refers to the cash flow at end of Year 5 or beginning of Year 6 . Because cash flows are being discounted to the present-that is, end of Year 0 or beginning of Year 1 -the period of discounting $\mathrm{CF}_{0}$ is zero whereas the period of discounting for $\mathrm{CF}_{5}$ is 5 years.

To get the exact money-weighted rate of return (IRR), the following equation would be equal to zero. Instead of calculating, however, use the 5 percent return to see whether the value of the expression is positive or not. If it is positive, then the money-weighted rate of return is greater than 5 percent, because a 5 percent discount rate could not reduce the value to zero.

$$
\frac{-30.00}{(1.05)^{0}}+\frac{-10.50}{(1.05)^{1}}+\frac{22.75}{(1.05)^{2}}+\frac{-3.00}{(1.05)^{3}}+\frac{-6.25}{(1.05)^{4}}+\frac{36.05}{(1.05)^{5}}=1.1471
$$

Because the value is positive, the money-weighted rate of return is greater than 5 percent. Using a financial calculator, the exact money-weighted rate of return is 5.86 percent.

\section{TIME-WEIGHTED RATE OF RETURN}
compare the money-weighted and time-weighted rates of return and evaluate the performance of portfolios based on these measures

An investment measure that is not sensitive to the additions and withdrawals of funds is the time-weighted rate of return. The time-weighted rate of return measures the compound rate of growth of $\$ 1$ initially invested in the portfolio over a stated measurement period. For the evaluation of portfolios of publicly traded securities, the time-weighted rate of return is the preferred performance measure as it neutralizes the effect of cash withdrawals or additions to the portfolio, which are generally outside of the control of the portfolio manager. To compute an exact time-weighted rate of return on a portfolio, take the following three steps:

\begin{enumerate}
  \item Price the portfolio immediately prior to any significant addition or withdrawal of funds. Break the overall evaluation period into subperiods based on the dates of cash inflows and outflows.

  \item Calculate the holding period return on the portfolio for each subperiod.

  \item Link or compound holding period returns to obtain an annual rate of return for the year (the time-weighted rate of return for the year). If the investment is for more than one year, take the geometric mean of the annual returns to obtain the time-weighted rate of return over that measurement period.

\end{enumerate}

Let us return to our dividend stock money-weighted example and calculate the time-weighted rate of return for that investor's portfolio. In that example, we computed the holding period returns on the portfolio, Step 2 in the procedure for finding the time-weighted rate of return. Given that the portfolio earned returns of 15 percent during the first year and 6.67 percent during the second year, what is the portfolio's time-weighted rate of return over an evaluation period of two years?

We find this time-weighted return by taking the geometric mean of the two holding period returns, Step 3 in the procedure above. The calculation of the geometric mean exactly mirrors the calculation of a compound growth rate. Here, we take the product of 1 plus the holding period return for each period to find the terminal value at $t=$ 2 of $\$ 1$ invested at $t=0$. We then take the square root of this product and subtract 1 to get the geometric mean. We interpret the result as the annual compound growth rate of $\$ 1$ invested in the portfolio at $t=0$. Thus we have

$(1+\text { Time-weighted return })^{2}=(1.15)(1.0667)$

Time-weighted return $=\sqrt{(1.15)(1.0667)}-1=10.76 \%$

The time-weighted return on the portfolio was 10.76 percent, compared with the money-weighted return of 9.39 percent, which gave larger weight to the second year's return. We can see why investment managers find time-weighted returns more meaningful. If a client gives an investment manager more funds to invest at an unfavorable time, the manager's money-weighted rate of return will tend to be depressed. If a client adds funds at a favorable time, the money-weighted return will tend to be elevated. The time-weighted rate of return removes these effects.

In defining the steps to calculate an exact time-weighted rate of return, we said that the portfolio should be valued immediately prior to any significant addition or withdrawal of funds. With the amount of cash flow activity in many portfolios, this task can be costly. We can often obtain a reasonable approximation of the time-weighted rate of return by valuing the portfolio at frequent, regular intervals, particularly if additions and withdrawals are unrelated to market movements. The more frequent the valuation, the more accurate the approximation. Daily valuation is commonplace. Suppose that a portfolio is valued daily over the course of a year. To compute the time-weighted return for the year, we first compute each day's holding period return. We compute 365 such daily returns, denoted $r_{1}, r_{2}, \ldots, r_{365}$. We obtain the annual return for the year by linking the daily holding period returns in the following way: $\left(1+r_{1}\right) \times\left(1+r_{2}\right) \times \ldots \times\left(1+r_{365}\right)-1$. If withdrawals and additions to the portfolio happen only at day's end, this annual return is a precise time-weighted rate of return for the year. Otherwise, it is an approximate time-weighted return for the year.

If we have a number of years of data, we can calculate a time-weighted return for each year individually, as above. If $r_{i}$ is the time-weighted return for year $i$, we calculate an annualized time-weighted return as the geometric mean of $N$ annual returns, as follows:

$$
r_{\mathrm{TW}}=\left[\left(1+r_{1}\right) \times\left(1+r_{2}\right) \times \ldots \times\left(1+r_{N}\right)\right]^{1 / N}-1
$$

Example 2 illustrates the calculation of the time-weighted rate of return.

\section{EXAMPLE 2}
\section{Time-Weighted Rate of Return}
Strubeck Corporation sponsors a pension plan for its employees. It manages part of the equity portfolio in-house and delegates management of the balance to Super Trust Company. As chief investment officer of Strubeck, you want to review the performance of the in-house and Super Trust portfolios over the last four quarters. You have arranged for outflows and inflows to the portfolios to be made at the very beginning of the quarter. Exhibit 5 summarizes the inflows and outflows as well as the two portfolios' valuations. In the table, the ending value is the portfolio's value just prior to the cash inflow or outflow at the beginning of the quarter. The amount invested is the amount each portfolio manager is responsible for investing.

\section{Exhibit 5: Cash Flows for the In-House Strubeck Account and the}
 Super Trust Account\begin{center}
\begin{tabular}{lcccc}
\hline
 &  & \multicolumn{2}{c}{Quarter} &  \\
\hline
 & $\mathbf{1}(\mathbf{\$})$ & $\mathbf{2} \mathbf{( \$ )}$ & $\mathbf{3}(\mathbf{\$})$ & $\mathbf{4} \mathbf{( \$ )}$ \\
\hline
In-House Account &  &  &  &  \\
Beginning value & $4,000,000$ & $6,000,000$ & $5,775,000$ & $6,720,000$ \\
Beginning of period inflow & $1,000,000$ & $(500,000)$ & 225,000 & $(600,000)$ \\
(outflow) &  &  &  &  \\
Amount invested & $5,000,000$ & $5,500,000$ & $6,000,000$ & $6,120,000$ \\
Ending value & $6,000,000$ & $5,775,000$ & $6,720,000$ & $5,508,000$ \\
Super Trust Account &  &  &  &  \\
Beginning value & $10,000,000$ & $13,200,000$ & $12,240,000$ & $5,659,200$ \\
Beginning of period inflow & $2,000,000$ & $(1,200,000)$ & $(7,000,000)$ & $(400,000)$ \\
(outflow) &  &  &  &  \\
Amount invested & $12,000,000$ & $12,000,000$ & $5,240,000$ & $5,259,200$ \\
Ending value & $13,200,000$ & $12,240,000$ & $5,659,200$ & $5,469,568$ \\
\hline
\end{tabular}
\end{center}

Based on the information given, address the following. 1. Calculate the time-weighted rate of return for the in-house account.

\section{Solution}
To calculate the time-weighted rate of return for the in-house account, we compute the quarterly holding period returns for the account and link them into an annual return. The in-house account's time-weighted rate of return is 27 percent, calculated as follows:

1Q HPR: $r_{1}=(\$ 6,000,000-\$ 5,000,000) / \$ 5,000,000=0.20$

2Q HPR: $r_{2}=(\$ 5,775,000-\$ 5,500,000) / \$ 5,500,000=0.05$

3Q HPR: $r_{3}=(\$ 6,720,000-\$ 6,000,000) / \$ 6,000,000=0.12$

4Q HPR: $r_{4}=(\$ 5,508,000-\$ 6,120,000) / \$ 6,120,000=-0.10$

$\left(1+r_{1}\right)\left(1+r_{2}\right)\left(1+r_{3}\right)\left(1+r_{4}\right)-1=(1.20)(1.05)(1.12)$

$(0.90)-1=0.27$ or $27 \%$

\begin{enumerate}
  \setcounter{enumi}{1}
  \item Calculate the time-weighted rate of return for the Super Trust account.
\end{enumerate}

\section{Solution}
The account managed by Super Trust has a time-weighted rate of return of 26 percent, calculated as follows:

1Q HPR: $r_{1}=(\$ 13,200,000-\$ 12,000,000) / \$ 12,000,000=0.10$

2Q HPR: $r_{2}=(\$ 12,240,000-\$ 12,000,000) / \$ 12,000,000=0.02$

3Q HPR: $r_{3}=(\$ 5,659,200-\$ 5,240,000) / \$ 5,240,000=0.08$

4Q HPR: $r_{4}=(\$ 5,469,568-\$ 5,259,200) / \$ 5,259,200=0.04$

$\left(1+r_{1}\right)\left(1+r_{2}\right)\left(1+r_{3}\right)\left(1+r_{4}\right)-1=(1.10)(1.02)(1.08)$

$(1.04)-1=0.26$ or $26 \%$

The in-house portfolio's time-weighted rate of return is higher than the Super Trust portfolio's by 100 basis points.

Having worked through this exercise, we are ready to look at a more detailed case.

\section{EXAMPLE 3}
\section{Time-Weighted and Money-Weighted Rates of Return Side}
 by SideYour task is to compute the investment performance of the Walbright Fund during 2014. The facts are as follows:

\begin{itemize}
  \item On 1 January 2014, the Walbright Fund had a market value of $\$ 100$ million.

  \item During the period 1 January 2014 to 30 April 2014, the stocks in the fund showed a capital gain of $\$ 10$ million.

  \item On 1 May 2014, the stocks in the fund paid a total dividend of $\$ 2$ million. All dividends were reinvested in additional shares.

  \item Because the fund's performance had been exceptional, institutions invested an additional $\$ 20$ million in Walbright on 1 May 2014, raising assets under management to $\$ 132$ million $(\$ 100+\$ 10+\$ 2+\$ 20)$.

  \item On 31 December 2014, Walbright received total dividends of $\$ 2.64$ million. The fund's market value on 31 December 2014, not including the $\$ 2.64$ million in dividends, was $\$ 140$ million. - The fund made no other interim cash payments during 2014.

\end{itemize}

Based on the information given, address the following.

\begin{enumerate}
  \item Compute the Walbright Fund's time-weighted rate of return.
\end{enumerate}

\section{Solution}
Because interim cash flows were made on 1 May 2014, we must compute two interim total returns and then link them to obtain an annual return. Exhibit 6 lists the relevant market values on 1 January, 1 May, and 31 December as well as the associated interim four-month (1 January to 1 May) and eight-month (1 May to 31 December) holding period returns.

\section{Exhibit 6: Cash Flows for the Walbright Fund}
1 May 2014 1 January 2014

Beginning portfolio value $=\$ 100$ million

Dividends received before additional investment = $\$ 2$ million Ending portfolio value $=\$ 110$ million

Holding period return $=\frac{\$ 2+\$ 10}{\$ 100}=12 \%$

New investment $=\$ 20$ million

Beginning market value for last $2 / 3$ of year $=\$ 132$ million

31 December 2014 Dividends received $=\$ 2.64$ million

Ending portfolio value $=\$ 140$ million

Holding period return $=\frac{\$ 2.64+\$ 140-\$ 132}{\$ 132}$

$=8.06 \%$

Now we must geometrically link the four- and eight-month returns to compute an annual return. We compute the time-weighted return as follows:

Time-weighted return $=1.12 \times 1.0806-1=0.2103$

In this instance, we compute a time-weighted rate of return of 21.03 percent for one year. The four-month and eight-month intervals combine to equal one year. (Taking the square root of the product $1.12 \times 1.0806$ would be appropriate only if 1.12 and 1.0806 each applied to one full year.)

\begin{enumerate}
  \setcounter{enumi}{1}
  \item Compute the Walbright Fund's money-weighted rate of return.
\end{enumerate}

\section{Solution}
To calculate the money-weighted return, we find the discount rate that sets the sum of the present value of cash inflows and outflows equal to zero. The initial market value of the fund and all additions to it are treated as cash outflows. (Think of them as expenditures.) Withdrawals, receipts, and the ending market value of the fund are counted as inflows. (The ending market value is the amount investors receive on liquidating the fund.) Because interim cash flows have occurred at four-month intervals, we must solve for the four-month internal rate of return. Exhibit 6 details the cash flows and their timing.

$\mathrm{CF}_{0}=-100$

$\mathrm{CF}_{1}=-20$

\section{$\mathrm{CF}_{2}=0$
$\mathrm{CF}_{3}=142.64$}
$\mathrm{CF}_{0}$ refers to the initial investment of $\$ 100$ million made at the beginning of the first four-month interval on 1 January 2014. $\mathrm{CF}_{1}$ refers to the cash flows made at end of the first four-month interval or the beginning of the second four-month interval on 1 May 2014. Those cash flows include a cash inflow of $\$ 2$ million for the dividend received and cash outflows of $\$ 22$ million for the dividend reinvested and additional investment respectively. The second four-month interval had no cash flow so $\mathrm{CF}_{2}$ is equal to zero. $\mathrm{CF}_{3}$ refers to the cash inflows at the end of the third four-month interval. Those cash inflows include a $\$ 2.64$ million dividend received and the fund's terminal market value of $\$ 140$ million

Using a spreadsheet or IRR-enabled calculator, we use $-100,-20,0$, and $\$ 142.64$ for the $t=0, t=1, t=2$, and $t=3$ net cash flows, respectively. Using either tool, we get a four-month IRR of 6.28 percent. The quick way to annualize this is to multiply by 3 . A more accurate way is $(1.0628)^{3}-1=0.20$ or 20 percent.

$$
\begin{aligned}
& \frac{\mathrm{CF}_{0}}{(1+\mathrm{IRR})^{0}}+\frac{\mathrm{CF}_{1}}{(1+\text { IRR })^{1}}+\frac{\mathrm{CF}_{2}}{(1+\mathrm{IRR})^{2}}+\frac{\mathrm{CF}_{3}}{(1+\mathrm{IRR})^{3}} \\
= & \frac{-100}{1}+\frac{-20}{(1+\text { IRR })^{1}}+\frac{0}{\left(1+\mathrm{IRR}^{2}\right.}+\frac{142.64}{\left(1+\mathrm{IRR}^{3}\right.}
\end{aligned}
$$

IRR $=6.28 \%$

\begin{enumerate}
  \setcounter{enumi}{2}
  \item Interpret the differences between the time-weighted and money-weighted rates of return.
\end{enumerate}

\section{Solution}
In this example, the time-weighted return (21.03 percent) is greater than the money-weighted return ( 20 percent). The Walbright Fund's performance was relatively poorer during the eight-month period, when the fund owned more shares, than it was overall. This fact is reflected in a lower money-weighted rate of return compared with time-weighted rate of return, as the money-weighted return is sensitive to the timing and amount of withdrawals and additions to the portfolio.

The accurate measurement of portfolio returns is important to the process of evaluating portfolio managers. In addition to considering returns, however, analysts must also weigh risk. When we worked through Example 2, we stopped short of suggesting that in-house management was superior to Super Trust because it earned a higher time-weighted rate of return. A judgement as to whether performance was "better" or "worse" must include the risk dimension, which will be covered later in your study materials.

\section{ANNUALIZED RETURN}
calculate and interpret major return measures and describe their appropriate uses The period during which a return is earned or computed can vary and often we have to annualize a return that was calculated for a period that is shorter (or longer) than one year. You might buy a short-term treasury bill with a maturity of 3 months, or you might take a position in a futures contract that expires at the end of the next quarter. How can we compare these returns? In, many cases, it is most convenient to annualize all available returns. Thus, daily weekly, monthly, and quarterly returns are converted to annualize all available returns. Many formulas used for calculating certain values or prices also require all returns and periods to be expressed as annualized rates of return. For example, the most common version of the Black-Scholes option-pricing model requires annualized returns and periods to be in years.

To annualize any return for a period shorter than one year, the return for the period must be compounded by the number of periods in a year. A monthly return is compounded 12 times, a weekly return is compounded 52 times, and a quarterly return is compounded 4 times. Daily returns are normally compounded 365 times. For an uncommon number of days, we compound by the ratio of 365 to the number of days.

If the weekly return is 0.2 percent, then the compound annual return is computed as shown because there are 52 weeks in a year:

$$
\begin{aligned}
& r_{\text {annual }}=\left(1+r_{\text {weekly }}\right)^{52}-1=(1+0.2 \%)^{52}-1 \\
& =(1.002)^{52}-1=0.1095=10.95 \%
\end{aligned}
$$

If the return for 15 days is 0.4 percent, the annualized return is computed assuming 365 days in a year. Thus,

$$
\begin{aligned}
& r_{\text {annual }}=\left(1+r_{15}\right)^{365 / 15}-1=(1+0.4 \%)^{365 / 15}-1 \\
& =(1.004)^{365 / 15}-1=0.1020=10.20 \%
\end{aligned}
$$

A general equation to annualize returns is given, where $c$ is the number of periods in a year. For a quarter, $c=4$ and for a month, $c=12$ :

$$
r_{\text {annual }}=\left(1+r_{\text {period }}\right)^{c}-1
$$

How can we annualize a return when the holding period return is more than one year? For example, how do we annualize an 18-month holding period return? Because one year contains two-thirds of 18-month periods, $c=2 / 3$ in the above equation. An 18-month return of 20 percent can be annualized, as shown:

$$
r_{\text {annual }}=\left(1+r_{18 \text { month }}\right)^{2 / 3}-1=(1+0.20)^{2 / 3}-1=0.1292=12.92 \%
$$

Similar expressions can be constructed when quarterly or weekly returns are needed for comparison instead of annual returns. In such cases, $c$ is equal to the number of holding periods in a quarter or in a week. For example, assume that you want to convert daily returns to weekly returns or annual returns to weekly returns for comparison between weekly returns. For converting daily returns to weekly returns, $c=$ 5 , assuming that there are five trading days in a week. For converting annual returns to weekly returns, $c=1 / 52$. The expressions for annual returns can then be rewritten as expressions for weekly returns, as shown:

$$
r_{\text {weekly }}=\left(1+r_{\text {daily }}\right)^{5}-1 ; r_{\text {weekly }}=\left(1+r_{\text {annual }}\right)^{1 / 52}-1
$$

One major limitation of annualizing returns is the implicit assumption that returns can be repeated precisely, that is, money can be reinvested repeatedly while earning a similar return. This type of return is not always possible. An investor may earn a return of 5 percent during a week because the market went up that week or he got lucky with his stock, but it is highly unlikely that he will earn a return of 5 percent every week for the next 51 weeks, resulting in an annualized return of $1,164.3$ percent $\left(=1.05^{52}-1\right)$. Therefore, it is important to annualize short-term returns with this limitation in mind.

\section{EXAMPLE 4}
\section{Annualized Returns}
An analyst is trying to evaluate three securities that have been in her portfolio for different periods of time.

\begin{itemize}
  \item In the last 100 days, Security A has earned a return of 6.2 percent.

  \item Security B has earned 2 percent over the last 4 weeks.

  \item Security $C$ has earned a return of 5 percent over the last 3 months

\end{itemize}

\begin{enumerate}
  \item The analyst is trying to assess the relative performance of the 3 securities.
\end{enumerate}

\section{Solution}
Annualized return for Security A: $R_{S A}=(1+0.062)^{365 / 100}-1=0.2455=$ $24.55 \%$

Annualized return for Security B: $R_{S B}=(1+0.02)^{52 / 4}-1=0.2936=29.36 \%$

Annualized return for Security C: $R_{S C}=(1+0.05)^{4}-1=0.2155=21.55 \%$

Security B has the highest annualized return.

\section{OTHER MAJOR RETURN MEASURES AND THEIR APPLICATIONS}
calculate and interpret major return measures and describe their appropriate uses

The statistical measures of return discussed in the previous section are generally applicable across a wide range of assets and time periods. Special assets, however, such as mutual funds, and other considerations, such as taxes or inflation, may require return measures that are specific to a particular application.

Although it is not possible to consider all types of special applications, we will discuss the effect of fees (gross versus net returns), taxes (pre-tax and after-tax returns), inflation (nominal and real returns), and leverage. Many investors use mutual funds or other external entities (i.e., investment vehicles) for investment. In those cases, funds charge management fees and expenses to the investors. Consequently, gross and net-of-fund-expense returns should also be considered. Of course, an investor may be interested in the net-of-expenses after-tax real return, which is in fact what an investor truly receives. We consider these additional return measures in the following sections.

\section{Gross and Net Return}
A gross return is the return earned by an asset manager prior to deductions for management expenses, custodial fees, taxes, or any other expenses that are not directly related to the generation of returns but rather related to the management and administration of an investment. These expenses are not deducted from the gross return because they may vary with the amount of assets under management or may vary because of the tax status of the investor. Trading expenses, however, such as commissions, are accounted for in (i.e., deducted from) the computation of gross return because trading expenses contribute directly to the return earned by the manager. Thus, gross return is an appropriate measure for evaluating and comparing the investment skill of asset managers because it does not include any fees related to the management and administration of an investment.

Net return is a measure of what the investment vehicle (mutual fund, etc.) has earned for the investor. Net return accounts for (i.e., deducts) all managerial and administrative expenses that reduce an investor's return. Because individual investors are most concerned about the net return (i.e., what they actually receive), small mutual funds with a limited amount of assets under management are at a disadvantage compared with the larger funds that can spread their largely fixed administrative expenses over a larger asset base. As a result, many small mutual funds waive part of the expenses to keep the funds competitive.

\section{Pre-tax and After-tax Nominal Return}
All return measures discussed previously are pre-tax nominal returns-that is, no adjustment has been made for taxes or inflation. In general, all returns are pre-tax nominal returns unless they are otherwise designated.

Many investors are concerned about the possible tax liability associated with their returns because taxes reduce the net return that they receive. Capital gains and income may be taxed differently, depending on the jurisdiction. Capital gains come in two forms: short-term capital gains and long-term capital gains. Long-term capital gains receive preferential tax treatment in a number of countries. Interest income is taxed as ordinary income in most countries. Dividend income may be taxed as ordinary income, may have a lower tax rate, or may be exempt from taxes depending on the country and the type of investor. The after-tax nominal return is computed as the total return minus any allowance for taxes on dividends, interest and realized gains.

Because taxes are paid on realized capital gains and income, the investment manager can minimize the tax liability by selecting appropriate securities (e.g., those subject to more favorable taxation, all other investment considerations equal) and reducing trading turnover. Therefore, taxable investors evaluate investment managers based on the after-tax nominal return.

\section{Real Returns}
A nominal return $(r)$ consists of three components: a real risk-free return as compensation for postponing consumption $\left(r_{r F}\right)$, inflation as compensation for loss of purchasing power $(\pi)$, and a risk premium for assuming risk $(R P)$. Thus, nominal return and real return can be expressed as:

$$
\begin{aligned}
& (1+r)=\left(1+r_{r F}\right) \times(1+\pi) \times(1+R P) \\
& \left(1+r_{\text {real }}\right)=\left(1+r_{r F}\right) \times(1+R P) \text { or } \\
& \left(1+r_{\text {real }}\right)=(1+r) \div(1+\pi)
\end{aligned}
$$

Often the real risk-free return and the risk premium are combined to arrive at the real "risky" rate as given in the second equation above, simply referred to as the real return. Real returns are particularly useful in comparing returns across time periods because inflation rates may vary over time. Real returns are also useful in comparing returns among countries when returns are expressed in local currencies instead of a constant investor currency and when inflation rates vary between countries (which are usually the case). Finally, the after-tax real return is what the investor receives as compensation for postponing consumption and assuming risk after paying taxes on investment returns. As a result, the after-tax real return becomes a reliable benchmark for making investment decisions. Although it is a measure of an investor's benchmark return, it is not commonly calculated by asset managers because it is difficult to estimate a general tax component applicable to all investors. For example, the tax component depends on an investor's specific taxation rate (marginal tax rate), how long the investor holds an investment (long-term versus short-term), and the type of account the asset is held in (tax-exempt, tax-deferred, or normal).

\section{EXAMPLE 5}
\section{Computation of Special Returns}
Let's return to Example 1. After reading this section, Mr. Lohrmann decided that he was not being fair to the fund manager by including the asset management fee and other expenses because the small size of the fund would put it at a competitive disadvantage. He learns that the fund spends a fixed amount of $€ 500,000$ every year on expenses that are unrelated to the manager's performance. Mr. Lohrmann has become concerned that both taxes and inflation may reduce his return. Based on the current tax code, he expects to pay 20 percent tax on the return he earns from his investment. Historically, inflation has been around 2 percent and he expects the same rate of inflation to be maintained.

\begin{enumerate}
  \item Estimate the annual gross return for the first year by adding back the fixed expenses.
\end{enumerate}

\section{Solution}
The gross return for the first year is higher by 1.67 percent (= $€ 500,000 / € 30,000,000)$ than the investor return reported by the fund. Thus, the gross return is 16.67 percent $(=15 \%+1.67 \%)$.

\begin{enumerate}
  \setcounter{enumi}{1}
  \item What is the net return that investors in the Rhein Valley Superior Fund earned during the five-year period?
\end{enumerate}

\section{Solution}
The investor return reported by the mutual fund is the net return of the fund after accounting for all direct and indirect expenses. The net return is also the pre-tax nominal return because it has not been adjusted for taxes or inflation. The net return for the five-year holding period was 42.35 percent.

\begin{enumerate}
  \setcounter{enumi}{2}
  \item What is the after-tax net return for the first year that investors earned from the Rhein Valley Superior Fund? Assume that all gains are realized at the end of the year and the taxes are paid immediately at that time.
\end{enumerate}

\section{Solution}
The net return earned by investors during the first year was 15 percent. Applying a 20 percent tax rate, the after-tax return that accrues to the investors is 12 percent $[=15 \%-(0.20 \times 15 \%)]$.

\begin{enumerate}
  \setcounter{enumi}{3}
  \item What is the anticipated after-tax real return that investors would have earned in the fifth year?
\end{enumerate}

\section{Solution}
As in Part 3, the after-tax return earned by investors in the fifth year is 2.4 percent $[=3 \%-(0.20 \times 3 \%)]$. Inflation reduces the return by 2 percent so the after-tax real return earned by investors in the fifth year is 0.39 percent, as shown:

$$
\frac{(1+2.40 \%)}{(1+2.00 \%)}-1=\frac{(1+0.0240)}{(1+0.0200)}-1=1.0039-1=0.0039=0.39 \%
$$

Note that taxes are paid before adjusting for inflation.

\section{Leveraged Return}
In the previous calculations, we have assumed that the investor's position in an asset is equal to the total investment made by an investor using his or her own money. This section differs in that the investor creates a leveraged position. There are two ways of creating a claim on asset returns that are greater than the investment of one's own money. First, an investor may trade futures contracts in which the money required to take a position may be as little as 10 percent of the notional value of the asset. In this case, the leveraged return, the return on the investor's own money, is 10 times the actual return of the underlying security. Both the gains and losses are amplified by a factor of 10 .

Investors can also invest more than their own money by borrowing money to purchase the asset. This approach is easily done in stocks and bonds, and very common when investing in real estate. If half ( 50 percent) of the money invested is borrowed, then the gross return to the investor is doubled but the interest to be paid on borrowed money must be deducted in order to calculate the net return.

\section{HISTORICAL RETURN AND RISK}
describe characteristics of the major asset classes that investors consider in forming portfolios

At this time, it is helpful to look at historical risk and returns for the three main asset categories: stocks, bonds, and Treasury bills. Stocks refer to corporate ownership, bonds refer to long-term fixed-income securities, and Treasury bills refer to short-term government debt securities. Although there is generally no expectation of default on government securities, long-term government bond prices are volatile (risky) because of possible future changes in interest rates. In addition, bondholders also face the risk that inflation will reduce the purchasing power of their cash flows.

\section{Historical Mean Return and Expected Return}
Before examining historical data, it is useful to distinguish between the historical mean return and expected return, which are very different concepts but easy to confuse. Historical return is what was actually earned in the past, whereas expected return is what an investor anticipates to earn in the future.

Expected return is the nominal return that would cause the marginal investor to invest in an asset based on the real risk-free interest rate $\left(r_{r F}\right)$, expected inflation $[E(\pi)]$, and expected risk premium for the risk of the asset $[E(R P)]$. The real risk-free interest rate is expected to be positive as compensation for postponing consumption. Similarly, the risk premium is expected to be positive in most cases. ${ }^{1}$ The expected inflation rate is generally positive, except when the economy is in a deflationary state

1 There are exceptions when an asset reduces overall risk of a portfolio. We will consider those exceptions in Section 14. and prices are falling. Thus, expected return is generally positive. The relationship between the expected return and the real risk-free interest rate, inflation rate, and risk premium can be expressed by the following equation:

$$
1+E(R)=\left(1+r_{r F}\right) \times[1+E(\pi)] \times[1+E(R P)]
$$

The historical mean return for investment in a particular asset, however, is obtained from the actual return that was earned by an investor. Because the investment is risky, there is no guarantee that the actual return will be equal to the expected return. In fact, it is very unlikely that the two returns are equal for a specific time period being considered. Given a long enough period of time, we can expect that the future (expected) return will equal the average historical return. Unfortunately, we do not know how long that period is -10 years, 50 years, or 100 years. As a practical matter, we often assume that the historical mean return is an adequate representation of the expected return, although this assumption may not be accurate. For example, Exhibit 7 shows that the historical equity returns in the last eight years (2010-2017) for large US company stocks were positive whereas the actual return was negative the prior decade, but nearly always positive historically. Nonetheless, longer-term returns (1926-2017) were positive and could be consistent with expected return. Though it is unknown if the historical mean returns accurately represent expected returns, it is an assumption that is commonly made.

\section{Exhibit 7: Risk and Return for US Asset Classes by Decade (\%)}
\begin{center}
\begin{tabular}{|c|c|c|c|c|c|c|c|c|c|c|c|}
\hline
 &  & 1930s & 1940s & 1950s & 1960s & 1970s & 1980s & 1990s & $2000 s$ & 2010s* & $1926-2017$ \\
\hline
\multirow{2}{*}{$\begin{array}{l}\text { Large com- } \\
\text { pany stocks }\end{array}$} & Return & -0.1 & 9.2 & 19.4 & 7.8 & 5.9 & 17.6 & 18.2 & -1.0 & 13.9 & 10.2 \\
\hline
 & Risk & 41.6 & 17.5 & 14.1 & 13.1 & 17.2 & 19.4 & 15.9 & 16.3 & 13.6 & 19.8 \\
\hline
\multirow{2}{*}{$\begin{array}{l}\text { Small com- } \\
\text { pany stocks }\end{array}$} & Return & 1.4 & 20.7 & 16.9 & 15.5 & 11.5 & 15.8 & 15.1 & 6.3 & 14.8 & 12.1 \\
\hline
 & Risk & 78.6 & 34.5 & 14.4 & 21.5 & 30.8 & 22.5 & 20.2 & 26.1 & 19.4 & 31.7 \\
\hline
\multirow{2}{*}{$\begin{array}{l}\text { Long-term } \\
\text { corporate } \\
\text { bonds }\end{array}$} & Return & 6.9 & 2.7 & 1 & 1.7 & 6.2 & 13 & 8.4 & 7.7 & 8.3 & 6.1 \\
\hline
 & Risk & 5.3 & 1.8 & 4.4 & 4.9 & 8.7 & 14.1 & 6.9 & 11.7 & 8.8 & 8.3 \\
\hline
\multirow{2}{*}{$\begin{array}{l}\text { Long-term } \\
\text { government } \\
\text { bonds }\end{array}$} & Return & 4.9 & 3.2 & -0.1 & 1.4 & 5.5 & 12.6 & 8.8 & 7.7 & 6.8 & 5.5 \\
\hline
 & Risk & 5.3 & 2.8 & 4.6 & 6 & 8.7 & 16 & 8.9 & 12.4 & 10.8 & 9.9 \\
\hline
\multirow{2}{*}{$\begin{array}{l}\text { Treasury } \\
\text { bills }\end{array}$} & Return & 0.6 & 0.4 & 1.9 & 3.9 & 6.3 & 8.9 & 4.9 & 2.8 & 0.2 & 3.4 \\
\hline
 & Risk & 0.2 & 0.1 & 0.2 & 0.4 & 0.6 & 0.9 & 0.4 & 0.6 & 0.1 & 3.1 \\
\hline
\multirow[t]{2}{*}{Inflation} & Return & -2.0 & 5.4 & 2.2 & 2.5 & 7.4 & 5.1 & 2.9 & 2.5 & 1.7 & 2.9 \\
\hline
 & Risk & 2.5 & 3.1 & 1.2 & 0.7 & 1.2 & 1.3 & 0.7 & 1.6 & 1.1 & 4.0 \\
\hline
\end{tabular}
\end{center}

\begin{itemize}
  \item Through 31 December 2017
\end{itemize}

Note: Returns are measured as annualized geometric mean returns.

Risk is measured by annualizing monthly standard deviations.

Source: 2018 SBBI Yearbook (Exhibits 1.2,1.3, 2.3 and 6.2).

Going forward, be sure to distinguish between expected return and historical mean return. We will alert the reader whenever historical returns are used to estimate expected returns.

\section{Nominal Returns of Major US Asset Classes}
We focus on three major asset categories in Exhibit 7: stocks, bonds, and T-bills. The mean nominal returns for US asset classes are reported decade by decade since the 1930s. The total for the 1926-2017 period is in the last column. All returns are annual geometric mean returns. Large company stocks had an overall annual return of 10.2 percent during the 92-year period. The return was negative in the 1930s and 2000s, and positive in all remaining decades. The 1950s and 1990s were the best decades for large company stocks. Small company stocks fared even better. The nominal return was never negative for any decade, and had double-digit growth in all decades except two, leading to an overall 92-year annual return of 12.1 percent.

Long-term corporate bonds and long-term government bonds earned overall returns of 6.1 percent and 5.5 percent, respectively. The corporate bonds did not have a single negative decade, although government bonds recorded a negative return in the 1950s when stocks were doing extremely well. Bonds also had some excellent decades, earning double-digit returns in the 1980s and 2000s.

Treasury bills (short-term government securities) did not earn a negative return in any decade. In fact, Treasury bills earned a negative return only in $1938(-0.02$ percent) when the inflation rate was -2.78 percent. Consistently positive returns for Treasury bills are not surprising because nominal interest rates are almost never negative and the Treasury bills suffer from little interest rate or inflation risk. Since the Great Depression, there has been no deflation in any decade, although inflation rates were highly negative in 1930 (-6.03 percent), 1931 (-9.52 percent), and 1932 $(-10.30$ percent). Conversely, inflation rates were very high in the late 1970 s and early 1980s, reaching 13.31 percent in 1979. Inflation rates have been largely range bound between 1 and 3 percent from 1991 to 2017 . Overall, the inflation rate was 2.9 percent for the 92-year period.

\section{Real Returns of Major US Asset Classes}
Because annual inflation rates can vary greatly, from -10.30 percent to +13.31 percent in the last 92 years, comparisons across various time periods are difficult and misleading using nominal returns. Therefore, it is more effective to rely on real returns. Real returns on stocks, bonds, and T-bills are reported from 1900 in Exhibit 8 and Exhibit 9. Exhibit 8: Cumulative Returns on US Asset Classes in Real Terms, 1900-2017

\begin{center}
\includegraphics[max width=\textwidth]{2023_05_04_36535b8d80b32081d422g-569}
\end{center}

Source: E. Dimson, P. Marsh, and M. Staunton, Credit Suisse Global Investment Returns Yearbook 2018, Credit Suisse Research Institute (February 2018). This chart is updated annually and can be found at \href{https://www.credit-suisse.com/media/assets/corporate/docs/about-us/media/media-release/2018/02/}{https://www.credit-suisse.com/media/assets/corporate/docs/about-us/media/media-release/2018/02/} giry-summary-2018.pdf.

Exhibit 8 shows that $\$ 1$ would have grown to $\$ 1,654$ if invested in stocks, to only $\$ 10.20$ if invested in bonds, and to $\$ 2.60$ if invested in T-bills. The difference in growth among the three asset categories is huge, although the difference in real returns does not seem that large: 6.5 percent per year for equities compared with 2.0 percent per year for bonds. This difference represents the effect of compounding over a 118-year period.

Exhibit 9 reports real rates of return. As we discussed earlier and as shown in the table, geometric mean is never greater than the arithmetic mean. Our analysis of returns focuses on the geometric mean because it is a more accurate representation of returns for multiple holding periods than the arithmetic mean. We observe that the real returns for stocks are higher than the real returns for bonds.

\section{Exhibit 9: Real Returns and Risk Premiums for Asset Classes (1900-2017)}
\begin{center}
\begin{tabular}{|c|c|c|c|c|c|c|c|c|c|c|}
\hline
 & \multirow[b]{2}{*}{Asset} & \multicolumn{3}{|c|}{United States} & \multicolumn{3}{|c|}{World} & \multicolumn{3}{|c|}{$\begin{array}{c}\text { World excluding United } \\
\text { States }\end{array}$} \\
\hline
 &  & GM (\%) & $\begin{array}{l}A M \\ (\%)\end{array}$ & $\begin{array}{l}S D \\ (\%)\end{array}$ & GM (\%) & $\begin{array}{l}A M \\ (\%)\end{array}$ & $\begin{array}{l}S D \\ (\%)\end{array}$ & GM (\%) & $\begin{array}{l}A M \\ (\%)\end{array}$ & SD (\%) \\
\hline
\multirow[t]{2}{*}{Real Returns} & Equities & 6.5 & 8.4 & 20.0 & 5.2 & 6.6 & 17.4 & 4.5 & 6.2 & 18.9 \\
\hline
 & Bonds & 2.0 & 2.5 & 10.4 & 2.0 & 2.5 & 11.0 & 1.7 & 2.7 & 14.4 \\
\hline
Premiums & $\begin{array}{l}\text { Equities vs. } \\ \text { bonds }\end{array}$ & 4.4 & 6.5 & 20.7 & 3.2 & 4.4 & 15.3 & 2.8 & 3.8 & 14.4 \\
\hline
\end{tabular}
\end{center}

Note: All returns are in percent per annum measured in US\$. GM = geometric mean, AM = arithmetic mean, $\mathrm{SD}=$ standard deviation.

"World" consists of 21 developed countries: Australia, Austria, Belgium, Canada, Denmark, Finland, France, Germany, Ireland, Italy, Japan, the Netherlands, New Zealand, Norway, Portugal, South Africa, Spain, Sweden, Switzerland, United Kingdom, and the United States. Weighting is by each country's relative market capitalization size. See source for details of calculations Source: Credit Suisse Global Investment Returns Sourcebook, 2018.

\section{Nominal and Real Returns of Asset Classes in Major Countries}
Along with US returns, real returns of major asset classes for a 21-country world and the world excluding the United States are also presented in Exhibit 9. Equity returns are weighted by each country's GDP before 1968 because of a lack of reliable market capitalization data. Returns are weighted by a country's market capitalization beginning with 1968. Similarly, bond returns are defined by a 21-country bond index, except GDP is used to create the weights because equity market capitalization weighting is inappropriate for a bond index and bond market capitalizations were not readily available.

The real geometric mean return for the world stock index over the last 117 years was 5.2 percent, and bonds had a real geometric mean return of 2.0 percent. The real geometric mean return for the world excluding the United States were 4.5 percent for stocks and 1.7 percent for bonds. For both stocks and bonds, the United States earned higher returns than the world excluding the United States. Similarly, real returns for stocks and bonds in the United States were higher than the real returns for rest of the world.

\section{Risk of Major Asset Classes}
Risk for major asset classes in the United States is reported for 1926-2017 in Exhibit 7, and the risk for major asset classes for the United States, the world, and the world excluding the United States are reported for 1900-2017 in Exhibit 9. Exhibit 7 shows that US small company stocks had the highest risk, 31.7 percent, followed by US large company stocks, 19.8 percent. Long-term government bonds and long-term corporate bonds had lower risk at 9.9 percent and 8.3 percent, with Treasury bills having the lowest risk at about 3.1 percent.

Exhibit 9 shows that the risk for world stocks is 17.4 percent and for world bonds is 11.0 percent. The world excluding the United States has risks of 18.9 percent for stocks and 14.4 percent for bonds. The effect of diversification is apparent when world risk is compared with US risk and world excluding US risk. Although the risk of US stocks is 20.0 percent and the risk of world excluding US stocks is 18.9 percent, the combination gives a risk of only 17.4 percent for world stocks.

\section{Risk-Return Trade-off}
The expression "risk-return trade-off" refers to the positive relationship between expected risk and return. In other words, a higher return is not possible to attain in efficient markets and over long periods of time without accepting higher risk. Expected returns should be greater for assets with greater risk.

The historical data presented above show the risk-return trade-off. Exhibit 7 shows for the United States that small company stocks had higher risk and higher return than large company stocks. Large company stocks had higher returns and higher risk than both long-term corporate bonds and government bonds. Bonds had higher returns and higher risk than Treasury bills. Uncharacteristically, however, long-term government bonds had higher total risk than long-term corporate bonds, although the returns of corporate bonds were slightly higher. These factors do not mean that long-term government bonds had greater default risk, just that they were more variable than corporate bonds during this historic period.

Exhibit 9 reveals that the risk and return for stocks were the highest of the asset classes, and the risk and return for bonds were lower than stocks for the United States, the world, and the world excluding the United States.

Another way of looking at the risk-return trade-off is to focus on the risk premium, which is the extra return investors can expect for assuming additional risk, after accounting for the risk-free interest rate. The nominal risk premium is the nominal risky return minus the nominal risk-free rate (which includes both compensation for expected inflation and the real risk-free interest rate). The real risk premium is the real risky return minus the real risk-free rate. Worldwide equity risk premiums reported at the bottom of Exhibit 9 show that equities outperformed bonds. Investors in equities earned a higher return than investors in bonds because of the higher risk in equities.

A more dramatic representation of the risk-return trade-off is shown in Exhibit 8, which shows the cumulative returns of US asset classes in real terms. The line representing T-bills is much less volatile than the other lines. Adjusted for inflation, the average real return on T-bills was 0.8 percent per year. The line representing bonds is more volatile than the line for T-bills but less volatile than the line representing stocks. The total return for equities including dividends and capital gains shows how $\$ 1$ invested at the beginning of 1900 grows to $\$ 1,654$, generating an annualized return of 6.5 percent in real terms.

Over long periods of time, we observe that higher risk does result in higher mean returns. Thus, it is reasonable to claim that, over the long term, market prices reward higher risk with higher returns, which is a characteristic of a risk-averse investor, a topic that we discuss in Section 9.

\section{OTHER INVESTMENT CHARACTERISTICS}
describe characteristics of the major asset classes that investors
consider in forming portfolios

In evaluating investments using only the mean (expected return) and variance (risk), we are implicitly making two important assumptions: 1 ) that the returns are normally distributed and can be fully characterized by their means and variances and 2) that markets are not only informationally efficient but that they are also operationally efficient. To the extent that these assumptions are violated, we need to consider additional investment characteristics. These are discussed below.

\section{Distributional Characteristics}
As explained in an earlier reading, a normal distribution has three main characteristics: its mean and median are equal; it is completely defined by two parameters, mean and variance; and it is symmetric around its mean with:

\begin{itemize}
  \item 68 percent of the observations within $\pm 1 \sigma$ of the mean,

  \item 95 percent of the observations within $\pm 2 \sigma$ of the mean, and

  \item 99 percent of the observations within $\pm 3 \sigma$ of the mean.

\end{itemize}

Using only mean and variance would be appropriate to evaluate investments if returns were distributed normally. Returns, however, are not normally distributed; deviations from normality occur both because the returns are skewed, which means they are not symmetric around the mean, and because the probability of extreme events is significantly greater than what a normal distribution would suggest. The latter deviation is referred to as kurtosis or fat tails in a return distribution. The next sections discuss these deviations more in-depth. Skewness

Skewness refers to asymmetry of the return distribution, that is, returns are not symmetric around the mean. A distribution is said to be left skewed or negatively skewed if most of the distribution is concentrated to the right, and right skewed or positively skewed if most is concentrated to the left. Exhibit 10 shows a typical representation of negative and positive skewness, whereas Exhibit 11 demonstrates the negative skewness of stock returns by plotting a histogram of US large company stock returns for $1926-2017$.

\section{Exhibit 10: Skewness}
\begin{center}
\includegraphics[max width=\textwidth]{2023_05_04_36535b8d80b32081d422g-572(1)}
\end{center}

Distribution Skewed to the Right (Positively Skewed)

\begin{center}
\includegraphics[max width=\textwidth]{2023_05_04_36535b8d80b32081d422g-572}
\end{center}

Distribution Skewed to the Left (Negatively Skewed)

Source: Reprinted from Fixed Income Readings for the Chartered Financial Analyst Program. Copyright CFA Institute.

Exhibit 11: Histogram of US Large Company Stock Returns, 1926-2017 (Percent)

\begin{center}
\includegraphics[max width=\textwidth]{2023_05_04_36535b8d80b32081d422g-572(2)}
\end{center}

Source: 2018 SBBI Yearbook (Appendix A1)

\section{Kurtosis}
Kurtosis refers to fat tails or higher than normal probabilities for extreme returns and has the effect of increasing an asset's risk that is not captured in a mean-variance framework, as illustrated in Exhibit 12. Investors try to evaluate the effect of kurtosis by using such statistical techniques as value at risk $(\mathrm{VaR})$ and conditional tail expectations. Several market participants note that the probability and the magnitude of extreme events is underappreciated and was a primary contributing factor to the financial crisis of 2008. The higher probability of extreme negative outcomes among stock returns can also be observed in Exhibit 11 .

\section{Exhibit 12: Kurtosis}
\begin{center}
\includegraphics[max width=\textwidth]{2023_05_04_36535b8d80b32081d422g-573}
\end{center}

Standard Deviations

Source: Reprinted from Fixed Income Readings for the Chartered Financial Analyst ${ }^{\circ}$ Program. Copyright CFA Institute.

\section{Market Characteristics}
In the previous analysis, we implicitly assumed that markets are both informationally and operationally efficient. Although informational efficiency of markets is a topic beyond the purview of this reading, we should highlight certain operational limitations of the market that affect the choice of investments. One such limitation is liquidity.

The cost of trading has three main components-brokerage commission, bid-ask spread, and price impact. Liquidity affects the latter two. Stocks with low liquidity can have wide bid-ask spreads. The bid-ask spread, which is the difference between the buying price and the selling price, is incurred as a cost of trading a security. The larger the bid-ask spread, the higher the cost of trading. If a $\$ 100$ stock has a spread of 10 cents, the bid-ask spread is only 0.1 percent $(\$ 0.10 / \$ 100)$. On the other hand, if a $\$ 10$ stock has a spread of 10 cents, the bid-ask spread is 1 percent. Clearly, the $\$ 10$ stock is more expensive to trade and an investor will need to earn 0.9 percent extra to make up the higher cost of trading relative to the $\$ 100$ stock.

Liquidity also has implications for the price impact of trade. Price impact refers to how the price moves in response to an order in the market. Small orders usually have little impact, especially for liquid stocks. For example, an order to buy 100 shares of a $\$ 100$ stock with a spread of 1 cent may have no effect on the price. On the other hand, an order to buy 100,000 shares may have a significant impact on the price as the buyer has to induce more and more stockholders to tender their shares. The extent of the price impact depends on the liquidity of the stock. A stock that trades millions of shares a day may be less affected than a stock that trades only a few hundred thousand shares a day. Investors, especially institutional investors managing large sums of money, must keep the liquidity of a stock in mind when making investment decisions.

Liquidity is a bigger concern in emerging markets than in developed markets because of the smaller volume of trading in those markets. Similarly, liquidity is a more important concern in corporate bond markets and especially for bonds of lower credit quality than in equity markets because an individual corporate bond issue may not trade for several days or weeks. This certainly became apparent during the global financial crisis.

There are other market-related characteristics that affect investment decisions because they might instill greater confidence in the security or might affect the costs of doing business. These include analyst coverage, availability of information, firm size, etc. These characteristics about companies and financial markets are essential components of investment decision making.

\section{RISK AVERSION AND PORTFOLIO SELECTION}
\section{explain risk aversion and its implications for portfolio selection}
As we have seen, stocks, bonds, and T-bills provide different levels of returns and have different levels of risk. Although investment in equities may be appropriate for one investor, another investor may not be inclined to accept the risk that accompanies a share of stock and may prefer to hold more cash. In the last section, we considered investment characteristics of assets in understanding their risk and return. In this section, we consider the characteristics of investors, both individual and institutional, in an attempt to pair the right kind of investors with the right kind of investments.

First, we discuss risk aversion and utility theory. Later we discuss their implications for portfolio selection.

\section{The Concept of Risk Aversion}
The concept of risk aversion is related to the behavior of individuals under uncertainty. Assume that an individual is offered two alternatives: one where he will get $\pounds 50$ for sure and the other is a gamble with a 50 percent chance that he gets $\pounds 100$ and 50 percent chance that he gets nothing. The expected value in both cases is $\pounds 50$, one with certainty and the other with uncertainty. What will an investor choose? There are three possibilities: an investor chooses the gamble, the investor chooses $\pounds 50$ with certainty, or the investor is indifferent. Let us consider each in turn. However, please understand that this is only a representative example, and a single choice does not determine the risk aversion of an investor.

\section{Risk Seeking}
If an investor chooses the gamble, then the investor is said to be risk loving or risk seeking. The gamble has an uncertain outcome, but with the same expected value as the guaranteed outcome. Thus, an investor choosing the gamble means that the investor gets extra "utility" from the uncertainty associated with the gamble. How much is that extra utility worth? Would the investor be willing to accept a smaller expected value because he gets extra utility from risk? Indeed, risk seekers will accept less return because of the risk that accompanies the gamble. For example, a risk seeker may choose a gamble with an expected value of $\pounds 45$ in preference to a guaranteed outcome of $\pounds 50$.

There is a little bit of gambling instinct in many of us. People buy lottery tickets although the expected value is less than the money they pay to buy it. Or people gamble at casinos with the full knowledge that the expected return is negative, a characteristic of risk seekers. These or any other isolated actions, however, cannot be taken at face value except for compulsive gamblers.

\section{Risk Neutral}
If an investor is indifferent about the gamble or the guaranteed outcome, then the investor may be risk neutral. Risk neutrality means that the investor cares only about return and not about risk, so higher return investments are more desirable even if they come with higher risk. Many investors may exhibit characteristics of risk neutrality when the investment at stake is an insignificant part of their wealth. For example, a billionaire may be indifferent about choosing the gamble or a $\pounds 50$ guaranteed outcome.

\section{Risk Averse}
If an investor chooses the guaranteed outcome, he/she is said to be risk averse because the investor does not want to take the chance of not getting anything at all. Depending on the level of aversion to risk, an investor may be willing to accept a guaranteed outcome of $\pounds 45$ instead of a gamble with an expected value of $\pounds 50$.

In general, investors are likely to shy away from risky investments for a lower, but guaranteed return. That is why they want to minimize their risk for the same amount of return, and maximize their return for the same amount of risk. The risk-return trade-off discussed earlier is an indicator of risk aversion. A risk-neutral investor would maximize return irrespective of risk and a risk-seeking investor would maximize both risk and return.

Data presented in the last section illustrate the historically positive relationship between risk and return, which demonstrates that market prices were based on transactions and investments by risk-averse investors and reflect risk aversion. Therefore, for all practical purposes and for our future discussion, we will assume that the representative investor is a risk-averse investor. This assumption is the standard approach taken in the investment industry globally.

\section{Risk Tolerance}
Risk tolerance refers to the amount of risk an investor can tolerate to achieve an investment goal. The higher the risk tolerance, the greater is the willingness to take risk. Thus, risk tolerance is negatively related to risk aversion.

\section{UTILITY THEORY AND INDIFFERENCE CURVES}
$\square \quad$ explain risk aversion and its implications for portfolio selection

Continuing with our previous example, a risk-averse investor would rank the guaranteed outcome of $\pounds 50$ higher than the uncertain outcome with an expected value of $\pounds 50$. We can say that the utility that an investor or an individual derives from the guaranteed outcome of $\pounds 50$ is greater than the utility or satisfaction or happiness he/ she derives from the alternative. In general terms, utility is a measure of relative satisfaction from consumption of various goods and services or in the case of investments, the satisfaction that an investor derives from a portfolio.

Because individuals are different in their preferences, all risk-averse individuals may not rank investment alternatives in the same manner. Consider the $\pounds 50$ gamble again. All risk-averse individuals will rank the guaranteed outcome of $\pounds 50$ higher than the gamble. What if the guaranteed outcome is only $\pounds 40$ ? Some risk-averse investors might consider $\pounds 40$ inadequate, others might accept it, and still others may now be indifferent about the uncertain $\pounds 50$ and the certain $\pounds 40$.

A simple implementation of utility theory allows us to quantify the rankings of investment choices using risk and return. There are several assumptions about individual behavior that we make in the definition of utility given in the equation below. We assume that investors are risk averse. They always prefer more to less (greater return to lesser return). They are able to rank different portfolios in the order of their preference and that the rankings are internally consistent. If an individual prefers $\mathrm{X}$ to $\mathrm{Y}$ and $\mathrm{Y}$ to $\mathrm{Z}$, then he/she must prefer $\mathrm{X}$ to $\mathrm{Z}$. This property implies that the indifference curves (see Exhibit 13) for the same individual can never touch or intersect. An example of a utility function is given below

$$
U=E(r)-\frac{1}{2} A \sigma^{2}
$$

where, $U$ is the utility of an investment, $E(r)$ is the expected return, and $\sigma^{2}$ is the variance of the investment.

In the above equation, $A$ is a measure of risk aversion, which is measured as the marginal reward that an investor requires to accept additional risk. More risk-averse investors require greater compensation for accepting additional risk. Thus, $A$ is higher for more risk-averse individuals. As was mentioned previously, a risk-neutral investor would maximize return irrespective of risk and a risk-seeking investor would maximize both risk and return.

We can draw several conclusions from the utility function. First, utility is unbounded on both sides. It can be highly positive or highly negative. Second, higher return contributes to higher utility. Third, higher variance reduces the utility but the reduction in utility gets amplified by the risk aversion coefficient, $A$. Utility can always be increased, albeit marginally, by getting higher return or lower risk. Fourth, utility does not indicate or measure satisfaction itself-it can be useful only in ranking various investments. For example, a portfolio with a utility of 4 is not necessarily two times better than a portfolio with a utility of 2 . The portfolio with a utility of 4 could increase our happiness 10 times or just marginally. But we do prefer a portfolio with a utility of 4 to a portfolio with a utility of 2 . Utility cannot be compared among individuals or investors because it is a very personal concept. From a societal point of view, by the same argument, utility cannot be summed among individuals.

Let us explore the utility function further. The risk aversion coefficient, $A$, is greater than zero for a risk-averse investor. So any increase in risk reduces his/her utility. The risk aversion coefficient for a risk-neutral investor is 0 , and changes in risk do not affect his/her utility. For a risk lover, the risk aversion coefficient is negative, creating an inverse situation so that additional risk contributes to an increase in his/her utility. Note that a risk-free asset $\left(\sigma^{2}=0\right)$ generates the same utility for all individuals.

\section{Indifference Curves}
An indifference curve plots the combinations of risk-return pairs that an investor would accept to maintain a given level of utility (i.e., the investor is indifferent about the combinations on any one curve because they would provide the same level of overall utility). Indifference curves are thus defined in terms of a trade-off between expected rate of return and variance of the rate of return. Because an infinite number of combinations of risk and return can generate the same utility for the same investor, indifference curves are continuous at all points.

\section{Exhibit 13: Indifference Curves for Risk-Averse Investors}
\begin{center}
\includegraphics[max width=\textwidth]{2023_05_04_36535b8d80b32081d422g-577}
\end{center}

A set of indifference curves is plotted in Exhibit 13. By definition, all points on any one of the three curves have the same utility. An investor does not care whether he/ she is at Point $\mathbf{a}$ or Point $\mathbf{b}$ on indifference Curve 1. Point $\mathbf{a}$ has lower risk and lower return than Point $\mathbf{b}$, but the utility of both points is the same because the higher return at Point $\mathbf{b}$ is offset by the higher risk.

Like Curve 1, all points on Curve 2 have the same utility and an investor is indifferent about where he/she is on Curve 2. Now compare Point $\mathbf{c}$ with Point $\mathbf{b}$. Point c has the same risk but significantly lower return than Point $\mathbf{b}$, which means that the utility at Point $\mathbf{c}$ is less than the utility at Point $\mathbf{b}$. Given that all points on Curve 1 have the same utility and all points on Curve 2 have the same utility and Point $\mathbf{b}$ has higher utility than Point $\mathbf{c}$, Curve 1 has higher utility than Curve 2. Therefore, a risk-averse investor with indifference Curves 1 and 2 will prefer Curve 1 to Curve 2 . The utility of a risk-averse investor always increases as you move northwest-higher return with lower risk. Because all investors prefer more utility to less, investors want to move northwest to the indifference curve with the highest utility.

The indifference curve for risk-averse investors runs from the southwest to the northeast because of the risk-return trade-off. If risk increases (going east) then it must be compensated by higher return (going north) to generate the same utility. The indifference curves are convex because of diminishing marginal utility of return (or wealth). As risk increases, an investor needs greater return to compensate for higher risk at an increasing rate (i.e., the curve gets steeper). The upward-sloping convex indifference curve has a slope coefficient closely related to the risk aversion coefficient. The greater the slope, the higher is the risk aversion of the investor as a greater increment in return is required to accept a given increase in risk. Indifference curves for investors with different levels of risk aversion are plotted in Exhibit 14. The most risk-averse investor has an indifference curve with the greatest slope. As volatility increases, this investor demands increasingly higher returns to compensate for risk. The least risk-averse investor has an indifference curve with the least slope and so the demand for higher return as risk increases is not as acute as for the more risk-averse investor. The risk-loving investor's indifference curve, however, exhibits a negative slope, implying that the risk-lover is happy to substitute risk for return. For a risk lover, the utility increases both with higher risk and higher return. Finally, the indifference curves of risk-neutral investors are horizontal because the utility is invariant with risk.

\section{Exhibit 14: Indifference Curves for Various Types of Investors}
\begin{center}
\includegraphics[max width=\textwidth]{2023_05_04_36535b8d80b32081d422g-578}
\end{center}

In the remaining parts of this reading, all investors are assumed to be risk averse unless stated otherwise.

\section{EXAMPLE 6}
\section{Comparing a Gamble with a Guaranteed Outcome}
Assume that you are given an investment with an expected return of 10 percent and a risk (standard deviation) of 20 percent, and your risk aversion coefficient is 3 .

\begin{enumerate}
  \item What is your utility of this investment?
\end{enumerate}

\section{Solution}
$U=0.10-0.5 \times 3 \times 0.20^{2}=0.04$ 2. What must be the minimum risk-free return you should earn to get the same utility?

\section{Solution}
A risk-free return's $\sigma$ is zero, so the second term disappears. To get the same utility (0.04), the risk-free return must be at least 4 percent. Thus, in your mind, a risky return of 10 percent is equivalent to a risk-free return or a guaranteed outcome of 4 percent.

\section{EXAMPLE 7}
\section{Computation of Utility}
Based on investment information given below and the utility formula $U=E(r)$ $-0.5 A \sigma^{2}$, answer the following questions. Returns and standard deviations are both expressed as percent per year. When using the utility formula, however, returns and standard deviations must be expressed in decimals.

\begin{center}
\begin{tabular}{lcc}
\hline
Investment & Expected Return $\boldsymbol{E}(\boldsymbol{r})$ & Standard Deviation $\boldsymbol{\sigma}$ \\
\hline
1 & $12 \%$ & $30 \%$ \\
2 & 15 & 35 \\
3 & 21 & 40 \\
4 & 24 & 45 \\
\hline
\end{tabular}
\end{center}

\begin{enumerate}
  \item Which investment will a risk-averse investor with a risk aversion coefficient of 4 choose, and which investment will a risk-averse investor with a risk aversion coefficient of 2 choose?
\end{enumerate}

\section{Solution}
The utility for risk-averse investors with $A=4$ and $A=2$ for each of the four investments are shown in the following table. Complete calculations for Investment 1 with $A=4$ are as follows: $U=0.12-0.5 \times 4 \times 0.30^{2}=-0.06$.

\begin{center}
\begin{tabular}{lcccc}
\hline
Investment & $\begin{array}{c}\text { Expected } \\ \text { Return } \boldsymbol{E}(\boldsymbol{r})\end{array}$ & $\begin{array}{c}\text { Standard } \\ \text { Deviation } \boldsymbol{\sigma}\end{array}$ & Utility $\boldsymbol{A}=\mathbf{4}$ & Utility $\boldsymbol{A}=\mathbf{2}$ \\
\hline
1 & $12 \%$ & $30 \%$ & -0.0600 & 0.0300 \\
2 & 15 & 35 & -0.0950 & 0.0275 \\
3 & 21 & 40 & -0.1100 & 0.0500 \\
4 & 24 & 45 & -0.1650 & 0.0375 \\
\hline
\end{tabular}
\end{center}

The risk-averse investor with a risk aversion coefficient of 4 should choose Investment 1 . The risk-averse investor with a risk aversion coefficient of 2 should choose Investment 3.

\begin{enumerate}
  \setcounter{enumi}{1}
  \item Which investment will a risk-neutral investor choose?
\end{enumerate}

\section{Solution}
A risk-neutral investor cares only about return. In other words, his risk aversion coefficient is 0 . Therefore, a risk-neutral investor will choose Investment 4 because it has the highest return. 3. Which investment will a risk-loving investor choose?

\section{Solution}
A risk-loving investor likes both higher risk and higher return. In other words, his risk aversion coefficient is negative. Therefore, a risk-loving investor will choose Investment 4 because it has the highest return and highest risk among the four investments.

APPLICATION OF UTILITY THEORY TO PORTFOLIO SELECTION

$$
\begin{aligned}
& \text { explain risk aversion and its implications for portfolio selection } \\
& \text { explain the selection of an optimal portfolio, given an investor's } \\
& \text { utility (or risk aversion) and the capital allocation line }
\end{aligned}
$$

The simplest application of utility theory and risk aversion is to a portfolio of two assets, a risk-free asset and a risky asset. The risk-free asset has zero risk and a return of $R_{f}$ The risky asset has a risk of $\sigma_{i}(>0)$ and an expected return of $E\left(R_{i}\right)$. Because the risky asset has risk that is greater than that of the risk-free asset, the expected return from the risky asset will be greater than the return from the risk-free asset, that is, $E\left(R_{i}\right)>R_{f}$

We can construct a portfolio of these two assets with a portfolio expected return, $E\left(R_{p}\right)$, and portfolio risk, $\sigma_{p}$, based on the formulas provided below. In the equations given below, $w_{1}$ is the weight in the risk-free asset and $\left(1-w_{1}\right)$ is the weight in the risky asset. Because $\sigma_{f}=0$ for the risk-free asset, the first and third terms in the formula for variance are zero leaving only the second term. We arrive at the last equation by taking the square root of both sides, which shows the expression for standard deviation for a portfolio of two assets when one asset is the risk-free asset:

$$
\begin{aligned}
& E\left(R_{p}\right)=w_{1} R_{f}+\left(1-w_{1}\right) E\left(R_{i}\right) \\
& \sigma_{P}^{2}=w_{1}^{2} \sigma_{f}^{2}+\left(1-w_{1}\right)^{2} \sigma_{i}^{2}+2 w_{1}\left(1-w_{1}\right) \rho_{12} \sigma_{f} \sigma_{i}=\left(1-w_{1}\right)^{2} \sigma_{i}^{2} \\
& \sigma_{p}=\left(1-w_{1}\right) \sigma_{i}
\end{aligned}
$$

The two-asset portfolio is drawn in Exhibit 15 by varying $w_{1}$ from 0 percent to 100 percent. The portfolio standard deviation is on the horizontal axis and the portfolio return is on the vertical axis. If only these two assets are available in the economy and the risky asset represents the market, the line in Exhibit 15 is called the capital allocation line. The capital allocation line represents the portfolios available to an investor. The equation for this line can be derived from the above two equations by rewriting the second equation as $w_{1}=1-\frac{\sigma_{p}}{\sigma_{i}}$ Substituting the value of $w_{1}$ in the equation for expected return, we get the following equation for the capital allocation line:

$$
E\left(R_{p}\right)=\left(1-\frac{\sigma_{p}}{\sigma_{i}}\right) R_{f}+\frac{\sigma_{p}}{\sigma_{i}} E\left(R_{i}\right)
$$

This equation can be rewritten in a more usable form:

$$
E\left(R_{p}\right)=R_{f}+\frac{\left(E\left(R_{i}\right)-R_{f}\right)}{\sigma_{i}} \sigma_{p}
$$

The capital allocation line has an intercept of $R_{f}$, and a slope of $\frac{\left(E\left(R_{i}\right)-R_{f}\right)}{\sigma_{i}}$, which is the additional required return for every increment in risk, and is sometimes referred to as the market price of risk.

Exhibit 15: Capital Allocation Line with Two Assets

\begin{center}
\includegraphics[max width=\textwidth]{2023_05_04_36535b8d80b32081d422g-581}
\end{center}

Portfolio Standard Deviation

Because the equation is linear, the plot of the capital allocation line is a straight line. The line begins with the risk-free asset as the leftmost point with zero risk and a risk-free return, $R_{f}$. At that point, the portfolio consists of only the risk-free asset. If 100 percent is invested in the portfolio of all risky assets, however, we have a return of $E\left(R_{i}\right)$ with a risk of $\sigma_{i}$

We can move further along the line in pursuit of higher returns by borrowing at the risk-free rate and investing the borrowed money in the portfolio of all risky assets. If 50 percent is borrowed at the risk-free rate, then $w_{1}=-0.50$ and 150 percent is placed in the risky asset, giving a return $=1.50 E\left(R_{i}\right)-0.50 R_{f}$, which is $>E\left(R_{i}\right)$ because $E\left(R_{i}\right)>R_{f}$

The line plotted in Exhibit 15 is comprised of an unlimited number of risk-return pairs or portfolios. Which one of these portfolios should be chosen by an investor? The answer lies in combining indifference curves from utility theory with the capital allocation line from portfolio theory. Utility theory gives us the utility function or the indifference curves for an individual, as in Exhibit 13, and the capital allocation line gives us the set of feasible investments. Overlaying each individual's indifference curves on the capital allocation line will provide us with the optimal portfolio for that investor. Exhibit 16 illustrates this process of portfolio selection.

\section{Exhibit 16: Portfolio Selection}
\begin{center}
\includegraphics[max width=\textwidth]{2023_05_04_36535b8d80b32081d422g-582}
\end{center}

The capital allocation line consists of the set of feasible portfolios. Points under the capital allocation line may be attainable but are not preferred by any investor because the investor can get a higher return for the same risk by moving up to the capital allocation line. Points above the capital allocation line are desirable but not achievable with available assets.

Three indifference curves for the same individual are also shown in Exhibit 16. Curve 1 is above the capital allocation line, Curve 2 is tangential to the line, and Curve 3 intersects the line at two points. Curve 1 has the highest utility and Curve 3 has the lowest utility. Because Curve 1 lies completely above the capital allocation line, points on Curve 1 are not achievable with the available assets on the capital allocation line. Curve 3 intersects the capital allocation line at two Points, $\mathbf{a}$ and $\mathbf{b}$. The investor is able to invest at either Point $\mathbf{a}$ or $\mathbf{b}$ to derive the risk-return trade-off and utility associated with Curve 3. Comparing points with the same risk, observe that Point $\mathbf{n}$ on Curve 3 has the same risk as Point $\mathbf{m}$ on Curve 2, yet Point $\mathbf{m}$ has the higher expected return. Therefore, all investors will choose Curve 2 instead of Curve 3. Curve 2 is tangential to the capital allocation line at Point $\mathbf{m}$. Point $\mathbf{m}$ is on the capital allocation line and investable. Point $\mathbf{m}$ and the utility associated with Curve 2 is the best that the investor can do because he/she cannot move to a higher utility indifference curve. Thus, we have been able to select the optimal portfolio for the investor with indifference Curves 1, 2, and 3. Point $\mathbf{m}$, the optimal portfolio for one investor, may not be optimal for another investor. We can follow the same process, however, for finding the optimal portfolio for other investors: the optimal portfolio is the point of tangency between the capital allocation line and the indifference curve for that investor. In other words, the optimal portfolio maximizes the return per unit of risk (as it is on the capital allocation line), and it simultaneously supplies the investor with the most satisfaction (utility).

As an illustration, Exhibit 17 shows two indifference curves for two different investors: Kelly with a risk aversion coefficient of 2 and Jane with a risk aversion coefficient of 4. The indifference curve for Kelly is to the right of the indifference curve for Jane because Kelly is less risk averse than Jane and can accept a higher amount of risk, i.e. has a higher tolerance for risk. Accordingly, their optimal portfolios are different: Point $\mathbf{k}$ is the optimal portfolio for Kelly and Point $\mathbf{j}$ is the optimal portfolio for Jane. In addition, for the same return, the slope of Jane's curve is higher than Kelly's suggesting that Jane needs greater incremental return as compensation for accepting an additional amount of risk compared with Kelly.

\section{Exhibit 17: Portfolio Selection for Two Investors with Various Levels of Risk}
Aversion

\begin{center}
\includegraphics[max width=\textwidth]{2023_05_04_36535b8d80b32081d422g-583}
\end{center}

\section{PORTFOLIO RISK \& PORTFOLIO OF TWO RISKY}
 ASSETScalculate and interpret the mean, variance, and covariance (or correlation) of asset returns based on historical data calculate and interpret portfolio standard deviation describe the effect on a portfolio's risk of investing in assets that are less than perfectly correlated

We have seen before that investors are risk averse and demand a higher return for a riskier investment. Therefore, ways of controlling portfolio risk without affecting return are valuable. As a precursor to managing risk, this section explains and analyzes the components of portfolio risk. In particular, it examines and describes how a portfolio consisting of assets with low correlations have the potential of reducing risk without necessarily reducing return.

\section{Portfolio of Two Risky Assets}
The return and risk of a portfolio of two assets was introduced in Sections $2-8$ of this reading. In this section, we briefly review the computation of return and extend the concept of portfolio risk and its components.

\section{Portfolio Return}
When several individual assets are combined into a portfolio, we can compute the portfolio return as a weighted average of the returns in the portfolio. The portfolio return is simply a weighted average of the returns of the individual investments, or assets. If Asset 1 has a return of 20 percent and constitutes 25 percent of the portfolio's investment, then the contribution to the portfolio return is 5 percent $(=25 \%$ of $20 \%)$. In general, if Asset $i$ has a return of $R_{i}$ and has a weight of $w_{i}$ in the portfolio, then the portfolio return, $R_{P}$, is given as:

$$
R_{P}=\sum_{i=1}^{N} w_{i} R_{i}, \quad \sum_{i=1}^{N} w_{i}=1
$$

Note that the weights must add up to 1 because the assets in a portfolio, including cash, must account for 100 percent of the investment. Also, note that these are single period returns, so there are no cash flows during the period and the weights remain constant.

When two individual assets are combined in a portfolio, we can compute the portfolio return as a weighted average of the returns of the two assets. Consider Assets 1 and 2 with weights of 25 percent and 75 percent in a portfolio. If their returns are 20 percent and 5 percent, the weighted average return $=(0.25 \times 20 \%)+(0.75 \times 5 \%)$ $=8.75 \%$. More generally, the portfolio return can be written as below, where $R_{p}$ is return of the portfolio, $w_{1}$ and $w_{2}$ are the weights of the two assets, and $R_{1}, R_{2}$ are returns on the two assets:

$$
R_{p}=w_{1} R_{1}+\left(1-w_{1}\right) R_{2}
$$

\section{Portfolio Risk}
Like a portfolio's return, we can calculate a portfolio's variance. Although the return of a portfolio is simply a weighted average of the returns of each security, this is not the case with the standard deviation of a portfolio (unless all securities are perfectly correlated-that is, correlation equals one). Variance can be expressed more generally for $N$ securities in a portfolio using the notation from the portfolio return calculation above:

$$
\begin{aligned}
& \sum_{i=1}^{N} w_{i}=1 \\
& \sigma_{P}^{2}=\operatorname{Var}\left(R_{P}\right)=\operatorname{Var}\left(\sum_{i=1}^{N} w_{i} R_{i}\right)
\end{aligned}
$$

Note that the weights must add up to 1 . The right side of the equation is the variance of the weighted average returns of individual securities. Weight is a constant, but the returns are variables whose variance is shown by $\operatorname{Var}\left(R_{i}\right)$. We can rewrite the equation as shown next. Because the covariance of an asset with itself is the variance of the asset, we can separate the variances from the covariances in the second equation:

$$
\begin{aligned}
\sigma_{P}^{2} & =\sum_{i, j=1}^{N} w_{i} w_{j} \operatorname{Cov}\left(R_{i}, R_{j}\right) \\
\sigma_{P}^{2} & =\sum_{i=1}^{N} w_{i}^{2} \operatorname{Var}\left(R_{i}\right)+\sum_{i, j=1, i \neq j}^{N} w_{i} w_{j} \operatorname{Cov}\left(R_{i}, R_{j}\right)
\end{aligned}
$$

$\operatorname{Cov}\left(R_{i}, R_{j}\right)$ is the covariance of returns, $R_{i}$ and $R_{j}$, and can be expressed as the product of the correlation between the two returns $\left(\rho_{1,2}\right)$ and the standard deviations of the two assets. Thus, $\operatorname{Cov}\left(R_{i}, R_{j}\right)=\rho_{i j} \sigma_{i} \sigma_{j}$.

For a two asset portfolio, the expression for portfolio variance simplifies to the following using covariance and then using correlation:

$$
\begin{gathered}
\sigma_{P}^{2}=w_{1}^{2} \sigma_{1}^{2}+w_{2}^{2} \sigma_{2}^{2}+2 w_{1} w_{2} \operatorname{Cov}\left(R_{1}, R_{2}\right) \\
\sigma_{P}^{2}=w_{1}^{2} \sigma_{1}^{2}+w_{2}^{2} \sigma_{2}^{2}+2 w_{1} w_{2} \rho_{12} \sigma_{1} \sigma_{2}
\end{gathered}
$$

The standard deviation of a two asset portfolio is given by the square root of the portfolio's variance:

$$
\sigma_{P}=\sqrt{w_{1}^{2} \sigma_{1}^{2}+w_{2}^{2} \sigma_{2}^{2}+2 w_{1} w_{2} \operatorname{Cov}\left(R_{1}, R_{2}\right)}
$$

or,

$$
\sigma_{P}=\sqrt{w_{1}^{2} \sigma_{1}^{2}+w_{2}^{2} \sigma_{2}^{2}+2 w_{1} w_{2} \rho_{12} \sigma_{1} \sigma_{2}}
$$

\section{EXAMPLE 8}
\section{Return and Risk of a Two-Asset Portfolio}
\begin{enumerate}
  \item Assume that as a US investor, you decide to hold a portfolio with 80 percent invested in the S\&P 500 US stock index and the remaining 20 percent in the MSCI Emerging Markets index. The expected return is 9.93 percent for the S\&P 500 and 18.20 percent for the Emerging Markets index. The risk (standard deviation) is 16.21 percent for the S\&P 500 and 33.11 percent for the Emerging Markets index. What will be the portfolio's expected return and risk given that the covariance between the S\&P 500 and the Emerging Markets index is 0.5 percent or 0.0050 ? Note that units for covariance and variance are written as $\%^{2}$ when not expressed as a fraction. These are units of measure like squared feet and the numbers themselves are not actually squared.
\end{enumerate}

Solution:

Portfolio return, $R_{P}=w_{1} R_{1}+\left(1-w_{1}\right), R_{2}=(0.80 \times 0.0993)+(0.20 \times 0.1820)$ $=0.1158=11.58 \%$.

Portfolio risk $=\sigma_{P}=\sqrt{w_{1}^{2} \sigma_{1}^{2}+w_{2}^{2} \sigma_{2}^{2}+2 w_{1} w_{2} \operatorname{Cov}\left(R_{1}, R_{2}\right)}$

$\sigma_{p}^{2}=w_{U S}^{2} \sigma_{U S}^{2}+w_{E M}^{2} \sigma_{E M}^{2}+2 w_{U S} w_{E M} \operatorname{Cov}_{U S, E M}$

$\sigma_{p}^{2}=\left(0.80^{2} \times 0.1621^{2}\right)+\left(0.20^{2} \times 0.3311^{2}\right)$

$+(2 \times 0.80 \times 0.20 \times 0.0050)$

$\sigma_{p}^{2}=0.01682+0.00439+0.00160=0.02281$

$\sigma_{p}=0.15103=15.10 \%$

The portfolio's expected return is 11.58 percent and the portfolio's risk is 15.10 percent. Look at this example closely. It shows that we can take the portfolio of a US investor invested only in the S\&P 500, combine it with a riskier portfolio consisting of emerging markets securities, and the return of the US investor increases from 9.93 percent to 11.58 percent while the risk of the portfolio actually falls from 16.21 percent to 15.10 percent. Exhibit 18 depicts how the combination of the two assets results in a superior riskreturn trade-off. Not only does the investor get a higher return, but he also gets it at a lower risk. That is the power of diversification as you will see later in this reading.

\section{Exhibit 18: Combination of Two Assets}
\begin{center}
\includegraphics[max width=\textwidth]{2023_05_04_36535b8d80b32081d422g-586}
\end{center}

\section{Covariance and Correlation}
The covariance in the formula for portfolio standard deviation can be expanded as $\operatorname{Cov}\left(R_{1}, R_{2}\right)=\rho_{12} \sigma_{1} \sigma_{2}$ where $\rho_{12}$ is the correlation between returns, $R_{1}, R_{2}$. Although covariance is important, it is difficult to interpret because it is unbounded on both sides. It is easier to understand the correlation coefficient $\left(\rho_{12}\right)$, which is bounded but provides similar information.

Correlation is a measure of the consistency or tendency for two investments to act in a similar way. The correlation coefficient, $\rho_{12}$, can be positive or negative and ranges from -1 to +1 . Consider three different values of the correlation coefficient:

\begin{itemize}
  \item $\rho_{12}=+1$ : Returns of the two assets are perfectly positively correlated. Assets 1 and 2 move together 100 percent of the time.

  \item $\rho_{12}=-1$ : Returns of the two assets are perfectly negatively correlated. Assets 1 and 2 move in opposite directions 100 percent of the time.

  \item $\rho_{12}=0$ : Returns of the two assets are uncorrelated. Movement of Asset 1 provides no prediction regarding the movement of Asset 2.

\end{itemize}

The correlation coefficient between two assets determines the effect on portfolio risk when the two assets are combined. To see how this works, consider two different values of $\rho_{12}$. You will find that portfolio risk is unaffected when the two assets are perfectly correlated $\left(\rho_{12}=+1\right)$. In other words, the portfolio's standard deviation is simply a weighted average of the standard deviations of the two assets and as such a portfolio's risk is unchanged with the addition of assets with the same risk parameters. Portfolio risk falls, however, when the two assets are not perfectly correlated $\left(\rho_{12}<\right.$ +1). Sufficiently low values of the correlation coefficient can make the portfolio riskless under certain conditions.

$$
\text { First, let } \rho_{12}=+1
$$

$$
\begin{aligned}
& \sigma_{p}^{2}=w_{1}^{2} \sigma_{1}^{2}+w_{2}^{2} \sigma_{2}^{2}+2 w_{1} w_{2} \rho_{12} \sigma_{1} \sigma_{2}=w_{1}^{2} \sigma_{1}^{2}+w_{2}^{2} \sigma_{2}^{2}+2 w_{1} w_{2} \sigma_{1} \sigma_{2} \\
&=\left(w_{1} \sigma_{1}+w_{2} \sigma_{2}\right)^{2} \\
& \sigma_{p}=w_{1} \sigma_{1}+w_{2} \sigma_{2}
\end{aligned}
$$

The first set of terms on the right side of the first equation contain the usual terms for portfolio variance. Because the correlation coefficient is equal to +1 , the right side can be rewritten as a perfect square. The third row shows that portfolio risk is a weighted average of the risks of the individual assets' risks. We showed earlier that the portfolio return is a weighted average of the assets' returns. Because both risk and return are just weighted averages of the two assets in the portfolio there is no reduction in risk when $\rho_{12}=+1$.

\section{Now let $\rho_{12}<+1$}
The above analysis showed that portfolio risk is a weighted average of asset risks when $\rho_{12}=+1$. When $\rho_{12}<+1$, the portfolio risk is less than the weighted average of the individual assets' risks.

To show this, we begin by reproducing the general formula for portfolio risk, which is expressed by the terms to the left of the " $<$ " sign below. The term to the right of " $<$ " shows the portfolio risk when $\rho_{12}=+1$ :

$$
\begin{aligned}
& \sigma_{p}=\sqrt{w_{1}^{2} \sigma_{1}^{2}+w_{2}^{2} \sigma_{2}^{2}+2 w_{1} w_{2} \rho_{12} \sigma_{1} \sigma_{2}}<\sqrt{w_{1}^{2} \sigma_{1}^{2}+w_{2}^{2} \sigma_{2}^{2}+2 w_{1} w_{2} \sigma_{1} \sigma_{2}} \\
&=\left(w_{1} \sigma_{1}+w_{2} \sigma_{2}\right) \\
& \sigma_{p}<\left(w_{1} \sigma_{1}+w_{2} \sigma_{2}\right)
\end{aligned}
$$

The left side is smaller than the right side because the correlation coefficient on the left side for the new portfolio is $<1$. Thus, the portfolio risk is less than the weighted average of risks while the portfolio return is still a weighted average of returns.

As you can see, we have achieved diversification by combining two assets that are not perfectly correlated. For an extreme case in which $\rho_{12}=-1$ (that is, the two asset returns move in opposite directions), the portfolio can be made risk free.

\section{EXAMPLE 9}
\section{Effect of Correlation on Portfolio Risk}
Two stocks have the same return and risk (standard deviation): 10 percent return with 20 percent risk. You form a portfolio with 50 percent each of Stock 1 and Stock 2 to examine the effect of correlation on risk.

\begin{enumerate}
  \item Calculate the portfolio return and risk if the correlation is 1.0.
\end{enumerate}

\section{Solution}
$$
\begin{aligned}
& R_{1}=R_{2}=10 \%=0.10 ; \sigma_{1}=\sigma_{2}=20 \%=0.20 ; w_{1}=w_{2}=50 \% \\
& =0.50 . \text { Case } 1: \rho_{12}=+1 \\
& R_{p}=w_{1} R_{1}+w_{2} R_{2} \\
& R_{p}=(0.5 \times 0.1)+(0.5 \times 0.1)=0.10=10 \% \\
& \sigma_{p}^{2}=w_{1}^{2} \sigma_{1}^{2}+w_{2}^{2} \sigma_{2}^{2}+2 w_{1} w_{2} \sigma_{1} \sigma_{2} \rho_{12} \\
& \sigma_{p}^{2}=\left(0.5^{2} \times 0.2^{2}\right)+\left(0.5^{2} \times 0.2^{2}\right)+(2 \times 0.5 \times 0.5 \times 0.2 \times 0.2 \times 1)=0.04 \\
& \sigma_{p}=\sqrt{0.04}=0.20=20 \%
\end{aligned}
$$

This equation demonstrates the earlier point that with a correlation of 1.0 the risk of the portfolio is the same as the risk of the individual assets. 2. Calculate the portfolio return and risk if the correlation is 0.0 .

\section{Solution}
$$
\begin{aligned}
& \rho_{12}=0 \\
& R_{p}=w_{1} R_{1}+w_{2} R_{2}=0.10=10 \% \\
& \sigma_{p}^{2}=w_{1}^{2} \sigma_{1}^{2}+w_{2}^{2} \sigma_{2}^{2}+2 w_{1} w_{2} \sigma_{1} \sigma_{2} \rho_{12} \\
& \sigma_{p}^{2}=\left(0.5^{2} \times 0.2^{2}\right)+\left(0.5^{2} \times 0.2^{2}\right) \\
& \quad+(2 \times 0.5 \times 0.5 \times 0.2 \times 0.2 \times 0)=0.02 \\
& \sigma_{p}=\sqrt{0.02}=0.14=14 \%
\end{aligned}
$$

This equation demonstrates the earlier point that, when assets have correlations of less than 1.0, they can be combined in a portfolio that has less risk than either of the assets individually.

\begin{enumerate}
  \setcounter{enumi}{2}
  \item Calculate the portfolio return and risk if the correlation is -1.0 .
\end{enumerate}

\section{Solution}
$$
\begin{aligned}
& \rho_{12}=-1 \\
& R_{p}=w_{1} R_{1}+w_{2} R_{2}=0.10=10 \% \\
& \sigma_{p}^{2}=w_{1}^{2} \sigma_{1}^{2}+w_{2}^{2} \sigma_{2}^{2}+2 w_{1} w_{2} \sigma_{1} \sigma_{2} \rho_{12} \\
& \sigma_{p}^{2}=\left(0.5^{2} \times 0.2^{2}\right)+\left(0.5^{2} \times 0.2^{2}\right) \\
& \quad+(2 \times 0.5 \times 0.5 \times 0.2 \times 0.2 \times-1)=0 \\
& \sigma_{p}=0 \%
\end{aligned}
$$

This equation demonstrates the earlier point that, if the correlation of assets is low enough, in this case 100 percent negative correlation or -1.00 (exactly inversely related), a portfolio can be designed that eliminates risk. The individual assets retain their risk characteristics, but the portfolio is risk free.

\begin{enumerate}
  \setcounter{enumi}{3}
  \item Compare the return and risk of portfolios with different correlations.
\end{enumerate}

\section{Solution}
The expected return is 10 percent in all three cases; however, the returns will be more volatile in Case 1 and least volatile in Case 3 . In the first case, there is no diversification of risk (same risk as before of 20 percent) and the return remains the same. In the second case, with a correlation coefficient of 0 , we have achieved diversification of risk (risk is now 14 percent instead of 20 percent), again with the same return. In the third case with a correlation coefficient of -1 , the portfolio is risk free, although we continue to get the same return of 10 percent. This example shows the power of diversification that we expand on further in Section 14.

\section{Relationship between Portfolio Risk and Return}
The previous example illustrated the effect of correlation on portfolio risk while keeping the weights in the two assets equal and unchanged. In this section, we consider how portfolio risk and return vary with different portfolio weights and different correlations.

Asset 1 has an annual return of 7 percent and annualized risk of 12 percent, whereas Asset 2 has an annual return of 15 percent and annualized risk of 25 percent. The relationship is tabulated in Exhibit 19 for the two assets and graphically represented in Exhibit 20.

\section{Exhibit 19: Relationship between Risk and Return}
\begin{center}
\begin{tabular}{lccccc}
\hline
Weight in & Portfolio & \multicolumn{5}{r}{Portfolio Risk with Correlation of} \\
\cline { 3 - 5 }
Asset 1 (\%) & Return & $\mathbf{1 . 0}$ & $\mathbf{0 . 5}$ & $\mathbf{0 . 2}$ & $\mathbf{- 1 . 0}$ &  \\
\hline
0 & 15.0 & 25.0 & 25.0 & 25.0 & 25.0 &  \\
10 & 14.2 & 23.7 & 23.1 & 22.8 & 21.3 &  \\
20 & 13.4 & 22.4 & 21.3 & 20.6 & 17.6 &  \\
30 & 12.6 & 21.1 & 19.6 & 18.6 & 13.9 &  \\
40 & 11.8 & 19.8 & 17.9 & 16.6 & 10.2 &  \\
50 & 11.0 & 18.5 & 16.3 & 14.9 & 6.5 &  \\
60 & 10.2 & 17.2 & 15.0 & 13.4 & 2.8 &  \\
70 & 9.4 & 15.9 & 13.8 & 12.3 & 0.9 &  \\
80 & 8.6 & 14.6 & 12.9 & 11.7 & 4.6 &  \\
90 & 7.8 & 13.3 & 12.2 & 11.6 & 8.3 &  \\
100 & 7.0 & 12.0 & 12.0 & 12.0 & 12.0 &  \\
\hline
\end{tabular}
\end{center}

\section{Exhibit 20: Relationship between Risk and Return}
\begin{center}
\includegraphics[max width=\textwidth]{2023_05_04_36535b8d80b32081d422g-589}
\end{center}

The table shows the portfolio return and risk for four correlation coefficients ranging from +1.0 to -1.0 and 11 weights ranging from 0 percent to 100 percent. The portfolio return and risk are 15 percent and 25 percent, respectively, when 0 percent is invested in Asset 1 , versus 7 percent and 12 percent when 100 percent is invested in Asset 1 . The portfolio return varies with weights but is unaffected by the correlation coefficient. Portfolio risk becomes smaller with each successive decrease in the correlation coefficient, with the smallest risk when $\rho_{12}=-1$. The graph in Exhibit 20 shows that the risk-return relationship is a straight line when $\rho_{12}=+1$. As the correlation falls, the risk becomes smaller and smaller as in the table. The curvilinear nature of a portfolio of assets is recognizable in all investment opportunity sets (except at the extremes where $\rho_{12}=-1$ or +1 ).

\section{EXAMPLE 10}
\section{Portfolio of Two Assets}
Assume you are a UK investor holding a portfolio invested $60 \%$ in UK large-capitalization equities (as proxied by the FTSE 100 Index) and $40 \%$ in local medium-duration Treasury bonds ("gilts"). The expected return on the FTSE 100 is $5.5 \%$ and on the medium-duration gilts it is $0.7 \%$. The risk (standard deviation of returns) is $13.2 \%$ and $4.2 \%$, respectively. The correlation between the two assets is -0.01 .

The expected return of this portfolio is

$$
R_{p}=w_{1} \times R_{1}+\left(1-w_{1}\right) \times R_{2}=0.6 \times 0.055+0.4 \times 0.07=0.0358 \approx 3.6 \% .
$$

The risk of this portfolio is

$$
\begin{aligned}
& \sigma_{p}=\sqrt{w_{1}^{2} \sigma_{1}^{2}+w_{2}^{2} \sigma_{2}^{2}+2 \times w_{1} w_{2} \times \rho \times \sigma_{1} \sigma_{2} .} \\
& \sigma_{p}= \\
& \sqrt{\left(0.6^{2} \times 0.132^{2}\right)+\left(0.4^{2} \times 0.042^{2}\right)+2 \times 0.6 \times 0.4 \times-0.01 \times 0.132 \times 0.042} . \\
& \sigma_{p}=0.0808 \approx 8.1 \%
\end{aligned}
$$

You notice that compared with US Treasury bonds, the expected return on gilts is lower and the risk of gilts is higher. US Treasury bonds have an expected return for a US-based investor of $1.5 \%$ and a risk of $4.0 \%$. You wonder whether replacing the gilts in your portfolio with US Treasury bonds (“Treasuries") would improve the risk and return profile of your portfolio.

\begin{enumerate}
  \item Do the given risk and return assumptions for US Treasury bonds allow you as a UK-based investor to calculate the expected return and risk of your portfolio with US Treasury bonds replacing UK gilts?
\end{enumerate}

\section{Solution:}
No. The expected return and risk for Treasuries apply to a US investor, who invests in US dollars. To calculate expected return and risk in sterling for a UK-based portfolio of FTSE 100 equities and US Treasuries, one needs to take into account the exchange rate between the US dollar and UK pound sterling. This exchange rate has a volatility (risk) of its own, and a return expectation for the GBP/USD exchange rate has to be specified. For the purpose of calculating the return and risk of a foreign asset in a domestic investor's portfolio, the foreign asset can be seen as a "portfolio" of two assets. The return of a foreign asset in domestic (i.e., non-foreign) currency can be decomposed into a local currency return component and an exchange rate component: $R_{D}=\left(1+R_{l c}\right) \times\left(1+R_{F X}\right)-1$

Because the portfolio is fully exposed to the movement in both the asset's value in local currency and the currency exchange rate, the foreign currency and the asset each have a $100 \%$ portfolio weight. Note that the exchange rate must be specified as domestic currency/foreign currency to convert the foreign currency return into the investor's domestic currency. The risk can be calculated as follows:

$\sigma_{D}=$

$\sqrt{w_{1}^{2} \sigma_{l c}^{2}+w_{2}^{2} \sigma_{F X}^{2}+2 \times w_{1} w_{2} \times \rho \times \sigma_{l c} \sigma_{F X}}=\sqrt{\sigma_{l c}^{2}+\sigma_{F X}^{2}+2 \times \rho \times \sigma_{l c} \times \sigma_{F X}}$.

Assume in what follows that the risk (measured as expected standard deviation) of the GBP/USD currency exchange rate is $9.0 \%$ and the returns on Treasuries have a correlation with the GBP/USD exchange rate of 0.33 . Assume also that you have no forecast for the future value of the USD/GBP exchange rate, and hence assume a $0 \%$ return.

\begin{enumerate}
  \setcounter{enumi}{1}
  \item What would be the expected risk of US Treasuries to you as a UK investor?
\end{enumerate}

\section{Solution}
$\sigma_{D}=$

$\sqrt{\sigma_{l c}^{2}+\sigma_{F X}^{2}+2 \times \rho \times \sigma_{l c} \times \sigma_{F X}}=\sqrt{0.040^{2}+0.090^{2}+2 \times 0.33 \times 0.040+0.090}$

$\sigma_{D}=0.110=11.0 \%$

The correlations between the FTSE 100, US Treasuries, and the USD/GBP exchange rate are as depicted in the following correlation matrix.

\begin{center}
\begin{tabular}{llll}
\hline
 & FTSE 100 & US Treasuries & GBP/USD \\
\hline
FTSE 100 & 1.00 & -0.32 & -0.06 \\
US Treasuries & -0.32 & 1.00 & 0.33 \\
GBP/USD & -0.06 & 0.33 & 1.00 \\
\hline
\end{tabular}
\end{center}

\begin{enumerate}
  \setcounter{enumi}{2}
  \item What would be the expected return and risk for your portfolio if you replace the UK gilts with US Treasuries?
\end{enumerate}

\section{Solution:}
The expected return is the weighted average of the expected returns in British pound sterling (GBP) of UK large-capitalization equities and of US Treasuries. Recall that the return of a foreign asset in domestic currency consists of a foreign currency component and an asset component. All expected returns can be found above.

$$
\begin{aligned}
& R_{p}=w_{1} \times R_{1}+\left(1-w_{1}\right) \times\left[\left(1+R_{l c}\right) \times\left(1+R_{F X}\right)-1\right] \\
& R_{p}=0.6 \times 0.055+0.4 \times[(1+0.015) \times(1+0.0)-1]=0.039=3.9 \% .
\end{aligned}
$$

Calculation of the risk of the portfolio involves a slightly more complicated formula. Recall that the risk of a two-asset portfolio depends on the risk and the weights of the individual assets and the co-movements between the two. For a three-asset portfolio (an equity portion, a foreign fixed-income portion, and the associated foreign currency exposure), the calculation is es- sentially the same, however there are three pairs of co-movements between assets, rather than one.

The formula for the standard deviation of a three-asset portfolio is therefore

$$
\begin{aligned}
& \sigma_{p}= \\
& \sqrt{w_{1}^{2} \sigma_{1}^{2}+w_{2}^{2} \sigma_{2}^{2}+w_{3}^{2} \sigma_{3}^{2}+2 \rho_{1,2} w_{1} w_{2} \sigma_{1} \sigma_{2}} \\
& +2 \rho_{1,3} w_{1} w_{3} \sigma_{1} \sigma_{3}+2 \rho_{2,3} w_{2} w_{3} \sigma_{2} \sigma_{3} .
\end{aligned}
$$

The portfolio weight of the foreign currency exposure is equal to the portfolio weight of the US Treasuries.

Using the information provided above, we can calculate the risk of the portfolio with UK large-capitalization equities and US Treasuries as follows:

$$
\begin{aligned}
& \sigma_{p}=\left(0.6^{2} \times 0.132^{2}+0.4^{2} \times 0.040^{2}+0.4^{2} \times 0.090^{2}+2 \times-0.32 \times 0.6 \times 0.4 \times\right. \\
& 0.132 \times 0.040+2 \times-0.06 \times 0.6 \times 0.4 \times 0.132 \times 0.090+2 \times 0.33 \times 0.4 \times 0.4 \times \\
& 0.040 \times 0.090)^{1 / 2} . \\
& \sigma_{p}=0.0841 \approx 8.4 \% .
\end{aligned}
$$

Compared to the UK equity/gilt portfolio, the UK equity/US Treasury portfolio has a higher expected return, because the UK gilts were replaced with an asset with superior return expectations. The risk of the new portfolio, however, is slightly higher despite the lower risk in local currency terms of US Treasuries compared to gilts. Owning US Treasuries as a non-US investor means being exposed to exchange rate risk, which should be considered when evaluating the risk profile.

\section{PORTFOLIO OF MANY RISKY ASSETS}
calculate and interpret portfolio standard deviation describe the effect on a portfolio's risk of investing in assets that are less than perfectly correlated

In the previous section, we discussed how the correlation between two assets can affect the risk of a portfolio and the smaller the correlation the lower is the risk. The above analysis can be extended to a portfolio with many risky assets $(N)$. Recall the previous equations for portfolio return and variance:

$$
E\left(R_{p}\right)=\sum_{i=1}^{N} w_{i} E\left(R_{i}\right), \quad \sigma_{P}^{2}=\left(\sum_{i=1}^{N} w_{i}^{2} \sigma_{i}^{2}+\sum_{i, j=1, i \neq j}^{N} w_{i} w_{j} \operatorname{Cov}(i, j)\right), \sum_{i=1}^{N} w_{i}=1
$$

To examine how a portfolio with many risky assets works and the ways in which we can reduce the risk of a portfolio, assume that the portfolio has equal weights $(1 / N)$ for all $N$ assets. In addition, assume that $\bar{\sigma}^{2}$ and $\overline{\operatorname{Cov}}$ are the average variance and average covariance. Given equal weights and average variance/covariance, we can rewrite the portfolio variance as below (intermediate steps are omitted to focus on the main result):

$$
\begin{aligned}
& \sigma_{P}^{2}=\left(\sum_{i=1}^{N} w_{i}^{2} \sigma_{i}^{2}+\sum_{i, j=1, i \neq j}^{N} w_{i} w_{j} \operatorname{Cov}(i, j)\right) \\
& \sigma_{P}^{2}=\frac{\bar{\sigma}^{2}}{N}+\frac{(N-1) \overline{\operatorname{Cov}}}{N}
\end{aligned}
$$

The equation in the second line shows that as $N$ becomes large, the first term on the right side with the denominator of $N$ becomes smaller and smaller, implying that the contribution of one asset's variance to portfolio variance gradually becomes negligible. The second term, however, approaches the average covariance as $N$ increases. It is reasonable to say that for portfolios with a large number of assets, covariance among the assets accounts for almost all of the portfolio's risk.

\section{Importance of Correlation in a Portfolio of Many Assets}
The analysis becomes more instructive and interesting if we assume that all assets in the portfolio have the same variance and the same correlation among assets. In that case, the portfolio risk can then be rewritten as:

$$
\sigma_{p}=\sqrt{\frac{\sigma^{2}}{N}+\frac{(N-1)}{N} \rho \sigma^{2}}
$$

The first term under the root sign becomes negligible as the number of assets in the portfolio increases leaving the second term (correlation) as the main determining factor for portfolio risk. If the assets are unrelated to one another, the portfolio can have close to zero risk. In the next section, we review these concepts to learn how portfolios can be diversified.

\section{THE POWER OF DIVERSIFICATION}
describe characteristics of the major asset classes that investors consider in forming portfolios

describe the effect on a portfolio's risk of investing in assets that are less than perfectly correlated

Diversification is one of the most important and powerful concepts in investments. Because investors are risk averse, they are interested in reducing risk preferably without reducing return. In other cases, investors may accept a lower return if it will reduce the chance of catastrophic losses. In previous sections of this reading, you learned the importance of correlation and covariance in managing risk. This section applies those concepts to explore ways for risk diversification. We begin with a simple but intuitive example.

\section{EXAMPLE 11}
\section{Diversification with Rain and Shine}
Assume a company Beachwear rents beach equipment. The annual return from the company's operations is 20 percent in years with many sunny days but falls to 0 percent in rainy years with few sunny days. The probabilities of a sunny year and a rainy year are equal at 50 percent. Thus, the average return is 10 percent, with a 50 percent chance of 20 percent return and a 50 percent chance of 0 percent return. Because Beachwear can earn a return of 20 percent or 0 percent, its average return of 10 percent is risky. You are excited about investing in Beachwear but do not like the risk. Having heard about diversification, you decide to add another business to the portfolio to reduce your investment risk.

\begin{itemize}
  \item There is a snack shop on the beach that sells all the healthy food you like. You estimate that the annual return from the Snackshop is also 20 percent in years with many sunny days and 0 percent in other years.
\end{itemize}

As with the Beachwear shop, the average return is 10 percent.

You decide to invest 50 percent each in Snackshop and Beachwear. The average return is still 10 percent, with 50 percent of 10 percent from Snackshop and 50 percent of 10 percent from Beachwear. In a sunny year, you would earn 20 percent ( $=50 \%$ of $20 \%$ from Beachwear $+50 \%$ of $20 \%$ from Snackshop). In a rainy year, you would earn 0 percent $(=50 \%$ of $0 \%$ from Beachwear $+50 \%$ of $0 \%$ from Snackshop). The results are tabulated in Exhibit 21.

\section{Exhibit 21}
\begin{center}
\begin{tabular}{|c|c|c|c|c|c|}
\hline
Type & Company & $\begin{array}{l}\text { Percent } \\ \text { Invested }\end{array}$ & $\begin{array}{c}\text { Return } \\ \text { in } \\ \text { Sunny } \\ \text { Year } \\ \text { (\%) }\end{array}$ & $\begin{array}{c}\text { Return } \\ \text { in } \\ \text { Rainy } \\ \text { Year } \\ \text { (\%) }\end{array}$ & $\begin{array}{l}\text { Average } \\ \text { Return } \\ \text { (\%) }\end{array}$ \\
\hline
Single stock & Beachwear & 100 & 20 & 0 & 10 \\
\hline
Single stock & Snackshop & 100 & 20 & 0 & 10 \\
\hline
\multirow{3}{*}{$\begin{array}{l}\text { Portfolio of two } \\
\text { stocks }\end{array}$} & Beachwear & 50 & 20 & 0 & 10 \\
\hline
 & Snackshop & 50 & 20 & 0 & 10 \\
\hline
 & Total & 100 & 20 & 0 & 10 \\
\hline
\end{tabular}
\end{center}

These results seem counterintuitive. You thought that by adding another business you would be able to diversify and reduce your risk, but the risk is exactly the same as before. What went wrong? Note that both businesses do well when it is sunny and both businesses do poorly when it rains. The correlation between the two businesses is +1.0 . No reduction in risk occurs when the correlation is +1.0 .

\begin{itemize}
  \item To reduce risk, you must consider a business that does well in a rainy year. You find a company that rents DVDs. DVDrental company is similar to the Beachwear company, except that its annual return is 20 percent in a rainy year and 0 percent in a sunny year, with an average return of 10 percent. DVDrental's 10 percent return is also risky just like Beachwear's return.
\end{itemize}

If you invest 50 percent each in DVDrental and Beachwear, then the average return is still 10 percent, with 50 percent of 10 percent from DVDrental and 50 percent of 10 percent from Beachwear. In a sunny year, you would earn 10 percent $(=50 \%$ of $20 \%$ from Beachwear $+50 \%$ of $0 \%$ from DVDrental). In a rainy year also, you would earn 10 percent $(=50 \%$ of $0 \%$ from Beachwear + $50 \%$ of $20 \%$ from DVDrental). You have no risk because you earn 10 percent in both sunny and rainy years. Thus, by adding DVDrental to Beachwear, you have reduced (eliminated) your risk without affecting your return. The results are tabulated in Exhibit 22.

\section{Exhibit 22}
\begin{center}
\begin{tabular}{|c|c|c|c|c|c|}
\hline
Type & Company & $\begin{array}{l}\text { Percent } \\ \text { Invested }\end{array}$ & $\begin{array}{l}\text { Return } \\ \text { in } \\ \text { Sunny } \\ \text { Year } \\ \text { (\%) }\end{array}$ & $\begin{array}{l}\text { Return } \\ \text { in } \\ \text { Rainy } \\ \text { Year } \\ \text { (\%) }\end{array}$ & $\begin{array}{l}\text { Average } \\ \text { Return } \\ \text { (\%) }\end{array}$ \\
\hline
Single stock & Beachwear & 100 & 20 & 0 & 10 \\
\hline
Single stock & DVDrental & 100 & 0 & 20 & 10 \\
\hline
\multirow{3}{*}{$\begin{array}{l}\text { Portfolio of two } \\
\text { stocks }\end{array}$} & Beachwear & 50 & 20 & 0 & 10 \\
\hline
 & DVDrental & 50 & 0 & 20 & 10 \\
\hline
 & Total & 100 & 10 & 10 & 10 \\
\hline
\end{tabular}
\end{center}

In this case, the two businesses have a correlation of -1.0 . When two businesses with a correlation of -1.0 are combined, risk can always be reduced to zero.

\section{Correlation and Risk Diversification}
Correlation is the key in diversification of risk. Notice that the returns from Beachwear and DVDrental always go in the opposite direction. If one of them does well, the other does not. Therefore, adding assets that do not behave like other assets in your portfolio is good and can reduce risk. The two companies in the above example have a correlation of -1.0 .

Even when we expand the portfolio to many assets, correlation among assets remains the primary determinant of portfolio risk. Lower correlations are associated with lower risk. Unfortunately, most assets have high positive correlations. The challenge in diversifying risk is to find assets that have a correlation that is much lower than +1.0 .

\section{Historical Risk and Correlation}
When we previously discussed asset returns, we were careful to distinguish between historical or past returns and expected or future returns because historical returns may not be a good indicator of future returns. Returns may be highly positive in one period and highly negative in another period depending on the risk of that asset. Exhibit 7 showed that returns for large US company stocks were high in the 1990s but were very low in the 2000s.

Risk for an asset class, however, does not usually change dramatically from one period to the next. Stocks have been risky even in periods of low returns. T-bills are always less risky even when they earn high returns. From Exhibit 7, we can see that risk has typically not varied much from one decade to the next, except that risk for bonds has been much higher in recent decades when compared with earlier decades. Therefore, it is not unreasonable to assume that historical risk can work as a good proxy for future risk.

As with risk, correlations are quite stable among assets of the same country Intercountry correlations, however, have been on the rise in the last few decades as a result of globalization and the liberalization of many economies. A correlation above 0.90 is considered high because the assets do not provide much opportunity for diversification of risk Low correlations-generally less than 0.50-are desirable for portfolio diversification.

\section{Historical Correlation among Asset Classes}
Correlations among major US asset classes and international stocks are reported in Exhibit 23 for 1970-2017. The highest correlation is between US large company stocks and US small company stocks at about 70 percent, whereas the correlation between US large company stocks and international stocks is approximately 66 percent. Although these are the highest correlations, they still provide diversification benefits because the correlations are less than 100 percent. The correlation between international stocks and US small company stocks is lower, at 50 percent. The lowest correlations are between stocks and bonds, with some correlations being negative, such as that between US small company stocks and US long-term government bonds. Similarly, the correlation between T-bills and stocks is close to zero. ${ }^{2}$

Exhibit 23: Correlation Among US Assets and International Stocks (1970-2017)

\begin{center}
\begin{tabular}{|c|c|c|c|c|c|c|c|}
\hline
Series & $\begin{array}{c}\text { International } \\ \text { Stocks }\end{array}$ & $\begin{array}{c}\text { US Large } \\ \text { I Company } \\ \text { Stocks }\end{array}$ & $\begin{array}{c}\text { US Small } \\ \text { Company } \\ \text { Stocks }\end{array}$ & $\begin{array}{c}\text { US } \\ \text { Long-Term } \\ \text { Corporate } \\ \text { Bonds }\end{array}$ & $\begin{array}{c}\text { US } \\ \text { Long-Term } \\ \text { Treasury } \\ \text { Bonds }\end{array}$ & $\begin{array}{c}\text { US } \\ \text { T-Bills }\end{array}$ & $\begin{array}{c}\text { US } \\ \text { Inflation }\end{array}$ \\
\hline
$\begin{array}{l}\text { International } \\ \text { stocks }\end{array}$ & 1.00 &  &  &  &  &  &  \\
\hline
$\begin{array}{l}\text { US large com- } \\ \text { pany stocks }\end{array}$ & 0.66 & 1.00 &  &  &  &  &  \\
\hline
$\begin{array}{l}\text { US small com- } \\ \text { pany stocks }\end{array}$ & 0.50 & 0.72 & 1.00 &  &  &  &  \\
\hline
$\begin{array}{l}\text { US long-term } \\ \text { corporate bonds }\end{array}$ & 0.02 & 0.23 & 0.06 & 1.00 &  &  &  \\
\hline
$\begin{array}{l}\text { US long-term } \\ \text { Treasury bonds }\end{array}$ & -0.13 & 0.01 & -0.15 & 0.89 & 1.00 &  &  \\
\hline
US T-bills & 0.01 & 0.04 & 0.02 & 0.05 & 0.09 & 1.00 &  \\
\hline
US inflation & -0.06 & -0.11 & 0.04 & -0.32 & -0.26 & 0.69 & 1.00 \\
\hline
\end{tabular}
\end{center}

Source: 2018 SBBI Yearbook (Exhibit 12.13).

The low correlations between stocks and bonds are attractive for portfolio diversification. Similarly, including international securities in a portfolio can also control portfolio risk. It is not surprising that most diversified portfolios of investors contain domestic stocks, domestic bonds, foreign stocks, foreign bonds, real estate, cash, and other asset classes.

\section{Avenues for Diversification}
The reason for diversification is simple. By constructing a portfolio with assets that do not move together, you create a portfolio that reduces the ups and downs in the short term but continues to grow steadily in the long term. Diversification thus makes a portfolio more resilient to gyrations in financial markets.

2 In any short period, T-bills are riskless and uncorrelated with other asset classes. For example, a 3-month US Treasury bill is redeemable at its face value upon maturity irrespective of what happens to other assets. When we consider multiple periods, however, returns on T-bills may be related to other asset classes because short-term interest rates vary depending on the strength of the economy and outlook for inflation. We describe a number of approaches for diversification, some of which have been discussed previously and some of which might seem too obvious. Diversification, however, is such an important part of investing that it cannot be emphasized enough, especially when we continue to meet and see many investors who are not properly diversified.

\begin{itemize}
  \item Diversify with asset classes. Correlations among major asset classes ${ }^{3}$ are not usually high, as can be observed from the few US asset classes listed in Exhibit 23. Correlations for other asset classes and other countries are also typically low, which provides investors the opportunity to benefit from diversifying among many asset classes to achieve the biggest benefit from diversification. A partial list of asset classes includes domestic large caps, domestic small caps, growth stocks, value stocks, domestic corporate bonds, long-term domestic government bonds, domestic Treasury bills (cash), emerging market stocks, emerging market bonds, developed market stocks (i.e., developed markets excluding domestic market), developed market bonds, real estate, and gold and other commodities. In addition, industries and sectors are used to diversify portfolios. For example, energy stocks may not be well correlated with health care stocks. The exact proportions in which these assets should be included in a portfolio depend on the risk, return, and correlation characteristics of each and the home country of the investor.

  \item Diversify with index funds. Diversifying among asset classes can become costly for small portfolios because of the number of securities required. For example, creating diversified exposure to a single category, such as a domestic large company asset class, may require a group of at least 30 stocks.

\end{itemize}

Exposure to 10 asset classes may require 300 securities, which can be expensive to trade and track. Instead, it may be effective to use exchange-traded funds or mutual funds that track the respective indexes, which could bring down the costs associated with building a well-diversified portfolio. Therefore, many investors should consider index mutual funds as an investment vehicle as opposed to individual securities.

\begin{itemize}
  \item Diversification among countries. Countries are different because of industry focus, economic policy, and political climate. The US economy produces many financial and technical services and invests a significant amount in innovative research. The Chinese and Indian economies, however, are focused on manufacturing. Countries in the European Union are vibrant democracies whereas East Asian countries are experimenting with democracy. Thus, financial returns in one country over time are not likely to be highly correlated with returns in another country. Country returns may also be different because of different currencies. In other words, the return on a foreign investment may be different when translated to the home country's currency. Because currency returns are uncorrelated with stock returns, they may help reduce the risk of investing in a foreign country even when that country, in isolation, is a very risky emerging market from an equity investment point of view. Investment in foreign countries is an essential part of a well-diversified portfolio.

  \item Diversify by not owning your employer's stock. Companies encourage their employees to invest in company stock through employee stock plans and retirement plans. You should evaluate investing in your company, however, just as you would evaluate any other investment. In addition, you should

\end{itemize}

3 Major asset classes are distinguished from sub-classes, such as US value stocks and US growth stocks. consider the nonfinancial investments that you have made, especially the human capital you have invested in your company. Because you work for your employer, you are already heavily invested in it-your earnings depend on your employer. The level of your earnings, whether your compensation improves or whether you get a promotion, depends on how well your employer performs. If a competitor drives your employer out of the market, you will be out of a job. Additional investments in your employer will concentrate your wealth in one asset even more so and make you less diversified.

\begin{itemize}
  \item Evaluate each asset before adding to a portfolio. Every time you add a security or an asset class to the portfolio, recognize that there is a cost associated with diversification. There is a cost of trading an asset as well as the cost of tracking a larger portfolio. In some cases, the securities or assets may have different names but belong to an asset class in which you already have sufficient exposure. A general rule to evaluate whether a new asset should be included to an existing portfolio is based on the following risk-return trade-off relationship:
\end{itemize}

$$
E\left(R_{n e w}\right)=R_{f}+\frac{\sigma_{\text {new }} \rho_{\text {new }, p}}{\sigma_{p}} \times\left[E\left(R_{p}\right)-R_{f}\right]
$$

where $E(R)$ is the return from the asset, $R_{f}$ is the return on the risk-free asset, $\sigma$ is the standard deviation, $\rho$ is the correlation coefficient, and the subscripts new and $p$ refer to the new stock and existing portfolio. If the new asset's risk-adjusted return benefits the portfolio, then the asset should be included. The condition can be rewritten using the Sharpe ratio on both sides of the equation as:

$$
\frac{E\left(R_{\text {new }}\right)-R_{f}}{\sigma_{\text {new }}}>\frac{E\left(R_{p}\right)-R_{f}}{\sigma_{p}} \times \rho_{\text {new }, p}
$$

If the Sharpe ratio of the new asset is greater than the Sharpe ratio of the current portfolio times the correlation coefficient, it is beneficial to add the new asset.

\begin{itemize}
  \item Buy insurance for risky portfolios. It may come as a surprise, but insurance is an investment asset-just a different kind of asset. Insurance has a negative correlation with your assets and is thus very valuable. Insurance gives you a positive return when your assets lose value, but pays nothing if your assets maintain their value. Over time, insurance generates a negative average return. Many individuals, however, are willing to accept a small negative return because insurance reduces their exposure to an extreme loss. In general, it is reasonable to add an investment with a negative return if that investment significantly reduces risk (an example of a classic case of the risk-return trade-off).
\end{itemize}

Alternatively, investments with negative correlations also exist. Historically, gold has a negative correlation with stocks; however, the expected return is usually small and sometimes even negative. Investors often include gold and other commodities in their portfolios as a way of reducing their overall portfolio risk, including currency risk and inflation risk.

Buying put options is another way of reducing risk. Because put options pay when the underlying asset falls in value (negative correlation), they can protect an investor's portfolio against catastrophic losses. Of course, put options cost money, and the expected return is zero or marginally negative.

\textbackslash section\{EFFICIENT FRONTIER: INVESTMENT OPPORTUNITY SET \& MINIMUM VARIANCE PORTFOLIOS

describe the effect on a portfolio's risk of investing in assets that are less than perfectly correlated describe and interpret the minimum-variance and efficient frontiers of risky assets and the global minimum-variance portfolio

In this section, we formalize the effect of diversification and expand the set of investments to include all available risky assets in a mean-variance framework. The addition of a risk-free asset generates an optimal risky portfolio and the capital allocation line. We can then derive an investor's optimal portfolio by overlaying the capital allocation line with the indifference curves of investors.

\section{Investment Opportunity Set}
If two assets are perfectly correlated, the risk-return opportunity set is represented by a straight line connecting those two assets. The line contains portfolios formed by changing the weight of each asset invested in the portfolio. This correlation was depicted by the straight line (with $\rho=1$ ) in Exhibit 20. If the two assets are not perfectly correlated, the portfolio's risk is less than the weighted average risk of the components, and the portfolio formed from the two assets bulges on the left as shown by curves with the correlation coefficient ( $\rho$ ) less than 1.0 in Exhibit 20. All of the points connecting the two assets are achievable (or feasible). The addition of new assets to this portfolio creates more and more portfolios that are either a linear combination of the existing portfolio and the new asset or a curvilinear combination, depending on the correlation between the existing portfolio and the new asset.

As the number of available assets increases, the number of possible combinations increases rapidly. When all investable assets are considered, and there are hundreds and thousands of them, we can construct an opportunity set of investments. The opportunity set will ordinarily span all points within a frontier because it is also possible to reach every possible point within that curve by judiciously creating a portfolio from the investable assets.

We begin with individual investable assets and gradually form portfolios that can be plotted to form a curve as shown in Exhibit 24. All points on the curve and points to the right of the curve are attainable by a combination of one or more of the investable assets. This set of points is called the investment opportunity set. Initially, the opportunity set consists of domestic assets only and is labeled as such in Exhibit 24 .

\section{Exhibit 24: Investment Opportunity Set}
\begin{center}
\includegraphics[max width=\textwidth]{2023_05_04_36535b8d80b32081d422g-600}
\end{center}

\section{Addition of Asset Classes}
Exhibit 24 shows the effect of adding a new asset class, such as international assets. As long as the new asset class is not perfectly correlated with the existing asset class, the investment opportunity set will expand out further to the northwest, providing a superior risk-return trade-off.

The investment opportunity set with international assets dominates the opportunity set that includes only domestic assets. Adding other asset classes will have the same impact on the opportunity set. Thus, we should continue to add asset classes until they do not further improve the risk-return trade-off. The benefits of diversification can be fully captured in this way in the construction of the investment opportunity set, and eventually in the selection of the optimal portfolio.

In the discussion that follows in this section, we will assume that all investable assets available to an investor are included in the investment opportunity set and no special attention needs to be paid to new asset classes or new investment opportunities.

\section{Minimum-Variance Portfolios}
The investment opportunity set consisting of all available investable sets is shown in Exhibit 25. There are a large number of portfolios available for investment, but we must choose a single optimal portfolio. In this subsection, we begin the selection process by narrowing the choice to fewer portfolios.

\section{Exhibit 25: Minimum-Variance Frontier}
\begin{center}
\includegraphics[max width=\textwidth]{2023_05_04_36535b8d80b32081d422g-601}
\end{center}

\section{Minimum-Variance Frontier}
Risk-averse investors seek to minimize risk for a given return. Consider Points $\mathrm{A}, \mathrm{B}$, and $\mathrm{X}$ in Exhibit 25 and assume that they are on the same horizontal line by construction. Thus, the three points have the same expected return, $E\left(R_{1}\right)$, as do all other points on the imaginary line connecting $A, B$, and $X$. Given a choice, an investor will choose the point with the minimum risk, which is Point $X$. Point $X$, however, is unattainable because it does not lie within the investment opportunity set. Thus, the minimum risk that we can attain for $E\left(R_{1}\right)$ is at Point $\mathrm{A}$. Point $\mathrm{B}$ and all points to the right of Point A are feasible but they have higher risk. Therefore, a risk-averse investor will choose only Point A in preference to any other portfolio with the same return.

Similarly, Point $C$ is the minimum variance point for the return earned at $C$. Points to the right of $C$ have higher risk. We can extend the above analysis to all possible returns. In all cases, we find that the minimum-variance portfolio is the one that lies on the solid curve drawn in Exhibit 25 . The entire collection of these minimum-variance portfolios is referred to as the minimum-variance frontier. The minimum-variance frontier defines the smaller set of portfolios in which investors would want to invest. Note that no risk-averse investor will choose to invest in a portfolio to the right of the minimum-variance frontier because a portfolio on the minimum-variance frontier can give the same return but at a lower risk.

\section{Global Minimum-Variance Portfolio}
The left-most point on the minimum-variance frontier is the portfolio with the minimum variance among all portfolios of risky assets, and is referred to as the global minimum-variance portfolio. An investor cannot hold a portfolio consisting of risky assets that has less risk than that of the global minimum-variance portfolio. Note the emphasis on "risky" assets. Later, the introduction of a risk-free asset will allow us to relax this constraint.

\section{Efficient Frontier of Risky Assets}
The minimum-variance frontier gives us portfolios with the minimum variance for a given return. However, investors also want to maximize return for a given risk. Observe Points $\mathrm{A}$ and $\mathrm{C}$ on the minimum-variance frontier shown in Exhibit 25. Both of them have the same risk. Given a choice, an investor will choose Portfolio A because it has a higher return. No one will choose Portfolio C. The same analysis applies to all points on the minimum-variance frontier that lie below the global minimum-variance portfolio. Thus, portfolios on the curve below the global minimum-variance portfolio and to the right of the global minimum-variance portfolio are not beneficial and are inefficient portfolios for an investor.

The curve that lies above and to the right of the global minimum-variance portfolio is referred to as the Markowitz efficient frontier because it contains all portfolios of risky assets that rational, risk-averse investors will choose.

An important observation that is often ignored is the slope at various points on the efficient frontier. As we move right from the global minimum-variance portfolio (Point $\mathrm{Z}$ ) in Exhibit 25, there is an increase in risk with a concurrent increase in return. The increase in return with every unit increase in risk, however, keeps decreasing as we move from left to the right because the slope continues to decrease. The slope at Point $\mathrm{D}$ is less than the slope at Point $\mathrm{A}$, which is less than the slope at Point $\mathrm{Z}$. The increase in return by moving from Point $\mathrm{Z}$ to Point $\mathrm{A}$ is the same as the increase in return by moving from Point $\mathrm{A}$ to Point $\mathrm{D}$. It can be seen that the additional risk in moving from Point $A$ to Point $\mathrm{D}$ is 3 to 4 times more than the additional risk in moving from Point $\mathrm{Z}$ to Point A. Thus, investors obtain decreasing increases in returns as they assume more risk.

\section{EFFICIENT FRONTIER: A RISK-FREE ASSET AND MANY RISKY ASSETS}
explain the selection of an optimal portfolio, given an investor's utility (or risk aversion) and the capital allocation line

Until now, we have only considered risky assets in which the return is risky or uncertain. Most investors, however, have access to a risk-free asset, most notably from securities issued by the government. The addition of a risk-free asset makes the investment opportunity set much richer than the investment opportunity set consisting only of risky assets.

\section{Capital Allocation Line and Optimal Risky Portfolio}
By definition, a risk-free asset has zero risk so it must lie on the $y$-axis in a mean-variance graph. A risk-free asset with a return of $R_{f}$ is plotted in Exhibit 26. This asset can now be combined with a portfolio of risky assets. The combination of a risk-free asset with a portfolio of risky assets is a straight line, such as in Section 11 (see Exhibit 15). Unlike in Section 11, however, we have many risky portfolios to choose from instead of a single risky portfolio.

\section{Exhibit 26: Optimal Risky Portfolio}
\begin{center}
\includegraphics[max width=\textwidth]{2023_05_04_36535b8d80b32081d422g-603}
\end{center}

Portfolio Standard Deviation

All portfolios on the efficient frontier are candidates for being combined with the risk-free asset. Two combinations are shown in Exhibit 26: one between the risk-free asset and efficient Portfolio $A$ and the other between the risk-free asset and efficient Portfolio P. Comparing capital allocation line A and capital allocation line P reveals that there is a point on $\operatorname{CAL}(\mathrm{P})$ with a higher return and same risk for each point on CAL(A). In other words, the portfolios on CAL(P) dominate the portfolios on CAL(A). Therefore, an investor will choose CAL(P) over CAL(A). We would like to move further northwest to achieve even better portfolios. None of those portfolios, however, is attainable because they are above the efficient frontier.

What about other points on the efficient frontier? For example, Point $\mathrm{X}$ is on the efficient frontier and has the highest return of all risky portfolios for its risk. However, Point $\mathrm{Y}$ on CAL(P), achievable by leveraging Portfolio $\mathrm{P}$ as seen in Section 11, lies above Point $X$ and has the same risk but higher return. In the same way, we can observe that not only does $\operatorname{CAL}(\mathrm{P})$ dominate $\operatorname{CAL}(\mathrm{A})$ but it also dominates the Markowitz efficient frontier of risky assets.

$\operatorname{CAL}(\mathrm{P})$ is the optimal capital allocation line and Portfolio $\mathrm{P}$ is the optimal risky portfolio. Thus, with the addition of the risk-free asset, we are able to narrow our selection of risky portfolios to a single optimal risky portfolio, P, which is at the tangent of CAL(P) and the efficient frontier of risky assets.

\section{The Two-Fund Separation Theorem}
The two-fund separation theorem states that all investors regardless of taste, risk preferences, and initial wealth will hold a combination of two portfolios or funds: a risk-free asset and an optimal portfolio of risky assets. ${ }^{10}$

The separation theorem allows us to divide an investor's investment problem into two distinct steps: the investment decision and the financing decision. In the first step, as in the previous analysis, the investor identifies the optimal risky portfolio. The optimal risky portfolio is selected from numerous risky portfolios without considering the investor's preferences. The investment decision at this step is based on the optimal risky portfolio's (a single portfolio) return, risk, and correlations. The capital allocation line connects the optimal risky portfolio and the risk-free asset. All optimal investor portfolios must be on this line. Each investor's optimal portfolio on the CAL(P) is determined in the second step. Considering each individual investor's risk preference, using indifference curves, determines the investor's allocation to the risk-free asset (lending) and to the optimal risky portfolio. Portfolios beyond the optimal risky portfolio are obtained by borrowing at the risk-free rate (i.e., buying on margin). Therefore, the individual investor's risk preference determines the amount of financing (i.e., lending to the government instead of investing in the optimal risky portfolio or borrowing to purchase additional amounts of the optimal risky portfolio).

\section{EXAMPLE 12}
\section{Choosing the Right Portfolio}
In Exhibit 27, the risk and return of the points marked are as follows:

\begin{center}
\begin{tabular}{lccccc}
\hline
Point & Return (\%) & Risk (\%) & Point (\%) & Return (\%) & Risk (\%) \\
\hline
A & 15 & 10 & B & 11 & 10 \\
C & 15 & 30 & D & 25 & 30 \\
F & 4 & 0 & G (gold) & 10 & 30 \\
P & 16 & 17 &  &  &  \\
\hline
\end{tabular}
\end{center}

\section{Exhibit 27}
\begin{center}
\includegraphics[max width=\textwidth]{2023_05_04_36535b8d80b32081d422g-604}
\end{center}

Answer the following questions with reference to the points plotted on Exhibit 27 and explain your answers. The investor is choosing one portfolio based on the graph.

\begin{enumerate}
  \item Which of the above points is not achievable?
\end{enumerate}

\section{Solution:}
Portfolio A is not attainable because it lies outside the feasible set and not on the capital allocation line. 2. Which of these portfolios will not be chosen by a rational, risk-averse investor?

\section{Solution:}
Portfolios G and C will not be chosen because D provides higher return for the same risk. $\mathrm{G}$ and $\mathrm{C}$ are the only investable points that do not lie on the capital allocation line.

\begin{enumerate}
  \setcounter{enumi}{2}
  \item Which of these portfolios is most suitable for a risk-neutral investor?
\end{enumerate}

\section{Solution:}
Portfolio D is most suitable because a risk-neutral investor cares only about return and portfolio $\mathrm{D}$ provides the highest return. $\mathrm{A}=0$ in the utility formula.

\begin{enumerate}
  \setcounter{enumi}{3}
  \item Gold is on the inefficient part of the feasible set. Nonetheless, gold is owned by many rational investors as part of a larger portfolio. Why?
\end{enumerate}

\section{Solution:}
Gold may be owned as part of a portfolio (not as the portfolio) because gold has low or negative correlation with many risky assets, such as stocks. Being part of a portfolio can thus reduce overall risk even though its standalone risk is high and return is low. Note that gold's price is not stable-its return is very risky (30 percent). Even risk seekers will choose D over G, which has the same risk but higher return.

\begin{enumerate}
  \setcounter{enumi}{4}
  \item What is the utility of an investor at point $\mathrm{P}$ with a risk aversion coefficient of 3 ?
\end{enumerate}

\section{Solution:}
$\mathrm{U}=E(r)-0.5 \mathrm{~A} \sigma^{2}=0.16-0.5 \times 3 \times 0.0289=0.1167=11.67 \%$.

\section{EFFICIENT FRONTIER: OPTIMAL INVESTOR PORTFOLIO}
explain the selection of an optimal portfolio, given an investor's utility (or risk aversion) and the capital allocation line

The CAL(P) in Exhibit 28 contains the best possible portfolios available to investors. Each of those portfolios is a linear combination of the risk-free asset and the optimal risky portfolio. Among the available portfolios, the selection of each investor's optimal portfolio depends on the risk preferences of an investor. In Sections 9-11, we discussed that the individual investor's risk preferences are incorporated into their indifference curves. These can be used to select the optimal portfolio.

Exhibit 28 shows an indifference curve that is tangent to the capital allocation line, CAL(P). Indifference curves with higher utility than this one lie above the capital allocation line, so their portfolios are not achievable. Indifference curves that lie below this one are not preferred because they have lower utility. Thus, the optimal portfolio for the investor with this indifference curve is portfolio $\mathrm{C}$ on $\mathrm{CAL}(\mathrm{P})$, which is tangent to the indifference curve.

\section{Exhibit 28: Optimal Investor Portfolio}
\begin{center}
\includegraphics[max width=\textwidth]{2023_05_04_36535b8d80b32081d422g-606}
\end{center}

\section{EXAMPLE 13}
\section{Comprehensive Example on Portfolio Selection}
This comprehensive example reviews many concepts learned in this reading. The example begins with simple information about available assets and builds an optimal investor portfolio for the Lohrmanns.

Suppose the Lohrmanns can invest in only two risky assets, A and B. The expected return and standard deviation for asset $\mathrm{A}$ are 20 percent and 50 percent, and the expected return and standard deviation for asset $B$ are 15 percent and 33 percent. The two assets have zero correlation with one another.

\begin{enumerate}
  \item Calculate portfolio expected return and portfolio risk (standard deviation) if an investor invests 10 percent in A and the remaining 90 percent in B.
\end{enumerate}

\section{Solution}
The subscript " $r p$ " means risky portfolio.

$$
\begin{aligned}
& R_{r p}=[0.10 \times 20 \%]+[(1-0.10) \times 15 \%]=0.155=15.50 \% \\
& \sigma_{r p}=\sqrt{w_{A}^{2} \sigma_{A}^{2}+w_{B}^{2} \sigma_{B}^{2}+2 w_{A} w_{B} \rho_{A B} \sigma_{A} \sigma_{B}} \\
= & \sqrt{\left(0.10^{2} \times 0.50^{2}\right)+\left(0.90^{2} \times 0.33^{2}\right)+(2 \times 0.10 \times 0.90 \times 0.0 \times 0.50 \times 0.33)} \\
= & 0.3012=30.12 \%
\end{aligned}
$$

Note that the correlation coefficient is 0 , so the last term for standard deviation is zero.

\begin{enumerate}
  \setcounter{enumi}{1}
  \item Generalize the above calculations for portfolio return and risk by assuming an investment of $w_{A}$ in Asset A and an investment of $\left(1-w_{A}\right)$ in Asset B.
\end{enumerate}

\section{Solution}
$$
\begin{aligned}
& R_{r p}=w_{A} \times 20 \%+\left(1-w_{A}\right) \times 15 \%=0.05 w_{A}+0.15 \\
& \sigma_{r p}=\sqrt{w_{A}^{2} \times 0.5^{2}+\left(1-w_{A}\right)^{2} \times 0.33^{2}}=\sqrt{0.25 w_{A}^{2}+0.1089\left(1-2 w_{A}+w_{A}^{2}\right)} \\
&= \sqrt{0.3589 w_{A}^{2}-0.2178 w_{A}+0.1089}
\end{aligned}
$$

The investment opportunity set can be constructed by using different weights in the expressions for $E\left(R_{r p}\right)$ and $\sigma_{r p}$ in Part 1 of this example. Exhibit 29 shows the combination of Assets A and B.

\section{Exhibit 29}
\begin{center}
\includegraphics[max width=\textwidth]{2023_05_04_36535b8d80b32081d422g-607}
\end{center}

\begin{enumerate}
  \setcounter{enumi}{2}
  \item Now introduce a risk-free asset with a return of 3 percent. Write an equation for the capital allocation line in terms of $w_{A}$ that will connect the risk-free asset to the portfolio of risky assets. (Hint: use the equation in Section 11 and substitute the expressions for a risky portfolio's risk and return from Part 2 above).
\end{enumerate}

\section{Solution}
The equation of the line connecting the risk-free asset to the portfolio of risky assets is given below (see Section 11), where the subscript " $r p$ " refers to the risky portfolio instead of " $i$ ", and the subscript " $p$ " refers to the new portfolio of two risky assets and one risk-free asset.

$E\left(R_{p}\right)=R_{f}+\frac{E\left(R_{i}\right)-R_{f}}{\sigma_{i}} \sigma_{p}$

Rewritten as

$E\left(R_{p}\right)=R_{f}+\frac{E\left(R_{r p}\right)-R_{f}}{\sigma_{r p}} \sigma_{p}$

$=0.03+\frac{0.05 w_{A}+0.15-0.03}{\sqrt{0.3589 w_{A}^{2}-0.2178 w_{A}+0.1089}} \sigma_{p}$

$=0.03+\frac{0.05 w_{A}+0.12}{\sqrt{0.3589 w_{A}^{2}-0.2178 w_{A}+0.1089}} \sigma_{p}$

The capital allocation line is the line that has the maximum slope because it is tangent to the curve formed by portfolios of the two risky assets. Exhibit 30 shows the capital allocation line based on a risk-free asset added to the group of assets.

\section{Exhibit 30}
\begin{center}
\includegraphics[max width=\textwidth]{2023_05_04_36535b8d80b32081d422g-608}
\end{center}

\begin{enumerate}
  \setcounter{enumi}{3}
  \item The slope of the capital allocation line is maximized when the weight in Asset A is 38.20 percent. What is the equation for the capital allocation line using $w_{A}$ of 38.20 percent?
\end{enumerate}

\section{Solution}
By substituting 38.20 percent for $w_{A}$ in the equation in Part 3, we get $E\left(R_{p}\right)=$ $0.03+0.4978 \sigma_{p}$ as the capital allocation line.

\begin{enumerate}
  \setcounter{enumi}{4}
  \item Having created the capital allocation line, we turn to the Lohrmanns. What is the standard deviation of a portfolio that gives a 20 percent return and is on the capital allocation line? How does this portfolio compare with asset A?
\end{enumerate}

\section{Solution}
Solve the equation for the capital allocation line to get the standard deviation: $0.20=0.03+0.4978 \sigma_{p} . \sigma_{p}=34.2 \%$. The portfolio with a 20 percent return has the same return as Asset A but a lower standard deviation, 34.2 percent instead of 50.0 percent.

\begin{enumerate}
  \setcounter{enumi}{5}
  \item What is the risk of portfolios with returns of 3 percent, 9 percent, 15 percent, and 20 percent?
\end{enumerate}

\section{Solution}
You can find the risk of the portfolio using the equation for the capital allocation line: $E\left(R_{p}\right)=0.03+0.4978 \sigma_{p}$.

For a portfolio with a return of 15 percent, write $0.15=0.03+0.4978 \sigma_{p}$. Solving for $\sigma_{p}$ gives 24.1 percent. You can similarly calculate risks of other portfolios with the given returns.

The risk of the portfolio for a return of 3 percent is 0.0 percent, for a return of 9 percent is 12.1 percent, for a return of 15 percent is 24.1 percent, and for a return of 20 percent is 34.2 percent. The points are plotted in Exhibit 31.

\section{Exhibit 31}
\begin{center}
\includegraphics[max width=\textwidth]{2023_05_04_36535b8d80b32081d422g-609(1)}
\end{center}

\begin{enumerate}
  \setcounter{enumi}{6}
  \item What is the utility that the Lohrmanns derive from a portfolio with a return of 3 percent, 9 percent, 15 percent, and 20 percent? The risk aversion coefficient for the Lohrmanns is 2.5 .
\end{enumerate}

\section{Solution}
To find the utility, use the utility formula with a risk aversion coefficient of 2.5 :

Utility $=E\left(R_{p}\right)-0.5 \times 2.5 \sigma_{p}^{2}$

Utility $(3 \%)=0.0300$

Utility $(9 \%)=0.09-0.5 \times 2.5 \times 0.121^{2}=+0.0717$

Utility $(15 \%)=0.15-0.5 \times 2.5 \times 0.241^{2}=+0.0774$

Utility $(20 \%)=0.20-0.5 \times 2.5 \times 0.341^{2}=+0.0546$

Based on the above information, the Lohrmanns choose a portfolio with a return of 15 percent and a standard deviation of 24.1 percent because it has the highest utility: 0.0774. Finally, Exhibit 32 shows the indifference curve that is tangent to the capital allocation line to generate Lohrmanns' optimal investor portfolio.

\section{Exhibit 32}
\begin{center}
\includegraphics[max width=\textwidth]{2023_05_04_36535b8d80b32081d422g-609}
\end{center}

\section{Investor Preferences and Optimal Portfolios}
The location of an optimal investor portfolio depends on the investor's risk preferences. A highly risk-averse investor may invest a large proportion, even 100 percent, of his/her assets in the risk-free asset. The optimal portfolio in this investor's case will be located close to the $y$-axis. A less risk-averse investor, however, may invest a large portion of his/her wealth in the optimal risky asset. The optimal portfolio in this investor's case will lie closer to Point $\mathrm{P}$ in Exhibit 28.

Some less risk-averse investors (i.e., with a high risk tolerance) may wish to accept even more risk because of the chance of higher return. Such an investor may borrow money to invest more in the risky portfolio. If the investor borrows 25 percent of his wealth, he/she can invest 125 percent in the optimal risky portfolio. The optimal investor portfolio for such an investor will lie to the right of Point $\mathrm{P}$ on the capital allocation line.

Thus, moving from the risk-free asset along the capital allocation line, we encounter investors who are willing to accept more risk. At Point P, the investor is 100 percent invested in the optimal risky portfolio. Beyond Point P, the investor accepts even more risk by borrowing money and investing in the optimal risky portfolio.

Note that we are able to accommodate all types of investors with just two portfolios: the risk-free asset and the optimal risky portfolio. Exhibit 28 is also an illustration of the two-fund separation theorem. Portfolio $\mathrm{P}$ is the optimal risky portfolio that is selected without regard to investor preferences. The optimal investor portfolio is selected on the capital allocation line by overlaying the indifference curves that incorporate investor preferences.

\section{SUMMARY}
This reading provides a description and computation of investment characteristics, such as risk and return, that investors use in evaluating assets for investment. This was followed by sections about portfolio construction, selection of an optimal risky portfolio, and an understanding of risk aversion and indifference curves. Finally, the tangency point of the indifference curves with the capital allocation line allows identification of the optimal investor portfolio. Key concepts covered in the reading include the following:

\begin{itemize}
  \item Holding period return is most appropriate for a single, predefined holding period.

  \item Multiperiod returns can be aggregated in many ways. Each return computation has special applications for evaluating investments.

  \item Risk-averse investors make investment decisions based on the risk-return trade-off, maximizing return for the same risk, and minimizing risk for the same return. They may be concerned, however, by deviations from a normal return distribution and from assumptions of financial markets' operational efficiency.

  \item Investors are risk averse, and historical data confirm that financial markets price assets for risk-averse investors.

  \item The risk of a two-asset portfolio is dependent on the proportions of each asset, their standard deviations and the correlation (or covariance) between the assets' returns. As the number of assets in a portfolio increases, the correlation among asset risks becomes a more important determinate of portfolio risk. - Combining assets with low correlations reduces portfolio risk.

  \item The two-fund separation theorem allows us to separate decision making into two steps. In the first step, the optimal risky portfolio and the capital allocation line are identified, which are the same for all investors. In the second step, investor risk preferences enable us to find a unique optimal investor portfolio for each investor.

  \item The addition of a risk-free asset creates portfolios that are dominant to portfolios of risky assets in all cases except for the optimal risky portfolio.

\end{itemize}

By successfully understanding the content of this reading, you should be comfortable calculating an investor's optimal portfolio given the investor's risk preferences and universe of investable assets available.

\section{PRACTICE PROBLEMS}
\begin{enumerate}
  \item An investor purchased 100 shares of a stock for $\$ 34.50$ per share at the beginning of the quarter. If the investor sold all of the shares for $\$ 30.50$ per share after receiving a $\$ 51.55$ dividend payment at the end of the quarter, the holding period return is closest to:
A. $-13.0 \%$.
B. $-11.6 \%$.
C. $-10.1 \%$.

  \item An analyst obtains the following annual rates of return for a mutual fund:

\end{enumerate}

\begin{center}
\begin{tabular}{lc}
\hline
Year & Return (\%) \\
\hline
2008 & 14 \\
2009 & -10 \\
2010 & -2 \\
\hline
\end{tabular}
\end{center}

The fund's holding period return over the three-year period is closest to:
A. $0.18 \%$.
B. $0.55 \%$.
C. $0.67 \%$.

\begin{enumerate}
  \setcounter{enumi}{2}
  \item An analyst observes the following annual rates of return for a hedge fund:
\end{enumerate}

\begin{center}
\begin{tabular}{lc}
\hline
Year & Return (\%) \\
\hline
2008 & 22 \\
2009 & -25 \\
2010 & 11 \\
\hline
\end{tabular}
\end{center}

The hedge fund's annual geometric mean return is closest to:
A. $0.52 \%$.
B. $1.02 \%$.
C. $2.67 \%$.

\begin{enumerate}
  \setcounter{enumi}{3}
  \item Which of the following return calculating methods is best for evaluating the annualized returns of a buy-and-hold strategy of an investor who has made annual deposits to an account for each of the last five years?
\end{enumerate}

A. Geometric mean return.

B. Arithmetic mean return.

C. Money-weighted return.

\begin{enumerate}
  \setcounter{enumi}{4}
  \item An investor performs the following transactions on the shares of a firm.
\end{enumerate}

\begin{itemize}
  \item At $t=0$, she purchases a share for $\$ 1,000$. - At $t=1$, she receives a dividend of $\$ 25$ and then purchases three additional shares for $\$ 1,055$ each.

  \item At $t=2$, she receives a total dividend of $\$ 100$ and then sells the four shares for $\$ 1,100$ each.

\end{itemize}

The money-weighted rate of return is closest to:
A. $4.5 \%$.
B. $6.9 \%$.
C. $7.3 \%$.

\begin{enumerate}
  \setcounter{enumi}{5}
  \item A fund receives investments at the beginning of each year and generates returns as shown in the table.
\end{enumerate}

\begin{center}
\begin{tabular}{lcc}
\hline
Year of Investment & $\begin{array}{c}\text { Assets Under Management } \\ \text { at the beginning of each } \\ \text { year }\end{array}$ & $\begin{array}{c}\text { Return during Year of } \\ \text { Investment }\end{array}$ \\
\hline
1 & $\$ 1,000$ & $15 \%$ \\
2 & $\$ 4,000$ & $14 \%$ \\
3 & $\$ 45,000$ & $-4 \%$ \\
\hline
\end{tabular}
\end{center}

Which return measure over the three-year period is negative?
A. Geometric mean return
B. Time-weighted rate of return
C. Money-weighted rate of return

\begin{enumerate}
  \setcounter{enumi}{6}
  \item At the beginning of Year 1 , a fund has $\$ 10$ million under management; it earns a return of $14 \%$ for the year. The fund attracts another $\$ 100$ million at the start of Year 2 and earns a return of $8 \%$ for that year. The money-weighted rate of return is most likely:
\end{enumerate}

A. less than the time-weighted rate of return.

B. the same as the time-weighted rate of return.

C. greater than the time-weighted rate of return.

\begin{enumerate}
  \setcounter{enumi}{7}
  \item An investor evaluating the returns of three recently formed exchange-traded funds gathers the following information:
\end{enumerate}

\begin{center}
\begin{tabular}{lcc}
\hline
ETF & Time Since Inception & Return Since Inception (\%) \\
\hline
1 & 146 days & 4.61 \\
2 & 5 weeks & 1.10 \\
3 & 15 months & 14.35 \\
\hline
\end{tabular}
\end{center}

The ETF with the highest annualized rate of return is:
A. ETF 1 .
B. ETF 2 .
C. ETF 3 .

\section{The following information relates to questions}
\section{9-10}
An analyst observes the following historic geometric returns:

\begin{center}
\begin{tabular}{lc}
\hline
Asset Class & Geometric Return (\%) \\
\hline
Equities & 8.0 \\
Corporate Bonds & 6.5 \\
Treasury bills & 2.5 \\
Inflation & 2.1 \\
\hline
\end{tabular}
\end{center}

\begin{enumerate}
  \setcounter{enumi}{8}
  \item The real rate of return for equities is closest to:
A. $5.4 \%$.
B. $5.8 \%$.
C. $5.9 \%$.

  \item The real rate of return for corporate bonds is closest to:
A. $4.3 \%$.
B. $4.4 \%$.
C. $4.5 \%$.

  \item The risk premium for equities is closest to:
A. $5.4 \%$.
B. $5.5 \%$.
C. $5.6 \%$.

  \item The risk premium for corporate bonds is closest to:
A. $3.5 \%$.
B. $3.9 \%$.
C. $4.0 \%$.

  \item With respect to trading costs, liquidity is least likely to impact the:
A. stock price.
B. bid-ask spreads.
C. brokerage commissions.

  \item Evidence of risk aversion is best illustrated by a risk-return relationship that is:
A. negative.
B. neutral.
C. positive. 15. With respect to risk-averse investors, a risk-free asset will generate a numerical utility that is:
A. the same for all individuals.
B. positive for risk-averse investors.
C. equal to zero for risk seeking investors.

  \item With respect to utility theory, the most risk-averse investor will have an indifference curve with the:
A. most convexity.
B. smallest intercept value.
C. greatest slope coefficient.

  \item With respect to an investor's utility function expressed as: $U=E(r)-\frac{1}{2} A \sigma^{2}$, which of the following values for the measure for risk aversion has the least amount of risk aversion?
A. -4 .
B. 0 .
C. 4 .

\end{enumerate}

\section{The following information relates to questions}
\section{8-19}
A financial planner has created the following data to illustrate the application of utility theory to portfolio selection:

\begin{center}
\begin{tabular}{lcc}
\hline
Investment & $\begin{array}{c}\text { Expected } \\ \text { Return (\%) }\end{array}$ & $\begin{array}{c}\text { Expected } \\ \text { Standard Deviation (\%) }\end{array}$ \\
\hline
1 & 18 & 2 \\
2 & 19 & 8 \\
3 & 20 & 15 \\
4 & 18 & 30 \\
\hline
\end{tabular}
\end{center}

\begin{enumerate}
  \setcounter{enumi}{17}
  \item A risk-neutral investor is most likely to choose:
A. Investment 1.
B. Investment 2 .
C. Investment 3 .

  \item If an investor's utility function is expressed as $U=E(r)-\frac{1}{2} A \sigma^{2}$ and the measure for risk aversion has a value of -2 , the risk-seeking investor is most likely to choose:

\end{enumerate}

A. Investment 2.

B. Investment 3. C. Investment 4.

\begin{enumerate}
  \setcounter{enumi}{19}
  \item If an investor's utility function is expressed as $U=E(r)-\frac{1}{2} A \sigma^{2}$ and the measure for risk aversion has a value of 2 , the risk-averse investor is most likely to choose:
A. Investment 1.
B. Investment 2 .
C. Investment 3 .

  \item If an investor's utility function is expressed as $U=E(r)-\frac{1}{2} A \sigma^{2}$ and the measure for risk aversion has a value of 4 , the risk-averse investor is most likely to choose:
A. Investment 1.
B. Investment 2 .
C. Investment 3 .

  \item With respect to the mean-variance portfolio theory, the capital allocation line, CAL, is the combination of the risk-free asset and a portfolio of all:
A. risky assets.
B. equity securities.
C. feasible investments.

  \item Two individual investors with different levels of risk aversion will have optimal portfolios that are:

\end{enumerate}

A. below the capital allocation line.

B. on the capital allocation line.

C. above the capital allocation line.

\begin{enumerate}
  \setcounter{enumi}{23}
  \item With respect to capital market theory, which of the following asset characteristics is least likely to impact the variance of an investor's equally weighted portfolio?
\end{enumerate}

A. Return on the asset.

B. Standard deviation of the asset.

C. Covariances of the asset with the other assets in the portfolio.

\begin{enumerate}
  \setcounter{enumi}{24}
  \item A portfolio manager creates the following portfolio:
\end{enumerate}

\begin{center}
\begin{tabular}{lcc}
\hline
Security & Security Weight (\%) & $\begin{array}{c}\text { Expected } \\ \text { Standard Deviation } \\ (\%)\end{array}$ \\
\hline
1 & 30 & 20 \\
2 & 70 & 12 \\
\hline
\end{tabular}
\end{center}

If the correlation of returns between the two securities is 0.40 , the expected standard deviation of the portfolio is closest to:

A. $10.7 \%$. B. $11.3 \%$.

C. $12.1 \%$.

\begin{enumerate}
  \setcounter{enumi}{25}
  \item A portfolio manager creates the following portfolio:
\end{enumerate}

\begin{center}
\begin{tabular}{lcc}
\hline
Security & Security Weight (\%) & $\begin{array}{c}\text { Expected } \\ \text { Standard Deviation } \\ (\%)\end{array}$ \\
\hline
1 & 30 & 20 \\
2 & 70 & 12 \\
\hline
\end{tabular}
\end{center}

If the covariance of returns between the two securities is -0.0240 , the expected standard deviation of the portfolio is closest to:
A. $2.4 \%$.
B. $7.5 \%$.
C. $9.2 \%$.

\section{The following information relates to questions}
 27-28A portfolio manager creates the following portfolio:

\begin{center}
\begin{tabular}{lcc}
\hline
Security & Security Weight (\%) & $\begin{array}{c}\text { Expected } \\ \text { Standard Deviation (\%) }\end{array}$ \\
\hline
1 & 30 & 20 \\
2 & 70 & 12 \\
\hline
\end{tabular}
\end{center}

\begin{enumerate}
  \setcounter{enumi}{26}
  \item If the standard deviation of the portfolio is $14.40 \%$, the correlation between the two securities is equal to:
A. -1.0 .
B. 0.0 .
C. 1.0 .

  \item If the standard deviation of the portfolio is $14.40 \%$, the covariance between the two securities is equal to:
A. 0.0006 .
B. 0.0240 .
C. 1.0000 .

\end{enumerate}

\section{The following information relates to questions 29-31}
A portfolio manager creates the following portfolio:

\begin{center}
\begin{tabular}{lcc}
\hline
Security & Expected Annual Return (\%) & Expected Standard Deviation (\%) \\
\hline
1 & 16 & 20 \\
2 & 12 & 20 \\
\hline
\end{tabular}
\end{center}

\begin{enumerate}
  \setcounter{enumi}{28}
  \item If the portfolio of the two securities has an expected return of $15 \%$, the proportion invested in Security 1 is:
A. $25 \%$.
B. $50 \%$.
C. $75 \%$.

  \item If the correlation of returns between the two securities is -0.15 , the expected standard deviation of an equal-weighted portfolio is closest to:
A. $13.04 \%$.
B. $13.60 \%$.
C. $13.87 \%$.

  \item If the two securities are uncorrelated, the expected standard deviation of an equal-weighted portfolio is closest to:
A. $14.00 \%$.
B. $14.14 \%$.
C. $20.00 \%$.

\end{enumerate}

\section{The following information relates to questions}
 32-33An analyst has made the following return projections for each of three possible outcomes with an equal likelihood of occurrence:

\begin{center}
\begin{tabular}{lcccc}
\hline
Asset & $\begin{array}{c}\text { Outcome 1 } \\ (\%)\end{array}$ & $\begin{array}{c}\text { Outcome 2 } \\ (\%)\end{array}$ & $\begin{array}{c}\text { Outcome 3 } \\ (\%)\end{array}$ & $\begin{array}{c}\text { Expected Return } \\ (\%)\end{array}$ \\
\hline
1 & 12 & 0 & 6 & 6 \\
2 & 12 & 6 & 0 & 6 \\
3 & 0 & 6 & 12 & 6 \\
\hline
\end{tabular}
\end{center}

\begin{enumerate}
  \setcounter{enumi}{31}
  \item If the analyst constructs two-asset portfolios that are equally-weighted, which pair of assets has the lowest expected standard deviation?
\end{enumerate}

A. Asset 1 and Asset 2. B. Asset 1 and Asset 3 .

C. Asset 2 and Asset 3 .

\begin{enumerate}
  \setcounter{enumi}{32}
  \item If the analyst constructs two-asset portfolios that are equally weighted, which pair of assets provides the least amount of risk reduction?
\end{enumerate}

A. Asset 1 and Asset 2.

B. Asset 1 and Asset 3.

C. Asset 2 and Asset 3 .

\begin{enumerate}
  \setcounter{enumi}{33}
  \item As the number of assets in an equally-weighted portfolio increases, the contribution of each individual asset's variance to the volatility of the portfolio:
A. increases.
B. decreases.
C. remains the same.

  \item With respect to an equally weighted portfolio made up of a large number of assets, which of the following contributes the most to the volatility of the portfolio?
A. Average variance of the individual assets.
B. Standard deviation of the individual assets.
C. Average covariance between all pairs of assets.

  \item The correlation between assets in a two-asset portfolio increases during a market decline. If there is no change in the proportion of each asset held in the portfolio or the expected standard deviation of the individual assets, the volatility of the portfolio is most likely to:
A. increase.
B. decrease.
C. remain the same.

  \item Which of the following statements is least accurate? The efficient frontier is the set of all attainable risky assets with the:

\end{enumerate}

A. highest expected return for a given level of risk.

B. lowest amount of risk for a given level of return.

C. highest expected return relative to the risk-free rate.

\begin{enumerate}
  \setcounter{enumi}{37}
  \item The portfolio on the minimum-variance frontier with the lowest standard deviation is:
A. unattainable.
B. the optimal risky portfolio.
C. the global minimum-variance portfolio.

  \item The set of portfolios on the minimum-variance frontier that dominates all sets of portfolios below the global minimum-variance portfolio is the:

\end{enumerate}

A. capital allocation line.

B. Markowitz efficient frontier.

C. set of optimal risky portfolios.

\begin{enumerate}
  \setcounter{enumi}{39}
  \item The dominant capital allocation line is the combination of the risk-free asset and the:
\end{enumerate}

A. optimal risky portfolio.

B. levered portfolio of risky assets.

C. global minimum-variance portfolio.

\begin{enumerate}
  \setcounter{enumi}{40}
  \item Compared to the efficient frontier of risky assets, the dominant capital allocation line has higher rates of return for levels of risk greater than the optimal risky portfolio because of the investor's ability to:
\end{enumerate}

A. lend at the risk-free rate.

B. borrow at the risk-free rate.

C. purchase the risk-free asset.

\begin{enumerate}
  \setcounter{enumi}{41}
  \item With respect to the mean-variance theory, the optimal portfolio is determined by each individual investor's:
A. risk-free rate.
B. borrowing rate.
C. risk preference.
\end{enumerate}

\section{SOLUTIONS}
\begin{enumerate}
  \item $\mathrm{C}$ is correct. $-10.1 \%$ is the holding period return, which is calculated as: $(3,050$ $-3,450+51.55) / 3,450$, which is comprised of a dividend yield of $1.49 \%=51.55 /$ $(3,450)$ and a capital loss yield of $-11.59 \%=-400 /(3,450)$.

  \item B is correct. $[(1+0.14)(1-0.10)(1-0.02)]-1=0.0055=0.55 \%$.

  \item A is correct. $[(1+0.22)(1-0.25)(1+0.11)]^{(1 / 3)}-1=1.0157^{(1 / 3)}-1=0.0052=$ $0.52 \%$

  \item A is correct. The geometric mean return compounds the returns instead of the amount invested.

  \item B is correct. Computation of the money-weighted return, $r$, requires finding the discount rate that sums the present value of cash flows to zero.

\end{enumerate}

The first step is to group net cash flows by time. For this example, we have $-\$ 1,000$ for the $t=0$ net cash flow, $-\$ 3,140=-\$ 3,165+\$ 25$ for the $t=1$ net cash flow, and $\$ 4,500=\$ 4,400+\$ 100$ for the $t=2$ net cash flow

Solving for $r$,

$\mathrm{CF}_{0}=-1,000$

$\mathrm{CF}_{1}=-3,140$

$\mathrm{CF}_{2}=+4,500$

$\frac{\mathrm{CF}_{0}}{\left(1+\mathrm{IRR}^{0}\right.}+\frac{\mathrm{CF}_{1}}{(1+\mathrm{IRR})^{1}}+\frac{\mathrm{CF}_{2}}{(1+\mathrm{IRR})^{2}}$

$=\frac{-1,000}{1}+\frac{-3,140}{(1+\text { IRR })^{1}}+\frac{4,500}{(1+\text { IRR })^{2}}=0$

results in a value of $r=6.91 \%$

\begin{enumerate}
  \setcounter{enumi}{5}
  \item $\mathrm{C}$ is correct. The money-weighted rate of return considers both the timing and amounts of investments into the fund. To calculate the money-weighted rate of return, tabulate the annual returns and investment amounts to determine the cash flows
\end{enumerate}

\begin{center}
\begin{tabular}{lccc}
\hline
Year & $\mathbf{1}$ & $\mathbf{2}$ & $\mathbf{3}$ \\
\hline
Balance from previous year & 0 & $\$ 1,150$ & $\$ 4,560$ \\
New investment & $\$ 1,000$ & $\$ 2,850$ & $\$ 40,440$ \\
Net balance at the beginning of year & $\$ 1,000$ & $\$ 4,000$ & $\$ 45,000$ \\
Investment return for the year & $15 \%$ & $14 \%$ & $-4 \%$ \\
Investment gain (loss) & $\$ 150$ & $\$ 560$ & $-\$ 1,800$ \\
Balance at the end of year & $\$ 1,150$ & $\$ 4,560$ & $\$ 43,200$ \\
\hline
\end{tabular}
\end{center}

$\mathrm{CF}_{0}=-\$ 1,000, \mathrm{CF}_{1}=-\$ 2,850, \mathrm{CF}_{2}=-\$ 40,440, \mathrm{CF}_{3}=+\$ 43,200$

Each cash inflow or outflow occurs at the end of each year. Thus, $\mathrm{CF}_{0}$ refers to the cash flow at the end of Year 0 or beginning of Year 1, and $\mathrm{CF}_{3}$ refers to the cash flow at end of Year 3 or beginning of Year 4. Because cash flows are being discounted to the present - that is, end of Year 0 or beginning of Year 1 -the period of discounting $\mathrm{CF}_{0}$ is zero whereas the period of discounting for $\mathrm{CF}_{3}$ is 3 years.

Solving for $r$

$$
\begin{aligned}
& \mathrm{CF}_{0}=-1,000 \\
& \mathrm{CF}_{1}=-2,850 \\
& \mathrm{CF}_{2}=-40,440 \\
& \mathrm{CF}_{3}=+43,200 \\
& \frac{\mathrm{CF}_{0}}{(1+\mathrm{IRR})^{0}}+\frac{\mathrm{CF}_{1}}{(1+\mathrm{IRR})^{1}}+\frac{\mathrm{CF}_{2}}{(1+\mathrm{IRR})^{2}}+\frac{\mathrm{CF}_{3}}{(1+I R R)^{3}} \\
&= \frac{-1,000}{1}+\frac{-2,850}{(1+\mathrm{IRR})^{1}}+\frac{-40,440}{(1+\mathrm{IRR})^{2}}+\frac{43,200}{\left(1+\mathrm{IRR}^{3}\right.}=0 \\
& \text { results in a value of } r=-2.22 \% .
\end{aligned}
$$

Note that $B$ is incorrect because the time-weighted rate of return (TWR) of the fund is the same as the geometric mean return of the fund and is thus positive:

$\mathrm{TWR}=\sqrt[3]{(1.15)(1.14)(0.96)}-1=7.97 \%$

\begin{enumerate}
  \setcounter{enumi}{6}
  \item A is correct. Computation of the money-weighted return, $r$, requires finding the discount rate that sums the present value of cash flows to zero. Because most of the investment came during Year 2, the measure will be biased toward the performance of Year 2. The cash flows are as follows:
\end{enumerate}

$$
\begin{aligned}
& \mathrm{CF}_{0}=-10 \\
& \mathrm{CF}_{1}=-100 \\
& \mathrm{CF}_{2}=+120.31
\end{aligned}
$$

The terminal value is determined by summing the investment returns for each period $[(10 \times 1.14 \times 1.08)+(100 \times 1.08)]$

$\frac{\mathrm{CF}_{0}}{(1+\mathrm{IRR})^{0}}+\frac{\mathrm{CF}_{1}}{(1+\mathrm{IRR})^{1}}+\frac{\mathrm{CF}_{2}}{(1+\mathrm{IRR})^{2}}$

$=\frac{-10}{1}+\frac{-100}{(1+\text { IRR })^{1}}+\frac{120.31}{(1+\text { IRR })^{2}}$

results in a value of $r=8.53 \%$.

The time-weighted return of the fund is $=\sqrt[2]{(1.14)(1.08)}-1=10.96 \%$.

\begin{enumerate}
  \setcounter{enumi}{7}
  \item B is correct. The annualized rate of return for ETF 2 is $12.05 \%=\left(1.0110^{52 / 5}\right)-1$, which is greater than the annualized rate of ETF $1,11.93 \%=\left(1.0461^{365 / 146}\right)-1$, and ETF 3, 11.32\% $=\left(1.1435^{12 / 15}\right)-1$. Despite having the lowest value for the periodic rate, ETF 2 has the highest annualized rate of return because of the reinvestment rate assumption and the compounding of the periodic rate.

  \item B is correct. $(1+0.080) /(1+0.0210)-1=5.8 \%$

  \item A is correct. $(1+0.065) /(1+0.0210)-1=4.3 \%$

  \item A is correct. $(1+0.080) /(1+0.0250)-1=5.4 \%$

  \item $\mathrm{B}$ is correct. $(1+0.0650) /(1+0.0250)-1=3.9 \%$

  \item $C$ is correct. Brokerage commissions are negotiated with the brokerage firm. A security's liquidity impacts the operational efficiency of trading costs. Specifically, liquidity impacts the bid-ask spread and can impact the stock price (if the ability to sell the stock is impaired by the uncertainty associated with being able to sell the stock).

  \item $\mathrm{C}$ is correct. Historical data over long periods of time indicate that there exists a positive risk-return relationship, which is a reflection of an investor's risk aver- sion.

  \item A is correct. A risk-free asset has a variance of zero and is not dependent on whether the investor is risk neutral, risk seeking or risk averse. That is, given that the utility function of an investment is expressed as $U=E(r)-\frac{1}{2} A \sigma^{2}$, where $A$ is the measure of risk aversion, then the $\operatorname{sign}$ of $A$ is irrelevant if the variance is zero (like that of a risk-free asset).

  \item $C$ is correct. The most risk-averse investor has the indifference curve with the greatest slope.

  \item A is correct. A negative value in the given utility function indicates that the investor is a risk seeker.

  \item $\mathrm{C}$ is correct. Investment 3 has the highest rate of return. Risk is irrelevant to a risk-neutral investor, who would have a measure of risk aversion equal to 0 . Given the utility function, the risk-neutral investor would obtain the greatest amount of utility from Investment 3.

\end{enumerate}

\begin{center}
\begin{tabular}{lccc}
\hline
Investment & $\begin{array}{c}\text { Expected } \\ \text { Return (\%) }\end{array}$ & Standard Deviation (\%) & $\begin{array}{c}\text { Expected } \\ \boldsymbol{A}=\mathbf{0}\end{array}$ \\
\hline
1 & 18 & 2 & 0.1800 \\
2 & 19 & 8 & 0.1900 \\
3 & 20 & 15 & 0.2000 \\
4 & 18 & 30 & 0.1800 \\
\hline
\end{tabular}
\end{center}

\begin{enumerate}
  \setcounter{enumi}{18}
  \item $C$ is correct. Investment 4 provides the highest utility value $(0.2700)$ for a risk-seeking investor, who has a measure of risk aversion equal to -2 .
\end{enumerate}

\begin{center}
\begin{tabular}{lccc}
\hline
Investment & $\begin{array}{c}\text { Expected } \\ \text { Return (\%) }\end{array}$ & $\begin{array}{c}\text { Expected } \\ \text { Standard Deviation (\%) }\end{array}$ & $\begin{array}{c}\text { Utility } \\ \boldsymbol{A}=-2\end{array}$ \\
\hline
1 & 18 & 2 & 0.1804 \\
2 & 19 & 8 & 0.1964 \\
3 & 20 & 15 & 0.2225 \\
4 & 18 & 30 & 0.2700 \\
\hline
\end{tabular}
\end{center}

\begin{enumerate}
  \setcounter{enumi}{19}
  \item $\mathrm{B}$ is correct. Investment 2 provides the highest utility value $(0.1836)$ for a risk-averse investor who has a measure of risk aversion equal to 2.
\end{enumerate}

\begin{center}
\begin{tabular}{lccc}
\hline
 & $\begin{array}{c}\text { Expected } \\ \text { Return (\%) }\end{array}$ & $\begin{array}{c}\text { Expected } \\ \text { Standard Deviation (\%) }\end{array}$ & $\begin{array}{c}\text { Utility } \\ \boldsymbol{A}=\mathbf{2}\end{array}$ \\
\hline
1 & 18 & 2 & 0.1796 \\
2 & 19 & 8 & 0.1836 \\
3 & 20 & 15 & 0.1775 \\
4 & 18 & 30 & 0.0900 \\
\hline
\end{tabular}
\end{center}

\begin{enumerate}
  \setcounter{enumi}{20}
  \item A is correct. Investment 1 provides the highest utility value (0.1792) for a risk-averse investor who has a measure of risk aversion equal to 4 .
\end{enumerate}

\begin{center}
\begin{tabular}{lccc}
\hline
Investment & $\begin{array}{c}\text { Expected } \\ \text { Return (\%) }\end{array}$ & $\begin{array}{c}\text { Expected } \\ \text { Standard Deviation (\%) }\end{array}$ & $\begin{array}{c}\text { Utility } \\ \boldsymbol{A}=\mathbf{4}\end{array}$ \\
\hline
1 & 18 & 2 & 0.1792 \\
2 & 19 & 8 & 0.1772 \\
3 & 20 & 15 & 0.1550 \\
4 & 18 & 30 & 0.0000 \\
\hline
\end{tabular}
\end{center}

\begin{enumerate}
  \setcounter{enumi}{21}
  \item A is correct. The CAL is the combination of the risk-free asset with zero risk and the portfolio of all risky assets that provides for the set of feasible investments. Allowing for borrowing at the risk-free rate and investing in the portfolio of all risky assets provides for attainable portfolios that dominate risky assets below the CAL.

  \item B is correct. The CAL represents the set of all feasible investments. Each investor's indifference curve determines the optimal combination of the risk-free asset and the portfolio of all risky assets, which must lie on the CAL.

  \item A is correct. The asset's returns are not used to calculate the portfolio's variance [only the assets' weights, standard deviations (or variances), and covariances (or correlations) are used].

  \item C is correct.

\end{enumerate}

$$
\begin{aligned}
& \sigma_{\text {port }}=\sqrt{w_{1}^{2} \sigma_{1}^{2}+w_{2}^{2} \sigma_{2}^{2}+2 w_{1} w_{2} \rho_{1,2} \sigma_{1} \sigma_{2}} \\
= & \sqrt{(0.3)^{2}(20 \%)^{2}+(0.7)^{2}(12 \%)^{2}+2(0.3)(0.7)(0.40)(20 \%)(12 \%)} \\
= & (0.3600 \%+0.7056 \%+0.4032 \%)^{0.5}=(1.4688 \%)^{0.5}=12.11 \%
\end{aligned}
$$

\begin{enumerate}
  \setcounter{enumi}{25}
  \item A is correct.
\end{enumerate}

$$
\begin{aligned}
& \sigma_{\text {port }}=\sqrt{w_{1}^{2} \sigma_{1}^{2}+w_{2}^{2} \sigma_{2}^{2}+2 w_{1} w_{2} \operatorname{Cov}\left(R_{1} R_{2}\right)} \\
= & \sqrt{(0.3)^{2}(20 \%)^{2}+(0.7)^{2}(12 \%)^{2}+2(0.3)(0.7)(-0.0240)} \\
= & (0.3600 \%+0.7056 \%-1.008 \%)^{0.5}=(0.0576 \%)^{0.5}=2.40 \%
\end{aligned}
$$

\begin{enumerate}
  \setcounter{enumi}{26}
  \item $\mathrm{C}$ is correct. A portfolio standard deviation of $14.40 \%$ is the weighted average, which is possible only if the correlation between the securities is equal to 1.0.

  \item B is correct. A portfolio standard deviation of $14.40 \%$ is the weighted average, which is possible only if the correlation between the securities is equal to 1.0. If the correlation coefficient is equal to 1.0 , then the covariance must equal 0.0240 , calculated as: $\operatorname{Cov}\left(R_{1}, R_{2}\right)=\rho_{12} \sigma_{1} \sigma_{2}=(1.0)(20 \%)(12 \%)=2.40 \%=0.0240$.

  \item C is correct.

\end{enumerate}

$$
\begin{aligned}
& R_{p}=w_{1} \times R_{1}+\left(1-w_{1}\right) \times R_{2} \\
& R_{p}=w_{1} \times 16 \%+\left(1-w_{1}\right) \times 12 \% \\
& 15 \%=0.75(16 \%)+0.25(12 \%)
\end{aligned}
$$

\begin{enumerate}
  \setcounter{enumi}{29}
  \item A is correct.
\end{enumerate}

$$
\begin{aligned}
& \sigma_{\text {port }}=\sqrt{w_{1}^{2} \sigma_{1}^{2}+w_{2}^{2} \sigma_{2}^{2}+2 w_{1} w_{2} \rho_{1,2} \sigma_{1} \sigma_{2}} \\
= & \sqrt{(0.5)^{2}(20 \%)^{2}+(0.5)^{2}(20 \%)^{2}+2(0.5)(0.5)(-0.15)(20 \%)(20 \%)} \\
= & (1.0000 \%+1.0000 \%-0.3000 \%)^{0.5}=(1.7000 \%)^{0.5}=13.04 \%
\end{aligned}
$$

\begin{enumerate}
  \setcounter{enumi}{30}
  \item $\mathrm{B}$ is correct.
\end{enumerate}

$$
\begin{aligned}
& \sigma_{\text {port }}=\sqrt{w_{1}^{2} \sigma_{1}^{2}+w_{2}^{2} \sigma_{2}^{2}+2 w_{1} w_{2} \rho_{1,2} \sigma_{1} \sigma_{2}} \\
= & \sqrt{(0.5)^{2}(20 \%)^{2}+(0.5)^{2}(20 \%)^{2}+2(0.5)(0.5)(0.00)(20 \%)(20 \%)} \\
= & (1.0000 \%+1.0000 \%-0.0000 \%)^{0.5}=(2.0000 \%)^{0.5}=14.14 \%
\end{aligned}
$$

\begin{enumerate}
  \setcounter{enumi}{31}
  \item $\mathrm{C}$ is correct. An equally weighted portfolio of Asset 2 and Asset 3 will have the lowest portfolio standard deviation, because for each outcome, the portfolio has the same expected return (they are perfectly negatively correlated).

  \item A is correct. An equally weighted portfolio of Asset 1 and Asset 2 has the highest level of volatility of the three pairs. All three pairs have the same expected return; however, the portfolio of Asset 1 and Asset 2 provides the least amount of risk reduction.

  \item B is correct. The contribution of each individual asset's variance (or standard deviation) to the portfolio's volatility decreases as the number of assets in the equally weighted portfolio increases. The contribution of the co-movement measures between the assets increases (i.e., covariance and correlation) as the number of assets in the equally weighted portfolio increases. The following equation for the varianceofanequallyweightedportfolioillustratesthesepoints: $\sigma_{p}^{2}=\frac{\bar{\sigma}^{2}}{N}+\frac{N-1}{N}$ $C O V=\frac{\bar{\sigma}^{2}}{N}+\frac{N-1}{N} \bar{\rho} \bar{\sigma}^{2}$

  \item $\mathrm{C}$ is correct. The co-movement measures between the assets increases (i.e., covariance and correlation) as the number of assets in the equally weighted portfolio increases. The contribution of each individual asset's variance (or standard deviation) to the portfolio's volatility decreases as the number of assets in the equally weighted portfolio increases. The following equation for the variance of an equally weighted portfolio illustrates these points:

\end{enumerate}

$\sigma_{p}^{2}=\frac{\bar{\sigma}^{2}}{N}+\frac{N-1}{N} \overline{C O V}=\frac{\bar{\sigma}^{2}}{N}+\frac{N-1}{N} \bar{\rho} \bar{\sigma}^{2}$

\begin{enumerate}
  \setcounter{enumi}{35}
  \item A is correct. Higher correlations will produce less diversification benefits provided that the other components of the portfolio standard deviation do not change (i.e., the weights and standard deviations of the individual assets).

  \item $\mathrm{C}$ is correct. The efficient frontier does not account for the risk-free rate. The efficient frontier is the set of all attainable risky assets with the highest expected return for a given level of risk or the lowest amount of risk for a given level of return.

  \item $\mathrm{C}$ is correct. The global minimum-variance portfolio is the portfolio on the minimum-variance frontier with the lowest standard deviation. Although the portfolio is attainable, when the risk-free asset is considered, the global minimum-variance portfolio is not the optimal risky portfolio.

  \item B is correct. The Markowitz efficient frontier has higher rates of return for a given level of risk. With respect to the minimum-variance portfolio, the Markowitz efficient frontier is the set of portfolios above the global minimum-variance port- folio that dominates the portfolios below the global minimum-variance portfolio.

  \item A is correct. The use of leverage and the combination of a risk-free asset and the optimal risky asset will dominate the efficient frontier of risky assets (the Markowitz efficient frontier).

  \item B is correct. The CAL dominates the efficient frontier at all points except for the optimal risky portfolio. The ability of the investor to purchase additional amounts of the optimal risky portfolio by borrowing (i.e., buying on margin) at the risk-free rate makes higher rates of return for levels of risk greater than the optimal risky asset possible.

  \item $\mathrm{C}$ is correct. Each individual investor's optimal mix of the risk-free asset and the optimal risky asset is determined by the investor's risk preference. LEARNING MODULE

\end{enumerate}

\begin{center}
\includegraphics[max width=\textwidth]{2023_05_04_36535b8d80b32081d422g-627}
\end{center}

\section{Portfolio Risk and Return: Part II}
by Vijay Singal, PhD, CFA.

Vijay Singal, PhD, CFA, is at Virginia Tech (USA).

\section{LEARNING OUTCOME}
\begin{center}
\begin{tabular}{c|l}
Mastery & The candidate should be able to: \\
\hline
$\square$ & $\begin{array}{l}\text { describe the implications of combining a risk-free asset with a } \\ \text { portfolio of risky assets } \\ \text { explain the capital allocation line (CAL) and the capital market line } \\ \text { (CML) } \\ \text { explain systematic and nonsystematic risk, including why an } \\ \text { investor should not expect to receive additional return for bearing } \\ \text { nonsystematic risk } \\ \text { explain return generating models (including the market model) and } \\ \text { their uses } \\ \text { calculate and interpret beta } \\ \text { explain the capital asset pricing model (CAPM), including its } \\ \text { assumptions, and the security market line (SML) } \\ \text { calculate and interpret the expected return of an asset using the } \\ \text { CAPM } \\ \text { describe and demonstrate applications of the CAPM and the SML } \\ \text { calculate and interpret the Sharpe ratio, Treynor ratio, } M^{2}, \text { and } \\ \text { Jensen's alpha }\end{array}$ \\
$\square$ &  \\
\end{tabular}
\end{center}

\section{INTRODUCTION}
Our objective in this reading is to identify the optimal risky portfolio for all investors by using the capital asset pricing model (CAPM). The foundation of this reading is the computation of risk and return of a portfolio and the role that correlation plays in diversifying portfolio risk and arriving at the efficient frontier. The efficient frontier and the capital allocation line consist of portfolios that are generally acceptable to all investors. By combining an investor's individual indifference curves with the market-determined capital allocation line, we are able to illustrate that the only optimal risky portfolio for an investor is the portfolio of all risky assets (i.e., the market). Additionally, we discuss the capital market line, a special case of the capital allocation line that is used for passive investor portfolios. We also differentiate between systematic and nonsystematic risk, and explain why investors are compensated for bearing systematic risk but receive no compensation for bearing nonsystematic risk. We discuss in detail the CAPM, which is a simple model for estimating asset returns based only on the asset's systematic risk. Finally, we illustrate how the CAPM allows security selection to build an optimal portfolio for an investor by changing the asset mix beyond a passive market portfolio.

The reading is organized as follows. In Section 2, we discuss the consequences of combining a risk-free asset with the market portfolio and provide an interpretation of the capital market line. Section 3 decomposes total risk into systematic and nonsystematic risk and discusses the characteristics of and differences between the two kinds of risk. We also introduce return-generating models, including the single-index model, and illustrate the calculation of beta. In Section 4, we introduce the capital asset pricing model and the security market line. Our focus on the CAPM does not suggest that the CAPM is the only viable asset pricing model. Although the CAPM is an excellent starting point, more advanced readings expand on these discussions and extend the analysis to other models that account for multiple explanatory factors. Section 5 covers several post-CAPM developments in theory. Section 6 covers measures for evaluating the performance of a portfolio which take account of risk. Section 7 covers some applications of the CAPM in portfolio construction. A summary and practice problems conclude the reading.

\section{CAPITAL MARKET THEORY: RISK-FREE AND RISKY ASSETS}
\begin{center}
\includegraphics[max width=\textwidth]{2023_05_04_36535b8d80b32081d422g-628}
\end{center}

You have learned how to combine a risk-free asset with one risky asset and with many risky assets to create a capital allocation line. In this section, we will expand our discussion of multiple risky assets and consider a special case of the capital allocation line, called the capital market line. While discussing the capital market line, we will define the market and its role in passive portfolio management. Using these concepts, we will illustrate how leveraged portfolios can enhance both risk and return.

\section{Portfolio of Risk-Free and Risky Assets}
Although investors desire an asset that produces the highest return and carries the lowest risk, such an asset does not exist. As the risk-return capital market theory illustrates, one must assume higher risk in order to earn a higher return. We can improve an investor's portfolio, however, by expanding the opportunity set of risky assets because this allows the investor to choose a superior mix of assets.

Similarly, an investor's portfolio improves if a risk-free asset is added to the mix. In other words, a combination of the risk-free asset and a risky asset can result in a better risk-return trade-off than an investment in only one type of asset because the risk-free asset has zero correlation with the risky asset. The combination is called the capital allocation line (and is depicted in Exhibit 2). Superimposing an investor's indifference curves on the capital allocation line will lead to the optimal investor portfolio.

Investors with different levels of risk aversion will choose different portfolios. Highly risk-averse investors choose to invest most of their wealth in the risk-free asset and earn low returns because they are not willing to assume higher levels of risk. Less risk-averse investors, in contrast, invest more of their wealth in the risky asset, which is expected to yield a higher return. Obviously, the higher return cannot come without higher risk, but the less risk-averse investor is willing to accept the additional risk.

\section{Combining a Risk-Free Asset with a Portfolio of Risky Assets}
We can extend the analysis of one risky asset to a portfolio of risky assets. For convenience, assume that the portfolio contains all available risky assets $(N)$, although an investor may not wish to include all of these assets in the portfolio because of the investor's specific preferences. If an asset is not included in the portfolio, its weight will be zero. The risk-return characteristics of a portfolio of $N$ risky assets are given by the following equations:

$$
\begin{aligned}
& E\left(R_{p}\right)=\sum_{i=1}^{N} w_{i} E\left(R_{i}\right) \\
& \sigma_{p}^{2}=\left(\sum_{i=1, j=1}^{N} w_{i} w_{j} \operatorname{Cov}(i, j)\right), \text { and } \sum_{i=1}^{N} w_{i}=1
\end{aligned}
$$

The expected return on the portfolio, $E\left(R_{p}\right)$, is the weighted average of the expected returns of individual assets, where $w_{i}$ is the fractional weight in asset $i$ and $R_{i}$ is the expected return of asset $i$. The risk of the portfolio $\left(\sigma_{p}\right)$, however, depends on the weights of the individual assets, the risk of the individual assets, and their interrelationships. The covariance between assets $i$ and $j, \operatorname{Cov}(i, j)$, is a statistical measure of the interrelationship between each pair of assets in the portfolio and can be expressed as follows, where $\rho_{i j}$ is the correlation between assets $i$ and $j$ and $\sigma_{i}$ is the risk of asset $i$ :

$$
\operatorname{Cov}(i, j)=\rho_{i j} \sigma_{i} \sigma_{j}
$$

Note from the equation below that the correlation of an asset with itself is 1 ; therefore:

$$
\operatorname{Cov}(i, i)=\rho_{i i} \sigma_{i} \sigma_{i}=\sigma_{i}^{2}
$$

By substituting the above expressions for covariance, we can rewrite the portfolio variance equation as

$$
\sigma_{p}^{2}=\left(\sum_{i=1}^{N} w_{i}^{2} \sigma_{i}^{2}+\sum_{i, j=1, i \neq j}^{N} w_{i} w_{j} \rho_{i j} \sigma_{i} \sigma_{j}\right)
$$

The suggestion that portfolios have lower risk than the assets they contain may seem counterintuitive. These portfolios can be constructed, however, as long as the assets in the portfolio are not perfectly correlated. As an illustration of the effect of asset weights on portfolio characteristics, consider a simple two-asset portfolio with zero weights in all other assets. Assume that Asset 1 has a return of 10 percent and a standard deviation (risk) of 20 percent. Asset 2 has a return of 5 percent and a standard deviation (risk) of 10 percent. Furthermore, the correlation between the two assets is zero. Exhibit 1 shows risks and returns for Portfolio $\mathrm{X}$ with a weight of 25 percent in Asset 1 and 75 percent in Asset 2, Portfolio $Y$ with a weight of 50 percent in each of the two assets, and Portfolio $\mathrm{Z}$ with a weight of 75 percent in Asset 1 and 25 percent in Asset 2.

\section{Exhibit 1: Portfolio Risk and Return}
\begin{center}
\begin{tabular}{lcccc}
\hline
 & $\begin{array}{c}\text { Weight in } \\ \text { Asset } \mathbf{1} \\ (\%)\end{array}$ & $\begin{array}{c}\text { Weight in } \\ \text { Asset } 2 \\ (\%)\end{array}$ & $\begin{array}{c}\text { Portfolio } \\ \text { Return } \\ (\%)\end{array}$ & $\begin{array}{c}\text { Portfolio } \\ \text { Standard } \\ \text { Deviation (\%) }\end{array}$ \\
\hline
Portfolio & 25.0 & 75.0 & 6.25 & 9.01 \\
X & 50.0 & 50.0 & 7.50 & 11.18 \\
$\mathrm{Y}$ & 75.0 & 25.0 & 8.75 & 15.21 \\
Return &  &  &  &  \\
Standard deviation & 10.0 & 5.0 &  &  \\
Correlation between Assets & 20.0 & 10.0 &  &  \\
1 and 2 &  &  &  &  \\
\hline
\end{tabular}
\end{center}

From this example we observe that the three portfolios are quite different in terms of their risk and return. Portfolio $\mathrm{X}$ has a 6.25 percent return and only 9.01 percent standard deviation, whereas the standard deviation of Portfolio $\mathrm{Z}$ is more than two-thirds higher (15.21 percent), although the return is only slightly more than one-third higher (8.75 percent). These portfolios may become even more dissimilar as other assets are added to the mix.

Consider three portfolios of risky assets, A, B, and C, as in Exhibit 2, that may have been presented to a representative investor by three different investment advisers. Each portfolio is combined with the risk-free asset to create three capital allocation lines, $\operatorname{CAL}(\mathrm{A}), \operatorname{CAL}(\mathrm{B})$, and $\operatorname{CAL}(\mathrm{C})$. The exhibit shows that Portfolio $\mathrm{C}$ is superior to the other two portfolios because it has a greater expected return for any given level of risk. As a result, an investor will choose the portfolio that lies on the capital allocation line for Portfolio C. The combination of the risk-free asset and the risky Portfolio C that is selected for an investor depends on the investor's degree of risk aversion.

Exhibit 2: Risk-Free Asset and Portfolio of Risky Assets

\begin{center}
\includegraphics[max width=\textwidth]{2023_05_04_36535b8d80b32081d422g-630}
\end{center}

Portfolio Standard Deviation $\left(\sigma_{p}\right)$

\section{Does a Unique Optimal Risky Portfolio Exist?}
We assume that all investors have the same economic expectation and thus have the same expectations of prices, cash flows, and other investment characteristics. This assumption is referred to as homogeneity of expectations. Given these investment characteristics, everyone goes through the same calculations and should arrive at the same optimal risky portfolio. Therefore, assuming homogeneous expectations, only one optimal portfolio exists. If investors have different expectations, however, they might arrive at different optimal risky portfolios. To illustrate, we begin with an expression for the price of an asset:

$$
P=\sum_{t=0}^{T} \frac{\mathrm{CF}_{t}}{\left(1+r_{t}\right)^{t}}
$$

where $C F_{t}$ is the cash flow at the end of period $t$ and $r_{t}$ is the discount rate or the required rate of return for that asset for period $t$. Period $t$ refers to all periods beginning from now until the asset ceases to exist at the end of time $T$. Because the current time is the end of period 0 , which is the same as the beginning of period 1 , there are $(T+1)$ cash flows and $(T+1)$ required rates of return. These conditions are based on the assumption that a cash flow, such as an initial investment, can occur now ( $t=$ $0)$. Ordinarily, however, $C F_{0}$ is zero.

We use the formula for the price of an asset to estimate the intrinsic value of an asset. Assume that the asset we are valuing is a share of Siemens AG which trades on Xetra. In the case of corporate stock, there is no expiration date, so $T$ could be extremely large, meaning we will need to estimate a large number of cash flows and rates of return. Fortunately, the denominator reduces the importance of distant cash flows, so it may be sufficient to estimate, say, 20 annual cash flows and 20 rates of returns. How much will Siemens earn next year and the year after next? What will the product markets Siemens operates in look like in five years' time? Different analysts and investors will have their own estimates that may be quite different from one another. Also, as we delve further into the future, more serious issues in estimating future revenue, expenses, and growth rates arise. Therefore, to assume that cash flow estimates for Siemens will vary among these investors is reasonable. In addition to the numerator (cash flows), it is also necessary to estimate the denominator, the required rates of return. We know that riskier companies will require higher returns because risk and return are positively correlated. Siemens stock is riskier than a risk-free asset, but by how much? And what should the compensation for that additional risk be? Again, it is evident that different analysts will view the riskiness of Siemens differently and, therefore, arrive at different required rates of return.

Siemens closed at $€ 111.84$ on Xetra on 31 August 2018. The traded price represents the value that a marginal investor attaches to a share of Siemens, say, corresponding to Analyst A's expectation. Analyst B may think that the price should be $€ 95$, however, and Analyst $C$ may think that the price should be $€ 125$. Given a price of $€ 111.84$, the expected returns of Siemens are quite different for the three analysts. Analyst B, who believes the price should be $€ 95$, concludes that Siemens is overvalued and may assign a weight of zero to Siemens in the recommended portfolio even though the market capitalization of Siemens was in excess of $€ 100$ billion as of the date of the quotation. In contrast, Analyst $C$, with a valuation of $€ 125$, thinks Siemens is undervalued and may significantly overweight Siemens in a portfolio.

Our discussion illustrates that analysts can arrive at different valuations that necessitate the assignment of different asset weights in a portfolio. Given the existence of many asset classes and numerous assets in each asset class, one can visualize that each investor will have his or her own optimal risky portfolio depending on his or her assumptions underlying the valuation computations. Therefore, market participants will have their own and possibly different optimal risky portfolios. If investors have different valuations of assets, then the construction of a unique optimal risky portfolio is not possible. If we make a simplifying assumption of homogeneity in investor expectations, we will have a single optimal risky portfolio as previously mentioned. Even if investors have different expectations, market prices are a proxy of what the marginal, informed investor expects, and the market portfolio becomes the base case, the benchmark, or the reference portfolio that other portfolios can be judged against. For Siemens, the market price was $€ 111.84$ per share and the market capitalization was about $€ 108$ billion. In constructing the market portfolio, Siemens's weight in the market portfolio will be equal to its market value divided by the value of all other assets included in the market portfolio.

CAPITAL MARKET THEORY: THE CAPITAL MARKET LINE

explain the capital allocation line (CAL) and the capital market line $(\mathrm{CML})$

In the previous section, we discussed how the risk-free asset could be combined with a risky portfolio to create a capital allocation line (CAL). In this section, we discuss a specific CAL that uses the market portfolio as the optimal risky portfolio and is known as the capital market line. We also discuss the significance of the market portfolio and applications of the capital market line (CML).

\section{Passive and Active Portfolios}
In the above subsection, we hypothesized three possible valuations for each share of Siemens: $€ 95$, $€ 111.84$, and $€ 125$. Which one is correct?

If the market is an informationally efficient market, the price in the market, $€ 111.84$, is an unbiased estimate of all future discounted cash flows (recall the formula for the price of an asset). In other words, the price aggregates and reflects all information that is publicly available, and investors cannot expect to earn a return that is greater than the required rate of return for that asset. If, however, the price reflects all publicly available information and there is no way to outperform the market, then there is little point in investing time and money in evaluating Siemens to arrive at your price using your own estimates of cash flows and rates of return.

In that case, a simple and convenient approach to investing is to rely on the prices set by the market. Portfolios that are based on the assumption of unbiased market prices are referred to as passive portfolios. Passive portfolios most commonly replicate and track market indexes, which are passively constructed on the basis of market prices and market capitalizations. Examples of market indexes are the S\&P 500 Index, the Nikkei 300, and the CAC 40. Passive portfolios based on market indexes are called index funds and generally have low costs because no significant effort is expended in valuing securities that are included in an index.

In contrast to passive investors' reliance on market prices and index funds, active investors may not rely on market valuations. They have more confidence in their own ability to estimate cash flows, growth rates, and discount rates. Based on these estimates, they value assets and determine whether an asset is fairly valued. In an actively managed portfolio, assets that are undervalued, or have a chance of offering above-normal returns, will have a positive weight (i.e., overweight compared to the market weight in the benchmark index), whereas other assets will have a zero weight, or even a negative weight if short selling is permitted (i.e., some assets will be underweighted compared with the market weight in the benchmark index). (Short selling is a transaction in which borrowed securities are sold with the intention to repurchase them at a lower price at a later date and return them to the lender.) This style of investing is called active investment management, and the portfolios are referred to as active portfolios. Most open-end mutual funds and hedge funds practice active investment management, and most analysts believe that active investing adds value. Whether these analysts are right or wrong is the subject of continuing debate.

\section{What Is the "Market"?}
In the previous discussion, we referred to the "market" on numerous occasions without actually defining the market. The optimal risky portfolio and the capital market line depend on the definition of the market. So what is the market?

Theoretically, the market includes all risky assets or anything that has value, which includes stocks, bonds, real estate, and even human capital. Not all assets are tradable, however, and not all tradable assets are investable. For example, the Taj Mahal in India is an asset but is not a tradable asset. Similarly, human capital is an asset that is not tradable. Moreover, assets may be tradable but not investable because of restrictions placed on certain kinds of investors. For example, all stocks listed on the Shanghai Stock Exchange are tradable. However, whereas Class A shares are listed in RMB and open to domestic investors and qualified foreign investors, Class B shares are listed in USD and open to foreign investors and domestic investors holding foreign currency dealing accounts.

If we consider all stocks, bonds, real estate assets, commodities, etc., probably hundreds of thousands of assets are tradable and investable. The "market" should contain as many assets as possible; we emphasize the word "possible" because it is not practical to include all assets in a single risky portfolio. Even though advancements in technology and interconnected markets have made it much easier to span the major equity markets, we are still not able to easily invest in other kinds of assets like bonds and real estate except in the most developed countries.

For the rest of this reading, we will define the "market" quite narrowly because it is practical and convenient to do so. Typically, a local or regional stock market index is used as a proxy for the market because of active trading in stocks and because a local or regional market is most visible to the local investors. For our purposes, we will use the S\&P 500 Index as the market's proxy. The S\&P 500 is commonly used by analysts as a benchmark for market performance throughout the United States. It contains 500 of the largest stocks that are domiciled in the United States, and these stocks are weighted by their market capitalization (price times the number of outstanding shares).

As of mid-2018, the stocks in the S\&P 500 account for approximately 80 percent of the total equity market capitalization in the United States, and because the US stock markets represent about 40 percent of the world markets, the S\&P 500 represents roughly 32 percent of worldwide publicly traded equity. Our definition of the market does not include non-US stock markets, bond markets, real estate, and many other asset classes, and therefore, "market" return and the "market" risk premium refer to US equity return and the US equity risk premium, respectively. The use of this proxy, however, is sufficient for our discussion, and is relatively easy to expand to include other tradable assets.

\section{The Capital Market Line (CML)}
A capital allocation line includes all possible combinations of the risk-free asset and an investor's optimal risky portfolio. The capital market line is a special case of the capital allocation line, where the risky portfolio is the market portfolio. The risk-free asset is a debt security with no default risk, no inflation risk, no liquidity risk, no interest rate risk, and no risk of any other kind. US Treasury bills are usually used as a proxy of the risk-free return, $R_{f}$

The S\&P 500 is a proxy of the market portfolio, which is the optimal risky portfolio. Therefore, the expected return on the risky portfolio is the expected market return, expressed as $E\left(R_{m}\right)$. The capital market line is shown in Exhibit 3, where the standard deviation $\left(\sigma_{p}\right)$, or total risk, is on the $x$-axis and expected portfolio return, $E\left(R_{p}\right)$, is on the $y$-axis. Graphically, the market portfolio is the point on the Markowitz efficient frontier where a line from the risk-free asset is tangent to the Markowitz efficient frontier. All points on the interior of the Markowitz efficient frontier are inefficient portfolios in that they provide the same level of return with a higher level of risk or a lower level of return with the same amount of risk. When plotted together, the point at which the CML is tangent to the Markowitz efficient frontier is the optimal combination of risky assets, on the basis of market prices and market capitalizations. The optimal risky portfolio is the market portfolio.

\section{Exhibit 3: Capital Market Line}
\begin{center}
\includegraphics[max width=\textwidth]{2023_05_04_36535b8d80b32081d422g-634}
\end{center}

The CML's intercept on the $y$-axis is the risk-free return $\left(R_{f}\right)$ because that is the return associated with zero risk. The CML passes through the point represented by the market return, $E\left(R_{m}\right)$. With respect to capital market theory, any point above the CML is not achievable and any point below the CML is dominated by and inferior to any point on the CML. Note that we identify the CML and CAL as lines even though they are a combination of two assets. Unlike a combination of two risky assets, which is usually not a straight line, a combination of the risk-free asset and a risky portfolio is a straight line, as illustrated below by computing the combination's risk and return.

Risk and return characteristics of the portfolio represented by the CML can be computed by using the return and risk expressions for a two-asset portfolio:

$E\left(R_{p}\right)=w_{1} R_{f}+\left(1-w_{1}\right) E\left(R_{m}\right)$,

and

$$
\sigma_{p}=\sqrt{w_{1}^{2} \sigma_{f}^{2}+\left(1-w_{1}\right)^{2} \sigma_{m}^{2}+2 w_{1}\left(1-w_{1}\right) \operatorname{Cov}\left(R_{f}, R_{m}\right)}
$$

The proportion invested in the risk-free asset is given by $w_{1}$, and the balance is invested in the market portfolio, $\left(1-w_{1}\right)$. The risk of the risk-free asset is given by $\sigma_{f}$ the risk of the market is given by $\sigma_{m}$, the risk of the portfolio is given by $\sigma_{p}$, and the covariance between the risk-free asset and the market portfolio is represented by $\operatorname{Cov}\left(R_{f} R_{m}\right)$.

By definition, the standard deviation of the risk-free asset is zero. Because its risk is zero, the risk-free asset does not co-vary or move with any other asset. Therefore, its covariance with all other assets, including the market portfolio, is zero, making the first and third terms under the square root sign zero. As a result, the portfolio return and portfolio standard deviation can be simplified and rewritten as:

$$
E\left(R_{p}\right)=w_{1} R_{f}+\left(1-w_{1}\right) E\left(R_{m}\right)
$$

and

$$
\sigma_{p}=\left(1-w_{1}\right) \sigma_{m}
$$

By substitution, we can express $E\left(R_{p}\right)$ in terms of $\sigma_{p}$. Substituting for $w_{1}$, we get:

$$
E\left(R_{p}\right)=R_{f}+\left(\frac{E\left(R_{m}\right)-R_{f}}{\sigma_{m}}\right) \times \sigma_{p}
$$

Note that the expression is in the form of a line, $y=a+b x$. The $y$-intercept is the risk-free rate, and the slope of the line referred to as the market price of risk is $\left[E\left(R_{m}\right)\right.$ $-R_{f} / \sigma_{m}$. The CML has a positive slope because the market's risky return is larger than the risk-free return. As the amount of the total investment devoted to the market increases-that is, as we move up the line-both standard deviation (risk) and expected return increase.

\section{EXAMPLE 1}
\section{Risk and Return on the CML}
Mr. Miles is a first time investor and wants to build a portfolio using only US T-bills and an index fund that closely tracks the S\&P 500 Index. The T-bills have a return of 5 percent. The S\&P 500 has a standard deviation of 20 percent and an expected return of 15 percent. 1. Draw the CML and mark the points where the investment in the market is 0 percent, 25 percent, 75 percent, and 100 percent.

\section{Solution}
We calculate the equation for the CML as $E\left(R_{p}\right)=5 \%+0.50 \times \sigma_{p}$ by substituting the given information into the general CML equation. The intercept of the line is 5 percent, and its slope is 0.50 . We can draw the CML by arbitrarily taking any two points on the line that satisfy the above equation.

Alternatively, the CML can be drawn by connecting the risk-free return of 5 percent on the $y$-axis with the market portfolio at (20 percent, 15 percent). The CML is shown in Exhibit 4.

\section{Exhibit 4: Risk and Return on the CML}
\begin{center}
\includegraphics[max width=\textwidth]{2023_05_04_36535b8d80b32081d422g-636}
\end{center}

Standard Deviation of Portfolio o

\begin{enumerate}
  \setcounter{enumi}{1}
  \item Mr. Miles is also interested in determining the exact risk and return at each point.
\end{enumerate}

\section{Solution:}
Return with 0 percent invested in the market $=5$ percent, which is the risk-free return.

Standard deviation with 0 percent invested in the market $=0$ percent because T-bills are not risky.

Return with 25 percent invested in the market $=(0.75 \times 5 \%)+(0.25 \times 15 \%)=$ $7.5 \%$.

Standard deviation with 25 percent invested in the market $=0.25 \times 20 \%=5 \%$.

Return with 75 percent invested in the market $=(0.25 \times 5 \%)+(0.75 \times 15 \%)=$ $12.50 \%$.

Standard deviation with 75 percent invested in the market $=0.75 \times 20 \%=15 \%$.

Return with 100 percent invested in the market $=15$ percent, which is the return on the $\mathrm{S} \& \mathrm{P} 500$. Standard deviation with 100 percent invested in the market $=20$ percent, which is the risk of the S\&P 500 .

\section{CAPITAL MARKET THEORY: CML - LEVERAGED PORTFOLIOS}
explain the capital allocation line (CAL) and the capital market line
(CML)

In the previous example, Mr. Miles evaluated an investment of between 0 percent and 100 percent in the market and the balance in T-bills. The line connecting $R_{f}$ and $M$ (market portfolio) in Exhibit 4 illustrates these portfolios with their respective levels of investment. At $R_{f}$ an investor is investing all of his or her wealth into risk-free securities, which is equivalent to lending 100 percent at the risk-free rate. At Point $M$ he or she is holding the market portfolio and not lending any money at the risk-free rate. The combinations of the risk-free asset and the market portfolio, which may be achieved by the points between these two limits, are termed "lending" portfolios. In effect, the investor is lending part of his or her wealth at the risk-free rate.

If Mr. Miles is willing to take more risk, he may be able to move to the right of the market portfolio (Point $M$ in Exhibit 4) by borrowing money and purchasing more of Portfolio M. Assume that he is able to borrow money at the same risk-free rate of interest, $R_{f}$ at which he can invest. He can then supplement his available wealth with borrowed money and construct a borrowing portfolio. If the straight line joining $R_{f}$ and $M$ is extended to the right of Point $M$, this extended section of the line represents borrowing portfolios. As one moves further to the right of Point $M$, an increasing amount of borrowed money is being invested in the market. This means that there is negative investment in the risk-free asset, which is referred to as a leveraged position in the risky portfolio. The particular point chosen on the CML will depend on the individual's utility function, which, in turn, will be determined by his risk and return preferences.

\section{EXAMPLE 2}
\section{Risk and Return of a Leveraged Portfolio with Equal Lending and Borrowing Rates}
\begin{enumerate}
  \item Mr. Miles decides to set aside a small part of his wealth for investment in a portfolio that has greater risk than his previous investments because he anticipates that the overall market will generate attractive returns in the future. He assumes that he can borrow money at 5 percent and achieve the same return on the S\&P 500 as before: an expected return of 15 percent with a standard deviation of 20 percent.
\end{enumerate}

Calculate his expected risk and return if he borrows 25 percent, 50 percent, and 100 percent of his initial investment amount.

\section{Solution:}
The leveraged portfolio's standard deviation and return can be calculated in the same manner as before with the following equations:

$$
E\left(R_{p}\right)=w_{1} R_{f}+\left(1-w_{1}\right) E\left(R_{m}\right)
$$

and

$$
\sigma_{p}=\left(1-w_{1}\right) \sigma_{m}
$$

The proportion invested in T-bills becomes negative instead of positive because Mr. Miles is borrowing money. If 25 percent of the initial investment is borrowed, $w_{1}=-0.25$, and $\left(1-w_{1}\right)=1.25$, etc.

Return with $w_{1}=-0.25=(-0.25 \times 5 \%)+(1.25 \times 15 \%)=17.5 \%$.

Standard deviation with $w_{1}=-0.25=1.25 \times 20 \%=25 \%$.

Return with $w_{1}=-0.50=(-0.50 \times 5 \%)+(1.50 \times 15 \%)=20.0 \%$.

Standard deviation with $w_{1}=-0.50=1.50 \times 20 \%=30 \%$.

Return with $w_{1}=-1.00=(-1.00 \times 5 \%)+(2.00 \times 15 \%)=25.0 \%$.

Standard deviation with $w_{1}=-1.00=2.00 \times 20 \%=40 \%$.

Note that negative investment (borrowing) in the risk-free asset provides a higher expected return for the portfolio but that higher return is also associated with higher risk.

\section{Leveraged Portfolios with Different Lending and Borrowing Rates}
Although we assumed that Mr. Miles can borrow at the same rate as the US government, it is more likely that he will have to pay a higher interest rate than the government because his ability to repay is not as certain as that of the government. Now consider that although Mr. Miles can invest (lend) at $R_{f}$ he can borrow at only $R_{b}$, a rate that is higher than the risk-free rate.

With different lending and borrowing rates, the CML will no longer be a single straight line. The line will have a slope of $\left[E\left(R_{m}\right)-R_{f}\right] / \sigma_{m}$ between Points $R_{f}$ and $M$, where the lending rate is $R_{f}$ but will have a smaller slope of $\left[E\left(R_{m}\right)-R_{b}\right] / \sigma_{m}$ at points to the right of $M$, where the borrowing rate is $R_{b}$. Exhibit 5 illustrates the CML with different lending and borrowing rates.

\section{Exhibit 5: CML with Different Lending and Borrowing Rates}
\begin{center}
\includegraphics[max width=\textwidth]{2023_05_04_36535b8d80b32081d422g-639}
\end{center}

The equations for the two lines are given below.

$$
w_{1} \geq 0: E\left(R_{p}\right)=R_{f}+\left(\frac{E\left(R_{m}\right)-R_{f}}{\sigma_{m}}\right) \times \sigma_{p}
$$

and

$$
w_{1}<0: E\left(R_{p}\right)=R_{b}+\left(\frac{E\left(R_{m}\right)-R_{b}}{\sigma_{m}}\right) \times \sigma_{p}
$$

The first equation is for the line where the investment in the risk-free asset is zero or positive-that is, at $M$ or to the left of $M$ in Exhibit 5. The second equation is for the line where borrowing, or negative investment in the risk-free asset, occurs. Note that the only difference between the two equations is in the interest rates used for borrowing and lending.

All passive portfolios will lie on the kinked CML, although the investment in the risk-free asset may be positive (lending), zero (no lending or borrowing), or negative (borrowing). Leverage allows less risk-averse investors to increase the amount of risk they take by borrowing money and investing more than 100 percent in the passive portfolio.

\section{EXAMPLE 3}
\section{Leveraged Portfolio with Different Lending and Borrowing Rates}
\begin{enumerate}
  \item Mr. Miles approaches his broker to borrow money against securities held in his portfolio. Even though Mr. Miles' loan will be secured by the securities in his portfolio, the broker's rate for lending to customers is 7 percent. Assuming a risk-free rate of 5 percent and a market return of 15 percent with a standard deviation of 20 percent, estimate Mr. Miles' expected return and risk if he invests 25 percent and 75 percent in the market and if he decides to borrow 25 percent and 75 percent of his initial investment and invest the money in the market.
\end{enumerate}

Solution:

The unleveraged portfolio's standard deviation and return are calculated using the same equations as before:

$$
E\left(R_{p}\right)=w_{1} R_{f}+\left(1-w_{1}\right) E\left(R_{m}\right),
$$

and

$$
\sigma_{p}=\left(1-w_{1}\right) \sigma_{m}
$$

The results are unchanged. The slope of the line for the unleveraged portfolio is 0.50 , just as before:

Return with 25 percent invested in the market $=(0.75 \times 5 \%)+(0.25 \times 15 \%)=$ $7.5 \%$.

Standard deviation with 25 percent invested in the market $=0.25 \times 20 \%=5 \%$.

Return with 75 percent invested in the market $=(0.25 \times 5 \%)+(0.75 \times 15 \%)=$ $12.5 \%$.

Standard deviation with 75 percent invested in the market $=0.75 \times 20 \%=15 \%$.

For the leveraged portfolio, everything remains the same except that $R_{f}$ is replaced with $R_{b}$.

$$
E\left(R_{p}\right)=w_{1} R_{b}+\left(1-w_{1}\right) E\left(R_{m}\right)
$$

and

$$
\sigma_{p}=\left(1-w_{1}\right) \sigma_{m}
$$

Return with $w_{1}=-0.25=(-0.25 \times 7 \%)+(1.25 \times 15 \%)=17.0 \%$.

Standard deviation with $w_{1}=-0.25=1.25 \times 20 \%=25 \%$.

Return with $w_{1}=-0.75=(-0.75 \times 7 \%)+(1.75 \times 15 \%)=21.0 \%$.

Standard deviation with $w_{1}=-0.75=1.75 \times 20 \%=35 \%$.

The risk and return of the leveraged portfolio is higher than that of the unleveraged portfolio. As Mr. Miles borrows more money to invest in the market, the expected return increases but so does the standard deviation of the portfolio. The slope of the line for the leveraged portfolio is 0.40 , compared with 0.50 for the unleveraged portfolio, which means that for every 1 percent increase in risk, the investor gets a 0.40 percent increase in expected return in the leveraged part of the portfolio, compared with a 0.50 percent increase in expected return in the unleveraged part of the portfolio. Only investors who are less risk averse will choose leveraged portfolios.

\section{SYSTEMATIC AND NONSYSTEMATIC RISK}
explain systematic and nonsystematic risk, including why an investor should not expect to receive additional return for bearing nonsystematic risk

In constructing a portfolio, it is important to understand the concept of correlation and how less than perfect correlation can diversify the risk of a portfolio. As a consequence, the risk of an asset held alone may be greater than the risk of that same asset when it is part of a portfolio. Because the risk of an asset varies from one environment to another, which kind of risk should an investor consider and how should that risk be priced? This section addresses the question of pricing of risk by decomposing the total risk of a security or a portfolio into systematic and nonsystematic risk. The meaning of these risks, how they are computed, and their relevance to the pricing of assets are also discussed.

\section{Systematic Risk and Nonsystematic Risk}
Systematic risk, also known as non-diversifiable or market risk, is the risk that affects the entire market or economy. In contrast, nonsystematic risk is the risk that pertains to a single company or industry and is also known as company-specific, industry-specific, diversifiable, or idiosyncratic risk.

Systematic risk is risk that cannot be avoided and is inherent in the overall market. It is non-diversifiable because it includes risk factors that are innate within the market and affect the market as a whole. Examples of factors that constitute systematic risk include interest rates, inflation, economic cycles, political uncertainty, and widespread natural disasters. These events affect the entire market, and there is no way to avoid their effect. Systematic risk can be magnified through selection or by using leverage, or diminished by including securities that have a low correlation with the portfolio, assuming they are not already part of the portfolio.

Nonsystematic risk is risk that is local or limited to a particular asset or industry that need not affect assets outside of that asset class. Examples of nonsystematic risk could include the failure of a drug trial or an airliner crash. All these events will directly affect their respective companies and possibly industries, but have no effect on assets that are far removed from these industries. Investors can avoid nonsystematic risk through diversification by forming a portfolio of assets that are not highly correlated with one another.

We will derive expressions for each kind of risk later in this reading. You will see that the sum of systematic variance and nonsystematic variance equals the total variance of the security or portfolio:

Total variance $=$ Systematic variance + Nonsystematic variance

Although the equality relationship is between variances, you will find frequent references to total risk as the sum of systematic risk and nonsystematic risk. In those cases, the statements refer to variance, not standard deviation.

\section{Pricing of Risk}
Pricing or valuing an asset is equivalent to estimating its expected rate of return. If an asset has a known terminal value, such as the face value of a bond, then a lower current price implies a higher future return and a higher current price implies a lower future return. The relationship between price and return can also be observed in the valuation expression shown in Section 2.1.2. Therefore, we will occasionally use price and return interchangeably when discussing the price of risk.

Consider an asset with both systematic and nonsystematic risk. Assume that both kinds of risk are priced - that is, you receive a return for both systematic risk and nonsystematic risk. What will you do? Realizing that nonsystematic risk can be diversified away, you would buy assets that have a large amount of nonsystematic risk. Once you have bought those assets with nonsystematic risk, you would diversify, or reduce that risk, by including other assets that are not highly correlated. In the process, you will minimize nonsystematic risk and eventually eliminate it altogether from your portfolio. You would now have a diversified portfolio with only systematic risk, yet you would be compensated for nonsystematic risk that you no longer have. Just like everyone else, you would have an incentive to take on more and more diversifiable risk because you are compensated for it even though you can get rid of it. The demand for diversifiable risk would keep increasing until its price becomes infinite and its expected return falls to zero. This means that our initial assumption of a non-zero return for diversifiable risk was incorrect and that the correct assumption is zero return for diversifiable risk. Therefore, according to theory, in an efficient market no incremental reward is earned for taking on diversifiable risk.

We have argued that investors should not be compensated for taking on nonsystematic risk. Therefore, investors who have nonsystematic risk must diversify it away by investing in many industries, many countries, and many asset classes. Because future returns are unknown and it is not possible to pick only winners, diversification helps in offsetting poor returns in one asset class by garnering good returns in another asset class, thereby reducing the overall risk of the portfolio. In contrast, investors must be compensated for accepting systematic risk because that risk cannot be diversified away. If investors do not receive a return commensurate with the amount of systematic risk they are taking, they will refuse to accept systematic risk.

In summary, according to theory, systematic or non-diversifiable risk is priced and investors are compensated for holding assets or portfolios based only on that investment's systematic risk. Investors do not receive any return for accepting nonsystematic or diversifiable risk. Therefore, it is in the interest of risk-averse investors to hold only well-diversified portfolios.

\section{EXAMPLE 4}
\section{Systematic and Nonsystematic Risk}
\begin{enumerate}
  \item Describe the systematic and nonsystematic risk components of the following assets:
\end{enumerate}

A. A risk-free asset, such as a three-month Treasury bill

B. The market portfolio, such as the S\&P 500 .

\begin{enumerate}
  \setcounter{enumi}{1}
  \item Consider two assets, $A$ and $\mathrm{B}$. Asset $\mathrm{A}$ has twice the amount of total risk as Asset B. For Asset A, systematic risk comprises two-thirds of total risk. For Asset B, all of total risk is systematic risk. Which asset should have a higher expected rate of return?
\end{enumerate}

\section{Solution to $1 \mathrm{~A}$ :}
By definition, a risk-free asset has no risk. Therefore, a risk-free asset has zero systematic risk and zero nonsystematic risk.

\section{Solution to $1 \mathrm{~B}$ :}
As we mentioned earlier, a market portfolio is a diversified portfolio, one in which no more risk can be diversified away. We have also described it as an efficient portfolio. Therefore, a market portfolio does not contain any nonsystematic risk.

\section{Solution to 2:}
Based on the facts given, Asset A's systematic risk is one-third greater than Asset B's systematic risk. Because only systematic risk is priced or receives a return, the expected rate of return must be higher for Asset A.

\section{RETURN GENERATING MODELS}
$$
\square \quad \begin{aligned}
& \text { explain return generating models (including the market model) and } \\
& \text { their uses }
\end{aligned}
$$

As previously mentioned, in order to form the market portfolio, you should combine all available risky assets. Knowledge of the correlations among those assets allows us to estimate portfolio risk. You also learned that a fully diversified portfolio will include all asset classes and essentially all assets in those asset classes. The work required for construction of the market portfolio is formidable. For example, for a portfolio of 1,000 assets, we will need 1,000 return estimates, 1,000 standard deviation estimates, and 499,500 $(1,000 \times 999 \div 2)$ correlations. Other related questions that arise with this analysis are whether we really need all 1,000 assets and what happens if there are errors in these estimates.

An alternate method of constructing an optimal portfolio is simpler and easier to implement. An investor begins with a known portfolio, such as the S\&P 500, and then adds other assets one at a time on the basis of the asset's standard deviation, expected return, and impact on the portfolio's risk and return. This process continues until the addition of another asset does not have a significant impact on the performance of the portfolio. The process requires only estimates of systematic risk for each asset because investors will not be compensated for nonsystematic risk. Expected returns can be calculated by using return-generating models, as we will discuss in this section. In addition to using return-generating models, we will also decompose total variance into systematic variance and nonsystematic variance and establish a formal relationship between systematic risk and return. In the next section, we will expand on this discussion and introduce the CAPM as the preferred return-generating model.

\section{Return-Generating Models}
A return-generating model is a model that can provide an estimate of the expected return of a security given certain parameters. If systematic risk is the only relevant parameter for return, then the return-generating model will estimate the expected return for any asset given the level of systematic risk.

As with any model, the quality of estimates of expected return will depend on the quality of input estimates and the accuracy of the model. Because it is difficult to decide which factors are appropriate for generating returns, the most general form of a return-generating model is a multi-factor model. A multi-factor model allows more than one variable to be considered in estimating returns and can be built using different kinds of factors, such as macroeconomic, fundamental, and statistical factors. Macroeconomic factor models use economic factors that are correlated with security returns. These factors may include economic growth, the interest rate, the inflation rate, productivity, employment, and consumer confidence. Past relationships with returns are estimated to obtain parameter estimates, which are, in turn, used for computing expected returns. Fundamental factor models analyze and use relationships between security returns and the company's underlying fundamentals, such as, for example, earnings, earnings growth, cash flow generation, investment in research, advertising, and number of patents. Finally, in a statistical factor model, historical and cross-sectional return data are analyzed to identify factors that explain variance or covariance in observed returns. These statistical factors, however, may or may not have an economic or fundamental connection to returns. For example, the conference to which the American football Super Bowl winner belongs, whether the American Football Conference or the National Football Conference, may be a factor in US stock returns, but no obvious economic connection seems to exist between the winner's conference and US stock returns. Moreover, data mining may generate many spurious factors that are devoid of any economic meaning. Because of this limitation, analysts prefer the macroeconomic and fundamental factor models for specifying and estimating return-generating models.

A general return-generating model is expressed in the following manner:

$$
E\left(R_{i}\right)-R_{f}=\sum_{j=1}^{k} \beta_{i j} E\left(F_{j}\right)=\beta_{i 1}\left[E\left(R_{m}\right)-R_{f}\right]+\sum_{j=2}^{k} \beta_{i j} E\left(F_{j}\right)
$$

The model has $k$ factors, $E\left(F_{1}\right), E\left(F_{2}\right), \ldots E\left(F_{k}\right)$. The coefficients, $\beta_{i j}$, are the factor weights (sometimes called factor loadings) associated with each factor. The left-hand side of the model has the expected excess return (i.e., the expected return over the risk-free rate). The right-hand side provides the risk factors that would generate the return or premium required to assume that risk. We have separated out one factor, $E\left(R_{m}\right)$, which represents the market return. All models contain return on the market portfolio as a key factor.

\section{Three-Factor and Four-Factor Models}
Eugene Fama and Kenneth French ${ }^{1}$ suggested that a return-generating model for stock returns should include relative market capitalization of the company ("size") relative book-to-market value of the company in addition to beta. Fama and French found that past returns could be explained better with their model than with other models available at that time, most notably, the capital asset pricing model. Mark Carhart (1997) extended the Fama and French model by adding another factor: momentum, defined as relative past stock returns.

\section{The Single-Index Model}
The simplest form of a return-generating model is a single-factor linear model, in which only one factor is considered. The most common implementation is a single-index model, which uses the market factor in the following form: $E\left(R_{i}\right)-R_{f}=\beta_{i}\left[E\left(R_{m}\right)-R_{f}\right]$.

Although the single-index model is simple, it fits nicely with the capital market line. Recall that the CML is linear, with an intercept of $R_{f}$ and a slope of $\left[E\left(R_{m}\right)-R_{f}\right] / \sigma_{m}$. We can rewrite the CML by moving the intercept to the left-hand side of the equation, rearranging the terms, and generalizing the subscript from $p$ to $i$, for any security:

$$
E\left(R_{i}\right)-R_{f}=\left(\frac{\sigma_{i}}{\sigma_{m}}\right)\left[E\left(R_{m}\right)-R_{f}\right]
$$

1 Fama and French (1992) The factor loading or factor weight, $\sigma_{i} / \sigma_{m}$, refers to the ratio of total security risk to total market risk. To obtain a better understanding of factor loading and to illustrate that the CML reduces to a single-index model, we decompose total risk into its components.

\section{Decomposition of Total Risk for a Single-Index Model}
With the introduction of return-generating models, particularly the single-index model, we are able to decompose total variance into systematic and nonsystematic variances. Instead of using expected returns in the single index, let us use realized returns. The difference between expected returns and realized returns is attributable to non-market changes, as an error term, $e_{i}$, in the second equation below:

$$
E\left(R_{i}\right)-R_{f}=\beta_{i}\left[E\left(R_{m}\right)-R_{f}\right]
$$

and

$$
R_{i}-R_{f}=\beta_{i}\left(R_{m}-R_{f}\right)+e_{i}
$$

The variance of realized returns can be expressed in the equation below (note that $R_{f}$ is a constant). We can further drop the covariance term in this equation because, by definition, any non-market return is uncorrelated with the market. Thus, we are able to decompose total variance into systematic and nonsystematic variances in the second equation below:

$$
\sigma_{i}^{2}=\beta_{i}^{2} \sigma_{m}^{2}+\sigma_{e}^{2}+2 \operatorname{Cov}\left(R_{m}, e_{i}\right)
$$

Total variance $=$ Systematic variance + Nonsystematic variance, which can be written as

$$
\sigma_{i}^{2}=\beta_{i}^{2} \sigma_{m}^{2}+\sigma_{e}^{2}
$$

Total risk can be expressed as

$$
\sigma_{i}=\sqrt{\beta_{i}^{2} \sigma_{m}^{2}+\sigma_{e}^{2}}
$$

Because nonsystematic risk is zero for well-diversified portfolios, such as the market portfolio, the total risk of a market portfolio and other similar portfolios is only systematic risk, which is $\beta_{i} \sigma_{m}$. We can now return to the CML discussed in the previous subsection and replace $\sigma_{i}$ with $\beta_{i} \sigma_{m}$ because the CML assumes that the market is a diversified portfolio. By making this substitution for the above equation, we get the following single-index model:

$$
\begin{aligned}
& E\left(R_{i}\right)-R_{f}=\left(\frac{\sigma_{i}}{\sigma_{m}}\right) \times\left[E\left(R_{m}\right)-R_{f}\right]=\left(\frac{\beta_{i} \sigma_{m}}{\sigma_{m}}\right) \times\left[E\left(R_{m}\right)-R_{f}\right], \\
& E\left(R_{i}\right)-R_{f}=\beta_{i}\left[E\left(R_{m}\right)-R_{f}\right]
\end{aligned}
$$

Thus, the CML, which holds only for well-diversified portfolios, is fully consistent with a single-index model.

In summary, total variance may be decomposed into systematic and nonsystematic variances and the CML is the same as a single-index model for diversified portfolios.

\section{Return-Generating Models: The Market Model}
The most common implementation of a single-index model is the market model, in which the market return is the single factor or single index. In principle, the market model and the single-index model are similar. The difference is that the market model is easier to work with and is normally used for estimating beta risk and computing abnormal returns. The market model is

$$
R_{i}=\alpha_{i}+\beta_{i} R_{m}+e_{i}
$$

To be consistent with the previous section, $\alpha_{i}=R_{f}(1-\beta)$. The intercept, $\alpha_{i}$, and slope coefficient, $\beta_{i}$, can be estimated by using historical security and market returns. These parameter estimates are then used to predict company-specific returns that a security may earn in a future period. Assume that a regression of Wal-Mart's historical daily returns on S\&P 500 daily returns gives an $\alpha_{i}$ of 0.0001 and a $\beta_{i}$ of 0.9 . Thus, Wal-Mart's expected daily return $=0.0001+0.90 \times R_{m}$. If, on a given day the market rises by 1 percent and Wal-Mart's stock rises by 2 percent, then Wal-Mart's company-specific return $\left(e_{i}\right)$ for that day $=R_{i}-E\left(R_{i}\right)=R_{i}-\left(\alpha_{i}+\beta_{i} R_{m}\right)=0.02-(0.0001+0.90 \times 0.01)$ $=0.0109$, or $1.09 \%$. In other words, Wal-Mart earned an abnormal return of 1.09 percent on that day.

CALCULATION AND INTERPRETATION OF BETA

calculate and interpret beta

We begin with the single-index model introduced earlier using realized returns and rewrite it as

$$
R_{i}=\left(1-\beta_{i}\right) R_{f}+\beta_{i} \times R_{m}+e_{i}
$$

Because systematic risk depends on the correlation between the asset and the market, we can arrive at a measure of systematic risk from the covariance between $R_{i}$ and $R_{m}$, where $R_{i}$ is defined using the above equation. Note that the risk-free rate is a constant, so the first term in $R_{i}$ drops out.

$$
\begin{aligned}
& \operatorname{Cov}\left(R_{i}, R_{m}\right)=\operatorname{Cov}\left(\beta_{i} \times R_{m}+e_{i}, R_{m}\right) \\
= & \beta_{i} \operatorname{Cov}\left(R_{m}, R_{m}\right)+\operatorname{Cov}\left(e_{i}, R_{m}\right) \\
= & \beta_{i} \sigma_{m}^{2}+0
\end{aligned}
$$

The first term is beta multiplied by the variance of $R_{m}$. Because the error term is uncorrelated with the market, the second term drops out. Then, we can rewrite the equation in terms of beta as follows:

$$
\beta_{i}=\frac{\operatorname{Cov}\left(R_{i}, R_{m}\right)}{\sigma_{m}^{2}}=\frac{\rho_{i, m} \sigma_{i} \sigma_{m}}{\sigma_{m}^{2}}=\frac{\rho_{i, m} \sigma_{i}}{\sigma_{m}}
$$

The above formula shows the expression for beta, $\beta_{i}$, which is similar to the factor loading in the single-index model presented earlier. For example, if the correlation between an asset and the market is 0.70 and the asset and market have standard deviations of return of 0.25 and 0.15 , respectively, the asset's beta would be $(0.70)(0.25) / 0.15=1.17$. If the asset's covariance with the market and market variance were given as 0.026250 and 0.02250 , respectively, the calculation would be $0.026250 / 0.02250=1.17$. The beta in the market model includes an adjustment for the correlation between asset $i$ and the market because the market model covers all assets whereas the CML works only for fully diversified portfolios.

As shown in the above equation, beta is a measure of how sensitive an asset's return is to the market as a whole and is calculated as the covariance of the return on $i$ and the return on the market divided by the variance of the market return; that expression is equivalent to the product of the asset's correlation with the market with a ratio of standard deviations of return (i.e., the ratio of the asset's standard deviation to the market's). As we have shown, beta captures an asset's systematic risk, or the portion of an asset's risk that cannot be eliminated by diversification. The variances and correlations required for the calculation of beta are usually based on historical returns. A positive beta indicates that the return of an asset follows the general market trend, whereas a negative beta shows that the return of an asset generally follows a trend that is opposite to that of the market. In other words, a positive beta indicates that the return of an asset moves in the same direction of the market, whereas a negative beta indicates that the return of an asset moves in the opposite direction of the market. A risk-free asset's beta is zero because its covariance with other assets is zero. In other words, a beta of zero indicates that the asset's return has no correlation with movements in the market. The market's beta can be calculated by substituting $\sigma_{m}$ for $\sigma_{i}$ in the numerator. Also, any asset's correlation with itself is 1 , so the beta of the market is 1 :

$$
\beta_{i}=\frac{\rho_{i, m} \sigma_{i}}{\sigma_{m}}=\frac{\rho_{m, m} \sigma_{m}}{\sigma_{m}}=1
$$

Because the market's beta is 1 , the average beta of stocks in the market, by definition, is 1 . In terms of correlation, most stocks, especially in developed markets, tend to be highly correlated with the market, with correlations in excess of 0.70 . Some US broad market indexes, such as the S\&P 500, the Dow Jones 30, and the NASDAQ 100, have even higher correlations that are in excess of 0.90 . The correlations among different sectors are also high, which shows that companies have similar reactions to the same economic and market changes. As a consequence and as a practical matter, finding assets that have a consistently negative beta is unusual because of the market's broad effects on all assets.

\section{EXAMPLE 5}
\section{Calculation of Beta}
Assuming that the risk (standard deviation) of the market is 25 percent, calculate the beta for the following assets:

We use the formula for beta in answering the above questions: $\beta_{i}=\frac{\rho_{i, m} \sigma_{i}}{\sigma_{m}}$

\begin{enumerate}
  \item A short-term US Treasury bill.
\end{enumerate}

\section{Solution:}
By definition, a short-term US Treasury bill has zero risk. Therefore, its beta is zero.

\begin{enumerate}
  \setcounter{enumi}{1}
  \item Gold, which has a standard deviation equal to the standard deviation of the market but a zero correlation with the market.
\end{enumerate}

\section{Solution:}
Because the correlation of gold with the market is zero, its beta is zero.

\begin{enumerate}
  \setcounter{enumi}{2}
  \item A new emerging market that is not currently included in the definition of "market"-the emerging market's standard deviation is 60 percent, and the correlation with the market is -0.1 .
\end{enumerate}

\section{Solution:}
Beta of the emerging market is $-0.1 \times 0.60 \div 0.25=-0.24$. 4. An initial public offering or new issue of stock with a standard deviation of 40 percent and a correlation with the market of 0.7 (IPOs are usually very risky but have a relatively low correlation with the market).

Solution:

Beta of the initial public offering is $0.7 \times 0.40 \div 0.25=1.12$.

\section{Estimation of Beta}
An alternative and more practical approach is to estimate beta directly by using the market model described above. The market model, $R_{i}=\alpha_{i}+\beta_{i} R_{m}+e_{i}$, is estimated by using regression analysis, which is a statistical process that evaluates the relationship between a given variable (the dependent variable) and one or more other (independent) variables. Historical security returns $\left(R_{i}\right)$ and historical market returns $\left(R_{m}\right)$ are inputs used for estimating the two parameters $\alpha_{i}$ and $\beta_{i}$.

Regression analysis is similar to plotting all combinations of the asset's return and the market return $\left(R_{i}, R_{m}\right)$ and then drawing a line through all points such that it minimizes the sum of squared linear deviations from the line. Exhibit 6 illustrates the market model and the estimated parameters. The intercept, $\alpha_{i}$ (sometimes referred to as the constant), and the slope term, $\beta_{i}$, are all that is needed to define the security characteristic line and obtain beta estimates.

Exhibit 6: Beta Estimation Using a Plot of Security and Market Returns

\begin{center}
\includegraphics[max width=\textwidth]{2023_05_04_36535b8d80b32081d422g-648}
\end{center}

Although beta estimates are important for forecasting future levels of risk, there is much concern about their accuracy. In general, shorter periods of estimation (e.g., 12 months) represent betas that are closer to the asset's current level of systematic risk. Shorter period beta estimates, however, are also less accurate than beta estimates measured over three to five years because they may be affected by special events in that short period. Although longer period beta estimates are more accurate, they may be a poor representation of future expectations, especially if major changes in the asset have occurred. Therefore, it is necessary to recognize that estimates of beta, whether obtained through calculation or regression analysis, may or may not represent current or future levels of an asset's systematic risk.

\section{Beta and Expected Return}
Although the single-index model, also called the capital asset pricing model (CAPM), will be discussed in greater detail in the next section, we will use the CAPM in this section to estimate returns, given asset betas. The CAPM is usually written with the risk-free rate on the right-hand side:

$$
E\left(R_{i}\right)=R_{f}+\beta_{i}\left[E\left(R_{m}\right)-R_{f}\right]
$$

The model shows that the primary determinant of expected return for a security is its beta, or how well the security correlates with the market. The higher the beta of an asset, the higher its expected return will be. Assets with a beta greater than 1 have an expected return that is higher than the market return, whereas assets with a beta of less than 1 have an expected return that is less than the market return.

In certain cases, assets may require a return less than the risk-free return. For example, if an asset's beta is negative, the required return will be less than the risk-free rate. When combined with the market, the asset reduces the risk of the overall portfolio, which makes the asset very valuable. Insurance is one such asset. Insurance gives a positive return when the insured's wealth is reduced because of a catastrophic loss. In the absence of such a loss or when the insured's wealth is growing, the insured is required to pay an insurance premium. Thus, insurance has a negative beta and a negative expected return, but helps in reducing overall risk.

\section{EXAMPLE 6}
\section{Calculation of Expected Return}
\begin{enumerate}
  \item Alpha Natural Resources (ANR), a coal producer, buys a large but privately held coal producer in China. As a result of the cross-border acquisition of a private company, ANR's standard deviation of returns is reduced from 50 percent to 30 percent and its correlation with the market falls from 0.95 to 0.75 . Assume that the standard deviation and return of the market remain unchanged at 25 percent and 10 percent, respectively, and that the risk-free rate is 3 percent.
\end{enumerate}

A. Calculate the beta of ANR stock and its expected return before the acquisition.

B. Calculate the expected return after the acquisition.

\section{Solution to 1A:}
Using the formula for $\beta_{i}$, we can calculate $\beta_{i}$ and then the return.

$$
\begin{aligned}
& \beta_{i}=\frac{\rho_{i, m} \sigma_{i}}{\sigma_{m}}=\frac{0.95 \times 0.50}{0.25}=1.90 \\
& E\left(R_{i}\right)=R_{f}+\beta_{i}\left[E\left(R_{m}\right)-R_{f}\right]=0.03+1.90 \times(0.10-0.03)=0.163=16.3 \%
\end{aligned}
$$

\section{Solution to $1 \mathrm{~B}$ :}
We follow the same procedure but with the after-acquisition correlation and risk.

$$
\beta_{i}=\frac{\rho_{i, m} \sigma_{i}}{\sigma_{m}}=\frac{0.75 \times 0.30}{0.25}=0.90
$$

$$
E\left(R_{i}\right)=R_{f}+\beta_{i}\left[E\left(R_{m}\right)-R_{f}\right]=0.03+0.90 \times(0.10-0.03)=0.093=9.3 \%
$$

The market risk premium is 7 percent $(10 \%-3 \%)$. As the beta changes, the change in the security's expected return is the market risk premium multiplied by the change in beta. In this scenario, ANR's beta decreased by 1.0, so the new expected return for ANR is 7 percentage points lower.

\begin{enumerate}
  \setcounter{enumi}{1}
  \item Mr. Miles observes the strong demand for iPods and iPhones and wants to invest in Apple stock. Unfortunately, Mr. Miles doesn't know the return he should expect from his investment. He has been given a risk-free rate of 3 percent, a market return of 10 percent, and Apple's beta of 1.5 .
\end{enumerate}

A. Calculate Apple's expected return.

B. An analyst looking at the same information decides that the past performance of Apple is not representative of its future performance. He decides that, given the increase in Apple's market capitalization, Apple acts much more like the market than before and thinks Apple's beta should be closer to 1.1. What is the analyst's expected return for Apple stock?

\section{Solution to $2 \mathrm{~A}$ :}
$$
E\left(R_{i}\right)=R_{f}+\beta_{i}\left[E\left(R_{m}\right)-R_{f}\right]=0.03+1.5 \times(0.10-0.03)=0.135=13.5 \%
$$

\section{Solution to 2B:}
$$
E\left(R_{i}\right)=R_{f}+\beta_{i}\left[E\left(R_{m}\right)-R_{f}\right]=0.03+1.1 \times(0.10-0.03)=0.107=10.7 \%
$$

This example illustrates the lack of connection between estimation of past returns and projection into the future. Investors should be aware of the limitations of using past returns for estimating future returns.

\section{CAPITAL ASSET PRICING MODEL: ASSUMPTIONS AND THE SECURITY MARKET LINE}
explain the capital asset pricing model (CAPM), including its assumptions, and the security market line (SML) calculate and interpret the expected return of an asset using the CAPM

The capital asset pricing model is one of the most significant innovations in portfolio theory. The model is simple, yet powerful; is intuitive, yet profound. The CAPM was introduced independently by William Sharpe, John Lintner, Jack Treynor, and Jan Mossin and builds on Harry Markowitz's earlier work on diversification and modern portfolio theory. ${ }^{2}$ The model provides a linear expected return-beta relationship that precisely determines the expected return given the beta of an asset. In doing so, it makes the transition from total risk to systematic risk, the primary determinant of expected return. Recall the following equation:

2 See, for example, Markowitz (1952), Sharpe (1964), Lintner (1965a, 1965b), Treynor (1961, 1962), and $\operatorname{Mossin}(1966)$

$$
E\left(R_{i}\right)=R_{f}+\beta_{i}\left[E\left(R_{m}\right)-R_{f}\right]
$$

The CAPM asserts that the expected returns of assets vary only by their systematic risk as measured by beta. Two assets with the same beta will have the same expected return irrespective of the nature of those assets. Given the relationship between risk and return, all assets are defined only by their beta risk, which we will explain as the assumptions are described.

In the remainder of this section, we will examine the assumptions made in arriving at the CAPM and the limitations those assumptions entail. Second, we will implement the CAPM through the security market line to price any portfolio or asset, both efficient and inefficient. Finally, we will discuss ways in which the CAPM can be applied to investments, valuation, and capital budgeting.

\section{Assumptions of the CAPM}
Similar to all other models, the CAPM ignores many of the complexities of financial markets by making simplifying assumptions. These assumptions allow us to gain important insights into how assets are priced without complicating the analysis. Once the basic relationships are established, we can relax the assumptions and examine how our insights need to be altered. Some of these assumptions are constraining, whereas others are benign. And other assumptions affect only a particular set of assets or only marginally affect the hypothesized relationships.

\section{Investors are risk-averse, utility-maximizing, rational individuals.}
Risk aversion means that investors expect to be compensated for accepting risk. Note that the assumption does not require investors to have the same degree of risk aversion; it only requires that they are averse to risk. Utility maximization implies that investors want higher returns, not lower returns, and that investors always want more wealth (i.e., investors are never satisfied). Investors are understood to be rational in that they correctly evaluate and analyze available information to arrive at rational decisions. Although rational investors may use the same information to arrive at different estimates of expected risk and expected returns, homogeneity among investors (see Assumption 4) requires that investors be rational individuals.

Risk aversion and utility maximization are generally accepted as reflecting a realistic view of the world. Yet, rationality among investors has been questioned because investors may allow their personal biases and experiences to disrupt their decision making, resulting in suboptimal investments.

Nonetheless, the model's results are unaffected by such irrational behavior as long as it does not affect prices in a significant manner (i.e., the trades of irrational investors cancel each other or are dominated by the trades of rational investors).

\begin{enumerate}
  \setcounter{enumi}{1}
  \item Markets are frictionless, including no transaction costs and no taxes.
\end{enumerate}

Frictionless markets allow us to abstract the analysis from the operational characteristics of markets. In doing so, we do not allow the risk-return relationship to be affected by, for example, the trading volume on the New York Stock Exchange or the difference between buying and selling prices. Specifically, frictionless markets do not have transaction costs, taxes, or any costs or restrictions on short selling. We also assume that borrowing and lending at the risk-free rate is possible.

The transaction costs of many large institutions are negligible, and many institutions do not pay taxes. Even the presence of non-zero transaction costs, taxes, or the inability to borrow at the risk-free rate does not materially affect the general conclusions of the CAPM. Costs of short selling or restrictions on short selling, however, can introduce an upward bias in asset prices, potentially jeopardizing important conclusions of the CAPM.

\section{Investors plan for the same single holding period.}
The CAPM is a single-period model, and all investor decisions are made on the basis of that one period. The assumption of a single period is applied for convenience because working with multi-period models is more difficult. A single-period model, however, does not allow learning to occur, and bad decisions can persist. In addition, maximizing utility at the end of a multi-period horizon may require decisions in certain periods that may seem suboptimal when examined from a single-period perspective. Nonetheless, the single holding period does not severely limit the applicability of the CAPM to multi-period settings.

\section{Investors have homogeneous expectations or beliefs.}
This assumption means that all investors analyze securities in the same way using the same probability distributions and the same inputs for future cash flows. In addition, given that they are rational individuals, the investors will arrive at the same valuations. Because their valuations of all assets are identical, they will generate the same optimal risky portfolio, which we call the market portfolio.

The assumption of homogeneous beliefs can be relaxed as long as the differences in expectations do not generate significantly different optimal risky portfolios.

\begin{enumerate}
  \setcounter{enumi}{4}
  \item All investments are infinitely divisible.
\end{enumerate}

This assumption implies that an individual can invest as little or as much as he or she wishes in an asset. This supposition allows the model to rely on continuous functions rather than on discrete jump functions. The assumption is made for convenience only and has an inconsequential impact on the conclusions of the model.

\begin{enumerate}
  \setcounter{enumi}{5}
  \item Investors are price takers.
\end{enumerate}

The CAPM assumes that there are many investors and that no investor is large enough to influence prices. Thus, investors are price takers, and we assume that security prices are unaffected by investor trades. This assumption is generally true because even though investors may be able to affect prices of small stocks, those stocks are not large enough to affect the primary results of the CAPM.

The main objective of these assumptions is to create a marginal investor who rationally chooses a mean-variance-efficient portfolio in a predictable fashion. We assume away any inefficiency in the market from both operational and informational perspectives. Although some of these assumptions may seem unrealistic, relaxing most of them will have only a minor influence on the model and its results. Moreover, the CAPM, with all its limitations and weaknesses, provides a benchmark for comparison and for generating initial return estimates.

\section{The Security Market Line}
In this subsection, we apply the CAPM to the pricing of securities. The security market line (SML) is a graphical representation of the capital asset pricing model with beta, reflecting systematic risk, on the $x$-axis and expected return on the $y$-axis. Using the same concept as the capital market line, the SML intersects the $y$-axis at the risk-free rate of return, and the slope of this line is the market risk premium, $R_{m}-R_{f}$. Recall that the capital market line (CML) does not apply to all securities or assets but only to portfolios on the efficient frontier. The efficient frontier gives optimal combinations of expected return and total risk. In contrast, the security market line applies to any security, efficient or not. Total risk and systematic risk are equal only for efficient portfolios because those portfolios have no diversifiable risk remaining.

Exhibit 7 is a graphical representation of the CAPM, the security market line. As shown earlier in this reading, the beta of the market is 1 ( $x$-axis) and the market earns an expected return of $R_{m}(y$-axis). Using this line, it is possible to calculate the expected return of an asset. The next example illustrates the beta and return calculations.

\section{Exhibit 7: The Security Market Line}
\begin{center}
\includegraphics[max width=\textwidth]{2023_05_04_36535b8d80b32081d422g-653}
\end{center}

\section{EXAMPLE 7}
\section{Security Market Line and Expected Return}
\begin{enumerate}
  \item Suppose the risk-free rate is 3 percent, the expected return on the market portfolio is 13 percent, and its standard deviation is 23 percent. An Indian company, Bajaj Auto, has a standard deviation of 50 percent but is uncorrelated with the market. Calculate Bajaj Auto's beta and expected return.
\end{enumerate}

\section{Solution:}
Using the formula for $\beta_{i}$, we can calculate $\beta_{i}$ and then the return.

$$
\beta_{i}=\frac{\rho_{i, m} \sigma_{i}}{\sigma_{m}}=\frac{0.0 \times 0.50}{0.23}=0
$$

$$
E\left(R_{i}\right)=R_{f}+\beta_{i}\left[E\left(R_{m}\right)-R_{f}\right]=0.03+0 \times(0.13-0.03)=0.03=3.0 \%
$$

Because of its zero correlation with the market portfolio, Bajaj Auto's beta is zero. Because the beta is zero, the expected return for Bajaj Auto is the risk-free rate, which is 3 percent.

\begin{enumerate}
  \setcounter{enumi}{1}
  \item Suppose the risk-free rate is 3 percent, the expected return on the market portfolio is 13 percent, and its standard deviation is 23 percent. A German company, Mueller Metals, has a standard deviation of 50 percent and a correlation of 0.65 with the market. Calculate Mueller Metal's beta and expected return.
\end{enumerate}

\section{Solution:}
Using the formula for $\beta_{i}$, we can calculate $\beta_{i}$ and then the return.

$$
\begin{aligned}
& \beta_{i}=\frac{\rho_{i, m} \sigma_{i}}{\sigma_{m}}=\frac{0.65 \times 0.50}{0.23}=1.41 \\
& E\left(R_{i}\right)=R_{f}+\beta_{i}\left[E\left(R_{m}\right)-R_{f}\right]=0.03+1.41 \times(0.13-0.03)=0.171=17.1 \%
\end{aligned}
$$

Because of the high degree of correlation with the market, the beta for Mueller Metals is 1.41 and the expected return is 17.1 percent. Because Mueller Metals has systematic risk that is greater than that of the market, it has an expected return that exceeds the expected return of the market.

\section{Portfolio Beta}
As we stated above, the security market line applies to all securities. But what about a combination of securities, such as a portfolio? Consider two securities, 1 and 2, with a weight of $w_{i}$ in Security 1 and the balance in Security 2. The return for the two securities and return of the portfolio can be written as:

$$
\begin{aligned}
& E\left(R_{1}\right)=R_{f}+\beta_{1}\left[E\left(R_{m}\right)-R_{f}\right] \\
E\left(R_{2}\right)= & R_{f}+\beta_{2}\left[E\left(R_{m}\right)-R_{f}\right] \\
& E\left(R_{p}\right)=w_{1} E\left(R_{1}\right)+w_{2} E\left(R_{2}\right) \\
= & w_{1} R_{f}+w_{1} \beta_{1}\left[E\left(R_{m}\right)-R_{f}\right]+w_{2} R_{f}+w_{2} \beta_{2}\left[E\left(R_{m}\right)-R_{f}\right] \\
= & R_{f}+\left(w_{1} \beta_{1}+w_{2} \beta_{2}\right)\left[E\left(R_{m}\right)-R_{f}\right]
\end{aligned}
$$

The last equation gives the expression for the portfolio's expected return. From this equation, we can conclude that the portfolio's beta $=w_{1} \beta_{1}+w_{2} \beta_{2}$. In general, the portfolio beta is a weighted sum of the betas of the component securities and is given by:

$$
\beta_{p}=\sum_{i=1}^{n} w_{i} \beta_{i} ; \sum_{i=1}^{n} w_{i}=1
$$

The portfolio's return given by the CAPM is

$$
E\left(R_{p}\right)=R_{f}+\beta_{p}\left[E\left(R_{m}\right)-R_{f}\right]
$$

This equation shows that a linear relationship exists between the expected return of a portfolio and the systematic risk of the portfolio as measured by $\beta_{p}$.

\section{EXAMPLE 8}
\section{Portfolio Beta and Return}
\begin{enumerate}
  \item You invest 20 percent of your money in the risk-free asset, 30 percent in the market portfolio, and 50 percent in RedHat, a US stock that has a beta of 2.0. Given that the risk-free rate is 4 percent and the market return is 16 percent, what are the portfolio's beta and expected return?
\end{enumerate}

\section{Solution:}
The beta of the risk-free asset $=0$, the beta of the market $=1$, and the beta of RedHat is 2.0. The portfolio beta is

$$
\begin{aligned}
& \beta_{p}=w_{1} \beta_{1}+w_{2} \beta_{2}+w_{3} \beta_{3}=(0.20 \times 0.0)+(0.30 \times 1.0)+(0.50 \times 2.0)=1.30 \\
& E\left(R_{i}\right)=R_{f}+\beta_{i}\left[E\left(R_{m}\right)-R_{f}\right]=0.04+1.30 \times(0.16-0.04)=0.196=19.6 \%
\end{aligned}
$$

The portfolio beta is 1.30 , and its expected return is 19.6 percent.

\section{Alternate Method:}
Another method for calculating the portfolio's return is to calculate individual security returns and then use the portfolio return formula (i.e., weighted average of security returns) to calculate the overall portfolio return.

Return of the risk-free asset $=4$ percent; return of the market $=16$ percent

RedHat's return based on its beta $=0.04+2.0 \times(0.16-0.04)=0.28$

Portfolio return $=(0.20 \times 0.04)+(0.30 \times 0.16)+(0.50 \times 0.28)=0.196=$ $19.6 \%$

Not surprisingly, the portfolio return is 19.6 percent, as calculated in the first method.

\section{CAPITAL ASSET PRICING MODEL: APPLICATIONS}
calculate and interpret the expected return of an asset using the CAPM

describe and demonstrate applications of the CAPM and the SML

The CAPM offers powerful and intuitively appealing predictions about risk and the relationship between risk and return. The CAPM is not only important from a theoretical perspective but is also used extensively in practice. In this section, we will discuss some common applications of the model. When applying these tools to different scenarios, it is important to understand that the CAPM and the SML are functions that give an indication of what the return in the market should be, given a certain level of risk. The actual return may be quite different from the expected return. Applications of the CAPM include estimates of the expected return for capital budgeting, comparison of the actual return of a portfolio or portfolio manager with the CAPM return for performance appraisal, and the analysis of alternate return estimates and the CAPM returns as the basis for security selection. The applications are discussed in more detail in this section.

\section{Estimate of Expected Return}
Given an asset's systematic risk, the expected return can be calculated using the CAPM. Recall that the price of an asset is the sum of all future cash flows discounted at the required rate of return, where the discount rate or the required rate of return is commensurate with the asset's risk. The expected rate of return obtained from the CAPM is normally the first estimate that investors use for valuing assets, such as stocks, bonds, real estate, and other similar assets. The required rate of return from the CAPM is also used for capital budgeting and determining the economic feasibility of projects. Again, recall that when computing the net present value of a project, investments and net revenues are considered cash flows and are discounted at the required rate of return. The required rate of return, based on the project's risk, is calculated using the CAPM.

Because risk and return underlie almost all aspects of investment decision making, it is not surprising that the CAPM is used for estimating expected return in many scenarios. Other examples include calculating the cost of capital for regulated companies by regulatory commissions and setting fair insurance premiums. The next example shows an application of the CAPM to capital budgeting.

\section{EXAMPLE 9}
\section{Application of the CAPM to Capital Budgeting}
GlaxoSmithKline Plc is examining the economic feasibility of developing a new medicine. The initial investment in Year 1 is $\$ 500$ million. The investment in Year 2 is $\$ 200$ million. There is a 50 percent chance that the medicine will be developed and will be successful. If that happens, GlaxoSmithKline must spend another $\$ 100$ million in Year 3, but its income from the project in Year 3 will be $\$ 500$ million, not including the third-year investment. In Years 4,5 , and 6 , it will earn $\$ 400$ million a year if the medicine is successful. At the end of Year 6, it intends to sell all rights to the medicine for $\$ 600$ million. If the medicine is unsuccessful, none of GlaxoSmithKline's investments can be salvaged. Assume that the market return is 12 percent, the risk-free rate is 2 percent, and the beta risk of the project is 2.3 . All cash flows occur at the end of each year.

\begin{enumerate}
  \item Calculate the expected annual cash flows using the probability of success.
\end{enumerate}

\section{Solution:}
There is a 50 percent chance that the cash flows in Years 3-6 will occur.

Taking that into account, the expected annual cash flows are:

Year 1: - $\$ 500$ million (outflow)

Year 2: -\$200 million (outflow)

Year 3: 50\% of $-\$ 100$ million (outflow) $+50 \%$ of $\$ 500$ million $=\$ 200$ million

Year 4: $50 \%$ of $\$ 400$ million $=\$ 200$ million Year $5: 50 \%$ of $\$ 400$ million $=\$ 200$ million

Year $6: 50 \%$ of $\$ 400$ million $+50 \%$ of $\$ 600$ million $=\$ 500$ million

\begin{enumerate}
  \setcounter{enumi}{1}
  \item Calculate the expected return.
\end{enumerate}

\section{Solution to 2:}
The expected or required return for the project can be calculated using the CAPM, which is $=0.02+2.3 \times(0.12-0.02)=0.25$.

\begin{enumerate}
  \setcounter{enumi}{2}
  \item Calculate the net present value.
\end{enumerate}

\section{Solution:}
The net present value is the discounted value of all cash flows:

$$
\begin{aligned}
& N P V=\sum_{t=0}^{T} \frac{C F_{t}}{\left(1+r_{t}\right)^{t}} \\
= & \frac{-500}{(1+0.25)}+\frac{-200}{(1+0.25)^{2}}+\frac{200}{(1+0.25)^{3}}+\frac{200}{(1+0.25)^{4}} \\
& +\frac{200}{(1+0.25)^{5}}+\frac{500}{(1+0.25)^{6}} \\
= & -400-128+102.40+81.92+65.54+131.07=-147.07 .
\end{aligned}
$$

Because the net present value is negative (-\$147.07 million), the project should not be accepted by GlaxoSmithKline.

\section{BEYOND CAPM: LIMITATIONS AND EXTENSIONS OF CAPM}
describe and demonstrate applications of the CAPM and the SML

In general, return-generating models allow us to estimate an asset's return given its characteristics, where the asset characteristics required for estimating the return are specified in the model. Estimating an asset's return is important for investment decision making. These models are also important as a benchmark for evaluating portfolio, security, or manager performance. The return-generating models were briefly introduced in Section 3.2.1, and one of those models, the capital asset pricing model, was discussed in detail in Section 4.

The purpose of this section is to make readers aware that, although the CAPM is an important concept and model, the CAPM is not the only return-generating model. In this section, we revisit and highlight the limitations of the CAPM and preview return-generating models that address some of those limitations.

\section{Limitations of the CAPM}
The CAPM is subject to theoretical and practical limitations. Theoretical limitations are inherent in the structure of the model, whereas practical limitations are those that arise in implementing the model.

\section{Theoretical Limitations of the CAPM}
\begin{itemize}
  \item Single-factor model: Only systematic risk or beta risk is priced in the CAPM. Thus, the CAPM states that no other investment characteristics should be considered in estimating returns. As a consequence, it is prescriptive and easy to understand and apply, although it is very restrictive and inflexible.

  \item Single-period model: The CAPM is a single-period model that does not consider multi-period implications or investment objectives of future periods, which can lead to myopic and suboptimal investment decisions. For example, it may be optimal to default on interest payments in the current period to maximize current returns, but the consequences may be negative in the next period. A single-period model like the CAPM is unable to capture factors that vary over time and span several periods.

\end{itemize}

\section{Practical Limitations of the CAPM}
In addition to the theoretical limitations, implementation of the CAPM raises several practical concerns, some of which are listed below.

\begin{itemize}
  \item Market portfolio: The true market portfolio according to the CAPM includes all assets, financial and nonfinancial, which means that it also includes many assets that are not investable, such as human capital and assets in closed economies. Richard Roll $^{3}$ noted that one reason the CAPM is not testable is that the true market portfolio is unobservable.

  \item Proxy for a market portfolio: In the absence of a true market portfolio, market participants generally use proxies. These proxies, however, vary among analysts, the country of the investor, etc. and generate different return estimates for the same asset, which is impermissible in the CAPM.

  \item Estimation of beta risk: A long history of returns (three to five years) is required to estimate beta risk. The historical state of the company, however, may not be an accurate representation of the current or future state of the company. More generally, the CAPM is an ex ante model, yet it is usually applied using ex post data. In addition, using different periods for estimation results in different estimates of beta. For example, a three-year beta is unlikely to be the same as a five-year beta, and a beta estimated with daily returns is unlikely to be the same as the beta estimated with monthly returns. Thus, we are likely to estimate different returns for the same asset depending on the estimate of beta risk used in the model.

  \item The CAPM is a poor predictor of returns: If the CAPM is a good model, its estimate of asset returns should be closely associated with realized returns. However, empirical support for the CAPM is weak. ${ }^{4}$ In other words, tests of the CAPM show that asset returns are not determined only by systematic risk. Poor predictability of returns when using the CAPM is a serious limitation because return-generating models are used to estimate future returns.

  \item Homogeneity in investor expectations: The CAPM assumes that homogeneity exists in investor expectations for the model to generate a single optimal risky portfolio (the market) and a single security market line. Without this assumption, there will be numerous optimal risky portfolios and numerous security market lines. Clearly, investors can process the same information in a rational manner and arrive at different optimal risky portfolios.

\end{itemize}

3 Roll (1977).

\begin{enumerate}
  \setcounter{enumi}{3}
  \item See, for example, Fama and French (1992).
\end{enumerate}

\section{Extensions to the CAPM}
Given the limitations of the CAPM, it is not surprising that other models have been proposed to address some of these limitations. These new models are not without limitations of their own, which we will mention while discussing the models. We divide the models into two categories - theoretical models and practical models-and provide one example of each type.

\section{Theoretical Models}
Theoretical models are based on the same principle as the CAPM but expand the number of risk factors. The best example of a theoretical model is the arbitrage pricing theory (APT), which was developed by Stephen Ross. ${ }^{5}$ Like the CAPM, APT proposes a linear relationship between expected return and risk:

$$
E\left(R_{p}\right)=R_{F}+\lambda_{1} \beta_{p, 1}+\ldots+\lambda_{K} \beta_{p, K}
$$

where

$$
\begin{aligned}
& E\left(R_{p}\right)=\text { the expected return of portfolio } p \\
& R_{F}=\text { the risk-free rate } \\
& \lambda_{j}=\text { the risk premium (expected return in excess of the risk-free rate) for }
\end{aligned}
$$

Unlike the CAPM, however, APT allows numerous risk factors-as many as are relevant to a particular asset. Moreover, other than the risk-free rate, the risk factors need not be common and may vary from one asset to another. A no-arbitrage condition in asset markets is used to determine the risk factors and estimate betas for the risk factors.

Although it is theoretically elegant, flexible, and superior to the CAPM, APT is not commonly used in practice because it does not specify any of the risk factors and it becomes difficult to identify risk factors and estimate betas for each asset in a portfolio. So from a practical standpoint, the CAPM is preferred to APT.

\section{Practical Models}
If beta risk in the CAPM does not explain returns, which factors do? Practical models seek to answer this question through extensive research. As mentioned in Section 3.2.1, the best example of such a model is the four-factor model proposed by Fama and French (1992) and Carhart (1997).

Based on an analysis of the relationship between past returns and a variety of different factors, Fama and French (1992) proposed that three factors seem to explain asset returns better than just systematic risk. Those three factors are relative size, relative book-to-market value, and beta of the asset. With Carhart's (1997) addition of relative past stock returns, the model can be written as follows:

$$
E\left(R_{i t}\right)=\alpha_{i}+\beta_{i, M K T} M K T_{t}+\beta_{i, S M B} S M B_{t}+\beta_{i, H M L} H M L_{t}+\beta_{i, U M D} U M D_{t}
$$

where

$E\left(R_{i}\right)=$ the return on an asset in excess of the one-month T-bill return

$M K T=$ the excess return on the market portfolio

$S M B=$ the difference in returns between small-capitalization stocks and large-capitalization stocks (size)

$H M L=$ the difference in returns between high-book-to-market stocks and low-book-to-market stocks (value versus growth)

$U M D=$ the difference in returns of the prior year's winners and losers (momentum)

Historical analysis shows that the coefficient on $M K T$ is not significantly different from zero, which implies that stock return is unrelated to the market. The factors that explain stock returns are size (smaller companies outperform larger companies), book-to-market ratio (value companies outperform glamour companies), and momentum (past winners outperform past losers).

The four-factor model has been found to predict asset returns much better than the CAPM and is extensively used in estimating returns for US stocks.

Two observations are in order. First, the model is not underpinned by a theory of market equilibrium, as is the case for the CAPM. Second, there is no assurance that the model will continue to work well in the future.

\section{1}
\section{PORTFOLIO PERFORMANCE APPRAISAL MEASURES}
calculate and interpret the Sharpe ratio, Treynor ratio, $M^{2}$, and Jensen's alpha

In the investment industry, performance evaluation refers to the measurement, attribution, and appraisal of investment results. In particular, performance evaluation provides information about the return and risk of investment portfolios over specified investment period(s). By providing accurate data and analysis on investment decisions and their consequences, performance evaluation allows portfolio managers to take corrective measures to improve investment decision-making and management processes. Performance evaluation information helps in understanding and controlling investment risk and should, therefore, lead to improved risk management. Performance evaluation seeks to answer the following questions:

\begin{itemize}
  \item What was the investment portfolio's past performance, and what may be expected in the future?
\end{itemize}

Answering this question is the subject of performance measurement. Performance measurement is concerned with the measurement of return and risk.

\begin{itemize}
  \item How did the investment portfolio produce its observed performance, and what are the expected sources of expected future performance? Answering this question is the subject of performance attribution. Performance attribution is concerned with identifying and quantifying the sources of performance of a portfolio.

  \item Was the observed investment portfolio's performance the result of investment skill or luck?

\end{itemize}

Answering this question is the subject of performance appraisal. Performance appraisal is concerned with identifying and measuring investment skill.

The information provided by performance evaluation is of great interest to all stakeholders in the investment management process because of its value in evaluating the overall quality of the investment management process as well as individual investment decisions.

In this reading, performance appraisal is based only on the CAPM. However, it is easy to extend this analysis to multi-factor models that may include industry or other special factors. Four ratios are commonly used in performance appraisal.

\section{The Sharpe Ratio}
Performance has two components, risk and return. Although return maximization is a laudable objective, comparing just the return of a portfolio with that of the market is not sufficient. Because investors are risk averse, they will require compensation for higher risk in the form of higher returns. A commonly used measure of performance is the Sharpe ratio, which is defined as the portfolio's risk premium divided by its risk. An appealing feature of the Sharpe ratio is that its use can be justified on a theoretical ex ante (before the fact) basis and ex post (after the fact) values can easily be determined by using readily available market data. The Sharpe ratio is also easy to interpret, essentially being an efficiency ratio relating reward to risks taken. It is the most widely recognized and used appraisal measure.

The equation below defines the ex ante Sharpe ratio in terms of three inputs: (1) the portfolio's expected return, $E\left(R_{p}\right)$; (2) the risk-free rate of interest, $R_{j}$ and (3) the portfolio's ex ante standard deviation of returns (return volatility), $\sigma_{p}$, a quantitative measure of total risk.

$$
\mathrm{SR}=\frac{E\left(R_{p}\right)-R_{F}}{\sigma_{p}}
$$

The Sharpe ratio can also be used on an ex post basis to evaluate historical risk-adjusted returns. Assume we have a sample of historical data that can be used to determine the sample mean portfolio return, $\bar{R}_{p}$; the standard deviation of the sample returns, here denoted by $\hat{\sigma}_{p}\left(s_{p}\right.$ is a familiar notation in other contexts); and the sample mean risk-free rate, $\bar{R}_{f}$. The ex post (or realized or historical) Sharpe ratio can then be determined by using the following:

$$
\widehat{S R}=\frac{\bar{R}_{p}-\bar{R}_{F}}{\hat{\sigma}_{p}}
$$

Recalling the CAL from earlier in the reading, one can see that the Sharpe ratio, also called the reward-to-variability ratio, is simply the slope of the capital allocation line. Note, however, that the ratio uses the total risk of the portfolio, not its systematic risk. The use of total risk is appropriate if the portfolio is an investor's total portfolio-that is, the investor does not own any other assets. Sharpe ratios of the market and other portfolios can also be calculated in a similar manner. The portfolio with the highest Sharpe ratio has the best risk-adjusted performance, and the one with the lowest Sharpe ratio has the worst risk-adjusted performance, provided that the numerator is positive for all comparison portfolios. If the numerator is negative, the ratio will be less negative for riskier portfolios, resulting in incorrect rankings. The Sharpe ratio, however, suffers from two limitations. First, it uses total risk as a measure of risk when only systematic risk is priced. Second, the ratio itself (e.g., 0.2 or 0.3 ) is not informative. To rank portfolios, the Sharpe ratio of one portfolio must be compared with the Sharpe ratio of another portfolio. Nonetheless, the ease of computation makes the Sharpe ratio a popular tool.

\section{The Treynor Ratio}
The Treynor ratio is a simple extension of the Sharpe ratio and resolves the Sharpe ratio's first limitation by substituting beta (systematic risk) for total risk. The ex ante and ex post Treynor ratios are provided below.

$$
\begin{aligned}
T R & =\frac{E\left(R_{p}\right)-R_{f}}{\beta_{p}} \\
\widehat{T R} & =\frac{\bar{R}_{p}-\bar{R}_{f}}{\widehat{\beta}_{p}}
\end{aligned}
$$

Just like the Sharpe ratio, the numerators must be positive for the Treynor ratio to give meaningful results. In addition, the Treynor ratio does not work for negative-beta assets-that is, the denominator must also be positive for obtaining correct estimates and rankings. Although both the Sharpe and Treynor ratios allow for ranking of portfolios, neither ratio gives any information about the economic significance of differences in performance. For example, assume the Sharpe ratio of one portfolio is 0.75 and the Sharpe ratio for another portfolio is 0.80 . The second portfolio is superior, but is that difference meaningful? In addition, we do not know whether either of the portfolios is better than the passive market portfolio. The remaining two measures, $\mathrm{M}^{2}$ and Jensen's alpha, attempt to address that problem by comparing portfolios while also providing information about the extent of the overperformance or underperformance.

\section{$M^{2}$ : Risk-Adjusted Performance (RAP)}
$\mathbf{M}^{\mathbf{2}}$ provides a measure of portfolio return that is adjusted for the total risk of the portfolio relative to that of some benchmark. In 1997, Nobel Prize winner Franco Modigliani and his granddaughter, Leah Modigliani, developed what they called a risk-adjusted performance measure, or RAP. The RAP measure has since become more commonly known as $\mathrm{M}^{2}$ reflecting the Modigliani names. It is related to the Sharpe ratio and ranks portfolios identically, but it has the useful advantage of being denominated in familiar terms of percentage return advantage assuming the same level of total risk as the market

$\mathrm{M}^{2}$ borrows from capital market theory by assuming a portfolio is leveraged or de-leveraged until its volatility (as measured by standard deviation) matches that of the market. This adjustment produces a portfolio-specific leverage ratio that equates the portfolio's risk to that of the market. The portfolio's excess return times the leverage ratio plus the risk-free rate is then compared with the markets actual return to determine whether the portfolio has outperformed or underperformed the market on a risk-adjusted basis.

The equations below provide the ex ante and ex post formulas for $\mathrm{M}^{2}$, where $\sigma_{m}$ is the standard deviation of the market portfolio and $\sigma_{m} / \sigma_{p}$ is the portfolio-specific leverage ratio. Because the Sharpe ratio is defined as $\frac{E\left(R_{p}\right)-R_{f}}{\sigma_{p}}$, the equation shows that $\mathrm{M}^{2}$ can be thought of as a rescaling of the Sharpe ratio that allows for easier comparisons among different portfolios. The reason that $\mathrm{M}^{2}$ and Sharpe ratios rank portfolios identically is because, in a given time period-and for any given comparison of the market portfolio-both the risk-free rate and the market volatility are constant across all comparisons. Only the Sharpe ratio differs, so it determines all rankings.

$$
\begin{aligned}
& \mathrm{M}^{2}=\left[E\left(R_{p}\right)-R_{f}\right] \frac{\sigma_{m}}{\sigma_{p}}+R_{f}=\mathrm{SR} \times \sigma_{m}+R_{f}(\text { ex ante }) \\
& \left.\widehat{\mathrm{M}^{2}}=\left(\bar{R}_{p}-\bar{R}_{f}\right) \widehat{\sigma}_{m} \widehat{\sigma}_{p}+R_{f}=\widehat{\mathrm{SR}} \times \widehat{\sigma}_{m}+R_{f} \text { (ex post }\right)
\end{aligned}
$$

For example, assume that $\bar{R}_{f}=4.0 \%, \bar{R}_{p}=14.0 \%, \hat{\sigma}_{p}=25.0 \%$ and $\hat{\sigma}_{m}=20.0 \%$. The Sharpe ratio is $0.4, \widehat{S R}=\frac{0.14-0.04}{0.25}=0.4$, and $\widehat{M^{2}}$ is $12.0 \%, \widehat{M^{2}}=0.4(0.2)+0.04=$ $0.12=12.0 \%$. If the market return was $10 \%$, then the portfolio outperformed the market on a risk-adjusted basis by $12.0 \%-10.0 \%=2.0 \%$. This difference between the risk-adjusted performance of the portfolio and the performance of the market is frequently referred to as $\mathbf{M}^{\mathbf{2}}$ alpha.

The Sharpe ratio of the market portfolio is $\widehat{S R}=\frac{0.10-0.04}{0.20}=0.3$. Comparing the Sharpe ratio of the portfolio with the Sharpe ratio of the market portfolio shows that the fund outperformed the market. But the $2.0 \%$ difference between $\mathrm{M}^{2}$ and the market's return tells us the risk-adjusted outperformance as a percentage return.

\section{Jensen's Alpha}
Like the Treynor ratio, Jensen's alpha is based on systematic risk. We can measure a portfolio's systematic risk by estimating the market model, which is done by regressing the portfolio's daily return on the market's daily return. The coefficient on the market return is an estimate of the beta risk of the portfolio. We can calculate the risk-adjusted return of the portfolio using the beta of the portfolio and the CAPM. The difference between the actual portfolio return and the calculated risk-adjusted return is a measure of the portfolio's performance relative to the market portfolio and is called Jensen's alpha. By definition, $\alpha_{m}$ of the market is zero. Jensen's alpha is also the vertical distance from the SML measuring the excess return for the same risk as that of the market and is given by

$$
\alpha_{p}=R_{p}-\left\{R_{f}+\beta_{p}\left[\mathrm{E}\left(R_{m}\right)-R_{f}\right]\right\}
$$

If the period is long, it may contain different risk-free rates, in which case $R_{f}$ represents the average risk-free rate. Furthermore, the returns in the equation are all realized, actual returns. The sign of $\alpha_{p}$ indicates whether the portfolio has outperformed the market. If $\alpha_{p}$ is positive, then the portfolio has outperformed the market; if $\alpha_{p}$ is negative, the portfolio has underperformed the market. Jensen's alpha is commonly used for evaluating most institutional managers, pension funds, and mutual funds. Values of alpha can be used to rank different managers and the performance of their portfolios, as well as the magnitude of underperformance or overperformance. For example, if a portfolio's alpha is 2 percent and another portfolio's alpha is 5 percent, the second portfolio has outperformed the first portfolio by 3 percentage points and the market by 5 percentage points. Jensen's alpha is the maximum amount that you should be willing to pay the manager to manage your money. As with other performance appraisal measures, Jensen's alpha has ex ante and ex post forms. The use context usually clarifies which one is being referred to. Where we want to underscore a reference to ex post Jensen's alpha based on an estimated beta, $\widehat{\beta}_{p}$, and an average market return, the notation $\hat{\alpha}_{p}$ is used.

\section{EXAMPLE 10}
\section{Portfolio Performance Evaluation}
\begin{enumerate}
  \item A British pension fund has employed three investment managers, each of whom is responsible for investing in one-third of all asset classes so that the pension fund has a well-diversified portfolio. Information about the managers is given below.
\end{enumerate}

\begin{center}
\begin{tabular}{lccc}
\hline
Manager & Average Return & $\hat{\sigma}$ & $\hat{\beta}$ \\
\hline
X & $10 \%$ & $20 \%$ & 1.1 \\
Y & 11 & 10 & 0.7 \\
Z & 12 & 25 & 0.6 \\
Market $(M)$ & 9 & 19 &  \\
Risk-free rate $\left(R_{f}\right)$ & 3 &  &  \\
\hline
\end{tabular}
\end{center}

Calculate the expected return for each manager, based on using the average market return and the CAPM. Then also calculate for the managers (ex post) Sharpe ratio, Treynor ratio, $\mathrm{M}^{2}$ alpha, and Jensen's alpha. Analyze your results and plot the returns and betas of these portfolios.

\section{Solution:}
In each case, the calculations are shown only for Manager X. All answers are tabulated below. Note that the $\beta$ of the market is 1 and the $\sigma$ and $\beta$ of the risk-free rate are both zero.

Expected return: $\quad E\left(R_{X}\right)=R_{f}+\beta_{X}\left[E\left(R_{m}\right)-R_{f}\right]=0.03+1.10$

$$
\begin{aligned}
& \times(0.09-0.03)=0.096=9.6 \% \\
& \widehat{\mathrm{SR}}=\frac{\bar{R}_{x}-\bar{R}_{f}}{\hat{\sigma}_{x}}=\frac{0.10-0.03}{0.20}=0.35 \\
& \widehat{\mathrm{TR}}=\frac{\bar{R}_{x}-\bar{R}_{f}}{\widehat{\beta}_{x}}=\frac{0.10-0.03}{1.1}=0.064 \\
& \widehat{M^{2}}=\left(\bar{R}_{x}-\bar{R}_{f}\right) \frac{\hat{\sigma}_{m}}{\hat{\sigma}_{x}}+\bar{R}_{f}=\widehat{S R} \times \hat{\sigma}_{m}+\bar{R}_{f} \\
& =0.35 \times 0.19+0.03=0.0965=9.65 \%
\end{aligned}
$$

Since the market return is $9 \%, \mathrm{M}^{2}$ alpha is $0.65 \%$ (9.65\%-9\%). $\widehat{\alpha}_{X}=R_{X}-\left[\bar{R}_{f}+\hat{\beta}_{X}\left(\bar{R}_{m}-\bar{R}_{f}\right)\right]=0.10-(0.03+1.1 \times 0.06)$ $=0.004=0.40 \%$

\section{Exhibit 8: Measures of Portfolio Performance Evaluation}
\begin{center}
\begin{tabular}{|c|c|c|c|c|c|c|c|c|}
\hline
Manager & $\bar{R}_{i}$ & $\hat{\sigma}_{i}$ & $\hat{\beta}_{i}$ & $E\left(R_{\boldsymbol{j}}\right)$ & Sharpe Ratio & Treynor Ratio & $M^{2}$ alpha & $\hat{a}_{i}$ \\
\hline
$\mathrm{X}$ & $10.0 \%$ & $20.0 \%$ & 1.10 & $9.6 \%$ & 0.35 & 0.064 & $0.65 \%$ & $0.40 \%$ \\
\hline
Y & 11.0 & 10.0 & 0.70 & 7.2 & 0.80 & 0.114 & 9.20 & 3.80 \\
\hline
Z & 12.0 & 25.0 & 0.60 & 6.6 & 0.36 & 0.150 & 0.84 & 5.40 \\
\hline
M & 9.0 & 19.0 & 1.00 & 9.0 & 0.32 & 0.060 & 0.00 & 0.00 \\
\hline
$R_{f}$ & 3.0 & 0.0 & 0.00 & 3.0 & - & - & - & 0.00 \\
\hline
\end{tabular}
\end{center}

Let us begin with an analysis of the risk-free asset. Because the risk-free asset has zero risk and a beta of zero, calculating the Sharpe ratio, Treynor ratio, or $\mathrm{M}^{2}$ is not possible because they all require the portfolio risk in the denominator. The risk-free asset's alpha, however, is zero. Turning to the market portfolio, we see that the absolute measures of performance, the Sharpe ratio and the Treynor ratio, are positive for the market portfolio. These ratios are positive as long as the portfolio earns a return that is in excess of that of the risk-free asset. $\widehat{\mathrm{M}}^{2}$ and $\widehat{\alpha}_{i}$ are performance measures relative to the market, so they are both equal to zero for the market portfolio. All three managers have Sharpe and Treynor ratios greater than those of the market, and all three managers' $\mathrm{M}^{2}$ alpha and $\alpha_{i}$ are positive; therefore, the pension fund should be satisfied with their performance. Among the three managers, Manager $\mathrm{X}$ has the worst performance, irrespective of whether total risk or systematic risk is considered for measuring performance. The relative rankings are depicted in Exhibit 9.

\section{Exhibit 9: Ranking of Portfolios by Performance Measure}
\begin{center}
\begin{tabular}{lcccc}
\hline
Rank & Sharpe Ratio & Treynor Ratio & $\mathbf{M}^{2}$ alpha & $\boldsymbol{a}_{\boldsymbol{i}}$ \\
\hline
1 & $\mathrm{Y}$ & $\mathrm{Z}$ & $\mathrm{Y}$ & $\mathrm{Z}$ \\
2 & $\mathrm{Z}$ & $\mathrm{Y}$ & $\mathrm{Z}$ & $\mathrm{Y}$ \\
3 & $\mathrm{X}$ & $\mathrm{X}$ & $\mathrm{X}$ & $\mathrm{X}$ \\
4 & $\mathrm{M}$ & $\mathrm{M}$ & $\mathrm{M}$ & $R_{f}$ \\
\hline
\end{tabular}
\end{center}

Comparing $\mathrm{Y}$ and $\mathrm{Z}$, we can observe that $\mathrm{Y}$ performs much better than $\mathrm{Z}$ when total risk is considered. $Y$ has a Sharpe ratio of 0.80 , compared with a Sharpe ratio of 0.36 for $\mathrm{Z}$. Similarly, $\mathrm{M}^{2}$ alpha is higher for $\mathrm{Y}$ ( 9.20 percent) than for $\mathrm{Z}$ (0.84 percent). In contrast, when systematic risk is used, $\mathrm{Z}$ outperforms Y. The Treynor ratio is higher for $\mathrm{Z}(0.150)$ than for $Y(0.114)$, and Jensen's alpha is also higher for $\mathrm{Z}$ (5.40 percent) than for $\mathrm{Y}$ (3.80 percent), which indicates that $Z$ has done a better job of generating excess return relative to systematic risk than $\mathrm{Y}$.

Exhibit 10 confirms these observations in that all three managers outperform the benchmark because all three points lie above the SML. Among the three portfolios, $\mathrm{Z}$ performs the best when we consider risk-adjusted returns because it is the point in Exhibit 10 that is located northwest relative to the portfolios $\mathrm{X}$ and $\mathrm{Y}$.

\section{Exhibit 10: Portfolios Along the SML}
\begin{center}
\includegraphics[max width=\textwidth]{2023_05_04_36535b8d80b32081d422g-666}
\end{center}

When do we use total risk performance measures like the Sharpe ratio and $\mathrm{M}^{2}$, and when do we use beta risk performance measures like the Treynor ratio and Jensen's alpha? Total risk is relevant for an investor when he or she holds a portfolio that is not fully diversified, which is not a desirable portfolio. In such cases, the Sharpe ratio and $\mathrm{M}^{2}$ are appropriate performance measures. Thus, if the pension fund were to choose only one fund manager to manage all its assets, it should choose Manager Y. Performance measures relative to beta risk-Treynor ratio and Jensen's alpha-are relevant when the investor holds a well-diversified portfolio with negligible diversifiable risk. In other words, if the pension fund is well diversified and only the systematic risk of the portfolio matters, the fund should choose Manager $\mathrm{Z}$.

The measures of performance evaluation assume that the market portfolio is the correct benchmark. As a result, an error in the benchmark may cause the results to be misleading. For example, evaluating a real estate fund against the S\&P 500 is incorrect because real estate has different characteristics than equity. In addition to errors in benchmarking, errors could occur in the measurement of risk and return of the market portfolio and the portfolios being evaluated. Finally, many estimates are based on historical data. Any projections based on such estimates assume that this level of performance will continue in the future.

APPLICATIONS OF THE CAPM IN PORTFOLIO CONSTRUCTION

calculate and interpret the expected return of an asset using the CAPM

describe and demonstrate applications of the CAPM and the SML This section introduces applications of the CAPM in portfolio construction. First, the security characteristic line, which graphically indicates ex post Jensen's alpha, is described. If we relax the assumption that investors have the same expectations about risk and return, a positive Jensen's alpha can be interpreted as an indication of superior information or investment ability. The section on security selection covers that possibility. The last section summarizes how the CAPM and related concepts can be applied to portfolio construction.

\section{Security Characteristic Line}
Similar to the SML, we can draw a security characteristic line (SCL) for a security. The SCL is a plot of the excess return of the security on the excess return of the market. In Exhibit 8, Jensen's alpha is the intercept and the beta is the slope. The equation of the line can be obtained by rearranging the terms in the expression for Jensen's alpha and replacing the subscript $p$ with $i$ :

$$
R_{i}-R_{f}=\alpha_{i}+\beta_{i}\left(R_{m}-R_{f}\right)
$$

As an example, the SCL is drawn in Exhibit 11 using Manager X's portfolio from Exhibit 8. The security characteristic line can also be estimated by regressing the excess security return, $R_{i}-R_{f}$ on the excess market return, $R_{m}-R_{f}$

\section{Exhibit 11: The Security Characteristic Line}
\begin{center}
\includegraphics[max width=\textwidth]{2023_05_04_36535b8d80b32081d422g-667}
\end{center}

\section{Security Selection}
When discussing the CAPM, we assumed that investors have homogeneous expectations and are rational, risk-averse, utility-maximizing investors. With these assumptions, we were able to state that all investors assign the same value to all assets and, therefore, have the same optimal risky portfolio, which is the market portfolio. In other words, we assumed that there is commonality among beliefs about an asset's future cash flows and the required rate of return. Given the required rate of return, we can discount the future cash flows of the asset to arrive at its current value, or price, which is agreed upon by all or most investors.

In this section, we introduce heterogeneity in beliefs of investors. Because investors are price takers, it is assumed that such heterogeneity does not significantly affect the market price of an asset. The difference in beliefs can relate to future cash flows, the systematic risk of the asset, or both. Because the current price of an asset is the discounted value of the future cash flows, the difference in beliefs could result in an investor-estimated price that is different from the CAPM-calculated price. The CAPM-calculated price is the current market price because it reflects the beliefs of all other investors in the market. If the investor-estimated current price is higher (lower) than the market price, the asset is considered undervalued (overvalued). Therefore, the CAPM is an effective tool for determining whether an asset is undervalued or overvalued and whether an investor should buy or sell the asset.

Although portfolio performance evaluation is backward looking and security selection is forward looking, we can apply the concepts of portfolio evaluation to security selection. The best measure to apply is Jensen's alpha because it uses systematic risk and is meaningful even on an absolute basis. A positive Jensen's alpha indicates a superior security, whereas a negative Jensen's alpha indicates a security that is likely to underperform the market when adjusted for risk.

Another way of presenting the same information is with the security market line. Potential investors can plot a security's expected return and beta against the SML and use this relationship to decide whether the security is overvalued or undervalued in the market. ${ }^{6}$ Exhibit 12 shows a number of securities along with the SML. All securities that reflect the consensus market view are points directly on the SML (i.e., properly valued). If a point representing the estimated return of an asset is above the SML (Points A and C), the asset has a low level of risk relative to the amount of expected return and would be a good choice for investment. In contrast, if the point representing a particular asset is below the SML (Point B), the stock is considered overvalued. Its return does not compensate for the level of risk and should not be considered for investment. Of course, a short position in Asset B can be taken if short selling is permitted.

6 In this reading, we do not consider transaction costs, which are important whenever deviations from a passive portfolio are considered. Thus, the magnitude of undervaluation or overvaluation should be considered in relation to transaction costs prior to making an investment decision. Exhibit 12: Security Selection Using SML

\begin{center}
\includegraphics[max width=\textwidth]{2023_05_04_36535b8d80b32081d422g-669}
\end{center}

\section{Implications of the CAPM for Portfolio Construction}
Based on the CAPM, investors should hold a combination of the risk-free asset and the market portfolio. The true market portfolio consists of a large number of securities, and an investor would have to own all of them in order to be completely diversified. Because owning all existing securities is not practical, in this section, we will consider an alternate method of constructing a portfolio that may not require a large number of securities and will still be sufficiently diversified. Exhibit 13 shows the reduction in risk as we add more and more securities to a portfolio. As can be seen from the exhibit, much of the nonsystematic risk can be diversified away in as few as 30 securities. These securities, however, should be randomly selected and represent different asset classes for the portfolio to effectively diversify risk. Otherwise, one may be better off using an index (e.g., the S\&P 500 for a diversified large-cap equity portfolio and other indexes for other asset classes).

\section{Exhibit 13: Diversification with Number of Stocks}
\begin{center}
\includegraphics[max width=\textwidth]{2023_05_04_36535b8d80b32081d422g-670}
\end{center}

Let's begin constructing the optimal portfolio with a portfolio of securities like the S\&P 500. Although the S\&P 500 is a portfolio of 500 securities, it is a good starting point because it is readily available as a single security for trading. In contrast, it represents only the large corporations that are traded on the US stock markets and, therefore, does not encompass the global market entirely. Because the S\&P 500 is the base portfolio, however, we treat it is as the market for the CAPM.

Any security not included in the S\&P 500 can be evaluated to determine whether it should be integrated into the portfolio. That decision is based on the $\alpha_{i}$ of the security, which is calculated using the CAPM with the S\&P 500 as the market portfolio. Note that security $i$ may not necessarily be priced incorrectly for it to have a non-zero $\alpha_{i}$; $\alpha_{i}$ can be positive merely because it is not well correlated with the S\&P 500 and its return is sufficient for the amount of systematic risk it contains. For example, assume a new stock market, ABC, opens to foreign investors only and is being considered for inclusion in the portfolio. We estimate ABC's model parameters relative to the S\&P 500 and find an $\alpha_{i}$ of approximately 3 percent, with a $\beta_{i}$ of 0.60 . Because $\alpha_{i}$ is positive, ABC should be added to the portfolio. Securities with a significantly negative $\alpha_{i}$ may be short sold to maximize risk-adjusted return. For convenience, however, we will assume that negative positions are not permitted in the portfolio.

In addition to the securities that are correctly priced but enter the portfolio because of their risk-return superiority, securities already in the portfolio (S\&P 500) may be undervalued or overvalued based on investor expectations that are incongruent with the market. Securities in the S\&P 500 that are overvalued (negative $\alpha_{i}$ ) should be dropped from the S\&P 500 portfolio, if it is possible to exclude individual securities, and positions in securities in the S\&P 500 that are undervalued (positive $\alpha_{i}$ ) should be increased.

This brings us to the next question: What should the relative weight of securities in the portfolio be? Because we are concerned with maximizing risk-adjusted return, securities with a higher $\alpha_{i}$ should have a higher weight, and securities with greater nonsystematic risk should be given less weight in the portfolio. A complete analysis of portfolio optimization is beyond the scope of this reading, but the following principles are helpful. The weight in each nonmarket security should be proportional to $\frac{\alpha_{i}}{\sigma_{e i}^{2}}$, where the denominator is the nonsystematic variance of security $i$. The total weight of nonmarket securities in the portfolio is proportional to $\frac{\sum_{i=1}^{N} w_{i} \alpha_{i}}{\sum_{i=1}^{N} w_{i}^{2} \sigma_{e i}^{2}}$. The weight in the market portfolio is a function of $\frac{E\left(R_{m}\right)}{\sigma_{m}^{2}}$. The information ratio, $\frac{\alpha_{i}}{\sigma_{e i}}$ (i.e., alpha divided by nonsystematic risk), measures the abnormal return per unit of risk added by the security to a well-diversified portfolio. The larger the information ratio is, the more valuable the security.

\section{EXAMPLE 11}
\section{Optimal Investor Portfolio with Heterogeneous Beliefs}
A Japanese investor is holding the Nikkei 225 index, which is her version of the market. She thinks that three stocks, P, Q, and R, which are not in the Nikkei 225, are undervalued and should form a part of her portfolio. She has the following information about the stocks, the Nikkei 225, and the risk-free rate (the information is given as expected return, standard deviation, and beta):

P: $15 \%, 30 \%, 1.5$

Q: 18\%, 25\%, 1.2

R: $16 \%, 23 \%, 1.1$

Nikkei 225: 12\%, 18\%, 1.0

Risk-free rate: $2 \%, 0 \%, 0.0$

\begin{enumerate}
  \item Calculate Jensen's alpha for $P, Q$, and $R$.
\end{enumerate}

\section{Solution:}
Stock P's $\alpha: R_{i}-\left[R_{f}+\beta_{i}\left(R_{m}-R_{f}\right)\right]=0.15-(0.02+1.5 \times 0.10)=-0.02$

Stock Q’s $\alpha: R_{i}-\left[R_{f}+\beta_{i}\left(R_{m}-R_{f}\right)\right]=0.18-(0.02+1.2 \times 0.10)=0.04$

Stock R's $\alpha: R_{i}-\left[R_{f}+\beta_{i}\left(R_{m}-R_{f}\right)\right]=0.16-(0.02+1.1 \times 0.10)=0.03$

\begin{enumerate}
  \setcounter{enumi}{1}
  \item Calculate nonsystematic variance for $P, Q$, and $R$.
\end{enumerate}

\section{Solution:}
Total variance $=$ Systematic variance + Nonsystematic variance. From Section 3.2.2, we write the equation as $\sigma_{e i}^{2}=\sigma_{i}^{2}-\beta_{i}^{2} \sigma_{m}^{2}$.

Stock P's nonsystematic variance $=(0.30 \times 0.30)-(1.5 \times 1.5 \times 0.18 \times 0.18)=$ $0.09-0.0729=0.0171$

Stock Q's nonsystematic variance $=(0.25 \times 0.25)-(1.2 \times 1.2 \times 0.18 \times 0.18)=$ $0.0625-0.0467=0.0158$

Stock R's nonsystematic variance $=(0.23 \times 0.23)-(1.1 \times 1.1 \times 0.18 \times 0.18)=$ $0.0529-0.0392=0.0137$ 3. Should any of the three stocks be included in the portfolio? If so, which stock should have the highest weight in the portfolio?

Solution:

Stock $P$ has a negative $\alpha$ and should not be included in the portfolio, unless a negative position can be assumed through short selling. Stocks $Q$ and $R$ have a positive $\alpha$; therefore, they should be included in the portfolio with positive weights.

The relative weight of $\mathrm{Q}$ is $0.04 / 0.0158=2.53$.

The relative weight of $R$ is $0.03 / 0.0137=2.19$.

Stock $\mathrm{Q}$ will have the largest weight among the nonmarket securities to be added to the portfolio. In relative terms, the weight of $\mathrm{Q}$ will be 15.5 percent greater than the weight of $\mathrm{R}(2.53 / 2.19=1.155)$. As the number of securities increases, the analysis becomes more complex. However, the contribution of each additional security toward improvement in the risk-return trade-off will decrease and eventually disappear, resulting in a well-diversified portfolio.

\section{SUMMARY}
In this reading, we discussed the capital asset pricing model in detail and covered related topics such as the capital market line. The reading began with an interpretation of the CML, uses of the market portfolio as a passive management strategy, and leveraging of the market portfolio to obtain a higher expected return. Next, we discussed systematic and nonsystematic risk and why one should not expect to be compensated for taking on nonsystematic risk. The discussion of systematic and nonsystematic risk was followed by an introduction to beta and return-generating models. This broad topic was then broken down into a discussion of the CAPM and, more specifically, the relationship between beta and expected return. The final section included applications of the CAPM to capital budgeting, portfolio performance evaluation, and security selection. The highlights of the reading are as follows.

\begin{itemize}
  \item The capital market line is a special case of the capital allocation line, where the efficient portfolio is the market portfolio.

  \item Obtaining a unique optimal risky portfolio is not possible if investors are permitted to have heterogeneous beliefs because such beliefs will result in heterogeneous asset prices.

  \item Investors can leverage their portfolios by borrowing money and investing in the market.

  \item Systematic risk is the risk that affects the entire market or economy and is not diversifiable.

  \item Nonsystematic risk is local and can be diversified away by combining assets with low correlations.

  \item Beta risk, or systematic risk, is priced and earns a return, whereas nonsystematic risk is not priced.

  \item The expected return of an asset depends on its beta risk and can be computed using the CAPM, which is given by $E\left(R_{i}\right)=R_{f}+\beta_{i}\left[E\left(R_{m}\right)-R_{f}\right]$. - The security market line is an implementation of the CAPM and applies to all securities, whether they are efficient or not.

  \item Expected return from the CAPM can be used for making capital budgeting decisions.

  \item Portfolios can be evaluated by several CAPM-based measures, such as the Sharpe ratio, the Treynor ratio, $M^{2}$, and Jensen's alpha.

  \item The SML can assist in security selection and optimal portfolio construction.

\end{itemize}

By successfully understanding the content of this reading, you should feel comfortable decomposing total variance into systematic and nonsystematic variance, analyzing beta risk, using the CAPM, and evaluating portfolios and individual securities.

\section{REFERENCES}
Carhart, Mark. 1997. “On Persistence in Mutual Fund Performance." Journal of Finance, vol. 52, no. 1:57-82. 10.2307/2329556

Fama, Eugene, Kenneth French. 1992. "The Cross-Section of Expected Stock Returns." Journal of Finance, vol. 47 , no. 2:427-466. 10.2307/2329112

\section{PRACTICE PROBLEMS}
\begin{enumerate}
  \item The line depicting the total risk and expected return of portfolio combinations of a risk-free asset and any risky asset is the:
\end{enumerate}

A. security market line.

B. capital allocation line.

C. security characteristic line.

\begin{enumerate}
  \setcounter{enumi}{1}
  \item The portfolio of a risk-free asset and a risky asset has a better risk-return tradeoff than investing in only one asset type because the correlation between the risk-free asset and the risky asset is equal to:
A. -1.0 .
B. 0.0
C. 1.0 .

  \item With respect to capital market theory, an investor's optimal portfolio is the combination of a risk-free asset and a risky asset with the highest:
A. expected return.
B. indifference curve.
C. capital allocation line slope.

  \item Highly risk-averse investors will most likely invest the majority of their wealth in:
A. risky assets.
B. risk-free assets.
C. the optimal risky portfolio.

  \item The capital market line (CML) is the graph of the risk and return of portfolio combinations consisting of the risk-free asset and:
A. any risky portfolio.
B. the market portfolio.
C. the leveraged portfolio.

  \item Which of the following statements most accurately defines the market portfolio in capital market theory? The market portfolio consists of all:
A. risky assets.
B. tradable assets.
C. investable assets.

  \item With respect to capital market theory, the optimal risky portfolio:

\end{enumerate}

A. is the market portfolio. B. has the highest expected return.

C. has the lowest expected variance.

\begin{enumerate}
  \setcounter{enumi}{7}
  \item Relative to portfolios on the CML, any portfolio that plots above the CML is considered:
A. inferior.
B. inefficient.
C. unachievable.

  \item A portfolio on the capital market line with returns greater than the returns on the market portfolio represents a $(\mathrm{n})$ :

\end{enumerate}

A. lending portfolio.

B. borrowing portfolio.

C. unachievable portfolio.

\begin{enumerate}
  \setcounter{enumi}{9}
  \item With respect to the capital market line, a portfolio on the CML with returns less than the returns on the market portfolio represents $\mathrm{a}(\mathrm{n})$ :
A. lending portfolio.
B. borrowing portfolio.
C. unachievable portfolio.

  \item Which of the following types of risk is most likely avoided by forming a diversified portfolio?
A. Total risk.
B. Systematic risk.
C. Nonsystematic risk.

  \item Which of the following events is most likely an example of nonsystematic risk?
A. A decline in interest rates.
B. The resignation of chief executive officer.
C. An increase in the value of the US dollar.

  \item With respect to the pricing of risk in capital market theory, which of the following statements is most accurate?

\end{enumerate}

A. All risk is priced.

B. Systematic risk is priced.

C. Nonsystematic risk is priced.

\begin{enumerate}
  \setcounter{enumi}{13}
  \item The sum of an asset's systematic variance and its nonsystematic variance of returns is equal to the asset's:
\end{enumerate}

A. beta. B. total risk.

C. total variance.

\section{The following information relates to questions}
 15-17An analyst gathers the following information:

\begin{center}
\begin{tabular}{|c|c|c|c|}
\hline
Security & $\begin{array}{c}\text { Expected } \\ \text { Annual Return (\%) }\end{array}$ & $\begin{array}{c}\text { Expected } \\ \text { Standard Deviation (\%) }\end{array}$ & $\begin{array}{c}\text { Correlation between } \\ \text { Security and the } \\ \text { Market }\end{array}$ \\
\hline
Security 1 & 11 & 25 & 0.6 \\
\hline
Security 2 & 11 & 20 & 0.7 \\
\hline
Security 3 & 14 & 20 & 0.8 \\
\hline
Market & 10 & 15 & 1.0 \\
\hline
\end{tabular}
\end{center}

\begin{enumerate}
  \setcounter{enumi}{14}
  \item Which security has the highest total risk?
A. Security 1 .
B. Security 2 .
C. Security 3 .

  \item Which security has the highest beta measure?
A. Security 1 .
B. Security 2
C. Security 3 .

  \item Which security has the least amount of market risk?
A. Security 1
B. Security 2 .
C. Security 3 .

  \item With respect to return-generating models, the intercept term of the market model is the asset's estimated:
A. beta.
B. alpha.
C. variance.

  \item With respect to return-generating models, the slope term of the market model is an estimate of the asset's:

\end{enumerate}

A. total risk. B. systematic risk.

C. nonsystematic risk.

\begin{enumerate}
  \setcounter{enumi}{19}
  \item With respect to return-generating models, which of the following statements is most accurate? Return-generating models are used to directly estimate the:
\end{enumerate}

A. expected return of a security.

B. weights of securities in a portfolio.

C. parameters of the capital market line.

\begin{enumerate}
  \setcounter{enumi}{20}
  \item With respect to capital market theory, the average beta of all assets in the market is:
\end{enumerate}

A. less than 1.0.

B. equal to 1.0.

C. greater than 1.0.

\begin{enumerate}
  \setcounter{enumi}{21}
  \item With respect to the capital asset pricing model, the primary determinant of expected return of an individual asset is the:
\end{enumerate}

A. asset's beta.

B. market risk premium.

C. asset's standard deviation.

\begin{enumerate}
  \setcounter{enumi}{22}
  \item With respect to the capital asset pricing model, which of the following values of beta for an asset is most likely to have an expected return for the asset that is less than the risk-free rate?
A. -0.5
B. 0.0
C. 0.5

  \item With respect to the capital asset pricing model, the market risk premium is:
A. less than the excess market return.
B. equal to the excess market return.
C. greater than the excess market return.

  \item The graph of the capital asset pricing model is the:
A. capital market line.
B. security market line.
C. security characteristic line.

  \item With respect to capital market theory, correctly priced individual assets can be plotted on the:

\end{enumerate}

A. capital market line. B. security market line.

C. capital allocation line.

\section{The following information relates to questions}
 27-30An analyst gathers the following information:

\begin{center}
\begin{tabular}{lcc}
\hline
Security & $\begin{array}{c}\text { Expected } \\ \text { Standard Deviation (\%) }\end{array}$ & Beta \\
\hline
Security 1 & 25 & 1.50 \\
Security 2 & 15 & 1.40 \\
Security 3 & 20 & 1.60 \\
\hline
\end{tabular}
\end{center}

\begin{enumerate}
  \setcounter{enumi}{26}
  \item With respect to the capital asset pricing model, if the expected market risk premium is $6 \%$ and the risk-free rate is $3 \%$, the expected return for Security 1 is closest to:
A. $9.0 \%$.
B. $12.0 \%$.
C. $13.5 \%$.

  \item With respect to the capital asset pricing model, if expected return for Security 2 is equal to $11.4 \%$ and the risk-free rate is $3 \%$, the expected return for the market is closest to:
A. $8.4 \%$.
B. $9.0 \%$.
C. $10.3 \%$.

  \item With respect to the capital asset pricing model, if the expected market risk premium is $6 \%$ the security with the highest expected return is:
A. Security 1 .
B. Security 2 .
C. Security 3.

  \item With respect to the capital asset pricing model, a decline in the expected market return will have the greatest impact on the expected return of:
A. Security 1 .
B. Security 2 .
C. Security 3 .

  \item With respect to capital market theory, which of the following statements best describes the effect of the homogeneity assumption? Because all investors have the same economic expectations of future cash flows for all assets, investors will invest in:

\end{enumerate}

A. the same optimal risky portfolio.

B. the Standard and Poor's 500 Index.

C. assets with the same amount of risk.

\begin{enumerate}
  \setcounter{enumi}{31}
  \item With respect to capital market theory, which of the following assumptions allows for the existence of the market portfolio? All investors:
\end{enumerate}

A. are price takers.

B. have homogeneous expectations.

C. plan for the same, single holding period.

\begin{enumerate}
  \setcounter{enumi}{32}
  \item Three equity fund managers have performance records summarized in the following table:
\end{enumerate}

\begin{center}
\begin{tabular}{lcc}
\hline
 & Mean Annual Return (\%) & $\begin{array}{c}\text { Standard Deviation of } \\ \text { Return (\%) }\end{array}$ \\
\hline
Manager 1 & 14.38 & 10.53 \\
Manager 2 & 9.25 & 6.35 \\
Manager 3 & 13.10 & 8.23 \\
\hline
\end{tabular}
\end{center}

Given a risk-free rate of return of $2.60 \%$, which manager performed best based on the Sharpe ratio?
A. Manager 1
B. Manager 2
C. Manager 3

\begin{enumerate}
  \setcounter{enumi}{33}
  \item Which of the following performance measures is consistent with the CAPM?
A. $M^{2}$.
B. Sharpe ratio.
C. Jensen's alpha.

  \item Which of the following performance measures does not require the measure to be compared to another value?
A. Sharpe ratio.
B. Treynor ratio.
C. Jensen's alpha.

  \item Which of the following performance measures is most appropriate for an investor who is not fully diversified?
A. $M^{2}$.
B. Treynor ratio. C. Jensen's alpha.

  \item The slope of the security characteristic line is an asset's:
A. beta.
B. excess return.
C. risk premium.

  \item Analysts who have estimated returns of an asset to be greater than the expected returns generated by the capital asset pricing model should consider the asset to be:
A. overvalued.
B. undervalued.
C. properly valued.

  \item The intercept of the best fit line formed by plotting the excess returns of a manager's portfolio on the excess returns of the market is best described as Jensen's:
A. beta.
B. ratio.
C. alpha.

  \item Portfolio managers who are maximizing risk-adjusted returns will seek to invest more in securities with:
A. lower values of Jensen's alpha.
B. values of Jensen's alpha equal to 0 .
C. higher values of Jensen's alpha.

  \item Portfolio managers, who are maximizing risk-adjusted returns, will seek to invest less in securities with:
A. lower values for nonsystematic variance.
B. values of nonsystematic variance equal to 0 .
C. higher values for nonsystematic variance.

\end{enumerate}

\section{SOLUTIONS}
\begin{enumerate}
  \item B is correct. A capital allocation line (CAL) plots the expected return and total risk of combinations of the risk-free asset and a risky asset (or a portfolio of risky assets).

  \item B is correct. A portfolio of the risk-free asset and a risky asset or a portfolio of risky assets can result in a better risk-return tradeoff than an investment in only one type of an asset, because the risk-free asset has zero correlation with the risky asset.

  \item B is correct. Investors will have different optimal portfolios depending on their indifference curves. The optimal portfolio for each investor is the one with highest utility; that is, where the CAL is tangent to the individual investor's highest possible indifference curve.

  \item B is correct. Although the optimal risky portfolio is the market portfolio, highly risk-averse investors choose to invest most of their wealth in the risk-free asset.

  \item B is correct. Although the capital allocation line includes all possible combinations of the risk-free asset and any risky portfolio, the capital market line is a special case of the capital allocation line, which uses the market portfolio as the optimal risky portfolio.

  \item A is correct. The market includes all risky assets, or anything that has value; however, not all assets are tradable, and not all tradable assets are investable.

  \item A is correct. The optimal risky portfolio is the market portfolio. Capital market theory assumes that investors have homogeneous expectations, which means that all investors analyze securities in the same way and are rational. That is, investors use the same probability distributions, use the same inputs for future cash flows, and arrive at the same valuations. Because their valuations of all assets are identical, all investors will invest in the same optimal risky portfolio (i.e., the market portfolio).

  \item C is correct. Theoretically, any point above the CML is not achievable and any point below the CML is dominated by and inferior to any point on the CML.

  \item B is correct. As one moves further to the right of point $\mathrm{M}$ on the capital market line, an increasing amount of borrowed money is being invested in the market portfolio. This means that there is negative investment in the risk-free asset, which is referred to as a leveraged position in the risky portfolio.

  \item A is correct. The combinations of the risk-free asset and the market portfolio on the CML where returns are less than the returns on the market portfolio are termed 'lending' portfolios.

  \item $\mathrm{C}$ is correct. Investors are capable of avoiding nonsystematic risk by forming a portfolio of assets that are not highly correlated with one another, thereby reducing total risk and being exposed only to systematic risk.

  \item B is correct. Nonsystematic risk is specific to a firm, whereas systematic risk affects the entire economy.

  \item B is correct. Only systematic risk is priced. Investors do not receive any return for accepting nonsystematic or diversifiable risk.

  \item $C$ is correct. The sum of systematic variance and nonsystematic variance equals the total variance of the asset. References to total risk as the sum of systematic risk and nonsystematic risk refer to variance, not to risk.

  \item A is correct. Security 1 has the highest total risk $=0.25$ compared to Security 2 and Security 3 with a total risk of 0.20 .

  \item $C$ is correct. Security 3 has the highest beta value; $1.07=\frac{\rho_{3, m} \sigma_{3}}{\sigma_{m}}=\frac{(0.80)(20 \%)}{15 \%}$ compared to Security 1 and Security 2 with beta values of 1.00 and 0.93 , respectively.

  \item B is correct. Security 2 has the lowest beta value; $0.93=\frac{\rho_{2, m} \sigma_{2}}{\sigma_{m}}=\frac{(0.70)(20 \%)}{15 \%}$ compared to Security 1 and 3 with beta values of 1.00 and 1.07 , respectively.

  \item B is correct. In the market model, $R_{i}=\alpha_{i}+\beta_{i} R_{m}+e_{i}$, the intercept, $\alpha_{i}$ and slope coefficient, $\beta_{i}$, are estimated using historical security and market returns.

  \item B is correct. In the market model, $R_{i}=\alpha_{i}+\beta_{i} R_{m}+e_{i}$, the slope coefficient, $\beta_{i}$, is an estimate of the asset's systematic or market risk.

  \item A is correct. In the market model, $R_{i}=\alpha_{i}+\beta_{i} R_{m}+e_{i}$, the intercept, $\alpha_{i}$, and slope coefficient, $\beta_{i}$, are estimated using historical security and market returns. These parameter estimates then are used to predict firm-specific returns that a security may earn in a future period.

  \item B is correct. The average beta of all assets in the market, by definition, is equal to 1.0.

  \item A is correct. The CAPM shows that the primary determinant of expected return for an individual asset is its beta, or how well the asset correlates with the market.

  \item A is correct. If an asset's beta is negative, the required return will be less than the risk-free rate in the CAPM. When combined with a positive market return, the asset reduces the risk of the overall portfolio, which makes the asset very valuable. Insurance is an example of a negative beta asset.

  \item B is correct. In the CAPM, the market risk premium is the difference between the return on the market and the risk-free rate, which is the same as the return in excess of the market return.

  \item B is correct. The security market line (SML) is a graphical representation of the capital asset pricing model, with beta risk on the $x$-axis and expected return on the $y$-axis.

  \item B is correct. The security market line applies to any security, efficient or not. The CAL and the CML use the total risk of the asset (or portfolio of assets) rather than its systematic risk, which is the only risk that is priced.

  \item B is correct. The expected return of Security 1 , using the CAPM, is $12.0 \%=3 \%+$ $1.5(6 \%) ; E\left(R_{i}\right)=R_{f}+\beta_{i}\left[E\left(R_{m}\right)-R_{f}\right]$.

  \item B is correct. The expected risk premium for Security 2 is $8.4 \%,(11.4 \%-3 \%)$, indicates that the expected market risk premium is $6 \%$; therefore, since the risk-free rate is $3 \%$ the expected rate of return for the market is $9 \%$. That is, using the CAPM, $E\left(R_{i}\right)=R_{f}+\beta_{i}\left[E\left(R_{m}\right)-R_{f}\right], 11.4 \%=3 \%+1.4(\mathrm{X} \%)$, where $\mathrm{X} \%=(11.4 \%-$ $3 \%) / 1.4=6.0 \%=$ market risk premium .

  \item $C$ is correct. Security 3 has the highest beta; thus, regardless of the value for the risk-free rate, Security 3 will have the highest expected return:

\end{enumerate}

$$
E\left(R_{i}\right)=R_{f}+\beta_{i}\left[E\left(R_{m}\right)-R_{f}\right]
$$

\begin{enumerate}
  \setcounter{enumi}{29}
  \item C is correct. Security 3 has the highest beta; thus, regardless of the risk-free rate the expected return of Security 3 will be most sensitive to a change in the expected market return.

  \item A is correct. The homogeneity assumption refers to all investors having the same economic expectation of future cash flows. If all investors have the same expectations, then all investors should invest in the same optimal risky portfolio, therefore implying the existence of only one optimal portfolio (i.e., the market portfolio).

  \item B is correct. The homogeneous expectations assumption means that all investors analyze securities in the same way and are rational. That is, they use the same probability distributions, use the same inputs for future cash flows, and arrive at the same valuations. Because their valuation of all assets is identical, they will generate the same optimal risky portfolio, which is the market portfolio.

  \item $C$ is correct. The Sharpe ratio $(\widehat{S R})$ is the mean excess portfolio return per unit of risk, where a higher Sharpe ratio indicates better performance:

\end{enumerate}

$$
\begin{aligned}
& \widehat{S R}_{1}=\frac{\bar{R}_{p}-\bar{R}_{f}}{\widehat{\sigma}_{p}}=\frac{14.38-2.60}{10.53}=1.12 \\
& \widehat{S R}_{2}=\frac{\bar{R}_{p}-\bar{R}_{f}}{\widehat{\sigma}_{p}}=\frac{9.25-2.60}{6.35}=1.05 \\
& \widehat{S R}_{3}=\frac{\bar{R}_{p}-\bar{R}_{f}}{\hat{\sigma}_{p}}=\frac{13.10-2.60}{8.23}=1.28
\end{aligned}
$$

\begin{enumerate}
  \setcounter{enumi}{33}
  \item C is correct. Jensen's alpha adjusts for systematic risk, and $M^{2}$ and the Sharpe Ratio adjust for total risk.

  \item $\mathrm{C}$ is correct. The sign of Jensen's alpha indicates whether or not the portfolio has outperformed the market. If alpha is positive, the portfolio has outperformed the market; if alpha is negative, the portfolio has underperformed the market.

  \item A is the correct. $M^{2}$ adjusts for risk using standard deviation (i.e., total risk).

  \item A is correct. The security characteristic line is a plot of the excess return of the security on the excess return of the market. In such a graph, Jensen's alpha is the intercept and the beta is the slope.

  \item B is correct. If the estimated return of an asset is above the SML (the expected return), the asset has a lower level of risk relative to the amount of expected return and would be a good choice for investment (i.e., undervalued).

  \item $C$ is correct. This is because of the plot of the excess return of the security on the excess return of the market. In such a graph, Jensen's alpha is the intercept and the beta is the slope.

  \item $\mathrm{C}$ is correct. Since managers are concerned with maximizing risk-adjusted returns, securities with a higher value of Jensen's alpha, $\alpha_{i}$, should have a higher weight.

  \item C is correct. Since managers are concerned with maximizing risk-adjusted returns, securities with greater nonsystematic risk should be given less weight in the portfolio.

\end{enumerate}

\end{document}