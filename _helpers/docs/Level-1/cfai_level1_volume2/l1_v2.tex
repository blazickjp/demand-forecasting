\documentclass[10pt]{article}
\usepackage[utf8]{inputenc}
\usepackage[T1]{fontenc}
\usepackage{amsmath}
\usepackage{amsfonts}
\usepackage{amssymb}
\usepackage[version=4]{mhchem}
\usepackage{stmaryrd}
\usepackage{hyperref}
\hypersetup{colorlinks=true, linkcolor=blue, filecolor=magenta, urlcolor=cyan,}
\urlstyle{same}
\usepackage{graphicx}
\usepackage[export]{adjustbox}
\graphicspath{ {./images/} }
\usepackage{multirow}

\title{ECONOMICS FINANCIAL STATEMENT ANALYSIS }


\author{by Elaine Henry, PhD, CFA, J. Hennie van Greuning, DCom, CFA, and}
\date{}


\begin{document}
\maketitle
CFA $^{\circledR}$ Program Curriculum 2023 • LEVEL 1 • VOLUME 2 @2022 by CFA Institute. All rights reserved. This copyright covers material written expressly for this volume by the editor/s as well as the compilation itself. It does not cover the individual selections herein that first appeared elsewhere. Permission to reprint these has been obtained by CFA Institute for this edition only. Further reproductions by any means, electronic or mechanical, including photocopying and recording, or by any information storage or retrieval systems, must be arranged with the individual copyright holders noted.

CFA $^{\circ}$, Chartered Financial Analyst ${ }^{\circ}$, AIMR-PPS $^{\circ}$, and GIPS ${ }^{\circ}$ are just a few of the trademarks owned by CFA Institute. To view a list of CFA Institute trademarks and the Guide for Use of CFA Institute Marks, please visit our website at \href{http://www.cfainstitute.org}{www.cfainstitute.org}.

This publication is designed to provide accurate and authoritative information in regard to the subject matter covered. It is sold with the understanding that the publisher is not engaged in rendering legal, accounting, or other professional service. If legal advice or other expert assistance is required, the services of a competent professional should be sought.

All trademarks, service marks, registered trademarks, and registered service marks are the property of their respective owners and are used herein for identification purposes only.

ISBN 978-1-950157-97-6 (paper)

ISBN 978-1-953337-24-5 (ebook)

2022

\section{CONTENTS}
How to Use the CFA Program Curriculum

Errata

Designing Your Personal Study Program

CFA Institute Learning Ecosystem (LES)

Feedback

\section{Economics}
\section{Learning Module 1}
Topics in Demand and Supply Analysis

3

Introduction 3

Demand Concepts $\quad 4$

Demand Concepts $\quad 4$

Price Elasticity of Demand $\quad 6$

Extremes of Price Elasticity $\quad 8$

Predicting Demand Elasticity, Price Elasticity and Total Expenditure $\quad 9$

Elasticity and Total Expenditure $\quad 11$

Income Elasticity of Demand, Cross-Price Elasticity of Demand 12

Cross-Price Elasticity of Demand 13

Substitution and Income Effects; Normal Goods, Inferior Goods and

Special Cases 15

Normal and Inferior Goods $\quad 17$

Supply Analysis: Cost, Marginal Return, and Productivity $\quad 20$

$\begin{array}{lr}\text { Marginal Returns and Productivity } & 20\end{array}$

Economic Profit Versus Accounting Profit 25

Economic Cost vs. Accounting Cost 25

Marginal Revenue, Marginal Cost and Profit Maximization; Short-Run Cost

Curves: Total, Variable, Fixed, and Marginal Costs 26

Understanding the Interaction between Total, Variable, Fixed, and Marginal Cost and Output

28

Perfect and Imperfect Competition, Profit Maximization 32

Profit-Maximization, Breakeven, and Shutdown Points of Production $\quad 34$

Breakeven Analysis and Shutdown Decision 35

The Shutdown Decision $\quad 36$

Economies and Diseconomies of Scale with Short-Run and Long-Run Cost

Analysis

40

$\begin{array}{lr}\text { Short- and Long-Run Cost Curves } & 41\end{array}$

Defining Economies of Scale and Diseconomies of Scale $\quad 42$

Summary 45

Practice Problems 49

$\begin{array}{ll}\text { Solutions } & 57\end{array}$

\section{Learning Module 2}
\section{The Firm and Market Structures}
61

Introduction and Analysis of Market Structures $\quad 61$

$\begin{array}{lr}\text { Analysis of Market Structures } & 62\end{array}$

$\begin{array}{ll}\text { Perfect Competition } & 67\end{array}$ Demand Analysis in Perfectly Competitive Markets $\quad 67$

$\begin{array}{lr}\text { Elasticity of Demand } & 69\end{array}$

$\begin{array}{ll}\text { Other Factors Affecting Demand } & 71\end{array}$

Consumer Surplus: Value Minus Expenditure 73

Supply Analysis, Optimal Price, and Output in Perfectly Competitive Markets 76

Optimal Price and Output in Perfectly Competitive Markets $\quad 77$

Long-Run Equilibrium In Perfectly Competitive Markets $\quad 81$

Monopolistic Competition $\quad 84$

Demand Analysis in Monopolistically Competitive Markets $\quad 85$

Supply Analysis in Monopolistically Competitive Markets 86

Optimal Price and Output in Monopolistically Competitive Markets $\quad 86$

Long-Run Equilibrium In Monopolistic Competition $\quad 87$

Oligopoly and Pricing Strategies 88

Demand Analysis and Pricing Strategies in Oligopoly Markets 89

$\begin{array}{ll}\text { The Cournot Assumption } & 91\end{array}$

$\begin{array}{ll}\text { The Nash Equilibrium } & 94\end{array}$

Oligopoly Markets: Optimal Price, Output, and Long-Run Equilibrium 97

$\begin{array}{lr}\text { Optimal Price and Output in Oligopoly Markets } & 98\end{array}$

Factors Affecting Long-Run Equilibrium in Oligopoly Markets 98

Monopoly Markets: Demand/Supply and Optimal Price/Output 99

Demand Analysis in Monopoly Markets 101

$\begin{array}{lr}\text { Supply Analysis in Monopoly Markets } & 102\end{array}$

Optimal Price and Output in Monopoly Markets 103

Price Discrimination and Consumer Surplus 105

Monopoly Markets: Long-Run Equilibrium 107

Identification of Market Structure 108

$\begin{array}{lr}\text { Econometric Approaches } & 109\end{array}$

$\begin{array}{lr}\text { Simpler Measures } & 109\end{array}$

$\begin{array}{lr}\text { Summary } & 111\end{array}$

References 113

Practice Problems 114

$\begin{array}{ll}\text { Solutions } & 119\end{array}$

Learning Module $3 \quad$ Aggregate Output, Prices, and Economic Growth 121

Introduction 122

Aggregate Output and Income 123

Gross Domestic Product 124

The Components of GDP 131

GDP, National Income, Personal Income, and Personal Disposable

Income

135

Relationship among Saving, Investment, the Fiscal Balance and the

Trade Balance

140

Aggregate Demand and Aggregate Supply 144

Aggregate Demand 144

Aggregate Supply 148

Shifts in the Aggregate Demand Curve $\quad 150$

Equilibrium GDP and Prices 162

Economic Growth and Sustainability 173

The Production Function and Potential GDP 175 Sources of Economic Growth

Measuring Sustainable Growth $\quad 185$

Summary $\quad 188$

References 192

Practice Problems 193

$\begin{array}{ll}\text { Solutions } & 199\end{array}$

\section{Learning Module 4}
Understanding Business Cycles

Introduction $\quad 203$

$\begin{array}{ll}\text { Overview of the Business Cycle } & 204\end{array}$

Phases of the Business Cycle $\quad 205$

Leads and Lags in Business and Consumer Decision Making 209

$\begin{array}{ll}\text { Market Conditions and Investor Behavior } & 209\end{array}$

$\begin{array}{ll}\text { Credit Cycles and Their Relationship to Business Cycles } & 211\end{array}$

$\begin{array}{ll}\text { Applications of Credit Cycles } & 211\end{array}$

Consequences for Policy $\quad 213$

Business Cycle Fluctuations from a Firm's Perspective $\quad 214$

The Workforce and Company Costs $\quad 214$

$\begin{array}{ll}\text { Fluctuations in Capital Spending } & 215\end{array}$

$\begin{array}{ll}\text { Fluctuations in Inventory Levels } & 217\end{array}$

$\begin{array}{ll}\text { Consumer Behavior } & 219\end{array}$

$\begin{array}{lr}\text { Consumer Confidence } & 219\end{array}$

$\begin{array}{lr}\text { Measures of Consumption } & 219\end{array}$

$\begin{array}{ll}\text { Income Growth } & 220\end{array}$

Saving Rates $\quad 221$

Housing Sector Behavior $\quad 222$

Available Statistics $\quad 222$

Sensitivity to Interest Rates and Relationship to Credit Cycle 222

The Role of Demographics $\quad 222$

Impact on the Economic Cycle 223

External Trade Sector Behavior $\quad 223$

$\begin{array}{lr}\text { Cyclical Fluctuations of Imports and Exports } & 224\end{array}$

The Role of the Exchange Rate $\quad 224$

$\begin{array}{lr}\text { Overall Effect on Exports and Imports } & 224\end{array}$

Theoretical Considerations $\quad 226$

Historical Context $\quad 226$

\begin{center}
\begin{tabular}{lr}
Neoclassical Economics & 226 \\
\hline
Rhe Austian School & 227 \\
\hline
\end{tabular}
\end{center}

The Austrian School $\quad 228$

Monetarism 228

Keynesianism $\quad 229$

Modern Approach to Business Cycles $\quad 231$

Economic Indicators $\quad 232$

Types of Indicators $\quad 233$

Composite Indicators $\quad 234$

Leading Indicators $\quad 234$

Using Economic Indicators $\quad 235$

Other Composite Leading Indicators $\quad 236$

Surveys $\quad 237$ The Use of Big Data in Economic Indicators 238

Nowcasting 238

GDPNow $\quad 239$

Unemployment $\quad 242$

Unemployment $\quad 242$

Inflation 246

Deflation, Hyperinflation, and Disinflation $\quad 247$

Measuring Inflation: The Construction of Price Indexes $\quad 248$

$\begin{array}{ll}\text { Price Indexes and Their Usage } & 250\end{array}$

Explaining Inflation $\quad 254$

Summary $\quad 260$

References $\quad 263$

Practice Problems $\quad 264$

$\begin{array}{ll}\text { Solutions } & 264\end{array}$

\section{Learning Module 5}
Monetary and Fiscal Policy

Introduction to Monetary and Fiscal Policy $\quad 278$

$\begin{array}{lr}\text { Monetary Policy } & 280\end{array}$

Money: Functions, Creation, and Definition $\quad 281$

$\begin{array}{ll}\text { The Functions of Money } & 281\end{array}$

$\begin{array}{ll}\text { Paper Money and the Money Creation Process } & 282\end{array}$

Definitions of Money $\quad 285$

Money: Quantity Theory, Supply and Demand, Fisher Effect 287

The Demand for Money $\quad 288$

$\begin{array}{lr}\text { The Supply and Demand for Money } & 289\end{array}$

The Fisher Effect $\quad 291$

Roles of Central Banks \& Objectives of Monetary Policy $\quad 294$

The Objectives of Monetary Policy $\quad 297$

The Costs of Inflation $\quad 299$

Monetary Policy Tools 301

$\begin{array}{ll}\text { Open Market Operations } & 301\end{array}$

The Central Bank's Policy Rate $\quad 301$

$\begin{array}{ll}\text { Reserve Requirements } & 302\end{array}$

The Transmission Mechanism $\quad 303$

$\begin{array}{lr}\text { Inflation Targeting } & 305\end{array}$

Central Bank Independence $\quad 306$

$\begin{array}{ll}\text { Credibility } & 306\end{array}$

Transparency $\quad 307$

Exchange Rate Targeting $\quad 311$

Monetary Policies: Contractionary, Expansionary, Limitations 314

What's the Source of the Shock to the Inflation Rate? 315

Limitations of Monetary Policy $\quad 315$

$\begin{array}{ll}\text { Roles and Objectives of Fiscal Policy } & 321\end{array}$

Roles and Objectives of Fiscal Policy $\quad 321$

Deficits and the National Debt $\quad 325$

Fiscal Policy Tools $\quad 329$

The Advantages and Disadvantages of Using the Different Tools of

Fiscal Policy Learning Module 6

Introduction to Geopolitics

Introduction $\quad 357$

National Governments and Political Cooperation 358

$\begin{array}{ll}\text { Features of Political Cooperation } & 359\end{array}$

Motivations for Cooperation $\quad 359$

$\begin{array}{ll}\text { Resource Endowment, Standardization, and Soft Power } & 361\end{array}$

The Role of Institutions $\quad 362$

Hierarchy of Interests and Costs of Cooperation 362

Power of the Decision Maker $\quad 363$

$\begin{array}{ll}\text { Political Non-Cooperation } & 364\end{array}$

Non-State Actors and the Forces of Globalization 365

$\begin{array}{ll}\text { Features of Globalization } & 366\end{array}$

Motivations for Globalization $\quad 368$

$\begin{array}{ll}\text { Costs of Globalization and Threats of Rollback } & 369\end{array}$

Assessing Geopolitical Actors and Risk $\quad 372$

Autarky 373

Hegemony $\quad 373$

$\begin{array}{lr}\text { Multilateralism } & 375\end{array}$

Bilateralism $\quad 376$

$\begin{array}{ll}\text { The Tools of Geopolitics } & 376\end{array}$

$\begin{array}{lr}\text { National Security Tools } & 378\end{array}$

Economic Tools $\quad 380$

$\begin{array}{lr}\text { Financial Tools } & 380\end{array}$

Multi-Tool Approaches $\quad 381$

Geopolitical Risk and Comparative Advantage $\quad 382$

Incorporating Geopolitical Risk into the Investment Process $\quad 383$

Types of Geopolitical Risk $\quad 383$

Assessing Geopolitical Threats $\quad 385$

$\begin{array}{ll}\text { Tracking risks according to signposts } & 389\end{array}$

Manifestations of Geopolitical Risk $\quad 390$

$\begin{array}{ll}\text { Acting on Geopolitical Risk } & 392\end{array}$

Summary 393

Practice Problems $\quad 396$

Solutions 399

Tariffs 423

Quotas $\quad 425$

Export Subsidies 426

Trading Blocs, Common Markets, and Economic Unions 428

Capital Restrictions 433

Balance of Payments- Accounts and Components 436

$\begin{array}{lr}\text { Balance of Payments Accounts } & 437\end{array}$

Balance of Payment Components 438

Paired Transactions in the BOP Bookkeeping System 440

Commercial Exports: Transactions (ia) and (ib) 441

Commercial Imports: Transaction (ii) 441

Loans to Borrowers Abroad: Transaction (iii) 441

Purchases of Home-Country Currency by Foreign Central Banks:

Transaction (iv) 442

Receipts of Income from Foreign Investments: Transaction (v) 443

Purchase of Non-financial Assets: Transaction (vi) 443

National Economic Accounts and the Balance of Payments 444

$\begin{array}{ll}\text { Trade Organizations } & 449\end{array}$

$\begin{array}{lr}\text { International Monetary Fund } & 450\end{array}$

World Bank Group 452

World Trade Organization 453

$\begin{array}{ll}\text { Summary } & 456\end{array}$

$\begin{array}{lr}\text { References } & 459\end{array}$

$\begin{array}{lr}\text { Practice Problems } & 460\end{array}$

$\begin{array}{lr}\text { Solutions } & 465\end{array}$

Currency Exchange Rates $\quad 469$

Introduction \& The Foreign Exchange Market $\quad 470$

The Foreign Exchange Market 471

$\begin{array}{ll}\text { Market Functions } & 477\end{array}$

Market Participants, Size and Composition 483

Market Size and Composition 486

Exchange Rate Quotations 488

Exchange Rate Quotations 489

$\begin{array}{lr}\text { Cross-Rate Calculations } & 492\end{array}$

Forward Calculations 496

Exchange Rate Regimes- Ideals and Historical Perspective 503

$\begin{array}{lr}\text { The Ideal Currency Regime } & 504\end{array}$

$\begin{array}{ll}\text { Historical Perspective on Currency Regimes } & 505 \\ \text { A Taxonomy of Currency Regimes } & 507 \\ \text { Arrangements with No Separate Legal Tender } & 509 \\ \text { Currency Board System } & 509 \\ \text { Fixed Parity } & 510 \\ \text { Target Zone } & 511 \\ \text { Active and Passive Crawling Pegs } & 511 \\ \text { Fixed Parity with Crawling Bands } & 511 \\ \text { Managed Float } & 511 \\ \text { Independently Floating Rates } & 511 \\ \text { Exchange Rates and the Trade Balance: Introduction } & 515 \\ \text { Exchange Rates and the Trade Balance: The Elasticities Approach } & 517 \\ \text { Exchange Rates and the Trade Balance: The Absorption Approach } & 521 \\ \text { Summary } & 525 \\ \text { References } & 529 \\ \text { Practice Problems } & 530 \\ \text { Solutions }\end{array}$

\section{Financial Statement Analysis}
\section{Learning Module 1}
Learning Module 2 Introduction to Financial Statement Analysis

Introduction

Scope of Financial Statement Analysis

Major Financial Statements - Balance Sheet

Financial Statements and Supplementary Information

Statement of Comprehensive Income

Income Statement

Other Comprehensive Income

Statement of Changes in Equity and Cash Flow Statement

Cash Flow Statement

Financial Notes, Supplementary Schedules, and Management Commentary Management Commentary or Management's Discussion and Analysis

Auditor's Reports

Other Sources of Information

Financial Statement Analysis Framework

Articulate the Purpose and Context of Analysis

Collect Data

Process Data

Analyze/Interpret the Processed Data

Develop and Communicate Conclusions/Recommendations

Follow-Up

Summary

References

Practice Problems

Solutions

Financial Reporting Standards

Introduction

The Objective of Financial Reporting

539 Accounting Standards Boards International Organization of Securities Commissions $\quad 583$ The Securities and Exchange Commission (US) 584 Capital Markets Regulation in Europe $\quad 588$

The International Financial Reporting Standards Framework 588 $\begin{array}{ll}\text { Qualitative Characteristics of Financial Reports } & 589\end{array}$ $\begin{array}{lr}\text { Constraints on Financial Reports } & 590\end{array}$

The Elements of Financial Statements $\quad 591$ Underlying Assumptions in Financial Statements 592 Recognition of Financial Statement Elements 592 Measurement of Financial Statement Elements 593

General Requirements for Financial Statements 593 $\begin{array}{lr}\text { Required Financial Statements } & 594\end{array}$ General Features of Financial Statements $\quad 595$ Structure and Content Requirements 596 Comparison of IFRS with Alternative Reporting Systems 597 Monitoring Developments in Financial Reporting Standards 598 $\begin{array}{lr}\text { New Products or Types of Transactions } & 599\end{array}$ $\begin{array}{lr}\text { Evolving Standards and the Role of CFA Institute } & 599\end{array}$

$\begin{array}{ll}\text { Summary } & 600\end{array}$

$\begin{array}{lr}\text { Practice Problems } & 602\end{array}$

$\begin{array}{ll}\text { Solutions } & 605\end{array}$

\section{How to Use the CFA Program Curriculum}
The CFA Program exams measure your mastery of the core knowledge, skills, and abilities required to succeed as an investment professional. These core competencies are the basis for the Candidate Body of Knowledge $\left(\mathrm{CBOK}^{\mathrm{m}}\right)$ ). The CBOK consists of four components:

\begin{itemize}
  \item A broad outline that lists the major CFA Program topic areas (www. \href{http://cfainstitute.org/programs/cfa/curriculum/cbok}{cfainstitute.org/programs/cfa/curriculum/cbok})

  \item Topic area weights that indicate the relative exam weightings of the top-level topic areas (\href{http://www.cfainstitute.org/programs/cfa/curriculum}{www.cfainstitute.org/programs/cfa/curriculum})

  \item Learning outcome statements (LOS) that advise candidates about the specific knowledge, skills, and abilities they should acquire from curriculum content covering a topic area: LOS are provided in candidate study sessions and at the beginning of each block of related content and the specific lesson that covers them. We encourage you to review the information about the LOS on our website (\href{http://www.cfainstitute.org/programs/cfa/curriculum/}{www.cfainstitute.org/programs/cfa/curriculum/} study-sessions), including the descriptions of LOS "command words" on the candidate resources page at \href{http://www.cfainstitute.org}{www.cfainstitute.org}.

  \item The CFA Program curriculum that candidates receive upon exam registration

\end{itemize}

Therefore, the key to your success on the CFA exams is studying and understanding the CBOK. You can learn more about the CBOK on our website: www.cfainstitute. org/programs/cfa/curriculum/cbok.

The entire curriculum, including the practice questions, is the basis for all exam questions and is selected or developed specifically to teach the knowledge, skills, and abilities reflected in the CBOK.

\section{ERRATA}
The curriculum development process is rigorous and includes multiple rounds of reviews by content experts. Despite our efforts to produce a curriculum that is free of errors, there are instances where we must make corrections. Curriculum errata are periodically updated and posted by exam level and test date online on the Curriculum Errata webpage (\href{http://www.cfainstitute.org/en/programs/submit-errata}{www.cfainstitute.org/en/programs/submit-errata}). If you believe you have found an error in the curriculum, you can submit your concerns through our curriculum errata reporting process found at the bottom of the Curriculum Errata webpage.

\section{DESIGNING YOUR PERSONAL STUDY PROGRAM}
An orderly, systematic approach to exam preparation is critical. You should dedicate a consistent block of time every week to reading and studying. Review the LOS both before and after you study curriculum content to ensure that you have mastered the applicable content and can demonstrate the knowledge, skills, and abilities described by the LOS and the assigned reading. Use the LOS self-check to track your progress and highlight areas of weakness for later review.

Successful candidates report an average of more than 300 hours preparing for each exam. Your preparation time will vary based on your prior education and experience, and you will likely spend more time on some study sessions than on others.

\section{CFA INSTITUTE LEARNING ECOSYSTEM (LES)}
Your exam registration fee includes access to the CFA Program Learning Ecosystem (LES). This digital learning platform provides access, even offline, to all of the curriculum content and practice questions and is organized as a series of short online lessons with associated practice questions. This tool is your one-stop location for all study materials, including practice questions and mock exams, and the primary method by which CFA Institute delivers your curriculum experience. The LES offers candidates additional practice questions to test their knowledge, and some questions in the LES provide a unique interactive experience.

\section{FEEDBACK}
Please send any comments or feedback to \href{mailto:info@cfainstitute.org}{info@cfainstitute.org}, and we will review your suggestions carefully.

\section{Economics}
\section*{LEARNING MODULE 
 1 }
\section{Topics in Demand and Supply Analysis}
Richard V. Eastin, PhD, is at the University of Southern California (USA). Gary L.

Arbogast, PhD, CFA (USA).

\section{LEARNING OUTCOME}
\begin{center}
\begin{tabular}{|c|c|}
\hline
Mastery & The candidate should be able to: \\
\hline
$\square$ & $\begin{array}{l}\text { calculate and interpret price, income, and cross-price elasticities of } \\ \text { demand and describe factors that affect each measure }\end{array}$ \\
\hline
$\square$ & compare substitution and income effects \\
\hline
$\square$ & contrast normal goods with inferior goods \\
\hline
$\square$ & describe the phenomenon of diminishing marginal returns \\
\hline
$\square$ & $\begin{array}{l}\text { determine and interpret breakeven and shutdown points of } \\ \text { production }\end{array}$ \\
\hline
 & $\begin{array}{l}\text { describe how economies of scale and diseconomies of scale affect } \\ \text { costs }\end{array}$ \\
\hline
\end{tabular}
\end{center}

\section{INTRODUCTION}
In a general sense, economics is the study of production, distribution, and consumption and can be divided into two broad areas of study: macroeconomics and microeconomics. Macroeconomics deals with aggregate economic quantities, such as national output and national income, and is rooted in microeconomics, which deals with markets and decision making of individual economic units, including consumers and businesses. Microeconomics is a logical starting point for the study of economics.

Microeconomics classifies private economic units into two groups: consumers (or households) and firms. These two groups give rise, respectively, to the theory of the consumer and the theory of the firm as two branches of study. The theory of the consumer deals with consumption (the demand for goods and services) by utility-maximizing individuals (i.e., individuals who make decisions that maximize the satisfaction received from present and future consumption). The theory of the firm deals with the supply of goods and services by profit-maximizing firms.

It is expected that candidates will be familiar with the basic concepts of demand and supply. This material is covered in detail in the recommended prerequisite readings. In this reading, we will explore how buyers and sellers interact to determine transaction prices and quantities. The reading is organized as follows: Section 2 discusses the consumer or demand side of the market model, and Section 3 discusses the supply side of the consumer goods market, paying particular attention to the firm's costs. Section 4 provides a summary of key points in the reading.

\section{2}
\section{DEMAND CONCEPTS}
The fundamental model of the private-enterprise economy is the demand and supply model of the market. In this section, we examine three important topics concerning the demand side of the model: (1) elasticities, (2) substitution and income effects, and (3) normal and inferior goods. The candidate is assumed to have a basic understanding of the demand and supply model and to understand how a market discovers the equilibrium price at which the quantity willingly demanded by consumers at that price is just equal to the quantity willingly supplied by firms. Here, we explore more deeply some of the concepts underlying the demand side of the model.

\section{Demand Concepts}
The quantity of a good that consumers are willing to buy depends on a number of different variables. Perhaps the most important of those variables is the item's own price. In general, economists believe that as the price of a good rises, buyers will choose to buy less of it, and as its price falls, they buy more. This opinion is so nearly universal that it has come to be called the law of demand.

Although a good's own price is important in determining consumers' willingness to purchase it, other variables also influence that decision. Consumers' incomes, their tastes and preferences, and the prices of other goods that serve as substitutes or complements are just a few of the other variables that influence consumers' demand for a product or service. Economists attempt to capture all these influences in a relationship called the demand function. (A function is a relationship that assigns a unique value to a dependent variable for any given set of values of a group of independent variables.)

Equation 1 is an example of a demand function. In Equation 1, we are saying, "The quantity demanded of good $X$ depends on (is a function of) the price of good $X$, consumers' income, and the price of good $Y^{\prime \prime}$ :

$$
Q_{x}^{d}=f\left(P_{x}, I, P_{y}\right)
$$

where

$Q_{x}^{d}=$ the quantity demanded of some good $X$ (such as per household demand for gasoline in liters per month)

$$
\begin{aligned}
P_{x} & =\text { the price per unit of good } X \text { (such as } € \text { per liter) } \\
I & =\text { consumers' income (as in } € 1,000 \text { s per household annually) } \\
P_{y} & =\text { the price of another good, } Y . \text { (There can be many other goods, not just }
\end{aligned}
$$

one, and they can be complements or substitutes.)

Often, economists use simple linear equations to approximate real-world demand and supply functions in relevant ranges. Equation 2 illustrates a hypothetical example of our function for gasoline demand:

$$
Q_{x}^{d}=84.5-6.39 P_{x}+0.25 I-2 P_{y}
$$

where the quantity of gasoline demanded $\left(Q_{x}^{d}\right)$ is a function of the price of a liter of gasoline $\left(P_{x}\right)$, consumers' income in $€ 1,000 \mathrm{~s}(I)$, and the average price of an automobile in $€ 1,000 \mathrm{~s}\left(P_{y}\right)$.

The signs of the coefficients on gasoline price (negative) and consumers' income (positive) reflect the relationship between those variables and the quantity of gasoline consumed. The negative sign on average automobile price indicates that if automobiles go up in price, fewer will likely be purchased and driven; hence, less gasoline will be consumed. (As discussed later, such a relationship would indicate that gasoline and automobiles have a negative cross-price elasticity of demand and are thus complements.)

To continue our example, suppose that the price of gasoline $\left(P_{x}\right)$ is $€ 1.48$ per liter, per household income $(I)$ is $€ 50,000$, and the price of the average automobile $\left(P_{y}\right)$ is $€ 20,000$. In this case, this function would predict that the per-household monthly demand for gasoline would be 47.54 liters, calculated as follows:

$$
Q_{x}^{d}=84.5-6.39(1.48)+0.25(50)-2(20)=47.54
$$

recalling that income and automobile prices are measured in thousands. Note that the sign on the "own-price" variable $\left(P_{x}\right)$ is negative; thus, as the price of gasoline rises, per household consumption would decrease by 6.39 liters per month for every $€ 1$ increase in gas price. Own price is used by economists to underscore that the reference is to the price of a good itself and not the price of some other good.

In our example, there are three independent variables in the demand function and one dependent variable. If any one of the independent variables changes, so does the quantity demanded. It is often desirable to concentrate on the relationship between the dependent variable and just one of the independent variables at a time. To accomplish this goal, we can hold the other independent variables constant and rewrite the equation.

For example, to concentrate on the relationship between the quantity demanded of the good and its own price, $P_{x}$, we hold constant the values of income and the price of good $Y$. In our example, those values are 50 and 20, respectively. The equation would then be rewritten as

$$
Q_{x}^{d}=84.5-6.39 P_{x}+0.25(50)-2(20)=57-6.39 P_{x}
$$

The quantity of gasoline demanded is a function of the price of gasoline (6.39 per liter), per household income $(€ 50,000)$, and the average price of an automobile $(€ 20,000)$. Notice that income and the price of automobiles are not ignored; they are simply held constant, and they are "collected" in the new constant term, 57 [84.5 + (0.25)(50) - (2) (20)]. Notice also that we can solve for $P_{x}$ in terms of $Q_{x}^{d}$ by rearranging Equation 3, which gives us Equation 4:

$$
P_{x}=8.92-0.156 Q_{x}^{d}
$$

Equation 4 gives the price of gasoline as a function of the quantity of gasoline consumed per month and is referred to as the inverse demand function. $Q_{x}$ in Equation 4 must be restricted to be less than or equal to 57 so that price is not negative. The graph of the inverse demand function is called the demand curve and is shown in Exhibit 1.1

1 Following usual practice, we show linear demand curves intersecting the quantity axis at a price of zero. Real-world demand functions may be non-linear in some or all parts of their domain. Thus, linear demand functions in practical cases are approximations of the true demand function that are useful for a relevant range of values.

\section{Exhibit 1: Household Demand Curve for Gasoline}
\begin{center}
\includegraphics[max width=\textwidth]{2023_05_04_ccb578c5089fd41615beg-018}
\end{center}

The demand curve represents the highest quantity willingly purchased at each price as well as the highest price willingly paid for each quantity. In this example, this household would be willing to purchase 47.54 liters of gasoline per month at a price of $€ 1.48$ per liter. If price were to rise to $€ 2.48$ per liter, the household would be willing to purchase only 41.15 liters per month.

This demand curve is drawn with price on the vertical axis and quantity on the horizontal axis. It can be correctly interpreted as specifying either the highest quantity a household would buy at a given price or the highest price it would be willing to pay for a given quantity. In our example, at a price of $€ 1.48$ per liter, households would each be willing to buy 47.54 liters per month. Alternatively, the highest price they would be willing to pay for 47.54 liters per month is $€ 1.48$ per liter. If the price were to rise by $€ 1$, households would reduce the quantity they each bought by 6.39 units, to 41.15 liters. The slope of the demand curve is measured as the change in price, $P$, divided by the change in quantity, $Q(\Delta P / \Delta Q$, where $\Delta$ stands for "the change in"). In this case, the slope of the demand curve is $1 /-6.39$, or -0.156 .

The general model of demand and supply can be highly useful in understanding directional changes in prices and quantities that result from shifts in one curve or the other. Often, though, we need to measure how sensitive quantity demanded or supplied is to changes in the independent variables that affect them. This is the concept of elasticity of demand and elasticity of supply. Fundamentally, all elasticities are calculated in the same way: They are ratios of percentage changes. Let us begin with the sensitivity of quantity demanded to changes in the own price.

\section{PRICE ELASTICITY OF DEMAND}
calculate and interpret price, income, and cross-price elasticities of demand and describe factors that affect each measure

In Equation 1, we expressed the quantity demanded of some good as a function of several variables, one of which was the price of the good itself (the good's "own-price").

In Equation 3, we introduced a hypothetical household demand function for gasoline, assuming that the household's income and the price of another good (automobiles) were held constant. That function was given by the simple linear expression $Q_{x}^{d}=57-6.39 P_{x}$. Using this expression, if we were asked how sensitive the quantity of gasoline demanded is to changes in price, we might say that whenever price changes by one unit, quantity changes by 6.39 units in the opposite direction; for example, if price were to rise by $€ 1$, quantity demanded would fall by 6.39 liters per month. The coefficient on the price variable $(-6.39)$ could be the measure of sensitivity we are seeking.

There is a drawback associated with that measure, however. It is dependent on the units in which we measured $Q$ and $P$. When we want to describe the sensitivity of demand, we need to recall the specific units in which $Q$ and $P$ were measured-liters per month and euros per liter-in our example. This relationship cannot readily be extrapolated to other units of measure-for example, gallons and dollars. Economists, therefore, prefer to use a gauge of sensitivity that does not depend on units of measure. That metric is called elasticity. Elasticity is a general measure of how sensitive one variable is to any other variable, and it is expressed as the ratio of percentage changes in each variable: $\% \Delta y / \% \Delta x$. In the case of own-price elasticity of demand, that measure is illustrated in Equation 5:

$$
E_{p_{x}}^{d}=\frac{\% \Delta Q_{x}^{d}}{\% \Delta P_{x}}
$$

This equation expresses the sensitivity of the quantity demanded to a change in price. $E_{p_{x}}^{d}$ is the good's own-price elasticity and is equal to the percentage change in quantity demanded divided by the percentage change in price. This measure is independent of the units in which quantity and price are measured. If quantity demanded falls by $8 \%$ when price rises by $10 \%$, then the elasticity of demand is simply -0.8 . It does not matter whether we are measuring quantity in gallons per week or liters per day, and it does not matter whether we measure price in dollars per gallon or euros per liter; $10 \%$ is $10 \%$, and $8 \%$ is $8 \%$. So the ratio of the first to the second is still -0.8 .

We can expand Equation 5 algebraically by noting that the percentage change in any variable $x$ is simply the change in $x(\Delta x)$ divided by the level of $x$. So, we can rewrite Equation 5, using a few simple steps, as

$$
E_{p_{x}}^{d}=\frac{\% \Delta Q_{x}^{d}}{\% \Delta P_{x}}=\frac{\frac{\Delta Q_{x}^{d}}{Q_{x}^{d}}}{\frac{\Delta P_{x}}{P_{x}}}=\left(\frac{\Delta Q_{x}^{d}}{\Delta P_{x}}\right)\left(\frac{P_{x}}{Q_{x}^{d}}\right)
$$

To get a better idea of price elasticity, it might be helpful to illustrate using our hypothetical demand function: $Q_{x}^{d}=57-6.39 P_{x}$. When the relationship between two variables is linear, $\Delta Q_{x}^{d} / \Delta P_{x}$ is equal to the slope coefficient on $P_{x}$ in the demand function. Thus, in our example, the elasticity of demand is -6.39 multiplied by the ratio of price to quantity. We need to choose a price at which to calculate the elasticity coefficient. Using our hypothetical original price of $€ 1.48$, we can find the quantity associated with that particular price by inserting 1.48 into the demand function as given in Equation 3:

$$
Q=57-(6.39)(1.48)=47.54
$$

and we find that $Q=47.54$ liters per month.

The result of our calculation is that at a price of 1.48 , the elasticity of our market demand function is $-6.39(1.48 / 47.54)=-0.2$. How do we interpret that value? It means, simply, that when price equals 1.48 , a $1 \%$ rise in price would result in a fall in quantity demanded of $0.2 \%$.

In our example, when the price is $€ 1.48$ per liter, demand is not very sensitive to changes in price because a $1 \%$ rise in price would reduce quantity demanded by only $0.2 \%$. In this case, we would say that demand is inelastic. To be precise, when the magnitude (ignoring algebraic sign) of the own-price elasticity coefficient has a value of less than one, demand is said to be inelastic. When that magnitude is greater than one, demand is said to be elastic. And when the elasticity coefficient is equal to negative one, demand is said to be unit elastic, or unitary elastic. Note that if the law of demand holds, own-price elasticity of demand will always be negative because a rise in price will be associated with a fall in quantity demanded, but it can be either elastic (very sensitive to a change in price) or inelastic (insensitive to a change in price). In our hypothetical example, suppose the price of gasoline was very high, say, $€ 5$ per liter. In this case, the elasticity coefficient would be -1.28 :

$\mathrm{Q}=57-(6.39)(5)=25.05$

and

$$
-6.39(5 / 25.05)=-1.28
$$

Because the magnitude of the elasticity coefficient is greater than one, we know that demand is elastic at that price. ${ }^{2}$ In other words, at lower prices ( $€ 1.48$ per liter), a slight change in the price of gasoline does not have much effect on the quantity demanded, but when gasoline is expensive ( $€ 5$ per liter), consumer demand for gas is highly affected by changes in price.

By examining Equation 6 more closely, we can see that for a linear demand curve the elasticity depends on where on the curve we calculate it. The first term, $\Delta Q / \Delta P$, which is the inverse of the slope of the demand curve, remains constant along the entire demand curve. But the second term, $P / Q$, changes depending on where we are on the demand curve. At very low prices, $P / Q$ is very small, so demand is inelastic. But at very high prices, $Q$ is low and $P$ is high, so the ratio $P / Q$ is very high and demand is elastic. Exhibit 2 illustrates a characteristic of all negatively sloped linear demand curves. Above the midpoint of the curve, demand is elastic; below the midpoint, demand is inelastic; and at the midpoint, demand is unit elastic.

\section{Exhibit 2: The Elasticity of a Linear Demand Curve}
\begin{center}
\includegraphics[max width=\textwidth]{2023_05_04_ccb578c5089fd41615beg-020}
\end{center}

Note: For all negatively sloped, linear demand curves, elasticity varies depending on where it is calculated.

\section{Extremes of Price Elasticity}
There are two special cases in which linear demand curves have the same elasticity at all points: vertical demand curves and horizontal demand curves. Consider a vertical demand curve, as in Panel A of Exhibit 3, and a horizontal demand curve, as in Panel

2 If interested, evidence on price elasticities of demand for gasoline can be found in Molly Espey, "Explaining the Variation in Elasticity Estimates of Gasoline Demand in the United States: A Meta-analysis," Energy Journal, vol. 17, no. 3 (1996): 49-60. The robust estimates were about -0.26 for short-run elasticity-less than one year-and -0.58 for more than a year. B. In the first case, the quantity demanded is the same, regardless of price. There is no demand curve that is perfectly vertical at all possible prices, but it is reasonable to assume that, over some range of prices, the same quantity would be purchased at a slightly higher price or a slightly lower price. Thus, in that price range, quantity demanded is not at all sensitive to price, and we would say that demand is perfectly inelastic in that range.

\section{Exhibit 3: The Extremes of Price Elasticity}
Panel $A$

Panel B

\begin{center}
\includegraphics[max width=\textwidth]{2023_05_04_ccb578c5089fd41615beg-021(1)}
\end{center}

Note: A vertical demand has zero elasticity and is called perfectly inelastic.

\begin{center}
\includegraphics[max width=\textwidth]{2023_05_04_ccb578c5089fd41615beg-021}
\end{center}

Note: A horizontal demand has infinite elasticity and is called perfectly elastic.

In the second case, the demand curve is horizontal at some given price. It implies that even a minute price increase will reduce demand to zero, but at that given price, the consumer would buy some large, unknown amount. This situation is a reasonable description of the demand curve facing an individual seller in a perfectly competitive market, such as the wheat market. At the current market price of wheat, an individual farmer could sell all she has. If, however, she held out for a price above market price, it is reasonable to believe that she would not be able to sell any at all; other farmers' wheat is a perfect substitute for hers, so no one would be willing to buy any of hers at a higher price. In this case, we would say that the demand curve facing a seller under conditions of perfect competition is perfectly elastic.

\section{PREDICTING DEMAND ELASTICITY, PRICE ELASTICITY AND TOTAL EXPENDITURE}
calculate and interpret price, income, and cross-price elasticities of demand and describe factors that affect each measure

Own-price elasticity of demand is a measure of how sensitive the quantity demanded is to changes in the price of a good or service, but what characteristics of a good or its market might be informative in determining whether demand is highly elastic? Perhaps the most important characteristic is whether there are close substitutes for the good in question. If there are close substitutes for the good, then if its price rises even slightly, a consumer would tend to purchase much less of this good and switch to the less costly substitute. If there are no substitutes, however, then it is likely that the demand is much less elastic. Consider a consumer's demand for some broadly defined product, such as bread. There really are no close substitutes for the entire category of bread, which includes all types from French bread to pita bread to tortillas and so on. So, if the price of all bread were to rise, perhaps a consumer would purchase a little less of it each week, but probably not a significantly smaller amount. Now, consider that the consumer's demand is for a particular baker's specialty bread instead of the category "bread" as a whole. Surely, there are close substitutes for Baker Bob's Whole Wheat Bread with Sesame Seeds than for bread in general. We would expect, then, that the demand for Baker Bob's special loaf is much more elastic than for the entire category of bread.

In addition to the degree of substitutability, other characteristics tend to be generally predictive of a good's elasticity of demand. These include the portion of the typical budget that is spent on the good, the amount of time that is allowed to respond to the change in price, the extent to which the good is seen as necessary or optional, and so on. In general, if consumers tend to spend a very small portion of their budget on a good, their demand tends to be less elastic than if they spend a very large part of their income. Most people spend only a little on toothpaste each month, for example, so it really does not matter whether the price rises $10 \%$. They would probably still buy about the same amount. If the price of housing were to rise significantly, however, most households would try to find a way to reduce the quantity they buy, at least in the long run.

This example leads to another characteristic regarding price elasticity. For most goods and services, the long-run demand is much more elastic than the short-run demand. For example, if the price of gasoline rises, we probably would not be able to respond quickly to reduce the quantity we consume. In the short run, we tend to be locked into modes of transportation, housing and employment location, and so on. With a longer adjustment period, however, we can adjust the quantity consumed in response to the change in price by adopting a new mode of transportation or reducing the distance of our commute. Hence, for most goods, long-run elasticity of demand is greater than short-run elasticity. Durable goods, however, tend to behave in the opposite way. If the price of washing machines were to fall, people might react quickly because they have an old machine that they know will need to be replaced fairly soon anyway. So when price falls, they might decide to go ahead and make a purchase. If the price of washing machines were to stay low forever, however, it is unlikely that a typical consumer would buy more machines over a lifetime.

Knowing whether the good or service is seen to be discretionary or non-discretionary helps to understand its sensitivity to a price change. Faced with the same percentage increase in prices, consumers are much more likely to give up their Friday night restaurant meal (discretionary) than they are to cut back significantly on staples in their pantry (non-discretionary). The more a good is seen as being necessary, the less elastic its demand is likely to be.

In summary, own-price elasticity of demand is likely to be greater (i.e., more sensitive) for items that have many close substitutes, occupy a large portion of the total budget, are seen to be optional instead of necessary, or have longer adjustment times. Obviously, not all these characteristics operate in the same direction for all goods, so elasticity is likely to be a complex result of these and other characteristics. In the end, the actual elasticity of demand for a particular good turns out to be an empirical fact that can be learned only from careful observation and, often, sophisticated statistical analysis.

\section{Elasticity and Total Expenditure}
Because of the law of demand, an increase in price is associated with a decrease in the number of units demanded of some good or service. But what can we say about the total expenditure on that good? That is, what happens to price times quantity when price falls? Recall that elasticity is defined as the ratio of the percentage change in quantity demanded to the percentage change in price. So if demand is elastic, a decrease in price is associated with a larger percentage rise in quantity demanded. Although each unit of the good has a lower price, a sufficiently greater number of units are purchased so that total expenditure (price times quantity) would rise as price falls when demand is elastic.

If demand is inelastic, however, a given percentage decrease in price is associated with a smaller percentage rise in quantity demanded. Consequently, when demand is inelastic, a fall in price brings about a fall in total expenditure.

In summary, when demand is elastic, price and total expenditure move in opposite directions. When demand is inelastic, price and total expenditure move in the same direction. This relationship is easy to identify in the case of a linear demand curve. Recall that above the midpoint, demand is elastic, and below the midpoint, demand is inelastic. In the upper section of Exhibit 4 , total expenditure $(P \times Q)$ is measured as the area of a rectangle whose base is $Q$ and height is $P$. Notice that as price falls, the areas of the inscribed rectangles (each outlined with their own dotted or dashed line) at first grow in size, become largest at the midpoint of the demand curve, and thereafter become smaller as price continues to fall and total expenditure declines toward zero. In the lower section of Exhibit 4, total expenditure is shown for each quantity purchased.

\section{Exhibit 4: Elasticity and Total Expenditure}
\begin{center}
\includegraphics[max width=\textwidth]{2023_05_04_ccb578c5089fd41615beg-023}
\end{center}

Note: Figure depicts the relationship among changes in price, changes in quantity, and changes in total expenditure Maximum total expenditure occurs at the unit-elastic point on a linear demand curve (the cross-hatched rectangle). The relationships just described hold for any demand curve, so it does not matter whether we are dealing with the demand curve of an individual consumer, the demand curve of the market, or the demand curve facing any given seller. For a market, the total expenditure by buyers becomes the total revenue to sellers in that market. It follows, then, that if market demand is elastic, a fall in price will result in an increase in total revenue to sellers as a whole, and if demand is inelastic, a fall in price will result in a decrease in total revenue to sellers. If the demand faced by any given seller were inelastic at the current price, that seller could increase revenue by increasing its price. But because demand is negatively sloped, the increase in price would decrease total units sold, which would almost certainly decrease total production cost. If raising price both increases revenue and decreases cost, such a move would always be profit enhancing. Faced with inelastic demand, a one-product seller would always be inclined to raise the price until the point at which demand becomes elastic.

\section{INCOME ELASTICITY OF DEMAND, CROSS-PRICE ELASTICITY OF DEMAND}
calculate and interpret price, income, and cross-price elasticities of demand and describe factors that affect each measure

Elasticity is a measure of how sensitive one variable is to change in the value of another variable. Up to this point, we have focused on price elasticity, but the quantity demanded of a good is also a function of consumer income.

Income elasticity of demand is defined as the percentage change in quantity demanded $\left(\% \Delta Q_{x}^{d}\right)$ divided by the percentage change in income $(\% \Delta I)$, holding all other things constant, as shown in Equation 7:

$$
E_{I}^{d}=\frac{\% \Delta Q_{x}^{d}}{\% \Delta I}
$$

The structure of this expression is identical to the structure of own-price elasticity given in Equation 5. (All elasticity measures that we will examine have the same general structure; the only thing that changes is the independent variable of interest.) For example, if the income elasticity of demand for some good has a value of 0.8 , we would interpret that to mean that whenever income rises by $1 \%$, the quantity demanded at each price would rise by $0.8 \%$.

Although own-price elasticity of demand will almost always be negative, income elasticity of demand can be negative, positive, or zero. Positive income elasticity means that as income rises, quantity demanded also rises. Negative income elasticity of demand means that when people experience a rise in income, they buy less of these goods, and when their income falls, they buy more of the same good.

Goods with positive income elasticity are called "normal" goods. Goods with negative income elasticity are called "inferior" goods. Typical examples of inferior goods are rice, potatoes, or less expensive cuts of meat. We will discuss the concepts of normal and inferior goods in a later section.

In our discussion of the demand curve, we held all other things constant, including consumer income, to plot the relationship between price and quantity demanded. If income were to change, the entire demand curve would shift one way or the other. For normal goods, a rise in income would shift the entire demand curve upward and to the right. For inferior goods, however, a rise in income would result in a downward and leftward shift in the entire demand curve.

\section{Cross-Price Elasticity of Demand}
We previously discussed a good's own-price elasticity. However, the price of another good might also have an impact on the demand for that good or service, and we should be able to define an elasticity with respect to the other price $\left(P_{y}\right)$ as well. That elasticity is called the cross-price elasticity of demand and takes on the same structure as own-price elasticity and income elasticity of demand, as represented in Equation 8:

$$
E_{p_{y}}^{d}=\frac{\% \Delta Q_{x}^{d}}{\% \Delta P_{y}}
$$

Note how similar this equation is to the equation for own-price elasticity. The only difference is that the subscript on $P$ is now $y$, where $y$ indicates some other good. This cross-price elasticity of demand measures how sensitive the demand for good $X$ is to changes in the price of some other good, $Y$, holding all other things constant. For some pairs of goods, $X$ and $Y$, when the price of $Y$ rises, more of good $X$ is demanded; the cross-price elasticity of demand is positive. Those goods are referred to as substitutes. In economics, if the cross-price elasticity of two goods is positive, they are substitutes, irrespective of whether someone would consider them "similar."

This concept is intuitive if you think about two goods that are seen to be close substitutes, perhaps like two brands of beer. When the price of one of your favorite brands of beer rises, you would probably buy less of that brand and more of a cheaper brand, so the cross-price elasticity of demand would be positive. For substitute goods, an increase in the price of one good would shift the demand curve for the other good upward and to the right.

Alternatively, two goods whose cross-price elasticity of demand is negative are said to be complements. Typically, these goods tend to be consumed together as a pair, such as gasoline and automobiles or houses and furniture. When automobile prices fall, we might expect the quantity of autos demanded to rise, and thus we might expect to see a rise in the demand for gasoline.

Whether two goods are substitutes or complements might not be immediately intuitive. For example, grocery stores often put things like coffee on sale in the hope that customers will come in for coffee and end up doing their weekly shopping there as well. In that case, coffee and, say, cabbage could very well empirically turn out to be complements even though we would not think that the price of coffee has any relation to sales of cabbage. Regardless of whether someone would see two goods as related in some fashion, if the cross-price elasticity of two goods is negative, they are complements.

Although a conceptual understanding of demand elasticities is helpful in sorting out the qualitative and directional effects among variables, using an empirically estimated demand function can yield insights into the behavior of a market. For illustration, let us return to our hypothetical individual demand function for gasoline in Equation 2, duplicated here for convenience:

$$
Q_{x}^{d}=84.5-6.39 P_{x}+0.25 I-2 P_{y}
$$

The quantity demanded of a given good $\left(Q_{x}^{d}\right)$ is a function of its own price $\left(P_{x}\right)$, consumer income $(I)$, and the price of another good $\left(P_{y}\right)$.

To derive the market demand function, the individual consumers' demand functions are simply added together. If there were 1,000 individuals who represented a market and they all had identical demand functions, the market demand function would be the individual consumer's demand function multiplied by the number of consumers. Using the individual demand function given by Equation 2, the market demand function would be as shown in Equation 9:

$$
Q_{x}^{d}=84,500-6,390 P_{x}+250 I-2,000 P_{y}
$$

Earlier, when we calculated own-price elasticity of demand, we needed to choose a price at which to calculate the elasticity coefficient. Similarly, we need to choose actual values for the independent variables $-P_{x}, I$, and $P_{y}$ - and insert these values into the "estimated" market demand function to find the quantity demanded. Choosing $€ 1.48$ for $P_{x}, € 50$ (in thousands) for $I$, and $€ 20$ (in thousands) for $P_{y}$, we find that the quantity of gasoline demanded is 47,543 liters per month. We now have everything we need to calculate own-price, income, and cross-price elasticities of demand for our market. Those elasticities are expressed in Equation 10, Equation 11, and Equation 12. Each of those expressions has a term denoting the change in quantity divided by the change in each respective variable: own price, $\Delta Q_{x} / \Delta P_{x}$; income, $\Delta Q_{x} / \Delta I$, and cross price, $\Delta Q_{x} / \Delta P_{y}$

As we stated in the discussion of own-price elasticity, when the relationship between two variables is linear, the change in quantity $\left(\Delta Q_{x}^{d}\right)$ divided by the change in own price $\left(\Delta P_{x}\right)$, income $(\Delta I)$, or cross price $\left(\Delta P_{y}\right)$ is equal to the slope coefficient on that other variable. The elasticities are calculated by inserting the slope coefficients from Equation 9 into the elasticity formulas.

Own-price elasticity:

$$
E_{p_{x}}^{d}=\left(\frac{\Delta Q_{x}^{d}}{\Delta P_{x}}\right)\left(\frac{P_{x}}{Q_{x}^{d}}\right)=(-6,390)\left(\frac{1.48}{47,542.8}\right)=-0.20
$$

Income elasticity:

$$
E_{I}^{d}=\left(\frac{\Delta Q_{x}^{d}}{\Delta I}\right)\left(\frac{I}{Q_{x}^{d}}\right)=(250)\left(\frac{50}{47,542.8}\right)=0.26
$$

Cross-price elasticity:

$$
E_{p_{y}}^{d}=\left(\frac{\Delta Q_{x}^{d}}{\Delta P_{y}}\right)\left(\frac{P_{y}}{Q_{x}^{d}}\right)=(-2000)\left(\frac{20}{47,542.8}\right)=-0.84
$$

In our example, at a price of $€ 1.48$, the own-price elasticity of demand is -0.20 ; a $1 \%$ increase in the price of gasoline leads to a decrease in quantity demanded of about $0.20 \%$ (Equation 10). Because the absolute value of the own-price elasticity is less than one, we characterize demand as being inelastic at that price; for example, an increase in price would result in an increase in total expenditure on gasoline by consumers in that market. The income elasticity of demand is 0.26 (Equation 11): A 1\% increase in income would result in an increase of $0.26 \%$ in the quantity demanded of gasoline. Because that elasticity is positive (but small), we would characterize gasoline as a normal good. The cross-price elasticity of demand between gasoline and automobiles is -0.84 (Equation 12): If the price of automobiles rose by $1 \%$, the demand for gasoline would fall by $0.84 \%$. We would, therefore, characterize gasoline and automobiles as complements because the cross-price elasticity is negative. The magnitude is quite small, however, so we would conclude that the complementary relationship is weak.

\section{EXAMPLE 1}
\section{Calculating Elasticities from a Given Demand Function}
An individual consumer's monthly demand for downloadable e-books is given by the equation $Q_{e b}^{d}=2-0.4 P_{e b}+0.0005 I+0.15 P_{h b}$, where $Q_{e b}^{d}$ equals the number of e-books demanded each month, $I$ equals the household monthly income, $P_{e b}$ equals the price of e-books, and $P_{h b}$ equals the price of hardbound books. Assume that the price of e-books is $€ 10.68$, household income is $€ 2,300$, and the price of hardbound books is $€ 21.40$. 1. Determine the value of own-price elasticity of demand for e-books.

\section{Solution to 1:}
The own-price elasticity of demand is given by $\left(\Delta Q_{e b}^{d} / \Delta P_{e b}\right)\left(P_{e b} / Q_{e b}^{d}\right)$. Notice from the demand function that $\Delta Q_{e b}^{d} / \Delta P_{e b}=-0.4$. Inserting the given variable values into the demand function yields $Q_{e b}^{d}=2-(0.4)(10.68)$ $+(.0005)(2300)+(0.15)(21.4)=2.088$. So at a price of $€ 10.68$, the own-price elasticity of demand equals $(-0.4)(10.68 / 2.088)=-2.046$, which is elastic because in absolute value the elasticity coefficient is greater than 1.

\begin{enumerate}
  \setcounter{enumi}{1}
  \item Determine the income elasticity of demand for e-books.
\end{enumerate}

\section{Solution to 2:}
Recall that income elasticity of demand is given by $\left(\Delta Q_{e b}^{d} / \Delta I\right)\left(I / Q_{e b}^{d}\right)$. Notice from the demand function that $\Delta Q_{e b}^{d} / \Delta I=0.0005$. Inserting the values for $I$ and $Q_{e b}^{d}$ yields income elasticity of $(0.0005)(2,300 / 2.088)=0.551$, which is positive, so e-books are a normal good.

\begin{enumerate}
  \setcounter{enumi}{2}
  \item Determine the cross-price elasticity of demand for e-books with respect to the price of hardbound books.
\end{enumerate}

\section{Solution to 3:}
Recall that cross-price elasticity of demand is given by $\left(\Delta Q_{e b} / \Delta P_{h b}\right)\left(P_{h b} /\right.$ $\left.Q_{e b}\right)$, and notice from the demand function that $\Delta Q_{e b} / \Delta P_{h b}=0.15$. Inserting the values for $P_{h b}$ and $Q_{e b}$ yields a cross-price elasticity of demand for e-books of $(0.15)(21.40 / 2.088)=1.537$, which is positive, implying that e-books and hardbound books are substitutes.

\section{SUBSTITUTION AND INCOME EFFECTS; NORMAL GOODS, INFERIOR GOODS AND SPECIAL CASES}
compare substitution and income effects

contrast normal goods with inferior goods

The law of demand states that if nothing changes other than the price of a particular good or service itself, a decrease in that good's price will tend to result in a greater quantity of that good being purchased. Simply stated, it is the assumption that a demand curve has negative slope; that is, where price per unit is measured on the vertical axis and quantity demanded per time period is measured on the horizontal axis, the demand curve is falling from left to right, as shown in Exhibit 5.

\section{Exhibit 5: A Negatively Sloped Demand Curve-The Law of Demand}
\begin{center}
\includegraphics[max width=\textwidth]{2023_05_04_ccb578c5089fd41615beg-028}
\end{center}

There are two reasons why a consumer would be expected to purchase more of a good when its price falls and less of a good when its price rises. These two reasons are known as the substitution effect and the income effect of a change in price. We address these two effects separately and then examine the combination of the two.

When the price of something-say, gasoline-falls, that good becomes relatively less costly compared with other goods or services a consumer might purchase. For example, gasoline is used in driving to work, so when its price falls, it is relatively cheaper to drive to work than to take public transportation. Hence, the consumer is likely to substitute a little more driving to work for a little less public transportation. When the price of beef falls, it becomes relatively cheaper than chicken. The typical consumer is, therefore, likely to purchase a little more beef and a little less chicken.

On its own, the substitution effect suggests that when the price of something falls, consumers tend to purchase more of that good. But another influence is often at work as well-the income effect. Consider a consumer spending all of her "money income" on a given combination of goods and services. (Her money income is simply the quantity of dollars or euros, or other relevant currency, that is available to her to spend in any given time period.) Now suppose the price of something she was regularly purchasing falls while her money income and the prices of all other goods remain unchanged. Economists refer to this as an increase in purchasing power or real income. For most goods and services, consumers tend to buy more of them when their income rises. So when the price of a good-say, beef-falls, most consumers would tend to buy more beef because of the increase in their real income. Although the consumer's money income (the number on her paycheck) is assumed not to have changed, her real income has risen because she can now buy more beef-and other goods, too-as a result of the fall in the price of that one good. So, quite apart from the substitution effect of a fall in a good's price, the income effect tends to cause consumers to purchase more of that good as well.

Substitution and income effects work the other way, too. If the price of beef were to rise, the substitution effect would cause the consumer to buy less of it and substitute more chicken for the now relatively more expensive beef. Additionally, the rise in the price of beef results in a decrease in the consumer's real income because now she can buy less goods with the same amount of money income. If beef is a good that consumers tend to buy more of when their income rises and less of when their income falls, then the rise in beef price would have an income effect that causes the consumer to buy less of it.

\section{Normal and Inferior Goods}
Economists classify goods on various dimensions, one of which relates to how consumers' purchases of a good respond to changes in consumer income. Earlier, when discussing income elasticity of demand, we introduced the concept of normal goods and inferior goods. For most goods and services, an increase in income would cause consumers to buy more; these are called normal goods. But that does not hold true for all goods: There are goods that consumers buy less of when their income rises and goods that they buy more of when their incomes fall. These are called inferior goods. This section will contrast normal goods with inferior goods.

We previously discussed income and substitution effects of a change in price. If a good is normal, a decrease in price will result in the consumer buying more of that good. Both the substitution effect and the income effect are at play here:

\begin{itemize}
  \item A decrease in price tends to cause consumers to buy more of this good in place of other goods-the substitution effect.

  \item The increase in real income resulting from the decline in this good's price causes people to buy even more of this good when its price falls-the income effect.

\end{itemize}

So, we can say that for normal goods (restaurant meals, for example, as most people tend to eat out more often when their incomes rise), the substitution and the income effects reinforce one another to cause the demand curve to be negatively sloped.

For inferior goods (cheaper cuts of meat or generic beverages, for example, which most people buy less of as their incomes rise), an increase in income causes consumers to buy less, not more, and if their incomes fall, they buy more, not less. "Inferior" does not imply anything at all about the quality of the good; it is simply used to refer to a good for which an increase in income causes some people to buy less of it.

The same good could be normal for some consumers while it is inferior for others. Consider a very low-income segment of the population. For those consumers, an increase in their income might very well result in their buying more fast-food meals. They might take some of that added income and enjoy eating out at a fast-food restaurant a little more often. Now consider a high-income group. If their income rises, they might be much less inclined to eat at fast-food restaurants and instead do their dining out at a fashionable French bistro, for example. So, fast-food meals might be a normal good for some people and an inferior good for others.

Let us now consider the substitution and income effects of a change in the price of normal and inferior goods. The substitution effect says that if the price of a good falls, the consumer will substitute more of this good in the consumption bundle and buy less of some other good. The substitution effect is true for both normal and inferior goods. Next, we provide an example.

We begin with a hypothetical consumer with a certain money income (R $\$ 200,000)$. Given the prices for all goods, he makes a decision to buy a given amount of Good $X$, coffee. If the price of coffee falls, the consumer is better off than when the price was higher. We can assume that this consumer would have been willing to pay some amount of money each month to be able to buy coffee at the lower price. We now have two states of the world: In State 1, he spends his income on all the various goods, including his desired quantity of coffee at the original price. In State 2, he is able to buy coffee at the new lower price, but because he has paid a portion of his income to buy coffee at the lower price, he now has less money income to spend on all goods combined. If we adjusted the amount of money he would have to pay to lock in the lower price of coffee until he is just indifferent between the two states of the world, we would have exactly offset the "good" thing of the lower price with the "bad" thing of less income. This removes the income effect of the price decrease and allows us to isolate the pure substitution effect. We find that in State 2, he would buy more coffee than in State 1. The pure substitution effect is always in the direction of buying more at the lower relative price.

Continuing our example, assume that we give back to the consumer the amount of money he is willing to pay for the privilege of buying coffee at the lower price. Clearly, he is better off because now he can buy coffee at the lower price without having to pay for the privilege. We want to know whether, with this higher money income, he will now buy more or less coffee at the lower price. The answer depends on whether coffee is a normal or an inferior good for this consumer. Recall that for normal goods, an increase in income causes consumers to buy more, but for inferior goods, an increase in income causes consumers to buy less.

In conclusion, the substitution effect of a change in the price of a good will always be in the direction of buying more at a lower price or less at a higher price. The income effect of that same price change, however, depends on whether the good is normal or inferior. If the good is normal, the income effect reinforces the substitution effect, both leading to a negatively sloped demand curve. But if the good is inferior, the income effect and the substitution effect work in opposite directions; the income effect tends to mitigate the substitution effect.

Exhibit 6 summarizes the substitution and income effects for normal and inferior goods.

Exhibit 6: The Substitution and Income Effects of a Price Decrease on Normal and Inferior Goods

\begin{center}
\begin{tabular}{lll}
\hline
 & Substitution Effect & Income Effect \\
\hline
Normal good & $\begin{array}{l}\text { Buy more because the good } \\ \text { is relatively cheaper than its } \\ \text { substitutes. }\end{array}$ & $\begin{array}{l}\text { Buy more because the increase in } \\ \text { purchasing power raises the total } \\ \text { consumption level. }\end{array}$ \\
Inferior good & $\begin{array}{l}\text { Buy more because the good } \\ \text { is relatively cheaper than its }\end{array}$ & $\begin{array}{l}\text { Buy less because the increase in } \\ \text { real income prompts the con- } \\ \text { substitutes. }\end{array}$ \\
 & $\begin{array}{l}\text { sumer to buy less of the inferior } \\ \text { good in favor of its preferred } \\ \text { substitutes. }\end{array}$ &  \\
 &  &  \\
\end{tabular}
\end{center}

\section{Exceptions to the Law of Demand}
In virtually every case in the real world, the law of demand holds: A decrease in price results in an increase in quantity demanded, resulting in a negatively sloped demand curve. In a few unusual cases, however, we may find a positively sloped demand curve-a decrease (increase) in price may result in a decrease (increase) in the quantity demanded. These unusual cases are called Giffen goods and Veblen goods.

In theory, it is possible for the income effect to be so strong and so negative as to overpower the substitution effect. In such a case, more of a good would be consumed as the price rises and less would be consumed as the price falls. These goods are called Giffen goods, named for Robert Giffen based on his observations of the purchasing habits of the Victorian era poor. For many decades, no one really believed that a Giffen good actually existed anywhere other than in textbooks. But in recent years, studies have documented a few rare cases. One study was conducted in a poor rural community where individuals spend a very large portion of their incomes on rice. For these individuals, rice was an inferior good. Under the law of demand, the quantity of rice purchased would rise with the decline in price, but the rise in quantity would be partially offset by the income effect (a decrease in the amount of rice purchased as a result of rising incomes). What the experimenters discovered, however, was that for a certain subset of consumers, the quantity of rice purchased declined in absolute terms - the income effect actually overwhelmed the substitution effect. For consumers living at subsistence levels-incomes just barely sufficient to enable them to meet their caloric intake needs-a decline in the price of the staple enabled them to shift more of their consumption from rice to the alternate sources of calories in their diet (e.g., meat).

With some goods, the item's price tag itself might drive the consumer's preferences for it. Thorstein Veblen posited just such a circumstance in his concept of conspicuous consumption. According to this way of thinking, a consumer might derive utility out of being known by others to consume a so-called high-status good, such as a luxury automobile or a very expensive piece of jewelry. Importantly, it is the high price itself that partly imparts value to such a good. These are called Veblen goods, and they derive their value from the consumption of them as symbols of the purchaser's high status in society; they are certainly not inferior goods. It is argued that by increasing the price of a Veblen good, the consumer would be more inclined to purchase it, not less.

\section{EXAMPLE 2}
\section{Income and Substitution Effects of a Decrease in Price}
Monica has a monthly entertainment budget that she spends on (a) movies and (b) an assortment of other entertainment items. When the price of each movie is $\$ 8$, she spends a quarter of her budget on six movies a month and the rest of her budget on other entertainment. Monica was offered an opportunity to join a movie club at her local theater that allows her to purchase movies at half the regular price, and she can choose each month whether to join the movie club or not. There is a membership fee she must pay for each month she belongs to the club. Monica is exactly indifferent between (a) not buying the membership and, therefore, paying $\$ 8$ for movies and (b) buying the membership and paying $\$ 4$ per movie. So, she flips a coin each month to determine whether to join the club that month. In months that she does join the club, she sees eight movies. For her birthday, a friend gave her a one-month club membership as a gift, and that month she saw 12 movies.

\begin{enumerate}
  \item If there were no club and the price of movies were to simply fall from $\$ 8$ to $\$ 4$, how many more movies would Monica buy each month?
\end{enumerate}

\section{Solution to 1:}
Six movies. When her friend gave her a club membership, she bought 12 movies instead of her usual 6. With the gift of the club membership, Monica could buy movies at a price of $\$ 4$ without paying for that privilege. This is the same as if the price of each movie fell from $\$ 8$ to $\$ 4$.

\begin{enumerate}
  \setcounter{enumi}{1}
  \item Of the increased number of movies Monica would purchase if the price were to fall from $\$ 8$ to $\$ 4$, determine how much of the increase would be attributable to the substitution effect and how much to the income effect of that price decrease.
\end{enumerate}

\section{Solution to 2:}
When Monica pays the club membership herself, she buys eight movies, two more than usual. Because Monica is equally well off whether she joins the club for a monthly fee and thereby pays half price or whether she does not join the club and pays full price, we can say that the income effect of the price decrease has been removed by charging her the monthly fee. So the increase from six movies to eight is the result of the substitution effect. When Monica's friend gave her the gift of a club membership, allowing her to pay half price without paying for the privilege, Monica bought 12 movies, 6 more than usual and 4 more than she would have had she paid the membership fee. The increase from 8 movies to 12 is the result of the income effect.

\begin{enumerate}
  \setcounter{enumi}{2}
  \item For Monica, are movies a normal, inferior, or Giffen good?
\end{enumerate}

\section{Solution to 3:}
When the price fell from $\$ 8$ to $\$ 4$, Monica bought more movies, so clearly movies are not a Giffen good for her. Additionally, because the substitution effect and the income effect are in the same direction of buying more movies, they are a normal good for Monica. The substitution effect caused her to buy two more movies, and the income effect caused her to buy an additional four movies.

\section{SUPPLY ANALYSIS: COST, MARGINAL RETURN, AND PRODUCTIVITY}
$\square \quad$ describe the phenomenon of diminishing marginal returns

To fully comprehend the supply side of a consumer goods market, an analyst must understand the firm's costs. (As a reminder, this reading builds on the basics of the market model as covered in the recommended prerequisite reading material.)

The firm's marginal cost is the foundation of the firm's ability and willingness to offer a given quantity for sale, and its costs depend on both the productivity of its inputs and their prices. In this section, we will describe the firm's cost curves-total, average, and marginal costs in both the short run and in the long run-paying special attention to what economists call the law of diminishing marginal returns. We will then use this information to explore the conditions under which a firm would find it beneficial to continue operation, even if its economic profits are negative, and at what levels of production its shutdown and breakeven points occur. Long-run costs will be examined in the context of economies and diseconomies of scale.

\section{Marginal Returns and Productivity}
There is an economic phenomenon known as increasing marginal returns, in which marginal product - the productivity of each additional unit of a resource-increases as additional units of that input are employed.

Initially, a firm can experience increasing returns from adding labor to the production process because of the concepts of specialization and division of labor. At first, by having too few workers relative to total physical capital, the understaffing situation requires employees to multi-task and share duties. As more workers are added, employees can specialize, become more adept at their individual functions, and realize an increase in marginal productivity. But after a certain output level, the law of diminishing marginal returns becomes evident. When more and more workers are added to a fixed capital base, the marginal return of the labor factor eventually decreases because the fixed input restricts the output potential of additional workers. As an illustration, consider automobile production. When an auto manufacturing plant is operating at full capacity, adding additional labor will not increase production because the physical plant is already $100 \%$ employed. More labor hours will merely add to costs without adding to output. Assuming all workers are of equal quality and motivation, the decline in marginal product occurs in the short run, where all other resources (typically, plant size, physical capital, and technology) are fixed.

Marginal returns are directly related to input productivity, a measure of the output per unit of input.

\section{Productivity: The Relationship between Production and Cost}
The cost of producing anything depends on the amount of inputs, or factors of production (these terms are synonymous), and the input prices. Examples of factors of production are employee hours, machine hours, raw materials, and so on. For simplicity, economists typically concentrate on only two inputs, labor and capital, although obviously there can be many inputs to a particular production process. The labor input is simply employee time, and it is measured as labor hours per time period, such as per week or per month. We denote labor hours as $L$. If a firm is using two laborers per week and each laborer works 35 hours per week, then $L$ equals 70 labor hours per week. We denote hours of capital as $K$. If the firm is using three machines and each one is used for 12 hours per week, then $K$ equals 36 machine hours per week. That is, the capital input is measured as machine hours used per time period. In this way, capital and labor are stated in similar terms. They represent flows of services-labor hours and machine hours - that are used to produce a flow of output per time period.

Accordingly, the respective input prices would be the wage rate per labor hour (we use $w$ to denote wage rate) and the rental rate per machine hour (we use $r$ to denote the rental rate per machine hour). It is helpful to think of a firm as renting the services of labor and of machines. Although the firm might own its own machines, it could in theory rent its machines out to another user, so it is forgoing the rate it could earn elsewhere when it is using its machines internally instead of renting them out. So, a firm is not using its own machines "for free." It is incurring the opportunity cost of not being able to rent those machines to another user.

The total cost of production (TC) is the number of hours of labor multiplied by the wage rate plus the number of machine hours multiplied by the rental rate of machines:

$\mathrm{TC}=(w)(L)+(r)(K)$

This formula illustrates that the total cost is just the cost of all the firm's inputs. It is not a cost function, however, which is a relationship between the cost of production and the flow of output. The cost function $C=\mathrm{f}(Q)$, where $(Q)$ denotes the flow of output in units of production per time period, relates the production cost per time period to the number of units of output produced per time period.

Two things could cause the cost of producing any given level of output to fall: Either the price of one or both inputs could fall or the inputs themselves could become more productive and less of them would be needed (e.g., a worker is more productive when fewer hours of labor are needed to produce the same output). The reverse is true also: A rise in cost could result from either a rise in input prices or a fall in input productivity, or both.

Why is productivity important? Cost-minimization and profit-maximization behavior dictate that the firm strives to maximize productivity-for example, produce the most output per unit of input or produce any given level of output with the least amount of inputs. A firm that lags behind the industry in productivity is at a competitive disadvantage and is likely to face decreases in future earnings and shareholders' wealth. An increase in productivity lowers production costs, which leads to greater profitability and investment value. These productivity benefits can be fully or partially distributed to other stakeholders of the business, such as to consumers in the form of lower prices and to employees in the form of enhanced compensation. Transferring some or all of the productivity rewards to non-equity holders creates synergies that benefit shareholders over time.

The benefits from increased productivity are as follows:

\begin{itemize}
  \item lower business costs, which translate into increased profitability;

  \item an increase in the market value of equity and shareholders' wealth resulting from an increase in profit; and

  \item an increase in worker rewards, which motivates further productivity increases from labor.

\end{itemize}

Undoubtedly, increases in productivity reinforce and strengthen the competitive position of the firm over the long run. A fundamental analysis of a company should examine the firm's commitment to productivity enhancements and the degree to which productivity is integrated into the competitive nature of the industry or market. In some cases, productivity is not only an important promoter of growth in firm value over the long term but is also the key factor for economic survival. A business that lags the market in terms of productivity often finds itself less competitive, while at the same time confronting profit erosion and deterioration in shareholders' wealth. Typical productivity measures for a firm are based on the concepts of total product, average product, and marginal product of labor.

\section{Total, Average, and Marginal Product of Labor}
When measuring a firm's operating efficiency, it is easier and more practical to use a single resource factor as the input variable rather than a bundle of the different resources that the firm uses in producing units of output. As discussed in the previous section, labor is typically the input that is the most identifiable and calculable for measuring productivity. However, any input that is not difficult to quantify can be used. As an example, a business that manually assembles widgets has 50 workers, one production facility, and an assortment of equipment and hand tools. The firm would like to assess its productivity when using these three input factors to produce widgets. In this example, it is most appropriate to use labor as the input factor for determining productivity because the firm uses only one (fixed) plant building and a variety of other physical capital.

We will use labor as the input variable to illustrate the concepts of total product, average product, and marginal product. Exhibit 7 provides a summary of these three concepts.

\section{Exhibit 7: Definitions and Calculations for Total, Marginal, and Average Product of Labor}
\begin{center}
\begin{tabular}{ll}
\hline
Term & Calculation \\
\hline
Total product & $\begin{array}{l}\text { Sum of the output from all inputs during a time period; usually illustrated as the total output }(Q) \\ \text { using labor quantity }(L)\end{array}$ \\
Average product & $\begin{array}{l}\text { Total product divided by the quantity of a given input; measured as total product divided by the } \\ \text { number of worker hours used at that output level }(Q / L)\end{array}$ \\
\end{tabular}
\end{center}

\begin{center}
\begin{tabular}{ll}
\hline
Term & Calculation \\
\hline
Marginal product & $\begin{array}{l}\text { The amount of additional output resulting from using one more unit of input assuming other } \\ \text { inputs are fixed; measured by taking the difference in total product and dividing by the change in } \\ \text { the quantity of labor }(\Delta Q / \Delta L)\end{array}$ \\
\hline
\end{tabular}
\end{center}

Total product $(Q)$ is defined as the aggregate sum of production for a firm during a time period. As a measure of productivity, total product provides superficial information about how effective and efficient a firm is in terms of producing output. For instance, three firms-Company A, Company B, and Company C-that make up an entire industry have total output levels of 100,000 units, 180,000 units, and 200,000 units, respectively. Obviously, Company $\mathrm{C}$ dominates the market with a $41.7 \%$ share, followed by Company B's $37.5 \%$ share and Company A's $20.8 \%$ portion of the market. However, this information says little about how efficient each firm is in generating its total output level. Total product only provides insight into a firm's production volume relative to the industry; it does not show how efficient a firm is in producing its output.

Average product of labor $\left(\mathrm{AP}_{L}\right)$ measures the productivity of an input (in this case, labor) on average and is calculated by dividing total product by the total number of units for the given input that is used to generate that output. Average product is usually measured on the basis of the labor input. It is a representative or overall measure of labor's productivity: Some workers are more productive than average, and others are less productive than average.

Exhibit 8 compares the productivity of the three firms introduced earlier. Company A employs 100 worker hours and produces 100,000 widgets per hour. Company B employs 200 worker hours and produces 180,000 widgets per hour. Company C employs 250 worker hours and produces 200,000 widgets per hour.

Exhibit 8: Comparing Productivity

\begin{center}
\begin{tabular}{lccc}
\hline
 & $\begin{array}{c}\text { Output } \\ (\mathbf{Q})\end{array}$ & $\begin{array}{c}\text { Number of Worker } \\ \text { Hours }(\boldsymbol{L})\end{array}$ & $\begin{array}{c}\text { Average Product of Labor } \\ \left(\text { AP }_{\boldsymbol{L}}\right)\end{array}$ \\
\hline
Company A & 100,000 & 100 & 1,000 \\
Company B & 180,000 & 200 & 900 \\
Company C & 200,000 & 250 & 800 \\
\hline
\end{tabular}
\end{center}

Using this metric, it is apparent that Company $\mathrm{A}$, with $\mathrm{AP}_{L}$ equal to 1,000 , is the most efficient firm, despite having the lowest market share. Company $C$ has the largest market share, but it is the least efficient of the three, with $\mathrm{AP}_{L}$ equal to 800. Assuming that Company A can maintain its productivity advantage over the long run, it will be positioned to generate the greatest return on investment through lower costs and higher profit outcomes relative to the other firms in the market.

Marginal product of labor $\left(\mathrm{MP}_{L}\right)$, also known as marginal return, measures the productivity of each additional unit of input and is calculated by observing the difference in total product when adding another unit of input (assuming other resource quantities are held constant). It is a gauge of the productivity of the individual additional worker hour rather than an average across all workers.

Exhibit 9 provides a numerical illustration for total, average, and marginal products of labor.

\section{Exhibit 9: Total, Average, and Marginal Product of Labor}
\begin{center}
\begin{tabular}{lccc}
\hline
$\begin{array}{l}\text { Labor } \\ (\mathbf{L})\end{array}$ & $\begin{array}{c}\text { Total Product } \\ \left(\mathbf{Q}_{\mathbf{L}}\right)\end{array}$ & $\begin{array}{c}\text { Average Product } \\ \left(\mathbf{A P}_{\mathbf{L}}\right)\end{array}$ & $\begin{array}{c}\text { Marginal Product } \\ \left(\mathbf{M P}_{\mathbf{L}}\right)\end{array}$ \\
\hline
0 & 0 & - & - \\
1 & 100 & 100 & 100 \\
2 & 210 & 105 & 110 \\
3 & 300 & 100 & 90 \\
4 & 360 & 90 & 60 \\
5 & 400 & 80 & 40 \\
6 & 420 & 70 & 20 \\
7 & 350 & 50 & -70 \\
\hline
\end{tabular}
\end{center}

Total product increases as the firm adds each additional hour of labor-until the seventh labor hour, at which point total production declines by 70 units. Obviously, the firm would want to avoid negative worker productivity.

At an employment level of five labor hours, $\mathrm{AP}_{L}$ is 80 units (400/5) and $\mathrm{MP}_{L}$ is 40 units $[(400-360) /(5-4)]$. The average productivity for all five labor hours is 80 units, but the productivity of the fifth labor hour is only 40 units.

\section{EXAMPLE 3}
\section{Calculation and Interpretation of Total, Average, and Marginal Product}
Exhibit 10 illustrates the production relationship between the number of machine hours and total product.

\section{Exhibit 10}
\begin{center}
\begin{tabular}{lccc}
\hline
$\begin{array}{l}\text { Machine } \\ \text { Hours }(\boldsymbol{K})\end{array}$ & Total Product $\left(\mathbf{Q}_{\boldsymbol{K}}\right)$ & $\begin{array}{c}\text { Average Product } \\ \left(\mathbf{A P}_{\boldsymbol{K}}\right)\end{array}$ & $\begin{array}{c}\text { Marginal Product } \\ \left.\mathbf{M P P}_{\boldsymbol{K}}\right)\end{array}$ \\
\hline
0 & 0 & - & - \\
1 & 1,000 & 1,000 & 1,000 \\
2 & 2,500 & 1,250 & 1,500 \\
3 & 4,500 & 1,500 & 2,000 \\
4 & 6,400 & 1,600 & 1,900 \\
5 & 7,400 & 1,480 & 1,000 \\
6 & 7,500 & 1,250 & 100 \\
7 & 7,000 & 1,000 & -500 \\
\hline
\end{tabular}
\end{center}

\begin{enumerate}
  \item Interpret the results for total, average, and marginal product.
\end{enumerate}

\section{Solution to 1:}
Total product increases up to six machine hours, where it tops out at 7,500. Because total product declines from Hour 6 to Hour 7, the marginal product for Machine Hour 7 is negative 500 units. Average product peaks at 1,600 units with four machine hours. Average product increases at a steady pace with the addition of Machine Hours 2 and 3. The addition of Machine Hour 4 continues to increase average product but at a decreasing rate. Beyond four machine hours, average product decreases-at an increasing rate. Marginal product peaks with Machine Hour 3 and decreases thereafter.

\begin{enumerate}
  \setcounter{enumi}{1}
  \item Indicate at what point increasing marginal returns change to diminishing marginal returns.
\end{enumerate}

\section{Solution to 2:}
The marginal product, $\mathrm{MP}_{K}$, of Machine Hour 3 is 2,000. The marginal product of each additional machine hour beyond Machine Hour 3 declines. Diminishing marginal returns are evident beyond Machine Hour 3.

A firm has a choice of using total product, average product, marginal product, or some combination of the three to measure productivity. Because total product is simply an indication of a firm's output volume and potential market share, average product and marginal product are better gauges of a firm's productivity. Both can reveal competitive advantage through production efficiency. However, individual worker productivity is not easily measurable when workers perform tasks collectively. In this case, average product is the preferred measure of productivity performance.

Referring to the total product column in Exhibit 9, output is more than twice as great (210 widgets) when two hours of labor are used as opposed to only one hour (100 widgets.) In this range of production, there is an increase in return when employee hours are added to the production process. This is the phenomenon of increasing marginal returns.

\section{ECONOMIC PROFIT VERSUS ACCOUNTING PROFIT}
Two important considerations of any firm are its level of profitability and whether to 8 continue to operate in the current environment. Economists define profit differently than do accountants. Economic profit is defined as the difference between total revenue (TR) and total economic costs. Accounting profit is the difference between TR and total accounting cost. TR is the same from both an accounting standpoint and an economic standpoint; it is derived by multiplying the selling price per unit of output by the number of units: $\mathrm{TR}=(P)(Q)$. The difference between the two measures of profit, therefore, lies in an understanding of economic cost (also called "opportunity cost," which is defined in detail in the next section).

\section{Economic Cost vs. Accounting Cost}
The opportunity cost of a ny particular decision, such as to produce a given level of output, can be determined by measuring the benefit forgone by n ot implementing the next best alternative. Suppose that a firm is currently operating with hired labor and its own plant and equipment to produce output at some level. The firm must continuously decide either to keep the level of output the same or to change it. The decision to maintain the same output requires that the firm hire the same amount of labor input and use the same level of its capital inputs as before. The labor expense is both an economic cost and an accounting cost because the money spent on labor hours could have been used for something else (opportunity cost), and it is also a current expense for the firm (accounting cost.) Accountants typically attempt to recognize the cost of plant and equipment in the form of accounting depreciation, which is a means of distributing the historical cost of the fixed capital among the units of production for financial reporting purposes. The money spent in the past on the firm's plant and equipment is what economists call a "sunk cost." Because sunk costs cannot be altered, they cannot affect an optimal decision, which is forward looking. Sunk costs are therefore ignored, and the key management question is, Going forward, what are the opportunity costs and benefits of maintaining a given level of output?

Here is where economic depreciation comes into play. To understand the opportunity costs of using our plant and equipment-already bought and paid for-for one more period of time to produce output, we have to ask the question, What else could be done with that fixed capital if it were not used to produce our output? The answer might be that because there is no external market for our machines and buildings, we are forgoing nothing by using it to produce output. Or it might be that there is a market where we could rent out or sell our capital equipment elsewhere instead of using it to produce output. That rental rate is the economic depreciation associated with using our own equipment to produce output instead of renting or selling it elsewhere.

Economic depreciation is forward looking. It asks, What am I giving up if I use my resources to produce output in the coming period? Accounting depreciation is backward looking. It asks, How should I distribute the historical cost-that I have already paid-across units of output that I intend to produce this period? Both concepts are useful-one for making managerial decisions about output and the other as a way spreading historical costs for reporting or tax purposes-but there is not necessarily a direct relationship between the two.

\section{MARGINAL REVENUE, MARGINAL COST AND PROFIT MAXIMIZATION; SHORT-RUN COST CURVES: TOTAL, VARIABLE, FIXED, AND MARGINAL COSTS}
determine and interpret breakeven and shutdown points of
production

It is assumed that any for-profit firm's management is tasked with achieving the goal of shareholder wealth maximization. Put most simply, that translates into the goal of economic profit maximization. Hereafter, when the word profit is used, it will be economic profit that we have in mind. Because profit is defined as TR minus TC, anything that increases revenue more than cost or decreases cost more than revenue will increase profit. Before we address profit maximization, we must introduce two important concepts: marginal revenue and marginal cost.

Marginal revenue (MR) is the additional revenue the firm realizes from the decision to increase output by one unit per time period. That is, $M R=\Delta T R / \Delta Q$. If the firm is operating in what economists call a perfectly competitive market, it is one of many sellers of identical products in an environment characterized by low or non-existent barriers to entry. Under perfect competition, the firm has no pricing power because there are many perfect substitutes for the product it sells. If it were to attempt to raise the price even by a very small amount, it would lose all of its sales to competitors. On the other hand, it can sell essentially any amount of product it wants without lowering the price below the market price. Take the wheat market as an example of a perfectly competitive market. A seller of wheat would have no control over the market price of wheat; thus, because $\mathrm{TR}=$ $(P)(Q)$, MR for this firm is simply price per unit of output. This firm is said to face a perfectly horizontal (zero-sloped), or infinitely elastic, demand curve for its product. For example, if the firm is selling 1,000 bushels of wheat per week at a price of $\pounds 3$ per bushel, $T R$ is $\pounds 3,000$. If the firm were to increase its output by one unit, then TR would rise by exactly $\pounds 3$ because the firm would not have to lower its price to sell that added unit. So, for sellers in a market with perfect competition, $M R=P$.

In contrast, if a firm sells a product that is differentiated from other firms' products and that has a large market share, the firm is said to be operating in an environment of imperfect competition. In the extreme case of imperfect competition, there might be only one firm selling a product with no close substitutes. That firm holds a monopoly, and it is subject to the market demand curve for its product. Whether a monopoly or simply operating under imperfect competition, the firm faces a negatively sloped demand curve and must lower its price to sell another unit. Thus, MR will be lower than price.

To illustrate this concept, we will decompose MR. Recall from earlier in the reading that

$\mathrm{TR}=(P)(Q)$

and

$$
\mathrm{MR}=\Delta \mathrm{TR} / \Delta Q
$$

Change in total revenue $(\triangle T R)$, the numerator of the ratio, can be written as $(P)(\Delta Q)$ $+(Q)(\Delta P)$

There are two competing forces affecting revenue: (1) Additional units are sold at the new price, and (2) all units must now be sold at the lower price. The firm is selling more units, but it is selling all units at a lower price than before.

To find MR, we divide the change in TR by the change in quantity:

$\mathrm{MR}=\frac{(P)(\Delta Q)}{\Delta Q}+\frac{(Q)(\Delta P)}{\Delta Q}=P+Q \frac{(\Delta P)}{\Delta Q}$

In other words, $M R$ is equal to price but with an "adjustment" equal to $(Q)(\Delta P / \Delta Q)$.

Taking this one step further, recall that earlier we said $(\Delta P / \Delta Q)$ is the slope of the demand curve. From our expression just given, $M R=P+Q(\Delta P / \Delta Q)$; so, $M R$ is equal to price with an adjustment equal to quantity times the slope of the demand curve.

A perfectly competitive firm faces a demand curve with a slope of zero. Substituting 0 for $\Delta P / \Delta Q$ into the expression given, it becomes clear that $M R$ is equal to price for the perfectly competitive firm-it need not lower its price to sell an additional unit. For a firm in an imperfectly competitive market, however, the demand curve is negatively sloped $(\Delta P / \Delta Q<0)$. Substituting this negative number into the expression for $M R, P+Q(\Delta P / \Delta Q)$, it becomes clear that MR for an imperfectly competitive firm is less than price.

Marginal cost (MC) is the increase to total cost resulting from the firm's decision to increase output by one additional unit per time period: $\mathrm{MC}=\Delta \mathrm{TC} / \Delta Q$. Economists distinguish between short-run marginal cost (SMC) and long-run marginal cost (LMC). Labor is variable over the short run, but the quantity of capital cannot be changed in the short run because there is a lead time required to build or buy new plant equipment and put it in place. In the long run, all inputs are variable.

SMC is essentially the additional cost of the variable input, labor, that must be incurred to increase the level of output by one unit. LMC is the additional cost of all inputs necessary to increase the level of output, allowing the firm the flexibility of changing both labor and capital inputs in a way that maximizes efficiency. Understanding MC is aided by recalling that cost is directly related to input prices and inversely related to productivity. For example, if the wage rate were to rise, cost would also rise. If labor were to become more productive, cost would fall. This relationship can be captured in an expression that relates SMC to wage rate $(w)$ and $\mathrm{MP}_{L}: \mathrm{SMC}=w / \mathrm{MP}_{L}$

This relationship between cost and productivity also holds with average variable cost. Variable costs are all costs that fluctuate with the level of production and sales. Average variable cost (AVC) is the ratio of total variable cost to total output: $\mathrm{AVC}=$ $\mathrm{TVC} / Q$. Again, if labor's wage rises, AVC also rises; but if labor were to become more productive, AVC falls. This relationship is captured by the expression $\mathrm{AVC}=w / \mathrm{AP}_{L}$.

Earlier, we noted that over some range of low output, the firm might benefit from increasing marginal productivity of its labor input as workers begin to specialize. As the $\mathrm{MP}_{L}$ increases, SMCs decline. Eventually, as more and more labor is added to a fixed amount of capital, the $\mathrm{MP}_{L}$ must fall, causing SMCs to rise.

We began this section by stating that the goal of management is to maximize profit. We now address the conditions necessary for reaching that goal. Consider a firm currently producing 1,000 widgets each week and whose management is contemplating increasing that output incrementally. Would that additional unit increase profit? Clearly, profit would be increased (or losses reduced) if the additional revenue from that next unit were greater than the additional cost. So, a profit-seeking firm should increase $Q$ if $M R>M C$. Conversely, if the additional unit added more to cost than to revenue, the firm should reduce output because it would save more in cost than it would lose in revenue. Only if the additional cost were exactly equal to the additional revenue would the firm be maximizing its profit.

There is another condition (called a second-order condition) necessary for profit maximization: At the level of output at which $M R=M C, M C$ cannot be falling. This condition is fairly intuitive. If $\mathrm{MC}$ is falling with additional output, $\mathrm{MP}_{L}$ would be rising. (Recall that $\mathrm{SMC}=w / \mathrm{MP}_{L}$ ) If one additional hour of labor input causes $\mathrm{MC}$ to fall, the firm would want to add that hour and continue adding labor until SMC becomes positively sloped. We can sum up the profit-maximization decision for an operating firm as follows: Produce the level of output such that (1) MR = MC and (2) MC is not falling.

Understanding the Interaction between Total, Variable, Fixed, and Marginal Cost and Output

Exhibit 11 shows the graphical relationships between total cost, total fixed cost, and total variable cost. TC is the summation of all costs, where costs are classified on the basis of whether they are fixed or variable. Total fixed cost (TFC) is the summation of all expenses that do not change as the level of production varies. Total variable cost (TVC) is the summation of all variable expenses; TVC rises with increased production and falls with decreased production. At zero production, TC is equal to TFC because TVC at this output level is zero. The curve for TC always lies parallel to and above the TVC curve by the amount of TFC.

\section{Exhibit 11: Total Cost, Total Variable Cost, and Total Fixed Cost}
\begin{center}
\includegraphics[max width=\textwidth]{2023_05_04_ccb578c5089fd41615beg-041(1)}
\end{center}

Exhibit 12 shows the relationships between the average total cost (ATC), average variable cost (AVC), average fixed cost (AFC), and marginal cost (MC) curves in the short run. As output quantity increases, AFC declines because TFCs are spread over a larger number of units. Both ATC and AVC take on a bowl-shaped pattern in which each curve initially declines, reaches a minimum average cost output level, and then increases after that point. The MC curve intersects both the ATC and the AVC at their minimum points-points $S$ and $T$. When MC is less than AVC, AVC will be decreasing. When MC is greater than AVC, AVC will be increasing.

\section{Exhibit 12: Average Total Cost, Average Variable Cost, Average Fixed Cost,}
 and Marginal CostCost per Unit

\begin{center}
\includegraphics[max width=\textwidth]{2023_05_04_ccb578c5089fd41615beg-041}
\end{center}

$S$, the lowest point on the AVC curve, is where $M C$ equals AVC. Beyond quantity $Q_{A V C}, M C$ is greater than AVC; thus, the AVC curve begins to rise. Note that it occurs at a quantity lower than the minimum point on the ATC curve.

$T$, the lowest point on the ATC curve, is where MC equals ATC. Beyond quantity $Q_{A T C}$, MC is greater than ATC; thus, the ATC curve is rising.

$A$, the difference between ATC and AVC at output quantity $Q_{1}$, is the amount of AFC.

$R$ indicates the lowest point on the $\mathrm{MC}$ curve. Beyond this point of production, fixed input constraints reduce the productivity of labor.

$X$ indicates the difference between ATC and AVC at quantity $Q_{2}$. It is less than $A$ because AFC $(Y)$ falls with output. Exhibit 13 shows an example of how total, average, and marginal costs are derived. TC is calculated by summing TFC and TVC. MC is derived by observing the change in TC as the quantity variable changes. There is a relationship that always holds for average and marginal costs: If $\mathrm{MC}$ is less than average cost, average cost must fall, and if $\mathrm{MC}$ is greater than average cost, average cost must rise. For example, in Exhibit 13, AVC begins to increase as output rises from 2 to 3 units because MC (50) is greater than AVC (41.7). Also from Exhibit 13, ATC declines up to 3 units because MC is less than ATC. After 3 units, ATC increases because the MC of Unit 4 (85) exceeds the ATC of all prior units (75). Initially, the MC curve declines because of increasing marginal returns to labor, but at some point, it begins to increase because of the law of diminishing marginal returns.

\section{Exhibit 13: Total, Average, Marginal, Fixed, and Variable Costs}
\begin{center}
\begin{tabular}{lccccccc}
\hline
Quantity (Q) & TFC $^{2}$ & AFC & TVC & AVC & TC & ATC & MC \\
\hline
0 & 100 & - & 0 & - & 100 & - & - \\
1 & 100 & 100.0 & 50 & 50.0 & 150 & 150.0 & 50 \\
2 & 100 & 50.0 & 75 & 37.5 & 175 & 87.5 & 25 \\
3 & 100 & 33.3 & 125 & 41.7 & 225 & 75.0 & 50 \\
4 & 100 & 25.0 & 210 & 52.5 & 310 & 77.5 & 85 \\
5 & 100 & 20.0 & 300 & 60.0 & 400 & 80.0 & 90 \\
6 & 100 & 16.7 & 450 & 75.0 & 550 & 91.7 & 150 \\
7 & 100 & 14.3 & 650 & 92.9 & 750 & 107.1 & 200 \\
8 & 100 & 12.5 & 900 & 112.5 & 1,000 & 125.0 & 250 \\
9 & 100 & 11.1 & 1,200 & 133.3 & 1,300 & 144.4 & 300 \\
10 & 100 & 10.0 & 1,550 & 155.0 & 1,650 & 165.0 & 350 \\
\hline
\end{tabular}
\end{center}

${ }^{a}$ Includes all opportunity costs.

As stated earlier, TC increases as the firm expands output and decreases when production is cut. TC increases at a decreasing rate up to a certain output level. Thereafter, the rate of increase accelerates as the firm gets closer to full utilization of capacity. The rate of change in TC mirrors the rate of change in TVC. In Exhibit 13, TC at 5 units is 400 -of which 300 is variable cost and 100 is fixed cost. At 10 units, TC is 1,650 -of which 1,550 is variable cost and 100 is fixed cost.

Fixed costs typically are incurred whether the firm produces anything or not. Fixed costs may stay the same over a given range of production but can change to another constant level when production moves outside of that range. The latter is referred to as a quasi-fixed cost, although it remains categorized as part of TFC. Examples of fixed costs are debt service, real estate lease agreements, and rental contracts. Normal profit is also considered to be a fixed cost because it is a return required by investors on their equity capital regardless of output level. Quasi-fixed cost examples would be certain utilities and administrative salaries that could be lower or avoided altogether when output is zero but would rise to higher constant levels over different production ranges.

Other fixed costs evolve primarily from investments in such fixed assets as real estate, production facilities, and equipment. These fixed costs cannot be arbitrarily cut when production declines. When a firm downsizes, the last expense to be cut is usually fixed cost. TVC has a direct relationship with quantity. When quantity increases, TVC increases; when quantity decreases, TVC declines. At zero production, TVC is always zero. Variable cost examples are payments for labor, raw materials, and supplies. The change in TVC declines up to a certain output point and then increases as production approaches capacity limits. In Exhibit 13, TVC increases with an increase in quantity. However, the change from 1 to 2 units is 25 (75 - 50), and the change from 9 to 10 units is 350.

Exhibit 14 illustrates the relationships between MC, ATC, AVC, and AFC for the data presented in Exhibit 13.

Exhibit 14: Average Total Cost, Average Variable Cost, Average Fixed Cost, and Marginal Cost for Exhibit 13 Data

\begin{center}
\includegraphics[max width=\textwidth]{2023_05_04_ccb578c5089fd41615beg-043}
\end{center}

Dividing TFC by quantity yields AFC. AFC decreases throughout the production span, reflecting the spreading of a constant cost over more and more production units. At high production volumes, AFC may be so low that it is a small proportion of ATC. In Exhibit 13, AFC declines from 100 at 1 unit to 20 at 5 units, and then to 10 at an output level of 10 units.

In Exhibit 13, AVC at 5 units is 60 (300/5). Over an initial range of production, AVC declines and then reaches a minimum point. Thereafter, AVC increases as the firm uses more of its production capacity. This higher cost results primarily from production constraints imposed by the fixed assets at higher volume levels. The lowest AVC quantity does not correspond to the least-cost quantity for ATC because AFC is still declining. In Exhibit 13, AVC is minimized at 2 units, whereas ATC is minimized at 3 units.

ATC is calculated by dividing TC by quantity (or by summing AFC and AVC). In Exhibit 13, at 3 units, ATC is 75 (TC of 225/3 units of production or AFC of 33.3 $+\mathrm{AVC}$ of 41.7). This is the least average cost point of production and the minimum point on the ATC curve. Although cost-minimizing behavior on the part of the firm would dictate operating at the minimum point on its ATC curve, the profit-maximizing quantity may not correspond to this minimum ATC point. Profit per unit, but not necessarily total profit, is maximized at this point.

\section{EXAMPLE 4}
\section{Calculation and Interpretation of Total, Average, Marginal, Fixed, and Variable Costs}
\begin{enumerate}
  \item The first three columns of Exhibit 15 display data on quantity, TFC, and TVC, which are used to calculate TC, AFC, AVC, ATC, and MC. Examine the results for total, average, marginal, fixed, and variable costs. Identify the quantity levels at which the ATC, AVC, and MC values reach their minimum points. Explain the relationship between TFC and TC at a quantity of zero output.
\end{enumerate}

\section{Exhibit 15}
\begin{center}
\begin{tabular}{lccccccc}
\hline
$\mathbf{Q}$ & TFC $^{\mathbf{a}}$ & TVC & AFC & AVC & TC & ATC & MC \\
\hline
0 & 5,000 & 0 & - & - & 5,000 & - & - \\
1 & 5,000 & 2,000 & $5,000.0$ & 2,000 & 7,000 & $7,000.0$ & 2,000 \\
2 & 5,000 & 3,800 & $2,500.0$ & 1,900 & 8,800 & $4,400.0$ & 1,800 \\
3 & 5,000 & 5,400 & $1,666.7$ & 1,800 & 10,400 & $3,466.7$ & 1,600 \\
4 & 5,000 & 8,000 & $1,250.0$ & 2,000 & 13,000 & $3,250.0$ & 2,600 \\
5 & 5,000 & 11,000 & $1,000.0$ & 2,200 & 16,000 & $3,200.0$ & 3,000 \\
6 & 5,000 & 15,000 & 833.3 & 2,500 & 20,000 & $3,333.3$ & 4,000 \\
7 & 5,000 & 21,000 & 714.3 & 3,000 & 26,000 & $3,714.3$ & 6,000 \\
8 & 5,000 & 28,800 & 625.0 & 3,600 & 33,800 & $4,225.0$ & 7,800 \\
9 & 5,000 & 38,700 & 555.6 & 4,300 & 43,700 & $4,855.6$ & 9,900 \\
10 & 5,000 & 51,000 & 500.0 & 5,100 & 56,000 & $5,600.0$ & 12,300 \\
\hline
\end{tabular}
\end{center}

a Includes all opportunity costs

\section{Solution:}
TFC remains unchanged at 5,000 throughout the entire production range, whereas AFC continuously declines from 5,000 at 1 unit to 500 at 10 units. Both AVC and MC initially decline and then reach their lowest level at 3 units, with costs of 1,800 and 1,600, respectively. Beyond 3 units, both AVC and $\mathrm{MC}$ increase, indicating that the cost of production rises with greater output. The least-cost point for ATC is 3,200 at 5 units. At zero output, TC is 5,000 , which equals the amount of TFC (at zero output, the firm will need no variable inputs, but it is committed to its fixed plant and equipment in the short run).

PERFECT AND IMPERFECT COMPETITION, PROFIT MAXIMIZATION

determine and interpret breakeven and shutdown points of production Recall from our earlier discussion of profit-maximizing conditions that a firm can generally be classified as operating in either a perfectly competitive or an imperfectly competitive environment. The difference between the two manifests itself in the slope of the demand curve facing the firm. If the environment of the firm is perfectly competitive, it must take the market price of its output as given, so it faces a perfectly elastic, horizontal demand curve. In this case, as we saw previously, the firm's MR and the price of its product are identical. Additionally, the firm's average revenue (AR), or revenue per unit, is also equal to price per unit. However, a firm that faces a negatively sloped demand curve must lower its price to sell an additional unit, so its MR is less than price $(P)$.

These characteristics of MR are also applicable to the TR functions. Under conditions of perfect competition, TR (as always) is equal to price times quantity: $T R=$ $(P)(Q)$. But under conditions of perfect competition, price is dictated by the market; the firm has no control over price. As the firm sells one more unit, its TR rises by the exact amount of price per unit.

Under conditions of imperfect competition, price is a variable under the firm's control, and therefore price is a function of quantity: $P=\mathrm{f}(Q)$, and $T R=\mathrm{f}(Q) \times Q$. For simplicity, suppose the firm is monopolistic and faces the market demand curve, which we will assume is linear and negatively sloped. Because the monopolist is the only seller, its TR is identical to the total expenditure of all buyers in the market. Earlier, we noted what happens as price is reduced and quantity sold increases in this environment: At first, a decrease in price increases total expenditure by buyers and TR to the firm because the decrease in price is outweighed by the increase in units sold. But as price continues to fall, the decrease in price overshadows the increase in quantity, and total expenditure (revenue) falls. We can now depict the demand and TR functions for firms under conditions of perfect and imperfect competition, as shown in Exhibit 16.

Exhibit 16: Demand and Total Revenue Functions for Firms under

Conditions of Perfect and Imperfect Competition

A. Perfectly Competitive Firm

\begin{center}
\includegraphics[max width=\textwidth]{2023_05_04_ccb578c5089fd41615beg-045(1)}
\end{center}

B. Imperfectly Competitive Firm

\begin{center}
\includegraphics[max width=\textwidth]{2023_05_04_ccb578c5089fd41615beg-045}
\end{center}

\section{Topics in Demand and Supply Analysis}
Panel A of Exhibit 16 depicts the demand curve (upper graph) and total revenue curve (lower graph) for the firm under conditions of perfect competition. Notice that the vertical axis in the upper graph is price per unit (e.g., $\pounds /$ bushel), whereas TR is measured on the vertical axis in the lower graph (e.g., $\pounds /$ week.) The same is true for the respective axes in Panel B, which depicts the demand and total revenue curves for the monopolist. The TR curve for the firm under conditions of perfect competition is linear, with a slope equal to price per unit. The TR curve for the monopolist first rises (in the range where MR is positive and demand is elastic) and then falls (in the range where $M R$ is negative and demand is inelastic) with output.

\section{Profit-Maximization, Breakeven, and Shutdown Points of Production}
We can now combine the firm's short-run TC curves with its TR curves to represent profit maximization in the cases of perfect competition and imperfect competition. Exhibit 17 shows both the AR and average cost curves in one graph for the firm under conditions of perfect competition.

Exhibit 17: Demand and Average and Marginal Cost Curves for the Firm under Conditions of Perfect Competition

\begin{center}
\includegraphics[max width=\textwidth]{2023_05_04_ccb578c5089fd41615beg-046}
\end{center}

The firm is maximizing profit by producing $Q^{*}$, where price is equal to SMC and SMC is rising. (Note that there is another output level, $Q^{\prime}$, where $P=$ SMC, but at that point, SMC is still falling, so this cannot be a profit-maximizing solution.) If market price were to rise, the firm's demand and MR curve would simply shift upward, and the firm would reach a new profit-maximizing output level to the right of $Q^{*}$. If, on the other hand, market price were to fall, the firm's demand and MR curve would shift downward, resulting in a new and lower level of profit-maximizing output. As depicted, this firm is currently earning a positive economic profit because market price exceeds ATC at output level $Q$ *. This profit is possible in the short run, but in the long run, competitors would enter the market to capture some of those profits and would drive the market price down to a level equal to each firm's ATC.

Exhibit 18 depicts the cost and revenue curves for the monopolist that is facing a negatively sloped market demand curve. The MR and demand curves are not identical for this firm. But the profit-maximizing rule is still the same: Find the level of $Q$ that equates SMC to MR-in this case, $Q^{*}$. Once that level of output is determined, the optimal price to charge is given by the firm's demand curve at $P^{*}$. This monopolist is earning positive economic profit because its price exceeds its ATC. The barriers to entry that give this firm its monopolistic power mean that outside competitors would be unable to compete away this firm's profits.

Exhibit 18: Demand and Average and Marginal Cost Curves for the Monopolistic Firm

\begin{center}
\includegraphics[max width=\textwidth]{2023_05_04_ccb578c5089fd41615beg-047}
\end{center}

\section{BREAKEVEN ANALYSIS AND SHUTDOWN DECISION}
$\mid \begin{aligned} & \text { determine and interpret breakeven and shutdown points of } \\ & \text { production }\end{aligned}$

A firm is said to break even if its TR is equal to its TC. It can also be said that a firm breaks even if its price (AR) is exactly equal to its ATC, which is true under conditions of perfect and imperfect competition. Of course, the goal of management is not just to break even but to maximize profit. However, perhaps the best the firm can do is cover all of its economic costs. Economic costs are the sum of total accounting costs and implicit opportunity costs. A firm whose revenue is equal to its economic costs is covering the opportunity cost of all of its factors of production, including capital. Economists would say that such a firm is earning normal profit, but not positive economic profit. It is earning a rate of return on capital just equal to the rate of return that an investor could expect to earn in an equivalently risky alternative investment (opportunity cost). Firms that are operating in a very competitive environment with no barriers to entry from other competitors can expect, in the long run, to be unable to earn a positive economic profit; the excess rate of return would attract entrants who would produce more output and ultimately drive the market price down to the level at which each firm is at best just earning a normal profit. This situation, of course, does not imply that the firm is earning zero accounting profit.

Exhibit 19 depicts the condition for both a firm under conditions of perfect competition (Panel A) and a monopolist (Panel B) in which the best each firm can do is to break even. Note that at the level of output at which SMC is equal to MR, price is just equal to ATC. Hence, economic profit is zero, and the firms are breaking even. Exhibit 19: Examples of Firms under Perfect Competition and Monopolistic Firms That Can, at Best, Break Even

A. Perfect Competition

\begin{center}
\includegraphics[max width=\textwidth]{2023_05_04_ccb578c5089fd41615beg-048}
\end{center}

\section{B. Monopolist}
\begin{center}
\includegraphics[max width=\textwidth]{2023_05_04_ccb578c5089fd41615beg-048(1)}
\end{center}

\section{The Shutdown Decision}
In the long run, if a firm cannot earn at least a zero economic profit, it will not operate because it is not covering the opportunity cost of all of its factors of production, labor, and capital. In the short run, however, a firm might find it advantageous to continue to operate even if it is not earning at least a zero economic profit. The discussion that follows addresses the decision to continue to operate and earn negative profit or shut down operations.

Recall that typically some or all of a firm's fixed costs are incurred regardless of whether the firm operates. The firm might have a lease on its building that it cannot avoid paying until the lease expires. In that case, the lease payment is a sunk cost: It cannot be avoided, no matter what the firm does. Sunk costs must be ignored in the decision to continue to operate in the short run. As long as the firm's revenues cover at least its variable cost, the firm is better off continuing to operate. If price is greater than AVC, the firm is not only covering all of its variable cost but also a portion of fixed cost.

For example, suppose a firm is producing 100 widgets and selling them at a price of $€ 4$ each. Obviously, its TR is $€ 400$ per time period. Suppose, also, that at that level of output, its ATC is $€ 7$, made up of AVC of $€ 3.75$ plus AFC of $€ 3.25$ per period. This firm is said to be earning negative economic profit (also referred to as economic loss, a condition in which revenues fall short of total opportunity cost) of $€ 300$ because its TC is $€ 700$ and its TR is only $€ 400$. Should this firm shut down immediately? If the fixed cost is unavoidable, then the firm is obligated to pay it whether it operates or not. The TFC is $€ 325$ ( $€ 3.25$ per unit on 100 units). If it shuts down and earns zero revenue, then its variable cost would be zero but its losses would still equal the $€ 325$ of unavoidable fixed cost. If, however, it continued to operate, it could earn revenue of $€ 400$ that would cover its variable cost of $€ 375$ and contribute $€ 25$ toward the fixed costs. In other words, this firm would lose less by continuing to operate $(€ 300)$ than by shutting down ( $€ 325)$.

In the long run, unless market price increases, this firm would exit the industry. But in the short run, it will continue to operate at a loss. Exhibit 20 depicts a firm under conditions of perfect competition facing three alternative market price ranges for its output. At any price above $P_{1}$, the firm can earn a positive profit and clearly should continue to operate. At a price below $P_{2}$, the minimum AVC, the firm could not even cover its variable cost and should shut down. At prices between $P_{2}$ and $P_{1}$, the firm should continue to operate in the short run because it is able to cover all of its variable cost and contribute something toward unavoidable fixed costs. Economists refer to the minimum AVC point as the shutdown point and the minimum ATC point as the breakeven point.

Exhibit 20: A Firm under Conditions of Perfect Competition Will Choose to Shut Down If Market Price Is Less Than Minimum AVC

\begin{center}
\includegraphics[max width=\textwidth]{2023_05_04_ccb578c5089fd41615beg-049}
\end{center}

\section{EXAMPLE 5}
\section{Breakeven Analysis and Profit Maximization When the Firm Faces a Negatively Sloped Demand Curve under Imperfect Competition}
Revenue and cost information for a future period is presented in Exhibit 21 for WR International, a newly formed corporation that engages in the manufacturing of low-cost, pre-fabricated dwelling units for urban housing markets in emerging economies. (Note that quantity increments are in blocks of 10 for a 250 change in price.) The firm has few competitors in a market setting of imperfect competition.

\section{Exhibit 21}
\begin{center}
\begin{tabular}{lcccc}
\hline
$\begin{array}{l}\text { Quantity } \\ (\mathbf{Q})\end{array}$ & $\begin{array}{c}\text { Price } \\ (\mathbf{)})\end{array}$ & $\begin{array}{c}\text { Total Revenue } \\ \text { (TR) }\end{array}$ & $\begin{array}{c}\text { Total Cost } \\ \text { (TC) }\end{array}$ & Profit \\
\hline
0 & 10,000 & 0 & 100,000 & $-100,000$ \\
10 & 9,750 & 97,500 & 170,000 & $-72,500$ \\
20 & 9,500 & 190,000 & 240,000 & $-50,000$ \\
30 & 9,250 & 277,500 & 300,000 & $-22,500$ \\
40 & 9,000 & 360,000 & 360,000 & 0 \\
50 & 8,750 & 437,500 & 420,000 & 17,500 \\
60 & 8,500 & 510,000 & 480,000 & 30,000 \\
70 & 8,250 & 577,500 & 550,000 & 27,500 \\
80 & 8,000 & 640,000 & 640,000 & 0 \\
90 & 7,750 & 697,500 & 710,000 & $-12,500$ \\
100 & 7,500 & 750,000 & 800,000 & $-50,000$ \\
\hline
\end{tabular}
\end{center}

a Includes all opportunity costs

\begin{enumerate}
  \item How many units must WR International sell to initially break even?
\end{enumerate}

\section{Solution to 1:}
WR International will initially break even at 40 units of production, where TR and TC equal 360,000.

\begin{enumerate}
  \setcounter{enumi}{1}
  \item Where is the region of profitability?
\end{enumerate}

\section{Solution to 2:}
The region of profitability will range from greater than 40 units to less than 80 units. Any production quantity of less than 40 units and any quantity greater than 80 units will result in an economic loss. 3. At what point will the firm maximize profit? At what points are there economic losses?

\section{Solution to 3:}
Maximum profit of 30,000 will occur at 60 units. Lower profit will occur at any output level that is higher or lower than 60 units. From 0 units to less than 40 units and for quantities greater than 80 units, economic losses occur.

Given the relationships between TR, TVC, and TFC, Exhibit 22 summarizes the decisions to operate, shut down production, or exit the market in both the short run and the long run. The firm must cover its variable cost to remain in business in the short run; if $\mathrm{TR}$ cannot cover $\mathrm{TVC}$, the firm shuts down production to minimize loss. The loss would be equal to the amount of fixed cost. If $\mathrm{TVC}$ exceeds $\mathrm{TR}$ in the long run, the firm will exit the market to avoid the loss associated with fixed cost at zero production. By exiting the market, the firm's investors do not suffer the erosion of their equity capital from economic losses. When TR is enough to cover TVC but not all of TFC, the firm can continue to produce in the short run but will be unable to maintain financial solvency in the long run.

\section{Exhibit 22: Short Run and Long Run Decisions to Operate or Not}
\begin{center}
\begin{tabular}{lcc}
\hline
Revenue-Cost Relationship & Short-Run Decision & Long-Term Decision \\
\hline
$\mathrm{TR}=\mathrm{TC}$ & Stay in market & Stay in market \\
$\mathrm{TR}=\mathrm{TVC}$ but $<\mathrm{TC}$ & Stay in market & Exit market \\
$\mathrm{TR}<\mathrm{TVC}$ & Shut down production & Exit market \\
\hline
\end{tabular}
\end{center}

\section{EXAMPLE 6}
\section{Shutdown Analysis}
For the most recent financial reporting period, a business domiciled in Ecuador (which recognizes the US dollar as an official currency) has revenue of $\$ 2$ million and TC of $\$ 2.5$ million, which are or can be broken down into TFC of $\$ 1$ million and TVC of $\$ 1.5$ million. The net loss on the firm's income statement is reported as $\$ 500,000$ (ignoring tax implications). In prior periods, the firm had reported profits on its operations.

\begin{enumerate}
  \item What decision should the firm make regarding operations over the short term?
\end{enumerate}

\section{Solution to 1:}
In the short run, the firm is able to cover all of its TVC but only half of its $\$ 1$ million in TFC. If the business ceases to operate, its loss would be $\$ 1$ million, the amount of TFC, whereas the net loss by operating would be minimized at $\$ 500,000$. The firm should attempt to operate by negotiating special arrangements with creditors to buy time to return operations back to profitability. 2. What decision should the firm make regarding operations over the long term?

\section{Solution to 2:}
If the revenue shortfall is expected to persist over time, the firm should cease operations, liquidate assets, and pay debts to the extent possible. Any residual for shareholders would decrease the longer the firm is allowed to operate unprofitably.

\begin{enumerate}
  \setcounter{enumi}{2}
  \item Assume the same business scenario except that revenue is now $\$ 1.3$ million, which creates a net loss of $\$ 1.2$ million. What decision should the firm make regarding operations in this case?
\end{enumerate}

\section{Solution to 3:}
The firm would minimize loss at $\$ 1$ million of TFC by shutting down. If the firm decided to continue to do business, the loss would increase to $\$ 1.2$ million. Shareholders would save $\$ 200,000$ in equity value by pursuing this option. Unquestionably, the business would have a rather short life expectancy if this loss situation were to continue.

When evaluating profitability, particularly of start-up firms and businesses using turnaround strategies, analysts should consider highlighting breakeven and shutdown points in their financial research. Identifying the unit sales levels at which the firm enters or leaves the production range for profitability and at which the firm can no longer function as a viable business entity provides invaluable insight for investment decisions.

\section{ECONOMIES AND DISECONOMIES OF SCALE WITH SHORT-RUN AND LONG-RUN COST ANALYSIS}
$$
\begin{aligned}
& \text { describe how economies of scale and diseconomies of scale affect } \\
& \text { costs }
\end{aligned}
$$

Rational behavior dictates that the firm select an operating size or scale that maximizes profit over any time frame. The time frame that defines the short run and long run for any firm is based on the ability of the firm to adjust the quantities of the fixed resources it uses. The short run is the time period during which at least one of the factors of production, such as technology, physical capital, and plant size, is fixed. The long run is defined as the time period during which all factors of production are variable. Additionally, in the long run, firms can enter or exit the market based on decisions regarding profitability. The long run is often referred to as the "planning horizon" in which the firm can choose the short-run position or optimal operating size that maximizes profit over time. The firm is always operating in the short run but planning in the long run.

The time required for long-run adjustments varies by industry. For example, the long run for a small business using very little technology and physical capital may be less than a year, whereas for a capital-intensive firm, the long run may be more than a decade. Given enough time, however, all production factors are variable, which allows the firm to choose an operating size or plant capacity based on different technologies and physical capital. In this regard, costs and profits will differ between the short run and the long run.

\section{Short- and Long-Run Cost Curves}
Recall that when we addressed the short-run cost curves of the firm, we assumed that the capital input was held constant. That meant that the only way to vary output in the short run is to change the level of the variable input-in our case, labor. If the capital input-namely, plant and equipment-were to change, however, we would have an entirely new set of short-run cost curves, one for each level of capital input.

The short-run total cost includes all the inputs-labor and capital-the firm is using to produce output. For reasons discussed earlier, the typical short-run total cost (STC) curve might rise with output, first at a decreasing rate because of specialization economies and then at an increasing rate, reflecting the law of diminishing marginal returns to labor. Total fixed cost (the quantity of capital input multiplied by the rental rate on capital) determines the vertical intercept of the STC curve. At higher levels of fixed input, TFC is greater but the production capacity of the firm is also greater. Exhibit 23 shows three different STC curves for the same technology but using three distinct levels of capital input-Points 1, 2, and 3 on the vertical axis.

\section{Exhibit 23: Short-Run Total Cost Curves for Various Plant Sizes}
\begin{center}
\includegraphics[max width=\textwidth]{2023_05_04_ccb578c5089fd41615beg-053}
\end{center}

Plant Size 1 is the smallest and, of course, has the lowest fixed cost; hence, its STC $_{1}$ curve has the lowest vertical intercept. But note that $\mathrm{STC}_{1}$ begins to rise more steeply with output, reflecting the lower plant capacity. Plant Size 3 is the largest of the three and reflects that size with both a higher fixed cost and a lower slope at any level of output. If a firm decided to produce an output between zero and $Q_{a}$, it would plan on building Plant Size 1 because for any output level in that range, its cost is less than it would be for Plant Size 2 or 3. Accordingly, if the firm were planning to produce output greater than $Q_{b}$, it would choose Plant Size 3 because its cost for any of those levels of output would be lower than for Plant Size 1 or 2 . And of course, Plant Size 2 would be chosen for output levels between $Q_{a}$ and $Q_{b}$. The long-run total cost curve is derived from the lowest level of STC for each level of output because in the long run, the firm is free to choose which plant size it will operate. This curve is called an "envelope curve." In essence, this curve envelopes-encompasses-all possible combinations of technology, plant size, and physical capital.

For each STC curve, there is also a corresponding short-run average total cost (SATC) curve and a corresponding long-run average total cost (LRAC) curve, the envelope curve of all possible short-run average total cost curves. The shape of the LRAC curve reflects an important concept called economies of scale and diseconomies of scale.

\section{Defining Economies of Scale and Diseconomies of Scale}
When a firm increases all of its inputs in order to increase its level of output (obviously, a long-run concept), it is said to scale up its production. Scaling down is the reverse-decreasing all of its inputs in order to produce less in the long run. Economies of scale occur if, as the firm increases its output, cost per unit of production falls. Graphically, this definition translates into a LRAC curve with a negative slope. Exhibit 24 depicts several SATC curves, one for each plant size, and the LRAC curve representing economies of scale.

Exhibit 24: Short-run Average Total Cost Curves for Various Plant Sizes and Their Envelope Curve, LRAC: Economies of Scale

\begin{center}
\includegraphics[max width=\textwidth]{2023_05_04_ccb578c5089fd41615beg-054}
\end{center}

Diseconomies of scale occur if cost per unit rises as output increases. Graphically, diseconomies of scale translate into an LRAC curve with a positive slope. Exhibit 25 depicts several SATC curves, one for each plant size, and their envelope curve, the LRAC curve, representing diseconomies of scale. Exhibit 25: Short-run Average Total Cost Curves for Various Plant Sizes and Their Envelope Curve, LRAC: Diseconomies of Scale

\begin{center}
\includegraphics[max width=\textwidth]{2023_05_04_ccb578c5089fd41615beg-055}
\end{center}

As the firm grows in size, economies of scale and a lower ATC can result from the following factors:

\begin{itemize}
  \item Increasing returns to scale, which is when a production process allows for increases in output that are proportionately larger than the increase in inputs.

  \item Having a division of labor and management in a large firm with numerous workers, which allows each worker to specialize in one task rather than perform many duties, as in the case of a small business (as such, workers in a large firm become more proficient at their jobs).

  \item Being able to afford more expensive, yet more efficient equipment and to adapt the latest in technology that increases productivity.

  \item Effectively reducing waste and lowering costs through marketable byproducts, less energy consumption, and enhanced quality control.

  \item Making better use of market information and knowledge for more effective managerial decision making.

  \item Obtaining discounted prices on resources when buying in larger quantities.

\end{itemize}

A classic example of a business that realizes economies of scale through greater physical capital investment is an electric utility. By expanding output capacity to accommodate a larger customer base, the utility company's per-unit cost will decline. Economies of scale help explain why electric utilities have naturally evolved from localized entities to regional and multi-region enterprises. Wal-Mart is an example of a business that uses bulk purchasing power to obtain deep discounts from suppliers to keep costs and prices low. Wal-Mart also uses the latest technology to monitor point-of-sale transactions to gather timely market information to respond to changes in customer buying behavior, which leads to economies of scale through lower distribution and inventory costs.

The factors that can lead to diseconomies of scale, inefficiencies, and rising costs when a firm increases in size include the following:

\begin{itemize}
  \item Decreasing returns to scale, which is when a production process leads to increases in output that are proportionately smaller than the increase in inputs.

  \item Being so large that it cannot be properly managed.

  \item Overlapping and duplication of business functions and product lines.

  \item Higher resource prices because of supply constraints when buying inputs in large quantities. Before its restructuring, General Motors (GM) was an example of a business that had realized diseconomies of scale by becoming too large. Scale diseconomies occurred through product overlap and duplication (i.e., similar or identical automobile models), and the fixed cost for these models was not spread over a large volume of output. In 2009, GM decided to discontinue three brands (Saturn, Pontiac, and Hummer), and in 2018 it was considering dropping various low-volume product models that overlapped with other models by 2020. GM had numerous manufacturing plants throughout the world and sold vehicles in more than a hundred countries. Given this geographical dispersion in production and sales, the company had communication and management coordination problems, which resulted in higher costs. In 2017, GM sold its European arm, Opel, to Groupe PSA, the maker of Peugeot and Citroën. GM also had significantly higher labor costs than its competitors. As the largest producer in the market, it had been a target of labor unions for higher compensation and benefits packages relative to other firms.

\end{itemize}

Economies and diseconomies of scale can occur at the same time; the impact on long-run average total cost (LRAC) depends on which dominates. If economies of scale dominate, LRAC decreases with increases in output. The reverse holds true when diseconomies of scale prevail. There may be a range of output over which LRAC falls (economies of scale) and then a range over which LRAC might be constant, followed by a range over which diseconomies of scale prevail, as depicted in Exhibit 26.

The minimum point on the LRAC curve is referred to as the minimum efficient scale. The minimum efficient scale is the optimal firm size under perfect competition over the long run. Theoretically, perfect competition forces the firm to operate at the minimum point on the LRAC curve because the market price will be established at this level over the long run. If the firm is not operating at this least-cost point, its long-term viability will be threatened.

\section{Exhibit 26: LRAC Can Exhibit Economies and Diseconomies of Scale}
\begin{center}
\includegraphics[max width=\textwidth]{2023_05_04_ccb578c5089fd41615beg-056(1)}
\end{center}

Units/Time Period

\section{EXAMPLE 7}
\section{Long-Run Average Total Cost Curve}
\begin{enumerate}
  \item Exhibit 27 displays the long-run average total cost curve (LRAC $\mathrm{US}$ ) and the short-run average total cost curves for three hypothetical US-based automobile manufacturers-Starr Vehicles (Starr), Rocket Sports Cars (Rocket), and General Auto (GenAuto). The LRAC curve for foreign-owned automobile
\end{enumerate}

\begin{center}
\includegraphics[max width=\textwidth]{2023_05_04_ccb578c5089fd41615beg-056}
\end{center}

cated in the graph. (The market structure implicit in the exhibit is imperfect competition.)

To what extent are the cost relationships depicted in Exhibit 27 useful for an economic and financial analysis of the three US-based auto firms?

\section{Exhibit 27}
\begin{center}
\includegraphics[max width=\textwidth]{2023_05_04_ccb578c5089fd41615beg-057}
\end{center}

\section{Solution:}
First, it is observable that the foreign auto companies have a lower LRAC compared with that of the US automobile manufacturers. This competitive position places the US firms at a cost-and possibly, pricing-disadvantage in the market, with the potential to lose market share to the lower-cost foreign competitors. Second, only Rocket operates at the minimum point of the $\mathrm{LRAC}_{\mathrm{US}}$, whereas GenAuto is situated in the region of diseconomies of scale and Starr is positioned in the economies of scale portion of the curve. To become more efficient and competitive, GenAuto needs to downsize and restructure, which means moving down the LRA $C_{\mathrm{US}}$ curve to a smaller, yet lower-cost production volume. In Contrast, Starr has to grow in size to become more efficient and competitive by lowering per-unit costs.

From a long-term investment prospective and given its cost advantage, Rocket has the potential to create more investment value relative to GenAuto and Starr. Over the long run, if GenAuto and Starr can lower their ATC, they will become more attractive to investors. But if any of the three US auto companies cannot match the cost competitiveness of the foreign firms, they may be driven from the market. In the long run, the lower-cost foreign automakers pose a severe competitive challenge to the survival of the US manufacturers and their ability to maintain and grow shareholders' wealth.

\section{SUMMARY}
This reading addressed several important concepts that extend the basic market model of demand and supply to assist the analyst in assessing a firm's breakeven and shutdown points of production. Demand concepts covered include own-price elasticity of demand, cross-price elasticity of demand, and income elasticity of demand. Supply concepts covered include total, average, and marginal product of labor; total, variable, and marginal cost of labor; and total and marginal revenue. These concepts are used to calculate the breakeven and shutdown points of production.

\begin{itemize}
  \item Elasticity of demand is a measure of how sensitive quantity demanded is to changes in various variables.

  \item Own-price elasticity of demand is the ratio of percentage change in quantity demanded to percentage change in a good or service's own price.

  \item If own-price elasticity of demand is greater than one in absolute terms, demand is elastic and a decline in price will result in higher total expenditure on that good.

  \item If own-price elasticity of demand is less than one in absolute terms, demand is inelastic and a decline in price will result in a lower total expenditure on that good.

  \item If own-price elasticity of demand is equal to negative one, demand is unit, or unitary, elastic and total expenditure on that good is independent of price.

  \item Own-price elasticity of demand will almost always be negative.

  \item Income elasticity of demand is the ratio of the percentage change in quantity demanded to the percentage change in consumer income.

  \item Demand is negatively sloped because of either the substitution effect or the income effect.

  \item The substitution effect is the phenomenon in which, as a good's price falls, more of this good is substituted for other, more expensive goods.

  \item The income effect is the phenomenon in which, as a good's price falls, real income rises and, if this good is normal, more of it will be purchased.

  \item If the good is inferior, the income effect will partially or fully offset the substitution effect.

  \item There are two exceptions to the law of demand: Giffen goods and Veblen goods.

  \item Giffen goods are highly inferior and make up a large portion of the consumer budget. As price falls, the substitution effect tends to cause more of the good to be consumed, but the highly negative income effect overwhelms the substitution effect. Demand curves for Giffen goods are positively sloped.

  \item Veblen goods are highly valued high-priced "status" goods; consumers may tend to buy more of a good if its price rises.

  \item If income elasticity of demand is positive, the good is a normal good. If income elasticity of demand is negative, the good is an inferior good.

  \item Cross-price elasticity of demand is the ratio of the percentage change in quantity demanded of one good to the percentage change in the price of a related good.

  \item If cross-price elasticity between two goods is positive, they are substitutes, and if cross-price elasticity between two goods is negative, they are complements.

  \item The law of demand states that a decrease in price will cause an increase in quantity demanded.

  \item Total product of labor is a short-run concept that is the total quantity that is able to be produced for each level of labor input, holding all other inputs constant. - Average product of labor (APL) is the total product of labor divided by number of labor hours.

  \item Marginal product of labor $\left(\mathrm{MP}_{L}\right)$ is the change in total product divided by the change in labor hours. $\mathrm{MP}_{L}$ might rise as more labor is added to a fixed amount of capital.

  \item The law of diminishing returns dictates that additional output must fall as more and more labor is added to a fixed amount of capital.

  \item Production costs increase as input prices rise and fall as inputs become more productive.

  \item Short-run total cost (STC) is the total expenditure on fixed capital plus the total expenditure on labor.

  \item Short-run marginal cost $(S M C)$ equals the ratio of wage to marginal product of labor $\left(\mathrm{MP}_{L}\right)$

  \item Average variable cost $(\mathrm{AVC})$ is the ratio of wage to average product of labor (APL).

  \item Average total cost (ATC) is total cost (TC) divided by the number of units produced.

  \item Revenue is price times quantity sold.

  \item Marginal revenue (MR) is the ratio of change in revenue to change in output.

  \item Firms under conditions of perfect competition have no pricing power and, therefore, face a perfectly horizontal demand curve at the market price. For firms under conditions of perfect competition, price is identical to marginal revenue (MR).

  \item Firms under conditions of imperfect competition face a negatively sloped demand curve and have pricing power. For firms under conditions of imperfect competition, marginal revenue (MR) is less than price.

  \item Economic profit equals total revenue (TR) minus total economic cost, whereas accounting profit equals $\mathrm{TR}$ minus total accounting cost.

  \item Economic cost takes into account the total opportunity cost of all factors of production.

  \item Opportunity cost is the next best alternative forgone in making a decision.

  \item Maximum economic profit requires that (1) marginal revenue (MR) equals marginal cost $(\mathrm{MC})$ and (2) MC not be falling with output.

  \item The breakeven point occurs when total revenue (TR) equals total cost (TC), otherwise stated as the output quantity at which average total cost (ATC) equals price.

  \item Shutdown occurs when a firm is better off not operating than continuing to operate.

  \item If all fixed costs are sunk costs, then shutdown occurs when the market price falls below minimum average variable cost. After shutdown, the firm incurs only fixed costs and loses less money than it would operating at a price that does not cover variable costs.

  \item In the short run, it may be rational for a firm to continue to operate while earning negative economic profit if some unavoidable fixed costs are covered.

  \item Economies of scale is defined as decreasing long-run cost per unit as output increases. Diseconomies of scale is defined as increasing long-run cost per unit as output increases. - Long-run average total cost is the cost of production per unit of output under conditions in which all inputs are variable.

  \item Specialization efficiencies and bargaining power in input price can lead to economies of scale.

  \item Bureaucratic and communication breakdowns and bottlenecks that raise input prices can lead to diseconomies of scale.

  \item The minimum point on the long-run average total cost curve defines the minimum efficient scale for the firm.

\end{itemize}

\section{PRACTICE PROBLEMS}
\begin{enumerate}
  \item If the price elasticity coefficient of the demand curve for paper clips is equal to -1 , demand is:
A. elastic.
B. inelastic.
C. unit elastic.

  \item The demand for membership at a local health club is determined by the following equation:

\end{enumerate}

$Q_{h m}^{d}=400-5 P_{h m}$

where $Q_{h m}^{d}$ is the number of health club members and $P_{h m}$ is the price of membership. If the price of health club membership is $\$ 35$, the price elasticity of demand is closest to:
A. -0.778 .
B. -0.500 .
C. -0.438

\section{The following information relates to questions}
3-6

The market demand function for four-year private universities is given by the equation

$Q_{p r}^{d}=84-3.1 P_{p r}+0.8 I+0.9 P_{p u}$

where $Q_{p r}^{d}$ is the number of applicants to private universities per year in thousands, $P_{p r}$ is the average price of private universities (in thousands of USD), $I$ is the household monthly income (in thousands of USD), and $P_{p u}$ is the average price of public (government-supported) universities (in thousands of USD). Assume that $P_{p r}$ is equal to $38, I$ is equal to 100 , and $P_{p u}$ is equal to 18 .

\begin{enumerate}
  \setcounter{enumi}{2}
  \item The price elasticity of demand for private universities is closest to:
A. -3.1 .
B. -1.9 .
C. 0.6 .

  \item The income elasticity of demand for private universities is closest to:
A. 0.5 .
B. 0.8 .
C. 1.3 . 5. The cross-price elasticity of demand for private universities with respect to the price of public universities is closest to:
A. 0.3 .
B. 3.1.
C. 3.9.

  \item If the cross-price elasticity between two goods is negative, the two goods are classified as:
A. normal.
B. substitutes.
C. complements.

  \item Price elasticity of demand for a good will most likely be greater if:

\end{enumerate}

A. there are no substitutes for the good.

B. consumers consider the good as discretionary.

C. consumers spend a small portion of their budget on the good.

\begin{enumerate}
  \setcounter{enumi}{7}
  \item If the income elasticity of demand for a product is -0.6 , a:
A. $1 \%$ increase in income will result in a $0.6 \%$ increase in demand.
B. $1 \%$ increase in income will result in a $0.6 \%$ decrease in demand.
C. $0.6 \%$ increase in income will result in a $1 \%$ decrease in demand.

  \item An individual's demand for onions is given by the following equation:

\end{enumerate}

$$
Q_{o}^{d}=3-0.05 P_{o}+0.009 I-0.16 P_{t}
$$

where $Q_{o}^{d}$ is the number of onions demanded, $P_{o}$ is the price per pound of onions, $I$ is the household income, and $P_{t}$ is the price per pound of tomatoes.

If the price of onions is $\$ 1.25$, household income is $\$ 2,500$, and the price of tomatoes is $\$ 3.75$, the cross-price elasticity of demand for onions with respect to the price of tomatoes is closest to:

A. -1.0597 .

B. -0.0242 .

C. -0.0081 .

\begin{enumerate}
  \setcounter{enumi}{9}
  \item Movement along the demand curve for good $X$ occurs due to a change in:
\end{enumerate}

A. income.

B. the price of $\operatorname{good} X$.

C. the price of a substitute for $\operatorname{good} X$.

\begin{enumerate}
  \setcounter{enumi}{10}
  \item A wireless phone manufacturer introduced a next-generation phone that received a high level of positive publicity. Despite running several high-speed production assembly lines, the manufacturer is still falling short in meeting demand for the phone nine months after introduction. Which of the following statements is the most plausible explanation for the demand/supply imbalance?
A. The phone price is low relative to the equilibrium price.
B. Competitors introduced next-generation phones at a similar price.
C. Consumer incomes grew faster than the manufacturer anticipated.

  \item In the case of a normal good with a decrease in own price, which of the following statements is most likely true?

\end{enumerate}

A. Both the substitution and income effects lead to an increase in the quantity purchased.

B. The substitution effect leads to an increase in the quantity purchased, while the income effect has no impact.

C. The substitution effect leads to an increase in the quantity purchased, while the income effect leads to a decrease.

\begin{enumerate}
  \setcounter{enumi}{12}
  \item For a Giffen good, the:
\end{enumerate}

A. demand curve is positively sloped.

B. substitution effect overwhelms the income effect.

C. income and substitution effects are in the same direction.

\begin{enumerate}
  \setcounter{enumi}{13}
  \item A company plans to hire additional factory employees. In the short run, marginal returns are most likely to decrease if:
\end{enumerate}

A. the factory is operating at full capacity.

B. the factory is experiencing a labor shortage.

C. workers are required to multitask and share duties.

\begin{enumerate}
  \setcounter{enumi}{14}
  \item The production relationship between the number of machine hours and total product for a company is presented below.
\end{enumerate}

\begin{center}
\begin{tabular}{lcc}
\hline
Machine Hours & Total Product & Average Product \\
\hline
1 & 3 & 3.00 \\
2 & 8 & 4.00 \\
3 & 14 & 4.67 \\
4 & 19 & 4.75 \\
5 & 21 & 4.20 \\
\hline
\end{tabular}
\end{center}

Diminishing marginal returns first occur beyond machine hour:
A. 3 .
B. 4 .
C. 5 .

\section{The following information relates to questions}
\section{6-17}
The manager of a small manufacturing firm gathers the following information about the firm's labor utilization and production:

\section{Exhibit 28: Exhibit 1}
\begin{center}
\begin{tabular}{lc}
\hline
Labor $(L)$ & Total Product (TP) \\
\hline
0 & 0 \\
1 & 150 \\
2 & 320 \\
3 & 510 \\
4 & 660 \\
5 & 800 \\
\hline
\end{tabular}
\end{center}

\begin{enumerate}
  \setcounter{enumi}{15}
  \item Refer to the data in Exhibit 2. The number of workers resulting in the highest level of average product of labor is closest to:
A. 3.
B. 4 .
C. 5 .

  \item Refer to the data in Exhibit 2. The marginal product of labor demonstrates increasing returns for the firm if the number of workers is closest to but not more than:
A. 2 .
B. 3.
C. 4 .

  \item Normal profit is best described as:
A. zero economic profit.
B. total revenue minus all explicit costs.
C. the sum of accounting profit plus economic profit.

  \item The marketing director for a Swiss specialty equipment manufacturer estimates the firm can sell 200 units and earn total revenue of CHF500,000. However, if 250 units are sold, revenue will total CHF600,000. The marginal revenue per unit associated with marketing 250 units instead of 200 units is closest to:
A. CHF 2,000.
B. CHF 2,400 .
C. CHF 2,500 . 20. An operator of a ski resort is considering offering price reductions on weekday ski passes. At the normal price of $€ 50$ per day, 300 customers are expected to buy passes each weekday. At a discounted price of $€ 40$ per day 450 customers are expected to buy passes each weekday. The marginal revenue per customer earned from offering the discounted price is closest to:
A. $€ 20$.
B. $€ 40$.
C. $€ 50$.

\end{enumerate}

\section{The following information relates to questions 21-23}
A firm's director of operations gathers the following information about the firm's cost structure at different levels of output:

Exhibit 29: Exhibit 1

\begin{center}
\begin{tabular}{lcc}
\hline
$\begin{array}{l}\text { Quantity } \\ \text { (Q) }\end{array}$ & $\begin{array}{c}\text { Total Fixed Cost } \\ \text { (TFC) }\end{array}$ & $\begin{array}{c}\text { Total Variable Cost } \\ \text { (TVC) }\end{array}$ \\
\hline
0 & 200 & 0 \\
1 & 200 & 100 \\
2 & 200 & 150 \\
3 & 200 & 200 \\
4 & 200 & 240 \\
5 & 200 & 320 \\
\hline
\end{tabular}
\end{center}

\begin{enumerate}
  \setcounter{enumi}{20}
  \item Refer to the data in Exhibit 1. When quantity produced is equal to 4 units, the average fixed cost $(A F C)$ is closest to:
A. 50 .
B. 60 .
C. 110 .

  \item Refer to the data in Exhibit 1. When the firm increases production from 4 to 5 units, the marginal cost $(M C)$ is closest to:
A. 40 .
B. 64
C. 80 .

  \item Refer to the data in Exhibit 1. The level of unit production resulting in the lowest average total cost $(A T C)$ is closest to:
A. 3 .
B. 4 .
C. 5 .

  \item An agricultural firm operating in a perfectly competitive market supplies wheat to manufacturers of consumer food products and animal feeds. If the firm were able to expand its production and unit sales by $10 \%$ the most likely result would be:

\end{enumerate}

A. a $10 \%$ increase in total revenue.

B. a $10 \%$ increase in average revenue.

C. an increase in total revenue of less than $10 \%$.

\begin{enumerate}
  \setcounter{enumi}{24}
  \item The marginal revenue per unit sold for a firm doing business under conditions of perfect competition will most likely be:
A. equal to average revenue.
B. less than average revenue.
C. greater than average revenue.

  \item A profit maximum is least likely to occur when:
A. average total cost is minimized.
B. marginal revenue equals marginal cost.
C. the difference between total revenue and total cost is maximized.

  \item The short-term breakeven point of production for a firm operating under perfect competition will most likely occur when:
A. price is equal to average total cost.
B. marginal revenue is equal to marginal cost.
C. marginal revenue is equal to average variable costs.

  \item The short-term shutdown point of production for a firm operating under perfect competition will most likely occur when:

\end{enumerate}

A. price is equal to average total cost.

B. marginal revenue is equal to marginal cost.

C. marginal revenue is equal to average variable costs.

\begin{enumerate}
  \setcounter{enumi}{28}
  \item Under conditions of perfect competition, a company will break even when market price is equal to the minimum point of the:
A. average total cost curve.
B. average variable cost curve.
C. short-run marginal cost curve.

  \item A company will shut down production in the short run if total revenue is less than total:
A. fixed costs.
B. variable costs.
C. opportunity costs.

  \item A company has total variable costs of $\$ 4$ million and fixed costs of $\$ 3$ million.

\end{enumerate}

Based on this information, the company will stay in the market in the long term if total revenue is at least:
A. $\$ 3.0$ million.
B. $\$ 4.5$ million.
C. $\$ 7.0$ million.

\begin{enumerate}
  \setcounter{enumi}{31}
  \item When total revenue is greater than total variable costs but less than total costs, in the short term a firm will most likely:
A. exit the market.
B. stay in the market.
C. shut down production.

  \item Under conditions of perfect competition, in the long run firms will most likely earn:
A. normal profits.
B. positive economic profits.
C. negative economic profits.

  \item A firm that increases its quantity produced without any change in per-unit cost is experiencing:
A. economies of scale.
B. diseconomies of scale.
C. constant returns to scale.

  \item A company is experiencing economies of scale when:
A. cost per unit increases as output increases.
B. it is operating at a point on the LRAC curve where the slope is negative.
C. It is operating beyond the minimum point on the long-run average total cost curve.

  \item Diseconomies of scale most likely result from:
A. specialization in the labor force.
B. overlap of business functions and product lines.
C. discounted prices on resources when buying in larger quantities. 37. A firm is operating beyond minimum efficient scale in a perfectly competitive industry. To maintain long-term viability the most likely course of action for the firm is to:
A. operate at the current level of production.
B. increase its level of production to gain economies of scale.
C. decrease its level of production to the minimum point on the long-run aver- age total cost curve.

\end{enumerate}

\section{SOLUTIONS}
\begin{enumerate}
  \item C is correct. When the price elasticity of demand coefficient is -1 , demand is said to be unit elastic, or unitary elastic.

  \item A is correct. Inserting the price of $\$ 35$ into the demand function, quantity demanded is calculated as

\end{enumerate}

$Q_{h m}^{d}=400-5(35)=225$

At a price of $\$ 35$ per health club membership, the elasticity of demand is

Price elasticity of demand $=\left(\Delta Q_{h m}^{d} / \Delta P_{h m}\right) \times\left(P_{h m} / Q_{h m}^{d}\right)$

Price elasticity of demand $=-5 \times(35 / 225)=-0.778$

\begin{enumerate}
  \setcounter{enumi}{2}
  \item B is correct. From the demand function:
\end{enumerate}

Solve for $Q_{p r}^{d}$ :

$\Delta Q_{p r}^{d} / \Delta P_{p r}=-3.1$ (the coefficient in front of own price)

$Q_{p r}^{d}=84-3.1 P_{p r}+0.8 I+0.9 P_{p u}$

$=84-3.1(38)+0.8(100)+0.9(18)$

$=62.4$

At $P_{p r}=38$

price elasticity of demand $=\left(\Delta Q_{p r}^{d} / \Delta P_{p r}\right)\left(P_{p r} / Q_{p r}^{d}\right)$

$=(-3.1)(38 / 62.4)$

$=-1.9$

\begin{enumerate}
  \setcounter{enumi}{3}
  \item $C$ is correct. From the demand function:
\end{enumerate}

Solve for $Q_{p r}^{d}$ :

$\Delta Q_{p r}^{d} / \Delta I=0.8$ (coefficient in front of the income variable)

$Q_{p r}^{d}=84-3.1 P_{p r}+0.8 I+0.9 P_{p u}$

$=84-3.1(38)+0.8(100)+0.9(18)$

$=62.4$

At $I=100$,

the income elasticity of demand $=\left(\Delta Q_{p r}^{d} / \Delta I\right)\left(I / Q_{p r}^{d}\right)$

$=(0.8)(100 / 62.4)$

$=1.3$

\begin{enumerate}
  \setcounter{enumi}{4}
  \item A is correct. From the demand function:
\end{enumerate}

Solve for $Q_{p r}^{d}$ :

$\Delta Q_{p r}^{d} / \Delta P_{p u}=0.9$ (the coefficient in front of $P_{p u}$ )

$Q_{p r}^{d}=84-3.1 P_{p r}+0.8 I+0.9 P_{p u}$

$$
\begin{aligned}
& =84-3.1(38)+0.8(100)+0.9(18) \\
& =62.4
\end{aligned}
$$

At $P=38$, and $P_{p u}=18$,

the cross-price elasticity of demand $=\left(\Delta Q_{p r}^{d} / \Delta P_{p u}\right)\left(P_{p u} / Q_{p r}^{d}\right)$

$$
\begin{aligned}
& =(0.9)(18 / 62.4) \\
& =0.3
\end{aligned}
$$

\begin{enumerate}
  \setcounter{enumi}{5}
  \item C is correct. With complements, consumption goes up or down together. With a negative cross-price elasticity, as the price of one good goes up, the demand for both falls.

  \item B is correct. Price elasticity of demand is likely to be greater for items that are seen as optional or discretionary.

  \item B is correct. Income elasticity is a measure of how sensitive quantity demanded is to a change in income. If the income elasticity of demand for the product is -0.6 , whenever income increases by $1 \%$, the quantity demanded of the product at each price decreases by $0.6 \%$. Consequently, as income rises, consumers will purchase less of the product.

  \item B is correct. The cross-price elasticity of demand measures the responsiveness of the demand for onions in response to a change in the price of tomatoes. From the demand function equation:

\end{enumerate}

$Q_{o}^{d}=3-0.05 P_{o}+0.009 I-0.16 P_{t}$

$Q_{o}^{d}=3-0.05(1.25)+0.009(2,500)-0.16(3.75)=24.8375$

At a price of onions of $\$ 1.25$ and a price of tomatoes of $\$ 3.75$, the cross-price elasticity of demand is calculated as follows:

Cross-price elasticity of demand $=\left(\Delta Q_{o}^{d} / \Delta P_{t}\right) \times\left(P_{t} / Q_{o}^{d}\right)$

Cross-price elasticity of demand $=-0.16 \times(3.75 / 24.8375)=-0.0242$

\begin{enumerate}
  \setcounter{enumi}{9}
  \item B is correct. The demand curve shows quantity demanded as a function of own price only.

  \item A is correct. The situation described is one of excess demand because, in order for markets to clear at the given level of quantity supplied, the company would need to raise prices.

  \item A is correct. In the case of normal goods, the income and substitution effects are reinforcing, leading to an increase in the amount purchased after a drop in price.

  \item A is correct. The income effect overwhelms the substitution effect such that an increase in the price of the good results in greater demand for the good, resulting in a positively sloped demand curve.

  \item A is correct. The law of diminishing returns occurs in the short run when additional output falls as more and more labor is added to a fixed amount of capital. When a factory is operating at full capacity, adding additional employees will not increase production because the physical plant is already $100 \%$ employed. More labor hours will add to costs without adding to output, thus resulting in diminishing marginal returns. 15. A is correct. Diminishing marginal returns occur when the marginal product of a resource decreases as additional units of that input are employed. Marginal product, which is the additional output resulting from using one more unit of input, is presented below.

\end{enumerate}

\begin{center}
\begin{tabular}{lccc}
\hline
Machine Hours & Total Product & Average Product & Marginal Product \\
\hline
1 & 3 & 3.00 & 3 \\
2 & 8 & 4.00 & 5 \\
3 & 14 & 4.67 & 6 \\
4 & 19 & 4.75 & 5 \\
5 & 21 & 4.20 & 2 \\
\hline
\end{tabular}
\end{center}

The marginal product of the third machine hour is 6 and declines thereafter. Consequently, diminishing marginal returns are first evident beyond three machine hours.

\begin{enumerate}
  \setcounter{enumi}{15}
  \item A is correct. Three workers produce the highest average product equal to 170 . AP $=510 / 3=170$.

  \item B is correct. Marginal product is equal to the change in total product divided by the change in labor. The increase in MP from 2 to 3 workers is 190: MP = $\Delta \mathrm{TP} / \Delta L=(510-320) /(3-2)=190 / 1=190$.

  \item A is correct. Normal profit is the level of accounting profit such that implicit opportunity costs are just covered; thus, it is equal to a level of accounting profit such that economic profit is zero.

  \item A is correct. Marginal revenue per unit is defined as the change in total revenue divided by the change in quantity sold. $M R=\Delta \mathrm{TR} \div \Delta Q$. In this case, change in total revenue equals CHF100,000, and change in total units sold equals 50 . CHF100,000 $\div 50=$ CHF2,000.

  \item A is correct. Marginal revenue per unit is defined as the change in total revenues divided by the change in quantity sold. $M R=\Delta \mathrm{TR} \div \Delta Q$. In this case, change in total revenue per day equals $€ 3,000[(450 \times € 40)-(300 \times € 50)]$, and change in units sold equals $150(450-300)$. $€ 3,000 \div 150=€ 20$.

  \item A is correct. Average fixed cost is equal to total fixed cost divided by quantity produced: $\mathrm{AFC}=\mathrm{TFC} / \mathrm{Q}=200 / 4=50$.

  \item $\mathrm{C}$ is correct. Marginal cost is equal to the change in total cost divided by the change in quantity produced. $\mathrm{MC}=\Delta \mathrm{TC} / \Delta Q=80 / 1=80$.

  \item $\mathrm{C}$ is correct. Average total cost is equal to total cost divided by quantity produced. At 5 units produced the average total cost is 104 . ATC $=\mathrm{TC} / \mathrm{Q}=520 / 5=$ 104.

  \item A is correct. In a perfectly competitive market, an increase in supply by a single firm will not affect price. Therefore, an increase in units sold by the firm will be matched proportionately by an increase in revenue.

  \item A is correct. Under perfect competition, a firm is a price taker at any quantity supplied to the market, and $A R=M R=$ Price.

  \item A is correct. The quantity at which average total cost is minimized does not necessarily correspond to a profit maximum. 27. A is correct. Under perfect competition, price equals marginal revenue. A firm breaks even when marginal revenue equals average total cost.

  \item C is correct. The firm should shut down production when marginal revenue is less than or equal to average variable cost.

  \item A is correct. A company is said to break even if its total revenue is equal to its total cost. Under conditions of perfect competition, a company will break even when market price is equal to the minimum point of the average total cost curve.

  \item B is correct. A company will shut down production in the short run when total revenue is below total variable costs.

  \item $\mathrm{C}$ is correct. A company will stay in the market in the long term if total revenue is equal to or greater than total cost. Because total costs are $\$ 7$ million ( $\$ 4$ million variable costs and $\$ 3$ million fixed costs), the company will stay in the market in the long term if total revenue equals at least $\$ 7$ million.

  \item B is correct. When total revenue is enough to cover variable costs but not total fixed costs in full, the firm can survive in the short run but would be unable to maintain financial solvency in the long run.

  \item A is correct. Competition should drive prices down to long-run marginal cost, resulting in only normal profits being earned.

  \item C is correct. Output increases in the same proportion as input increases occur at constant returns to scale.

  \item B is correct. Economies of scale occur if, as the firm increases output, cost per unit of production falls. Graphically, this definition translates into a long-run average cost curve (LRAC) with a negative slope.

  \item B is correct. As the firm increases output, diseconomies of scale and higher average total costs can result when there is overlap and duplication of business functions and product lines.

  \item $\mathrm{C}$ is correct. The firm operating at greater than long-run efficient scale is subject to diseconomies of scale. It should plan to decrease its level of production.

\end{enumerate}

\section*{LEARNING MODULE 2 }
\section{The Firm and Market Structures}
Richard Fritz, PhD, is at the School of Economics at Georgia Institute of Technology (USA). Michele Gambera, PhD, CFA, is at UBS Asset Management and the University of Illinois at Urbana-Champaign (USA).

\section{LEARNING OUTCOME}
\begin{center}
\begin{tabular}{c|l}
Mastery & The candidate should be able to: \\
$\square \square$ & $\begin{array}{l}\text { describe characteristics of perfect competition, monopolistic } \\ \text { competition, oligopoly, and pure monopoly } \\ \text { explain relationships between price, marginal revenue, marginal cost, } \\ \text { economic profit, and the elasticity of demand under each market } \\ \text { structure } \\ \text { describe a firm's supply function under each market structure } \\ \text { describe and determine the optimal price and output for firms under } \\ \text { each market structure } \\ \text { describe pricing strategy under each market structure } \\ \text { explain factors affecting long-run equilibrium under each market } \\ \text { structure } \\ \text { describe the use and limitations of concentration measures in } \\ \text { identifying market structure } \\ \text { identify the type of market structure within which a firm operates }\end{array}$ \\
$\square$ & $\square$ \\
\end{tabular}
\end{center}

\section{INTRODUCTION AND ANALYSIS OF MARKET STRUCTURES}
$$
\begin{aligned}
& \text { describe characteristics of perfect competition, monopolistic } \\
& \text { competition, oligopoly, and pure monopoly }
\end{aligned}
$$

The purpose of this reading is to build an understanding of the importance of market structure. As different market structures result in different sets of choices facing a firm's decision makers, an understanding of market structure is a powerful tool in analyzing issues such as a firm's pricing of its products and, more broadly, its potential to increase profitability. In the long run, a firm's profitability will be determined by the forces associated with the market structure within which it operates. In a highly competitive market, long-run profits will be driven down by the forces of competition. In less competitive markets, large profits are possible even in the long run; in the short run, any outcome is possible. Therefore, understanding the forces behind the market structure will aid the financial analyst in determining firms' short- and long-term prospects.

Section 2 introduces the analysis of market structures. The section addresses questions such as: What determines the degree of competition associated with each market structure? Given the degree of competition associated with each market structure, what decisions are left to the management team developing corporate strategy? How does a chosen pricing and output strategy evolve into specific decisions that affect the profitability of the firm? The answers to these questions are related to the forces of the market structure within which the firm operates.

Sections $3,4,5$, and 6 analyze demand, supply, optimal price and output, and factors affecting long-run equilibrium for perfect competition, monopolistic competition, oligopoly, and pure monopoly, respectively.

Section 7 reviews techniques for identifying the various forms of market structure. For example, there are accepted measures of market concentration that are used by regulators of financial institutions to judge whether or not a planned merger or acquisition will harm the competitive nature of regional banking markets. Financial analysts should be able to identify the type of market structure a firm is operating within. Each different structure implies a different long-run sustainability of profits. A summary and practice problems conclude the reading.

\section{Analysis of Market Structures}
Traditionally, economists classify a market into one of four structures: perfect competition, monopolistic competition, oligopoly, and monopoly. Section 2.1 explains that four-way classification in more detail. Section 2.2 completes the introduction by providing and explaining the major points to evaluate in determining the structure to which a market belongs.

\section{Economists' Four Types of Structure}
Economists define a market as a group of buyers and sellers that are aware of each other and can agree on a price for the exchange of goods and services. While the internet has extended a number of markets worldwide, certain markets are limited by geographic boundaries. For example, the internet search engine Google operates in a worldwide market. In contrast, the market for premixed cement is limited to the area within which a truck can deliver the mushy mix from the plant to a construction site before the compound becomes useless. Thomas L. Friedman's international best seller The World Is Flat ${ }^{1}$ challenges the concept of the geographic limitations of the market. If the service being provided by the seller can be digitized, its market expands worldwide. For example, a technician can scan your injury in a clinic in Switzerland. That radiographic image can be digitized and sent to a radiologist in India to be read. As a customer (i.e., patient), you may never know that part of the medical service provided to you was the result of a worldwide market.

Some markets are highly concentrated, with the majority of total sales coming from a small number of firms. For example, in the market for internet search, three firms controlled 98.9 percent of the US market (Google 63.5 percent, Microsoft 24

1 Friedman (2006). percent, and Oath (formerly Yahoo) 11.4 percent) as of January $2018 .{ }^{2}$ Other markets are very fragmented, such as automobile repairs, where small independent shops often dominate and large chains may or may not exist. New products can lead to market concentration: It is estimated that the Apple iPod had a world market share of over 70 percent among MP3 players in 2009.

\section{THE IMPORTANCE OF MARKET STRUCTURE}
Consider the evolution of television broadcasting. As the market environment for television broadcasting evolved, the market structure changed, resulting in a new set of challenges and choices. In the early days, there was only one choice: the "free" analog channels that were broadcast over the airwaves. In most countries, there was only one channel, owned and run by the government. In the United States, some of the more populated markets were able to receive more channels because local channels were set up to cover a market with more potential viewers. By the 1970s, new technologies made it possible to broadcast by way of cable connectivity and the choices offered to consumers began to expand rapidly. Cable television challenged the "free" broadcast channels by offering more choice and a better-quality picture. The innovation was expensive for consumers and profitable for the cable companies. By the 1990s, a new alternative began to challenge the existing broadcast and cable systems: satellite television. Satellite providers offered a further expanded set of choices, albeit at a higher price than the free broadcast and cable alternatives. In the early 2000 s, satellite television providers lowered their pricing to compete directly with the cable providers.

Today, cable program providers, satellite television providers, and terrestrial digital broadcasters that offer premium and pay-per-view channels compete for customers who are increasingly finding content on the internet and on their mobile devices. Companies like Netflix, Apple, and Amazon offered alternative ways for consumers to access content. By 2018 , these companies had moved beyond the repackaging of existing shows to developing their own content, mirroring the evolution of cable channels such as $\mathrm{HBO}$ and ESPN a decade earlier.

This is a simple illustration of the importance of market structure. As the market for television broadcasting became increasingly competitive, managers have had to make decisions regarding product packaging, pricing, advertising, and marketing in order to survive in the changing environment. In addition, mergers and acquisitions as a response to these competitive pressures have changed the essential structure of the industry.

Market structure can be broken down into four distinct categories: perfect competition, monopolistic competition, oligopoly, and monopoly.

We start with the most competitive environment, perfect competition. Unlike some economic concepts, perfect competition is not merely an ideal based on assumptions. Perfect competition is a reality-for example, in several commodities markets, where sellers and buyers have a strictly homogeneous product and no single producer is large enough to influence market prices. Perfect competition's characteristics are well recognized and its long-run outcome unavoidable. Profits under the conditions of perfect competition are driven to the required rate of return paid by the entrepreneur to borrow capital from investors (so-called normal profit or rental cost of capital). This does not mean that all perfectly competitive industries are doomed to extinction by a lack of profits. On the contrary, millions of businesses that do very well are living under the pressures of perfect competition.

Monopolistic competition is also highly competitive; however, it is considered a form of imperfect competition. Two economists, Edward H. Chamberlin (US) and Joan Robinson (UK), identified this hybrid market and came up with the term because there are not only strong elements of competition in this market structure but also some monopoly-like conditions. The competitive characteristic is a notably large

2 Source: \href{http://www.statista.com/statistics/267161/market-share-of-search-engines-in-the-united-states/}{www.statista.com/statistics/267161/market-share-of-search-engines-in-the-united-states/}. number of firms, while the monopoly aspect is the result of product differentiation. That is, if the seller can convince consumers that its product is uniquely different from other, similar products, then the seller can exercise some degree of pricing power over the market. A good example is the brand loyalty associated with soft drinks such as Coca-Cola. Many of Coca-Cola's customers believe that their beverages are truly different from and better than all other soft drinks. The same is true for fashion creations and cosmetics.

The oligopoly market structure is based on a relatively small number of firms supplying the market. The small number of firms in the market means that each firm must consider what retaliatory strategies the other firms will pursue when prices and production levels change. Consider the pricing behavior of commercial airline companies. Pricing strategies and route scheduling are based on the expected reaction of the other carriers in similar markets. For any given route-say, from Paris, France, to Chennai, India-only a few carriers are in competition. If one of the carriers changes its pricing package, others will likely retaliate. Understanding the market structure of oligopoly markets can help in identifying a logical pattern of strategic price changes for the competing firms.

Finally, the least competitive market structure is monopoly. In pure monopoly markets, there are no other good substitutes for the given product or service. There is a single seller, which, if allowed to operate without constraint, exercises considerable power over pricing and output decisions. In most market-based economies around the globe, pure monopolies are regulated by a governmental authority. The most common example of a regulated monopoly is the local electrical power provider. In most cases, the monopoly power provider is allowed to earn a normal return on its investment and prices are set by the regulatory authority to allow that return.

\section{Factors That Determine Market Structure}
Five factors determine market structure:

\begin{enumerate}
  \item The number and relative size of firms supplying the product;

  \item The degree of product differentiation;

  \item The power of the seller over pricing decisions;

  \item The relative strength of the barriers to market entry and exit; and

  \item The degree of non-price competition.

\end{enumerate}

The number and relative size of firms in a market influence market structure. If there are many firms, the degree of competition increases. With fewer firms supplying a good or service, consumers are limited in their market choices. One extreme case is the monopoly market structure, with only one firm supplying a unique good or service. Another extreme is perfect competition, with many firms supplying a similar product. Finally, an example of relative size is the automobile industry, in which a small number of large international producers (e.g., Volkswagen and Toyota) are the leaders in the global market, and a number of small companies either have market power because they are niche players (e.g., Ferrari or McLaren) or have little market power because of their narrow range of models or limited geographical presence (e.g., Mazda or Fiat-Chrysler).

In the case of monopolistic competition, there are many firms providing products to the market, as with perfect competition. However, one firm's product is differentiated in some way that makes it appear better than similar products from other firms. If a firm is successful in differentiating its product, the differentiation will provide pricing leverage. The more dissimilar the product appears, the more the market will resemble the monopoly market structure. A firm can differentiate its product through aggressive advertising campaigns; frequent styling changes; the linking of its product with other, complementary products; or a host of other methods. When the market dictates the price based on aggregate supply and demand conditions, the individual firm has no control over pricing. The typical hog farmer in Nebraska and the milk producer in Bavaria are price takers. That is, they must accept whatever price the market dictates. This is the case under the market structure of perfect competition. In the case of monopolistic competition, the success of product differentiation determines the degree with which the firm can influence price. In the case of oligopoly, there are so few firms in the market that price control becomes possible. However, the small number of firms in an oligopoly market invites complex pricing strategies. Collusion, price leadership by dominant firms, and other pricing strategies can result.

The degree to which one market structure can evolve into another and the difference between potential short-run outcomes and long-run equilibrium conditions depend on the strength of the barriers to entry and the possibility that firms fail to recoup their original costs or lose money for an extended period of time and are therefore forced to exit the market. Barriers to entry can result from very large capital investment requirements, as in the case of petroleum refining. Barriers may also result from patents, as in the case of some electronic products and drug formulas. Another entry consideration is the possibility of high exit costs. For example, plants that are specific to a special line of products, such as aluminum smelting plants, are non-redeployable, and exit costs would be high without a liquid market for the firm's assets. High exit costs deter entry and are therefore also considered barriers to entry. In the case of farming, the barriers to entry are low. Production of corn, soybeans, wheat, tomatoes, and other produce is an easy process to replicate; therefore, those are highly competitive markets.

Non-price competition dominates those market structures where product differentiation is critical. Therefore, monopolistic competition relies on competitive strategies that may not include pricing changes. An example of non-price competition is product differentiation through marketing. In other circumstances, non-price competition may occur because the few firms in the market feel dependent on each other. Each firm fears retaliatory price changes that would reduce total revenue for all of the firms in the market. Because oligopoly industries have so few firms, each firm feels dependent on the pricing strategies of the others. Therefore, non-price competition becomes a dominant strategy.

\section{Exhibit 1: Characteristics of Market Structure}
\begin{center}
\begin{tabular}{llllll}
\hline
Market Structure & $\begin{array}{l}\text { Number of } \\ \text { Sellers }\end{array}$ & $\begin{array}{l}\text { Degree of Product } \\ \text { Differentiation }\end{array}$ & $\begin{array}{l}\text { Barriers to } \\ \text { Entry }\end{array}$ & $\begin{array}{l}\text { Pricing Power } \\ \text { of Firm }\end{array}$ & $\begin{array}{l}\text { Non-price } \\ \text { Competition }\end{array}$ \\
\hline
Perfect competition & Many & $\begin{array}{l}\text { Homogeneous/ } \\ \text { Standardized }\end{array}$ & Very Low & None & None \\
Monopolistic competition & Many & Differentiated & Low & Some & $\begin{array}{l}\text { Advertising } \\ \text { and Product } \\ \text { Differentiation } \\ \text { Advertising } \\ \text { and Product } \\ \text { Differentiation }\end{array}$ \\
Oligopoly & Few & $\begin{array}{l}\text { Homogeneous/ } \\ \text { Standardized }\end{array}$ & High & $\begin{array}{l}\text { Some or } \\ \text { Considerable }\end{array}$ & Advertising \\
\hline
Monopoly & One & Unique Product & Very High & Considerable & Ad \\
\hline
\end{tabular}
\end{center}

From the perspective of the owners of the firm, the most desirable market structure is that with the most control over price, because this control can lead to large profits. Monopoly and oligopoly markets offer the greatest potential control over price; monopolistic competition offers less control. Firms operating under perfectly competitive market conditions have no control over price. From the consumers' perspective, the most desirable market structure is that with the greatest degree of competition because prices are generally lower. Thus, consumers would prefer as many goods and services as possible to be offered in competitive markets.

As often happens in economics, there is a trade-off. While perfect competition gives the largest quantity of a good at the lowest price, other market forms may spur more innovation. Specifically, there may be high costs in researching a new product, and firms will incur such costs only if they expect to earn an attractive return on their research investment. This is the case often made for medical innovations, for example-the cost of clinical trials and experiments to create new medicines would bankrupt perfectly competitive firms but may be acceptable in an oligopoly market structure. Therefore, consumers can benefit from less-than-perfectly-competitive markets.

\section{PORTER'S FIVE FORCES AND MARKET STRUCTURE}
A financial analyst aiming to establish market conditions and consequent profitability of incumbent firms should start with the questions framed by Exhibit 1: How many sellers are there? Is the product differentiated? and so on. Moreover, in the case of monopolies and quasi monopolies, the analyst should evaluate the legislative and regulatory framework: Can the company set prices freely, or are there governmental controls? Finally, the analyst should consider the threat of competition from potential entrants.

This analysis is often summarized by students of corporate strategy as "Porter's five forces," named after Harvard Business School professor Michael E. Porter. His book, Competitive Strategy, presented a systematic analysis of the practice of market strategy. Porter (2008) identified the five forces as:

\begin{itemize}
  \item Threat of entry;

  \item Power of suppliers;

  \item Power of buyers (customers);

  \item Threat of substitutes; and

  \item Rivalry among existing competitors.

\end{itemize}

It is easy to note the parallels between four of these five forces and the columns in Exhibit 1. The only "orphan" is the power of suppliers, which is not at the core of the theoretical economic analysis of competition, but which has substantial weight in the practical analysis of competition and profitability.

Some stock analysts (e.g., Dorsey 2004) use the term "economic moat" to suggest that there are factors protecting the profitability of a firm that are similar to the moats (ditches full of water) that were used to protect some medieval castles. A deep moat means that there is little or no threat of entry by invaders, i.e. competitors. It also means that customers are locked in because of high switching costs.

\section*{PERFECT COMPETITION }
Perfect competition is characterized by the five conditions presented in Exhibit 1, above:

\begin{enumerate}
  \item There are a large number of potential buyers and sellers.

  \item The products offered by the sellers are virtually identical.

  \item There are few or easily surmountable barriers to entry and exit.

  \item Sellers have no market-pricing power.

  \item Non-price competition is absent.

\end{enumerate}

While few markets achieve the distinction of being perfectly competitive, it is useful to establish the outcome associated with this market structure as a benchmark against which other market structures can be compared. The most typical example of perfect competition is found in certain aspects of the agriculture industry, such as the large number of farmers growing corn for animal feed. Corn is a primary source of food for pork, beef, and poultry production. A bushel of corn from Farmer Brown is virtually identical to a bushel of corn from Farmer Lopez. If a hog farmer needs corn to feed his hogs, it does not matter whether the corn comes from Farmer Brown or Farmer Lopez. Furthermore, the aggregate corn market is well defined, with active futures and spot markets. Information about the corn market is easy and inexpensive to access, and there is no way to differentiate the product, such as by advertising. Agribusiness is capital intensive, but where arable land is relatively abundant and water is available, the barriers to entry (e.g., capital and expertise) for corn production are relatively low.

\section{Demand Analysis in Perfectly Competitive Markets}
The price of a homogeneous product sold in a competitive market is determined by the demand and supply in that market. Economists usually represent demand and supply in a market through demand and supply curves in a two-axis plane, where quantity and price are shown on the $x$-axis and $y$-axis, respectively. Economists believe that demand functions have negative slopes, as shown in Exhibit 2. That is, at high prices, less is demanded. For normal goods and services, as the price declines, the quantity demanded increases. This concept is based on two effects: the income effect and the substitution effect. The income effect results from the increased purchasing power the consumer has when prices fall. With lower prices, the consumer can afford to purchase more of the product. The substitution effect comes from the increasing attractiveness of the lower-priced product. If soybean prices are unchanged and corn prices decrease, hog farmers will substitute corn for soybeans as feed for their animals.

\section{Exhibit 2: Market Demand in Perfect Competition}
\begin{center}
\includegraphics[max width=\textwidth]{2023_05_04_ccb578c5089fd41615beg-080}
\end{center}

Assume the demand for this product can be specified as

$$
Q_{D}=50-2 P
$$

where $Q_{D}$ is the quantity of demand and $P$ is the product's price. This demand function can be rearranged in terms of price:

$$
P=25-0.5 Q_{D}
$$

In this form, total revenue (TR) is equal to price times quantity, or $P \times Q_{D}$. Thus,

$$
\mathrm{TR}=P Q_{D}=25 Q_{D}-0.5 Q_{D}^{2}
$$

Average revenue (AR) can be found by dividing TR by $Q_{D}$. Therefore,

$$
\mathrm{AR}=\mathrm{TR} / Q_{D}=\left(25 Q_{D}-0.5 Q_{D}^{2}\right) / Q_{D}=25-0.5 Q_{D}
$$

Note that the AR function is identical to the market demand function. The assumption here is that the relationship between price and quantity demanded is linear. Clearly, that may not be the case in the real market. Another simplifying assumption made is that the price of the product is the only determinant of demand. Again, that is not likely in the real market. For example, economic theory suggests that consumer income is another important factor in determining demand. The prices of related goods and services, such as substitutes and complements, are also considered factors affecting demand for a specific product.

Marginal revenue (MR) is the change in total revenue per extra increment sold when the quantity sold changes by a small increment, $\Delta Q_{D}$. Substituting $\left(Q_{D}+\Delta Q_{D}\right)$ into the total revenue (TR) equation, marginal revenue can be expressed as:

$$
\begin{aligned}
& \mathrm{MR}=\frac{\Delta \mathrm{TR}}{\Delta Q_{D}}=\frac{\left[25\left(Q_{D}+\Delta Q_{D}\right)-0.5\left(Q_{D}^{2}+2 Q_{D} \Delta Q_{D}+\Delta Q_{D}^{2}\right)\right]-\left[25 Q_{D}-0.5 Q_{D}^{2}\right]}{\Delta Q_{D}} \\
& =\frac{25 \Delta Q_{D}-Q_{D} \Delta Q_{D}-0.5 \Delta Q_{D}^{2}}{\Delta Q_{D}}=25-Q_{D}-0.5 \Delta Q_{D}
\end{aligned}
$$

For example, suppose $Q_{D}=5$ and $\Delta Q_{D}=1$, then total revenue increases from $112.50\left[=25(5)-0.5\left(5^{2}\right)\right]$ to $132\left[=25(6)-0.5\left(6^{2}\right)\right]$, and marginal revenue is $19.5=$ $(132-112.5) / 1$. Note that marginal revenue is equal to $\left(25-Q_{D}-0.5 \Delta Q_{D}\right)$. Now suppose that $\Delta Q_{D}$ is much smaller, for example $\Delta Q_{D}=0.1$. In this case, total revenue increases to $114.495\left[=25(5.1)-0.5\left(5.1^{2}\right)\right]$, and marginal revenue is $1.995 / 0.1$ = 19.95. It is straightforward to confirm that as $\Delta Q_{D}$ gets smaller marginal revenue gets closer to $20=25-Q_{D}$. So, for very small changes in the quantity sold we can write marginal revenue as ${ }^{3}$

$$
\mathrm{MR}=25-Q_{D}
$$

Although we have introduced the concept of marginal revenue in the context of the demand curve for the market as a whole, its usefulness derives from its role in the output and pricing decisions of individual firms. As we will see, marginal revenue and an analogous concept, marginal cost, are critical in determining firms' profit-maximizing strategies.

\section{ELASTICITY OF DEMAND}
\begin{center}
\includegraphics[max width=\textwidth]{2023_05_04_ccb578c5089fd41615beg-081}
\end{center}

Consumers respond differently to changes in the price of different kinds of products and services. The quantity demanded for some products is very price sensitive, while for other products, price changes result in little change in the quantity demanded. Economists refer to the relationship between changes in price and changes in the quantity demanded as the price elasticity of demand. Therefore, the demand for the former group of products - those that are very price sensitive-is said to have high price elasticity, whereas the demand for the latter group is said to have low price elasticity. Understanding the sensitivity of demand changes to changes in price is critical to understanding market structures.

Price elasticity of demand measures the percentage change in the quantity demanded given a percentage change in the price of a given product. Because the relationship of demand to price is negative, the price elasticity of demand would be negative. Many economists, however, present the price elasticity as an absolute value, so that price elasticity has a positive sign. We will follow that convention. Higher price elasticity indicates that consumers are very responsive to changes in price. Lower values for price elasticity imply that consumers are not very responsive to price changes. Price elasticity can be measured with the following relationship:

$\varepsilon_{P}=-\left(\%\right.$ change in $\left.Q_{D}\right) \div(\%$ change in $P)$

where $\varepsilon_{P}$ is price elasticity of demand, $Q_{D}$ is the quantity demanded, and $P$ is the product's price.

Price elasticity of demand falls into three categories. When demand is very responsive to price change, it is identified as elastic. When demand is not responsive to price change, it is identified as inelastic. When the percentage change in quantity demanded is exactly the same as the percentage change in price, the demand is called unitary elastic.

$\varepsilon_{P}>1$ Demand is elastic

3 Readers who are familiar with calculus will recognize this as the derivative of total revenue with respect to the quantity sold. $\varepsilon_{P}=1$ Demand is unitary elastic

$\varepsilon_{P}<1$ Demand is inelastic

Price elasticity of demand depends on several factors. Price elasticity will be higher if there are many close substitutes for the product. If a product has many good alternatives, consumers will be more sensitive to price changes. For example, carbonated beverages ("soft drinks") have many close substitutes. It takes strong brand loyalty to keep customer demand high in the soft drink market when one brand's price is strategically lowered; the price elasticity of demand for Coca-Cola has been estimated to be 3.8. For products with numerous close substitutes, demand is highly elastic. For products with few close substitutes, demand is lower in price elasticity and would be considered price inelastic. The demand for first-class airline tickets is often seen as inelastic because only very wealthy people are expected to buy them; the demand for economy-class tickets is elastic because the typical consumer for this product is more budget-conscious. Consumers do not consider economy-class airline tickets a close substitute for first-class accommodations, particularly on long flights.

The airline ticket example introduces another determinant of price elasticity of demand. The greater the share of the consumer's budget spent on the item, the higher the price elasticity of demand. Expensive items, such as durable goods (e.g., refrigerators and televisions), tend to have higher elasticity measures, while less expensive items, such as potatoes and salt, have lower elasticity values. Consumers will not change their normal salt consumption if the price of salt decreases by 10 percent. Instead, they will buy their next package of salt when they run out, with very little regard to the price change.

The airline ticket also makes a good example for the final factor determining price elasticity. Price elasticity of demand also depends on the length of time within which the demand schedule is being considered. Holiday airline travel is highly price elastic. Consumers shop vigorously for vacation flights because they have time to plan their holiday. Business airline travelers typically have less flexibility in determining their schedules. If your business requires a face-to-face meeting with a client, then the price of the ticket is somewhat irrelevant. If gasoline prices increase, there is very little you can do in the short run but pay the higher price. However, evidence of commuter choices indicates that many use alternative transportation methods after the gasoline price spikes. In the long run, higher gasoline prices will lead consumers to change their modes of transportation, trading in less efficient vehicles for automobiles with higher gas mileage or public transit options where available.

There are two extreme cases of price elasticity of demand. One extreme is the horizontal demand schedule. This term implies that at a given price, the response in the quantity demanded is infinite. This is the demand schedule faced by a perfectly competitive firm because it is a price taker, as in the case of a corn farmer. If the corn farmer tried to charge a higher price than the market price, nobody would buy her product. On the other hand, the farmer has no incentive to sell at a lower price because she can sell all she can produce at the market price. In a perfectly competitive market the quantity supplied by an individual firm has a negligible effect on the market price. In the case of perfect price elasticity, the measure is $\varepsilon_{P}=\infty$.

The other extreme is the vertical demand schedule. The vertical demand schedule implies that some fixed quantity is demanded, regardless of price. An example of such demand is the diabetic consumer with the need for a certain amount of insulin. If the price of insulin goes up, the patient will not consume less of it. The amount desired is set by the patient's medical condition. The measure for perfect price inelasticity is $\varepsilon_{P}=0$.

The nature of the elasticity calculation and consumer behavior in the marketplace imply that for virtually any product (excluding cases of perfect elasticity and perfect inelasticity), demand is more elastic at higher prices and less elastic (more inelastic) at lower prices. For example, at current low prices, the demand for table salt is very inelastic. However, if table salt increased in price to hundreds of dollars per ounce, consumers would become more responsive to its price changes. Exhibit 3 reports several empirical estimates of price elasticity of demand.

\section{Exhibit 3: Empirical Price Elasticities}
Commodity (Good/Service)

Price Elasticity of Market

Demand

\begin{center}
\begin{tabular}{lc}
Alcoholic beverages consumed at home &  \\
Beer & 0.84 \\
Wine & 0.55 \\
Liquor & 0.50 \\
Coffee & 0.16 \\
Regular & 0.36 \\
Instant & 2.44 \\
Credit charges on bank cards & 3.04 \\
Furniture & 1.20 \\
Glassware/china & 1.20 \\
International air transportation United States/Europe & 0.73 \\
Shoes & 1.65 \\
Soybean meal & 2.22 \\
\hline
Tomatoes &  \\
\hline
\end{tabular}
\end{center}

Note: Various sources, as noted in McGuigan, Moyer, and Harris (2008), p. 95. These are the elasticities with respect to the product's own price; by convention, they are shown here as positive numbers.

\section{OTHER FACTORS AFFECTING DEMAND}
describe characteristics of perfect competition, monopolistic competition, oligopoly, and pure monopoly

explain relationships between price, marginal revenue, marginal cost, economic profit, and the elasticity of demand under each market structure

There are two other important forces that influence shifts in consumer demand. One influential factor is consumer income and the other is the price of a related product. For normal goods, as consumer income increases, the demand increases. The degree to which consumers respond to higher incomes by increasing their demand for goods and services is referred to as income elasticity of demand. Income elasticity of demand measures the responsiveness of demand to changes in income. The calculation is similar to that of price elasticity, with the percentage change in income replacing the percentage change in price. Note the new calculation below:

$\varepsilon_{Y}=\left(\%\right.$ change in $\left.Q_{D}\right) \div(\%$ change in $Y)$ where $\varepsilon_{Y}$ is income elasticity of demand, $Q_{D}$ is the quantity demanded, and $Y$ is consumer income. For normal goods, the measure $\varepsilon_{Y}$ will be a positive value. That is, as consumers' income rises, more of the product is demanded. For products that are considered luxury items, the measure of income elasticity will be greater than one. There are other goods and services that are considered inferior products. For inferior products, as consumer income rises, less of the product is demanded. Inferior products will have negative values for income elasticity. For example, a person on a small income may watch television shows, but if this person had more income, she would prefer going to live concerts and theater performances; in this example, television shows would be the inferior good.

As a technical issue, the difference between price elasticity of demand and income elasticity of demand is that the demand adjustment for price elasticity represents a movement along the demand schedule because the demand schedule represents combinations of price and quantity. The demand adjustment for income elasticity represents a shift in the demand curve because with a higher income one can afford to purchase more of the good at any price. For a normal good, an increase in income would shift the demand schedule out to the right, away from the origin of the graph, and a decrease in income would shift the demand curve to the left, toward the origin.

The final factor influencing demand for a product is the change in price of a related product, such as a strong substitute or a complementary product. If a close competitor in the beverage market lowers its price, consumers will substitute that product for your product. Thus, your product's demand curve will shift to the left, toward the origin of the graph. Cross-price elasticity of demand is the responsiveness of the demand for product $A$ that is associated with the change in price of product $B$ :

$$
\varepsilon_{X}=\left(\% \text { change in } Q_{D A}\right) \div\left(\% \text { change in } P_{B}\right)
$$

where $\varepsilon_{X}$ is cross-price elasticity of demand, $Q_{D A}$ is the quantity demanded of product $A$, and $P_{B}$ is the price of product $B$.

When the cross-price elasticity of demand between two products is positive, the two products are considered to be substitutes. For example, you may expect to have positive cross-price elasticity between honey and sugar. If the measure of cross-price elasticity is negative, the two products are referred to as complements of each other. For example, if the price of DVDs goes up, you would expect consumers to buy fewer DVD players. In this case, the cross-price elasticity of demand would have a negative value.

Reviewing cross-price elasticity values provides a simple test for the degree of competition in the market. The more numerous and the closer the substitutes for a product, the lower the pricing power of firms selling in that market; the fewer the substitutes for a product, the greater the pricing power. One interesting application was a US Supreme Court case involving the production and sale of cellophane by DuPont. ${ }^{4}$ The court noted that the relevant product market for DuPont's cellophane was the broader flexible packaging materials market. The Supreme Court found the cross-price elasticity of demand between cellophane and other flexible packaging materials to be sufficiently high and exonerated DuPont from a charge of monopolizing the market.

Because price elasticity of demand relates changes in price to changes in the quantity demanded, there must be a logical relationship between marginal revenue and price elasticity. Recall that marginal revenue equals the change in total revenue given a change in output or sales. An increase in total revenue results from a decrease in price that results in an increase in sales. In order for the increase in the quantity demanded to be sufficient to offset the decline in price, the percentage change in quantity demanded must be greater than the percentage decrease in price. The relationship between TR and price elasticity is as follows: $\varepsilon_{P}>1$ Demand is elastic $\uparrow P \rightarrow \operatorname{TR} \downarrow$ and $\downarrow P \rightarrow \operatorname{TR} \uparrow$

$\varepsilon_{P}=1$ Demand is unitary elastic $\downarrow P \rightarrow$ no change in TR

$0<\varepsilon_{P}<1$ Demand is inelastic $\uparrow P \rightarrow \mathrm{TR} \uparrow$ and $\downarrow P \rightarrow \mathrm{TR} \downarrow$

Total revenue is maximized when marginal revenue is zero. The logic is that as long as marginal revenue is positive (i.e., each additional unit sold contributes to additional total revenue), total revenue will continue to increase. Only when marginal revenue becomes negative will total revenue begin to decline. Therefore, the percentage decrease in price is greater than the percentage increase in quantity demanded. The relationship between marginal revenue (MR) and price elasticity can be expressed as

$$
\mathrm{MR}=P\left[1-\left(1 / \varepsilon_{P}\right)\right]
$$

An understanding of price elasticity of demand is an important strategic tool. It would be very useful to know in advance what would happen to your firm's total revenue if you increased the product's price. If you are operating in the inelastic portion of the demand curve, increasing the price of the product will increase total revenue. On the other hand, if you are operating in the elastic portion of the product's demand curve, increasing the price will decrease total revenue.

Decision makers can also use the relationship between marginal revenue and price elasticity of demand in other ways. For example, suppose you are a farmer considering planting soybeans or some other feed crop, such as corn. From Exhibit 3 , we know that soybean meal's price elasticity of demand has been estimated to be 1.65. We also know that the current (May 2018) soybean meal price is $\$ 465.00$ per metric ton. ${ }^{5}$ Therefore, by solving the equation above, we find that the expected marginal revenue per metric ton of soybean meal is $\$ 183.16$. Soybeans may prove to be a profitable crop for the farmer. Just a few years earlier, in May of 2014, the price of a metric ton of soybean meal was $\$ 578.75$. Given the crop's price elasticity of demand, the estimated marginal revenue per metric ton was then $\$ 227.97$. The higher price translates into higher marginal revenue and might have induced the farmer to plant even more soybeans rather than another feed crop instead.

How do business decision makers decide what level of output to bring to the market? To answer that question, the firm must understand its cost of resources, its production relations, and its supply function. Once the supply function is well defined and understood, it is combined with the demand analysis to determine the profit-maximizing levels of output.

\section*{CONSUMER SURPLUS: VALUE MINUS EXPENDITURE }
\begin{abstract}
describe characteristics of perfect competition, monopolistic competition, oligopoly, and pure monopoly explain relationships between price, marginal revenue, marginal cost, economic profit, and the elasticity of demand under each market structure
\end{abstract}

To this point, we have discussed the fundamentals of supply and demand curves and explained a simple model of how a market can be expected to arrive at an equilibrium combination of price and quantity. While it is certainly necessary for the analyst to understand the basic workings of the market model, it is also crucial to have a sense of why we might care about the nature of the equilibrium. In this section we review the concept of consumer surplus, which is helpful in understanding and evaluating business pricing strategies. Consumer surplus is defined as the difference between the value that a consumer places on the units purchased and the amount of money that was required to pay for them. It is a measure of the value gained by the buyer from the transaction.

To get an intuitive feel for the concept of consumer surplus, consider the last thing you purchased. Whatever it was, think of how much you paid for it. Now contrast that price with the maximum amount you would have been willing to pay rather than go without the item altogether. If those two numbers are different, we say you received some consumer surplus from your purchase. You got a "bargain" because you would have been willing to pay more than you had to pay.

Earlier, we referred to the law of demand, which says that as price falls, consumers are willing to buy more of the good. This observation translates into a negatively sloped demand curve. Alternatively, we could say that the highest price that consumers are willing to pay for an additional unit declines as they consume more and more of a good. In this way, we can interpret their willingness to pay as a measure of how much they value each additional unit of the good. This is a very important point: To purchase a unit of some good, consumers must give up something else they value. So, the price they are willing to pay for an additional unit of a good is a measure of how much they value that unit, in terms of the other goods they must sacrifice to consume it.

If demand curves are negatively sloped, it must be because the value of each additional unit of the good falls as more of the good is consumed. We shall explore this concept further below, but for now, it is enough to recognize that the demand curve can therefore be considered a marginal value curve, because it shows the highest price consumers would be willing to pay for each additional unit. In effect, the demand curve is the willingness of consumers to pay for each additional unit.

This interpretation of the demand curve allows us to measure the total value of consuming any given quantity of a good: It is the sum of all the marginal values of each unit consumed, up to and including the last unit. Graphically, this measure translates into the area under the consumer's demand curve, up to and including the last unit consumed, as shown in Exhibit 4, where the consumer is choosing to buy $Q_{1}$ units of the good at a price of $P_{1}$. The marginal value of the $Q_{1}{ }^{\text {th }}$ unit is clearly $P_{1}$ because that is the highest price the consumer is willing to pay for that unit. Importantly, however, the marginal value of each unit up to the $Q_{1}{ }^{\text {th }}$ is greater than $P_{1}$.

\section{Exhibit 4: Consumer Surplus}
\begin{center}
\includegraphics[max width=\textwidth]{2023_05_04_ccb578c5089fd41615beg-086}
\end{center}

Note: Consumer surplus is the area beneath the demand curve and above the price paid. Because the consumer would have been willing to pay more for each of those units than she paid $\left(P_{1}\right)$, we can say she received more value than the cost to her of buying them. This extra value is the buyer's consumer surplus. The total value of quantity $Q_{1}$ to the buyer is the area of the vertically crosshatched trapezoid in Exhibit 4 . The total expenditure is only the area of the rectangle with height $P_{1}$ and base $Q_{1}$ (bottom section). The total consumer surplus received from buying $Q_{1}$ units at a level price of $P_{1}$ per unit is the difference between the area under the demand curve and the area of the rectangle $P_{1} \times Q_{1}$. The resulting area is shown as the lightly shaded triangle (upper section).

\section{EXAMPLE 1}
\section{Consumer Surplus}
\begin{enumerate}
  \item A market demand function is given by the equation $Q_{D}=180-2 P$. Find the value of consumer surplus if price is equal to 65 .
\end{enumerate}

\section{Solution:}
First, input 65 into the demand function to find the quantity demanded at that price: $Q_{D}=180-2(65)=50$. Then, to make drawing the demand curve easier, invert the demand function by solving for $P$ in terms of $Q_{D}: P=90-$ $0.5 Q_{D}$. Note that the price intercept is 90 and the quantity intercept is 180 .

Draw the demand curve:

\begin{center}
\includegraphics[max width=\textwidth]{2023_05_04_ccb578c5089fd41615beg-087}
\end{center}

Find the area of the triangle above the price of 65 and below the demand curve, up to quantity 50: Area $=1 / 2$ (Base)(Height $)=1 / 2(50)(25)=625$.

\section*{SUPPLY ANALYSIS, OPTIMAL PRICE, AND OUTPUT IN PERFECTLY COMPETITIVE MARKETS }
Consider two corn farmers, Mr. Brown and Ms. Lopez. They both have land available to them to grow corn and can sell at one price, say 3 currency units per kilogram. They will try to produce as much corn as is profitable at that price. If the price is driven up to 5 currency units per kilogram by new consumers entering the market-say, ethanol producers-Mr. Brown and Ms. Lopez will try to produce more corn. To increase their output levels, they may have to use less productive land, increase irrigation, use more fertilizer, or all three. Their production costs will likely increase. They will both still try to produce as much corn as possible to profit at the new, higher price of 5 currency units per kilogram. Exhibit 5 illustrates this example. Note that the supply functions for the individual firms have positive slopes. Thus, as prices increase, the firms supply greater quantities of the product.

\section{Exhibit 5: Firm and Market Supply in Perfect Competition}
Panel A. Brown

\begin{center}
\includegraphics[max width=\textwidth]{2023_05_04_ccb578c5089fd41615beg-088(2)}
\end{center}

Panel B. Lopez

\begin{center}
\includegraphics[max width=\textwidth]{2023_05_04_ccb578c5089fd41615beg-088}
\end{center}

Panel C. Industry

\begin{center}
\includegraphics[max width=\textwidth]{2023_05_04_ccb578c5089fd41615beg-088(1)}
\end{center}

Notice that the market supply curve is the sum of the supply curves of the individual firms-Brown, Lopez, and others-that make up the market. Assume that the supply function for the market can be expressed as a linear relationship, as follows:

$$
Q_{S}=10+5 P \text {, or } P=-2+0.2 Q_{S} \text {, }
$$

where $Q_{S}$ is the quantity supplied and $P$ is the price of the product.

Before we analyze the optimal supply level for the firm, we need to point out that economic costs and profits differ from accounting costs and profits in a significant way. Economic costs include all the remuneration needed to keep the productive resource in its current employment or to acquire the resource for productive use. To evaluate the remuneration needed to keep the resource in its current use and attract new resources for productive use, economists refer to the resource's opportunity cost. Opportunity cost is measured by determining the resource's next best opportunity. If a corn farmer could be employed in an alternative position in the labor market with an income of 50,000, then the opportunity cost of the farmer's labor is 50,000 . Similarly, the farmer's land and capital could be leased to another farmer or sold and reinvested in another type of business. The return foregone by not doing so is an opportunity cost. In economic terms, total cost includes the full normal market return on all the resources utilized in the business. Economic profit is the difference between TR and total cost (TC). Economic profit differs from accounting profit because accounting profit does not include opportunity cost. Accounting profit includes only explicit payments to outside providers of resources (e.g. workers, vendors, lenders) and depreciation based on the historic cost of physical capital.

\section{Optimal Price and Output in Perfectly Competitive Markets}
Carrying forward our examples from Sections 3.1 and 3.2, we can now combine the market supply and demand functions to solve for the equilibrium price and quantity, where $Q^{*}$ represents the equilibrium level of both supply and demand.

$$
\begin{aligned}
& P=25-0.5 Q_{D}=-2+0.2 Q_{S}=P \\
& 25-0.5 Q_{D}=-2+0.2 Q_{S} \\
& 27=0.7 Q^{*} \\
& Q^{*}=38.57
\end{aligned}
$$

According to the market demand curve, the equilibrium price is

$$
P=25-0.5 Q^{*}=25-0.5(38.57)=25-19.29=5.71 \text {. }
$$

With many firms in the market and total output in the market of almost 39 units of the product, the effective market price would be 5.71. This result becomes the demand function for each perfectly competitive firm. Even if a few individual producers could expand production, there would not be a noticeable change in the market equilibrium price. In fact, if any one firm could change the equilibrium market price, the market would not be in perfect competition. Therefore, the demand curve that each perfectly competitive firm faces is a horizontal line at the equilibrium price, as shown in Exhibit 6, even though the demand curve for the whole market is downward sloping.

\section{Exhibit 6: Individual Firm's Demand in Perfect Competition}
\begin{center}
\includegraphics[max width=\textwidth]{2023_05_04_ccb578c5089fd41615beg-090}
\end{center}

\section{EXAMPLE 2}
\section{Demand Curves in Perfect Competition}
\begin{enumerate}
  \item Is it possible that the demand schedule faced by Firm A is horizontal while the demand schedule faced by the market as a whole is downward sloping?
\end{enumerate}

A. No, because Firm A can change its output based on demand changes.

B. No, because a horizontal demand curve means that elasticity is infinite.

C. Yes, because consumers can go to another firm if Firm A charges a higher price, and Firm A can sell all it produces at the market price.

\section{Solution:}
$\mathrm{C}$ is correct. Firm A cannot charge a higher price and has no incentive to sell at a price below the market price.

To analyze the firm's revenue position, recall that average revenue is equivalent to the firm's demand function. Therefore, the horizontal line that represents the firm's demand curve is the firm's AR schedule.

Marginal revenue is the incremental increase in total revenue associated with each additional unit sold. For every extra unit the firm sells, it receives 5.71. Thus, the firm's MR schedule is also the horizontal line at 5.71. TR is calculated by multiplying AR by the quantity of products sold. Total revenue is the area under the AR line at the point where the firm produces the output. In the case of perfect competition, the following conditions hold for the individual firm:

$$
\text { Price }=\text { Average revenue }=\text { Marginal revenue }
$$

The next step is to develop the firm's cost functions. The firm knows that it can sell the entire product it produces at the market's equilibrium price. How much should it produce? That decision is determined by analysis of the firm's costs and revenues. A corn farmer uses three primary resources: land, labor, and capital. In economics, capital is any man-made aid to production. For the corn farmer, his or her capital includes the irrigation system, tractors, harvesters, trucks, grain bins, fertilizer, and so forth. The labor includes the farmer, perhaps members of the farmer's family, and hired labor. In the initial stages of production, only the farmer and the farmer's family are cultivating the land, with a significant investment in capital. They have a tractor, fertilizer, irrigation equipment, grain bins, seed, and a harvester. The investment in land and capital is relatively high compared with the labor input. In this production phase, the average cost of producing a bushel of corn is high. As they begin to expand by adding labor to the collection of expensive land and capital, the average cost of producing corn begins to decline-for example, because one tractor can be used more intensively to plow a larger amount of land. When the combination of land, labor, and capital approaches an efficient range, the average cost of producing a bushel of corn declines.

Given a certain level of technology, there is a limit to the increase in productivity. Eventually something begins to cause declining marginal productivity. That is, each additional unit of input produces a progressively smaller increase in output. This force is called the law of diminishing returns. This "law" helps define the shape of the firm's cost functions. Average cost and marginal cost will be U-shaped. Over the initial stages of output, average and marginal costs will decline. At some level of output, the law of diminishing returns will overtake the efficiencies in production and average and marginal costs will increase.

Average cost (AC) is Total cost (TC) divided by Output $(Q)$. Therefore,

$\mathrm{AC}=\mathrm{TC} / Q$

Note that we have defined average cost (AC) in terms of total costs. Many authors refer to this as "average total cost" to distinguish it from a related concept, "average variable cost," which omits fixed costs. In the remainder of this reading, average cost should be understood to mean average total cost.

Marginal cost $(\mathrm{MC})$ is the change in TC associated with an incremental change in output:

$\mathrm{MC}=\Delta \mathrm{TC} / \Delta Q$

By definition, fixed costs do not vary with output, so marginal cost reflects only changes in variable costs. ${ }^{6} \mathrm{MC}$ declines initially because processes can be made more efficient and specialization makes workers more proficient at their tasks. However, at some higher level of output, $M C$ begins to increase (e.g., must pay workers a higher wage to have them work overtime and, in agriculture, less fertile land must be brought into production). MC and $A C$ will be equal at the level of output where $A C$ is minimized. This is a mathematical necessity and intuitive. If you employ the least expensive labor in the initial phase of production, average and marginal cost will decline. Eventually, additional labor will be more costly. For example, if the labor market is at or near full employment, in order to attract additional workers, you must pay higher wages than they are currently earning elsewhere. Thus, the additional (marginal) labor is more costly, and the higher cost increases the overall average as soon as MC exceeds AC. Exhibit 7 illustrates the relationship between AC and MC.

6 Readers who are familiar with calculus will recognize that MC is simply the derivative of total cost with respect to quantity produced.

\section{Exhibit 7: Individual Firm's Short-Run Cost Schedules}
\begin{center}
\includegraphics[max width=\textwidth]{2023_05_04_ccb578c5089fd41615beg-092}
\end{center}

Quantity (firm)

Now combine the revenue and cost functions from Exhibit 6 and Exhibit 7. In short-run equilibrium, the perfectly competitive firm can earn an economic profit (or an economic loss). In this example, the equilibrium price, 5.71, is higher than the minimum AC. The firm will always maximize profit at an output level where $M R=$ MC. Recall that in perfect competition, the horizontal demand curve is the marginal revenue and average revenue schedules. By setting output at point A in Exhibit 8, where $M R=M C$, the firm will maximize profits. Total revenue is equal to $P \times Q-$ in this case, 5.71 times $Q_{C}$. Total cost is equal to $Q_{C}$ times the average cost of producing $Q_{C}$, at point B in Exhibit 8. The difference between the two areas is economic profit.

\section{Exhibit 8: Perfectly Competitive Firm's Short-Run Equilibrium}
\begin{center}
\includegraphics[max width=\textwidth]{2023_05_04_ccb578c5089fd41615beg-093}
\end{center}

\section{LONG-RUN EQUILIBRIUM IN PERFECTLY COMPETITIVE MARKETS}
explain factors affecting long-run equilibrium under each market structure

describe pricing strategy under each market structure

In the long run, economic profit will attract other entrepreneurs to the market, resulting in the production of more output. The aggregate supply will increase, shifting the industry supply $\left(S_{1}\right)$ curve to the right, away from the origin of the graph. For a given demand curve, this increase in supply at each price level will lower the equilibrium price, as shown in Exhibit 9.

\section{Exhibit 9: Perfectly Competitive Market with Increased Supply}
\section{Panel A. Brown Panel B. Lopez Panel C. Industry}
\begin{center}
\includegraphics[max width=\textwidth]{2023_05_04_ccb578c5089fd41615beg-094}
\end{center}

In the long run, the perfectly competitive firm will operate at the point where marginal cost equals the minimum of average cost, because at that point, entry is no longer profitable: In equilibrium, price equals not only marginal cost (firm equilibrium) but also minimum average cost, so that total revenues equal total costs. This result implies that the perfectly competitive firm operates with zero economic profit. That is, the firm receives its normal profit (rental cost of capital), which is included in its economic costs. Recall that economic profits occur when total revenue exceeds total cost (and therefore differ from accounting profits). With low entry cost and homogeneous products to sell, the perfectly competitive firm earns zero economic profit in the long run.

Exhibit 10 illustrates the long-run equilibrium position of the perfectly competitive firm. Note that total revenue equals price ( $\$ 4.50)$ times quantity $\left(Q_{E}\right)$ and total cost equals average cost $(\$ 4.50)$ times quantity $\left(Q_{E}\right)$.

\section{Exhibit 10: Perfectly Competitive Firm's Long-Run Equilibrium}
\begin{center}
\includegraphics[max width=\textwidth]{2023_05_04_ccb578c5089fd41615beg-095(1)}
\end{center}

The long-run marginal cost schedule is the perfectly competitive firm's supply curve. The firm's demand curve is dictated by the aggregate market's equilibrium price. The basic rule of profit maximization is that $M R=M C$, as is the case in long-run equilibrium. The firm's demand schedule is the same as the firm's marginal revenue and average revenue. Given its cost of operation, the only decision the perfectly competitive firm faces is how much to produce. The answer is the level of output that maximizes its return, and that level is where $\mathrm{MR}=\mathrm{MC}$. The demand curve is perfectly elastic. Of course, the firm constantly tries to find ways to lower its cost in the long run.

\section{SCHUMPETER ON INNOVATION AND PERFECT COMPETITION}
The Austrian-American economist Joseph A. Schumpeter ${ }^{7}$ pointed out that technical change in economics can happen in two main ways:

\begin{enumerate}
  \item Innovation of process: a new, more efficient way to produce an existing good or service.

  \item Innovation of product: a new product altogether or an innovation upon an existing product.

\end{enumerate}

Innovation of process is related to production methods. For example, instead of mixing cement by hand, since the invention of the electric engine it has been possible to use electric mixers. A more recent innovation has been to use the internet to provide technical support to personal computer users: A technician can remotely log on to the customer's PC and fix problems instead of providing instructions over the phone. The result is likely the same, but the process is more efficient.

\begin{center}
\includegraphics[max width=\textwidth]{2023_05_04_ccb578c5089fd41615beg-095}
\end{center}

Innovation of product is related to the product itself. MP3 players, smart phones, robot surgery, and GPS vehicle monitoring have existed only for a few years. They are new products and services. While portable music players existed before the MP3 player, no similar service existed before GPS monitoring of personal vehicles and freight trucks was invented.

How does the reality of continuous innovation of product and process, which is a characteristic of modern economies, fit into the ideal model of perfect competition, where the product is made by a huge number of tiny, anonymous suppliers? This seems a contradiction because the tiny suppliers cannot all be able to invent new productsand indeed, the markets for portable music players and smart phones do not look like perfect competition.

Schumpeter suggested that perfect competition is more of a long-run type of market. In the short run, a company develops a new process or product and is the only one to take advantage of the innovation. This company likely will have high profits and will outpace any competitors. A second stage is what Schumpeter called the swarming (as when a group of bees leaves a hive to follow a queen): In this case, some entrepreneurs notice the innovation and follow the innovator through imitation. Some of them will fail, while others will succeed and possibly be more successful than the initial innovator. The third stage occurs when the new technology is no longer new because everyone has imitated it. At this point, no economic profits are realized, because the new process or product is no longer a competitive advantage, in the sense that everyone has it-which is when perfect competition prevails and we have long-run equilibrium until a new innovation of process or product is introduced.

\section{MONOPOLISTIC COMPETITION}
$\square \quad \begin{aligned} & \text { describe characteristics of perfect competition, monopolistic } \\ & \text { competition, oligopoly, and pure monopoly } \\ & \text { explain relationships between price, marginal revenue, marginal cost, } \\ & \text { economic profit, and the elasticity of demand under each market } \\ & \text { structure } \\ & \text { describe a firm's supply function under each market structure } \\ & \square\end{aligned} \quad \begin{aligned} & \text { describe and determine the optimal price and output for firms under } \\ & \text { each market structure } \\ & \text { describe pricing strategy under each market structure }\end{aligned}$
$\square \quad$

Early in the 20th century, economists began to realize that most markets did not operate under the conditions of perfect competition. ${ }^{8}$ Many market structures exhibited characteristics of strong competitive forces; however, other distinct non-competitive factors played important roles in the market. As the name implies, monopolistic competition is a hybrid market. The most distinctive factor in monopolistic competition is product differentiation. Recall the characteristics from Exhibit 1:

\begin{enumerate}
  \item There are a large number of potential buyers and sellers.

  \item The products offered by each seller are close substitutes for the products offered by other firms, and each firm tries to make its product look different.

  \item Entry into and exit from the market are possible with fairly low costs. 4. Firms have some pricing power.

  \item Suppliers differentiate their products through advertising and other non-price strategies.

\end{enumerate}

While the market is made up of many firms that compose the product group, each producer attempts to distinguish its product from that of the others. Product differentiation is accomplished in a variety of ways. For example, consider the wide variety of communication devices available today. Decades ago, when each communication market was controlled by a regulated single seller (the telephone company), all telephones were alike. In the deregulated market of today, the variety of physical styles and colors is extensive. All versions accomplish many of the same tasks.

The communication device manufacturers and providers differentiate their products with different colors, styles, networks, bundled applications, conditional contracts, functionality, and more. Advertising is usually the avenue pursued to convince consumers there is a difference between the goods in the product group. Successful advertising and trademark branding result in customer loyalty. A good example is the brand loyalty associated with Harley-Davidson motorcycles. Harley-Davidson's customers believe that their motorcycles are truly different from and better than all other motorcycles. The same kind of brand loyalty exists for many fashion creations and cosmetics.

The extent to which the producer is successful in product differentiation determines pricing power in the market. Very successful differentiation results in a market structure that resembles the single-seller market (monopoly). However, because there are relatively low entry and exit costs, competition will, in the long run, drive prices and revenues down toward an equilibrium similar to perfect competition. Thus, the hybrid market displays characteristics found in both perfectly competitive and monopoly markets.

\section{Demand Analysis in Monopolistically Competitive Markets}
Because each good sold in the product group is somewhat different from the others, the demand curve for each firm in the monopolistic competition market structure is downward sloping to the right. Price and the quantity demanded are negatively related. Lowering the price will increase the quantity demanded, and raising the price will decrease the quantity demanded. There will be ranges of prices within which demand is elastic and (lower) prices at which demand is inelastic. Exhibit 11 illustrates the demand, marginal revenue, and cost structures facing a monopolistically competitive firm in the short run.

\section{Exhibit 11: Short-Run Equilibrium in Monopolistic Competition}
\begin{center}
\includegraphics[max width=\textwidth]{2023_05_04_ccb578c5089fd41615beg-098}
\end{center}

In the short run, the profit-maximizing choice is the level of output where $M R=M C$. Because the product is somewhat different from that of the competitors, the firm can charge the price determined by the demand curve. Therefore, in Exhibit 11, $Q_{1}$ is the ideal level of output and $P_{1}$ is the price consumers are willing to pay to acquire that quantity. Total revenue is the area of the rectangle $P_{1} \times Q_{1}$.

\section{Supply Analysis in Monopolistically Competitive Markets}
In perfect competition, the firm's supply schedule is represented by the marginal cost schedule. In monopolistic competition, there is no well-defined supply function. The information used to determine the appropriate level of output is based on the intersection of MC and MR. However, the price that will be charged is based on the market demand schedule. The firm's supply curve should measure the quantity the firm is willing to supply at various prices. That information is not represented by either marginal cost or average cost.

\section{Optimal Price and Output in Monopolistically Competitive Markets}
As seen in Section 4.1, in the short run, the profit-maximizing choice is the level of output where $\mathrm{MR}=\mathrm{MC}$ and total revenue is the area of the rectangle $P_{1} \times Q_{1}$ in Exhibit 11 .

The average cost of producing $Q_{1}$ units of the product is $C_{1}$, and the total cost is the area of the rectangle $C_{1} \times Q_{1}$. The difference between TR and TC is economic profit. The profit relationship is described as

$\pi=\mathrm{TR}-\mathrm{TC}$

where $\pi$ is total profit, TR is total revenue, and TC is total cost.

\section{THE BENEFITS OF IMPERFECT COMPETITION}
Is monopolistic competition indeed imperfect —that is, is it a bad thing? At first, one would say that it is an inefficient market structure because prices are higher and the quantity supplied is less than in perfect competition. At the same time, in the real world, we see more markets characterized by monopolistic competition than markets meeting the strict conditions of perfect competition. If monopolistic competition were that inefficient, one wonders, why would it be so common?

A part of the explanation goes back to Schumpeter. Firms try to differentiate their products to meet the needs of customers. Differentiation provides a profit incentive to innovate, experiment with new products and services, and potentially improve the standard of living.

Moreover, because each customer has differing tastes and preferences, slight variations of each good or service are likely to capture the niche of the market that prefers them. An example is the market for candy, where one can find chocolate, licorice, mint, fruit, and many other flavors.

A further reason why monopolistic competition may be good is that people like variety. Traditional economic theories of international trade suggested that countries should buy products from other countries that they cannot produce domestically. Therefore, Norway should buy bananas from a tropical country and sell crude oil in exchange. But this is not the only kind of exchange that happens in reality: For example, Germany imports Honda, Subaru, and Toyota cars from Japan and sells Volkswagen, Porsche, Mercedes, and BMW cars to Japan. In theory, this should not occur because each of the countries produces good cars domestically and does not need to import them. The truth, however (see, for example, Krugman 1989), is that consumers in both countries enjoy variety. Some Japanese drivers prefer to be at the steering wheel of a BMW; others like Hondas, and the same happens in Germany. Variety and product differentiation, therefore, are not necessarily bad things.

\section{LONG-RUN EQUILIBRIUM IN MONOPOLISTIC COMPETITION}
explain factors affecting long-run equilibrium under each market structure describe pricing strategy under each market structure

Because TC includes all costs associated with production, including opportunity cost, economic profit is a signal to the market, and that signal will attract more competition. Just as with the perfectly competitive market structure, with relatively low entry costs, more firms will enter the market and lure some customers away from the firm making an economic profit. The loss of customers to new entrant firms will drive down the demand for all firms producing similar products. In the long run for the monopolistically competitive firm, economic profit will fall to zero. Exhibit 12 illustrates the condition of long-run equilibrium for monopolistic competition.

\section{Exhibit 12: Long-Run Equilibrium in Monopolistic Competition}
\begin{center}
\includegraphics[max width=\textwidth]{2023_05_04_ccb578c5089fd41615beg-100}
\end{center}

In long-run equilibrium, output is still optimal at the level where $M R=M C$, which is $Q_{1}$ in Exhibit 12. Again, the price consumers are willing to pay for any amount of the product is determined from the demand curve. That price is $P_{1}$ for the quantity $Q_{1}$ in Exhibit 12, and total revenue is the area of the rectangle $P_{1} \times Q_{1}$. Notice that unlike long-run equilibrium in perfect competition, in the market of monopolistic competition, the equilibrium position is at a higher level of average cost than the level of output that minimizes average cost. Average cost does not reach its minimum until output level $Q_{2}$ is achieved. Total cost in this long-run equilibrium position is the area of the rectangle $C_{1} \times Q_{1}$. Economic profit is total revenue minus total cost. In Exhibit 12, economic profit is zero because total revenue equals total cost: $P_{1} \times Q_{1}=C_{1} \times Q_{1}$.

In the hybrid market of monopolistic competition, zero economic profit in long-run equilibrium resembles perfect competition. However, the long-run level of output, $Q_{1}$, is less than $Q_{2}$, which corresponds to the minimum average cost of production and would be the long run level of output in a perfectly competitive market. In addition, the economic cost in monopolistic competition includes some cost associated with product differentiation, such as advertising. In perfect competition, there are no costs associated with advertising or marketing because all products are homogeneous. Prices are lower, but consumers may have little variety.

\section{OLIGOPOLY AND PRICING STRATEGIES}
describe characteristics of perfect competition, monopolistic competition, oligopoly, and pure monopoly

explain relationships between price, marginal revenue, marginal cost, economic profit, and the elasticity of demand under each market structure

describe pricing strategy under each market structure An oligopoly market structure is characterized by only a few firms doing business in a relevant market. The products must all be similar and generally be substitutes for one another. In some oligopoly markets, the goods or services may be differentiated by marketing and strong brand recognition, as in the markets for breakfast cereals and for bottled or canned beverages. Other examples of oligopoly markets are made up of homogeneous products with little or no attempt at product differentiation, such as petroleum and cement. The most distinctive characteristic of oligopoly markets is the small number of firms that dominate the market. There are so few firms in the relevant market that their pricing decisions are interdependent. That is, each firm's pricing decision is based on the expected retaliation by the other firms. Recall from Exhibit 1 the characteristics of oligopoly markets:

\begin{enumerate}
  \item There are a small number of potential sellers.

  \item The products offered by each seller are close substitutes for the products offered by other firms and may be differentiated by brand or homogeneous and unbranded.

  \item Entry into the market is difficult, with fairly high costs and significant barriers to competition.

  \item Firms typically have substantial pricing power.

  \item Products are often highly differentiated through marketing, features, and other non-price strategies.

\end{enumerate}

Because there are so few firms, each firm can have some degree of pricing power, which can result in substantial profits. Another by-product of the oligopoly market structure is the attractiveness of price collusion. Even without price collusion, a dominant firm may easily become the price maker in the market. Oligopoly markets without collusion typically have the most sophisticated pricing strategies. Examples of non-colluding oligopolies include the US tobacco market and the Thai beer market. In 2004, four firms controlled 99 percent of the US tobacco industry. ${ }^{9}$ Brands owned by Singha Co. and by ThaiBev controlled over 90 percent of the Thai beer market in 2009. (This situation is expected to change soon, as the Association of Southeast Asian Nations trade agreement will open the doors to competition from other ASEAN producers.) Perhaps the most well-known oligopoly market with collusion is the OPEC cartel, which seeks to control prices in the petroleum market by fostering agreements among oil-producing countries.

\section{Demand Analysis and Pricing Strategies in Oligopoly Markets}
Oligopoly markets' demand curves depend on the degree of pricing interdependence. In a market where collusion is present, the aggregate market demand curve is divided up by the individual production participants. Under non-colluding market conditions, each firm faces an individual demand curve. Furthermore, non-colluding oligopoly market demand characteristics depend on the pricing strategies adopted by the participating firms. There are three basic pricing strategies: pricing interdependence, the Cournot assumption, and the Nash equilibrium.

The first pricing strategy is to assume pricing interdependence among the firms in the oligopoly. A good example of this situation is any market where there are "price wars," such as the commercial airline industry. For example, flying out of their hubs in Atlanta, both Delta Air Lines and AirTran Airways jointly serve several cities. AirTran is a low-cost carrier and typically offers lower fares to destinations out of Atlanta.

9 These examples are based on "Industry Surveys," Net Advantage Database, Standard \& Poor's; and Market Share Reports, Gale Research, annual issues, as noted in McGuigan, Moyer, and Harris (2016). Delta tends to match the lower fares for those cities also served by AirTran when the departure and arrival times are similar to its own. However, when Delta offers service to the same cities at different time slots, Delta's ticket prices are higher.

The most common pricing strategy assumption in these price war markets is that competitors will match a price reduction and ignore a price increase. The logic is that by lowering its price to match a competitor's price reduction, the firm will not experience a reduction in customer demand. Conversely, by not matching the price increase, the firm stands to attract customers away from the firm that raised its prices. The oligopolist's demand relationship must represent the potential increase in market share when rivals' price increases are not matched and no significant change in market share when rivals' price decreases are matched.

Given a prevailing price, the price elasticity of demand will be much greater if the price is increased and less if the price is decreased. The firm's customers are more responsive to price increases because its rivals have lower prices. Alternatively, the firm's customers are less responsive to price decreases because its rivals will match its price change.

This implies that the oligopolistic firm faces two different demand structures, one associated with price increases and another relating to price reductions. Each demand function will have its own marginal revenue structure as well. Consider the demand and marginal revenue functions in Exhibit 13. The functions $\mathrm{D}_{\mathrm{P} \uparrow}$ and $\mathrm{MR}_{\mathrm{P} \uparrow}$ represent the demand and marginal revenue schedules associated with higher prices, while the functions $D_{P \downarrow}$ and $M_{P}$ represent the lower prices' demand and marginal revenue schedules. The two demand schedules intersect at the prevailing price (i.e., the price where price increase and price decrease are both equal to zero).

\section{Exhibit 13: Kinked Demand Curve in Oligopoly Market}
\begin{center}
\includegraphics[max width=\textwidth]{2023_05_04_ccb578c5089fd41615beg-102}
\end{center}

This oligopolistic pricing strategy results in a kinked demand curve, with the two segments representing the different competitor reactions to price changes. The kink in the demand curve also yields a discontinuous marginal revenue structure, with one part associated with the price increase segment of demand and the other relating to the price decrease segment. Therefore, the firm's overall demand equals the relevant portion of $\mathrm{D}_{\mathrm{P} \uparrow}$ and the relevant portion of $\mathrm{D}_{\mathrm{P} \downarrow}$. Exhibit 14 represents the firm's new demand and marginal revenue schedules. The firm's demand schedule in Exhibit 14 is segment $D_{P \uparrow}$ and $D_{P \downarrow}$, where overall demand $D=D_{P \uparrow}+D_{P \downarrow}$.

\section{Exhibit 14: Kinked Demand Curve in Oligopoly Market}
\begin{center}
\includegraphics[max width=\textwidth]{2023_05_04_ccb578c5089fd41615beg-103}
\end{center}

Notice in Exhibit 14 that a wide variety of cost structures are consistent with the prevailing price. If the firm has relatively low marginal costs, $\mathrm{MC}_{1}$, the profit-maximizing pricing rule established earlier, $\mathrm{MR}=\mathrm{MC}$, still holds for the oligopoly firm. Marginal cost can rise to $\mathrm{MC}_{2}$ and $\mathrm{MC}_{3}$ before the firm's profitability is challenged. If the marginal cost curve $\mathrm{MC}_{2}$ passes through the gap in marginal revenue, the most profitable price and output combination remains unchanged at the prevailing price and original level of output.

Criticism of the kinked demand curve analysis focuses on its inability to determine what the prevailing price is from the outset. The kinked demand curve analysis does help explain why stable prices have been observed in oligopoly markets and is therefore a useful tool for analyzing such markets. However, because it cannot determine the original prevailing price, it is considered an incomplete pricing analysis.

\section{THE COURNOT ASSUMPTION}
\begin{center}
\includegraphics[max width=\textwidth]{2023_05_04_ccb578c5089fd41615beg-103(1)}
\end{center}

The second pricing strategy was first developed by French economist Augustin Cournot in 1838. In the Cournot assumption, each firm determines its profit-maximizing production level by assuming that the other firms' output will not change. This assumption simplifies pricing strategy because there is no need to guess what the other firm will do to retaliate. It also provides a useful approach to analyzing real-world behavior in oligopoly markets. Take the most basic oligopoly market situation, a two-firm duopoly market. ${ }^{10}$ In equilibrium, neither firm has an incentive to change output, given the other firm's production level. Each firm attempts to maximize its own profits under the assumption that the other firm will continue producing the same level of output in the future. The Cournot strategy assumes that this pattern continues until each firm reaches its long-run equilibrium position. In long-run equilibrium, output and price are stable: There is no change in price or output that will increase profits for either firm.

Consider this example of a duopoly market. Assume that the aggregate market demand has been estimated to be

$$
Q_{D}=450-P
$$

The supply function is represented by constant marginal cost $\mathrm{MC}=30$.

The Cournot strategy's solution can be found by setting $Q_{D}=q_{1}+q_{2}$, where $q_{1}$ and $q_{2}$ represent the output levels of the two firms. Each firm seeks to maximize profit, and each firm believes the other firm will not change output as it changes its own output (Cournot's assumption). The firm will maximize profit where $M R=M C$. Rearranging the aggregate demand function in terms of price, we get:

$$
P=450-Q_{D}=450-q_{1}-q_{2} \text {, and } \mathrm{MC}=30
$$

Total revenue for each of the two firms is found by multiplying price and quantity:

$$
\begin{gathered}
\mathrm{TR}_{1}=P q_{1}=\left(450-q_{1}-q_{2}\right) q_{1}=450 q_{1}-q_{1}^{2}-q_{1} q_{2}, \text { and } \\
\mathrm{TR}_{2}=P q_{2}=\left(450-q_{1}-q_{2}\right) q_{2}=450 q_{2}-q_{2} q_{1}-q_{2}^{2}
\end{gathered}
$$

Marginal revenue is defined as the change in total revenue, given a change in sales $\left(q_{1}\right.$ or $\left.q_{2}\right) .{ }^{11}$ For the profit-maximizing output, set $\mathrm{MR}=\mathrm{MC}$, or

$$
450-2 q_{1}-q_{2}=30
$$

and

$$
450-q_{1}-2 q_{2}=30
$$

Find the simultaneous equilibrium for the two firms by solving the two equations with two unknowns:

$$
450-2 q_{1}-q_{2}=450-q_{1}-2 q_{2}
$$

Because $q_{2}=q_{1}$ under Cournot's assumption, insert this solution into the demand function and solve as

$$
450-2 q_{1}-q_{1}=450-3 q_{1}=30
$$

Therefore, $q_{1}=140, q_{2}=140$, and $Q=280$.

The price is $P=450-280=170$.

In the Cournot strategic pricing solution, the market equilibrium price will be 170 and the aggregate output will be 280 units. This result, known as the Cournot equilibrium, differs from the perfectly competitive market equilibrium because the perfectly competitive price will be lower and the perfectly competitive output will be higher. In general, non-competitive markets have higher prices and lower levels of output in equilibrium when compared with perfect competition. In competition, the equilibrium is reached where price equals marginal cost.

$P_{C}=\mathrm{MR}_{C}=\mathrm{MC}$, so $450-Q=30$

where $P_{C}$ is the competitive firm's equilibrium price.

10 The smallest possible oligopoly market is a duopoly, which is made up of only two sellers.

11 The marginal revenue formulas can be obtained using the technique introduced in Section 3.1. For the market demand function, total revenue is $P \times Q=450 Q-Q^{2}$ and our technique yields $M R=\Delta T R / \Delta Q=$ $450-2 Q$. For the individual firms in the Cournot duopoly, $\mathrm{MR}_{1}=\Delta \mathrm{TR}_{1} / \Delta q_{1}=450-2 q_{1}-q_{2}$, and $\mathrm{MR}_{2}$ $=\Delta \mathrm{TR}_{2} / \Delta q_{2}=450-q_{1}-2 q_{2}$. Each of these marginal revenue formulas is, of course, the derivative of the relevant total revenue formula with respect to the relevant quantity.

$$
Q=420, \text { and } P_{C}=30 .
$$

Exhibit 15 describes the oligopoly, competitive, and monopoly market equilibrium positions, where $P_{M}$ is the monopoly optimum price, $P_{C}$ is the competitive price, and $P_{\text {Cournot }}$ is the oligopoly price under the Cournot assumption.

\section{Exhibit 15: Cournot Equilibrium in Duopoly Market}
\begin{center}
\includegraphics[max width=\textwidth]{2023_05_04_ccb578c5089fd41615beg-105}
\end{center}

In the later discussion regarding monopoly market structure, equilibrium will be established where MR = MC. That solution is also shown in Exhibit 15. The monopoly firm's demand schedule is the aggregate market demand schedule. Therefore, the solution is

$$
\mathrm{MR}=\mathrm{MC}
$$

From Footnote 10, $M R=450-2 Q$; therefore,

$$
450-2 Q=30 \text { and } Q=210
$$

From the aggregate demand function, solve for price:

$$
P_{M}=450-210=240
$$

Note that the Cournot solution falls between the competitive equilibrium and the monopoly solution.

It can be shown that as the number of firms increases from two to three, from three to four, and so on, the output and price equilibrium positions move toward the competitive equilibrium solution. This result has historically been the theoretical basis for the antitrust policies established in the United States.

\section*{THE NASH EQUILIBRIUM }
The third pricing strategy is attributed to one of the 1994 Nobel Prize winners, John Nash, who first developed the general concepts. In the previous analysis, the concept of market equilibrium occurs when firms are achieving their optimum remuneration under the circumstances they face. In this optimum environment, the firm has no motive to change price or output level. Existing firms are earning a normal return (zero economic profit), leaving no motive for entry to or exit from the market. All firms in the market are producing at the output level where price equals the average cost of production.

In game theory (the set of tools that decision makers use to consider responses by rival decision makers), the Nash equilibrium is present when two or more participants in a non-cooperative game have no incentive to deviate from their respective equilibrium strategies after they have considered and anticipated their opponent's rational choices or strategies. In the context of oligopoly markets, the Nash equilibrium is an equilibrium defined by the characteristic that none of the oligopolists can increase its profits by unilaterally changing its pricing strategy. The assumption is made that each participating firm does the best it can, given the reactions of its rivals. Each firm anticipates that the other firms will react to any change made by competitors by doing the best they can under the altered circumstances. The firms in the oligopoly market have interdependent actions. The actions are non-cooperative, with each firm making decisions that maximize its own profits. The firms do not collude in an effort to maximize joint profits. The equilibrium is reached when all firms are doing the best they can, given the actions of their rivals.

Exhibit 16 illustrates the duopoly result from the Nash equilibrium. Assume there are two firms in the market, ArcCo and BatCo. ArcCo and BatCo can charge high prices or low prices for the product. The market outcomes are shown in Exhibit 16.

\section{Exhibit 16: Nash Equilibrium in Duopoly Market}
\begin{center}
\includegraphics[max width=\textwidth]{2023_05_04_ccb578c5089fd41615beg-107}
\end{center}

For example, the top left solution indicates that when both ArcCo and BatCo offer the product at low prices, ArcCo earns a profit of 50 and BatCo earns 70. The top right solution shows that if ArcCo offers the product at a low price, BatCo earns zero profits. The solution with the maximum joint profits is the lower right equilibrium, where both firms charge high prices for the product. Joint profits are 800 in this solution.

However, the Nash equilibrium requires that each firm behaves in its own best interest. BatCo can improve its position by offering the product at low prices when ArcCo is charging high prices. In the lower left solution, BatCo maximizes its profits at 350. While ArcCo can earn 500 in its best solution, it can do so only if BatCo also agrees to charge high prices. This option is clearly not in BatCo's best interest because it can increase its return from 300 to 350 by charging lower prices.

This scenario brings up the possibility of collusion. If ArcCo agrees to share at least 51 of its 500 when both companies are charging high prices, BatCo should also be willing to charge high prices. While, in general, such collusion is unlawful in most countries, it remains a tempting alternative. Clearly, conditions in oligopolistic industries encourage collusion, with a small number of competitors and interdependent pricing behavior. Collusion is motivated by several factors: increased profits, reduced cash flow uncertainty, and improved opportunities to construct barriers to entry.

When collusive agreements are made openly and formally, the firms involved are called a cartel. In some cases, collusion is successful; other times, the forces of competition overpower collusive behavior. There are six major factors that affect the chances of successful collusion. ${ }^{12}$

\begin{enumerate}
  \item The number and size distribution of sellers. Successful collusion is more likely if the number of firms is small or if one firm is dominant. Collusion becomes more difficult as the number of firms increases or if the few firms have similar market shares. When the firms have similar market shares, the competitive forces tend to overshadow the benefits of collusion.

  \item The similarity of the products. When the products are homogeneous, collusion is more successful. The more differentiated the products, the less likely it is that collusion will succeed.

  \item Cost structure. The more similar the firms' cost structures, the more likely it is that collusion will succeed. 4. Order size and frequency. Successful collusion is more likely when orders are frequent, received on a regular basis, and relatively small. Frequent small orders, received regularly, diminish the opportunities and rewards for cheating on the collusive agreement.

  \item The strength and severity of retaliation. Oligopolists will be less likely to break the collusive agreement if the threat of retaliation by the other firms in the market is severe.

  \item The degree of external competition. The main reason to enter into the formal collusion is to increase overall profitability of the market, and rising profits attract competition. For example, in 2016 the average extraction cost of a barrel of crude oil from Saudi Arabia was approximately $\$ 9$, while the average cost from United States shale oil fields was roughly $\$ 23.50$. The cost of extracting oil from the Canadian tar sands in 2016 was roughly $\$ 27$ per barrel. It is more likely that crude oil producers in the gulf countries will successfully collude because of the similarity in their cost structures (roughly \$9-\$10 per barrel). If OPEC had held crude oil prices down below $\$ 30$ per barrel, there would have been a viable economic argument to develop US shale oil fields through fracking or expand extraction from Canada's tar sands. OPEC's successful cartel raised crude oil prices to the point where outside sources became economically possible and in doing so increased the competition the cartel faces. ${ }^{13}$

\end{enumerate}

There are other possible oligopoly strategies that are associated with decision making based on game theory. The Cournot equilibrium and the Nash equilibrium are examples of specific strategic games. A strategic game is any interdependent behavioral choice employed by individuals or groups that share a common goal (e.g., military units, sports teams, or business decision makers). Another prominent decision-making strategy in oligopolistic markets is the first-mover advantage in the Stackelberg model, named after the economist who first conceptualized the strategy. ${ }^{14}$ The important difference between the Cournot model and the Stackelberg model is that Cournot assumes that in a duopoly market, decision making is simultaneous, while Stackelberg assumes that decisions are made sequentially. In the Stackelberg model, the leader firm chooses its output first and then the follower firm chooses after observing the leader's output. It can be shown that the leader firm has a distinct advantage, being a first mover. ${ }^{15}$ In the Stackelberg game, the leader can aggressively overproduce to force the follower to scale back its production or even punish or eliminate the weaker opponent. This approach is sometimes referred to as a "top dog" strategy. ${ }^{16}$ The leader earns more than in Cournot's simultaneous game, while the follower earns less. Many other strategic games are possible in oligopoly markets. The important conclusion is that the optimal strategy of the firm depends on what its adversary does. The price and marginal revenue the firm receives for its product depend on both its decisions and its adversary's decisions.

13 "Barrel Breakdown," Wall Street Journal, April 15, 2016.

14 Von Stackelberg (1952). See also Kelly (2011) for a comparison between the Cournot and Stackelberg equilibriums.

15 Nicholson and Snyder (2016).

16 Fudenberg and Tirole (1984).

\section{OLIGOPOLY MARKETS: OPTIMAL PRICE, OUTPUT, AND LONG-RUN EQUILIBRIUM}
As in monopolistic competition, the oligopolist does not have a well-defined supply function. That is, there is no way to determine the oligopolist's optimal levels of output and price independent of demand conditions and competitor's strategies. However, the oligopolist still has a cost function that determines the optimal level of supply. Therefore, the profit-maximizing rule established earlier is still valid: The level of output that maximizes profit is where $M R=M C$. The price to charge is determined by what price consumers are willing to pay for that quantity of the product. Therefore, the equilibrium price comes from the demand curve, while the output level comes from the relationship between marginal revenue and marginal cost.

Consider an oligopoly market in which one of the firms is dominant and thus able to be the price leader. Dominant firms generally have 40 percent or greater market share. When one firm dominates an oligopoly market, it does so because it has greater capacity, has a lower cost structure, was first to market, or has greater customer loyalty than other firms in the market.

Assuming there is no collusion, the dominant firm becomes the price maker, and therefore its actions are similar to monopoly behavior in its segment of the market. The other firms in the market follow the pricing patterns of the dominant firm. Why wouldn't the price followers attempt to gain market share by undercutting the dominant firm's price? The most common explanation is that the dominant firm's supremacy often stems from a lower cost of production. Usually, the price followers would rather charge a price that is even higher than the dominant firm's price choice. If they attempt to undercut the dominant firm, the followers risk a price war with a lower-cost producer that can threaten their survival. Some believe that one explanation for the price leadership position of the dominant firm is simply convenience. Only one firm has to make the pricing decisions, and the others can simply follow its lead.

Exhibit 17 establishes the dominant firm's pricing decision. The dominant firm's demand schedule, $D_{L}$, is a substantial share of the total market demand, $D_{T}$. The low-cost position of the dominant firm is represented by its marginal cost, $M C_{L}$. The sum of the marginal costs of the price followers is established as $\Sigma \mathrm{MC}_{F}$ and represents a higher cost of production than that of the price leader.

\section{Exhibit 17: Dominant Oligopolist's Price Leadership}
\begin{center}
\includegraphics[max width=\textwidth]{2023_05_04_ccb578c5089fd41615beg-109}
\end{center}

There is an important reason why the total demand curve and the leader demand curve are not parallel in Exhibit 17: Remember that the leader is the low-cost producer. Therefore, as price decreases, fewer of the smaller suppliers will be able to profitably remain in the market, and several will exit because they do not want to sell below cost. Therefore, the leader will have a larger market share as $P$ decreases, which implies that $Q_{L}$ increases at a low price, exactly as shown by a steeper $D_{T}$ in the diagram.

\includegraphics[max width=\textwidth, center]{2023_05_04_ccb578c5089fd41615beg-110}
output $Q_{L}$. This is the quantity it wants to supply; however, the price it will charge is determined by its segment of the total demand function, $D_{L}$. At price $P_{L}$, the dominant firm will supply quantity $Q_{L}$ of total demand, $D_{T}$. The price followers will supply the difference to the market, $\left(Q_{T}-Q_{L}\right)=Q_{F}$. Therefore, neither the dominant firm nor the follower firms have a single functional relationship that determines the quantity supplied at various prices.

\section{Optimal Price and Output in Oligopoly Markets}
From the discussion above, clearly there is no single optimum price and output analysis that fits all oligopoly market situations. The interdependence among the few firms that make up the oligopoly market provides a complex set of pricing alternatives, depending on the circumstances in each market. In the case of the kinked demand curve, the optimum price is the prevailing price at the kink in the demand function. However, as previously noted, the kinked demand curve analysis does not provide insight into what established the prevailing price in the first place.

Perhaps the case of the dominant firm, with the other firms following the price leader, is the most obvious. In that case, the optimal price is determined at the output level where $M R=M C$. The profit-maximizing price is then determined by the output position of the segment of the demand function faced by the dominant firm. The price followers have little incentive to change the leader's price. In the case of the Cournot assumption, each firm assumes that the other firms will not alter their output following the dominant firm's selection of its price and output level.

Therefore, again, the optimum price is determined by the output level where MR = MC. In the case of the Nash equilibrium, each firm will react to the circumstances it faces, maximizing its own profit. These adjustments continue until there are stable prices and levels of output. Because of the interdependence, there is no certainty as to the individual firm's price and output level.

\section{Factors Affecting Long-Run Equilibrium in Oligopoly Markets}
Long-run economic profits are possible for firms operating in oligopoly markets. However, history has shown that, over time, the market share of the dominant firm declines. Profits attract entry by other firms into the oligopoly market. Over time, the marginal costs of the entrant firms decrease because they adopt more efficient production techniques, the dominant firm's demand and marginal revenue shrink, and the profitability of the dominant firm declines. In the early 1900s, J.P. Morgan, Elbert Gary, Andrew Carnegie, and Charles M. Schwab created the United States Steel Corporation (US Steel). When it was first formed in 1901, US Steel controlled 66 percent of the market. By 1920, US Steel's market share had declined to 46 percent, and by 1925 its market share was 42 percent.

In the long run, optimal pricing strategy must include the reactions of rival firms. History has proven that pricing wars should be avoided because any gains in market share are temporary. Decreasing prices to drive competitors away lowers total revenue to all participants in the oligopoly market. Innovation may be a way-though sometimes an uneconomical one-to maintain market leadership.

\section{OLIGOPOLIES: APPEARANCE VERSUS BEHAVIOR}
When is an oligopoly not an oligopoly? There are two extreme cases of this situation. A normal oligopoly has a few firms producing a differentiated good, and this differentiation gives them pricing power.

At one end of the spectrum, we have the oligopoly with a credible threat of entry. In practice, if the oligopolists are producing a good or service that can be easily replicated, has limited economies of scale, and is not protected by brand recognition or patents, they will not be able to charge high prices. The easier it is for a new supplier to enter the market, the lower the margins. In practice, this oligopoly will behave very much like a perfectly competitive market.

At the opposite end of the spectrum, we have the case of the cartel. Here, the oligopolists collude and act as if they were a single firm. In practice, a very effective cartel enacts a cooperative strategy. As shown in Section 5.1, instead of going to a Nash equilibrium, the cartel participants go to the more lucrative (for them) cooperative equilibrium.

A cartel may be explicit (that is, based on a contract) or implicit (based on signals). An example of signals in a duopoly would be that one of the firms reduces its prices and the other does not. Because the firm not cutting prices refuses to start a price war, the firm that cut prices may interpret this signal as a "suggestion" to raise prices to a higher level than before, so that profits may increase for both.

\section{MONOPOLY MARKETS: DEMAND/SUPPLY AND}
 OPTIMAL PRICE/OUTPUTdescribe characteristics of perfect competition, monopolistic competition, oligopoly, and pure monopoly

explain relationships between price, marginal revenue, marginal cost, economic profit, and the elasticity of demand under each market structure describe a firm's supply function under each market structure describe and determine the optimal price and output for firms under each market structure

explain factors affecting long-run equilibrium under each market structure describe pricing strategy under each market structure

Monopoly market structure is at the opposite end of the spectrum from perfect competition. For various reasons, there are significant barriers to entry such that a single firm produces a highly specialized product and faces no threat of competition. There are no good substitutes for the product in the relevant market, and the market demand function is the same as the individual firm's demand schedule. The distinguishing characteristics of monopoly are that a single firm represents the market and significant barriers to entry exist. Exhibit 1 identified the characteristics of monopoly markets:

\begin{enumerate}
  \item There is a single seller of a highly differentiated product.

  \item The product offered by the seller has no close substitute.

  \item Entry into the market is very difficult, with high costs and significant barriers to competition.

  \item The firm has considerable pricing power. 5. The product is differentiated through non-price strategies such as advertising.

\end{enumerate}

Monopoly markets are unusual. With a single seller dominating the market, power over price decisions is significant. For a single seller to achieve this power, there must be factors that allow the monopoly to exist. One obvious source of monopoly power would be a patent or copyright that prevents other firms from entering the market. Patent and copyright laws exist to reward intellectual capital and investment in research and development. In so doing, they provide significant barriers to entry.

Another possible source of market power is control over critical resources used for production. One example is De Beers Consolidated Mines Limited. De Beers owned or controlled all diamond mining operations in South Africa and established pricing agreements with other important diamond producers. In doing so, De Beers was able to control the prices for cut diamonds for decades. Technically, De Beers was a near-monopoly dominant firm rather than a pure monopoly, although its pricing procedure for cut diamonds resembled monopoly behavior.

Perhaps the most common form of monopolistic market power occurs as the result of government-controlled authorization. In most urban areas, a single source of water and sewer services is offered. In some cases, these services are offered by a government-controlled entity. In other cases, private companies provide the services under government regulation. Such "natural" monopolies require a large initial investment that benefits from economies of scale; therefore, government may authorize a single seller to provide a certain service because having multiple sellers would be too costly. For example, electricity in most markets is provided by a single seller. Economies of scale result when significant capital investment benefits from declining long-run average costs. In the case of electricity, a large gas-fueled power plant producing electricity for a large area is substantially more efficient than having a small diesel generator for every building. That is, the average cost of generating and delivering a kilowatt of electricity will be substantially lower with the single power station, but the initial fixed cost of building the power station and the lines delivering electricity to each home, factory, and office will be very high.

In the case of natural monopolies, limiting the market to a single seller is considered beneficial to society. One water and sewer system is deemed better for the community than dozens of competitors because building multiple infrastructures for running water and sewer service would be particularly expensive and complicated. One electrical power grid supplying electricity for a community can make large capital investments in generating plants and lower the long-run average cost, while multiple power grids would lead to a potentially dangerous maze of wires. Clearly, not all monopolies are in a position to make significant economic profits. Regulators, such as public utility commissions in the United States, attempt to determine what a normal return for the monopoly owners' investment should be, and they set prices accordingly. Nevertheless, monopolists attempt to maximize profits.

Not all monopolies originate from "natural" barriers. For some monopolists, barriers to entry do not derive from increasing returns to scale. We mentioned that marketing and brand loyalty are sources of product differentiation in monopolistic competition. In some highly successful cases, strong brand loyalty can become a formidable barrier to entry. For example, if the Swiss watchmaker Rolex is unusually successful in establishing brand loyalty, so that its customers think there is no close substitute for its product, then the company will have monopoly-like pricing power over its market.

The final potential source of market power is the increasing returns associated with network effects. Network effects result from synergies related to increasing market penetration. By achieving a critical level of adoption, Microsoft was able to extend its market power through the network effect-for example, because most computer users know how to use Microsoft Word. Therefore, for firms, Word is cheaper to adopt than other programs because almost every new hire will be proficient in using the software and will need no further training. At some level of market share, a network-based product or service (think of Facebook or eBay) reaches a point where each additional share point increases the probability that another user will adopt. ${ }^{17}$ These network effects increase the value to other potential adopters. In Microsoft's case, the network effects crowded out other potential competitors, including Netscape's internet browser, that might have led to applications bypassing Windows. Eventually, Microsoft's operating system's market share reached 92 percent of the global market. Similar situations occur in financial markets: If a publicly listed share or a derivative contract is more frequently traded on a certain exchange, market participants wishing to sell or buy the security will go to the more liquid exchange because they expect to find a better price and faster execution there.

\section{Demand Analysis in Monopoly Markets}
The monopolist's demand schedule is the aggregate demand for the product in the relevant market. Because of the income effect and the substitution effect, demand is negatively related to price, as usual. The slope of the demand curve is negative and therefore downward sloping. The general form of the demand relationship is

$$
Q_{D}=a-b P \quad \text { or, rewritten, } \quad P=a / b-(1 / b) Q_{D}
$$

Therefore, total revenue $=\mathrm{TR}=P \times Q=(a / b) Q_{D}-(1 / b) Q_{D}^{2}$

Marginal revenue is the change in revenue given a change in the quantity demanded. Because an increase in quantity requires a lower price, the marginal revenue schedule is steeper than the demand schedule. If the demand schedule is linear, then the marginal revenue curve is twice as steep as the demand schedule. ${ }^{18}$

$$
\mathrm{MR}=\Delta \mathrm{TR} / \Delta Q=(a / b)-(2 / b) Q_{D}
$$

The demand and marginal revenue relationship is expressed in Exhibit 18.

\section{Exhibit 18: Monopolist's Demand and Marginal Revenue}
\begin{center}
\includegraphics[max width=\textwidth]{2023_05_04_ccb578c5089fd41615beg-113}
\end{center}

17 When a network-based device reaches a 30 percent share, the next 50 percentage points are cheaper to promote, according to McGuigan, Moyer, and Harris (2016).

18 Marginal revenue can be found using the technique shown in Section 3.1 or, for readers who are familiar with calculus, by taking the derivative of the total revenue function: $\mathrm{MR}=\Delta \mathrm{TR} / \Delta Q=(a / b)-(2 / b) Q_{D}$. Suppose a company operating on a remote island is the single seller of natural gas. Demand for its product can be expressed as

$$
\begin{aligned}
Q_{D} & =400-0.5 P, \text { which can be rearranged as } \\
P & =800-2 Q_{D}
\end{aligned}
$$

Total revenue is $P \times Q=\mathrm{TR}=800 Q_{D}-2 Q_{D}^{2}$, and marginal revenue is $M R=800$ $-4 Q_{D} \cdot 19$

In Exhibit 18, the demand curve's intercept is 800 and the slope is -2 . The marginal revenue curve in Exhibit 18 has an intercept of 800 and a slope of -4 .

Average revenue is TR $/ Q_{D}$; therefore, $A R=800-2 Q_{D}$, which is the same as the demand function. In the monopoly market model, average revenue is the same as the market demand schedule.

\section{Supply Analysis in Monopoly Markets}
A monopolist's supply analysis is based on the firm's cost structure. As in the market structures of monopolistic competition and oligopoly, the monopolist does not have a well-defined supply function that determines the optimal output level and the price to charge. The optimal output is the profit-maximizing output level. The profit-maximizing level of output occurs where marginal revenue equals marginal cost, $M R=M C$.

Assume the natural gas company has determined that its total cost can be expressed as

$\mathrm{TC}=20,000+50 Q+3 Q^{2}$

Marginal cost is $\Delta \mathrm{TC} / \Delta Q=\mathrm{MC}=50+6 Q .{ }^{20}$

Supply and demand can be combined to determine the profit-maximizing level of output. Exhibit 19 combines the monopolist's demand and cost functions.

Exhibit 19: Monopolist's Demand, Marginal Revenue, and Cost Structures

\begin{center}
\includegraphics[max width=\textwidth]{2023_05_04_ccb578c5089fd41615beg-114}
\end{center}

$19 \mathrm{MR}=\Delta \mathrm{TR} / \Delta Q=800-4 Q$.

20 The marginal cost equation can be found in this case by applying the technique used to find the marginal revenue equation in Section 3.1, or by taking the derivative of the total cost function. In Exhibit 19, the demand and marginal revenue functions are clearly defined by the aggregate market. However, the monopolist does not have a supply curve. The quantity that maximizes profit is determined by the intersection of MC and $M R, Q_{D E}$.

The price consumers are willing to pay for this level of output is $P_{E}$, as determined by the demand curve, $P_{E}$.

The profit-maximizing level of output is $M R=M C: 800-4 Q_{D}=50+6 Q_{D}$; therefore, $Q_{D}=75$ when profit is maximized.

Total profit equals total revenue minus total cost:

$$
\pi=800 Q-2 Q_{D}^{2}-\left(20,000+50 Q_{D}+3 Q_{D}^{2}\right)=-20,000+750 Q_{D}-5 Q_{D}^{2}
$$

Profit is represented by the difference between the area of the rectangle $Q_{D E} \times P_{E}$, representing total revenue, and the area of the rectangle $Q_{D E} \times A C_{E}$, representing total cost.

\section{MONOPOLISTS AND THEIR INCENTIVES}
In theoretical models, which usually take product quality and technology as given, monopolists can choose to vary either price or quantity. In real life, they also can vary their product.

A monopolist can choose to limit quality if producing a higher-quality product is costly and higher quality does not increase profits accordingly. For example, the quality of domestically produced cars in most developed countries improved dramatically once imports became more available. Before the opening of borders to imports, the single incumbent that dominated the market (for example, Fiat in Italy) or the small group of incumbents acting as a collusive oligopoly (such as the Detroit "Big Three" in the United States) were the effective monopolists of their domestic automobile markets. Rust corrosion, limited reliability, and poor gas mileage were common. ${ }^{21}$

Similarly, regulated utilities may have limited incentives to innovate. Several studies, including Gómez-lbáñez (2003), have found that state-owned and other monopoly telephone utilities tended to provide very poor service before competition was introduced. Poor service may not be limited to poor connection quality but may also include extensive delays in adding new users and limited introduction of new services, such as caller ID or automatic answering services.

Intuitively, a monopolist will not spend resources on quality control, research and development, or even customer relations unless there is a threat of entry of a new competitor or unless there is a clear link between such expenses and a profit increase. In contrast, in competitive markets, including oligopoly, innovation and quality are often ways to differentiate the product and increase profits.

\section{Optimal Price and Output in Monopoly Markets}
Continuing the natural gas example from above, the total profit function can be solved using the quadratic formula. ${ }^{22}$ Another method to solve the profit function is to evaluate $\Delta \pi / \Delta Q_{D}$ and set it equal to zero. This identifies the point at which profit is unaffected by changes in output. ${ }^{23}$ Of course, this will give the same result as we found by equating marginal revenue with marginal cost. The monopoly will maximize profits when $Q^{*}=75$ units of output and the price is set from the demand curve at 650 .

$$
P^{*}=800-2(75)=650 \text { per unit }
$$

To find total maximum profits, substitute these values into the profit function above:

21 For more on this topic, see Banker, Khosla, and Sinha (1998).

22 The quadratic formula, where $a Q^{2}+b Q+c=0$, is $Q=\left\{-b \pm \sqrt{\left(b^{2}-4 a c\right)}\right\} / 2 a$.

23 Maximum profit occurs where $\Delta \pi / \Delta Q_{D}=0=750-10 Q_{D}$. Therefore, profits are maximized at $Q_{D}=75$

$$
\pi=-20,000+750 Q_{D}-5 Q_{D}^{2}=-20,000+750(75)-5\left(75^{2}\right)=8,125
$$

Note that the price and output combination that maximizes profit occurs in the elastic portion of the demand curve in Exhibit 19. This must be so because marginal revenue and marginal cost will always intersect where marginal revenue is positive. This fact implies that quantity demanded responds more than proportionately to prices changes, i.e. demand is elastic, at the point at which $M C=M R$. As noted earlier, the relationship between marginal revenue and price elasticity, $E_{P}$, is:

$$
\mathrm{MR}=P\left[1-1 / E_{P}\right]
$$

In monopoly, $\mathrm{MR}=\mathrm{MC}$; therefore,

$$
P\left[1-1 / E_{P}\right]=\mathrm{MC}
$$

The firm can use this relationship to determine the profit-maximizing price if the firm knows its cost structure and the price elasticity of demand, $E_{P}$. For example, assume the firm knows that its marginal cost is constant at 75 and recent market analysis indicates that price elasticity is estimated to be 1.5 . The optimal price is solved as

$$
\begin{aligned}
& P[1-1 / 1.5]=75 \text { and } \\
& P=225
\end{aligned}
$$

Exhibit 19 indicated that the monopolist wants to produce at $Q_{E}$ and charge the price of $P_{E}$. Suppose this is a natural monopoly that is operating as a government franchise under regulation. Natural monopolies are usually found where production is based on significant economies of scale and declining cost structure in the market. Examples include electric power generation, natural gas distribution, and the water and sewer industries. These are often called public utilities. Exhibit 20 illustrates such a market in long-run equilibrium.

\section{Exhibit 20: Natural Monopoly in a Regulated Pricing Environment}
\begin{center}
\includegraphics[max width=\textwidth]{2023_05_04_ccb578c5089fd41615beg-116}
\end{center}

In Exhibit 20, three possible pricing and output solutions are presented. The first is what the monopolist would do without regulation: The monopolist would seek to maximize profits by producing $Q_{M}$ units of the product, where long-run marginal cost equals marginal revenue, $L R M C=M R$. To maximize profits, the monopolist would raise the price to the level the demand curve will accept, $P_{M}$. In perfect competition, the price and output equilibrium occurs where price is equal to the marginal cost of producing the incremental unit of the product. In a competitive market, the quantity produced would be higher, $Q_{C}$, and the price lower, $P_{C}$. For this regulated monopoly, the competitive solution would be unfair because at output $Q_{C}$, the price $P_{C}$ would not cover the average cost of production. One possibility is to subsidize the difference between the long-run average cost, LRAC, and the competitive price, $P_{C}$, for each unit sold.

Another solution is for the regulator to set the price at the point where long-run average cost equals average revenue. Recall that the demand curve represents the average revenue the firm receives at each output level. The government regulator will attempt to determine the monopolistic firm's long-run average cost and set the output and price so that the firm receives a fair return on the owners' invested capital. The regulatory solution is output level $Q_{R}$, with the price set at $P_{R}$. Therefore, the regulatory solution is found between the unregulated monopoly equilibrium and the competitive equilibrium.

\section{PRICE DISCRIMINATION AND CONSUMER SURPLUS}
describe pricing strategy under each market structure

Monopolists can either be more or less effective in taking advantage of their market structure. At one extreme, we have a monopolist that charges prices and supplies quantities that are the same as they would be in perfect competition; this scenario may be a result of regulation or threat of entry (if the monopolist charged more, another company could come in and price the former monopolist out of the market). At the opposite extreme, hated by all consumers and economists, is the monopolist that extracts the entire consumer surplus. This scenario is called first-degree price discrimination, where a monopolist can charge each customer the highest price the customer is willing to pay. This is called price discrimination because the monopolist charges a different price to each client. How can this be? For example, if the monopolist knows the exact demand schedule of the customer, then the monopolist can capture the entire consumer surplus. In practice, the monopolist can measure how often the product is used and charges the customer the highest price the consumer is willing to pay for that unit of good. Another possibility is that public price disclosure is non-existent, so that no customer knows what the other customers are paying. Interestingly, not every consumer is worse off in this case, because some consumers may be charged a price that is below that of monopoly, as long as the marginal revenue exceeds the marginal cost.

In second-degree price discrimination the monopolist offers a menu of quantity-based pricing options designed to induce customers to self-select based on how highly they value the product. Such mechanisms include volume discounts, volume surcharges, coupons, product bundling, and restrictions on use. In practice, producers can use not just the quantity but also the quality (e.g., "professional grade") to charge more to customers that value the product highly.

Third-degree price discrimination happens when customers are segregated by demographic or other traits. For example, some econometric software is licensed this way: A student version can handle only small datasets and is sold for a low price; a professional version can handle very large datasets and is sold at a much higher price because corporations need to compute the estimates for their business and are therefore willing to pay more for a license. Another example is that airlines know that passengers who want to fly somewhere and come back the same day are most likely business people; therefore, one-day roundtrip tickets are generally more expensive than tickets with a return flight at a later date or over a weekend.

Price discrimination has many practical applications when the seller has pricing power. The best way to understand how this concept works is to think of consumer surplus: As seen in this reading, a consumer may be willing to pay more for the first unit of a good, but to buy a second unit she will want to pay a lower price, thus getting a better deal on the first unit. In practice, sellers can sometimes use income and substitution effects to their advantage. Think of something you often buy, perhaps lunch at your favorite café. How much would you be willing to pay for a "lunch club membership card" that would allow you to purchase lunches at, say, half price? If the café could extract from you the maximum amount each month that you would be willing to pay for the half-price option, then it would successfully have removed the income effect from you in the form of a monthly fixed fee. Notice that a downward-sloping demand curve implies that you would end up buying more lunches each month than before you purchased the discount card, even though you would be no better or worse off than before. This is a way that sellers are sometimes able to extract consumer surplus by means of creative pricing schemes. It's a common practice among big-box retailers, sports clubs, and other users of what is called "two-part tariff pricing," as in the example below.

\section{EXAMPLE 3}
\section{Price Discrimination}
Nicole's monthly demand for visits to her health club is given by the following equation: $Q_{D}=20-4 P$, where $Q_{D}$ is visits per month and $P$ is euros per visit. The health club's marginal cost is fixed at $€ 2$ per visit.

\begin{enumerate}
  \item Draw Nicole's demand curve for health club visits per month.
\end{enumerate}

\section{Solution to 1:}
$Q_{D}=20-4 P$, so when $P=0, Q_{D}=20$. Inverting, $P=5-0.25 Q_{D}$, so when $Q$ $=0, P=5$.

\begin{enumerate}
  \setcounter{enumi}{1}
  \item If the club charged a price per visit equal to its marginal cost, how many visits would Nicole make per month?
\end{enumerate}

Solution to 2:

$Q_{D}=20-4(2)=12$. Nicole would make 12 visits per month at a price of $€ 2$ per visit.

\begin{enumerate}
  \setcounter{enumi}{2}
  \item How much consumer surplus would Nicole enjoy at that price?
\end{enumerate}

\section{Solution to 3:}
Nicole's consumer surplus can be measured as the area under her demand curve and above the price she pays for a total of 12 visits, or $(0.5)(12)(3)=$ 18. Nicole would enjoy a consumer surplus of $€ 18$ per month. 4. How much could the club charge Nicole each month for a membership fee?

\section{Solution to 4:}
The club could extract all of Nicole's consumer surplus by charging her a monthly membership fee of $€ 18$ plus a per-visit price of $€ 2$. This pricing method is called a two-part tariff because it assesses one price per unit of the item purchased plus a per-month fee (sometimes called an "entry fee") equal to the buyer's consumer surplus evaluated at the per-unit price.

\section{MONOPOLY MARKETS: LONG-RUN EQUILIBRIUM}
explain factors affecting long-run equilibrium under each market structure

The unregulated monopoly market structure can produce economic profits in the long run. In the long run, all factors of production are variable, while in the short run, some factors of production are fixed. Generally, the short-run factor that is fixed is the capital investment, such as the factory, machinery, production technology, available arable land, and so forth. The long-run solution allows for all inputs, including technology, to change. In order to maintain a monopoly market position in the long run, the firm must be protected by substantial and ongoing barriers to entry. If the monopoly position is the result of a patent, then new patents must be continuously added to prevent the entry of other firms into the market.

For regulated monopolies, such as natural monopolies, there are a variety of long-run solutions. One solution is to set the price equal to marginal cost, $P=\mathrm{MC}$. However, that price will not likely be high enough to cover the average cost of production, as Exhibit 19 illustrated. The answer is to provide a subsidy sufficient to compensate the firm. The national rail system in the United States, Amtrak, is an example of a regulated monopoly operating with a government subsidy.

National ownership of the monopoly is another solution. Nationalization of the natural monopoly has been a popular solution in Europe and other parts of the world. The United States has generally avoided this potential solution. One problem with this arrangement is that once a price is established, consumers are unwilling to accept price increases, even as factor costs increase. Politically, raising prices on products from government-owned enterprises is highly unpopular.

Establishing a governmental entity that regulates an authorized monopoly is another popular solution. Exhibit 19 illustrated the appropriate decision rule. The regulator sets price equal to long-run average cost, $P_{R}=$ LRAC. This solution assures that investors will receive a normal return for the risk they are taking in the market. Given that no other competitors are allowed, the risk is lower than in a highly competitive market environment. The challenge facing the regulator is determining the authentic risk-related return and the monopolist's realistic long-run average cost.

The final solution is to franchise the monopolistic firm through a bidding war. Again, the public goal is to select the winning firm based on price equaling long-run average cost. Retail outlets at rail stations and airports and concession outlets at stadiums are examples of government franchises. The long-run success of the monopoly franchise depends on its ability to meet the goal of pricing its products at the level of its long-run average cost.

\section{EXAMPLE 4}
\section{Monopolies and Efficiency}
\begin{enumerate}
  \item Are monopolies always inefficient?
\end{enumerate}

A. No, because if they charge more than average cost they are nationalized.

B. Yes, because they charge all consumers more than perfectly competitive markets would.

C. No, because economies of scale and regulation (or threat of entry) may give a better outcome for buyers than perfect competition.

Solution:

$\mathrm{C}$ is correct. Economies of scale and regulation may make monopolies more efficient than perfect competition.

\section{IDENTIFICATION OF MARKET STRUCTURE}
describe the use and limitations of concentration measures in identifying market structure

identify the type of market structure within which a firm operates

Monopoly markets and other situations where companies have pricing power can be inefficient because producers constrain output to cause an increase in prices. Therefore, there will be less of the good being consumed and it will be sold at a higher price, which is generally inefficient for the overall market. As a result, many countries have introduced competition law to regulate the degree of competition in many industries.

Market power in the real world is not always as clear as it is in textbook examples. Governments and regulators often have the difficult task of measuring market power and establishing whether a firm has a dominant position that may resemble a monopoly. A few historical examples of this are as follows:

\begin{enumerate}
  \item In the 1990s, US regulators prosecuted agricultural corporation Archer Daniels Midland for conspiring with Japanese competitors to fix the price of lysine, an amino acid used as an animal feed additive. The antitrust action resulted in a settlement that involved over US\$100 million in fines paid by the cartel members.

  \item In the 1970s, US antitrust authorities broke up the local telephone monopoly, leaving AT\&T the long-distance business (and opening that business to competitors), and required AT\&T to divest itself of the local telephone companies it owned. This antitrust decision brought competition, innovation, and lower prices to the US telephone market.

  \item European regulators (specifically, the European Commission) have affected the mergers and monopoly positions of European corporations (as in the case of the companies Roche, Rhone-Poulenc, and BASF, which were at the center of a vitamin price-fixing case) as well as non-European companies (such as Intel) that do business in Europe. Moreover, the merger between the US company General Electric and the European company Honeywell was denied by the European Commission on grounds of excessive market concentration.

\end{enumerate}

Quantifying excessive market concentration is difficult. Sometimes, regulators need to measure whether something that has not yet occurred might generate excessive market power. For example, a merger between two companies might allow the combined company to be a monopolist or quasi monopolist in a certain market.

A financial analyst hearing news about a possible merger should always consider the impact of competition law (sometimes called antitrust law)-that is, whether a proposed merger may be blocked by regulators in the interest of preserving a competitive market.

\section{Econometric Approaches}
How should one measure market power? The theoretical answer is to estimate the elasticity of demand and supply in a market. If demand is very elastic, the market must be very close to perfect competition. If demand is rigid (inelastic), companies may have market power. This is the approach taken in the cellophane case mentioned in Section 3.1.2.

From the econometric point of view, this estimation requires some attention. The problem is that observed price and quantity are the equilibrium values of price and quantity and do not represent the value of either supply or demand. Technically, this is called the problem of endogeneity, in the sense that the equilibrium price and quantity are jointly determined by the interaction of demand and supply. Therefore, to have an appropriate estimation of demand and supply, we will need to use a model with two equations, namely, an equation of demanded quantity (as a function of price, income of the buyers, and other variables) and an equation of supplied quantity (as a function of price, production costs, and other variables). The estimated parameters will then allow us to compute elasticity.

Regression analysis is useful in computing elasticity but requires a large number of observations. Therefore, one may use a time-series approach and, for example, look at 20 years of quarterly sales data for a market. However, the market structure may have changed radically over those 20 years, and the estimated elasticity may not apply to the current situation. Moreover, the supply curve may change due to a merger among large competitors, and the estimation based on past data may not be informative regarding the future state of the market post merger.

An alternative approach is a cross-sectional regression analysis. Instead of looking at total sales and average prices in a market over time (the time-series approach mentioned above), we can look at sales from different companies in the market during the same year, or even at single transactions from many buyers and companies. Clearly, this approach requires substantial data-gathering effort, and therefore, this estimation method can be complicated. Moreover, different specifications of the explanatory variables (for example, using total GDP rather than median household income or per-capita GDP to represent income) may sometimes lead to dramatically different estimates.

\section{Simpler Measures}
Trying to avoid the above drawbacks, analysts often use simpler measures to estimate elasticity. The simplest measure is the concentration ratio, which is the sum of the market shares of the largest $N$ firms. To compute this ratio, one would, for example, add the sales values of the largest 10 firms and divide this figure by total market sales. This number is always between zero (perfect competition) and 100 percent (monopoly). The main advantage of the concentration ratio is that it is simple to compute, as shown above. The disadvantage is that it does not directly quantify market power. In other words, is a high concentration ratio a clear signal of monopoly power? The analysis of entry in Section 2 explains clearly that this is not the case: A company may be the only incumbent in a market, but if the barriers to entry are low, the simple presence of a potential entrant may be sufficient to convince the incumbent to behave like a firm in perfect competition. For example, a sugar wholesaler may be the only one in a country, but the knowledge that other large wholesalers in the food industry might easily add imported sugar to their range of products should convince the sugar wholesaler to price its product as if it were in perfect competition.

Another disadvantage of the concentration ratio is that it tends to be unaffected by mergers among the top market incumbents. For example, if the largest and second-largest incumbents merge, the pricing power of the combined entity is likely to be larger than that of the two pre-existing companies. But the concentration ratio may not change much.

\section{CALCULATING THE CONCENTRATION RATIO}
Suppose there are eight producers of a certain good in a market. The largest producer has 35 percent of the market, the second largest has 25 percent, the third has 20 percent, the fourth has 10 percent, and the remaining four have 2.5 percent each. If we computed the concentration ratio of the top three producers, it would be $35+25+20$ $=80$ percent, while the concentration ratio of the top four producers would be $35+25$ $+20+10=90$ percent.

If the two largest companies merged, the new concentration ratio for the top three producers would be 60 (the sum of the market shares of the merged companies) $+20+$ $10=90$ percent, and the concentration ratio for the four top producers would be 92.5 percent. Therefore, this merger affects the concentration ratio very mildly, even though it creates a substantial entity that controls 60 percent of the market.

For example, the effect of consolidation in the US retail gasoline market has resulted in increasing degrees of concentration. In 1992, the top four companies in the US retail gasoline market shared 33 percent of the market. By 2001, the top four companies controlled 78 percent of the market (Exxon Mobil 24 percent, Shell 20 percent, BP/Amoco/ Arco 18 percent, and Chevron/Texaco 16 percent).

To avoid the known issues with concentration ratios, economists O.C. Herfindahl and A.O. Hirschman suggested an index where the market shares of the top $N$ companies are first squared and then added. If one firm controls the whole market (a monopoly), the Herfindahl-Hirschman index (HHI) equals 1. If there are $M$ firms in the industry with equal market shares, then the $\mathrm{HHI}$ equals $(1 / M)$. This provides a useful gauge for interpreting an HHI. For example, an HHI of 0.20 would be analogous to having the market shared equally by 5 firms.

The HHI for the top three companies in the example in the box above would be $0.35^{2}+0.25^{2}+0.20^{2}=0.225$ before the merger, while after the merger, it would be $0.60^{2}+0.20^{2}+0.10^{2}=0.410$, which is substantially higher than the initial 0.225 . The $\mathrm{HHI}$ is widely used by competition regulators; however, just like the concentration ratio, the HHI does not take the possibility of entry into account, nor does it consider the elasticity of demand. Therefore, the HHI has limited use for a financial analyst trying to estimate the potential profitability of a company or group of companies.

\section{EXAMPLE 5}
\section{The Herfindahl-Hirschman Index}
\begin{enumerate}
  \item Suppose a market has 10 suppliers, each of them with 10 percent of the market. What are the concentration ratio and the HHI of the top four firms?
A. Concentration ratio 4 percent and $\mathrm{HHI} 40$
B. Concentration ratio 40 percent and HHI 0.4
C. Concentration ratio 40 percent and $\mathrm{HHI} 0.04$
\end{enumerate}

Solution:

$\mathrm{C}$ is correct. The concentration ratio for the top four firms is $10+10+10+$ $10=40$ percent, and the $\mathrm{HHI}$ is $0.10^{2} \times 4=0.01 \times 4=0.04$.

\section{SUMMARY}
In this reading, we have surveyed how economists classify market structures. We have analyzed the distinctions between the different structures that are important for understanding demand and supply relations, optimal price and output, and the factors affecting long-run profitability. We also provided guidelines for identifying market structure in practice. Among our conclusions are the following:

\begin{itemize}
  \item Economic market structures can be grouped into four categories: perfect competition, monopolistic competition, oligopoly, and monopoly.

  \item The categories differ because of the following characteristics: The number of producers is many in perfect and monopolistic competition, few in oligopoly, and one in monopoly. The degree of product differentiation, the pricing power of the producer, the barriers to entry of new producers, and the level of non-price competition (e.g., advertising) are all low in perfect competition, moderate in monopolistic competition, high in oligopoly, and generally highest in monopoly.

  \item A financial analyst must understand the characteristics of market structures in order to better forecast a firm's future profit stream.

  \item The optimal marginal revenue equals marginal cost. However, only in perfect competition does the marginal revenue equal price. In the remaining structures, price generally exceeds marginal revenue because a firm can sell more units only by reducing the per unit price.

  \item The quantity sold is highest in perfect competition. The price in perfect competition is usually lowest, but this depends on factors such as demand elasticity and increasing returns to scale (which may reduce the producer's marginal cost). Monopolists, oligopolists, and producers in monopolistic competition attempt to differentiate their products so that they can charge higher prices.

  \item Typically, monopolists sell a smaller quantity at a higher price. Investors may benefit from being shareholders of monopolistic firms that have large margins and substantial positive cash flows. - Competitive firms do not earn economic profit. There will be a market compensation for the rental of capital and of management services, but the lack of pricing power implies that there will be no extra margins.

  \item While in the short run firms in any market structure can have economic profits, the more competitive a market is and the lower the barriers to entry, the faster the extra profits will fade. In the long run, new entrants shrink margins and push the least efficient firms out of the market.

  \item Oligopoly is characterized by the importance of strategic behavior. Firms can change the price, quantity, quality, and advertisement of the product to gain an advantage over their competitors. Several types of equilibrium (e.g., Nash, Cournot, kinked demand curve) may occur that affect the likelihood of each of the incumbents (and potential entrants in the long run) having economic profits. Price wars may be started to force weaker competitors to abandon the market.

  \item Measuring market power is complicated. Ideally, econometric estimates of the elasticity of demand and supply should be computed. However, because of the lack of reliable data and the fact that elasticity changes over time (so that past data may not apply to the current situation), regulators and economists often use simpler measures. The concentration ratio is simple, but the HHI, with little more computation required, often produces a better figure for decision making.

\end{itemize}

\section{REFERENCES}
Banker, R.D., I. Khosla, K.K. Sinha. 1998. "Quality and Competition." Management Science, vol. 44, no. 9:1179-1192. 10.1287/mnsc.44.9.1179

Chamberlin, Edward H. 1933. The Theory of Monopolistic Competition. Cambridge, MA: Harvard University Press.

Dorsey, Pat. 2004. The Five Rules for Successful Stock Investing: Morningstar's Guide to Building Wealth and Winning in the Market. Hoboken, NJ: John Wiley \& Sons.

Friedman, Thomas L. 2006. The World Is Flat: A Brief History of the Twenty-first Century. New York: Farrar, Straus and Giroux

Fudenberg, Drew, Jean Tirole. 1984. “The Fat Cat Effect, the Puppy Dog Ploy and the Lean and Hungry Look." American Economic Review, vol. 74, no. 2:361-366.

Kelly, Anthony. 2011. Decision Making Using Game Theory: An Introduction for Managers. Cambridge, UK: Cambridge University Press.

Krugman, Paul R. 1989. "Industrial Organization and International Trade." In Handbook of Industrial Organization, vol. 2. Edited by Richard Schmalensee and Robert Willig. Amsterdam: Elsevier B.V.

McGuigan, James R., R. Charles Moyer, Frederick H. Harris. 2016. Managerial Economics: Applications, Strategy and Tactics. 14th ed. Mason, $\mathrm{OH}$ : Thomson South-Western.

Nicholson, Walter, Christopher M. Snyder. 2016. Microeconomic Theory: Basic Principles and Extensions. 12th ed. Mason, OH: Thomson South-Western.

Schumpeter, Joseph A. 1942. Capitalism, Socialism and Democracy. New York: HarperCollins. von Stackelberg, Heinrich. 1952. The Theory of the Market Economy. New York: Oxford University Press.

\section{PRACTICE PROBLEMS}
\begin{enumerate}
  \item A market structure characterized by many sellers with each having some pricing power and product differentiation is best described as:
A. oligopoly.
B. perfect competition.
C. monopolistic competition.

  \item A market structure with relatively few sellers of a homogeneous or standardized product is best described as:
A. oligopoly.
B. monopoly.
C. perfect competition.

  \item The demand schedule in a perfectly competitive market is given by $\mathrm{P}=93-1.5 \mathrm{Q}$ (for $\mathrm{Q} \leq 62$ ) and the long-run cost structure of each company is:

\end{enumerate}

Total cost: $256+2 Q+4 Q^{2}$

Average cost: $256 / Q+2+4 Q$

Marginal cost: $2+8 Q$

New companies will enter the market at any price greater than:
A. 8 .
B. 66 .
C. 81 .

\begin{enumerate}
  \setcounter{enumi}{3}
  \item If companies earn economic profits in a perfectly competitive market, over the long run the supply curve will most likely:
\end{enumerate}

A. shift to the left.

B. shift to the right.

C. remain unchanged.

\begin{enumerate}
  \setcounter{enumi}{4}
  \item A company doing business in a monopolistically competitive market will most likely maximize profits when its output quantity is set such that:
\end{enumerate}

A. average cost is minimized.

B. marginal revenue equals average cost.

C. marginal revenue equals marginal cost.

\begin{enumerate}
  \setcounter{enumi}{5}
  \item Oligopolistic pricing strategy most likely results in a demand curve that is:
\end{enumerate}

A. kinked.

B. vertical.
C. horizontal.

\begin{enumerate}
  \setcounter{enumi}{6}
  \item Collusion is less likely in a market when:
A. the product is homogeneous.
B. companies have similar market shares.
C. the cost structures of companies are similar.

  \item In an industry comprised of three companies, which are small-scale manufacturers of an easily replicable product unprotected by brand recognition or patents, the most representative model of company behavior is:
A. oligopoly.
B. perfect competition.
C. monopolistic competition.

  \item Deep River Manufacturing is one of many companies in an industry that make a food product. Deep River units are identical up to the point they are labeled. Deep River produces its labeled brand, which sells for $\$ 2.20$ per unit, and "house brands" for seven different grocery chains which sell for $\$ 2.00$ per unit. Each grocery chain sells both the Deep River brand and its house brand. The best characterization of Deep River's market is:
A. oligopoly.
B. perfect competition.
C. monopolistic competition.

  \item SigmaSoft and ThetaTech are the dominant makers of computer system software. The market has two components: a large mass-market component in which demand is price sensitive, and a smaller performance-oriented component in which demand is much less price sensitive. SigmaSoft's product is considered to be technically superior. Each company can choose one of two strategies:

\end{enumerate}

\begin{itemize}
  \item Open architecture (Open): Mass market focus allowing other software venders to develop products for its platform.

  \item Proprietary (Prop): Allow only its own software applications to run on its platform.

\end{itemize}

Depending upon the strategy each company selects, their profits would be:

\begin{center}
\includegraphics[max width=\textwidth]{2023_05_04_ccb578c5089fd41615beg-128}
\end{center}

The Nash equilibrium for these companies is:
A. proprietary for SigmaSoft and proprietary for ThetaTech.
B. open architecture for SigmaSoft and proprietary for ThetaTech.
C. proprietary for SigmaSoft and open architecture for ThetaTech.

\begin{enumerate}
  \setcounter{enumi}{10}
  \item Companies most likely have a well-defined supply function when the market structure is:
A. oligopoly.
B. perfect competition.
C. monopolistic competition.

  \item Aquarius, Inc. is the dominant company and the price leader in its market. One of the other companies in the market attempts to gain market share by undercutting the price set by Aquarius. The market share of Aquarius will most likely:

\end{enumerate}

A. increase.

B. decrease.

C. stay the same.

\begin{enumerate}
  \setcounter{enumi}{12}
  \item Over time, the market share of the dominant company in an oligopolistic market will most likely:
A. increase.
B. decrease.
C. remain the same.

  \item Market competitors are least likely to use advertising as a tool of differentiation in an industry structure identified as:
A. monopoly.
B. perfect competition.
C. monopolistic competition. 15. Upsilon Natural Gas, Inc. is a monopoly enjoying very high barriers to entry. Its marginal cost is $\$ 40$ and its average cost is $\$ 70$. A recent market study has determined the price elasticity of demand is 1.5 . The company will most likely set its price at:
A. $\$ 40$.
B. $\$ 70$,
C. $\$ 120$

  \item A government entity that regulates an authorized monopoly will most likely base regulated prices on:

\end{enumerate}

A. marginal cost.

B. long run average cost.

C. first degree price discrimination.

\begin{enumerate}
  \setcounter{enumi}{16}
  \item An analyst gathers the following market share data for an industry:
\end{enumerate}

\begin{center}
\begin{tabular}{lc}
\hline
Company & $\begin{array}{c}\text { Sales } \\ \text { (in millions of } € \text { ) }\end{array}$ \\
\hline
ABC & 300 \\
Brown & 250 \\
Coral & 200 \\
Delta & 150 \\
Erie & 100 \\
All others & 50 \\
\hline
\end{tabular}
\end{center}

The industry's four-company concentration ratio is closest to:
A. $71 \%$.
B. $86 \%$.
C. $95 \%$.

\begin{enumerate}
  \setcounter{enumi}{17}
  \item An analyst gathered the following market share data for an industry comprised of five companies:
\end{enumerate}

\begin{center}
\begin{tabular}{lc}
\hline
Company & Market Share (\%) \\
\hline
Zeta & 35 \\
Yusef & 25 \\
Xenon & 20 \\
Waters & 10 \\
Vlastos & 10 \\
\hline
\end{tabular}
\end{center}

The industry's three-firm Herfindahl-Hirschmann Index is closest to:
A. 0.185 .
B. 0.225 .
C. 0.235 . 19. One disadvantage of the Herfindahl-Hirschmann Index is that the index:
A. is difficult to compute.
B. fails to reflect low barriers to entry.
C. fails to reflect the effect of mergers in the industry.

\section{SOLUTIONS}
\begin{enumerate}
  \item C is correct. Monopolistic competition is characterized by many sellers, differentiated products, and some pricing power.

  \item A is correct. Few sellers of a homogeneous or standardized product characterizes an oligopoly.

  \item B is correct. The long-run competitive equilibrium occurs where $\mathrm{MC}=\mathrm{AC}=P$ for each company. Equating $M C$ and AC implies $2+8 Q=256 / Q+2+4 Q$.

\end{enumerate}

Solving for $Q$ gives $Q=8$. Equating $M C$ with price gives $P=2+8 Q=66$. Any price above 66 yields an economic profit because $P=\mathrm{MC}>\mathrm{AC}$, so new companies will enter the market.

\begin{enumerate}
  \setcounter{enumi}{3}
  \item B is correct. The economic profit will attract new entrants to the market and encourage existing companies to expand capacity.

  \item $\mathrm{C}$ is correct. The profit maximizing choice is the level of output where marginal revenue equals marginal cost.

  \item A is correct. The oligopolist faces two different demand structures, one for price increases and another for price decreases. Competitors will lower prices to match a price reduction, but will not match a price increase. The result is a kinked demand curve.

  \item B is correct. When companies have similar market shares, competitive forces tend to outweigh the benefits of collusion.

  \item B is correct. The credible threat of entry holds down prices and multiple incumbents are offering undifferentiated products.

  \item $\mathrm{C}$ is correct. There are many competitors in the market, but some product differentiation exists, as the price differential between Deep River's brand and the house brands indicates.

  \item C is correct. In the Nash model, each company considers the other's reaction in selecting its strategy. In equilibrium, neither company has an incentive to change its strategy. ThetaTech is better off with open architecture regardless of what SigmaSoft decides. Given this choice, SigmaSoft is better off with a proprietary platform. Neither company will change its decision unilaterally.

  \item B is correct. A company in a perfectly competitive market must accept whatever price the market dictates. The marginal cost schedule of a company in a perfectly competitive market determines its supply function.

  \item A is correct. As prices decrease, smaller companies will leave the market rather than sell below cost. The market share of Aquarius, the price leader, will increase.

  \item B is correct. The dominant company's market share tends to decrease as profits attract entry by other companies.

  \item B is correct. The product produced in a perfectly competitive market cannot be differentiated by advertising or any other means.

  \item $C$ is correct. Profits are maximized when $M R=M C$. For a monopoly, $M R=P[1-$ $\left.1 / E_{p}\right]$. Setting this equal to $\mathrm{MC}$ and solving for $P$ :

\end{enumerate}

$\$ 40=P[1-(1 / 1.5)]=P \times 0.333$

$P=\$ 120$

\begin{enumerate}
  \setcounter{enumi}{15}
  \item B is correct. This allows the investors to receive a normal return for the risk they are taking in the market.

  \item B is correct. The top four companies in the industry comprise 86 percent of industry sales: $(300+250+200+150) /(300+250+200+150+100+50)=$ $900 / 1050=86 \%$.

  \item B is correct. The three-firm Herfindahl-Hirschmann Index is $0.35^{2}+0.25^{2}+$ $0.20^{2}=0.225$.

  \item B is correct. The Herfindahl-Hirschmann Index does not reflect low barriers to entry that may restrict the market power of companies currently in the market. LEARNING MODULE

\end{enumerate}

\begin{center}
\includegraphics[max width=\textwidth]{2023_05_04_ccb578c5089fd41615beg-133}
\end{center}

\section{Aggregate Output, Prices, and Economic Growth}
by Paul R. Kutasovic, PhD, CFA, and Richard Fritz, PhD.

Paul R. Kutasovic, PhD, CFA, is at New York Institute of Technology (USA). Richard Fritz, PhD, is at the School of Economics at Georgia Institute of Technology (USA).

\section{LEARNING OUTCOME}
\begin{center}
\begin{tabular}{|c|c|}
\hline
Mastery & The candidate should be able to: \\
\hline
$\square$ & $\begin{array}{l}\text { calculate and explain gross domestic product (GDP) using } \\ \text { expenditure and income approaches }\end{array}$ \\
\hline
$\square$ & $\begin{array}{l}\text { compare the sum-of-value-added and value-of-final-output methods } \\ \text { of calculating GDP }\end{array}$ \\
\hline
$\square$ & $\begin{array}{l}\text { compare nominal and real GDP and calculate and interpret the GDP } \\ \text { deflator }\end{array}$ \\
\hline
$\square$ & $\begin{array}{l}\text { compare GDP, national income, personal income, and personal } \\ \text { disposable income }\end{array}$ \\
\hline
$\square$ & $\begin{array}{l}\text { explain the fundamental relationship among saving, investment, the } \\ \text { fiscal balance, and the trade balance }\end{array}$ \\
\hline
$\square$ & explain how the aggregate demand curve is generated \\
\hline
$\square$ & explain the aggregate supply curve in the short run and long run \\
\hline
$\square$ & $\begin{array}{l}\text { explain causes of movements along and shifts in aggregate demand } \\ \text { and supply curves }\end{array}$ \\
\hline
$\square$ & $\begin{array}{l}\text { describe how fluctuations in aggregate demand and aggregate supply } \\ \text { cause short-run changes in the economy and the business cycle }\end{array}$ \\
\hline
$\square$ & $\begin{array}{l}\text { distinguish among the following types of macroeconomic equilibria: } \\ \text { long-run full employment, short-run recessionary gap, short-run } \\ \text { inflationary gap, and short-run stagflation }\end{array}$ \\
\hline
$\square$ & $\begin{array}{l}\text { explain how a short-run macroeconomic equilibrium may occur at a } \\ \text { level above or below full employment }\end{array}$ \\
\hline
$\square$ & $\begin{array}{l}\text { analyze the effect of combined changes in aggregate supply and } \\ \text { demand on the economy }\end{array}$ \\
\hline
\end{tabular}
\end{center}

\section{LEARNING OUTCOME}
\begin{center}
\begin{tabular}{c|l}
Mastery & The candidate should be able to: \\
$\square$ & $\begin{array}{l}\text { describe sources, measurement, and sustainability of economic } \\ \text { growth } \\ \text { describe the production function approach to analyzing the sources } \\ \text { of economic growth } \\ \text { define and contrast input growth with growth of total factor } \\ \text { productivity as components of economic growth }\end{array}$ \\
$\square$ &  \\
\end{tabular}
\end{center}

\section{INTRODUCTION}
In the field of economics, microeconomics is the study of the economic activity and behavior of individual economic units, such as a household, a company, or a market for a particular good or service, and macroeconomics is the study of the aggregate activities of households, companies, and markets. Macroeconomics focuses on national aggregates, such as total investment, the amount spent by all businesses on plant and equipment; total consumption, the amount spent by all households on goods and services; the rate of change in the general level of prices; and the overall level of interest rates.

Macroeconomic analysis examines a nation's aggregate output and income, its competitive and comparative advantages, the productivity of its labor force, its price level and inflation rate, and the actions of its national government and central bank. The objective of macroeconomic analysis is to address such fundamental questions as the following:

\begin{itemize}
  \item What is an economy's aggregate output, and how is aggregate income measured?

  \item What factors determine the level of aggregate output/income for an economy?

  \item What are the levels of aggregate demand and aggregate supply of goods and services within the country?

  \item Is the level of output increasing or decreasing, and at what rate?

  \item Is the general price level stable, rising, or falling?

  \item Is unemployment rising or falling?

  \item Are households spending or saving more?

  \item Are workers able to produce more output for a given level of inputs?

  \item Are businesses investing in and expanding their productive capacity?

  \item Are exports (imports) rising or falling?

\end{itemize}

From an investment perspective, investors must be able to evaluate a country's current economic environment and to forecast its future economic environment in order to identify asset classes and securities that will benefit from economic trends occurring within that country. Macroeconomic variables-such as the level of inflation, unemployment, consumption, government spending, and investment-have different effects on the growth and profitability of industries within a country, the companies within those industries, and the returns of the securities issued by those companies. The following section describes gross domestic product and related measures of domestic output and income. Short-run and long-run aggregate demand and supply curves are discussed next, followed by the causes of shifts and movements along those curves, as well as factors that affect equilibrium levels of output, prices, and interest rates. The last section discusses sources, sustainability, and measures of economic growth.

\section{AGGREGATE OUTPUT AND INCOME}
$\square \quad \begin{aligned} & \text { calculate and explain gross domestic product (GDP) using } \\ & \text { expenditure and income approaches } \\ & \text { compare the sum-of-value-added and value-of-final-output methods } \\ & \text { of calculating GDP } \\ & \text { compare nominal and real GDP and calculate and interpret the GDP } \\ & \text { deflator } \\ & \text { compare GDP, national income, personal income, and personal } \\ & \text { disposable income } \\ & \text { explain the fundamental relationship among saving, investment, the } \\ & \text { fiscal balance, and the trade balance }\end{aligned}$

The aggregate output of an economy is the value of all the goods and services produced during a specified period. The aggregate income of an economy is the value of all the payments earned by the suppliers of factors used in the production of goods and services. Because the value of the output produced must accrue to the factors of production, aggregate output and aggregate income within an economy must be equal.

There are four broad forms of payments (i.e., income): compensation of employees, rent, interest, and profits. Compensation of employees includes wages and benefits (primarily employer contributions to private pension plans and health insurance) that individuals receive in exchange for providing labor. Rent is payment for the use of property. Interest is payment for lending funds. Profit is the return that owners of a company receive for the use of their capital and the assumption of financial risk when making their investments. We can think of the sum of rent, interest, and profit as the operating surplus of a company. It represents the return on all capital used by the business.

Although businesses are the direct owners of much of the property and physical capital in the economy, by virtue of owning the businesses, households are the ultimate owners of these assets and hence the ultimate recipients of the profits. In reality, of course, a portion of profits are usually retained within businesses to help finance maintenance and expansion of capacity. Similarly, because the government is viewed as operating on a non-profit basis, any revenue it receives from ownership of companies and/or property may be viewed as being passed back to households in the form of lower taxes. Therefore, for simplicity, it is standard in macroeconomics to attribute all income to the household sector unless the analysis depends on a more precise accounting.

Aggregate expenditure, the total amount spent on the goods and services produced in the (domestic) economy during the period, must also be equal to aggregate output and aggregate income. Some of this expenditure, however, may come from other countries in the form of net exports. Thus, aggregate output, aggregate income, and aggregate expenditure all refer to different ways of decomposing the same quantity. Exhibit 1 illustrates the flow of inputs, output, income, and expenditures in a very simple economy. Households supply the factors of production (labor and capital) to businesses in exchange for wages and profit (aggregate income) totaling $\pounds 100$. These flows are shown by the top two arrows. Companies use the inputs to produce goods and services (aggregate output) that they sell to households (aggregate expenditure) for $\pounds 100$. The output and expenditure flows are shown by the bottom two arrows. Aggregate output, income, and expenditure are all equal to $\pounds 100$.

In this simplified example, households spend all of their income on domestically produced goods and services. They do not buy non-domestic goods, save for the future, or pay taxes. Similarly, businesses do not sell internationally or to the government, and they do not invest to increase their productive capacity. These important components of the economy will be added in a later section. But first we need to discuss how output and income are measured.

Exhibit 1: Output, Income, and Expenditure in a Simple Economy: The Circular Flow

\begin{center}
\includegraphics[max width=\textwidth]{2023_05_04_ccb578c5089fd41615beg-136}
\end{center}

Consumption Expenditure $\pounds 100$

\section{Gross Domestic Product}
Gross domestic product (GDP) measures

\begin{itemize}
  \item the market value of all final goods and services produced within the economy during a given period (output definition) or, equivalently,

  \item the aggregate income earned by all households, all companies, and the government within the economy during a given period (income definition).

\end{itemize}

Intuitively, GDP measures the flow of output and income in the economy. GDP represents the broadest measure of the value of economic activity occurring within a country during a given period.

Therefore, GDP can be determined in two different ways. In the income approach, GDP is calculated as the total amount earned by households and companies in the economy. In the expenditure approach, GDP is calculated as the total amount spent on the goods and services produced within the economy during a given period. For the economy as a whole, total income must equal total expenditures, so the two approaches yield the same result. Many developed countries use a standardized methodology for measuring GDP. This methodology is described in the official handbook of the Organisation for Economic Co-Operation and Development (Paris: OECD Publishing). The OECD reports the national accounts for many developed nations. In the United States, the National Income and Product Accounts (also called NIPA, or national accounts, for short) is the official US government accounting of all the income and expenditure flows in the US economy. The national accounts are the responsibility of the US Department of Commerce and are published in its Survey of Current Business. In Canada, similar data are available from Statistics Canada, whereas in China, the National Bureau of Statistics of China provides GDP data.

To ensure that GDP is measured consistently over time and across countries, the following three broad criteria are used:

\begin{itemize}
  \item All goods and services included in the calculation of GDP must be produced during the measurement period. Therefore, items produced in previous periods-such as houses, cars, machinery, or equipment-are excluded. In addition, transfer payments from the government sector to individuals, such as unemployment compensation or welfare benefits, are excluded. Capital gains that accrue to individuals when their assets appreciate in value are also excluded.

  \item The only goods and services included in the calculation of GDP are those whose value can be determined by being sold in the market. This criterion enables the price of goods or services to be objectively determined. For example, a liter of extra virgin olive oil is more valuable than a liter of spring water because extra virgin olive oil has a higher market price than spring water does. The value of labor used in activities that are not sold on the market, such as commuting or gardening, is also excluded from GDP. Byproducts of production processes are also excluded if they have no explicit market value, such as air pollution, water pollution, and acid rain.

  \item Only the market value of final goods and services is included in GDP. Final goods and services are those that are not resold. Intermediate goods are goods that are resold or used to produce another good. Note that "intermediate goods" should not be confused with inventories and "final goods" should not be confused with so-called final sales. The value of intermediate goods is excluded from GDP because additional value is added during the production process, and all the value added during the entire production process is reflected in the final sale price of the finished good. An alternative approach to measuring GDP is summing all the value added during the production and distribution processes. The most direct approach, however, is to sum the market value of all the final goods and services produced within the economy during a given period.

\end{itemize}

Two distinct, but closely related, measurement methods can be used to calculate GDP based on expenditures: value of final output and sum of value added. Exhibit 2 illustrates these two methods. In this example, a farmer sells wheat to a miller. The miller grinds the wheat into flour and sells it to a baker, who makes bread and sells it to a retailer. Finally, the bread is sold to retail customers. The wheat and flour are both intermediate goods in this example because they are used as inputs to produce another good. Thus, they are not counted (directly) in GDP. For the purposes of GDP, the value of the final product is $€ 1.00$, which includes the value added by the bread retailer as a distributor of the bread. If, in contrast, the baker sold directly to the public, the value counted in GDP would be the price at which the baker sold the bread, $€ 0.78$. The left column of the exhibit shows the total revenue received at each stage of the process, whereas the right column shows the value added at each stage. Note that the market value of the final product $(€ 1.00)$ is equal to the sum of the value added at each of the stages. Thus, the contribution to GDP can be measured as either the final sale price or the sum of the value added at each stage.

Exhibit 2: Value of Final Product Equals Income Created

\begin{center}
\begin{tabular}{|c|c|c|c|}
\hline
 & $\begin{array}{c}\text { Receipts at Each } \\ \text { Stage ( } €)\end{array}$ & $\begin{array}{l}\text { Value Added } \\ \text { (= Income Created) } \\ \text { at Each Stage }(€)\end{array}$ &  \\
\hline
Receipts of farmer from miller & 0.15 & 0.15 & Value added by farmer \\
\hline
Receipts of miller from baker & 0.46 & 0.31 & Value added by miller \\
\hline
Receipts of baker from retailer & 0.78 & 0.32 & Value added by baker \\
\hline
\multirow{3}{*}{$\begin{array}{l}\text { Receipts of retailer from final } \\
\text { customer }\end{array}$} & 1.00 & 0.22 & Value added by retaile \\
\hline
 & 1.00 & 1.00 &  \\
\hline
 & Value of final output & $\begin{array}{l}\text { Total value added }= \\ \text { Total income created }\end{array}$ &  \\
\hline
\end{tabular}
\end{center}

\section{EXAMPLE 1}
\section{Contribution of Automobile Production to GDP}
\begin{enumerate}
  \item Exhibit 3 provides simplified information on the cost of producing an automobile in the United States at various stages of the production process. The example assumes the automobile is produced and sold domestically and assumes no imported material is used. Calculate the contribution of automobile production to GDP using the value-added method, and show that it is equivalent to the expenditure method. What effect would the use of imported steel or plastics have on GDP?
\end{enumerate}

\section{Exhibit 3: Cost of Producing Automobiles}
Stage of Production

Sales Value (\$)

\begin{enumerate}
  \item Production of basic materials
\end{enumerate}

$\begin{array}{ll}\text { Steel } & 1,000\end{array}$

$\begin{array}{lr}\text { Plastics } & 3,000\end{array}$

$\begin{array}{lr}\text { Semiconductors } & 1,000\end{array}$

\begin{enumerate}
  \setcounter{enumi}{1}
  \item Assembly of automobile (manufacturer price) $\quad 15,000$
\end{enumerate}

$\begin{array}{lr}\text { 3. Wholesale price for automobile dealer } & 16,000\end{array}$

\begin{enumerate}
  \setcounter{enumi}{3}
  \item Retail price $\quad 18,000$
\end{enumerate}

\section{Solution:}
GDP includes only the value of final goods and ignores intermediate goods in order to avoid double counting. Thus, the final sale price of $\$ 18,000$ and not the total sales value of $\$ 54,000$ (summing sales at all the levels of production) would be included in GDP. Alternatively, we can avoid double counting by calculating and summing the value added at each stage. At each stage of production, the difference between what a company pays for its inputs and what it receives for the product is its contribution to GDP. The value added for each stage of production is computed as follows:

\section{Stage of Production}
Sales Value (\$)

Value Added

\begin{enumerate}
  \item Production of basic materials
\end{enumerate}

Steel

Plastics

Semiconductors

Total inputs

\begin{enumerate}
  \setcounter{enumi}{1}
  \item Assembly of car (manufacturer price)
\end{enumerate}

(\$)

\begin{enumerate}
  \setcounter{enumi}{2}
  \item Wholesale price for car dealer

  \item Retail price

\end{enumerate}

\begin{center}
\begin{tabular}{ccl}
1,000 & 1,000 &  \\
3,000 & 3,000 &  \\
1,000 & 1,000 &  \\
\cline { 2 - 2 }
 & 5,000 & (sum of three inputs) \\
15,000 & 10,000 & $=(15,000-5,000)$ \\
16,000 & 1,000 & $=(16,000-15,000)$ \\
18,000 & 2,000 & $=(18,000-16,000)$ \\
\end{tabular}
\end{center}

Total expenditures

18,000

Total value added

18,000

Thus, the sum of the value added by each stage of production is equal to $\$ 18,000$, which is equal to the final selling price of the automobile. If some of the inputs (steel, plastics, or semiconductors) are imported, the value added would be reduced by the amount paid for the imports.

\section{Goods and Services Included at Imputed Values}
As a general rule, only the value of goods and services whose value can be determined by being sold in the market are included in the measurement of GDP. Owner-occupied housing and government services, however, are two examples of services that are not sold in the marketplace but are still included in the measurement of GDP.

When a household (individual) rents a place to live, she is buying housing services. The household pays the owner of the property rent in exchange for shelter. The income that a property owner receives is included in the calculation of GDP. When a household purchases a home, however, it is implicitly paying itself in exchange for the shelter. As a result, the government must estimate (impute) a value for this owner-occupied rent, which is then added to GDP.

The value of government services provided by police officers, firemen, judges, and other government officials is a key factor that affects the level of economic activity. Valuing these services is difficult, however, because they are not sold in a market like other services; individual customers cannot decide how much to consume or how much they are willing to pay. Therefore, these services are simply included in GDP at their cost (e.g., wages paid) with no value added attributed to the production process.

For simplicity and global comparability, the number of goods and services with imputed values that are included in the measurement of GDP are limited. In general, non-market activity is excluded from GDP. Thus, activities performed for one's own benefit, such as cooking, cleaning, and home repair, are excluded. Activities in the so-called underground economy are also excluded. The underground economy reflects economic activity that people hide from the government either because it is illegal or because they are attempting to evade taxation. Undocumented laborers who are paid "off the books" are one example. The illegal drug trade is another. Similarly, barter transactions, such as neighbors exchanging services with each other (e.g., helping your neighbor repair his fence in exchange for him plowing your garden), are excluded from GDP. Exhibit 4 shows a historical study on the estimated size of the underground economy in various countries as a percentage of nominal GDP. The estimates range from $8 \%$ in the United States to $60 \%$ in Peru. Based on these estimates, the US national income accounts fail to account for roughly 7.4\% $(=8 / 108)$ of economic activity, whereas in Peru, the national accounts miss roughly $37.5 \%$ (=60/160) of the economy. For most of the countries shown, the national accounts miss $12 \%-20 \%$ of the economy.

\section{Exhibit 4: Underground Economy As a Percentage of}
 Nominal GDP (2006)\begin{center}
\begin{tabular}{lc}
\hline
Country & $\begin{array}{c}\text { Underground Economy As a } \\ \text { Percentage of Nominal GDP (\%) }\end{array}$ \\
\hline
Peru & 60.0 \\
Mexico & 32.1 \\
South Korea & 27.5 \\
Costa Rica & 26.8 \\
Greece & 26.0 \\
India & 24.4 \\
Italy & 23.1 \\
Spain & 20.2 \\
Sweden & 16.3 \\
Germany & 15.4 \\
Canada & 14.1 \\
China & 14.0 \\
France & 13.2 \\
Japan & 8.9 \\
United States & 8.0 \\
\hline
\end{tabular}
\end{center}

Source: Friedrich Schneider and Andreas Buehm, Linz University, 2009.

It should be clear from these estimates of the underground economy that the reliability of official GDP data varies considerably across countries. Failure to capture a significant portion of activity is one problem. Poor data collection practices and unreliable statistical methods within the official accounts are also potential problems.

\section{Nominal and Real GDP}
To evaluate an economy's health, it is often useful to remove the effect of changes in the general price level on GDP because higher (lower) income driven solely by changes in the price level is not indicative of a higher (lower) level of economic activity. To account for these effects, economists use real GDP, which indicates what would have been the total expenditures on the output of goods and services if prices were unchanged. Per capita real GDP (real GDP divided by the size of the population) has often been used as a measure of the average standard of living in a country.

Suppose we are interested in measuring the GDP of an economy. For the sake of simplicity, suppose that the economy consists of a single automobile maker and that in 2019, 300,000 vehicles are produced with an average market price of $€ 18,750$. GDP in 2019 would be $€ 5,625,000,000$. Economists define the value of goods and services measured at current prices as nominal GDP. Suppose that in 2020, 300,000 vehicles are again produced but that the average market price for a vehicle increases by $7 \%$ to $€ 20,062.50$. GDP in 2020 would be $€ 6,018,750,000$. Even though no more cars were produced in 2020 than in 2019, it appears that the economy grew by $(€ 6,018,750,000 / € 5,625,000,000)-1=7 \%$ between 2019 and 2020, although it actually did not grow at all.

Nominal and real GDP can be expressed as

Nominal $\mathrm{GDP}_{t}=P_{t} \times Q_{t}$

where

$P_{t}=$ prices in year $t$

$Q_{t}=$ quantity produced in year $t$

\begin{center}
\includegraphics[max width=\textwidth]{2023_05_04_ccb578c5089fd41615beg-141(1)}
\end{center}

where

$$
P_{B}=\text { prices in the base year }
$$

Taking the base year to be 2019 and putting in the 2019 and 2020 numbers gives the following:

\begin{center}
\includegraphics[max width=\textwidth]{2023_05_04_ccb578c5089fd41615beg-141}
\end{center}

Real $\mathrm{GDP}_{2019}=(€ 18,750 \times 300,000)=€ 5,625,000,000$

Nominal $\mathrm{GDP}_{2020}=(€ 20,062.50 \times 300,000)=€ 6,018,750,000$

Real $\mathrm{GDP}_{2020}=(€ 18,750 \times 300,000)=€ 5,625,000,000$

In this example, real GDP did not change between 2019 and 2020 because the total output remained the same: 300,000 vehicles. The difference between nominal GDP in 2020 and real GDP in 2020 was the $7 \%$ inflation rate.

Now suppose that the auto manufacturer produced 3\% more vehicles in 2020 than in 2019 (i.e., production in 2020 was 309,000 vehicles). Real GDP would increase by 3\% from 2019 to 2020 . With a 7\% increase in prices, nominal GDP for 2020 would now be as follows:

Nominal GDP $_{2020}=(1.03 \times 300,000) \times(1.07 \times € 18,750)$

$$
\begin{aligned}
& =(309,000 \times € 20,062.50) \\
& =€ 6,199,312,500
\end{aligned}
$$

The implicit price deflator for GDP, or simply the GDP deflator, is defined as follows:

GDP deflator $=\frac{\text { Value of current year output at current year prices }}{\text { Value of current year output at base year prices }} \times 100$

Thus, in the example, the GDP deflator for 2020 is $[(309,000 \times € 20,062.50) /(309,000$ $x € 18,750)](100)=(1.07)(100)=107$. The GDP deflator broadly measures the aggregate changes in prices across the overall economy, and hence changes in the deflator provide a useful gauge of inflation within the economy.

Real GDP is equal to nominal GDP divided by the GDP deflator scaled by 100:

Real GDP $=$ Nominal GDP $/($ GDP deflator/100)

This relation gives the GDP deflator its name. That is, the measure of GDP in terms of current prices, nominal GDP, is adjusted for inflation by dividing it by the deflator. The expression also shows that the GDP deflator is the ratio of nominal GDP to real GDP scaled by 100 :

GDP deflator $=($ Nominal GDP $/$ Real GDP $) \times 100$ Thus, real GDP for 2020 would be

$$
\begin{aligned}
& \text { Real } \text { GDP }_{2020}=\text { Nominal GDP/(GDP deflator/100) } \\
&=€ 6,199,312,500 /(107 / 100) \\
&=€ 5,793,750,000
\end{aligned}
$$

Note that $€ 5,793,750,000$ represents 3\% real growth over 2019 GDP and 3\% higher real GDP for 2020 than under the assumption of no growth in unit car sales in 2020.

What would be the increase in nominal GDP for 2020 compared with 2019 with the $3 \%$ greater automobile production and $7 \%$ inflation?

\begin{center}
\includegraphics[max width=\textwidth]{2023_05_04_ccb578c5089fd41615beg-142}
\end{center}

$$
\begin{aligned}
& =(€ 6,199,312,500 / € 5,625,000,000)-1 \\
& =0.102
\end{aligned}
$$

So, nominal GDP would increase by $10.2 \%$, which equals $(1.07 \times 1.03)-1$, or approximately $7 \%+3 \%=10 \%$. Which number is more informative about growth in economic activity: $3 \%$ real growth or $10.2 \%$ nominal growth? The real growth rate is more informative because it exactly captures increases in output. Nominal growth, by blending price changes with output changes, is less directly informative about output \href{http://changes.In}{changes.In} summary, real economic growth is measured by the percentage change in real GDP. When measuring real economic activity or when comparing one nation's economy with another, real GDP and real GDP growth should be used because they more closely reflect the quantity of output available for consumption and investment.

\section{EXAMPLE 2}
\section{Calculating the GDP Deflator}
\begin{enumerate}
  \item John Lambert is an equity analyst with Equitytrust, an investment management firm that primarily invests in domestic stocks and bonds in a hypothetical country. Equitytrust's investment policy committee is concerned about the possibility of inflation. The implicit GDP deflator is an important measure of the overall price level in the economy, and changes in the deflator provide an important gauge of inflation within the economy. Released GDP data are shown in Exhibit 5. The committee asks Lambert to use the GDP data to calculate the implicit GDP price deflator from 2016 to 2020 and the inflation rate for 2020 .
\end{enumerate}

\section{Exhibit 5: Real and Nominal GDP}
Seasonally adjusted at annual rates (SAAR)

\begin{center}
\begin{tabular}{lccccc}
\hline
 & $\mathbf{2 0 1 6}$ & $\mathbf{2 0 1 7}$ & $\mathbf{2 0 1 8}$ & $\mathbf{2 0 1 9}$ \\
\hline
GDP at market prices & $1,822,808$ & $1,897,531$ & $1,990,183$ & $1,994,911$ \\
Real GDP & $1,659,195$ & $1,698,153$ & $1,747,478$ & $1,762,561$ \\
\hline
\end{tabular}
\end{center}

\section{Solution:}
The implicit GDP price deflator measures inflation across all sectors of the economy, including the consumer, business, government, exports, and imports. It is calculated as the ratio of nominal to real GDP and reported as an index number with the base year deflator equal to 100 . The implicit GDP price deflator for the economy for 2020 is calculated as $(2,035,506 / 1,786,677) \times 100=113.9$. The results for the other years are shown in the following table:

\begin{center}
\begin{tabular}{lccccc}
\hline
 & $\mathbf{2 0 1 6}$ & $\mathbf{2 0 1 7}$ & $\mathbf{2 0 1 8}$ & $\mathbf{2 0 1 9}$ \\
\hline
GDP at market prices (\$ millions) & $1,822,808$ & $1,897,531$ & $1,990,183$ & $1,994,911$ \\
Real GDP (2007 \$ millions) & $1,659,195$ & $1,698,153$ & $1,747,478$ & $1,762,561$ \\
Implicit GDP price deflator & 113.9 & 111.7 & 113.9 & $1,786,677$ \\
\hline
\end{tabular}
\end{center}

For 2016, the annual inflation rate is equal to [(113.9/113.2) - 1] or 0.62 percent. This shows that Canada experienced a very low rate of inflation in 2016.

\section{The Components of GDP}
Having defined GDP and discussed how it is measured, we can now consider the major components of GDP, the flows among the four major sectors of the economy-the household sector, the business sector, the government sector, and the foreign or external sector (consisting of transactions with the "rest of the world")-and the markets through which they interact. An expression for GDP, based on the expenditure approach, is

$$
\mathrm{GDP}=C+I+G+(X-M)=\left(C+G^{C}\right)+\left(I+G^{I}\right)+(X-M)
$$

where

$C=$ consumer spending on final goods and services

$I=$ gross private domestic investment, which in cludes business investment in capital goods (e.g., fixed capital such as plant and equipment) and changes in inventory (inventory investment)

$G=$ government spending on final goods and services for both current consumption and investment in capital goods $=G^{C}+G^{I}$

$$
\begin{aligned}
& X=\text { exports } \\
& M=\text { imports }
\end{aligned}
$$

Exhibit 6 shows the flow of expenditures, income, and financing among the four sectors of the economy and the three principal markets. In the exhibit, solid arrows point in the direction of expenditure on final goods and services. For simplicity, corresponding flows of output are not shown separately. The flow of factors of production is also shown with a solid arrow. Financial flows, including income and net taxes, are shown with dashed arrows pointing to the recipient of funds.

\section{Exhibit 6: Output, Income, and Expenditure Flows}
\begin{center}
\includegraphics[max width=\textwidth]{2023_05_04_ccb578c5089fd41615beg-144}
\end{center}

\section{The Household and Business Sectors}
The very top portion of Exhibit 6 shows the services of labor, land, and capital flowing through the factor market to business firms and the flow of income back from firms to households. Households spend part of their income on consumption $(C)$ and save (S) part of their income for future consumption. Current consumption expenditure flows through the goods market to the business sector. Household saving flows into the financial markets, where it provides funding for businesses that need to borrow or raise equity capital. Firms borrow or raise equity primarily to finance investment $(I)$ in inventory, property, plant, and equipment. Investment $(I)$ is shown flowing from firms through the goods market and back to firms because the business sector both demands and produces the goods needed to build productive capacity (capital goods).

In most developed economies, such as Italy and the United States, expenditures on capital goods represent a significant portion of GDP. Investments (expenditures) on capital goods accounted for approximately $12.3 \%$ of Italy's GDP in 2015, while in the United States investments accounted for approximately $16.2 \%$ of GDP during the same period. In some developing countries, notably China (45.4\%) and India (32.9\%), investment spending accounts for a substantially larger share of the economy (according to OECD.Stat). As we will examine in greater detail later, investment spending is an important determinant of an economy's long-term growth rate. At the same time, investment spending is the most volatile component of the economy, and changes in capital spending-especially spending on inventories-are one of the main factors causing short-run economic fluctuations.

\section{The Government Sector}
The government sector collects taxes from households and businesses. For simplicity, only the taxes collected from the household sector are shown in Exhibit 6. In turn, the government sector purchases goods and services $(G)$ for both consumption and investment from the business sector. For example, the government sector hires construction companies to build roads, schools, and other infrastructure goods. Government expenditure $(G)$ also reflects spending on the military, police and fire protection, the postal service, and other government services. To keep Exhibit 6 simple, however, we combine consumption and investment expenditures into government expenditure, $G$.

Governments also make transfer payments to households. In general, such payments are designed to address social objectives such as maintaining minimum living standards, providing health care, and assisting the unemployed with retraining and temporary support. In Exhibit 6, transfer payments are subtracted from taxes and reflected in net taxes $(T)$.

Transfer payments are not included in government expenditures on goods and services $(G)$ because they represent a monetary transfer by the government of tax revenue back to individuals, with no corresponding receipt of goods or services. The household spending facilitated by the transfer payments is, of course, included in consumption $(C)$ and, hence, GDP. It is worth noting that transfers do not always take the form of direct payments to beneficiaries. Instead, the government may pay for or even directly provide goods or services to individuals. For example, universal health care programs often work in this way.

If, as is usually the case, government expenditure $(G)$ exceeds net taxes $(T)$, then the government has a fiscal deficit and must borrow in the financial markets. Thus, the government may compete with businesses in the financial markets for the funds generated by household saving. The only other potential source of funds in an economy is capital flows from the rest of the world, discussed in the next section.

In 2015, the ratio of general government spending (which includes central government as well as state, provincial, and local government) to GDP in Italy was $50.4 \%$, while in the United States it was $37.7 \%$. In countries where the government provides more services, such as universal health care in Italy, the government's contribution to GDP is greater. France's government sector represents 57.0\% of GDP. In other countries, the public sector makes up a smaller share. In Mexico, for example, government spending is $24.5 \%$ of GDP. Exhibit 7 shows data on tax revenues, general government spending, and transfer payments as a share of nominal GDP.

\section{Exhibit 7: General Government Spending and Taxes as a Percentage of GDP (2015)}
General Government Spending as a Percentage of GDP

\section{Country}
General Government Tax Revenues as a Percentage of GDP Goods and Services and Debt Service

Transfer Payments

\begin{center}
\begin{tabular}{lcccc}
\hline
Canada & $39.8 \%$ & $41.1 \%$ & n.a. & n.a. \\
Mexico & 23.6 & 24.5 & n.a. & n.a. \\
United States & 33.5 & 37.7 & 29.9 & 7.8 \\
Japan & 35.8 & 39.4 & 23.3 & 16.1 \\
South Korea & 33.8 & 32.4 & n.a. &  \\
France & 53.4 & 57.0 & 24.5 &  \\
Germany & 44.7 & 44.0 & 19.0 &  \\
Greece & 48.3 & 54.2 & 20.5 &  \\
Italy & 47.7 & 50.4 & 21.4 &  \\
Sweden & 50.5 & 50.2 & 20.9 & 20.9 \\
Australia & 34.3 & 37.2 & 29.3 & 10.2 \\
\end{tabular}
\end{center}

\begin{center}
\begin{tabular}{|c|c|c|c|c|}
\hline
 & \multirow{2}{*}{$\begin{array}{c}\text { General Government } \\
\text { Tax Revenues as a } \\
\text { Percentage of GDP }\end{array}$} & \multicolumn{3}{|c|}{$\begin{array}{c}\text { General Government Spending } \\
\text { as a Percentage of GDP }\end{array}$} \\
\hline
Country &  & Total & $\begin{array}{c}\text { Goods and Services } \\ \text { and Debt Service }\end{array}$ & Transfer Payments \\
\hline
OECD - Average & 42.2 & 43.8 & 27.3 & 16.5 \\
\hline
\end{tabular}
\end{center}

Note: n.a. = not available

Source: Government at a Glance, 2017 edition (\href{http://stats.oecd.org}{stats.oecd.org}).

\section{The External Sector}
Trade and capital flows involving the rest of the world are shown in the bottom right quadrant of Exhibit 6. Net exports $(X-M)$ reflects the difference between the value of goods and services sold internationally-exports $(X)$-and the portion of domestic consumption $(C)$, investment $(I)$, and government expenditure $(G)$ that represents purchases of goods and services from the rest of the world-imports $(M)$.

A balance of trade deficit means that the domestic economy is spending more on international goods and services than other nations' economies are spending on domestic goods and services. It also means that the country is spending more than it produces because domestic saving is insufficient to finance domestic investment plus the government's fiscal balance. A trade deficit must be funded by borrowing from the rest of the world through the financial markets. The rest of the world is able to provide this financing because, by definition, it must be running a corresponding trade surplus and is spending less than it produces.

It bears emphasizing that trade and capital flows between an economy and the rest of the world must balance. One area's deficit is another's surplus, and vice versa. This dynamic is an accounting identity that must hold. In effect, having allowed a country to run a trade deficit, other nations must, in aggregate, finance it. The financing terms, however, may or may not be attractive.

Exhibit 8 reports trade balances for the United States with selected countries. Note that in 2016, China was the United States' largest trading partner, in terms of sending goods to the United States (US imports). Mexico and Canada, members of NAFTA along with the United States, were second and third in sending goods to the United Sates but were also important consumers of US goods (US exports). Over all its trading partners, the US balance of trade deficit in 2016 was US $\$ 734,316.3$ million.

Exhibit 8: US International Trade in Goods-Selected Countries, 2016 (US\$ millions)

\begin{center}
\begin{tabular}{lccc}
\hline
 & Exports & Imports & Balance \\
\hline
Total, all countries & $1,454,624.2$ & $2,188,940.5$ & $-734,316.3$ \\
Europe & $318,447.1$ & $483,454.8$ & $-165,007.7$ \\
Euro area & $200,166.6$ & $325,880.1$ & $-125,713.4$ \\
France & $30,941.2$ & $46,764.6$ & $-15,823.4$ \\
Germany & $49,362.0$ & $114,227.4$ & $-64,865.4$ \\
Italy & $16,753.6$ & $45,210.1$ & $-28,456.5$ \\
Canada & $266,826.7$ & $278,066.8$ & $-11,240.1$ \\
Mexico & $230,969.1$ & $294,151.0$ & $-63,191.9$ \\
China & $115,775.1$ & $462,813.0$ & $-347,037.9$ \\
\end{tabular}
\end{center}

\begin{center}
\begin{tabular}{lccc}
\hline
 & Exports & Imports & Balance \\
\hline
India & $21,689.0$ & $45,998.4$ & $-24,309.5$ \\
Japan & $63,264.3$ & $132,201.8$ & $-68,937.6$ \\
\hline
\end{tabular}
\end{center}

Source: FT-900 Supplement, US International Trade in Goods and Services, December 2016.

\section{GDP, National Income, Personal Income, and Personal Disposable Income}
This section examines the calculation of GDP and other income measures in detail by means of an analysis of data from Statistics Canada.

Exhibit 9 provides data on the level of Canadian GDP and its components measured at market prices (nominal GDP) as part of Example 3.

Exhibit 9: GDP Measures for the Canadian Economy

(C\$ millions at market prices, seasonally adjusted at annual rates)

\begin{center}
\begin{tabular}{|c|c|c|c|c|c|}
\hline
 & 2014 & 2015 & 2016 & 2017 & 2018 \\
\hline
\multicolumn{6}{|l|}{Expenditure-Based:} \\
\hline
Final consumption expenditure & $1,517,376$ & $1,566,215$ & $1,610,954$ & $1,684,563$ & $1,754,326$ \\
\hline
$\begin{array}{l}\text { Household final consumption } \\ \text { expenditure }\end{array}$ & $1,086,253$ & $1,122,079$ & $1,153,668$ & $1,208,438$ & $1,255,327$ \\
\hline
$\begin{array}{l}\text { Non-profit institutions serving house- } \\ \text { holds' final consumption expenditure }\end{array}$ & 26,826 & 28,575 & 30,951 & 31,976 & 33,207 \\
\hline
$\begin{array}{l}\text { General governments final consump- } \\ \text { tion expenditure }\end{array}$ & 404,297 & 415,561 & 426,335 & 444,149 & 465,792 \\
\hline
Gross fixed capital formation & 486,542 & 474,732 & 461,262 & 486,760 & 501,247 \\
\hline
Business gross fixed capital formation & 410,591 & 395,432 & 382,196 & 401,686 & 409,835 \\
\hline
$\begin{array}{l}\text { Non-profit institutions serving house- } \\ \text { holds' gross fixed capital formation }\end{array}$ & 3,145 & 2,956 & 2,212 & 2,290 & 2,325 \\
\hline
$\begin{array}{l}\text { General governments' gross fixed capi- } \\ \text { tal formation }\end{array}$ & 72,806 & 76,344 & 76,854 & 82,784 & 89,087 \\
\hline
Investment in inventories & 9,603 & -577 & -228 & 17,088 & 11,904 \\
\hline
Exports to other countries & 633,112 & 633,955 & 638,095 & 672,533 & 713,351 \\
\hline
Less: Imports from other countries & 651,176 & 683,019 & 685,868 & 719,909 & 756,913 \\
\hline
Statistical discrepancy & -559 & -865 & 1,320 & 72 & -59 \\
\hline
GDP at market prices & $1,994,898$ & $1,990,441$ & $2,025,535$ & $2,141,107$ & $2,223,856$ \\
\hline
\end{tabular}
\end{center}

Income-Based:

\begin{center}
\begin{tabular}{lccccc}
\hline
Compensation of employees & $\mathbf{9 9 8 , 4 6 3}$ & $\mathbf{1 , 0 2 6 , 8 4 6}$ & $\mathbf{1 , 0 2 6 , 4 8 3}$ & $\mathbf{1 , 0 6 9 , 9 5 6}$ & $\mathbf{1 , 1 2 0 , 5 2 0}$ \\
Gross operating surplus & $\mathbf{5 5 7 , 2 4 4}$ & $\mathbf{5 0 5 , 1 6 7}$ & $\mathbf{5 2 6 , 0 2 4}$ & $\mathbf{5 7 3 , 2 5 7}$ &  \\
Net operating surplus: corporations & 289,123 & 220,212 & 233,763 & 280,267 &  \\
Consumption of fixed capital: & 203,080 & 217,285 & 222,954 & 221,894 &  \\
corporations &  &  &  &  &  \\
Consumption of fixed capital: general & 65,041 & 67,280 & 228,666 & 71,096 &  \\
\end{tabular}
\end{center}

governments and non-profit institu-

tions serving households

\begin{center}
\begin{tabular}{|c|c|c|c|c|c|}
\hline
 & 2014 & 2015 & 2016 & 2017 & 2018 \\
\hline
Gross mixed income & 227,166 & 235,365 & 244,403 & 256,757 & 265,147 \\
\hline
Net mixed income & 172,079 & 177,997 & 183,577 & 194,463 & 199,212 \\
\hline
$\begin{array}{l}\text { Consumption of fixed capital: unincor- } \\ \text { porated businesses }\end{array}$ & 55,087 & 57,368 & 60,826 & 62,294 & 65,935 \\
\hline
Taxes less subsidies on production & 84,321 & 88,570 & 90,516 & 95,563 & 98,698 \\
\hline
$\begin{array}{l}\text { Taxes less subsidies on products and } \\ \text { imports }\end{array}$ & 127,144 & 133,627 & 139,429 & 145,647 & 148,921 \\
\hline
Statistical discrepancy & 560 & 866 & $-1,320$ & -73 & 60 \\
\hline
$\begin{array}{l}\text { Gross domestic income at market } \\ \text { prices }\end{array}$ & $1,994,898$ & $1,990,441$ & $2,025,535$ & $2,141,107$ & $2,223,856$ \\
\hline
\end{tabular}
\end{center}

Source: Statistics Canada. Table 36-10-0222-01 and Table: 36-10-0221-01 GDP, expenditure-based \& income-based, annual $(x \mathrm{C} \$ 1,000,000)$.

Exhibit 9 shows the two approaches to measuring GDP: 1) expenditures on final output measured as the sum of sales to the final users and 2) the sum of the factor incomes generated in the production of final output. In theory, the two approaches should provide the same estimate of GDP. As shown in the exhibit, however, they differ in practice because of the use of different data sources. The difference is accounted for by a statistical discrepancy. Market analysts more closely follow the expenditure approach because the expenditure data are more timely and reliable than data for the income components.

Using the expenditure approach, Statistics Canada measures Canadian GDP as follows:

GDP $=$ Consumer spending on goods and services

\begin{itemize}
  \item Business gross fixed investment

  \item Change in inventories

  \item Government spending on goods and services

  \item Government gross fixed investment

  \item Exports - Imports

  \item Statistical discrepancy

\end{itemize}

Canadian national income accounts explicitly allocate government expenditures between consumption expenditure and gross fixed capital formation. Not all countries make this distinction. The United States, for example, does not. Also note that the investment in business inventories must be included in expenditures. Otherwise, goods produced but not yet sold would be left out of GDP.

The income-based approach calculates gross domestic income (GDI) as the sum of factor incomes and essentially measures the cost of producing final output. Two of the costs entering into the gross value of output, however, are not really earned by a factor of production. These items, depreciation and indirect taxes, are discussed next. GDP is estimated in the income approach as follows (note that construction of the national income accounts varies across countries):

$$
\begin{aligned}
& \text { GDP }=\text { Gross domestic income }(\mathrm{GDI}) \\
&=\text { Net domestic income }+ \text { Consumption of fixed capital }(\mathrm{CFC})+\text { Statistical } \\
& \text { discrepancy }
\end{aligned}
$$

where gross domestic income is the income received by all factors of production used in the generation of final output: Gross domestic income $=$ Compensation of employees

\begin{itemize}
  \item Gross operating surplus

  \item Gross mixed income

  \item Taxes less subsidies on production

  \item Taxes less subsidies on products and imports

\end{itemize}

Compensation of employees consists of 1 ) wages and salaries including direct compensation in cash or in kind plus 2) "employers' social contributions" that supplement wages, which are primarily payments for government-sponsored social security schemes including pensions and health insurance.

Gross operating surplus is related to corporate profits and includes private corporations, non-profit corporations, and government corporations. Gross operating surplus is the surplus arising from operations and does not take out charges for rent, interest, or similar charges on financial assets or natural resources used by the business. As such, gross operating surplus essentially measures the return on capital used by the business as a whole, rather than the return to the owners of the business (profit).

Gross mixed income is the same concept applied to unincorporated business in the economy. It is measured in the same way as gross operating surplus and has three major components: 1) farm income, 2) non-farm income excluding rent, and 3) rental income.

"Indirect business taxes less subsidies" reflects taxes and subsidies included in the final price of the good or service. It is the (net) portion of national income that is directly paid to the government. In the Canadian accounts, these are measured in two ways: 1) "taxes less subsidies on products and imports," which include sales taxes, fuel taxes, and import duties, and 2) "taxes less subsidies on factors of production," which are mainly property taxes and payroll taxes.

The consumption of fixed capital (CFC) is a measure of the wear and tear (depreciation) of the capital stock that occurs in the production of goods and services. This measure acknowledges the fact that some income/output must be allocated to the replacement of the existing capital stock as it wears out. Loosely speaking, one may think of Profit + CFC as the gross surplus earned by capital, with the CFC being the amount that must be earned and reinvested to maintain the existing productivity of the capital.

Along with the GDP report, Statistics Canada and other government statistical agencies provide information on personal income (called Primary Household Income in the Canadian accounts) and personal saving (called Household Net Saving), as shown in Exhibit 10. Personal income is a broad measure of household income and measures the ability of consumers to make purchases. As such, it is one of the key determinants of consumption spending. Primary Household Income includes all income received by households, whether earned or unearned. It includes compensation of employees plus net mixed income from unincorporated businesses plus net property income.

Exhibit 10: Household Income and Saving for the Canadian Economy

(C\$ millions at market prices, seasonally adjusted at annual rates)

\begin{center}
\begin{tabular}{|c|c|c|c|c|c|}
\hline
 & 2014 & 2015 & 2016 & 2017 & 2018 \\
\hline
Compensation of employees & 996,247 & $1,024,743$ & $1,024,142$ & $1,066,600$ & $1,117,091$ \\
\hline
Plus: net mixed income & 172,079 & 177,997 & 183,577 & 194,463 & fill in \\
\hline
Plus: net property income & 126,019 & 142,185 & 133,098 & 140,942 & 140,370 \\
\hline
Primary household income & $1,294,345$ & $1,344,925$ & $1,340,817$ & $1,402,005$ & calc \\
\hline
Plus: current transfers received & 266,153 & 288,200 & 307,077 & 320,861 & 333,602 \\
\hline
\end{tabular}
\end{center}

\begin{center}
\begin{tabular}{|c|c|c|c|c|c|}
\hline
 & 2014 & 2015 & 2016 & 2017 & 2018 \\
\hline
Less: current transfers paid & 478,843 & 503,245 & 515,136 & 533,256 & 555,884 \\
\hline
Household disposable income & $1,081,655$ & $1,129,880$ & $1,132,758$ & $1,189,610$ & calc \\
\hline
$\begin{array}{l}\text { Less: household final consumption } \\ \text { expenditure }\end{array}$ & $1,086,253$ & $1,122,079$ & $1,153,668$ & $1,208,438$ & fill in \\
\hline
Plus: change in pension entitlements & 46,439 & 43,260 & 43,139 & 43,391 & 43,087 \\
\hline
Household net saving & 41,841 & 51,061 & 22,229 & 24,563 & calc \\
\hline
Household saving rate (\%) & 3.9 & 4.5 & 2.0 & 2.1 & calc \\
\hline
\end{tabular}
\end{center}

Source: Statistics Canada. Table 36-10-0224-01 Household sector, current accounts, provincial and territorial, annual.

Household disposable income (HDI) is equal to personal income less personal taxes. In reality, households both pay taxes to the government and receive some income from governments (transfer payments such as social insurance payments, unemployment compensation, and disability payments) that is not earned. Thus HDI is calculated as household primary income less net current transfers paid. This calculation measures the amount of after-tax income that households have to spend on goods and services or to save. Thus, it is the most relevant, and most closely watched, measure of income for household spending and saving decisions.

Finally, household net saving is equal to HDI adjusted for two items: subtracting household final consumption expenditures and adding the net change in pension entitlements.

\section{EXAMPLE 3}
\section{Canadian GDP Release and Other Measures of Production and Income}
\begin{enumerate}
  \item The investment policy committee at Equitytrust asks John Lambert to review the Canadian GDP data shown in Exhibit 9 and Exhibit 10.

  \item Calculate 2018 GDP using the expenditure approach, and indicate how the expenditures are represented in Exhibit 6.

  \item Calculate 2018 GDP using the income approach.

  \item Calculate personal income for 2018 .

  \item Using the Canadian data for 2018, calculate the level of household saving $(S)$, the saving rate, and net payments $(T)$ by the household sector to government.

  \item Calculate the impact of international trade on the Canadian economy in 2018 and Canada's net foreign borrowing/lending in 2018.

  \item Calculate the amount of net fixed capital formation across all sectors in 2018.

\end{enumerate}

\section{Solutions:}
(all numbers in millions)

\section{Solution to 1:}
In the expenditure approach, nominal GDP is calculated as the sum of spending by the major sectors in the economy:

GDP = Final consumption expenditures + Gross fixed capital formation + Investment in Inventories + Exports - Imports + Statistical discrepancySubstituting the numbers from Exhibit 9 ,

GDP $=1,754,326+501,247+11,904+713,351-756,913-59$

$=\mathrm{C} \$ 2,223,856$.

In Exhibit 6, these expenditures are represented by the arrows pointing to the goods market and by the arrow pointing back to firms labeled as $C+$ $I+G+(X-M)$. Note that $G$ in Exhibit 6 is captured here partly in final consumption expenditures and partly in gross fixed capital formation for the Canadian national income accounts.

\section{Solution to 2:}
On the income side, nominal GDP is equal to gross domestic income-the sum of income received by the factors of production-given by the following equation:

Gross domestic income $=$ Compensation of employees + Gross operating surplus + Gross mixed income+ Indirect business taxes - Subsidies + Statistical discrepancy

Substituting in the numbers from Exhibit 9, we get indirect business taxes less subsidies as equal to the sum of taxes less subsidies on production plus taxes less subsidies on products and imports $=98,698+148,921=$ C $\$ 247,619$. Using this result,

$$
\begin{aligned}
& \text { GDP }=\text { GDI }=1,120,520+590,510+265,147+247,619+60 \\
& =\mathrm{C} \$ 2,223,856 .
\end{aligned}
$$

\section{Solution to 3:}
Personal income (household primary income) is calculated as

Household Primary Income $=$ Compensation of employees + Net mixed income + Net property incomeSubstituting in the numbers from Exhibit 10,

Household primary income $=1,117,091+199,212+140,370$

$$
=\mathrm{C} \$ 1,456,673 \text {. }
$$

\section{Solution to 4:}
Household saving is equal to personal disposable income less household final consumption expenditures plus the change in pension entitlements. Household final consumption $(C)$ is given in Exhibit 9 as $\mathrm{C} \$ 1,255,327$. Substituting in the appropriate numbers, Household net saving = Household primary income - Net current transfers paid - Household final consumption expenditure + Net change in pension entitlements $=1,456,673$ $(555,884-333,602)-1,255,327+43,087=C \$ 22,151$. The Canadian saving rate for $2018=(22,151 / 1,117,091)=2 \%$.

Net "taxes" paid by the household sector consists of two components: 1 ) transfers paid by households to the government minus 2) government transfer payments received by households. From Exhibit 10, government transfer payments to households for 2018 were $C \$ 333,602$. The transfer payments by households to government in 2018 was $C \$ 555,884$. Thus the net payments to government by households in 2018 was:

$$
555,884-333,602=C \$ 222,282
$$

\section{Solution to 5:}
It is clear from Exhibit 9 that the international sector generally has a large effect each year on the Canadian economy. In 2018, exports were $\mathrm{C} \$ 713,351$, or roughly $32 \%$ of Canadian GDP. Imports were higher than exports at $\mathrm{C} \$ 56,913$ in 2018, indicating that Canada had a trade deficit of $713,351-756,913=C \$ 43,562$. This figure is roughly the same size as the trade deficit in 2017.

Canada funds its trade deficit by borrowing from the rest of the world through the financial markets. In 2018, this involved borrowing $C \$ 43,562$ from the rest of the world. Trade and capital flows between an economy and the rest of the world must balance. A trade deficit must be funded by a capital inflow.

\section{Solution to 6:}
Net fixed capital formation is equal to gross fixed capital formation found in the expenditure accounts less consumption of fixed capital (CFC) by corporations, government and non-profits, and unincorporated businesses found in the income accounts. In 2018, gross fixed capital formation was C $\$ 501,247$, and CFC was $(228,666+74,564+65,935)$ for corporations, government and non-profits, and unincorporated businesses, respectively. Thus net fixed capital formation in the Canadian economy for 2018 was $501,247-369,165=\mathrm{C} \$ 132,082$.

\section{Relationship among Saving, Investment, the Fiscal Balance and the Trade Balance}
Total expenditure on domestically produced output comes from four sources: household consumption $(C)$, investments (I), government spending $(G)$, and net exports $(X-M)$. This can be expressed as follows:

Expenditure $=C+I+G+(X-M)$

Personal disposable income is equal to GDP $(Y)$ plus transfer payments $(F)$ minus retained earnings and depreciation (= business saving, $S_{B}$ ) minus direct and indirect taxes $(R)$. Households allocate disposable income between consumption of goods and services $(C)$ and household saving $\left(S_{H}\right)$. Therefore,

$$
Y+F-S_{B}-R=C+S_{H}
$$

Rearranging this equation, we get

$$
Y=C+S+T
$$

where $T=(R-F)$ denotes net taxes and $S=\left(S_{B}+S_{H}\right)$ denotes total private sector saving. Because total expenditures must be identical to aggregate income $(Y)$, we have the following relationship:

$C+S+T=C+I+G+(X-M)$

By rearranging this equation, we get the following fundamental relationship among domestic saving, investment, the fiscal balance, and the trade balance:

$$
S=I+(G-T)+(X-M)
$$

This equation shows that domestic private saving is used or absorbed in one of three ways: investment spending $(I)$, financing government deficits $(G-T)$, and building up financial claims against overseas economies [positive trade balance, $(X-M)>0$ ]. If there is a trade deficit $[(X-M)<0]$, then domestic private saving is being supplemented by inflows of foreign saving and other countries' economies are building up financial claims against the domestic economy.

By rearranging the identity, we can examine the implications of government deficits and surpluses:

$G-T=(S-I)-(X-M)$

A fiscal deficit $[(G-T)>0]$ implies that the private sector must save more than it invests $[(S-I)>0]$ or the country must run a trade deficit $[(X-M)<0]$ with corresponding inflow of foreign saving-or both.

\section{EXAMPLE 4}
\section{Foreign Capital Inflows Help Finance Government Deficits}
\begin{enumerate}
  \item The budgetary situation changed dramatically in Canada during 2009, the first year of the financial crisis. The Department of Finance Canada reported that in 2009, the combined federal-provincial governments had a deficit of $C \$ 84,249$ (all amounts in $C \$$ millions). Thus, the government sector operated at a deficit that needed to be financed. How was this deficit financed?
\end{enumerate}

\section{Solution:}
Using the formula $G-T=(S-I)-(X-M)$ shows that a budget deficit is financed through either higher domestic saving $(S)$, lower business investment $(I)$, or international borrowing $(X-M)$.

In 2009, private sector saving exceeded investment spending by $C \$ 58,588$ $(319,802-277,574)$. Thus, domestic private saving financed more than $69.5 \%$ of the 2009 government deficit $(58,588 / 84,249)$.

To finance the rest of the government deficit, foreign imports $(M)$ would have to exceed exports $(X)$ by $C \$ 25,661$. The actual trade deficit (amount of foreign borrowing) in 2009 was $C \$ 26,169$, slightly greater than the amount required. This difference largely results from the statistical discrepancy caused by different data sources being used for expenditure-based and income-based estimates of GDP.

Equation 2 is the key relationship that must hold in order for aggregate income and aggregate expenditure to be equal. Up to this point, we have treated it as simply an accounting identity. We now need to think of it as the outcome of explicit decisions on the part of households, businesses, government, and people in other countries. When we do so, we face the question of what underlies these decisions and how the requisite balance is established. Economists have found that the dominant determinant of consumption spending is disposable income $\left(Y-S_{B}-T\right)$. This relationship can be expressed formally by indicating that consumption is a function $C(\cdot)$ of disposable income,

$$
C=C\left(Y-S_{B}-T\right)
$$

or, dropping the technically correct but practically insignificant adjustment for retained earnings and depreciation $\left(S_{B}\right)$, a function of GDP minus net taxes:

$$
C=C(Y-T)
$$

When households receive an additional unit of income, some proportion of this additional income is spent and the remainder is saved. The marginal propensity to consume (MPC) represents the proportion of an additional unit of disposable income that is consumed or spent. Because the amount that is not spent is saved, the marginal propensity to save (MPS) is MPS $=1-$ MPC.

According to the consumption function, either an increase in real income or a decrease in taxes will increase aggregate consumption. Somewhat more sophisticated models of consumption recognize that consumption depends not only on current disposable income but also on wealth. Except for the very rich, individuals tend to spend a higher fraction of their current income as their wealth increases because with higher current wealth, there is less need to save to provide for future consumption.

Exhibit 11 shows household consumption expenditures as a percentage of GDP for selected countries.

Exhibit 11: Household Final Consumption Expenditures as a Percentage of GDP, 2018

\begin{center}
\begin{tabular}{ll}
\hline
United States & $68.0 \%$ \\
Mexico & 64.8 \\
Italy & 60.3 \\
Japan & 55.6 \\
Canada & 57.9 \\
France & 53.9 \\
Germany & 52.1 \\
\hline
\end{tabular}
\end{center}

Source: The World Bank, \href{https://data.worldbank.org/indicator/NE.CON.PRVT.ZS}{https://data.worldbank.org/indicator/NE.CON.PRVT.ZS}

These figures reflect the average propensity to consume (APC)-that is, the ratio $C / Y$-rather than a measure of how the next unit of income would be divided between spending and saving, the MPC. However, they are reasonable proxies for the MPC in each country. Comparing Germany's 55.0\% APC with Mexico's 67.8\%, the implication is that the Mexican economy is more sensitive to changes in disposable household income than is the German economy. All else being equal, macroeconomic policies that increase disposable household income, such as lowering government taxes, would have a larger effect on the economies of Mexico (67.8\%) and the United States (68.3\%) than similar policies would have in Germany (55.0\%) or France (55.4\%).

Companies are the primary source of investment spending $(I)$. They make investment decisions in order to expand their stock of physical capital, such as building new factories or adding new equipment to existing facilities. A definition of physical capital is any manmade aid to production. Companies also buy investment goods, such as manufacturing plants and equipment to replace existing facilities and equipment that wear out. Total investment, including replacement of worn-out capital, is called gross investment, as opposed to net investment, which reflects only the addition of new capacity. GDP includes gross investment; hence the name gross domestic product. Total investment spending in such developed countries as Italy, Germany, the United Kingdom, and the United States ranged between 12\% and 16\% of GDP in 2015 (OECD. Stat Extracts: Country Statistical Profiles 2017, \href{http://stats.oecd.org}{stats.oecd.org}).

Investment decisions depend primarily on two factors: the level of interest rates and aggregate output/income. The level of interest rates reflects the cost of financing investment. The level of aggregate output serves as a proxy for the expected profitability of new investments. When an economy is underutilizing its resources, interest rates are typically very low, and yet investment spending often remains dormant because the expected return on new investments is also low. Conversely, when output is high and companies have little spare capacity, the expected return on new investments is high. Thus, investment decisions may be modeled as a decreasing function $\boldsymbol{I}(\cdot, \cdot)$ of the real interest rate (nominal interest rate minus the expected rate of inflation) and an increasing function of the level of aggregate output. Formally,

$$
I=\boldsymbol{I}(r, Y)
$$

where $I$ is investment spending, $r$ is the real interest rate, and $Y$ is, as usual, aggregate income. This investment function leaves out some important drivers of investment decisions, such as the availability of new and better technology. Nonetheless, it reflects the two most important considerations: the cost of funding (represented by the real interest rate) and the expected profitability of the new capital (proxied by the level of aggregate output).

Many government spending decisions are insensitive to the current level of economic activity, the level of interest rates, the currency exchange rate, and other economic factors. Thus, economists often treat the level of government spending on goods and services $(G)$ as an exogenous policy variable determined outside the macroeconomic model. In essence, this means that the adjustments required to maintain the balance among aggregate spending, income, and output must occur primarily within the private sector.

Tax policy may also be viewed as an exogenous policy tool. The actual amount of net taxes $(T)$ collected, however, is closely tied to the level of economic activity. Most countries impose income taxes or value-added taxes (VAT) or both that increase with the level of income or expenditure. Similarly, at least some transfer payments to the household sector are usually based on economic need and are hence inversely related to aggregate income. Each of these factors makes net taxes ( $T)$ rise and fall with aggregate income, $Y$. The government's fiscal balance can be represented as

$$
G-T=\bar{G}-t(Y)
$$

where $\overline{\boldsymbol{G}}$ is the exogenous level of government expenditure and $t(Y)$ indicates that net taxes are an (increasing) function of aggregate income, $Y$. The fiscal balance decreases (smaller deficit or larger surplus) as aggregate income ( $Y$ ) increases and increases as income declines. This effect is called an automatic stabilizer because it tends to mitigate changes in aggregate output.

Net exports $(X-M)$ are primarily a function of income in the domestic country and in the rest of the world and of the relative prices of domestic and non-domestic goods and services. As domestic income rises, some of the additional demand that is induced will be for imported goods. Thus, net exports will decline. An increase in income in the rest of the world will lead to an increase in demand for the domestic country's products and hence an increase net exports. A decrease in the relative price of domestically produced goods and services, perhaps because of a depreciation in the domestic currency, will shift demand toward these products and hence increase net exports.

\section{AGGREGATE DEMAND AND AGGREGATE SUPPLY}
explain how the aggregate demand curve is generated
explain the aggregate supply curve in the short run and long run
explain causes of movements along and shifts in aggregate demand
and supply curves
describe how fluctuations in aggregate demand and aggregate supply
cause short-run changes in the economy and the business cycle
distinguish among the following types of macroeconomic equilibria:
long-run full employment, short-run recessionary gap, short-run
inflationary gap, and short-run stagflation
explain how a short-run macroeconomic equilibrium may occur at a
level above or below full employment
analyze the effect of combined changes in aggregate supply and
demand on the economy

Short-term fluctuations in an economy's output arise for a number of reasons. The fluctuations of output and price level can have a significant effect on issuers of securities and therefore come under close scrutiny by practitioners and policymakers.

We will build a model of aggregate demand and aggregate supply and use it to discuss how aggregate output and the level of prices are determined in the economy. Aggregate demand (AD) represents the quantity of goods and services that households, businesses, government, and international customers want to buy at any given level of prices. Aggregate supply (AS) represents the quantity of goods and services that producers are willing to supply at any given level of prices. It also reflects the amount of labor and capital that households are willing to offer into the marketplace at given real wage rates and cost of capital.

\section{Aggregate Demand}
As we will see, the aggregate demand curve looks like the ordinary demand curves that we encounter in microeconomics: quantity demanded increases as the price level declines. But our intuitive understanding of that relationship-lower price allows us to buy more of a good with a given level of income-does not apply here because income is not fixed. Instead, aggregate income/expenditure (GDP) is to be determined within the model along with the price level. Thus, we need to explain the relationship between price and quantity demanded somewhat differently.

The aggregate demand curve represents the combinations of aggregate income and the price level at which two conditions are satisfied. First, aggregate expenditure equals aggregate income. As indicated in our discussion of GDP accounting, this relation must always be true after the fact. Second, the available real money supply is willingly held by households and businesses. Recall that GDP is defined as

$$
\mathrm{GDP}=C+I+G+(X-M)
$$

where

$$
\begin{aligned}
C & =\text { consumer spending on final goods and services } \\
I & =\text { gross private domestic investment } \\
G & =\text { government spending on final goods and services } \\
X & =\text { exports } \\
M & =\text { imports }
\end{aligned}
$$

We assume that government spending $(G)$ is exogenous, set by government policy. To explain why the aggregate demand curve slopes downward, we analyze how the price level in the economy affects the consumption $(C)$, investment $(I)$, and net export $(X-M)$ components of GDP. The downward slope of the aggregate demand curve results from three effects: the wealth effect, the interest rate effect, and the real exchange rate effect. We assume that the nominal money supply is held constant.

\section{Wealth Effect}
The wealth effect is based on the concept of purchasing power of nominal wealth, including nominal value of the money held by consumers, physically or in a bank account. Nominal wealth does not change: One British pound is always worth one British pound. Similarly, one euro is worth one euro, and one yen is worth one yen. The real value, however-the value of money in terms of goods and services-is not fixed. It fluctuates with the prices of goods and services. Assume that our favorite loaf of bread costs one pound $(\pounds 1=$ one loaf of bread). If we have $\pounds 10$ in a pocket, then we have money that is equivalent to 10 loaves of bread. The real money holdings are therefore 10 loaves of bread.

$$
\frac{\pounds 10}{\left(\frac{\pounds 1}{\text { loaf }}\right)}=10 \text { loaves }
$$

If the price of a loaf of bread increases to $\pounds 2$ ( $\pounds 2=$ one loaf of bread), the same $\pounds 10$ is now worth only five loaves of bread. Our real money holdings, and therefore our real wealth, have decreased. Conversely, if the price of a loaf of bread decreases to one-half pound ( $\pounds 0.50=$ one loaf of bread), the same $\pounds 10$ is now worth 20 loaves of bread. Our real money holdings, and therefore our real wealth, have increased.

The effect of the change in the price level and the resulting change in the real value of money holdings give the wealth effect its name, because as the real value of our money holdings changes, so does our real wealth.

The wealth effect is one reason that the aggregate demand curve is downward sloping. An increase in the price level decreases the quantity of goods and services that can be purchased with the fixed quantity of nominal wealth-consumers are less wealthy (in real terms) and therefore demand fewer goods and services. Exhibit 12 illustrates this dynamic as the movement from Point A to Point B. An increase in the price level from $P_{1}$ to $P_{2}$ reduces income from $Y_{1}$ to $Y_{2}$. Conversely, a decrease in the price level increases the quantity of goods and services that can be purchased with the fixed quantity of money available-consumers are wealthier (in real terms) and therefore demand more goods and services.

\section{Exhibit 12: The Aggregate Demand Curve}
\begin{center}
\includegraphics[max width=\textwidth]{2023_05_04_ccb578c5089fd41615beg-158}
\end{center}

\section{The Interest Rate Effect}
When the price level changes, the demand for money also changes. For example, if the price for your lunch today is 10 euros ( $€ 10=$ lunch), then we need to have 10 euros in our pocket to eat lunch. Now, assume that the price of lunch doubles to 20 euros $(€ 20=$ lunch). If we want to eat lunch, we need to get more money-our demand for money increases. With a fixed supply of money, the price of money increases. Because the price of money is the interest rate, then as the price level increases, the interest rate increases.

The converse is also true. If the price of lunch decreases to five euros ( $€ 5=$ lunch), we don't need to carry around as much money as before - our demand for money decreases. With a fixed supply of money, the price of money decreases. Therefore, as the price level decreases, the interest rate decreases.

Why not just keep extra "lunches" (money) in our pocket? Because holding extra money in such liquid form means that we are giving up the interest that we could be earning by lending the money out to others (e.g. through purchasing a government or corporate bond or just leaving the money in a bank account and letting the bank do the lending for us). The interest we would earn is the opportunity cost of keeping money in our pocket. So, when the price level decreases, we will want to keep more of our money in interest-bearing assets and our demand for money decreases.

Changes in the interest rate affect the quantity of goods and services demanded. If interest rates increase, businesses invest less because their borrowing costs increase. This shift leads to fewer profitable projects to invest in and also negatively affects the demand for commercial real estate. In addition, higher interest rates lead to lower consumption-especially for large purchases such as automobiles or residential real estate, which are usually purchased with loans. If interest rates decrease, then businesses have more profitable projects to invest in, and consumers would be expected to spend more.

This leads us to the interest rate effect. A higher price level creates greater demand for money, which raises the interest rate. The higher interest rate decreases demand for investment and consumption expenditures, which leads to less demand for goods and services. Conversely, a lower price level leads to a lower demand for money, which leads to a lower interest rate. The lower interest rate increases demand for investment and consumption expenditures, which leads to more demand for goods and services.

\section{The Real Exchange Rate Effect}
An increase in the domestic price level causes appreciation of the real exchange rate. Real exchange rates will be explained later, but for the purpose of this discussion, it is important to understand that the real exchange rate is the observed, quoted nominal exchange rate adjusted for price level. So a higher price level affects the real exchange rate and makes domestic goods more expensive in other countries, reducing exports. It also makes non-domestic goods less expensive domestically, increasing imports. The result is lower demand for domestic goods and services. Conversely, a decrease in the domestic price level (assuming the price level abroad remains unchanged) leads to a depreciation of the real exchange rate. This decrease in the real exchange rate makes domestic goods less expensive in other countries, increasing exports, and non-domestic goods more expensive domestically, decreasing imports. The result is higher demand for domestic goods and services. This dynamic gives us the real exchange rate effect.

An additional channel affecting exchange rates involves the interest rate effect. When interest rates increase (because of a higher price level), non-domestic investors increase their demand for the domestic currency in the foreign exchange market because they want to earn the higher return on their savings. This increased demand causes the domestic currency to appreciate, which in turn increases the real exchange rate.

When interest rates decrease (because of a lower price level), a similar but opposite effect occurs. Domestic savers seek higher returns in non-domestic markets. This increased supply of the domestic currency in the foreign exchange market causes the domestic currency to depreciate, which in turn decreases the real exchange rate.

The interest rate effect just described is an additional channel through which a higher price level leads to a higher real exchange rate. The two effects are complementary, operate in the same direction and magnify the impact of the change in price level on aggregate demand. Exhibit 13 provides a summary.

\section{Exhibit 13: Deriving Aggregate Demand Curve: Summary of Key Effects}
\begin{center}
\includegraphics[max width=\textwidth]{2023_05_04_ccb578c5089fd41615beg-159}
\end{center}

\section{EXAMPLE 5}
\section{Slope of the Aggregate Demand Curve}
\begin{enumerate}
  \item Considering the movement along the aggregate demand curve, which of the following statements best explains the interest rate effect when the price level decreases?
\end{enumerate}

A. Decrease in money demand, with a fixed money supply, leads to lower interest rates. Lower interest rates induce businesses to invest more because there is now a greater number of profitable projects. Consumers also spend more when interest rates are lower. Lower interest rates also weaken the exchange rate, boosting net exports. These effects cause output to be higher at a lower price level.

B. Decrease in money demand, with a fixed money supply, leads to lower interest rates. Lower interest rates induce businesses to invest less because there are now fewer profitable projects. Lower interest rates also weaken the exchange rate, boosting net exports.

C. Decrease in money demand, with a fixed money supply, leads to higher interest rates, which induce businesses to invest less and consumers to save more. Higher interest rates cause the exchange rate to appreciate, boosting net exports.

\section{Solution:}
A is correct. B is incorrect because the statement suggests that lower interest rates lead to less business investment. $C$ is incorrect because the statement suggests that a lower price level leads to higher interest rates.

\section{Aggregate Supply}
Aggregate demand tells us only the relationship between the price level and the amount of output demanded at those prices. To understand what price and output level will prevail in the economy, we need to add aggregate supply, the amount of output producers are willing to provide at various prices. The aggregate supply curve (AS curve) represents the level of domestic output that companies will produce at each price level. Unlike the demand side, we must distinguish between the short- and long-run AS curves, which differ with respect to how wages and other input prices respond to changes in final output prices. "Long run" and "short run" are relative terms and are necessarily imprecise with respect to calendar time. The "long run" is long enough that wages, prices, and expectations can adjust but not long enough that physical capital is a variable input. Capital and the available technology to use that capital remain fixed. This condition implies a period of at least a few years and perhaps a decade. The truly long run, in which even the capital stock is variable, may be thought of as covering multiple decades. Consideration of the very long run is postponed to our subsequent discussion of economic growth.

In the very short run, perhaps a few months or quarters, companies will increase or decrease output to some degree without changing prices. Exhibit 14 illustrates this dynamic with the horizontal line labeled VSRAS. If demand is somewhat stronger than expected, companies earn higher profit by increasing output as long as they can cover their variable costs. So they will run their plant and equipment more intensively, demand more effort from their salaried employees, and increase the hours of employees who are paid on the basis of hours worked. If demand is somewhat weaker than projected, companies can run their plants less intensively, cut labor hours, and use staff to perform maintenance and carry out efficiency-enhancing projects that are often postponed during busier periods.

\section{Exhibit 14: Aggregate Supply Curve}
\begin{center}
\includegraphics[max width=\textwidth]{2023_05_04_ccb578c5089fd41615beg-161}
\end{center}

Over somewhat longer periods, the AS curve is upward sloping because more costs become variable. This dynamic is represented by the short-run aggregate supply (SRAS) curve in Exhibit 14. In most businesses, wages are adjusted once a year, but for companies with union contracts, several years may pass before the contracts expire. The prices for raw materials and other inputs may also be established under long-term contracts. Hence, wages and other input costs are relatively inflexible in the short run and do not fully adjust to changes in output prices. As the price level rises, most companies enjoy higher profit margins and hence expand production. In Exhibit 14, when prices move from $P_{1}$ to $P_{2}$, the quantity of aggregate output supplied increases from $Y_{1}$ to $Y_{2}$. Conversely, a reduction in the price level squeezes profit margins and causes companies to reduce production.

Over time, however, wages and other input prices tend to "catch up" with the prices of final goods and services. In other words, wages and prices that are inflexible or slow to adjust in the short run adjust to changes in the price level over the long run. Thus, over the long run, when the aggregate price level changes, wages and other input prices change proportionately so that the higher aggregate price level has no effect on aggregate supply. This dynamic is illustrated by the vertical long-run aggregate supply (LRAS) curve in Exhibit 14 . As prices move from $P_{1}$ to $P_{2}$, the quantity of output supplied remains at $Y_{1}$ in the long run. The only change that occurs is that prices shift to a higher level (from $P_{1}$ to $P_{2}$ ).

The position of the LRAS curve is determined by the potential output of the economy. The amount of output produced depends on the fixed amount of capital and labor and the available technology. This classical model of aggregate supply can be expressed as

$$
Y=\boldsymbol{F}(\bar{K}, \bar{L})=\bar{Y}
$$

where $\bar{K}$ is the fixed amount of capital and $\bar{L}$ is the available labor supply. The stock of capital is assumed to incorporate the existing technological base. Note that investment (I) reflects replacement of worn-out capital plus the change in capital, $\Delta K$. Over short periods, net investment is assumed to have a negligible effect on aggregate supply. The available labor supply is also held constant, and workers are assumed to have a given set of skills. The long-run equilibrium level of output, $Y_{1}$ in Exhibit 14, is referred to as the full employment, or natural, level of output. At this level of output, the economy's resources are deemed to be fully employed, and (labor) unemployment is at its natural rate. This concept of a natural rate of unemployment assumes the macroeconomy is currently operating at an efficient and unconstrained level of production. Companies have enough spare capacity to avoid bottlenecks, and there is a modest, stable pool of unemployed workers (job seekers equal job vacancies) looking for and transitioning into new jobs.

\section{Shifts in the Aggregate Demand Curve}
In the next two sections, the aggregate demand (AD) and aggregate supply (AS) models are used to address three critical macroeconomic questions:

\begin{enumerate}
  \item What causes an economy to expand or contract?

  \item What causes inflation and changes in the level of unemployment?

  \item What determines an economy's rate of sustainable growth, and how can it be measured?

\end{enumerate}

Before addressing these questions, we need to distinguish between 1) the long-run growth rate of real GDP and 2) short-run fluctuations in real GDP around this long-run trend.

The business cycle is a direct result of short-term fluctuations of real GDP. It consists of periods of economic expansion and contraction. In an expansion, real GDP is increasing, the unemployment rate is declining, and capacity utilization is rising. In a contraction, real GDP is decreasing, the unemployment rate is rising, and capacity utilization is declining. Shifts in the AD and AS curves determine the short-run changes in the economy associated with the business cycle. In addition, the AD-AS model provides a framework for estimating the sustainable growth rate of an economy, which is addressed in Section 4.

From an asset allocation perspective, it is important to determine the current phase of the business cycle as well as how fast the economy is growing relative to its sustainable growth rate. The expected rate of return on equities and fixed-income securities, for example, depends on estimates of the growth rate of GDP and inflation. For equities, GDP growth is the primary determinant of aggregate corporate profits. For fixed-income securities, the expected rate of inflation determines the spread between real and nominal rates of return. In order to use the AD and AS model to analyze the economy and to make investment decisions, we need to first understand what factors cause the curves to shift.

\section{Shifts in Aggregate Demand}
In addition to price, factors that influence the level of spending by households, companies, governments, and people in other countries (i.e., the aggregate level of expenditures) will cause the AD curve to shift. A shift to the right represents an increase in aggregate demand at any price level. Exhibit 15 shows this dynamic as a shift from $\mathrm{AD}_{1}$ to $\mathrm{AD}_{2}$. A shift to the left represents a decrease in aggregate demand at any price level. This dynamic is indicated by a move from $\mathrm{AD}_{1}$ to $\mathrm{AD}_{3}$. Key factors that directly or indirectly influence the level of aggregate expenditures and cause the aggregate demand curve to shift include changes in the following:

\begin{itemize}
  \item household wealth,

  \item consumer and business expectations,

  \item capacity utilization,

  \item monetary policy,

  \item the exchange rate,

  \item growth in the global economy, and

  \item fiscal policy (government spending and taxes).

\end{itemize}

\section{Exhibit 15: Shifts in the Aggregate Demand Curve}
\begin{center}
\includegraphics[max width=\textwidth]{2023_05_04_ccb578c5089fd41615beg-163}
\end{center}

Income, Output, Y

\section{Household Wealth}
Household wealth includes the value of both financial assets (e.g., cash, savings accounts, investment securities, and pensions) and real assets (e.g., real estate). The primary reason households save a portion of their current income is to accumulate wealth for consumption in the future. The proportion of disposable income that households save depends partly on the value of the financial and real assets that they have already accumulated. If these assets increase in value, households will tend to save less and spend a greater proportion of their income because they will still be able to meet their wealth accumulation goals. As a result, an increase in household wealth increases consumer spending and shifts the aggregate demand curve to the right. In contrast, a decline in wealth will reduce consumer spending and shift the AD curve to the left. This dynamic, often referred to as the wealth effect, provides one explanation for how changes in equity prices affect economic activity. Higher equity prices increase household wealth, which increases consumer spending and reduces the amount saved out of current income. Economic studies estimate that an increase or decrease in wealth in developed countries increases or decreases annual consumer spending by 3\%-7\% of the change in wealth (Case, Quigley, and Shiller 2005). A smaller but still statistically significant wealth effect has been found in a number of emerging markets (developing countries) (Funke 2004). Exhibit 16: Housing Prices and the Saving Rate in the United Kingdom, 2000-2019

\begin{center}
\includegraphics[max width=\textwidth]{2023_05_04_ccb578c5089fd41615beg-164}
\end{center}

Source: Office of National Statistics, United Kingdom.

\section{EXAMPLE 6}
\section{The Wealth Effect on Saving and Consumption}
\begin{enumerate}
  \item The importance of the wealth effect on consumption, and its relationship to housing prices, was evident in the sharp global economic downturn. A major factor associated with the economic downturn was the sharp fall in housing prices, especially in countries that experienced a housing boom earlier in the decade, such as the United States, the United Kingdom, Spain, and Ireland. In each of these countries, consumers reduced spending sharply and raised the level of saving in response to the decline in wealth. Do the data in Exhibit 16 provide support for the wealth effect?
\end{enumerate}

\section{Solution:}
Housing prices in the United Kingdom rose by nearly $111 \%$ [(211.1 100)/100] between 2000 and 2007. During this period, the saving rate averaged 7.2\% of income up until 2008. As predicted, when housing prices fell by $11 \%$ between 2008 and 2009, the saving rate rose dramatically from $7.2 \%$ in 2008 to $12.4 \%$ in 2010 . Of course, the decline in housing prices was not the only factor contributing to the increase in the saving rate. Stock prices also declined during this period, further reducing wealth in the United Kingdom, and the slowdown raised uncertainty over future jobs and income. Then, as the economy recovered, house prices rose again from 2013, which led to a (lagged) decline in the saving rate.

\section{Consumer and Business Expectations}
Psychology has an important impact on consumer and business spending. When consumers are confident about their future income and the stability/safety of their jobs, they tend to spend a higher portion of their disposable income. This dynamic shifts the AD curve to the right. Consumer spending declines and the AD curve shifts to the left when consumers become less confident. Similarly, when businesses are optimistic about their future growth and profitability, they spend (invest) more on capital projects, which also shifts the AD curve to the right.

\section{Capacity Utilization}
Capacity utilization is a measure of how fully an economy's production capacity is being used. Companies with excess capacity have little incentive to invest in new property, plant, and equipment. In contrast, when companies are operating at or near full capacity, they will need to increase investment spending in order to expand production. Data from the OECD and the US Federal Reserve indicate that when aggregate capacity utilization reaches $82 \%-85 \%$, production blockages arise, prompting companies to increase their level of investment spending. This dynamic shifts the AD curve to the right.

Fiscal Policy

Fiscal policy is the use of taxes and government spending to affect the level of aggregate expenditures. An increase in government spending, one of the direct components of $\mathrm{AD}$, shifts the $\mathrm{AD}$ curve to the right, whereas a decrease in government spending shifts the AD curve to the left. Taxes affect GDP indirectly through their effect on consumer spending and business investment. Lower taxes will increase the proportion of personal income and corporate pre-tax profits that consumers and businesses have available to spend and will shift the AD curve to the right. In contrast, higher taxes will shift the AD curve to the left.

\section{Monetary Policy}
Money is generally defined as currency in circulation plus deposits at commercial banks. Monetary policy refers to action taken by a nation's central bank to affect aggregate output and prices through changes in bank reserves, reserve requirements, or its target interest rate.

Most countries have fractional reserve banking systems in which each bank must hold reserves (vault cash plus deposits at the central bank) at least equal to the required reserve ratio times its customer deposits. Banks with excess reserves can lend them to banks that need reserves to meet their reserve requirements. The central bank can increase the money supply by 1) buying securities from banks, 2) lowering the required reserve ratio if such a tool is available, and/or 3) reducing its target for the interest rate at which banks borrow and lend reserves among themselves. In each case, the opposite action would decrease the money supply.

When the central bank buys securities from banks in an open-market operation, it pays for them with a corresponding increase in bank reserves. This dynamic increases the amount of deposits banks can accept from their customers-that is, the money supply. Similarly, cutting the required reserve ratio increases the level of deposits (i.e., money) consistent with a given level of reserves in the system. If the central bank chooses to target an interbank lending rate, as the Federal Reserve targets the federal funds rate in the United States, then it must add or drain reserves via open-market operations to maintain the target interest rate. If it raises (lowers) its target interest rate, it will have to drain (add) reserves in order to make reserves more (less) expensive in the interbank market. Thus, open-market operations and interest rate targeting are very closely related. The main distinction is whether the central bank chooses to target a level of reserves and let the market determine the interest rate or chooses to target the interest rate and let the market (i.e., banks) determine the level of reserves they desire to hold at that rate.

An increase in the money supply shifts the AD curve to the right so that each price level corresponds to a higher level of income and expenditure. There are various channels through which the additional expenditures may be induced. For example, the interest rate reduction required to induce investors to hold the additional money balances will encourage companies to invest more and households to borrow to purchase durable goods, such as cars. In addition, banks may facilitate greater expenditure by raising credit limits and loosening credit standards. Conversely, a reduction in the money supply shifts the AD curve to the left.

Exhibit 17 illustrates the short-run and long-run effect of expansionary monetary policy. Suppose the central bank expands the money supply in an attempt to stimulate demand when the economy is already in long-run equilibrium. The expansionary policy will shift the $\mathrm{AD}$ curve to the right, from $\mathrm{AD}_{1}$ to $\mathrm{AD}_{2}$. In the very short run, output will expand from $Y_{1}$ to $Y_{2}$ without an increase in the price level. After operating at higher-than-normal production rates for a few months or quarters, companies will begin to push for price increases and input prices will begin to rise as well. The aggregate supply curve will steepen, and prices will increase to $P_{3}$ while output declines to $Y_{3}$. As input prices become more flexible, the AS curve will steepen until, in the long run, it is vertical and output has returned to the long-run natural level, $Y_{1}$, with prices rising to $P_{4}$. Thus, expanding the money supply increases output in the short run, but in the long run it affects only the price level.

\section{Exhibit 17: Short-Run and Long-Run Effects of Monetary Expansion}
\begin{center}
\includegraphics[max width=\textwidth]{2023_05_04_ccb578c5089fd41615beg-166}
\end{center}

\section{Exchange Rate}
Changes in the exchange rate affect the price of exports and imports and thus aggregate demand. For example, a lower euro relative to other currencies makes European exports cheaper in world markets and non-European products sold in Europe (European imports) more expensive. Therefore, a lower euro should cause European exports to increase and imports to decline, leading the AD curve to shift to the right. Conversely, a stronger euro reduces exports and raises imports, and the AD curve shifts to the left.

\section{Growth in the Global Economy}
International trade is what links countries together and creates a global economy. Faster economic growth in non-domestic markets encourages people in other countries to buy more products from domestic producers and increases exports. For example, rapid GDP growth in ASEAN member countries has increased their demand for products from other nations. Japan has benefited from this rapid growth because it has exported more products to ASEAN countries. In terms of the AD and AS model, the AD curve for Japan has shifted to the right because of increased demand for Japanese products in ASEAN countries, resulting in higher exports. A decline in the growth rates of ASEAN members' economies would have a negative effect on the Japanese economy because exports would be lower. This change would cause the Japanese AD curve to shift to the left.

What happens to interest rates when the AD curve shifts? In the case of an increase in the money supply, the interest rate declines at each price level because the increase in income $(Y)$ increases saving, and rates must decline to induce a corresponding increase in investment spending $(I)$. In each of the other cases considered above, a rightward shift in the AD curve will increase the interest rate at each price level. With the real money supply held constant, the interest rate must rise as income increases. The increase in the interest rate reduces the demand for money at each level of expenditure/income and, therefore, allows expenditure/income to increase without an increase in the money supply. In terms of the quantity theory of money equation, this shift corresponds to a higher velocity of money, $V$.

Exhibit 18 summarizes the main factors that shift the AD curve. In each case, the impact of the factor is considered in isolation. In practice, however, various factors may be at work simultaneously and there may be interaction among them. This is especially true with regard to expectational factors-consumer and business confidence-which are likely to be influenced by other developments.

\section{Exhibit 18: Impact of Factors Shifting Aggregate Demand}
\begin{center}
\begin{tabular}{|c|c|c|}
\hline
An Increase in the Following Factors & Shifts the AD Curve & Reason \\
\hline
Stock prices & Rightward: Increase in AD & Higher consumption \\
\hline
Housing prices & Rightward: Increase in AD & Higher consumption \\
\hline
Consumer confidence & Rightward: Increase in AD & Higher consumption \\
\hline
Business confidence & Rightward: Increase in AD & Higher investment \\
\hline
Capacity utilization & Rightward: Increase in AD & Higher investment \\
\hline
Government spending & Rightward: Increase in AD & Government spending a component of AD \\
\hline
Taxes & Leftward: Decrease in AD & Lower consumption and investment \\
\hline
Bank reserves & Rightward: Increase in AD & $\begin{array}{l}\text { Lower interest rate, higher investment and } \\ \text { possibly higher consumption }\end{array}$ \\
\hline
$\begin{array}{l}\text { Exchange rate (foreign currency per unit } \\ \text { domestic currency) }\end{array}$ & Leftward: Decrease in AD & Lower exports and higher imports \\
\hline
Global growth & Rightward: Increase in AD & Higher exports \\
\hline
\end{tabular}
\end{center}

\section{EXAMPLE 7}
\section{Shifts in Aggregate Demand}
Francois Ubert is a portfolio manager with EuroWorld, a French investment management firm. Ubert is considering increasing his clients' portfolio exposure to Brazilian equities. Before doing so, he asks you to prepare a report on the following recent economic events in Brazil and to summarize the effect of each event on the Brazilian economy and on Brazilian equity and fixed-income securities.

\begin{enumerate}
  \item The Brazilian central bank reduced bank reserves, resulting in a lower money supply.
\end{enumerate}

\section{Solution to 1:}
This monetary policy action is designed to reduce consumption and business investment spending. The reduction in real money balances will increase interest rates and discourage lending within the banking system. Higher interest rates and tighter credit will reduce both investment and consumption expenditures and shift the AD curve to the left. The prices of fixed-income securities will fall because of the rise in interest rates. The reduction in aggregate output should lower corporate profits, and it is likely that equity prices will also fall.

\begin{enumerate}
  \setcounter{enumi}{1}
  \item The capacity utilization rate in Brazil is currently estimated to be $86.4 \%$, a $2.7 \%$ increase from the previous year.
\end{enumerate}

\section{Solution to 2:}
Capacity utilization is a key factor determining the level of investment spending. A current utilization rate of more than $86 \%$ and an increase from the previous year indicate a growing lack of spare capacity in the Brazilian economy. As a result, businesses will probably increase their level of capital spending. This change will increase $\mathrm{AD}$ and shift the $\mathrm{AD}$ curve to the right. Higher economic activity (income/output) will cause upward pressure on interest rates and may have a negative effect on fixed-income securities. Higher income/output should increase corporate profits and is likely to have a positive effect on equity securities.

\begin{enumerate}
  \setcounter{enumi}{2}
  \item Corporate profits reported by Brazilian companies increased by $30 \%$ over last year's levels, and corporations have revised their forecasts of future profitability upward.
\end{enumerate}

\section{Solution to 3:}
Expected corporate profits are an important determinant of the level of investment spending. The large increase in expected profits will raise the level of investment spending and increase aggregate demand. This change will shift the AD curve to the right. The increase in corporate profits and the resulting increase in economic output should have a positive effect on equities. The increase in output will put upward pressure on interest rates and downward pressure on the prices of fixed-income securities.

\begin{enumerate}
  \setcounter{enumi}{3}
  \item The government recently announced that it plans to start construction on a number of hydroelectric projects to increase renewable energy sources.
\end{enumerate}

\section{Solution to 4:}
Fiscal policy uses government spending to influence the level and growth rate of economic activity. The announcement indicates an increase in government spending, which is a direct component of AD. Therefore, higher spending on the projects will increase AD and shift the AD curve to the right. The increase in output and expenditure should be positive for equi- ties. But it will be negative for existing fixed-income investments because higher interest rates will be required to induce investors to buy and hold the government debt issued to fund the new projects.

\begin{enumerate}
  \setcounter{enumi}{4}
  \item Forecasts by private sector economists project that the European economy will enter a sharp slowdown and decline in the next year.
\end{enumerate}

\section{Solution to 5:}
A sharp slowdown in Europe will decrease the demand for Brazilian exports by European households and businesses and shift the AD curve to the left. The resulting decline in income and downward pressure on prices will be positive for fixed-income securities but negative for equities.

\section{Shifts in the Short-Run Aggregate Supply Curve}
Factors that change the cost of production or expected profit margins will cause the SRAS curve to shift. These factors include changes in

\begin{itemize}
  \item nominal wages,

  \item input prices, including the price of natural resources,

  \item expectations about future output prices and the overall price level,

  \item business taxes and subsidies, and

  \item the exchange rate.

\end{itemize}

In addition, factors that shift the long-run AS curve (discussed elsewhere) will also shift the SRAS curve by a corresponding amount because the SRAS and LRAS reflect the same underlying resources and technology. As the economy's resources and technology change, the full employment (or natural) level of output changes, and both the LRAS and SRAS shift accordingly.

\section{Change in Nominal Wages}
Changes in nominal wages shift the short-run AS curve because wages are often the largest component of a company's costs. An increase in nominal wages raises production costs, resulting in a decrease in AS and a leftward shift in the SRAS curve. Lower wages shift the AS curve to the right. It is important to note that changes in nominal wages have no effect on the LRAS curve.

A better way to measure the impact of labor costs on the AS curve is to measure the change in unit labor cost. We define the change in unit labor cost as

$\%$ Change in unit labor cost $=\%$ Change in nominal wages

$-\%$ Change in productivity

\section{EXAMPLE 8}
\section{Unit Labor Cost and Short-Run Aggregate Supply}
\begin{enumerate}
  \item Suppose Finnish workers are paid $€ 20$ an hour and are able to produce 100 cell phones in an hour. The labor cost per cell phone is $€ 0.20$ ( $€ 20$ divided by 100 units). If the wages per hour for Finnish workers rise by $10 \%$ from $€ 20$ to $€ 22$ and they are able to raise their productivity by $10 \%$, what is the impact on unit labor cost and the short-run aggregate supply curve?
\end{enumerate}

Solution:

The workers can now produce 110 cell phones per hour, and unit labor cost will not change $(22 / 110=0.20)$. In this case, the SRAS curve will remain in its original position. If wages had increased by $20 \%$ instead of $10 \%$, then unit labor cost would have increased and the SRAS would shift to the left. Conversely, if the wage increase were only $5 \%$, then unit labor cost would have decreased and the SRAS would shift to the right.

\section{Change in Input Prices}
The price of raw materials is an important component of cost for many businesses. Lower input prices reduce the cost of production, which, in turn, makes companies willing to produce more at any output price. This dynamic is reflected in a rightward shift of the SRAS curve. Conversely, higher input prices increase production costs, which, in turn, causes companies to reduce production at any output price. This dynamic shifts the SRAS curve to the left. During the 1970s, high oil prices caused the SRAS curve in most countries to shift to the left. In contrast, in the mid-1980s, declining oil prices lowered the cost of production and shifted the SRAS curve in most countries to the right. Oil prices currently have a smaller effect on the global economy than in the 1970s and 1980s because most countries have reduced their reliance on oil and improved their energy efficiency so that they now use less energy per unit of GDP.

\section{Change in Expectations about Future Prices}
The impact of expected future prices on current output decisions is not as straightforward as it might seem. First, each company is primarily concerned about the price of its own output rather than the general price level. The latter may be more reflective of its costs. If it expects its own output price to rise (fall) relative to the general price level, then it may increase (decrease) production in response to the perceived change in its profit margin. As more and more companies become optimistic (pessimistic) about their ability to raise the relative price of their product, the SRAS will shift to the right (left). In the aggregate, of course, companies can neither raise nor lower their prices relative to the general price level. Hence, shifts in the SRAS driven by such price expectations are likely to be modest and temporary. Second, considering future prices introduces a temporal aspect into decision making. If the future price level is expected to be higher, companies may decide to produce more today in order to expand inventory available for future sale. But they will only do so if the cost of carrying inventory (financing, storage, and spoilage) is less than they expect to save on production costs by producing more today and less in the future. Conversely, they may cut current production and sell out of existing inventory if they expect future prices (and costs) to be lower.

The upshot is that expectations of higher (lower) future prices are likely to shift the SRAS curve to the right (left), but the effect may be modest and/or temporary.

\section{Change in Business Taxes and Subsidies}
Higher business taxes increase production costs per unit and shift the short-run AS curve to the left. Business subsidies are a payment from the government to the producer. Subsidies for businesses lower their production costs and shift the SRAS curve to the right.

\section{Change in the Exchange Rate}
Many countries import raw materials, including energy and intermediate goods. As a result, changes in the exchange rate can affect the cost of production and, therefore, aggregate supply. A higher yen relative to the euro, for instance, would lower the cost of raw materials and intermediate goods imported to Japan from Europe. This, in turn, will lower the production costs of Japanese producers and shift the AS curve in Japan to the right. A lower yen would have the opposite effect.

\section{Shifts in the Long-Run Aggregate Supply Curve}
As discussed, the position of the LRAS curve is determined by the potential output of the economy. Potential GDP measures the productive capacity of the economy and is the level of real GDP that can be produced at full employment. Potential GDP is not a static concept but can increase each year at a steady rate as the economy's resource capacity grows. Therefore, any factor increasing the resource base of an economy causes the LRAS curve to shift as shown in Exhibit 19.

\section{Exhibit 19: Shift in the Long-Run Aggregate Supply (LRAS) Curve}
\begin{center}
\includegraphics[max width=\textwidth]{2023_05_04_ccb578c5089fd41615beg-171}
\end{center}

These factors include changes in

\begin{itemize}
  \item supply of labor and quality of labor forces (human capital),

  \item supply of natural resources,

  \item supply of physical capital, and

  \item productivity and technology.

\end{itemize}

\section{Supply of Labor}
The larger the supply of labor, the more output the economy can produce. The labor supply depends on growth in the population, the labor force participation rate (the percentage of the population working or looking for work), and net immigration. The determinants of the labor supply are discussed in more detail later in our coverage. Increases in the labor supply shift the LRAS curve to the right. Decreases shift the curve to the left.

\section{Supply of Natural Resources}
Natural resources are essential inputs to the production process and include everything from available land to oil to water. Increased availability of natural resources shifts the LRAS curve to the right.

\section{Supply of Physical Capital}
Investment in new property, plant, equipment, and software is an essential ingredient for growth. An increase in the stock of physical capital will increase the capacity of the economy to produce goods and services. Simply put, if workers are provided with more and better equipment to use, they should be able to produce more output than they could with the older equipment. Thus, strong growth in business investment, which increases the supply of physical capital, shifts the LRAS curve to the right.

\section{Supply of Human Capital}
Another way to raise the productive capacity of a country is to increase human capitalthe quality of the labor force-through training, skills development, and education. Improvement in the quality of the labor force shifts the LRAS curve to the right.

\section{Labor Productivity and Technology}
Another important factor affecting the productive capacity of an economy is how efficient labor is in transforming inputs into final goods and services. Productivity measures the efficiency of labor and is the amount of output produced by workers during a given period-for example, output per hour worked. An increase in productivity decreases labor cost, improves profitability, and results in higher output. Two of the main drivers of labor productivity-physical capital per worker and the quality of the workforce-have been discussed already. The third key determinant of productivity is technology. Advances in technology shift the LRAS curve to the right.

\section{EXAMPLE 9}
\section{Unit Labor Cost and Long-Run Aggregate Supply}
\begin{enumerate}
  \item Finnish workers are paid $€ 20$ per hour and are able to produce 100 cell phones in an hour. If workers develop a new technique for assembly and are able to produce 200 cell phones per hour, what is the effect on the long-run aggregate supply curve?
\end{enumerate}

\section{Solution:}
Labor cost per unit will decline to $€ 0.10(20 / 200=€ 0.10$ per cell phone). As a result, profit per unit will rise and companies will have an incentive to increase production. Thus, the LRAS curve will shift to the right.

Exhibit 20 summarizes the factors shifting the AS curve. Rightward shifts in the SRAS or LRAS curves are defined as an increase in supply, while leftward shifts represent a decrease in supply.

\section{Exhibit 20: Impact of Factors Shifting Aggregate Supply}
\begin{center}
\begin{tabular}{llll}
\hline
An Increase in & Shifts SRAS & Shifts LRAS & Reason \\
\hline
Supply of labor & Rightward & Rightward & Increases resource base \\
Supply of natural resources & Rightward & Rightward & Increases resource base \\
Supply of human capital & Rightward & Rightward & Increases resource base \\
Supply of physical capital & Rightward & Rightward & Increases resource base \\
Productivity and technology & Rightward & Rightward & Improves efficiency of inputs \\
Nominal wages & Leftward & No effect & Increases labor cost \\
Input prices (e.g., energy) & Leftward & No effect & Increases cost of production \\
\end{tabular}
\end{center}

\begin{center}
\begin{tabular}{|c|c|c|c|}
\hline
An Increase in & Shifts SRAS & Shifts LRAS & Reason \\
\hline
Expectation of future prices & Rightward & No effect & $\begin{array}{l}\text { Anticipation of higher costs and/or perception of } \\ \text { improved pricing power }\end{array}$ \\
\hline
Business taxes & Leftward & No effect & Increases cost of production \\
\hline
Subsidy & Rightward & No effect & Lowers cost of production \\
\hline
Exchange rate & Rightward & No effect & Lowers cost of production \\
\hline
\end{tabular}
\end{center}

As with our summary of factors that shift the AD curve, Exhibit 20 considers each of the factors affecting aggregate supply in isolation. In practice, various factors will be at work simultaneously, and there may be interaction among them. This is especially important with respect to interaction between factors listed as affecting only SRAS and those that also affect LRAS.

For example, consider an increase in the cost of natural resource inputs (e.g., energy). This change shifts the SRAS curve to the left, but according to Exhibit 20, it has no effect on LRAS. This presumes that the relative prices of the factors of production have not changed permanently. If a permanent change has occurred, companies will be forced to conserve on the now more expensive input and will be unable to produce as efficiently. The LRAS curve would, therefore, shift to the left, just as it would if the available supply of natural resources had declined relative to the supply of other inputs. Indeed, that is the most likely cause of a permanent change in relative input prices.

\section{EXAMPLE 10}
\section{Shifts in Aggregate Supply}
In this hypothetical scenario, John Donovan is a portfolio manager for a global mutual fund. Currently, his fund has $10 \%$ of its assets invested in Chinese equities. He is considering increasing the fund's allocation to the Chinese equity market. His decision will be based on an analysis of the following economic developments and their effect on the Chinese economy and equity market. What is the effect on SRAS and LRAS from the following factors?

\begin{enumerate}
  \item Global oil prices, currently near their longer-run trend at US $\$ 75$ a barrel, have increased from US $\$ 35$ a barrel over the last three years because of strong demand from emerging markets.
\end{enumerate}

\section{Solution to 1:}
Higher energy prices cause a decrease in short-run AS and shift the SRAS curve to the left. Because oil prices are back to their longer-run trend, the leftward shift in SRAS essentially reverses a previous shift that occurred when oil prices fell to US 35 a barrel, and it is likely that there will be no effect on the LRAS curve. Lower output and profit are likely to have a negative effect on Chinese equity prices.

\begin{enumerate}
  \setcounter{enumi}{1}
  \item The number of students studying engineering has dramatically increased at Chinese universities during the last decade.
\end{enumerate}

\section{Solution to 2:}
More students studying engineering indicates an improvement in the quality of the labor force-an increase in human capital. As a result, AS increases and the AS curve shifts to the right. Both short-run and long-run curves are affected. Higher output and profits may be expected to have a positive effect on Chinese equity prices.

\begin{enumerate}
  \setcounter{enumi}{2}
  \item Wages for China's workers are rising, leading some multinational companies to consider shifting their investments to Vietnam or Cambodia.
\end{enumerate}

\section{Solution to 3:}
The increase in wages increases labor costs for businesses, causes short-run aggregate supply to decline, and shifts the SRAS curve to the left. Lower output and profit should have a negative effect on Chinese equity prices.

\begin{enumerate}
  \setcounter{enumi}{3}
  \item Data show that business investment as a share of GDP is more than $40 \%$ in China.
\end{enumerate}

\section{Solution to 4:}
The high level of business investment indicates that the capital stock in China is growing at a fast rate. This growth means that workers have more capital to use, which increases their productivity. Thus, AS increases and the AS curve shifts to the right. Both short-run AS and long-run AS are affected. Higher output should have a positive effect on Chinese equity prices.

\begin{enumerate}
  \setcounter{enumi}{4}
  \item The People's Bank of China is likely to permit the yuan to appreciate by $10 \%$ during the next year.
\end{enumerate}

\section{Solution to 5:}
The probable appreciation of the yuan means that the cost of imported raw materials, such as iron ore, coal, and oil, will be lower for Chinese companies. As a result, short-run AS increases and the SRAS curve shifts to the right. The LRAS curve may also shift to the right if the appreciation of the yuan is permanent and global commodity prices do not fully adjust. Higher output and profit should have a positive effect on Chinese equity prices. Note that the stronger yuan will also reduce export demand and shift the AD curve to the left. The combined effect of the AD and AS shifts on output, profit, and equity prices is ambiguous.

The implications of the aforementioned factors for equity investment in China are ambiguous. If the long-run effects dominate, however, then the net effect should be positive. The positive factors-the high level of investment and the growing pool of engineering students-have a lasting effect on output and profit. The negative factors-higher wages and oil pricesshould be temporary, because wages will realign with the price level and the increase in oil prices appears to offset a previous temporary decline. The reduction in raw material prices resulting from the stronger currency is positive for output, profit, and equities in the short run and perhaps in the long run as well.

\section{Equilibrium GDP and Prices}
Now that we have discussed the components of the AD and AS model, we can combine them to determine the real level of GDP and the price level. Equilibrium occurs where the $\mathrm{AD}$ and $\mathrm{AS}$ curves intersect. At this point, the quantity of aggregate output demanded (or the level of aggregate expenditures) is equal to the quantity of aggregate output supplied. In Exhibit 21, equilibrium price and GDP occur at $P_{1}$ and $Y_{1}$. If the price level is above $P_{1}$, then the quantity of output supplied exceeds the amount demanded. This situation would result in unsold inventories and would require a reduction in production and in prices. If the price level is below $P_{1}$, then the quantity of aggregate output demanded exceeds the quantity of aggregate output supplied. This situation would result in a shortage of goods that would put upward pressure on prices.

It is important to understand that short-run macroeconomic equilibrium may occur at a level above or below full employment. We consider four possible types of macroeconomic equilibrium:

\begin{enumerate}
  \item Long-run full employment

  \item Short-run recessionary gap

  \item Short-run inflationary gap

  \item Short-run stagflation

\end{enumerate}

From an investment perspective, the performance of asset classes and financial markets will differ in each case as the economy makes the adjustment toward the macroeconomic equilibrium. We look at these differences later in the reading.

\section{Long-Run Equilibrium}
Exhibit 21 shows the long-run full employment equilibrium for an economy. In this case, equilibrium occurs where the AD curve intersects the SRAS curve at a point on the LRAS curve. Because equilibrium occurs at a point on the LRAS curve, the economy is at potential real GDP. Both labor and capital are fully employed, and everyone who wants a job has one. In the long run, equilibrium GDP is equal to potential GDP.

\section{Exhibit 21: Long-Run Macroeconomic Equilibrium}
\begin{center}
\includegraphics[max width=\textwidth]{2023_05_04_ccb578c5089fd41615beg-175}
\end{center}

In practice, the level of potential GDP is difficult to measure with precision. Because of fluctuations arising from shifts in the AD and SRAS curves, the economy rarely operates at potential GDP. Thus, potential GDP is not observable from the data on actual GDP. In addition, potential GDP is determined by factors that are themselves difficult to measure. Thus, "bottom-up" estimates of the level of potential output are also quite imprecise. As we will cover subsequently, however, economists have confidence that the long-run growth rate of potential GDP can be estimated well enough to provide meaningful guidance for analysts and policymakers. Hence, in the short run, economists generally focus on factors that cause actual GDP to grow faster or slower than their estimate of the long-run growth rate of potential output. In addition, they focus on measures that indicate, albeit imprecisely, the extent to which the economy is operating above or below its productive capacity, such as unemployment and capacity utilization.

\section{Recessionary Gap}
Cyclical fluctuations in real GDP and prices are caused by shifts in both the AD and SRAS curves. A decline in AD or a leftward shift in the AD curve results in lower GDP and lower prices. Such declines in AD lead to economic contractions, and if such declines drive demand below the economy's potential GDP, the economy goes into a recession (defined as a period during which real GDP decreases for at least two consecutive quarters). In Exhibit 22, when aggregate demand falls, the equilibrium shifts from Point A to Point B. Real GDP contracts from $Y_{1}$ to $Y_{2}$, and the aggregate price level falls from $P_{1}$ to $P_{2}$. Because of the decline in demand, companies reduce their workforce and the unemployment rate rises. The economy is in recession, and the recessionary gap is measured as the difference between $Y_{2}$ and $Y_{1}$ or the amount by which equilibrium output is below potential GDP. Thus, a recessionary gap occurs when the AD curve intersects the short-run AS curve at a short-run equilibrium level of GDP below potential GDP. Most importantly, in contrast to full employment, equilibrium GDP is below potential GDP.

\section{Exhibit 22: Recessionary Gap}
\begin{center}
\includegraphics[max width=\textwidth]{2023_05_04_ccb578c5089fd41615beg-176}
\end{center}

Any of the factors discussed earlier could cause the shift in the AD curve. Tightening of monetary policy, higher taxes, more pessimistic consumers and businesses, and lower equity and housing prices all reduce $\mathrm{AD}$ and are all possible causes of a recession.

The question is, how does the economy return to full employment? There is considerable debate among economists about the answer to this question. Some economists argue that an automatic, self-correcting mechanism will push the economy back to its potential, without the need for government action. The idea is that because of the decline in prices and higher unemployment, workers will be willing to accept lower nominal wages. Workers will do this because each currency unit of wages now buys more goods and services because of their lower price. As a result, lower wages and input prices will shift the SRAS curve to the right (see Exhibit 20) and push the economy back to full employment and potential GDP.

The problem is that this price mechanism can take several years to work. As an alternative, government can use the tools of fiscal and monetary policy to shift the AD curve to the right (from Point B to Point A in Exhibit 22) and move the economy back to full employment. On the fiscal side, policymakers can reduce taxes or increase government spending. On the monetary side, the central bank can lower interest rates or increase the money supply. The problem, however, is that variable lags in the effectiveness of these policy measures imply that policy adjustments may end up reinforcing rather than counteracting underlying shifts in the economy.

\section{Investment Implications of a Decrease in AD}
Aggregate demand and aggregate supply are theoretical measures that are very hard to measure directly. Most governments, however, publish statistics that provide an indication of the direction that aggregate demand and supply are moving over time. For example, statistics on consumer sentiment, factory orders for durable and nondurable goods, the value of unfilled orders, the number of new housing starts, the number of hours worked, and changes in inventories provide an indication of the direction of aggregate demand. If these statistics suggest that a recession is caused by a decline in $\mathrm{AD}$, the following conditions are likely to occur:

\begin{itemize}
  \item Corporate profits will decline.

  \item Commodity prices will decline.

  \item Interest rates will decline.

  \item Demand for credit will decline.

\end{itemize}

This situation suggests the following investment strategy:

\begin{itemize}
  \item Reduce investments in cyclical companies, companies with sales and profits that regularly expand and contract with the business cycle or state of economy (for example, automobile and chemical companies), because their earnings are likely to decline the most in an economic slowdown.

  \item Reduce investments in commodities and/or commodity-oriented companies because the decline in commodity prices will slow revenue growth and reduce profit margins.

  \item Increase investments in defensive companies, companies with sales and profits that have little sensitivity to the business cycle or state of the economy (e.g., food and pharmaceutical companies), because they are likely to experience only modest earnings declines in an economic slowdown.

  \item Increase investments in investment-grade or government-issued fixed-income securities. The prices of these securities should increase as interest rates decline.

  \item Increase investments in long-maturity fixed-income securities because their prices will be more responsive to the decline in interest rates than the prices of shorter-maturity securities.

  \item Reduce investments in speculative equity securities and in fixed-income securities with low credit quality ratings.

\end{itemize}

As with most investment strategies, this strategy will be most successful if it is implemented before other market participants recognize the opportunities and asset prices adjust.

\section{EXAMPLE 11}
\section{Using AD and AS: A Historical Example: 2007-2009}
Many Asian economies were more adversely affected than the United States by the global recession that began in late 2007. In the first quarter of 2009, real GDP fell at an annualized rate of 16\% in Japan and 11\% in Singapore, compared with a $6 \%$ annualized decline in the United States. Using the data on exports as a share of GDP shown in Exhibit 23, explain how the following economic factors contributed to the recession in the Asian economies:

\section{Exhibit 23: Exports as a Share of GDP, 2007 and 2016}
\begin{center}
\begin{tabular}{|c|c|c|c|c|}
\hline
\multirow[b]{2}{*}{Economy} & \multicolumn{2}{|c|}{2007} & \multicolumn{2}{|c|}{2016} \\
\hline
 & $\begin{array}{c}\text { Exports as a } \\ \text { Percentage of GDP }\end{array}$ & $\begin{array}{l}\text { Percentage of } \\ \text { Exports Going to } \\ \text { United States }\end{array}$ & $\begin{array}{c}\text { Exports as a } \\ \text { Percentage of GDP }\end{array}$ & $\begin{array}{l}\text { Percentage of Exports } \\ \text { Going to United States }\end{array}$ \\
\hline
Hong Kong SAR & 186 & 11.2 & 192 & 8.3 \\
\hline
Singapore & 166 & 11.5 & 175 & 6.5 \\
\hline
Thailand & 62 & 11.6 & 69 & 11.2 \\
\hline
Germany & 53 & 10.9 & 47 & 9 \\
\hline
South Korea & 47 & 7.1 & 44 & 12 \\
\hline
Mexico & 37 & 26.4 & 37 & 81.2 \\
\hline
Canada & 28 & 80.2 & 31 & 76.2 \\
\hline
Chinese mainland & 27 & 19 & 21 & 19 \\
\hline
India & 17 & 20.1 & 20 & 16 \\
\hline
Japan & 14 & 17 & 17 & 20.2 \\
\hline
Kenya & 12 & - & 16 & 6.7 \\
\hline
United States & 12 & - & 12 & - \\
\hline
Ethiopia & 11 & 6.7 & 9 & 9.9 \\
\hline
\end{tabular}
\end{center}

Sources: World Bank: World Development Indicators NE.EXP.GNFS.ZS and \href{http://atlas.media.mit.edu/}{atlas.media.mit.edu/} en/profile/country.

\begin{enumerate}
  \item Collapse of house prices and home construction in the United States.
\end{enumerate}

\section{Solution to 1:}
The collapse in housing prices caused housing construction spending, a component of business investment, to decline in the United States. The decline in housing prices also caused a sharp fall in household wealth. As a result, consumption spending in the United States declined because of the wealth effect. The decline in both consumption and housing construction shifted the AD curve for the United States to the left, resulting in a US recession. The link to the Asian economies was through global trade because exports represented such a large share of the Asian economies' GDP (Exhibit 24). In turn, these economies exported a significant amount of goods and services to the United States. Thus, the recession in the United States and especially the decline in US consumption spending caused a sharp fall in exports among Asian economies. This decrease lowered their AD and shifted the AD curve to the left, resulting in a recessionary gap in these economies.

\begin{enumerate}
  \setcounter{enumi}{1}
  \item Oil prices rising from around US $\$ 30$ a barrel in 2004 to nearly US $\$ 150$ a barrel in 2008. (Note that most Asian economies rely on imports for almost all of their oil and energy needs. In contrast, the United States has a large domestic energy industry and imports only about half of its oil.)
\end{enumerate}

\section{Solution to 2:}
The rise in oil prices increased input cost and shifted the short-run AS curve to the left. Because the eastern Asian economies depend heavily on imported oil, their economies were more adversely affected than that of the United States.

\begin{enumerate}
  \setcounter{enumi}{2}
  \item The dramatic reduction in credit availability following the collapse or near collapse of major financial institutions in 2008.
\end{enumerate}

\section{Solution to 3:}
The decline in housing prices caused financial institutions in the United States to suffer large losses on housing-related loans and securities. Several large lenders collapsed, and the US Treasury and the Federal Reserve had to intervene to prevent a wave of bankruptcies among large financial institutions. As a result of the crisis, it became difficult for households and businesses to obtain credit to finance their spending. This dynamic caused AD to fall and increased the severity of the recession in the United States, resulting in a significant decline in US imports and thus exports from the Asian economies. In addition, the financial crisis made it more difficult to obtain trade finance, further reducing exports from Asia.

In summary, global investors need to be aware of the growing linkages among economies and the extent that one economy's growth depends on demand from within as well as from outside of that economy. Data on exports as a percentage of an economy's GDP provide an indication of this dependence. Although Japan is often viewed as an export-driven economy, Exhibit 24 shows that in 2018, exports were only $18 \%$ of its GDP. Similarly, the economy of India depends largely on domestic spending for growth because in 2018, exports accounted for only $20 \%$ of GDP.

\section{Inflationary Gap}
Increases in AD lead to economic expansions as real GDP and employment increase. If the expansion drives the economy beyond its production capacity, however, inflation will occur. The inflation rate is defined as the increase in the general price level from one period to the next. As summarized in Exhibit 18, higher government spending, lower taxes, a more optimistic outlook among consumers and businesses, a weaker domestic currency, rising equity and housing prices, and an increase in the money supply would each stimulate aggregate demand and shift the AD curve to the right. If aggregate supply does not increase to match the increase in AD, a rise in the overall level of prices will result.

As shown in Exhibit 24, an increase in AD will shift the equilibrium level of GDP from Point A to Point B. Real output increases from $Y_{1}$ to $Y_{2}$, and the aggregate price level rises from $P_{1}$ to $P_{2}$. As a result of the increase in aggregate demand, companies increase their production and hire more workers. The unemployment rate declines. Once an economy reaches its potential GDP, however, companies must pay higher wages and other input prices to further increase production. The economy now faces an inflationary gap, measured by the difference between $Y_{2}$ and $Y_{1}$ in Exhibit 24. An inflationary gap occurs when the economy's short-run level of equilibrium GDP is above potential GDP, resulting in upward pressure on prices.

\section{Exhibit 24: Inflationary Gap}
\begin{center}
\includegraphics[max width=\textwidth]{2023_05_04_ccb578c5089fd41615beg-180}
\end{center}

GDP cannot remain at $Y_{2}$ for long because the economy is overutilizing its resourcesthat is, extra shifts of workers are hired and plant and equipment are operating at their maximum capacity. Eventually, workers become tired and plants and equipment wear out. The increase in the general price level and input prices will set in motion the process of returning the economy back to potential GDP. Higher wages and input prices shift the SRAS curve to the left (from SRAS to SRAS $_{2}$ ), moving the economy to Point $C$ in Exhibit 24. Again, this self-correcting mechanism may work slowly.

A nation's government and/or its central bank can attempt to use the tools of fiscal and monetary policy to control inflation by shifting the $\mathrm{AD}$ curve to the left ( $\mathrm{AD}_{2}$ to $\mathrm{AD}_{1}$ in Exhibit 24) so that the return to full employment occurs without the price increase. From a fiscal perspective, policymakers can raise taxes or cut government spending. From a monetary perspective, the central bank can reduce bank reserves, resulting in a decrease in the growth of the money supply and higher interest rates.

\section{Investment Implications of an Increase in AD Resulting in an Inflationary Gap}
If economic statistics (consumer sentiment, factory orders for durable and nondurable goods, etc.) suggest that there is an expansion caused by an increase in AD, the following conditions are likely to occur:

\begin{itemize}
  \item Corporate profits will rise.

  \item Commodity prices will increase.

  \item Interest rates will rise.

  \item Inflationary pressures will build.

\end{itemize}

This situation suggests the following investment strategy:

\begin{itemize}
  \item Increase investment in cyclical companies because they are expected to have the largest increase in earnings.

  \item Reduce investments in defensive companies because they are expected to have only a modest increase in earnings.

  \item Increase investments in commodities and commodity-oriented equities because they will benefit from higher production and output.

  \item Reduce investments in fixed-income securities, especially longer-maturity securities, because they will decline in price as interest rates rise. Raise exposure to speculative fixed-income securities (junk bonds) because default risks decrease in an economic expansion.

\end{itemize}

\section{Stagflation: Both High Inflation and High Unemployment}
Structural fluctuations in real GDP are caused by fluctuations in SRAS. Declines in aggregate supply bring about stagflation-high unemployment and increased inflation. Increases in aggregate supply conversely give rise to high economic growth and low inflation.

Exhibit 25 shows the case of a decline in aggregate supply, perhaps caused by an unexpected increase in basic material and oil prices. The equilibrium level of GDP shifts from Point A to $\mathrm{B}$. The economy experiences a recession as GDP falls from $Y_{1}$ to $Y_{2}$, but the price level, instead of falling, rises from $P_{1}$ to $P_{2}$. Over time, the reduction in output and employment should put downward pressure on wages and input prices and shift the SRAS curve back to the right, re-establishing full employment equilibrium at Point A. However, this mechanism may be painfully slow. Policymakers may use fiscal and monetary policy to shift the AD curve to the right, as previously discussed, but at the cost of a permanently higher price level at Point $\mathrm{C}$.

\section{Exhibit 25: Stagflation}
\begin{center}
\includegraphics[max width=\textwidth]{2023_05_04_ccb578c5089fd41615beg-181}
\end{center}

The global economy experienced stagflation in the mid-1970s and early 1980s. Both unemployment and inflation soared. The problem was caused by a sharp decline in aggregate supply fueled by higher input prices, especially the price of oil. In 1973, the price of oil quadrupled. A steep global recession began in late 1973 and lasted through early 1975. The recession was unusual because prices rose rather than declined as would be expected in a typical demand-caused downturn. In 1979-1980, the price of oil doubled. Higher energy prices shifted the SRAS curve to the left, as shown in Exhibit 25, leading to a global recession in 1980-1982. In the United States, the contraction in output was reinforced by the Federal Reserve's decision to tighten monetary policy to fight the supply-induced inflation.

\section{Investment Implications of a Shift in AS}
Labor and raw material costs, including energy prices, determine the direction of shifts in short-run aggregate supply: Higher costs for labor, raw materials, and energy lead to a decrease in aggregate supply, resulting in lower economic growth and higher prices. Conversely, lower labor costs, raw material prices, and energy prices lead to an increase in aggregate supply, resulting in higher economic growth and a lower aggregate price level. Productivity is also an important factor. Higher rates of productivity growth shift the AS to the right, resulting in higher output and lower unit input prices. Lower rates of productivity growth do the opposite and shift the AS curve to the left. From an investment perspective, a decline in AS (leftward shift of the SRAS curve) suggests:

\begin{itemize}
  \item reducing investment in fixed income because rising output prices (i.e., inflation) put upward pressure on nominal interest rates,

  \item reducing investment in most equity securities because profit margins are squeezed and output declines, and

  \item increasing investment in commodities or commodity-based companies because prices and profits are likely to rise.

\end{itemize}

On the other hand, an increase in AS (rightward shift of the SRAS curve) resulting from higher productivity growth or lower labor, raw material, and energy costs is favorable for most asset classes other than commodities.

\section{Conclusions on $A D$ and $A S$}
The business cycle and the resulting fluctuations in real GDP are caused by shifts in the AD and AS curves. The effects of these shifts can be summarized as follows:

\begin{itemize}
  \item An increase in AD raises real GDP, lowers the unemployment rate, and increases the aggregate level of prices.

  \item A decrease in AD lowers real GDP, increases the unemployment rate, and decreases the aggregate level of prices.

  \item An increase in AS raises real GDP, lowers the unemployment rate, and lowers the aggregate level of prices.

  \item A decrease in AS lowers real GDP, raises the unemployment rate, and raises the aggregate level of prices.

\end{itemize}

If both curves shift, the effect is a combination of the aforementioned individual effects. We look at four possible cases:

\begin{enumerate}
  \item Both $A D$ and $A S$ increase. If both $A D$ and $A S$ increase, real GDP will increase but the effect on inflation is unclear unless we know the magnitude of the changes, because an increase in AD will increase the price level, whereas an increase in AS will decrease the price level. If AD increases more than AS, the price level will increase. If AS increases more than AD, however, the price level will decline.

  \item Both $A D$ and $A S$ decrease. If both AD and AS decrease, real GDP and employment will decline but the effect on the price level is unclear unless we know the magnitude of the changes, because a decrease in AD decreases the price level, whereas a decrease in AS increases the price level. If AD decreases more than AS, the price level will fall. If AS decreases more than $\mathrm{AD}$, the price level will rise.

  \item $A D$ increases and $A S$ decreases. If $\mathrm{AD}$ increases and $\mathrm{AS}$ declines, the price level will rise but the effect on real GDP is unclear unless we know the magnitude of the changes, because an increase in AD increases real GDP, whereas a decrease in AS decreases real GDP. If AD increases more than AS declines, GDP will rise. If AS decreases more than AD increases, real GDP will fall.

  \item $A D$ decreases and $A S$ increases. If $A D$ decreases and $A S$ increases, the price level will decline but the effect on real GDP is unclear unless we know the magnitudes of the changes, because a decrease in AD decreases real GDP, whereas an increase in AS increases real GDP. If AD decreases more than AS increases, real GDP will fall. If AS increases more than AD declines, real GDP will rise. Exhibit 26 summarizes these four cases.

\end{enumerate}

\section{Exhibit 26: Effect of Combined Changes in AS and AD}
\begin{center}
\begin{tabular}{lccc}
\hline
Change in AS & Change in AD & $\begin{array}{c}\text { Effect on } \\ \text { Real GDP }\end{array}$ & $\begin{array}{c}\text { Effect on } \\ \text { Aggregate Price Level }\end{array}$ \\
\hline
Increase & Increase & Increase & Indeterminate \\
Decrease & Decrease & Decrease & Indeterminate \\
Increase & Decrease & Indeterminate & Decrease \\
Decrease & Increase & Indeterminate & Increase \\
\hline
\end{tabular}
\end{center}

Whether the growth of the economy is demand- or supply-driven has an effect on asset prices. Demand-driven expansions are normally associated with rising interest rates and inflation, whereas contractions are associated with lower inflation and interest rates. Supply-driven expansions are associated with lower inflation and interest rates, whereas supply-driven contractions are associated with rising inflation and interest rates.

\section{EXAMPLE 12}
\section{Investment Strategy Based on AD and AS Curves}
An analyst is evaluating the possibility of investing in China, Italy, Mexico, or Brazil. What are the equity and fixed-income investment opportunities in these countries based on the following events?

\begin{enumerate}
  \item The Chinese government announced a spending plan of US $\$ 1.2$ trillion, or 13\% of GDP. In addition, the central bank of China eased monetary policy, resulting in a surge of lending.
\end{enumerate}

\section{Solution to 1:}
Stimulative fiscal and monetary policies should result in a demand-driven expansion. Investors should reduce investments in fixed-income securities and defensive companies and invest in cyclical companies and commodities. As a result, the prospects for growth-oriented equity investments look favorable in China.

\begin{enumerate}
  \setcounter{enumi}{1}
  \item The Italian government announced a decline in labor productivity, and it expects this trend to continue into the future.
\end{enumerate}

\section{Solution to 2:}
A decline in labor productivity will result in a decline in AS; that is, the AS curve will shift to the left. This scenario is typically a poor investment environment. Investors should reduce investments in both fixed-income and equity securities and invest in commodities. Entry into Italian stocks and bonds does not look attractive. 3. In response to rising inflationary pressure, the Mexican central bank tightened monetary policy, and the government announced tax increases and spending cuts to balance the budget.

Solution to 3:

The policy measures put in place by the Mexican government and central bank will cause a drop in AD and likely result in a recession. Investors should increase their investments in fixed-income securities because interest rates will most likely decline as the recession deepens. This scenario is a poor environment for equity securities.

\begin{enumerate}
  \setcounter{enumi}{3}
  \item A major discovery of oil off the coast of Brazil lowered oil prices, while the Brazilian government announced a major increase in spending on public infrastructure to stimulate the economy.
\end{enumerate}

\section{Solution to 4:}
In this situation, both the AD and AS curves will shift. The increase in spending on public infrastructure will shift the AD curve to the right, resulting in higher aggregate expenditures and prices. Lower oil prices will shift the AS curve to the right, resulting in higher GDP but lower prices. Thus, GDP will clearly increase, but the effect on prices and inflation is indeterminate. As a result, investors should increase their investment in equity securities; however, the effect on fixed-income securities is unclear.

\section{EXAMPLE 13}
\section{Historical Example: Using AD and AS to Explain Japan's Economic Problem}
Japan experienced sluggish growth in real GDP for nearly two decades following the bursting of an asset and investment bubble in the late 1980s. At the same time, Japan experienced deflation (declining prices) over the two-decade period. The reasons for this protracted period of stagnation continue to be debated among economists. Failure to recognize a change in the Japanese growth rate has hurt many investors, especially those taking a long-term perspective. From their peak in 1989, Japanese equity prices, as measured by the Nikkei index, fell by more than $60 \%$ before bottoming out in mid-1992. During the following two decades, the market was essentially flat despite considerable volatility.

The performance of the Japanese economy can be explained using the AD and AS model. The protracted slowdown of growth in Japan beginning in the early 1990s can be linked to the effect of the collapse of the equity and commercial real estate markets in the late 1980s and to excessive investment in capital goods (new factories and equipment) in the 1980s. These problems were compounded by persistent weakness in the banking sector, a profound lack of confidence among businesses and consumers, and negative demographics with slow growth in the working age population.

The sum of these developments caused a decline in both the AD and AS curves. Aggregate demand declined, shifting the AD curve to the left for the following reasons:

\begin{itemize}
  \item The wealth effect resulting from the decline in equity and real estate prices sharply reduced consumption spending. Asset prices took more than two decades to recover. - Excessive investment in capital goods caused a sharp decline in business investment during the period.

  \item Lack of confidence among businesses and consumers during the period.

  \item Problems in the banking sector made monetary policy ineffective because banks were unable to lend, which negatively affected both consumer and business spending.

\end{itemize}

AS declined for the following reasons:

\begin{itemize}
  \item Marked slowing in private investment spending during the period reduced the capital stock. This dynamic also reduced potential GDP.

  \item Slow population growth limited the growth in the labor supply, which also reduced potential GDP.

  \item Higher energy prices slowed growth during the period because of Japan's heavy dependence on imported energy.

\end{itemize}

As would be expected, the declines in both AD and AS resulted in slow GDP growth. The fact that prices fell indicates that the AD curve shifted more than the AS curve.

\section*{ECONOMIC GROWTH AND SUSTAINABILITY }
Understanding sources of economic growth is crucial for practitioners involved in various types of investment decision making. Economic growth is calculated as the annual percentage change in real GDP or the annual change in real per capita GDP:

\begin{itemize}
  \item Growth in real GDP measures how rapidly the total economy is expanding.

  \item Per capita GDP, defined as real GDP divided by population, measures the standard of living in each country and the ability of the average person to buy goods and services

\end{itemize}

Exhibit 27 illustrates growth in GDP in selected regions since 1990. It shows that although total GDP doubled in Latin America during the period, GDP per capita in the region increased by only about $50 \%$, as a result of population growth. Exhibit 27: Real GDP Growth and Per Capita GDP Growth for Selected Regions In PPP (constant 2017 US\$ billion)

\begin{center}
\includegraphics[max width=\textwidth]{2023_05_04_ccb578c5089fd41615beg-186}
\end{center}

\section{GDP per Capita, PPP (constant 2017 US\$)}
\begin{center}
\includegraphics[max width=\textwidth]{2023_05_04_ccb578c5089fd41615beg-186(1)}
\end{center}

Note: PPP is Purchasing Power Parity.

Source: World Bank data.

Economic growth is important because rapid growth in per capita real GDP can transform a poor nation into a wealthy one. Even small differences in the growth rate of per capita GDP, if sustained over time, have a large effect on an economy's standard of living. Nevertheless, there is a limit to how fast an economy can grow. Faster growth is not always better for an economy because there are costs associated with excess growth, such as higher inflation, potential environmental damage, and the lower consumption and higher savings needed to finance the growth.

These costs raise the issue of sustainable growth, which requires an understanding of the concept of potential GDP. Recall that potential GDP measures the economy's productive capacity and is the level of real GDP that an economy could produce if capital and labor are fully employed. In order to grow over time, an economy must add to its productive capacity. Thus, the sustainable rate of economic growth is measured by the rate of increase in the economy's productive capacity or potential GDP. It is important to note that economists cannot directly measure potential output. Instead, they estimate it using a variety of techniques that we will discuss.

For global investors, estimating the sustainable rate of economic growth for an economy is important in valuing a variety of assets. Growth in potential GDP matters for long-run equity market appreciation and for individual industries and company forecasts. Potential GDP can be used to gauge inflationary pressures, determine credit quality, and assess the likelihood of a change in central bank policy, all of which are relevant for fixed income investors. Economic growth in large economies or whole regions is also useful for forecasting demand for commodities.

\section{The Production Function and Potential GDP}
We focus on the simplest form of what is known as the neoclassical or Solow growth model, the framework used to understand and analyze the underlying sources of growth for an economy. The model shows that the economy's productive capacity and potential GDP increase for two reasons:

\begin{enumerate}
  \item accumulation of such inputs as capital, labor, and raw materials used in production, and

  \item discovery and application of new technologies that make the inputs in the production process more productive - that is, able to produce more goods and services for the same amount of input.

\end{enumerate}

\section{The Production Function}
The model is based on a production function that provides the quantitative link between the levels of output that the economy can produce and the inputs used in the production process, given the state of technology. A two-factor production function with labor and capital as the inputs is expressed mathematically as

$$
Y=A \boldsymbol{F}(L, K)
$$

where $Y$ denotes the level of aggregate output in the economy, $L$ is the quantity of labor or number of workers in the economy, $K$ is the capital stock or the equipment and structures used to produce goods and services, and $A$ represents total factor productivity (TFP). TFP is a scale factor that captures the exogenous effect of technological change and other factors that raise output beyond the measured contribution of the capital and labor inputs. Like potential GDP, TFP is not directly observed in the economy but instead is estimated.

The production function shows that output in the economy depends on inputs and the level of technology. The economy's capacity to produce goods grows when these inputs increase and/or technology advances. The more technologically advanced an economy is, the more output it is able to produce from a given amount of inputs.

\section{The Model Assumptions}
The neoclassical model makes three assumptions about the production function that provide a link to microeconomics. First, it assumes that the production function has constant returns to scale. This means that if all the inputs in the production process are increased by the same percentage, then output will rise by that percentage. Thus, doubling all inputs would double output. Second, the model assumes that the production function exhibits diminishing marginal productivity with respect to any individual input. Finally, the model assumes no positive or negative externalities are associated with the use of the inputs. These assumptions play an important role in assessing the contribution of labor and capital to economic growth. Marginal productivity looks at the extra output produced from a one-unit increase in an input if the other inputs are unchanged. It applies to any input as long as the other inputs are held constant. For example, if we have a factory of a fixed size and we add more workers to the factory, the marginal productivity of labor measures how much additional output each additional worker will produce. Diminishing marginal productivity means that at some point the extra output obtained from each additional unit of the input will decline. In the prior example, if we hire more workers at the existing factory (fixed capital input in this case), output will rise by a smaller and smaller amount with each additional worker.

In the case of capital, if we add more and more capital to a fixed number of workers, the amount of additional output contributed by each additional amount of capital will fall. Thus, if capital grows faster than labor, capital will become less productive, resulting in slower and slower growth. Diminishing marginal productivity of capital has two major implications for potential GDP and long-term growth:

\begin{enumerate}
  \item Long-term sustainable growth cannot rely solely on capital deepening investment that increases the stock of capital relative to labor. More generally, increasing the supply of some input(s) relative to other inputs will lead to diminishing returns and cannot be the basis for sustainable growth.

  \item Given that developing countries have relatively less capital, their productivity of capital is high. All else the same, the growth rates of developing countries should exceed those of developed countries. As a result, there should be a convergence of incomes between developed and developing countries over time.

\end{enumerate}

Because of the assumption of diminishing returns to capital, the only way to sustain growth in potential GDP per capita in the long run is through growth in TFP. This results in an upward shift in the production function: The economy produces more goods and services using the same level of labor and capital inputs. In terms of the formal production function $Y=A \boldsymbol{F}(L, K)$, this dynamic is reflected by an increase in the parameter $A$.

\section{Growth Accounting Equation}
Using the production function, Robert Solow developed a model that explained the contribution of labor, capital, and TFP to economic growth. The growth accounting equation shows that the rate of growth of potential output equals growth in TFP plus the weighted average growth rate of labor and capital.

Growth in potential GDP $=$ Growth in TFP $+W_{L}($ Growth in labor $)+$ $W_{C}$ (Growth in capital)

where $W_{L}$ and $W_{C}$ are the relative shares of capital and labor in national income. The capital share is the sum of corporate profits, net interest income, net rental income, and depreciation divided by GDP. The labor share is employee compensation divided by GDP. For many developed countries, $W_{L}$ and $W_{C}$ are roughly 0.7 and 0.3 , respectively.

The growth accounting equation highlights a key point: The contribution of labor and capital to long-term growth depends on their respective shares of national income. For the United States, because labor's share is higher, an increase in the growth rate of labor will have a significantly larger effect (roughly double) on potential GDP growth than will an equivalent increase in the growth rate of capital.

The growth accounting equation can be further modified to explain growth in per capita GDP. Because it measures the standard of living and purchasing power of the average person in an economy, per capita GDP is more relevant than the absolute level of GDP in comparing economic performance among countries. Transforming the growth accounting equation into per capita terms results in the following equation:

Growth in per capita potential GDP $=$ Growth in $\mathrm{TFP}+W_{C}$ (Growth in capital-t o-labor ratio) The capital-to-labor ratio measures the amount of capital available per worker and is weighted by the share of capital in national income. Because capital's share in national income in the US economy is 0.3 , a $1 \%$ increase in the amount of capital available for each worker increases per capita output by only $0.3 \%$.

\section{EXAMPLE 14}
\section{Impact of Capital Investment in Laos}
\begin{enumerate}
  \item Annual GDP growth in Laos has averaged about $7 \%$ over the 10 years since 2009. The economy has benefitted from foreign direct investment in hydroelectric dams and mining projects. Using the neoclassical growth model, an emerging market analyst states that these capital inflows are unlikely to boost the long-term potential GDP growth rate of Laos.
\end{enumerate}

Explain using the neoclassical model whether the analyst's statement is correct.

\section{Solution:}
The capital inflows in the form of foreign direct investment are increasing capital spending in Laos and, through capital deepening, will raise the short-term GDP growth rate. Long-term potential GDP, however, depends on growth in TFP, the key driver of which is technical progress. Adding to capital through the foreign direct investment will be subject to diminishing returns. Thus, the analyst is correct because long-term potential GDP growth cannot rely on capital accumulation or capital deepening. To increase potential GDP, Laos must implement policies to boost the growth in TFP.

\section{Sources of Economic Growth}
The foregoing growth accounting equation focuses on the main determinants of growth-capital, labor, and TFP (the key driver being technology)-and omits a number of other sources of growth to simplify the analysis. For many countries, however, human capital, public infrastructure, natural resources, and the economic environment and institutions play an important role in explaining economic growth. In the standard neoclassical model discussed earlier, these inputs are measured as part of TFP growth along with technological changes. Using this model, the sources of growth for an economy that determine its capacity to supply goods and services are as shown in Exhibit 28. Note that the amount of labor, measured as total hours worked, is determined by the size of the average hours worked per worker multiplied by the labor force. Labor force itself is a product of working age population and the participation rate (the proportion of working age population choosing to work or seeking work).

\section{Exhibit 28: Sources of Economic Growth}
Growth in potential GDP = Growth in TFP + WL (Growth in labor) + WC (Growth in capital)

\begin{center}
\includegraphics[max width=\textwidth]{2023_05_04_ccb578c5089fd41615beg-190}
\end{center}

\section{Technology}
The most important factor affecting economic growth is technology, especially in developed countries. Technology refers to the process a company uses to transform inputs into outputs. Technological advances are discoveries that make it possible to produce more or higher-quality goods and services with the same resources or inputs. At the same time, technological progress results in the creation of new goods and services. Finally, technological progress improves how efficiently businesses are organized and managed.

Technological advances allow an economy to overcome the limits imposed by diminishing marginal returns. Thus, an economy will face limits to growth if it relies exclusively on expanding the inputs or factors of production. One of the key drivers of growth in developed countries is the information technology (IT) sector, whose growth has been driven by technological innovation that has caused the price of key technologies, such as semiconductors, to fall dramatically. The steep declines in prices have encouraged investment in IT at the expense of other assets. The sector has grown very fast and has made a significant contribution to economic growth, employment, and exports.

Countries can innovate through expenditures, both public and private, on research and development (R\&D). Expenditures on R\&D and the number of patents issued, although not directly measuring innovation, provide some useful insight into innovative performance. Countries can also acquire new technology through imitation or copying the technology developed elsewhere. The embodiment of technology in capital goods can also enable relatively poor countries to catch up to the technology leaders.

TFP is often used as a proxy for technological progress and organizational innovation. TFP is the amount by which output would rise because of improvements in the production process. It is not directly observed and is calculated as a residual, the difference between the growth rate of potential output and the weighted average growth rate of capital and labor. Specifically,

TFP growth $=$ Growth in potential GDP $-\left[W_{L}(\right.$ Growth in labor $)+$ $W_{C}$ (Growth in capital)]

The interpretation of the TFP as a pure technology measure is incorrect. Given the measurement and estimation difficulties, TFP will measure not only technological change but also the effect of other inputs into the production process such as human capital, public infrastructure, and natural resources. In addition to the other inputs, TFP measures the effect of externalities and spillover effects, omitted inputs, organizational efficiencies, and institutional and governance factors. Nevertheless, it is a useful indicator of the underlying technological forces.

\section{Labor Supply}
Growth in the number of people available for work (quantity of workforce) is an important source of economic growth and partially accounts for the superior growth performance of the US economy versus the European and Japanese economies. Most developing countries, such as China, India, and Mexico, have a large potential labor supply. We can measure the potential size of the labor input as the total number of hours available for work, which is given by

Total hours worked $=$ Labor force $\times$ Average hours worked per worker

The labor force is defined as the portion of the working age population (over the age of 16) that is employed or available for work but not working (unemployed). The contribution of labor to overall output is also affected by changes in the average hours worked per worker. The average number of hours worked is highly sensitive to the business cycle. The long-term trend, however, has been toward a shorter work week in the advanced countries. This development is the result of legislation, collective bargaining agreements, and the growth of part-time and temporary work.

\section{Human Capital}
In addition to the quantity of labor (the " $L$ " in the earlier mentioned production function), the quality of the labor force is important. Human capital reflects the accumulated knowledge and skill that workers acquire from education, training, and life experience. It measures the quality of the workforce, and we treat it as part of TFP. In general, better-educated and skilled workers will be more productive and more adaptable to changes in technology.

An economy's human capital is increased through investment in education and on-the-job training. Like physical capital, investment in education is costly. Studies show, however, that education offers a significant return-that is, people with more education earn higher wages. Moreover, education may also have a spillover or externality effect: Increasing the educational level of one person not only raises that person's output but also the output of those around her. The spillover effect operates through the link between education and advances in technology. Education not only improves the quality of the labor force but also encourages growth through innovation. Investment in health is also a major contributor to human capital, especially in developing countries.

\section{Physical Capital Stock}
The physical capital stock (accumulated amount of buildings, machinery, and equipment used to produce goods and services) increases from year to year as long as net investment (gross investment less depreciation of capital) is positive. Thus, countries with a higher rate of investment should have a growing physical capital stock and a higher rate of GDP growth.

Exhibit 29 shows the level of business investment as a share of GDP across a selection of 18 developed and developing economies. The exhibit shows significant variation across countries. As is evident in Exhibit 29, the correlation between investment (as a percentage of GDP) and economic growth in the subsequent period (19 years after 2000 and 9 years after 2010) is high. Economies that devote a large share of GDP to investment, such as China, India, and South Korea (numerical data not shown), have high growth rates. Ireland, Europe's fastest-growing economy from 2010-2018, has among the highest investment-to-GDP ratio in Europe. Economies that devote a smaller share of GDP to investment, such as Brazil and Mexico, have slower growth rates. The data helps explain why the Chinese economy has expanded so rapidly, achieving an annual GDP growth rate of nearly $10 \%$ over the last two decades. Investment spending in China on new factories, equipment, and infrastructure as a percentage of GDP is the highest in the world.

Exhibit 29: Business Investment As a Percentage of GDP and GDP Growth in the Subsequent Period

\begin{center}
\includegraphics[max width=\textwidth]{2023_05_04_ccb578c5089fd41615beg-192}
\end{center}

Notes: The term "investment" is represented by non-residential gross fixed capital formation as a share of GDP used. GDP growth is the average annual real GDP growth in the decade following the year of the investment as a percentage of GDP data.

Sources: GDP: OECD National Accounts Statistics (database), April 2019. Gross fixed capital: World Development Indicator, ne.gdi.totl.zs

\section{Natural Resources}
Raw materials are an essential input to growth and include everything from available land to oil to water. Historically, consumption of raw materials has increased as economies have grown. There are two categories of natural resources:

\begin{enumerate}
  \item Renewable resources are those that can be replenished, such as a forest. For example, if a tree is cut, a seedling can be planted and a new forest harvested in the future.

  \item Non-renewable resources are finite resources that are depleted once they are consumed. Oil and coal are examples.

\end{enumerate}

Natural resources account for some of the differences in growth among countries. Today, such countries as Brazil and Australia, as well as those in the Middle East, have relatively high per capita incomes because of their resource base. Countries in the Middle East have large pools of oil. Brazil has an abundance of land suitable for large-scale agricultural production, making it a major exporter of coffee, soybeans, and beef.

Although natural resources are an important factor in growth, they are not necessary for a country to achieve a high level of income provided it can acquire the requisite inputs through trade. Countries in eastern Asia, such as Japan and South Korea, have in the past experienced rapid economic growth but own few natural resources. It is also important to note that having plentiful natural resources may hurt an economy's growth, a topic that will be explored at a later stage.

\section{Public Infrastructure}
Roads, water systems, mass transportation, airports, and utilities are all examples of public infrastructure or public capital. Even if the infrastructure assets are privately owned or operated, the economic impact will be the same. Infrastructure assets have few substitutes and generate significant economies of scale, so they have the characteristics of a natural monopoly. A key feature of public capital is that they create externalities. Externalities are spillover effects of production and consumption activities onto others who are not directly involved in a particular transaction, activity, or decision. In the case of public infrastructure, the externalities complement the production of private sector goods and services. The full benefit of public infrastructure investment may extend beyond its direct expenditure because it boosts the productivity of private capital.

\section{Other Factors Driving Growth}
Researchers have focused on the positive externalities associated with research and development and public education. Research and development efforts by one firm spill over and affect the stock of knowledge available to all firms, raising potential GDP growth. Thus, external or social benefits arise from increased spending on research and development that go beyond the benefit for the individual firm. The key point for economic growth is that there may be constant or increasing returns to the inputs, and thus the economy is no longer constrained by diminishing marginal productivity.

Externalities also have a negative effect on growth, with pollution being the primary example. Air and water quality in advanced countries have improved significantly over the last few decades, and there is a strong negative correlation between per capita GDP and pollution. But these success stories largely reflect the local economic effects of economic growth. The environmental problems we face today are global in nature and affect the entire planet. The burning of fossil fuels releases carbon dioxide into the atmosphere, causing a greenhouse effect that raises the earth's overall temperature. Climate change is closely linked to economic growth, because carbon dioxide emissions rise with economic growth. These emissions impose a cost on the global economy in terms of economic activity, food production, health, and habitability.

Without government action, individuals and firms may have no incentive to reduce these negative externalities. The more global the pollution problem, the less any single country benefits from reducing its emissions. Moreover, nations face an important trade-off on emissions: Reducing emissions today may result in lower short-term growth, but not reducing emissions today will negatively affect long-term sustainable GDP growth. Policy measures that create incentives to reduce the carbon intensity of output are the potential solution.

In addition to the aforementioned factors, a country's economic environment can play an important role in influencing economic growth. Lack of appropriate institutions and a poor legal and political environment restrains growth, especially in the developing countries. Factors such as internal competitive structure; poorly developed financial markets; a country's openness to trade; and its political stability, property rights, and rule of law can influence innovative activity and economic growth.

\section{EXAMPLE 15}
\section{Sources of Growth for Westerland}
Westerland is an advanced economy with a high capital-to-labor ratio. Working-age population growth has been close to zero for the last decade. During the same period, annual GDP growth has slowed to about $1 \%$. More than $70 \%$ of Westerland's electricity is produced by renewable energy resources. The nation has reduced carbon dioxide emissions by $25 \%$ over the last decade. Policymakers hope to raise Westerland's sustainable rate of growth or its potential GDP.

Explain how the following factors are a potential source of growth for Westerland.

\begin{enumerate}
  \item Tax reductions are proposed on business investment and R\&D.
\end{enumerate}

\section{Solution to 1:}
Because Westerland is an advanced economy operating at a high level of capital per worker, diminishing returns to capital are likely to be significant. If this is the case, additions to capital will have little effect on growth and are not a significant source of growth. In contrast, incentives encouraging increased R\&D expenditures by increasing the growth rate of TFP and technological change are a potential source of growth.

\begin{enumerate}
  \setcounter{enumi}{1}
  \item Labor market incentives are planned to create flexible work schedules and develop policies for maternity/parental leave and childcare.
\end{enumerate}

\section{Solution to 2:}
Westerland is facing a demographic challenge because slow population growth will limit its potential growth rate. To offset this issue, the country can encourage immigration and/or increase its workforce participation rate. The objective of creating flexible work schedules and developing policies for parental leave and childcare is to raise the labor force participation rate (note that Working age population $\times$ Participation rate $\times$ Average hours worked gives $L$, the amount of labor). In the short run, the labor market incentives will increase the growth rate in the labor supply and thus are a potential source of growth. After the initial response to these incentives, however, the labor force participation rate will level off. As a result, the policies will have no long-term effect on the labor input and will not increase sustainable growth.

\begin{enumerate}
  \setcounter{enumi}{2}
  \item Westerland plans to further reduce its carbon dioxide emissions, but global carbon dioxide emissions are expected to rise significantly as global GDP growth accelerates.
\end{enumerate}

\section{Solution to 3:}
Westerland has successfully reduced its carbon dioxide emissions and plans further reduction, probably at the cost of slower GDP growth. Nevertheless, it still faces the negative effects of climate change despite its own effort to reduce emissions. This dynamic highlights climate change as a classic example of a negative externality that no single country can solve and requires global cooperation. This international burden sharing presents a stumbling block for success.

\section{Measures of Sustainable Growth}
Measuring how fast an economy can grow is an important exercise. Economists project potential GDP in order to forecast the economy's sustainable growth path. Potential GDP is an unobserved concept that is approximated using a number of alternative methods. Note that estimates of the economy's potential growth can change as new data become available. Being able to understand such changes is critical for financial analysts, because long-term forecasts for cash flows that companies generate are dependent on the sustainable rate of economic growth.

\section{Labor Productivity}
One approach to estimating potential GDP uses the productivity of the labor force, which is generally reliable data. Labor productivity consists of the quantity of goods and services (real GDP) that a worker can produce in one hour of work. Our standard of living improves if we produce more goods and services for each hour of work. Labor productivity is calculated as real GDP for a given year divided by the total number of hours worked in that year, counting all workers. We use total hours, rather than the number of workers, to adjust for the fact that not everyone works the same number of hours.

Labor productivity $=$ Real GDP/Aggregate hours

Therefore, we need to understand the forces that make labor more productive. Productivity is determined by the factors that we examined earlier: education and skill of workers (human capital), investments in physical and public capital, natural resources, improvements in technology, and other factors. An increase in any of these factors will increase the productivity of the labor force. The factors determining labor productivity can be derived from the production functions under the assumption of constant returns to scale, where a doubling of inputs causes output to double as well. Dividing the production function by $L$, we obtain the following:

$$
Y / L=A \boldsymbol{F}(1, K / L)
$$

where $Y / L$ is output per worker, which is a measure of labor productivity. The equation states that labor productivity depends on physical capital per worker $(K / L)$ and total factor productivity, which largely measures technological change $(A)$. TFP is a scale factor that does not depend on the mix of inputs. Changes in TFP are measured as a residual, capturing growth that cannot be attributed to the labor supply and physical capital inputs. On the other hand, as shown in this equation, output per worker depends on both the general level of productivity (reflected in TFP) and the mix of inputs. Increases in either TFP or the capital-to-labor ratio boost output per worker. Because both output and labor input can be observed, labor productivity can be measured directly.

Labor productivity is a key concept for measuring the health and prosperity of an economy and its sustainable rate of growth. An analyst examining the growth prospects for an economy needs to focus on the labor productivity data for that country. Growth of labor productivity over time largely explains the differences in the living standards among countries. Labor productivity growth also helps analysts estimate the long-term sustainable growth rates among countries. The distinction between the level and growth rate of productivity is important to understand. Exhibit 30 shows that many countries have relatively high levels of productivity, even though productivity growth during the last decade or two has not been significant (e.g., Italy, the United Kingdom, Spain). Ireland, however, entered this century with a lower level of productivity and had a lower starting point compared with the other countries. Its substantial productivity growth in the last two decades has elevated the country to the top of this particular list. Exhibit 30: Labor Productivity: Level vs. Growth Rate in Select Countries

\section{A. 2018 GDP per Hour Worked}
USD

\begin{center}
\includegraphics[max width=\textwidth]{2023_05_04_ccb578c5089fd41615beg-196}
\end{center}

B. Annual Productivity Growth (\%)

USD

\begin{center}
\includegraphics[max width=\textwidth]{2023_05_04_ccb578c5089fd41615beg-196(1)}
\end{center}

Source: OECD Productivity Statistics (database), April 2019

\section{Level of Labor Productivity}
The higher the level of labor productivity, the more goods and services the economy can produce with the same number of workers. The level of labor productivity depends on the accumulated stock of human and physical capital and is much higher in the developed countries. For example, India has a population of about 1.3 billion people, compared with more than 83 million people in Germany (United Nations, World Population Prospects 2019). Because of its much larger population, India has significantly more workers than Germany; however, the German economy, as measured by real GDP, is larger. As shown in Exhibit 30, Germany has among the highest level of productivity in the world, producing more than US $\$ 70$ of GDP per hour worked. Similarly, workers in France, the United States, and Ireland have high levels of productivity. In comparison, Mexican workers produce only US $\$ 20.70$ worth of GDP per hour worked. Thus, German workers are more than three times more productive as Mexican workers.

\section{Growth Rate of Labor Productivity}
The growth rate of labor productivity is the percentage increase in productivity over a year. It is among the economic statistics that economists and financial analysts watch most closely. In contrast to the level of productivity, the growth rate of productivity is typically higher in developing countries where human and physical capital is scarce but growing rapidly.

If productivity growth is rapid, it means the same number of workers can produce more and more goods and services. In this case, companies can afford to pay higher future wages and still make a profit. Thus, high rates of productivity growth will translate into rising profits. Rising profits, if not already anticipated by the market, will support higher stock prices. In contrast, persistently low productivity growth suggests the economy is in bad shape. Without productivity gains, businesses have to either cut wages or boost prices in order to increase profit margins. Low rates of productivity growth should be associated with slow growth in profits and flat or declining stock prices.

\section{EXAMPLE 16}
\section{Prospects for Equity Returns in Mexico}
\begin{enumerate}
  \item John Todd, CFA, manages a global mutual fund with nearly $30 \%$ of its assets invested in Europe. Because of the low population growth rate, he is concerned about the long-term outlook for the European economies. With potentially slower economic growth in Europe, the environment for equities may be less attractive. Therefore, Todd is considering reallocating some of the assets from Europe to Mexico. Data shows that the level of business investment has been around $24 \%$ of GDP, while that of China is more than $40 \%$ and that of European countries is close to $20 \%$. The level of productivity for Europe as a whole is more than $\$ 60$ per hour worked (as of 2018). Data also shows that labor productivity since 2001 in Mexico is growing at about $0.2 \%$ annual rate, below that of Germany, France, and Spain. Do you think that investment opportunities are favorable in Mexico? According to the OECD, the Mexican population increased by $1.1 \%$ in 2018 , compared with a $0.2 \%$ increase in the European Union population (27 countries).
\end{enumerate}

\section{Solution:}
Other than the higher population growth rate, the potential sources of growth for Mexico are not favorable. The level of business investment in Mexico is quite low, especially compared with China, and not much higher than that of many of the advanced economies in Europe. The level of labor productivity in Mexico is well below that in most European countries. This is not surprising given that the amount of capital per worker in Mexico is much lower than that in Europe.

What is surprising and of concern is the low rate of labor productivity growth in Mexico. This means that the rightward shift in the aggregate supply curve is greater for the European countries than for Mexico, despite the more favorable demographic trend in the country. In addition, it implies that there is more potential for expanding profit margins in Europe than in Mexico. Thus, the analysis of potential growth does not suggest a favorable outlook for equity returns in Mexico. In this simplified example, in the absence of more favorable considerations-for example, compelling equity valuations-Todd should decide not to reallocate assets from Europe to Mexico.

\section{Measuring Sustainable Growth}
Labor productivity data can be used to estimate the rate of sustainable growth of the economy. A useful way to describe potential GDP is as a combination of aggregate hours worked and the productivity of those workers:

Potential GDP $=$ Aggregate hours worked $\times$ Labor productivity Transforming the above equation into growth rates, we obtain the following:

Potential growth rate = Long-term growth rate of aggregate hours worked + Long-term labor productivity growth rate

Thus, potential growth is a combination of the long-term growth rate of the aggregate hours worked and the long-term growth rate of labor productivity. Therefore, if aggregate hours worked are growing at $1 \%$ per year and productivity per worker is rising at $2 \%$ per year, then potential GDP (adjusted for inflation) is rising at 3\% per year.

\section{EXAMPLE 17}
\section{Estimating the Rate of Growth in Potential GDP}
\begin{enumerate}
  \item Exhibit 31 provides data on the growth rate in aggregate hours worked and productivity growth for Canada, Germany, Japan, and the United States. Estimate the growth rate of potential GDP for each country by averaging the growth rates for these variables since 2001. Also, describe the role that the labor input plays in determining potential growth for each country.
\end{enumerate}

Exhibit 31: Aggregate Hours Worked and Productivity: Average Annual Growth Rate

\begin{center}
\begin{tabular}{lcccc}
\hline
Aggregate Hours Worked & \multicolumn{2}{c}{Productivity} &  &  \\
\hline
 & $\mathbf{2 0 1 1 - 2 0 1 8}$ & $\mathbf{2 0 0 1 - 2 0 1 0}$ & $\mathbf{2 0 1 1 - 2 0 1 8}$ & $\mathbf{2 0 0 1 - 2 0 1 0}$ \\
\hline
Canada & $0.9 \%$ & $1.4 \%$ & $1.0 \%$ & $0.9 \%$ \\
Germany & 0.5 & 0.2 & 1.1 & 0.9 \\
Japan & 0.3 & -0.2 & 0.8 & 1.2 \\
United States & 0.7 & 0.7 & 1.5 & 0.8 \\
\hline
\end{tabular}
\end{center}

\section{Solution:}
Potential GDP is calculated as the sum of the trend growth rate in the aggregate hours worked (labor input) and the trend growth rate in labor productivity. The growth in the labor input depends on the population growth rate, changes in the labor force participation rate, and changes in hours worked per person. Estimating based on the average for the period from 2001-2018 gives the following data:

\begin{center}
\begin{tabular}{lccc}
\hline
 & $\begin{array}{c}\text { Projected Growth } \\ \text { in Aggregate } \\ \text { Hours Worked }\end{array}$ & $\begin{array}{c}\text { Projected } \\ \text { Growth in Labor } \\ \text { Productivity }\end{array}$ & $\begin{array}{c}\text { Projected Growth } \\ \text { in Potential GDP }\end{array}$ \\
\hline
Canada & $1.2 \%$ & $0.9 \%$ & $2.1 \%$ \\
Germany & 0.3 & 1.0 & 1.3 \\
Japan & 0.0 & 1.0 & 1.0 \\
United States & 0.7 & 1.1 & 1.8 \\
\hline
\end{tabular}
\end{center}

Note: Rounding used throughout.

Most of the difference between the growth rates in potential GDP among these countries can be explained by the labor input. The most significant result is the difference in the growth rate in aggregate hours in Germany and Japan, in contrast to that in the United States and Canada. The results suggest that Japan's sluggish economic growth is likely to continue because of the lack of growth in the labor input.

\section{EXAMPLE 18}
\section{Prospects for Fixed-Income Investments}
\begin{enumerate}
  \item As a fixed income analyst for a large Canadian bank, you are preparing the 2019-2020 forecast and have just received the latest GDP forecast from the OECD for Canada, Germany, Japan, and the United States. The forecast is given in Exhibit 32, along with the potential GDP growth estimates that the analyst has calculated.
\end{enumerate}

\section{Exhibit 32: Projected vs. Potential GDP Growth}
\begin{center}
\begin{tabular}{lcc}
\hline
 & $\begin{array}{c}\text { Projected Average } \\ \text { Annual GDP Growth } \\ (\mathbf{2 0 1 9 - 2 0 2 0 )}\end{array}$ & $\begin{array}{c}\text { Projected Growth in } \\ \text { Potential GDP }\end{array}$ \\
\hline
Canada & $4.0 \%$ & $2.1 \%$ \\
Germany & 1.5 & 1.3 \\
Japan & 0.0 & 1.0 \\
United States & 3.5 & 1.8 \\
\hline
\end{tabular}
\end{center}

To evaluate the future prospects for fixed-income investments, analysts must estimate the future rate of inflation and assess the possibility of changes in monetary policy by the central bank. An important indicator for both of these factors is the degree of slack in the economy, which can be measured by comparing the growth rates of actual GDP and potential GDP.

Based on the information in Exhibit 32, evaluate the prospects for fixed-income investments in each country.

\section{Solution:}
In comparing the OECD forecast for GDP growth with the estimated growth rate in potential GDP, there are two cases to consider:

\begin{enumerate}
  \item If actual GDP is growing at a faster rate than potential GDP, it signals growing inflationary pressures, and an increased likelihood that the central bank will raise interest rates.

  \item If actual GDP is growing at a slower rate than potential GDP, it signals growing resource slack, less inflationary pressures, and an increased likelihood that the central bank will reduce rates or leave them unchanged.

\end{enumerate}

Projected and potential GDP for the aforementioned countries can be compared referring to Exhibit 32.

The data suggest that inflationary pressure will grow in the United States and Canada and that both the Federal Reserve and the Bank of Canada will eventually raise interest rates. Thus, the environment for bond investing is not favorable in the United States and Canada, because bond prices are likely to decline.

With Germany growing at close to its potential rate of GDP growth, the rate of inflation should neither rise nor fall. Monetary policy is set by the European Central Bank (ECB), but data on the German economy play a big role in the ECB's decision. Based on the data in Exhibit 32, no change in ECB policy is likely. For bond investors, little change in bond prices is likely in Germany, so investors need to focus on the interest (coupon) income received from the bond.

Finally, growing resource slack in Japan will put downward pressure on inflation and may force the Bank of Japan to keep rates low. Bond prices should rise in this environment.

\section{SUMMARY}
\begin{itemize}
  \item GDP is the market value of all final goods and services produced within a country during a given period.

  \item GDP can be valued by looking at either the total amount spent on goods and services produced in the economy or the income generated in producing those goods and services.

  \item GDP counts only final purchases of newly produced goods and services during the current period. Transfer payments and capital gains are excluded from GDP.

  \item Intermediate goods are excluded from GDP in order to avoid double counting.

  \item GDP can be measured either from the value of final output or by summing the value added at each stage of the production and distribution process. The sum of the value added by each stage is equal to the final selling price of the good.

  \item Nominal GDP is the value of production using the prices of the current year. Real GDP measures production using the constant prices of a base year. The GDP deflator equals the ratio of nominal GDP to real GDP.

  \item Households earn income in exchange for providing-directly or indirectly through ownership of businesses-the factors of production (labor, capital, and natural resources including land). From this income, they consume, save, and pay net taxes.

  \item Businesses produce most of the economy's output/income and invest to maintain and expand productive capacity. Companies retain some earnings but pay out most of their revenue as income to the household sector and as taxes to the government.

  \item The government sector collects taxes from households and businesses and purchases goods and services, for both consumption and investment, from the private business sector.

  \item International trade consists of exports and imports. The difference between the two is net exports. If net exports are positive (negative), then the country spends less (more) than it earns. Net exports are balanced by accumulation of either claims on the rest of the world (net exports $>0$ ) or obligations to the rest of the world (net exports $<0$ ). - Capital markets provide a link between saving and investment in the economy.

  \item From the expenditure side, GDP includes personal consumption $(C)$, gross private domestic investment $(I)$, government spending $(G)$, and net exports $(X-M)$

  \item The major categories of expenditure are often broken down into subcategories. Gross private domestic investment includes both investment in fixed assets (plant and equipment) and the change in inventories. In some countries, government spending on investment is separated from other government spending.

  \item National income is the income received by all factors of production used in the generation of final output. It equals GDP minus the capital consumption allowance (depreciation) and a statistical discrepancy.

  \item Personal income reflects pre-tax income received by households. It equals national income plus transfers minus undistributed corporate profits, corporate income taxes, and indirect business taxes.

  \item Personal disposable income equals personal income minus personal taxes.

  \item Private saving must equal investment plus the fiscal and trade deficits. That is, $S=I+(G-T)+(X-M)$.

  \item Consumption spending is a function of disposable income. The marginal propensity to consume represents the fraction of an additional unit of disposable income that is spent.

  \item Investment spending depends on the average interest rate and the level of aggregate income. Government purchases and tax policy are often considered to be exogenous variables determined outside the macroeconomic model. Actual taxes collected depend on income and are, therefore, endogenous - that is, determined within the model.

  \item Aggregate demand and aggregate supply determine the level of real GDP and the price level.

  \item The downward slope of the aggregate demand curve arises as the result of three effects: the wealth effect, the interest rate effect, and the real exchange rate effect. The curve is drawn assuming a constant money supply.

  \item The aggregate demand curve will shift if there is a change in a factor, other than price, that affects aggregate demand. These factors include household wealth, consumer and business expectations, capacity utilization, monetary policy, fiscal policy, exchange rates, and other nations' GDP.

  \item The aggregate supply curve is the relationship between the quantity of real GDP supplied and the price level, keeping all other factors constant. Movements along the supply curve reflect the effects of price on supply.

  \item The short-run aggregate supply curve is upward sloping because higher prices result in higher profits and induce businesses to produce more and laborers to work more. In the short run, some prices are sticky, implying that some prices do not adjust to changes in demand.

  \item In the long run, all prices are assumed to be flexible. The long-run aggregate supply curve is vertical because input costs adjust to changes in output prices, leaving the optimal level of output unchanged. The position of the curve is determined by the economy's level of potential GDP.

  \item The level of potential output, also called the full employment or natural level of output, is unobservable and difficult to measure precisely. This concept represents an efficient and unconstrained level of production at which companies have enough spare capacity to avoid bottlenecks and there is a balance between the pool of unemployed workers and the pool of job openings.

  \item The long-run aggregate supply curve will shift because of changes in labor supply, supply of physical and human capital, and productivity/technology.

  \item The short-run supply curve will shift because of changes in potential GDP, nominal wages, input prices, expectations about future prices, business taxes and subsidies, and the exchange rate.

  \item Short-term fluctuations in GDP are caused by shifts in aggregate demand and aggregate supply.

  \item When the level of GDP in the economy is below potential GDP, such a recessionary situation exerts downward pressure on the aggregate price level.

  \item When the level of GDP is above potential GDP, such an overheated situation puts upward pressure on the aggregate price level.

  \item Stagflation, a combination of high inflation and weak economic growth, is caused by a decline in short-run aggregate supply.

  \item Growth in real GDP measures how rapidly the total economy is expanding. Per capita GDP, defined as real GDP divided by population, reflects a country's standard of living. Real GDP growth rates and levels of per capita GDP vary widely among countries.

  \item The sources of economic growth include the supply of labor, the supply of physical and human capital, raw materials, and technological knowledge.

  \item Output can be described in terms of a production function. For example, $Y=A \boldsymbol{F}(L, K)$, where $L$ is the quantity of labor, $K$ is the capital stock, and $A$ represents total factor productivity (primarily technology). The function $\boldsymbol{F}(\cdot)$ is assumed to exhibit constant returns to scale but diminishing marginal productivity for each input individually.

  \item Total factor productivity is a scale factor that reflects the portion of output growth that is not accounted for by changes in the capital and labor inputs. TFP is mainly a reflection of technological change.

  \item Based on a two-factor production function, Potential GDP growth = Growth in TFP $+W_{L}$ (Growth in labor) $+W_{C}$ (Growth in capital), where $W_{L}$ and $W_{C}\left(=1-W_{L}\right)$ are the shares of labor and capital in GDP.

  \item Diminishing marginal productivity implies that

  \item increasing the supply of some input(s) relative to other inputs will lead to diminishing returns and cannot be the basis for sustainable growth. In particular, long-term sustainable growth cannot rely solely on capital deepening (i.e., on increasing the stock of capital relative to labor).

  \item given the relative scarcity and hence high productivity of capital in developing countries, the growth rate of developing countries should exceed that of developed countries.

  \item The sustainable rate of economic growth is measured by the rate of increase in the economy's productive capacity or potential GDP.

  \item The labor supply is determined by both population growth (including immigration) and the labor force participation rate. The capital stock in a country increases with investment. Correlation between long-run economic growth and the rate of investment is high. - In addition to labor, capital, and technology, human capital-essentially, the quality of the labor force-and natural resources are important determinants of output and growth as is the state of public infrastructure, institutional, legal and political environment.

  \item Technological advances are discoveries that make it possible to produce more and/or higher-quality goods and services with the same resources or inputs. Technology is the main factor affecting economic growth in developed countries.

  \item Externalities and spillover effects are also recognized to play an important role in growth. Positive externalities associated with research and development and public education. Negative externalities such as pollution have a negative effect on growth.

  \item Climate change is closely linked to economic growth as carbon dioxide emissions rise with economic growth. These emissions impose a cost on the global economy in terms of economic activity, food production, health and habitability.

  \item The sustainable rate of growth in an economy is determined by the growth rate of the labor supply plus the growth rate of labor productivity.

\end{itemize}

\section{REFERENCES}
Case, K., J. Quigley, R. Shiller. 2005. "Comparing Wealth Effects: The Stock Market versus the Housing Market." Advances in Macroeconomics, vol. 5, no. 1.

Funke, N. 2004. "Is There a Stock Market Wealth Effect in Emerging Markets?" International Monetary Fund (March). 10.1016/j.econlet.2003.12.002

\section{PRACTICE PROBLEMS}
\begin{enumerate}
  \item Which of the following statements is the most appropriate description of gross domestic product (GDP)?
\end{enumerate}

A. The total income earned by all households, firms, and the government whose value can be verified

B. The total amount spent on all final goods and services produced within the economy during a given period

C. The total market value of resalable and final goods and services produced within the economy during a given period

\begin{enumerate}
  \setcounter{enumi}{1}
  \item The component least likely to be included in a measurement of gross domestic product (GDP) is:
\end{enumerate}

A. the value of owner occupied rent.

B. the annual salary of a local police officer.

C. environmental damage caused by production.

\begin{enumerate}
  \setcounter{enumi}{2}
  \item Which of the following would be included in Canadian GDP for a given year? The market value of:
A. wine grown in Canada by US citizens.
B. electronics made in Japan and sold in Canada.
C. movies produced outside Canada by Canadian filmmakers.

  \item Suppose a painting is produced and sold in 2018 for $\pounds 5,000$. The expenses involved in producing the painting amounted to $\pounds 2,000$. According to the sum-of-value-added method of calculating GDP, the value added by the final step of creating the painting was:
A. $\pounds 2,000$
B. $\pounds 3,000$.
C. $\pounds 5,000$

  \item Which of the following conditions is least likely to increase a country's GDP?
A. An increase in net exports
B. Increased investment in capital goods
C. Increased government transfer payments

  \item The most accurate description of nominal GDP is:
A. a measure of total expenditures at current prices.
B. the value of goods and services at constant prices.
C. a measure to compare one nation's economy to another. 7. From the beginning to the ending years of a decade, the annual value of final goods and services for country $X$ increased from $€ 100$ billion to $€ 300$ billion. During that period, the GDP deflator increased from 111 to 200 . Over the decade, real GDP for country $X$ increased by approximately:
A. $50 \%$.
B. $67 \%$.
C. $200 \%$.

  \item If the GDP deflator values for Year 1 and Year 3 were 190 and 212.8, respectively, which of the following best describes the annual growth rate of the overall price level?
A. $5.8 \%$.
B. $6 \%$.
C. $12 \%$.

  \item The numerator of the GDP price deflator reflects:

\end{enumerate}

A. the value of base year output at current prices.

B. the value of current year output at current prices.

C. the value of current year output at base year prices.

\begin{enumerate}
  \setcounter{enumi}{9}
  \item Consider the following data for a hypothetical country:
\end{enumerate}

\begin{center}
\begin{tabular}{lc}
\hline
Account name & Amount (\$ trillions) \\
\hline
Consumption & 15.0 \\
Capital consumption allowance & 1.5 \\
Government spending & 3.8 \\
Imports & 1.7 \\
Gross private domestic investment & 4.0 \\
Exports & 1.5 \\
\hline
\end{tabular}
\end{center}

Based only on the data given, the gross domestic product and national income are respectively closest to:
A. 21.1 and 20.6 .
B. 22.6 and 21.1.
C. 22.8 and 20.8 .

\begin{enumerate}
  \setcounter{enumi}{10}
  \item In calculating personal income for a given year, which of the following would not be subtracted from national income?
A. Indirect business taxes
B. Undistributed corporate profits
C. Unincorporated business net income

  \item Equality between aggregate expenditure and aggregate output implies that the government's fiscal deficit must equal:

\end{enumerate}

A. Private saving - Investment - Net exports.

B. Private saving - Investment + Net exports.

C. Investment - Private saving + Net exports.

\begin{enumerate}
  \setcounter{enumi}{12}
  \item Because of a sharp decline in real estate values, the household sector has increased the fraction of disposable income that it saves. If output and investment spending remain unchanged, which of the following is most likely?
\end{enumerate}

A. A decrease in the government deficit

B. A decrease in net exports and increased capital inflow

C. An increase in net exports and increased capital outflow

\begin{enumerate}
  \setcounter{enumi}{13}
  \item A GDP deflator less than 1 indicates that an economy has experienced:
A. inflation
B. deflation.
C. stagflation.

  \item In explaining the downward slope of the aggregate demand curve, the interest rate effect suggests that a higher price level will:

\end{enumerate}

A. lead to lower interest rates, which will lead to higher levels of consumption and investment.

B. lead to higher interest rates, which will lead to lower levels of consumption and investment.

C. not affect interest rates and therefore will lead to a higher level of aggregate demand.

\begin{enumerate}
  \setcounter{enumi}{15}
  \item One of the reasons aggregate demand curve is downward sloping is because a lower price level means that the real exchange rate:
\end{enumerate}

A. depreciates, making domestic goods cheaper in other countries and imports less competitive, resulting in a higher level of net exports.

B. appreciates, making domestic goods more expensive in other countries and imports more competitive, resulting in a higher level of net exports.

C. appreciates, making the country's exports and imports less competitive and leading to lower net exports.

\begin{enumerate}
  \setcounter{enumi}{16}
  \item The full employment, or natural, level of output is best described as:
A. the maximum level obtainable with existing resources.
B. the level at which all available workers have jobs consistent with their skills.
C. a level with a modest, stable pool of unemployed workers transitioning to new jobs.

  \item Which of the following best describes the aggregate supply curve in the short run (one to two years)? The short-run aggregate supply curve is:

\end{enumerate}

A. flat, because output is more flexible than prices in the short run.

B. vertical, because wages and other input prices fully adjust to the price level.

C. upward sloping, because input prices do not fully adjust to the price level in the short run.

\begin{enumerate}
  \setcounter{enumi}{18}
  \item If wages were automatically adjusted for changes in the price level, the short-run aggregate supply curve would most likely be:
A. flatter.
B. steeper.
C. unchanged.

  \item The least likely cause of a decrease in aggregate demand is:
A. higher taxes.
B. a weak domestic currency.
C. a fall in capacity utilization.

  \item Which of the following is most likely to cause the long-run aggregate supply curve to shift to the left?

\end{enumerate}

A. Higher nominal wages

B. A decline in productivity

C. An increase in corporate taxes

\begin{enumerate}
  \setcounter{enumi}{21}
  \item Increased household wealth will most likely cause an increase in:
\end{enumerate}

A. household saving.

B. investment expenditures.

C. consumption expenditures.

\begin{enumerate}
  \setcounter{enumi}{22}
  \item The most likely outcome when both aggregate supply and aggregate demand increase is:
A. a rise in inflation.
B. higher employment.
C. an increase in nominal GDP.

  \item Which of the following is least likely to be caused by a shift in aggregate demand?
A. Stagflation
B. A recessionary gap
C. An inflationary gap

  \item Following a sharp increase in the price of energy, the overall price level is most likely to rise in the short run:

\end{enumerate}

A. and remain elevated indefinitely unless the central bank tightens.

B. but remain unchanged in the long run unless the money supply is increased.

C. and continue to rise until all prices have increased by the same proportion.

\begin{enumerate}
  \setcounter{enumi}{25}
  \item Among developed economies, which of the following sources of economic growth is most likely to explain superior growth performance?
A. Technology
B. Capital stock
C. Labor supply

  \item Which of the following can be measured directly?
A. Potential GDP
B. Labor productivity
C. Total factor productivity

  \item The sustainable growth rate is best estimated as:
A. the weighted average of capital and labor growth rates.
B. growth in the labor force plus growth of labor productivity.
C. growth in total factor productivity plus growth in the capital-to-labor ratio.

  \item In the neoclassical or Solow growth model, an increase in total factor productivity reflects an increase in:
A. returns to scale.
B. output for given inputs.
C. the sustainable growth rate.

\end{enumerate}

\section{The following information relates to questions}
\section{0-31}
An economic forecasting firm has estimated the following equation from historical data based on the neoclassical growth model:

Potential output growth $=1.5+0.72 \times$ Growth of labor $+0.28 \times$ Growth of capital

\begin{enumerate}
  \setcounter{enumi}{29}
  \item The intercept (1.5) in this equation is best interpreted as:
A. the long-run sustainable growth rate.
B. the growth rate of total factor productivity.
C. above-trend historical growth that is unlikely to be sustained. 31. The coefficient on the growth rate of labor (0.72) in this equation is best interpreted as:
\end{enumerate}

A. the labor force participation rate.

B. the marginal productivity of labor.

C. the share of income earned by labor.

\begin{enumerate}
  \setcounter{enumi}{31}
  \item Convergence of incomes over time between emerging market countries and developed countries is most likely the result of:
A. total factor productivity.
B. diminishing marginal productivity of capital.
C. the exhaustion of non-renewable resources.
\end{enumerate}

\section{SOLUTIONS}
\begin{enumerate}
  \item B is correct. GDP is the total amount spent on all final goods and services produced within the economy during a specific period.

  \item C is correct. Byproducts of production processes that have no explicit market value are not included in GDP.

  \item A is correct. Canadian GDP is the total market value of all final goods and services produced during a given period within Canada. The wine was produced in Canada and counts toward Canadian GDP.

  \item B is correct. The value added by the artist is $\pounds 5,000-\pounds 2,000=\pounds 3,000$.

  \item C is correct. Government transfer payments, such as unemployment compensation or welfare benefits, are excluded from GDP.

  \item A is correct. Nominal GDP is defined as the value of goods and services measured at current prices. Expenditure is used synonymously with the value of goods and services because aggregate expenditures must equal aggregate output of an economy.

  \item B is correct. Real GDP in the first year was $€ 100$ billion/1.11 $=€ 90$ billion, and in the last year it was $€ 300$ billion/2.00 $=€ 150$ billion. Thus, $(€ 150-€ 90) / € 90=$ 0.67 , or $67 \%$.

  \item A is correct: $(212.8 / 190)^{1 / 2}-1=0.0583$, or $5.8 \%$.

  \item B is correct.

\end{enumerate}

GDP deflator $=\frac{\text { Value of current year output at current year prices }}{\text { Value of current year output at base year prices }} \times 100$

\begin{enumerate}
  \setcounter{enumi}{9}
  \item $\mathrm{B}$ is correct. GDP $=$ Consumption + Gross private domestic investment + Government spending + (Exports - Imports $)=15+4+3.8+(1.5-1.7)=22.6$. National income $=$ GDP $-\mathrm{CCA}=22.6-1.5=21.1$

  \item C is correct. Unincorporated business net income, also known as proprietor's income, is included in personal income.

  \item A is correct. The fundamental relationship among saving, investment, the fiscal balance, and the trade balance is $S=I+(G-T)+(X-M)$. This form of the relationship shows that private saving must fund investment expenditures, the government fiscal balance, and net exports (= net capital outflows). Rearranging gives $G-T=(S-I)-(X-M)$. The government's fiscal deficit $(G-T)$ must equal the private sector's saving/investment balance $(S-I)$ minus net exports.

  \item $C$ is correct. The fundamental relationship among saving, investment, the fiscal balance, and the trade balance is $S=I+(G-T)+(X-M)$. Given the levels of output and investment spending, an increase in saving (reduction in consumption) must be offset by either an increase in the fiscal deficit or an increase in net exports. Increasing the fiscal deficit is not one of the choices, so an increase in net exports and corresponding increase in net capital outflows (increased lending internationally and/or increased international purchases of assets) is the correct response.

  \item B is correct. The GDP deflator $=$ Nominal GDP/Real GDP. To obtain a ratio less than 1, real GDP must exceed nominal GDP, which indicates that prices have decreased and, accordingly, deflation has occurred.

  \item B is correct. Assuming a fixed supply of money, a higher price level will lead to a higher price of money. Because the price of money is the interest rate, as the price level increases, the interest rate increases. Higher interest rates will lead to lower consumption and investment.

  \item A is correct. A lower price level results in a weaker real exchange rate, making domestic goods cheaper to people in other countries and imports less competitive, resulting in a higher level of net exports.

  \item C is correct. At the full employment, or natural, level of output, the economy is operating at an efficient and unconstrained level of production. Companies have enough spare capacity to avoid bottlenecks, and there is a modest, stable pool of unemployed workers (job seekers equal job vacancies) looking for and transitioning into new jobs.

  \item $\mathrm{C}$ is correct. Because of long-term contracts and other rigidities, wages and other input costs do not fully adjust to changes in the price level in the short run. Given input prices, firms respond to output price changes by expanding or contracting output to maximize profit. Hence, the SRAS is upward sloping.

  \item B is correct. The slope of the short-run aggregate supply curve reflects the extent to which wages and other input costs adjust to the overall price level. Automatic adjustment of wages would mitigate the impact of price changes on profitability. Hence, firms would not adjust output as much in response to changing output prices-the SRAS curve would be steeper.

  \item B is correct. A weak domestic currency will result in an increase in aggregate demand at each price level-a rightward shift in the AD curve. A weaker currency will cause a country's exports to be cheaper in global markets. Conversely, imports will be more expensive for domestic buyers. Hence, the net exports component of aggregate demand will increase.

  \item B is correct. Productivity measures the efficiency of labor and is the amount of output produced by workers during a given period. A decline in productivity implies decreased efficiency. A decline in productivity increases labor costs, decreases profitability, and results in lower output at each output price level-a leftward shift in both the short-run and long-run aggregate supply curves.

  \item $\mathrm{C}$ is correct. The wealth effect explains the impact of increases or decreases in household wealth on economic activity. Household wealth includes financial and real assets. As asset values increase, consumers save less and spend more out of current income because they will still be able to meet their wealth accumulation goals. Therefore, an increase in household wealth results in a rightward shift in the aggregate demand curve.

  \item B is correct. Higher aggregate demand and higher aggregate supply raise real GDP and lower unemployment, meaning employment levels increase.

  \item A is correct. Stagflation occurs when output is declining and prices are rising. This dynamic most likely results from a decline in aggregate supply-a leftward shift of the SRAS curve. Depending on the source of the shift, the LRAS may shift too.

  \item B is correct. An increase in energy prices will shift the short-run aggregate supply curve (SRAS) to the left, reducing output and increasing prices. If the aggregate demand curve does not change-in particular, if the central bank does not expand the money supply-slack in the economy will put downward pressure on in input prices, shifting the SRAS back to its original position. In the long run, the price level will be unchanged.

  \item A is correct. Technology is the most important factor affecting economic growth for developed countries. Technological advances are very important because they allow an economy to overcome the limits imposed by diminishing marginal returns.

  \item B is correct. Labor productivity can be directly measured as output/hour.

  \item B is correct. Output growth is equal to the growth rate of the labor force plus the growth rate of labor productivity-that is, output per worker. Unlike total factor productivity, output per worker is observable, so this measure is the most practical way to approach estimation of sustainable growth.

  \item B is correct. Total factor productivity is a scale factor primarily reflecting technology. An increase in TFP means that output increases for any level of factor inputs.

  \item B is correct. The estimated equation is the standard Solow growth accounting equation. The intercept is the growth rate of total factor productivity.

  \item $C$ is correct. In the standard Solow growth accounting equation, the coefficient on each factor's growth rate is its share of income.

  \item B is correct. Diminishing marginal productivity of capital means that as a country accumulates more capital per worker, the incremental boost to output declines. Thus, all else equal, economies grow more slowly as they become more capital intensive. Given the relative scarcity and hence high marginal productivity of capital in developing countries, these economies tend to grow more rapidly than developed countries. This dynamic leads to convergence in income levels over time.

\end{enumerate}

\section*{LEARNING MODULE 
 4 }
\section{Understanding Business Cycles}


\section{LEARNING OUTCOME}
\begin{center}
\begin{tabular}{|c|c|}
\hline
Mastery & The candidate should be able to: \\
\hline
$\square$ & describe the business cycle and its phases \\
\hline
$\square$ & describe credit cycles \\
\hline
$\square$ & $\begin{array}{l}\text { describe how resource use, consumer and business activity, housing } \\ \text { sector activity, and external trade sector activity vary as an economy } \\ \text { moves through the business cycle }\end{array}$ \\
\hline
$\square$ & describe theories of the business cycle \\
\hline
$\square$ & $\begin{array}{l}\text { interpret a set of economic indicators, and describe their uses and } \\ \text { limitations }\end{array}$ \\
\hline
$\square$ & $\begin{array}{l}\text { describe types of unemployment, and compare measures of } \\ \text { unemployment }\end{array}$ \\
\hline
$\square$ & explain inflation, hyperinflation, disinflation, and deflation \\
\hline
$\square$ & explain the construction of indexes used to measure inflation \\
\hline
$\square$ & compare inflation measures, including their uses and limitations \\
\hline
$\square$ & contrast cost-push and demand-pull inflation \\
\hline
\end{tabular}
\end{center}

\section{INTRODUCTION}
A typical economy's output of goods and services fluctuates around its longer-term path. We now turn our attention to those recurring, cyclical fluctuations in economic output. Some of the factors that influence short-term changes in the economy-such as changes in population, technology, and capital-are the same as those that affect long-term sustainable economic growth. But forces that cause shifts in aggregate demand and aggregate supply curves-such as expectations, political developments, natural disasters, and fiscal and monetary policy decisions-influence economies particularly in the short run. We first describe a typical business cycle and its phases. While each cycle is different, analysts and investors need to be familiar with the typical cycle phases and what they mean for the expectations and decisions of businesses and households that influence the performance of sectors and companies. These behaviors also impact financial conditions and risk appetite, thus impacting the setting of expectations and choices of portfolio exposures to different investment sectors or styles.

In the sections that follow, we describe credit cycles, introduce several theories of business cycles, and explain how different economic schools of thought interpret the business cycle and their recommendations with respect to it. We also discuss variables that demonstrate predictable relationships with the economy, focusing on those whose movements have value in predicting the future course of the economy. We then proceed to explain measures and features of unemployment and inflation.

\section{OVERVIEW OF THE BUSINESS CYCLE}
describe the business cycle and its phases

Business cycles are recurrent expansions and contractions in economic activity affecting broad segments of the economy. Burns and Mitchell (1946) define the business cycle as follows:

Business cycles are a type of fluctuation found in the aggregate economic activity of nations that organize their work mainly in business enterprises: a cycle consists of expansions occurring at about the same time in many economic activities, followed by similarly general recessions, contractions, and revivals which merge into the expansion phase of the next cycle; this sequence of events is recurrent but not periodic; in duration, business cycles vary from more than one year to 10 or 12 years.

This definition is rich with important insights. First, business cycles are typical of economies that rely mainly on business enterprises-therefore, not agrarian societies or centrally planned economies. Second, a cycle has an expected sequence of phases, alternating between expansion and contraction, or upswings and downturns. Third, such phases occur at about the same time throughout the economy. Finally, cycles are recurrent; they happen again and again over time but not in a periodic way; they do not all have the exact same intensity and duration. Exhibit 1 provides an illustration of the pattern of economic growth rate in developed markets.

\section{Exhibit 1: Fluctuations of Growth in OECD Countries over Time}
\begin{center}
\includegraphics[max width=\textwidth]{2023_05_04_ccb578c5089fd41615beg-217}
\end{center}

Note: The Organisation for Economic Co-Operation and Development (OECD) includes more than 30 large countries

Source: OECD.Stat (htttps://stats.oecd.org/) YoY change in quarterly GDP in OECD countries.

Burns' and Mitchell's definition remains helpful. History never repeats itself in quite the same way, but it certainly does offer patterns that can be used when analyzing the present and forecasting the future. Business cycle analysis is a wide-ranging topic with conflicting perspectives held by industry participants.

\section{Phases of the Business Cycle}
Business cycles are recurring sequences of alternating upswings and downturns. There are various ways to break the business cycle into phases. The most obvious is to divide it into two primary segments: the expansion, or the upswing, and the contraction, or the downturn, with two key turning points, or peaks and troughs (see Exhibit 2 and Exhibit 3). These two periods are fairly easy to identify in retrospect. Subdividing the cycle more finely, however, becomes ambiguous, even in retrospect, because it requires identifying more nuanced changes, such as acceleration or deceleration of growth without a change in its direction. It is thus useful to divide the cycle into several phases distinguished through both economic and financial market characteristics. Our focus is on economic characteristics of the different phases, but we will highlight their implication for the behavior of different segments of the financial markets along the way.

The timing of these periods will depend on the type of cycle. Before moving on to the description of the four distinct phases to which we will refer in the subsequent sections, we first explain the different cycle concepts that analysts should be aware of given the range of different opinions, interpretations, and descriptions that practitioners use.

\section{Types of Cycles}
\begin{itemize}
  \item Classical cycle refers to fluctuations in the level of economic activity (e.g., measured by GDP in volume terms). The contraction phases between peaks and troughs are often short, while expansion phases are much longer. Exhibit 2 shows the classical cycle of economic activity. In practice, the classical cycle is not used extensively by academics and practitioners because it does not easily allow the breakdown of movements in GDP between short-term fluctuations and long-run trends. In addition, an absolute decline in activity between peaks and troughs does not occur frequently.
\end{itemize}

\section{Exhibit 2: Classical Cycle}
\begin{center}
\includegraphics[max width=\textwidth]{2023_05_04_ccb578c5089fd41615beg-218}
\end{center}

Output (left scale)

\begin{itemize}
  \item Growth cycle refers to fluctuations in economic activity around the long-term potential or trend growth level. The focus is on how much actual economic activity is below or above trend growth in economic activity. The dashed "wave" in the lower part of Exhibit 3 captures the fluctuation of actual activity from trend growth activity. Exhibit 3 shows "gaps" between actual and trend output. The growth cycle definition comes closest to how mainstream economists think: It dissects overall economic activity into a part driven by long-run trends and a part reflecting short-run fluctuations. Compared to the classical view of business cycles, peaks are generally reached earlier and troughs later in time. The time periods below and above trend growth are of similar length.
\end{itemize}

\section{Exhibit 3: Classical and Growth Cycles}
\begin{center}
\includegraphics[max width=\textwidth]{2023_05_04_ccb578c5089fd41615beg-219(1)}
\end{center}

\begin{itemize}
  \item Growth rate cycle refers to fluctuations in the growth rate of economic activity (e.g., GDP growth rate). Peaks and troughs are mostly recognized earlier than when using the other two definitions (see Exhibit 4). An advantage is that there is not necessarily a need to first estimate a long-run growth path. Nevertheless, economists often refer to economic growth being above or below potential growth rate, reflecting upswings or downturns.
\end{itemize}

\section{Exhibit 4: Classical, Growth, and Growth Rate Cycles}
\begin{center}
\includegraphics[max width=\textwidth]{2023_05_04_ccb578c5089fd41615beg-219}
\end{center}

Notes: The vertical lines indicate troughs and peaks when using either the classical, growth, or growth rate cycle definition of a business cycle. The growth cycle reflects the percentage deviation of output relative to its trend. The growth rates in the growth rate cycle are calculated as annualized month-over-month growth rates.

\section{Practical Issues}
In practice, the definitions of a business cycle are used interchangeably, which often causes confusion with regard to how one labels the phases and their timing. The classical cycle definition is rarely used. In line with how most economists and practitioners view the cycle, we will generally be using the growth cycle concept in which business cycles can be thought of as fluctuations around potential output (the trend in potential output is shown as the upward sloping dotted line in Exhibit 3).

\section{Four Phases of the Cycle}
The overall business cycle can be split into four phases:

Recovery: The economy is going through the "trough" of the cycle, where actual output is at its lowest level relative to potential output. Economic activity, including consumer and business spending, is below potential but is starting to increase, closing the negative output gap.

Expansion: The recovery gathers pace, output increases, and the rate of growth is above average. Actual output rises above potential output, and the economy enters the so-called "boom" phase. Consumers increase spending, and companies increase production, employment, and investment. Prices and interest rates may start increasing. As the expansion continues, the economy may start to experience shortages in factors of production. Overinvestment in productive capacity may lead companies to reduce further investment spending.

Slowdown: Output of the economy reaches its highest level relative to potential output (largest positive output gap). The growth rate begins to slow relative to potential output growth, and the positive output gap begins to narrow. Consumers remain optimistic about the economy, but companies may rely on overtime rather than using new hires to meet demand. Inflation slows at some point, and price levels may decrease.

Contraction: Actual economic output falls below potential economic output. Consumer and business confidence declines. Companies reduce costs by eliminating overtime and reducing employment. The economy may experience declines in absolute economic activity, a recession, or if the fall in activity is particularly large, a depression. If the decline is moderate, this phase tends to be shorter than the expansion phase.

Exhibit 5 provides a summary of the key characteristics of each phase and describes how several important economic variables evolve through the course of a business cycle.

\section{Exhibit 5: Business Cycle Phase Characteristics}
\begin{center}
\includegraphics[max width=\textwidth]{2023_05_04_ccb578c5089fd41615beg-221}
\end{center}

\begin{center}
\begin{tabular}{|c|c|c|c|c|}
\hline
Phase & Recovery & Expansion & Slowdown & Contraction \\
\hline
Description & $\begin{array}{l}\text { Economy going through } \\ \text { a trough. Negative } \\ \text { output gap starts to } \\ \text { narrow. }\end{array}$ & $\begin{array}{l}\text { Economy enjoying an } \\ \text { upswing. Positive out- } \\ \text { put gap opens. }\end{array}$ & $\begin{array}{l}\text { Economy going through } \\ \text { a peak. Positive output } \\ \text { gap starts to narrow. }\end{array}$ & $\begin{array}{l}\text { Economy weakens and } \\ \text { may go into a recession. } \\ \text { Negative output gap } \\ \text { opens. }\end{array}$ \\
\hline
$\begin{array}{l}\text { Activity levels - } \\ \text { consumers and } \\ \text { businesses }\end{array}$ & $\begin{array}{l}\text { Activity levels are below } \\ \text { potential but start to } \\ \text { increase. }\end{array}$ & $\begin{array}{l}\text { Activity measures show } \\ \text { above-average growth } \\ \text { rates. }\end{array}$ & $\begin{array}{l}\text { Activity measures are } \\ \text { above average but } \\ \text { decelerating. Moving to } \\ \text { below-average rates of } \\ \text { growth. }\end{array}$ & $\begin{array}{l}\text { Activity measures are } \\ \text { below potential. Growth } \\ \text { is lower than normal. }\end{array}$ \\
\hline
Employment & $\begin{array}{l}\text { Layoffs slow. Businesses } \\ \text { rely on overtime before } \\ \text { moving to hiring. } \\ \text { Unemployment remains } \\ \text { higher than average. }\end{array}$ & $\begin{array}{l}\text { Businesses move from } \\ \text { using overtime and } \\ \text { temporary employees to } \\ \text { hiring. Unemployment } \\ \text { rate stabilizes and starts } \\ \text { falling. }\end{array}$ & $\begin{array}{l}\text { Business continue hir- } \\ \text { ing but at a slower pace. } \\ \text { Unemployment rate } \\ \text { continues to fall but at } \\ \text { decreasing rates. }\end{array}$ & $\begin{array}{l}\text { Businesses first cut } \\ \text { hours, eliminate over- } \\ \text { time, and freeze hiring, } \\ \text { followed by outright } \\ \text { layoffs. Unemployment } \\ \text { rate starts to rise. }\end{array}$ \\
\hline
Inflation & $\begin{array}{l}\text { Inflation remains } \\ \text { moderate. }\end{array}$ & $\begin{array}{l}\text { Inflation picks up } \\ \text { modestly. }\end{array}$ & $\begin{array}{l}\text { Inflation further } \\ \text { accelerates. }\end{array}$ & $\begin{array}{l}\text { Inflation decelerates but } \\ \text { with a lag. }\end{array}$ \\
\hline
\end{tabular}
\end{center}

\section{Leads and Lags in Business and Consumer Decision Making}
The behavior of businesses and households is key to the cycle and frequently incorporates leads and lags relative to what are established as turning points. For example, at the beginning of an expansion phase, companies may want to fully use their existing workforce and wait to hire new employees until they are sure that the economy is indeed growing. However, gradually all economic variables are going to revert toward their normal range of values (e.g., GDP growth will be close to potential, or average, growth).

\section{Market Conditions and Investor Behavior}
Many economic variables and sectors of the economy have distinctive cyclical patterns. Knowledge of these patterns can offer insight into likely cyclical directions overall, or it can be particularly applicable to an investment strategy that requires more specific rather than general cyclical insights for investment success.

\section{Recovery phase}
When asset markets expect the end of a recession and the beginning of an expansion phase, risky assets will be repriced upward. When an expansion is expected, the markets will start incorporating higher profit expectations into the prices of corporate bonds and stocks. Typically, equity markets will hit a trough about three to six months before the economy bottoms and well before the economic indicators turn up. Indeed, as we will see later, the equity market is classified as a leading indicator of the economy.

\section{Expansion phase}
When an economy's expansion is well-established, a later part of an expansion, referred to as a "boom," often follows. The boom is an expansionary phase characterized by economic growth "testing the limits" of the economy, strong confidence, profit, and credit growth. For example, companies may expand so much that they have difficulty finding qualified workers and will compete with other prospective employers by raising wages and continuing to expand capacity, relying on strong cash flows and borrowing. The government and/or central bank may step in if it is concerned about the economy overheating.

\section{Slowdown phase}
During the boom, the riskiest assets will often have substantial price increases. Safe assets, such as government bonds that were more highly prized during recession, may have lower prices and thus higher yields. In addition, investors may fear higher inflation, which also contributes to higher nominal yields.

\section{Contraction phase}
During contraction, investors place relatively high values on such safer assets as government securities and shares of companies with steady (or growing) positive cash flows, such as utilities and producers of staple goods. Such preferences reflect the fact that the marginal utility of a safe income stream increases in periods when employment is insecure or declining.

\section{EXAMPLE 1}
\section{When Do Recessions Begin and End?}
A simple and commonly referred to rule is the following: A recession has started when a country or region experiences two consecutive quarters of negative real GDP growth. Real GDP growth is a measure of the "real" or "inflation-adjusted" growth of the overall economy. This rule can be misleading because it does not indicate a recession if real GDP growth is negative in one quarter, slightly positive the next quarter, and again negative in the next quarter. Many analysts question this result. This issue is why some countries have statistical and economic committees that apply the principles stated by Burns and Mitchell to several macroeconomic variables-not just real GDP growth-as a basis to identify business cycle peaks and troughs. The National Bureau of Economic Research (NBER) is an organization that dates business cycles in the United States. Interestingly, the economists and statisticians on NBER's Business Cycle Dating Committee analyze numerous time series of data focusing on employment, industrial production, and sales. Because the data are available with a delay (preliminary data releases can be revised even several years after the period they refer to), it also means that the Committee's determinations may take place well after the business cycle turning points have occurred. As we will see later in the reading, there are practical indicators that may help economists understand in advance if a cyclical turning point is about to happen.

\begin{enumerate}
  \item Which of the following rules is mostlikely to be used to determine whether the economy is in a recession?
\end{enumerate}

A. The central bank has run out of foreign reserves.

B. Real GDP has two consecutive quarters of negative growth.

C. Economic activity experiences a significant decline in two business sectors.

\section{Solution to 1:}
B is correct. GDP is a measure of economic activity for the whole economy. Changes in foreign reserves or a limited number of sectors may not have a material impact on the whole economy.

\begin{enumerate}
  \setcounter{enumi}{1}
  \item Suppose you are interested in forecasting earnings growth for a company active in a country where no official business cycle dating committee (such as the NBER) exists. The variables you are most likely to consider to identify peaks and troughs of a country's business cycle are:
\end{enumerate}

A. inflation, interest rates, and unemployment.

B. stock market values and money supply.

C. unemployment, GDP growth, industrial production, and inflation.

\section{Solution to 2:}
$\mathrm{C}$ is correct. Unemployment, GDP growth, industrial production, and inflation are measures of economic activity. The discount rate, the monetary base, and stock market indexes are not direct measures of economic activities. The first two are determined by monetary policy, which react to economic activities, whereas the stock market indexes tend to be forward looking or leading indicators of the economy.

\section{CREDIT CYCLES AND THEIR RELATIONSHIP TO BUSINESS CYCLES}
$$
\square \quad \text { describe credit cycles }
$$

Whereas business cycles mostly use GDP as a measure of economic activity, a literature has emerged in which cyclical developments of financial variables are analyzed separately. This is most commonly done in terms of credit and property prices. Credit cycles describe the changing availability-and pricing-of credit. They describe growth in private sector credit (availability and usage of loans), which is essential for business investments and household purchases of real estate. They are therefore connected to real economic activity captured by business cycles that describe fluctuations in real GDP.

When the economy is strong or improving, the willingness of lenders to extend credit, and on favorable terms, is high. Conversely, when the economy is weak or weakening, lenders pull back, or "tighten" credit, by making it less available and more expensive. This frequently contributes to the decline of such asset values as real estate, causing further economic weakness and higher defaults. This is because of the importance of credit in the financing of construction and the purchase of property. Credit cycles are a subset of a wider family of so-called financial cycles, a topic that goes beyond the scope of our coverage.

\section{Applications of Credit Cycles}
Financial factors were for a long time not prominent on the radar screens of macroeconomists. Monetary and financial phenomena were largely seen as a veil that could be ignored when trying to understand the economy. But loose private sector credit is considered to have contributed to a number of financial crises, such as the Latam crisis of the 1980s; the Mexican, Brazilian, and Russian crises of the 1990s; the Asian crisis of 1997-1998; and the global financial crisis of 2008-2009. Loose credit conditions often lead to asset price and real estate bubbles that burst when capital market outflows and drawdowns occur mostly due to weaker fundamentals.

It is recognized that in a world with financial frictions, business cycles can be amplified, with deeper recessions and more extensive expansions because of changes in access to external financing. In line with this, it is found that the duration and magnitude of recessions and recoveries are often shaped by linkages between business and credit cycles (Claessens, Kose, and, Terrones 2012). In particular, recessions accompanied by financial disruption episodes (notably, house and equity price busts), tend to be longer and deeper. Recoveries combined with rapid growth in credit, risk-taking, and house prices tend to be stronger.

However, whereas financial variables tend to co-vary closely with each other and can often help explain the size of an economic expansion or contraction, they are not always synchronized with the traditional business cycle. Credit cycles tend to be longer, deeper, and sharper than business cycles. Although the length of a business cycle varies from peak to trough, the average length of a credit cycle is mostly found to be longer than that of the business cycle. Exhibit 6 illustrates how credit cycles can be visualized.

\section{VISUALIZING FINANCIAL CYCLES}
In an October 2019 Working Paper titled "Predicting Recessions: Financial Cycle versus Term Spread," the Bank for International Settlements (BIS) provided a visual presentation of credit cycles, which is reproduced in Exhibit 6. It shows that such cycles tend to boom before recessions.

\section{Exhibit 6}
Financial cycles tend to boom ahead of recessions ${ }^{1}$
\includegraphics[max width=\textwidth, center]{2023_05_04_ccb578c5089fd41615beg-225}

The shaded areas represent recessions

1 Financial cycles are measured by the composite financial cycle proxy calculated from frequency-based (bandpass) filters capturing mediumterm cycles in real credit, the credit-to-GDP ratio and real house prices.

Source: BIS Material (available on the BIS website free of charge: \href{http://www.bis.org}{www.bis.org}).

Notes: Credit cycles are measured by a (composite) proxy calculated from variables that include credit-to-GDP ratio and real house prices. The axis on the right shows the year-on-year change in the proxy.

\section{Consequences for Policy}
Investors pay attention to the stage in the credit cycle because (1) it helps them understand developments in the housing and construction markets; (2) it helps them assess the extent of business cycle expansions as well as contractions, particularly the severity of a recession if it coincides with the contraction phase of the credit cycle; and (3) it helps them better anticipate policy makers' actions. Whereas monetary and fiscal policy traditionally concentrate on reducing the volatility of business cycles, macro-prudential stabilization policies that aim to dampen financial booms have gained importance. This is further stressed by findings that strong peaks in credit cycles are closely associated with subsequent systemic banking crises (Schüler, Hiebert, and Peltonen 2017).

\section{EXAMPLE 2}
\begin{enumerate}
  \item A senior portfolio manager at Carnara Asset Management explains her analysis of business cycles to a junior analyst. She makes two statements:
\end{enumerate}

\begin{itemize}
  \item Statement 1: Business cycles measure activity by GDP, whereas credit cycles combine a range of financial variables, such as the amount of and pricing of credit.

  \item Statement 2: Credit cycles and business cycles are unrelated and serve different purposes.

\end{itemize}

A. Only Statement 1 is true.

B. Only Statement 2 is true C. Both statements are true

\section{Solution:}
A is correct. Only Statement 1 is true. Statement 2 is not true because researchers have found linkages between financial and business cycles that help explain the magnitude of business cycle expansions and contractions depending on the state of the credit cycle.

\section{BUSINESS CYCLE FLUCTUATIONS FROM A FIRM'S PERSPECTIVE}
describe how resource use, consumer and business activity, housing sector activity, and external trade sector activity vary as an economy moves through the business cycle

This section provides a broad overview of how the use of resources needed to produce goods and services typically evolves during a business cycle. We start by focusing on circumstances of firms and explore some of the links between fluctuations in inventory, employment, and investment in physical capital with economic fluctuations.

\section{The Workforce and Company Costs}
The pattern of hiring and employment is shown in Exhibit 7. When the economy enters contraction, companies reduce costs and eliminate overtime. They may try to retain workers rather than reducing employment only to increase it later. Finding and training new workers is costly, and it may be more cost efficient to keep workers on the payroll even if they are not fully utilized. Companies may also benefit from an implicit bond of loyalty between a company and its workers, boosting productivity in the process. In prolonged contractions, companies will start reducing costs more aggressively-terminating consultants, advertising campaigns, and workers beyond the strict minimum. Capacity utilization will be low, and few companies will invest in new equipment. Companies will try to liquidate their inventories of unsold products. In addition, banks will be reluctant to lend because bankruptcy risks are perceived to be higher, adding to the weakness in the economy.

Decreases in aggregate demand are likely to depress wages or wage growth as well as prices of inputs and capital goods. After a while, all of these input prices will be relatively low. In addition, interest rates may be cut to try to revive the economy.

As prices and interest rates decrease, consumers and companies may begin to purchase more and aggregate demand may begin to rise. This stage is the turning point of the business cycle: Aggregate demand starts to increase and economic activity increases. Exhibit 7: Business Cycle Phases-Levels of Employment

\begin{center}
\includegraphics[max width=\textwidth]{2023_05_04_ccb578c5089fd41615beg-227}
\end{center}

\begin{center}
\begin{tabular}{|c|c|c|c|c|}
\hline
Phase & Recovery & Expansion & Slowdown & Contraction \\
\hline
$\begin{array}{l}\text { Description of activ- } \\ \text { ity levels }\end{array}$ & $\begin{array}{l}\text { Economy starts at } \\ \text { trough and output } \\ \text { below potential. Activity } \\ \text { picks up, and gap starts } \\ \text { to close. }\end{array}$ & $\begin{array}{l}\text { Economy enjoying an } \\ \text { upswing, with activ- } \\ \text { ity measures showing } \\ \text { above-average growth } \\ \text { rates. }\end{array}$ & $\begin{array}{l}\text { Economy at peak. } \\ \text { Activity above average } \\ \text { but decelerating. The } \\ \text { economy may expe- } \\ \text { rience shortages of } \\ \text { factors of production } \\ \text { as demand may exceed } \\ \text { supply. }\end{array}$ & $\begin{array}{l}\text { Economy goes into a } \\ \text { contraction, (recession, if } \\ \text { severe). Activity mea- } \\ \text { sures are below potential. } \\ \text { Growth is lower than } \\ \text { normal. }\end{array}$ \\
\hline
\multirow[t]{2}{*}{Employment} & $\begin{array}{l}\text { Layoffs slow. Businesses } \\ \text { rely on overtime before } \\ \text { moving to hiring. } \\ \text { Unemployment remains } \\ \text { higher than average. }\end{array}$ & $\begin{array}{l}\text { Businesses move from } \\ \text { using overtime and } \\ \text { temporary employees to } \\ \text { hiring. Unemployment } \\ \text { rate stabilizes and starts } \\ \text { falling. }\end{array}$ & $\begin{array}{l}\text { Businesses continue } \\ \text { hiring but at a slower } \\ \text { pace. Unemployment } \\ \text { rate continues to fall } \\ \text { but at slowly decreasing } \\ \text { rates. }\end{array}$ & $\begin{array}{l}\text { Businesses first cut } \\ \text { hours, eliminate over- } \\ \text { time, and freeze hiring, } \\ \text { followed by outright } \\ \text { layoffs. Unemployment } \\ \text { rate starts to rise. }\end{array}$ \\
\hline
 & \multicolumn{4}{|c|}{Levels of employment lag the cycle} \\
\hline
\end{tabular}
\end{center}

\section{Fluctuations in Capital Spending}
Capital spending -spending on tangible goods, such as property, plant, and equipmenttypically fluctuates with the business cycle. Because business profits and cash flows are sensitive to changes in economic activity, capital spending is also highly sensitive to changes in economic activity. In fact, investment is one of the most procyclical and volatile components of GDP. Companies' spending decisions are driven by business conditions, expectations, and levels of capacity utilization, all of which fluctuate over the cycle. With regard to efficiency, firms will run "lean production" to generate maximum output with the fewest number of workers at the end of contractions. Exhibit 8 provides a description of capital spending over the cycle. Note that new orders statistics include orders that will be delivered over several years. For example, it is common for airlines to order 40 airplanes to be delivered over five years. Where relevant, analysts use "core" orders that exclude defense and aircrafts for a better understanding of the economy's trend.

\section{Exhibit 8: Capital Spending during the Economic Cycle}
\begin{center}
\begin{tabular}{|c|c|c|c|c|}
\hline
Phase of the Cycle & Recovery & Expansion & Slowdown & Contraction \\
\hline
$\begin{array}{l}\text { Business conditions } \\ \text { and expectations }\end{array}$ & $\begin{array}{l}\text { Excess capacity during } \\ \text { trough, low utilization, } \\ \text { little need for capacity } \\ \text { expansion. } \\ \text { Interest rates tend to } \\ \text { be low-supporting } \\ \text { investment. }\end{array}$ & $\begin{array}{l}\text { Companies enjoy favor- } \\ \text { able conditions. } \\ \text { Capacity utilization } \\ \text { increases from low levels. } \\ \text { Over time, productive } \\ \text { capacity may begin to } \\ \text { limit ability to respond to } \\ \text { demand. } \\ \text { Growth in earnings and } \\ \text { cash flow gives businesses } \\ \text { the financial ability to } \\ \text { increase investment } \\ \text { spending. }\end{array}$ & $\begin{array}{l}\text { Business conditions at } \\ \text { peak, with healthy cash } \\ \text { flow. } \\ \text { Interest rates tend to be } \\ \text { higher-aimed at reduc- } \\ \text { ing overheating and } \\ \text { encouraging investment } \\ \text { slowdown. }\end{array}$ & $\begin{array}{l}\text { Companies experience } \\ \text { fall in demand, profits } \\ \text { and cash flows. }\end{array}$ \\
\hline
Capital spending & $\begin{array}{l}\text { Low but increasing } \\ \text { as companies start to } \\ \text { enjoy better condi- } \\ \text { tions. Capex focus on } \\ \text { efficiency rather than } \\ \text { capacity. } \\ \text { Upturn most pro- } \\ \text { nounced in orders for } \\ \text { light producer equip- } \\ \text { ment. } \\ \text { Typically, the orders } \\ \text { initially reinstated are } \\ \text { for equipment with a } \\ \text { high rate of obsoles- } \\ \text { cence, such as software, } \\ \text { systems, and techno- } \\ \text { logical hardware. }\end{array}$ & $\begin{array}{l}\text { Customer orders and } \\ \text { capacity utilization } \\ \text { increase. Companies } \\ \text { start to focus on capacity } \\ \text { expansion. } \\ \text { The composition of the } \\ \text { economy's capacity may } \\ \text { not be optimal for the } \\ \text { current structure of } \\ \text { demand, necessitating } \\ \text { spending on new types of } \\ \text { equipment. } \\ \text { Orders precede actual } \\ \text { shipments, so orders for } \\ \text { capital equipment are a } \\ \text { widely watched indicator } \\ \text { of the future direction of } \\ \text { capital spending. }\end{array}$ & $\begin{array}{l}\text { New orders intended to } \\ \text { increase capacity may be } \\ \text { an early indicator of the } \\ \text { late stage of the expan- } \\ \text { sion phase. } \\ \text { Companies continue } \\ \text { to place new orders as } \\ \text { they operate at or near } \\ \text { capacity. }\end{array}$ & $\begin{array}{l}\text { New orders halted, } \\ \text { and some existing } \\ \text { orders canceled (no } \\ \text { need to expand). } \\ \text { Initial cutbacks may } \\ \text { be sharp and exag- } \\ \text { gerate the economy's } \\ \text { downturn. As the } \\ \text { general cyclical bust } \\ \text { matures, cutbacks in } \\ \text { spending on heavy } \\ \text { equipment further } \\ \text { intensify the contrac- } \\ \text { tion. Maintenance } \\ \text { scaled back. }\end{array}$ \\
\hline
Examples & $\begin{array}{l}\text { Software, systems, and } \\ \text { hardware (high rates of } \\ \text { obsolescence) orders } \\ \text { placed or re-instated } \\ \text { first. }\end{array}$ & $\begin{array}{l}\text { Heavy and complex } \\ \text { equipment, warehouses, } \\ \text { and factories. } \\ \text { A company may need } \\ \text { warehouse space in loca- } \\ \text { tions different from where } \\ \text { existing facilities are. }\end{array}$ & $\begin{array}{l}\text { Fiber-optic overinvest- } \\ \text { ment in late } 1990 \text { s that } \\ \text { peaked with the "tech- } \\ \text { nology, media, telecoms } \\ \text { bubble." }\end{array}$ & $\begin{array}{l}\text { Technology and light } \\ \text { equipment with short } \\ \text { lead times get cut first } \\ \text { Cuts in construction } \\ \text { and heavy equipment } \\ \text { follow. }\end{array}$ \\
\hline
\end{tabular}
\end{center}

\section{EXAMPLE 3}
\section{Capital Spending}
\begin{enumerate}
  \item Levels of capacity utilization are one of the factors that determine companies' aggregate need for additional capital expenditure. Which of the following is another factor that impacts the capital expenditure decision?
\end{enumerate}

A. The rate of unemployment.

B. The composition of the economy's capacity in relation to how it can satisfy demand.

C. The ability to re-instate orders canceled during the contraction stage.

\section{Solution to 1:}
$\mathrm{B}$ is correct. The composition of the current productive capacity may not be optimal of the current structure of demand. $\mathrm{C}$ is incorrect because the ability to re-instate canceled orders is a matter that is relevant once the decision to increase capital expenditure is made.

\begin{enumerate}
  \setcounter{enumi}{1}
  \item Orders for technology and light equipment decline before construction orders in a contraction because:
\end{enumerate}

A. businesses are uncertain about cyclical directions.

B. they are easier to cancel than large construction contracts.

C. businesses value light equipment less than structures and heavy machinery.

\section{Solution to 2:}
B is correct. Because it usually takes much longer time to plan and complete large construction projects than for equipment orders, construction projects may be less influenced by business cycles.

\section{Fluctuations in Inventory Levels}
The aggregate size of inventories is small relative to the size of the economy, but their accumulation and cutbacks by businesses can occur with substantial speed and frequency. Changing inventories reflect differences between the growth (or decline) in sales and the growth (or decline) in production. A key indicator in this area is the inventory-sales ratio that measures the inventories available for sale to the level of sales. Analysts pay attention to inventories to gauge the position of the economy in the cycle. Exhibit 9 shows how production, sales, and inventories typically move through the phases of a cycle.

Exhibit 9: Inventories throughout the Cycle

\begin{center}
\begin{tabular}{lllll}
\hline
Phase of the Cycle & Recovery & Expansion & Slowdown & Contraction \\
\hline
Sales and production & Sales decline slows. Sales & Sales increase. & Sales slow faster than & Businesses produce at \\
 & subsequently recover. & Production rises fast & production; invento- & rates below the sales \\
 & Production upturn & to keep up with sales & ries increase. & volumes necessary to \\
 & follows but lags behind & growth and to replen- & Economic slowdown & dispose of unwanted \\
 & sales growth. & ish inventories of & leads to production & inventories. \\
 & Over time, production & finished products. This & cutbacks and order &  \\
 & approaches normal lev- & increases the demand & cancellations. &  \\
 & els as excess inventories & for intermediate &  &  \\
 & from the downturn are & products. “Inventory &  &  \\
 & cleared. & rebuilding or restock- &  &  \\
 &  & ing stage." &  &  \\
 &  &  &  &  \\
\end{tabular}
\end{center}

\begin{center}
\begin{tabular}{lllll}
\hline
Phase of the Cycle & Recovery & Expansion & Slowdown & \multicolumn{1}{l}{Contraction} \\
\hline
Inventory-sales ratio & Begins to fall as sales & Ratio stable. & Ratio increases. & Regins to fall back \\
 & $\begin{array}{l}\text { recovery outpaces } \\ \text { production. }\end{array}$ &  & economy. &  \\
\hline
\end{tabular}
\end{center}

\section{EXAMPLE 4}
\section{Inventory Fluctuation}
\begin{enumerate}
  \item Although a small part of the overall economy, changes in inventories can influence economic growth measures significantly because they:
A. reflect general business sentiment.
B. tend to move forcefully up or down.
C. determine the availability of goods for sale.
\end{enumerate}

\section{Solution to 1:}
$\mathrm{B}$ is correct. Inventory levels can fluctuate dramatically over the business cycle.

\begin{enumerate}
  \setcounter{enumi}{1}
  \item Inventories tend to rise when:
A. inventory-sales ratios are low.
B. inventory-sales ratios are high.
C. economic activity begins to rebound.
\end{enumerate}

\section{Solution to 2:}
A is correct. When the economy starts to recover, sales of inventories can outpace production, which results in low inventory-sales ratios. Companies then need to accumulate more inventories to restore the ratio to normal level. $C$ is incorrect because in the early stages of a recovery, inventories are likely to fall as sales increase faster than production.

\begin{enumerate}
  \setcounter{enumi}{2}
  \item Inventories will often fall early in a recovery because:
\end{enumerate}

A. businesses need profit.

B. sales outstrip production.

C. businesses ramp up production because of increased economic activity.

\section{Solution to 3:}
$\mathrm{B}$ is correct. The companies are slow to increase production in the early recovery phase because they first want to confirm the recession is over. Increasing output also takes time after the downsizing during the recession.

\section{CONSUMER BEHAVIOR}
Households represent the largest single sector of most economies. As a result, patterns of household consumption determine overall economic direction more than any other sector. For this reason, the patterns of household consumption are important to a range of practitioners, not just to analysts focusing on companies serving consumers.

\section{Consumer Confidence}
Consumer confidence plays a significant role in spending decisions and reflects expectations of future incomes and employment prospects. Beyond direct observations of consumer spending and its mix, practitioners can also gauge future directions of spending by analyzing measures of consumer confidence or sentiment to ascertain how aggressive consumers may be in their spending. Usually, such information is in the form of surveys intended to provide practitioners with a general guide to trends.

\section{Measures of Consumption}
Two primary measures of household consumption are retail sales (physical as well as online) and broad-based indicators of consumer spending that also include purchases outside purely retail establishments, such as utilities and household services. Some additional measures make finer distinctions, such as tracking spending, both real and nominal, of a specific group of consumer products. Exhibit 10 shows how consumer spending changes through the economic cycle, and it splits consumer spending into three parts: (1) durable goods, (2) non-durable goods, and (3) services. Spending on durables is the most cyclical part, while spending on non-durables is the least cyclical. Spending on services, which include both more and less cyclical sub-components, in aggregate fits between the durable and non-durable goods categories. It is also worth noting that consumer spending, including that on durables, while cyclical, is less cyclical than investment spending by firms.

\section{Exhibit 10: Consumer Behavior}
\begin{center}
\includegraphics[max width=\textwidth]{2023_05_04_ccb578c5089fd41615beg-232}
\end{center}

\begin{center}
\begin{tabular}{|c|c|c|c|c|}
\hline
Phase of the Cycle & Recovery & Expansion & Slowdown & Contraction \\
\hline
$\begin{array}{l}\text { Incomes, employment, and } \\ \text { confidence }\end{array}$ & $\begin{array}{l}\text { Unemployment } \\ \text { remains above } \\ \text { average. Layoffs } \\ \text { slow. Businesses rely } \\ \text { on overtime before } \\ \text { moving to hiring. } \\ \text { Consumer confidence } \\ \text { starts improving. }\end{array}$ & $\begin{array}{l}\text { Hiring restarts. } \\ \text { Unemployment rate } \\ \text { stabilizes and starts } \\ \text { falling. Consumers } \\ \text { experience rising } \\ \text { incomes, healthy } \\ \text { employment prospects, } \\ \text { and greater confidence. }\end{array}$ & $\begin{array}{l}\text { Businesses continue } \\ \text { hiring but at a slower } \\ \text { pace. Unemployment } \\ \text { rate continues to fall. } \\ \text { Incomes are still grow- } \\ \text { ing. Consumers remain } \\ \text { confident. }\end{array}$ & $\begin{array}{l}\text { Businesses first cut } \\ \text { hours of overtime } \\ \text { prior to freezing hiring } \\ \text { and starting layoffs. } \\ \text { Employment levels } \\ \text { decline, and consumer } \\ \text { confidence weakens. }\end{array}$ \\
\hline
$\begin{array}{l}\text { Spending on consumer } \\ \text { durables }\end{array}$ & $\begin{array}{l}\text { Spending limited as } \\ \text { households postpone }\end{array}$ & Spending increases. & $\begin{array}{l}\text { Spending above } \\ \text { average. }\end{array}$ & $\begin{array}{l}\text { Purchases postponed; } \\ \text { spending decreasing. }\end{array}$ \\
\hline
\end{tabular}
\end{center}

(autos, motorcycles, appli- spending.

ances, furniture)

Consumer non-durables

(i.e., medicines, food,

household products)

Services

(entertainment, outdoor

eating, communications,

Spending below

average.

Spending increases. Spending above

average. Spending declines.

personal services)

Analysts often compare trends in durable purchases with those in the other categories to understand the economy's progress through the cycle. Because durable goods purchases usually replace items with longer useful lives, households usually postpone such purchases during economic downturns. Therefore, a weakness in durables spending may be an early indication of general economic weakness, and an increase in such spending may signal a more general cyclical recovery.

\section{Income Growth}
Growth in income is normally a good indicator of consumption prospects, and, albeit not as timely as consumer surveys, household income figures are widely available in most countries. Especially relevant is after-tax income, known as disposable income. Some analysts chart consumer spending based on a concept termed permanent income. Permanent income excludes temporary income (windfall or sudden, unexpected) and unsustainable losses and gains and tries to capture the income flow on which households believe they can rely. The basic level of consumption reflects this notion of permanent income. However, spending on durables tends to rise and fall with disposable income, regardless of the source, not just permanent income.

\section{Saving Rates}
Consumer spending patterns frequently diverge from trends in income, no matter how income is measured. An analysis of the saving rates can assist practitioners in this regard. Changes in saving rates can capture consumers' intent to reduce spending out of current income. The saving rate may also reflect future income uncertainties perceived by consumers (precautionary savings). Therefore, a higher saving rate may indicate consumers' ability to spend despite possible lower income in the future. A rise in the saving rate, usually measured as a percentage of income, may indicate caution among households and signal economic weakening. At the same time, the greater the stock of savings in the household sector and the wider the gap between ongoing income and spending, the greater the capacity for households to increase their spending. So, although unusually high savings may at first say something negative about the cyclical outlook, they point longer-term to the potential for recovery.

\section{EXAMPLE 5}
\section{Consumer Behavior}
\begin{enumerate}
  \item Durable goods have the most pronounced cyclical behavior because:
\end{enumerate}

A. they have a longer useful life.

B. their purchase cannot be delayed.

C. they are needed more than non-durable goods or services.

\section{Solution to 1:}
A is correct. Durable goods are usually big ticket items, the life span of which can be extended with repairs and without incurring high replacement costs. So, consumers tend to delay replacement when the economic outlook is not favorable.

\begin{enumerate}
  \setcounter{enumi}{1}
  \item Permanent income provides a better guide to:
\end{enumerate}

A. saving rates.

B. spending on services.

C. spending on durable goods.

\section{Solution to 2:}
B is correct. Households adjust consumption of discretionary goods and services based on the perceived permanent income level rather than temporary earning fluctuations. Saving rates and durable goods consumption are more related to the short-term uncertainties caused by recessions.

\section{HOUSING SECTOR BEHAVIOR}
Housing sector activity includes new and existing home sales and residential construction activity. It is a small part of the overall economy compared to consumer spending, but because housing activity experiences dramatic swings, it often counts more in overall economic movements than the sector's relatively small size might suggest. In many countries, such as the United States and the United Kingdom, changing property values are closely related to consumer wealth and confidence.

\section{Available Statistics}
Many economies offer statistics on a range of housing activities, including the inventory of unsold homes on the market and average or median price of homes (sometimes recorded by type of housing unit and sometimes as the price per square foot or square meter). The relationships in this area typically follow fairly regular cyclical patterns.

\section{Sensitivity to Interest Rates and Relationship to Credit Cycle}
Because many home buyers finance their purchase with a loan (called mortgage in some countries), the sector is especially sensitive to interest rates. Home buying and consequently construction activity expand in response to lower loan interest rates and contract in response to higher loan interest rates.

Beyond such interest rate effects, housing also follows its own internal cycle, seen as part of the credit cycle. When housing prices are low relative to average incomes, and especially when mortgage rates are also low, the cost of owning a house falls and demand for housing increases. Often indicators of the cost of owning a house are available to compare household incomes with the cost of supporting an average house, both its price and the expense of a typical mortgage. Commonly, housing prices and mortgage rates rise disproportionately as expansionary cycles mature, bringing on an increase in relative housing costs even as household incomes rise. The resulting slowdown of house sales can lead to a cyclical downturn first in buying and then, as the inventory of unsold houses builds, in actual construction activity.

These links, clear as they are, are far from mechanical. If housing prices have risen rapidly in the recent past, for instance, many people will buy to gain exposure to the expected further price gains even as the purchase in other respects becomes harder to rationalize. Such behavior can extend the cycle upward and may later result in a more severe correction. This result occurs because "late buying" activity invites overbuilding. The large inventory of unsold homes eventually puts downward pressure on real estate prices, catching late buyers, who have stretched their resources. This pattern occurred in many countries during the 2008-2009 global financial crisis.

\section{The Role of Demographics}
Cyclical behavior in housing occurs around the long0run growth trend in housing determined by demographics, such as family and household formation. Where data on family formation are unavailable, analysts can gauge the rate of family formation by assessing the age structure of the population, as household formation commonly occurs in the 25-40-year-old age group. Adjusted for older people who are vacating existing homes, such calculations serve as an indicator of underlying, longer-term, secular housing demand.

\section{Impact on the Economic Cycle}
Although such measures have little to do with business cycles, they do offer a gauge, along with affordability, of how quickly the housing market can correct excess and return to growth. As some faster-growing developing economies experience urbanization (people moving from the countryside to cities), they witness fast growth in demand for housing units. Such housing demand may quickly reverse any cyclical weakness in the economy, more so than in such economies as Italy or Japan where net new family formation is relatively slight.

\section{EXAMPLE 6}
\section{Housing Sector Behavior}
\begin{enumerate}
  \item Housing is more sensitive than other sectors of the economy to:
\end{enumerate}

A. interest rates.

B. permanent income.

C. government spending.

\section{Solution to 1:}
A is correct. Because real estate purchases are usually financed with mortgage loans, interest rate changes directly influence the monthly payment amounts.

\begin{enumerate}
  \setcounter{enumi}{1}
  \item Apart from questions of affordability, house buying is affected by:
\end{enumerate}

A. the rate of family formation.

B. expectation of housing price increases.

C. both the rate of family formation and expectation of housing price increases.

\section{Solution to 2:}
$\mathrm{C}$ is correct. Family formation constitutes the actual need for housing, whereas buying on the expectation of housing price increases reflects the fact that real estate has investment value.

\section{EXTERNAL TRADE SECTOR BEHAVIOR}
The external trade sector varies tremendously in size and importance from one economy to another. In such places as Singapore or smaller open economies, such as Lithuania, where almost all inputs are imported and the bulk of the economy's output finds its way to the export market, trade (the sum of both exports and imports) easily exceeds its GDP. External conditions, especially the stage of the cycle in key export regions and in primary trading partners, play a crucial role. In other places, such as Brazil or the United States, external trade assumes a much smaller part of GDP. Since the 1970s, global trade has risen substantially, so the business cycles of the large economies in the world can be more easily transmitted to other economies.

\section{Cyclical Fluctuations of Imports and Exports}
Typically, imports rise, all else equal, with the pace of domestic GDP growth because rising domestic demand increases purchases of goods and services, by consumers and business, from abroad. Thus, imports respond to the domestic cycle. Exports are more dependent on cycles in the rest of the world. If these external cycles are strong, all else equal, exports will grow even if the domestic economy should experience a decline in growth. To understand the impact of exports, financial analysts need to understand the strength of the major trading partners of the economy under consideration. The net effect of trade may offset cyclical weakness and, depending on the importance of exports to the economy, could erase it altogether. For these reasons, such differences can mean the pattern of external trade balances is entirely different from the rest of the domestic economic cycle. Exhibit 11 provides a short summary.

\section{Exhibit 11: External Trade}
\begin{center}
\begin{tabular}{lllll}
\hline
$\begin{array}{l}\text { Phase of the Cycle } \\ \text { (domestic economy) }\end{array}$ & Recovery & \multicolumn{1}{c}{Expansion} & \multicolumn{1}{c}{Slowdown} & Contraction \\
\hline
Exports &  & Driven by external demand &  &  \\
\hline
$\begin{array}{lllll}\text { Imports ("assuming } \\ \text { exchange rate remains } \\ \text { unchanged) }\end{array}$ & $\begin{array}{l}\text { Imports } \\ \text { below aver- } \\ \text { age, start to } \\ \text { increase. }\end{array}$ & $\begin{array}{l}\text { Imports } \\ \text { increase. }\end{array}$ & $\begin{array}{l}\text { Imports peak } \\ \text { and start to } \\ \text { decline. }\end{array}$ & $\begin{array}{l}\text { Imports in } \\ \text { decline to } \\ \text { below-average } \\ \text { levels. }\end{array}$ \\
\hline
\end{tabular}
\end{center}

\section{The Role of the Exchange Rate}
The currency exchange rate also has an independent effect that can move trade in directions strikingly different from the domestic economic cycle. When a nation's currency appreciates (the currency gains in strength relative to other currencies), foreign goods seem cheaper than domestic goods to the domestic population, prompting, all else equal, a relative rise in imports. At the same time, such currency appreciation makes that nation's exports more expensive in global markets and may reduce exports. Of course, currency depreciation has the opposite effect. Although currency moves may be volatile and on occasion extreme, they have a significant effect on trade and the balance of payments only when they cumulate in a single direction for some time. Moves from one month or quarter to the next, however great, have a minimal effect until they persist. Thus, cumulative currency movements that take place over a period of years will have an impact on trade flows that will persist even if the currency subsequently moves in the opposite direction for a temporary period.

\section{Overall Effect on Exports and Imports}
Financial analysts need to consider a wide range of variables, both in the domestic economy and abroad, to assess relative GDP growth rates and then factor in currency considerations to ascertain whether they reinforce or counteract other cyclical forces. Generally, GDP growth differentials in global economic growth rates between countries have the most immediate and straightforward effects: Domestic changes in economic activity raise or reduce imports, and foreign economic activity changes raise or reduce exports. Currency moves have a more complex and, despite the interim short-term currency moves, a more gradual effect.

\section{EXAMPLE 7}
\section{External Trade}
\begin{enumerate}
  \item Imports most clearly respond to:
A. the level of exports.
B. domestic industrial policy.
C. domestic GDP growth rate.
\end{enumerate}

\section{Solution to 1:}
$\mathrm{C}$ is correct. As a part of aggregate demand, imports reflect the domestic needs for foreign goods, which vary together with domestic economic growth.

\begin{enumerate}
  \setcounter{enumi}{1}
  \item Exports generally respond to the:
A. government expenditure
B. global GDP growth rates.
C. domestic GDP growth rates.
\end{enumerate}

\section{Solution to 2:}
B is correct. Exports reflect the foreign demands on domestic output, which depend on the conditions of the global economy.

\section{EXAMPLE 8}
\section{Investment Implications}
\begin{enumerate}
  \item A junior portfolio manager is reviewing portfolio holdings. He focuses on three specific companies that he was tasked with reviewing. He has just received an analysis of the stage of the business cycle from a team of the firm's economists. To set the context of his analysis, he needs to understand the behavior of the three companies as they have exposures to different parts of the economy. One is focused on consumer goods (electronics and small vehicles) sold on the local market; one is focused on equipment sold to domestic companies; and one makes household and personal care consumer products, which it sells both domestically and abroad.
\end{enumerate}

Which of the these three companies' financial performance fluctuates with the business cycle of their domestic economy the most?
A. Consumer goods company
B. Business equipment company
C. Producer of household and personal care products

\section{Solution:}
$B$ is correct. Business investment fluctuates more than consumer spending and exports. A and $\mathrm{C}$ are incorrect because although consumer goods companies are impacted by the cycle, they are impacted less than companies exposed to business investment.

\section*{THEORETICAL CONSIDERATIONS }
In this section we review the main theories that try to explain causes of business cycles and mechanisms through which the economy returns (or does not return) to the longer-term growth path.

\section{Historical Context}
Modern mainstream macroeconomists use the aggregate demand/aggregate supply (AD/AS) framework to synthesize the different schools of thought that have developed over the decades. In all of the key theories, movements in either of these curves create business cycles over time. Until the 1930s, the classical view has been that the aggregate supply curve is very steep (see Exhibit 12), prices and wages adjust quickly, and therefore recessions are mild and only short-lived. There was no need for government policy to intervene in the economy to try to smooth the business cycle as full employment level of output would be quickly restored. On the contrary, a policy of active demand management might, because of reaction lags that we will explore later, actually trigger business cycles rather than smooth them.

\section{Exhibit 12: Classical Aggregate Supply Curve}
\begin{center}
\includegraphics[max width=\textwidth]{2023_05_04_ccb578c5089fd41615beg-238}
\end{center}

The depth and severity of the Great Depression in the 1930s triggered the development of so-called Keynesian theories, in which fiscal policies play an important role in stabilizing the aggregate demand curve. Because prices and wages are seen as sticky and do not adjust, particularly on the way down (shown in Exhibit 13), the economy may experience a significant output gap without intervention.

Rising inflation in the late 1960s and the subsequent stagflation periods triggered by the oil price shocks in the 1970s were an incubator for another approach: monetarism. This school of thought highlights the importance of monetary policy and advocates stable money growth to keep the position of the aggregate demand curve stable, inflation under control, and reduce business cycle volatility.

Since the mid-1980s, we witnessed the emergence of micro-founded theories and the role of rational expectations that spurred the development of neoclassical (within which are Real Business Cycle, RBC, theorists) and new-Keynesian types of macroeconomic models to explain short-term fluctuations. Despite the "neo" and "new" in front of the classical and Keynesian theories, the fundamental differences remain the same: Classical-oriented economists have more faith in the working of the "invisible hand" (i.e., the proper functioning of markets), while Keynesian-oriented economists see rigidities and market failures as more prevalent and the government in a position to actively intervene.

We will now explain the two broad groups of theories through this "non-intervention" vs "intervention" lens and describe each in terms of a) the role of demand and supply shocks; b) the completeness of markets and thereby the stickiness of prices; c) the importance of asymmetric information problems; and d) the role of policy lags.

\section{Neoclassical Economics}
Neoclassical economists, but in particular the so-called Real Business Cycle (RBC) advocates, stress the importance of movements in the supply curve. Technological progress, natural disasters, changes in relative prices of key inputs, other supply-side constraints, or changes in firm expectations lead to changes in the position of the supply curve. RBC models of the business cycle conclude that expansions and contractions represent efficient operation of the economy in response to external real shocks; the level of economic activity at any time is consistent with maximizing expected utility. For that reason, the policy recommendation of RBC theorists, and neoclassical economists in general, is for governments not to intervene in the economy with discretionary fiscal and monetary policy. This is stressed even further by arguing that monetary and fiscal policy will affect the economy with substantial delay, referred to as "lags": There is a lag between when policy makers realize what is happening with the economy and when they develop a plan and implement it. Finally, it takes time for policy to actually have an impact on the economy. It is possible that any policy action will impact the economy too late, and instead of bringing output closer to the long-term path, it may amplify the next cycle.

In a more mainstream context, reducing economic volatility is translated into stressing the importance of expectation and information management. Removing asymmetric information problems should speed up price adjustments. If all participants in a market are well informed, market forces can work efficiently. Information problems, however, create inefficiencies. If a firm cannot distinguish between an increase in the relative price of the good it is offering and general inflation, it might make inefficient production decisions. If the price increase in the market arises due to an increase in demand, then the firm should increase production. However, if price increases merely reflect general inflation, the firm should not increase production. Removing such information problems reduces economic volatility. Similarly, fostering agents in the economy to form precise and well-anchored expectations about inflation, for example, can lead to self-fulfilling outcomes and stabilize the position of the supply curve. Neoclassical economists would also argue that policies should not prevent Schumpeterian "creative destruction" from taking place. The term "creative destruction" describes the disruptive process of transformation that accompanies innovation, originally described by Joseph Schumpeter in the 1950s. When an inventor presents a new product or a new, better way to produce an existing good or service (e.g., radio frequency identification tracking of inventories), then the entrepreneur that introduces the new discovery will likely have bigger profits and may drive the existing producers out of business. Therefore, innovations can generate movements in the aggregate supply curve that policy makers should not try to correct.

\section{The Austrian School}
The so-called Austrian school, another "non-intervention" theory, shares many views of the neoclassical economists. In contrast to the neoclassical school, however, it focuses on the role of money, which they believe does more than facilitate the exchange of goods and services. Austrian economists argue that low interest rates and excessive credit growth during boom times result in over-investment in projects with low returns, causing failure and a move into recession. They argue that following a period of overinvestment, companies will realize that they have overinvested, resulting in a sudden drop in investment with a negative shock for aggregate demand. Because the Austrian school sees business cycles arising from government (and central bank) policies as the cause of overinvestment and subsequent failure, they suggest that policy makers should rarely intervene in the economy.

These classical views stress that markets are in general flexible and prices will easily adjust to changing situations. This makes the aggregate supply curve very steep, even in the short run (see Exhibit 12). Policy should be designed to allow markets to function and prices to adjust with flexibility.

\section{Monetarism}
For monetarists, other than maintaining steady growth of the money supply, government should play only a very limited role in the economy. They reject active management of aggregate demand. Fiscal and monetary policy should be clear and consistent over time so all economic agents can forecast government actions. In this way, the uncertainty of economic fluctuations would not be increased by any uncertainty about the timing and magnitude of economic policies and their lagged effects.

According to the monetarist school, business cycles may occur both because of exogenous shocks and because of government intervention. So, it is better to let aggregate demand and supply find their own equilibrium than to risk causing further economic fluctuations. In contrast to neoclassical economists, a key part of monetarist thought is that the money supply needs to continue to grow at a moderate rate. If it falls, as occurred in the 1930s, the economic downturn could be severe; whereas if money grows too fast, inflation will follow.

Monetarists also see business cycles largely stemming from movements in the aggregate demand curve. But their conclusions differ from Keynesian interventionism, which they reject, for three main reasons:

\begin{enumerate}
  \item Unlike the monetarist model, the Keynesian model does not recognize the supreme importance of the money supply. If the money supply grows too fast, there will be an unsustainable boom, and if it grows too slowly, there will be a recession.

  \item Monetarists consider the long-term costs of government intervention (e.g., growing government debt and a high cost of interest on this debt). 3. The timing of a government's economic policy responses is uncertain, and the stimulative effects of a fiscal expansion may take effect after the crisis was over, thus causing more harm than good.

\end{enumerate}

\section{Keynesianism}
Keynesians see an active role for the government in managing aggregate demand. Keynesian-oriented theories give limited importance to the supply curve. Instead of thinking about it as being rather steep, these theories see prices being sticky in the short run, leading to a relatively flat AS curve (see the horizontal section of the AS curve in Exhibit 13). It is argued that nominal wages, as one of the most important prices in the economy, are generally rigid and not downward flexible; workers do not want to see their nominal compensation decrease. Movements in the AD curve, therefore, have large consequences on the overall output of the economy.

But even if workers agreed to accept lower salaries in times of recession, this situation might exacerbate the crisis. Lower wage expectations would hurt confidence, and workers would cut back on their spending-reducing aggregate demand, causing further weakening of confidence, and leading to contractions in demand for goods and services. These contractions would start with the more expensive items, such as durable goods, and move in a "domino effect" through the economy (the downward spiral of the aggregate demand curve continuously shifting left). Exhibit 13 shows how a decrease in aggregate demand (from $\mathrm{AD}_{0}$ to $\mathrm{AD}_{1}$ ) may result in an output gap and an equilibrium in recession (ER) from which it may take a long time to recover and restore full employment levels of output (at equilibrium $\mathrm{E}_{0}$ ).

\section{Exhibit 13: Keynesian AS/AD Framework}
\begin{center}
\includegraphics[max width=\textwidth]{2023_05_04_ccb578c5089fd41615beg-242}
\end{center}

Furthermore, there could be circumstances in which lower interest rates would not reignite growth because business confidence or "animal spirit" (consumer's lack of confidence) was too low. Therefore, Keynesian economists advocate government intervention in the form of fiscal policy. While Keynes accepted the possibility that markets would reach the equilibrium envisioned by neoclassical economists over the long run, he famously quipped that "in the long run, we are all dead"-that is, the human suffering is excessive while waiting for all shocks to be absorbed and for the economy to return to equilibrium.

In Keynesians' opinion, when crises occur, the government should intervene to keep capital and labor employed by deliberately running a larger fiscal deficit. This intervention would limit the damage caused by major recessions. Although this concept continues to be a highly politically charged debate, many economists agree that government expenditure can limit the negative effect of major economic crises in the short term. The practical criticisms that are often expressed about Keynesian fiscal policy are:

\begin{enumerate}
  \item Fiscal deficits mean higher government debt that needs to be serviced and repaid eventually.

  \item Keynesian cyclical policies are focused on the short term. In the long run, the economy may come back and the presence of the expansionary policy may cause it to "overheat" - that is, to have unsustainably fast economic growth, which causes inflation and other problems. This result is because of the typical lags involved in expansionary policy taking effect on the economy. 3. Fiscal policy takes time to implement. Quite often, by the time stimulatory fiscal policy kicks in, the economy is already recovering.

\end{enumerate}

\section{THE PERSPECTIVE OF HYMAN MINSKY}
A different view of business cycles came from Hyman Minsky. Minsky believed that excesses in financial markets exacerbate economic fluctuations. For example, a rapid growth of credit, often given to risky ventures in the late expansion phase of the cycle, will be followed by a "credit crunch" during the down-swing phase. Minsky traced excesses to a type of complacency in which people underestimate the risk of events that have not occurred in a while. Therefore, if the economy has been in a long expansion, people may think that the market works very well and that the expansion will last forever-that is, extrapolating past experiences. In this sense, Minsky's theories could be seen as a precursor of behavioral finance, which is the branch of finance that studies how cognitive and emotional biases, such as overconfidence and short memory, induce investors to be overconfident and make suboptimal choices.

The term Minsky moment has been coined for a point in the business cycle when, after individuals become overextended in borrowing to finance speculative investments, people start realizing that something is likely to go wrong and a panic ensues leading to asset sell-offs. The subprime crisis that affected many industrialized countries starting in 2008 has been represented as a "Minsky moment" (see McCulley 2009) because it came after years in which risk premiums (e.g., the differentials, or spreads, between very risky bonds and very safe bonds) were at historically low levels. Typically, low risk premiums suggest that no adverse events are expected; in other words, investors believe that because the economy and the markets have been enjoying a protracted expansion, there is no reason to worry about the future. As a consequence, many market observers suggest that business cycles are being tamed. This kind of view of the world leads people to underestimate risk-for example, by not doing the appropriate diligent research before granting a loan or before purchasing a security (i.e., complacency).

The "Minsky moment" has been compared with a cartoon in which a cartoon character walks over a cliff without realizing it. When he looks down and sees that he is walking on thin air, he panics and falls to the bottom of the canyon-just like the world economy in 2008.

\section{Modern Approach to Business Cycles}
Today's mainstream economics can be described as a synthesis of three macroeconomic theories: neoclassical economics (including RBC theories), Keynesian economics, and monetarism. As the different models do not necessarily rule each other out, mainstream economists in practice use an analytical framework that encompasses all three broad theories.

The debate about business cycles often receives a politically partisan treatment in the press because some experts and commentators are generally against government intervention in the economy (for example, because it may lead to large fiscal deficits) and others are in favor (for example, because it may alleviate the effects of a large economic shock). Financial analysts should base investment decisions on analysis and try to put personal biases aside.

However, there is little doubt that central banks try to manage the business cycle by moving the aggregate demand curve through raising interest rates when the economy is growing rapidly and inflation accelerates and cutting rates when the economy is weak. In the 2008-2009 downturn, when official interest rates first approached zero, central bankers extended the range of their actions and tools to influence aggregate demand and stimulate the economy through the use of unconventional policies that will be covered later.

\section{EXAMPLE 9}
\section{Theories of Business Cycle}
\begin{enumerate}
  \item The main difference between neoclassical (RBC) and Keynesian models is that the neoclassical models:
\end{enumerate}

A. focus on the role of money in the economy.

B. focus on the aggregate supply curve in explaining the causes of economic cycles.

C. assume that prices adjust slowly, whereas Keynesians assume that prices adjust quickly.

\section{Solution to 1:}
$B$ is correct. Neoclassical (including RBC) economists suggest that business cycles are caused by factors that shift the aggregate supply curve. A is incorrect as monetarists focus on the role of money and the growth in money supply in the economy. $\mathrm{C}$ is incorrect because Keynesians believe that prices adjust slowly.

\begin{enumerate}
  \setcounter{enumi}{1}
  \item Which of the following theories would advocate government intervention to stimulate aggregate demand during economic contractions?
\end{enumerate}

A. The Austrian school

B. Keynesians

C. Real Business Cycle theorists

Solution to 2:

B is correct. Keynesians would advocate government intervention to stimulate aggregate demand. A and C are incorrect as advocates of the Austrian school and the RBC theorists would discourage government intervention.

\section{ECONOMIC INDICATORS}
interpret a set of economic indicators, and describe their uses and limitations

Economic indicators are variables that provide information on the state of the overall economy. They are statistics or data readings that reflect economic circumstances of a country, group of countries, region, or sector. Economic indicators are used by policy makers and analysts to understand and assess the existing condition of the economy and its position in the cycle. They can also be used to help predict or confirm the turning points in the cycle. Such knowledge allows analysts to better predict the financial and market performance of stocks and bonds of issuers operating in different sectors of the economy with different sensitivity to the economic cycle.

\section{Types of Indicators}
Economic indicators are often classified according to whether they lag, lead, or coincide with changes in an economy's growth.

\begin{itemize}
  \item Leading economic indicators have turning points that usually precede those of the overall economy. They are believed to have value for predicting the economy's future state, usually near-term.

  \item Coincident economic indicators have turning points that are usually close to those of the overall economy. They are believed to have value for identifying the economy's present state.

  \item Lagging economic indicators have turning points that take place later than those of the overall economy. They are believed to have value in identifying the economy's past condition and only change after a trend has been established.

\end{itemize}

Exhibit 14 provides an illustration of several leading, lagging, and coincident indicators. The leading indicators observed at a point in time labeled as time " 1 " indicate the direction of the of the activity (output) at a future point in time, such as time "2." The lagging indicators, released around time "3," refer to and help confirm what the state of the economy was at time "2."

\section{Exhibit 14: Types of Economic Indicators}
\begin{center}
\includegraphics[max width=\textwidth]{2023_05_04_ccb578c5089fd41615beg-245}
\end{center}

" Leading indicators will be explored in the subsequent section.

\section{Composite Indicators}
An economic indicator either consists of a single variable, like industrial production or the total value of outstanding building permits, or can be a composite of different variables that all tend to move together. The latter are regularly labelled composite indicators. Traditionally, most composite indicators to measure the cyclical state of the economy consist of up to a dozen handpicked variables published by organizations like the OECD or national research institutes. The exact variables combined into these composites vary from one economy to the other. In each case, however, they bring together various economic and financial measures that have displayed a consistently leading, coincident, or lagging relationship to that economy's general cycle.

\section{Leading Indicators}
In the United States, for instance, The Conference Board, a US industry research organization, publishes a composite leading indicator known as the Conference Board Leading Economic Index (LEI) that consists of 10 component parts (it uses the classical business cycle as the underlying concept). Exhibit 15 presents the 10 components used in the LEI. In addition to naming the indicators, it also offers a general description of why each measure is leading the business cycle.

\section{Exhibit 15: Index of Leading Economic Indicators-United States}
Leading indicators

Average weekly hours, manufacturing

Average weekly initial claims for unemployment insurance

Manufacturers' new orders for consumer goods and materials

\section{Reason for use}
Businesses will cut overtime before laying off workers in a downturn and increase it before rehiring in a cyclical upturn. Moves up and down before the general economy.

A very sensitive test of initial layoffs and rehiring.

Businesses cannot wait too long to meet demand without ordering. Orders tend to lead at upturns and downturns \& captures business sentiment.

The Institute of Supply Management (ISM)

polls its members to build indexes of manufacturing orders, output, employment, pricing, and comparable gauges for services.

\begin{center}
\includegraphics[max width=\textwidth]{2023_05_04_ccb578c5089fd41615beg-247}
\end{center}

Average consumer expectations for Survey based business conditions

Reflects the month on month change in new orders for final sales. Decline of new orders can signal weak demand and can lead to recession.

Captures business expectations and offers first signal of movement up or down. Important sector.

Signals new construction activity as permits required before new building can begin.

Stocks tends to anticipate economic turning points; useful early signal.

A vulnerable financial system can amplify the effects of negative shocks, causing widespread recessions.

LT (10 or 30 year) bond yields express market expectations about the direction of short-term interest rates. As rates ultimately follow the economic cycle up and down, a wider spread, by anticipating short rate increases, also anticipates an economic upswing and vice versa.

Optimism tends to increase spending. Provides early insight into the direction ahead for the whole economy.

\section{Using Economic Indicators}
Exhibit 16 shows a simplified process that an analyst could use to identify business cycle phases. The conclusions can then be used to make investment decisions-for example, to decide in what sectors companies are likely to see improving or deteriorating cash flows, which could impact the investment performance of the equity and debt securities issued by the companies. Note that the order of the steps shown does not have to follow this particular sequence.

\section{Exhibit 16: Use of Statistics to Identify Business Cycle Phase}
\section{Step 1}
Data release: Analyst notes an increase in the reported level of consumer instalment debt to income.

Analysis: The above indicator normally lags cyclical upturns.

Possible conclusion: Initial evidence that an upturn has been underway.

\section{Step 2}
Data release: Industrial Production Index and non-farm payrolls (employees on non-agricultural payrolls) are moving higher.

Analysis: These are coincident indicators that suggest activity is picking up.

Possible conclusion: Further evidence that expansion is underway.

\section{Step 3}
Observation: Equity market index has been trending higher. Equity index is a leading indicator. Analyst checks the aggregate LEI Index.

Analysis: If the aggregate LEI is moving higher too, evidence that recovery is underway. -> Confirmation that output is moving higher.

Or

Aggregate LEI is not moving higher. Analyst cannot draw conclusions about recovery.

\section{Other Composite Leading Indicators}
For about 30 countries and several aggregates, such as the EU and G-7, the Organisation for Economic Co-Operation and Development (OECD) calculates OECD Composite Leading Indicator (CLI), which gauge the state of the business cycle in the economy using the growth cycle concept. One of the interesting features of OECD CLI is that the underlying methodology is consistent across several countries. It can therefore be compared more easily to see how each region is faring. Exhibit 17 shows the eight components of CLI used by the OECD. As is usually the case with leading indicators, some are based on surveys while others are based on reported market or economic data.

\section{Exhibit 17: OECD Euro Area CLI Components}
\begin{center}
\includegraphics[max width=\textwidth]{2023_05_04_ccb578c5089fd41615beg-249}
\end{center}

The parallels between many of these components and those used in the United States are clear, but the Euro area has a services component in its business activity measures that the United States lacks; the Euro area forgoes many of the overtime and employment gauges that the United States includes. The OECD CLI for Japan is again similar, but it does include labor market indicators (unlike the Eura area) and adds a measure of business failures not included in the other two.

The German ifo survey is a widely used index capturing business climate in Germany and is published monthly. Exhibit 18 shows how the index moved ahead of quarterly-reported YoY changes in German GDP. It also shows an uptick in 2018 despite the GDP growth downturn. These indicators are useful, but they are not foolproof.

\section{Exhibit 18: Business climate survey and GDP}
\begin{center}
\includegraphics[max width=\textwidth]{2023_05_04_ccb578c5089fd41615beg-249(1)}
\end{center}

Sources: The ifo Institute and the Federal Statistical Office of Germany.

\section{Surveys}
The composite indicators for region- or country-specific business cycles often make use of economic tendency surveys. These monthly or quarterly surveys carried out by central banks, research institutes, statistical offices, and trade associations are conducted among either businesses, consumers, or experts. The mostly qualitative questions posed are often on the state of their finances, level of activity, and their confidence in the future.

CIRET (a forum for leading economists and institutions that conducts and analyzes business and/or consumer survey data), the United Nations, the OECD, and the European Commission are at different regional levels harmonizing and exchanging knowledge regarding these business tendency and consumer surveys (United Nations 2015). The Directorate-General for Economic and Financial Affairs (DG ECFIN) publishes fully harmonized results for different sectors of the EU member and applicant economies. The Bank of Japan carries out similar surveys and releases the findings in what is called the "Tankan Report." These diverse sources multiply within and across economies. Over the last decade, so-called "purchasing managers" indexes along similar lines, albeit with often much smaller survey samples, have been introduced in a wide range of countries and regions, including Europe and China.

These economic tendency surveys are often aggregated into or are a part of composite indicators. IHS Markit publishes a global Purchasing Managers' Index (PMI) indicator using results of its national business surveys (see Exhibit 17). A global consumer confidence indicator is published by The Conference Board. Some other institutions collect national survey data and calculate supra-national results. For example, the OECD publishes a business confidence indicator and a consumer confidence indicator for the OECD aggregate in its report "Main Economic Indicators." In addition, the European Commission calculates various survey-based indicators for the EU, reflecting the situation in Europe. Based upon economic tendency surveys and using the growth rate cycle concept, KOF and FGV publish a coincident and a leading indicator for the world economy called the Global Economic Barometers. These incorporate hundreds of survey variables from all over the world.

\section{The Use of Big Data in Economic Indicators}
The vast increase in this information and academic developments regarding the use of big(ger) data have in recent years increased the number of variables that go into these composite indicators. For instance, using a statistical technique called "principal components analysis," the Federal Reserve Bank of Chicago computes the Chicago Fed National Activity Index (CFNAI) using 85 monthly macroeconomic series. These series cover industrial production, personal income, capital utilization, employment by sectors, housing starts, retail sales, and so on. Principal components analysis "extracts" the underlying trend that is common to most of these variables, thus distilling the essence of the US business cycle. Similarly, the Bank of Italy in conjunction with the Centre for Economic Policy Research (CEPR) produces the EuroCOIN statistic, which is also based on principal component analysis. There are more than 100 macroeconomic series included in $€$-coin. $€$-coin also includes data derived from surveys, interest rates, and other financial variables.

\section{Nowcasting}
Policy makers and market practitioners use real-time monitoring of economic and financial variables to continuously assess current conditions. To overcome the publication delays (for example, GDP numbers are published with a substantial delay) and forecast the "present," they make use of a large variety of data-such as financial market transactions, data from the usage of the large amounts of timely internet searches, and electronic payment data-to provide estimates for key low frequency (monthly or quarterly) economic indicators. What they produce is a nowcast, an estimate of the current state. We refer to the process of producing such an estimate as nowcasting. It can be applied to various macroeconomic variables of interest, such as GDP growth, inflation, or unemployment.

\section{GDPNow}
Nowcasts are produced by a number of entities in investment banking and asset management for their internal or client use, but they are also published by institutions, such as the Atlanta Fed (the Federal Reserve Bank of Atlanta, one of the 12 Federal Reserve Banks in the United States). According to the Atlanta Fed, "GDPNow" is "best viewed as a running estimate of real GDP growth based on available data for the current measured quarter." The objective is to forecast GDP for the current quarter (which will not be released until after quarter-end) in real time based on data as they are released throughout the quarter. To do this, the Atlanta Fed attempts to use the same methodology and data as will be used by the US Bureau of Economic Analysis (BEA) to estimate GDP, replacing data that have not yet been released with forecasts based on the data already observed. As the quarter progresses, more of the actual data will have been observed. GDPNow should, at least on average, converge to what will be released by the BEA as their "advance" estimate of quarterly GDP about four weeks after quarter-end.

\section{DIFFUSION INDEX OF ECONOMIC INDICATORS}
In the United States, The Conference Board also compiles a monthly diffusion index of the leading, lagging, and coincident indicators. The diffusion index reflects the proportion of the index's components that are moving in a pattern consistent with the overall index. Analysts often rely on these diffusion indexes to provide a measure of the breadth of the change in a composite index.

For example, The Conference Board tracks the growth of each of the 10 constituents of its Leading Economic Index, assigning a value of 1.0 to each indicator that rises by more than $0.05 \%$ during the monthly measurement period, a value of 0.5 for each component indicator that changes by less than $0.05 \%$, and a value of 0 for each component indicator that falls by more than $0.05 \%$. These assigned values, which of course differ in other indexes in other countries, are then summed and divided by 10 (the number of components). To make the overall measure resemble the more familiar indexes, the Board multiplies the result by 100 .

A simple numerical example will help explain. Say, for ease of exposition, the indicator has only four component parts: stock prices, money growth, orders, and consumer confidence. In one month, stock prices rise $2.0 \%$, money growth rises $1.0 \%$, orders are flat and consumer confidence falls by $0.6 \%$. Using The Conference Board's assigned values, these would contribute respectively: $1.0+1.0+0.5+0$ to create a numerator of 2.5 . When divided by four (the number of components) and multiplied by 100 , it generates an indicator of 62.5 for that month.

Assume that the following month stock prices fall $0.8 \%$, money grows by $0.5 \%$, orders pick up $0.5 \%$, and consumer confidence grows 3.5\%. Applying the appropriate values, the components would add to $0+1.0+1.0+1.0=3.0$. Divided by the number of components and multiplied by 100 , it yields an index value of 75 . The $20.0 \%$ increase in the index value means more components of the composite index are rising. Given this result, an analyst can be more confident that the higher composite index value actually represents broader movements in the economy. In general, a diffusion index does not reflect outliers in any component (like a straight arithmetic mean would do) but instead tries to capture the overall change common to all components.

\section{EXAMPLE 10}
\section{Economic Indicators}
\begin{enumerate}
  \item Leading, lagging, and coincident indicators are:
\end{enumerate}

A. the same worldwide.

B. based on historical cyclical observations.

C. based on Keynesian and/or Monetarist theory.

Solution to 1:

$\mathrm{B}$ is correct. The recognition of economic indicators is based on empirical observations for an economy.

\begin{enumerate}
  \setcounter{enumi}{1}
  \item A diffusion index:
\end{enumerate}

A. measures growth.

B. reflects the consensus change in economic indicators.

C. is roughly analogous to the indexes used to measure industrial production.

\section{Solution to 2:}
$\mathrm{B}$ is correct. The diffusion indexes are constructed to reflect the common trends embedded in the movements of all the indicators included in such an index.

\begin{enumerate}
  \setcounter{enumi}{2}
  \item In the morning business news, a financial analyst, Kevin Durbin, learned that average hourly earnings had increased last month. The most appropriate action for Durbin is to:
\end{enumerate}

A. call his clients to inform them of a good trading opportunity today.

B. examine other leading indicators to see any confirmation of a possible turning point for the economy.

C. use the news in his research report as a confirmation for his belief that the economy has recovered from a recession.

\section{Solution to 3:}
$B$ is correct. Financial analysts need to synthesize the information from various indicators in order to gather a reliable reading of the economic trends.

\begin{enumerate}
  \setcounter{enumi}{3}
  \item The following table shows the trends in various economic indicators in the two most recent quarters:
\end{enumerate}

\section{Economic Indicator}
Interest rate spread between long-term government bonds and overnight borrowing rate

New orders for capital goods

Residential building permits

Employees on non-agricultural payrolls

Manufacturing and trade sales

Average duration of unemployment

Change in unit labor costs Trend

Narrowing

Declining

Declining

Turned from rising to falling

Stable

Small decline

Rising Given the information, this economy is most likely experiencing a:
A. continuing recession.
B. peak in the business cycle.
C. strong recovery out of a trough.

\section{Solution to 4:}
$B$ is correct. The first three indicators are leading indicators, and all of them are indicating an impending recession, which means the economy has reached the peak in this cycle. Non-agricultural payrolls and manufacturing and trade sales are coincident indicators. The trends in these two variables further indicate that the economy may begin to decline. The trends in the last two indicators-both lagging indicators-indicate that the economy may either continue to grow or it may be close to a peak. Aggregating the signals given by all three groups of economic indicators, it appears the economy may be near the peak of a business cycle.

\begin{enumerate}
  \setcounter{enumi}{4}
  \item The indicator indexes created by various organizations or research agencies:
A. include only leading indicators to compute their value.
B. are highly reliable signals on the phase of business cycles.
C. evolve over time in terms of composition and computation formula.
\end{enumerate}

\section{Solution to 5:}
$\mathrm{C}$ is correct. The indicator indexes are constantly updated for their composition and methodology based on the accumulation of empirical knowledge, and they can certainly include more than just leading indicators.

\begin{enumerate}
  \setcounter{enumi}{5}
  \item Which one of the following trends in various economic indicators is most consistent with a recovery from a recession?
\end{enumerate}

A. A declining inventory-sales ratio and stable industrial production index

B. A rising broad stock market index and unit labor costs turning from increasing to decreasing

C. A decrease in average weekly initial claims for unemployment insurance and an increase in aggregate real personal income

\section{Solution to 6:}
$\mathrm{C}$ is correct. The improving leading indicator, average weekly initial claims for unemployment insurance, and the improving coincident indicator, aggregate real personal income, are most consistent with an economic recovery. Even though a declining inventory-to-sales ratio, a lagging indicator, is consistent with an early recovery, the coincident indicator, the stable industrial production index, does not support that conclusion. Although a rising stock market index can signal economic expansion, the lagging indicator, the unit labor costs, has peaked, which is more consistent with a recession.

\begin{enumerate}
  \setcounter{enumi}{6}
  \item Which of the following statements gives the best description of nowcasting?
\end{enumerate}

A. It is a method used to forecast future trends in economic variables based on their past and current values. B. It is a method used for real-time monitoring of economic and financial variables to continuously assess current conditions and provide an estimate of the current state.

C. It is a technique used to study past relationships between variables to explain which ones have explained the path of a particular variable of interest.

\section{Solution to 7:}
B is correct. Nowcasting involves the use of techniques to estimate the present state. A is incorrect because nowcasting aims to estimate the present, not forecast the future. $\mathrm{C}$ is incorrect because the focus of nowcasting is to estimate the current, present value of a variable, such as GDP.

\section{UNEMPLOYMENT}
describe types of unemployment, and compare measures of unemployment

Many governments and central banks have economic policy objectives related to limiting the rate at which citizens are unemployed and containing price inflation (i.e., preserving the purchasing power of a domestic currency). The relationships of these variables to the business cycle are discussed in the following sections. In general, unemployment is at its highest somewhat after the recovery has started and is at its lowest somewhat after the peak of the economy.

\section{Unemployment}
An overheated economy often leads to inflation when unemployment is very low. Workers ask for higher wages because they expect prices of goods and services to keep going up; at the same time, they have market power against employers because there are few available workers to be hired. The upward pressure on wages coupled with the impact of wage escalator clauses (automatic increases in wages as the consumer price index grows) trigger a price-wage inflationary spiral. This issue was a particular problem in industrialized countries during the 1960s and 1970s and to a degree remains relevant today.

A key aspect in this process is inflation expectations. Because inflation expectations are high, the request for higher wages is stronger, which induces employers to increase prices in advance to keep their profit margins stable. This avalanche process grows with time. Central banks act, sometimes drastically, to slow down the economy and reset inflationary expectations throughout the economy at a low level so that if everyone expects low inflation, the inflationary spiral itself will stop. These actions may trigger a deep recession. Therefore, whenever a financial analyst sees signs of a price-wage spiral in the making, a reasonable response would be to consider the effect of both high inflation and sharp tightening of monetary policy.

This example shows that measures of labor market conditions are important in assessing whether an economy is at risk of cyclical downturn. The following are definitions of a few terms used to summarize the state of the labor market:

\begin{itemize}
  \item Employed: The number of people with a job. This figure normally does not include people working in the informal sector (e.g., unlicensed cab drivers, illegal workers).

  \item Labor force: The number of people who either have a job or are actively looking for a job. This number excludes retirees, children, stay-at-home parents, full-time students, and other categories of people who are neither employed nor actively seeking employment.

  \item Unemployed: People who are actively seeking employment but are currently without a job. Some special subcategories include:

  \item Long-term unemployed: People who have been out of work for a long time (more than three to four months in many countries) but are still looking for a job.

  \item Frictionally unemployed: People who are not working at the time of filling out the statistical survey because they are taking time to search for a job that matches their skills, interests, and other preferences better than what is currently available, or people who have left one job and are about to start another job. The frictionally unemployed includes people who have voluntarily left their previous positions to change their jobsin other words, they are "between jobs"-and those new entrants or re-entrants into the labor force who have not yet found work. Frictional unemployment is short-term and transitory in nature.

  \item Unemployment rate: The ratio of unemployed to labor force.

  \item Participation rate (or activity ratio): The ratio of labor force to total population of working age (i.e., those between 16 and 64 years of age).

  \item Underemployed: A person who has a job but either has the qualifications to work at a significantly higher-paying job or works part-time but desires a full-time position. For example, a lawyer who is out of work and takes a job in a bookstore could call herself underemployed. This lawyer would not count as unemployed for the computation of the unemployment rate (she does have a job, even if it may not be her highest-paying job). Although the unemployment rate statistic is criticized for not taking the issue of underemployment into account, it may be difficult to classify whether a person is truly underemployed-for example, the lawyer may find legal work too stressful and prefers working at the bookstore.

  \item Discouraged worker: A person who has stopped looking for a job. Perhaps because of a weak economy, the discouraged worker has given up seeking employment. Discouraged workers are statistically outside the labor force (similar to children and retirees), which means they are not counted in the official unemployment rate. During prolonged recessions, the unemployment rate may actually decrease because many discouraged workers stop seeking work. It is important to observe the participation rate together with the unemployment rate to understand if unemployment is decreasing because of an improved economy or because of an increase in discouraged workers. Discouraged workers and underemployed people may be considered examples of "hidden unemployment."

  \item Voluntarily unemployed: A person voluntarily outside the labor force, such as a jobless worker refusing an available vacancy for which the wage is lower than their threshold or those who retired early.

\end{itemize}

\section{The Unemployment Rate}
The unemployment rate is certainly the most quoted measure of unemployment; it attempts to measure those people who have no work but would work if they could find it, generally stated as a percentage of the overall labor force. In the United States, the indicator emerges from a monthly survey of households by the US Bureau of Labor Statistics. The survey asks how many household members have jobs and how many of working age do not have jobs but are seeking work. Other statistical bureaus rely on other sources for the calculation, using claims for unemployment assistance, for instance, or their equivalent. Some statistical bureaus measure the labor force simply as those of working age, regardless of whether they are ready or willing to work. These differences can make precise international comparisons problematic. One solution is to use the International Labour Organization (ILO) statistics that try to estimate on a consistent basis. As indicated earlier, some statistical agencies add perspective with other measures-for example, what proportion of those who have ceased work are discouraged, underemployed, or have opted out of the labor force for other reasons or are working part-time.

Although these various unemployment measures provide insight to the state of the economy, they are inaccurate in pointing to cyclical directions for two primary reasons, both of which make unemployment a lagging economic indicator of the business cycle.

One reason is that the unemployment rate tends to point to a past economic condition-that is, it lags the cycle-because the labor force expands and declines in response to the economic environment. Compounding the inaccuracy, discouraged workers cease searching for work when times get hard, reducing the number typically counted as unemployed and making the job market look stronger than it really is. Conversely, when the job market picks up, these people return to the search; however, because they seldom find work immediately, they at least initially raise the calculation of those unemployed, giving the false impression of the lack of recovery in the jobs market, when, in fact, it is the improvement that brought these people back into the labor force in the first place. Sometimes this cyclical flow of new job seekers is so great that the unemployment rate actually rises even as the economic recovery gains momentum. Those agencies that measure the labor force in terms of the working-age population avoid this bias because the working-age population measure is unaffected by economic conditions in the labor market. But this approach introduces biases of its own, such as counting as unemployed those people who have severe disabilities and could never seek work.

Another reason the unemployment indicator tends to lag the cycle comes from the typical reluctance of businesses to lay off people. The reluctance may stem from a desire to retain good workers for the long run, or it may just reflect constraints written into labor contracts that make layoffs expensive. The reluctance makes the various measures of unemployment rise more slowly as the economy slides into recession than they otherwise might. Then as the recovery develops, a business waits to hire until it has fully employed the workers it has kept on the payroll during the recession; this delay causes decreases in the unemployment rate to lag in the cyclical recovery, sometimes for a long time.

\section{Overall Payroll Employment and Productivity Indicators}
To get a better picture of the employment cycle, practitioners often rely on more straightforward measures of payroll growth. By measuring the size of payrolls, practitioners sidestep such issues as the ebb and flow of discouraged workers. These statistics, however, have biases of their own. It is hard, for instance, to count employment in smaller businesses, which may be significant drivers of employment growth. Still, there is a clear indication of economic trouble when payrolls shrink and a clear indication of recovery when they rise. The examination of other measures can also assist in understanding the employment situation and its use in determining cyclical directions. Two additional measures are hours worked, especially overtime, and the use of temporary workers. A business does not want to make mistakes with full-time staff, either hiring or firing. Thus, at the first signs of economic weakness, managers cut back hours, especially overtime. Such movements can simply reflect minor month-to-month production shifts, but if followed by cutbacks in part-time and temporary staff, the picture gives a strong signal of economic weakness, especially if confirmed by other independent indicators. Similarly, on the cyclical upswing, a business turns first to increases in overtime and hours. If a business then increases temporary staffing, it gives a good signal of economic recovery long before any movement in rehiring full-time staff again, especially if confirmed by independent cyclical indicators.

Productivity measures also offer insight into this cyclical process. Because productivity is usually measured by dividing output by hours worked, a business's tendency to keep workers on the payroll even as output falls usually prompts a reduction in measured productivity. If measures are available promptly enough, this sign of cyclical weakness might precede even the change in hours. This drop in productivity precedes any change in full-time payrolls. Productivity also responds promptly when business conditions improve and the business first begins to utilize its underemployed workers, which occurs earlier than any upturn in full-time payrolls.

On a more fundamental level, productivity can also pick up in response to technological breakthroughs or improved training techniques. As already mentioned, such changes affect potential GDP. If strong enough, they can negatively affect employment trends, keeping them slower than they would be otherwise by relieving the need for additional staff to increase production. But these influences usually unfold over decades and mean little to cyclical considerations, which, at most, unfold over years. What is more, there are few statistical indicators to gauge the onset of technological change, confining analysts to the use of anecdotal evidence or occasional longitudinal studies.

\section{EXAMPLE 11}
\section{Analyzing Unemployment}
\begin{enumerate}
  \item At the peak of the business cycle, if the unemployment rate is low, the majority of the unemployed are most likely:
\end{enumerate}

A. discouraged workers.

B. long-term unemployed.

C. frictionally unemployed.

\section{Solution to 1:}
$\mathrm{C}$ is correct. At the peak of a business cycle, the labor market is usually tight, and people become unemployed largely because they are either "between jobs" or they have entered or reentered the labor force but have not yet found work.

\begin{enumerate}
  \setcounter{enumi}{1}
  \item As an economy starts to recover from a trough in the business cycle, the unemployment rate is most likely to:
\end{enumerate}

A. continue to rise with a decline in the number of discouraged workers.

B. start to decline with an increase in the number of discouraged workers. C. continue to rise with an increase in the number of discouraged workers.

\section{Solution to 2:}
A is correct. As the economy starts to recover, discouraged workers return to the labor force and start looking for jobs, which increases both the number of unemployed and the size of the labor force. The unemployment rate rises because the rise in the unemployed population is proportionately larger than the increase in the size of the labor force. $B$ and $C$ are incorrect because an increase in the number of discouraged workers typically occurs when the economy is contracting.

\begin{enumerate}
  \setcounter{enumi}{2}
  \item An analyst observes that the unemployment rate is high and rising, whereas productivity and hours worked have declined. The analyst is most likely to conclude that the labor market is signaling the:
\end{enumerate}

A. end of a recession.

B. deepening of a recession.

C. peak of the business cycle.

\section{Solution to 3:}
B is correct. High and rising unemployment, declining hours worked, and falling productivity are all signs of a weak economy getting weaker. When the economy first slows down, businesses cut back employees' hours. As the recession deepens, they then lay off employees, leading to a higher unemployment rate. Yet, because workforce turnover is costly for businesses, the scale of the layoff can be less than the decline in output, resulting in a decline in productivity. A is incorrect because toward the end of a recession, businesses are hesitant to increase hiring and instead use more overtime, increasing both productivity and the hours worked. $\mathrm{C}$ is incorrect because at the peak of a business cycle, the unemployment rate is usually low and the level of hours worked is high.

\section{INFLATION}
explain inflation, hyperinflation, disinflation, and deflation explain the construction of indexes used to measure inflation compare inflation measures, including their uses and limitations contrast cost-push and demand-pull inflation

The overall price level changes at varying rates during different phases of a business cycle. Thus, when studying business cycles, it is important to understand this phenomenon. In general, the inflation rate is pro-cyclical (that is, it goes up and down with the cycle), but with a lag of a year or more.

Inflation refers to a sustained rise in the overall level of prices in an economy. Economists use various price indexes to measure the overall price level, also called the aggregate price level. The inflation rate is the percentage change in a price index-that is, the speed of overall price level movements. Investors follow the inflation rate closely not only because it can help to infer the state of the economy, but also because an unexpected change may result in a change in monetary policy, which can have a large and immediate impact on asset prices. In developing countries, very high inflation rates can lead to social unrest or even shifts of political power, which constitutes political risk for investments in those economies.

Central banks, the monetary authority in most economies, monitor the domestic inflation rates closely when conducting monetary policy. Monetary policy determines interest rates and the available quantities of money and loans in an economy. A high inflation rate combined with fast economic growth and low unemployment usually indicates the economy is overheating, which may trigger some policy movements to cool it down. However, if a high inflation rate is combined with a high level of unemployment and a slowdown of the economy-an economic state known as stagflation (stagnation plus inflation) - the economy will typically be left to correct itself because no short-term economic policy is deemed effective.

\section{Deflation, Hyperinflation, and Disinflation}
Various terms are related to the levels and changes of the inflation rate:

\begin{itemize}
  \item Deflation: A sustained decrease in aggregate price level, which corresponds to a negative inflation rate-that is, an inflation rate of less than $0 \%$.

  \item Hyperinflation: An extremely fast increase in aggregate price level, which corresponds to an extremely high inflation rate-for example, $500 \%$ to $1,000 \%$ per year.

  \item Disinflation: A decline in the inflation rate, such as from around 15\% to $20 \%$ to $5 \%$ or $6 \%$. Disinflation is very different from deflation because even after a period of disinflation, the inflation rate remains positive and the aggregate price level keeps rising (although at a slower speed).

\end{itemize}

Inflation means that the same amount of money can purchase less real goods or services in the future. So, the value of money or the purchasing power of money decreases in an inflationary environment. When deflation occurs, the value of money actually increases. Because most debt contracts are written in fixed monetary amounts, the liability of a borrower also rises in real terms during deflation. As the price level falls, the revenue of a typical company also falls during a recession. Facing increasing real debt, a company that is short of cash usually cuts its spending, investment, and workforce sharply. Less spending and high unemployment further exacerbate the economic contraction. To avoid getting too close to deflation, the consensus on the preferred inflation rate is around $2 \%$ per year for developed economies. Deflation occurred in the United States during the Great Depression and briefly during the recession following the global financial crisis of 2008-2009. Since the late 1990s, Japan has experienced several episodes of deflation.

Hyperinflation usually occurs when large scale government spending is not backed by real tax revenue and the monetary authority accommodates government spending by increasing the money supply. Hyperinflation may also be caused by the shortage of supply created during or after a war, economic regime transition, or prolonged economic distress of an economy caused by political instability. During hyperinflation, people are eager to change their cash into real goods because prices are rising very fast. As a result, money changes hands at an extremely high frequency. The government also has to print more money to support its increased spending. As more cash chases a limited supply of goods and services, the rate of price increases accelerates. After World War I, a famous case of hyperinflation occurred in Germany from 1923 to 1924. During the peak of this episode, prices doubled every 3.7 days. After World War II, Hungary experienced a severe hyperinflation during which prices doubled every 15.6 hours at its peak in 1946. In 1993, the inflation rate in Ukraine peaked at $10,155 \%$ per year. In January 1994, the monthly inflation rate peaked at 313 million\% in Yugoslavia. The most recent hyperinflation in Zimbabwe reached a peak of monthly inflation at 79.6 billion\% in the middle of November 2008. Because the basic cause for hyperinflation is too much money in circulation, regaining control of the money supply is the key to ending hyperinflation.

Exhibit 19 shows recent episodes of disinflation in selected countries around the world. The first episode happened during the early 1980s. Because of the two oil crises in the 1970s, many countries around the world were experiencing high levels of inflation. In Exhibit 19, the annual inflation rates in most countries around 1980 ranged between $10 \%$ and $20 \%$. Even though this level is still far from hyperinflation, it generated social pressure against inflationary monetary policy. At the cost of a severe recession early in the 1980s, these countries brought inflation rates down to around $5 \%$, on average, by 1985 . In the early 1990s, inflationary experiences varied widely in world markets as some countries entered recessions, such as the United States and the United Kingdom, and others boomed. However, from the beginning to the end of the 1990s, there was a broad-based decline in inflation rates; in some countries, annual inflation rates were below $2 \%$ by the end of the decade. In many countries, the decline in inflation was attributed to high productivity growth rates.

\section{Exhibit 19: Two Episodes of Disinflation around the World Annual Inflation Rates}
\begin{center}
\begin{tabular}{|c|c|c|c|c|c|c|c|c|c|}
\hline
\multirow[b]{2}{*}{Year} & \multicolumn{5}{|c|}{First Episode} & \multicolumn{4}{|c|}{Second Episode} \\
\hline
 & 1979 & 1980 & 1983 & 1984 & 1985 & 1990 & 1991 & 1998 & 1999 \\
\hline
\multicolumn{10}{|l|}{Country} \\
\hline
Australia & 9.1 & 10.2 & 10.1 & 3.9 & 6.7 & 7.3 & 3.2 & 0.9 & 1.5 \\
\hline
Canada & 9.1 & 10.1 & 5.9 & 4.3 & 4.0 & 4.8 & 5.6 & 1.0 & 1.7 \\
\hline
Finland & 7.5 & 11.6 & 8.4 & 7.1 & 5.2 & 6.1 & 4.3 & 1.4 & 1.2 \\
\hline
France & 10.6 & 13.6 & 9.5 & 7.7 & 5.8 & 3.2 & 3.2 & 0.6 & 0.5 \\
\hline
Germany & 4.0 & 5.4 & 3.3 & 2.4 & 2.1 & 2.7 & 4.0 & 1.0 & 0.6 \\
\hline
Italy & 14.8 & 21.1 & 14.6 & 10.8 & 9.2 & 6.5 & 6.3 & 2.0 & 1.7 \\
\hline
Japan & 3.7 & 7.8 & 1.9 & 2.3 & 2.0 & 3.1 & 3.3 & 0.7 & -0.3 \\
\hline
South Korea & 18.3 & 28.7 & 3.4 & 2.3 & 2.5 & 8.6 & 9.3 & 7.5 & 0.8 \\
\hline
Spain & 15.7 & 15.6 & 12.2 & 11.3 & 8.8 & 6.7 & 5.9 & 1.8 & 2.3 \\
\hline
Sweden & 7.2 & 13.7 & 8.9 & 8.0 & 7.4 & 10.4 & 9.4 & -0.3 & 0.5 \\
\hline
United Kingdom & 13.4 & 18.0 & 4.6 & 5.0 & 6.1 & 7.0 & 7.5 & 1.6 & 1.3 \\
\hline
United States & 11.3 & 13.5 & 3.2 & 4.3 & 3.5 & 5.4 & 4.2 & 1.6 & 2.2 \\
\hline
Average & 10.4 & 14.1 & 7.2 & 5.8 & 5.3 & 6.0 & 5.5 & 1.6 & 1.2 \\
\hline
G-7 Countries & 9.6 & 12.5 & 4.6 & 4.6 & 3.9 & 4.8 & 4.4 & 1.3 & 1.4 \\
\hline
\end{tabular}
\end{center}

Source: The Organisation for Economic Co-Operation and Development (OECD).

\section{Measuring Inflation: The Construction of Price Indexes}
Because the inflation rate is measured as the percentage change of a price index, it is important to understand how a price index is constructed so that the inflation rate derived from that index can be accurately interpreted. A price index represents the average prices of a basket of goods and services, and various methods can be used to average the different prices. Exhibit 20 shows a simple example of the change of a consumption basket over time.

\section{Exhibit 20: Consumption Basket and Prices over Two Months}
\begin{center}
\begin{tabular}{lcccc}
\hline
Time & \multicolumn{2}{c}{January $\mathbf{2 0 1 9}$} & \multicolumn{2}{c}{February 2019} \\
\hline
Goods & Quantity & Price & Quantity & Price \\
\hline
Rice & $50 \mathrm{~kg}$ & $¥ 3 / \mathrm{kg}$ & $70 \mathrm{~kg}$ & $¥ 4 / \mathrm{kg}$ \\
Gasoline & 70 liters & $¥ 4.4 / \mathrm{liter}$ & 60 liters & $¥ 4.5 / \mathrm{liter}$ \\
\hline
\end{tabular}
\end{center}

For January 2019, the total value of the consumption basket is:

Value of rice + Value of gasoline $=(50 \times 3)+(70 \times 4.4)=¥ 458$.

A price index uses the relative weight of a good in a basket to weight the price in the index. Therefore, the same consumption basket in February 2019 is worth:

Value of rice + Value of gasoline $=(50 \times 4)+(70 \times 4.5)=¥ 515$.

The price index in the base period is usually set to 100 . So, if the price index in January 2019 is 100, then the price index in February 2019 is

Price index in February $2019=\frac{515}{458} \times 100=¥ 112.45$ and

Inflation rate $=\frac{112.45}{100}-1=0.1245=12.45 \%$.

A price index created by holding the composition of the consumption basket constant is called a Laspeyres index. Most price indexes around the world are Laspeyres indexes because the survey data on the consumption basket are only available with a lag. In many countries, the basket is updated every five years. Because most price indexes are created to measure the cost of living, however, simply using a fixed basket of goods and services has three serious biases:

\begin{itemize}
  \item Substitution bias: As the price of one good or service rises, people may substitute it with other goods or services that have a lower price. This substitution will result in an upward bias in the measured inflation rate based on a Laspeyres index.

  \item Quality bias: As the quality of the same product improves over time, it satisfies people's needs and wants better. One such example is the quality of cars. Over the years, the prices of cars have been rising while the safety and reliability of cars have been enhanced. If not adjusted for quality, the measured inflation rate will experience another upward bias.

  \item New product bias: New products are frequently introduced, but a fixed basket of goods and services will not include them. In general, this situation again creates an upward bias in the inflation rate.

\end{itemize}

It is relatively easy to resolve the quality bias and new product bias. Many countries adjust for the quality of the products in a basket, a practice called hedonic pricing. New products can be introduced into the basket over time. The substitution bias can be somewhat resolved by using the chained price index formula. One such example is the Fisher index, which is the geometric mean of the Laspeyres index and the Paasche index. The latter is an index formula using the current composition of the basket. Using the consumption basket for February 2019 in Exhibit 20, the value of the Paasche index is

$$
\begin{aligned}
& \text { Paasche Index }{ }_{02 / 2019}=I_{P}=\frac{(70 \times 4)+(60 \times 4.5)}{(70 \times 3)+(60 \times 4.4)} \times 100 \\
& \quad=\frac{550}{474} \times 100=116.03
\end{aligned} .
$$

The value of the Fisher index is

Fisher $\operatorname{Index}_{02 / 2019}=\sqrt{I_{P} \times I_{L}}=\sqrt{116.03 \times 112.45}=114.23$,

where $I_{L}$ is the Laspeyres index.

\section{Price Indexes and Their Usage}
Most countries use a consumer price index (CPI) specific to the domestic economy to track inflation. Exhibit 21 shows the different weights for various categories of goods and services in the consumer price indexes of different countries.

\begin{center}
\begin{tabular}{|c|c|c|c|c|c|c|}
\hline
Country & China & India & India & Germany & $\begin{array}{l}\text { United } \\ \text { States }\end{array}$ & $\begin{array}{l}\text { Unitec } \\ \text { States }\end{array}$ \\
\hline
Name of Index & $\mathrm{CPI}$ & CPI(UNME) & $\operatorname{CPI}(\text { Urban })^{\mathrm{c}}$ & $\mathrm{HICP}$ & CPI-U & PCE \\
\hline
Year $^{\mathrm{a}}$ & $2016^{\mathrm{b}}$ & $1984 / 85$ & 2012 & $2017 / 18$ & $2017 / 18$ & 2009 \\
\hline
\multicolumn{7}{|l|}{Category (\%):} \\
\hline
Food and Beverage & 30 & 47.1 & 37.7 & 14 & 15.3 & 13.4 \\
\hline
Housing and Utility & 21 & 21.9 & 27.2 & 31.7 & 37.9 & 17.3 \\
\hline
Furniture & 6 & 2 & 3.9 & 5 & 4.2 & 3.4 \\
\hline
Apparel & 7.5 & 7 & 5.6 & 4.5 & 3.3 & 3.3 \\
\hline
Medical Care & 8 & 2.5 & 4.8 & 4.4 & 7.7 & 16.9 \\
\hline
Transportation and Communication & 13 & 5.2 & 9.7 & 16.5 & 15.3 & $9.3^{\mathrm{d}}$ \\
\hline
Education and Recreation & 10.5 & 6.8 & 7.6 & 12.3 & 9.1 & $8.9^{\mathrm{e}}$ \\
\hline
Others & 4 & 7.5 & 3.5 & 11.6 & 7.2 & 27.6 \\
\hline
\end{tabular}
\end{center}

\section{Exhibit 21: The Consumption Basket of Different Consumer Price Indexes}
a The base year of the weights, where it is appropriate.

b Weights for China are not released publicly. Imputed numbers given as of 2018 are from the Reserve Bank of Australia.

${ }^{\mathrm{c}}$ India redefined the CPI bundle in 2012 for urban consumers.

$\mathrm{d}$ Includes only transportation expenditures by consumers.

e Includes only recreational expenditures by consumers.

Sources: Government websites and authors' calculations.

As shown in Exhibit 21, in different countries the consumer price indexes have different names and different weights on various categories of goods and services. For example, food weights are higher in the CPI for China and India but less for the developed countries; a greater proportion of income of the average consumer goes to food in the developing countries of China and India than in the developed countries shown in Exhibit 21. For India, the weights across categories change dramatically over time as the country has developed. Weights across the categories change across time in developed countries too, but these changes tend to be much smaller. The scope of the index is also different among countries. For China and Germany, the surveys used to collect data for CPI cover both urban and rural areas. The CPI for the United States covers only urban areas using a household survey, which is why it is called the CPI-U. On the other hand, using business surveys, the personal consumption expenditure (PCE) price index covers all personal consumption in the United States. The producer price index (PPI) is another important inflation measure. The PPI reflects the price changes experienced by domestic producers in a country. Because price increases may eventually pass through to consumers, the PPI can influence the future CPI. The items in the PPI include fuels, farm products (such as grains and meat), machinery and equipment, chemical products (such as drugs and paints), transportation equipment, metals, pulp and paper, and so on. These products are usually further grouped by stage-of-processing categories: crude materials, intermediate materials, and finished goods. Similar to the CPI, scope and weights vary among countries. The differences in the weights can be much more dramatic for the PPI than for the CPI because different countries may specialize in different industries. In some countries, the PPI is called the wholesale price index (WPI).

As an important inflation indicator, many economic activities are indexed to a certain price index. For example, in the United States, Treasury Inflation-Protected Securities (TIPS) adjusts the bond's principal according to the US CPI-U index. The terms of labor contracts and commercial real estate leases may adjust periodically according to the CPI. Recurring payments in business contracts can be linked to the PPI or its sub-indexes for a particular category of products.

Central banks usually use a consumer price index to monitor inflation. For example, the European Central Bank (ECB), the central bank for the European Union (EU), focuses on the Harmonised Index of Consumer Prices (HICP). Each member country in the EU first reports their own individual HICP, and then Eurostat, the statistical office for the EU, aggregates the country-level HICPs with country weights. But there are exceptions. The Reserve Bank of India follows the inflation in India using a WPI. Because food items only represent about $27 \%$ in the India WPI (much lower than the $70 \%$ in the India rural CPI), the rural CPIs can rise faster than the WPI when there is high food price inflation. Besides the weight differences, the wholesale prices in the WPI also understate market prices because they do not consider retail margins (markups). The choice of inflation indicator may also change over time. The central bank of the United States, known as the Federal Reserve Board (the Fed), once focused on the CPI-U produced by the Bureau of Labor Statistics under the US Department of Labor. Because the CPI-U is a Laspeyres index and has the previously discussed upward biases, however, the Fed switched in 2000 to the PCE index, a Fisher index produced by the Bureau of Economic Analysis under the US Department of Commerce. The PCE index also has the advantage that it covers the complete range of consumer spending rather than just a basket.

\section{HEADLINE AND CORE INFLATION}
Headline inflation refers to the inflation rate calculated based on the price index that includes all goods and services in an economy. Core inflation usually refers to the inflation rate calculated based on a price index of goods and services except food and energy. Policy makers often choose to focus on the core inflation rate when reading the trend in the economy and making economic policies As they try to avoid overreaction to short-term fluctuations in food and energy prices that may not have a significant impact on future headline inflation.

The ultimate goal for policy makers is to control headline inflation, which reflects the actual cost of living. The fluctuations in the prices of food and energy are often the result of short-term changes in supply and demand. These changes in the prices of energy, particularly oil, are internationally determined and not necessarily reflective of the domestic business cycle. These imbalances may not persist, or even if some changes are permanent, the economy may be able to absorb them over time. These possibilities make headline inflation a noisy predictor. The core inflation rate may be a better signal of the trend in domestically driven inflation. To the extent that some trends in the headline inflation rate are permanent, policy makers need to pay attention to these as well. Besides tracking inflation, financial analysts also use the price index to deflate GDP (i.e., to eliminate the price effect in nominal GDP data so as to identify trends in real economic growth). Many countries publish a particular price index, called the GDP deflator, for that purpose. Sub-indexes are also commonly available and may prove more valuable to an analyst with an interest in a particular industry or company.

\section{SUB-INDEXES AND RELATIVE PRICES}
As mentioned previously, a sub-index refers to the price index for a particular category of goods or services. Relative price is the price of a specific good or service in comparison with those of other goods and services. Good examples for relative prices include the prices for food and energy. The movements in a sub-index or a relative price may be difficult to detect in the headline inflation rate. Because macroeconomic policy decision makers rely heavily on the headline inflation rate, they may not be aware of price movements at the sub-index level. These price movements, however, can be very useful for analyzing the prospects of an industry or a company. For example, if the producer price index for the machinery used by an industry rises quickly, the allowable capital depreciation permitted by the existing tax code may not generate sufficient tax benefits for the companies in that industry to meet future replacement expenses. The future profitability of the industry may decline for this reason. The decline in prices for flat screen televisions provides an example of relative price movements. The price drop for these TVs may help to lower inflation pressure but can hurt manufacturers' profits.

\section{EXAMPLE 12}
\section{Inflation}
\begin{enumerate}
  \item Which of the following statements regarding the movements of overall price levels is most accurate?
\end{enumerate}

A. Disinflation means that the overall price level declines.

B. Deflation occurs when the inflation rate turns negative.

C. When the price of chicken rises, the inflation rate will increase.

\section{Solution to 1:}
$\mathrm{B}$ is correct. When the inflation rate falls below zero-that is, the overall price level declines-the economy is experiencing deflation. A is incorrect, because disinflation indicates that the overall price level is rising but at a slower pace. $C$ is incorrect because inflation measures are designed to reflect changes in the overall price level. Consumption baskets in modern economies usually contain a large number of goods and services, thus the price of a particular product usually cannot significantly influence the overall price level.

\begin{enumerate}
  \setcounter{enumi}{1}
  \item Deflation can exacerbate a recession because firms may reduce their investments and hiring when:
\end{enumerate}

A. the slower pace of inflation lowers aggregate demand.

B. their revenues decline but their debt burden rises in real terms.

C. the prices of their products continue to fall because of intense competition.

\section{Solution to 2:}
B is correct. As the prices of the output of firms fall, the firms receive lower revenues. Because the nominal amount of debt that firms carry is usually fixed, lower general price levels leads to higher debt balances in real terms. These two forces push firms closer to default, so they may scale back spending on investments and labor, which further lowers the aggregate demand and pushes the general price level even lower. In macroeconomic analysis, it is usually the changes in aggregate demand that influence inflation instead of the reverse causality. Furthermore, neither inflation fluctuations nor aggregate demand shifts explain the potential damaging effect of deflation. Price decline attributable to the competitive environment is a microeconomic phenomenon that is not sufficient to explain the macroeconomic impact of deflation.

\begin{enumerate}
  \setcounter{enumi}{2}
  \item Which one of the following economic phenomena related to inflation cannot be determined by using observations of the inflation rate alone?
\end{enumerate}

A. Deflation

B. Stagflation

C. Hyperinflation

\section{Solution to 3:}
B is correct. A high inflation rate alone does not indicate stagflation, which happens if high unemployment occurs together with high inflation.

\begin{enumerate}
  \setcounter{enumi}{3}
  \item If a price index is calculated based on a fixed basket of goods, in an inflationary environment the inflation rate calculated based on this index over time will:
\end{enumerate}

A. overstate the actual cost of living.

B. understate the actual cost of living.

C. track the actual cost of living quite closely.

\section{Solution to 4:}
A is correct. Upward biases, such as the substitution bias or quality bias, will overstate the actual cost of living.

\begin{enumerate}
  \setcounter{enumi}{4}
  \item To adjust nominal economic growth for general price level changes in a country, an analyst would prefer to use:
A. the CPI.
B. the GDP deflator.
C. the Personal Consumption Expenditures (PCE) index.
\end{enumerate}

\section{Solution to 5:}
B is correct. The GDP deflator reflects the prices of the goods and services produced domestically. Both the CPI and PCE indexes are constructed using consumption baskets, and the components of a consumption basket can be very different from the components of output of that same country.

\begin{enumerate}
  \setcounter{enumi}{5}
  \item To estimate the trends in sales and production costs of a given industry, an analyst would prefer to collect data on:
\end{enumerate}

A. the sub-indexes of the wholesale price index (WPI) for that industry. B. the sub-indexes of both the CPI and WPI that are relevant to the industry.

C. the sub-indexes of the CPI relevant to the output and inputs of that industry.

Solution to 6:

B is correct. A sub-index of the CPI reflects the market price changes of the products of an industry, whereas a sub-index of the WPI reflects the price changes of the inputs of an industry. The different composition of outputs and inputs of an industry need to be appropriately accounted for when selecting a price series. Furthermore, the WPI may not take the markups set by the industry into account.

\begin{enumerate}
  \setcounter{enumi}{6}
  \item Compared with core inflation, headline inflation:
\end{enumerate}

A. has an upward bias.

B. is more subject to short-term market conditions.

C. can more accurately predict future inflation.

Solution to 7:

B is correct. Headline inflation is heavily influenced by food and energy price fluctuations, which are affected by short-term supply and demand changes in these markets. These market conditions may not persist. It is also possible for an economy to absorb the price changes so that they will not have long-lasting impact on the headline inflation rate. This means headline inflation contains a great deal of noise and is not a reliable predictor of future inflation trends. The biases in various inflation measures are inherent in the index construction methodology and are not related to the price movements of the goods and services.

\section{Explaining Inflation}
Economists describe two types of inflation: cost-push, in which rising costs, usually wages, compel businesses to raise prices generally; and demand-pull, in which increasing demand raise prices generally, which then are reflected in a business's costs as workers demand wage hikes to catch up with the rising cost of living. Whatever the sequence by which prices and costs rise in an economy, the fundamental cause is the same: excessive demands-either for raw materials, finished goods, or labor-that outstrip the economy's ability to respond. The initial signs appear in the areas with the greatest constraints: the labor market, the commodity market, or some area of final output. Even before examining particular cost and price measures, when considering inflation, practitioners look to indicators that might reveal when the economy faces such constraints.

\section{Cost-Push Inflation}
Considering cost-push inflation, analysts may look at commodity prices because commodities are an input to production. But because wages are the single biggest cost to businesses, practitioners focus most particularly on wage-push inflation, which is tied to the labor market. Because the object is to gauge demand for labor relative to capacity, the unemployment rate is key, as well as measures of the number of workers available to meet the economy's expanding needs. Obviously, the higher the unemployment rate, the lower the likelihood that shortages will develop in labor markets; whereas the lower the unemployment rate, the greater the likelihood shortages will drive up wages. Because the unemployment rate generally only counts people who are looking for work, some practitioners argue that it fails to account for the economy's full labor potential. They state that a tight labor market will bring people out in search of work and ease any potential wage strains. To account for this issue and to modify the unemployment rate indicator, these practitioners also look at the participation rate of people in the labor force, arguing that it gives a fuller and more accurate picture of potential than the unemployment rate.

Analysis in this area recognizes that not all labor is alike. Structural factors related to training deficiencies, cultural patterns in all or some of the population, inefficiencies in the labor market, and the like can mean that the economy will effectively face labor shortages long before the unemployment rate reaches very low figures. This effective unemployment rate, below which pressure emerges in labor markets, is frequently referred to as the non-accelerating inflation rate of unemployment (NAIRU) or, drawing on the work of the Nobel Prize winner Milton Friedman, the natural rate of unemployment (NARU). Of course, these rates vary from one economy to another and over time in a single economy. It is this rate rather than full employment that determines when an economy will experience bottlenecks in the labor market and wage-push inflationary pressures.

Take, for example, the technology sector. It has grown so rapidly in some economies that training in the labor force cannot keep up with demand. This sector can, therefore, face shortages of trained workers and attendant wage pressures even though the economy as a whole seems to have considerable slack in the overall labor market. Until training (supply) catches up with demand, that economy may experience wage and inflation pressure at rates of unemployment that in other places and circumstances might suggest ample slack in the labor market and much less wage-push pressure.

Assessments of wage-push inflation also consider direct observations of wage trends that, when they accelerate, might force businesses to raise prices (initiating the wage-price spiral mentioned earlier in this reading). Statistical agencies provide a wide array of wage-cost indicators, such as hourly wage gauges, weekly earnings, and overall labor costs, including the outlays for benefits. Some of these indicators include the effects of special overtime pay or bonuses, and others do not. And although these measures give an idea of the cost to businesses and hence the kind of wage-push inflationary pressure, a complete picture only emerges when practitioners examine such trends alongside productivity measures.

Productivity, or output per hour, is an essential part of wage-push inflation analysis because the output available from each worker determines the number of units over which businesses can spread the cost of worker compensation. The greater each worker's output per hour, the lower the price businesses need to charge for each unit of output to cover hourly labor costs. And by extension, the faster output per hour grows, the faster labor compensation can expand without putting undue pressure on businesses' costs per unit of output. The equation for this unit labor cost (ULC) indicator, as it is called, is as follows:

$\mathrm{ULC}=W / O$

where

$$
\begin{aligned}
\mathrm{ULC} & =\text { unit labor costs } \\
O & =\text { output per hour per worker } \\
W & =\text { total labor compensation per hour per worker }
\end{aligned}
$$

Many factors can affect labor productivity across time and between economies. The cyclical swings have already been described, as have the effects of technology and training. The pace of development also tends to increase worker productivity because the more sophisticated equipment, systems, and technologies workers have at their disposal, the higher their output per hour. Whatever causes the productivity growth, if it fails to keep up with worker compensation, unit costs to a business rise, and as a business tries to protect its profit margins, prices generally come under increasing upward pressure. Generally, this situation occurs because heavy demand for labor relative to available labor resources has pushed up compensation faster than productivity. Practitioners use a variety of indicators to identify cost- or wage-push inflationary pressure.

\section{EXAMPLE 13}
\section{Unemployment Too High}
\begin{enumerate}
  \item Which of the following is not a problem with the NARU and NAIRU?
\end{enumerate}

A. They only work in monetarist models.

B. They may change over time given changes in technology and economic structure.

C. They do not account for bottlenecks in segments of the labor market (e.g., college graduates).

\section{Solution:}
A is correct. The NARU and NAIRU are the unemployment rates at which the inflation rate will not rise because of a shortage of labor. This concept does not tie to a particular school of macroeconomic models.

\section{Demand-Pull Inflation}
The search for indicators from the demand-pull side of the inflation question brings practitioners back to the relationship between actual and potential real GDP and industrial capacity utilization. The higher the rate of capacity utilization or the closer actual GDP is to potential, the more likely an economy will suffer shortages, bottlenecks, a general inability to satisfy demand, and hence, price increases. The more an economy operates below its potential or the lower the rate of capacity utilization, the less such supply pressure will exist and the greater likelihood of a slowdown in inflation, or outright deflation. In addition to these macro indicators, practitioners will also look for signs of inflationary pressure in commodity prices-in part because they are a cost to business, but more as a general sign of excess demand. For an individual economy, observations of commodity prices could be misleading because commodities trade in a global market and accordingly reflect global economic conditions more than those in an individual economy.

Taking a different perspective, Monetarists contend that inflation is fundamentally a monetary phenomenon. A surplus of money, they argue, will inflate the money price of everything in the economy. Stated in terms of straightforward supply and demand relationships, a surplus of money would bring down its value just as a surplus in any market would bring down the price of the product in excess. Because the price of money is stated in terms of the products it can buy, its declining value would have an expression in generally higher prices - that is, in inflation. This Monetarist argument, as it is called, finds a simpler expression in the old saying: "Inflation results when too much money chases too few goods." Although it seems distant from other explanations of inflation, in practice it is not that distinct. Excess money causes inflationary pressure by increasing liquidity, which ultimately causes a rapid rise in demand. In this sense, the Monetarist argument is a special case under the more general heading of demand-pull concepts of inflation. The practical distinction between the monetarist and other approaches is in identifying the initial cause of the demand excess.

Practitioners can track this effect by examining various money supply indicators, usually provided by the central bank. To detect inflationary or deflationary pressure, practitioners note acceleration or deceleration in monetary growth based on past trends. Obviously in the absence of a special explanation, acceleration signals the potential for inflationary pressure. In applying this approach, practitioners also compare monetary growth with the growth of the nominal economy, represented by nominal GDP. If monetary growth is outpacing the growth of the nominal economy, there is deemed to be inflationary potential. This is especially the case if monetary growth has also accelerated from its trend. There is a disinflationary or deflationary potential if monetary growth lags the economy's rate of expansion, especially if it has also decelerated from its trend.

\section{INFLATION AND VELOCITY}
Some practitioners view the likelihood of inflationary pressure from the vantage point of the ratio of nominal GDP to money supply, commonly called the "velocity of money." If this ratio remains stable around a constant or a historical trend, they see reason to look for relative price stability. If velocity falls, it could suggest a surplus of money that might have inflationary potential, but much depends on why it has declined. If velocity has fallen because a cyclical correction has brought down the GDP numerator relative to the money denominator, then practitioners view prospects as more likely to lead to a cyclical upswing to reestablish the former relationship than inflationary pressure. If velocity has fallen, however, because of an increase in the money denominator, then inflationary pressure becomes more likely. If velocity rises, financial analysts might be concerned about a shortage of money in the economy and disinflation or deflation. The 2008-2009 global recession and financial crisis offers an extreme example of these velocity ambiguities. As the global economy slipped into recession, which held back the GDP numerator in velocity measures, central banks, most notably the Federal Reserve in the United States, tried to help financial institutions cope by injecting huge amounts of money into their respective financial systems, raising the velocity denominator. Velocity measures plummeted accordingly. The expectation is that subsequent GDP growth as economies and financial markets heal will bring velocity back to a more normal level and trend. That said, the fear is that the monetary surge will, over the very long run, lead to inflation. For policy makers, this situation has created a very difficult policy choice. On the one side, they need to sustain the supply of money to help their respective economies cope with the aftereffects of the financial crisis. On the other side, they need ultimately to withdraw any monetary excess to preclude potential inflationary pressures.

\section{Inflation Expectations}
Beyond demand-pull, monetary, and cost-push inflation considerations, practitioners also need to account for the effect of inflation expectations. Once inflation becomes embedded in an economy, businesses, workers, consumers, and economic actors of every kind begin to expect it and build those expectations into their actions. This reaction, in turn, creates an inflationary momentum of its own. Such expectations give inflation something of a self-sustaining character and cause it to persist in an economy even after its initial cause has disappeared. High inflation rates persisted in the 1970s and early 1980s in Europe and the United States on the basis of expectations-even after these economies had sunk into recession. The resulting slow or negative economic growth combined with high unemployment and rising inflation was termed "stagflation."

Measuring inflation expectations is not easy. Some practitioners gauge expectations by relying on past inflation trends and on the assumption that market participants largely extrapolate their past experiences. In some markets, surveys of inflation expectations are available, although these are often biased by the way the questions are asked. Another indicator of inflation expectations becomes available when governments issue bonds, such as Treasury Inflation-Protected Securities (TIPS), that adjust in various ways to compensate holders for inflation. By comparing the interest available on these bonds with other government bonds that do not offer such inflation-linked adjustments, practitioners can gauge the general level of inflation expectations among market participants and factor it into their own inflation forecasts and strategies.

For example, if today's yield on the 10-year nominal bond of a certain country is $3.5 \%$ and the yield on the 10-year inflation-protected bond of the same country is $1.5 \%$, we infer that the market is pricing in a $3.5 \%-1.5 \%=2 \%$ average annual inflation over the next 10 years. However, this calculation needs to be treated with caution because the market for inflation-linked bonds is relatively small and thus yields can be influenced by other factors, such as the very strong demand from US pension funds seeking to match their liabilities.

\section{EXAMPLE 14}
\section{Inflation}
\begin{enumerate}
  \item To examine whether there is inflationary pressure caused by rising costs, an analyst will most likely gather data on:
\end{enumerate}

A. the growth rates of money supply and nominal GDP.

B. the unemployment rate, the NAIRU, and productivity growth.

C. commodity prices, past inflation trends, and expected inflation surveys.

\section{Solution to 1:}
$\mathrm{B}$ is correct. Comparing the current unemployment rate with the NAIRU and productivity growth with the wage growth can help an analyst determine whether inflation may be rising because of higher costs (cost-push inflation). Comparing the monetary growth with nominal GDP growth is helpful to determine whether high demand is creating inflationary pressure (demand-pull inflation). Commodity price increases could be an indicator of either cost-push or demand-pull inflation but may contain limited information in some situations. Past inflation trends and surveys on inflation expectations can help to gauge expected inflation rates; inflation expectations can be a driver of inflation even in the absence of the original underlying cause.

\begin{enumerate}
  \setcounter{enumi}{1}
  \item The most recent macroeconomic data for an economy are given in the following table:
\end{enumerate}

\begin{center}
\begin{tabular}{lc}
Variable & Value \\
\hline
Hourly wage growth rate & $3.4 \%$ \\
Unit labor cost growth rate & $-0.25 \%$ \\
Nominal GDP growth rate & $3.4 \%$ \\
Money supply growth rate & $6.7 \%$ \\
Implied inflation rate from government issued inflation-linked & $2.2 \%$ \\
securities &  \\
\hline
\end{tabular}
\end{center}

Based on the information in the table, an analyst will conclude that current inflation pressure in this economy is most likely caused by:
A. rising wages.
B. rising inflation expectations.
C. bottlenecks in increasing supply to satisfy demand.

\section{Solution to 2:}
$C$ is correct. The table shows that growth in the money supply has outpaced nominal GDP growth, which can result in too much money chasing too few goods. In other words, inflation pressure results from demand beyond the economy's current capacity to produce. Although the wage rate is rising, the negative unit labor cost growth rate indicates an increase in productivity. Thus, it is unlikely the economy will experience cost-push inflation. The implied inflation rate is very modest, which is unlikely to lead to a rising inflation rate.

\begin{enumerate}
  \setcounter{enumi}{2}
  \item Cost-push inflation most likely occurs when:
A. unemployment rates are low.
B. unemployment rates are high.
C. unemployment is either high or low.
\end{enumerate}

\section{Solution to 3:}
A is correct. When unemployment is below the NAIRU, there is a shortage of labor that pushes up labor cost.

\begin{enumerate}
  \setcounter{enumi}{3}
  \item Unit labor costs measure:
A. hourly wage rates.
B. total labor compensation per hour.
C. a combination of hourly wages and output.
\end{enumerate}

\section{Solution to 4:}
$C$ is correct. Unit labor costs reflect the labor cost in each unit of output.

\begin{enumerate}
  \setcounter{enumi}{4}
  \item Demand-pull inflation:
A. is a discredited concept.
B. depends on the movements in commodity prices.
C. reflects the state of economic activity relative to potential.
\end{enumerate}

\section{Solution to 5:}
$C$ is correct. When the economy is operating above its potential capacity allowed by the resources available, inflation will start to rise.

\begin{enumerate}
  \setcounter{enumi}{5}
  \item Monetarists believe inflation:
A. reflects the growth of money.
B. is driven by the level of interest rates.
C. is largely a cost-push phenomenon.
\end{enumerate}

\section{Solution to 6:}
A is correct. Monetarists emphasize the role of money growth in determining the inflation rate, especially in the long run. As Milton Friedman famously put it: "Inflation is always and everywhere a monetary phenomenon."

\begin{enumerate}
  \setcounter{enumi}{6}
  \item The inflationary potential of a particular inflation rate depends on the economy's NAIRU or NARU, which, in turn, depends in part on:
\end{enumerate}

A. the intensity of past cyclical swings.

B. the bargaining power of trade unions.

C. the skill set of the labor force relative to the economy's industrial mix.

Solution to 7:

$\mathrm{C}$ is correct. If the skill set of a large part of the labor force cannot satisfy the hiring need from the employers, the NAIRU of such an economy can be quite high.

\begin{enumerate}
  \setcounter{enumi}{7}
  \item Which of the following is not a problem with the NARU and NAIRU?
\end{enumerate}

A. They are not observable directly.

B. They work only in monetarist models.

C. They change over time given changes in technology and economic structure.

Solution to 8:

$B$ is correct. The NAIRU or NARU reflects the potential of an economy and thus cannot be directly observed from the economic data. They also change over time depending on technological progress and social factors.

\section{SUMMARY}
\begin{itemize}
  \item Business cycles are recurrent expansions and contractions in economic activity affecting broad segments of the economy.

  \item Business cycles are a fundamental feature of market economies, but their amplitude and/or length vary considerably.

  \item Business cycles can be split into many different phases. The investment industry typically refers to four phases of the cycle: recovery, expansion, slowdown, and contraction, with the peak in output occurring during the slowdown phase and the trough in output occurring in the recovery phase.

  \item Classical cycle refers to fluctuations in the level of economic activity when measured by GDP in volume terms.

  \item Growth cycle refers to fluctuations in economic activity around the long-term potential trend growth level with focus on how much actual economic activity is below or above trend growth in economic activity.

  \item Growth rate cycle refers to fluctuations in the growth rate of economic activity.

  \item Credit cycles describe the changing availability-and pricing-of credit. When the economy is strong or improving, the willingness of lenders to extend credit, and on favorable terms, is high. When the economy is weak or weakening, lenders pull back, or "tighten" credit, by making it less available and more expensive, contributing to asset values and further economic weakness and higher defaults.

  \item Credit cycles are relevant due to the importance of credit in the financing of construction and the purchase of property.

  \item The extent of business cycle fluctuations and the duration of recessions and recoveries are often shaped by linkages between business and credit cycles.

  \item Key economic variables change through the business cycle. The level of business investment shows significant changes over the cycle.

  \item Employment levels follow the cycle with a delay as companies initially use overtime before hiring after the onset of recovery and then reduce overtime before reducing employment as the economy passes its peak and enters contraction.

  \item Consumer spending, the largest component of output, follows cyclical patterns as workers make decisions based on their levels of income, wage growth, and employment outlook.

  \item Spending on consumer durables is the most cyclical consumer activity, followed by services and consumer staples, which are less affected by the cyclicality of the economy.

  \item The size of inventories is small relative to the size of the economy, but they have a much greater effect on economic growth than justified by their relatively small aggregate size relative to the economy as a whole.

  \item Inventory-sales ratio measures the inventories available for sale to the level of sales. Analysts pay attention to inventories to gauge the position of the economy in the cycle.

  \item Neoclassical and Real Business Cycle (RBC) theories focus on fluctuations of aggregate supply (AS). If AS shifts left because of an input price increase or right because of a price decrease or technical progress, the economy will gradually converge to its new equilibrium. Government intervention is generally not necessary because it may exacerbate the fluctuation or delay the convergence to equilibrium.

  \item Keynesian theories focus on fluctuations of aggregate demand (AD). If AD shifts left, Keynesians advocate government intervention to restore full employment and avoid a deflationary spiral.

  \item Monetarists argue that the timing of the impact from government policies is uncertain, and it is generally better to let the economy find its new equilibrium unassisted while ensuring that the money supply is kept growing at an even pace.

  \item Economic indicators are statistics on macroeconomic variables that help in understanding which stage of the business cycle an economy is at. Of particular importance are the leading indicators, which suggest where the economy is likely to be in the near future. No economic indicator is perfect, and many of these statistics are subject to periodic revisions.

  \item Leading economic indicators have turning points that usually precede those of the overall economy. They include survey-based indicators, such as (ISM), new orders, and average consumer expectations. They also include average weekly hours, initial claims for unemployment insurance, new building permits, the stock market index, and the difference between yields on short-term and long-term bonds. - Coincident economic indicators have turning points that are usually close to those of the overall economy. They are believed to have value for identifying the economy's present state. They include industrial production indexes, manufacturing and trade sales indexes, aggregate real personal income, and non-agricultural employment.

  \item Lagging economic indicators have turning points that take place later than those of the overall economy; they change after a trend has been established. They include average duration of unemployment, inventory-sales ratio, change in unit labor costs, average bank prime lending rate, commercial and industrial loans outstanding, ratio of consumer installment debt to income, and change in consumer price index for services.

  \item Policy makers and market practitioners use real-time monitoring of economic and financial variables to continuously assess current conditions and produce a nowcast. Nowcasting produces an estimate of the present state of the economy. It is useful because the actual data on such measures as GDP are only published with delay, after the end of the time period under consideration.

  \item Unemployed are those people who are actively seeking employment but are currently without a job.

  \item Long-term unemployed are those who have been out of work for a long time (more than three to four months in many countries) but are still looking for a job.

  \item Frictionally unemployed are those people who are not working at the time of filling out the statistical survey because they are taking time to search for a job that matches their skills, interests, and other preferences better than what is currently available, or people who have left one job and are about to start another job.

  \item There are different types of inflation. Hyperinflation indicates a high (e.g., $100 \%$ annual) and increasing rate of inflation; deflation indicates a negative inflation rate (prices decrease); imported inflation is associated with increasing cost of inputs that come from abroad; demand inflation is caused by constraints in production that prevent companies from making as many goods as the market demands (it is sometimes called wartime inflation, because in times of war, goods tend to be rationed).

  \item Inflation is measured by many indexes. Consumer price indexes reflect the prices of a basket of goods and services that is typically purchased by a normal household. Producer price indexes measure the cost of a basket of raw materials, intermediate inputs, and finished products. GDP deflators measure the price of the basket of goods and services produced within an economy in a given year. Core indexes exclude volatile items, such as agricultural products and energy, whose prices tend to vary more than other goods.

  \item Price levels are affected by real factors and monetary factors. Real factors include aggregate supply (an increase in supply leads to lower prices) and aggregate demand (an increase in demand leads to higher prices). Monetary factors include the supply of money (i.e., more money circulating, if the economy is in equilibrium, will lead to higher prices) and the velocity of money (i.e., higher velocity, if the economy is in equilibrium, will lead to higher prices).

  \item Economists describe two types of inflation: cost-push, in which rising costs, usually wages, compel businesses to raise prices generally; and demand-pull, in which increasing demand raises prices generally, which then are reflected in a business's costs as workers demand wake hikes to catch up with the rising cost of living.

\end{itemize}

\section{REFERENCES}
United Nations 2015. Handbook on Economic Tendency Surveys. United Nations Statistical Papers Series M, no. 96. \href{https://unstats.un.org/unsd/nationalaccount/docs/ETS_Handbook_}{https://unstats.un.org/unsd/nationalaccount/docs/ETS\_Handbook\_} wCover.pdf.

\section{PRACTICE PROBLEMS}
\begin{enumerate}
  \item The characteristic business cycle patterns of trough, expansion, peak, and contraction are:
\end{enumerate}

A. periodic.

B. recurrent.

C. of similar duration.

\begin{enumerate}
  \setcounter{enumi}{1}
  \item During the contraction phase of a business cycle, it is most likely that:
\end{enumerate}

A. inflation indicators are stable.

B. aggregate economic activity relative to potential output is decreasing.

C. investor preference for government securities declines.

\begin{enumerate}
  \setcounter{enumi}{2}
  \item An economic peak is most closely associated with:
A. accelerating inflation.
B. stable unemployment.
C. declining capital spending.

  \item Based on typical labor utilization patterns across the business cycle, productivity (output per hours worked) is most likely to be highest:

\end{enumerate}

A. at the peak of a boom.

B. into a maturing expansion

C. at the bottom of a recession.

\begin{enumerate}
  \setcounter{enumi}{4}
  \item As the expansion phase of the business cycle advances from early stage to late stage, businesses most likely experience a decrease in:
\end{enumerate}

A. labor costs.

B. capital investment.

C. availability of qualified workers.

\begin{enumerate}
  \setcounter{enumi}{5}
  \item An analyst writes in an economic report that the current phase of the business cycle is characterized by accelerating inflationary pressures and borrowing by companies. The analyst is most likely referring to the:
\end{enumerate}

A. peak of the business cycle.

B. contraction phase of the business cycle.

C. early expansion phase of the business cycle.

\begin{enumerate}
  \setcounter{enumi}{6}
  \item With which sector of the economy would analysts most commonly associate credit cycles?
\end{enumerate}

A. Exports
B. Construction and purchases of property
C. Food retail

\begin{enumerate}
  \setcounter{enumi}{7}
  \item The reason analysts follow developments in the availability of credit is that:
A. loose private sector credit may contribute to the extent of asset price and real estate bubbles and subsequent crises.
B. loose credit helps reduce the extent of asset price and real estate bubbles.
C. credit cycles are of same length and depth as business cycles.

  \item In a recession, companies are most likely to adjust their stock of physical capital by:
A. selling it at fire sale prices.
B. not maintaining equipment.
C. quickly canceling orders for new construction equipment.

  \item The inventory-sales ratio is most likely to be rising:
A. as a contraction unfolds.
B. partially into a recovery.
C. near the top of an economic cycle.

  \item A decrease in a country's total imports is most likely caused by:
A. an increase in the pace of domestic GDP growth.
B. a cyclical downturn in the economies of primary trading partners.
C. persistent currency depreciation relative to primary trading partners.

  \item Monetarists favor a limited role for the government because they argue:
A. government policy responses may lag.
B. firms take time to adjust to systemic shocks to the economy.
C. resource use is efficient with marginal revenue and cost equal.

  \item A national government responds to a severe recession by funding numerous infrastructure projects using deficit spending. Which school of economic thought is most consistent with such action?
A. Keynesian
B. Monetarist
C. Neoclassical

  \item According to Real Business Cycle models, an economic contraction is most likely caused by:
A. sticky wages.
B. rising energy prices. C. a contraction in the money supply.

  \item The Austrian economic school attributes the primary cause of the business cycle to:

\end{enumerate}

A. misguided government intervention.

B. the creative destruction of technological progress.

C. sticky price and wage expectations that exaggerate trends.

\begin{enumerate}
  \setcounter{enumi}{15}
  \item According to the Austrian school, the most appropriate government response to an economic recession is to:
\end{enumerate}

A. allow the market to adjust naturally.

B. maintain steady growth in the money supply.

C. decrease the market rate of interest below its natural value.

\begin{enumerate}
  \setcounter{enumi}{16}
  \item Which of the following statements is the best description of the characteristics of economic indicators?
\end{enumerate}

A. Leading indicators are important because they track the entire economy.

B. Lagging indicators, in measuring past conditions, do not require revisions.

C. A combination of leading and coincident indicators can offer effective forecasts.

\begin{enumerate}
  \setcounter{enumi}{17}
  \item When the spread between 10-year US Treasury yields and the federal funds rate narrows and at the same time the prime rate stays unchanged, this mix of indicators most likely forecasts future economic:
\end{enumerate}

A. growth.

B. decline.

C. stability.

\begin{enumerate}
  \setcounter{enumi}{18}
  \item Current economic statistics indicating little change in services inflation, rising residential building permits, and increasing average duration of unemployment are best interpreted as:
\end{enumerate}

A. conflicting evidence about the direction of the economy.

B. evidence that a cyclical upturn is expected to occur in the future.

C. evidence that a cyclical downturn is expected to occur in the future.

\begin{enumerate}
  \setcounter{enumi}{19}
  \item If relative to prior values of their respective indicators, the inventory-sales ratio has risen, unit labor cost is stable, and real personal income has decreased, it is most likely that a peak in the business cycle:
\end{enumerate}

A. has occurred.

B. is just about to occur.

C. will occur sometime in the future.

\begin{enumerate}
  \setcounter{enumi}{20}
  \item When aggregate real personal income, industrial output, and the S\&P 500 Index all increase in a given period, it is most accurate to conclude that a cyclical upturn is:
A. occurring.
B. about to end.
C. about to begin.

  \item Which of the following is most likely to increase after an increase in aggregate real personal income?
A. Equity prices
B. Building permits for new private housing units
C. The ratio of consumer installment debt to income

  \item Which of the following indicators is most appropriate in predicting a turning point in the economy?
A. The Industrial Production Index
B. The average bank prime lending rate
C. Average weekly hours, manufacturing

  \item The unemployment rate is considered a lagging indicator because:
A. new job types must be defined to count their workers.
B. multi-worker households change jobs at a slower pace.
C. businesses are slow to hire and fire due to related costs.

  \item During an economic recovery, a lagging unemployment rate is most likely attributable to:
A. businesses quickly rehiring workers.
B. new job seekers entering the labor force.
C. underemployed workers transitioning to higher-paying jobs.

  \item A decrease in both the labor force participation ratio and the unemployment rate is most likely caused by:

\end{enumerate}

A. an increase in discouraged workers.

B. an increase in underemployed workers.

C. a decrease in voluntarily unemployed persons.

\begin{enumerate}
  \setcounter{enumi}{26}
  \item The treasury manager of a large company has recently left his position to accept a promotion with a competitor six months from now. A statistical employment survey conducted now should categorize the status of the former treasury manager as:
\end{enumerate}

A. underemployed.

B. voluntarily unemployed. C. frictionally unemployed.

\begin{enumerate}
  \setcounter{enumi}{27}
  \item The discouraged worker category is defined to include people who:
A. are overqualified for their job.
B. could look for a job but choose not to.
C. currently look for work without finding it.

  \item The category of persons who would be most likely to be harmed by an increase in the rate of inflation is:
A. homeowners with fixed 30-year mortgages.
B. retirees relying on a fixed annuity payment.
C. workers employed under contracts with escalator clauses.

  \item The term that describes when inflation declines but nonetheless remains at a positive level is:
A. deflation.
B. stagflation.
C. disinflation.

  \item Deflation is most likely to be associated with:
A. a shortage of government revenue.
B. substantial macroeconomic contraction.
C. explicit monetary policy to combat inflation.

  \item The least likely consequence of a period of hyperinflation is the:
A. reduced velocity of money.
B. increased supply of money.
C. possibility of social unrest.

\end{enumerate}

The following information relates to questions 33-34

\section{Exhibit 22: Exhibit 1: Consumption Baskets and Prices over Two Months}
\begin{center}
\begin{tabular}{lcccc}
\hline
 & \multicolumn{2}{c}{November $\mathbf{2 0 1 0}$} & \multicolumn{2}{c}{December $\mathbf{2 0 1 0}$} \\
\hline
Goods & Quantity & Price & Quantity & Price \\
\hline
Sugar & $70 \mathrm{~kg}$ & $€ 0.90 / \mathrm{kg}$ & $120 \mathrm{~kg}$ & $€ 1.00 / \mathrm{kg}$ \\
\hline
\end{tabular}
\end{center}

\begin{center}
\begin{tabular}{lcccc}
 & \multicolumn{2}{c}{November $\mathbf{2 0 1 0}$} & \multicolumn{2}{c}{December $\mathbf{2 0 1 0}$} \\
\hline
Goods & Quantity & Price & Quantity & Price \\
\hline
Cotton & $60 \mathrm{~kg}$ & $€ 0.60 / \mathrm{kg}$ & $50 \mathrm{~kg}$ & $€ 0.80 / \mathrm{kg}$ \\
\hline
\end{tabular}
\end{center}

\begin{enumerate}
  \setcounter{enumi}{32}
  \item Assuming the base period for 2010 consumption is November and the initial price index is set at 100 , then the inflation rate after calculating the December price index as a Laspeyres index is closestto:
A. $19.2 \%$.
B. $36.4 \%$.
C. $61.6 \%$.

  \item For the December consumption basket in Exhibit 1, the value of the Paasche index is closest to:
A. 116
B. 148 .
C. 160 .

  \item A central bank will most likely allow the economy to self-correct in periods of:
A. high inflation, fast economic growth, and low unemployment.
B. low inflation, slow economic growth, and high unemployment.
C. high inflation, slow economic growth, and high unemployment.

  \item Disinflation is best described as a:
A. decline in price levels.
B. negative inflation rate.
C. decline in the inflation rate.

  \item The characteristic of national consumer price indexes that is most likely shared across major economies worldwide is:
A. the geographic areas covered in their surveys.
B. the weights they place on covered goods and services.
C. their use in the determination of macroeconomic policy.

  \item Of the following statements regarding the Producer Price Index (PPI), which is the least likely? The PPI:
A. can influence the future CPI.
B. category weights can vary more widely than analogous CPI terms.
C. is used more frequently than CPI as a benchmark for adjusting labor con- tract payments. 39. The following presents selected commodity price data for July-August 20X0:

\end{enumerate}

\begin{center}
\begin{tabular}{lcccc}
 & \multicolumn{2}{c}{July 20X0} & \multicolumn{2}{c}{August 20X0} \\
\hline
Goods & Quantity & Price & Quantity & Price \\
\hline
Milk & 18 liters & $€ 1.00 /$ liter & 17 liters & $€ 1.00 /$ liter \\
Orange juice & 6 liters & $€ 2.00 /$ liter & 4 liters & $€ 2.50 /$ liter \\
\hline
\end{tabular}
\end{center}

Given the consumption basket and prices presented, which type of price index will result in the highest calculated inflation rate over a two-month time period?

A. One that uses a current consumption basket

B. One that uses a constant consumption basket

C. One reflecting substitutions made by consumers over time

\begin{enumerate}
  \setcounter{enumi}{39}
  \item The inflation rate most likely relied on to determine public economic policy is:
A. core inflation.
B. headline inflation.
C. index of food and energy prices.

  \item What is the most important effect of labor productivity in a cost-push inflation scenario?

\end{enumerate}

A. Rising productivity indicates a strong economy and a bias towards inflation.

B. The productivity level determines the economy's status relative to its "natural rate of unemployment."

C. As productivity growth proportionately exceeds wage increases, product price increases are less likely.

\begin{enumerate}
  \setcounter{enumi}{41}
  \item A product is part of a price index based on a fixed consumption basket. If, over time, the product's quality improves while its price stays constant, the measured inflation rate is most likely:
\end{enumerate}

A. unaffected.

B. biased upward.

C. biased downward.

\begin{enumerate}
  \setcounter{enumi}{42}
  \item A price index of goods and services that excludes food and energy is most likely used to calculate:
A. core inflation.
B. the GDP deflator.
C. headline inflation.

  \item An economist expects the following:

\end{enumerate}

\begin{itemize}
  \item The decline in the unemployment rate will result in higher revenues for home retailers. - A tighter labor market will put upward pressure on wages, compelling home retailers to raise prices.
\end{itemize}

Which type of inflation best corresponds to the economist's expectations?
A. Stagflation
B. Cost-push inflation
C. Demand-pull inflation

\section{SOLUTIONS}
\begin{enumerate}
  \item B is correct. The stages of the business cycle occur repeatedly over time.

  \item B is correct. The net trend during contraction is negative.

  \item A is correct. Inflation is rising at peaks.

  \item C is correct. At the end of a recession, firms will run "lean production" to generate maximum output with the fewest number of workers.

  \item C is correct. When an economy's expansion is well established, businesses often have difficulty finding qualified workers.

  \item A is correct. Accelerating inflation and rapidly expanding capital expenditures typically characterize the peak of the business cycle. During such times, many businesses finance their capital expenditures with debt to expand their production capacity.

  \item B is correct. Credit cycles are associated with availability of credit, which is important in the financing of construction and the purchase of property.

  \item A is correct. Studies have shown that loose credit conditions contribute to the extent of asset price and real estate bubbles that tend to be followed by crises.

  \item B is correct. Physical capital adjustments to downturns come through aging of equipment plus lack of maintenance.

  \item $\mathrm{C}$ is correct. Near the top of a cycle, sales begin to slow before production is cut, leading to an increase in inventories relative to sales.

  \item $\mathrm{C}$ is correct. When a nation's currency depreciates, domestic goods seem cheaper than foreign goods, placing downward pressure on demand for imports. When the depreciation persists for some time, the country's total imports are likely to decrease.

  \item A is correct. Monetarists caution that policy effects can occur long after the need for which they were implemented is no longer an issue.

  \item A is correct. Keynesian economics is based on government intervention in the form of fiscal policy. The national government responds to the recession by using deficit spending to fund infrastructure projects.

  \item B is correct. Real Business Cycle models conclude that expansions and contractions of the economy are responses to external shocks, such as supply shocks arising from advances in technology or changes in the relative prices of inputs (e.g., energy prices). An increase in energy prices shifts short-run aggregate supply to the left, resulting in higher prices and lower GDP.

  \item A is correct. Austrian economists see monetary policy mistakes as leading to booms and busts.

  \item A is correct. Austrian economists advocate limited government intervention in the economy. They advise that the best thing to do in a recession is to allow the necessary market adjustment to take place.

  \item $\mathrm{C}$ is correct. While no single indicator is definitive, a mix of them-which can be affected by various economic determinants-can offer the strongest signal of performance.

  \item B is correct. The narrowing spread of this leading indicator foretells a drop in short-term rates and a fall in economic activity. The prime rate is a lagging indicator and typically moves after the economy turns.

  \item B is correct. Rising building permits-a leading indicator-indicate that an upturn is expected to occur or continue. Increasing average duration of unemployment-a lagging indicator-indicates that a downturn has occurred, whereas the lack of any change in services inflation-also a lagging indicator-is neither negative nor positive for the direction of the economy. Taken together, these statistics indicate that a cyclical upturn may be expected to occur.

  \item A is correct. Both inventory-sales and unit labor costs are lagging indicators that decline somewhat after a peak. Real personal income is a coincident indicator that by its decline shows a slowdown in business activity.

  \item A is correct. Aggregate real personal income and industrial output are coincident indicators, whereas the S\&P 500 is a leading indicator. An increase in aggregate personal income and industrial output signals that an expansion is occurring whereas an increase in the S\&P 500 signals that an expansion will occur or is expected to continue. Taken together, these statistics indicate that a cyclical upturn is occurring.

  \item $\mathrm{C}$ is correct. Aggregate real personal income is a coincident indicator of the business cycle, and the ratio of consumer installment debt to income is a lagging indicator. Increases in the ratio of consumer installment debt follow increases in average aggregate income during the typical business cycle.

  \item C is correct. Leading economic indicators have turning points that usually precede those of the overall economy. Average weekly hours, manufacturing is a leading economic indicator. The Industrial Production Index is a coincident economic indicator, and the average bank prime lending rate is a lagging economic indicator.

  \item C is correct. This effect makes unemployment rise more slowly as recessions start and fall more slowly as recoveries begin.

  \item B is correct. In an economic recovery, new job seekers return to the labor force, and because they seldom find work immediately, their return may initially raise the unemployment rate.

  \item A is correct. Discouraged workers have given up seeking employment and are statistically outside the labor force. Therefore, an increase in discouraged workers will decrease the labor force and thus the labor participation ratio, which is the ratio of labor force to total working age population. Additionally, an increase in discouraged workers will decrease the unemployment rate because discouraged workers are not counted in the official unemployment rate.

  \item $C$ is correct. Frictionally unemployed people are not working at the time of the employment survey but have recently left one job and are about to start another job. The frictionally unemployed have a job waiting for them and are not $100 \%$ unemployed, it is just that they have not started the new job yet. Although the treasury manager has left his current employment, he has accepted a new position at another firm that starts in six months.

  \item B is correct. Discouraged workers are defined as persons who have stopped look- ing for work and are outside the labor force.

  \item B is correct. With inflation, a fixed amount of money buys fewer goods and services, thus reducing purchasing power.

  \item $\mathrm{C}$ is correct. Disinflation is known as a reduction of inflation from a higher to lower, but still above zero, level.

  \item B is correct. Deflation is connected to a vicious cycle of reduced spending and higher unemployment.

  \item A is correct. In hyperinflation, consumers accelerate their spending to beat price increases, and money circulates more rapidly.

  \item A is correct. The Laspeyres index is calculated with these inputs:

\end{enumerate}

\begin{itemize}
  \item November consumption bundle: $70 \times 0.9+60 \times 0.6=99$

  \item December consumption bundle: $70 \times 1+60 \times 0.8=118$

  \item December price index: $(118 / 99) \times 100=119.19$

  \item Inflation rate: $(119.19 / 100)-1=0.1919=19.19 \%$

\end{itemize}

\begin{enumerate}
  \setcounter{enumi}{33}
  \item A is correct. The Paasche index uses the current product mix of consumption combined with the variation of prices. For December, its value is
\end{enumerate}

$(120 \times 1+50 \times 0.8) /(120 \times 0.9+50 \times 0.6)=(160 / 138) \times 100=115.9$.

\begin{enumerate}
  \setcounter{enumi}{34}
  \item $\mathrm{C}$ is correct. This scenario is often referred to as stagflation. Here, the economy is likely to be left to self-correct because no short-term economic policy is thought to be effective.

  \item $\mathrm{C}$ is correct. Disinflation is a decline in the inflation rate-for example, from $7 \%$ to $4 \%$.

  \item C is correct. Central banks typically use consumer price indexes to monitor inflation and evaluate their monetary policies.

  \item $\mathrm{C}$ is correct. The CPI is typically used for this purpose, while the PPI is more closely connected to business contracts.

  \item B is correct. The inflation rate calculated by using a constant consumption basket (the Laspeyres index) is $10 \%$, derived as follows:

\end{enumerate}

July 20X0 consumption basket $=(18 \times € 1)+(6 \times € 2)=€ 30$

August 20X0 consumption basket $=(18 \times € 1)+(6 \times € 2.5)=€ 33$

Value of the Laspeyres index (IL) $=(€ 33 / € 30) \times 100=€ 110$

Inflation rate $=(110 / 100)-1=0.10=10 \%$

The inflation rate calculated using a current consumption basket (the Paasche index) is $8 \%$, derived as follows:

July $20 \mathrm{X} 0$ consumption basket $=(17 \times € 1)+(4 \times € 2)=€ 25$

August 20X0 consumption basket $=(17 \times € 1)+(4 \times € 2.5)=€ 27$

Value of the Paasche index (IP) $=(€ 27 / € 25) \times 100=€ 108$

Inflation rate $=(108 / 100)-1=0.08=8 \%$

The inflation rate calculated by "chaining" the monthly prices of consumption baskets as they change over time (the Fisher index) is derived as follows:

Value of the Fisher index $=\sqrt{I_{P} \times I_{L}}$

Value of the Fisher Index $=\sqrt{€ 110 \times € 108}=€ 108.99$

Inflation rate $=(108.99 / 100)-1=0.0899=8.99 \%$

\begin{enumerate}
  \setcounter{enumi}{39}
  \item A is correct. Core inflation is less volatile since it excludes food and energy prices and therefore will not be as likely to lead to policy overreactions when serving as a target.

  \item C is correct. For productivity, or output per hour, the faster that it can grow, the further that wages can rise without putting pressure on business costs per unit of output.

  \item B is correct. As the quality of a product improves, it satisfies people's needs and wants better. The measured inflation rate is skewed higher than otherwise unless an adjustment is made for the increase in the quality of the good. Even if the good's price had increased over time, the improvements in quality would still bias the measured inflation rate upward.

  \item A is correct. A price index of goods and services that excludes food and energy is used to calculate core inflation. Policy makers often use core inflation when reading the trend in the economy and making economic policies. The reason is because policy makers are trying to avoid overreaction to short-term fluctuations in prices as a result of short-term changes in supply and demand.

  \item B is correct. Cost-push inflation refers to the situation in which rising costs, usually wages, compel businesses to raise prices.

\end{enumerate}

\section*{LEARNING MODULE }
\section{5}
\section{Monetary and Fiscal Policy}
Andrew Clare, PhD, and Stephen Thomas, PhD, are at Cass Business School (United Kingdom).

\section{LEARNING OUTCOME}
\begin{center}
\begin{tabular}{|c|c|}
\hline
Mastery & The candidate should be able to: \\
\hline
 & compare monetary and fiscal policy \\
\hline
 & describe functions and definitions of money \\
\hline
 & explain the money creation process \\
\hline
 & describe theories of the demand for and supply of money \\
\hline
 & describe the Fisher effect \\
\hline
 & describe roles and objectives of central banks \\
\hline
 & contrast the costs of expected and unexpected inflation \\
\hline
 & describe tools used to implement monetary policy \\
\hline
 & describe the monetary transmission mechanism \\
\hline
 & $\begin{array}{l}\text { explain the relationships between monetary policy and economic } \\ \text { growth, inflation, interest, and exchange rates }\end{array}$ \\
\hline
 & describe qualities of effective central banks \\
\hline
 & $\begin{array}{l}\text { contrast the use of inflation, interest rate, and exchange rate } \\ \text { targeting by central banks }\end{array}$ \\
\hline
 & $\begin{array}{l}\text { determine whether a monetary policy is expansionary or } \\ \text { contractionary }\end{array}$ \\
\hline
 & describe limitations of monetary policy \\
\hline
 & describe roles and objectives of fiscal policy \\
\hline
 & $\begin{array}{l}\text { describe the arguments about whether the size of a national debt } \\ \text { relative to GDP matters }\end{array}$ \\
\hline
 & $\begin{array}{l}\text { describe tools of fiscal policy, including their advantages and } \\ \text { disadvantages }\end{array}$ \\
\hline
\end{tabular}
\end{center}

\section{LEARNING OUTCOME}
\begin{center}
\begin{tabular}{c|l}
Mastery & The candidate should be able to: \\
$\square$ & $\begin{array}{l}\text { explain the implementation of fiscal policy and difficulties of } \\ \text { implementation } \\ \text { determine whether a fiscal policy is expansionary or contractionary } \\ \text { explain the interaction of monetary and fiscal policy }\end{array}$ \\
\end{tabular}
\end{center}

\section{1}
\section{INTRODUCTION TO MONETARY AND FISCAL POLICY}
compare monetary and fiscal policy

The economic decisions of households can have a significant impact on an economy. For example, a decision on the part of households to consume more and to save less can lead to an increase in employment, investment, and ultimately profits. Equally, the investment decisions made by corporations can have an important impact on the real economy and on corporate profits. But individual corporations can rarely affect large economies on their own; the decisions of a single household concerning consumption will have a negligible impact on the wider economy.

By contrast, the decisions made by governments can have an enormous impact on even the largest and most developed of economies for two main reasons. First, the public sectors of most developed economies normally employ a significant proportion of the population, and they are usually responsible for a significant proportion of spending in an economy. Second, governments are also the largest borrowers in world debt markets. Exhibit 1 gives some idea of the scale of government borrowing and spending.

\section{Exhibit 1}
\section{Panel A. Central Government Debt to GDP, 2017}
\begin{center}
\includegraphics[max width=\textwidth]{2023_05_04_ccb578c5089fd41615beg-291}
\end{center}

\section{Panel B. Public Sector Spending to GDP, 2017}
\begin{center}
\includegraphics[max width=\textwidth]{2023_05_04_ccb578c5089fd41615beg-291(1)}
\end{center}

Source: IMF, World Economic Outlook Database, April 2018.

Government policy is ultimately expressed through its borrowing and spending activities. In this reading, we identify and discuss two types of government policy that can affect the macroeconomy and financial markets: monetary policy and fiscal policy.

Monetary policy refers to central bank activities that are directed toward influencing the quantity of money and credit in an economy. ${ }^{1}$ By contrast, fiscal policy refers to the government's decisions about taxation and spending. Both monetary and fiscal policies are used to regulate economic activity over time. They can be used to accelerate growth when an economy starts to slow or to moderate growth and activity when an economy starts to overheat. In addition, fiscal policy can be used to redistribute income and wealth.

The overarching goal of both monetary and fiscal policy is normally the creation of an economic environment where growth is stable and positive and inflation is stable and low. Crucially, the aim is therefore to steer the underlying economy so that it does not experience economic booms that may be followed by extended periods of low or negative growth and high levels of unemployment. In such a stable economic

1 Central banks can implement monetary policy almost completely independent of government interference and influence at one end of the scale, or simply as the agent of the government at the other end of the scale. environment, householders can feel secure in their consumption and saving decisions, while corporations can concentrate on their investment decisions, on making their regular coupon payments to their bond holders and on making profits for their shareholders.

The challenges to achieving this overarching goal are many. Not only are economies frequently buffeted by shocks (such as oil price jumps), but some economists believe that natural cycles in the economy also exist. Moreover, there are plenty of examples from history where government policies-either monetary, fiscal, or both-have exacerbated an economic expansion that eventually led to damaging consequences for the real economy, for financial markets, and for investors.

The balance of the reading is organized as follows. Section 2 provides an introduction to monetary policy and related topics. Section 3 presents fiscal policy. The interactions between monetary policy and fiscal policy are the subject of Section 4. A summary and practice problems conclude the reading.

\section{Monetary Policy}
As stated above, monetary policy refers to government or central bank activities that are directed toward influencing the quantity of money and credit in an economy. Before we can begin to understand how monetary policy is implemented, we must examine the functions and role of money. We can then explore the special role that central banks play in today's economies.

\section{EXAMPLE 1}
\section{Monetary and Fiscal Policy}
\begin{enumerate}
  \item Which of the following statements best describes monetary policy? Monetary policy:
\end{enumerate}

A. involves the setting of medium-term targets for broad money aggregates.

B. involves the manipulation by a central bank of the government's budget deficit.

C. seeks to influence the macro economy by influencing the quantity of money and credit in the economy.

\section{Solution to 1:}
$\mathrm{C}$ is correct. Choice $\mathrm{A}$ is incorrect because, although the setting of targets for monetary aggregates is a possible tool of monetary policy, monetary policy itself is concerned with influencing the overall, or macro, economy.

\begin{enumerate}
  \setcounter{enumi}{1}
  \item Which of the following statements best describes fiscal policy? Fiscal policy:
\end{enumerate}

A. is used by governments to redistribute wealth and incomes.

B. is the attempt by governments to balance their budgets from one year to the next.

C. involves the use of government spending and taxation to influence economy activity.

\section*{Solution to 2: }
$\mathrm{C}$ is correct. Note that governments may wish to use fiscal policy to redistribute incomes and balance their budgets, but the overriding goal of fiscal policy is usually to influence a broader range of economic activity.

\section{MONEY: FUNCTIONS, CREATION, AND DEFINITION}
To understand the nature, role, and development of money in modern economies, it is useful to think about a world without money-where to purchase any good or service, an individual would have to "pay" with another good or service. An economy where such economic agents as households, corporations, and even governments pay for goods and services in this way is known as a barter economy. There are many drawbacks to such an economy. First, the exchange of goods for other goods (or services) would require both economic agents in the transaction to want what the other is selling. This means that there has to be a double coincidence of wants. It might also be impossible to undertake transactions where the goods are indivisible-that is, where one agent wishes to buy a certain amount of another's goods, but that agent only has one indivisible unit of another good that is worth more than the good that the agent is trying to buy. Another problem occurs if economic agents do not wish to exchange all of their goods on other goods and services. This may not be a problem, however, when the goods they have to sell can be stored safely so that they retain their value for the future. But if these goods are perishable, they will not be able to store value for their owner. Finally, in a barter economy, there are many measures of value: the price of oranges in terms of pears; of pears in terms of bread; of bread in terms of milk; or of milk in terms of oranges. A barter economy has no common measure of value that would make multiple transactions simple.

\section{The Functions of Money}
The most generic definition of money is that it is any generally accepted medium of exchange. A medium of exchange is any asset that can be used to purchase goods and services or to repay debts. Money can thus eliminate the debilitating double coincidence of the "wants" problem that exists in a barter economy. When this medium of exchange exists, a farmer wishing to sell wheat for wine does not need to identify a wine producer in search of wheat. Instead, he can sell wheat to those who want wheat in exchange for money. The farmer can then exchange this money for wine with a wine producer, who in turn can exchange that money for the goods or services that she wants.

However, for money to act as this liberating medium of exchange, it must possess certain qualities. It must:

i. be readily acceptable,

ii. have a known value,

iii. be easily divisible, iv. have a high value relative to its weight, and

v. be difficult to counterfeit.

Qualities (i) and (ii) are closely related; the medium of exchange will only be acceptable if it has a known value. If the medium of exchange has quality (iii), then it can be used to purchase items of relatively little value and of relatively large value with equal ease. Having a high value relative to its weight is a practical convenience, meaning that people can carry around sufficient wealth for their transaction needs. Finally, if the medium of exchange can be counterfeited easily, then it would soon cease to have a value and would not be readily acceptable as a means of effecting transactions; in other words, it would not satisfy qualities (i) and (ii).

Given the qualities that money needs to have, it is clear why precious metals (particularly gold and silver) often fulfilled the role of medium of exchange in early societies, and as recently as the early part of the twentieth century. Precious metals were acceptable as a medium of exchange because they had a known value, were easily divisible, had a high value relative to their weight, and could not be easily counterfeited.

Thus, precious metals were capable of acting as a medium of exchange. But they also fulfilled two other useful functions that are essential for the characteristics of money. In a barter economy, it is difficult to store wealth from one year to the next when one's produce is perishable, or indeed, if it requires large warehouses in which to store it. Because precious metals like gold had a high value relative to their bulk and were not perishable, they could act as a store of wealth. However, their ability to act as a store of wealth not only depended on the fact that they did not perish physically over time, but also on the belief that others would always value precious metals. The value from year to year of precious metals depended on people's continued demand for them in ornaments, jewellery, and so on. For example, people were willing to use gold as a store of wealth because they believed that it would remain highly valued. However, if gold became less valuable to people relative to other goods and services year after year it would not be able to fulfill its role as a store of value, and as such might also lose its status as a medium of exchange.

Another important characteristic of money is that it can be used as a universal unit of account. As such, it can create a single unitary measure of value for all goods and services. In an economy where gold and silver are the accepted medium of exchange, all prices, debts, and wealth can be recorded in terms of their gold or silver coin exchange value. Money, in its role as a unit of account, drastically reduces the number of prices in an economy compared to barter, which requires that prices be established for a good in terms of all other goods for which it might be exchanged.

In summary, money fulfills three important functions, it:

\begin{itemize}
  \item acts as a medium of exchange;

  \item provides individuals with a way of storing wealth; and

  \item provides society with a convenient measure of value and unit of account.

\end{itemize}

\section{Paper Money and the Money Creation Process}
Although precious metals like gold and silver fulfilled the required functions of money relatively well for many years, and although carrying gold coins around was easier than carrying around one's physical produce, it was not necessarily a safe way to conduct business.

A crucial development in the history of money was the promissory note. The process began when individuals began leaving their excess gold with goldsmiths, who would look after it for them. In turn the goldsmiths would give the depositors a receipt, stating how much gold they had deposited. Eventually these receipts were traded directly for goods and services, rather than there being a physical transfer of gold from the goods buyer to the goods seller. Of course, both the buyer and seller had to trust the goldsmith because the goldsmith had all the gold and the goldsmith's customers had only pieces of paper. These depository receipts represented a promise to pay a certain amount of gold on demand. This paper money therefore became a proxy for the precious metals on which they were based, that is, they were directly related to a physical commodity. Many of these early goldsmiths evolved into banks, taking in excess wealth and in turn issuing promissory notes that could be used in commerce.

In taking in other people's gold and issuing depository receipts and later promissory notes, it became clear to the goldsmiths and early banks that not all the gold that they held in their vaults would be withdrawn at any one time. Individuals were willing to buy and sell goods and services with the promissory notes, but the majority of the gold that backed the notes just sat in the vaults-although its ownership would change with the flow of commerce over time. A certain proportion of the gold that was not being withdrawn and used directly for commerce could therefore be lent to others at a rate of interest. By doing this, the early banks created money.

The process of money creation is a crucial concept for understanding the role that money plays in an economy. Its potency depends on the amount of money that banks keep in reserve to meet the withdrawals of its customers. This practice of lending customers' money to others on the assumption that not all customers will want all of their money back at any one time is known as fractional reserve banking.

We can illustrate how it works through a simple example. Suppose that the bankers in an economy come to the view that they need to retain only 10 percent of any money deposited with them. This is known as the reserve requirement. ${ }^{2}$ Now consider what happens when a customer deposits $€ 100$ in the First Bank of Nations. This deposit changes the balance sheet of First Bank of Nations, as shown in Exhibit 2, and it represents a liability to the bank because it is effectively loaned to the bank by the customer. By lending 90 percent of this deposit to another customer the bank has two types of assets: (1) the bank's reserves of $€ 10$, and (2) the loan equivalent to $€ 90$. Notice that the balance sheet still balances; $€ 100$ worth of assets and $€ 100$ worth of liabilities are on the balance sheet.

Now suppose that the recipient of the loan of $€ 90$ uses this money to purchase some goods of this value and the seller of the goods deposits this $€ 90$ in another bank, the Second Bank of Nations. The Second Bank of Nations goes through the same process; it retains $€ 9$ in reserve and loans 90 percent of the deposit ( $€ 81)$ to another customer. This customer in turn spends $€ 81$ on some goods or services. The recipient of this money deposits it at the Third Bank of Nations, and so on. This example shows how money is created when a bank makes a loan.

\section{Exhibit 2: Money Creation via Fractional Reserve Banking}
First Bank of Nations

\begin{center}
\begin{tabular}{llll}
\hline
Assets &  & Liabilities &  \\
\hline
Reserves & $€ 10$ & Deposits & $€ 100$ \\
Loans & $€ 90$ &  &  \\
\hline
\end{tabular}
\end{center}

Second Bank of Nations

Assets

Liabilities

Reserves

2 This is an example of a voluntary reserve requirement because it is self-imposed. Second Bank of Nations

\begin{center}
\begin{tabular}{|c|c|c|c|}
\hline
\multicolumn{2}{|l|}{Assets} & \multicolumn{2}{|l|}{Liabilities} \\
\hline
Loans & $€ 81$ &  &  \\
\hline
\multicolumn{4}{|c|}{Third Bank of Nations} \\
\hline
\multicolumn{2}{|l|}{Assets} & \multicolumn{2}{|l|}{Liabilities} \\
\hline
Reserves & $€ 8.1$ & Deposits & $€ 81$ \\
\hline
Loans & $€ 72.9$ &  &  \\
\hline
\end{tabular}
\end{center}

This process continues until there is no more money left to be deposited and loaned out. The total amount of money 'created' from this one deposit of $€ 100$ can be calculated as:

New deposit/Reserve requirement $=€ 100 / 0.10=€ 1,000$

It is the sum of all the deposits now in the banking system. You should also note that the original deposit of $€ 100$, via the practice of reserve banking, was the catalyst for $€ 1,000$ worth of economic transactions. That is not to say that economic growth would be zero without this process, but instead that it can be an important component in economic activity.

The amount of money that the banking system creates through the practice of fractional reserve banking is a function of 1 divided by the reserve requirement, a quantity known as the money multiplier. ${ }^{3}$ In the case just examined, the money multiplier is $1 / 0.10=10$. Equation 1 implies that the smaller the reserve requirement, the greater the money multiplier effect.

In our simplistic example, we assumed that the banks themselves set their own reserve requirements. However, in some economies, the central bank sets the reserve requirement, which is a potential means of affecting money growth. In any case, a prudent bank would be wise to have sufficient reserves such that the withdrawal demands of their depositors can be met in stressful economic and credit market conditions.

Later, when we discuss central banks and central bank policy, we will see how central banks can use the mechanism just described to affect the money supply. Specifically, the central bank could, by purchasing $€ 100$ in government securities credited to the bank account of the seller, seek to initiate an increase in the money supply. The central bank may also lend reserves directly to banks, creating excess reserves (relative to any imposed or self-imposed reserve requirement) that can support new loans and money expansion.

\section{EXAMPLE 2}
\section{Money and Money Creation}
\begin{enumerate}
  \item To fulfill its role as a medium of exchange, money should:
A. be a conservative investment.
B. have a low value relative to its weight.
C. be easily divisible and a good store of value.
\end{enumerate}

3 This quantity, known as the simple money multiplier, represents a maximum expansion. To the extent that banks hold excess reserves or that money loaned out is not re-deposited, the money expansion would be less. More complex multipliers incorporating such factors are developed in more advanced texts.

\section{Solution to 1:}
$\mathrm{C}$ is correct. Money needs to have a known value and be easily divisible. It should also be readily acceptable, difficult to counterfeit, and have a high value relative to its weight.

\begin{enumerate}
  \setcounter{enumi}{1}
  \item If the reserve requirement for banks in an economy is 5 percent, how much money could be created with the deposit of an additional $\pounds 100$ into a deposit account?
A. $\pounds 500$
B. $\pounds 1,900$
C. $\pounds 2,000$
\end{enumerate}

\section{Solution to 2:}
$\mathrm{C}$ is correct. To calculate the increase in money from an additional deposit in the banking system, use the following expression: new deposit/reserve requirement.

\begin{enumerate}
  \setcounter{enumi}{2}
  \item Which of the following functions does money normally fulfill for a society? It:
\end{enumerate}

A. acts as a medium of exchange only.

B. provides economic agents with a means of storing wealth only.

C. provides society with a unit of account, acts as a medium of exchange, and acts as a store of wealth.

\section{Solution to 3:}
$\mathrm{C}$ is correct. Money needs to be able to fulfill the functions of acting as a unit of account, a medium of exchange, and a means of storing wealth.

\section{Definitions of Money}
The process of money creation raises a fundamental issue: What is money? In an economy with money but without promissory notes and fractional reserve banking, money is relatively easy to define: Money is the total amount of gold and silver coins in circulation, or their equivalent. The money creation process above, however, indicates that a broader definition of money might encompass all the notes and coins in circulation plus all bank deposits.

More generally, we might define money as any medium that can be used to purchase goods and services. Notes and coins can be used to fulfill this purpose, and yet such currency is not the only means of purchasing goods and services. Personal cheques can be written based on a bank chequing account, while debit cards can be used for the same purpose. But what about time deposits or savings accounts? Nowadays transfers can be made relatively easily from a savings account to a current account; therefore, these savings accounts might also be considered as part of the stock of money. Credit cards are also used to pay for goods and services; however, there is an important difference between credit card payments and those made by cheques and debit cards. Unlike a cheque or debit card payment, a credit card payment involves a deferred payment. Basically, the greater the complexity of any financial system, the harder it is to define money. The monetary authorities in most modern economies produce a range of measures of money (see Exhibit 3). But generally speaking, the money stock consists of notes and coins in circulation, plus the deposits in banks and other financial institutions that can be readily used to make purchases of goods and services in the economy. In this regard, economists often speak of the rate of growth of narrow money and/or broad money. By narrow money, they generally mean the notes and coins in circulation in an economy, plus other very highly liquid deposits. Broad money encompasses narrow money but also includes the entire range of liquid assets that can be used to make purchases.

Because financial systems, practice, and institutions vary from economy to economy, so do definitions of money; thus, it is difficult to make international comparisons. Still, most central banks produce both a narrow and broad measure of money, plus some intermediate ones too. Exhibit 3 shows the money definitions in four economies.

\section{Exhibit 3: Definitions of Money}
\section{Money Measures in the United States}
The US Federal Reserve produces two measures of money. The first is M1, which comprises notes and coins in circulation, travelers' cheques of non-bank issuers, demand deposits at commercial banks, plus other deposits on which cheques can be written. M2 is the broadest measure of money currently produced by the Federal Reserve and includes M1, plus savings and money market deposits, time deposit accounts of less than $\$ 100,000$, plus other balances in retail money market and mutual funds.

\section{Money Measures in the Eurozone}
The European Central Bank (ECB) produces three measures of euro area money supply. The narrowest is M1. M1 comprises notes and coins in circulation, plus all overnight deposits. M2 is a broader definition of euro area money that includes M1, plus deposits redeemable with notice up to three months and deposits with maturity up to two years. Finally, the euro area's broadest definition of money is M3, which includes M2, plus repurchase agreements, money market fund units, and debt securities with up to two years maturity.

\section{Money Measures in Japan}
The Bank of Japan calculates three measures of money. M1 is the narrowest measure and consists of cash currency in circulation. M2 incorporates M1 but also includes certificates of deposit (CDs). The broadest measure, M3, incorporates M2, plus deposits held at post offices, plus other savings and deposits with financial institutions. There is also a "broad measure of liquidity" that encompasses M3 as well as a range of other liquid assets, such as government bonds and commercial paper.

\section{Money Measures in the United Kingdom}
The United Kingdom produces a set of four measures of the money stock. M0 is the narrowest measure and comprises notes and coins held outside the Bank of England, plus Bankers' deposits at the Bank of England. M2 includes M0, plus (effectively) all retail bank deposits. M4 includes M2, plus wholesale bank and building society deposits and also certificates of deposit. Finally, the Bank of England produces another measure called M3H, which is a measure created to be comparable with money definitions in the EU (see above). M3H includes M4, plus UK residents' and corporations' foreign currency deposits in banks and building societies.

\section{MONEY: QUANTITY THEORY, SUPPLY AND DEMAND, FISHER EFFECT}
describe theories of the demand for and supply of money describe the Fisher effect

The previous section of this reading shows that there are many definitions of money. In this section, we explore the important relationship between money and the price level. This relationship is best expressed in the quantity theory of money, which asserts that total spending (in money terms) is proportional to the quantity of money. The theory can be explained in terms of Equation 2, known as the quantity equation of exchange:

$$
M \times V=P \times Y
$$

where $M$ is the quantity of money, $V$ is the velocity of circulation of money (the average number of times in a given period that a unit of currency changes hands), $P$ is the average price level, and $Y$ is real output. The expression is really just an accounting identity. Effectively, it says that over a given period, the amount of money used to purchase all goods and services in an economy, $M \times V$, is equal to monetary value of this output, $P \times Y$. If the velocity of money is approximately constant-which is an assumption of quantity theory-then spending $P \times Y$ is approximately proportional to $M$. The quantity equation can also be used to explain a consequence of money neutrality. If money neutrality holds, then an increase in the money supply, $M$, will not affect $Y$, real output, or the speed with which money changed hands, $V$, because if real output is unaffected, there would be no need for money to change hands more rapidly. ${ }^{4}$ However, it will cause the aggregate price level, $P$, to rise.

The simple quantity theory gave rise to the equally simple idea that the price level, or at least the rate of inflation, could be controlled by manipulating the rate of growth of the money supply. Economists who believe this are referred to as monetarists. They argue that there is a causal relationship running from money growth to inflation. In the past, some governments have tried to apply this logic in their efforts to control inflation, most notably and unsuccessfully the United Kingdom's government in 1979 (see Example 5). However, it is possible that causality runs the other way-that is, from real activity to the money supply. This means that the quantity of money in circulation is determined by the level of economic activity, rather than vice versa.

4 Note that the full version of the quantity theory of money uses the symbol $T$ rather than $Y$ to indicate transactions because money is used not just for buying goods and services but also for financial transactions. We will return to this point in the discussion of quantitative easing.

\section{The Demand for Money}
The amount of wealth that the citizens of an economy choose to hold in the form of money-as opposed to bonds or equities-is known as the demand for money. There are three basic motives for holding money:

\begin{itemize}
  \item transactions-related;

  \item precautionary; and

  \item speculative.

\end{itemize}

Money balances that are held to finance transactions are referred to as transactions money balances. The size of the transactions balances will tend to increase with the average value of transactions in an economy. Generally speaking, as gross domestic product (GDP) grows over time, transactions balances will also tend to grow; however, the ratio of transactions balances to GDP remains fairly stable over time.

As the name suggests, precautionary money balances are held to provide a buffer against unforeseen events that might require money. These balances will tend to be larger for individuals or organizations that enter into a high level of transactions over time. In other words, a precautionary buffer of $\$ 100$ for a company that regularly enters into transactions worth millions of dollars might be considered rather small. When we extend this logic to the overall economy, we can see that these precautionary balances will also tend to rise with the volume and value of transactions in the economy, and therefore, GDP as well.

Finally, the speculative demand for money (sometimes called the portfolio demand for money) relates to the demand to hold speculative money balances based on the potential opportunities or risks that are inherent in other financial instruments (e.g., bonds). Speculative money balances consist of monies held in anticipation that other assets will decline in value. But in choosing to hold speculative money balances rather than bonds, investors give up the return that could be earned from the bond or other financial assets. Therefore, the speculative demand for money will tend to fall as the returns available on other financial assets rises. However, it will tend to rise as the perceived risk in other financial instruments rises. In equilibrium, individuals will tend to increase their holdings of money relative to riskier assets until the marginal benefit of having a lower risk portfolio of wealth is equal to the marginal cost of giving up a unit of expected return on these riskier assets. In aggregate then, speculative balances will tend to be inversely related to the expected return on other financial assets and directly related to the perceived risk of other financial assets.

\section{EXAMPLE 3}
\section{Money}
\begin{enumerate}
  \item The transactions demand for money refers to the demand to hold money:
\end{enumerate}

A. as a buffer against unforeseen events.

B. to use in the purchase of goods and services.

C. based on the opportunity or risks available on other financial instruments.

\section{Solution to 1:}
$\mathrm{B}$ is correct. The transactions demand for money refers to the amount of money that economic agents wish to hold to pay for goods and services. 2. The speculative demand for money will tend to:
A. fall as the perceived risk on other assets rises.
B. rise as the expected returns on other assets fall.
C. be inversely related to the transactions demand for money.

\section{Solution to 2:}
$B$ is correct. If the expected return on other assets falls, then the opportunity cost of holding money also falls and can, in turn, lead to an increase in the speculative demand for money.

\begin{enumerate}
  \setcounter{enumi}{2}
  \item What is the difference between narrow and broad money? Broad money:
\end{enumerate}

A. is limited to those liquid assets most commonly used to make purchases.

B. can be used to purchase a wider range of goods and services than narrow money.

C. encompasses narrow money and refers to the stock of the entire range of liquid assets that can be used to make purchases.

\section{Solution to 3:}
$\mathrm{C}$ is correct. This is the definition of broad money. Broad money encompasses narrow money.

\section{The Supply and Demand for Money}
We have now discussed definitions of money, its relationship with the aggregate price level, and the demand for it. We now discuss the interaction between the supply of and demand for money.

As with most other markets, the supply of money and the demand to hold it will interact to produce an equilibrium price for money. In this market, the price of money is the nominal interest rate that could be earned by lending it to others. Exhibit 4 shows the supply and demand curves for money. The vertical scale represents the rate of interest; the horizontal scale plots the quantity of nominal money in the economy. The supply curve (MS) is vertical because we assume that there is a fixed nominal amount of money circulating at any one time. The demand curve (MD) is downward sloping because as interest rates rise, the speculative demand for money falls. The supply and demand for money are both satisfied at an equilibrium interest rate of $I_{0}$. $I_{0}$ is the rate of interest at which no excess money balances exist.

\section{Exhibit 4: The Supply and Demand for Money}
\begin{center}
\includegraphics[max width=\textwidth]{2023_05_04_ccb578c5089fd41615beg-302}
\end{center}

Quantity of Money

To see why $I_{0}$ is the equilibrium rate of interest where there are no excess money balances, consider the following. If the interest rate on bonds were $I_{1}$ instead of $I_{0}$, there would be excess supply of money $\left(M_{0}-M_{1}\right)$. Economic agents would seek to buy bonds with their excess money balances, which would force the price of bonds up and the interest rate back down to $I_{0}$. Similarly, if bonds offered a rate of interest, $I_{2}$, there would be an excess demand for money $\left(M_{2}-M_{0}\right)$. Corporations and individuals would seek to sell bonds so that individuals could increase their money holdings, but in doing so, the price of bonds would fall and the interest rate offered on them would rise until it reached $I_{0}$. Interest rates effectively adjust to bring the market into equilibrium ("clear the market"). In this simple example, we have also assumed that the supply of money and bonds is fixed as economic agents readjust their holdings. In practice, this may not be true, but the dynamics of the adjustment process described here essentially still hold.

Exhibit 4 also reemphasises the relationship between the supply of money and the aggregate price level, which we first encountered when discussing the quantity theory of money. Suppose that the central bank increases the supply of money from $M_{0}$ to $M_{2}$, so that the vertical supply curve shifts to the right. Because the increase in the supply of money makes it more plentiful and hence less valuable, its price (the interest rate) falls as the price level rises. This all sounds very simple, but in practice the effects of an increase in the money supply are more complex. The initial increase in the money supply will create excess supply of cash. People and companies could get rid of the excess by loaning the money to others by buying bonds, as implied above, but they might also deposit it in a bank or simply use it to buy goods and services. But an economy's capacity to produce goods and services depends on the availability of real things: notably, natural resources, capital, and labour-that is, factors of production supplied either directly or indirectly by households. Increasing the money supply does not change the availability of these real things. Thus, some economists believe that the long-run impact of an exogenous increase in the supply of money is an increase in the aggregate price level.

This phenomenon-whereby an increase in the money supply is thought in the long run simply to lead to an increase in the price level while leaving real variables like output and employment unaffected-is known as money neutrality. To see why in the long run money should have a neutral effect on real things, consider the following simple example.

Suppose the government declared today that $1 \mathrm{~kg}$ would henceforth be referred to as $2 \mathrm{~kg}$ and that $1.5 \mathrm{~kg}$ would be referred to as $3 \mathrm{~kg}$. In other words, suppose that they halved the "value" of a kilogram. Would anything real have changed? A $1 \mathrm{~kg}$ bag of sugar would not have changed physically, although it would be relabelled as a $2 \mathrm{~kg}$ bag of sugar. However, there might be some short-run effects; confused people might buy too little sugar, and some people might go on crash diets! But ultimately people would adjust. In the long run, the change wouldn't matter. There is a clear parallel here with the theory of money neutrality. Doubling the prices of everything-halving the value of a currency-does not change anything real. This is because, like kilograms, money is a unit of account. However, halving the value of a currency could affect real things in the short run.

There are two points worth making with regard to money neutrality. First, although the simple kilogram analogy above does suggest that money should not affect real things in the long run, as the British economist Keynes said: "In the long run we are all dead!" In practice, it is very difficult for economists to be sure that money neutrality holds in the long run. And second, we must assume that monetary authorities do believe that the money supply can affect real things in the short run. If they did not, then there would be almost no point to monetary policy.

\section{The Fisher Effect}
The Fisher effect is directly related to the concept of money neutrality. Named after the economist Irving Fisher, the Fisher effect states that the real rate of interest in an economy is stable over time so that changes in nominal interest rates are the result of changes in expected inflation. Thus, the nominal interest rate $\left(R_{\text {nom }}\right)$ in an economy is the sum of the required real rate of interest $\left(R_{\text {real }}\right)$ and the expected rate of inflation $\left(\pi^{e}\right)$ over any given time horizon:

$$
R_{\text {nom }}=R_{\mathrm{real}}+\pi^{e}
$$

According to money neutrality, over the long term the money supply and/or the growth rate in money should not affect $R_{\text {real }}$ but will affect inflation and inflation expectations.

The Fisher effect also demonstrates that embedded in every nominal interest rate is an expectation of future inflation. Suppose that 12-month US government T-bills offered a yield equal to 4 percent over the year. Suppose also that T-bill investors wished to earn a real rate of interest of 2 percent and expected inflation to be 2 percent over the next year. In this case, the return of 4 percent would be sufficient to deliver the investors' desired real return of 2 percent (so long as inflation did not exceed 2 percent). Now suppose that investors changed their view about future inflation and instead expected it to equal 3 percent over the next 12 months. To compensate them for the higher expected inflation, the $\mathrm{T}$-bill rate would have to rise to 5 percent, thereby preserving the required 2 percent real return.

There is one caveat to this example. Investors can never be sure about future values of such economic variables as inflation and real growth. To compensate them for this uncertainty, they require a risk premium. The greater the uncertainty, the greater the required risk premium. So, all nominal interest rates are comprised of three components:

\begin{itemize}
  \item a required real return;

  \item a component compensating investors for expected inflation; and

  \item a risk premium to compensate them for uncertainty.

\end{itemize}

\section{EXAMPLE 4}
\section{Interest Rates and the Supply of Money}
\begin{enumerate}
  \item According to the quantity equation of exchange, an increase in the money supply can lead to an:
\end{enumerate}

A. increase in the aggregate price level, regardless of changes in the velocity of circulation of money.

B. increase in the aggregate price level as long as the velocity of circulation of money rises sufficiently to offset the increase in the money supply.

C. increase in the aggregate price level as long as the velocity of circulation of money does not fall sufficiently to offset the increase in the money supply and real output is unchanged.

\section{Solution to 1:}
$\mathrm{C}$ is correct. If the velocity of circulation of money does not change with an increase in the money supply and real output is fixed, then the aggregate price level should increase. If the velocity of circulation of money falls sufficiently, or if real output rises sufficiently, then the increase in money may have no impact on prices.

\begin{enumerate}
  \setcounter{enumi}{1}
  \item The nominal interest rate comprises a real rate of interest:
\end{enumerate}

A. plus a risk premium only.

B. plus a premium for expected inflation only.

C. compensation for both expected inflation and risk.

Solution to 2:

$\mathrm{C}$ is correct. Investors demand a real rate of interest and compensation for expected inflation and a risk premium to compensate them for uncertainty.

\begin{enumerate}
  \setcounter{enumi}{2}
  \item An expansion in the money supply would most likely:
\end{enumerate}

A. lead to a decline in nominal interest rates.

B. lead to an increase in nominal interest rates.

C. reduce the equilibrium amount of money that economic agents would wish to hold.

\section{Solution to 3:}
A is correct. Increasing the supply of money, all other things being equal, will reduce its "price," that is, the interest rate on money balances.

\section{EXAMPLE 5}
Mrs. Thatcher's Monetary Experiment

\section{The Background}
Over the 1970s, the United Kingdom had one of the worst inflation records of any developed economy. Retail price inflation averaged 12.6 percent over that decade and peaked at 26.9 percent in August 1975. Over this period, then-Prime Minister Margaret Thatcher and her advisers had become convinced that inflation could not be controlled by the income and price policies used in the United Kingdom in the past. Instead, they believed that inflation could be tamed by controlling the rate of growth of the money supply. Mrs. Thatcher's first administration took power in May 1979 with the intention of pursuing a monetarist agenda-that is, a macroeconomic policy that would be underpinned by targets for money supply growth.

\section{The Medium-Term Financial Strategy}
Targets for monetary growth were set for a definition of the money supply known as Sterling M3 (£M3), which was to be kept in the range of 7-11 percent for the period 1980-1981 and then gradually reduced to within $4-8$ percent by 1983-1984. This set of targets was known as the Medium Term Financial Strategy (MTFS). The idea was simple: Control the rate of growth of the money supply, and the rate of growth of prices (i.e., inflation) would remain under control too. The instrument of control was the Bank of England's policy interest rate that would be set to achieve the desired rate of growth of the money supply. This was a macroeconomic policy built, however imperfectly, on an interpretation of the quantity theory of money.

The theory was simple, but the practice proved to be less so. Over the first two and a half years of the MTFS, $\pounds M 3$ overshot its target by 100 percent. The inability of the monetary authorities to control the rate of growth of the broad money supply was largely caused by Thatcher's abolition of exchange controls in 1979. By abolishing these controls, there was a significant increase in foreign exchange business that came into the British banking system, which changed the velocity of money and therefore meant that the relationship between broad money and nominal incomes had changed fundamentally. ${ }^{5}$

Despite the inability to control the money supply, in 1983 the Thatcher administration reasserted its confidence in the policy and published a further set of monetary targets for several years ahead. However, the persistent failure to meet these targets, too, eventually led to the abandonment of any type of monetary targeting by the summer of 1985 .

The experience of the UK monetary authorities over this period emphasizes how unstable the relationship between money and the policy interest rate could be along with the relationship between money and aggregate demand-particularly

5 See Goodhart (1989) for a discussion in an economy that is experiencing rapid financial innovation, as the UK economy was following the abolition of exchange controls and the introduction of greater competition within the banking industry.

Today the Bank of England is responsible for the operation and implementation of monetary policy in the United Kingdom. The trends in money supply are watched very carefully, but they are not the subject of targets, per se.

\section*{ROLES OF CENTRAL BANKS \& OBJECTIVES OF MONETARY POLICY }
Central banks play several key roles in modern economies. Generally, a central bank is the monopoly supplier of the currency, the banker to the government and the bankers' bank, the lender of last resort, the regulator and supervisor of the payments system, the conductor of monetary policy, and the supervisor of the banking system. Let us examine these roles in turn.

In its earliest form, money could be exchanged for a pre-specified precious commodity, usually gold, and promissory notes were issued by many private banks. Today, however, state-owned institutions-usually central banks-are designated in law as being the monopoly suppliers of a currency. Initially, these monopolists supplied money that could be converted into a pre-specified amount of gold; they adhered to a gold standard. For example, up until 1931, bank notes issued by Britain's central bank, the Bank of England, could be redeemed at the bank for a pre-specified amount of gold. But Britain, like most other major economies, abandoned this convertibility principle in the first half of the twentieth century. Money in all major economies today is not convertible by law into anything else, but it is, in law, legal tender. This means that it must be accepted when offered in exchange for goods and services. Money that is not convertible into any other commodity is known as fiat money. Fiat money derives its value via government decree and because people accept it for payment of goods and services and for debt repayment.

As long as fiat money is acceptable to everyone as a medium of exchange, and it holds its value over time, then it will also be able to serve as a unit of account. However, once an economy has moved to a system of fiat money, the role of the supplier of that money becomes even more crucial because they could, for example, expand the supply of this money indefinitely should they wish to do so. Central banks therefore play a crucial role in modern economies as the suppliers and guardians of the value of their fiat currencies and as institutions charged with the role of maintaining confidence in their currencies. As the monopoly suppliers of an economy's currency, central banks are at the centre of economic life. As such, they assume other roles in addition to being the suppliers and guardians of the value of their currencies.

Most central banks act as the banker to the government and to other banks. They also act as a lender of last resort to banks. Because the central bank effectively has the capacity to print money, it is in the position to be able to supply the funds to banks that are facing a damaging shortage. The facts that economic agents know that the central bank stands ready to provide the liquidity required by any of the banks under its jurisdiction and that they trust government bank deposit insurance help to prevent bank runs in the first place. However, the recent financial crisis has shown that this knowledge is not always sufficient to deter a bank run.

\section{EXAMPLE 6}
\section{The Northern Rock Bank Run}
In the latter part of the summer of 2007, the fall in US house prices and the related implosion of the US sub-prime mortgage market became the catalyst for a global liquidity crisis. Banks began to hoard cash and refused to lend to other banks at anything other than extremely punitive interest rates through the interbank market. This caused severe difficulties for a UK mortgage bank, Northern Rock. Northern Rock's mortgage book had expanded rapidly in the preceding years as it borrowed aggressively from the money markets. It is now clear that this expansion was at the expense of loan quality. The then UK regulatory authority, the Financial Services Authority (FSA), ${ }^{6}$ later reported in 2008 that Northern Rock's lending practices did not pay due regard to either the credit quality of the mortgagees or the values of the properties on which the mortgages were secured. Being at the worst end of banking practice, and relying heavily on international capital markets for its funding, Northern Rock was therefore very susceptible to a global reduction in liquidity. As the liquidity crisis took hold, Northern Rock found that it could not replace its maturing money market borrowings. On 12 September 2007, in desperate need of liquidity, Northern Rock's board approached the UK central bank to ask for the necessary funds.

However, the news of Northern Rock's perilous liquidity position became known by the public and, more pertinently, by Northern Rock's retail depositors. On 14 September, having heard the news, queues began to form outside Northern Rock branches as depositors tried to withdraw their savings. On that day, it was estimated that Northern Rock depositors withdrew around $\pounds 1 \mathrm{bn}$, representing 5 percent of Northern Rock's deposits. Further panic ensued as investors in "internet only" Northern Rock accounts could not withdraw their money because of the collapse of Northern Rock's website. A further $\pounds 1 \mathrm{bn}$ was withdrawn over the next two days.

Northern Rock's share price dropped rapidly, as did the share prices of other similar UK banks. The crisis therefore threatened to engulf more than one bank. To prevent contagion, the Chancellor of the Exchequer announced on 17 September that the UK government would guarantee all Northern Rock deposits. This announcement was enough to stabilize the situation, and given that lending to Northern Rock was now just like lending to the government, deposits actually started to rise again.

Eventually Northern Rock was nationalized by the UK government, with the hope that at some time in the future it could be privatized once its balance sheet had been repaired.

Central banks are also often charged by the government to supervise the banking system, or at least to supervise those banks that they license to accept deposits. However, in some countries, this role is undertaken by a separate authority. In other countries, the central bank can be jointly responsible with another body for the supervision of its banks.

6 In 2013, the Financial Services Authority was replaced by two new regulatory authorities, the Financial Conduct Authority (FCA) and the Prudential Regulation Authority (PRA). Exhibit 5 lists the banking supervisors in the G-10 countries; central banks are underlined. As the exhibit shows, most but not all bank systems have a single supervisor, which is not necessarily a central bank. A few countries, such as Germany and the United States, have more than one supervisor.

\section{Exhibit 5: Banking Supervision in the G10}
\begin{center}
\begin{tabular}{ll}
\hline
Country & Institution(s) \\
\hline
Belgium & Banking and Finance Commission \\
Canada & Office of the Superintendent of Financial Institutions \\
France & Commission Bancaire \\
Germany & Federal Banking Supervisory Office; Deutsche Bundesbank \\
Italy & $\underline{\text { Bank of Italy }}$ \\
Japan & $\underline{\text { Financial Services Agency }}$ \\
Netherlands & $\underline{\text { Bank of Netherlands }}$ \\
Sweden & Swedish Financial Supervisory Authority \\
Switzerland & Federal Commission \\
United Kingdom & $\underline{\text { Bank of England }}$ \\
United States & Office of the Comptroller of the Currency; Federal Reserve; Federal \\
 & Deposit Insurance Corporation \\
\end{tabular}
\end{center}

The United Kingdom is an interesting case study in this regard. Until May 1997, the Bank of England had statutory responsibility for banking supervision in the United Kingdom. In May 1997, banking supervision was removed from the Bank of England and assigned to a new agency, the Financial Services Authority (FSA). However, the removal of responsibility for banking supervision from the central bank was seen by some as being a contributory factor in the run on the mortgage bank Northern Rock, and generally as a contributory factor in the recent banking crisis. Because of this perceived weakness in the separation of the central bank from banking supervision, the Bank of England regained responsibility for banking supervision and regulation in 2013.

Perhaps the least appreciated role of a central bank is its role in the payments system. Central banks are usually asked to oversee, regulate, and set standards for a country's payments system. Every day millions of financial transactions take place in a modern economy. For the system to work properly, procedures must be robust and standardized. The central bank will usually oversee the payments system and will also be responsible for the successful introduction of any new processes. Given the international nature of finance, the central bank will also be responsible for coordinating payments systems internationally with other central banks.

Most central banks will also be responsible for managing their country's foreign currency reserves and also its gold reserves. With regard to the latter, even though countries abandoned the gold standard in the early part of the twentieth century, the world's central bankers still hold large quantities of gold. As such, if central banks were to decide to sell significant proportions of their gold reserves, it could potentially depress gold prices.

Finally, central banks are usually responsible for the operation of a country's monetary policy. This is arguably the highest profile role that these important organizations assume. Recall that monetary policy refers to central bank activities that are directed toward influencing the quantity of money and credit in an economy. As the monopoly supplier of a country's currency, central banks are in the ideal position to implement and/or determine monetary policy.

To summarise, central banks assume a range of roles and responsibilities. They do not all assume responsibility for the supervision of the banks, but all of the other roles listed below are normally assumed by the central bank:

\begin{itemize}
  \item Monopoly supplier of the currency;

  \item Banker to the government and the bankers' bank;

  \item Lender of last resort;

  \item Regulator and supervisor of the payments system;

  \item Conductor of monetary policy; and

  \item Supervisor of the banking system.

\end{itemize}

\section{The Objectives of Monetary Policy}
Central banks fulfill a variety of important roles, but for what overarching purpose? A brief perusal of the websites of the world's central banks will reveal a wide range of explanations of their objectives. Their objectives are clearly related to their roles, and so there is frequent mention of objectives related to the stability of the financial system and to the payments systems. Some central banks are charged with doing all they can to maintain full employment and output. But some also have related but less tangible roles, like "maintaining confidence in the financial system," or even to "promote understanding of the financial sector." But there is one overarching objective that most seem to acknowledge explicitly, and that is the objective of maintaining price stability.

So, although central banks usually have to perform many roles, most specify an overarching objective. Exhibit 6 lists what we might call the primary objective(s) of a number of central banks, from both developed and developing economies.

\section{Exhibit 6: The Objectives of Central Banks}
\section{The Central Bank of Brazil}
Its "institutional mission" is to "ensure the stability of the currency's purchasing power and a solid and efficient financial system."

\section{The European Central Bank}
"[T]o maintain price stability is the primary objective of the Euro system and of the single monetary policy for which it is responsible. This is laid down in the Treaty on the Functioning of the European Union, Article 127 (1)."

"Without prejudice to the objective of price stability", the euro system will also "support the general economic policies in the Community with a view to contributing to the achievement of the objectives of the Community." These include a "high level of employment" and "sustainable and non-inflationary growth."

\section{The US Federal Reserve}
“The Federal Reserve sets the nation's monetary policy to promote the objectives of maximum employment, stable prices, and moderate long-term interest rates."

\section{The Reserve Bank of Australia}
"It is the duty of the Reserve Bank Board, within the limits of its powers, to ensure that the monetary and banking policy of the Bank is directed to the greatest advantage of the people of Australia and that the powers of the Bank ... are exercised in such a manner as, in the opinion of the Reserve Bank Board, will best contribute to:

a. the stability of the currency of Australia;

b. the maintenance of full employment in Australia; and

c. the economic prosperity and welfare of the people of Australia."

\section{The Bank of Korea}
"The primary purpose of the Bank, as prescribed by the Bank of Korea Act of 1962 , is the pursuit of price stability."

Source: Central bank websites found at \href{http://www.bis.org/cbanks.htm}{http://www.bis.org/cbanks.htm}.

\section{EXAMPLE 7}
\section{Central Banks}
\begin{enumerate}
  \item A central bank is normally not the:
\end{enumerate}

A. lender of last resort.

B. banker to the government and banks.

C. body that sets tax rates on interest on savings.

\section{Solution to 1:}
$\mathrm{C}$ is correct. A central bank is normally the lender of last resort and the banker to the banks and government, but the determination of all tax rates is normally the preserve of the government and is a fiscal policy issue.

\begin{enumerate}
  \setcounter{enumi}{1}
  \item Which of the following best describes the overarching, long-run objective of most central banks?
\end{enumerate}

A. Price stability

B. Fast economic growth

C. Current account surplus

\section{Solution to 2:}
A is correct. Central banks normally have a variety of objectives, but the overriding one is nearly always price stability.

As we have already discussed, one of the essential features of a monetary system is that the medium of exchange should have a relatively stable value from one period to the next. Arguably then, the overarching goal of most central banks in maintaining price stability is the associated goal of controlling inflation. But before we explore the tools central banks use to control inflation, we should first consider the potential costs of inflation. In other words, we should ask why it is that central bankers believe that it is so important to control a nominal variable.

\section{THE COSTS OF INFLATION}
contrast the costs of expected and unexpected inflation

Huge efforts have been put into controlling inflation since the major economies experienced such high levels of inflation in the 1970s. From the early 1970s then, inflation has been seen as a very bad thing. But why? What are the costs of inflation? The debate around the "costs" of inflation really centers on the distinction between expected inflation and unexpected inflation. Expected inflation is clearly the level of inflation that economic agents expect in the future. Unexpected inflation can be defined as the level of inflation that we experience that is either below or above that which we expected; it is the component of inflation that is a surprise.

At a micro level, high inflation means that businesses constantly have to change the advertised prices of their goods and services. These are known as menu costs. There also exists what economists refer to as "shoe leather" costs of inflation. In times of high inflation, people would naturally tend to hold less cash and would therefore wear out their shoe leather (or more likely the engines of their cars) in making frequent trips to the bank to withdraw cash. But these are relatively old arguments, used to demonstrate that inflation is bad. In a modern economy, with the internet and with transactions becoming increasingly cashless, these costs associated with inflation will be lower today than they may have been in the past.

To demonstrate the potentially more significant costs of inflation, consider the following. Imagine a world where inflation is high but where all prices (including asset prices) in an economy are perfectly indexed to inflation, and that technology has eliminated the issues surrounding the menu and shoe leather costs of inflation. In such a world, would economic agents care about inflation? Probably not. If the average price of goods and services rose by 10 percent, people's salaries (and all other prices) would rise by the same amount, which would therefore make economic agents indifferent to the rise in prices.

In practice, however, all prices, wages, salaries, rents, and so forth are not indexed, in which case economic agents would certainly need to think about inflation more carefully. But what if inflation in this world where prices are no longer perfectly indexed is high, but perfectly predictable? In this alternative, imaginary world, economic agents would have to think about inflation, but not too hard as long as they were capable of calculating the impact of the known inflation rate on all future prices. So, if everyone knew that inflation was going to be 10 percent over the next year, then everyone could bargain for a 10 percent increase in their salaries to accommodate this, and companies could plan to put up the prices of their goods and services by 10 percent. In this world, an expectation of 10 percent inflation would become a self-fulfilling prophecy.

However, economic agents would worry about inflation in a world where all prices were not indexed and, crucially, where inflation was high and unpredictable. In fact, this is a crude description the inflationary backdrop in many developed economies over the 1970s and 1980s, including the United States, France, the United Kingdom, Italy, and Canada.

Arguably it is unexpected inflation that is most costly. Inflation that is fully anticipated can be factored into wage negotiations and priced into business and financial contracts. But when inflation turns out to be higher than is anticipated, then borrowers benefit at the expense of lenders because the real value of their borrowing declines. Conversely, when inflation is lower than is anticipated, lenders benefit at the expense of borrowers because the real value of the payment on debts rises. Furthermore, if inflation is very uncertain or very volatile, then lenders will ask for a premium to compensate them for this uncertainty. As a result, the costs of borrowing will be higher than would otherwise have been the case. Higher borrowing costs could in turn reduce economic activity, for example, by discouraging investment.

It is also possible that inflation uncertainty can exacerbate the economic cycle. Inflation uncertainty is the degree to which economic agents view future rates of inflation as hard to forecast. Take for example the case of an imaginary television manufacturer. Suppose one day that the manufacturer looks out at the market for televisions and sees that the market price of televisions has risen by 10 percent. Armed with this information, the manufacturer assumes that there has been an increase in demand for televisions or maybe a reduction in supply. So, to take advantage of the new, higher prices, the manufacturer extends the factory, employs more workers, and begins to produce more televisions.

Having now increased the output of the factory, the manufacturer then attempts to sell the extra televisions that the factory has produced. But to its horror, the manufacturer finds out that there is no extra demand for televisions. Instead, the 10 percent rise in television prices was caused by a generalized 10 percent increase in all consumer prices across the economy. The manufacturer realizes that it has surplus stock, surplus factory capacity, and too many workers. So, it cuts back on production, lays off some of the workforce, and realizes that it won't need to invest in new plant or machinery for a long time.

This example emphasizes the potentially destabilizing impact of unexpected inflation. It demonstrates how unanticipated inflation can reduce the information content of market prices for economic agents. If we scale this example up, it should not be too difficult to imagine how unanticipated increases or decreases in the general price level could help to exacerbate-and in some extreme cases cause-economic booms and busts.

Over the last two to three decades the consensus among economists has been that unanticipated and high levels of inflation can have an impact on real things like employment, investment and profits, and therefore that controlling inflation should be one of the main goals of macroeconomic policy. In summary:

Expected inflation can give rise to:

\begin{itemize}
  \item menu costs and

  \item shoe leather costs.

\end{itemize}

Unanticipated (unexpected) inflation can in addition:

\begin{itemize}
  \item lead to inequitable transfers of wealth between borrowers and lenders (including losses to savings);

  \item give rise to risk premia in borrowing rates and the prices of other assets; and

  \item reduce the information content of market prices.

\end{itemize}

\section*{MONETARY POLICY TOOLS }
Central banks have three primary tools available to them: open market operations, the refinancing rate, and reserve requirements. ${ }^{7}$

\section{Open Market Operations}
One of the most direct ways for a central bank to increase or reduce the amount of money in circulation is via open market operations. Open market operations involve the purchase and sale of government bonds from and to commercial banks and/or designated market makers. For example, when the central bank buys government bonds from commercial banks, this increases the reserves of private sector banks on the asset side of their balance sheets. If banks then use these surplus reserves by increasing lending to corporations and households, then via the money multiplier process explained in Section 2.1.2, broad money growth expands. Similarly, the central bank can sell government bonds to commercial banks. By doing this, the reserves of commercial banks decline, reducing their capacity to make loans (i.e., create credit) to households and corporations and thus causing broad money growth to decline through the money multiplier mechanism. In using open market operations, the central bank may target a desired level of commercial bank reserves or a desired interest rate for these reserves.

\section{The Central Bank's Policy Rate}
The most obvious expression of a central bank's intentions and views comes via the interest rate it sets. The name of the official interest rate (or official policy rate or just policy rate) varies from central bank to central bank, but its purpose is to influence short- and long-term interest rates and ultimately real economic activity.

The interest rate that a central bank sets and that it announces publicly is normally the rate at which it is willing to lend money to the commercial banks (although practices do vary from country to country). This policy rate can be achieved by using short-term collateralized lending rates, known as repo rates. For example, if the central bank wishes to increase the supply of money, it might buy bonds (usually government bonds) from the banks, with an agreement to sell them back at some time in the future. This transaction is known as a repurchase agreement. Normally, the maturity of repo agreements ranges from overnight to two weeks. In effect, this represents a secured loan to the banks, and the lender (in this case the central bank) earns the repo rate.

Suppose that a central bank announces an increase in its official interest rate. Commercial banks would normally increase their base rates at the same time. A commercial bank's base rate is the reference rate on which it bases lending rates to all other customers. For example, large corporate clients might pay the base rate plus

7 Monetary policy tools and operations often vary considerably from economy to economy. We have tried to describe the generics of the process here. For a more-detailed review of monetary operations across the world, see Gray and Talbot (2006) 1 percent on their borrowing from a bank, while the same bank might lend money to a small corporate client at the base rate plus 3 percent. But why would commercial banks immediately increase their base or reference rates just because the central bank's refinancing rate had increased?

The answer is that commercial banks would not want to have lent at a rate of interest that would be lower than they might be charged by the central bank. Effectively, the central bank can force commercial banks to borrow from it at this rate because it can conduct open market operations that create a shortage of money, forcing the banks to sell bonds to it with a pre-agreed repurchase price (i.e., do a repurchase agreement). The repo rate would be such that the central bank earned the official refinancing rate on the transactions.

The name of each central bank's official refinancing rate varies. The Bank of England's refinancing rate is the two-week repo rate. In other words, the Bank of England fixes the rate at which it is willing to lend two-week money to the banking sector. The ECB's official policy rate is known as the refinancing rate and defines the rate at which it is willing to lend short-term money to the euro area banking sector.

The corresponding rate in the United States is the discount rate, which is the rate for member banks borrowing directly from the Federal Reserve System. But the most important interest rate used in US monetary policy is the federal funds rate. The federal funds rate (or fed funds rate) is the interbank lending rate on overnight borrowings of reserves. The Federal Open Market Committee (FOMC) seeks to move this rate to a target level by reducing or adding reserves to the banking system by means of open market operations. The level of the rate is reviewed by the FOMC at its meetings held every six weeks (although the target can be changed between meetings, if necessary).

Through the setting of a policy rate, a central bank can manipulate the amount of money in the money markets. Generally speaking, the higher the policy rate, the higher the potential penalty that banks will have to pay to the central bank if they run short of liquidity, the greater will be their willingness to reduce lending, and the more likely that broad money growth will shrink.

\section{Reserve Requirements}
The third primary way in which central banks can limit or increase the supply of money in an economy is via their reserve requirements. We have already seen that the money creation process is more powerful the lower the percentage reserve requirement of banks. So, a central bank could restrict money creation by raising the reserve requirements of banks. However, this policy tool is not used much nowadays in developed economies. Indeed, some central banks, such as the Bank of England, do not even set minimum reserve requirements for the banks under their jurisdiction anymore. Changing reserve requirements frequently is disruptive for banks. For example, if a central bank increased the reserve requirements, a bank that was short on reserves might have to cease its lending activities until it had built up the necessary reserves, because deposits would be unlikely to rise quickly enough for the bank to build its reserves in this way. However, reserve requirements are still actively used in many developing countries to control lending and remain a potential policy tool for those central banks that do not currently use it.

To summarize, central banks can manipulate the money supply in one of three ways:

\begin{itemize}
  \item open market operations;

  \item its official policy rate and associated actions in the repo market; and

  \item manipulation of official reserve requirements.

\end{itemize}

\section{The Transmission Mechanism}
The overarching goal of a central bank is to maintain price stability. We demonstrated above how a central bank can manipulate the money supply and growth of the money supply. We also indicated how policy rates set and targeted by the central banks are usually very short term in nature; often they target overnight interest rates. However, most businesses and individuals in the real economy borrow and lend over much longer time frames than this. It may not be obvious, then, how changing short-term interest rates can influence the real economy, particularly if money neutrality holds in the long run. The fact that central bankers believe that they can affect real economic variables, in particular economic growth, by influencing broad money growth suggests that they believe that money is not neutral-at least not in the short run.

Exhibit 7 presents a stylized representation of the monetary transmission mechanism. This is the process whereby a central bank's interest rate gets transmitted through the economy and ultimately affects the rate of increase of prices-that is, inflation.

\section{Exhibit 7: A Stylized Representation of the Monetary Transmission Mechanism}
\begin{center}
\includegraphics[max width=\textwidth]{2023_05_04_ccb578c5089fd41615beg-315}
\end{center}

Source: Bank of England.

Suppose that a central bank announces an increase in its official interest rate. The implementation of the policy may begin to work through the economy via four interrelated channels. Those channels include bank lending rates, asset prices, agents' expectations, and exchange rates. First, as described above, the base rates of commercial banks and interbank rates should rise in response to the increase in the official rate. Banks would, in turn, increase the cost of borrowing for individuals and companies over both short- and long-term horizons. Businesses and consumers would then tend to borrow less as interest rates rise. An increase in short-term interest rates could also cause the price of such assets as bonds or the value of capital projects to fall as the discount rate for future cash flows rises.

Market participants would then come to the view that higher interest rates will lead to slower economic growth, reduced profits, and reduced borrowing to finance asset purchases. Exporters' profits might decline if the rise in interest rates causes the country's exchange rate to appreciate, because this would make domestic exports more expensive to overseas buyers and dampen demand to purchase them. The fall in asset prices as well as an increase in prices would reduce household financial wealth and therefore lead to a reduction in consumption growth. Expectations regarding interest rates can play a significant role in the economy. Often companies and individuals will make investment and purchasing decisions based on their interest rate expectations, extrapolated from recent events. If the central bank's interest rate move is widely expected to be followed by other interest rate increases, investors and companies will act accordingly. Consumption, borrowing, and asset prices may all decline as a result of the revision in expectations.

There is a whole range of interconnected ways in which a rise in the central bank's policy rate can reduce real domestic demand and net external demand (that is, the difference between export and import consumption). Weaker total demand would tend to put downward pressure on the rate of domestic inflation-as would a stronger currency, which would reduce the prices of imports. Taken together, these might begin to put downward pressure on the overall measure of inflation.

To summarize, the central bank's policy rate works through the economy via any one, and often all, of the following interconnected channels:

\begin{itemize}
  \item Short-term interest rates;

  \item Changes in the values of key asset prices;

  \item The exchange rate; and

  \item The expectations of economic agents.

\end{itemize}

\section{EXAMPLE 8}
\section{Central Bank Tools}
\begin{enumerate}
  \item Which of the following variables are most likely to be affected by a change in a central bank's policy rate?
\end{enumerate}

A. Asset prices only

B. Expectations about future interest rates only

C. Both asset prices and expectations about future interest rates

Solution to 1:

$\mathrm{C}$ is correct. The price of equities, for example, might be affected by the expectation of future policy interest rate changes. In other words, a rate change may be taken as a signal of the future stance of monetary policycontractionary or expansionary.

\begin{enumerate}
  \setcounter{enumi}{1}
  \item Which of the following does a central bank seek to influence directly via the setting of its official interest rate?
\end{enumerate}

A. Inflation expectations

B. Import prices

C. Domestic inflation

Solution to 2:

A is correct. By setting its official interest rate, a central bank could expect to have a direct influence on inflation expectations-as well as on other market interest rates, asset prices, and the exchange rate (where this is freely floating). If it can influence these factors, it might ultimately hope to influence import prices (via changes in the exchange rate) and also domestically generated inflation (via its impact on domestic and/or external demand). The problem is that the workings of the transmission mechanism-from the official interest rate to inflation-are complex and can change over time. INFLATION TARGETING

Over the 1990s, a consensus began to build among both central bankers and politicians that the best way to control inflation and thereby maintain price stability was to target a certain level of inflation and to ensure that this target was met by monitoring a wide range of monetary, financial, and real economic variables. Nowadays, inflation-targeting frameworks are the cornerstone of monetary policy and macroeconomic policy in many economies. Exhibit 8 shows the growth in the number of inflation-targeting monetary policy regimes over time.

The inflation-targeting framework that is now commonly practiced was pioneered in New Zealand. In 1988, the New Zealand Minister of Finance, Roger Douglas, announced that economic policy would focus on bringing inflation down from the prevailing level of around 6.0 percent to a target range of 0 to 2 percent. This goal was given legal status by the Reserve Bank of New Zealand Act 1989. As part of the Act, the Reserve Bank of New Zealand (RBNZ) was given the role of pursuing this target. The bank was given operational independence; it was free to set interest rates in the way that it thought would best meet the inflation target. Although the RBNZ had independent control of monetary policy, it was still accountable to the government and was charged with communicating its decisions in a clear and transparent way. As Exhibit 8 shows, the New Zealand model was widely copied.

\section*{Exhibit 8: The Progressive Adoption of Inflation Targeting by Central Banks }
\begin{center}
\begin{tabular}{llllll}
\hline
1989 & New Zealand &  &  &  &  \\
1990 & Chile & Canada &  &  &  \\
1991 & Israel & United Kingdom &  &  &  \\
1992 & Sweden & Finland & Australia &  &  \\
1995 & Spain &  &  &  &  \\
1998 & Czech Republic & South Korea & Poland &  &  \\
1999 & Mexico & Brazil & Colombia & ECB &  \\
2000 & South Africa & Thailand &  &  &  \\
2001 & Iceland & Norway & Hungary & Peru & Philippines \\
2005 & Guatemala & Indonesia & Romania &  &  \\
2006 & Turkey & Serbia &  &  &  \\
2007 & Ghana &  &  &  &  \\
\hline
\end{tabular}
\end{center}

Note: Spain and Finland later joined the EMU.

Sources: For 2001 and earlier, Truman (2003). For 2002 to 2007, Roger (2010).

Although these inflation-targeting regimes vary a little from economy to economy, their success is thought to depend on three key concepts: central bank independence, credibility, and transparency.

\section{Central Bank Independence ${ }^{8}$}
In most cases, the central bank that is charged with targeting inflation has a degree of independence from its government. This independence is thought to be important. It is conceivable that politicians could announce an inflation target and direct the central bank to set interest rates accordingly. Indeed, this was the process adopted in the United Kingdom between 1994 and 1997. But politicians have a constant eye on re-election and might be tempted, for example, to keep rates "too low" in the lead up to an election in the hope that this might help their re-election prospects. As a consequence, this might lead to higher inflation. Thus, it is now widely believed that monetary policy decisions should rest in the hands of an organization that is remote from the electoral process. The central bank is the natural candidate to be the monopoly supplier of a currency.

However, there are degrees of independence. For example, the head of the central bank is nearly always chosen by government officials. The Chairman of the US Federal Reserve's Board of Governors is appointed by the President of the United States of America; the Head of the ECB is chosen by the committee of Euro area finance ministers; while the Governor of the Bank of England is chosen by the Chancellor of the Exchequer. So, in practice, separating control from political influence completely is probably an impossible (although a desirable) goal.

There are further degrees of independence. Some central banks are both operationally and target independent. This means that they not only decide the level of interest rates, but they also determine the definition of inflation that they target, the rate of inflation that they target, and the horizon over which the target is to be achieved. The ECB has independence of this kind. By contrast, other central banksincluding those in New Zealand, Sweden, and the United Kingdom-are tasked to hit a definition and level of inflation determined by the government. These central banks are therefore only operationally independent.

\section{Credibility}
The independence of the central bank and public confidence in it are key in the design of an inflation-targeting regime.

To illustrate the role of credibility, suppose that instead of the central bank, the government assumes the role of targeting inflation but the government is heavily indebted. Given that higher inflation reduces the real value of debt, the government would have an incentive to avoid reaching the inflation target or to set a high inflation target such that price stability and confidence in the currency could be endangered. As a result, few would believe the government was really intent on controlling inflation; thus, the government would lack credibility. Many governments have very large levels of debt, especially since the 2008-2009 global financial crisis. In such a situation, economic agents might expect a high level of inflation, regardless of the actual, stated target. The target might have little credibility if the organization's likelihood of sticking to it is in doubt.

If a respected central bank assumes the inflation-targeting role and if economic agents believe that the central bank will hit its target, the belief itself could become self-fulfilling. If everyone believes that the central bank will hit an inflation target of 2 percent next year, this expectation might be built into wage claims and other nominal contracts that would make it hit the 2 percent target. It is for this reason that central

8 For information about the degree of independence of any central bank, the roles that it assumes in an economy, and the framework in which it operates, analysts should go to a central bank's website. A list of central bank websites can be found at \href{http://www.bis.org/cbanks.htm}{http://www.bis.org/cbanks.htm}. bankers pay a great deal of attention to inflation expectations. If these expectations were to rise rapidly, perhaps following a rapid increase in oil prices, unchecked expectations could get embedded into wage claims and eventually cause inflation to rise.

\section{Transparency}
One way of establishing credibility is for a central bank to be transparent in its decision making. Many, if not all, independent inflation-targeting central banks produce a quarterly assessment of their economies. These Inflation reports, as they are usually known, give central banks' views on the range of indicators that they watch when they come to their (usually) monthly interest rate decision. They will consider and outline their views on the following subjects, usually in this order:

\begin{itemize}
  \item Broad money aggregates and credit conditions;

  \item Conditions in financial markets

  \item Developments in the real economy (e.g., the labour market); and

  \item Evolution of prices

\end{itemize}

Consideration of all of these important components of an economy is then usually followed by a forecast of growth and inflation over a medium-term horizon, usually two years.

By explaining their views on the economy and by being transparent in decision making, the independent, inflation-targeting central banks seek to gain reputation and credibility, making it easier to influence inflation expectations and hence ultimately easier to meet the inflation target.

\section{The Target}
Whether the target is set by the central bank or by the government for the central bank to hit, the level of the target and the horizon over which the target is to be hit is a crucial consideration in all inflation-targeting frameworks.

\section{Exhibit 9: A Range of Inflation Targets}
\section{Country/Region}
Australia

Canada

Euro-area

South Korea

New Zealand Australian Federal Reserve's target is inflation between $2.0 \%$ and $3.0 \%$.

Bank of Canada's target is CPI inflation within the $1.0 \%$ and $3.0 \%$ range.

ECB's target is CPI inflation close to, but below, a ceiling of $2 \%$.

Bank of Korea's target for 2010-2012 is CPI inflation within \textbackslash pm 1.0 percentage of $2.0 \%$.

Reserve Bank of New Zealand's target is to keep future inflation between $1.0 \%$ and $3.0 \%$ with a focus on the average future inflation rate near $2.0 \%$.

Sweden

Riksbank's target is CPI inflation within \textbackslash pm 1.0 percentage point of $2.0 \%$.

United Kingdom Bank of England's target is CPI inflation within \textbackslash pm 1.0 percentage point of $2.0 \%$

Source: Central bank websites (\href{http://www.bis.org/cbanks.htm}{http://www.bis.org/cbanks.htm}). Exhibit 9 shows that many central banks in developed economies target an inflation rate of 2 percent based on a consumer price index. Given that the operation of monetary policy is both art and science, the banks are normally allowed a range around the central target of +1 percent or -1 percent. For example, with a 2 percent target, they would be tasked to keep inflation between 1 percent and 3 percent. But why target 2 percent and not 0 percent?

The answer is that aiming to hit 0 percent could result in negative inflation, known as deflation. One of the limitations of monetary policy that we discuss below is its ability or inability to deal with periods of deflation. If deflation is something to be avoided, why not target 10 percent? The answer to this question is that levels of inflation that high would not be consistent with price stability; such a high inflation rate would further tend to be associated with high inflation volatility and uncertainty. Central bankers seem to agree that 2 percent is far enough away from the risks of deflation and low enough not to lead to destabilizing inflation shocks.

Finally, we should keep in mind that the headline inflation rate that is announced in most economies every month, and which is the central bank's target, is a measure of how much a basket of goods and services has risen over the previous twelve months. It is history. Furthermore, interest rate changes made today will take some time to have their full effect on the real economy as they make their way through the monetary transmission mechanism. It is for these two reasons that inflation targeters do not target current inflation but instead usually focus on inflation two years ahead.

Although inflation-targeting mandates may vary from country to country, they have common elements: the specification of an explicit inflation target, with permissible bounds, and a requirement that the central bank should be transparent in its objectives and policy actions. This is all usually laid out in legislation that imposes statutory obligations on the central bank. As mentioned earlier, New Zealand pioneered the inflation-targeting approach to monetary policy that has since been copied widely. Below is New Zealand's Policy Targets Agreement, which specifies the inflation-targeting mandate of its central bank, the Reserve Bank of New Zealand.

\section{Exhibit 10: New Zealand's Policy Targets Agreement}
"This agreement between the Minister of Finance and the Governor of the Reserve Bank of New Zealand (the Bank) is made under section 9 of the Reserve Bank of New Zealand Act 1989 (the Act). The Minister and the Governor agree as follows:

\section{Price stability}
a. Under Section 8 of the Act the Reserve Bank is required to conduct monetary policy with the goal of maintaining a stable general level of prices.

b. The Government's economic objective is to promote a growing, open and competitive economy as the best means of delivering permanently higher incomes and living standards for New Zealanders. Price stability plays an important part in supporting this objective.

\section{Policy target}
a. In pursuing the objective of a stable general level of prices, the Bank shall monitor prices as measured by a range of price indexes. The price stability target will be defined in terms of the All Groups Consumers Price Index (CPI), as published by Statistics New Zealand.

b. For the purpose of this agreement, the policy target shall be to keep future CPI inflation outcomes between 1 per cent and 3 per cent on average over the medium term.

\section{Inflation variations around target}
a. For a variety of reasons, the actual annual rate of CPI inflation will vary around the medium-term trend of inflation, which is the focus of the policy target. Amongst these reasons, there is a range of events whose impact would normally be temporary. Such events include, for example, shifts in the aggregate price level as a result of exceptional movements in the prices of commodities traded in world markets, changes in indirect taxes, ${ }^{9}$ significant government policy changes that directly affect prices, or a natural disaster affecting a major part of the economy.

b. When disturbances of the kind described in clause 3(a) arise, the Bank will respond consistent with meeting its medium-term target.

\section{Communication, implementation and accountability}
a. On occasions when the annual rate of inflation is outside the medium-term target range, or when such occasions are projected, the Bank shall explain in Policy Statements made under section 15 of the Act why such outcomes have occurred, or are projected to occur, and what measures it has taken, or proposes to take, to ensure that inflation outcomes remain consistent with the medium-term target.

b. In pursuing its price stability objective, the Bank shall implement monetary policy in a sustainable, consistent and transparent manner and shall seek to avoid unnecessary instability in output, interest rates and the exchange rate.

c. The Bank shall be fully accountable for its judgments and actions in implementing monetary policy."

Source: \href{http://www.rbnz.govt.nz/}{http://www.rbnz.govt.nz/}.

To summarize, an inflation-targeting framework normally has the following set of features:

\begin{itemize}
  \item An independent and credible central bank;

  \item A commitment to transparency;

  \item A decision-making framework that considers a wide range of economic and financial market indicators; and

  \item A clear, symmetric and forward-looking medium-term inflation target, sufficiently above 0 percent to avoid the risk of deflation but low enough to ensure a significant degree of price stability.

\end{itemize}

Indeed, independence, credibility, and transparency are arguably the crucial ingredients for an effective central bank, whether they target inflation or not.

\section{The Main Exceptions to the Inflation-Targeting Rule}
Although the practice of inflation targeting is widespread, there are two prominent central banks that have not adopted a formal inflation target along the lines of the New Zealand model: the Bank of Japan and the US Federal Reserve System.

\section{The Bank of Japan}
Japan's central bank, the Bank of Japan (BoJ), does not target an explicit measure of inflation. Japan's government and its monetary authorities have been trying to combat deflation for much of the last two decades. However, despite their efforts-including the outright printing of money-inflation

9 "Indirect taxes" refer to such taxes as sales taxes and value-added taxes that are levied on goods and services rather than directly on individuals and companies. has remained very weak. Inflation targeting is seen very much as a way of combating and controlling inflation; as such, it would seem to have no place in an economy that suffers from persistent deflation.

Some economists have argued, however, that an inflation target is exactly what the Japanese economy needs. By announcing that positive inflation of say 3 percent is desired by the central bank, this might become a self-fulfilling prophecy if Japanese consumers and companies factor this target into nominal wage and price contracts. But for economic agents to believe that the target will be achieved, they have to believe that the central bank is capable of achieving it. Given that the BoJ has failed to engineer persistent, positive inflation, it is debatable how much credibility Japanese households and corporations would afford such an inflation-targeting policy.

\section{The US Federal Reserve System}
It is perhaps rather ironic that the world's most influential central bank, the US Federal Reserve, which controls the supply of the world's de facto reserve currency, the US dollar, does not have an explicit inflation target. However, it is felt that the single-minded pursuit of inflation might not be compatible with the Fed's statutory goal as laid out in the Federal Reserve Act, which charges the Fed's board to:

"promote effectively the goals of maximum employment, stable prices, and moderate long-term interest rates."

In other words, it has been argued that inflation targeting might compromise the goal of "maximum employment." In practice, however, the Fed has indicated that it sees core inflation measured by the personal consumption expenditure (PCE) deflator of about, or just below, 2 percent as being compatible with "stable prices." Financial markets therefore watch this US inflation gauge very carefully in order to try and anticipate the rate actions of the Fed.

\section{Monetary Policy in Developing Countries}
Developing economies often face significant impediments to the successful operation of any monetary policy-that is, the achievement of price stability. These include:

\begin{itemize}
  \item the absence of a sufficiently liquid government bond market and developed interbank market through which monetary policy can be conducted;

  \item a rapidly changing economy, making it difficult to understand what the neutral rate might be and what the equilibrium relationship between monetary aggregates and the real economy might be;

  \item rapid financial innovation that frequently changes the definition of the money supply;

  \item a poor track record in controlling inflation in the past, making monetary policy intentions less credible; and

  \item an unwillingness of governments to grant genuine independence to the central bank.

\end{itemize}

Taken together, any or all of these impediments might call into question the effectiveness of any developing economy's monetary policy framework, making any related monetary policy goals difficult to achieve.

\section{EXAMPLE 9}
\section{Central Bank Effectiveness}
\begin{enumerate}
  \item The reason some inflation-targeting banks may target low inflation and not 0 percent inflation is best described by which of the following statements?
\end{enumerate}

A. Some inflation is viewed as being good for an economy.

B. Targeting zero percent inflation runs a higher risk of a deflationary outcome.

C. It is very difficult to eliminate all inflation from a modern economy.

\section{Solution to 1:}
B is correct. Inflation targeting is art, not science. Sometimes inflation will be above target and sometimes below. Were central banks to target zero percent, then inflation would almost certainly be negative on some occasions. If a deflationary mindset then sets in among economic agents, it might be difficult for the central bank to respond to this because they cannot cut interest rates much below zero.

\begin{enumerate}
  \setcounter{enumi}{1}
  \item The degree of credibility that a central bank is afforded by economic agents is important because:
A. they are the lender of last resort.
B. their targets can become self-fulfilling prophecies.
C. they are the monopolistic suppliers of the currency.
\end{enumerate}

\section{Solution to 2:}
$B$ is correct. If a central bank operates within an inflation-targeting regime and if economic agents believe that it will achieve its target, this expectation will become embedded into wage negotiations, for example, and become a self-fulfilling prophecy. Also, banks need to be confident that the central bank will lend them money when all other sources are closed to them; otherwise, they might curtail their lending drastically, leading to a commensurate reduction in money and economic activity.

\section{EXCHANGE RATE TARGETING}
explain the relationships between monetary policy and economic growth, inflation, interest, and exchange rates

contrast the use of inflation, interest rate, and exchange rate targeting by central banks

Many developing economies choose to operate monetary policy by targeting their currency's exchange rate, rather than an explicit level of domestic inflation. Such targeting involves setting a fixed level or band of values for the exchange rate against a major currency, with the central bank supporting the target by buying and selling the national currency in foreign exchange markets. There are recent examples of developed economies using such an approach. In the $1980 \mathrm{~s}$, following the failure of its policy of trying to control UK inflation by setting medium-term goals for money supply growth (see Example 5), the UK government decided to operate monetary policy such that the sterling's exchange rate equalled a pre-determined value in terms of German deutschemarks. The basic idea is that by tying a domestic economy's currency to that of an economy with a good track record on inflation, the domestic economy would effectively "import" the inflation experience of the low inflation economy.

Suppose that a developing country wished to maintain the value of its currency against the US dollar. The government and/or central bank would announce the currency exchange rate that they wished to target. To simplify matters, let us assume that the domestic inflation rates are very similar in both countries and that the monetary authorities of the developing economy have set an exchange rate target that is consistent with relative price levels in the two economies. Under these (admittedly unlikely) circumstances, in the absence of shocks, there would be no reason for the exchange rate to deviate significantly from this target level. So as long as domestic inflation closely mirrors US inflation, the exchange rate should remain close to its target (or within a target band). It is in this sense that a successful exchange rate policy imports the inflation of the foreign economy.

Now suppose that economic activity in the developing economy starts to rise rapidly and that domestic inflation in the developing economy rises above the level in the United States. With a freely floating exchange rate regime, the currency of the developing economy would start to fall against the dollar. To arrest this fall, and to protect the exchange rate target, the developing economy's monetary authority sells foreign currency reserves and buys its own currency. This has the effect of reducing the domestic money supply and increasing short-term interest rates. The developing economy experiences a monetary policy tightening which, if expected to bring down inflation, will cause its exchange rate to rise against the dollar.

By contrast, in a scenario in which inflation in the developing country fell relative to the United States, the central bank would need to sell the domestic currency to support the target, tending to increase the domestic money supply and reduce the rate of interest.

In practice, the interventions of the developing economy central bank will simply stabilize the value of its currency, with many frequent adjustments. But this simplistic example should demonstrate one very important fact: When the central bank or monetary authority chooses to target an exchange rate, interest rates and conditions in the domestic economy must adapt to accommodate this target and domestic interest rates and money supply can become more volatile.

The monetary authority's commitment to and ability to support the exchange rate target must be credible for exchange rate targeting to be successful. If that is not the case, then speculators may trade against the monetary authority. Speculative attacks forced sterling out of the European Exchange Rate Mechanism in 1992. The fixed exchange rate regime was abandoned and the United Kingdom allowed its currency to float freely. Eventually, the UK government adopted a formal inflation target in 1997. Similarly, in the Asian financial crisis of 1997-1998, Thailand's central bank tried to defend the Thai baht against speculative attacks for much of the first half of 1997 but then revealed at the beginning of July that it had no reserves left. The subsequent devaluation triggered a debt crisis for banks and companies that had borrowed in foreign currency, and contagion spread throughout Asia.

Despite these risks, many currencies are pegged to other currencies, most notably the US dollar. Exhibit 11 shows a list of some of the currencies that were pegged to (fixed against) the US dollar at the end of 2018. Other currencies operate under a "managed exchange rate policy," where they are allowed to fluctuate within a range that is maintained by a monetary authority via market intervention. Dollarization occurs when a country adopts the US dollar as their functional currency. This is stronger than pegging to the dollar because under dollarization the US dollar replaces the previous national currency. Exhibit 11 breaks out countries that peg their currency to the dollar and those that have adopted the US dollar as their currency.

\section{Exhibit 11: Select Currencies Pegged to the US Dollar, as of December 2018}
Pegged to USD

\begin{itemize}
  \item Bermuda

  \item Bahamas

  \item Lebanon

  \item Hong Kong SAR - Saudi Arabia

  \item Qatar

  \item United Arab Emirates

\end{itemize}

\section{Dollarized}
\begin{itemize}
  \item Panama (1904)

  \item Ecuador (2000)

  \item East Timor (2001) - El Salvador (2000)

  \item Caribbean Netherlands (2011)

\end{itemize}

\section{EXAMPLE 10}
\section{Exchange Rate Targeting}
\begin{enumerate}
  \item When the central bank chooses to target a specific value for its exchange rate:
\end{enumerate}

A. it must also target domestic inflation.

B. it must also set targets for broad money growth.

C. conditions in the domestic economy must adapt to accommodate this target.

\section{Solution to 1:}
$\mathrm{C}$ is correct. The adoption of an exchange rate target requires that the central bank set interest rates to achieve this target. If the target comes under pressure, domestic interest rates may have to rise, regardless of domestic conditions. It may have a "target" level of inflation in mind as well as "targets" for broad money growth, but as long as it targets the exchange rate, domestic inflation and broad money trends must simply be allowed to evolve.

\begin{enumerate}
  \setcounter{enumi}{1}
  \item With regard to monetary policy, what is the hoped for benefit of adopting an exchange rate target?
\end{enumerate}

A. Freedom to pursue redistributive fiscal policy

B. Freedom to set interest rates according to domestic conditions

C. To "import" the inflation experience of the economy whose currency is being targeted

\section{Solution to 2:}
$\mathrm{C}$ is correct. Note that interest rates have to be set to achieve this target and are therefore subordinate to the exchange rate target and partially dependent on economic conditions in the foreign economy. 3. Which of the following is least likely to be an impediment to the successful implementation of monetary policy in developing economies?

A. Fiscal deficits

B. Rapid financial innovation

C. Absence of a liquid government bond market

\section*{Solution to 3: }
A is correct. Note that the absence of a liquid government bond market through which a central bank can enact open market operations and/or repo transactions will inhibit the implementation of monetary policy-as would rapid financial innovation because such innovation can change the relationship between money and economic activity. Fiscal deficits, on the other hand, are not normally an impediment to the implementation of monetary policy, although they could be if they were perceived to be unsustainable.

\section{Q}
\section{MONETARY POLICIES: CONTRACTIONARY, EXPANSIONARY, LIMITATIONS}
Most central banks will adjust liquidity conditions by adjusting their official policy rate. ${ }^{10}$ When they believe that economic activity is likely to lead to an increase in inflation, they might increase interest rates, thereby reducing liquidity. In these cases, market analysts describe such actions as contractionary because the policy is designed to cause the rate of growth of the money supply and the real economy to contract (see Exhibit 7 for the possible transmission mechanism here). Conversely, when the economy is slowing and inflation and monetary trends are weakening, central banks may increase liquidity by cutting their target rate. In these circumstances, monetary policy is said to be expansionary.

Thus, when policy rates are high, monetary policy may be described as contractionary; when low, they may be described as expansionary. But what are they "high" and "low" in comparison to?

The neutral rate of interest is often taken as the point of comparison. One way of characterizing the neutral rate is to say that it is that rate of interest that neither spurs on nor slows down the underlying economy. As such, when policy rates are above the neutral rate, monetary policy is contractionary; when they are below the neutral rate, monetary policy is expansionary. The neutral rate should correspond to the average policy rate over a business cycle.

10 Although, if they have reduced their policy rate to 0 percent, to increase liquidity further they have to resort to less-conventional monetary policy measures. However, economists' views of the neutral rate for any given economy might differ, and therefore, their view of whether monetary policy is contractionary, neutral, or expansionary might differ too. What economists do agree on is that the neutral policy rate for any economy comprises two components:

\begin{itemize}
  \item Real trend rate of growth of the underlying economy, and

  \item Long-run expected inflation.

\end{itemize}

The real trend rate of growth of an economy is also difficult to discern, but it corresponds to that rate of economic growth that is achievable in the long run that gives rise to stable inflation. If we are thinking about an economy with a credible inflation-targeting regime, where the inflation target is say 2 percent per year and where an analyst believes that the economy can grow sustainably over the long term at a rate of 2.5 percent per year, then they might also estimate the neutral rate to be:

Neutral rate $=$ Trend growth + Inflation target $=2.5 \%+2 \%=4.5 \%$

The analyst would therefore describe the central bank's monetary policy as being contractionary when its policy rate is above 4.5 percent and expansionary when it is below this level.

In practice, central banks often indicate what they believe to be the neutral rate of interest for their economy too. But determining this "neutral rate" is more art than science. For example, many analysts have recently revised down their estimates of trend growth for many western countries following the collapse of the credit bubble, because in many cases, the governments and private individuals of these economies are now being forced to reduce consumption levels and pay down their debts.

\section{What's the Source of the Shock to the Inflation Rate?}
An important aspect of monetary policy for those charged with its conduct is the determination of the source of any shock to the inflation rate. Suppose that the monetary authority sees that inflation is rising beyond its target, or simply in a way that threatens price stability. If this rise was caused by an increase in the confidence of consumers and business leaders, which in turn has led to increases in consumption and investment growth rates, then we could think of it as being a demand shock. In this instance, it might be appropriate to tighten monetary policy in order to bring the inflationary pressures generated by these domestic demand pressures under control.

However, suppose instead that the rise in inflation was caused by a rise in the price of oil (for the sake of argument). In this case, the economy is facing a supply shock, and raising interest rates might make a bad situation worse. Consumers are already facing an increase in the cost of fuel prices that might cause profits and consumption to fall and eventually unemployment to rise. Putting up interest rates in this instance might simply exacerbate the oil price-induced downturn, which might ultimately cause inflation to fall sharply.

It is important, then, for the monetary authority to try to identify the source of the shock before engineering a contractionary or expansionary monetary policy phase.

\section{Limitations of Monetary Policy}
The limitations of monetary policy include problems in the transmission mechanism and the relative ineffectiveness of interest rate adjustment as a policy tool in deflationary environments.

\section{Problems in the Monetary Transmission Mechanism}
In Exhibit 7, we presented a stylized representation of the monetary policy transmission mechanism, including the channels of bank lending rates, asset prices, expectations, and exchange rates. The implication of the diagram is that there are channels through which the actions of the central bank or monetary authority are transmitted to both the nominal and real economy. However, there may be some occasions when the will of the monetary authority is not transmitted seamlessly through the economy.

Suppose that a central bank raises interest rates because it is concerned about the strength of underlying inflationary pressures. Long-term interest rates are influenced by the path of expected short-term interest rates, so the outcome of the rate hike will depend on market expectations. Suppose that bond market participants think that short-term rates are already too high, that the monetary authorities are risking a recession, and that the central bank will likely undershoot its inflation target. This fall in inflation expectations could cause long-term interest rates to fall. That would make long-term borrowing cheaper for companies and households, which could in turn stimulate economic activity rather than cause it to contract.

Arguably, the more credible the monetary authority, the more stable the long end of the yield curve; moreover, the monetary authority will be more confident that its "policy message" will be transmitted throughout the economy. A term recently used in the marketplace is bond market vigilantes. These "vigilantes" are bond market participants who might reduce their demand for long-term bonds, thus pushing up their yields, if they believe that the monetary authority is losing its grip on inflation. That yield increase could act as a brake on any loose monetary policy stance. Conversely, the vigilantes may push long-term rates down by increasing their demand for long-dated government bonds if they expect that tight monetary policy is likely to cause a sharp slowdown in the economy, thereby loosening monetary conditions for long-term borrowers in the economy.

A credible monetary policy framework and authority will tend not to require the vigilantes to do the work for it.

In very extreme instances, there may be occasions where the demand for money becomes infinitely elastic-that is, where the demand curve is horizontal and individuals are willing to hold additional money balances without any change in the interest rate-so that further injections of money into the economy will not serve to further lower interest rates or affect real activity. This is known as a liquidity trap. In this extreme circumstance, monetary policy can become completely ineffective. The economic conditions for a liquidity trap are associated with the phenomenon of deflation.

\section{Interest Rate Adjustment in a Deflationary Environment and Quantitative Easing as a Response}
Deflation is a pervasive and persistent fall in a general price index and is more difficult for conventional monetary policy to deal with than inflation. This is because cutting nominal interest rates much below zero to stimulate the economy is difficult. ${ }^{11}$ It is at this point that the economic conditions for a liquidity trap arise.

Deflation raises the real value of debt, while the persistent fall in prices can encourage consumers to put off consumption today, leading to a fall in demand that leads to further deflationary pressure. Thus a deflationary "trap" can develop, which is characterized by weak consumption growth, falling prices, and increases in real debt levels. Japan eventually found itself in such a position following the collapse of its property bubble in the early 1990s.

11 Interest rates were cut to below zero in several European countries in 2014 and subsequently in Japan in 2016. If conventional monetary policy-the adjustment of short-term interest rates-is no longer capable of stimulating the economy once the zero or even negative nominal interest rate bound has been reached, is monetary policy useless?

In the aftermath of the collapse of the high-tech bubble in November 2002, Federal Reserve Governor (now Chairman) Ben Bernanke gave a speech entitled "Deflation: Making Sure 'It' Doesn't Happen Here.' In this speech, Bernanke stated that inflation was always and everywhere a monetary phenomenon, and he expressed great confidence that by expanding the money supply by various means (including dropping it out of a helicopter on the population below), the Federal Reserve as the monopoly supplier of money could always engineer positive inflation in the US economy. He said:

I am confident that the Fed would take whatever means necessary to prevent significant deflation in the United States and, moreover, that the US central bank, in cooperation with other parts of the government as needed, has sufficient policy instruments to ensure that any deflation that might occur would be both mild and brief.

Following the collapse of the credit bubble in 2008, a number of governments along with their central banks cut rates to (near) zero, including those in the United States and the United Kingdom. However, there was concern that the underlying economies might not respond to this drastic monetary medicine, mainly because the related banking crisis had caused banks to reduce their lending drastically. In order to kick start the process, both the Federal Reserve and the Bank of England effectively printed money and pumped it in to their respective economies. This "unconventional" approach to monetary policy, known as quantitative easing (QE), is operationally similar to open market purchase operations but conducted on a much larger scale.

The additional reserves created by central banks in a policy of quantitative easing can be used to buy any assets. The Bank of England chose to buy gilts (bonds issued by the UK government), where the focus was on gilts with three to five years maturity. The idea was that this additional reserve would kick-start lending, causing broad money growth to expand, which would eventually lead to an increase in real economic activity. But there is no guarantee that banks will respond in this way. In a difficult economic climate, it may be better to hold excess reserves rather than to lend to households and businesses that may default.

In the United States, the formal plan for QE mainly involved the purchase of mortgage bonds issued or guaranteed by Freddie Mac and Fannie Mae. Part of the intention was to push down mortgage rates to support the US housing market, as well as to increase the growth rate of broad money. Before implementing this formal program, the Federal Reserve intervened in several other markets that were failing for lack of liquidity, including interbank markets and the commercial paper market. These interventions had a similar effect on the Federal Reserve's balance sheet and the money supply as the later $\mathrm{QE}$ program.

This first round of $\mathrm{QE}$ by the Federal Reserve was then followed by a further round of QE, known as QE2. In November 2010, the Federal Reserve judged that the US economy had not responded sufficiently to the first round of QE (QE1). The Fed announced that it would create $\$ 600$ billion and use this money to purchase long-dated US Treasuries in equal tranches over the following eight months. The purpose of QE2 was to ensure that long bond yields remained low in order to encourage businesses and households to borrow for investment and consumption purposes, respectively.

The final round of QE, known as QE3, was implemented in September 2012 to provide $\$ 40$ billion per month to purchase agency mortgage-backed securities "until the labor market improved substantially." QE3 lasted until December 2013, when the Federal Reserve announced it was tapering back on these purchases. These purchases, and quantitative easing, ended 10 months later in October 2014. As long as central banks have the appropriate authority from the government, they can purchase any assets in a quantitative easing program. But the risks involved in purchasing assets with credit risk should be clear. In the end, the central bank is just a special bank. If it accumulates bad assets that then turn out to create losses, it could face a fatal loss of confidence in its main product: fiat money.

\section{Limitations of Monetary Policy: Summary}
The ultimate problem for monetary authorities as they try to manipulate the supply of money in order to influence the real economy is that they cannot control the amount of money that households and corporations put in banks on deposit, nor can they easily control the willingness of banks to create money by expanding credit. Taken together, this also means that they cannot always control the money supply. Therefore, there are definite limits to the power of monetary policy.

\section{EXAMPLE 11}
\section{The Limits of Monetary Policy: The Case of Japan}
\section{The Background}
Between the 1950s and 1980s, Japan's economy achieved faster real growth than any other G7 economy. But the terrific success of the economy sowed the seeds of the problems that were to follow. The very high real growth rates achieved by Japan over four decades became built in to asset prices, particularly equity and commercial property prices. Toward the end of the 1980s, asset prices rose to even higher levels when the Bank of Japan followed a very easy monetary policy as it tried to prevent the Japanese yen from appreciating too much against the US dollar. However, when interest rates went up in 1989-1990 and the economy slowed, investors eventually came to believe that the growth assumptions that were built in to asset prices and other aspects of the Japanese economy were unrealistic. This realization caused Japanese asset prices to collapse. For example, the Nikkei 225 stock market index reached 38,915 in 1989; by the end of March 2003, it had fallen by 80 percent to 7,972. The collapse in asset prices caused wealth to decline dramatically. Consumer confidence understandably fell sharply too, and consumption growth slowed. Corporate spending also fell, while bank lending contracted sharply in the weak economic climate. Although many of these phenomena are apparent in all recessions, the situation was made worse when deflation set in. In an environment when prices are falling, consumers may put off discretionary spending today until tomorrow; by doing this, however, they exacerbate the deflationary environment. Deflation also raises the real value of debts; as deflation takes hold, borrowers find the real value of their debts rising and may try to increase their savings accordingly. Once again, such actions exacerbate the recessionary conditions.

\section{The Monetary Policy Response}
Faced with such a downturn, the conventional monetary policy response is to cut interest rates to try to stimulate real economic activity. The Japanese central bank, the Bank of Japan, cut rates from 8 percent in 1990 to 1 percent by 1996. By February 2001, the Japanese policy rate was cut to zero where it stayed.

Once rates are at or near zero, there are two broad approaches suggested by theory, though the two are usually complementary. First, the central bank can try to convince markets that interest rates will remain low for a long time, even after the economy and inflation pick up. This will tend to lower interest rates along the yield curve. Second, the central bank can try to increase the money supply by purchasing assets from the private sector, so-called quantitative easing. The Bank of Japan (BoJ) did both in 2001. It embarked on a program of quantitative easing supplemented by an explicit promise not to raise short-term interest rates until deflation had given way to inflation.

Quantitative easing simply involves the printing of money by the central bank. In practice, this involved the BoJ using open market operations to add reserves to the banking system through the direct purchase of government securities in the open market.

The reserve levels became the new target. The BoJ's monetary policy committee determined the level of reserves and the quantity of bond purchases that should be undertaken, rather than voting on the policy rate.

The success of this policy is difficult to judge. As the chart below shows, although deflation turned to inflation for a while, it returned to deflation in 2008-2009 when the Japanese economy suffered a sharp recession along with much of the rest of the world. At that time, having reversed its QE policy during 2004-2008 by reducing its bond holdings, the Bank of Japan began to buy again.

The Bank of Japan ramped up its asset purchases starting in 2013, when other central banks began to unwind their QE programs. At the beginning of 2013, BoJ Assets to Japanese GDP were approximately $30 \%$. By mid-2018, BoJ assets to Japanese GDP were almost 100\%! Economists debate the point, but arguably, even the Bank of Japan's much larger program of $\mathrm{QE}$ has not been able to eliminate deflation. The Japanese experience suggests that there may be limits to the power of monetary policy.

\section{Exhibit 12: Inflation and Deflation in Japan}
\begin{center}
\includegraphics[max width=\textwidth]{2023_05_04_ccb578c5089fd41615beg-331}
\end{center}

Source: \href{http://www.statbureau.org/en/japan/inflation-tables}{www.statbureau.org/en/japan/inflation-tables}.

\section{EXAMPLE 12}
\section{Evaluating Monetary Policy}
\begin{enumerate}
  \item If an economy's trend GDP growth rate is 3 percent and its central bank has a 2 percent inflation target, which policy rate is most consistent with an expansionary monetary policy?
A. 4 percent
B. 5 percent
C. 6 percent
\end{enumerate}

Solution to 1:

A is correct. The neutral rate of interest, which in this example is 5 percent, is considered to be that rate of interest that neither spurs on nor slows down the underlying economy. As such, when policy rates are above the neutral rate, monetary policy is contractionary; when they are below the neutral rate, monetary policy is expansionary. It comprises two components: the real trend rate of growth of the underlying economy (in this example, 3 percent) and long-run expected inflation (in this example, 2 percent).

\begin{enumerate}
  \setcounter{enumi}{1}
  \item An increase in a central bank's policy rate might be expected to reduce inflationary pressures by:
A. reducing consumer demand.
B. reducing the foreign exchange value of the currency.
C. driving up asset prices leading to an increase in personal sector wealth.
\end{enumerate}

\section{Solution to 2:}
A is correct. If an increase in the central bank's policy rate is successfully transmitted via the money markets to other parts of the financial sector, consumer demand might decline as the rate of interest on mortgages and other credit rises. This decline in consumer demand should, all other things being equal and amongst other affects, lead to a reduction in upward pressure on consumer prices.

\begin{enumerate}
  \setcounter{enumi}{2}
  \item Which of the following statements best describes a fundamental limitation of monetary policy? Monetary policy is limited because central bankers:
\end{enumerate}

A. cannot control the inflation rate perfectly.

B. are appointed by politicians and are therefore never truly independent.

C. cannot control the amount of money that economic agents put in banks, nor the willingness of banks to make loans.

\section{Solution to 3:}
$\mathrm{C}$ is correct. Central bankers do not control the decisions of individuals and banks that can influence the money creation process.

\section{ROLES AND OBJECTIVES OF FISCAL POLICY}
describe roles and objectives of fiscal policy

The second set of tools used for influencing economic activity consists of the tools associated with fiscal policy. These involve the use of government spending and changing tax revenue to affect a number of aspects of the economy:

\begin{itemize}
  \item Overall level of aggregate demand in an economy and hence the level of economic activity.

  \item Distribution of income and wealth among different segments of the population.

  \item Allocation of resources between different sectors and economic agents.

\end{itemize}

Often, a discussion of fiscal policy focuses on the impact of changes in the difference between government spending and revenue on the aggregate economy, rather than on the actual levels of spending and revenue themselves.

\section{Roles and Objectives of Fiscal Policy}
A primary aim for fiscal policy is to help manage the economy through its influence on aggregate national output, that is, real GDP.

\section{Fiscal Policy and Aggregate Demand}
Aggregate demand is the amount companies and households plan to spend. We can consider a number of ways that fiscal policy can influence aggregate demand. For example, an expansionary policy could take one or more of the following forms:

\begin{itemize}
  \item Cuts in personal income tax raise disposable income with the objective of boosting aggregate demand.

  \item Cuts in sales (indirect) taxes to lower prices which raises real incomes with the objective of raising consumer demand.

  \item Cuts in corporation (company) taxes to boost business profits, which may raise capital spending.

  \item Cuts in tax rates on personal savings to raise disposable income for those with savings, with the objective of raising consumer demand.

  \item New public spending on social goods and infrastructure, such as hospitals and schools, boosting personal incomes with the objective of raising aggregate demand.

\end{itemize}

We must stress, however, that the reliability and magnitude of these relationships will vary over time and from country to country. For example, in a recession with rising unemployment, it is not always the case that cuts in income taxes will raise consumer spending because consumers may wish to raise their precautionary (rainy day) saving in anticipation of further deterioration in the economy. Indeed, in very general terms economists are often divided into two camps regarding the workings of fiscal policy: Keynesians believe that fiscal policy can have powerful effects on aggregate demand, output, and employment when there is substantial spare capacity in an economy. Monetarists believe that fiscal changes only have a temporary effect on aggregate demand and that monetary policy is a more effective tool for restraining or boosting inflationary pressures. Monetarists tend not to advocate using monetary policy for countercyclical adjustment of aggregate demand. This intellectual division will naturally be reflected in economists' divergent views on the efficacy of the large fiscal expansions observed in many countries following the credit crisis of 2008, along with differing views on the possible impact of quantitative easing.

\section{Government Receipts and Expenditure in Major Economies}
In Exhibit 13, we present the total government revenues as a percentage of GDP for some major economies. This is the share of a country's output that is gathered by the government through taxes and such related items as fees, charges, fines, and capital transfers. It is often considered as a summary measure of the extent to which a government is involved both directly and indirectly in the economic activity of a country.

Taxes are formally defined as compulsory, unrequited payments to the general government (they are unrequited in the sense that benefits provided by a government to taxpayers are usually not related to payments). Exhibit 13 contains taxes on incomes and profits, social security contributions, indirect taxes on goods and services, employment taxes, and taxes on the ownership and transfer of property.

Exhibit 13: General Government Revenues as Percent of GDP

\begin{center}
\begin{tabular}{lllllll}
\hline
 & $\mathbf{1 9 9 5}$ & $\mathbf{2 0 0 0}$ & $\mathbf{2 0 0 5}$ & $\mathbf{2 0 0 8}$ & $\mathbf{2 0 1 0}$ & $\mathbf{2 0 1 5}$ \\
\hline
Australia & 34.5 & 36.1 & 36.5 & 35.3 & 32.4 & 34.9 \\
Germany & 45.1 & 46.4 & 43.6 & 43.8 & 43.0 & 44.5 \\
Japan & 31.2 & 31.4 & 31.7 & 34.4 & 30.6 & 35.7 \\
United Kingdom & 38.2 & 40.3 & 40.8 & 42.2 & 38.2 & 38.0 \\
United States & 33.8 & 35.4 & 33.0 & 32.3 & 30.9 & 33.4 \\
OECD & 37.9 & 39.0 & 37.7 & 37.9 & 39.8 & 40.9 \\
\hline
\end{tabular}
\end{center}

Source: Organisation for Economic Co-Operation and Development (OECD).

Taxes on income and profits have been fairly constant for the Organisation for Economic Co-Operation and Development (OECD) countries overall at around 12.5-13 percent of GDP since the mid-1990s, while taxes on goods and services have been steady at about 11 percent of GDP for that period. Variations between countries can be substantial; taxes on goods and services are around 5 percent of GDP for the United States and Japan but over 16 percent for Denmark.

Exhibit 14 shows the percentage of GDP represented by government expenditure in a variety of major economies over time. Generally, these have been fairly constant since 1995, though Germany had a particularly high number at the start of the period because of reunification costs. The impacts of governments' fiscal stimulus programs in the face of the 2008-2009 financial crisis show up as significant increases in government expenditures in Exhibit 14 and increases in government deficits in Exhibit 15 between 2008 and 2010.

\section{Exhibit 14: General Government Expenditures as Percent of GDP}
\begin{center}
\begin{tabular}{lcccccc}
\hline
 & $\mathbf{1 9 9 5}$ & $\mathbf{2 0 0 0}$ & $\mathbf{2 0 0 5}$ & $\mathbf{2 0 0 8}$ & $\mathbf{2 0 1 0}$ & $\mathbf{2 0 1 5}$ \\
\hline
Australia & 38.2 & 35.2 & 34.8 & 34.3 & 34.4 & 36.2 \\
Germany & 54.8 & 45.1 & 46.9 & 43.8 & 47.3 & 43.9 \\
Japan & 36.0 & 39.0 & 38.4 & 37.1 & 39.6 & 39.4 \\
United Kingdom & 44.1 & 36.6 & 44.0 & 47.5 & 47.6 & 42.2 \\
\end{tabular}
\end{center}

\begin{center}
\begin{tabular}{lcccccc}
\hline
 & $\mathbf{1 9 9 5}$ & $\mathbf{2 0 0 0}$ & $\mathbf{2 0 0 5}$ & $\mathbf{2 0 0 8}$ & $\mathbf{2 0 1 0}$ & $\mathbf{2 0 1 5}$ \\
\hline
United States & 37.1 & 33.9 & 36.2 & 38.8 & 42.9 & 37.6 \\
OECD & 42.7 & 38.7 & 40.5 & 41.4 & 45.2 & 41.8 \\
\hline
\end{tabular}
\end{center}

Source: OECD.

Clearly, the possibility that fiscal policy can influence output means that it may be an important tool for economic stabilization. In a recession, governments can raise spending (expansionary fiscal policy) in an attempt to raise employment and output. In boom times-when an economy has full employment and wages and prices are rising too fast-then government spending may be reduced and taxes raised (contractionary fiscal policy).

Hence, a key concept is the budget surplus/deficit, which is the difference between government revenue and expenditure for a fixed period of time, such as a fiscal or calendar year. Government revenue includes tax revenues net of transfer payments; government spending includes interest payments on the government debt. Analysts often focus on changes in the budget surplus or deficit from year to year as indicators of whether the fiscal policy is getting tighter or looser. An increase in a budget surplus would be associated with contractionary fiscal policy, while a rise in a deficit is an expansionary fiscal policy. Of course, over the course of a business cycle the budget surplus will vary automatically in a countercyclical way. For example, as an economy slows and unemployment rises, government spending on social insurance and unemployment benefits will also rise and add to aggregate demand. This is known as an automatic stabilizer. Similarly, if boom conditions ensue and employment and incomes are high, then progressive income and profit taxes are rising and also act as automatic stabilizers increasing budget surplus or reducing budget deficit. The great advantage of automatic stabilizers is that they are indeed automatic, not requiring the identification of shocks to which policymakers must consider a response. By reducing the responsiveness of the economy to shocks, these automatic stabilizers reduce output fluctuations. Automatic stabilizers should be distinguished from discretionary fiscal policies, such as changes in government spending or tax rates, which are actively used to stabilize aggregate demand. If government spending and revenues are equal, then the budget is balanced.

Exhibit 15: General Government Net Borrowing or Lending as Percent of GDP

\begin{center}
\begin{tabular}{lcccccc}
\hline
 & $\mathbf{1 9 9 5}$ & $\mathbf{2 0 0 0}$ & $\mathbf{2 0 0 5}$ & $\mathbf{2 0 0 8}$ & $\mathbf{2 0 1 0}$ & $\mathbf{2 0 1 5}$ \\
\hline
Australia & -3.7 & 0.9 & 1.7 & -3.8 & -4.4 & -2.2 \\
Germany & -9.7 & 1.3 & -3.3 & -0.2 & -4.2 & 0.8 \\
Japan & -4.7 & -7.6 & -6.7 & -4.1 & -9.1 & -3.6 \\
United Kingdom & -5.8 & 3.7 & -3.3 & -5.1 & -9.4 & -4.2 \\
United States & -3.3 & 1.5 & -3.3 & -7.0 & -12.0 & -4.2 \\
OECD & -4.8 & 0.2 & -2.7 & -1.5 & -5.1 & -1.9 \\
\hline
\end{tabular}
\end{center}

Source: OECD.

\section{EXAMPLE 13}
\section{Sources and Uses of Government Cash Flows: The Case of the United Kingdom}
The precise components of revenue and expenditure will of course vary over time and between countries. But, as an example of the breakdown of expenditure and revenue, in Exhibit 16 and Exhibit 17 we have presented the budget projections of the United Kingdom for $2018 / 2019$. The budget projected that total spending would come to $\pounds 808 \mathrm{bn}$, while total revenue would only be $\pounds 769 \mathrm{bn}$. The government was therefore forecasting a budget shortfall of $\pounds 39 \mathrm{bn}$ for the fiscal year, meaning that it had an associated need to borrow $\pounds 39 \mathrm{bn}$ from the private sector in the United Kingdom or the private and public sectors of other economies.

Exhibit 16: Where Does the Money Go? The United Kingdom, 2018-2019

Note: All values are in billions of pounds.

\begin{center}
\includegraphics[max width=\textwidth]{2023_05_04_ccb578c5089fd41615beg-336}
\end{center}

Source: HM Treasury, United Kingdom. Exhibit 17: Where Does the Money Come From? The United Kingdom, 2018-2019

Note: All values are in billions of pounds.

\begin{center}
\includegraphics[max width=\textwidth]{2023_05_04_ccb578c5089fd41615beg-337}
\end{center}

Source: HM Treasury, United Kingdom.

\section{DEFICITS AND THE NATIONAL DEBT}
describe the arguments about whether the size of a national debt relative to GDP matters

Government deficits are the difference between government revenues and expenditures over a period of calendar time, usually a year. Government (or national) debt is the accumulation over time of these deficits. Government deficits are financed by borrowing from the private sector, often via private pension and insurance fund portfolio investments. We saw above that governments are more likely to have deficits than surpluses over long periods of time. As a result, there may exist a large stock of outstanding government debt owned by the private sector. This will vary as the business cycle ebbs and flows. Exhibit 18 shows the time path of the ratio of public debt to GDP for the United Kingdom over several hundred years. It can be clearly seen that the major cause of fluctuations in that ratio through history has been the financing of wars, in particular the Napoleonic Wars of 1799-1815 and the First and Second World Wars of 1914-1918 and 1939-1945.

\section{Exhibit 18: UK National Debt as Percent of GDP (1692-2018)}
Debt as Percent of GDP

\begin{center}
\includegraphics[max width=\textwidth]{2023_05_04_ccb578c5089fd41615beg-338}
\end{center}

Source: \href{http://ukpublicspending.co.uk}{http://ukpublicspending.co.uk}.

With the onset of the credit crisis of 2008, governments actively sought to stimulate their economies through increased expenditures without raising taxes and revenues. This led to increased borrowing, shown in Exhibits 15 and Exhibit 19, which has become a concern in the financial markets in 2010 for such countries as Greece. Indeed, between 2008 and 2009, central government debt rose from $\$ 1.2$ trillion to $\$ 1.6$ trillion in the United Kingdom and from $\$ 5.8$ trillion to $\$ 7.5$ trillion for the United States. ${ }^{12}$ The fiscal expansion by governments in the face of the financial crisis seems to have significantly raised the General Government Debt to GDP ratio over the long term for many countries, as illustrated in Exhibit 19.

\section{Exhibit 19: General Government Debt as Percent of GDP}
\begin{center}
\begin{tabular}{lcccccc}
\hline
 & $\mathbf{1 9 9 5}$ & $\mathbf{2 0 0 0}$ & $\mathbf{2 0 0 5}$ & $\mathbf{2 0 0 8}$ & $\mathbf{2 0 1 0}$ & $\mathbf{2 0 1 5}$ \\
\hline
Australia & 57.3 & 41.1 & 30.0 & 30.0 & 41.9 & 64.1 \\
Germany & 54.1 & 59.5 & 70.1 & 68.1 & 84.5 & 78.9 \\
Japan & 94.7 & 142.6 & 176.2 & 181.6 & 207.5 & 237.4 \\
United Kingdom & 51.4 & 48.7 & 51.3 & 63.3 & 88.8 & 111.7 \\
United States & 83.2 & 61.7 & 79.0 & 93.2 & 117.0 & 125.3 \\
OECD & 65.8 & 59.9 & 59.5 & 60.8 & 73.0 & 85.3 \\
\hline
\end{tabular}
\end{center}

Source: \href{http://www.oecd.org}{www.oecd.org}.

12 \href{http://www.oecd.org}{www.oecd.org}. Ultimately, if the ratio of debt to GDP rises beyond a certain unknown point, then the solvency of the country comes into question. An additional indicator for potential insolvency is the ratio of interest rate payments to GDP, which is shown for some major economies in Exhibit 20. These represent payments required of governments to service their debts as a percentage of national output and as such reflect both the size of debts and the interest charged on them. Such ratios could rise rapidly with the growing debt ratios of 2009 and 2010, particularly if the interest rates on the debt were to rise from the historically low levels.

\section{Exhibit 20: General Government Net Debt Interest Payments as Percent of GDP}
\begin{center}
\begin{tabular}{lcccccc}
\hline
 & $\mathbf{1 9 9 5}$ & $\mathbf{2 0 0 0}$ & $\mathbf{2 0 0 5}$ & $\mathbf{2 0 0 8}$ & $\mathbf{2 0 1 0}$ & $\mathbf{2 0 1 5}$ \\
\hline
Australia & 3.5 & 1.7 & 1.0 & -0.5 & 0.0 & 0.3 \\
Germany & 2.9 & 2.7 & 2.4 & 2.3 & 2.1 & 0.9 \\
Japan & 1.3 & 1.5 & 0.8 & 0.3 & 0.6 & 0.4 \\
United Kingdom & 3.1 & 2.4 & 1.8 & 1.7 & 2.6 & 2.0 \\
United States & 3.5 & 2.5 & 1.8 & 2.6 & 2.9 & 2.8 \\
OECD & 3.6 & 2.5 & 1.8 & 1.9 & 2.1 & 1.9 \\
\hline
\end{tabular}
\end{center}

Source: OECD.

Governments' spending was far in excess of revenues following the credit crisis of 2007-2010 as governments tried to stimulate their economies; this level of spending raised concerns in some quarters about the scale of governmental debt accumulation. Exhibit 19 shows that government debt relative to GDP for the OECD countries overall rose from 59.5 percent in 2005 to 85.3 percent in 2015. In Japan, where fiscal spending has been used to stimulate the economy from the early 1990s, the ratio has risen from 94.7 percent in 1995 to 237.4 percent in 2015. If an economy grows in real terms, so do the real tax revenues and hence the ability to service a growing real debt at constant tax rate levels. However, if the real growth in the economy is lower than the real interest rate on the debt, then the debt ratio will worsen even though the economy is growing because the debt burden (i.e., the real interest rate times the debt) grows faster than the economy. Hence, an important issue for governments and their creditors is whether their additional spending leads to sufficiently higher tax revenues to pay the interest on the debt used to finance the extra spending.

However, within a national economy, the real value of the outstanding debt will fall if the overall price level rises (i.e., inflation, and hence a rise in nominal GDP even if real GDP is static) and thus the ratio of debt to GDP may not be rising. But if the general price level falls (i.e., deflation), then the ratio may stay elevated for longer. If net interest payments rise rapidly and investors lose confidence in a government's ability to honour its debts, then financing costs may escalate even more quickly and make the situation unstable.

Should we be concerned about the size of a national debt (relative to GDP)? There are strong arguments both for and against:

The arguments against being concerned about national debt (relative to GDP) are as follows:

\begin{itemize}
  \item The scale of the problem may be overstated because the debt is owed internally to fellow citizens. This is certainly the case in Japan and South Korea, where 93 percent is owned by local residents. Canada is similar with 90 percent owned by residents. However, other countries have a much lower percentage owned internally. In the United States and United Kingdom, the figures are 53 percent and 73 percent, respectively, while Italy has only 46 percent owned by local residents. ${ }^{13}$

  \item A proportion of the money borrowed may have been used for capital investment projects or enhancing human capital (e.g., training, education); these should lead to raised future output and tax revenues.

  \item Large fiscal deficits require tax changes which may actually reduce distortions caused by existing tax structures.

  \item Deficits may have no net impact because the private sector may act to offset fiscal deficits by increasing saving in anticipation of future increased taxes. This argument is known as "Ricardian equivalence" and is discussed in more detail later.

  \item If there is unemployment in an economy, then the debt is not diverting activity away from productive uses (and indeed the debt could be associated with an increase in employment).

\end{itemize}

The arguments in favour of being concerned are:

\begin{itemize}
  \item High levels of debt to GDP may lead to higher tax rates in the search for higher tax revenues. This may lead to disincentives to economic activity as the higher marginal tax rates reduce labour effort and entrepreneurial activity, leading to lower growth in the long run.

  \item If markets lose confidence in a government, then the central bank may have to print money to finance a government deficit. This may lead ultimately to high inflation, as evidenced by the economic history of Germany in the 1920s and more recently in Zimbabwe.

  \item Government borrowing may divert private sector investment from taking place (an effect known as crowding out); if there is a limited amount of savings to be spent on investment, then larger government demands will lead to higher interest rates and lower private sector investing.

\end{itemize}

An important distinction to make is between long- and short-run effects. Over short periods of time (say, a few years), crowding out may have little effect. If it lasts for a longer time, however, then capital accumulation in an economy may be damaged. Similarly, tax distortions may not be too serious over the short-term but will have a more substantial impact over many years.

\section{EXAMPLE 14}
\section{Types of Fiscal Policies}
\begin{enumerate}
  \item Which of the following is not associated with an expansionary fiscal policy?
\end{enumerate}

A. A rise in capital gains taxes

B. Cuts in personal income taxes

C. New capital spending by the government on road building

13 These data come from the Bank for International Settlements (BIS), IMF, and central bank websites. All figures are as of 2018 .

\section{Solution to 1:}
A is correct. A rise in capital gains taxes reduces income available for spending and hence reduces aggregate demand, other things being equal. Cutting income tax raises disposable income, while new road building raises employment and incomes; in both cases, aggregate demand rises and hence policy is expansionary.

\begin{enumerate}
  \setcounter{enumi}{1}
  \item Fiscal expansions will most likely have the most impact on aggregate output when the economy is in which of the following states?
\end{enumerate}

A. Full employment

B. Near full employment

C. Considerable unemployment

\section{Solution to 2:}
$\mathrm{C}$ is correct. When an economy is close to full employment a fiscal expansion raising aggregate demand can have little impact on output because there are few spare unused resources (e.g., labour or idle factories); instead, there will be upward pressure on prices (i.e., inflation).

\begin{enumerate}
  \setcounter{enumi}{2}
  \item Which one of the following is most likely a reason to not use fiscal deficits as an expansionary tool?
\end{enumerate}

A. They may crowd out private investment.

B. They may facilitate tax changes to reduce distortions in an economy.

C. They may stimulate employment when there is substantial unemployment in an economy.

\section{Solution to 3:}
A is correct. A frequent argument against raises in fiscal deficits is that the additional borrowing to fund the deficit in financial markets will displace private sector borrowing for investment (i.e., "crowd it out").

FISCAL POLICY TOOLS

describe tools of fiscal policy, including their advantages and disadvantages

We now look at the nature of the fiscal tools available to a government. Government spending can take a variety of forms:

\begin{itemize}
  \item Transfer payments are welfare payments made through the social security system and, depending on the country, comprise payments for state pensions, housing benefits, tax credits and income support for poorer families, child benefits, unemployment benefits, and job search allowances. Transfer payments exist to provide a basic minimum level of income for low-income households, and they also provide a means by which a government can change the overall income distribution in a society. Note that these payments are not included in the definition of GDP because they do not reflect a reward to a factor of production for economic activity. Also, they are not considered to be part of general government spending on goods and services.

  \item Current government spending involves spending on goods and services that are provided on a regular, recurring basis-including health, education, and defense. Clearly, such spending will have a big impact on a country's skill level and overall labour productivity.

  \item Capital expenditure includes infrastructure spending on roads, hospitals, prisons, and schools. This investment spending will add to a nation's capital stock and affect productive potential for an economy.

\end{itemize}

Government spending can be justified on both economic and social grounds:

\begin{itemize}
  \item To provide such services as defense that benefit all citizens equally.

  \item For infrastructure capital spending (e.g., roads) to help a country's economic growth.

  \item To guarantee a minimum level of income for poorer people and hence redistribute income and wealth (e.g., welfare and related benefits).

  \item To influence a government's economic objectives of low inflation and high employment and growth (e.g., management of aggregate demand).

  \item To subsidize the development of innovative and high-risk new products or markets (e.g., alternative energy sources).

\end{itemize}

Government revenues can take several forms:

\begin{itemize}
  \item Direct taxes are levied on income, wealth, and corporate profits and include capital gains taxes, national insurance (or labour) taxes, and corporate taxes. They may also include a local income or property tax for both individuals and businesses. Inheritance tax on a deceased's estate will have both revenue-raising and wealth-redistribution aspects.

  \item Indirect taxes are taxes on spending on a variety of goods and services in an economy-such as the excise duties on fuel, alcohol, and tobacco as well as sales (or value-added tax)-and often exclude health and education products on social grounds. In addition, taxes on gambling may also be considered to have a social aspect in deterring such activity, while fuel duties will have an environmental purpose in making fuel consumption and hence travel more expensive.

\end{itemize}

Taxes can be justified both in terms of raising revenues to finance expenditures and in terms of income and wealth redistribution policies. Economists typically consider four desirable attributes of a tax policy:

\begin{itemize}
  \item Simplicity: This refers to ease of compliance by the taxpayer and enforcement by the revenue authorities. The final liability should be certain and not easily manipulated.

  \item Efficiency: Taxation should interfere as little as possible in the choices individuals make in the market place. Taxes affect behaviour and should, in general, discourage work and investment as little as possible. A major philosophical issue among economists is whether tax policy should deliberately deviate from efficiency to promote "good" economic activities, such as savings, and discourage harmful ones, such as tobacco consumption. Although most would accept a limited role in guiding consumer choices, some will question if policymakers are equipped to decide on such objectives and whether there will be unwanted ancillary effects, such as giving tax breaks for saving among people who already save and whose behaviour does not change.

  \item Fairness: This refers to the fact that people in similar situations should pay the same taxes ("horizontal equity") and that richer people should pay more taxes ("vertical equity"). Of course, the concept of fairness is really subjective. Still, most would agree that income tax rates should be progressivethat is, that households and corporations should pay proportionately more as their incomes rise. However, some people advocate "flat" tax rates, whereby all should pay the same proportion of taxable income.

  \item Revenue sufficiency: Although revenue sufficiency may seem obvious as a criterion for tax policy, there may be a conflict with fairness and efficiency. For example, one may believe that increasing income tax rates to reduce fiscal deficits reduces labour effort and that tax rate increases are thus an inefficient policy tool.

\end{itemize}

\section{EXAMPLE 15}
\section{Some Issues with Tax Policy}
\begin{enumerate}
  \item Incentives. Some economists believe that income taxes reduce the incentive to work, save, and invest and that the overall tax burden has become excessive. These ideas are often associated with supply-side economics and the US economist Arthur Laffer. A variety of income tax cuts and simplifications have taken place in the United States since 1981, and although there is substantial controversy, some claim that work effort did rise (although tax cuts had little impact on savings). Similarly, some found that business investment did rise, while others claimed it was independent of such cuts.

  \item Fairness. How do we judge the fairness of the tax system? One way is to calibrate the tax burden falling on different groups of people ranked by their income and to assess how changes in taxes affect these groups. Of course, this imposes huge data demands on investigators and must be considered incomplete. In the United States, it has been found that the federal system is indeed highly progressive. Many countries use such methods to analyze the impact of tax changes on different income groups when they announce their annual fiscal policy plans.

  \item Tax reform. There is continuous debate on reforming tax policy. Should there be a flat-rate tax on labour income? Should all investment be immediately deducted for corporate taxes? Should more revenue be sourced from consumption taxes? Should taxes be indexed to inflation? Should dividends be taxed when profits have already been subject to tax? Should estates be taxed at all? Many of these issues are raised in the context of their impact on economic growth.

\end{enumerate}

\section{EXAMPLE 16}
\section{Fiscal Tools}
\begin{enumerate}
  \item Which of the following is not a tool of fiscal policy?
\end{enumerate}

A. A rise in social transfer payments

B. The purchase of new equipment for the armed forces

C. An increase in deposit requirements for the buying of houses

Solution to 1:

$\mathrm{C}$ is correct. Rises in deposit requirements for house purchases are intended to reduce the demand for credit for house purchases and hence would be considered a tool of monetary policy. This is a policy used actively in several countries, and is under consideration by regulators in other countries to constrain house price inflation.

\begin{enumerate}
  \setcounter{enumi}{1}
  \item Which of the following is not an indirect tax?
\end{enumerate}

A. Excise duty

B. Value-added Tax

C. Employment taxes

Solution to 2:

$\mathrm{C}$ is correct. Both excise duty and VAT are applied to prices, whereas taxes on employment apply to labour income and hence are not indirect taxes.

\begin{enumerate}
  \setcounter{enumi}{2}
  \item Which of the following statements is most accurate?
\end{enumerate}

A. Direct taxes are useful for discouraging alcohol consumption.

B. Because indirect taxes cannot be changed quickly, they are of no use in fiscal policy.

C. Government capital spending decisions are slow to plan, implement, and execute and hence are of little use for short-term economic stabilization.

\section{Solution to 3:}
$\mathrm{C}$ is correct. Capital spending is much slower to implement than changes in indirect taxes; and indirect taxes affect alcohol consumption more directly than direct taxes.

\section{The Advantages and Disadvantages of Using the Different Tools of Fiscal Policy}
The different tools used to expedite fiscal policy as a means to try to put or keep an economy on a path of positive, stable growth with low inflation have both advantages and disadvantages:

\section{Advantages:}
\begin{itemize}
  \item Indirect taxes can be adjusted almost immediately after they are announced and can influence spending behaviour instantly and generate revenue for the government at little or no cost to the government. - Social policies, such as discouraging alcohol or tobacco use, can be adjusted almost instantly by raising such taxes.
\end{itemize}

\section{Disadvantages:}
\begin{itemize}
  \item Direct taxes are more difficult to change without considerable notice, often many months, because payroll computer systems will have to be adjusted (although the announcement itself may well have a powerful effect on spending behaviour more immediately). The same may be said for welfare and other social transfers.

  \item Capital spending plans take longer to formulate and implement, typically over a period of years. For example, building a road or hospital requires detailed planning, legal permissions, and implementation. This is often a valid criticism of an active fiscal policy and was widely heard during the US fiscal stimulus in 2009-2010. On the other hand, such policies add to the productive potential of an economy, unlike a change in personal or indirect taxes. Of course, the slower the impact of a fiscal change, the more likely other exogenous changes will already be influencing the economy before the fiscal change kicks in.

\end{itemize}

The above-mentioned tools may also have expectational effects at least as powerful as the direct effects. The announcement of future income tax rises a year ahead could potentially lead to reduced consumption immediately. Such delayed tax rises were a feature of UK fiscal policy of 2009-2010; however, the evidence is anecdotal because spending behaviour changed little until the delayed tax changes actually came into force.

We may also consider the relative potency of the different fiscal tools. Direct government spending has a far bigger impact on aggregate spending and output than income tax cuts or transfer increases; however, if the latter are directed at the poorest in society (basically, those who spend all their income), then this will give a relatively strong boost. Further discussion and examples of these comparisons are given in section 4 below on the interaction between monetary and fiscal policy.

Modeling the Impact of Taxes and Government Spending: The Fiscal Multiplier

The conventional macroeconomic model has government spending, $G$, adding directly to aggregate demand, $A D$, and reducing it via taxes, $T$; these comprise both indirect taxes on expenditures and direct taxes on factor incomes. Further government spending is increased via the payment of transfer benefits, $B$, such as social security payments. Hence, the net impact of the government sector on aggregate demand is:

$$
G-T+B=\text { Budget surplus OR deficit }
$$

Net taxes (NT; taxes less transfers) reduce disposable income $(Y D)$ available to individuals relative to national income or output $(Y)$ as follows:

$$
Y D=Y-N T=(1-t) Y
$$

where $t$ is the net tax rate. Net taxes are often assumed to be proportional to national income, $Y$, and hence total tax revenue from net taxes is $t Y$. If $t=20 \%$ or 0.2 , then for every $\$ 1$ rise in national income, net tax revenue will rise by 20 cents and household disposable income will rise by 80 cents.

The fiscal multiplier is important in macroeconomics because it tells us how much output changes as exogenous changes occur in government spending or taxation. The recipients of the increase in government spending will typically save a proportion $1-c$ of each additional dollar of disposable income, where $c$ is the marginal propensity to consume (MPC) this additional income. Ignoring income taxes, we can see that $\$ c$ will, in turn, be spent by these recipients on more goods and services. The recipients of this $\$ c$ will themselves spend a proportion $c$ of this additional income (i.e., $\$ c \times c$, or $c$-squared). This process continues with income and spending growing at a constant rate of $c$ as it passes from hand to hand through the economy. This is the familiar geometric progression with constant factor $c$, where $0<c<1$. The sum of this geometric series is $1 /(1-c)$.

We define $s$ as the marginal propensity to save (MPS), the amount saved out of an additional dollar of disposable income. Because $c+s=1$, hence $s=1-c$.

\section{Exhibit 21: Disposable Income, Saving, and the MPC}
\begin{center}
\begin{tabular}{lcccc}
\hline
Income & Income tax & Disposable income & Consumption & Saving \\
\hline
$\$ 100$ & $\$ 20$ & $\$ 80$ & $\$ 72$ & $\$ 8$ \\
\hline
\end{tabular}
\end{center}

In Exhibit 21, the MPC out of disposable income is $90 \%$ or $0.9(72 / 80)$. The MPS is therefore $1-0.9$ or 0.1

For every dollar of new (additional) spending, total incomes and spending rises by $\$ 1 /(1-c)$. And because $0<c<1$, this must be $>1$; this is the multiplier. If $c=0.9$ (or individuals spend 90 percent of additions to income), then the multiplier $=1 /$ $(1-0.9)=10$.

A formal definition of the multiplier would be the ratio of the change in equilibrium output to the change in autonomous spending that caused the change. This is a monetary measure, but because prices are assumed to be constant in this analysis, real and monetary amounts are identical. Given that fiscal policy is about changes in government spending, $G$, net taxes, $N T$, and tax rates, $t$, we can see that the multiplier is an important tool for calibrating the possible impact of policy changes on output. How can we introduce tax changes into the multiplier concept? We do this by introducing the idea of disposable income, $Y D$, defined as income less income taxes net of transfers, $Y-N T$.

Households spend a proportion $c$ of disposable income, $Y D$, that is, $c Y D$ or $c(Y$ $-N T)$ or $c(1-t) Y$. The marginal propensity to consume in the presence of taxes is then $c(1-t)$. If the government increases spending, say on road building, by an amount, $G$, then disposable income rises by $(1-t) G$ and consumer spending by $c(1-t)$ G. Provided there are unused sources of capital and labour in the economy, this leads to a rise in aggregate demand and output; the recipients of this extra consumption spending will have $(1-t) c(1-t) G$ extra disposable income available and will spend $c$ of it. This cumulative extra spending and income will continue to spread through the economy at a decreasing rate as $0<c(1-t)<1$. The overall final impact on aggregate demand and output will effectively be the sum of this decreasing geometric series with common ratio $c(1-t)$, and this sums to $1 /[1-c(1-t)]$. This is known as the fiscal multiplier and is very relevant to studies of fiscal policy as changes in $G$ or tax rates will affect output in an economy through the value of the multiplier.

For example, if the tax rate is 20 percent, or 0.2 , and the marginal propensity to consume is 90 percent, or 0.9 , then the fiscal multiplier will be: $1 /[1-0.9(1-0.2)]$ or $1 / 0.28=3.57$. In other words, if the government raises $G$ by $\$ 1$ billion, total incomes and spending rise by $\$ 3.57$ billion.

Discretionary fiscal policy (see below) will involve changes in these variables with a view to influencing $Y$.

\section{The Balanced Budget Multiplier}
If a government increases $G$ by the same amount as it raises taxes, the aggregate output actually rises. Why is this?

It is because the marginal propensity to consume out of disposable income is less than 1 , and hence for every dollar less in $Y D$, spending only falls $\$ c$. Hence, aggregate spending falls less than the tax rise by a factor of $c$. A balanced budget leads to a rise in output, which in turn leads to further rises in output and incomes via the multiplier effect.

Suppose an economy has an equilibrium output or income level of $\$ 1,000$ consisting of $\$ 900$ of consumption and $\$ 100$ of investment spending, which is fixed and not related to income. If government spending is set at $\$ 200$, financed by a tax rate of 20 percent (giving tax revenue of $\$ 200$ ), what will happen to output? First, additional government spending of $\$ 200$ will raise output by that amount; but will taxes of $\$ 200$ reduce output by a similar amount? Not if the MPC is less than 1 ; suppose it is 0.9 , and hence spending will only fall by 90 percent of $\$ 200$, or $\$ 180$. The initial impact of the balanced fiscal package on aggregate demand will be to raise it by $\$ 200-\$ 180$ $=\$ 20$. This additional output will, in turn, lead to further increases in income and output through the multiplier effect.

Even though the above policy involved a combination of government spending and tax increases that initially left the government's budget deficit/surplus unchanged, the induced rise in output will lead to further tax revenue increases and a further change in the budget position. Could the government adjust the initial change in spending to offset exactly the eventual total change in tax revenues? The answer is "yes," and we can ask what will be the effect on output of this genuinely balanced budget change? This balanced budget multiplier always takes the value unity.

\section{EXAMPLE 17}
\section{Government Debt, Deficits, and Ricardo}
The total stock of government debt is the outstanding stock of IOUs issued by a government and not yet repaid. They are issued when the government has insufficient tax revenues to meet expenditures and has to borrow from the public. The size of the outstanding debt equals the cumulative quantity of net borrowing it has done, and the fiscal or budget deficit is added in the current period to the outstanding stock of debt. If the outstanding stock of debt falls, we have a negative deficit or a surplus.

If a government reduces taxation by $\$ 10$ billion one year and replaces that revenue with borrowing of $\$ 10$ billion from the public, will it have any real impact on the economy? The important issue here is how people perceive that action: Do they recognize what will happen over time as interest and bond principal have to be repaid out of future taxes? If so, they may think of the bond finance as equivalent to delayed taxation finance; thus, the reduction in current taxation will have no impact on spending because individuals save more in anticipation of higher future taxes to repay the bond. This is called Ricardian equivalence after the economist David Ricardo. If people do not correctly anticipate all the future taxes required to repay the additional government debt, then they feel wealthier when the debt is issued and may increase their spending, adding to aggregate demand.

Whether Ricardian equivalence holds in practice is ultimately an empirical issue and is difficult to calibrate conclusively given the number of things that are changing at any time in a modern economy. FISCAL POLICY IMPLEMENTATION

$$
\begin{aligned}
& \text { explain the implementation of fiscal policy and difficulties of } \\
& \text { implementation } \\
& \text { determine whether a fiscal policy is expansionary or contractionary }
\end{aligned}
$$

In the following, we discuss major issues in fiscal policy implementation.

\section{Deficits and the Fiscal Stance}
An important question is the extent to which the budget is a useful measure of the government's fiscal stance. Does the size of the deficit actually indicate whether fiscal policy is expansionary or contractionary? Clearly, such a question is important for economic policymakers insofar as the deficit can change for reasons unrelated to actual fiscal policy changes. For example, the automatic stabilizers mentioned earlier will lead to changes in the budget deficit unrelated to fiscal policy changes; a recession will cause tax revenues to fall and the budget deficit to rise. An observer may conclude that fiscal policy has been loosened and is expansionary and that no further government action is required.

To this end, economists often look at the structural (or cyclically adjusted) budget deficit as an indicator of the fiscal stance. This is defined as the deficit that would exist if the economy was at full employment (or full potential output). Hence, if we consider a period of relatively high unemployment, such as 2009-2010 with around 9-10 percent of the workforce out of work in the United States and Europe, then the budget deficits in those countries would be expected to be reduced substantially if the economies returned to full employment. At this level, tax revenues would be higher and social transfers lower. Recent data for major countries are given in Exhibit 22, where negative numbers refer to deficits and positive numbers are surpluses.

Exhibit 22: General Government Cyclically Adjusted Balances as Percent of GDP

\begin{center}
\begin{tabular}{lcccccc}
\hline
 & $\mathbf{1 9 9 5}$ & $\mathbf{2 0 0 0}$ & $\mathbf{2 0 0 5}$ & $\mathbf{2 0 0 8}$ & $\mathbf{2 0 1 0}$ & $\mathbf{2 0 1 5}$ \\
\hline
Australia & -3.1 & 0.9 & 2.0 & -0.4 & -3.8 & -0.1 \\
Germany & -9.5 & 0.9 & -2.6 & -0.8 & -3.3 & 0.7 \\
Japan & -4.6 & -6.4 & -4.1 & -4.0 & -8.2 & -3.6 \\
United Kingdom & -5.6 & 0.8 & -4.5 & -5.6 & -7.6 & -4.3 \\
United States & -2.9 & -0.4 & -5.4 & -7.1 & -10.0 & -3.5 \\
OECD & -4.6 & -1.2 & -3.6 & -4.5 & -6.9 & -2.0 \\
\hline
\end{tabular}
\end{center}

Source: OECD Economic Outlook, Volume 2018 Issue 1.

A further reason why actual government deficits may not be a good measure of fiscal stance is the distinction between real and nominal interest rates and the role of inflation adjustment when applied to budget deficits. Although national economic statistics treat the cash interest payments on debt as government expenditure it makes more sense to consider only the inflation-adjusted (or real) interest payments because the real value of the outstanding debt is being eroded by inflation. Automatic stabilizers-such as income tax, VAT, and social benefits-are important because as output and employment fall and reduce tax revenues, so net tax revenues also fall as unemployment benefits rise. This acts as a fiscal stimulus and serves to reduce the size of the multiplier, dampening the output response of whatever caused the fall in output in the first place. By their very nature, automatic stabilizers do not require policy changes; no policymaker has to decide that an economic shock has occurred and how to respond. Hence, the responsiveness of the economy to shocks is automatically reduced, as are movements in employment and output.

In addition to these automatic adjustments, governments also use discretionary fiscal adjustments to influence aggregate demand. These will involve tax changes and/or spending cuts or increases usually with the aim of stabilizing the economy. A natural question is why fiscal policy cannot stabilize aggregate demand completely, hence ensuring full employment at all times.

\section{Difficulties in Executing Fiscal Policy}
Fiscal policy cannot stabilize aggregate demand completely because the difficulties in executing fiscal policy cannot be completely overcome.

First, the policymaker does not have complete information on how the economy functions. It may take several months for policymakers to realize that an economy is slowing, because data appear with a considerable time lag and even then are subject to substantial revision. This is often called the recognition lag and has been likened to the problem of driving with the rear view mirror. Then, when policy changes are finally decided on, they may take many months to implement. This is the action lag. If a government decides to raise spending on capital projects to increase employment and incomes, for example, these may take many months to plan and put into action. Finally, the result of these actions on the economy will take additional time to become evident; this is the impact lag. These types of policy lags also occur in the case of discretionary monetary policy.

A second aspect of time in this process is the uncertainty of where the economy is heading independently of these policy changes. For example, a stimulus may occur simultaneously with a surprise rise in investment spending or in the demand for a country's exports just as discretionary government spending starts to rise. Macroeconomic forecasting models do not generally have a good track record for accuracy and hence cannot be relied on to aid the policy-making process in this context. In addition, when discretionary fiscal adjustments are announced (or are already underway), private sector behaviour may well change leading to rises in consumption or investment, both of which will reinforce the effects of a rise in government expenditure. Again, this will make it difficult to calibrate the required fiscal adjustment to secure full employment.

There are wider macroeconomic issues also involved here.

\begin{itemize}
  \item If the government is concerned with both unemployment and inflation in an economy, then raising aggregate demand toward the full employment level may also lead to a tightening labour market and rising wages and prices. The policymaker may be reluctant to further fine tune fiscal policy in an uncertain world because it might induce inflation.

  \item If the budget deficit is already large relative to GDP and further fiscal stimulus is required, then the necessary increase in the deficit may be considered unacceptable by the financial markets when government funding is raised, leading to higher interest rates on government debt and political pressure to tackle the deficit. - Of course, all this presupposes that we know the level of full employment, which is difficult to measure accurately. Fiscal expansion raises demand, but what if we are already at full employment, which will be changing as productive capacity changes and workers' willingness to work at various wage levels changes?

  \item If unused resources reflect a low supply of labour or other factors rather than a shortage of demand, then discretionary fiscal policy will not add to demand and will be ineffective, raising the risk of inflationary pressures in the economy.

  \item The issue of crowding out may occur: If the government borrows from a limited pool of savings, the competition for funds with the private sector may crowd out private firms with subsequent less investing and economic growth. In addition, the cost of borrowing may rise, leading to the cancellation of potentially profitable opportunities. This concept is the subject of continuing empirical debate and investigation.

\end{itemize}

\section{EXAMPLE 18}
\section{Evaluating Fiscal Policy}
\begin{enumerate}
  \item Which of the following statements is least accurate?
\end{enumerate}

A. The economic data available to policymakers have a considerable time lag.

B. Economic models always offer an unambiguous guide to the future path of the economy.

C. Surprise changes in exogenous economic variables make it difficult to use fiscal policy as a stabilization tool.

\section{Solution to 1:}
$\mathrm{B}$ is correct. Economic forecasts from models will always have an element of uncertainty attached to them and thus are not unambiguous or precise in their prescriptions. Once a fiscal policy decision has been made and implemented, unforeseen changes in other variables may affect the economy in ways that would lead to changes in the fiscal policy if we had perfect foresight. Note that it is true that official economic data may be available with substantial time lags, making fiscal judgements more difficult.

\begin{enumerate}
  \setcounter{enumi}{1}
  \item Which of the following statements is least accurate?
\end{enumerate}

A. Discretionary fiscal changes are aimed at stabilizing an economy.

B. In the context of implementing fiscal policy, the recognition lag is often referred to as "driving in the rear view mirror."

C. Automatic fiscal stabilizers include new plans for additional road building by the government.

\section{Solution to 2:}
$\mathrm{C}$ is correct. New plans for road building are discretionary and not automatic. 3. Which of the following statements regarding a fiscal stimulus is most accurate?

A. Accommodative monetary policy reduces the impact of a fiscal stimulus.

B. Different statistical models will predict different impacts for a fiscal stimulus.

C. It is always possible to predict precisely the impact of a fiscal stimulus on employment.

\section{Solution to 3:}
B is correct. Different models embrace differing views on how the economy works, including differing views on the impact of fiscal stimuli.

\begin{enumerate}
  \setcounter{enumi}{3}
  \item Which of the following statements is most accurate?
\end{enumerate}

A. An increase in the budget deficit is always expansionary.

B. An increase in government spending is always expansionary.

C. The structural deficit is always larger than the deficit below full employment.

\section{Solution to 4:}
$\mathrm{A}$ is correct. Note that increases in government spending may be accompanied by even bigger rises in tax receipts and hence may not be expansionary.

\begin{enumerate}
  \setcounter{enumi}{4}
  \item Crowding out refers to a:
\end{enumerate}

A. fall in interest rates that reduces private investment.

B. rise in private investment that reduces private consumption.

C. rise in government borrowing that reduces the ability of the private sector to access investment funds.

\section{Solution to 5:}
$C$ is correct. A fall in interest rates is likely to lead to a rise in investment. Crowding out refers to government borrowing that reduces the ability of the private sector to invest.

\begin{enumerate}
  \setcounter{enumi}{5}
  \item A contractionary fiscal policy will always involve which of the following?
\end{enumerate}

A. A balanced budget

B. A reduction in government spending

C. A fall in the budget deficit or rise in the surplus

\section{Solution to 6:}
$\mathrm{C}$ is correct. Note that a reduction in government spending could be accompanied by an even bigger fall in taxation, making it be expansionary.

\begin{enumerate}
  \setcounter{enumi}{6}
  \item Which one of the following statements is most accurate?
\end{enumerate}

A. Ricardian equivalence refers to individuals having no idea of future tax liabilities.

B. If there is high unemployment in an economy, then easy monetary and fiscal policies should lead to an expansion in aggregate demand. C. Governments do not allow political pressures to influence fiscal policies but do allow voters to affect monetary policies.

\section{Solution to 7:}
$\mathrm{B}$ is correct. Note that governments often allow pressure groups to affect fiscal policy and that Ricardian equivalence involves individuals correctly anticipating future taxes, so $\mathrm{A}$ and $\mathrm{C}$ are not correct choices.

\section{THE RELATIONSHIP BETWEEN MONETARY AND}
 FISCAL POLICYexplain the interaction of monetary and fiscal policy

Both monetary and fiscal policies can be used to try and influence the macroeconomy. But the impact of monetary policy on aggregate demand may differ depending on the fiscal policy stance. Conversely, the impact of fiscal policy might vary under various alternative monetary policy conditions. Clearly, policymakers need to understand this interaction. For example, they need to consider the impact of changes to the budget when monetary policy is accommodative as opposed to when it is restrictive: Can we expect the same impact on aggregate demand in both situations?

Although both fiscal and monetary policy can alter aggregate demand, they do so through differing channels with differing impact on the composition of aggregate demand. The two policies are not interchangeable. Consider the following cases in which the assumption is made that wages and prices are rigid:

\begin{itemize}
  \item Easy fiscal policy/tight monetary policy: If taxes are cut or government spending rises, the expansionary fiscal policy will lead to a rise in aggregate output. If this is accompanied by a reduction in money supply to offset the fiscal expansion, then interest rates will rise and have a negative effect on private sector demand. We have higher output and higher interest rates, and government spending will be a larger proportion of overall national income.

  \item Tight fiscal policy/easy monetary policy: If a fiscal contraction is accompanied by expansionary monetary policy and low interest rates, then the private sector will be stimulated and will rise as a share of GDP, while the public sector will shrink.

  \item Easy monetary policy/easy fiscal policy: If both fiscal and monetary policy are easy, then the joint impact will be highly expansionary-leading to a rise in aggregate demand, lower interest rates (at least if the monetary impact is larger), and growing private and public sectors.

  \item Tight monetary policy/tight fiscal policy: Interest rates rise (at least if the monetary impact on interest rates is larger) and reduce private demand. At the same time, higher taxes and falling government spending lead to a drop in aggregate demand from both public and private sectors.

\end{itemize}

\section{Factors Influencing the Mix of Fiscal and Monetary Policy}
Although governments are concerned about stabilizing the level of aggregate demand at close to the full employment level, they are also concerned with the growth of potential output. To this end, encouraging private investment will be important. It may best be achieved by accommodative monetary policy with low interest rates and a tight fiscal policy to ensure free resources for a growing private sector.

At other times, the lack of a good quality, trained workforce-or perhaps a modern capital infrastructure-will be seen as an impediment to growth; thus, an expansion in government spending in these areas may be seen as a high priority. If taxes are not raised to pay for this, then the fiscal stance will be expansionary. If a loose monetary policy is chosen to accompany this expansionary spending, then it is possible that inflation may be induced. Of course, it is an open question as to whether policymakers can judge the appropriate levels of interest rates or fiscal spending levels.

Clearly, the mix of policies will be heavily influenced by the political context. A weak government may raise spending to accommodate the demands of competing vested interests (e.g., subsidies to particular sectors, such as agriculture in the EC), and thus a restrictive monetary policy may be needed to hold back the possibly inflationary growth in aggregate demand through raised interest rates and less credit availability.

Both fiscal and monetary policies suffer from lack of precise knowledge of where the economy is today, because data appear initially subject to revision and with a time lag. However, fiscal policy suffers from two further issues with regard to its use in the short run.

As we saw earlier, it is difficult to implement quickly because spending on capital projects takes time to plan, procure, and put into practice. In addition, it is politically easier to loosen fiscal policy than to tighten it; in many cases, automatic stabilizers are the source of fiscal tightening, because tax rates are not changing and political opposition is muted. Similarly, the independence of many central banks means that decisions on raising interest rates are outside the hands of politicians and thus can be taken more easily.

The interaction between monetary and fiscal policies was also implicitly evident in our discussion of Ricardian equivalence because if tax cuts have no impact on private spending as individuals anticipate future higher taxes, then clearly this may lead policymakers to favour monetary tools.

Ultimately, the interaction of monetary and fiscal policies in practice is an empirical question, which we touched on earlier. In their detailed research paper using the IMF'S Global Integrated Monetary and Fiscal Model (IMF 2009), IMF researchers examined four forms of coordinated global fiscal loosening over a two-year period, which will be reversed gradually after the two years are completed. These are:

\begin{itemize}
  \item an increase in social transfers to all households,

  \item a decrease in tax on labour income,

  \item a rise in government investment expenditure, and

  \item a rise in transfers to the poorest in society.

\end{itemize}

The two types of monetary policy responses considered are:

\begin{itemize}
  \item no monetary accommodation, so rising aggregate demand leads to higher interest rates immediately; or

  \item interest rates are kept unchanged (accommodative policy) for the two years. The following important policy conclusions from this study emphasize the role of policy interactions:

  \item No monetary accommodation: Government spending increases have a much bigger effect (six times bigger) on GDP than similar size social transfers because the latter are not considered permanent, although real interest rates rise as monetary authorities react to rises in aggregate demand and inflation. Targeted social transfers to the poorest citizens have double the effect of the non-targeted transfers, while labour tax reductions have a slightly bigger impact than the latter.

  \item Monetary accommodation: Except for the case of the cut in labour taxes, fiscal multipliers are now much larger than when there is no monetary accommodation. The cumulative multiplier (i.e., the cumulative effect on real GDP over the two years divided by the percentage of GDP, which is a fiscal stimulus) is now 3.9 for government expenditure compared to 1.6 with no monetary accommodation. The corresponding numbers for targeted social transfer payments are 0.5 without monetary accommodation and 1.7 with it. The larger multiplier effects with monetary accommodation result from rises in aggregate demand and inflation, leading to falls in real interest rates and additional private sector spending (e.g., on investment goods). Labour tax cuts are less positive.

\end{itemize}

\section{Quantitative Easing and Policy Interaction}
What about the scenario of zero interest rates and deflation? Fiscal stimulus should still raise demand and inflation, lowering real interest rates and stimulating private sector demand. We saw earlier that quantitative easing has been a feature of major economies during 2009-2010. This involves the purchase of government or private securities by the central bank from individuals, institutions, or banks and substituting central bank balances for those securities. The ultimate aim is that recipients will subsequently increase expenditures, lending or borrowing in the face of raised cash balances and lower interest rates.

If the central bank purchases government securities on a large scale, it is effectively funding the budget deficit and the independence of monetary policy is an illusion. This so-called "printing of money" is feared by many economists as the monetization of the government deficit. Note that it is unrelated to the conventional inflation target of central banks, such as the Bank of England. Some economists question whether an independent central bank should engage in such activity.

\section{The Importance of Credibility and Commitment}
The IMF model implies that if governments run persistently high budget deficits, real interest rates rise and crowd out private investment, reducing each country's productive potential. As individuals realize that deficits will persist, inflation expectations and longer-term interest rates rise: This reduces the effect of the stimulus by half.

Further, if there is a real lack of commitment to fiscal discipline over the longer term, (e.g., because of aging populations) and the ratio of government debt to GDP rose by 10 percentage points permanently in the United States alone, then world real interest rates would rise by 0.14 percent-leading to a 0.6 percent permanent fall in world GDP.

\section{EXAMPLE 19}
\section{Interactions of Monetary and Fiscal Policy}
\begin{enumerate}
  \item In a world where Ricardian equivalence holds, governments would most likely prefer to use monetary rather than fiscal policy because under Ricardian equivalence:
\end{enumerate}

A. real interest rates have a more powerful effect on the real economy.

B. the transmission mechanism of monetary policy is better understood.

C. the future impact of fiscal policy changes are fully discounted by economic agents.

\section{Solution to 1:}
$\mathrm{C}$ is correct. If Ricardian equivalence holds, then economic agents anticipate that the consequence of any current tax cut will be future tax rises, which leads them to increase their saving in anticipation of this so that the tax cut has little effect on consumption and investment decisions. Governments would be forced to use monetary policy to affect the real economy on the assumption that money neutrality did not hold in the short term.

\begin{enumerate}
  \setcounter{enumi}{1}
  \item If fiscal policy is easy and monetary policy tight, then:
\end{enumerate}

A. interest rates would tend to fall, reinforcing the fiscal policy stance.

B. the government sector would tend to shrink as a proportion of total GDP.

C. the government sector would tend to expand as a proportion of total GDP.

\section{Solution to 2:}
$\mathrm{C}$ is correct. With a tight monetary policy, real interest rates should rise and reduce private sector activity, which could be at least partially offset by an expansion in government activity via the loosening of fiscal policy. The net effect, however, would be an expansion in the size of the public sector relative to the private sector.

\begin{enumerate}
  \setcounter{enumi}{2}
  \item Which of the following has the greatest impact on aggregate demand according to an IMF study? A 1 percent of GDP stimulus in:
A. government spending.
B. rise in transfer benefits.
C. cut in labour income tax across all income levels.
\end{enumerate}

\section{Solution to 3:}
A is correct. The study clearly showed that direct spending by the government leads to a larger impact on GDP than changes in taxes or benefits.

\section{SUMMARY}
In this reading, we have sought to explain the practices of both monetary and fiscal policy. Both can have a significant impact on economic activity, and it is for this reason that financial analysts need to be aware of the tools of both monetary and fiscal policy, the goals of the monetary and fiscal authorities, and most important the monetary and fiscal policy transmission mechanisms.

\begin{itemize}
  \item Governments can influence the performance of their economies by using combinations of monetary and fiscal policy. Monetary policy refers to central bank activities that are directed toward influencing the quantity of money and credit in an economy. By contrast, fiscal policy refers to the government's decisions about taxation and spending. The two sets of policies affect the economy via different mechanisms.

  \item Money fulfills three important functions: It acts as a medium of exchange, provides individuals with a way of storing wealth, and provides society with a convenient unit of account. Via the process of fractional reserve banking, the banking system can create money.

  \item The amount of wealth that the citizens of an economy choose to hold in the form of money-as opposed to, for example, bonds or equities-is known as the demand for money. There are three basic motives for holding money: transactions-related, precautionary, and speculative.

  \item The addition of 1 unit of additional reserves to a fractional reserve banking system can support an expansion of the money supply by an amount equal to the money multiplier, defined as $1 /$ reserve requirement (stated as a decimal).

  \item The nominal rate of interest is comprised of three components: a real required rate of return, a component to compensate lenders for future inflation, and a risk premium to compensate lenders for uncertainty (e.g., about the future rate of inflation).

  \item Central banks take on multiple roles in modern economies. They are usually the monopoly supplier of their currency, the lender of last resort to the banking sector, the government's bank and bank of the banks, and they often supervise banks. Although they may express their objectives in different ways, the overarching objective of most central banks is price stability.

  \item For a central bank to be able to implement monetary policy objectively, it should have a degree of independence from government, be credible, and be transparent in its goals and objectives.

  \item The ultimate challenge for central banks as they try to manipulate the supply of money to influence the economy is that they cannot control the amount of money that households and corporations put in banks on deposit, nor can they easily control the willingness of banks to create money by expanding credit. Taken together, this also means that they cannot always control the money supply. Therefore, there are definite limits to the power of monetary policy.

  \item The concept of money neutrality is usually interpreted as meaning that money cannot influence the real economy in the long run. However, by the setting of its policy rate, a central bank hopes to influence the real economy via the policy rate's impact on other market interest rates, asset prices, the exchange rate, and the expectations of economic agents. - Inflation targeting is the most common monetary policy-although exchange rate targeting is also used, particularly in developing economies. Quantitative easing attempts to spur aggregate demand by drastically increasing the money supply.

  \item Fiscal policy involves the use of government spending and revenue raising (taxation) to impact a number of aspects of the economy: the overall level of aggregate demand in an economy and hence the level of economic activity; the distribution of income and wealth among different segments of the population; and hence ultimately the allocation of resources between different sectors and economic agents.

  \item The tools that governments use in implementing fiscal policy are related to the way in which they raise revenue and the different forms of expenditure. Governments usually raise money via a combination of direct and indirect taxes. Government expenditure can be current on goods and services or can take the form of capital expenditure, for example, on infrastructure projects.

  \item As economic growth weakens, or when it is in recession, a government can enact an expansionary fiscal policy-for example, by raising expenditure without an offsetting increase in taxation. Conversely, by reducing expenditure and maintaining tax revenues, a contractionary policy might reduce economic activity. Fiscal policy can therefore play an important role in stabilizing an economy.

  \item Although both fiscal and monetary policy can alter aggregate demand, they work through different channels, the policies are therefore not interchangeable, and they conceivably can work against one another unless the government and central bank coordinate their objectives.

\end{itemize}

\section{REFERENCES}
Goodhart, Charles A.E. 1989. “The Conduct of Monetary Policy." Economic Journal, vol. 99, no. 396:293-346. 10.2307/2234028

Gray, Simon, Nick Talbot. 2006. Monetary Operations. London: Bank of England (\href{http://www}{http://www}. \href{http://bankofengland.co.uk/education/ccbs/handbooks/ccbshb24.htm}{bankofengland.co.uk/education/ccbs/handbooks/ccbshb24.htm}).

IMF (International Monetary Fund) 2009. “The Case for Global Fiscal Stimulus” (March): http:// \href{http://www.imf.org/external/pubs/ft/spn/2009/spn0903.pdf}{www.imf.org/external/pubs/ft/spn/2009/spn0903.pdf}.

Roger, Scott. 2010. "Inflation Targeting Turns 20." Finance and Development, vol. 47, no. 1:46-49(March).

Truman, Edwin. 2003. Inflation Targeting in the World Economy. Washington, DC: Institute for International Economics.

\section{PRACTICE PROBLEMS}
\begin{enumerate}
  \item As the reserve requirement increases, the money multiplier:
A. increases.
B. decreases.
C. remains the same.

  \item Which is the most accurate statement regarding the demand for money?

\end{enumerate}

A. Precautionary money demand is directly related to GDP.

B. Transactions money demand is inversely related to returns on bonds.

C. Speculative demand is inversely related to the perceived risk of other assets.

\begin{enumerate}
  \setcounter{enumi}{2}
  \item The following exhibit shows the supply and demand for money:
\end{enumerate}

\begin{center}
\includegraphics[max width=\textwidth]{2023_05_04_ccb578c5089fd41615beg-359}
\end{center}

There is an excess supply of money when the nominal rate of interest is:
A. $I_{0}$.
B. $I_{1}$.
C. $I_{2}$.

\begin{enumerate}
  \setcounter{enumi}{3}
  \item According to the theory of money neutrality, money supply growth does not affect variables such as real output and employment in:
A. the long run.
B. the short run.
C. the long and short run.

  \item Which of the following best describes a fundamental assumption when monetary policy is used to influence the economy?

\end{enumerate}

A. Financial markets are efficient. B. Money is not neutral in the short run.

C. Official rates do not affect exchange rates.

\begin{enumerate}
  \setcounter{enumi}{5}
  \item Monetarists are most likely to believe:
\end{enumerate}

A. there is a causal relationship running from inflation to money.

B. inflation can be affected by changing the money supply growth rate.

C. rapid financial innovation in the market increases the effectiveness of monetary policy.

\begin{enumerate}
  \setcounter{enumi}{6}
  \item The proposition that the real interest rate is relatively stable is most closely associated with:
\end{enumerate}

A. the Fisher effect.

B. money neutrality.

C. the quantity theory of money.

\begin{enumerate}
  \setcounter{enumi}{7}
  \item Which of the following equations is a consequence of the Fisher effect?
A. Nominal interest rate $=$ Real interest rate + Expected rate of inflation.
B. Real interest rate $=$ Nominal interest rate + Expected rate of inflation.
C. Nominal interest rate $=$ Real interest rate + Market risk premium .

  \item Which role is a central bank least likely to assume?
A. Lender of last resort.
B. Sole supervisor of banks.
C. Supplier of the currency.

  \item Central banks would typically be most concerned with costs of:
A. low levels of inflation that are anticipated.
B. moderate levels of inflation that are anticipated.
C. moderate levels of inflation that are not anticipated.

  \item Monetary policy is least likely to include:
A. setting an inflation rate target.
B. changing an official interest rate.
C. enacting a transfer payment program.

  \item Which is the most accurate statement regarding central banks and monetary policy?

\end{enumerate}

A. Central bank activities are typically intended to maintain price stability.

B. Monetary policies work through the economy via four independent channels. C. Commercial and interbank interest rates move inversely to official interest rates.

\begin{enumerate}
  \setcounter{enumi}{12}
  \item When a central bank announces a decrease in its official policy rate, the desired impact is an increase in:
A. investment.
B. interbank borrowing rates.
C. the national currency's value in exchange for other currencies.

  \item Which action is a central bank least likely to take if it wants to encourage businesses and households to borrow for investment and consumption purposes?
A. Sell long-dated government securities.
B. Purchase long-dated government treasuries.
C. Purchase mortgage bonds or other securities.

  \item A central bank's repeated open market purchases of government bonds:
A. decreases the money supply.
B. is prohibited in most countries.
C. is consistent with an expansionary monetary policy.

  \item A central bank that decides the desired levels of interest rates and inflation and the horizon over which the inflation objective is to be achieved is most accurately described as being:

\end{enumerate}

A. target independent and operationally independent.

B. target independent but not operationally independent.

C. operationally independent but not target independent.

\begin{enumerate}
  \setcounter{enumi}{16}
  \item A country that maintains a target exchange rate is most likely to have which outcome when its inflation rate rises above the level of the inflation rate in the target country?
\end{enumerate}

A. An increase in short-term interest rates.

B. An increase in the domestic money supply.

C. An increase in its foreign currency reserves.

\begin{enumerate}
  \setcounter{enumi}{17}
  \item In theory, setting the policy rate equal to the neutral interest rate should promote:
A. stable inflation.
B. balanced budgets.
C. greater employment.

  \item A prolonged period of an official interest rate very close to zero without an in- crease in economic growth most likely suggests:

\end{enumerate}

A. quantitative easing must be limited to be successful.

B. there may be limits to the effectiveness of monetary policy.

C. targeting reserve levels is more important than targeting interest rates.

\begin{enumerate}
  \setcounter{enumi}{19}
  \item Raising the reserve requirement is most likely an example of which type of monetary policy?
\end{enumerate}

A. Neutral.

B. Expansionary.

C. Contractionary.

\begin{enumerate}
  \setcounter{enumi}{20}
  \item Which of the following is a limitation on the ability of central banks to stimulate growth in periods of deflation?
\end{enumerate}

A. Ricardian equivalence.

B. The interaction of monetary and fiscal policy.

C. The fact that interest rates cannot fall significantly below zero.

\begin{enumerate}
  \setcounter{enumi}{21}
  \item The least likely limitation to the effectiveness of monetary policy is that central banks cannot:
\end{enumerate}

A. accurately determine the neutral rate of interest.

B. regulate the willingness of financial institutions to lend.

C. control amounts that economic agents deposit into banks.

\begin{enumerate}
  \setcounter{enumi}{22}
  \item Quantitative easing, the purchase of government or private securities by the central banks from individuals and/or institutions, is an example of which monetary policy stance?
A. Neutral.
B. Expansionary.
C. Contractionary.

  \item The least likely goal of a government's fiscal policy is to:
A. redistribute income and wealth.
B. influence aggregate national output.
C. ensure the stability of the purchasing power of its currency.

  \item Which of the following best represents a contractionary fiscal policy?

\end{enumerate}

A. Public spending on a high-speed railway.

B. A temporary suspension of payroll taxes.

C. A freeze in discretionary government spending.

\begin{enumerate}
  \setcounter{enumi}{25}
  \item A "pay-as-you-go" rule, which requires that any tax cut or increase in entitlement spending be offset by an increase in other taxes or reduction in other entitlement spending, is an example of which fiscal policy stance?
A. Neutral.
B. Expansionary.
C. Contractionary.

  \item The most likely argument against high national debt levels is that:
A. the debt is owed internally to fellow citizens.
B. they create disincentives for economic activity.
C. they may finance investment in physical and human capital.

  \item Which statement regarding fiscal deficits is most accurate?

\end{enumerate}

A. Higher government spending may lead to higher interest rates and lower private sector investing.

B. Central bank actions that grow the money supply to address deflationary conditions decrease fiscal deficits.

C. According to the Ricardian equivalence, deficits have a multiplicative effect on consumer spending.

\begin{enumerate}
  \setcounter{enumi}{28}
  \item Which of the following is the most likely example of a tool of fiscal policy?
A. Public financing of a power plant.
B. Regulation of the payment system.
C. Central bank's purchase of government bonds.

  \item Which statement regarding fiscal policy is most accurate?
A. To raise business capital spending, personal income taxes should be reduced.
B. Cyclically adjusted budget deficits are appropriate indicators of fiscal policy.
C. An increase in the budget surplus is associated with expansionary fiscal policy.

  \item The least likely explanation for why fiscal policy cannot stabilize aggregate demand completely is that:
A. private sector behavior changes over time.
B. policy changes are implemented very quickly.
C. fiscal policy focuses more on inflation than on unemployment.

  \item Given an independent central bank, monetary policy actions are more likely than fiscal policy actions to be:
A. implementable quickly.
B. effective when a specific group is targeted. C. effective when combating a deflationary economy.

  \item Which policy alternative is most likely to be effective for growing both the public and private sectors?

\end{enumerate}

A. Easy fiscal/easy monetary policy.

B. Easy fiscal/tight monetary policy.

C. Tight fiscal/tight monetary policy.

\section{SOLUTIONS}
\begin{enumerate}
  \item B is correct. There is an inverse relationship between the money multiplier and the reserve requirement. The money multiplier is equal to 1 divided by the reserve requirement.

  \item A is correct. Precautionary money demand is directly related to GDP. Precautionary money balances are held to provide a buffer against unforeseen events that might require money. Precautionary balances tend to rise with the volume and value of transactions in the economy, and therefore rise with GDP.

  \item B is correct. When the interest rate on bonds is $I_{1}$ there is an excess supply of money (equal to $M_{0}-M_{1}>0$ ). Economic agents would seek to buy bonds with their excess money balances, which would force the price of bonds up and the interest rate down to $I_{0}$.

  \item A is correct. According to the theory of money neutrality, an increase in the money supply ultimately leads to an increase in the price level and leaves real variables unaffected in the long run.

  \item B is correct. If money were neutral in the short run, monetary policy would not be effective in influencing the economy.

  \item B is correct. By definition, monetarists believe prices may be controlled by manipulating the money supply.

  \item A is correct. The Fisher effect is based on the idea that the real interest rate is relatively stable. Changes in the nominal interest rate result from changes in expected inflation.

  \item A is correct. The Fisher effect implies that changes in the nominal interest rate reflect changes in expected inflation, which is consistent with Nominal interest rate $=$ Real interest rate + Expected rate of inflation.

  \item B is correct. The supervision of banks is not a role that all central banks assume. When it is a central bank's role, responsibility may be shared with one or more entities.

  \item $\mathrm{C}$ is correct. Low levels of inflation has higher economic costs than moderate levels, all else equal; unanticipated inflation has greater costs than anticipated inflation.

  \item $\mathrm{C}$ is correct. Transfer payment programs represent fiscal, not monetary policy.

  \item A is correct. Central bank activities are typically intended to maintain price stability. Concerning choice B, note that the transmission channels of monetary policy are not independent.

  \item A is correct. Investment is expected to move inversely with the official policy rate.

  \item A is correct. Such action would tend to constrict the money supply and increase interest rates, all else equal.

  \item $\mathrm{C}$ is correct. The purchase of government bonds via open market operations increases banking reserves and the money supply; it is consistent with an expan- sionary monetary policy.

  \item A is correct. The central bank described is target independent because it set its own targets (e.g., the target inflation rate) and operationally independent because it decides how to achieve its targets (e.g., the time horizon).

  \item A is correct. Interest rates are expected to rise to protect the exchange rate target.

  \item A is correct. The neutral rate of interest is that rate of interest that neither stimulates nor slows down the underlying economy. The neutral rate should be consistent with stable long-run inflation.

  \item B is correct. A central bank would decrease an official interest rate to stimulate the economy. The setting in which an official interest rate is lowered to zero (or even slightly below zero) without stimulating economic growth suggests that there are limits to monetary policy.

  \item C is correct. Raising reserve requirements should slow money supply growth.

  \item $\mathrm{C}$ is correct. Deflation poses a challenge to conventional monetary policy because once the central bank has cut nominal interest rates to zero (or slightly less than zero) to stimulate the economy, they cannot cut them further.

  \item A is correct. The inability to determine exactly the neutral rate of interest does not necessarily limit the power of monetary policy.

  \item B is correct. Quantitative easing is an example of an expansionary monetary policy stance. It attempts to spur aggregate demand by drastically increasing the money supply.

  \item C is correct. Ensuring stable purchasing power is a goal of monetary rather than fiscal policy. Fiscal policy involves the use of government spending and tax revenue to affect the overall level of aggregate demand in an economy and hence the level of economic activity.

  \item $\mathrm{C}$ is correct. A freeze in discretionary government spending is an example of a contractionary fiscal policy.

  \item A is correct. A "pay-as-you-go" rule is a neutral policy because any increases in spending or reductions in revenues would be offset. Accordingly, there would be no net impact on the budget deficit/surplus.

  \item B is correct. The belief is that high levels of debt to GDP may lead to higher future tax rates which may lead to disincentives to economic activity.

  \item A is correct. Government borrowing may compete with private sector borrowing for investment purposes.

  \item A is correct. Public financing of a power plant could be described as a fiscal policy tool to stimulate investment.

  \item B is correct. Cyclically adjusted budget deficits are appropriate indicators of fiscal policy. These are defined as the deficit that would exist if the economy was at full employment (or full potential output).

  \item B is correct. Fiscal policy is subject to recognition, action, and impact lags.

  \item A is correct. Monetary actions may face fewer delays to taking action than fiscal policy, especially when the central bank is independent.

  \item A is correct. If both fiscal and monetary policies are "easy," then the joint impact will be highly expansionary, leading to a rise in aggregate demand, low interest rates, and growing private and public sectors.

\end{enumerate}

\section*{LEARNING MODULE }
6

\section{Introduction to Geopolitics}
Lauren Goodwin, CFA, is at New York Life Investments (USA)

\section{LEARNING OUTCOME}
\begin{center}
\begin{tabular}{|c|c|}
\hline
Mastery & The candidate should be able to: \\
\hline
$\square$ & $\begin{array}{l}\text { describe geopolitics from a cooperation versus competition } \\ \text { perspective }\end{array}$ \\
\hline
$\square$ & describe geopolitics and its relationship with globalization \\
\hline
$\square$ & $\begin{array}{l}\text { describe tools of geopolitics and their impact on regions and } \\ \text { economies }\end{array}$ \\
\hline
\includegraphics[max width=\textwidth]{2023_05_04_ccb578c5089fd41615beg-369}
 & describe geopolitical risk and its impact on investments \\
\hline
\end{tabular}
\end{center}

\section{INTRODUCTION}
The international environment is constantly evolving. Such trends as the growth of emerging market economies, globalization, and the rise of populism impact the range of opportunities and threats that companies, industries, nations, and regional groups face. They can be impacted by changes in regulation, disrupted trade flows, or even conflict.

Geopolitics is the study of how geography affects politics and international relations. Within the field of geopolitics, analysts study actors-the individuals, organizations, companies, and national governments that carry out political, economic, and financial activities-and how they interact with one another. These relations matter for investments because they contribute to important drivers of investment performance, including economic growth, business performance, market volatility, and transaction costs.

Geopolitical risk is the risk associated with tensions or actions between actors that affect the normal and peaceful course of international relations. Geopolitical risk tends to rise when the geographic and political factors underpinning country relations shift. A shift could arise from a change in policy, a natural disaster, a terrorist act, a theft, or war.

Investors study geopolitical risk because it has a tangible impact on investment outcomes. On a macroeconomic level, these risks impact capital markets conditions, including economic growth, interest rates, and market volatility. Changes in capital markets conditions can, in turn, have an important influence on asset allocation decisions, including an investor's choice of geographic exposure. On a portfolio level, geopolitical risk can influence the appropriateness of an investment security or strategy for an investor's goals, risk tolerance, and time horizon. A higher likelihood of geopolitical risk can raise or lower an asset class's expected return or impact a sector or company's operating environment, affecting its attractiveness for an investment strategy.

There is no shortage of ways in which geopolitical risk can impact a portfolio, so identifying, assessing, and tracking geopolitical risk can be difficult and time consuming. It is thus important for investors to map those potential risks to tangible investment outcomes. In this reading, we will build a framework by which investors can measure, assess, track, and react to geopolitical risk, with a goal of improving investment outcomes.

\section{NATIONAL GOVERNMENTS AND POLITICAL COOPERATION}
 describe geopolitics from a cooperation versus competition perspectiveRelationships within and among countries can be complex. To begin, there are many actors influencing international relations, political developments, and economic affairs. In the introduction, we defined actors as the individuals, organizations, companies, and national governments that carry out political, economic, and financial activities. This definition can be split into two types of actors relevant for geopolitical risk: state actors and non-state actors. State actors are typically national governments, political organizations, or country leaders that exert authority over a country's national security and resources. The South African President, Sultan of Brunei, Malaysia's Parliament, and the British Prime Minister are all examples of state actors. Non-state actors are those that participate in global political, economic, or financial affairs but do not directly control national security or country resources. Examples of non-state actors are non-governmental organizations (NGOs), multinational companies, charities, and even influential individuals, such as business leaders or cultural icons.

These actors are influenced not only by their relationship with one another but also by factors impacting their other allies and adversaries. For example, if Country A has a cooperative relationship with Country B but Country B attacks Country C, a close ally of Country A, then the relationship between Countries A and B may become strained or even broken. The opposite is also true: A country's cooperation across many political, economic, or financial channels may increase trust over time. As a result, the international system is comprised of multi-faceted webs of affairs that can shape events and decisions as well as economic and market outcomes. While there is no "one size fits all" model for geopolitical actors, understanding and categorizing the threats and opportunities that a country faces may help us to gauge the likelihood that geopolitical risk will arise.

Countries and their governments are heavily influenced by economic, financial, and national security considerations as well as by social and cultural factors and non-state actors. In fact, economic and financial considerations are inextricably linked with a country's national interest and political dynamic. That said, it is useful to isolate one of these areas-national actors and political cooperation-as a starting point in understanding actors' interwoven goals and the way they may impact geopolitical risk. In the next section, we will develop a sense of state actors and their motivations for cooperation (versus non-cooperation) before adding additional layers of geopolitical risk analysis.

\section{Features of Political Cooperation}
At the highest level, relations between countries or national governments (state actors) can be cooperative or competitive in nature. Cooperation is the process by which countries work together toward some shared goal or purpose. These goals may, and often do, vary widely-from strategic or military concerns to economic influence or cultural preferences. Given the expansive nature of country goals and interests, their interactions with one another may also be complex, generating the potential for geopolitical risk to arise.

We begin that framework by exploring one specific type of cooperation-political cooperation-which is the degree to which countries work toward agreements on rules and standardization for the activities and interactions between them. Within this context, a cooperative country is one who engages and reciprocates in rules standardization; harmonization of tariffs; international agreements on trade, immigration, or regulation; and allows for the free flow of information, including technology transfer. By contrast, a non-cooperative country is one with inconsistent and even arbitrary rules; restricted movement of goods, services, people, and capital across borders; retaliation; and limited technology exchange. We show the degree of political cooperation in Exhibit 1.

\section{Exhibit 1: Political Cooperation and Non-cooperation}
\begin{center}
\begin{tabular}{|c|c|}
\hline
Inconsistent Rules & Rules Standardization \\
\hline
Arbitrary Rule & Harmonization of Tariffs \\
\hline
Restricted Movement Across Borders & Free Movement Across Borders \\
\hline
Restricted Trade; Capital Controls & Permitted Movement of Goods, Services, and Capital \\
\hline
Retaliation & Reciprocation \\
\hline
Lack of Technology Exchange & Technology Exchange \\
\hline
Non-Cooperation & Cooperation \\
\hline
\end{tabular}
\end{center}

\section{Motivations for Cooperation}
There are many reasons why a country may want to cooperate with its neighbors or with other state actors. These reasons are typically defined by a country's national interest-its goals and ambitions-whether they be military, economic, or cultural.

\section{National Security or Military Interest}
National security or national defense involves protecting a country, including its citizens, economy, and institutions, from external threat. These threats may be broad in nature-from military attacks, to terrorism, crime, cyber-security, and even natural disasters. Geographic factors play an important role in shaping a country's approach to national security and the extent to which it will choose a cooperative approach. Landlocked countries, such as Switzerland, rely extensively on their neighbors for access to vital resources. This reliance may make cooperation more important for sustaining international access and growth, or even for survival. By the same token, countries highly connected to trade routes, such as Singapore, or countries acting as a conduit for trade, such as Panama, may use their geographic location as a lever of power in broader international dynamics.

\section{Economic Interest}
Over time, the concept of national security has expanded to include economic factors, including access to such resources as energy, food, or water. On a domestic level, growing national wealth and limiting income inequality can contribute to social stability, another important component of national security. On the international level, the ability of national firms to operate on a global scale is increasingly important as well. In this context, countries that choose to cooperate in support of their economic interest are likely focused on one of two factors: Either they would like to secure essential resources through trade, or they would like to level the global playing field for their companies or industries through standardization.

\section{KNOWLEDGE CHECK}
\begin{enumerate}
  \item True or false: Geographic factors do not influence the extent to which a country will choose a cooperative approach.
\end{enumerate}

Explain your reasoning.

A. True

B. False

\section{Solution}
B is correct. Geographic factors play an important role in shaping a country's approach to national security and the extent to which it will choose a cooperative approach. A country with limited access to resources because of its geographical characteristics might be more inclined to cooperate as it needs access to resources to thrive, whereas a country rich in internal resources might use its resources as leverage to influence international dynamics. Countries with direct access to the sea have an advantage related to trade and transport as well as the natural resources provided by the ocean.

\begin{enumerate}
  \setcounter{enumi}{1}
  \item Fill in the blank: Two factors on which a cooperative country is likely focused include and
\end{enumerate}

\section{Solution}
Two factors on which a cooperative country is likely focused include trade and standardization.

\section{Resource Endowment, Standardization, and Soft Power}
\section{Geophysical Resource Endowment}
At a basic level, geophysical resource endowment includes such factors as livable geography and climate as well as access to food and water, which are necessary for sustainable growth. Geophysical resource endowment is highly unequal among countries. Some countries, such as the United States, Russia, Australia, and China, are relatively self-sufficient in their resource use. Others, such as Western Europe, Japan, and Turkey, are highly reliant on others for key factors of production, such as fossil fuels. Still others, such as Saudi Arabia, have a plentiful endowment of fossil fuels but rely on others for many basic needs.

These different starting points create power dynamics that can impact the terms of engagement between states. A country heavily endowed with a resource may find itself with more political leverage when dealing with another country in desperate need of that resource. At the same time, a resource-rich country may become vulnerable if the use or sale of the resource benefits certain groups more than others, therefore contributing to internal political instability.

\section{Standardization}
Economic and financial activities may cross borders with or without explicit government support. As they do, governments have more incentive to cooperate with others in standardizing the rules of engagement. Standardization is the process of creating protocols for the production, sale, transport, or use of a product or service. Standardization occurs when relevant parties agree to follow these protocols together. It helps support expanded economic and financial activities across borders, such as trade and capital flows, which support higher economic growth and standards of living. Rules standardization can take many forms-from regulatory cooperation, to process standardization, to operational synchronization. Rules standardization may also be driven by non-state entities, such as industry groups or organizations. Exhibit 2 provides examples of standardization.

Exhibit 2: Types of Rules Standardization

\begin{center}
\begin{tabular}{|c|c|c|c|}
\hline
 & $\begin{array}{c}\text { Regulatory } \\ \text { Cooperation }\end{array}$ & $\begin{array}{c}\text { Process } \\ \text { Standardization }\end{array}$ & $\begin{array}{c}\text { Operational } \\ \text { Synchronization }\end{array}$ \\
\hline
Challenge & $\begin{array}{l}\text { As financial cooper- } \\ \text { ation expands, coun- } \\ \text { tries need a compre- } \\ \text { hensive standard for } \\ \text { governance and risk } \\ \text { management of the } \\ \text { banking sector. }\end{array}$ & $\begin{array}{l}\text { Financial transactions } \\ \text { across borders faced } \\ \text { higher costs and longer } \\ \text { wait times, increas- } \\ \text { ing the burden for } \\ \text { cross-border activity. }\end{array}$ & $\begin{array}{l}\text { Increasing internationa } \\ \text { trade created supply } \\ \text { chain bottlenecks as } \\ \text { containers of differ- } \\ \text { ent sizes and shapes } \\ \text { were sent to ports } \\ \text { worldwide. }\end{array}$ \\
\hline
Solution & $\begin{array}{l}\text { Basel Committee on } \\ \text { Banking Supervision } \\ \text { (BCBS) }\end{array}$ & $\begin{array}{l}\text { Society for Worldwide } \\ \text { Interbank Financial } \\ \text { Telecommunication } \\ \text { (SWIFT) }\end{array}$ & Containerization \\
\hline
Process & $\begin{array}{l}\text { The BCBS was estab- } \\ \text { lished in } 1974 \text { by the } \\ \text { G10 banking author- } \\ \text { ities. Membership } \\ \text { has since expanded } \\ \text { to the G20. }\end{array}$ & $\begin{array}{l}\text { SWIFT was established } \\ \text { in } 1973 \text { to provide } \\ \text { a global financial } \\ \text { infrastructure. }\end{array}$ & $\begin{array}{l}\text { Standards set for } \\ \text { containers of uniform } \\ \text { size and shape using } \\ \text { multi-modal forms of } \\ \text { transport (land, sea, air, } \\ \text { rail) and port cranes. }\end{array}$ \\
\hline
\end{tabular}
\end{center}

\section{Introduction to Geopolitics}
\begin{center}
\begin{tabular}{llll}
\hline
 & \multicolumn{1}{c}{$\begin{array}{c}\text { Regulatory } \\
\text { Cooperation }\end{array}$} & \multicolumn{1}{c}{$\begin{array}{c}\text { Process } \\
\text { Standardization }\end{array}$} & \multicolumn{1}{c}{$\begin{array}{c}\text { Operational } \\
\text { Synchronization }\end{array}$} \\
\hline
Benefit & $\begin{array}{l}\text { Allows for more } \\ \text { effective supervision } \\ \text { of the global banking } \\ \text { sector and interna- } \\ \text { tional capital flows. }\end{array}$ & $\begin{array}{l}\text { Facilitates global pay- } \\ \text { ments in more than 200 } \\ \text { countries and territo- } \\ \text { ries, servicing more } \\ \text { than 11,000 institutions }\end{array}$ & $\begin{array}{l}\text { Dramatically reduces } \\ \text { the time and cost of } \\ \text { shipping goods. }\end{array}$ \\
 &  & worldwide. &  \\
\hline
\end{tabular}
\end{center}

\section{Cultural Considerations and "Soft Power"}
Finally, countries may have cultural reasons for cooperating with others. These could be historical in nature, such as long-standing political ties, immigration patterns, shared experiences, or cultural similarities. In other cases, countries may engage in soft power, a means of influencing another country's decisions without force or coercion. Soft power can be built over time through such actions as cultural programs, advertisement, travel grants, and university exchange. For example, the European Union (EU) Erasmus+ program provides funding for exchange programs in education and sport in order to drive social inclusion and participation between participants from different countries and also to promote such priority policies as the EU's green and digital transitions. In another example, countries like South Korea advertise visiting Seoul, the country's capital city, in subway systems globally. These advertisements use popular Korean-made products, musical acts, and actors to encourage interaction with Korean culture and business.

\section{The Role of Institutions}
An institution is an established organization or practice in a society or culture. An institution can be a formal structure, such as a university, organization, or process backed by law; or, it can be informal, such as customs or behavioral patterns important to a society. Institutions can, but need not be, formed by national governments. Examples of institutions include non-governmental organizations, charities, religious customs, family units, the media, political parties, and educational practice.

Generally, strong institutions contribute to more stable internal and external political forces. That consistency, in turn, gives a country more opportunity to develop cooperative relationships. Countries with strong institutions-including organizations and structures promoting government accountability, rule of law, and property rightsallow them to act with more authority and independence in the international space. Stronger institutions also make cooperative relationships more durable. By integrating a cooperative relationship in multiple layers of society, strong institutions can reduce the likelihood that a country defects from its cooperative roles.

\section{Hierarchy of Interests and Costs of Cooperation}
The national interest of a country is its set of goals and ambitions. For some countries, national interest is viewed primarily in geospatial terms-the need for self-determination, survival as a nation state, the need for clear national borders, or expansion of the nation state itself. For others, national interest incorporates a wide array of interrelated factors, including the economic and social considerations discussed earlier. This broad approach can create conflicts among a country's many important needs, which complicates the assessment of geopolitical actors and their motivations. We can think of a country's national interest as a hierarchy of factors, with those essential for survival at the top of the hierarchy and nice-but-not-essential elements lower in the hierarchy. Governments use the hierarchy of interests to guide their behavior. They will choose to cooperate where it benefits the nation-state, but when two needs result in conflicting cooperation tactics, those higher on the hierarchy are prioritized. For example, cooperation in the form of tariff harmonization may benefit the country on a stand-alone basis; however, if those countries are in a military conflict, then there is a higher cost to cooperation. In this case, if military determination is higher on the countries' hierarchies, then cooperation may not be in their national interest despite potential benefits.

\section{EXAMPLE 1}
\section{Hierarchy of Interests}
Every country has different resources, goals, and leadership. This means that they will have different priorities, too. Some may prioritize access to food and water, like Country A in the following example. Others might prioritize independence from foreign influence, like Country B. What's more, these priorities may shift as political leadership turns over or as global events change. For investors, it's important to understand how these resources, goals, and leadership styles may interact or even conflict with one another over time. When countries' goals misalign or change, it may give rise to geopolitical risk.

\begin{center}
\includegraphics[max width=\textwidth]{2023_05_04_ccb578c5089fd41615beg-375}
\end{center}

\section{Power of the Decision Maker}
Some elements on the hierarchy of national interests may appear clear-cut; securing access to food and water may take precedence over funding a cultural program, for example. However, as basic societal needs are met, the hierarchy of national interests can become more subjective. One government may treat the prioritization of some interests-such as military buildup or providing health care-very differently from its predecessor. How governments weigh those issues will determine the depth and nature of political cooperation.

The length of a country's political cycle has an important impact on priority designation. Many countries have political cycles of just a few years, which means that long-term risks like climate change or addressing income inequality can be difficult to prioritize against projects or goals that can be achieved in a short-term horizon. Intrinsic in this reality is that governmental decision makers, whether political parties or individuals, have their own set of influences and needs. While we will not explore this idea in depth, it is important to acknowledge that, for the purpose of geopolitical risk analysis, decision makers' motivations can impact a country's cooperative and non-cooperative choices. This introduces a factor of psychology and non-predictability into choices along the hierarchy of a nation's needs that can shape geopolitical relationships.

\section{KNOWLEDGE CHECK}
\begin{enumerate}
  \item True or false: The length of a country's political cycle does not impact its priority designation.
\end{enumerate}

Explain your reasoning.

A. True

B. False

\section{Solution}
$B$ is correct. The length of a country's political cycle has an important impact on priority designation as governments with shorter cycles have little incentive to focus on longer-term priorities, even if those priorities would be beneficial for society. Australia, for example, has shorter than average parliamentary terms, which could potentially impact governmental priorities.

\section{Political Non-Cooperation}
We consider political cooperation and non-cooperation as existing along a spectrum. While it is in some countries' interest to be highly politically cooperative, for others it is less essential. Over time, most countries are cooperative on some rules standardization on the international scale. However, there are some instances of extreme non-cooperation-countries whose political self-determination is more important than the benefits of any cooperative actions. These extreme cases are rare, not least because the importance of cooperation for other state actors may result in attempts to coerce non-cooperative state actors into participation.

\section{EXAMPLE 2}
\section{International Sanctions against Venezuela (2015-)}
United States has been concerned about Venezuelan narcotics trafficking since 2005 and Venezuela's lack of cooperation in combatting terrorism since 2006. With support from Congress and in response to increasing political repression in Venezuela, US President Obama levied additional sanctions for human rights abuses, corruption, and antidemocratic actions. The European Union urged Venezuelan officials to work towards political reconciliation but ultimately joined the United States in targeted sanctions to encourage a credible and meaningful process towards re-starting cooperation.

Throughout this time, sanctions have included targeted restrictions on Venezuelan officials, blocking financial transactions, and an embargo on the oil sector, which is highly important to the country's economy. Some politicians continue to support economic sanctions as a means to pressure the Venezuelan government to meet international standards on human rights and political cooperation. Others are concerned about the increasing humanitarian cost of those sanctions.

Venezuela's non-cooperative stance in the international arena has resulted in it being subject to substantial international sanctions. Imposing sanctions is in itself a non-cooperative approach by the United Sates and the European Union meant to influence a country's or its political leadership's behavior. Venezuela's non-cooperative stance indicates that its political self-determination is a priority above that of the humanitarian cost being inflicted on its citizens. Political cooperation versus non-cooperation is only one lens through which geopolitical actors engage with the world, but it is an important one for understanding countries' priorities. Assessing a state actor's hierarchy of needs and how it may change may help us to understand its motivations and priorities. These factors impact not only countries' political and military actions but also their willingness to support economic and financial cooperation, which we will explore next.

\section{NON-STATE ACTORS AND THE FORCES OF GLOBALIZATION}
describe geopolitics and its relationship with globalization

Globalization is the process of interaction and integration among people, companies, and governments worldwide. It is marked by the spread of products, information, jobs, and culture across borders. Indeed, the spread of goods and services across borders has been increasing for decades. The World Bank Openness Index, a key measure of globalization, has risen from $27 \%$ in 1970 to more than $60 \%$ in 2019 , as shown in Exhibit 3. Since 2008 globalization has experienced headwinds; these include the impact of the Global Financial Crisis, which increased scrutiny of cross-border activity, and the rise in nationalism, which decreased some countries' appetite for using imported products or services. Capacity constraints may also create an important headwind. Global trade cannot make up $100 \%$ of global economic activity, meaning its expansion may slow even without specific disruptions.

\section{Exhibit 3: Globalization Has Been Expanding for Decades}
\begin{center}
\includegraphics[max width=\textwidth]{2023_05_04_ccb578c5089fd41615beg-377}
\end{center}

Source: World Bank DataBank (\href{https://data.worldbank.org}{https://data.worldbank.org}). Trade is defined as imports plus exports in the global economy.

In addition to its macroeconomic impact, globalization is also visible at microeconomic level. Take, for example, the production of an automobile. A car may be designed in Japan, with electrical parts made in Germany, steering systems designed in the United States, seatbelts manufactured in Sweden, climate system made in Belgium, and assembled in Mexico. In fact, automobiles are often assembled in one country with parts from all over the world, finished in a second country, and sold in a third

\section{Introduction to Geopolitics}
(see Exhibit 4). This extensive process provides opportunities for companies to find the best inputs for their product, whether in terms of quality or cost-effectiveness. The process of globalization also opens opportunities for investors worldwide, who may invest in aspects of engineering, production, or even the process of supply chain management and logistics.

\section{Exhibit 4: Who Makes the Automobile?}
\begin{center}
\includegraphics[max width=\textwidth]{2023_05_04_ccb578c5089fd41615beg-378}
\end{center}

Notes: Reprinted by permission. Copyright @ 2021 Bank of America Corporation ("BAC"). The use of the above in no way implies that BAC or any of its affiliates endorses the views or interpretation or the use of such information or acts as any endorsement of the use of such information. The information is provided "as is" and none of BAC or any of its affiliates warrants the accuracy or completeness of the information.

Source: Bank of America Merrill Lynch Global Research.

Globalization also has cultural and communicative features. Although it can be difficult to measure these features, such as the spread of information or culture, it is not difficult to see them in our daily lives. A grocery store in Warsaw may hold Italian cheese, Moroccan spices, Colombian coffee, and Indian sauces. Social media allows users in South Africa to collaborate on dances with a Japanese music group. Faster and more affordable travel has increased interactions between citizens of countries all over the world. Internet usage allows for the near-instantaneous spread of cultural information and context.

\section{Features of Globalization}
In Section 2, we considered political cooperation and non-cooperation as one lens by which to consider geopolitical actors-primarily national or state actors. Globalization, by contrast, is the result of economic and financial cooperation. It is carried out mostly by non-state actors, such as corporations, individuals, or organizations.

Globalization is marked by economic and financial cooperation, including the active trade of goods and services, capital flows, currency exchange, and cultural and information exchange. Actors participating in globalization are likely to reach beyond their national borders for access to new markets, talent, or learning. By contrast, anti-globalization or nationalism is the promotion of a country's own economic interests to the exclusion or detriment of the interests of other nations. Nationalism is marked by limited economic and financial cooperation. These actors may focus on national production and sales, limited cross-border investment and capital flows, and restricted currency exchange. Exhibit 5 shows a continuum from nationalism to globalization.

\section{Exhibit 5: Globalization: Economic and Financial Cooperation}
\begin{center}
\includegraphics[max width=\textwidth]{2023_05_04_ccb578c5089fd41615beg-379}
\end{center}

Globalization and cooperation tend to be correlated, meaning that globalization can be facilitated or accelerated by political cooperation, but globalization is also an independent process. Organic private sector forces can drive the exchange of products or ideas even without government support or harmonized rules.

\section{KNOWLEDGE CHECK}
\begin{enumerate}
  \item Fill in the blanks using the following words: government; private sector; national; sub-national.
\end{enumerate}

While political cooperation and non-cooperation can be driven by actors, globalization as the result of economic and financial cooperation is carried out mostly by actors, such as corporations, individuals, or organizations. Organic forces can drive the exchange of products or ideas even without support.

\section{Solution}
While political cooperation and non-cooperation can be driven by national actors, globalization as the result of economic and financial cooperation is carried out mostly by sub-national actors, such as corporations, individuals, or organizations. Organic private sector forces can drive the exchange of products or ideas even without government support.

\begin{enumerate}
  \setcounter{enumi}{1}
  \item True or False: Globalization is primarily carried out by governmental actors.
\end{enumerate}

Explain your reasoning.

A. True B. False

\section{Solution}
B is correct. Globalization is primarily carried out by non-governmental actors, such as corporations, individuals, or organizations, and is the result of economic and financial cooperation. Multi-national corporations, for example, want a competitive advantage leading them to seek new markets, talent, and learning as well as trade, capital, currency, and cultural and information exchange. Corporations' outsourcing of talent is an example of globalization.

\section{Motivations for Globalization}
Non-state actors, such as companies and investors, who choose to participate in globalization consider three potential gains:

\begin{itemize}
  \item Increasing profits

  \item Access to resources and markets

  \item Intrinsic gain

\end{itemize}

\section{Increasing Profits}
Profit can be increased by two means: increasing sales or reducing costs.

\section{Increasing sales}
The opportunity to generate higher profits may motivate companies to globalize. The first way to generate profit is to increase sales. Companies may choose to engage in globalization in order to access new customers for their goods and services. This process may require extensive investment, including acquiring or building property, plant, and equipment in the local market. It may also require hiring workers in new markets and offering training directly benefiting the countries involved.

\section{Reducing costs}
Another way to increase profits is to reduce costs. Globalization allows companies to access lower tax-operating environments, reduce labor costs, or seek other supply chain efficiency gains, all of which are cost-reduction measures. Of course, increasing sales and reducing costs can be closely intertwined. The consolidation of the global automobile industry is one instructive example. For capital-intensive production processes, companies experience economies of scale, reducing average costs, by producing more. By consolidating, automakers can benefit from a global sales opportunity while reducing costs per auto sold.

\section{Access to Resources and Markets}
In Section 2.2, we described the economic interest countries may have to cooperate, including access to resources. The same may be true of non-state actors, such as companies, seeking sustainable access to resources, such as talent or raw materials. If those resources are not readily available or affordable in the home country, then the non-state actor may globalize to improve access.

A non-state actor may also seek market access or investment opportunities abroad. For example, a country experiencing low domestic market returns and increasing wealth may experience a higher propensity to accept cross-border risk-a catalyst for seeking returns abroad. For investment professionals, there are two important types of flows. Portfolio investment flows are short-term investments in foreign assets, such as stocks or bonds. Alternatively, foreign direct investments (FDI) are long-term investments in the productive capacity of a foreign country. These concepts will be covered in more detail elsewhere in the curriculum.

As appetite for cross-border investment has increased, such globalizing actors as companies and organizations (i.e., industry groups, non-governmental or charitable organizations), have established processes for facilitating those needs, including foreign exchange, accounting services, and global investment management services. This globalizing force, while an independent one, has been facilitated and strengthened over time by the harmonization of foreign exchange markets and investment practices.

\section{Intrinsic Gain}
Intrinsic gain is a side effect or consequence of an activity that generates a benefit beyond profit itself. It is difficult to measure but contributes to globalization's momentum. It can also be a stabilizing force, increasing empathy between actors and reducing the likelihood that a geopolitical threat is levied. One example of intrinsic gain is the personal growth or education that individuals may receive from expanding their horizons, experiencing new places, or learning new ideas. Another example is accelerated productivity from learning new methods

Regardless of the motivation for globalization-profit driven or intrinsically driven-these processes and investments can have multiplicative effects. As globalization deepens, companies develop standards and processes to incorporate multiple cultures into their overall corporate culture. They must cooperate with multiple sets of rules, and they may establish standard procedures based on the groups of companies they operate in.

While these factors provide important motivating forces for globalization, they are not the only benefits of globalization. The process of reducing barriers between global businesses and organization can also provide aggregate economic benefits, such as increased choice, higher quality goods, increased competition among firms, higher efficiency, and increased labor mobility.

\section{Costs of Globalization and Threats of Rollback}
While globalization provides many benefits, there may also be costs associated with it for individual economies or sectors, depending on how it is implemented. Some of the potential disadvantages of globalization include the following:

\section{Unequal Accrual of Economic and Financial Gains}
Economic theory tells us that aggregate economic activity is improved when all actors seek profit maximization and efficiency. However, improvement on the aggregate does not mean improvement for everyone. If a company moves a factory to another country, it creates jobs in the new country but reduces them at home, while firms in the new country may have to compete with the foreign firm for labor. Some actors will benefit from this exchange, but others may suffer.

\section{Lower Environmental, Social, and Governance Standards}
Companies operating in lower-cost countries often operate in the local standards of those countries. If standards on environmental protection, social benefits, or corporate governance are lower in one country compared to another and companies ultimately reduce their standards of production in that context, then globalization can create a drain on human, administrative, and environmental resources. While the more measurable corporate profit factor shows a gain, the overall effect of that activity may be negative. For example, many European countries have stricter standards on carbon emissions than those elsewhere in the world. Imagine that a Europe-based company decides to produce in a different country with lighter environmental regulations and cheaper labor costs. If the company decides to act according to local standards rather than its home country standards, it may make more profit but reduce environmental quality.

\section{Political Consequences}
The two costs listed above can create a third cost of globalization: the political consequences of global expansion. While some individuals may enjoy global exposure, others may fear it. While some countries may benefit from improved labor force utilization, others may lose jobs as a company moves abroad. Therefore, globalization can contribute to income and wealth inequality, as well as differences in opportunity, within and between countries. These dynamics can manifest in countries' local politics, resulting in a force not only for reduced political and economic cooperation but also for a rollback in political cooperation.

\section{Interdependence}
Through the process of greater economic and financial cooperation, companies may become dependent on other countries' resources for their supply chains. On aggregate, this can result in the nation itself becoming dependent on other nations for certain resources. If there is a disruption to the supply chain, including via a moment of political non-cooperation, then firms may not be able to produce the good themselves. Commodities production illustrates this potential risk. Rare earth metals, used for light-emitting diode (LED) lights and most electronic displays, are largely produced in China. Copper, essential for renewable energy construction, is largely produced in Chile. Disruptions to production in these countries, including mining accidents or floods, can disrupt entire industries relying on those resources.

\section{EXAMPLE 3}
\section{The COVID-19 Pandemic and Supply Chain Shortages}
The COVID-19 pandemic, which began in late 2019 and reached global impact in 2020, provides an interesting example of how geopolitical disruptions can impact companies' supply chains and countries' access to important resources. In the early months of 2020, many countries enforced "stay at home" orders in hopes of protecting their citizens against the pandemic's spread. As a result, demand for many goods and services collapsed and manufacturing activity was restricted, resulting in a significant slowdown in production of many goods and services. Even as production resumed, the impact of production slowdown had ripple effects across industries.

Semiconductors were one such product. Semiconductor production is highly concentrated in China and is very important to the automobile industry. The chips are used for fuel-pressure sensors, navigation displays, and speedometers, among other automotive devices. Gradually, as the pandemic wore on, mobility began to improve but supply remained constrained. The result was a severe shortage of semiconductor supply, which contributed to high and rising prices for new, used, and rental automobiles in many countries across the world. This dependence of a global industry on one country's production illustrates how interconnected supply chains can create important financial risks. What's more, the effects of the semiconductor shortage soon moved beyond the automobile industry, as other industrial players struggled to secure chips for their devices.

\section{EXAMPLE 4}
\section{"America First" Policies: Is Globalization Reversing?}
The threat of deglobalization has grabbed headlines since 2018 when the Trump administration began a series of "America first" policies. Rooted in nationalism, isolationism, and concerns for national and economic security, trade wars escalated in fits and starts for several years. Targets evolved to include not only countries traditionally seen as US competitors but also long-standing allies, such as the European Union, via tariffs on imported products, and Canada and Mexico, through a renegotiation of the North America Free Trade Agreement (NAFTA). These developments have led investors and policymakers to consider whether globalization is reversing and what it means for the path of geopolitical risk.

Indeed, the impact of political non-cooperation on multinational companies is significant. Global design, manufacturing, and distribution systems are complex, making a change in the rules-such as a restriction on the free movement of goods and services-burdensome, which raises questions about profitability and efficiency. Multinational corporations are accustomed, however, to the political and operational risks of international production. This makes management rightfully slow to change carefully engineered processes over political disputes that may eventually be cleared up.

The experience of the COVID-19 pandemic built upon pre-existing challenges to global supply chain management. Supply disruptions revealed the pitfalls of over-relying on any single production location. Corporates found their supply chains were fragile as inputs sourced from foreign countries became unavailable or manufacturing moved offshore suddenly fell victim to lockdown orders.

That said, completely reversing globalization seems unlikely. Instead, companies are likely to use a combination of tactics to fortify their supply chains.

Reshoring the essentials: Shortages of prescription medication, personal protective equipment, and other essential items during the pandemic highlighted the need for certain 'essential' supply chains to be rebuilt domestically for emergency situations. Companies seeking to reduce manufacturing and procurement risk may relocate back to their home countries.

Re-globalizing production: The same concerns about production disruptions, rising labor costs, or political risk may instead prompt companies to duplicate or fortify their supply chains.

Doubling down on key markets: China has been the focus of US trade concerns, but the political risk is unlikely to change the need for some globalization. Not all manufacturing would be able to move outside of such key markets as China, Canada, and Mexico, particularly for companies that also seek to sell to those markets. While labor costs in some trading partners have risen over time, so too has the productivity of those workers. Add large market size, physical infrastructure supporting coordination, sophisticated supply chains, and the investment required to rebuild supply chains elsewhere, and some companies may consider doubling down on key markets. Developing production "In country, for the country," in combination with external supply chains, may be required.

\section*{ASSESSING GEOPOLITICAL ACTORS AND RISK }
Using the two axes we have discussed-political cooperation versus non-cooperation and globalization versus nationalism-investment analysts can assess geopolitical actors and the likelihood of threat to investment outcomes. Where countries stand in that balance shapes their standing compared with other geopolitical actors, the tools of geopolitics they can use, and the threats and opportunities they face.

\section{Exhibit 6: Archetypes of Globalization and Cooperation}
\begin{center}
\includegraphics[max width=\textwidth]{2023_05_04_ccb578c5089fd41615beg-384}
\end{center}

The framework shown in Exhibit 6 presents four archetypes of country behavior: autarky, hegemony, multilateralism, and bilateralism. Each archetype has its own costs, benefits, and tradeoffs with respect to geopolitical risk. In general terms, regions, countries, and industries that are more dependent on cross-border goods and capital flows will have growth rates and investment returns that benefit from greater global cooperation. The interdependent nature of their activities may also reduce the likelihood that collaborative countries levy economic, financial, or political attacks on one another. However, that same interdependence often makes cooperative actors more vulnerable to geopolitical risk than those less dependent on cooperation and trade. Diversifying the means of production provides some shield against risk but also provides more touchpoints by which risk can occur. In the remainder of the section, we describe the costs and benefits of each archetype, with some examples of each.

It is important to note that each of the axes explored in this framework-cooperation versus non-cooperation and globalization versus nationalism-represent a spectrum. Rarely do countries represent full extremes of either of these factors. And, as you can see from the examples provided, a country's place within the framework can be a moving target. As such, the examples provided in no way represent a value judgement of a country's goals or approach. For geopolitical risk analysis, it matters not only which quadrant a country falls in today but also its stability within that quadrant. A hegemon who is working to build more political cooperation may be less of a threat to investment results than a multilateral actor trying to break them.

The "America first" example given in Section 3.3 is highly relevant here. Adjusting previously agreed-upon rules may be in a country's national interest, but inconsistent application of those rules presents a risk for companies and economic growth. Investors must be aware of actors' movement within the framework in order to more appropriately assess the likelihood and impact of geopolitical risk.

\section{KNOWLEDGE CHECK}
\begin{enumerate}
  \item True or false: The relative permanence of geographic factors implies a fundamental stability in geopolitical risk analysis of the involved parties.
\end{enumerate}

Explain your selection.
A. True
B. False

\section{Solution}
B is correct. Geopolitical risk analysis is fundamentally dynamic, with analysts needing to be aware of the parties' movement within the relevant framework to most appropriately assess the likelihood and impact of geopolitical risk.

Geopolitical actors face a delicate balance between harnessing the potential intrinsic and profit gains from globalization while managing the many and sometimes unforeseen consequences. Further complicating the picture is the intricate connection between globalization and cooperation. A country's place within the globalization and cooperation framework can be a moving target, and for geopolitical risk analysis, it matters not only which quadrant a company falls in today but also its stability within that quadrant.

\section{Autarky}
Autarky describes countries seeking political self-sufficiency with little or no external trade or finance. State-owned enterprises control strategic domestic industries. The self-sufficiency of autarkic countries allows them to be stronger politically, including the ability to exercise complete control over the supply of technology, goods, and services, as well as media and political messaging. In some cases, periods of autarky can provide a country with swifter economic and political development. For example, for much of the 20th century, China could have been described as autarkic, exercising very little political cooperation or globalization. However, China's autarkic stance resulted in substantial poverty alleviation and an eventual move towards more economic and financial cooperation. That said, an autarkic stance does not come without costs. In other cases, such as North Korea and Venezuela described in Example 2, autarky has resulted in a gradual loss of economic and political development within the country.

\section{Hegemony}
Hegemonic countries tend to be regional or even global leaders, and they use their political or economic influence of others to control resources. State-owned enterprises tend to control key export markets. A hegemonic system can provide valuable benefits both to hegemonic countries themselves and to the international system. For the country itself, economic and political dominance may provide important influence on global affairs. For the global system, countries aligning with the hegemon's rules and standards may enjoy the rewards provided by the leader, including the stabilizing force of monitoring and enforcing the hegemon's standards. That said, there may be costs to hegemonic systems. As hegemons gain or lose influence in the international system, they may become more competitive, increasing the likelihood of geopolitical risk.

\section{EXAMPLE 5}
\section{China and Technology Transfer}
In 2011, China overtook Japan as the world's second largest economy. Having joined the World Trade Organization (WTO) in 2001, China's economy had globalized meaningfully. At the same time, the importance of China's economy gave it important influence on the rules of play in the international stage. One area where China has exerted its political influence is in technology transfer. For many years, China's stance was that in order to integrate into the Chinese production machine, including access to a skilled and plentiful labor supply, some companies would be required to transfer their technology ideas and processes to Chinese partners. For a smaller or less important economy, these rules might be difficult to enforce, but China's size and importance made it possible. These policies may have some globalizing effects. By having Chinese companies adopt more international technological ideas and rules, they become more globalized in their approach. At the same time, they can also create investment risks. If China were to become more restrictive, it could disrupt global supply chains and increase costs for global companies.

\section{EXAMPLE 6}
\section{Russia and Gas Disputes}
Russia's economy, and particularly its oil sector, is well-integrated into global supply chains. However, the country is also largely politically autonomous, which can contribute to geopolitical risk and uncertainty. For example, Russia's control of important natural gas pipelines gives it significant political leverage over other countries, particularly those in Europe that rely on it for fossil fuels. As a result, European countries may be less likely to confront Russia in other areas. Between February and March 2014, Russia annexed Crimea, Ukraine. Other countries condemned the annexation and considered it to be a violation of international law, including such previously politically cooperative agreements as the 1991 Belavezha Accords that established the Commonwealth of Independent States. Still, the annexation continues, in part because many countries are unwilling to disrupt an important economic relationship.

In addition to political uncertainty, the annexation contributed to important economic and market developments impacting investors. In the immediate aftermath of the annexation, the Russian ruble depreciated significantly, contributing to higher market volatility and inflation. Other significant economic impacts followed. For example, countries, including Canada, the United States, and European Union member states, imposed sanctions on Russian officials. Ukraine responded to the annexation by cutting off water to the area, contributing to crop failure.

\section{Multilateralism}
Multilateralism describes countries that participate in mutually beneficial trade relationships and extensive rules harmonization. Private firms are fully integrated into global supply chains with multiple trade partners. Examples of multilateral countries include Germany and Singapore.

\section{EXAMPLE 7}
\section{Singapore as Multilateral Actor}
Singapore is ranked as the most open economy in the world by the World Economic Forum, the third least corrupt by Transparency International, and the second most pro-business by the World Bank Doing Business Report. It has low tax rates and the second highest per capita GDP in terms of purchasing power of its citizens.

\begin{center}
\begin{tabular}{lcccc}
\hline
 & $\begin{array}{c}\text { World } \\ \text { Bank Ease } \\ \text { of Doing } \\ \text { Business } \\ \text { Rank (2019) }\end{array}$ & $\begin{array}{c}\text { Corruption } \\ \text { Perceptions } \\ \text { Index Rank } \\ \mathbf{( 2 0 2 0 )}\end{array}$ & $\begin{array}{c}\text { Corporate } \\ \text { Tax Rate } \\ \mathbf{( 2 0 2 1 )}\end{array}$ & $\begin{array}{c}\text { GDP (per } \\ \text { capita, in } \\ \text { international } \\ \text { dollars, 2021) }\end{array}$ \\
\hline
Singapore & 2 & 3 & $17.00 \%$ & 95,650 \\
United States & 6 & 25 & $21.00 \%$ & 63,594 \\
Germany & 22 & 9 & $30.00 \%$ & 53,024 \\
Australia & 18 & 11 & $30.00 \%$ & 51,102 \\
United Kingdom & 8 & 11 & $19.00 \%$ & 43,839 \\
Japan & 29 & 124 & $30.60 \%$ & 41,507 \\
Mexico & 60 & 78 & $30.00 \%$ & 18,867 \\
China & 31 & 94 & $34.00 \%$ & 17,624 \\
Brazil & 124 & $28.00 \%$ & 14,563 &  \\
South Africa & 82 & $30.00 \%$ & 11,582 &  \\
Philippines & 95 &  & 8,436 &  \\
\hline
\end{tabular}
\end{center}

Sources: World Bank (\href{http://doingbusiness.org}{doingbusiness.org}), Transparency International (\href{http://transparency.org}{transparency.org}), KPMG, International Monetary Fund (\href{http://imf.org}{imf.org}).

What is behind these impressive statistics? For starters, Singapore's factor endowment makes it highly dependent on cooperation and innovation to survive. The country has limited natural resources, including water and arable land, which means the country must rely on agrotechnology parks and reclaimed land for agricultural production as well as trade partners for the inputs to that production.

Geographic factors also contribute to Singapore's economic openness. The country is located at the intersection of many important global trade routes. Its population is highly ethnically and racially diverse, allowing for English as a key language for global business. In addition, its workforce is highly skilled, giving Singapore an important role as a center for trade and innovation throughout Asia.

Politically, Singapore has highly stable institutions. The government is strategically involved in its economic system to allow for consistent prioritization of economic activity and enforcement of business-friendly institutional governance. Singapore's openness has contributed to its attractiveness as a potential partner.

\section{Introduction to Geopolitics}
As a result of its location and factor endowments, Singapore is both capable of and highly dependent on international cooperation for its economic growth. The country generates higher economic growth rates due to greater global cooperation. However, that same cooperation may leave Singapore more vulnerable to geopolitical risk than those countries that are less dependent on cooperation and trade.

\section{KNOWLEDGE CHECK}
\begin{enumerate}
  \item Identify two of the factors accounting for Singapore's reliance on globalization to achieve economic success.
\end{enumerate}

\section{Solution}
First, Singapore's factor endowment of limited natural resources makes it dependent on cooperation and innovation to survive.

Second, Singapore's geographic location at the intersection of multiple important global trade routes helps make it an Asian center for world business. Third, culturally it has a highly ethnically and racially diverse population fluent in the key global business language of English, enhancing its collaborative capabilities.

Finally, it has highly stable political institutions with high governmental priorities on promoting economic activity and enforcing business-friendly institutional governance. These qualities make it an attractive potential business partner and likely sought out for global cooperation.

\section{Bilateralism}
Bilateralism is the conduct of political, economic, financial, or cultural cooperation between two countries. Countries engaging in bilateralism may have relations with many different countries, but they are one-at-a-time agreements without multiple partners. Typically, countries exist on a spectrum between bilateralism and multilateralism. In between the two extremes is regionalism, in which a group of countries cooperate with one another. Both bilateralism and regionalism can be conducted at the exclusion of other groups. For example, regional blocs may agree to provide trade benefits to one another and increase barriers for those outside of that group.

It is noteworthy that relatively few countries perfectly fit the bilateral mold. Moving towards stronger political cooperation tends to lead organically to globalization; it is common for non-state actors to globalize as long as the path is laid for them. What's more, innovations, such as the internet and digital transfer, have made it even easier for firms to globalize.

Bilateralism once nicely described Japan; its government engaged in substantial political cooperation for the sake of building a strong export market, but it did not globalize its imports. That said, as the importance of international capital markets has deepened, Japan has been a pioneer of globalizing its equity and bond markets for international players. Today it would be considered a multilateral player.

\section{THE TOOLS OF GEOPOLITICS}
\begin{center}
\includegraphics[max width=\textwidth]{2023_05_04_ccb578c5089fd41615beg-389}
\end{center}

Now that we understand the characteristics of geopolitical actors, we can examine the tools these actors use to manifest or reinforce their interests with respect to others. The tools an actor uses are ultimately the source of geopolitical risk as it impacts investors-shaping the likelihood of risk as well as the speed and size of impact, as we will explore in Section 6. As a result, understanding the tools themselves may be just as important as understanding the motivations for using them.

The tools of geopolitics may be separated into three types:

\begin{itemize}
  \item national security tools,

  \item economic tools, and

  \item financial tools.

\end{itemize}

Among each of these tools, there are choices that reflect or improve cooperation and those that reflect or escalate conflict (non-cooperation). Tools facilitating cooperation are those that increase flows between countries; this can mean an increased flow of goods, services, capital, or labor through treaties, trade agreements, capital provisions, and approved migration. Tools escalating conflict are those that reduce these flows between countries.

We can use the framework provided in Section 4 to assess the nature of tools used by geopolitical actors and to strengthen our designation of countries within that framework. The extent to which actors use these tools to improve cooperation or escalate conflict influences their position within the framework. As we described in Section 4, it is important for geopolitical risk analysts to track not only which quadrant best describes an actor but also its stability within that quadrant. Using different national security, economic, or financial tools may indicate that an actor is changing in character, which would increase or decrease geopolitical risk. Exhibit 7 shows where the tools of geopolitics fit in the four archetypes of country behavior (presented in Exhibit 6).

\section{Exhibit 7: Tools of Geopolitics}
\begin{center}
\includegraphics[max width=\textwidth]{2023_05_04_ccb578c5089fd41615beg-390}
\end{center}

\section{National Security Tools}
National security tools are those used to influence or coerce a state actor through direct or indirect impact on the country's resources, people, or borders. National security tools may be active, meaning they are being used at the time of analysis, or threatened, meaning they are not currently used but their use is likely enough to warrant concern.

The most extreme example of a national security tool is that of armed conflict. Armed conflict is a direct and active national security tool. It can be internal or external to a country and has two major impacts. The first is the disruption or destruction of physical infrastructure, which can inflict long-term damage on a country's capital stock and ability to rebuild that stock. The second impact is on migration away from areas of armed conflict, which can reshape international flows of goods, services, capital, and labor. It can also impact neighboring countries and states accepting refugees.

\section{EXAMPLE 8}
\section{Syrian Refugee Crisis (since 2011) and Impacts on Germany}
The Syrian refugee crisis began in March 2011 after a violent government crackdown on public demonstrations. Protests turned into an armed rebellion and escalated to civil war. As a result, Syria has experienced extreme deterioration of its physical infrastructure, including buildings transportation, and other structures. As conflict prolonged, more than 6.6 million Syrians have fled the country, with another 6.7 million having been internally displaced. Beyond the tragic human toll of this ongoing conflict, Syria has also experienced negative impact on its current and potential economic growth as a result of the deterioration of its capital and labor stock. According to the International Monetary Fund (IMF), Syria's economy grew by 4.5\%, on average, from 2000-2010. By 2016, the Syrian economy was less than half its size before the war. In investment terms, these developments have contributed to a sustained higher discount rate and lower bond and equity returns. The majority of refugees have found refuge in neighboring countries, such as Turkey, Lebanon, and Jordan. However, many also made their way, often by dangerous routes, to European countries. The refugee crisis attracted significant international attention in 2015 when German Chancellor Angela Merkel announced that the country would admit one million Syrian refugees. In the short-term, the decision was disruptive to domestic politics and international cooperation as not all citizens and neighboring countries were supportive of the approach. The decision also carried economic costs, including constructing housing and establishing resettlement programs. However, in the long-term, Germany appears to be benefiting from the cooperative decision, having attracted a younger and skilled demographic of migrants thus improving the demographic balance of an otherwise aging country. The resulting improvements in Germany's labor and capital stock may improve Germany's potential economic growth rate. In investment terms, a higher potential economic growth rate contributes to higher expected market returns, all else equal, and may contribute to industry developments or innovations.

Of course, not all tools are used in so direct a nature as armed conflict. For example, espionage, or the practice of using spies to obtain political or military information, is a necessarily indirect national security tool. Military alliances are often used to aid in direct conflict and also to deter conflict from arising in the first place.

What's more, not all national security tools are used in a non-cooperative way. State actors can combine forces to reduce the likelihood that national security tools are used. For example, the North Atlantic Treaty Organization (NATO), an alliance between the European Union, United States, United Kingdom, and Canada, is used to discuss and de-escalate potential conflict among members and between members and outside states. Originally constructed to provide collective security against the Soviet Union, NATO now serves as a collective effort to reduce nuclear proliferation and other common national security goals.

\section{KNOWLEDGE CHECK}
\begin{enumerate}
  \item True or false: While geopolitical financial tools can be used for both cooperative and non-cooperative reasons, national security tools are characteristically non-cooperative.
\end{enumerate}

Explain your selection.

A. True

B. False

Solution

$B$ is correct. National security tools are non-cooperative in cases like armed conflict. But collective security agreements, such as NATO, can also be used cooperatively to reduce the possibility of conflict among members and between members and outside states.

Financial tools may be cooperative in reducing geopolitical risk if they encourage cooperation in security, economic, or financial arenas. They may also tend to non-cooperation by creating vulnerabilities in the international system. US dominance is one such example, both promoting financial activity and making other countries vulnerable to US monetary policy changes.

\section{Economic Tools}
Economic tools are the actions used to reinforce cooperative or non-cooperative stances via economic means. Among state actors, economic tools can include multilateral trade agreements, such as the Southern Common Market (MERCOSUR), which facilitates trade among member countries in South America, or the global harmonization of tariff rules, as facilitated by the World Trade Organization (WTO). Highly cooperative economic tools may also include common markets, like the European Union, or a common currency, like the euro.

By contrast, economic tools can also be non-cooperative in nature. Nationalization, the process of transferring an activity or industry from private to state control, is a non-cooperative approach to asserting economic control. Nationalization is most common in sectors perceived as vital to economic security or competitiveness, such as the energy sector. For example, after a period of privatization, Argentina moved in 2012 to re-nationalize YPF, the nation's largest energy firm. In addition to controlling an important geophysical resource, Argentina's government looked to secure the sale of fossil fuels as an important source of foreign exchange, a scarcity at the time. Countries can engage in voluntary export restraints, meaning they refuse to trade as much of their goods and services as would meet demand. Countries can also choose to impose domestic content requirements, asserting that a certain proportion or type of domestic input be included in an exported good.

\section{Financial Tools}
Financial tools are the actions used to reinforce cooperative or non-cooperative stances via financial mechanisms. Examples of cooperative financial tools include the free exchange of currencies across borders and allowing foreign investment. Examples of non-cooperative financial tools include limiting access to local currency markets and restricting foreign investment. Sanctions, described in Section 2.7, provide a useful example of countries using financial tools to influence geopolitical outcomes.

Cooperative financial tools may reduce geopolitical risk if they encourage cooperation in security, economic, or financial arenas. However, the same tools may also create vulnerabilities in the international system. The dominance of the US dollar is one such example. The international interbank market, in which banks borrow and lend to one another, hosts transactions heavily denominated in US dollars. The market itself provides a tool of cooperation, and the free exchange of currency helps facilitate financial activity and cooperation more broadly. That said, the US dollar's importance to exchange also makes other countries vulnerable to changes in US monetary policy. Specifically, tighter US monetary policy can contribute to liquidity shortages in countries that do not or cannot maintain US dollar reserves.

\section{KNOWLEDGE CHECK}
\begin{enumerate}
  \item Describe the conditions under which cooperative financial tools may decrease geopolitical risk as well as create vulnerabilities in the international system.
\end{enumerate}

\section{Solution}
Cooperative financial tools may reduce geopolitical risk if they encourage cooperation in security, economic, and financial arenas. However, if the system becomes too dependent on a particular financial tool or if the tool becomes too dominant, it may introduce vulnerabilities in the international system that can have far-reaching implications.

\section{Multi-Tool Approaches}
Just as geopolitics is multi-faceted and includes many types of actors and features, so too are the tools of geopolitics. Systems of political, economic, and financial cooperation can be, and often are, intertwined. One interesting example is cabotage, or the right to transport passengers or goods within a country by a foreign firm. Many countries, including those with multilateral trade agreements, impose restrictions on cabotage across transportation subsectors-meaning that shippers, airlines, and truck drivers are not allowed to transport goods and services within another country's borders. Allowing cabotage requires coordination on areas like physical security and economic coordination, a highly multilateral process.

International organizations may also make use of multiple tools of geopolitics. For example, the Association of Southeast Asian Nations (ASEAN) is comprised of 10 members states and seeks to facilitate economic, political, security, military, educational, and cultural integration between its members. The European Union, which began as a 6-country economic bloc, has expanded to 27 member states (as of October 2021) and includes substantial financial and national security components. Generally, as actors incorporate more tools of collaboration, they are less likely to initiate conflict or use a non-cooperative tool against associated actors.

\section{EXAMPLE 9}
\section{European Union and Brexit}
One of the most pronounced transitions towards multilateral cooperation and globalization is in the European Union (EU). From 1945 until 2016, European countries exhibited fairly steady progress towards political cooperation-via shared security, free movement of goods and people, and harmonization of rules and regulations-and economic and financial globalization. EU member states offer European citizenship and the right for workers and companies from all EU member states to operate within other EU countries. The strongest manifestation of this multilateralism is exhibited in the creation of the common currency for the euro area, a subset of European countries.

Timeline: The Path of European Cooperation and Globalization

\begin{center}
\includegraphics[max width=\textwidth]{2023_05_04_ccb578c5089fd41615beg-393}
\end{center}

2016 marked an important disruption in the European integration process when the United Kingdom voted to exit the EU. Despite the benefits of cooperation and globalization, the costs became increasingly felt, impacting local politics. The United Kingdom's sudden reversal in some of its multilateral momentum was a surprise to many global actors and investors. This geopolitical risk resulted in near-term market volatility, particularly in financial and currency markets. It has also resulted in the dismantling of political cooperation in recent years; the United Kingdom's multilateral approach is, in some areas, shifting towards a bilateral approach.

\section{Geopolitical Risk and Comparative Advantage}
The geopolitical tools discussed in this section generate important risks and opportunities for investors, but they do not operate in a vacuum. In fact, geopolitical risk and the tools of geopolitics can shape actors' core priorities. Models of international trade will be explored later in the curriculum. (In this context, we reference them briefly to illustrate how geopolitical tools can shape, destroy, or build upon a country's core approach to other actors.

In the classic example of international trade, countries are endowed with certain resources, or factors, and technological capabilities. Not every country will be endowed with the same factors and capabilities and thus may benefit from cooperating via trade. For example, Ghana is rich in resources, such as hydrocarbons, gold, agricultural products, and several industrial metals. However, it relies heavily on other countries for both processing and industrial use of those resources. Each country specializes in areas defined by its resources and capabilities and then exchanges with the other in a way that benefits both parties. Through specialization and exchange, industries within each country experience greater economies of scale. Households and firms will therefore have a greater variety of products to choose from. Competition between firms is higher, and resources are allocated more efficiently.

Geopolitical risk and the tools of geopolitics can tilt comparative advantage in one direction or another. For example, countries or regions with limited geopolitical risk exposure may attract more labor and capital. By contrast, those with higher geopolitical risk exposure may suffer a loss of labor and capital. Similarly, a consistent threat of conflict may drive more regular volatility in asset prices, prompting investors to require higher compensation for risk taken (i.e., an increase in the required rate of return or the discount rate used in valuation).

\section{KNOWLEDGE CHECK}
\begin{enumerate}
  \item True or false: Germany's reaction to the Syrian refugee crisis is an example of comparative advantage stemming from geopolitical risk.
\end{enumerate}

Explain your selection.

A. True

B. False

\section{Solution}
A is correct. Countries with a lower geopolitical risk exposure have the ability to attract resources, such as labor and capital. With its strong economic position in the EU and longstanding stability of political leadership, Germany was able to undertake the resettlement of one million Syrian refugees to it while assuming the short-term relocation costs and disruption of its domestic politics. This improves its long-term demographic balance by adding young and talented migrants, with the resulting increase in labor and capital stock potentially increasing Germany's economic growth rate.

\section{INCORPORATING GEOPOLITICAL RISK INTO THE INVESTMENT PROCESS}
describe geopolitical risk and its impact on investments

There is no shortage of attention to geopolitical risk in financial market analysis. However, the extent to which investors incorporate geopolitical risk into their decision making will vary widely with their investment objectives and risk tolerance. Some investors may be considered takers of geopolitical risk. These investors may only incorporate geopolitical risk into their analysis to the extent that it impacts the long-term attractiveness of asset classes or strategies. For other investors, geopolitical risk may be a central component of their investment process. For these portfolios, monitoring dislocations is an achievable and meaningful driver of alpha creation, where focused geopolitical risk analysis can reduce the impact and severity of adverse events and enhance the potential for upside growth. For example, an investor that anticipates an important political transition may enact portfolio hedges to shield against market volatility or may use the market volatility as a buying opportunity-in either case, improving investment outcomes.

What follows is a discussion of geopolitical risk's impact on the investment environment. We begin with a discussion of types of geopolitical risk, followed by a means of assessing those risks and the ways that they can manifest in a portfolio.

\section{Types of Geopolitical Risk}
There are three basic types of geopolitical risk: event risk, exogenous risk, and thematic risk.

Event risk evolves around set dates, such as elections, new legislation, or other date-driven milestones, such as holidays or political anniversaries, known in advance. Political events often result in changes to investor expectations related to a country's cooperative stance. As a result, risk analysts often use political calendars as a starting place for assessing event risk.

One example of event risk is the United Kingdom's referendum on European Union membership. This was a known event risk; it was planned for 23 June 2016, well in advance. The stakes of the election were well understood, with many years' worth of politically cooperative steps likely to unwind in the event of a "yes" vote. Most investors expected a "no" vote; when the results proved the opposite, investor expectations related to the United Kingdom's cooperative stance were drastically changed.

\section{EXAMPLE 10}
\section{Market Reaction to Event Risk (United Kingdom's EU}
Referendum, 2016)

\begin{center}
\begin{tabular}{lccc}
\hline
 & First Day & 30 Days & 1 Year \\
\hline
FTSE & $-3.1 \%$ & $6.2 \%$ & $17.1 \%$ \\
GBP & $-8.1 \%$ & $-11.9 \%$ & $-14.5 \%$ \\
10-year bond yield (Gilt) & $-21.0 \%$ & $-42.0 \%$ & $-25.0 \%$ \\
\hline
\end{tabular}
\end{center}

\section{Introduction to Geopolitics}
The United Kingdom's vote to end its European Union membership came as a surprise to many investors. Several asset classes were immediately impacted, including equities, the national currency, and government bonds. In the weeks that followed, investors adjusted to the news but became concerned about what a rollback in political cooperation might mean for long-term economic growth. As a result, equity prices recovered, but the British pound continued on a steady decline. Government bond yields declined precipitously in the month following the vote, recovering somewhat as the year progressed.

It is useful to note that the predictability of an event does not necessarily change its likelihood, its speed of impact, or the size of impact on investors; however, it does give investors more time to prepare a response. We will come back to this example and the others in this section when we discuss the assessment of geopolitical threats.

Exogenous risk is a sudden or unanticipated risk that impacts either a country's cooperative stance, the ability of non-state actors to globalize, or both. Examples include sudden uprisings, invasions, or the aftermath of natural disasters.

Exhibit 8: Market Reaction to Exogenous Risk (Japan's Fukushima Nuclear Disaster, 2011)

\begin{center}
\begin{tabular}{lccc}
\hline
 & First Day & $\mathbf{3 0}$ Days & $\mathbf{1}$ Year \\
\hline
Nikkei & $-6.2 \%$ & $-5.2 \%$ & $-3.6 \%$ \\
Yen & $-0.3 \%$ & $3.4 \%$ & $0.5 \%$ \\
Japanese 10-year & $-3.3 \%$ & $5.9 \%$ & $-22.3 \%$ \\
\end{tabular}
\end{center}

bond yield

On 11 March 2011, Japan was struck with an earthquake and subsequent tsunami wave, resulting in significant loss of human life, homes, and productive capital. The natural disaster also caused a significant nuclear accident that resulted in further human, property, and environmental damage and also disrupted supply chains. The initial market response reflected market concern: Equities fell, the currency depreciated, and bond prices rose. In the weeks that followed and as the toll of the environmental disaster became more apparent, Japanese equity markets continued to suffer, declining up to $20.4 \%$ (as of 25 November 2011) from their levels the day of the earthquake.

As the environmental costs of the accident became clear, this event contributed to a shifting stance on political cooperation on environmental issues. Less than three months after the incident, Germany decided to phase out nuclear power entirely by 2022 . Belgium confirmed plans to exit nuclear power by 2025 , and such countries as Italy, Spain, and Switzerland opted not to reintroduce nuclear energy programs.

Finally, thematic risks are known risks that evolve and expand over a period of time. Climate change, pattern migration, the rise of populist forces, and the ongoing threat of terrorism fall into this category.

Cyber threats are another example of thematic risk. Cyber risks include any attempt to expose, alter, disable, destroy, steal, or gain information through unauthorized access to or unauthorized use of computer systems. These threats began with the expansion of internet and computer use and have increased in number and scale. Now, the number of records stolen or impacted by cyberattacks is in the billions per year. While the basic nature of cyberattacks is consistent, the size, scale, and sophistication of attacks have increased over time.

Exhibit 9: Market Reaction to Thematic risk (Equifax Data Breach, 2017)

\begin{center}
\begin{tabular}{lccc}
\hline
 & First Day & 30 Days & 1 Year \\
\hline
Stock price & $-13.7 \%$ & $-22.0 \%$ & $-4.8 \%$ \\
Financial services & $0.9 \%$ & $10.3 \%$ & $18.6 \%$ \\
industry &  &  &  \\
\hline
\end{tabular}
\end{center}

In September 2017, the US consumer credit reporting company Equifax announced a data breach that exposed personal information including names, dates of birth, and personal identification numbers of approximately 147 million people. The initial market reaction was very negative. Equifax's equity price fell by $13.7 \%$ in one day and by $34.9 \%$ over the first week, reaching its low on 15 September 2017. Over time, the impact moderated somewhat, but Equifax still underperformed the financial services industry in the year after its data breach. At first, the data breach impacted Equifax the most, including instituting protections and credit monitoring for impacted customers, without significant impacts on the broader financial industry.

However, the event also triggered broader impacts over time. Other companies have increased spending on cyber security and instituted stronger processes, including software upgrades.

\section{KNOWLEDGE CHECK}
\begin{enumerate}
  \item Which of the following types of risks are known in advance? Select all that apply.
A. A. Event risk
B. B. Exogenous risk
C. C. Thematic risk
\end{enumerate}

\section{Solution}
A and $\mathrm{C}$ are correct.

Event risk evolves around set dates, and thematic risk is a known risk that evolves and expands over a period of time. Exogenous risk is a sudden or unanticipated risk.

\section{Assessing Geopolitical Threats}
Geopolitical risk is always present in the investment environment, and these risks can impact investments in many different ways-from broad macroeconomic levels, to industry impacts, down to individual companies. The question for investors is whether the particular geopolitical risk is relevant for their portfolio management decisions. To make this assessment, an investor considers geopolitical risk in terms of the following three areas:

\begin{itemize}
  \item likelihood it will occur, - velocity (speed) of its impact, and

  \item size and nature of that impact.

\end{itemize}

\section{Likelihood}
The likelihood of a risk is the probability that it will occur. Measuring likelihood is a challenging process. The highly unpredictable nature of many risks-their build over time, the many and conflicting motivations of actors involved-means this exercise can be more art than science. However, we may use the framework from Section 4 to assess the basic likelihood of risk occurring. Highly collaborative and globalized countries are, on balance, less likely to experience geopolitical risk because the political, economic, and financial costs of partners inflicting those risks are higher. That same interconnectedness, however, may also make multilateral countries more vulnerable to certain risks. Their operation in and cooperation with other countries means a risk posed to any of those countries may also impact itself. Similarly, multiple risk exposures may increase the impact of that risk when it occurs.

Of course, many other factors may increase the likelihood that a risk may occur. Internal political stability, economic need, and the motivations of governmental actors play an important role in increasing the likelihood of disruptive action. In fact, these considerations are so plentiful and intertwined that geopolitical risk monitoring tools have emerged as key components of many research offerings for investors. The numerous scenarios for any given risk make measuring risk likelihood a potentially never-ending task. As a result, investors must balance the time spent on this activity with the relative importance of its input to the investment process.

The examples used in Section 6.1 can help us put a finer point on this concept. Of the three risks described, a cyber risk may have been the most likely to occur and to impact a given investment strategy, whereas the United Kingdom's "Yes" vote and Russia's annexation of Crimea were less likely at the time in which they occurred. Of course, all three of these risks have different potential impacts on investors. As a result, likelihood should only be considered in conjunction with the velocity and impact of the risk.

\section{Velocity}
The velocity of geopolitical risk is the pace at which it impacts an investor portfolio. For the sake of simplicity, we explore short-term or "high velocity" impacts, medium-term, and long-term or "low velocity" impacts (see Exhibit 8).

In the short term, we may see volatility in the markets affecting entire industries or even the entire market. Exogenous or "black swan" events tend to fit into this category, causing market volatility and investor flight to quality. A black swan risk is an event that is rare and difficult to predict but has an important impact. Investors with the appropriate time horizon and risk tolerance may make tactical changes to their investment choices as a result of these events. Long-term changes are unlikely to be necessary.

Risks with a medium-term impact may begin to impair companies' processes, costs, and investment opportunities, resulting in lower valuations. These risks tend to be distributed towards specific sectors, meaning they will impact some companies much more than others.

Long-term risks may have important environmental, social, governance, and other impacts. This can impact an investor's asset allocation-including choice of asset classes and investment styles-for a long-term horizon; however, the immediate impact on portfolios is likely to be more limited. Exhibit 10: Risk Velocity

\begin{center}
\includegraphics[max width=\textwidth]{2023_05_04_ccb578c5089fd41615beg-399}
\end{center}

It is worth noting that some risks have more than one speed of impact on investments. The United Kingdom's referendum outcome to exit the EU had some immediate impacts, notably a decline in the value of the British pound sterling and other forms of market volatility. Over time, low-velocity impacts have become more apparent and more lasting for investors. Higher transaction costs and the unwind of previous forms of political cooperation-including the freedom of movement between the United Kingdom and the EU-are generating important impacts on investment outcomes and overall economic growth.

\section{KNOWLEDGE CHECK}
\begin{enumerate}
  \item Match the potential velocity of a geopolitical risk with the most likely investor reaction among those listed in the following table:
\end{enumerate}

Geopolitical Risk Velocity Investor Reaction

\begin{enumerate}
  \item Low
\end{enumerate}

A. Adjust investments in specific sectors

\begin{enumerate}
  \setcounter{enumi}{1}
  \item Medium
\end{enumerate}

B. Flight to quality

\begin{enumerate}
  \setcounter{enumi}{2}
  \item High
\end{enumerate}

C. Adjust asset allocation

\section{Solution}
Option 1 (Low) matches with C (Adjust asset allocation).

Option 2 (Medium) matches with A (Adjust investments in specific sectors).

Option 3 (High) matches with B (Flight to quality).

\begin{enumerate}
  \setcounter{enumi}{1}
  \item Describe the three areas that an investor should consider when assessing geopolitical risk.
\end{enumerate}

\section{Solution}
An investor should consider the following three areas when assessing geopolitical risk: the likelihood it will occur, the velocity of its impact, and the size and nature of that impact. The likelihood that it will occur considers the probability that geopolitical risk will occur. The velocity of its impact is the pace at which it impacts an investor portfolio. The impact can manifest in many ways and can be discrete in size or broad in nature.

\section{Introduction to Geopolitics}
\section{Impact}
A risk's impact on investor portfolios can manifest in many different ways. For the sake of this framework for assessing risk, suffice it to say that investors should consider the size of any risk's impact when gauging its importance to the investment process. A high-impact risk may merit extensive study of its drivers and motivations, whereas a low-impact risk may not. In addition, the size of a risk's impact may also be compounded by external factors. For example, risk tends to have a greater impact on markets experiencing a general contraction or economic downturn.

Impact may also be discrete or broad in nature. Discrete impacts are those that impact only one company or sector at a time, whereas broad impacts are felt more holistically by a sector, a country, or the global economy. Cyber risks may be considered in this light. In the event of a cyberattack, only the companies and investment strategies exposed to that company will be impacted. However, cyber risks may also have a broader impact by increasing monitoring, due diligence, and security costs for all companies and investors seeking to avoid them.

When assessing geopolitical risk for portfolio management, investors should consider all three geopolitical risk factors-likelihood, velocity, and size and nature of impact-together. For example, a highly likely risk with very little impact to the portfolio may not merit extensive analysis and investor attention. However, a highly impactful risk with a low likelihood of occurring may merit building a scenario for response but not regular monitoring and assessment. Between these extremes, investors must consider their goals and risk tolerance to identify high-priority risks.

\section{Scenario analysis}
Geopolitical risks seldom develop in linear fashion, making it difficult to monitor and forecast their likelihood, velocity, and impact on a portfolio as well as difficult to address those changes through portfolio action. As a result, many investors deploy an approach that includes scenario analysis and signposting rather than a single point forecast.

Scenario analysis is the process of evaluating portfolio outcomes across potential circumstances or states of the world. Scenarios help investment teams understand where they stand with respect to a risk that might cause them to change their behavior. Scenario analysis can strengthen a team's conviction about its prioritization and calls to action, thereby helping it make good investment choices at opportune moments.

Scenarios can take the form of qualitative analysis, quantitative measurement, or both. A simple framework for qualitative scenario building begins with developing a base case for the event. What is the most impactful outcome of the risk? How likely is that risk/outcome to occur in the first place? From there, investors can consider upside and downside scenarios. Is it a persistent tail risk or a short-term shock? How are markets likely to recover once the event has taken place? Considering alternate futures for key risks will drive more precise perspective around what constitutes key developments in the most important risks.

Quantitative scenarios can vary widely by sophistication. One form of a simple quantitative scenario is a stylized scenario in which portfolio sensitivity is measured against one key factor relevant to the portfolio, such as interest rates, asset prices, or exchange rates. Another involves using circumstances from extreme events to help build quantitative tests for portfolio resilience. It's important to have reasonable ambitions, however. Quantitative scenarios can be complicated due to the secondary and linked impacts of geopolitical risks to securities in the portfolio.

Good scenario building can prompt investors to alter their risk prioritization, making it a useful tool not only for tracking risks but also for deciding which portfolio actions may be valuable to take. Good scenario analysis also requires a consistent commitment of investors' time and resources. Teams that read similar research or speak with similar client groups may be impacted by groupthink, the practice of thinking or making decisions as a group in a way that discourages creativity or individual responsibility. For scenario analysis to be useful in portfolio management, teams must work hard to build creative processes, identify scenarios, track them, and assess the need for action on a regular cadence.

\section{Tracking risks according to signposts}
In order to build a portfolio's resilience to unexpected change, asset managers develop processes in advance that allow for rapid course correction. In other words, by creating plans for addressing priority risks as they occur, investors can help reduce the events' impact on investment outcomes.

One important process is identifying signposts for priority risk. A signpost is an indicator, market level, data piece, or event that signals a risk is becoming more or less likely. An analyst can think of signposts like a traffic light. If quantitative and qualitative evidence suggest that a risk is low in likelihood, velocity, or impact, then the signposts are flashing "green," or "no action needed." If signposts are flashing "amber," indicating that a risk is medium in likelihood, velocity, or impact, then higher caution and preparedness against that risk may be warranted. As a risk rises in likelihood, velocity, or impact, an action plan may be necessary.

Identifying signposts should equip a team to differentiate signal from noise and react when signposts flash red. For instance, when the market environment moves to either risk-on or risk-off, it's important to identify what actions should be taken next or communications made. Good signposts are anchored in the key assumptions made up-front around a scenario and mark whether a scenario is materializing.

Let's return to our example of the United Kingdom's Brexit referendum for context. Before 2014, signposts for geopolitical risk in the United Kingdom may have flashed "green." Attitudes towards European Union membership were divided, and while there is always a possibility of disruptive geopolitical change, a reduction in political cooperation with the European Union was not clearly defined. Then, when the referendum was announced in 2015, signposts for disruptive change became more likely and faster in potential velocity, resulting in a higher required level of portfolio manager attention. In May 2016, when phone polls suggested that the "leave" vote was moving to the majority, signposts flashed "red" and attentive portfolio managers prepared action plans for election day.

Identifying the right signposts can require some trial and error. A basic rule of thumb for distinguishing signal from noise is the distinction between politics and policy. For example, there can be a big difference in "politics" between two leaders, but the "policies" they enact are what create larger or more durable portfolio impacts. Political developments can serve as meaningful signposts, as they can indicate a change in the risk's likelihood or pace. However, analysts are frequently knocked off course by following developments that do not necessarily indicate a change in real economic or market outcomes. Instead, analysts should look for "policy" changes to guide their portfolio management decisions.

Some combinations of economic and financial market circumstances serve as strong warnings of potential trouble. For example, high inflation and deteriorating employment can signal political unrest. A pegged currency and rapidly declining export value (particularly for commodities exporters) can prompt a change in exchange rate policy. Often, particularly for emerging markets, these signposts will change before official data are released. If a portfolio relies on country-level economic conditions, data screens should be used to help identify any red flags early.

\section{Manifestations of Geopolitical Risk}
If geopolitical risk takes many forms, its impact on investor portfolios is just as multi-faceted. High-velocity risks are most likely to manifest in market volatility via prompt changes in asset prices. Commonly impacted asset prices are those of commodities, foreign exchange, equities, and bond prices (via changes in interest rates).

One recent example is the market response to economic shutdowns related to the COVID-19 pandemic. Using the United States as an example, the S\&P 500 Index fell from a level of 3,386 on 19 February 2020 to 2,237.4 on 23 March, a decline of nearly $34 \%$. Bonds also experienced volatility. Global investors' "flight to safety" pushed US bond prices up. The US 10-year Treasury yield fell from 1.5661 to a low of 0.5407 during that period, a decline of nearly $68 \%$. This volatility was not permanent but created ample risk-and opportunity-for investors during that time.

Exhibit 11: US Market Reaction to the COVID-19 Pandemic, 1 January 2020-23 March 2021

\begin{center}
\includegraphics[max width=\textwidth]{2023_05_04_ccb578c5089fd41615beg-402}
\end{center}

Source: Bloomberg Finance LP.

Low-velocity geopolitical risks can have a more prolonged impact on investor inputs. Sustained disruption may result in smaller revenues, higher costs, or both, which can negatively impact a company's valuation. Here, too, the COVID-19 pandemic is an instructive example. While risk asset valuations improved over the course of the pandemic, disruptions to mobility and consumption had long-lasting impacts on company revenues and supply chains.

For countries, regions, or sectors perceived to be at more consistent risk of geopolitical disruption, investors may require higher compensation, effectively increasing the discount rate investors use when valuing those securities. Portfolio investment flows face greater volatility due to geopolitical factors, and investors will factor in a higher risk premium. This dynamic is a key reason why asset prices in emerging and frontier markets are typically maintained at a discount to those in developed countries perceived to have a lower threat of risk.

\section{Exhibit 12}
\section{Geopolitical Risk Index (GPR)}
In a 2019 paper, two analysts at the US Federal Reserve Board of Governors built the Geopolitical Risk Index (GPR) based on a tally of news articles covering geopolitical tensions and their impact on economic events. The purpose of the index is to measure real-time geopolitical risk as perceived by the press, the public, global investors, and policymakers in a way that is consistent over time. GPR

VIX

\begin{center}
\includegraphics[max width=\textwidth]{2023_05_04_ccb578c5089fd41615beg-403}
\end{center}

Through their construction of the GPR, the authors made three important observations. First, they found that high levels of geopolitical risk reduce US investment, employment, and price level of the stock market. Second, and taking this observation deeper, the authors found that individual firm's investment falls more in industries positively exposed to geopolitical risk and that firms reduce investment in the wake of idiosyncratic geopolitical risk events. Finally, they studied the adverse effect of geopolitical events themselves as well as the threat of adverse events, finding that the threat of events had a larger impact over time.

\section{KNOWLEDGE CHECK}
\begin{enumerate}
  \item True or False: Higher-velocity geopolitical risks are most likely to have a prolonged impact on investor inputs.
\end{enumerate}

Explain your selection.

A. True

B. False

Solution

$B$ is correct. Higher-velocity risks are most likely to manifest in market volatility via prompt changes in asset prices. Lower -velocity geopolitical risks are likely to have a prolonged impact on investor inputs. The terrorist attacks of September 11, 2001, in the United States are an example of a highervelocity geopolitical risk as the market had a sharp downturn and rebounded in a relatively short amount of time.

\begin{enumerate}
  \setcounter{enumi}{1}
  \item True or false: The probability and impact of geopolitical risks influence relative asset price discount rates across emerging and developed markets.
\end{enumerate}

Explain your selection.

A. True B. False

Solution

A is correct. For countries, regions, or sectors perceived to be at more consistent risk of geopolitical disruption, investors may require higher compensation, effectively increasing the discount rate used in valuation. When portfolio investment flows face greater volatility due to geopolitical factors, investors will factor in a higher risk premium. This accounts for why asset prices in emerging markets are typically maintained at a discount to those in developed countries perceived to have a lower threat of geopolitical risk, the latter more likely to experience lower probability risks with lesser impacts.

\section{Acting on Geopolitical Risk}
Determining the likelihood, velocity, and impact of a risk may help an investor to assign priority to which risks might be most important. But if the risk does occur, what, if anything, can be done about it? Even if an investor could anticipate every risk and its impact on their portfolios, they must still consider whether and how to act in the face of such threats. A final step in incorporating geopolitical risk into portfolio management process thus requires that geopolitical risk analysis be translated into investment action as appropriate for investor goals, risk tolerance, and time horizon.

Taking a top-down approach, asset allocators may consider geopolitical risk in their asset allocation strategy. The likelihood, velocity, and impact of risks may affect key capital markets assumptions as well as an asset allocator's positioning in certain countries or regions. For example, countries with a long history of using a multilateral approach may be considered more reliable investments and see increased investor flows. By contrast, those countries experiencing consistent military threat may have lower economic and investment growth potential due to consistent disruptions. The asset allocator would thus allocate more capital to the countries with lower expected risk profiles.

At the portfolio management level, investors can consider geopolitical risk as a factor in multi-factor models. Imagine an analyst assessing global car manufacturers. A company with highly diversified production may be exposed to more risk (i.e., higher likelihood), but production would be less likely to come to a halt given the multiple production alternatives (i.e., lower impact). When making a buy or sell recommendation, the analyst may consider relative geopolitical risk exposure as a factor in their analysis. Disruptive threats may be used as a binary yes-or-no factor or they can impact the confidence intervals around factors related to momentum, valuation, market sentiment, or the economic cycle.

Ultimately, the importance of geopolitical risk to the investment process depends on investor objectives, risk tolerance, and time horizon. For an investor with low risk tolerance, reducing exposure to geopolitical risk may be appropriate, whether through low-volatility investment choices or through hedging.

For an investor with a long time horizon, a geopolitical event like an exogenous shock could be a buying opportunity. For an investor nearing retirement, however, that same exogenous shock can have a major negative impact on their terminal portfolio value.

The extensive political, economic, and financial cooperation in which countries, companies, and organizations participate may raise the stakes of geopolitical risk analysis for global investors. Changes in the style and momentum of international cooperation can have an important impact on capital markets. Global investors ignore those risks at their peril.

\section{SUMMARY}
\begin{itemize}
  \item Geopolitics is the study of how geography affects politics and international relations. Within the field of geopolitics, analysts study actors-the individuals, organizations, companies, and national governments that carry out political, economic, and financial activities-and how they interact with one another.

  \item State actors can be cooperative or non-cooperative. There are many reasons why a country may want to cooperate with its neighbors or with other state actors. These reasons are typically defined by a country's national interest.

  \item National security or national defense involves protecting a country, its citizens, economy, and institutions, from external threats. These threats may be broad in nature-from military attack, to terrorism, crime, cyber-security, and even natural disasters.

  \item Geographic factors play an important role in shaping a country's approach to national security and the extent to which it will choose a cooperative approach. Landlocked countries rely extensively on their neighbors for access to vital resources. Countries highly connected to trade routes or countries acting as a conduit for trade may use their geographic location as a lever of power.

  \item Generally, strong institutions contribute to more stable internal and external political forces. Countries with strong institutions, including organizations and structures promoting government accountability, rule of law, and property rights, allow them to act with more authority.

  \item Globalization is marked by economic and financial cooperation, including the active trade of goods and services, capital flows, currency exchange, and cultural and information exchange. By contrast, anti-globalization or nationalism is the promotion of a country's own economic interests to the exclusion or detriment of the interests of other nations. Nationalism is marked by limited economic and financial cooperation.

  \item Globalization provides potential gains, such as:

  \item increased profits-through increasing sales and/or reducing costs,

  \item access to resources-market access and investment opportunities, and

  \item intrinsic gains-an improved quality of life.

  \item Globalization also has some potential drawbacks, such as:

  \item unequal economic and financial gains,

  \item interdependence that can lead to supply chain disruption, and

  \item possible exploitation of social and environmental resources.

  \item In general terms, regions, countries, and industries that are more dependent on cross-border goods and capital flows will have higher levels of cooperation.

  \item The interdependent nature of globalization may reduce the likelihood that collaborative countries levy economic, financial, or political attacks on one another. However, interdependence can make cooperative actors more vulnerable to geopolitical risk than those less dependent on cooperation and trade

\end{itemize}

\section{Introduction to Geopolitics}
\begin{itemize}
  \item A geopolitical framework for analysis includes four archetypes of country behavior: autarky, hegemony, multilateralism, and bilateralism. Each archetype has its own costs, benefits, and tradeoffs with respect to geopolitical risk.

  \item The tools of geopolitics may be separated into three types:

  \item national security tools,

  \item economic tools, and

  \item financial tools.

  \item The most extreme example of a national security tool is that of armed conflict. Espionage is an indirect national security tool. Military alliances are often used either to aid in direct conflict or to deter conflict from arising in the first place.

  \item Economic tools are used to reinforce a cooperative or non-cooperative stance via economic means. Among state actors, economic tools can include multilateral trade agreements or the global harmonization of tariff rules. Economic tools can also be non-cooperative in nature. Nationalization is a non-cooperative approach to asserting economic control.

  \item Financial tools are the actions used to reinforce a cooperative or non-cooperative stance via financial mechanisms. Examples of cooperative financial tools include the free exchange of currencies across borders and allowing foreign investment. Examples of non-cooperative financial tools include limiting access to local currency markets and restricting foreign investment.

  \item Geopolitical risk is the risk associated with tensions or actions between actors (state and non-state) that affect the normal and peaceful course of international relations. Geopolitical risk tends to rise when the geographic and political factors underpinning country relations shift.

  \item There are three basic types of geopolitical risk:

  \item event risk,

  \item exogenous risk, and

  \item thematic risk.

  \item Event risk evolves around set dates known in advance. Political events, for example, often result in changes to investor expectations related to a country's cooperative stance. Brexit is an example of event risk.

  \item Exogenous risk is a sudden or unanticipated risk that can impact either a country's cooperative stance, the ability of non-state actors to globalize, or both. Examples include sudden uprisings, invasions, or the aftermath of natural disasters.

  \item Thematic risks are known risks that evolve and expand over a period of time. Climate change, cyber threats, and the ongoing threat of terrorism fall into this category.

  \item To make an assessment, an investor considers geopolitical risk in terms of the following three areas:

\end{itemize}

\begin{enumerate}
  \item likelihood it will occur,

  \item velocity (speed) of its impact, and

  \item size and nature of that impact. - Geopolitical risks seldom develop in linear fashion, making it difficult to monitor and forecast their likelihood, velocity, and size and nature of impact on a portfolio. As a result, many investors deploy an approach that includes scenario building and signposting rather than a single point forecast.

\end{enumerate}

\begin{itemize}
  \item Scenario analysis is the process of evaluating portfolio outcomes across potential circumstances or states of the world. Scenarios can take the form of qualitative analysis, quantitative measurement, or both.

  \item Investors study geopolitical risk because it has a tangible impact on investment outcomes. On a macroeconomic level, these risks can impact capital markets conditions, such as economic growth, interest rates, and market volatility.

  \item Changes in capital markets conditions can have an important influence on asset allocation decisions, including an investor's choice of geographic exposures.

  \item On a portfolio level, geopolitical risk can influence the appropriateness of an investment security or strategy for an investor's goals, risk tolerance, and time horizon.

\end{itemize}

\section{PRACTICE PROBLEMS}
\begin{enumerate}
  \item Which of these is likely lowest on a country's hierarchy of interests?
A. Tariff harmonization
B. Military determination
C. Cultural program development

  \item Which of the following actions by a country is most likely a form of geopolitical cooperation?

\end{enumerate}

A. Acting as a conduit for trade

B. Engaging in rules standardization

C. Opting to use soft power over military retaliation

\begin{enumerate}
  \setcounter{enumi}{2}
  \item Which of the following statements represents an aspect of geopolitical risk?
\end{enumerate}

A. Modeling geopolitical risk is relatively easy to standardize.

B. An engaged country can be considered cooperative, even if it does not reciprocate.

C. The strength of a country's institutions is relevant to the durability of its cooperative relationships.

\begin{enumerate}
  \setcounter{enumi}{3}
  \item Which of the following statements regarding a country's political cooperation is most accurate?
\end{enumerate}

A. If a country is engaged in military conflict, there is a higher cost to cooperation.

B. A country with few internal resources is not likely to rely on political cooperation.

C. Interest prioritization does not determine the depth and nature of political cooperation.

\begin{enumerate}
  \setcounter{enumi}{4}
  \item For the following contrasting pairs of archetypes of globalization and cooperation, which one reflects the greatest differences in country behavior?
\end{enumerate}

A. Bilateralism vs. autarky

B. Multilateralism vs. autarky

C. Multilateralism vs. hegemony

\begin{enumerate}
  \setcounter{enumi}{5}
  \item Which of these actions would do the most to increase geopolitical risk?
\end{enumerate}

A. Increase capital flows

B. Restrict foreign currency exchange

C. Engage in trade of goods and services 7. A consequence of one of the disadvantages of globalization is that:

A. pay differences between countries have narrowed.

B. emerging market trade flows have grown more important.

C. greater economic and financial cooperation has increased interdependence.

\begin{enumerate}
  \setcounter{enumi}{7}
  \item Which of the following outcomes is most likely a result of globalization?
A. Unequal gains
B. Increased independence
C. Decreased access to talent

  \item A US company expanding critical spare part inventories for local customers made at its existing Canadian facility after a supply chain disruption is most likely using the coping tactic of:

\end{enumerate}

A. reshoring the essentials.

B. re-globalizing production.

C. doubling down on key markets.

\begin{enumerate}
  \setcounter{enumi}{9}
  \item In the following table, match the geopolitical tool with the most appropriate example of each tool by drawing a line between the two.
\end{enumerate}

Geopolitical Tool Example

\begin{enumerate}
  \item Financial
\end{enumerate}

A. Nationalization

\begin{enumerate}
  \setcounter{enumi}{1}
  \item Economic
\end{enumerate}

B. Espionage

\begin{enumerate}
  \setcounter{enumi}{2}
  \item National security
\end{enumerate}

C. Free exchange of currency across borders

\begin{enumerate}
  \setcounter{enumi}{10}
  \item An example of a geopolitical multi-tool for furthering national interests is:
A. cabotage.
B. armed conflict.
C. nationalization of key export industries.

  \item Which of these is most likely to be described as an event risk?
A. An earthquake
B. An election
C. An ongoing civil war

  \item Exogenous risks are best described as those that:

\end{enumerate}

A. are known and evolve and expand over a period of time.

B. evolve around set dates, such as elections, new legislation, or other date-driven milestones, such as holidays or political anniversaries.

C. are sudden or unanticipated and impact either a country's cooperative stance, the ability of non-state actors to globalize, or both. 14. Which of the following statements about geopolitical threats in the investment environment is most accurate?

A. Geopolitical risk is not always present in the investment environment.

B. Highly collaborative, interconnected countries are vulnerable to geopolitical risk.

C. Geopolitical risk tends to have less of an impact on markets already experiencing a general contraction or economic downturn.

\begin{enumerate}
  \setcounter{enumi}{14}
  \item An applicable conclusion drawn from the Geopolitical Risk Index (GPR) is that:
A. high geopolitical risk results in tangible macroeconomic effects.
B. recurring geopolitical risk events lead to reduced corporate investment.
C. the adverse impact of actual events is greater over time than that of the threat of such events.

  \item Fill in the blank: The basic geopolitical risk type most likely in comparison to have the smallest degree of uncertainty is
A. exogenous risk.
B. event risk.
C. thematic risk.

\end{enumerate}

\section{SOLUTIONS}
\begin{enumerate}
  \item C is correct. Cultural program development is likely lowest on a country's hierarchy of interests. Military determination (B) is often a primary source of national security and key to a country's national interest. Tariff harmonization (A) may improve economic activity and improve cooperation. Cultural programs are important and influential but likely lower priority compared to A and B.

  \item B is correct. Political cooperation is associated with anything related to agreements of rules and standardization, with countries working together towards some shared goal. A cooperative country is one that engages and reciprocates in rules standardization. $A$ is incorrect because acting as a conduit of trade, like Panama, involves non-cooperatively using a country's geographic location as a lever of power in broader international dynamics. $C$ is incorrect because both soft power and military retaliation are examples of non-cooperative behavior, with the former being a less extreme means to influence another country's decisions without force or coercion.

  \item C is correct. The strength of a country's institutions can make cooperative relationships more durable. A is incorrect because modeling geopolitical risk is not easily standardized. B is incorrect because a cooperative country is one that is both engaged and reciprocates.

  \item A is correct. If a country is engaged in military conflict, there is a higher cost to cooperation. B is incorrect because a country with few internal resources is likely to rely on political cooperation. $\mathrm{C}$ is incorrect because interest prioritization does determine the depth and nature of political cooperation.

  \item B is correct. Multilateralism describes countries that participate in mutually beneficial trade relationships and extensive rules harmonization. Autarky describes countries seeking political self-sufficiency with little or no external trade or finance. In the Exhibit 6 display of these behavior patterns, these choices are most widely separated on both the globalization and cooperation continuums. A is incorrect because bilateral or regional approaches describe those countries leveraging regional trade relationships and may face the world as a group. Bilateralism shares with autarky a bias against globalization. These approaches diverge, however, regarding cooperation, with autarky being more non-cooperative. $C$ is incorrect because multilateralism describes countries that participate in mutually beneficial trade relationships and extensive rules harmonization. Hegemony represents countries exerting political or economic influence of others to control resources. Multilateralism shares with hegemony an inclination towards globalization, as shown in Exhibit 6, but diverges from hegemony regarding cooperation; hegemony is more non-cooperative.

  \item B is correct. Restricted foreign currency exchange-a characteristic of anti-globalization-would likely reduce political and economic cooperation and thus increase geopolitical risk. A is incorrect because an increase in capital flows would reduce geopolitical risk. $\mathrm{C}$ is incorrect because an increase in trade would reduce geopolitical risk.

  \item C is correct. Through the process of greater economic and financial cooperation, companies may become dependent on other countries' resources for their supply chains. On aggregate, this can result in the nation itself becoming dependent on other nations for certain resources (such as rare earth metals, cobalt, or copper). If there is a disruption to the supply chain, including via a moment of political

\end{enumerate}

\section{Introduction to Geopolitics}
non-cooperation, then firms may not be able to produce the good themselves.

A is incorrect because the narrowing of pay differences between countries results from a motivation to pursue globalization, which is related to one of its advantages. It is an important way for companies to increase profitability by reducing their costs. While wage differentials remain, they are decreasing. B is incorrect because the proportion of flows between developed economies as a share of overall trade continuing to decline as emerging market trade flows rise is a function of increased international investment. This increased investment has provided beneficial aggregate economic benefits, such as increased choice, higher quality goods, increased competition among firms, higher efficiency, and increased labor mobility.

\begin{enumerate}
  \setcounter{enumi}{7}
  \item A is correct. Unequal accrual of economic and financial gains is a cost of globalization because improvement on the aggregate does not mean improvement for everyone. B is incorrect because globalization leads to interdependency as companies may become dependent on other countries' resources for their supply chains. $C$ is incorrect because rather than decreased access to talent, a country might actually globalize to improve access to talent as cooperation and globalization can lead to increased access to resources.

  \item A is correct. The COVID-19 pandemic has highlighted the need for certain essential supply chains to be rebuilt domestically for emergency situations, with availability of critical spare parts being an analogy. The close integration of the US and Canadian economies through the revised NAFTA agreement effectively makes expanded production at an existing Canadian factory an example of reducing manufacturing risk by relocating to home countries via reshoring. B is incorrect because instead of reducing manufacturing risk by duplicating or fortifying its supply chain, the company is simply continuing to use its existing capacity more intensively. $\mathrm{C}$ is incorrect because although the production is intended to better supply its home market, there is no evidence that the company is expanding its presence in the US market or shifting its focus to the exclusion of available opportunities elsewhere.

  \item Option 1 (Financial) matches with C (Free exchange of currency across borders). Option 2 (Economic) matches with A (Nationalization. Option 3 (National security) matches with B (Espionage).

  \item A is correct. Cabotage is the right to transport passengers or goods within a country by a foreign firm. Many countries-including those with multilateral trade agreements-impose restrictions on cabotage across transportation subsectors, meaning that shippers, airlines, and truck drivers are not allowed to transport goods and services within another country's borders. Allowing cabotage requires coordination on areas like physical security and economic coordination, a highly multilateral (multi-tool) process. B is incorrect because armed conflict is the most extreme example of a national security tool. It can be either internal or external to a country in taking a direct and active approach to wielding influence. $\mathrm{C}$ is incorrect because nationalization of key export industries is an economic tool. This process of transferring an activity or industry from private to state control is a non-cooperative approach to asserting economic control.

  \item B is correct. Event risk evolves around set dates, such as elections and new legislation, or other date-driven milestones, such as holidays or political anniversaries known in advance. The other choices could not be known in advance. An earthquake (A) is an example of an exogenous risk. An ongoing civil war (C) is an example of a thematic risk.

  \item $\mathrm{C}$ is correct. Exogenous risks are best described as those that are sudden or unanticipated and that impact either a country's cooperative stance, the ability of non-state actors to globalize, or both. A is incorrect because thematic (not exogenous) risks are known risks that evolve and expand over a period of time. B is incorrect because event (not exogenous) risks are those that evolve around set dates, including elections, new legislation, or other date-driven milestones.

  \item B is correct. Although highly collaborative and globalized countries are, on balance, less likely to experience geopolitical risk because the political, economic, and financial costs of partners inflicting those risks are higher than less collaborative countries, the same interconnectedness may make them more vulnerable to geopolitical risk. A is incorrect because geopolitical risk is always present in the investment environment. $\mathrm{C}$ is incorrect because geopolitical risk tends to have a greater impact on markets already experiencing a general contraction or economic downturn.

  \item A is correct. The GPR index creators found that high levels of geopolitical risk reduce US investment, employment, and price level of the stock market. B is incorrect because firms reduce investment in the wake of idiosyncratic events, which would be unlikely to repeat. $C$ is incorrect because the threat of an event was shown to have a larger impact over time than that of the actual events themselves.

  \item B is correct. Event risk evolves around set dates, such as elections or new legislation, or other date-driven milestones, such as holidays or political anniversaries. Analysts can thus look to political calendars as a predictable starting point for determining the occurrence of event risk, with time to devise a suitable response. A is incorrect because exogenous risk is sudden and unanticipated. Examples include sudden uprisings, invasions, or the aftermath of natural disasters. The timing and range of its effects thus have the greatest unknowns. $C$ is incorrect because thematic risks are known risks that evolve and expand over a period of time. Climate change, pattern migration, the rise of populist forces, and the ongoing threat of terrorism fall into this category. These are more foreseeable than exogenous risks, but with their extended interval of exposure, planned responses will likely require continuing adjustments.

\end{enumerate}

\section*{LEARNING MODULE }
\section{7}
\section{International Trade and Capital Flows}
Usha Nair-Reichert, PhD, is at Georgia Institute of Technology (USA). Daniel Robert

Witschi, is at Dreyfus Sons \& Co Ltd, PhD, CFA (Switzerland).

\section{LEARNING OUTCOME}
\begin{center}
\begin{tabular}{|c|c|}
\hline
Mastery & The candidate should be able to: \\
\hline
; & $\begin{array}{l}\text { compare gross domestic product and gross national product } \\ \text { describe benefits and costs of international trade } \\ \text { contrast comparative advantage and absolute advantage } \\ \text { compare the Ricardian and Heckscher-Ohlin models of trade and } \\ \text { the source(s) of comparative advantage in each model } \\ \text { compare types of trade and capital restrictions and their economic } \\ \text { implications } \\ \text { explain motivations for and advantages of trading blocs, common } \\ \text { markets, and economic unions } \\ \text { describe common objectives of capital restrictions imposed by } \\ \text { governments } \\ \text { describe the balance of payments accounts including their } \\ \text { components } \\ \text { explain how decisions by consumers, firms, and governments affect } \\ \text { the balance of payments } \\ \text { describe functions and objectives of the international organizations } \\ \text { that facilitate trade, including the World Bank, the International } \\ \text { Monetary Fund, and the World Trade Organization }\end{array}$ \\
\hline
\end{tabular}
\end{center}

INTRODUCTION \& INTERNATIONAL TRADE-BASIC TERMINOLOGY

compare gross domestic product and gross national product Global investors must address two fundamentally interrelated questions: where to invest and in what asset classes? Some countries may be attractive from an equity perspective because of their strong economic growth and the profitability of particular domestic sectors or industries. Other countries may be attractive from a fixed income perspective because of their interest rate environment and price stability. To identify markets that are expected to provide attractive investment opportunities, investors must analyze cross-country differences in such factors as expected GDP growth rates, monetary and fiscal policies, trade policies, and competitiveness. From a longer term perspective investors also need to consider such factors as a country's stage of economic and financial market development, demographics, quality and quantity of physical and human capital (accumulated education and training of workers), and its area(s) of comparative advantage. ${ }^{1}$

This reading provides a framework for analyzing a country's trade and capital flows and their economic implications. International trade can facilitate economic growth by increasing the efficiency of resource allocation, providing access to larger capital and product markets, and facilitating specialization based on comparative advantage. The flow of financial capital (funds available for investment) between countries with excess savings and those where financial capital is scarce can increase liquidity, raise output, and lower the cost of capital. From an investment perspective, it is important to understand the complex and dynamic nature of international trade and capital flows because investment opportunities are increasingly exposed to the forces of global competition for markets, capital, and ideas.

This reading is organized as follows. Section 2 defines basic terminology used in the reading and describes patterns and trends in international trade and capital flows. It also discusses the benefits of international trade, distinguishes between absolute and comparative advantage, and explains two traditional models of comparative advantage. Section 3 describes trade restrictions and their implications and discusses the motivation for, and advantages of, trade agreements. Section 4 describes the balance of payments and Section 5 discusses the function and objectives of international organizations that facilitate trade. A summary of key points and practice problems conclude the reading.

\section{International Trade}
The following sections describe the role, importance, and possible benefits and costs of international trade. Before beginning those discussions, we define some basic terminology used in this area.

\section{Basic Terminology}
The aggregate output of a nation over a specified time period is usually measured as its gross domestic product or its gross national product. Gross domestic product (GDP) measures the market value of all final goods and services produced by factors of production (such as labor and capital) located within a country/economy during a given period of time, generally a year or a quarter. Gross national product (GNP), however, measures the market value of all final goods and services produced by factors of production (such as labor and capital) supplied by citizens of a country, regardless of whether such production takes place within the country or outside of the country. The difference between a country's GDP and its GNP is that GDP includes, and GNP excludes, the production of goods and services by foreigners within that country, whereas GNP includes, and GDP excludes, the production of goods and

1 Comparative advantage refers to a country's ability to produce a good at a relatively lower cost than other goods it produces, as compared with another country. It will be more precisely defined and illustrated in Section 2. services by its citizens outside of the country. Countries that have large differences between GDP and GNP generally have a large number of citizens who work abroad (for example, Pakistan and Portugal), and/or pay more for the use of foreign-owned capital in domestic production than they earn on the capital they own abroad (for example, Brazil and Canada). Therefore, GDP is more widely used as a measure of economic activity occurring within the country, which, in turn, affects employment, growth, and the investment environment.

Imports are goods and services that a domestic economy (i.e., households, firms, and government) purchases from other countries. For example, the US economy imports (purchases) cloth from India and wine from France. Exports are goods and services that a domestic economy sells to other countries. For example, South Africa exports (sells) diamonds to the Netherlands, and China exports clothing to the European Union. So how are services imported or exported? If a Greek shipping company transports the wine that the United States imports from France, the United States would classify the cost of shipping as an import of services from Greece and the wine would be classified as an import of goods from France. Similarly, when a British company provides insurance coverage to a South African diamond exporter, Britain would classify the cost of the insurance as an export of services to South Africa. Other examples of services exported/imported include engineering, consulting, and medical services.

The terms of trade are defined as the ratio of the price of exports to the price of imports, representing those prices by export and import price indexes, respectively. The terms of trade capture the relative cost of imports in terms of exports. If the prices of exports increase relative to the prices of imports, the terms of trade have improved because the country will be able to purchase more imports with the same amount of exports. ${ }^{2}$ For example, when oil prices increased during 2007-2008, major oil exporting countries experienced an improvement in their terms of trade because they had to export less oil in order to purchase the same amount of imported goods. In contrast, if the price of exports decreases relative to the price of imports, the terms of trade have deteriorated because the country will be able to purchase fewer imports with the same amount of exports. Because each country exports and imports a large number of goods and services, the terms of trade of a country are usually measured as an index number (normalized to 100 in some base year) that represents a ratio of the average price of exported goods and services to the average price of imported goods and services. Exhibit 1 shows the terms of trade reported by the World bank, aggregated by their categories for region and income group. A value over (under) 100 indicates that the country, or group of countries, experienced better (worse) terms of trade relative to the base year of 2000 .

Exhibit 1: Data on the Barter Terms of Trade for Industrial and Developing Countries

(Unit Export Value/Unit Import Value)

\begin{center}
\begin{tabular}{lccccccc}
\hline
 & $\mathbf{1 9 9 0}$ & $\mathbf{1 9 9 5}$ & $\mathbf{2 0 0 0}$ & $\mathbf{2 0 0 5}$ & $\mathbf{2 0 1 0}$ & $\mathbf{2 0 1 5}$ &  \\
\hline
High Income & 99.2 & 110.4 & 100.0 & 105.3 & 115.4 & 113.3 &  \\
Low Income & 125.9 & 118.2 & 100.0 & 98.0 & 117.2 & 113.8 & 132.4 \\
\end{tabular}
\end{center}

2 Although the prices of imports and exports are each stated in currency units, the currency units cancel out when we take the ratio, so the terms of trade reflect the relative price of imports and exports in real (i.e., quantity) terms: units of imports per unit of exports. To see this, note that if one unit of imports costs $P_{M}$ currency units and one unit of exports is priced at $P_{X}$ currency units, then the country can buy $P_{X} / P_{M}$ (= Terms of trade) units of imports for each unit of exports.

\begin{center}
\begin{tabular}{|c|c|c|c|c|c|c|}
\hline
 & 1990 & 1995 & 2000 & 2005 & 2010 & 2015 \\
\hline
Asia & 115.5 & 114.7 & 100.0 & 99.2 & 109.6 & 105.9 \\
\hline
Europe & 109.1 & 105.7 & 100.0 & 99.8 & 105.0 & 101.6 \\
\hline
Western Hemisphere & 98.7 & 104.7 & 100.0 & 104.9 & 115.9 & 107.5 \\
\hline
Middle East & 92.6 & 94.9 & 100.0 & 125.7 & 165.2 & 164.1 \\
\hline
\end{tabular}
\end{center}

As an example, Exhibit 1 indicates that from 2000 to 2015 both low-income and high-income countries generally experienced an increase in their terms of trade; however, low-income countries experienced more volatility over the period. Looking at the disaggregated data indicates that Africa also experienced improvement in its terms of trade, albeit with volatility, during this period. Countries in Asia and the Western Hemisphere experienced some increase in their terms of trade while those in Europe saw little change. In contrast, countries in the Middle East (which benefited from rising prices of their petroleum exports) experienced a substantial increase in their terms of trade.

Net exports is the difference between the value of a country's exports and the value of its imports (i.e., value of exports minus imports). If the value of exports equals the value of imports, then trade is balanced. If the value of exports is greater (less) than the value of imports, then there is a trade surplus (deficit). When a country has a trade surplus, it lends to foreigners or buys assets from foreigners reflecting the financing needed by foreigners running trade deficits with that country. Similarly, when a country has a trade deficit, it has to borrow from foreigners or sell some of its assets to foreigners. Section 4 on the balance of payments explains these relationships more fully.

Autarky is a state in which a country does not trade with other countries. This means that all goods and services are produced and consumed domestically. The price of a good or service in such an economy is called its autarkic price. An autarkic economy is also known as a closed economy because it does not trade with other countries. An open economy, in contrast, is an economy that trades with other countries. If there are no restrictions on trade, then members of an open economy can buy and sell goods and services at the price prevailing in the world market, the world price. An open economy can provide domestic households with a larger variety of goods and services, give domestic companies access to global markets and customers, and offer goods and services that are more competitively priced. In addition, it can offer domestic investors access to foreign capital markets, foreign assets, and greater investment opportunities. For capital intensive industries, such as automobiles and aircraft, manufacturers can take advantage of economies of scale because they have access to a much larger market. Free trade occurs when there are no government restrictions on a country's ability to trade. Under free trade, global aggregate demand and supply determine the equilibrium quantity and price of imports and exports. Government policies that impose restrictions on trade, such as tariffs and quotas (discussed later in the reading), are known as trade protection and prevent market forces (demand and supply) from determining the equilibrium price and quantity for imports and exports. According to Deardorff, globalization refers to the "increasing worldwide integration of markets for goods, services, and capital that began to attract special attention in the late 1990s."3 It also references "a variety of other changes that were perceived to occur at about the same time, such as an increased role for large corporations (multinational

3 Deardorff, Alan. "Deardorff's Glossary of International Economics" (\href{http://www-personal.umich.edu/}{www-personal.umich.edu/} alandear/ glossary) corporations) in the world economy and increased intervention into domestic policies and affairs by international institutions," such as the International Monetary Fund, the World Trade Organization, and the World Bank.

The levels of aggregate demand and supply and the quantities of imports and exports in an economy are related to the concepts of excess demand and excess supply. Exhibit 2 shows supply and demand curves for cars in the United Kingdom. E is the autarkic equilibrium at price $P^{A}$ and quantity $Q^{A}$, with the quantity of cars demanded equaling the quantity supplied. Now, consider a situation in which the country opens up to trade and the world price is $P_{1}$. At this price, the quantity demanded domestically is $Q^{K}$ while the quantity supplied is $Q^{J}$. Hence excess demand is $Q^{J} Q^{K}$. This quantity is satisfied by imports. For example, at a world price of $\$ 15,000$, the quantity of cars demanded in the United Kingdom might be 2 million and UK production of cars only 1.5 million. As a result, the excess demand of 500,000 would be satisfied by imports. Returning to Exhibit 2, now consider a situation in which the world price is $P_{2}$. The quantity demanded is $Q^{C}$ while the quantity supplied is $Q^{D}$. Hence, the domestic excess supply at world price $P_{2}$ is $Q^{C} Q^{D}$, which results in exports of $Q^{C} Q^{D}$.

\section{Exhibit 2: Excess Demand, Excess Supply, Imports and Exports}
\begin{center}
\includegraphics[max width=\textwidth]{2023_05_04_ccb578c5089fd41615beg-419}
\end{center}

\section{PATTERNS AND TRENDS IN INTERNATIONAL TRADE AND CAPITAL FLOWS}
The importance of trade in absolute and relative terms (trade-to-GDP ratio) is illustrated in Exhibit 3 through Exhibit 5. Exhibit 3 shows that trade as a percentage of regional GDP increased in all regions of the world during 1970-2006. Developing countries in Asia had the fastest growth in trade, increasing from less than 20 percent of GDP in 1970 to more than 70 percent of GDP in 2006.

\section{Exhibit 3: Trade in Goods and Services (Percent of Regional GDP)}
\begin{center}
\includegraphics[max width=\textwidth]{2023_05_04_ccb578c5089fd41615beg-420}
\end{center}

Note: Developing East Asia and Developing Europe \& Central Asia exclude all World Bank designated "High Income" countries in these regions.

Source: World Development Indicators

Exhibit 4 indicates that trade as a percentage of GDP and the GDP growth rate increased in most regions of the world during 1990-2009. However, data for 2010-2016 indicates a decline that, although consistent with the worldwide economic downturn, varied across country groups. High-income countries that are members of the Organisation for Economic Co-Operation and Development (OECD) experienced a growth rate of 2.4 percent during $2000-2006$, but had a growth rate of only 1.1 percent in between 2010-2016. The corresponding numbers for growth in non-OECD high-income countries are 5.0 percent and 2.0 percent, respectively; for lower-middle-income countries, they are 7.7 percent and 3.3 percent, respectively. The 2009 World Development Report affirmed the link between trade and growth and noted evidence that all rich and emerging economies are oriented to being open to trade. More specifically, the report indicated:

...When exports are concentrated in labor-intensive manufacturing, trade increases the wages for unskilled workers, benefiting poor people. It also encourages macroeconomic stability, again benefiting the poor, who are more likely to be hurt by inflation. And through innovation and factor accumulation, it enhances productivity and thus growth. There may be some empirical uncertainty about the strength of trade's relationship with growth. But essentially all rich and emerging economies have a strong trade orientation. (World Bank 2009)

Of course, trade is not the only factor that influences economic growth. Research has also identified such factors as the quality of institutions, infrastructure, and education; economic systems; the degree of development; and global market conditions (World Trade Organization 2008).

\section{Exhibit 4: Trade Openness and GDP Growth}
Trade as Percent of GDP (averaged over the period) Average GDP growth (\%)

\begin{center}
\begin{tabular}{|c|c|c|c|c|c|c|c|c|}
\hline
$\begin{array}{l}\text { Country } \\ \text { Group }\end{array}$ & $1980-1989$ & $1990-1999$ & $2000-2009$ & $2010-2016$ & $1980-1989$ & 1990-1999 & 2000-2009 & $2010-2016$ \\
\hline
World & 75.4 & 77.9 & 91.6 & 94.1 & $3.1 \%$ & $2.5 \%$ & $3.2 \%$ & $2.1 \%$ \\
\hline
\multicolumn{9}{|c|}{High income:} \\
\hline
All & 40.7 & 42.8 & 53.6 & 61.3 & $3.1 \%$ & $2.5 \%$ & $1.5 \%$ & $1.1 \%$ \\
\hline
OECD & 36.5 & 39.1 & 48.7 & 56.0 & $3.1 \%$ & $2.4 \%$ & $1.3 \%$ & $1.1 \%$ \\
\hline
Non-OECD & 126.3 & 116.5 & 134.9 & 138.8 & $3.9 \%$ & $3.4 \%$ & $3.4 \%$ & $2.0 \%$ \\
\hline
\multicolumn{9}{|c|}{Low and middle income:} \\
\hline
All & 30.2 & 42.3 & 56.3 & 52.7 & $3.4 \%$ & $2.6 \%$ & $5.5 \%$ & $3.0 \%$ \\
\hline
Middle & 30.0 & 42.2 & 56.2 & 52.6 & $3.4 \%$ & $2.6 \%$ & $5.5 \%$ & $3.0 \%$ \\
\hline
$\begin{array}{l}\text { Upper } \\ \text { middle }\end{array}$ & 28.9 & 41.5 & 55.7 & 51.5 & $2.1 \%$ & $2.4 \%$ & $5.5 \%$ & $2.9 \%$ \\
\hline
$\begin{array}{l}\text { Lower } \\ \text { middle }\end{array}$ & 33.7 & 44.7 & 57.9 & 56.3 & $6.0 \%$ & $3.0 \%$ & $5.5 \%$ & $3.3 \%$ \\
\hline
Low & 52.9 & 54.1 & 63.5 & 66.9 & $2.6 \%$ & $1.9 \%$ & $4.4 \%$ & $2.3 \%$ \\
\hline
\end{tabular}
\end{center}

Note: Averages indicate the average of the annual data for the period covered

Source: World Bank.

Exhibit 5 presents trade and foreign direct investment as a percentage of GDP for select countries for 1990-2017. Foreign direct investment (FDI) refers to direct investment by a firm in one country (the source country) in productive assets in a foreign country (the host country). When a firm engages in FDI, it becomes a multinational corporation (MNC) operating in more than one country or having subsidiary firms in more than one country. It is important to distinguish FDI from foreign portfolio investment (FPI), which refers to shorter-term investment by individuals, firms, and institutional investors (e.g., pension funds) in such foreign financial instruments as foreign stocks and foreign government bonds. Exhibit 5 shows that trade as a percentage of GDP for the world as a whole increased from 39 percent in 1990 to 57 percent in 2010. In Argentina, trade as a percentage of GDP increased from 15 percent in 1990 to 35 percent in 2010, while in India during this same period it increased from 16 percent to almost 50 percent. Among the more advanced economies, trade expanded sharply in Germany (from 46 percent to 87 percent between 1990 and 2017), but in the United States trade expanded more modestly (from 20 percent to 27 percent).

Exhibit 5: Increasing Global Interdependence

FDI and Trade as a percentage of GDP

\begin{center}
\begin{tabular}{llcccc}
\hline
Country & \multicolumn{1}{c}{Type of Flow} & $\mathbf{1 9 9 0}$ & $\mathbf{2 0 0 0}$ & $\mathbf{2 0 1 0}$ & $\mathbf{2 0 1 7}$ \\
\hline
World & Trade & 38.9 & 51.3 & 56.9 & $56.2^{*}$ \\
 & FDI: Net Inflows & 0.9 & 4.4 & 2.7 & 2.4 \\
 & FDI: Net Outflows & 1.3 & 4.1 & 2.6 & 2.0 \\
Argentina & Trade & 15.0 & 22.6 & 35.0 & 25.0 \\
 & FDI: Net Inflows & 1.3 & 3.7 & 2.7 & 1.9 \\
 & FDI: Net Outflows & 0.0 & 0.3 & 0.2 & 0.2 \\
\end{tabular}
\end{center}

\begin{center}
\begin{tabular}{llcccc}
\hline
Country & \multicolumn{1}{c}{Type of Flow} & $\mathbf{1 9 9 0}$ & $\mathbf{2 0 0 0}$ & $\mathbf{2 0 1 0}$ & $\mathbf{2 0 1 7}$ \\
\hline
Germany & Trade & 46.0 & 61.4 & 79.3 & 86.9 \\
 & FDI: Net Inflows & 0.2 & 12.7 & 2.5 & 2.1 \\
 & FDI: Net Outflows & 1.4 & 5.0 & 4.3 & 3.4 \\
India & Trade & 15.7 & 27.2 & 49.7 & 40.6 \\
 & FDI: Net Inflows & 0.1 & 0.8 & 1.7 & 1.5 \\
United States & 0.0 & 0.1 & 1.0 & 0.4 &  \\
 & FDI: Net Outflows & 19.8 & 25.0 & 28.2 & $26.6^{*}$ \\
 & Trade & 0.8 & 3.4 & 1.7 & 1.8 \\
FDI: Net Inflows & 1.0 & 1.8 & 2.3 & 2.2 &  \\
\hline
FDI: Net Outflows &  &  &  &  &  \\
\hline
\end{tabular}
\end{center}

\begin{itemize}
  \item Trade figures for 2016.
\end{itemize}

Source: World Development Indicators.

The increasing importance of multinational corporations is also apparent in Exhibit 5. Net FDI inflows and outflows increased as a percentage of GDP between 1990 and 2000 for each of the countries shown. Trade between multinational firms and their subsidiaries (i.e., intra-firm trade) has become an important part of world trade. For example, 46 percent of US imports occur between related parties (Bernard, Jensen, Redding, and Schott 2010). Globalization of production has increased the productive efficiency of manufacturing firms because they are able to decompose their value chain into individual components or parts, and then outsource their production to different locations where these components can be produced most efficiently. ${ }^{4}$ For example, in 2016 Apple's iPhone 6s was manufactured with components sourced from several locations around the world: the camera, display and storage were manufactured in Japan; the RAM and A9 processor were manufactured in Korea; the modem, battery, Wi-Fi module, radio frequency transceiver and chassis were manufactured in China; and much of the hardware and software was designed in the United States while the phone itself was assembled in China. ${ }^{5}$ Foreign direct investment and outsourcing have increased business investment in these economies and provided smaller and less developed economies the opportunity to participate in international trade. For example, in 2016 Intel had 10 fabrication plants and 101 assembly and testing sites in 8 countries/regions. These trends indicate the increasing global interdependence of economies, although the degree of interdependence varies. Greater interdependence also means that economies are now more exposed to global competition. As a result, they must be more flexible in their production structure in order to respond effectively to changes in global demand and supply.

The complexity of trading relationships has also increased with the development of sophisticated global supply chains that include not only final goods but also intermediate goods and services. Increased global interdependence has changed the risk and return profiles of many economies. Economies that have greater international links are more exposed to, and affected by, economic downturns and crises occurring in other parts of the world. The contagion effect of the Asian financial crisis, which began in Thailand in July 1997, spread to many other markets, such as Indonesia, Malaysia, South Korea, Philippines, Hong Kong SAR, Singapore, and Taiwan Region. It even affected

4 Hill and Hult (2019) explains the idea of the firm as a value chain: "The operations of the firms can be thought of as a value chain composed of a series of distinct value creation activities including production, marketing and sales, materials management, R\&D, human resources, information systems, and firm infrastructure." Production itself can be broken down into distinct components and each component outsourced separately.

5 "Here's where all the components of your iPhone come from" Skye Gould and Antonio Villas-Boas Apr. 12, 2016 \href{https://www.businessinsider.com/where-iphone-parts-come-from-2016-4}{https://www.businessinsider.com/where-iphone-parts-come-from-2016-4} Brazil and Russia to some degree, although there is less clarity about the mechanisms by which the crisis spread beyond Asia. Among the outward symptoms of the crisis were exchange rate problems, such as currency speculation and large depreciation of currencies, capital flight, and financial and industrial sector bankruptcies. However, recovery was surprisingly swift and all these economies exhibited positive growth by the second quarter of 1999 (Gerber 2017).

\section{BENEFITS AND COSTS OF INTERNATIONAL TRADE}
describe benefits and costs of international trade

The preceding sections have described the growth of world trade and the increasing interdependence of national economies. Has trade been beneficial? The benefits and costs of international trade have been widely debated. The most compelling arguments supporting international trade are: countries gain from exchange and specialization, industries experience greater economies of scale, households and firms have greater product variety, competition is increased, and resources are allocated more efficiently.

Gains from exchange occur when trade enables each country to receive a higher price for its exports (and greater profit) and/or pay a lower price for imported goods instead of producing these goods domestically at a higher cost (i.e., less efficiently). This exchange, in turn, leads to a more efficient allocation of resources by increasing production of the export good and reducing production of the import good in each country (trading partner). This efficiency allows consumption of a larger bundle of goods, thus increasing overall welfare. The fact that trade increases overall welfare does not, of course, mean that every individual consumer and producer is better off. What it does mean is that the winners could, in theory, compensate the losers and still be better off.

Trade also leads to greater efficiency by fostering specialization based on comparative advantage. Traditional trade models, such as the Ricardian model and the Heckscher-Ohlin model, focus on specialization and trade according to comparative advantage arising from differences in technology and factor endowments, respectively. These models will be discussed in the next section.

Newer models of trade focus on the gains from trade that result from economies of scale, greater product variety, and increased competition. In an open economy, increased competition from foreign firms reduces the monopoly power of domestic firms and forces them to become more efficient, as compared to a closed economy. Industries that exhibit increasing returns to scale (for example, the automobile and steel industries) benefit from increased market size as a country starts trading because the average cost of production declines as output increases in these industries. Monopolistically competitive models of trade have been used to explain why there is significant two-way trade (known as intra-industry trade) between countries within the same industry. Intra-industry trade occurs when a country exports and imports goods in the same product category or classification.

In a monopolistically competitive industry, there are many firms; each firm produces a unique or differentiated product, there are no exit or entry barriers, and long-run economic profits are zero. In such a model, even though countries may be similar, they gain from trade because each country focuses on the production and export of one or more varieties of the good and imports other varieties of the good. For example, the European Union exports and imports different types of cars. Consumers gain from having access to a greater variety of final goods. Firms benefit from greater economies of scale because firms both within and outside the EU are able to sell their goods in both markets. Hence, scale economies allow firms to benefit from the larger market size and experience lower average cost of production as a result of trade.

Research suggests that trade liberalization can lead to increased real (that is, inflation-adjusted) GDP although the strength of this relationship is still debated. The positive influence of trade on GDP can arise from more efficient allocation of resources, learning by doing, higher productivity, knowledge spillovers, and trade-induced changes in policies and institutions that affect the incentives for innovation. ${ }^{6}$ In industries where there is "learning by doing," such as the semiconductor industry, the cost of production per unit declines as output increases because of expertise and experience acquired in the process of production. Trade can lead to increased exchange of ideas, freer flow of technical expertise, and greater awareness of changing consumer tastes and preferences in global markets. It can also contribute to the development of higher quality and more effective institutions and policies that encourage domestic innovation. For example, Coe and Helpman (1995) show that foreign research and development (R\&D) has beneficial effects on domestic productivity. These effects become stronger the more open an economy is to foreign trade. They estimate that about a quarter of the benefits of R\&D investment in a G-7 country accrues to their trading partners. ${ }^{7}$ Hill (2007) discusses the case of Logitech, a Swiss company that manufactures computer mice. In order to win original equipment manufacturer (OEM) contracts from IBM and Apple, Logitech needed to develop innovative designs and provide high-volume production at a low cost. So in the late 1980s they moved to Taiwan Region, which had a highly qualified labor force, competent parts suppliers, a rapidly expanding local computer industry, and offered Logitech space in a science park at a very competitive rate. Soon thereafter, Logitech was able to secure the Apple contract.

Opponents of free trade point to the potential for greater income inequality and the loss of jobs in developed countries as a result of import competition. As a country moves toward free trade, there will be adjustments in domestic industries that are exporters as well as those that face import competition. Resources (investments) may need to be reallocated into or out of an industry depending on whether that industry is expanding (exporters) or contracting (i.e., facing import competition). As a result of this adjustment process, less-efficient firms may be forced to exit the industry, which may, in turn, lead to higher unemployment and the need for displaced workers to be retrained for jobs in expanding industries. The counter argument is that although there may be short-term and even some medium-term costs, these resources are likely to be more effectively (re-)employed in other industries in the long run. Nonetheless, the adjustment process is virtually certain to impose costs on some groups of stakeholders.

\section{EXAMPLE 1}
\section{Benefits of Trade}
Consider two countries that each produce two goods. Suppose the cost of producing cotton relative to lumber is lower in Cottonland than in Lumberland.

6 "Knowledge spillovers" occur when investments in knowledge creation generate benefits that extend beyond the investing entity and facilitate learning and innovation by other firms or entities.

7 G-7 countries include Canada, France, Germany, Italy, Japan, the United Kingdom, and the United States. 1. How would trade between the two countries affect the lumber industry in Lumberland?

\section{Solution to 1:}
The lumber industry in Lumberland would benefit from trade. Because the cost of producing lumber relative to producing cotton is lower in Lumberland than in Cottonland (i.e., lumber is relatively cheap in Lumberland), Lumberland will export lumber and the industry will expand.

\begin{enumerate}
  \setcounter{enumi}{1}
  \item How would trade between the two countries affect the lumber industry in Cottonland?
\end{enumerate}

\section{Solution to 2:}
The lumber industry in Cottonland would not benefit from trade, at least in the short run. Because lumber is relatively expensive to produce in Cottonland, the domestic lumber industry will shrink as lumber is imported from Lumberland.

\begin{enumerate}
  \setcounter{enumi}{2}
  \item What would happen to the lumber industry workers in Cottonland in the long run?
\end{enumerate}

\section{Solution to 3:}
The overall welfare effect in both countries is positive. However, in the short run, many lumber producers in Cottonland (and cotton producers in Lumberland) are likely to find themselves without jobs as the lumber industry in Cottonland and the cotton industry in Lumberland contract. Those with skills that are also needed in the other industry may find jobs fairly quickly. Others are likely to do so after some re-training. In the long run, displaced workers should be able to find jobs in the expanding export industry. However, those who remain in the import-competing industry may be permanently worse off because their industry-specific skills are now less valuable. Thus, even in the long run, trade does not necessarily make every stakeholder better off. But the winners could compensate the losers and still be better off, so the overall welfare effect of opening trade is positive.

\begin{enumerate}
  \setcounter{enumi}{3}
  \item What is the meaning of the expression "gains from trade"?
\end{enumerate}

\section{Solution to 4:}
Gains from trade imply that the overall benefits of trade outweigh the losses from trade. It does not mean that all stakeholders (producers, consumers, government) benefit (or benefit equally) from trade.

\begin{enumerate}
  \setcounter{enumi}{4}
  \item What are some of the benefits from trade?
\end{enumerate}

\section{Solution to 5:}
Some of the benefits from trade include: gains from exchange and specialization based on relative cost advantage; gains from economies of scale as the companies add new markets for their products; greater variety of products available to households and firms; greater efficiency from increased competition; and more efficient allocation of resources.

\begin{center}
\includegraphics[max width=\textwidth]{2023_05_04_ccb578c5089fd41615beg-426}
\end{center}

\section{COMPARATIVE ADVANTAGE AND THE GAINS FROM TRADE: ABSOLUTE AND COMPARATIVE ADVANTAGE}
contrast comparative advantage and absolute advantage

Up to this point, we have not been precise about what it means for a country to have a comparative advantage in the production of specific goods and services. In this section, we define comparative advantage, distinguish it from the notion of absolute advantage, and demonstrate the gains from trading in accordance with comparative advantage. We then explain two traditional models of trade-the Ricardian and Heckscher-Ohlin models-and the source of comparative advantage in each model.

\section{Gains from Trade: Absolute and Comparative Advantage}
A country has an absolute advantage in producing a good (or service) if it is able to produce that good at a lower cost or use fewer resources in its production than its trading partner. For example, suppose a worker in Brazil can produce either 20 pens or 40 pencils in a day. A worker in Vietnam can produce either 10 pens or 60 pencils. A Vietnamese worker produces 60 pencils a day while a Brazilian worker produces only 40 pencils a day. Hence, Vietnam produces pencils at a lower cost than Brazil, and has an absolute advantage in the production of pencils. Similarly, Brazil produces pens at a lower cost than Vietnam, and hence has an absolute advantage in the production of pens. A country has a comparative advantage in producing a good if its opportunity cost of producing that good is less than that of its trading partner. In our example, the opportunity cost of producing an extra pen in Vietnam is 6 pencils. It is the opportunity foregone; namely, the number of pencils Vietnam would have to give up to produce an extra pen. If Brazil does not trade and has to produce both pens and pencils, it will have to give up 2 pencils in order to produce a pen. Similarly, in Vietnam each pen will cost 6 pencils. Hence, the opportunity cost of a pen in Brazil is 2 pencils, whereas in Vietnam it is 6 pencils. Brazil has the lower opportunity cost and thus a comparative advantage in the production of pens. Vietnam has a lower opportunity cost (1 pencil costs 1/6th of a pen) than Brazil (1 pencil costs $1 / 2$ a pen) in the production of pencils and thus has a comparative advantage in the production of pencils. Example 2 further illustrates these concepts.

\section{EXAMPLE 2}
\begin{abstract}
Absolute and Comparative Advantages Suppose there are only two countries, India and the United Kingdom. India exports cloth to the United Kingdom and imports machinery. The output per worker per day in each country is shown in Exhibit 6:
\end{abstract}

Exhibit 6: Output per Worker per Day

\begin{center}
\begin{tabular}{lcc}
\hline
 & Machinery & Cloth (yards) \\
\hline
United Kingdom & 4 & 8 \\
India & 2 & $\mathbf{1 6}$ \\
\hline
\end{tabular}
\end{center}

Based only on the information given, address the following:

\begin{enumerate}
  \item Which country has an absolute advantage in the production of:
A. machinery?
B. cloth?

  \item Do the countries identified in Question 1 as having an absolute advantage in the production of A) machinery and B) cloth, also have a comparative advantage in those areas?

\end{enumerate}

\section{Solution to 1A:}
The United Kingdom has an absolute advantage in the production of machinery because it produces more machinery per worker per day than India.

\section{Solution to $1 \mathrm{~B}$ :}
India has an absolute advantage in the production of cloth because it produces more cloth per worker per day than the United Kingdom.

\section{Solution to $2 A$ and $2 B$ :}
In both cases, the answer is "yes." In the case of machinery, the opportunity cost of a machine in the United Kingdom is 2 yards of cloth $(8 \div 4$ or 1 machine $=$ 2 yards cloth). This amount is the autarkic price of machines in terms of cloth in the United Kingdom. In India, the opportunity cost of a machine is 8 yards of cloth $(16 \div 2$ or 1 machine $=8$ yards cloth). Thus, the United Kingdom has a comparative advantage in producing machines. In contrast, the opportunity cost of a yard of cloth in the United Kingdom and in India is $1 / 2$ and $1 / 8$ of a machine, respectively. India has a lower opportunity cost (1/8 of a machine) and, therefore, a comparative advantage in the production of cloth.

It is important to note that even if a country does not have an absolute advantage in producing any of the goods, it can still gain from trade by exporting the goods in which it has a comparative advantage. In Example 2, if India could produce only 6 yards of cloth per day instead of 16 yards of cloth, the United Kingdom would have an absolute advantage in both machines and cloth. However, India would still have a comparative advantage in the production of cloth because the opportunity cost of a yard of cloth in India, 1/3 of a machine in this case, would still be less than the opportunity cost of a yard of cloth in the United Kingdom (1/2 of a machine as before).

Let us now illustrate the gains from trading according to comparative advantage. In Example 2, if the United Kingdom could sell a machine for more than 2 yards of cloth and if India could purchase a machine for less than 8 yards of cloth, both countries would gain from trade. Although it is not possible to determine the exact world price without additional details regarding demand and supply conditions, both countries would gain from trade as long as the world price for machinery in terms of cloth is between the autarkic prices of the trading partners. In our example, this price corresponds to a price of between 2 and 8 yards of cloth for a machine. The further away the world price of a good or service is from its autarkic price in a given country, the more that country gains from trade. For example, if the United Kingdom was able to sell a machine to India for 7 yards of cloth (i.e., closer to India's autarkic price), it would gain 5 yards of cloth per machine sold to India compared with its own autarkic price (with no trade) of 1 machine for 2 yards of cloth. However, if the United Kingdom was able to sell a machine to India for only 3 yards of cloth (closer to the UK autarkic price), it would gain only 1 yard of cloth per machine sold to India compared with its own autarkic price. Exhibit 7 and Exhibit 8 provide the production and consumption schedules of both countries at autarky and after trade has commenced. In autarky (Exhibit 7), the United Kingdom produces and consumes 200 machines and 400 yards of cloth (without trade, consumption of each product must equal domestic production). Similarly, India produces 100 machines and 800 yards of cloth in autarky. In a world economy consisting of only these two countries, total output for each commodity is the sum of production in both countries. Therefore, total world output is 300 machines and 1,200 yards of cloth.

\section{Exhibit 7: Production and Consumption in Autarky}
\begin{center}
\begin{tabular}{lcc}
\hline
 & Autarkic Production & Autarkic Consumption \\
\hline
United Kingdom &  &  \\
Machinery (m) & 200 & 200 \\
Cloth (yards) (c) & 400 & 400 \\
India & 100 & 100 \\
Machinery & 800 & 800 \\
Cloth (yards) &  &  \\
Total World: & 300 & 300 \\
Machinery & 1200 & 1200 \\
Cloth (yards) &  &  \\
\hline
\end{tabular}
\end{center}

Now, assume that the United Kingdom and India start trading and that the world price of 1 machine is 4 yards of cloth $(1 \mathrm{~m}=4 \mathrm{c})$. This price is within the range of acceptable world trading prices discussed earlier because this price lies between the autarkic prices of the United Kingdom $(1 \mathrm{~m}=2 \mathrm{c})$ and India $(1 \mathrm{~m}=8 \mathrm{c})$. Exhibit 8 shows that in an open economy, the United Kingdom would specialize in machines and India would specialize in cloth. As a result, the United Kingdom produces 400 machines and no cloth, while India produces 1,600 yards of cloth and no machines. The United Kingdom exports 160 machines to India in exchange for 640 yards of cloth. After trade begins with India, the United Kingdom consumes 240 machines and 640 yards of cloth. Consumption in the United Kingdom increases by 40 machines and 240 yards of cloth. Similarly, India consumes 160 machines and 960 yards of cloth, an increase of 60 machines and 160 yards of cloth. World production and consumption is now 400 machines and 1,600 yards of cloth. Post-trade production and consumption exceeds the autarkic situation by 100 machines and 400 yards of cloth.

Exhibit 8: Gains from Trade

\begin{center}
\begin{tabular}{lccc}
\hline
 & $\begin{array}{c}\text { Post-trade } \\ \text { Production }\end{array}$ & $\begin{array}{c}\text { Post-trade } \\ \text { Consumption }\end{array}$ & $\begin{array}{c}\text { Change in Consumption } \\ \text { (compared with autarky) }\end{array}$ \\
\hline
UK & 400 & 240 & +40 \\
Machinery & 0 & 640 & +240 \\
Cloth (yards) & 0 & +60 &  \\
India & 0 & 160 & +160 \\
\end{tabular}
\end{center}

Total World:

\begin{center}
\begin{tabular}{lccc}
\hline
 & $\begin{array}{c}\text { Post-trade } \\ \text { Production }\end{array}$ & $\begin{array}{c}\text { Post-trade } \\ \text { Consumption }\end{array}$ & $\begin{array}{c}\text { Change in Consumption } \\ \text { (compared with autarky) }\end{array}$ \\
\hline
Machinery & 400 & 400 & +100 \\
Cloth (yards) & 1600 & 1600 & +400 \\
\hline
\end{tabular}
\end{center}

Exhibit 9 shows a more general case of gains from trade under increasing costs. In Panel A, the curve connecting the $X$ and $Y$ axes is the UK production possibilities frontier (PPF). ${ }^{8}$ That is, it represents the combinations of cloth and machinery that the United Kingdom can produce given its technology and resources (capital and labor). The slope of the PPF at any point is the opportunity cost of one good in terms of the other. The shape of the PPF indicates increasing opportunity cost in terms of machines as more cloth is produced and vice versa. To maximize the value of output, production occurs where the slope of the PPF equals the relative price of the goods. $P_{A}$ represents the autarkic price line, which is tangent to the PPF at $A$, the autarkic equilibrium. The slope of the autarkic price line represents the opportunity cost before trade. In autarky, the United Kingdom produces and consumes 60 machines and 60 thousand yards cloth, and is on indifference curve $I .{ }^{9}$ When the United Kingdom starts trading with India, it faces the world price line $P^{*}$. This new price line is tangent to the PPF at $B$. The change in relative prices of the goods encourages the United Kingdom to increase the production of the good in which it has comparative advantage (machines) and produce at $B$ instead of $A$. We note that at $B$ the United Kingdom has increased the production of machines to 120 units and reduced the production of cloth to 30 thousand yards. We also note that trade has expanded the UK consumption possibilities. The United Kingdom consumes at point $E$ after trade, exports 80 machines to India and imports 80 thousand yards of cloth from India. Note that $E$ is outside the PPF, but on the world price line that is tangent to the PPF at $B$. This line is also the trading possibilities line because trade occurs along this line. The slope of this line is the opportunity cost of a machine in terms of cloth in the world market. The United Kingdom has clearly increased its welfare through trade because it is able to consume at point $E$, which is on a higher indifference curve (III) and thus represents a higher level of welfare compared with the autarkic consumption point $A$ on indifference curve $I$.

Panel B shows the corresponding situation for India. When trade opens with the United Kingdom, India shifts production from $A^{\prime}$ to $B^{\prime}$, producing more cloth, the good in which it has a comparative advantage, and fewer machines. It now exports 80 thousand yards of cloth to the United Kingdom and imports 80 machines from the United Kingdom. India now consumes at $E^{\prime}$ which is on the world price line and also on a higher indifference curve, III', than the autarkic consumption point $\left(A^{\prime}\right)$ on indifference curve I'. Thus, by specializing (incompletely, as is typically the case with increasing production costs) in the good in which it has a comparative advantage, each country increases its welfare. We should also note that $P^{*}$ is the price at which trade is balanced. At this relative world price, the export of cloth from India equals the import of cloth into the United Kingdom (80 thousand yards) and the export of machines from the United Kingdom equals the imports of machines into India (80 machines).

8 Modified from Salvatore (2011).

9 An indifference curve represents the various combinations of goods (machines and cloth) that provide the same level of utility or welfare. Higher indifference curves represent higher levels of utility or welfare. Exhibit 9: Graphical Depiction of Gains from Trade with Increasing Costs

Panel A. United Kingdom

\begin{center}
\includegraphics[max width=\textwidth]{2023_05_04_ccb578c5089fd41615beg-430}
\end{center}

\section{Panel B. India}
\begin{center}
\includegraphics[max width=\textwidth]{2023_05_04_ccb578c5089fd41615beg-430(1)}
\end{center}

A country's comparative advantage can change over time as a result of structural shifts in its domestic economy, shifts in the global economy, the accumulation of physical or human capital, new technology, the discovery of such natural resources as oil, and so on. For example, an increase in skilled labor in China has led several multinational companies to establish R\&D facilities in China to benefit from its highly educated workforce.

\section{EXAMPLE 3}
\section{Changes in Comparative Advantage}
Exhibit 10 shows how the South Korea's comparative advantage changed over time as a result of an export-oriented development strategy it adopted during the 1960s. 10 The challenges of foreign competition created a "virtuous circle" that was self-reinforcing. South Korea's changing comparative advantage was the result of government policy, an increasingly skilled and productive workforce, and proactive firms that learned and adapted new technology.

Exhibit 10: Changes in Structure of South Korea's Exports, 1980-2015 (Percentage Shares)

\begin{center}
\begin{tabular}{|c|c|c|c|c|c|c|c|c|}
\hline
 & 1980 & 1985 & 1990 & 1995 & 2000 & 2005 & 2010 & 2015 \\
\hline
Agricultural products & $6.8 \%$ & $4.0 \%$ & $3.8 \%$ & $3.2 \%$ & $2.3 \%$ & $1.8 \%$ & $2.0 \%$ & $2.0 \%$ \\
\hline
Fuels and mining products & $1.0 \%$ & $3.1 \%$ & $1.6 \%$ & $2.7 \%$ & $6.1 \%$ & $6.9 \%$ & $8.9 \%$ & $8.1 \%$ \\
\hline
Manufactures & $69.3 \%$ & $74.2 \%$ & $76.9 \%$ & $81.8 \%$ & $82.2 \%$ & $86.9 \%$ & $86.5 \%$ & $87.5 \%$ \\
\hline
Chemicals & $3.3 \%$ &  & $3.2 \%$ & $6.4 \%$ & $7.3 \%$ & $9.3 \%$ & $10.3 \%$ & $10.9 \%$ \\
\hline
Pharmaceuticals &  &  & $0.1 \%$ & $0.2 \%$ & $0.2 \%$ & $0.2 \%$ & $0.3 \%$ & $0.4 \%$ \\
\hline
Machinery and transport equipment & $15.7 \%$ &  & $32.4 \%$ & $46.9 \%$ & $53.2 \%$ & $58.4 \%$ & $55.5 \%$ & $57.7 \%$ \\
\hline
Office and telecom equipment & $7.5 \%$ & $9.9 \%$ & $18.2 \%$ & $23.8 \%$ & $31.1 \%$ & $27.9 \%$ & $20.3 \%$ & $20.5 \%$ \\
\hline
$\begin{array}{l}\text { Integrated circuits \& electronic } \\ \text { components }\end{array}$ &  &  & $6.8 \%$ & $13.9 \%$ & $13.1 \%$ & $9.3 \%$ & $9.1 \%$ & $10.7 \%$ \\
\hline
Textiles & $9.8 \%$ & $6.8 \%$ & $7.7 \%$ & $8.8 \%$ & $6.7 \%$ & $3.5 \%$ & $2.3 \%$ & $2.0 \%$ \\
\hline
Clothing & $13.1 \%$ & $11.9 \%$ & $10.0 \%$ & $3.5 \%$ & $2.7 \%$ & $0.9 \%$ & $0.3 \%$ & $0.4 \%$ \\
\hline
\end{tabular}
\end{center}

Source: World Trade Organisation, WTO Statistics Database \href{http://stat.wto.org/Home/WSDBHome}{stat.wto.org/Home/WSDBHome}. aspx.

\begin{enumerate}
  \item How has South Korea's structure of exports changed over time?
\end{enumerate}

\section{Solution to 1:}
In 1980, agriculture and clothing accounted for 6.8 percent and 13 percent of South Korea's exports, respectively. By 2015, the corresponding figures were 2.0 percent and 0.4 percent. In contrast, by 2015 machinery and transport equipment were almost $60 \%$ of South Korea's merchandise exports from only about $16 \%$ in 1980 . Manufactures as a whole were $87.5 \%$, up from $69 \%$ in 1980.

\begin{enumerate}
  \setcounter{enumi}{1}
  \item How did increased foreign competition impact the economy?
\end{enumerate}

\section{Solution to 2:}
The challenges of foreign competition created a "virtuous circle" that was self-reinforcing. Success in export markets increased the confidence of South Korean firms and led to greater success in exports through increased productivity, higher-quality products, acquisition of new skills, and adoption of technologies.

10 Wikipedia: Trade Policy of South Korea. In 1962, South Korea first introduced an export promotion policy targeted at labor intensive industries like textiles and clothing. By the 1970s, this plan had shifted focus to heavy industries and chemicals as the main export targets. As South Korea developed in the 1980s and early 1990s, export policies shifted toward consumer products, electronics, and high tech in particular. 3. What were the factors that helped to change South Korea's comparative advantage?

Solution to 3:

The factors that helped change South Korea's comparative advantage included government policy, an increasingly skilled and productive workforce, and proactive firms that learned and adapted new technology.

From an investment perspective, it is critical for analysts to be able to examine a country's comparative and absolute advantages and to analyze changes in them. It is also important to understand changes in government policy and regulations, demographics, human capital, demand conditions, and other factors that may influence comparative advantage and production and trade patterns. This information can then be used to identify sectors, industries within those sectors, and companies within those industries that will benefit.

\section{RICARDIAN AND HECKSCHER-OHLIN MODELS OF COMPARATIVE ADVANTAGE}
compare the Ricardian and Heckscher-Ohlin models of trade and the source(s) of comparative advantage in each model

A discussion of absolute and comparative advantage and the gains from specialization would be incomplete without a discussion of two important theories of trade, the Ricardian Model and the Heckscher-Ohlin Model. These models are based on cross-country differences in technology and in factor endowments, respectively. These theoretical models are based on several assumptions, some of which may not be fully satisfied in the real world; nonetheless they provide extremely useful insights into the determinants and patterns of trade.

Adam Smith argued that a country could gain from trade if it had an absolute advantage in the production of a good. David Ricardo extended Smith's idea of the gains from trade by arguing that even if a country did not have an absolute advantage in the production of any good, it could still gain from trade if it had a comparative advantage in the production of a good. In the Ricardian model, labor is the only (variable) factor of production. Differences in labor productivity, reflecting underlying differences in technology, are the source of comparative advantage and hence the key driver of trade in this model. A country with a lower opportunity cost in the production of a good has a comparative advantage in that good and will specialize in its production. In our two-country model, if countries vary in size, the smaller country may specialize completely, but may not be able to meet the total demand for the product. Hence, the larger country may be incompletely specialized, producing and exporting the good in which it has a comparative advantage but still producing (and consuming) some of the good in which it has a comparative disadvantage. It is important to recognize that although differences in technology may be a major source of comparative advantage at a given point in time, other countries can close the technology gap or even gain a technological advantage. The shift of information technology services from developed countries to India is an example of comparative advantage shifting over time. ${ }^{11}$ This shift was facilitated by India's growing pool of highly skilled and relatively low-wage labor, the development and growth of its telecommunication infrastructure, and government policies that liberalized trade in the 1990s.

In the Heckscher-Ohlin Model (also known as the factor-proportions theory), both capital and labor are variable factors of production. That is, each good can be produced with varying combinations of labor and capital. According to this model, differences in the relative endowment of these factors are the source of a country's comparative advantage. This model assumes that technology in each industry is the same among countries, but it varies between industries. According to the theory, a country has a comparative advantage in goods whose production is intensive in the factor with which it is relatively abundantly endowed, and would tend to specialize in and export that good. Capital is relatively more (less) abundant in a country if the ratio of its endowment of capital to labor is greater (less) than that of its trading partner. ${ }^{12}$ This scenario means a country in which labor is relatively abundant would export relatively labor-intensive goods and import relatively capital-intensive goods. For example, because the manufacture of textiles and clothing is relatively labor intensive, they are exported by such countries as China and India where labor is relatively abundant.

Relative factor intensities in production can be illustrated with the following example. In 2002, capital per worker in the Canadian paper industry was $C \$ 118,777$, whereas in the clothing manufacturing sector it was $C \$ 8,954 .^{13}$ These amounts indicate that manufacturing paper is more capital intensive than clothing production. Canada trades with Thailand and, being relatively capital abundant compared with Thailand, it exports relatively capital-intensive paper to Thailand and imports relatively labor-intensive clothing from Thailand.

Because the Heckscher-Ohlin model has two factors of production, labor and capital, (unlike the Ricardian model that has only labor), it allows for the possibility of income redistribution through trade. The demand for an input is referred to as a derived demand because it is derived from the demand for the product it is used to produce. As a country opens up to trade, it has a favorable impact on the abundant factor, and a negative impact on the scarce factor. This result is because trade causes output prices to change; more specifically, the price of the export good increases and the price of the import good declines. These price changes affect the demand for factors used to produce the import and export goods, and hence affect the incomes received by each factor of production.

To illustrate this point, consider again the opening of trade between the United Kingdom and India in Exhibit 9. When trade opened, the United Kingdom expanded production of machines-which are assumed to be the capital-intensive industryand reduced production of clothing. India did the opposite. Machines became more expensive relative to clothing in the United Kingdom (line $P^{*}$ is steeper than line $P^{A}$ ). The relative price change, along with the shift in output it induces, leads to a redistribution of income from labor to capital in the United Kingdom. The opposite occurs in India-machines become cheaper relative to clothing (line $P^{*}$ is flatter than $P^{A^{\prime}}$ ), production shifts toward clothing, and income is redistributed from capital to labor.

Note that in each country, the relatively cheap good and the relatively cheap factor of production both get more expensive when trade is opened. That raises an interesting question: If free trade equalizes the prices of goods among countries, does it also

11 According to NASSCOM (India's prominent IT-BPO trade association), Indian firms offer a wide range of information technology services that include consulting, systems integration, IT outsourcing/managed services/hosting services, training, and support/maintenance. See \href{http://www.nasscom.in}{www.nasscom.in}.

12 Alternatively, factor abundance can be defined in terms of the relative factor prices that prevail in autarky. Under this definition, labor is more (less) abundant in a country if the cost of labor relative to the cost of capital is lower (higher) in that country.

13 Appleyard, Field, and Cobb (2010). equalize the prices of the factors of production? In the simple Heckscher-Ohlin world of homogeneous products, homogeneous inputs, and identical technologies among countries, the answer is yes: The absolute and relative factor prices are equalized in both countries if there is free trade. In the real world, we see that factor prices do not converge completely even if there is free trade because several assumptions of the models are not fully satisfied in the real world. Nonetheless, it is important to note that with international trade factor prices display a tendency to move closer together in the long run.

Changes in factor endowments can cause changes in the patterns of trade and can create profitable investment opportunities. For example, in 1967 Japan had a comparative advantage in unskilled-labor-intensive goods, such as textiles, apparel, and leather. Meier (1998) notes that by 1980, Japan had greatly increased its skilled labor and consequently had a comparative advantage in skill-intensive products, especially non-electrical machinery.

It is important to note that technological differences, as emphasized in the Ricardian trade model, and differences in factor abundance, as emphasized in the HeckscherOhlin model, are both important drivers of trade. They are complementary, not mutually exclusive. Tastes and preferences can also vary among countries and can change over time, leading to changes in trade patterns and trade flows.

\section{TRADE AND CAPITAL FLOWS: RESTRICTIONS \& AGREEMENTS- TARIFFS, QUOTAS AND EXPORT SUBSIDIES}
compare types of trade and capital restrictions and their economic
implications

Trade restrictions (or trade protection) are government policies that limit the ability of domestic households and firms to trade freely with other countries. Examples of trade restrictions include tariffs, import quotas, voluntary export restraints (VER), subsidies, embargoes, and domestic content requirements. Tariffs are taxes that a government levies on imported goods. Quotas restrict the quantity of a good that can be imported into a country, generally for a specified period of time. A voluntary export restraint is similar to a quota but is imposed by the exporting country. An export subsidy is paid by the government to the firm when it exports a unit of a good that is being subsidized. The goal here is to promote exports, but it reduces welfare by encouraging production and trade that is inconsistent with comparative advantage. Domestic content provisions stipulate that some percentage of the value added or components used in production should be of domestic origin. Trade restrictions are imposed by countries for several reasons including protecting established domestic industries from foreign competition, protecting new industries from foreign competition until they mature (infant industry argument), protecting and increasing domestic employment, protecting strategic industries for national security reasons, generating revenues from tariffs (especially for developing countries), and retaliation against trade restrictions imposed by other countries. Capital restrictions are defined as controls placed on foreigners' ability to own domestic assets and/or domestic residents' ability to own foreign assets. Thus, in contrast with trade restrictions, which limit the openness of goods markets, capital restrictions limit the openness of financial markets.

\section{Tariffs}
Tariffs are taxes that a government levies on imported goods. ${ }^{14}$ The primary objective of tariffs is to protect domestic industries that produce the same or similar goods. They may also aim to reduce a trade deficit. Tariffs reduce the demand for imported goods by increasing their price above the free trade price. The economic impact of a tariff on imports in a small country is illustrated in Exhibit 11. In this context, a small country is not necessarily small in size, population, or GDP. Instead, a small country is one that is a price taker in the world market for a product and cannot influence the world market price. For example, by many measures Brazil is a large country, but it is a price taker in the world market for cars. A large country, however, is a large importer of the product and can exercise some influence on price in the world market. When a large country imposes a tariff, the exporter reduces the price of the good to retain some of the market share it could lose if it did not lower its price. This reduction in price alters the terms of trade and represents a redistribution of income from the exporting country to the importing country. So, in theory it is possible for a large country to increase its welfare by imposing a tariff if 1 ) its trading partner does not retaliate and 2) the deadweight loss as a result of the tariff (see below) is smaller than the benefit of improving its terms of trade. However, there would still be a net reduction in global welfare-the large country cannot gain by imposing a tariff unless it imposes an even larger loss on its trading partner.

In Exhibit 11, the world price (free trade price) is $P^{*}$. Under free trade, domestic supply is $Q^{1}$, domestic consumption is $Q^{4}$, and imports are $Q^{1} Q^{4}$. After the imposition of a per-unit tariff $t$, the domestic price increases to $P_{t}$, which is the sum of the world price and the per-unit tariff $t$. At the new domestic price, domestic production increases to $Q^{2}$ and domestic consumption declines to $Q^{3}$, resulting in a reduction in imports to $Q^{2} Q^{3}$.

The welfare effects can be summarized as follows:

\begin{itemize}
  \item Consumers suffer a loss of consumer surplus because of the increase in price. ${ }^{15}$ This effect is represented by areas $\mathrm{A}+\mathrm{B}+\mathrm{C}+\mathrm{D}$ in Exhibit 11 .

  \item Local producers gain producer surplus from a higher price for their output. This effect is represented by area A.

  \item The government gains tariff revenue on imports $Q^{2} Q^{3}$. This effect is represented by area C.

\end{itemize}

The net welfare effect is the sum of these three effects. The loss in consumer surplus is greater than the sum of the gain in producer surplus and government revenue and results in a deadweight loss to the country's welfare of $B+D$.

Welfare Effects of an Import Tariff or Quota

\begin{center}
\begin{tabular}{lc}
\hline
 & Importing Country \\
\hline
Consumer surplus & $-(\mathrm{A}+\mathrm{B}+\mathrm{C}+\mathrm{D})$ \\
Producer surplus & $+\mathrm{A}$ \\
\end{tabular}
\end{center}

14 Governments may also impose taxes on exports, although they are less common

15 Consumer surplus, producer surplus, and deadweight loss are defined and discussed in the prerequisite reading "Demand and Supply Analysis: Introduction" available online in your Candidate Resources.

\section{Welfare Effects of an Import Tariff or Quota}
\section{Importing Country}
Tariff revenue or Quota rents

$$
+\mathrm{C}
$$

National welfare

$-B-D$

Tariffs create deadweight loss because they give rise to inefficiencies on both the consumption and production side. B represents inefficiencies in production. Instead of being able to import goods at the world price $P^{*}$, tariffs encourage inefficient producers whose cost of production is greater than $P^{*}$ to enter (or remain in) the market, leading to an inefficient allocation of resources. On the consumption side, tariffs prevent mutually beneficial exchanges from occurring because consumers who were willing to pay more than $P^{*}$ but less than $P_{t}$ are now unable to consume the good.

\section{Exhibit 11: Welfare Effects of Tariff and Import Quota}
Price

\begin{center}
\includegraphics[max width=\textwidth]{2023_05_04_ccb578c5089fd41615beg-436}
\end{center}

\section{EXAMPLE 4}
\section{Analysis of a Tariff}
South Africa manufactures 110,000 tons of paper. However, domestic demand for paper is 200,000 tons. The world price for paper is $\$ 5$ per ton. South Africa will import 90,000 tons of paper from the world market at free trade prices. If the South African government (a small country) decides to impose a tariff of 20 percent on paper imports, the price of imported paper will increase to $\$ 6$. Domestic production after the imposition of the tariff increases to 130,000 tons, while the quantity demanded declines to 170,000 tons. 1. Calculate the loss in consumer surplus arising from the imposition of the tariff.

\section{Solution to 1:}
The loss in consumer surplus $=\$ 1 \times 170,000+1 / 2 \times \$ 1 \times 30,000=\$ 185,000$. This calculation is represented by areas $\mathrm{A}+\mathrm{B}+\mathrm{C}+\mathrm{D}$ in Exhibit 11 .

\begin{enumerate}
  \setcounter{enumi}{1}
  \item Calculate the gain in producer surplus arising from the imposition of the tariff.
\end{enumerate}

\section{Solution to 2:}
Gain in producer surplus $=\$ 1 \times 110,000+1 / 2 \times(\$ 1 \times 20,000)=\$ 120,000$; Area A in Exhibit 11.

\begin{enumerate}
  \setcounter{enumi}{2}
  \item Calculate the gain in government revenue arising from the imposition of the tariff.
\end{enumerate}

\section{Solution to 3:}
Change in government revenue $=\$ 1 \times 40,000=\$ 40,000$; Area $C$ in Exhibit 11.

\begin{enumerate}
  \setcounter{enumi}{3}
  \item Calculate the deadweight loss arising from the imposition of the tariff.
\end{enumerate}

\section{Solution to 4:}
Deadweight loss because of the tariff $=1 / 2 \times \$ 1 \times 20,000+1 / 2 \times \$ 1 \times 30,000$ $=\$ 25,000$; Areas B + D in Exhibit 11.

\section{Quotas}
A quota restricts the quantity of a good that can be imported into a country, generally for a specified period of time. An import license specifies the quantity that can be imported. For example, the European Union operates a system of annual import quotas for steel producers who are not members of the World Trade Organization. The 2010 quota was 0.2 million tons a year for Kazakhstan. In the case of Russia, the 2010 quota of 3.2 million tons per year was a part of an EU-Russia agreement. ${ }^{16} \mathrm{~A}$ key difference between tariffs and quotas is that the government is able to collect the revenue generated from a tariff. This effect is uncertain under a quota. With quotas, foreign producers can often raise the price of their goods and earn greater profits than they would without the quota. These profits are called quota rents. In Exhibit 11, if the quota is $Q^{2} Q^{3}$, the equivalent tariff that will restrict imports to $Q^{2} Q^{3}$ is $t$ and the domestic price after the quota is $P_{t}$. This is the same as the domestic price after the tariff $t$ was imposed. Area $C$, however, is now the quota rent or profits that are likely to be captured by the foreign producer rather than tariff revenue that is captured by the domestic government. If the foreign producer or foreign government captures the quota rent, $C$, then the welfare loss to the importing country, represented by areas $B+D+C$ in Exhibit 11, under a quota is greater than under the equivalent tariff. If the government of the country that imposes the quota can capture the quota rents by auctioning the import licenses for a fee, then the welfare loss under the quota is similar to that of a tariff, represented by areas B $+\mathrm{D}$.

16 For more information, see \href{http://ec.europa.eu/trade/creating-opportunities/economic-sectors/}{http://ec.europa.eu/trade/creating-opportunities/economic-sectors/} industrial-goods/steel/. A voluntary export restraint (VER) is a trade barrier under which the exporting country agrees to limit its exports of the good to its trading partners to a specific number of units. The main difference between an import quota and a VER is that the former is imposed by the importer, whereas the latter is imposed by the exporter. The VER allows the quota rent resulting from the decrease in trade to be captured by the exporter (or exporting country), whereas in the case of an import quota there is ambiguity regarding who captures the quota rents. Hence, a VER results in welfare loss in the importing country. For example, in 1981 the Japanese government imposed VERs on automobile exports to the United States.

\section{Export Subsidies}
An export subsidy is a payment by the government to a firm for each unit of a good that is exported. Its goal is to stimulate exports. But it interferes with the functioning of the free market and may distort trade away from comparative advantage. Hence, it reduces welfare. Countervailing duties are duties that are levied by the importing country against subsidized exports entering the country. As an example, agricultural subsidies in developed countries, notably the EU, have been a contentious issue in trade negotiations with less-developed countries and developed countries that are agricultural exporters, such as New Zealand and Australia.

In the case of an export subsidy, the exporter has the incentive to shift sales from the domestic to the export market because it receives the international price plus the per-unit subsidy for each unit of the good exported. This scenario raises the price in the domestic market by the amount of the subsidy in the small country case (price before subsidy plus subsidy). In the large country case, the world price declines as the large country increases exports. The net welfare effect is negative in both the large and small country cases, with a larger decline in the large country case. This result is because in the large country case, the decline in world prices implies that a part of the subsidy is transferred to the foreign country, unlike in the small country case.

Exhibit 12 summarizes some of these effects.

\section{Exhibit 12}
Panel A. Effects of Alternative Trade Policies

\begin{center}
\begin{tabular}{lllll}
\hline
 & Tariff & Import Quota & Export Subsidy & VER \\
\hline
Impact on & Importing country & Importing country & Exporting country & Importing country \\
Producer surplus & Increases & Increases & Increases & Increases \\
Consumer surplus & Decreases & Decreases & Decreases & Decreases \\
Government revenue & Increases & Mixed (depends on & Falls (government & No change (rent to \\
 &  & whether the quota & spending rises) & foreigners) \\
 &  & rents are captured by &  &  \\
 & the importing country &  &  &  \\
 & through sale of licenses &  &  &  \\
 & or by the exporters) &  &  &  \\
 &  &  &  &  \\
\end{tabular}
\end{center}

Panel A. Effects of Alternative Trade Policies

\begin{center}
\begin{tabular}{lllll}
\hline
 & Tariff & Import Quota & Export Subsidy & VER \\
\hline
National welfare & Decreases in small & Decreases in small & Decreases &  \\
 & country & country &  &  \\
 & Could increase in & Could increase in large &  &  \\
 & large country & country &  &  \\
\hline
\end{tabular}
\end{center}

Panel B. Effects of Alternative Trade Policies on Price, Production, Consumption, and Trade

\begin{center}
\begin{tabular}{lllll}
\hline
 & Tariff & Import Quota & Export Subsidy & VER \\
\hline
Impact on & Importing country & Importing country & Exporting country & Importing country \\
Price & Increases & Increases & Increases & Increases \\
Domestic consumption & Decreases & Decreases & Decreases & Decreases \\
Domestic production & Increases & Increases & Increases & Increases \\
Trade & Imports decrease & Imports decrease & Exports increase & Imports decrease \\
\hline
\end{tabular}
\end{center}

\section{EXAMPLE 5}
\section{Tariffs, Quotas, and VERs}
Thailand, a small country, has to decide whether to impose a tariff or a quota on the import of computers. You are considering investing in a local firm that is a major importer of computers.

\begin{enumerate}
  \item What will be the impact of a tariff on prices, quantity produced, and quantity imported in Thailand (the importing country)?
\end{enumerate}

\section{Solution to 1:}
A tariff imposed by a small country, such as Thailand, raises the price of computers in the importing country, reduces the quantity imported, and increases domestic production.

\begin{enumerate}
  \setcounter{enumi}{1}
  \item If Thailand imposes a tariff, what will the impact be on prices in the exporting country?
\end{enumerate}

\section{Solution to 2:}
A tariff imposed by a small country would not change the price of computers in the exporting country.

\begin{enumerate}
  \setcounter{enumi}{2}
  \item How would a tariff affect consumer surplus, producer surplus, and government revenue in Thailand?
\end{enumerate}

\section{Solution to 3:}
When a small country imposes a tariff, it reduces consumer surplus, increases producer surplus, and increases government revenue in that country. 4. Explain whether the net welfare effect of a tariff is the same as that of a quota.

Solution to 4:

The quota can lead to a greater welfare loss than a tariff if the quota rents are captured by the foreign government or foreign firms.

\begin{enumerate}
  \setcounter{enumi}{4}
  \item Which policy, a tariff or a quota, would be most beneficial to the local importer in which you may invest and why?
\end{enumerate}

\section{Solution to 5:}
A tariff will hurt importers because it will reduce their share of the computer market in Thailand. The impact of a quota depends on whether the importers can capture a share of the quota rents. Assuming importers can capture at least part of the rents, they will be better off with a quota.

\begin{enumerate}
  \setcounter{enumi}{5}
  \item If Thailand were to negotiate a VER with the countries from which it imports computers, would this be better or worse than an import quota for the local importing firm in which you may invest? Why?
\end{enumerate}

\section{Solution to 6:}
The VER would not be better for the local importer than the import quota and would most likely be worse. Under the VER, all of the quota rents will be captured by the exporting countries whereas with an import quota at least part of the quota rents may be captured by local importers.

It is important to understand existing trade policies and the potential for policy changes that may impact return on investment. Changes in the government's trade policy can affect the pattern and value of trade and may result in changes in industry structure. These changes may have important implications for firm profitability and growth because they can affect the goods a firm can import/export, change demand for its products, impact its pricing policies, and create delays through increased paperwork, procurement of licenses, approvals, and so on. For example, changes in import policies that affect the ability of a firm to import vital inputs for production may increase the cost of production and reduce firm profitability.

\section{TRADING BLOCS, COMMON MARKETS, AND ECONOMIC UNIONS}
explain motivations for and advantages of trading blocs, common markets, and economic unions

There has been a proliferation of trading blocs or regional trading agreements (RTA) in recent years. Important examples of regional integration include the North American Free Trade Agreement (NAFTA) and the European Union (EU). A regional trading bloc is a group of countries that have signed an agreement to reduce and progressively eliminate barriers to trade and movement of factors of production among the members of the bloc. It may or may not have common trade barriers against countries that are not members of the bloc. There are many different types of regional trading blocs, depending on the level of integration that takes place. Free trade areas (FTA) are one of the most prevalent forms of regional integration in which all barriers to the flow of goods and services among members have been eliminated. However, each country maintains its own polices against non-members. The North American Free Trade Agreement (NAFTA) among the United States, Canada, and Mexico is an example of a FTA. A customs union extends the FTA by not only allowing free movement of goods and services among members but also creating a common trade policy against non-members. In 1947, Belgium, the Netherlands, and Luxemburg ("Benelux") formed a customs union that became a part of the European Community in 1958. The common market is the next level of economic integration that incorporates all aspects of the customs union and extends it by allowing free movement of factors of production among members. The Southern Cone Common Market (MERCOSUR) of Argentina, Brazil, Paraguay, and Uruguay is an example of a common market. ${ }^{17}$ An economic union requires an even greater degree of integration. It incorporates all aspects of a common market and in addition requires common economic institutions and coordination of economic policies among members. The European Community became the European Union in 1993. If the members of the economic union decide to adopt a common currency, then it is also a monetary union. For example, with the adoption of the euro, $19 \mathrm{EU}$ member countries also formed a monetary union. ${ }^{18}$

\section{EXAMPLE 6}
\section{Trading Blocs}
\begin{enumerate}
  \item Chile and Australia have a free trade with each other but have separate trade barriers on imports from other countries. Chile and Australia are a part of $\mathrm{a}(\mathrm{n})$
A. FTA.
B. Economic union.
C. Customs union.
D. Common market.
\end{enumerate}

\section{Solution to 1:}
A is correct. Chile and Australia do not a have customs union because they do not have a common trade policy with respect to other trade partners ( $\mathrm{C}$ is incorrect). A common market or a economic union entail even more integration (B and $\mathrm{D}$ are incorrect).

\begin{enumerate}
  \setcounter{enumi}{1}
  \item An RTA that removes all tariffs on imports from member countries, has common external tariffs against all non-members, but does not advance further in deepening economic integration is called $\mathrm{a}(\mathrm{n})$
\end{enumerate}

A. FTA.

17 For more information, visit the OECD website, \href{http://stats.oecd.org/glossary/}{http://stats.oecd.org/glossary/}.

18 On 1 January 1999, Austria, Belgium, Finland, France, Germany, Ireland, Italy, Luxembourg, the Netherlands, Portugal, and Spain adopted the euro. This adoption meant that these countries had to surrender control over their domestic monetary policy to the European Central Bank. Greece joined in 2001. Euro coins and notes went into circulation on 1 January 2002, and these countries gave up the last vestiges of their national currencies. Other members now include Slovenia (2007), Cyprus (2008), Malta (2008), Slovakia (2009), Estonia (2011), Latvia (2014), and Lithuania (2015). The eurozone (i.e., the monetary union) is only a subset of the EU membership because some EU members, notably the United Kingdom, have not adopted the euro. B. Economic union.

C. Customs union.

D. Common market.

\section{Solution to 2:}
$\mathrm{C}$ is correct. A basic FTA does not entail common external tariffs (A is incorrect), whereas a common market and an economic union entail integration beyond common external tariffs (B and D are incorrect).

Regional integration is popular because eliminating trade and investment barriers among a small group of countries is easier, politically less contentious, and quicker than multilateral trade negotiations under the World Trade Organization (WTO). The WTO is a negotiating forum that deals with the rules of global trade between nations and where member countries can go to sort out trade disputes. The latest rounds of trade negotiations launched by the WTO in 2001 at Doha, Qatar, included several contentious issues of specific concern to developing countries, such as the cost of implementing trade policy reform in developing countries, market access in developed countries for developing countries' agricultural products, and access to affordable pharmaceuticals in developing countries. After nearly a decade of negotiations, very limited progress has been made on the major issues. Hence, it is not surprising to see a renewed interest in bilateral and multilateral trade liberalization on a smaller scale. Policy coordination and harmonization are also easier among a smaller group of countries. Regional integration can be viewed as a movement toward freer trade.

Regional integration results in preferential treatment for members compared with non-members and can lead to changes in the patterns of trade. Member countries move toward freer trade by eliminating or reducing trade barriers against each other, leading to a more efficient allocation of resources. But regional integration may also result in trade and production being shifted from a lower-cost non-member who still faces trade barriers to a higher-cost member who faces no trade barriers. This shift leads to a less-efficient allocation of resources and could reduce welfare. Hence, there are two static effects that are direct results of the formation of the customs union: trade creation and trade diversion.

Trade creation occurs when regional integration results in the replacement of higher-cost domestic production by lower-cost imports from other members. For example, consider two hypothetical countries, Qualor and Vulcan. Qualor produces 10 million shirts annually and imports 2 million shirts from Vulcan, which has a lower cost of production. Qualor has 10 percent tariffs on imports from Vulcan. Qualor and Vulcan then agree to form a customs union. Qualor reduces its production of shirts to 7 million and now imports 11 million shirts from Vulcan. The decline in Qualor's domestic production (from 10 million to 7 million shirts) is replaced by importing 3 million additional shirts from the low-cost producer, Vulcan. This scenario represents trade creation. The rest of the additional imports (6 million shirts) represent increased consumption by Qualor's consumers because the price of shirts declines after formation of the custom union.

Trade diversion occurs when lower-cost imports from nonmember countries are replaced with higher-cost imports from members. In the example in the preceding paragraph, suppose Qualor initially imposes a 10 percent tariff on imports from both Vulcan and Aurelia. Aurelia is the lowest-cost producer of shirts, so Qualor initially imports 2 million shirts from Aurelia instead of from Vulcan. Qualor and Vulcan then form a customs union, which eliminates tariffs on imports from Vulcan but maintains a 10 percent tariff on imports from Aurelia. Now trade diversion could occur if the free trade price on imports from Vulcan is lower than the price on imports from Aurelia. Even though Aurelia is the lowest-cost producer, it may be a higher-priced source of imports because of the tariff. If this is the case, then Qualor will stop importing from Aurelia, a non-member, and divert its imports to Vulcan, a member of the RTA. Both trade creation and trade diversion are possible in an RTA. If trade creation is larger than trade diversion, then the net welfare effect is positive. However, there are concerns that this may not always be the case.

The benefits ascribed to free trade-greater specialization according to comparative advantage, reduction in monopoly power because of foreign competition, economies of scale from larger market size, learning by doing, technology transfer, knowledge spillovers, greater foreign investment, and better quality intermediate inputs at world prices-also apply to regional trading blocs. In addition, fostering greater interdependence among members of the regional trading bloc reduces the potential for conflict. Members of the bloc also have greater bargaining power and political clout in the global economy by acting together instead of as individual countries.

The 2009 World Development Report points to spillover of growth across borders as one of the main benefits of regional integration (Collier and O'Connell 2007). There is evidence of considerable spillovers among OECD countries, which are highly integrated both as a group and within their own geographic regions. The long-run growth of integrated countries is interconnected because members have greater access to each other's markets. Strong growth in any RTA country could have a positive impact on growth in other RTA member countries. RTAs also enhance the benefits of good policy and lead to convergence in living standards. For example, growth spillovers are likely to be much smaller among Sub-Saharan African countries because of a lack of integration arising from deficiencies in RTAs and inadequate levels of transportation and telecommunications infrastructure. Roberts and Deichmann (2008) estimated what the cumulative loss in real GDP between 1970 and 2000 would have been if Switzerland, which is landlocked and fully integrated with both its immediate neighbors and the world economy, had been subject to the same level of spillovers as the Central African Republic. Under such a scenario, Switzerland's GDP per capita in 2000 would have been 9.3 percent lower. The cumulative GDP loss would have been $\$ 334$ billion (constant US dollars, 2000), which was the equivalent of 162 percent of Switzerland's real GDP in 2000.

Although regional integration has many advantages, it may impose costs on some groups. For example, there was significant concern in the United States that NAFTA and especially low-skilled-labor intensive imports from Mexico could hurt low-skilled workers. Adjustment costs arose as import competition caused inefficient firms to exit the market, and the workers in those firms were at least temporarily unemployed as they sought new jobs. However, the surviving firms experienced an increase in productivity, and US consumers benefited from the increase in product varieties imported from Mexico. Feenstra and Taylor (2008) estimated that the product varieties exported from Mexico to the United States had grown by an average of 2.2 percent a year across all industries. They estimated that NAFTA imposed private costs of nearly $\$ 5.4$ billion a year in the United States during 1994-2002, but that these costs were offset by an average welfare gain of $\$ 5.5$ billion a year accruing from increased varieties imported from Mexico. Consumer gains from more varieties of products continued over time as long as the imports continued, while adjustment costs arising from job losses declined over time. In 2003, the gain from increased product varieties from Mexico was $\$ 11$ billion, far exceeding the adjustment costs of $\$ 5.4$ billion. ${ }^{19}$ Their analysis concluded:

19 Feenstra and Taylor (2008) discuss in their book the data limitations and various assumptions they made in their analysis. ...Thus the consumer gains from increased product variety, when summed over the years, considerably exceed the private loss from displacement. This outcome is guaranteed to occur because the gains from expanded import varieties occur every year that the imports are available, whereas labor displacement is a temporary phenomenon. (Feenstra and Taylor 2008, p. 208)

It is important to recognize, however, that workers displaced by regional integration may have to bear long-term losses if they are unable to find jobs with wages comparable with the jobs they lost or they remain unemployed for a long period. For example, although import competition was certainly not the only factor that led to a dramatic contraction of the US automobile industry, the impact on employment in that industry is likely to be permanent and many former autoworkers, especially older workers, may never find comparable jobs.

Concerns regarding national sovereignty, especially where big and small nations may be part of the same bloc, have also been an impediment to the formation of FTAs. The proposal for a South Asian regional bloc has faced challenges regarding India's role because it is one of the biggest economies in the region.

Regional integration is important from an investment perspective because it offers new opportunities for trade and investment. The cost of doing business in a large, single, regional market is lower and firms can benefit from economies of scale. However, it is important to note that differences in tastes, culture, and competitive conditions still exist among members of a trading bloc. These differences may limit the potential benefits from investments within the bloc. In addition, depending on the level of integration and the safeguards in place, problems faced by individual member countries in an RTA may quickly spread to other countries in the bloc.

There are at least two challenges in the formation of an RTA and in its potential progression from a free trade area to deeper integration in the form of a customs union, common market, or economic union. First, cultural differences and historical considerations-for example, wars and conflicts-may complicate the social/political process of integration. Second, maintaining a high degree of economic integration limits the extent to which member countries can pursue independent economic and social policies. Free trade and mobility of labor and capital tend to thwart policies aimed at controlling relative prices and/or quantities within a country, while balance of payments and fiscal credibility considerations limit the viability of divergent macroeconomic policies. This situation is especially true in the case of a monetary union because monetary policy is not under the control of individual countries and currency devaluation/revaluation is not available as a tool to correct persistent imbalances. ${ }^{20}$ When persistent imbalances do arise, they may lead to a crisis that spills over to other countries facing similar problems. A recent example is the fear of contagion caused by the Greek fiscal crisis in 2010. In May 2010, Standard \& Poor's reduced the credit ratings on Greece's government from investment grade to junk status. It also downgraded the government debt of Spain and Portugal. These countries were suffering from a combination of high government deficits and slow GDP growth. The credit downgrades increased fears that Greece, in particular, would default on its debt and cause economic turmoil not only among the healthier countries in the EU but also in the United States and Asia. The EU and the International Monetary Fund (IMF) agreed on a USD145 billion (EUR110 billion) bailout for Greece in May 2010, and provided Ireland with a financing package of about USD113 billion (EUR85 billion) in November 2010. As of late 2010, there were continuing concerns about the financial

20 These limitations are inherent in any system with fixed exchange rates and a high degree of capital mobility. They are not unique to a monetary union (i.e., a common currency). For a discussion of currency regimes, see the Level I curriculum reading on Currency Exchange Rates. health of Greece, Ireland, Portugal, and Spain. The EU, which created the European Financial Stability Facility (EFSF) in 2010 to help EU countries in need, has been debating the need for an expansion in the scope and financing capacity of the EFSF.

\section{EXAMPLE 7}
\section{Trade Agreements}
Bagopia, Cropland, and Technopia decide to enter into an RTA. In the first stage, they decide to sign a free trade agreement (FTA). After several successful years, they decide that it is time to form a common market.

\begin{enumerate}
  \item Does an FTA make exporting firms in member countries more attractive as investment options?
\end{enumerate}

\section{Solution to 1:}
The first stage, where there is free movement of goods and services among RTA members, is called a free trade area. It makes exporting firms a more attractive investment proposition because they are able to serve markets in member countries without the additional costs imposed by trade barriers.

\begin{enumerate}
  \setcounter{enumi}{1}
  \item How does the common market affect firms doing business in these countries compared with an FTA?
\end{enumerate}

\section{Solution to 2:}
Unlike an FTA, a common market allows for free movement of factors of production, such as labor and capital, among the member economies. Like an FTA, it provides access to a much larger market and free movement of goods and services. But the common market can create more profitable opportunities for firms than an FTA by allowing them to locate production in and purchase components from anywhere in the common market according to comparative advantage.

\section{CAPITAL RESTRICTIONS}
$$
\begin{aligned}
& \text { describe common objectives of capital restrictions imposed by } \\
& \text { governments }
\end{aligned}
$$

There are many reasons for governments to restrict inward and outward flow of capital. For example, the government may want to meet some objective regarding employment or regional development, or it may have a strategic or defense-related objective. Many countries require approval for foreigners to invest in their country and for citizens to invest abroad. Control over inward investment by foreigners results in restrictions on how much can be invested, and on the type of industries in which capital can be invested. For example, such strategic industries as defense and telecommunications are often subject to ownership restrictions. Outflow restrictions can include restrictions on repatriation of capital, interest, profits, royalty payments, and license fees. Citizens are often limited in their ability to invest abroad, especially in foreign exchange-scarce economies, and there can be deadlines for repatriation of income earned from any investments abroad.

Economists consider free movement of financial capital to be beneficial because it allows capital to be invested where it will earn the highest return. Inflows of capital also allow countries to invest in productive capacity at a rate that is higher than could be achieved with domestic savings alone, and it can enable countries to achieve a higher rate of growth. Longer-term investments by foreign firms that establish a presence in the local economy can bring in not only much needed capital but also new technology, skills, and advanced production and management practices as well as create spillover benefits for local firms. Investment by foreign firms can also create a network of local suppliers if they source some of their components locally. Such suppliers may receive advanced training and spillover benefits from a close working relationship with the foreign firms. On the one hand, increased competition from foreign firms in the market may force domestic firms to become more efficient. On the other hand, it is possible that the domestic industry may be hurt because domestic firms that are unable to compete are forced to exit the market.

In times of macroeconomic crisis, capital mobility can result in capital flight out of the country, especially if most of the inflow reflects short-term portfolio flows into stocks, bonds, and other liquid assets rather than foreign direct investment (FDI) in productive assets. In such circumstances, capital restrictions are often used in conjunction with other policy instruments, such as fixed exchange rate targets. Capital restrictions and fixed exchange rate targets are complementary instruments because in a regime of perfect capital mobility, governments cannot achieve domestic and external policy objectives simultaneously using only standard monetary and fiscal policy tools. ${ }^{21}$ By limiting the free flow of capital, capital controls provide a way to exercise control over a country's external balance whereas more traditional macro-policy tools are used to address other objectives.

Modern capital controls were developed by the belligerents in World War I as a method to finance the war effort. At the start of the war, all major powers restricted capital outflows (i.e., the purchase of foreign assets or loans abroad). These restrictions raised revenues by keeping capital in the domestic economy, facilitating the taxation of wealth, and producing interest income. Moreover, capital controls helped to maintain a low level of interest rates, reducing the government's borrowing costs on its liabilities. Since WWI, controls on capital outflows have been used similarly in other countries, mostly developing nations, to generate revenue for governments or to permit them to allocate credit in the domestic economy without risking capital flight. In broad terms, a capital restriction is any policy designed to limit or redirect capital flows. Such restrictions may take the form of taxes, price or quantity controls, or outright prohibitions on international trade in assets. Price controls may take the form of special taxes on returns to international investment, taxes on certain types of transactions, or mandatory reserve requirements-that is, a requirement forcing foreign parties wishing to deposit money in a domestic bank account to deposit some percentage of the inflow with the central bank for a minimum period at zero interest. Quantity restrictions on capital flows may include rules imposing ceilings or requiring special authorization for new or existing borrowing from foreign creditors. Or there may be administrative controls on cross-border capital movements in which a government agency must approve transactions for certain types of assets.

Effective implementation of capital restrictions may entail non-trivial administration costs, particularly if the measures have to be broadened to close potential loopholes. There is also the risk that protecting the domestic financial markets by capital

21 Section 4.1 of the Level I curriculum reading on Currency Exchange Rates provides a concise discussion of the policy implications of capital mobility with fixed versus floating exchange rates. restrictions may postpone necessary policy adjustments or impede private-sector adaptation to changing international circumstances. Most importantly, controls may give rise to negative market perceptions, which may, in turn, make it more costly and difficult for the country to access foreign funds.

In a study on the effectiveness of capital controls, the International Monetary Fund considered restrictions on capital outflows and inflows separately. ${ }^{22}$ The authors concluded that for restrictions on capital inflows to be effective (i.e., not circumvented), the coverage needs to be comprehensive and the controls need to be implemented forcefully. Considerable administrative costs are incurred in continuously extending, amending, and monitoring compliance with the regulations. Although controls on inflows appeared to be effective in some countries, it was difficult to distinguish the impact of the controls from the impact of other policies, such as strengthening of prudential regulations, increased exchange rate flexibility, and adjustment of monetary policy. In the case of capital outflows, the imposition of controls during episodes of financial crisis seems to have produced mixed results, providing only temporary relief of varying duration to some countries, while successfully shielding others (e.g., Malaysia) and providing them with sufficient time to restructure their economies.

\section{EXAMPLE 8}
\section{Historical Example-Capital Restrictions: Malaysia's Capital Controls in 1998-2001}
After the devaluation of the Thai baht in July 1997, Southeast Asia suffered from significant capital outflows that led to falling local equity and real estate prices and declining exchange rates. To counter the outflows of capital, the IMF urged many of the countries in the region to increase interest rates, thus making their assets more attractive to foreign investors. Higher interest rates, however, weighed heavily on the domestic economies. In response to this dilemma, Malaysia imposed capital controls on 1 September 1998. These controls prohibited transfers between domestic and foreign accounts, eliminated credit facilities to offshore parties, prevented repatriation of investment until 1 September 1999, and fixed the exchange rate of the Malaysian ringgit at 3.8 per US dollar. In February 1999, a system of taxes on capital flows replaced the prohibition on repatriation of capital. Although the details were complex, the net effect was to discourage short-term capital flows while permitting long-term transactions. By imposing capital controls, Malaysia hoped to regain monetary independence, and to be able to cut interest rates without provoking a fall in the value of its currency as investors avoided Malaysian assets. The imposition of outflow controls indeed curtailed speculative capital outflows and allowed interest rates to be reduced substantially. At the same time, under the umbrella of the capital controls, the authorities pursued bank and corporate restructuring and achieved a strong economic recovery in 1999 and 2000. With the restoration of economic and financial stability, administrative controls on portfolio outflows were replaced by a two-tier, price-based exit system in February 1999, which was finally eliminated in May 2001. Although Malaysia's capital controls did contribute to a stabilization of its economy, they came with long-term costs associated with the country's removal from the MSCI developed equity market index, an important benchmark in the institutional asset management industry, and its relegation to the emerging market universe. The Malaysian market was no longer seen as on par with developed equity markets whose institutional

22 Ariyoshi, et al. (2000). and regulatory frameworks provide a higher standard of safety for investors. As a consequence, it became more difficult for Malaysia to attract net long-term capital inflows (Kawai and Takagi 2003).

\begin{enumerate}
  \item Under what economic circumstances were Malaysia's capital restrictions imposed?
\end{enumerate}

\section{Solution to 1:}
As a result of the Southeast Asian crisis, Malaysia suffered substantial net capital outflows pushing up the domestic interest rate level.

\begin{enumerate}
  \setcounter{enumi}{1}
  \item What was the ultimate objective of Malaysia's capital restrictions?
\end{enumerate}

\section{Solution to 2:}
The restrictions were designed to limit and redirect capital flows to allow the government to reduce interest rates and pursue bank and corporate restructurings.

\begin{enumerate}
  \setcounter{enumi}{2}
  \item How successful were the country's capital restrictions?
\end{enumerate}

\section{Solution to 3:}
Although the capital controls helped stabilize Malaysia's economy, they contributed to a change in investors' perception of Malaysian financial markets and removal of the Malaysian equity market from the MSCI benchmark universe of developed equity markets. This situation undermined international demand for Malaysian equities and made it more difficult to attract net long-term capital inflows.

\section{BALANCE OF PAYMENTS- ACCOUNTS AND COMPONENTS}
$$
\begin{aligned}
& \text { describe the balance of payments accounts including their } \\
& \text { components }
\end{aligned}
$$

The balance of payments (BOP) is a double-entry bookkeeping system that summarizes a country's economic transactions with the rest of the world for a particular period of time, typically a calendar quarter or year. In this context, a transaction is defined as "an economic flow that reflects the creation, transformation, exchange, or extinction of economic value and involves changes in ownership of goods and/or financial assets, the provision of services, or the provision of labour and capital." ${ }^{23} \mathrm{In}$ other words, the BOP reflects payments for exports and imports as well as financial transactions and financial transfers. Analyzing the BOP is an important element in assessing a country's macroeconomic environment, its monetary and fiscal policies, and its long-term growth potential. Investors use data on trade and capital flows to

23 IMF Balance of Payments Handbook, chapter II, page 6. evaluate a country's overall level of capital investment, profitability, and risk. The following section describes the balance of payments, the factors that influence it, and its impact on exchange rates, interest rates, and capital market transactions.

\section{Balance of Payments Accounts}
The BOP is a double-entry system in which every transaction involves both a debit and credit. In principle, the sum of all debit entries should equal the sum of all credit entries, and the net balance of all entries on the BOP statement should equal zero. In practice, however, this is rarely the case because the data used to record balance of payments transactions are often derived from different sources.

Debit entries reflect purchases of imported goods and services, purchases of foreign financial assets, payments received for exports, and payments (interest and principal) received from debtors. Credit entries reflect payments for imported goods and services, payments for purchased foreign financial assets, and payments to creditors (see Exhibit 13, Panel A). Put differently, a debit represents an increase in a country's assets (the purchase of foreign assets or the receipt of cash from foreigners) or a decrease in its liabilities (the amount owed to foreigners); a credit represents a decrease in assets (the sale of goods and services to foreigners or the payment of cash to foreigners) or an increase in liabilities (an amount owed to foreigners).

For example, as shown in Panel B of Exhibit 13, on 1 September Country A purchases $\$ 1$ million of goods from Country $B$ and agrees to pay for these goods on 1 December. On 1 September, Country A would record in its BOP a \$1 million debit to reflect the value of the goods purchased (i.e., increase in assets) and $\$ 1$ million credit to reflect the amount owed to Country B. On 1 December, Country A would record in its BOP a $\$ 1$ million debit to reflect a decrease in the amount owed (liability) to Country B and $\$ 1$ million a credit to reflect the actual payment to Country B (decrease in assets).

From Country B's perspective, on 1 September it would record in its BOP a $\$ 1$ million debit to reflect the amount owed by Country A and a $\$ 1$ million credit to reflect the sale of goods (exports). On 1 December, Country B would record a $\$ 1$ million debit to reflect the cash received from Country $\mathrm{A}$, and $\$ 1$ million credit to reflect the fact that it is no longer owed $\$ 1$ million by Country $A$.

\section{Exhibit 13: Basic Entries in a BOP Context}
Panel A

DEBITS

CREDITS

Increase in Assets, Decrease in Liabilities

Decrease in Assets, Increase in Liabilities

\begin{itemize}
  \item Value of imported goods and services

  \item Purchases of foreign financial assets

  \item Receipt of payments from foreigners

  \item Increase in debt owed by foreigners

  \item Payment of debt owed to foreigners - Payments for imports of goods and services

  \item Payments for foreign financial assets

  \item Value of exported goods and services

  \item Payment of debt by foreigners

  \item Increase in debt owed to foreigners

\end{itemize}

Panel B

Country $A$

Debits

Credits

1 September

$\$ 1$ million

Goods purchased from Country B

(increase in real assets) $\$ 1$ million

Short-term liability for goods purchased from Country B (increase in financial liabilities)

\begin{center}
\begin{tabular}{|c|c|c|}
\hline
Country A & Debits & Credits \\
\hline
1 December & $\begin{array}{l}\text { \$1 million } \\ \text { Elimination of short-term liability for goods } \\ \text { purchased from Country B } \\ \text { (decrease in financial liabilities) }\end{array}$ & $\begin{array}{l}\text { \$1 million } \\ \text { Payment for goods purchased from Country B } \\ \text { (decrease in financial assets) }\end{array}$ \\
\hline
Country B & Debits & Credits \\
\hline
1 September & $\begin{array}{l}\$ 1 \text { million } \\ \text { Short-term claim for goods delivered to } \\ \text { Country A } \\ \text { (increase in financial assets) }\end{array}$ & $\begin{array}{l}\text { \$1 million } \\ \text { Goods delivered to Country A } \\ \text { (decrease in real assets) }\end{array}$ \\
\hline
1 December & $\begin{array}{l}\text { \$1 million } \\ \text { Receipt of payment for goods delivered to } \\ \text { Country A } \\ \text { (increase in financial assets) }\end{array}$ & $\begin{array}{l}\text { \$1 million } \\ \text { Elimination of claim for goods delivered to Country A } \\ \text { (decrease in financial assets) }\end{array}$ \\
\hline
\end{tabular}
\end{center}

\section{Balance of Payment Components}
The BOP is composed of the current account that measures the flow of goods and services, the capital account that measures transfers of capital, and the financial account that records investment flows. These accounts are further disaggregated into sub-accounts:

\section{Current Account}
The current account can be decomposed into four sub-accounts:

\begin{enumerate}
  \item Merchandise trade consists of all commodities and manufactured goods bought, sold, or given away.

  \item Services include tourism, transportation, engineering, and business services, such as legal services, management consulting, and accounting. Fees from patents and copyrights on new technology, software, books, and movies are also recorded in the services category.

  \item Income receipts include income derived from ownership of assets, such as dividends and interest payments; income on foreign investments is included in the current account because that income is compensation for services provided by foreign investments. When a German company builds a plant in China, for instance, the services the plant generates are viewed as a service export from Germany to China equal in value to the profits the plant yields for its German owner.

  \item Unilateral transfers represent one-way transfers of assets, such as worker remittances from abroad to their home country and foreign direct aid or gifts.

\end{enumerate}

\section{Capital Account}
The capital account consists of two sub-accounts:

\begin{enumerate}
  \item Capital transfers include debt forgiveness and migrants' transfers (goods and financial assets belonging to migrants as they leave or enter the country). ${ }^{24}$ Capital transfers also include the transfer of title to fixed assets and the transfer of funds linked to the sale or acquisition of fixed assets, gift and inheritance taxes, death duties, uninsured damage to fixed assets, and legacies.

  \item Sales and purchases of non-produced, non-financial assets, such as the rights to natural resources, and the sale and purchase of intangible assets, such as patents, copyrights, trademarks, franchises, and leases.

\end{enumerate}

\section{Financial Account}
The financial account can be broken down in two sub-accounts: financial assets abroad and foreign-owned financial assets within the reporting country.

\begin{enumerate}
  \item A country's assets abroad are further divided into official reserve assets, government assets, and private assets. These assets include gold, foreign currencies, foreign securities, the government's reserve position in the International Monetary Fund, ${ }^{25}$ direct foreign investment, and claims reported by resident banks.

  \item Foreign-owned assets in the reporting country are further divided into official assets and other foreign assets. These assets include securities issued by the reporting country's government and private sectors (e.g., bonds, equities, mortgage-backed securities), direct investment, and foreign liabilities reported by the reporting country's banking sector.

\end{enumerate}

\section{EXAMPLE 9}
\section{US Current Account Balance}
\begin{enumerate}
  \item Exhibit 14 shows a simplified version of the US balance of payments for 1970-2009.
\end{enumerate}

Exhibit 14: US International Transactions Accounts Data

\begin{center}
\begin{tabular}{|c|c|c|c|c|c|c|}
\hline
(Credits+, Debits-) & 1970 & 1980 & 1990 & 2000 & 2009 & 2017 \\
\hline
\multicolumn{7}{|l|}{Current Account} \\
\hline
$\begin{array}{l}\text { Exports of goods and services and } \\ \text { income receipts }\end{array}$ & 68,387 & 344,440 & 706,975 & $1,421,515$ & $2,159,000$ & $3,279,190$ \\
\hline
Exports of goods and services & 56,640 & 271,834 & 535,233 & $1,070,597$ & $1,570,797$ & $2,351,072$ \\
\hline
Income receipts & 11,747 & 72,606 & 171,742 & 350,918 & 588,203 & 928,118 \\
\hline
$\begin{array}{l}\text { Imports of goods and services and } \\ \text { income payments }\end{array}$ & $-59,901$ & $-333,774$ & $-759,290$ & $-1,779,241$ & $-2,412,489$ & $-3,609,734$ \\
\hline
Imports of goods and services & $-54,386$ & $-291,241$ & $-616,097$ & $-1,449,377$ & $-1,945,705$ & $-2,903,349$ \\
\hline
Income payments & $-5,515$ & $-42,532$ & $-143,192$ & $-329,864$ & $-466,783$ & $-706,385$ \\
\hline
\end{tabular}
\end{center}

24 Immigrants bring with them goods and financial assets already in their possession. Hence, these goods are imported on grounds other than commercial transactions.

25 These are in effect official currency reserves held with the International Monetary Fund.

\begin{center}
\begin{tabular}{|c|c|c|c|c|c|c|}
\hline
(Credits+, Debits-) & 1970 & 1980 & 1990 & 2000 & 2009 & 2017 \\
\hline
Unilateral current transfers, net & $-6,156$ & $-8,349$ & $-26,654$ & $-58,645$ & $-124,943$ & $-118,597$ \\
\hline
\multicolumn{7}{|l|}{Capital Account} \\
\hline
Capital account transactions, net & $\ldots$ & $\ldots$ & $-7,220$ & -1 & -140 & $-24,746$ \\
\hline
\multicolumn{7}{|l|}{Financial Account} \\
\hline
$\begin{array}{l}\text { US-owned assets abroad, ex deriva- } \\ \text { tives (increase/financial outflow }(-) \text { ) }\end{array}$ & $-9,337$ & $-86,967$ & $-81,234$ & $-560,523$ & $-140,465$ & $-1,182,749$ \\
\hline
$\begin{array}{l}\text { Foreign-owned assets in the United } \\ \text { States, ex derivatives (increase/finan- } \\ \text { cial inflow }(+) \text { ) }\end{array}$ & 7,226 & 62,037 & 139,357 & $1,038,224$ & 305,736 & $1,537,682$ \\
\hline
Financial derivatives, net & NA & NA & NA & NA & 50,804 & 23,074 \\
\hline
$\begin{array}{l}\text { Statistical discrepancy (sum of above } \\ \text { items with sign reversed) }\end{array}$ & -219 & 22,613 & 28,066 & $-61,329$ & 162,497 & $-95,880$ \\
\hline
\end{tabular}
\end{center}

Based only on the information given, address the following:

\begin{enumerate}
  \item Calculate the current account balance for each year.

  \item Calculate the financial account balance for each year.

  \item Describe the long-term change in the current account balance.

  \item Describe the long-term change in the financial account balance.

\end{enumerate}

Solutions to 1 and 2:

\begin{center}
\begin{tabular}{lcccccc}
\hline
(Credits+, Debits-) & $\mathbf{1 9 7 0}$ & $\mathbf{1 9 8 0}$ & $\mathbf{1 9 9 0}$ & $\mathbf{2 0 0 0}$ & $\mathbf{2 0 0 9}$ \\
\hline
Current Account & 2,330 & 2,317 & $-78,969$ & $-416,371$ & $-378,432$ \\
Financial Account & $-2,111$ & $-24,930$ & 58,123 & 477,701 & 216,075 \\
\hline
\end{tabular}
\end{center}

\section{Solution to 3:}
The United States had a current account surplus until 1980. After 1990, the US current account had an increasing deficit as a result of strong import growth.

\section{Solution to 4:}
Mirroring the growing US current account deficit, the US financial account, after 1990, registered increasing net capital inflows in similar proportions to the deficit in the current account. SYSTEM

describe the balance of payments accounts including their components The following examples illustrate how some typical cross-border transactions are recorded in the BOP framework outlined previously. They include commercial exports and imports, the receipt of income from foreign investments, loans made to borrowers abroad, and purchases of home-country currency by foreign central banks. Exhibit 15 illustrates the various individual bookkeeping entries from the perspective of an individual country, in this case Germany.

\section{Commercial Exports: Transactions (ia) and (ib)}
A company in Germany sells technology equipment to a South Korean auto manufacturer for a total price of EUR50 million, including freight charges of EUR1 million to be paid within 90 days. The merchandise will be shipped via a German cargo ship. In this case, Germany is exporting two assets: equipment and transportation services. The cargo shipped is viewed as being created in Germany and used by South Korean customers. In return for relinquishing these two assets, Germany acquires a financial asset-the promise by the South Korean manufacturer to pay for the equipment in 90 days.

Germany would record a EUR50 million debit to an account called "private short-term claims" to show an increase in this asset. It would also record a credit of EUR49 million to "goods" and another credit of EUR1 million to "services." Both credit entries are listed in the export category and show the decrease in assets available to German residents. These figures are entered as credits on lines 2 and 3 and as a debit on 19 in Exhibit 15 and are marked with (ia) to identify a typical commercial export transaction. To pay for the technology equipment purchased from Germany, the South Korean auto manufacturer may purchase euros from its local bank (i.e., a EUR demand deposit held by the Korean bank in a German bank) and then transfer them to the German exporter. As a result, German liabilities to South Korean residents (i.e., South Korean private short-term claims) would be debited. The respective entries, marked with (ib) are on lines 19 and 23 in Exhibit 15.

\section{Commercial Imports: Transaction (ii)}
A German utility company imports gas from Russia valued at EUR 45 million (ii), and agrees to pay the Russian company within three months. The imported gas generates a debit on line 6 . The obligation pay is recorded as a credit to foreign private short-term claims on line 23.

\section{Loans to Borrowers Abroad: Transaction (iii)}
A German commercial bank purchases EUR 100 million in intermediate-term bonds issued by a Ukrainian steel company. The bonds are denominated in euros, so payment is made in euros (i.e., by transferring EUR demand deposits). A debit entry on line 18 records the increase in German holdings of Ukrainian bonds, and a credit entry on line 23 records the increase in demand deposits held by Ukrainians in German banks.

Exhibit 15: Hypothetical Transactions between German Residents and Foreigners

\begin{center}
\begin{tabular}{lccc}
\hline
Item no & Account & Debit & Credit \\
\hline
$+/-$ &  &  &  \\
\hline
\end{tabular}
\end{center}

\begin{center}
\begin{tabular}{|c|c|c|c|c|}
\hline
Item no & Account & Debit & Credit & Balance \\
\hline
 &  & - & + & $+/-$ \\
\hline
2 & Goods &  & 49 (ia) & 49 \\
\hline
3 & Services &  & 1 (ia) & 1 \\
\hline
4 & Income on residents' investments abroad &  & $5(v)$ & 5 \\
\hline
5 & Imports of goods and services, income paid &  &  & -45 \\
\hline
6 & Goods & 45 (ii) &  & -45 \\
\hline
7 & Services &  &  &  \\
\hline
8 & Income on foreign investments in home country &  &  &  \\
\hline
9 & Unilateral transfers &  &  &  \\
\hline
10 & Changes in residents' claims on foreigners &  &  & -105 \\
\hline
11 & Official reserve assets &  &  &  \\
\hline
12 & Gold &  &  &  \\
\hline
13 & Foreign currency balances &  &  &  \\
\hline
14 & Other &  &  &  \\
\hline
15 & Government claims &  &  &  \\
\hline
16 & Private claims &  &  &  \\
\hline
17 & Direct investments &  &  &  \\
\hline
18 & Other private long-term claims & 100 (iii) &  & -100 \\
\hline
19 & Private short-term claims & $50(\mathrm{ia}), 5(\mathrm{v})$ & $50(\mathrm{ib})$ & -5 \\
\hline
20 & Changes in foreign claims on residents &  &  & 195 \\
\hline
21 & Foreign official claims &  & 20 (iv) & 20 \\
\hline
22 & Foreign private long-term claims &  &  &  \\
\hline
23 & Foreign private short-term claims & 20 (iv), 50 (ib) & $\begin{array}{c}45 \text { (ii), } 100 \text { (iii) } \\ 100(\mathrm{vi})\end{array}$ & 175 \\
\hline
24 & Other & $100(\mathrm{vi})$ &  & -100 \\
\hline
 & Total & 270370 & 270370 & 0 \\
\hline
 & Current Account: $(1)+(5)+(9)$ &  &  & 10 \\
\hline
 & Capital Account: (24) &  &  & -100 \\
\hline
 & Financial Account: $(10)+(20)$ &  &  & 90 \\
\hline
\end{tabular}
\end{center}

\section{Purchases of Home-Country Currency by Foreign Central Banks: Transaction (iv)}
Private foreigners may not wish to retain euro balances acquired in earlier transactions. Those who are holding foreign currency, in our example euro claims, typically do so for purposes of financing purchases from Germany (or other euro area member countries). Assume for instance, that Swiss residents attempt to sell EUR20 million in exchange for their native currency, the Swiss franc (CHF), but there is a lack of demand for EUR funds in Switzerland. In such circumstances, the CHF would appreciate against the EUR. To prevent an undesired CHF appreciation, the Swiss National Bank (SNB) might sell CHF in exchange for EUR balances.

Suppose that the Swiss National Bank purchased EUR20 million, typically in the form of a EUR demand deposit held with a German bank, from local commercial banks in Switzerland. The German BOP would register an increase of EUR20 million in German liabilities held by foreign monetary authorities, the Swiss National Bank (line 21), and an equivalent decline in short-term liabilities held by private foreigners (i.e., Swiss private investors, line 23). It may be noteworthy that when the SNB purchases EUR funds from Swiss commercial banks, it also credits them the CHF equivalent of EUR20 million. The SNB's liabilities to Swiss commercial banks arising from this transaction are in fact reserve deposits that Swiss banks can use when they expand their lending business and create new deposits. Currency interventions by central banks, therefore, can contribute to an increase in a country's overall money supply, all else remaining unchanged.

\section{Receipts of Income from Foreign Investments: Transaction (v)}
Each year, residents of Germany receive billions of EUR in interest and dividends from capital invested in foreign securities and other financial claims. German residents receive these payments in return for allowing foreigners to use German capital that otherwise could be put to work in Germany. Foreign residents, in turn, receive similar returns for their investments in Germany. Assume that a German firm has a long-term capital investment in a profitable subsidiary abroad, and that the subsidiary transfers to its German parent EUR5 million in dividends in the form of funds held in a foreign bank. The German firm then has a new (or increased) demand deposit in a foreign bank as compensation for allowing its capital to be used by its subsidiary. A debit entry on line 19 shows German private short-term claims on foreigners have increased by EUR5 million, and a credit entry on line 4 reflects the fact that German residents have given up an asset (the services of capital covered over the period) valued at EUR5 million.

\section{Purchase of Non-financial Assets: Transaction (vi)}
In a move to safeguard its long-term supply of uranium, a German utility company purchases the rights to exploit a uranium mine from the government of Kazakhstan. It agrees to pay within three months. The respective entries are on lines 23 and 24. Because a non-financial, non-produced asset is involved in this transaction, it is recorded in Germany's capital account.

Note that the sum of all BOP entries in Exhibit 15 is 0 . Transactions (i)-(iv) produce a current account surplus of EUR10 million, a capital account deficit of EUR100 million, and a financial account surplus of EUR90 million.

Although it is important to understand the detailed structure of official balance of payments accounts as described in the preceding paragraphs, this example is not necessarily how investment professionals think about the balance of payments day-to-day. Practitioners often think of the current account as roughly synonymous with the trade balance (merchandise trade + services) and lump all the financing flows (financial account + capital account) into one category that is usually referred to simply as the "capital account." They then think of the capital account as consisting of two types of flows-portfolio investment flows and foreign direct investment (FDI). The former are shorter-term investments in foreign assets (stocks, bonds, etc.), whereas the latter are long-term investments in production capacity abroad. Although not completely accurate, this way of thinking about the balance of payments focuses attention on the components-trade, portfolio flows, and FDI-that are most sensitive to, and most likely to affect, market conditions, prices of goods and services, asset prices, and exchange rates. In addition, this perspective fits well with the role that the balance of payments plays in the macroeconomy.

\section{NATIONAL ECONOMIC ACCOUNTS AND THE BALANCE OF PAYMENTS}
explain how decisions by consumers, firms, and governments affect the balance of payments

In a closed economy, all output $Y$ is consumed or invested by the private sectordomestic households and businesses-or purchased by the government. Letting $Y$ denote GDP, $C$ private consumption, $I$ investment, and $G$ government purchases of goods and services, the national income identity for a closed economy is given by:

$$
Y=C+I+G
$$

Once foreign trade is introduced, however, some output is purchased by foreigners (exports) whereas some domestic spending is used for purchases of foreign goods and services (imports). The national income identity for an open economy is thus

$$
Y=C+I+G+X-M
$$

where $X$ denotes exports and $M$ denotes imports.

For most countries, exports rarely equal imports. Net exports or the difference between exports and imports $(X-M)$ is the equivalent of the current account balance from a BOP perspective. ${ }^{26}$ When a country's imports exceed its exports, the current account is in deficit. When a country's exports exceed its imports, the current account is in surplus. As the right side of Equation 2 shows, a current account surplus or deficit can affect GDP (and also employment). The balance of the current account is also important because it measures the size and direction of international borrowing.

In order for the balance of payments to balance, a deficit or surplus in the current account must be offset by an opposite balance in the sum of the capital and financial accounts. This requirement means that a country with a current account deficit has to increase its net foreign debts by the amount of the current account deficit. For example, the United States has run current account deficits for many years while accumulating net foreign liabilities: The current account deficit was financed by net capital imports (i.e., direct investments by foreigners), loans by foreign banks, and the sale of US equities and fixed-income securities to foreign investors. By the same token, an economy with a current account surplus is earning more for its exports than it spends for its imports. Japan, Germany, and China are traditional current account surplus countries accumulating substantial net foreign claims, especially against the United States. An economy with a current account surplus finances the current account deficit of its trading partners by lending to them-that is, granting bank loans and investing in financial and real assets. As a result, the foreign wealth of a surplus country rises because foreigners pay for imports by issuing liabilities that they will eventually have to redeem.

By rearranging Equation 2, we can define the current account balance from the perspective of the national income accounts as:

$$
C A=X-M=Y-(C+I+G)
$$

26 Strictly speaking, net exports as defined here is the trade balance rather than the current account balance because it excludes income receipts and unilateral transfers. This distinction arises because we have defined income $Y$ as GDP rather than GNP (see section 2.1). Because the trade balance is usually the dominant component of the current account, the terms "trade balance" and "current account" are often used interchangeably. We will do so here unless the distinction is important to the discussion. Only by borrowing money from foreigners can a country have a current account deficit and consume more output than it produces. If it consumes less output than it produces, it has a current account surplus and can (indeed must) lend the surplus to foreigners. International capital flows essentially reflect an inter-temporal trade. An economy with a current account deficit is effectively importing present consumption and exporting future consumption.

Let us now turn to the relationship between output $Y$ and disposable income $Y^{d}$. We have to recognize that part of income is spent on taxes $T$, and that the private sector receives net transfers $R$ in addition to (national) income. Disposable income $Y^{d}$ is thus equal to income plus transfers minus taxes:

$$
Y^{d}=Y+R-T
$$

Disposable income, in turn, is allocated to consumption and saving so that we can write

$$
Y^{d}=C+S_{p}
$$

where $S_{p}$ denotes private sector saving. Combining Equation 4 and Equation 5 allows us to write consumption as income plus transfers minus taxes and saving.

$$
C=Y^{d}-S_{p}=Y+R-T-S_{p}
$$

We can now use the right side of Equation 6 to substitute for $C$ in Equation 3 . With some rearrangement we obtain

$C A=S_{p}-I+(T-G-R)$

Because $(T-G-R)$ is taxes minus government spending and transfers, it is the government surplus, or put differently, government savings $S_{g}$. Equation 7 can therefore be restated as

$$
S_{p}+S_{g}=I+C A
$$

Equation 8 highlights an essential difference between open and closed economies: An open economy can use its saving for domestic investment or for foreign investment (i.e., by exporting its savings and acquiring foreign assets), while in a closed economy savings can only be used for domestic investment. Put another way, an open economy with promising investment opportunities is not constrained by its domestic savings rate in order to exploit these opportunities. As Equation 8 shows, it can raise investment by increasing foreign borrowing (a reduction in CA) without increasing domestic savings. For example, if India decides to build a network of high-speed trains, it can import all the required materials it needs from France and then borrow the funds, perhaps also from France, to pay for the materials. This transaction increases India's domestic investment because the imported materials contribute to the expansion in the country's capital stock. All else being equal, this transaction will also produce a current account deficit for India by an amount equal to the increase in investment. India's savings does not have to increase, even though investment increases. This example can be interpreted as an inter-temporal trade, in which India imports present consumption (when it borrows to fund current expenditure) and exports future consumption (when it repays the loan).

Rearranging Equation 8, we can write

$S_{p}=I+C A-S_{g}$

Equation 9 states that an economy's private savings can be used in three ways: (1) investment in domestic capital (I), (2) purchases of assets from foreigners $(C A)$, and (3) net purchases (or redemptions) of government debt $\left(-S_{g}\right)$.

Finally, we can rearrange Equation 8 again to illustrate the macroeconomic sources of a current account imbalance:

$$
C A=S_{p}+S_{g}-I
$$

A current account deficit tends to result from low private savings, high private investment, a government deficit $\left(S_{g}<0\right)$, or a combination of the three. Alternatively, a current account surplus reflects high private savings, low private investment, or a government surplus.

As outlined above, trade deficits can result from a lack of private or government savings or booming investments. If trade deficits primarily reflect high private or government consumption (i.e., scarce savings $=S_{p}+S_{g}$ ), the deficit country's capacity to repay its liabilities from future production remains unchanged. If a trade deficit primarily reflects strong investments $(I)$, however, the deficit country can increase its productive resources and its ability to repay its liabilities.

We can also see from Equation 3 that a current account deficit tends to reflect a strong domestic economy (elevated consumer, government, and investment spending), which is usually accompanied by elevated domestic credit demand and high interest rates. In such an environment, widening interest rate differentials vis-à-vis other countries can lead to growing net capital imports and produce an appreciating currency. In the long run, however, a persistent current account deficit leads to a permanent increase in the claims held by other countries against the deficit country. As a result, foreign investors may require rising risk premiums for such claims, a process that appears to lead to a depreciating currency.

\section{EXAMPLE 10}
\section{Historical Example: The United Kingdom Budget}
A financial newspaper had the following item:

The UK's budget deficit is the highest in the G-20; in Europe, only Ireland borrows more. These are the stark facts facing Chancellor of the Exchequer George Osborne as he plans his first Budget tomorrow. He intends to tackle the problem even if that involves severe spending cuts and large tax increases.

Source: Financial Times, 21 June 2010.

\begin{enumerate}
  \item What are the likely consequences for the UK current account balance from the planned fiscal policy moves mentioned in the above article?
\end{enumerate}

\section{Solution to 1:}
The combination of spending cuts and tax increases will, all else the same, lead to an improvement in the UK current account position.

\begin{enumerate}
  \setcounter{enumi}{1}
  \item Describe the impact spending cuts and tax increases are likely to have on UK imports.
\end{enumerate}

\section{Solution to 2:}
UK imports are likely to be reduced by tax increases and spending cuts because government demand for foreign goods will fall and growth in private household income, which finances private imports, will be restricted as more household income goes to taxes.

\section{EXAMPLE 11}
\section{Global Current Account Imbalances since 1996}
As a result of growing financial integration and trade liberalization, the world economy has entered a period of rapid growth in cross-border trade since the late 1980s. In synch with surging international trade, current account imbalances widened substantially in the 1990s and the first decade of the new millennium. Exhibit 16 shows current account balances for 1996-2017 for five specific groups-the United States, the top 20 Oil exporting countries as of 2016, Germany and Japan (DEU + JPN), China and emerging Asia (CHN + EMA)-and two broad categories: all other High Income countries and all remaining Low and Middle Income countries. The United States ran a current account deficit in every year, and in every year its deficit represented most of the aggregate value of such deficits worldwide. Only after the 2007-2009 recession has the US deficit declined both in absolute terms and relative to the global aggregate of current account deficits. In the first half of the 1990s, Germany and Japan were the traditional current account surplus countries, providing net exports of goods and services to and accumulating net claims against the United States. Since the late 1990s, among the largest current account surpluses are those of China and emerging Asia. Oil exporting countries, who have traditionally had significant current account surpluses, saw a change to current account deficits in 2015.

\section{Exhibit 16: Global Imbalances (Current Account Balance in Percent of World GDP)}
\begin{center}
\includegraphics[max width=\textwidth]{2023_05_04_ccb578c5089fd41615beg-460}
\end{center}

Note: 1. CHN+EMA includes the following economies: Chinese mainland, Hong Kong SAR, Indonesia, Korea, Malaysia, Philippines, Singapore, Thailand.

\begin{enumerate}
  \setcounter{enumi}{1}
  \item The Oil group consists of the 20 largest oil exporting countries in 2016excluding the United States-as listed in the Economist Intelligence Unit World Factbook 2017.

  \item The High Income as well as the Low and Middle Income groups' definitions include all countries that do not fall into any of the other categories.

\end{enumerate}

Source: IMF Balance of Payments Statistics Yearbook, 2018.

As illustrated in Equation 10, current account deficits or surpluses reflect imbalances between national savings (including government savings) and investments. Current account deficits are often related to expansionary fiscal policy and government deficits. In the 1980s, for instance, the growing deficit in the US current account was widely seen as the consequence of tax cuts and rising defense spending adopted by the Reagan administration. Since the mid-1990s, however, the current account imbalances depicted in Exhibit 16 appear to reflect other, more complex factors. Exhibit 17 illustrates US net savings $(S-I)$ for private domestic businesses, households and non-profits, and the government (i.e., federal, state, and local) from the first quarter of 1996 to the first quarter of 2018. The exhibit indicates that business sector net savings and government net savings as a percentage of GDP were near mirror-images between 1996 to 2008. During the technology bubble businesses invested heavily and ran progressively larger savings deficits while the government moved to a surplus. After the bubble burst the pattern reversed with businesses moving to net positive savings and the government fiscal balance deteriorating sharply. Meanwhile, the household sector gradually reduced its savings rate. After the global financial crisis in late 2008, households and businesses cut spending and increased savings sharply while the government deficit exploded to more than 12 percent of GDP. As the economy recovered from the financial crisis from 2012 onwards, government borrowing fell to five percent of GDP. Both households and business maintained a higher rate of saving during this period than before the financial crisis.

Exhibit 17: United States: Sectorial Saving-Investment Balance since 1996 (Net Savings in Percent of GDP)

\begin{center}
\includegraphics[max width=\textwidth]{2023_05_04_ccb578c5089fd41615beg-461}
\end{center}

Source: Federal Reserve Board, flow-of-funds data.

\section{TRADE ORGANIZATIONS}
describe functions and objectives of the international organizations that facilitate trade, including the World Bank, the International Monetary Fund, and the World Trade Organization

During the Great Depression in the 1930s, countries attempted to support their failing economies by sharply raising barriers to foreign trade, devaluing their currencies to compete against each other for export markets, and restricting their citizens' freedom to hold foreign exchange. These attempts proved to be self-defeating. World trade declined dramatically and employment and living standards fell sharply in many countries. By the 1940s, it had become a wide-spread conviction that the world economy was in need of organizations that would help promote international economic cooperation. In July 1944, during the United Nations Monetary and Financial Conference in Bretton Woods, New Hampshire, representatives of 45 governments agreed on a framework for international economic cooperation. Two crucial, multinational organizations emanated from this conference-the World Bank, which was founded during the conference, and the International Monetary Fund (IMF), which came into formal existence in December 1945. Although the IMF was founded with the goal to stabilize exchange rates and assist the reconstruction of the world's international payment system, the World Bank was created to facilitate post-war reconstruction and development.

A third institution, the International Trade Organization (ITO), was to be created to handle the trade side of international economic cooperation, joining the other two "Bretton Woods" institutions. The draft ITO charter was ambitious, extending beyond world trade regulations to include rules on employment, commodity agreements, restrictive business practices, international investment, and services. The objective was to create the ITO at a United Nations Conference on Trade and Employment in Havana, Cuba in 1947. Meanwhile, 15 countries had begun negotiations in December 1945 to reduce and regulate customs tariffs. With World War II only barely ended, they wanted to give an early boost to trade liberalization and begin to correct the legacy of protectionist measures that had remained in place since the early 1930s. The group had expanded to 23 nations by the time the deal was signed on 30 October 1947 and the General Agreement on Tariffs and Trade (GATT) was born. The Havana conference began on 21 November 1947, less than a month after GATT was signed. The ITO charter was finally approved in Havana in March 1948, but ratification in some national legislatures proved impossible. The most serious opposition was in the US Congress, even though the US government had been one of the driving forces. In 1950, the United States government announced that it would not seek congressional ratification of the Havana Charter, and the ITO was effectively dead. As a consequence, the GATT became the only multilateral instrument governing international trade from 1948 until the World Trade Organization (WTO) was officially established in 1995.

\section{International Monetary Fund}
As we saw earlier, current account deficits reflect a shortage of net savings in an economy and can be addressed by policies designed to rein in domestic demand. This approach could, however, have adverse consequences for domestic employment. The IMF stands ready to lend foreign currencies to member countries to assist them during periods of significant external deficits. A pool of gold and currencies contributed by members provides the IMF with the resources required for these lending operations. The funds are only lent under strict conditions and borrowing countries' macroeconomic policies are continually monitored. The IMF's main mandate is to ensure the stability of the international monetary system, the system of exchange rates and international payments that enables countries to buy goods and services from each other. More specifically, the IMF:

\begin{itemize}
  \item provides a forum for cooperation on international monetary problems;

  \item facilitates the growth of international trade and promotes employment, economic growth, and poverty reduction;

  \item supports exchange rate stability and an open system of international payments; and

  \item lends foreign exchange to members when needed, on a temporary basis and under adequate safeguards, to help them address balance of payments problems. The global financial crisis of 2007-2009 demonstrated that domestic and international financial stability cannot be taken for granted, even in the world's most developed countries. In light of these events, the IMF has redefined and deepened its operations by: ${ }^{27}$

  \item enhancing its lending facilities: The IMF has upgraded its lending facilities to better serve its members. As part of a wide-ranging reform of its lending practices, it has also redefined the way it engages with countries on issues related to structural reform of their economies. In this context, it has doubled member countries' access to fund resources and streamlined its lending approach to reduce the stigma of borrowing for countries in need of financial help.

  \item improving the monitoring of global, regional, and country economies: The IMF has taken several steps to improve economic and financial surveillance, which is its framework for providing advice to member countries on macroeconomic policies and warning member countries of risks and vulnerabilities in their economies.

  \item helping resolve global economic imbalances: The IMF's analysis of global economic developments provides finance ministers and central bank governors with a common framework for discussing the global economy.

  \item analyzing capital market developments: The IMF is devoting more resources to the analysis of global financial markets and their links with macroeconomic policy. It also offers training to country officials on how to manage their financial systems, monetary and exchange regimes, and capital markets.

  \item assessing financial sector vulnerabilities: Resilient, well-regulated financial systems are essential for macroeconomic stability in a world of ever-growing capital flows. The IMF and the World Bank jointly run an assessment program aimed at alerting countries to vulnerabilities and risks in their financial sectors.

\end{itemize}

From an investment perspective, the IMF helps to keep country-specific market risk and global systemic risk under control. The Greek sovereign debt crisis, which threatened to destabilize the entire European banking system, is a recent example. In early 2010, the Greek sovereign debt rating was downgraded to non-investment grade by leading rating agencies as a result of serious concerns about the sustainability of Greece's public sector debt load. Yields on Greek government bonds rose substantially following the downgrading and the country's ability to refinance its national debt was seriously questioned in international capital markets. Bonds issued by some other European governments fell and equity markets worldwide declined in response to spreading concerns of a Greek debt default. The downgrading of Greek sovereign debt was the ultimate consequence of persistent and growing budget deficits the Greek government had run before and after the country had joined the European Monetary Union (EMU) in 2001. Most of the budget shortfalls reflected elevated outlays for public-sector jobs, pensions, and other social benefits as well as persistent tax evasion. Reports that the Greek government had consistently and deliberately misreported the country's official economic and budget statistics contributed to further erosion of confidence in Greek government bonds in international financial markets. Facing default, the Greek government requested that a joint European Union/IMF bailout package be activated, and a loan agreement was reached between Greece, the other EMU member countries, and the IMF. The deal consisted of an immediate EUR45 billion in loans to be provided in 2010, with more funds available later. A total of EUR110 billion was

\begin{center}
\includegraphics[max width=\textwidth]{2023_05_04_ccb578c5089fd41615beg-463}
\end{center}

agreed depending on strict economic policy conditions that included cuts in wages and benefits, an increase in the retirement age for public-sector employees, limits on public pensions, increases in direct and indirect taxes, and a substantial reduction in state-owned companies. By providing conditional emergency lending facilities to the Greek government and designing a joint program with the European Union on how to achieve fiscal consolidation, the IMF prevented a contagious wave of sovereign debt crises in global capital markets.

Another example of IMF activities is the East Asian Financial Crisis in the late 1990s. It began in July 1997, when Thailand was forced to abandon its currency's peg with the US dollar. Currency devaluation subsequently hit other East Asian countries that had similar balance of payment problems, such as South Korea, Malaysia, the Philippines, and Indonesia. They had run persistent and increasing current account deficits, financed mainly with short-term capital imports, in particular, domestic banks borrowing in international financial markets. External financing was popular because of the combination of lower foreign, especially US, interest rates and fixed exchange rates. Easy money obtained from abroad led to imprudent investment, which contributed to overcapacities in several industries and inflated prices on real estate and stock markets. The IMF came to the rescue of the affected countries with considerable loans, accompanied by policies designed to control domestic demand, which included fiscal austerity and tightened monetary reins.

\section{World Bank Group}
The World Bank's main objective is to help developing countries fight poverty and enhance environmentally sound economic growth. For developing countries to grow and attract business, they have to

\begin{itemize}
  \item strengthen their governments and educate their government officials;

  \item implement legal and judicial systems that encourage business;

  \item protect individual and property rights and honour contracts;

  \item develop financial systems robust enough to support endeavours ranging from micro credit to financing larger corporate ventures; and

  \item combat corruption.

\end{itemize}

Given these targets, the World Bank provides funds for a wide range of projects in developing countries worldwide and financial and technical expertise aimed at helping those countries reduce poverty.

The World Bank's two closely affiliated entities - the International Bank for Reconstruction and Development (IBRD) and the International Development Association (IDA)-provide low or no-interest loans and grants to countries that have unfavourable or no access to international credit markets. Unlike private financial institutions, neither the IBRD nor the IDA operates for profit. The IBRD is market-based, and uses its high credit rating to pass the low interest it pays for funds on to its borrowers-developing countries. It pays for its own operating costs because it does not look to outside sources to furnish funds for overhead.

IBRD lending to developing countries is primarily financed by selling AAA-rated bonds in the world's financial markets. Although the IBRD earns a small margin on this lending, the greater proportion of its income comes from lending out its own capital. This capital consists of reserves built up over the years and money paid in from the Bank's 185 member country shareholders. IBRD's income also pays for World Bank operating expenses and has contributed to IDA and debt relief. IDA is the world's largest source of interest-free loans and grant assistance to the poorest countries. IDA's funds are replenished every three years by 40 donor countries. Additional funds are regenerated through repayments of loan principal on 35-to-40-year, no-interest loans, which are then available for re-lending. At the end of September 2010, the IBRD had net loans outstanding of USD125.5 billion, while its borrowings amounted to USD132 billion.

Besides acting as a financier, the World Bank also provides analysis, advice, and information to its member countries to enable them to achieve the lasting economic and social improvements their people need. Another of the Bank's core functions is to increase the capabilities of its partners, people in developing countries, and its own staff. Links to a wide range of knowledge-sharing networks have been set up by the Bank to address the vast need for information and dialogue about development.

From an investment perspective, the World Bank helps to create the basic economic infrastructure that is essential for the creation of domestic financial markets and a well-functioning financial industry in developing countries. Moreover, the IBRD is one of the most important supranational borrowers in the international capital markets. Because of its strong capital position and its very conservative financial, liquidity, and lending policies, it enjoys the top investment-grade rating from the leading agencies and investors have confidence in its ability to withstand adverse events. As a result, IBRD bonds denominated in various major currencies are widely held by institutional and private investors.

\section{World Trade Organization}
The WTO provides the legal and institutional foundation of the multinational trading system. It is the only international organization that regulates cross-border trade relationships among nations on a global scale. It was founded on 1 January 1995, replacing the General Agreement on Tariffs and Trade (GATT) that had come into existence in 1947. The GATT was the only multilateral body governing international trade from 1947 to 1995. It operated for almost half a century as a quasi-institutionalized, provisional system of multilateral treaties. Several rounds of negotiations took place under the GATT, of which the Tokyo round and the Uruguay round may have been the most far reaching. The Tokyo round was the first major effort to address a wide range of non-tariff trade barriers, whereas the Uruguay round focused on the extension of the world trading system into several new areas, particularly trade in services and intellectual property, but also to reform trade in agricultural products and textiles. The GATT still exists in an updated 1994 version and is the WTO's principal treaty for trade in goods. The GATT and the General Agreement on Trade in Services (GATS) are the major agreements within the WTO's body of treaties that encompasses a total of about 60 agreements, annexes, decisions, and understandings.

In November 2001, the most recent and still ongoing round of negotiations was launched by the WTO in Doha, Qatar. The Doha round was an ambitious effort to enhance globalization by slashing barriers and subsidies in agriculture and addressing a wide range of cross-border services. So far, under GATS, which came into force in January 1995, banks, insurance companies, telecommunication firms, tour operators, hotel chains, and transport companies that want to do business abroad can enjoy the same principles of free and fair trade that had previously applied only to international trade in goods. No final agreement has been reached in the Doha round as of mid-2018, however, it marked one of the most crucial events in global trade over the past several decades: China's accession to the WTO in December 2001. The inability to reach agreement in the Doha round has led to an increasing number of bilateral and multi-lateral trade agreements, such as the Trans-Pacific Partnership with Japan, Vietnam and nine other countries.

The WTO's most important functions are the implementation, administration, and operation of individual agreements; acting as a platform for negotiations; and settling disputes. Moreover, the WTO has the mandate to review and propagate its members' trade policies and ensure the coherence and transparency of trade policies through surveillance in a global policy setting. The WTO also provides technical cooperation and training to developing, least-developed, and low-income countries to assist with their adjustment to WTO rules. In addition, the WTO is a major source of economic research and analysis, producing ongoing assessments of global trade in its publications and research reports on special topics. Finally, the WTO is in close cooperation with the other two Bretton Woods institutions, the IMF and the World Bank.

From an investment perspective, the WTO's framework of global trade rules provides the major institutional and regulatory base without which today's global multinational corporations would be hard to conceive. Modern financial markets would look different without the large, multinational companies whose stocks and bonds have become key elements in investment portfolios. In the equity universe, for instance, investment considerations focusing on global sectors rather than national markets would make little sense without a critical mass of multinational firms competing with each other in a globally defined business environment.

\section{EXAMPLE 12}
\section{Historical Example: Function and Objective of International Organizations}
On 10 May 2010, the Greek government officially applied for emergency lending facilities extended by the International Monetary Fund. It sent the following letter to the IMF's Managing Director:

Request for Stand-By Arrangement

This paper was prepared based on the information available at the time it was completed on Monday, May 10, 2010. The views expressed in this document are those of the staff team and do not necessarily reflect the views of the government of Greece or the Executive Board of the IMF. The policy of publication of staff reports and other documents by the IMF allows for the deletion of market-sensitive information.

May 3, 2010

Managing Director

International Monetary Fund

Washington DC

The attached Memorandum of Economic and Financial Policies (MEFP) ${ }^{28}$ outlines the economic and financial policies that the Greek government and the Bank of Greece, respectively, will implement during the remainder of 2010 and in the period 2011-2013 to strengthen market confidence and Greece's fiscal and financial position during a difficult transition period toward a more open and competitive economy. The government is fully committed to the policies stipulated in this document and its attachments, to frame tight budgets in the coming years with the aim to reduce the fiscal deficit to below 3 percent in 2014 and achieve a downward trajectory in the public debt-GDP ratio beginning in 2013, to safeguard the stability of the Greek financial system, and to implement structural reforms to boost competitiveness and the economy's capacity to produce, save, and export. (...) The government is strongly determined to lower the fiscal deficit, (...) by achieving higher and more equitable tax collections,

28 The detailed memorandum is available from \href{http://www.imf.org/external/pubs/ft/scr/2010/cr10111.pdf}{www.imf.org/external/pubs/ft/scr/2010/cr10111.pdf}. and constraining spending in the government wage bill and entitlement outlays, among other items. In view of these efforts and to signal the commitment to effective macroeconomic policies, the Greek government requests that the Fund supports this multi-year program under a Stand-By Arrangement (SBA) for a period of 36 months in an amount equivalent to SDR26.4 billion. ${ }^{29}$ (...) A parallel request for financial assistance to euro area countries for a total amount of $€ 80$ billion has been sent. The implementation of the program will be monitored through quantitative performance criteria and structural benchmarks as described in the attached MEFP and Technical Memorandum of Understanding (TMU). There will be twelve quarterly reviews of the program supported under the SBA by the Fund, (...) to begin with the first review that is expected to be completed in the course of the third calendar quarter of 2010, and then every quarter thereafter until the last quarterly review envisaged to be completed during the second calendar quarter of 2013, to assess progress in implementing the program and reach understandings on any additional measures that may be needed to achieve its objectives. (....) The Greek authorities believe that the policies set forth in the attached memorandum are adequate to achieve the objectives of the economic program, and stand ready to take any further measures that may become appropriate for this purpose. The authorities will consult with the Fund in accordance with its policies on such consultations, (...) and in advance of revisions to the policies contained in the MEFP. All information requested by the Fund (....) to assess implementation of the program will be provided.

(...)

Sincerely,

George Papaconstantinou George Provopoulos

Minister of Finance $\quad$ Governor of the Bank of Greece

\begin{enumerate}
  \item What is the objective of the IMF's emergency lending facilities?
\end{enumerate}

\section{Solution to 1:}
The program seeks to safeguard the stability of the Greek financial system and to implement structural reforms to boost competitiveness and the economy's capacity to produce, save and export.

\begin{enumerate}
  \setcounter{enumi}{1}
  \item What are the macroeconomic policy conditions under which the IMF provides emergency lending to Greece?
\end{enumerate}

\section{Solution to 2:}
The Greek government has to reduce the country's fiscal deficit by achieving higher and more equitable tax collections as well as constrain spending in the government wage bill and entitlement outlays.

29 A SDR (special drawing right) is a basket of four leading currencies: Japanese yen (JPY), US dollar (USD), British pound (GBP), and euro (EUR). It consists of 18.4 yen, 0.6320 USD, 0.0903 GBP, and 0.41 EUR. One SDR was worth 1.4975 USD or 1.1547 EUR on 10 May 2010. 3. What is the amount Greece requests from the IMF as emergency funds?

\section{Solution to 3:}
Greece applied for a standby arrangement in an amount equivalent to SDR26.4 billion (approximately USD39.5 billion, based on the 10 May 2010 exchange rate).

\section{SUMMARY}
This reading provides a framework for analyzing a country's trade and capital flows and their economic implications. It examines basic models that explain trade based on comparative advantage and provides a basis for understanding how international trade can affect the rate and composition of economic growth as well as the attractiveness of investment in various sectors.

\begin{itemize}
  \item The benefits of trade include

  \item gains from exchange and specialization;

  \item gains from economies of scale as companies add new markets for their products;

  \item greater variety of products available to households and firms; and

  \item increased competition and more efficient allocation of resources.

  \item A country has an absolute advantage in producing a good (or service) if it is able to produce that good at a lower absolute cost or use fewer resources in its production than its trading partner. A country has a comparative advantage in producing a good if its opportunity cost of producing that good is less than that of its trading partner.

  \item Even if a country does not have an absolute advantage in the production of any good, it can gain from trade by producing and exporting the good(s) in which it has a comparative advantage and importing good(s) in which it has a comparative disadvantage.

  \item In the Ricardian model of trade, comparative advantage and the pattern of trade are determined by differences in technology between countries. In the Heckscher-Ohlin model of trade, comparative advantage and the pattern of trade are determined by differences in factor endowments between countries. In reality, technology and factor endowments are complementary, not mutually exclusive, determinants of trade patterns.

  \item Trade barriers prevent the free flow of goods and services among countries. Governments impose trade barriers for various reasons including: to promote specific developmental objectives, to counteract certain imperfections in the functioning of markets, or to respond to problems facing their economies.

  \item For purposes of international trade policy and analysis, a small country is defined as one that cannot affect the world price of traded goods. A large country's production and/or consumption decisions do alter the relative prices of trade goods.

  \item In a small country, trade barriers generate a net welfare loss arising from distortion of production and consumption decisions and the associated inefficient allocation of resources. - Trade barriers can generate a net welfare gain in a large country if the gain from improving its terms of trade (higher export prices and lower import prices) more than offsets the loss from the distortion of resource allocations. However, the large country can only gain if it imposes an even larger welfare loss on its trading partner(s).

  \item An import tariff and an import quota have the same effect on price, production, and trade. With a quota, however, some or all of the revenue that would be raised by the equivalent tariff is instead captured by foreign producers (or the foreign government) as quota rents. Thus, the welfare loss suffered by the importing country is generally greater with a quota.

  \item A voluntary export restraint is imposed by the exporting country. It has the same impact on the importing country as an import quota from which foreigners capture all of the quota rents.

  \item An export subsidy encourages firms to export their product rather than sell it in the domestic market. The distortion of production, consumption, and trade decisions generates a welfare loss. The welfare loss is greater for a large country because increased production and export of the subsidized product reduces its global price-that is, worsens the country's terms of trade.

  \item Capital restrictions are defined as controls placed on foreigners' ability to own domestic assets and/or domestic residents' ability to own foreign assets. In contrast to trade restrictions, which limit the openness of goods markets, capital restrictions limit the openness of financial markets.

  \item A regional trading bloc is a group of countries who have signed an agreement to reduce and progressively eliminate barriers to trade and movement of factors of production among the members of the bloc.

  \item They may or may not have common trade barriers against those countries that are not members of the bloc. In a free trade area all barriers to the flow of goods and services among members are eliminated, but each country maintains its own polices against non-members.

  \item A customs union extends the FTA by not only allowing free movement of goods and services among members but also creating a common trade policy against non-members

  \item A common market incorporates all aspects of a customs union and extends it by allowing free movement of factors of production among members.

  \item An economic union incorporates all aspects of a common market and requires common economic institutions and coordination of economic policies among members.

  \item Members of a monetary union adopt a common currency.

  \item From an investment perspective, it is important to understand the complex and dynamic nature of trading relationships because they can help identify potential profitable investment opportunities as well as provide some advance warning signals regarding when to disinvest in a market or industry.

  \item The major components of the balance of payments are the

  \item current account balance, which largely reflects trade in goods and services.

  \item capital account balance, which mainly consists of capital transfers and net sales of non-produced, non-financial assets. - financial account, which measures net capital flows based on sales and purchases of domestic and foreign financial assets.

  \item Decisions by consumers, firms, and governments influence the balance of payments.

  \item Low private savings and/or high investment tend to produce a current account deficit that must be financed by net capital imports; high private savings and/or low investment, however, produce a current account surplus, balanced by net capital exports.

  \item All else the same, a government deficit produces a current account deficit and a government surplus leads to a current account surplus.

  \item All else the same, a sustained current account deficit contributes to a rise in the risk premium for financial assets of the deficit country. Current account surplus countries tend to enjoy lower risk premiums than current account deficit countries.

  \item Created after WWII, the International Monetary Fund, the World Bank, and the World Trade Organization are the three major international organizations that provide necessary stability to the international monetary system and facilitate international trade and development.

  \item The IMF's mission is to ensure the stability of the international monetary system, the system of exchange rates and international payments that enables countries to buy goods and services from each other. The IMF helps to keep country-specific market risk and global systemic risk under control.

  \item The World Bank helps to create the basic economic infrastructure essential for creation and maintenance of domestic financial markets and a well-functioning financial industry in developing countries.

  \item The World Trade Organization's mission is to foster free trade by providing a major institutional and regulatory framework of global trade rules without which today's global multinational corporations would be hard to conceive.

\end{itemize}

\section{REFERENCES}
Appleyard, Dennis, Alfred Field, Steven Cobb. 2010. International Economics. 7th edition. Boston: McGraw-Hill/Irwin.

Ariyoshi, Akira, Karl Habermeier, Bernard Laurens, Inci Otker-Robe, Jorge Iván Canales-Kriljenko, Andrei Kirilenko. 2000. "Capital Controls: Country Experiences with Their Use and Liberalization." IMF Occasional Paper 190, Washington, DC (May 17).

Bernard, Andrew B., J. Bradford Jensen, Stephen J. Redding, Peter K. Schott. 2010. "Intrafirm Trade and Product Contractibility." American Economic Review, vol. 100, no. 2 (May):444-448. $10.1257 /$ aer.100.2.444

Coe, David T., Elhanan Helpman. 1995. "International R\&D Spillovers." European Economic Review, vol. 39, no. 5 (May):859-887. 10.1016/0014-2921(94)00100-E

Collier, Paul, Stephen A. O'Connell. 2007. "Opportunities and Choices." In The Political Economy of Economic Growth in Africa, 1960-2000, vol. 1. Edited by Benno J. Ndulu, Stephen A. O'Connell, Robert H. Bates, Paul Collier, and Charles C. Soludo. Cambridge, U.K.: Cambridge University Press.

Feenstra, Robert C., Alan M. Taylor. 2008. International Economics. New York: Worth Publishers.

Gerber, James. 2017. International Economics. 7th edition. New York: Prentice Hall.

Hill, Charles W.L., G. Tomas M. Hult. 2019. International Business: Competing in the Global Marketplace. 12th edition. Boston: Irwin/McGraw-Hill.

Kawai, Masahiro, Shinji Takagi. 2003. "Rethinking Capital Controls: The Malaysian Experience." PRI Discussion Paper Series No. 03A-05, Policy Research Institute, Ministry of Finance Japan, Tokyo (May).

Meier, Gerald M. 1998. The International Environment of Business: Competition and Governance in the Global Economy. New York: Oxford University Press.

Roberts, Mark, Uwe Deichmann. 2008. "Regional Spillover Estimation." Background paper for the World Development Report 2009: Reshaping Economic Geography, World Bank.

Salvatore, Dominick. 2011. Introduction to International Economics. 3rd edition. Hoboken, NJ: John Wiley \& Sons.

World Bank 2009. World Development Report 2009: Reshaping Economic Geography. Washington, DC: World Bank.

World Trade Organization 2008. World Trade Report 2008: Trade in a Globalizing World. Geneva: World Trade Organization.

\section{PRACTICE PROBLEMS}
\begin{enumerate}
  \item Which of the following statements best describes the benefits of international trade?
\end{enumerate}

A. Countries gain from exchange and specialization.

B. Countries receive lower prices for their exports and pay higher prices for imports.

C. Absolute advantage is required for a country to benefit from trade in the long term.

\begin{enumerate}
  \setcounter{enumi}{1}
  \item Which of the following statements best describes the costs of international trade?
\end{enumerate}

A. Countries without an absolute advantage in producing a good cannot benefit significantly from international trade.

B. Resources may need to be allocated into or out of an industry and less-efficient companies may be forced to exit an industry, which in turn may lead to higher unemployment.

C. Loss of manufacturing jobs in developed countries as a result of import competition means that developed countries benefit far less than developing countries from trade.

\begin{enumerate}
  \setcounter{enumi}{2}
  \item Suppose the cost of producing tea relative to copper is lower in Tealand than in Copperland. With trade, the copper industry in Copperland would most likely:
A. expand.
B. contract.
C. remain stable.

  \item A country has a comparative advantage in producing a good if:

\end{enumerate}

A. it is able to produce the good at a lower cost than its trading partner.

B. its opportunity cost of producing the good is less than that of its trading partner.

C. its opportunity cost of producing the good is more than that of its trading partner.

\begin{enumerate}
  \setcounter{enumi}{4}
  \item Suppose Mexico exports vegetables to Brazil and imports flashlights used for mining from Brazil. The output per worker per day in each country is as follows:
\end{enumerate}

\begin{center}
\begin{tabular}{lcc}
\hline
 & Flashlights & Vegetables \\
\hline
Mexico & 20 & 60 \\
Brazil & 40 & 80 \\
\hline
\end{tabular}
\end{center}

Which country has a comparative advantage in the production of vegetables and what is the most relevant opportunity cost?

A. Brazil: 2 vegetables per flashlight. B. Mexico: 1.5 vegetables per flashlight.

C. Mexico: $1 / 3$ flashlight per vegetable.

\begin{enumerate}
  \setcounter{enumi}{5}
  \item Suppose three countries produce bananas and pencils with output per worker per day in each country as follows:
\end{enumerate}

\begin{center}
\begin{tabular}{lcc}
\hline
 & Bananas & Pencils \\
\hline
Mexico & 20 & 40 \\
Brazil & 30 & 90 \\
Canada & 40 & 160 \\
\hline
\end{tabular}
\end{center}

Which country has the greatest comparative advantage in the production of bananas?
A. Canada.
B. Brazil.
C. Mexico.

\begin{enumerate}
  \setcounter{enumi}{6}
  \item In the Ricardian trade model, a country captures more of the gains from trade if:
A. it produces all products while its trade partner specializes in one good.
B. the terms of trade are closer to its autarkic prices than to its partner's autar- kic prices.
C. the terms of trade are closer to its partner's autarkic prices than to its autar- kic prices.

  \item Germany has much more capital per worker than Portugal. In autarky each country produces and consumes both machine tools and wine. Production of machine tools is relatively capital intensive whereas winemaking is labor intensive. According to the Heckscher-Ohlin model, when trade opens:

\end{enumerate}

A. Germany should export machine tools and Portugal should export wine.

B. Germany should export wine and Portugal should export machine tools.

C. Germany should produce only machine tools and Portugal should produce only wine.

\begin{enumerate}
  \setcounter{enumi}{8}
  \item According to the Heckscher-Ohlin model, when trade opens:
A. the scarce factor gains relative to the abundant factor in each country.
B. the abundant factor gains relative to the scarce factor in each country.
C. income is redistributed between countries but not within each country.

  \item Which type of trade restriction would most likely increase domestic government revenue?
A. Tariff.
B. Import quota.
C. Export subsidy. 11. Which of the following trade restrictions is likely to result in the greatest welfare loss for the importing country?
A. A tariff.
B. An import quota.
C. A voluntary export restraint.

  \item A large country can:
A. benefit by imposing a tariff.
B. benefit with an export subsidy.
C. not benefit from any trade restriction.

  \item If Brazil and South Africa have free trade with each other, a common trade policy against all other countries, but no free movement of factors of production between them, then Brazil and South Africa are part of a:

\end{enumerate}

A. customs union.

B. common market.

C. free trade area (FTA).

\begin{enumerate}
  \setcounter{enumi}{13}
  \item Which of the following factors best explains why regional trading agreements are more popular than larger multilateral trade agreements?
\end{enumerate}

A. Minimal displacement costs.

B. Trade diversions benefit members.

C. Quicker and easier policy coordination.

\begin{enumerate}
  \setcounter{enumi}{14}
  \item The sale of mineral rights would be captured in which of the following balance of payments components?
\end{enumerate}

A. Capital account.

B. Current account.

C. Financial account.

\begin{enumerate}
  \setcounter{enumi}{15}
  \item Patent fees and legal services are recorded in which of the following balance of payments components?
\end{enumerate}

A. Capital account.

B. Current account.

C. Financial account.

\begin{enumerate}
  \setcounter{enumi}{16}
  \item During the most recent quarter, a steel company in South Korea had the following transactions
\end{enumerate}

\begin{itemize}
  \item Bought iron ore from Australia for AUD50 million.

  \item Sold finished steel to the United States for USD65 million.

  \item Borrowed AUD50 million from a bank in Sydney, Australia. - Received a USD10 million dividend from US subsidiary.

  \item Paid KRW550 million to a Korean shipping company.

\end{itemize}

Which of the following would be reflected in South Korea's current account balance for the quarter?
A. The loan.
B. The shipping.
C. The dividend.

\begin{enumerate}
  \setcounter{enumi}{17}
  \item Which of the following most likely contributes to a current account deficit?
A. High taxes.
B. Low private savings.
C. Low private investment.

  \item Which of the following chronic deficit conditions is least alarming to the deficit country's creditors?
A. High consumption.
B. High private investment.
C. High government spending.

  \item Which of the following international trade organizations regulates cross-border exchange among nations on a global scale?
A. World Bank Group (World Bank).
B. World Trade Organization (WTO).
C. International Monetary Fund (IMF).

  \item Which of the following international trade organizations has a mission to help developing countries fight poverty and enhance environmentally sound economic growth?
A. World Bank Group (World Bank).
B. World Trade Organization (WTO).
C. International Monetary Fund (IMF).

  \item Which of the following organizations helps to keep global systemic risk under control by preventing contagion in scenarios such as the 2010 Greek sovereign debt crisis?
A. World Bank Group (World Bank).
B. World Trade Organization (WTO).
C. International Monetary Fund (IMF).

  \item Which of the following international trade bodies was the only multilateral body governing international trade from 1948 to $1995 ?$

\end{enumerate}

A. World Trade Organization (WTO). B. International Trade Organization (ITO).

C. General Agreement on Tariffs and Trade (GATT).

\section{SOLUTIONS}
\begin{enumerate}
  \item A is correct. Countries gain from exchange when trade enables each country to receive a higher price for exported goods and/or pay a lower price for imported goods. This leads to more efficient resource allocation and allows consumption of a larger variety of goods.

  \item B is correct. Resources may need to be reallocated into or out of an industry, depending on whether that industry is an exporting sector or an import-competing sector of that economy. As a result of this adjustment process, less-efficient companies may be forced to exit the industry, which in turn may lead to higher unemployment and the need for retraining in order for displaced workers to find jobs in expanding industries.

  \item A is correct. The copper industry in Copperland would benefit from trade. Because the cost of producing copper relative to producing tea is lower in Copperland than in Tealand, Copperland will export copper and the industry will expand.

  \item B is correct. Comparative advantage is present when the opportunity cost of producing a good is less than that of a trading partner.

  \item C is correct. While Brazil has an absolute advantage in the production of both flashlights and vegetables, Mexico has a comparative advantage in the production of vegetables. The opportunity cost of vegetables in Mexico is $1 / 3$ per flashlight, while the opportunity cost of vegetables in Brazil is $1 / 2$ per flashlight.

  \item C is correct. Mexico has the lowest opportunity cost to produce an extra banana. The opportunity cost is 2 pencils per banana in Mexico, 3 pencils per banana in Brazil, and 4 pencils per banana in Canada.

  \item $\mathrm{C}$ is correct. A country gains if trade increases the price of its exports relative to its imports as compared to its autarkic prices, i.e. the final terms of trade are more favorable than its autarkic prices. If the relative price of exports and imports remains the same after trade opens, then the country will consume the same basket of goods before and after trade opens, and it gains nothing from the ability to trade. In that case, its trade partner will capture all of the gains. Of course, the opposite is true if the roles are reversed. More generally, a country captures more of the gains from trade the more the final terms of trade differ from its autarkic prices.

  \item A is correct. In the Heckscher-Ohlin model a country has a comparative advantage in goods whose production is intensive in the factor with which it is relatively abundantly endowed. In this case, capital is relatively abundant in Germany so Germany has a comparative advantage in producing the capital-intensive product: machine tools. Portugal is relatively labor abundant, hence should produce and export the labor-intensive product: wine.

  \item B is correct. As a country opens up to trade, it has a favorable impact on the abundant factor, and a negative impact on the scarce factor. This is because trade causes the output mix to change and therefore changes the relative demand for the factors of production. Increased output of the export product increases demand for the factor that is used intensively in its production, while reduced output of the import product decreases demand for the factor used intensively in its production. Because the export (import) product uses the abundant (scarce) factor intensively, the abundant factor gains relative to the scarce factor in each country.

  \item A is correct. The imposition of a tariff will most likely increase domestic government revenue. A tariff is a tax on imports collected by the importing country's government.

  \item $\mathrm{C}$ is correct. With a voluntary export restraint, the price increase induced by restricting the quantity of imports (= quota rent for equivalent quota = tariff revenue for equivalent tariff) accrues to foreign exporters and/or the foreign government.

  \item A is correct. By definition, a large country is big enough to affect the world price of its imports and exports. A large country can benefit by imposing a tariff if its terms of trade improve by enough to outweigh the welfare loss arising from inefficient allocation of resources.

  \item A is correct. A customs union extends a free trade area (FTA) by not only allowing free movement of goods and services among members, but also creating common trade policy against non-members. Unlike a more integrated common market, a customs union does not allow free movement of factors of production among members.

  \item C is correct. Regional trading agreements are politically less contentious and quicker to establish than multilateral trade negotiations (for example, under the World Trade Organization). Policy coordination and harmonization is easier among a smaller group of countries.

  \item A is correct. The capital account measures capital transfers and sale and purchase of non-produced, non-financial assets such as mineral rights and intangible assets.

  \item B is correct. The current account measures the flows of goods and services (including income from foreign investments). Patent fees and legal services are both captured in the services sub-account of the current account.

  \item $\mathrm{C}$ is correct. The current account includes income received on foreign investments. The Korean company effectively "exported" the use of its capital during the quarter to its US subsidiary, and the dividend represents payment for those services.

  \item B is correct. A current account deficit tends to result from low private saving, high private investment, a government deficit, or a combination of the three. Of the choices, only low private savings contributes toward a current account deficit.

  \item B is correct. A current account deficit tends to result from low private saving, high private investment, low government savings, or a combination of the three. Of these choices, only high investments can increase productive resources and improve future ability to repay creditors.

  \item B is correct. The WTO provides the legal and institutional foundation of the multinational trading system and is the only international organization that regulates cross-border trade relations among nations on a global scale. The WTO's mission is to foster free trade by providing a major institutional and regulatory framework of global trade rules. Without such global trading rules, today's global transnational corporations would be hard to conceive.

  \item A is correct. The World Bank's mission is to help developing countries fight poverty and enhance environmentally sound economic growth. The World Bank helps to create the basic economic infrastructure essential for creation and maintenance of domestic financial markets and a well-functioning financial industry in developing countries.

  \item $\mathrm{C}$ is correct. From an investment perspective, the IMF helps to keep country-specific market risk and global systemic risk under control. The Greek sovereign debt crisis on 2010, which threatened to destabilize the entire European banking system, is a recent example. The IMF's mission is to ensure the stability of the international monetary system, the system of exchange rates and international payments which enables countries to buy goods and services from each other.

  \item $\mathrm{C}$ is correct. The GATT was the only multilateral body governing international trade from 1948 to 1995 . It operated for almost half a century as a quasi-institutionalized, provisional system of multilateral treaties and included several rounds of negotiations.

\end{enumerate}

\section{LEARNING MODULE 8}
\section{Currency Exchange Rates}
by William A. Barker, PhD, CFA, Paul D. McNelis, and Jerry Nickelsburg, PhD.

William A. Barker, PhD, CFA (Canada). Paul D. McNelis is at Gabelli School of Business, Fordham University (USA). Jerry Nickelsburg, PhD, is at the Anderson School of

Management, University of California, Los Angeles (USA).

\section{LEARNING OUTCOME}
\begin{center}
\includegraphics[max width=\textwidth]{2023_05_04_ccb578c5089fd41615beg-481}
\end{center}

\section*{INTRODUCTION \& THE FOREIGN EXCHANGE MARKET }
\begin{abstract}
define an exchange rate and distinguish between nominal and real exchange rates and spot and forward exchange rates calculate and interpret the percentage change in a currency relative to another currency
\end{abstract}

Measured by daily turnover, the foreign exchange (FX) market-the market in which currencies are traded against each other - is by far the world's largest market. Current estimates put daily turnover at approximately USD5.1 trillion for 2016 . This is about 10 to 15 times larger than daily turnover in global fixed-income markets and about 50 times larger than global turnover in equities.

The FX market is also a truly global market that operates 24 hours a day, each business day. It involves market participants from every time zone connected through electronic communications networks that link players as large as multibillion-dollar investment funds and as small as individuals trading for their own account-all brought together in real time. International trade would be impossible without the trade in currencies that facilitates it, and so too would cross-border capital flows that connect all financial markets globally through the FX market.

These factors make foreign exchange a key market for investors and market participants to understand. The world economy is increasingly transnational in nature, with both production processes and trade flows often determined more by global factors than by domestic considerations. Likewise, investment portfolio performance increasingly reflects global determinants because pricing in financial markets responds to the array of investment opportunities available worldwide, not just locally. All of these factors funnel through, and are reflected in, the foreign exchange market. As investors shed their "home bias" and invest in foreign markets, the exchange rate-the price at which foreign-currency-denominated investments are valued in terms of the domestic currency-becomes an increasingly important determinant of portfolio performance.

Even investors adhering to a purely "domestic" portfolio mandate are increasingly affected by what happens in the foreign exchange market. Given the globalization of the world economy, most large companies depend heavily on their foreign operations (for example, by some estimates about 30 percent of S\&P 500 Index earnings are from outside the United States). Almost all companies are exposed to some degree of foreign competition, and the pricing for domestic assets-equities, bonds, real estate, and others - will also depend on demand from foreign investors. All of these various influences on investment performance reflect developments in the foreign exchange market.

This reading introduces the foreign exchange market, providing the basic concepts and terminology necessary to understand exchange rates as well as some of the basics of exchange rate economics.

The reading is divided up as follows. Section 2 describes the organization of the foreign exchange market and discusses the major players-who they are, how they conduct their business, and how they respond to exchange rate changes. Section 3 takes up the mechanics of exchange rates: definitions, quotes, and calculations. This section shows that the reader has to pay close attention to conventions used in various foreign exchange markets around the world because they can vary widely. Sometimes exchange rates are quoted in the number of domestic currency units per unit of foreign currency, and sometimes they are quoted in the opposite way. The exact notation used to represent exchange rates can vary widely as well, and occasionally the same exchange rate notation will be used by different sources to mean completely different things. The notation used here may not be the same as that encountered elsewhere. Therefore, the focus should be on understanding the underlying concepts rather than relying on rote memorization of formulas. We also show how to calculate cross-exchange rates and how to compute the forward exchange rate given either the forward points or the percentage forward premium or discount. In Section 4, we discuss alternative exchange rate regimes operating throughout the world. Finally, in Section 5, we discuss how exchange rates affect a country's international trade (exports and imports) and capital flows. A summary and practice problems conclude the reading.

\section{The Foreign Exchange Market}
To understand the FX market, it is necessary to become familiar with some of its basic conventions. Individual currencies are often referred to by standardized three-letter codes that the market has agreed upon through the International Organization for Standardization (ISO). Exhibit 1 lists some of the major global currencies and their identification codes.

\section{Exhibit 1: Standard Currency Codes}
\begin{center}
\begin{tabular}{ll}
\hline
Three-Letter Currency Code & Currency \\
\hline
USD & US dollar \\
EUR & Euro \\
JPY & Japanese yen \\
GBP & British pound \\
CHF & Swiss franc \\
CAD & Canadian dollar \\
AUD & Australian dollar \\
NZD & New Zealand dollar \\
ZAR & South African rand \\
SEK & Swedish krona \\
NOK & Norwegian krone \\
BRL & Brazilian real \\
SGD & Singapore dollar \\
MXN & Mexican peso \\
CNY & Chinese yuan \\
HKD & Hong Kong dollar \\
INR & Indian rupee \\
KRW & Russian ruble \\
\hline
\end{tabular}
\end{center}

It is important to understand that there is a difference between referring to an individual currency and an exchange rate. One can hold an individual currency (for example, in a EUR100 million deposit), but an exchange rate refers to the price of one currency in terms of another (for example, the exchange rate between the EUR and USD). An individual currency can be singular, but there are always two currencies involved in an exchange rate: the price of one currency relative to another. The exchange rate is the number of units of one currency (called the price currency) that one unit of another currency (called the base currency) will buy. An equivalent way of describing the exchange rate is as the cost of one unit of the base currency in terms of the price currency.

This distinction between individual currencies and exchange rates is important because, as we will see in a later section, these three-letter currency codes can be used both ways. (For example, when used as an exchange rate in the professional FX market, EUR is understood to be the exchange rate between the euro and US dollar). But be aware of the context (either as a currency or as an exchange rate) in which these three-letter currency codes are being used. To avoid confusion, this reading will identify exchange rates using the convention of "A/B," referring to the number of units of currency A that one unit of currency B will buy. For example, a USD/EUR exchange rate of 1.1700 means that 1 euro will buy 1.1700 US dollars (i.e., 1 euro costs 1.1700 US dollars). ${ }^{1}$ In this case, the euro is the base currency and the US dollar is the price currency. A decrease in this exchange rate would mean that the euro costs less or that fewer US dollars are needed to buy one euro. In other words, a decline in this exchange rate indicates that the USD is appreciating against the EUR or, equivalently, the EUR is depreciating against the USD.

The exchange rates described above are referred to as nominal exchange rates. This is to distinguish them from real exchange rates, which are indexes often constructed by economists and other market analysts to assess changes in the relative purchasing power of one currency compared with another. Creating these indexes requires adjusting the nominal exchange rate by using the price levels in each country of the currency pair (hence the name "real exchange rates") in order to compare the relative purchasing power between countries.

In a world of homogenous goods and services and with no market frictions or trade barriers, the relative purchasing power across countries would tend to equalize: Why would you pay more, in real terms, domestically for a "widget" if you could import an identical "widget" from overseas at a cheaper price? This basic concept is the intuition behind a theory known as "purchasing power parity" (PPP), which describes the long-term equilibrium of nominal exchange rates. PPP asserts that nominal exchange rates adjust so that identical goods (or baskets of goods) will have the same price in different markets. Or, put differently, the purchasing power of different currencies is equalized for a standardized basket of goods.

In practice, the conditions required to enforce PPP are not satisfied: Goods and services are not identical across countries; countries typically have different baskets of goods and services produced and consumed; many goods and services are not traded internationally; there are trade barriers and transaction costs (e.g., shipping costs and import taxes); and capital flows are at least as important as trade flows in determining nominal exchange rates. As a result, nominal exchange rates exhibit persistent deviations from PPP. Moreover, relative purchasing power among countries displays a weak, if any, tendency toward long-term equalization. A simple example of a cross-country comparison of the purchasing power of a standardized good is the "Big Mac" index produced by the Economist, which shows the relative price of this standardized hamburger in different countries. The Big Mac index shows that fast-food hamburger prices can vary widely internationally and that this difference in purchasing power is typical of most goods and services. Hence, movements in real exchange rates provide meaningful information about changes in relative purchasing power among countries.

1 This convention is consistent with the meaning of "/" in mathematics and the straightforward interpretation of "A/B" as "A per B" is helpful in understanding exchange rates as the price of one currency in terms of another. Nevertheless, other notation conventions exist. "B/A" and "B:A" are sometimes used to denote what this reading denotes as "A/B." Careful attention to the context will usually make the convention clear. Consider the case of an individual who wants to purchase goods from a foreign country. The individual would be able to buy fewer of these goods if the nominal spot exchange rate for the foreign currency appreciated or if the foreign price level increased. Conversely, the individual could buy more foreign goods if the individual's domestic income increased. (For this example, we will assume that changes in the individual's income are proportional to changes in the domestic price level.) Hence, in real purchasing power terms, the real exchange rate that an individual faces is an increasing function of the nominal exchange rate (quoted in terms of the number of units of domestic currency per one unit of foreign currency) and the foreign price level and a decreasing function of the domestic price level. The higher the real exchange rate that this individual faces, the fewer foreign goods, in real terms, the individual can purchase and the lower that individual's relative purchasing power compared with the other country.

An equivalent way of viewing the real exchange rate is that it represents the relative price levels in the domestic and foreign countries. Mathematically, we can represent the foreign price level in terms of the domestic currency as:

Foreign price level in domestic currency $=S_{d / f} \times P_{f}$

where $S_{d / f}$ is the spot exchange rate (quoted in terms of the number of units of domestic currency per one unit of foreign currency) and $P_{f}$ is foreign price level quoted in terms of the foreign currency. We can define the domestic price level, in terms of the domestic currency, as $P_{d}$. Hence, the ratio between the foreign and domestic price levels is:

Real exchange $\operatorname{rate}_{(d / f)}=\left(S_{d / f} \times P_{f}\right) / P_{d}=S_{d l f} \times\left(P_{f} / P_{d}\right)$

For example, for a British consumer wanting to buy goods made in the Eurozone, the real exchange rate (defined in GBP/EUR terms; note that the domestic currency for the United Kingdom is the price currency, not the base currency) will be an increasing function of the nominal spot exchange rate (GBP/EUR) and the Eurozone price level and a decreasing function of the UK price level. This is written as:

Real exchange rate $\frac{G B P}{E U R}=S_{\frac{G B P}{}}^{E U R} \times\left(\frac{C P I_{\text {eur }}}{C P I_{U K}}\right)$

Let's examine the effect of movements in the domestic and foreign price levels, and the nominal spot exchange rate, on the real purchasing power of an individual in the United Kingdom wanting to purchase Eurozone goods. Assume that the nominal spot exchange rate (GBP/EUR) increases by 10 percent, the Eurozone price level by 5 percent, and the UK price level by 2 percent. The change in the real exchange rate is then:

$$
\left(1+\frac{\Delta S \frac{d}{f}}{S \frac{d}{f}}\right) \times \frac{\left(1+\frac{\Delta P_{f}}{P_{f}}\right)}{\left(1+\frac{\Delta P_{d}}{P_{d}}\right)}-1=(1+10 \%) \times \frac{1+5 \%}{1+2 \%}-1 \approx 10 \%+5 \%-2 \% \approx 13 \%
$$

In this case, the real exchange rate for the UK-based individual has increased about 13 percent, meaning that it now costs more, in real terms, to buy Eurozone goods. Or put differently, the UK individual's real purchasing power relative to Eurozone goods has declined by about 13 percent. An easy way to remember this relationship is to consider the real exchange rate (stated with the domestic currency as the price currency) as representing the real price you face in order to purchase foreign goods and services: The higher the price (real exchange rate), the lower your relative purchasing power

The real exchange rate for a currency can be constructed for the domestic currency relative to a single foreign currency or relative to a basket of foreign currencies. In either case, these real exchange rate indexes depend on the assumptions made by the analyst creating them. Several investment banks and central banks create proprietary measures of real exchange rates. It is important to note that real exchange rates are not quoted or traded in global FX markets: They are only indexes created by analysts to understand the international competitiveness of an economy and the real purchasing power of a currency.

In this context, real exchange rates can be useful for understanding trends in international trade and capital flows and hence can be seen as one of the influences on nominal spot exchange rates. As an example, consider the exchange rate between the Indian rupee and the US dollar. During 2018, the nominal rupee exchange rate against the US dollar (INR/USD) rose by approximately 6.7 percent-meaning that the US dollar appreciated against the rupee. However, the annual inflation rates in the United States and India were different during 2018-approximately 2.5 percent for the United States and 4.7 percent for India. This means that the real exchange rate (in INR/ USD terms) was depreciating less rapidly than the nominal INR/USD exchange rate:

$$
\left(1+\% \Delta S_{\frac{I N R}{U S D}}\right) \times \frac{\left(1+\% \Delta P_{U S}\right)}{\left(1+\% \Delta P_{\text {India }}\right)}-1 \approx+6.7 \%+2.5 \%-4.7 \% \approx 4.5 \%
$$

This combination of a much weaker rupee and a higher Indian inflation rate meant that the real exchange rate faced by India was increasing, thus decreasing Indian purchasing power in USD terms.

Movements in real exchange rates can have a similar effect as movements in nominal exchange rates in terms of affecting relative prices and hence trade flows. Even if the nominal spot exchange rate does not move, differences in inflation rates between countries affect their relative competitiveness.

Although real exchange rates can exert some influence on nominal exchange rate movements, they are only one of many factors; it can be difficult to disentangle all of these inter-relationships in a complex and dynamic FX market. As discussed earlier, PPP is a poor guide to predicting future movements in nominal exchange rates because these rates can deviate from PPP equilibrium-and even continue to trend away from their PPP level-for years at a time. Hence, it should not be surprising that real exchange rates, which reflect changes in relative purchasing power, have a poor track record as a predictor of future nominal exchange rate movements.

\section{EXAMPLE 1}
\section{Nominal and Real Exchange Rates}
An investment adviser located in Sydney, Australia, is meeting with a local client who is looking to diversify her domestic bond portfolio by adding investments in fixed-rate, long-term bonds denominated in HKD. The client frequently visits Hong Kong SAR, and many of her annual expenses are denominated in HKD. The client, however, is concerned about the foreign currency risks of offshore investments and whether the investment return on her HKD-denominated investments will maintain her purchasing power-both domestically (i.e., for her AUD-denominated expenses) and in terms of her foreign trips (i.e., denominated in HKD, for her visits to Hong Kong SAR). The investment adviser explains the effect of changes in nominal and real exchange rates to the client and illustrates this explanation by making the following statements:

Statement 1 All else equal, an increase in the nominal AUD/ HKD exchange rate will lead to an increase in the AUD-denominated value of your foreign investment. Statement 2 All else equal, an increase in the nominal AUD/HKD exchange rate means that your relative purchasing power for your Hong Kong SAR trips will increase (based on paying for your trip with the income from your HKD-denominated bonds).

Statement 3 All else equal, an increase in the Australian inflation rate will lead to an increase in the real exchange rate (AUD/HKD). A higher real exchange rate means that the relative purchasing power of your AUD-denominated income is higher.

Statement 4 All else equal, a decrease in the nominal exchange rate (AUD/HKD) will decrease the real exchange rate (AUD/ HKD) and increase the relative purchasing power of your AUD-denominated income.

To demonstrate the effects of the changes in inflation and nominal exchange rates on relative purchasing power, the adviser uses the following scenario: "Suppose that the AUD/HKD exchange rate increases by 5 percent, the price of goods and services in Hong Kong SAR goes up by 5 percent, and the price of Australian goods and services goes up by 2 percent."

\begin{enumerate}
  \item Statement 1 is:
\end{enumerate}

A. correct.

B. incorrect, because based on the quote convention the investment's value would be decreasing in AUD terms.

C. incorrect, because the nominal AUD value of the foreign investments will depend on movements in the Australian inflation rate.

\section{Solution to 1:}
A is correct. Given the quoting convention, an increase in the AUD/HKD rate means that the base currency (HKD) is appreciating (one HKD will buy more AUD). This is increasing the nominal value of the HKD-denominated investments when measured in AUD terms.

\begin{enumerate}
  \setcounter{enumi}{1}
  \item Statement 2 is:
\end{enumerate}

A. correct.

B. incorrect, because purchasing power is not affected in this case.

C. incorrect, because based on the quote convention, the client's relative purchasing power would be decreasing.

\section{Solution to 2:}
$B$ is correct. When paying for $\mathrm{HKD}$-denominated expenses with HKD-denominated income, the value of the AUD/HKD spot exchange rate (or any other spot exchange rate) would not be relevant. In fact, this is a basic principle of currency risk management: reducing FX risk exposures by denominating assets and liabilities (or income and expenses) in the same currency.

\begin{enumerate}
  \setcounter{enumi}{2}
  \item Statement 3 is:
\end{enumerate}

A. correct.

B. incorrect with respect to the real exchange rate only. C. incorrect with respect to both the real exchange rate and the purchasing power of AUD-denominated income.

\section{Solution to 3:}
$\mathrm{C}$ is correct. An increase in the Australian (i.e., domestic) inflation rate means that the real exchange rate (measured in domestic/foreign, or AUD/ HKD, terms) would be decreasing, not increasing. Moreover, an increase in the real exchange rate $\left(R_{A U D / H K D}\right)$ would be equivalent to a reduction of the purchasing power of the Australian client: Goods and services denominated in HKD would cost more.

\begin{enumerate}
  \setcounter{enumi}{3}
  \item Statement 4 is:
\end{enumerate}

A. correct.

B. incorrect with respect to the real exchange rate.

C. incorrect with respect to the purchasing power of AUD-denominated income.

\section{Solution to 4:}
A is correct. As the spot AUD/HKD exchange rate decreases, the HKD is depreciating against the AUD; or equivalently, the AUD is appreciating against the HKD. This is reducing the real exchange rate $\left(R_{A U D / H K D}\right)$ and increasing the Australian client's purchasing power.

\begin{enumerate}
  \setcounter{enumi}{4}
  \item Based on the adviser's scenario and assuming that the HKD value of the HKD bonds remained unchanged, the nominal AUD value of the client's HKD investments would:
\end{enumerate}

A. decrease by about 5 percent.

B. increase by about 5 percent.

C. remain approximately the same.

\section{Solution to 5:}
B is correct. As the AUD/HKD spot exchange rate increases by 5 percent, the HKD is appreciating against the AUD by 5 percent and, all else equal, the value of the HKD-denominated investment is increasing by 5 percent in AUD terms.

\begin{enumerate}
  \setcounter{enumi}{5}
  \item Based on the adviser's scenario, the change in the relative purchasing power of the client's AUD-denominated income is closest to:
A. -8 percent.
B. +8 percent.
C. +12 percent.
\end{enumerate}

\section{Solution to 6:}
A is correct. The real exchange rate $\left(R_{A U D / H K D}\right)$ is expressed as:

$$
R_{\frac{A U D}{H K D}}=S_{\frac{A U D}{H K D}} \times \frac{P_{H K D}}{P_{A U D}}
$$

The information in the adviser's scenario can be expressed as:

$$
\% \triangle R \frac{A U D}{H K D} \approx \% \Delta S \frac{A U D}{H K D}+\% \triangle P_{H K D}-\% \Delta P_{A U D} \approx+5 \%+5 \%-2 \% \approx+8 \%
$$

Because the real exchange rate (expressed in AUD/HKD terms) has gone up by about 8 percent, the real purchasing power of the investor based in Australia has declined by about 8 percent. This can be seen from the fact that HKD has appreciated against the AUD in nominal terms, and the Hong Kong SAR price level has also increased. This increase in the cost of Hong Kong SAR goods and services (measured in AUD) is only partially offset by the small (2 percent) increase in the investor's income (assumed equal to the change in the Australian price level).

\section{MARKET FUNCTIONS}
define an exchange rate and distinguish between nominal and real exchange rates and spot and forward exchange rates describe functions of and participants in the foreign exchange
market

FX markets facilitate international trade in goods and services, where companies and individuals need to make transactions in foreign currencies. This would cover everything from companies and governments buying and selling products in other countries, to tourists engaged in cross-border travel (for example, a German tourist selling euros and buying sterling for a visit to London). Although this is an important dimension of FX markets, and despite the growth of global trade in recent years, an even larger proportion of the daily turnover in FX markets is accounted for by capital market transactions, where investors convert between currencies for the purpose of moving funds into (or out of) foreign assets. These types of transactions cover the range from direct investments (for example, companies buying such fixed assets as factories) in other countries to portfolio investments (the purchase of stocks, bonds, and other financial assets denominated in foreign currencies). Because capital is extremely mobile in modern financial markets, this ebb and flow of money across international borders and currencies generates a huge and growing volume of FX transactions.

Regardless of the underlying motivation for the FX transaction, it will eventually require that one currency be exchanged for another in the FX market. In advance of that required transaction, market participants are exposed to the risk that the exchange rate will move against them. Often they will try to reduce (hedge) this risk through a variety of FX instruments (described in more detail later). Conversely, market participants may form opinions about future FX movements and undertake speculative FX risk exposures through a variety of $F X$ instruments in order to profit from their views.

The distinction between hedging and speculative positions is not always clear cut. For example, consider the case of a corporation selling its products overseas. This creates an FX risk exposure because the revenue from foreign sales will ultimately need to be converted into the corporation's home currency. This risk exposure is typically hedged, and corporate hedging often accounts for large FX flows passing through the market. The amount and timing of foreign revenue, however, are generally hard to predict with precision: They will depend on the pace of foreign sales, the sales prices realized, the pace at which foreign clients pay for their purchases, and so forth. In the face of this uncertainty, the corporate treasury will estimate the timing and amount of foreign revenue and will then hedge a portion of this estimated amount. Many corporate treasuries have hedging targets based on this estimate, but they also have the flexibility to under-hedge or over-hedge based on their opinions about future FX rate movements. In order to judge the effectiveness of these discretionary trades, the performance of the corporate treasury is compared with a benchmark, usually stated in terms of a fixed amount hedged relative to total sales. (For example, the benchmark may be a 100 percent fully hedged position. The profitability of the hedge actually implemented-which, based on the treasury's discretion, can vary above or below 100 percent-is then compared with what would have been achieved with a passive, 100 percent fully hedged position.) Treasury managers' performance is judged based on gains or losses relative to the benchmark, just as an investment fund manager's performance is benchmarked against performance targets.

At the other end of the spectrum between hedging and speculation, consider the archetypical speculative account: a hedge fund. Although it is true that hedge funds will seek out, accept, and manage risk for profit, a hedge fund is, after all, a hedge fund: Strict risk control procedures are critical to the fund's success, especially when leverage is involved. This mixture of speculative and hedging motives is common throughout the FX space as market participants shape their FX exposures to suit their market forecasts, operational mandates, and appetites for risk.

The FX market provides a variety of products that provide the flexibility to meet this varied and complex set of financial goals. Spot transactions involve the exchange of currencies for immediate delivery. For most currencies, this corresponds to " $\mathrm{T}+2$ " delivery, meaning that the exchange of currencies is settled two business days after the trade is agreed to by the two sides of the deal. (One exception is the Canadian dollar, for which spot settlement against the US dollar is on a $\mathrm{T}+1$ basis.) The exchange rate used for these spot transactions is referred to as the spot exchange rate, and it is the exchange rate that most people refer to in their daily lives (for example, this is the exchange rate usually quoted by the financial press, on the evening news, and so forth).

It is important to realize, however, that spot transactions make up only a minority of total daily turnover in the global FX market: The rest is accounted for by trade in outright forward contracts, FX swaps, and FX options. Although these products will be covered in more depth in a subsequent section, and at Level II of the CFA curriculum, we will provide a brief introduction to these products here.

Outright forward contracts (often referred to simply as forwards) are agreements to deliver foreign exchange at a future date at an exchange rate agreed upon today. For example, suppose that a UK-based company expects to receive a payment of 100 million euros in 85 days. Although it could convert these euros to British pounds with a spot transaction (the spot rate would be the GBP/EUR rate in 83 days, because of $T+2$ settlement), this future spot rate is currently unknown and represents a foreign exchange risk to the company. The company can avoid this risk by entering into a transaction with a foreign exchange dealer to sell 100 million euros against the British pound for settlement 85 days from today at a rate-the forward exchange rate-agreed upon today.

As such, forward contracts are any exchange rate transactions that occur with currency settlement longer than the usual $\mathrm{T}+2$ settlement for spot delivery. Each of these contracts requires two specifications: the date at which the currencies are to be exchanged and the exchange rate to be applied on the settlement date. Accordingly, exchange rates for these transactions are called forward exchange rates to distinguish them from spot rates.

Dealers will typically quote forward rates for a variety of standard forward settlement dates (for example, one week, one month, or 90 days) on their dealing screens. In an over-the-counter (OTC) market, however, traders can arrange forward settlement at any future date they agree upon, with the forward exchange rate scaled appropriately for the specific term to settlement. Standard forward settlement dates (such as three months) are defined in terms of the spot settlement date, which is generally $\mathrm{T}$ + 2. For example, if today is 18 October and spot settlement is for 20 October, then a three-month forward settlement would be defined as 20 January of the following year. Note as well that these standard forward settlement dates may not always be good business days: 20 January could be a weekend or a holiday. In that case, the forward settlement date is set to the closest good business day. Traders always confirm the exact forward settlement date when making these types of trades, and the forward rate is scaled by the exact number of days to settlement.

In an OTC market, the size of the forward contracts can also be any size that the two counterparties agree upon. In general, however, liquidity in forward markets declines the longer the term to maturity and the larger the trade size. The concept of the forward exchange rate and exchange hedging is developed further in Section 3.

Although the OTC market accounts for the majority of foreign exchange trades with future (i.e., greater than $\mathrm{T}+$ 2) settlement dates, there is also a deep, liquid market in exchange-traded futures contracts for currencies. Although there are technical differences between futures and forward contracts, the basic concept is the same: The price is set today for settlement on a specified future date. Futures contracts on currencies trade on several exchanges globally, but the majority of volume in exchange-traded currency futures contracts is found on the International Monetary Market (IMM) division of the Chicago Mercantile Exchange (CME). Futures contracts differ from OTC forward contracts in several important ways: They trade on exchanges (such as the CME) rather than OTC; they are only available for fixed contract amounts and fixed settlement dates; the exchanges demand that a fixed amount of collateral be posted against the futures contract trade; and this collateral is marked-to-market daily, with counterparties asked to post further collateral if their positions generate losses. On balance, futures contracts are somewhat less flexible than forward contracts. Nonetheless, they provide deep, liquid markets for deferred delivery with a minimum of counterparty (i.e., default) risk-a proposition that many FX traders find attractive. Accordingly, daily turnover in FX futures contracts is huge. As of 2010, the average daily trading volume of FX futures on the CME alone was estimated to be about USD140 billion, which is almost comparable in size to the interbank volume of spot transactions.

Because forward contracts eventually expire, existing speculative positions or FX hedges that need to be extended must be rolled prior to their settlement dates. This typically involves a spot transaction to offset (settle) the expiring forward contract and a new forward contract to be set at a new, more distant settlement date. The combination of an offsetting spot transaction and a new forward contract is referred to as an FX swap. ${ }^{2}$

An FX swap is best illustrated by an example. Suppose that a trader sells 100 million euros with settlement 95 days from today at a forward exchange rate (USD/EUR) of 1.2000. In 93 days, the forward contract is two days from settlement, specifically the T +2 days to spot settlement. To roll the forward contract, the trader will engage in the following FX swap. First, the trader will need to buy 100 million euros spot, for which $\mathrm{T}+2$ settlement will fall on day 95 , the same day as the settlement of the expiring forward contract. The purchase of the 100 million euros spot will be used to satisfy the delivery of the 100 million euros sold in the expiring forward contract. Because 100 million euros are being both bought and sold on day 95, there is no exchange of euros between counterparties on that day: The amounts net to zero. However, there will be an exchange of US dollars, reflecting the movement in exchange rates between

2 Note that an "FX swap" is not the same as a "currency swap." An FX swap is simply the combination of a spot and a forward FX transaction (i.e., only two settlement dates-spot and forward-are involved). A currency swap is generally used for multiple periods and payments. the date the forward contract was agreed to (day 0 ) and day 93 . Suppose that on day 93 the spot exchange rate for USD/EUR is 1.1900 . This means that the trader will see a cash flow on day 95 of USD1,000,000. This is calculated as follows:

EUR100,000,000 $\times(1.2000-1.1900)=$ USD1,000,000

The trader receives USD1,000,000 from the counterparty because the euro was sold forward to day 95 at a price of 1.2000; it was bought (on day 93) for spot settlement on day 95 at a price of 1.1900 . This price movement in the euro indicates a profit to the trader, but because the euro quantities exchanged on day 95 net to zero $(100,000,000$ euros both bought and sold), this cash flow is realized in US dollars. The second leg of the FX swap is then to initiate a new forward sale of 100 million euros at the USD/ EUR forward exchange rate being quoted on day 93. This renews the forward position (a forward sale of the euro) to a new date.

FX swaps will be dealt with in more detail at Level II in the curriculum. For the purposes of this reading, it is only necessary to understand that (1) an FX swap consists of a simultaneous spot and forward transaction; (2) these swap transactions can extend (roll) an existing forward position to a new future date; and (3) rolling the position forward leads to a cash flow on settlement day. This cash flow can be thought of as a mark-to-market on the forward position. FX swaps are a large component of daily FX market turnover because market participants have to roll over existing speculative or hedging positions as the underlying forward contracts mature in order to extend the hedge or speculative position (otherwise, the position is closed out on the forward settlement date).

One other area where FX swaps are used in FX markets also bears mentioning: They are often used by market participants as a funding source (called swap funding). Consider the case of a UK-based firm that needs to borrow GBP100 million for 90 days, starting 2 days from today. One way to do this is simply to borrow 90 -day money in GBP-denominated funds starting at $\mathrm{T}+2$. An alternative is to borrow in US dollars and exchange these for British pounds in the spot FX market (both with $\mathrm{T}+2$ settlement) and then sell British pounds 90 days forward against the US dollar. (Recall that the maturity of a forward rate contract is defined in terms of the spot settlement date, so the 90-day forward rate would be for settlement in 92 days from today.) The company has the use of GBP100 million for 90 days, starting on $\mathrm{T}+2$, and at the end of this period can pay off the US dollar loan at a known, pre-determined exchange rate (the 90-day forward rate). By engaging in simultaneous spot and forward transactions (i.e., an FX swap), the company has eliminated any FX risk from the foreign borrowing. The all-in financing rate using an FX swap will typically be close to that of domestic borrowing, usually within a few basis points. This near equivalence is enforced by an arbitrage relationship that will be described in Section 3.3. On large borrowing amounts, however, even a small differential can add up to substantial cost savings.

Another way to hedge FX exposures, or implement speculative FX positions, is to use options on currencies. FX options are contracts that, for an upfront premium or fee, give the purchaser the right, but not the obligation, to make an FX transaction at some future date at an exchange rate agreed upon today (when the contract is agreed to). The holder of an FX option will exercise the option only if it is advantageous to do so-that is, if the agreed upon exchange rate for the FX option contract is better than the FX rate available in the market at option expiry. As such, options are extremely flexible tools for managing FX exposures and account for a large percentage of daily turnover in the FX market.

Another concept to bear in mind is that spot, forward, swap, and option products are typically not used in isolation. Most major market participants manage their FX transactions and FX risk exposures through concurrent spot, forward, swap, and option positions. Taken together, these instruments (the building blocks of the FX market) provide an extremely flexible way for market participants to shape their FX risk exposures to match their operational mandate, risk tolerance, and market opinion. Moreover, FX transactions are often made in conjunction with transactions in other financial markets-such as equities, fixed income, and commodities. These markets have a variety of instruments as well, and market participants jointly tailor their overall position simultaneously using the building blocks of the FX market and these other markets.

\section{EXAMPLE 2}
\section{Spot and Forward Exchange Rates}
The investment adviser based in Sydney, Australia, continues her meeting with the local client who has diversified her domestic bond portfolio by adding investments in fixed-rate, long-term bonds denominated in $\mathrm{HKD}$. Given that the client spends most of the year in Australia, she remains concerned about the foreign exchange risk of her foreign investments and asks the adviser how these might be managed. The investment adviser explains the difference between spot and forward exchange rates and their role in determining foreign exchange risk exposures. The investment adviser suggests the following investment strategy to the client: "You can exchange AUD for HKD in the spot exchange market, invest in a risk-free, one-year $\mathrm{HKD}$-denominated zero coupon bond, and use a one-year forward contract for converting the proceeds back into AUD."

\begin{center}
\begin{tabular}{ll}
Spot exchange rate (AUD/HKD) & 0.1714 \\
One-year HKD interest rate & $2.20 \%$ \\
One-year forward exchange rate (AUD/HKD) & 0.1724 \\
\hline
\end{tabular}
\end{center}

\begin{enumerate}
  \item Which of the following statements is most correct? Over a one-year horizon, the exchange rate risk of the client's investment in HKD-denominated bonds is determined by uncertainty over:
A. today's AUD/HKD forward rate.
B. the $\mathrm{AUD} / \mathrm{HKD}$ spot rate one year from now.
C. the AUD/HKD forward rate one year from now.
\end{enumerate}

\section{Solution to 1:}
$B$ is correct. The exchange rate risk (for an unhedged investment) is defined by the uncertainty over future spot rates. In this case, the relevant spot rate is that which would prevail one year from now. Forward rates that would be in effect one year from now would be irrelevant, and the current forward rate is known with certainty.

\begin{enumerate}
  \setcounter{enumi}{1}
  \item To reduce the exchange rate risk of the Hong Kong SAR investment, the client should:
\end{enumerate}

A. sell AUD spot.

B. sell AUD forward.

C. sell HKD forward.

\section{Solution to 2:}
$\mathrm{C}$ is correct. The Australian-based investor owns HKD-denominated bonds, meaning that she is long HKD exposure. To hedge this exposure, she could enter into a forward contract to sell the HKD against the AUD for future de- livery (that is, match a long HKD exposure in the cash market with a short HKD exposure in the derivatives market). The forward rate is established at the time the forward contract is entered into, eliminating any uncertainty about what exchange rate would be used to convert HKD-denominated cash flows back into AUD.

\begin{enumerate}
  \setcounter{enumi}{2}
  \item Over a one-year horizon, the investment proposed by the investment adviser is most likely:
\end{enumerate}

A. risk free.

B. exposed to interest rate risk.

C. exposed to exchange rate risk.

Solution to 3:

$\mathrm{A}$ is correct. The investment is risk free because the investment is based on a risk-free, one-year, zero coupon, HKD-denominated bond-meaning there is no default or reinvestment risk. The investment will mature in one-year at par; there is no interest rate risk. The use of a forward contract to convert the HKD-denominated proceeds back to AUD eliminates any exchange rate risk.

\begin{enumerate}
  \setcounter{enumi}{3}
  \item To set up the investment proposed by the adviser, the client would need to:
\end{enumerate}

A. sell AUD spot; sell a one-year, HKD-denominated bond; and buy AUD forward.

B. buy AUD spot; buy a one-year, HKD-denominated bond; and sell AUD forward.

C. sell AUD spot; buy a one-year, HKD-denominated bond; and buy AUD forward.

Solution to 4:

$\mathrm{C}$ is correct. To create the investment, the client needs to convert AUD to HKD in the spot exchange market, invest in (buy) the one-year HKD bond, and sell the HKD forward/buy the AUD forward. Note that this process is directly comparable to the swap financing approach described in this section of the reading.

\begin{enumerate}
  \setcounter{enumi}{4}
  \item The return (in AUD) on the investment proposed by the investment adviser is closest to:
A. 2.00 percent.
B. 3.00 percent.
C. 5.00 percent.
\end{enumerate}

Solution to 5:

$\mathrm{B}$ is correct. Converting one AUD to HKD in the spot market gives the client $(1 / 0.1714)=$ HKD5.83. Investing this for one year leads to $5.83 \times(1.022)$ $=$ HKD5.96. Selling this amount of HKD at the forward rate gives $5.96 \times$ $0.1724=$ AUD1.028 (rounding to three decimal places). This implies an AUD-denominated return of 2.8 percent which rounds up to 3 percent.

\section{MARKET PARTICIPANTS, SIZE AND COMPOSITION}
describe functions of and participants in the foreign exchange
market

We now turn to the counterparties that participate in FX markets. As mentioned previously, there is an extremely diverse range of market participants, ranging in size from multi-billion-dollar investment funds down to individuals trading for their own account (including foreign tourists exchanging currencies at airport kiosks).

To understand the various market participants, it is useful to separate them into broad categories. One broad distinction is between what the market refers to as the buy side and the sell side. The sell side generally consists of large FX trading banks (such as Citigroup, UBS, and Deutsche Bank); the buy side consists of clients who use these banks to undertake FX transactions (i.e., buy FX products) from the sell-side banks.

The buy side can be further broken down into several categories:

\begin{itemize}
  \item Corporate accounts: Corporations of all sizes undertake FX transactions during cross-border purchases and sales of goods and services. Many of their FX flows can also be related to cross-border investment flows-such as international mergers and acquisitions (M\&A) transactions, investment of corporate funds in foreign assets, and foreign currency borrowing.

  \item Real money accounts: These are investment funds managed by insurance companies, mutual funds, pension funds, endowments, exchange-traded funds (ETFs), and other institutional investors. These accounts are referred to as real money because they are usually restricted in their use of leverage or financial derivatives. This distinguishes them from leveraged accounts (discussed next); although, many institutional investors often engage in some form of leverage, either directly through some use of borrowed funds or indirectly using financial derivatives.

  \item Leveraged accounts: This category, often referred to as the professional trading community, consists of hedge funds, proprietary trading shops, commodity trading advisers (CTAs), high-frequency algorithmic traders, and the proprietary trading desks at banks-and indeed, almost any active trading account that accepts and manages FX risk for profit. The professional trading community accounts for a large and growing proportion of daily FX market turnover. These active trading accounts also have a wide diversity of trading styles. Some are macro-hedge funds that take longer term FX positions based on their views of the underlying economic fundamentals of a currency. Others are high-frequency algorithmic traders that use technical trading strategies (such as those based on moving averages or Fibonacci levels) and whose trading cycles and investment horizons are sometimes measured in milliseconds.

  \item Retail accounts: The simplest example of a retail account is the archetypical foreign tourist exchanging currency at an airport kiosk. However, it is important to realize that as electronic trading technology has reduced the barriers to entry into FX markets and the costs of FX trading, there has been a huge surge in speculative trading activity by retail accountsconsisting of individuals trading for their own accounts as well as smaller hedge funds and other active traders. This also includes households using electronic trading technology to move their savings into foreign currencies (this is relatively widespread among households in Japan, for example). It is estimated that retail trading accounts for as much as 10 percent of all spot transactions in some currency pairs and that this proportion is growing.

  \item Governments: Public entities of all types often have FX needs, ranging from relatively small (e.g., maintaining consulates in foreign countries) to large (e.g., military equipment purchases or maintaining overseas military bases). Sometimes these flows are purely transactional-the business simply needs to be done-and sometimes government FX flows reflect, at least in part, the public policy goals of the government. Some government FX business resembles that of investment funds, although sometimes with a public policy mandate as well. In some countries, public sector pension plans and public insurance schemes are run by a branch of the government. One example is the Caisse de dépôt et placement du Québec, which was created by the Québec provincial government in Canada to manage that province's public sector pension plans. The Caisse, as it is called, is a relatively large player in financial markets, with about CAD308 billion of assets under management as of mid-2018. Although it has a mandate to invest these assets for optimal return, it is also called upon to help promote the economic development of Québec. It should be noted that many governments-both at the federal and provincial/state levels-issue debt in foreign currencies; this, too, creates FX flows. Such supranational agencies as the World Bank and the African Development Bank issue debt in a variety of currencies as well.

  \item Central banks: These entities sometimes intervene in FX markets in order to influence either the level or trend in the domestic exchange rate. This often occurs when the central banks judge their domestic currency to be too weak and when the exchange rate has overshot any concept of equilibrium level (e.g., because of a speculative attack) to the degree that the exchange rate no longer reflects underlying economic fundamentals. Alternatively, central banks also intervene when the FX market has become so erratic and dysfunctional that end-users such as corporations can no longer transact necessary FX business. Conversely, sometimes central banks intervene when they believe that their domestic currency has become too strong, to the point that it undercuts that country's export competitiveness. The Bank of Japan intervened against yen strength versus the US dollar in 2004 and again in March 2011 after the massive earthquake and nuclear disaster. Similarly, in 2010, 2013, and again in 2015, the Swiss National Bank intervened against strength in the Swiss franc versus the euro by selling the Swiss franc on the euro-Swiss (CHF/EUR) cross-rate. Central bank reserve managers are also frequent participants in FX markets in order to manage their country's FX reserves. In this context, they act much like real money investment fundsalthough generally with a cautious, conservative mandate to safeguard the value of their country's foreign exchange reserves. The foreign exchange reserves of some countries are enormous, and central bank participation in FX markets can sometimes have a material impact on exchange rates even when these reserve managers are not intervening for public policy purposes. Exhibit 2 provides information on central bank reserve holdings as of the second quarter of $2015 .^{3}$ Total central bank reserve holdings have held steady for several years and as of the first quarter of 2018 were $\$ 11,594$ billion.

\end{itemize}

3 See International Monetary Fund (2018) Currency Composition of Foreign Exchange Reserves (COFER) Tables 1-3. Exhibit 2: Currency Composition of Official Foreign Exchange Reserves, as of 1st Quarter 2015 (USD billion)

Total foreign exchange holdings globally

11,433

$\begin{array}{lr}\text { Held by advanced economies } & 3,946\end{array}$

\begin{center}
\begin{tabular}{lr}
Held by emerging and developing economies & 7,487 \\
\hline
\end{tabular}
\end{center}

$\begin{array}{ll}\text { Percent of global holdings held in the US dollar }{ }^{\mathrm{a}} & 66 \%\end{array}$

a This percentage is calculated using that amount of global currency reserves for which the currency composition can be identified.

Note that the amount of foreign exchange reserves now held by emerging economies comfortably exceeds those held by developed economies. This largely reflects the rapid growth in foreign reserves held by Asian central banks, because these countries typically run large current account surpluses with the United States and other developed economies. Reserve accumulation by energy exporting countries in the Middle East and elsewhere is also a factor. Most of the global currency reserves are held in US dollars; the percentage held in USD is more than twice the portion held in the euro, the second most widely held currency in central bank foreign exchange reserves.

\begin{itemize}
  \item Sovereign wealth funds (SWFs): Many countries with large current account surpluses have diverted some of the resultant international capital flows into SWFs rather than into foreign exchange reserves managed by central banks. Although SWFs are government entities, their mandate is usually more oriented to purely investment purposes rather than public policy purposes. As such, SWFs can be thought of as akin to real money accounts, although some SWFs can employ derivatives or engage in aggressive trading strategies. It is generally understood that SWFs use their resources to help fulfill the public policy mandate of their government owners. The SWFs of many current account surplus countries (such as exporting countries in East Asia or oil-exporting countries) are enormous, and their FX flows can be an important determinant of exchange rate movements in almost all of the major currency pairs.
\end{itemize}

As mentioned, the sell side generally consists of the FX dealing banks that sell FX products to the buy side. Even here, however, distinctions can be made.

\begin{itemize}
  \item A large and growing proportion of the daily FX turnover is accounted for by the very largest dealing banks, such as Deutsche Bank, Citigroup, UBS, HSBC, and a few other multinational banking behemoths. Maintaining a competitive advantage in FX requires huge fixed-cost investments in the electronic technology that connects the FX market, and it also requires a broad, global client base. As a result, only the largest banks are able to compete successfully in providing competitive price quotes to clients across the broad range of FX products. In fact, among the largest FX dealing banks, a large proportion of their business is crossed internally, meaning that these banks are able to connect buyers and sellers within their own extremely diverse client base and have no need to show these FX flows outside of the bank.

  \item All other banks fall into the second and third tier of the FX market sell side. Many of these financial institutions are regional or local banks with well-developed business relationships, but they lack the economies of scale, broad global client base, or information technology (IT) expertise required to offer competitive pricing across a wide range of currencies and FX products. In many cases, these are banks in emerging markets that don't have the business connections or credit lines required to access the FX market on a cost-effective basis on their own. As a result, these banks often outsource FX services by forming business relationships with the larger tier-one banks; otherwise, they depend on the deep, competitive liquidity provided by the largest FX market participants.

\end{itemize}

The categories presented are based on functions that are closely associated with the named groups. However, in some cases, functions typifying a group may also be assumed by or shared with another group. For example, sell-side banks provide FX price quotes. However, hedge funds and other large players may access the professional FX market on equal terms with the dealing banks and effectively act as market makers.

One of the most important ideas to draw from this categorization of market participants is that there is an extremely wide variety of FX market participants, reflecting a complex mix of trading motives and strategies that can vary with time. Most market participants reflect a combination of hedging and speculative motives in tailoring their FX risk exposures. Among public sector market participants, public policy motives may also be a factor. The dynamic, complex interaction of FX market participants and their trading objectives makes it difficult to analyze or predict movements in FX rates with any precision, or to describe the FX market adequately with simple characterizations.

\section{Market Size and Composition}
In this section, we present a descriptive overview of the global FX market drawn from the 2016 Triennial Survey undertaken by the Bank for International Settlements (BIS). The BIS is an umbrella organization for the world's central banks. Every three years, participating central banks undertake a survey of the FX market in their jurisdictions, the results of which are aggregated and compiled at the BIS. The most recent survey, taken in April 2016, gives a broad indication of the current size and distribution of global FX market flows.

As of April 2016, the BIS estimates that average daily turnover in the traditional FX market (comprised of spot, outright forward, and FX swap transactions) totaled approximately USD5.1 trillion. Exhibit 3 shows the approximate percentage allocation among FX product types, including both traditional FX products and exchange-traded FX derivatives. Note that this table of percentage allocations adds exchange-traded derivatives to the BIS estimate of average daily turnover of USD5.1 trillion; the "Spot" and "Outright forwards" categories include only transactions that are not executed as part of a swap transaction.

\section{Exhibit 3: FX Turnover by Instrument}
\begin{center}
\begin{tabular}{lc}
\hline
Spot & $33 \%$ \\
Outright forwards & 14 \\
Swaps $^{a}$ & 49 \\
FX options & 5 \\
Total & $\mathbf{1 0 0 \%}$ \\
\hline
\end{tabular}
\end{center}

a Includes both FX and currency swaps.

The survey also provides a percentage breakdown of the average daily flows between sell-side banks (called the interbank market), between banks and financial customers (all non-bank financial entities, such as real money and leveraged accounts, SWFs, and central banks), and between banks and non-financial customers (such as corporations, retail accounts, and governments). The breakdown is provided in Exhibit 4 . It bears noting that the proportion of average daily FX flow accounted for by financial clients is much larger than that for non-financial clients. The BIS also reports that the proportion of financial client flows has been growing rapidly, and in 2010 it exceeded interbank trading volume for the first time. This underscores the fact that only a minority of the daily FX flow is accounted for by corporations and individuals buying and selling foreign goods and services. Huge investment pools and professional traders are accounting for a large and growing proportion of the FX business.

\section{Exhibit 4: FX Flows by Counterparty}
\begin{center}
\begin{tabular}{lc}
Interbank & $42 \%$ \\
Financial clients & 51 \\
Non-financial clients & 8 \\
\hline
\end{tabular}
\end{center}

The 2016 BIS survey also identifies the top five currency pairs in terms of their percentage share of average daily global FX turnover. These are shown in Exhibit 5. Note that each of these most active pairs includes the US dollar (USD).

\section{Exhibit 5: FX Turnover by Currency Pair}
\begin{center}
\begin{tabular}{lc}
\hline
Currency Pair & $\%$ of Market \\
\hline
USD/EUR & $23.1 \%$ \\
JPY/USD & $17.8 \%$ \\
USD/GBP & $9.3 \%$ \\
USD/AUD & $5.2 \%$ \\
CAD/USD & $4.3 \%$ \\
\hline
\end{tabular}
\end{center}

The largest proportion of global FX trading occurs in London, followed by New York. This means that FX markets are most active between approximately 8:00 a.m. and 11:30 a.m. New York time, when banks in both cities are open. (The official London close is at 11:00 a.m. New York time, but London markets remain relatively active for a period after that.) Tokyo is the third-largest FX trading hub.

\section{EXAMPLE 3}
\section{Market Participants and Composition of Trades}
The investment adviser based in Sydney, Australia, makes the following statements to her client when describing some of the basic characteristics of the foreign exchange market:

Statement 1 "Foreign exchange transactions for spot settlement see the most trade volume in terms of average daily turnover because the FX market is primarily focused on settling international trade flows."

Statement 2 "The most important foreign exchange market participants on the buy side are corporations engaged in international trade; on the sell side they are the local banks that service their FX needs." 1. Statement 1 is:

A. correct.

B. incorrect with respect to the importance of spot settlements.

C. incorrect both with respect to the importance of spot settlements and international trade flows.

\section{Solution to 1:}
$\mathrm{C}$ is correct. Although the media generally focus on the spot market when discussing foreign exchange, the majority of average daily trade volume involves the FX swap market as market participants either roll over or modify their existing hedging and speculative positions (or engage in FX swap financing). Although it is true that all international trade transactions eventually result in some form of spot settlement, this typically generates a great deal of hedging (and speculative) activity in advance of spot settlement. Moreover, an important group of FX market participants engages in purely speculative positioning with no intention of ever delivering/receiving the principal amount of the trades. Most FX trading volume is not related to international trade: Portfolio flows (cross-border capital movements) and speculative activities dominate.

\begin{enumerate}
  \setcounter{enumi}{1}
  \item Statement 2 is:
\end{enumerate}

A. correct.

B. incorrect with respect to corporations engaged in international trade.

C. incorrect with respect to both corporations and the local banks that service their trade needs.

\section{Solution to 2:}
$\mathrm{C}$ is correct. As of 2016, the most important foreign exchange market participants in terms of average daily turnover are found not among corporations engaged in international trade but among huge investment managers, both private (e.g., pension funds) and public (e.g., central bank reserve managers or sovereign wealth funds). A large and growing amount of daily turnover is also being generated by high-frequency traders who use computer algorithms to automatically execute extremely high numbers of speculative trades (although their individual ticket sizes are generally small, they add up to large aggregate flows). On the sell side, the largest money center banks (e.g., Deutsche Bank, Citigroup, HSBC, UBS) are increasingly dominating the amount of trading activity routed through dealers. Regional and local banks are increasingly being marginalized in terms of their share of average daily turnover in FX markets.

\section{EXCHANGE RATE QUOTATIONS}
calculate and interpret the percentage change in a currency relative to another currency

\section{Exchange Rate Quotations}
Exchange rates represent the relative price of one currency in terms of another. This price can be represented in two ways: 1 ) currency A buys how many units of currency B; or 2) currency B buys how many units of currency A. Of course, these two prices are simply the inverse of each other.

To distinguish between these two prices, market participants sometimes distinguish between direct and indirect exchange rates. In the quoting convention $\mathrm{A} / \mathrm{B}$ (where there is a certain number of units of currency A per one unit of currency B), we refer to currency $A$ as the price currency (or quote currency); currency B is referred to as the base currency. (The reason for this choice of names will become clearer below.) The base currency is always set at a quantity of one. A direct currency quote takes the domestic country as the price currency and the foreign country as the base currency. For example, for a Paris-based trader, the domestic currency would be the euro (EUR) and a foreign currency would be the UK pound (GBP). For this Paris-based trader, a direct quote would be EUR/GBP. An exchange rate quote of EUR/GBP = 1.1211 means that 1 GBP costs 1.1211 EUR. For this Paris-based trader, an indirect quote has the domestic currency-the euro-as the base currency. An indirect quote of GBP/EUR $=0.8920$ means that 1 EUR costs 0.8920 GBP. Direct and indirect quotes are just the inverse (reciprocal) of each other.

It can be confusing to describe exchange rates as either being direct or indirect because determining the domestic currency and the foreign currency depends on where one is located. For a London-based market participant, the UK pound (GBP) is the domestic currency and the euro (EUR) is a foreign currency. For a Paris-based market participant, it would be the other way around.

To avoid confusion, the professional FX market has developed a set of market conventions that all market participants typically adhere to when making and asking for FX quotes. Exhibit 6 displays some of these for the major currencies: the currency code used for obtaining exchange rate quotes, how the market lingo refers to this currency pair, and the actual ratio-price currency per unit of base currencyrepresented by the quote.

\section{Exhibit 6: Exchange Rate Quote Conventions}
\begin{center}
\begin{tabular}{lll}
\hline
$\begin{array}{l}\text { FX Rate Quote } \\ \text { Convention }\end{array}$ & Name Convention & $\begin{array}{l}\text { Actual Ratio } \\ \text { (Price currency/Base currency) }\end{array}$ \\
\hline
EUR & Euro & USD/EUR \\
JPY & Dollar-yen & JPY/USD \\
GBP & Sterling & USD/GBP \\
CAD & Dollar-Canada & CAD/USD \\
AUD & Aussie & USD/AUD \\
NZD & Kiwi & USD/NZD \\
CHF & Swiss franc & CHF/USD \\
EURJPY & Euro-yen & JPY/EUR \\
EURGBP & Euro-sterling & GBP/EUR \\
EURCHF & Euro-Swiss & CHF/EUR \\
GBPJPY & Sterling-yen & JPY/GBP \\
EURCAD & Euro-Canada & CAD/EUR \\
\end{tabular}
\end{center}

\begin{center}
\begin{tabular}{lll}
\hline
$\begin{array}{l}\text { FX Rate Quote } \\ \text { Convention }\end{array}$ & Name Convention & $\begin{array}{l}\text { Actual Ratio } \\ \text { (Price currency/Base currency) }\end{array}$ \\
\hline
CADJPY & Canada-yen & JPY/CAD \\
\hline
\end{tabular}
\end{center}

Several things should be noted in this exhibit. First, the three-letter currency codes in the first column (for FX rate quotes) refer to what are considered the major exchange rates. Remember that an exchange rate is the price of one currency in terms of another: There are always two currencies involved in the price. This is different from referring to a single currency in its own right. For example, one can refer to the euro (EUR) as a currency; but if we refer to a euro exchange rate (EUR), it is always the price of the euro in terms of another currency, in this case the US dollar. This is because in the professional FX market, the three-letter code EUR is always taken to refer to the euro-US dollar exchange rate, which is quoted in terms of the number of US dollars per euro (USD/EUR). Second, where there are six-letter currency codes in the first column, these refer to some of the major cross-rates. This topic will be covered in the next section, but generally these are secondary exchange rates and they are not as common as the main exchange rates. (It can be noted that three-letter codes are always in terms of an exchange rate involving the US dollar, while the six-letter codes are not.) Third, when both currencies are mentioned in the code or the name convention, the base currency is always mentioned first, the opposite order of the actual ratio (price currency/base currency). Thus, the code for "Sterling-yen" is "GBPJPY," but the actual number quoted is the number of yen per sterling (JPY/GBP). It should also be noted that the codes may appear in a variety offormats that all mean the same thing. For example, GBPJPY might instead appear as GBP:JPY or GBP-JPY. Fourth, regardless of where a market participant is located, there is always a mix of direct and indirect quotes in common market usage. For example, a trader based in Toronto will typically refer to the euro-Canada and Canada-yen exchange rates-a mixture of direct (CAD/EUR) and indirect (JPY/CAD) quotes for a Canadian-based trader. There is no overall consistency in this mixture of direct and indirect quoting conventions in the professional FX market; a market participant just has to get familiar with how the conventions are used. ${ }^{4}$

Another concept involving exchange rate quotes in professional $\mathrm{FX}$ markets is that of a two-sided price. When a client asks a bank for an exchange rate quote, the bank will provide a "bid" (the price at which the bank is willing to buy the currency) and an "offer" (the price at which the bank is willing to sell the currency). But there are two currencies involved in an exchange rate quote, which is always the price of one currency relative to the other. So, which one is being bought and sold in this two-sided price quote? This is where the lingo involving the price currency (or quote currency) and the base currency, explained above, becomes useful. The two-sided price quoted by the dealer is in terms of buying/selling the base currency. It shows the number of units of the price currency that the client will receive from the dealer for one unit of the base currency (the bid) and the number of units of the price currency that the client must sell to the dealer to obtain one unit of the base currency (the offer). Consider the case of a client that is interested in a transaction involving the Swiss franc (CHF) and the euro (EUR). As we have seen above, the market convention is to quote this as euro-Swiss (CHF/EUR). The EUR is the base currency, and the two-sided quote (price) shows the number of units of the price currency (CHF) that must be paid or will be received for 1 euro. For example, a two-sided price in euro-Swiss (CHF/

4 In general, however, there is a hierarchy for quoting conventions. For quotes involving the EUR, it serves as the base currency (e.g., GBP/EUR). Next in the priority sequence, for quotes involving the GBP (but not the EUR) it serves as the base currency (e.g., USD/GBP). Finally, for quotes involving the USD (but not the GBP or EUR) it serves as the base currency (e.g., CAD/USD). Exceptions among the major currencies are the AUD and NZD: they serve as the base currency when quoted against the USD (i.e., USD/AUD, USD/NZD). EUR) might look like: 1.1583-1.1585. The client will receive CHF1.1583 for selling EUR1 to the dealer and must pay CHF1.1585 to the dealer to buy EUR1. Note that the price is shown in terms of the price currency and that the bid is always less than the offer: The bank buys the base currency (EUR, in this case) at the low price and sells the base currency at the high price. Buying low and selling high is profitable for banks, and spreading clients-trying to widen the bid/offer spread-is how dealers try to increase their profit margins. However, it should be noted that the electronic dealing systems currently used in professional FX markets are extremely efficient in connecting buyers and sellers globally. Moreover, this worldwide competition for business has compressed most bid/offer spreads to very tight levels. For simplicity, in the remainder of this reading we will focus on exchange rates as a single number (with no bid/offer spread).

One last thing that can be pointed out in exchange rate quoting conventions is that most major spot exchange rates are typically quoted to four decimal places. One exception among the major currencies involves the yen, for which spot exchange rates are usually quoted to two decimal places. (For example, using spot exchange rates from the middle of 2018, a USD/EUR quote would be expressed as 1.1701, while a JPY/ EUR quote would be expressed as 130.9761.) This difference involving the yen comes from the fact that the units of yen per unit of other currencies is typically relatively large already, and hence extending the exchange rate quote to four decimal places is viewed as unnecessary.

Regardless of what quoting convention is used, changes in an exchange rate can be expressed as a percentage appreciation of one currency against the other: One simply has to be careful in identifying which currency is the price currency and which is the base currency. For example, let's suppose the exchange rate for the euro (USD/ EUR) increases from 1.1500 to 1.2000 . This represents an (un-annualized) percentage change of:

$\frac{1.2000}{1.1500}-1=4.35 \%$

This represents a 4.35 percent appreciation in the euro against the US dollar (and not an appreciation of the US dollar against the euro) because the USD/EUR exchange rate is expressed with the dollar as the price currency.

Note that this appreciation of the euro against the US dollar can also be expressed as a depreciation of the US dollar against the euro; but in this case, the depreciation is not equal to 4.35 percent. Inverting the exchange rate quote from USD/EUR to EUR/ USD, so that the euro is now the price currency, leads to:

$$
\frac{\left(\frac{1}{1.2000}\right)}{\left(\frac{1}{1.1500}\right)}-1=\frac{1.1500}{1.2000}-1=-4.17 \%
$$

Note that the US dollar depreciation is not the same, in percentage terms, as the euro appreciation. This will always be true; it is simply a matter of arithmetic.

\section{EXAMPLE 4}
\section{Exchange Rate Conventions}
A dealer based in New York City provides a spot exchange rate quote of 18.8590

MXN/USD to a client in Mexico City. The inverse of 18.8590 is 0.0530 .

\begin{enumerate}
  \item From the perspective of the Mexican client, the most accurate statement is that the:
A. direct exchange rate quotation is equal to 0.0530 .
B. direct exchange rate quotation is equal to 18.8590 .
\end{enumerate}

\section{C. indirect exchange rate quotation is equal to 18.8590 .}
\section{Solution to 1:}
$\mathrm{B}$ is correct. A direct exchange rate uses the domestic currency as the price currency and the foreign currency as the base currency. For an MXN/USD quote, the $\mathrm{MXN}$ is the price currency; therefore, the direct quote for the Mexican client is 18.8590 (it costs 18.8590 pesos to purchase 1 US dollar). Another way of understanding a direct exchange rate quote is that it is the price of one unit of foreign currency in terms of your own currency. This purchase of a unit of foreign currency can be thought of as a purchase much like any other you might make; think of the unit of foreign currency as just another item that you might be purchasing with your domestic currency. For example, for someone based in Canada, a liter of milk currently costs about CAD1.25 and USD1 costs about CAD1.30. This direct currency quote uses the domestic currency (the Canadian dollar, in this case) as the price currency and simply gives the price of a unit of foreign currency that is being purchased.

\begin{enumerate}
  \setcounter{enumi}{1}
  \item If the bid/offer quote from the dealer was $18.8580 \sim 18.8600$ MXN/USD, then the bid/offer quote in USD/MXN terms would be closest to:
A. $0.05302 \sim 0.05303$.
B. $0.05303 \sim 0.05302$.
C. $0.053025 \sim 0.053025$.
\end{enumerate}

\section{Solution to 2:}
A is correct. An MXN/USD quote means the amount of MXN the dealer is bidding (offering) to buy (sell) USD1. The dealer's bid to buy USD1 at MXN18.8580 is equivalent to the dealer paying MXN18.8580 to buy USD1. Dividing both terms by 18.8580 means the dealer is paying (i.e., selling) MXN1 to buy USD0.05303. This is the offer in USD/MXN terms: The dealer offers to sell MXN1 at a price of USD0.05303. In USD/MXN terms, the dealer's bid for MXN1 is 0.05302 , calculated by inverting the offer of 18.8600 in MXN/USD terms $(1 / 18.8600=0.05302)$. Note that in any bid/offer quote, no matter what the base or price currencies, the bid is always lower than the offer.

\section{CROSS-RATE CALCULATIONS}
calculate and interpret currency cross-rates

Given two exchange rates involving three currencies, it is possible to back out what the cross-rate must be. For example, as we have seen, the FX market convention is to quote the exchange rate between the US dollar and the euro as euro-dollar (USD/ EUR). The FX market also quotes the exchange rate between the Canadian dollar and US dollar as dollar-Canada (CAD/USD). Given these two exchange rates, it is possible to back out the cross-rate between the euro and the Canadian dollar, which according to market convention is quoted as euro-Canada (CAD/EUR). This calculation is shown as:

$$
\frac{\mathrm{CAD}}{\mathrm{USD}} \times \frac{\mathrm{USD}}{\mathrm{EUR}}=\frac{\mathrm{CAD}}{\mathrm{USD}} \times \frac{\mathrm{US}}{\mathrm{EUR}}=\frac{\mathrm{CAD}}{\mathrm{EUR}}
$$

Hence, to get a euro-Canada (CAD/EUR) quote, we must multiply the dollar-Canada (CAD/USD) quote by the euro-dollar (USD/EUR) quote. For example, assume the exchange rate for dollar-Canada is 1.3020 and the exchange rate for euro-dollar is 1.1701. Using these sample spot exchange rates, calculating the euro-Canada cross-rate equals:

$$
1.3020 \times 1.1701=1.5235 \mathrm{CAD} \text { per EUR }
$$

It is best to avoid talking in terms of direct or indirect quotes because, as noted above, these conventions depend on where one is located and hence what the domestic and foreign currencies are. Instead, focus on how the math works: Sometimes it is necessary to invert one of the quotes in order to get the intermediary currency to cancel out in the equation to get the cross-rate. For example, to get a Canada-yen (JPY/CAD) quote, one is typically using the dollar-Canada (CAD/USD) rate and dollar-yen (JPY/ USD) rate, which are the market conventions. This Canada-yen calculation requires that the dollar-Canada rate (CAD/USD) be inverted to a USD/CAD quote for the calculations to work, as shown below:

$$
\left(\frac{\mathrm{CAD}}{\mathrm{USD}}\right)^{-1} \times \frac{\mathrm{JPY}}{\mathrm{USD}}=\frac{\mathrm{USD}}{\mathrm{CAD}} \times \frac{\mathrm{JPY}}{\mathrm{USD}}=\frac{\mathrm{US}}{\mathrm{CAD}} \times \frac{\mathrm{JPY}}{\mathrm{USD}}=\frac{\mathrm{JPY}}{\mathrm{CAD}}
$$

Hence, to get a Canada-yen (JPY/CAD) quote, we must first invert the dollar-Canada (CAD/USD) quote before multiplying by the dollar-yen (JPY/USD) quote. As an example, let's assume that we have spot exchange rates of 1.3020 for dollar-Canada (CAD/USD) and 111.94 for dollar-yen (JPY/USD). The dollar-Canada rate of 1.3020 inverts to 0.7680 ; multiplying this value by the dollar-yen quote of 111.94 gives a Canada-yen quote of:

$$
0.7680 \times 111.94=85.97 \mathrm{JPY} \text { per CAD }
$$

Market participants asking for a quote in a cross-rate currency pair typically will not have to do this calculation themselves: Either the dealer or the electronic trading platform will provide a quote in the specified currency pair. (For example, a client asking for a quote in Canada-yen will receive that quote from the dealer; he will not be given separate dollar-Canada and dollar-yen quotes in order to do the math.) But be aware that dealers providing the quotes often have to do this calculation themselves if only because the dollar-Canada and dollar-yen currency pairs often trade on different trading desks and involve different traders. Electronic dealing machines used in both the interbank market and bank-to-client markets often provide this mathematical operation to calculate cross-rates automatically.

Because market participants can receive both a cross-rate quote (for example, Canada-yen) as well as the component underlying exchange rate quotes (for example, dollar-Canada and dollar-yen), these cross-rate quotes must be consistent with the above equation; otherwise, the market will arbitrage the mispricing. Extending our example above, we calculate a Canada-yen (JPY/CAD) rate of 85.97 based on underlying dollar-Canada (CAD/USD) and dollar-yen (JPY/USD) rates of 1.3020 and 111.94, respectively. Now suppose that at the same time a misguided dealer quotes a Canada-yen rate of 86.20. This is a different price in JPY/CAD for an identical service: converting yen into Canadian dollars. Hence, any trader could buy CAD1 at the lower price of JPY85.97and then turn around and sell CAD1 at JPY86.20(recall our earlier discussion of how price and base currencies are defined). The riskless arbitrage profit is JPY0.23 per CAD1. The arbitrage-called triangular arbitrage, "tri-," because it involves three currencies-would continue until the price discrepancy was removed.

In reality, however, these discrepancies in cross-rates almost never occur because both human traders and automatic trading algorithms are constantly on alert for any pricing inefficiencies. In practice, and for the purposes of this reading, we can consider cross-rates as being consistent with their underlying exchange rate quotes and that given any two exchange rates involving three currencies, we can back out the third cross-rate.

\section{EXAMPLE 5}
\section{Cross Exchange Rates and Percentage Changes}
A research report produced by a dealer includes the following exhibit:

\begin{center}
\begin{tabular}{lcc}
\hline
 & Spot Rate & $\begin{array}{c}\text { Expected Spot } \\ \text { Rate in One Year }\end{array}$ \\
\hline
USD/EUR & 1.1701 & 1.1619 \\
CHF/USD & 0.9900 & 0.9866 \\
USD/GBP & 1.3118 & 1.3066 \\
\hline
\end{tabular}
\end{center}

\begin{enumerate}
  \item The spot CHF/EUR cross-rate is closest to:
A. 0.8461 .
B. 0.8546 .
C. 1.1584 .
\end{enumerate}

Solution to 1:

$\mathrm{C}$ is correct:

$$
\frac{\mathrm{CHF}}{\mathrm{EUR}}=\frac{\mathrm{USD}}{\mathrm{EUR}} \times \frac{\mathrm{CHF}}{\mathrm{USD}}=1.1701 \times 0.9900=1.1584
$$

\begin{enumerate}
  \setcounter{enumi}{1}
  \item The spot GBP/EUR cross-rate is closest to:
A. 0.8920 .
B. 1.1211
C. 1.4653
\end{enumerate}

\section{Solution to 2:}
A is correct:

$$
\frac{\mathrm{GBP}}{\mathrm{EUR}}=\frac{\mathrm{USD}}{\mathrm{EUR}} \times\left(\frac{\mathrm{USD}}{\mathrm{GBP}}\right)^{-1}=\frac{\mathrm{USD}}{\mathrm{EUR}} \times \frac{\mathrm{GBP}}{\mathrm{USD}}=\frac{1.1701}{1.3118}=0.8920
$$

\begin{enumerate}
  \setcounter{enumi}{2}
  \item Based on the exhibit, the euro is expected to appreciate by how much against the US dollar over the next year?
A. -0.7 percent
B. +0.7 percent
C. +1.0 percent
\end{enumerate}

\section{Solution to 3:}
A is correct. The euro is the base currency in the USD/EUR quote, and the expected decrease in the USD/EUR rate indicates that the EUR is depreciating (in one year it will cost less USD to buy one EUR). Mathematically:

$$
\frac{1.1619}{1.1701}-1=-0.7 \%
$$

\begin{enumerate}
  \setcounter{enumi}{3}
  \item Based on the exhibit, the US dollar is expected to appreciate by how much against the British pound over the next year?
A. +0.6 percent
B. -0.4 percent
C. +0.4 percent
\end{enumerate}

\section{Solution to 4:}
$\mathrm{C}$ is correct. The GBP is the base currency in the USD/GBP quote, and the expected decrease in the USD/GBP rate means that the GBP is expected to depreciate against the USD. Or equivalently, the USD is expected to appreciate against the GBP. Mathematically:

$$
\left(\frac{1.3066}{1.3118}\right)^{-1}-1=\frac{1.3118}{1.3066}-1=+0.4 \%
$$

\begin{enumerate}
  \setcounter{enumi}{4}
  \item Over the next year, the Swiss franc is expected to:
\end{enumerate}

A. depreciate against the GBP.

B. depreciate against the EUR.

C. appreciate against the GBP, EUR, and USD.

\section{Solution to 5:}
$\mathrm{C}$ is correct: Because the question does not require calculating the magnitude of the appreciation or depreciation, we can work with CHF as either the price currency or the base currency. In this case, it is easiest to use it as the price currency. According to the table, CHF/USD is expected to decline from 0.9900 to 0.9866 , so CHF is expected to be stronger (i.e., it should appreciate against the USD). CHF/EUR is currently 1.1584 (see the solution to problem 1) and is expected to be $1.1463(=0.9866 \times 1.1619)$, so CHF is expected to appreciate against the EUR. CHF/GBP is currently 1.2987 (= $0.9900 \times 1.3118)$ and is expected to be $1.2891(=0.9866 \times 1.3066)$, so $\mathrm{CHF}$ is also expected to appreciate against the GBP.

Alternatively, we can derive this answer intuitively. The table shows that the CHF/USD rate is expected to decline: That is, the USD is expected to depreciate against the CHF, or alternatively, the CHF is expected to appreciate against the USD. The table also shows that the USD/EUR and USD/GBP rates are also decreasing, meaning that the EUR and GBP are expected to depreciate against the USD, or alternatively, the USD is expected to appreciate against the EUR and GBP. If the CHF is expected to appreciate against the USD and the USD is expected to appreciate against both the EUR and GBP, it follows that the CHF is expected to appreciate against both the EUR and GBP.

\begin{enumerate}
  \setcounter{enumi}{5}
  \item Based on the exhibit, which of the following lists the three currencies from strongest to weakest over the next year?
A. USD, GBP, EUR
B. USD, EUR, GBP
C. EUR, USD, GBP
\end{enumerate}

\section{Solution to 6:}
A is correct. According to the table, USD/EUR is expected to decline from 1.1701 to 1.1619, while USD/GBP is expected to decline from 1.3118 to 1.3066. So, the USD is expected to be stronger than both the EUR and GBP. GBP/EUR is currently $0.8920\left[=(1.3118)^{-1} \times 1.1701\right]$ and is expected to be $0.8893\left[=(1.3066)^{-1} \times 1.1619\right]$, so the GBP is expected to be stronger than the EUR.

\begin{enumerate}
  \setcounter{enumi}{6}
  \item Based on the exhibit, which of the following lists the three currencies in order of appreciating the most to appreciating the least (in percentage terms) against the USD over the next year?
\end{enumerate}

A. GBP, CHF, EUR

B. CHF, GBP, EUR

C. EUR, CHF, GBP

\section{Solution to 7:}
B is correct. The USD/EUR rate depreciates by -0.7 percent $(=$ $[1.1619 / 1.1701]-1)$, which is the depreciation of the base currency EUR against the USD. The USD/GBP rate declines -0.4 percent (= [1.3066/1.3118] - 1), which is the depreciation of the GBP against the USD. Inverting the CHF/USD rate to a USD/CHF convention shows that the base currency CHF appreciates by +0.35 percent against the USD $(=$ $[1.0136 / 1.0101]-1)$.

\section{FORWARD CALCULATIONS}
calculate an outright forward quotation from forward quotations expressed on a points basis or in percentage terms

explain the arbitrage relationship between spot rates, forward rates, and interest rates

calculate and interpret a forward discount or premium

calculate and interpret the forward rate consistent with the spot rate and the interest rate in each currency

In professional FX markets, forward exchange rates are typically quoted in terms of points (also sometimes referred to as "pips"). The points on a forward rate quote are simply the difference between the forward exchange rate quote and the spot exchange rate quote, with the points scaled so that they can be related to the last decimal in the spot quote. When the forward rate is higher than the spot rate, the points are positive and the base currency is said to be trading at a forward premium. Conversely, if the forward rate is less than the spot rate, the points (forward rate minus spot rate) are negative and the base currency is said to be trading at a forward discount. Of course, if the base currency is trading at a forward premium, then the price currency is trading at a forward discount, and vice versa. This can best be explained by means of an example. Mid-2018, the spot euro-dollar exchange rate (USD/EUR) was 1.15885 and the one-year forward rate was 1.19532 . Hence, the forward rate was trading at a premium to the spot rate (the forward rate was larger than the spot rate) and the one-year forward points were quoted as +364.7 . This +364.7 comes from:

$$
1.19532-1.15885=+0.03647
$$

Recall that most non-yen exchange rates are quoted to four decimal places, so in this case we would scale up by four decimal places (multiply by 10,000) so that this +0.03647 would be represented as +364.7 points. Notice that the points are scaled to the size of the last digit in the spot exchange rate quote-usually the fourth decimal place. Notice as well that points are typically quoted to one (or more) decimal places, meaning that the forward rate will typically be quoted to five or more decimal places. The exception among the major currencies is the yen, which is typically quoted to two decimal places for spot rates. Here, forward points are scaled up by two decimal places-the last digit in the spot rate quote-by multiplying the difference between forward and spot rates by 100 .

Typically, quotes for forward rates are shown as the number of forward points at each maturity. 5 These forward points are also called swap points because an FX swap consists of simultaneous spot and forward transactions. In the middle of 2018, a trader would have faced a spot rate and forward points in the euro-dollar (USD/ EUR) currency pair similar to those in Exhibit 7:

\section{Exhibit 7: Sample Spot and Forward Quotes}
\begin{center}
\begin{tabular}{lc}
\hline
Maturity & Spot Rate or Forward Points \\
\hline
Spot & 1.15885 \\
One week & +5.6 \\
One month & +27.1 \\
Three months & +80.9 \\
Six months & +175.6 \\
Twelve months & +364.7 \\
\hline
\end{tabular}
\end{center}

Notice that the absolute number of points generally increases with maturity. This is because the number of points is proportional to the yield differential between the two countries (the Eurozone and the United States, in this case) scaled by the term to maturity. Given the interest rate differential, the longer the term to maturity, the greater the absolute number of forward points. Similarly, given the term to maturity, a wider interest rate differential implies a greater absolute number of forward points. (This will be explained and demonstrated in more detail later in this section.)

To convert any of these quoted forward points into a forward rate, one would divide the number of points by 10,000 (to scale down to the fourth decimal place, the last decimal place in the spot quote) and then add the result to the spot exchange rate quote. ${ }^{6}$ For example, using the data in Exhibit 7 , the three-month forward rate in this case would be:

$$
1.15885+\left(\frac{+80.9}{10,000}\right)=1.15885+0.00809=1.16694
$$

5 As mentioned earlier, "maturity" is defined in terms of the time between spot settlement (usually $\mathrm{T}+2$ ) and the settlement of the forward contract.

6 Because the JPY/USD exchange rate is only quoted to two decimal places, forward points for the dollaryen currency pair are divided by 100. Occasionally, one will see the forward rate or forward points represented as a percentage of the spot rate rather than as an absolute number of points. Continuing our example from above, the three-month forward rate for USD/EUR can be represented as:

$$
\frac{1.15885+0.00809}{1.15885}-1=\left(\frac{1.16694}{1.15885}\right)-1=+0.698 \%
$$

This shows that either the forward rate or the forward points can be used to calculate the percentage discount (or premium) in the forward market-in this case, +0.698 percent rounding to three decimal places. To convert a spot quote into a forward quote when the points are shown as a percentage, one simply multiples the spot rate by one plus the percentage premium or discount:

$$
1.15885 \times(1+0.698 \%)=1.15885 \times(1.0000+0.00698) \approx 1.16694
$$

Note that, rounded to the fifth decimal place, this is equal to our previous calculation. However, it is typically the case in professional FX markets that forward rates will be quoted in terms of pips rather than percentages.

We now turn to the relationship between spot rates, forward rates, and interest rates and how their relationship is derived. Forward exchange rates are based on an arbitrage relationship that equates the investment return on two alternative but equivalent investments. Consider the case of an investor with funds to invest. For simplicity, we will assume that there is one unit of the investor's domestic currency to be invested for one period. One alternative is to invest for one period at the domestic risk-free rate $\left(i_{d}\right)$; at the end of the period, the amount of funds held is equal to $\left(1+i_{d}\right)$. An alternative investment is to convert this one unit of domestic currency to foreign currency using the spot rate of $S_{f / d}$ (number of units of foreign currency per one unit of domestic currency). This can be invested for one period at the foreign risk-free rate; at the end of the period, the investor would have $S_{f / d}\left(1+i_{f}\right)$ units of foreign currency. These funds must then be converted back to the investor's domestic currency. If the exchange rate to be used for this end-of-period conversion was pre-contracted at the start of the period (i.e., a forward rate was used), it would eliminate any foreign exchange risk from converting at a future, unknown spot rate. Given the assumed exchange rate convention here (foreign/domestic), the investor would obtain $\left(1 / F_{f / d}\right)$ units of the domestic currency for each unit of foreign currency sold forward. Note that this process of converting domestic funds in the spot FX market, investing at the foreign risk-free rate, and then converting back to the domestic currency with a forward rate is identical to the concept of swap financing described in an earlier section of this reading.

Hence, we have two alternative investments-both risk-free because both are invested at risk-free interest rates and because any foreign exchange risk was eliminated (hedged) by using a forward rate. Because these two investments are equal in risk characteristics, they must have the same return. Bearing in mind that the currency quoting convention is the number of foreign currency units per single domestic unit (f/d), this relationship can be stated as:

$$
\left(1+i_{d}\right)=S_{f f d}\left(1+i_{f}\right)\left(\frac{1}{F_{f f d}}\right)
$$

This is an arbitrage relationship because it describes two alternative investments (one on either side of the equal sign) that should have equal returns. If they do not, a riskless arbitrage opportunity exists because an investor can sell short the investment with the lower return and invest the funds in the investment with the higher return; the difference between the two returns is pure profit. ${ }^{7}$

7 It is because of this arbitrage relationship that the all-in financing rate using swap financing is close to the domestic interest rate. This formula is perhaps the easiest and most intuitive way to remember the formula for the forward rate because it is based directly on the underlying intuition (the arbitrage relationship of two alternative but equivalent investments, one on either side of the equal sign). Also, the right-hand side of the equation, for the hedged foreign investment alternative, is arranged in proper time sequence: a) convert domestic to foreign currency; then b) invest the foreign currency at the foreign interest rate; and finally c) convert the foreign currency back to the domestic currency. ${ }^{8}$

This arbitrage equation can be re-arranged as needs require. For example, to get the formula for the forward rate, the above equation can be restated as:

$$
F_{f l d}=S_{f l d}\left(\frac{1+i_{f}}{1+i_{d}}\right)
$$

Another way of looking at this is, given the spot exchange rate and the domestic and foreign risk-free interest rates, the forward rate is whatever value completes this equation and eliminates any arbitrage opportunity. For example, let's assume that the spot exchange rate $\left(S_{f / d}\right)$ is 1.6535 , the domestic 12 -month risk-free rate is 3.50 percent, and the foreign 12-month risk-free rate is 5.00 percent. The 12-month forward rate $\left(F_{f / d}\right)$ must then be equal to:

$$
1.6535\left(\frac{1.0500}{1.0350}\right)=1.6775
$$

Suppose instead that, with the spot exchange rate and interest rates unchanged, you were given a quote on the 12 -month forward rate $\left(F_{f / d}\right)$ of 1.6900 . Because this misquoted forward rate does not agree with the arbitrage equation, it would present a riskless arbitrage opportunity. This can be seen by using the arbitrage equation to compute the return on the two alternative investment strategies. The return on the domestic-only investment approach is the domestic risk-free rate (3.50 percent). In contrast, the return on the hedged foreign investment when this misquoted forward rate is put into the arbitrage equation equals:

$$
S_{f l d}\left(1+i_{f}\right)\left(\frac{1}{F_{f l d}}\right)=1.6535(1.05)\left(\frac{1}{1.6900}\right)=1.0273
$$

This defines a return of 2.73 percent. Hence, the investor could make riskless arbitrage profits by borrowing at the higher foreign risk-free rate, selling the foreign currency at the spot exchange rate, hedging the currency exposure (buying the foreign currency back) at the misquoted forward rate, investing the funds at the lower domestic risk-free rate, and thereby getting a profit of 77 basis points $(3.50 \%-2.73 \%)$ for each unit of domestic currency involved-all with no upfront commitment of the investor's own capital. Any such opportunity in real-world financial markets would be quickly "arbed" away. It is interesting to note that in this example, the investor actually borrows at the higher of the two interest rates but makes a profit because the foreign currency is underpriced in the forward market.

The underlying arbitrage equation can also be re-arranged to show the forward rate as a percentage of the spot rate:

$$
\frac{F_{f l d}}{S_{f f d}}=\left(\frac{1+i_{f}}{1+i_{d}}\right)
$$

8 Recall that this equation is based on an $f / d$ exchange rate quoting convention. If the exchange rate data were presented in $d / f$ form, one could either invert these quotes back to $f / d$ form and use the above equation or use the following equivalent equation: $\left(1+i_{d}\right)=\left(1 / S_{d / f}\right)\left(1+i_{f}\right) F_{d / f}$. If this latter equation were used, remember that forward and spot exchange rates are now being quoted on a $d / f$ convention. This shows that, given an $f / d$ quoting convention, the forward rate will be higher than (be at a premium to) the spot rate if foreign interest rates are higher than domestic interest rates. More generally, and regardless of the quoting convention, the currency with the higher (lower) interest rate will always trade at a discount (premium) in the forward market.

One context in which forward rates are quoted as a percentage of spot rates occurs when forward rates are interpreted as expected future spot rates, or:

$$
F_{t}=\widehat{S}_{t+1}
$$

Substituting this expression into the previous equation and doing some re-arranging leads to:

$$
\frac{\hat{S}_{t+1}}{S_{t}}-1=\% \Delta \hat{S}_{t+1}=\left(\frac{i_{f}-i_{d}}{1+i_{d}}\right)
$$

This shows that if forward rates are interpreted as expected future spot rates, the expected percentage change in the spot rate is proportional to the interest rate differential $\left(i_{f}-i_{d}\right)$.

It is intuitively appealing to see forward rates as expected future spot rates. However, this interpretation of forward rates should be used cautiously. First, the direction of the expected change in spot rates is somewhat counter-intuitive. All else being equal, an increase in domestic interest rates (for example, the central bank tightens monetary policy) would typically be expected to lead to an increase in the value of the domestic currency. In contrast, the equation above indicates that, all else equal, a higher domestic interest rate implies slower expected appreciation (or greater expected depreciation) of the domestic currency (recall that this equation is based on an $f / d$ quoting convention).

More important, historical data show that forward rates are poor predictors of future spot rates. Although various econometric studies suggest that forward rates may be unbiased predictors of future spot rates (i.e., they do not systematically over- or under-estimate future spot rates), this is not particularly useful information because the margin of error for these forecasts is so large. As we have seen in our introductory section, the FX market is far too complex and dynamic to be captured by a single variable, such as the level of the yield differential between countries. Moreover, as can be seen in the formula above for the forward rate, forward rates are based on domestic and foreign interest rates. This means that anything that affects the level and shape of the yield curve in either the domestic or foreign market will also affect the relationship between spot and forward exchange rates. In other words, FX markets do not operate in isolation but will reflect almost all factors affecting other markets globally; anything that affects expectations or risk premia in these other markets will reverberate in forward exchange rates as well. Although the level of the yield differential is one factor that the market may look at in forming spot exchange rate expectations, it is only one of many factors. (Many traders look to the trend in the yield differential rather than the level of the differential.) Moreover, there is a lot of noise in FX markets that makes almost any model-no matter how complex-a relatively poor predictor of spot rates at any given point in the future. In practice, FX traders and market strategists do not base either their currency expectations or trading strategies solely on forward rates.

For the purposes of this reading, it is best to understand forward exchange rates simply as a product of the arbitrage equation outlined earlier and forward points as being related to the (time-scaled) interest rate differential between the two countries. Reading any more than that into forward rates or interpreting them as the "market forecast" can be potentially misleading. To understand the relationship between maturity and forward points, we need to generalize our arbitrage formula slightly. Suppose the investment horizon is a fraction, $\tau$, of the period for which the interest rates are quoted. Then the interest earned in the domestic and foreign markets would be $\left(i_{d} \tau\right)$ and $\left(i_{f} \tau\right)$, respectively. Substituting this into our arbitrage relationship and solving for the difference between the forward and spot exchange rates gives:

$$
F_{f l d}-S_{f l d}=S_{f l d}\left(\frac{i_{f}-i_{d}}{1+i_{d} \tau}\right) \tau
$$

This equation shows that forward points (appropriately scaled) are proportional to the spot exchange rate and to the interest rate differential and approximately (but not exactly) proportional to the horizon of the forward contract.

Let's demonstrate this using an example. Suppose that we wanted to determine the 30-day forward exchange rate given a 30-day domestic risk-free interest rate of 2.00 percent per year, a 30-day foreign risk-free interest rate of 3.00 percent per year, and a spot exchange rate $\left(S_{f / d}\right)$ of 1.6555 . The risk-free assets used in this arbitrage relationship are typically bank deposits quoted using the London Interbank Offered Rate (Libor) for the currencies involved. The day count convention for Libor deposits is Actual/360. ${ }^{9}$ Incorporating the fractional period ( $\tau$ ) as above and inserting the data into the forward rate equation leads to a 30-day forward rate of:

$$
F_{f l d}=S_{f / d}\left(\frac{1+i_{f} \tau}{1+i_{d} \tau}\right)=1.6555\left(\frac{1+0.0300\left[\frac{30}{360}\right]}{1+0.0200\left[\frac{30}{360}\right]}\right)=1.6569
$$

This means that, for a 30-day term, forward rates are trading at a premium of 14 pips $(1.6569-1.6555)$. This can also be calculated using the above formula for swap points:

$$
F_{f l d}-S_{f l d}=S_{f l d}\left(\frac{i_{f}-i_{d}}{1+i_{d} \tau}\right) \tau=1.6555\left(\frac{0.0300-0.0200}{1+0.0200\left[\frac{30}{360}\right]}\right)\left[\frac{30}{360}\right]=0.0014
$$

As should be clear from this expression, the absolute number of swap points will be closely related to the term of the forward contract (i.e., approximately proportional to $\mathrm{\tau}=$ Actual/360). For example, leaving the spot exchange rate and interest rates unchanged, let's set the term of the forward contract to 180 days:

$$
F_{f l d}-S_{f l d}=1.6555\left(\frac{0.0300-0.0200}{1+0.0200\left[\frac{180}{360}\right]}\right)\left[\frac{180}{360}\right]=0.0082
$$

This leads to the forward rate trading at a premium of 82 pips. The increase in the number of forward points is approximately proportional to the increase in the term of the contract (from 30 days to 180 days). Note that although the term of the 180-day forward contract is six times longer than that of a 30-day contract, the number of forward points is not exactly six times larger: $6 \times 14=84$.

Similarly, the number of forward points is proportional to the spread between foreign and domestic interest rates $\left(i_{f}-i_{d}\right)$. For example, with reference to the original 30-day forward contract, let's set the foreign interest rate to 4.00 percent leaving the domestic interest rate and spot exchange rate unchanged. This doubles the interest rate differential $\left(i_{f}-i_{d}\right)$ from 1.00 percent to 2.00 percent; it also doubles the forward points (rounding to four decimal places):

$$
F_{f l d}-S_{f l d}=1.6555\left(\frac{0.0400-0.0200}{1+0.0200\left[\frac{30}{360}\right]}\right)\left[\frac{30}{360}\right]=0.0028
$$

9 This means that for interest calculation purposes, it is assumed that there are 360 days in the year However, the actual number of days the funds are on deposit is used to calculate the interest payable.

\section{EXAMPLE 6}
\section{Forward Rates}
A French company has recently finalized a sale of goods to a UK-based client and expects to receive a payment of GBP50 million in 32 days. The corporate treasurer at the French company wants to hedge the foreign exchange risk of this transaction and receives the following exchange rate information from a dealer:

$\begin{array}{ll}\text { GBP/EUR spot rate } & 0.8752\end{array}$

One-month forward points $\quad-1.4$

\begin{enumerate}
  \item Given the above data, the treasurer could hedge the foreign exchange risk by:
\end{enumerate}

A. buying EUR (selling GBP) at a forward rate of 0.87380 .

B. buying EUR (selling GBP) at a forward rate of 0.87506 .

C. selling EUR (buying GBP) at a forward rate of 0.87506 .

\section{Solution to 1:}
B is correct. The French company would want to convert the GBP to its domestic currency, the EUR (it wants to sell GBP, buy EUR). The forward rate would be equal to: $0.8752+(-1.4 / 10,000)=0.87506$.

\begin{enumerate}
  \setcounter{enumi}{1}
  \item The best interpretation of the forward discount shown is that:
\end{enumerate}

A. the euro is expected to depreciate over the next 30 days.

B. one-month UK interest rates are higher than those in the Eurozone.

C. one-month Eurozone interest rates are higher than those in the United Kingdom.

\section{Solution to 2:}
$\mathrm{C}$ is correct. A forward discount indicates that interest rates in the base currency country (France in this case, which uses the euro) are higher than those in the price currency country (the United Kingdom).

\begin{enumerate}
  \setcounter{enumi}{2}
  \item If the 12-month forward rate is 0.87295 GBP/EUR, then based on the data the 12-month forward points are closest to:
A. -22.5 .
B. -2.25 .
C. -0.00225
\end{enumerate}

\section{Solution to 3:}
A is correct. The number of forward points is equal to the scaled difference between the forward rate and the spot rate. In this case: $0.87295-0.87520$ $=-0.00225$. This is then multiplied by 10,000 to convert to the number of forward points.

\begin{enumerate}
  \setcounter{enumi}{3}
  \item If a second dealer quotes GBP/EUR at a 12-month forward discount of 0.30 percent on the same spot rate, the French company could:
\end{enumerate}

A. trade with either dealer because the 12-month forward quotes are equivalent. B. lock in a profit in 12 months by buying EUR from the second dealer and selling it to the original dealer.

C. lock in a profit in 12 months by buying EUR from the original dealer and selling it to the second dealer.

\section{Solution to 4:}
B is correct. A 0.30 percent discount means that the second dealer will sell euros 12 months forward at $0.8752 \times(1-0.0030)=0.87257$, a lower price per euro than the original dealer's quote of 0.87295 . Buying euros at the cheaper 12-month for ward rate (0.87257) and selling the same amount of euros 12 months forward at the higher 12-month forward rate (0.87295) means a profit of $(0.87295-0.87257=$ GBP 0.00038) per euro transacted, receivable when both forward contracts settle in 12 months.

\begin{enumerate}
  \setcounter{enumi}{4}
  \item If the 270-day Libor rates (annualized) for the EUR and GBP are $1.370 \%$ and $1.325 \%$, respectively, and the spot GBP/EUR exchange rate is 0.8489 , then the number of forward points for a 270-day forward rate $\left(F_{G B P / E U R}\right)$ is closest to:
A. -22.8 .
B. -3.8 .
C. -2.8 .
\end{enumerate}

\section{Solution to 5:}
$C$ is correct, because the forward rate is calculated as:

$$
F \frac{G B P}{E U R}=S_{\frac{G B P}{E U R}}\left(\frac{1+i_{G B P}\left[\frac{\text { Actual }}{360}\right]}{1+i_{E U R}\left[\frac{\text { Actual }}{360}\right]}\right)=0.8489\left(\frac{1+0.01325\left[\frac{270}{360}\right]}{1+0.01370\left[\frac{270}{360}\right]}\right)=0.84862
$$

This shows that the forward points are at a discount of: $0.84862-0.84890$ $=-0.00028$, or -2.8 points. This can also be seen using the swap points formula:

$$
F \frac{G B P}{E U R}-S_{\frac{G B P}{E U R}}=0.8489\left(\frac{0.01325-0.01370}{1+0.01370\left[\frac{270}{360}\right]}\right)\left[\frac{270}{360}\right]=-0.00028
$$

The calculation of -3.8 points omits the day count (270/360), and -22.8 points gets the scaling wrong.

\section{EXCHANGE RATE REGIMES- IDEALS AND HISTORICAL PERSPECTIVE}
describe exchange rate regimes

Highly volatile exchange rates create uncertainty that undermines the efficiency of real economic activity and the financial transactions required to facilitate that activity. Exchange rate volatility also has a direct impact on investment decisions because it is a key component of the risk inherent in foreign (i.e., foreign-currency-denominated) assets. Exchange rate volatility is also a critical factor in selecting hedging strategies for foreign currency exposures.

The amount of foreign exchange rate volatility will depend, at least in part, on the institutional and policy arrangements associated with trade in any given currency. Virtually every exchange rate is managed to some degree by central banks. The policy framework that each central bank adopts is called an exchange rate regime. Although there are many potential variations, these regimes fall into a few general categories. Before describing each of these types, we consider the possibility of an ideal regime and provide some historical perspective on the evolution of currency arrangements.

\section{The Ideal Currency Regime}
The ideal currency regime would have three properties. First, the exchange rate between any two currencies would be credibly fixed. This would eliminate currency-related uncertainty with respect to the prices of goods and services as well as real and financial assets. Second, all currencies would be fully convertible (i.e., currencies could be freely exchanged for any purpose and in any amount). This condition ensures unrestricted flow of capital. Third, each country would be able to undertake fully independent monetary policy in pursuit of domestic objectives, such as growth and inflation targets.

Unfortunately, these three conditions are not consistent. If the first two conditions were satisfied-credibly fixed exchange rates and full convertibility—then there would really be only one currency in the world. Converting from one national currency to another would have no more significance (indeed less) than deciding whether to carry coins or paper currency in your wallet. Any attempt to influence interest rates, asset prices, or inflation by adjusting the supply of one currency versus another would be futile. Thus, it should be clear that independent monetary policy is not possible if exchange rates are credibly fixed and currencies are fully convertible. There can be no ideal currency regime.

The impact of the currency regime on a country's ability to exercise independent monetary policy is a recurring theme in open-economy macroeconomics. It will be covered in more detail in other readings; however, it is worthwhile to emphasize the basic point by considering what would happen in an idealized world of perfect capital mobility. If the exchange rate were credibly fixed, then any attempt to decrease default-free interest rates in one country below those in another-that is, to undertake independent, expansionary monetary policy-would result in a potentially unlimited outflow of capital because funds would seek the higher return. The central bank would be forced to sell foreign currency and buy domestic currency to maintain the fixed exchange rate. The loss of reserves and reduction in the domestic money supply would put upward pressure on domestic interest rates until rates were forced back to equality, negating the initial expansionary policy. Similarly, contractionary monetary policy (higher interest rates) would be thwarted by an inflow of capital.

The situation is quite different, however, with a floating exchange rate. A decrease in the domestic interest rate would make the domestic currency less attractive. The resulting depreciation of the domestic currency would shift demand toward domestically produced goods (i.e., exports rise and imports fall), reinforcing the expansionary impact of the initial decline in the interest rate. Similarly, a contractionary increase in the interest rate would be reinforced by appreciation of the domestic currency.

In practice, of course, capital is not perfectly mobile and the impact on monetary policy is not so stark. The fact remains, however, that fixed exchange rates limit the scope for independent monetary policy and that national monetary policy regains potency and independence, at least to some degree, if the exchange rate is allowed to fluctuate and/or restrictions are placed on convertibility. In general, the more freely the exchange rate is allowed to float and the more tightly convertibility is controlled, the more effective the central bank can be in addressing domestic macroeconomic objectives. The downside, of course, is the potential distortion of economic activity caused by exchange rate risk and inefficient allocation of financial capital.

\section{Historical Perspective on Currency Regimes}
How currencies exchange for one another has evolved over the centuries. At any point in time, different exchange rate systems may coexist; still, there tends to be one dominant system in the world economy. Throughout most of the 19th century and the early 20th century until the start of World War I, the US dollar and the UK pound sterling operated on the "classical gold standard." The price of each currency was fixed in terms of gold. Gold was the numeraire ${ }^{10}$ for each currency; therefore, it was indirectly the numeraire for all other prices in the economy. Many countries (e.g., the colonies of the United Kingdom) fixed their currencies relative to sterling and were therefore implicitly also operating on the classical gold standard.

The classical gold standard operated by what is called the price-specie-flow mechanism. This mechanism operated through the impact of trade imbalances on capital flows, namely gold. As countries experienced a trade surplus, they accumulated gold as payment, their domestic money supply expanded by the amount dictated by the fixed parity, prices rose, and exports fell. Similarly, when a country ran a trade deficit, there was an automatic outflow of gold, a contraction of the domestic money supply, and a fall in prices leading to increased exports.

In this system, national currencies were backed by gold. A country could only print as much money as its gold reserve warranted. The system was limited by the amount of gold, but it was self-adjusting and inspired confidence. With a fixed stock of gold, the price-specie-flow mechanism would work well. Still, new gold discoveries as well as more efficient methods of refining gold would enable a country to increase its gold reserves and increase its money supply apart from the effect of trade flows. In general, however, trade flows drove changes in national money supplies. 11

There is much disagreement among economic historians about the effect of the classical gold standard on overall macroeconomic stability. Was it destabilizing? On the one hand, monetary policy was tied to trade flows, so a country could not engage in expansionary policies when there was a downturn in the non-traded sector. On the other hand, it has been argued that tying monetary policy to trade flows kept inflation in check.

During the 1930s, the use of gold as a clearing device for settlement of trade imbalances, combined with increasing protectionism on the part of economies struggling with depression as well as episodes of deflation and hyperinflation, created a chaotic environment for world trade. As a consequence of these factors, world trade dropped by over 50 percent and the gold standard was abandoned.

In the later stages of World War II, a new system of fixed exchange rates with periodic realignments was devised by John Maynard Keynes and Harry Dexter White, representing the UK and US Treasuries, respectively. The Bretton Woods system, named after the town where it was negotiated, was adopted by 44 countries in 1944. From the end of the war until the collapse of the system in the early 1970s, the United States, Japan, and most of the industrialized countries of Europe maintained a system of fixed parities for exchange rates between currencies. When the parities were significantly

10 Economists refer to the unit of account in terms of which other goods, services, and assets are priced as the numeraire. Under the classical gold standard, the official value of each currency was expressed in ounces of gold.

11 The European inflation of the 17th century was an important exception. Discoveries of gold in South America led to an increase in the world gold stock and in prices throughout Europe. The impact was especially pronounced in Imperial Spain, the primary importing country. Historians have attributed the decline of the Spanish Empire, in part, to the loss of control of domestic prices. and persistently out of line with the balancing of supply and demand, there would be a realignment of currencies with some appreciating in value and others depreciating in value. These periodic realignments were viewed as a part of standard monetary policy.

By 1973, with chronic inflation taking hold throughout the world, most nations abandoned the Bretton Woods system in favor of a flexible exchange rate system under what are known as the Smithsonian Agreements. Milton Friedman had called for such a system as far back as the 1950s. ${ }^{12}$ His argument was that the fixed parity system with periodic realignments would become unsustainable. When the inevitable realignments were imminent, large speculative profit opportunities would appear. Speculators would force the hand of monetary policy authorities, and their actions would distort the data needed to ascertain appropriate trade-related parities. It is better, he argued, to let the market, rather than central bank governors and treasury ministers, determine the exchange rate.

After 1973, most of the industrialized world changed to a system of flexible exchange rates. The original thinking was that the forces that caused exchange rate chaos in the 1930s-poor domestic monetary policy and trade barriers-would not be present in a flexible exchange rate regime, and therefore exchange rates would move in response to the exchange of goods and services among countries. As it turned out, however, exchange rates moved around much more than anyone expected. Academic economists and financial analysts alike soon realized that the high degree of exchange rate volatility was the manifestation of a highly liquid, forward-looking asset market. ${ }^{13}$ Investment-driven FX transactions-for both long-term investment and short-term speculation-mattered much more in setting the spot exchange rate than anyone had previously imagined.

There are costs, of course, to a high degree of exchange rate volatility. These include difficulty planning without hedging exchange rate risks - a form of insurance cost, domestic price fluctuations, uncertain costs of raw materials, and short-term interruptions in financing transactions. For these reasons, in 1979 the European Economic Community opted for a system of limited flexibility, the European Exchange Rate Mechanism (ERM).

Initially, the system called for European currency values to fluctuate within a narrow band called "the snake." This did not last long. The end of the Cold War and the re-unification of Germany created conditions ripe for speculative attack. In the early 1990s, the United Kingdom was in a recession and the government's monetary policy leaned toward low interest rates to stimulate economic recovery. Germany was issuing large amounts of debt to pay for re-unification, and the German central bank (the Deutsche Bundesbank) opted for high interest rates to ensure price stability. Capital began to flow from sterling to Deutsche marks to obtain the higher interest rate. The Bank of England tried to lean against these flows and maintain the exchange rate within the Exchange Rate Mechanism, but eventually it began to run out of marks to sell. Because it was almost certain that devaluation would be required, holders of sterling rushed to purchase marks at the old rate and the speculative attack forced the United Kingdom out of the ERM in September 1992, only two years after it finally joined the system.

12 Friedman (1953).

13 Whether or not FX markets satisfy recognized definitions of market efficiency-correctly reflecting all available information-is debatable (e.g., some point to evidence of trending as a clear violation of efficiency). However, there is no doubt that FX market participants attempt to incorporate new information, which is often lumpy and difficult to decipher, into their expectations about the future. Changing expectationsaccurate or otherwise-affect the value that investors place on holding different currencies and, in a highly liquid market, lead to rapid and sometimes violent exchange rate movements. Despite these difficulties, 1999 saw the creation of a common currency for most Western European countries, without Switzerland or the United Kingdom, called the euro. ${ }^{14}$ The hope was that the common currency would increase transparency of prices across borders in Europe, enhance market competition, and facilitate more efficient allocation of resources. The drawback, of course, is that each member country lost the ability to manage its exchange rate and therefore to engage in independent monetary policy.

\section{A TAXONOMY OF CURRENCY REGIMES}
describe exchange rate regimes

Although the pros and cons of fixed and flexible exchange rate regimes continue to be debated, regimes have been adopted that lie somewhere between these polar cases. In some cases, the driving force is the lack of credibility with respect to sound monetary policy. An economy with a history of hyperinflation may be forced to adopt a form of fixed-rate regime because its promise to maintain a sound currency with a floating rate regime would not be credible. This has been a persistent issue in Latin America. In other cases, the driving force is as much political as economic. The decision to create the euro was strongly influenced by the desire to enhance political union within the European Community, whose members had been at war with each other twice in the 20th century.

As of April 2008, the International Monetary Fund (IMF) classified exchange rate regimes into the eight categories shown in Exhibit 8.

\section{Exhibit 8: Exchange Rate Regimes for Selected Economies ${ }^{15}$ As of 30 April 2008}
\begin{center}
\begin{tabular}{|c|c|c|c|}
\hline
\multirow[b]{2}{*}{Type of Regime} & \multicolumn{3}{|c|}{Currency Anchor} \\
\hline
 & USD & EUR & Basket/None \\
\hline
\multicolumn{4}{|c|}{No separate legal tender} \\
\hline
Dollarized & $\begin{array}{l}\text { Ecuador, El Salvador, } \\ \text { Marshall Islands, Micronesia, } \\ \text { Palau, Panama, Timor-Leste, } \\ \text { Zimbabwe }\end{array}$ & $\begin{array}{l}\text { Kosovo, Montenegro, San } \\ \text { Marino }\end{array}$ & Kiribati, Tuval \\
\hline
Monetary union &  & $\begin{array}{l}\text { EMU: Austria, Belgium, } \\ \text { Cyprus, Estonia, Finland, } \\ \text { France, Germany, Greece, } \\ \text { Ireland, Italy, Latvia, } \\ \text { Luxembourg, Lithuania, } \\ \text { Malta, Netherlands, Portugal, } \\ \text { Slovak Rep., Slovenia, Spain }\end{array}$ &  \\
\hline
\end{tabular}
\end{center}

14 The number of European countries adopting the euro has continued to expand since its inception; the most recent country to join the euro was Lithuania, on 1 January 2015.

15 The classifications are described in the IMF's Annual Report on Exchange Arrangements and Exchange Restrictions 2014. In some cases, the labels used by the IMF do not clearly distinguish among the regimes. Hence, the names applied here to the regimes differ somewhat from the IMF's original taxonomy.

\begin{center}
\begin{tabular}{|c|c|c|c|}
\hline
\multirow[b]{2}{*}{Type of Regime} & \multicolumn{3}{|c|}{Currency Anchor} \\
\hline
 & USD & EUR & Basket/None \\
\hline
Currency board & $\begin{array}{l}\text { Djibouti, Hong Kong SAR, } \\ \text { Antigua and Barbuda }\end{array}$ & $\begin{array}{l}\text { Bosnia and Herzegovina, } \\ \text { Bulgaria }\end{array}$ & Brunei Darussalam \\
\hline
Fixed parity & $\begin{array}{l}\text { Aruba, The Bahamas, } \\ \text { Bahrain, Barbados, Belize, } \\ \text { Curaçao and Saint Maarten, } \\ \text { Eritrea, Jordan, Oman, } \\ \text { Qatar, Saudi Arabia, South } \\ \text { Sudan, Turkmenistan, UAE, } \\ \text { Venezuela }\end{array}$ & $\begin{array}{l}\text { Cabo Verde, Comoros, } \\ \text { Denmark, São Tomé and } \\ \text { Príncipe } \\ \text { WAEMU: Benin, Burkina } \\ \text { Faso, Côte d'Ivoire, } \\ \text { Guinea-Bissau, Mali, Niger, } \\ \text { Senegal, Togo } \\ \text { CEMAC: Cameroon, Central } \\ \text { African Rep., Chad, Rep. of } \\ \text { Congo, Equatorial Guinea, } \\ \text { Gabon }\end{array}$ & $\begin{array}{l}\text { Fiji, Kuwait, Libya, Morocco, } \\ \text { Samoa, Bhutan, Lesotho, } \\ \text { Namibia, Nepal, Swaziland }\end{array}$ \\
\hline
Target zone &  & Slovak Republic & Syria \\
\hline
Crawling peg & Nicaragua &  & Botswana \\
\hline
Crawling band & Honduras, Jamaica & Croatia & $\begin{array}{l}\text { China, Ethiopia, Uzbekistan, } \\ \text { Armenia, Dominican Republic, } \\ \text { Guatemala, Argentina, Belarus } \\ \text { Haiti, Switzerland, Tunisia }\end{array}$ \\
\hline
Managed float & Cambodia, Liberia &  & $\begin{array}{l}\text { Algeria, Iran, Syria, The } \\ \text { Gambia, Myanmar, Nigeria, } \\ \text { Rwanda, Czech Rep., Costa } \\ \text { Rica, Malaysia, Mauritania } \\ \text { Pakistan, Russia, Sudan, } \\ \text { Vanuatu }\end{array}$ \\
\hline
$\begin{array}{l}\text { Independent float } \\ \text { Currency columns due to } \\ \text { floating }\end{array}$ & $\begin{array}{l}\text { Australia, Canada, Chile, } \\ \text { Japan, Mexico, Norway, } \\ \text { Poland, Sweden, United } \\ \text { Kingdom, Somalia, United } \\ \text { States }\end{array}$ & $\begin{array}{l}\text { Albania, Brazil, Colombia, } \\ \text { Georgia, Ghana, Hungary, } \\ \text { Iceland, Indonesia, Israel, } \\ \text { Korea, Moldova, New } \\ \text { Zealand, Paraguav, Peru }\end{array}$ & $\begin{array}{l}\text { Philippines, Romania, Serbia, } \\ \text { South Africa, Thailand, Turkey, } \\ \text { Uganda }\end{array}$ \\
\hline
\end{tabular}
\end{center}

It should be noted that global financial markets are too complex and diverse to be fully captured by this (or any other) classification system. A government's control over the domestic currency's exchange rate will depend on many factors; for example, the degree of capital controls used to prevent the free flow of funds in and out of the economy. Also, even under an "independent float" regime monetary authorities will occasionally intervene in foreign exchange markets in order to influence the value of their domestic currency. Additionally, the specifics of exchange rate policy implementation are subject to change.

This means that the classifications in Exhibit 8 are somewhat arbitrary and subject to interpretation, as well as change, over time. The important point to be drawn from this discussion is that the prices and flows in foreign exchange markets will, to varying degrees, reflect the legal and regulatory framework imposed by governments, not just "pure" market forces. Governments have a variety of motives and tools for attempting to manage exchange rates. The taxonomy in Exhibit 8 can be used to help understand the main distinctions among currency regimes and the rationales for adopting them, but the specific definitions should not be interpreted too rigidly. Instead, the focus should be on the diversity of foreign exchange markets globally as well as the implications of these various currency regimes for market pricing.

\section{Arrangements with No Separate Legal Tender}
The IMF identifies two types of arrangements in which a country does not have its own legal tender. In the first, known as dollarization, the country uses the currency of another nation as its medium of exchange and unit of account. In the second, the country participates in a monetary union whose members share the same legal tender. In either case, the country gives up the ability to conduct its own monetary policy.

In principle, a country could adopt any currency as its medium of exchange and unit of account, but the main reserve currency, the US dollar, is an obvious choicehence the name dollarization. Many countries are dollarized: East Timor, El Salvador, Ecuador, and Panama, for example. By adopting another country's currency as legal tender, a dollarized country inherits that country's currency credibility, but not its credit-worthiness. For example, although local banks may borrow, lend, and accept deposits in US dollars, they are not members of the US Federal Reserve System nor are they backed by deposit insurance from the Federal Deposit Insurance Corporation. Thus, interest rates on US dollars in a dollarized economy need not be, and generally are not, the same as on dollar deposits in the United States.

Dollarization imposes fiscal discipline by eliminating the possibility that the central bank will be induced to monetize government debt (i.e., to persistently purchase government debt with newly created local currency). For countries with a history of fiscal excess or lack of monetary discipline, dollarizing the economy can facilitate growth of international trade and capital flows if it creates an expectation of economic and financial stability. In the process, however, it removes another potential source of stabilization-domestic monetary policy.

The European Economic and Monetary Union (EMU) is the most prominent example of the second type of arrangement lacking separate legal tender. Each EMU member uses the euro as its currency. Although member countries cannot have their own monetary policies, they jointly determine monetary policy through their representation at the European Central Bank (ECB). As with dollarization, a monetary union confers currency credibility on members with a history of fiscal excess and/or a lack of monetary discipline. However, as shown by the 2010 EMU sovereign debt crisis, monetary union alone cannot confer credit-worthiness.

\section{Currency Board System}
The IMF defines a currency board system (CBS) as:

A monetary regime based on an explicit legislative commitment to exchange domestic currency for a specified foreign currency at a fixed exchange rate, combined with restrictions on the issuing authority to ensure fulfillment of its legal obligation. This implies that domestic currency will be issued only against foreign exchange and it remains fully backed by foreign assets.... 16

Hong Kong SAR has the leading example of a long-standing (since 1983) currency board. US dollar reserves are held to cover, at the fixed parity, the entire monetary base-essentially bank reserves plus all HKD notes and coins in circulation. ${ }^{17}$ Note that HKD-denominated bank deposits are not fully collateralized by US dollar reserves; to do so would mean that banks could not lend against their deposits. The Hong Kong Monetary Authority (HKMA) does not function as a traditional central bank under this system because the obligation to maintain 100 percent foreign currency reserves

16 International Monetary Fund (2006).

17 For a description of Hong Kong's currency board system, see Hong Kong Monetary Authority (2005). against the monetary base prevents it from acting as a lender-of-last-resort for troubled financial institutions. However, it can provide short-term liquidity by lending against foreign currency collateral.

A CBS works much like the classical gold standard in that expansion and contraction of the monetary base are directly linked to trade and capital flows. As with the gold standard, a CBS works best if domestic prices and wages are very flexible, non-traded sectors of the domestic economy are relatively small, and the global supply of the reserve asset grows at a slow, steady rate consistent with long-run real growth with stable prices. The first two of these conditions are satisfied in Hong Kong SAR. Until and unless Hong Kong SAR selects a new reserve asset, however, the third condition depends on US monetary policy.

In practice, the HKD exhibits modest fluctuations around the official parity of HKD/ USD $=7.80$ because the HKMA buys (sells) USD at a pre-announced level slightly below (above) the parity. Persistent flows on one side of this convertibility zone or the other result in interest rate adjustments rather than exchange rate adjustments. Inside the zone, however, the exchange rate is determined by the market and the HKMA is free to conduct limited monetary operations aimed at dampening transitory interest rate movements.

One of the advantages of a CBS as opposed to dollarization is that the monetary authority can earn a profit by paying little or no interest on its liability-the monetary base-and can earn a market rate on its asset-the foreign currency reserves. This profit is called seigniorage. ${ }^{18}$ Under dollarization, the seigniorage goes to the monetary authority whose currency is used.

\section{Fixed Parity}
A simple fixed-rate system differs from a CBS in two important respects. First, there is no legislative commitment to maintaining the specified parity. Thus, market participants know that the country may choose to adjust or abandon the parity rather than endure other, potentially more painful, adjustments. Second, the target level of foreign exchange reserves is discretionary; it bears no particular relationship to domestic monetary aggregates. Thus, although monetary independence is ultimately limited as long as the exchange peg is maintained, the central bank can carry out traditional functions, such as serving as lender of last resort.

In the conventional fixed-rate system, the exchange rate may be pegged to a single currency-for example, the US dollar-or to a basket index of the currencies of major trading partners. There is a band of up to \textbackslash pm 1 percent around the parity level within which private flows are allowed to determine the exchange rate. The monetary authority stands ready to spend its foreign currency reserves, or buy foreign currency, in order to maintain the rate within these bands.

The credibility of the fixed parity depends on the country's willingness and ability to offset imbalances in private sector demand for its currency. Both excess and deficient private demand for the currency can exert pressure to adjust or abandon the parity. Excess private demand for the domestic currency implies a rapidly growing stock of foreign exchange reserves, expansion of the domestic money supply, and potentially accelerating inflation. Deficient demand for the currency depletes foreign exchange reserves and exerts deflationary pressure on the economy. If market participants believe the foreign exchange reserves are insufficient to sustain the parity, then that belief

18 More generally, seigniorage is the profit earned when the value of money issued exceeds the cost of producing it. For physical currency, seigniorage arises when a coin is minted for a fraction of its face value and then issued (sold) at its face value. may be self-fulfilling because the resulting speculative attack will drain reserves and may force an immediate devaluation. Thus, the level of reserves required to maintain credibility is a key issue for a simple fixed exchange rate regime.

\section{Target Zone}
A target zone regime has a fixed parity with fixed horizontal intervention bands that are somewhat wider-up to \textbackslash pm 2 percent around the parity - than in the simple fixed parity regime. The wider bands provide the monetary authority with greater scope for discretionary policy.

\section{Active and Passive Crawling Pegs}
Crawling pegs for the exchange rate-usually against a single currency, such as the US dollar-were common in the 1980s in Latin America, particularly Brazil, during the high inflation periods. To prevent a run on the US dollar reserves, the exchange rate was adjusted frequently (weekly or daily) to keep pace with the inflation rate. Such a system was called a passive crawl. An adaptation used in Argentina, Chile, and Uruguay was the active crawl: The exchange rate was pre-announced for the coming weeks with changes taking place in small steps. The aim of the active crawl was to manipulate expectations of inflation. Because the domestic prices of many goods were directly tied to import prices, announced changes in the exchange rate would effectively signal future changes in the inflation rate of these goods.

\section{Fixed Parity with Crawling Bands}
A country can also have a fixed central parity with crawling bands. Initially, a country may fix its rates to a foreign currency to anchor expectations about future inflation but then gradually permit more and more flexibility in the form of a pre-announced widening band around the central parity. Such a system has the desirable property of allowing a gradual exit strategy from the fixed parity. A country might want to introduce greater flexibility and greater scope for monetary policy, but it may not yet have the credibility or financial infrastructure for full flexibility. So it maintains a fixed parity with slowly widening bands.

\section{Managed Float}
A country may simply follow an exchange rate policy based on either internal or external policy targets-intervening or not to achieve trade balance, price stability, or employment objectives. Such a policy, often called dirty floating, invites trading partners to respond likewise with their exchange rate policy and potentially decreases stability in foreign exchange markets as a whole. The exchange rate target, in terms of either a level or a rate of change, is typically not explicit.

\section{Independently Floating Rates}
In this case, the exchange rate is left to market determination and the monetary authority is able to exercise independent monetary policy aimed at achieving such objectives as price stability and full employment. The central bank also has latitude to act as a lender of last resort to troubled financial institutions, if necessary.

It should be clear from recent experience that the concepts of float, managed float, crawl, and target zone are not hard and fast rules. Central banks do occasionally engage, implicitly or explicitly, in regime switches-even in countries nominally following an independently floating exchange rate regime. For example, when the US dollar appreciated in the mid-1980s with record US trade deficits, then-US Treasury Secretary James Baker engineered the Plaza Accord, in which Japan and Germany engineered an appreciation of their currencies against the US dollar. (The "Plaza Accord" is so named because it was negotiated at the Plaza Hotel in New York City.) This 1985 policy agreement involved a combination of fiscal and monetary policy measures by the countries involved as well as direct intervention in foreign exchange markets. The Plaza Accord was a clear departure from a pure independently floating exchange rate system.

There are more recent examples of government intervention in foreign exchange markets. In September 2000, the European Central Bank, the Federal Reserve Board, the Bank of Japan, the Bank of England, and the Bank of Canada engaged in "concerted" intervention in order to support the value of the euro, a "freely floating" currency which was then under pressure within foreign exchange markets. (This intervention was described as "concerted" because it was pre-arranged and coordinated among the central banks involved.) During 2010, many countries engaged in unilateral intervention to prevent the rapid appreciation of their currencies against the US dollar. Several of these countries also employed various fiscal and regulatory measures (for example, taxes on capital inflows) in order to further affect exchange rate movements.

The important point to draw from this discussion is that exchange rates do not only reflect private sector market forces but will also, to varying degrees, be influenced by the legal and regulatory framework (currency regimes) within which foreign exchange markets operate. Moreover, they will occasionally be influenced by government polices (fiscal, monetary, and intervention) intended to manage exchange rates. All of these can vary widely among countries and are subject to change with time.

Nonetheless, the most widely traded currencies in foreign exchange markets (the US dollar, yen, euro, UK pound, Swiss franc, and the Canadian and Australian dollars) are typically considered to be free floating, although subject to relatively infrequent intervention.

\section{EXAMPLE 7}
\section{Currency Regimes}
An investment adviser in Los Angeles, USA, is meeting with a client who wishes to diversify her portfolio by including more international investments. In order to evaluate the suitability of international diversification for the client, the adviser attempts to explain some of the characteristics of foreign exchange markets. The adviser points out that exchange rate regimes affect the performance of domestic economies as well as the amount of foreign exchange risk posed by international investments.

The client and her adviser discuss potential investments in Hong Kong SAR, Panama, and Canada. The adviser notes that the currency regimes of Hong Kong SAR, Panama, and Canada are a currency board, dollarization, and a free float, respectively. The adviser tells his client that these regimes imply different degrees of foreign exchange risk for her portfolio.

The discussion between the investment adviser and his client then turns to potential investments in other markets with different currency regimes. The adviser notes that some markets are subject to fixed parity regimes against the US dollar. The client asks whether a fixed parity regime would imply less foreign currency risk for her portfolio than would a currency board. The adviser replies: "Yes, a fixed parity regime means a constant exchange rate and is more credible than a currency board." The adviser goes on to explain that in some markets exchange rates are allowed to vary, although with different degrees of foreign exchange market intervention to limit exchange rate volatility. Citing examples, he notes that mainland China has a crawling peg regime with reference to the US dollar, but the average daily percentage changes in mainland China/US exchange rate are very small compared with the average daily volatility for a freely floating currency. The adviser also indicates that Denmark has a target zone regime with reference to the euro, and South Korea usually follows a freely floating currency regime but sometimes switches to a managed float regime. The currencies of mainland China, Denmark, and South Korea are the yuan renminbi (CNY), krone (DKK), and won (KRW), respectively.

\begin{enumerate}
  \item Based solely on the exchange rate risk the client would face, what is the correct ranking (from most to least risky) of the following investment locations?
\end{enumerate}

A. Panama, Canada, Hong Kong SAR.

B. Canada, Hong Kong SAR, Panama.

C. Hong Kong SAR, Panama, Canada.

\section{Solution to 1:}
$\mathrm{B}$ is correct. The CAD/USD exchange rate is a floating exchange rate, and Canadian investments would therefore carry exchange rate risk for a US-based investor. Although Hong Kong SAR follows a currency board system, the HKD/USD exchange rate nonetheless does display some variation, albeit much less than in a floating exchange rate regime. In contrast, Panama has a dollarized economy (i.e., it uses the US dollar as the domestic currency); therefore, there is no foreign exchange risk for a US investor.

\begin{enumerate}
  \setcounter{enumi}{1}
  \item Based solely on their foreign exchange regimes, which investment location is least likely to import inflation or deflation from the United States?
A. Canada.
B. Panama.
C. Hong Kong SAR.
\end{enumerate}

\section{Solution to 2:}
A is correct. The Canadian dollar floats independently against the US dollar leaving the Bank of Canada able to adjust monetary policy to maintain price stability. Neither Hong Kong SAR (currency board) nor Panama (dollarized) can exercise independent monetary policy to buffer its economy from the inflationary/deflationary consequences of US monetary policy.

\begin{enumerate}
  \setcounter{enumi}{2}
  \item The adviser's statement about fixed parity regimes is incorrect with regard to:
A. credibility.
B. a constant exchange rate.
C. both a constant exchange rate and credibility.
\end{enumerate}

\section{Solution to 3:}
$\mathrm{C}$ is correct. A fixed exchange rate regime does not mean that the exchange rate is rigidly fixed at a constant level. In practice, both a fixed-rate regime and a currency board allow the exchange rate to vary within a band around the stated parity level. Thus, both regimes involve at least a modest amount of exchange rate risk. The fixed parity regime exposes the investor to the additional risk that the parity may not be maintained. In a fixed parity regime, the level of foreign currency reserves is discretionary and typically only a small fraction of the domestic money supply. With no legal obligation to maintain the parity, the monetary authority may adjust the parity (devalue or revalue its currency) or allow its currency to float if doing so is deemed to be less painful than other adjustment mechanisms (e.g., fiscal restraint). In contrast, a currency board entails a legal commitment to maintain the parity and to fully back the domestic currency with reserve currency assets. Hence, there is little risk that the parity will be abandoned.

\begin{enumerate}
  \setcounter{enumi}{3}
  \item Based on the adviser's categorization of mainland China's currency regime, if the USD is depreciating against the KRW, then it is most likely correct that the CNY is:
\end{enumerate}

A. fixed against the KRW.

B. appreciating against the KRW.

C. depreciating against the KRW.

Solution to 4:

$\mathrm{C}$ is correct. If the CNY is subject to a crawling peg with very small daily adjustments versus the USD and the USD is depreciating against the KRW, then the CNY would most likely be depreciating against the KRW as well. In fact, this was an important issue in foreign exchange markets through the latter part of 2010: As the USD depreciated against most Asian currencies (and less so against the CNY), many Asian countries felt that they were losing their competitive export advantage because the CNY was so closely tied to the USD. This led many Asian countries to intervene in FX markets against the strength of their domestic currencies in order not to lose an export pricing advantage against the Chinese mainland.

\begin{enumerate}
  \setcounter{enumi}{4}
  \item Based on the adviser's categorization of Denmark's currency regime, it would be most correct to infer that the:
\end{enumerate}

A. krone is allowed to float against the euro within fixed bands.

B. Danish central bank will intervene if the exchange rate strays from its target level.

C. target zone will be adjusted periodically in order to manage inflation expectations.

Solution to 5:

A is correct. A target zone means that the exchange rate between the euro and Danish krone (DKK) will be allowed to vary within a fixed band (as of 2010, the target zone for the DKK/EUR is a \textbackslash pm 2.5 percent band). This does not mean that the DKK/EUR rate is fixed at a certain level (B is incorrect) or that the target zone will vary in order to manage inflation expectations (this is a description of a crawling peg, which makes $\mathrm{C}$ incorrect).

\begin{enumerate}
  \setcounter{enumi}{5}
  \item Based on the adviser's categorization of South Korea's currency policy, it would be most correct to infer that the:
\end{enumerate}

A. Korean central bank is engineering a gradual exit from a fixed-rate regime. B. government is attempting to peg the exchange rate within a predefined zone.

C. won is allowed to float, but with occasional intervention by the Korean central bank.

\section{Solution to 6:}
$\mathrm{C}$ is correct. Similar to the monetary authorities responsible for many of the world's major currencies, the South Korean policy typically involves letting market forces determine the exchange rate (an independent floating rate regime). But this approach does not mean that market forces are the sole determinant of the won exchange rate. As with most governments, the South Korean policy is to intervene in foreign exchange markets when movements in the exchange rate are viewed as undesirable (a managed float). For example, during the later part of 2010, South Korea and many other countries intervened in foreign exchange markets to moderate the appreciation of their currencies against the US dollar. Answer A describes a fixed parity with a crawling bands regime, and B describes a target zone regime: Both answers are incorrect.

\section{EXCHANGE RATES AND THE TRADE BALANCE: INTRODUCTION}
explain the effects of exchange rates on countries' international trade and capital flows

Just as a family that spends more than it earns must borrow or sell assets to finance the excess, a country that imports more goods and services than it exports must borrow from foreigners or sell assets to foreigners to finance the trade deficit. Conversely, a country that exports more goods and services than it imports must invest the excess either by lending to foreigners or by buying assets from foreigners. Thus, a trade deficit (surplus) must be exactly matched by an offsetting capital account surplus (deficit). ${ }^{19}$ This implies that any factor that affects the trade balance must have an equal and opposite impact on the capital account, and vice versa. To put this somewhat differently, the impact of exchange rates and other factors on the trade balance must be mirrored by their impact on capital flows: They cannot affect one without affecting the other.

Using a fundamental identity from macroeconomics, the relationship between the trade balance and expenditure/saving decisions can be expressed as: ${ }^{20}$

$$
X-M=(S-I)+(T-G)
$$

19 In official balance of payments accounts, investment/financing flows are separated into two categories: the capital account and the financial account. Because the technical distinction is immaterial for present purposes, we will simply refer to the balance of investment/financing flows as the capital account. Similarly, we ignore the technical distinction between the trade balance and the current account balance. The details of balance of payments accounting are presented in the Level I curriculum reading on International Trade and Capital Flows.

20 This relationship is developed in the Level I curriculum reading on Aggregate Output, Prices, and Economic Growth. where $X$ represents exports, $M$ is imports, $S$ is private savings, $I$ is investment in plant and equipment, $T$ is taxes net of transfers, and $G$ is government expenditure. From this relationship, we can see that a trade surplus $(X>M)$ must be reflected in a fiscal surplus $(T>G)$, an excess of private saving over investment $(S>I)$, or both. Because a fiscal surplus can be viewed as government saving, we can summarize this relationship more simply by saying that a trade surplus means the country saves more than enough to fund its investment $(I)$ in plant and equipment. The excess saving is used to accumulate financial claims on the rest of the world. Conversely, a trade deficit means the country does not save enough to fund its investment spending $(I)$ and must reduce its net financial claims on the rest of the world.

Although this identity provides a key link between real expenditure/saving decisions and the aggregate flow of financial assets into or out of a country, it does not tell us what type of financial assets will be exchanged or in what currency they will be denominated. All that can be said is that asset prices and exchange rates at home and abroad must adjust so that all financial assets are willingly held by investors.

If investors anticipate a significant change in an exchange rate, they will try to sell the currency that is expected to depreciate and buy the currency that is expected to appreciate. This implies an incipient (i.e., potential) flow of capital from one country to the other, which must either be accompanied by a simultaneous shift in the trade balance or be discouraged by changes in asset prices and exchange rates. Because expenditure/saving decisions and prices of goods change much more slowly than financial investment decisions and asset prices, most of the adjustment usually occurs within the financial markets. That is, asset prices and exchange rates adjust so that the potential flow of financial capital is mitigated and actual capital flows remain consistent with trade flows. In a fixed exchange rate regime, the central bank offsets the private capital flows in the process of maintaining the exchange rate peg and the adjustment occurs in other asset prices, typically interest rates, until and unless the central bank is forced to allow the exchange rate to adjust. ${ }^{21}$ In a floating exchange rate regime, the main adjustment is often a rapid change in the exchange rate that dampens an investor's conviction that further movement will be forthcoming. Thus, capital flows-potential and actual-are the primary determinant of exchange rate movements in the short-to-intermediate term. Trade flows become increasingly important in the longer term as expenditure/saving decisions and the prices of goods and services adjust.

With the correspondence between the trade balance and capital flows firmly established, we can now examine the impact of exchange rate changes on the trade balance from two perspectives. The first approach focuses on the effect of changing the relative price of domestic and foreign goods. This approach, which is called the elasticities approach, highlights changes in the composition of spending. The second approach, called the absorption approach, focuses on the impact of exchange rates on aggregate expenditure/saving decisions.

21 A classic example of this occurred in September 1992, when the United Kingdom was forced to withdraw from the European Exchange Rate Mechanism, the forerunner of the current European Economic and Monetary Union (EMU).

\section{EXCHANGE RATES AND THE TRADE BALANCE: THE ELASTICITIES APPROACH}
explain the effects of exchange rates on countries' international trade and capital flows

The effectiveness of devaluation (in a fixed system) or depreciation (in a flexible system) of the currency for reducing a trade deficit depends on well-behaved demand and supply curves for goods and services. The condition that guarantees that devaluations improve the trade balance is called the Marshall-Lerner condition. The usual statement of this condition assumes that trade is initially balanced. We will present a generalization of the condition that allows for an initial trade imbalance and hence is more useful in addressing whether exchange rate movements will correct such imbalances.

Recall from microeconomics that the price elasticity of demand is given by: ${ }^{22}$

$\varepsilon=-\frac{\% \text { change in quantity }}{\% \text { change in price }}=-\frac{\% \Delta Q}{\% \Delta P}$

For example, a demand elasticity of 0.6 means that quantity demanded increases by 6 percent if price declines by 10 percent. Note that the elasticity of demand is defined so that it is a positive number. Because expenditure $(R)$ equals price multiplied by quantity $(P \times Q)$, by re-arranging the above expression to solve and substitute for $\% \Delta Q$, we can see that:

$$
\% \text { change in expenditure }=\% \Delta R=\% \Delta P+\% \Delta Q=(1-\varepsilon) \% \Delta P
$$

From this we can see that an increase in price decreases expenditure if $\varepsilon>1$, but it increases expenditure if $\varepsilon<1$. By convention, if $\varepsilon>1$ demand is described as being "elastic," while if $\varepsilon<1$ demand is described as "inelastic."

The basic idea behind the Marshall-Lerner condition is that demand for imports and exports must be sufficiently price-sensitive so that increasing the relative price of imports increases the difference between export receipts and import expenditures. The generalized Marshall-Lerner condition is:

$$
\omega_{X} \varepsilon_{X}+\omega_{M}\left(\varepsilon_{M}-1\right)>0
$$

where $\omega_{X}$ and $\omega_{M}$ are the shares of exports and imports, respectively, in total trade (i.e., imports + exports) and $\varepsilon_{X}$ and $\varepsilon_{M}$ are the price elasticities of foreign demand for domestic country exports and domestic country demand for imports, respectively. Note that $\left(\omega_{X}+\omega_{M}\right)=1$ and that an initial trade deficit implies $\omega_{M}>\omega_{X}$. If this condition is satisfied, a devaluation/depreciation of the domestic currency will move the trade balance toward surplus.

The first term in the generalized Marshall-Lerner condition reflects the change in export receipts assuming the domestic currency price of exports is unchanged (i.e., foreigners are billed in the domestic currency). It will be positive as long as export demand is not totally insensitive to price. Depreciation of the domestic currency makes exports cheaper in foreign currency and induces an increase in the quantity demanded by foreigners. This is reflected by the elasticity $\varepsilon_{\mathrm{x}}$. There is no direct price impact on domestic currency export revenue because the domestic currency price is assumed to be unchanged. Hence, the percentage change in export revenue corresponding to a 1 percent depreciation of the currency is simply $\varepsilon_{\mathrm{x}}$. The second term in the generalized Marshall-Lerner condition reflects the impact on import expenditures. Assuming that imports are billed in a foreign currency, the domestic currency price of imports

22 See the Level I curriculum reading Topics in Demand and Supply Analysis. rises as the domestic currency depreciates. The direct price effect increases import expenditures, while the induced reduction in the quantity of imports decreases import expenditures. The net effect depends on the elasticity of import demand, $\varepsilon_{M}$. Import expenditure declines only if import demand is elastic (i.e., $\varepsilon_{M}>1$ ).

Examination of the generalized Marshall-Lerner condition indicates that more elastic demand-for either imports or exports-makes it more likely that the trade balance will improve. Indeed, if the demand for imports is elastic, $\varepsilon_{M}>1$, then the trade balance will definitely improve. It should also be clear that the elasticity of import demand becomes increasingly important, and the export elasticity less important, as the initial trade deficit gets larger-that is, as $\omega_{M}$ increases. In the special case of initially balanced trade, $\omega_{X}=\omega_{M}$, the condition reduces to $\left(\varepsilon_{X}+\varepsilon_{M}>1\right)$, which is the classic Marshall-Lerner condition.

Exhibit 9 illustrates the impact of depreciation on the trade balance. For ease of reference, we assume the domestic currency is the euro. A 10 percent depreciation of the euro makes imports 10 percent more expensive in euro terms. With an import elasticity of 0.65 , this induces a 6.5 percent reduction in the quantity of imports. But import expenditures increase by 3.5 percent $[10 \% \times(1-0.65)]$ or $€ 21,000,000$ because the drop in quantity is not sufficient to offset the increase in price. On the export side, the euro price of exports does not change but the foreign currency price of exports declines by 10 percent. This induces a 7.5 percent increase in the quantity of exports given an elasticity of 0.75 . The euro value of exports therefore increases by 7.5 percent or $€ 30,000,000$. The net effect is a $€ 9,000,000$ improvement in the trade balance and a $€ 51,000,000$ increase in total trade.

\section{Exhibit 9: Marshall-Lerner Condition with a 10 Percent Depreciation of Domestic Currency ( $\epsilon$ )}
\begin{center}
\begin{tabular}{lcc}
\hline
Assumptions & Exports & Imports \\
\hline
Demand elasticity & 0.75 & 0.65 \\
Percent price change &  &  \\
In domestic currency $(€)$ & 0 & $10 \%$ \\
In foreign currency & $-10 \%$ & 0 \\
\hline
Results & Initial value( $\epsilon)$ & Change( $\boldsymbol{\epsilon})$ \\
\hline
Exports & $400,000,000$ & $30,000,000$ \\
Imports & $600,000,000$ & $21,000,000$ \\
Trade balance & $-200,000,000$ & $9,000,000$ \\
Total trade & $1,000,000,000$ & $51,000,000$ \\
\hline
\end{tabular}
\end{center}

The balance of trade improves after the depreciation of the euro because the MarshallLerner condition is satisfied: The increase in the euro-value of exports exceeds the increase in the value of imports. Based on the data in Exhibit $9, \omega_{M}=0.6$ (i.e., 600,000,000/1,000,000,000) and $\omega_{X}=0.4$ (i.e., $1-0.6$ ). Thus, the Marshall-Lerner equation is greater than zero:

$$
\omega_{X} \varepsilon_{X}+\omega_{M}\left(\varepsilon_{M}-1\right)=0.4 \times 0.75+0.6(0.65-1)=0.09
$$

The elasticity of demand for any good or service depends on at least four factors: 1) the existence or absence of close substitutes, 2) the structure of the market for that product (e.g., a monopoly or perfect competition), 3) its share in people's budgets, and 4) the nature of the product and its role in the economy. Demand for a product with close substitutes is highly price-sensitive, whereas demand for a unique product tends to be much less elastic. The demand curve faced by any producer also depends on the nature and level of competition among producers of that product. If there are many sellers of identical products, then each producer faces highly elastic demand for its output even if global demand for that product is insensitive to price. Producers who are able to differentiate their product, perhaps through branding, face somewhat less elastic demand. In markets with only a few sellers, each producer faces demand that is highly dependent upon strategic maneuvers by its competitors. If competitors match price decreases but not increases, then the producer loses market share by raising his price but fails to gain market share by reducing his price.

Price changes have two effects on demand. The substitution effect refers to changes in the composition of spending across different products. As a product gets more expensive (cheaper) relative to other products, customers demand less (more) of it. This is what people usually think of first when they consider the effect of a price change. The income effect refers to the fact that price changes affect real purchasing power. When the price of a good rises (falls), people's purchasing power is reduced (increased). The strength of this effect depends on the product's share in people's budgets - the more important the product, the stronger the income effect. The income effect also depends on the nature of the product. The demand for luxuries is highly sensitive to income, whereas the demand for necessities is fairly insensitive to income.

To illustrate the differential impact of the two drivers of the income effect-share of expenditure and nature of the product-consider the demand for food. Clearly, food is a necessity. Based on this fact, we would expect demand to be inelastic. However, the share of expenditures that go to food varies across countries. In poor countries, food represents a much larger share of expenditure than in rich countries. Hence, all else being equal, we would expect the demand for food to be more price elastic in poorer countries. Of course, even in rich countries, the composition of spending on food may change considerably even if overall demand for food does not.

A significant portion of international trade occurs in intermediate productsproducts that are used as inputs into the production of other goods. Demand for these products derives from supply and demand decisions for the final products. However, the same basic considerations apply for intermediate products as for final products. Are there close substitutes for it in the production process? If not, its demand will tend to be less elastic than would be the case if there were readily available substitutes. How important is it to the overall economy? All else equal, the larger its share in overall production costs for the economy, the bigger its impact on production decisions and therefore the more price-elastic its derived demand. Oil is a classic example of a widely used input with few readily adoptable substitutes, at least in the short run. Lack of substitutes tends to make oil demand price-inelastic. However, it is so important in modern industrial economies that changes in its price can induce expansion or contraction of aggregate output. This makes short-run oil demand somewhat more elastic-at least for significant price changes. In the longer run, the feasibility of substitution among energy sources enhances the price sensitivity of oil demand.

Exhibit 10 shows estimates of demand elasticity for various products. The estimates range from essentially zero for pediatric doctor visits-a necessity for which there is virtually no substitute-to 3.8 for Coca-Cola, a specific brand for which there are many substitutes. Note that the elasticity of demand for soft drinks in general is much lower than for Coca-Cola, roughly 0.9. The elasticity of demand for rice in Japan versus in Bangladesh clearly illustrates the impact of expenditure share on price sensitivity. Similarly, although air travel for pleasure (a luxury) is quite price elastic, demand for first-class air travel is fairly insensitive to price. This is most likely because many first-class passengers are either traveling on business (presumably deemed to have high value added) or wealthy enough that the cost of first-class airfare is inconsequential.

\section{Exhibit 10: Estimates of Demand Elasticities}
\begin{center}
\begin{tabular}{|c|c|c|}
\hline
Product Description & Elasticity & Rationale/Comment \\
\hline
\multicolumn{3}{|l|}{Travel and transport} \\
\hline
\multicolumn{3}{|l|}{Airline travel (US)} \\
\hline
For pleasure & 1.5 & Luxury \\
\hline
1st class & 0.3 & Business and wealthy travelers \\
\hline
Car fuel (US, long term) & 0.3 &  \\
\hline
Bus travel (US) & 0.2 &  \\
\hline
Ford compact car & 2.8 & Large purchase; specific brand \\
\hline
\multicolumn{3}{|l|}{Food and beverages} \\
\hline
Rice &  & Necessity; staple food \\
\hline
Bangladesh & 0.8 & Poor country \\
\hline
Japan & 0.3 & Wealthy country \\
\hline
\multicolumn{3}{|l|}{Soft drinks} \\
\hline
All & $0.8-1.0$ &  \\
\hline
Coca-Cola & 3.8 & Specific brand; competitive market \\
\hline
\multicolumn{3}{|l|}{Medical care (US)} \\
\hline
Health insurance & 0.3 &  \\
\hline
Pediatric doctor visit & $0.0-0.1$ & No good substitute \\
\hline
Materials and energy &  & Necessary inputs \\
\hline
Steel & $0.2-0.3$ &  \\
\hline
Oil & 0.4 &  \\
\hline
\end{tabular}
\end{center}

Sources: Various studies cited in Wikipedia, "Price Elasticity of Demand," accessed August 2018 (http:// \href{http://en.wikipedia.org/wiki/Price_elasticity_of_demand}{en.wikipedia.org/wiki/Price\_elasticity\_of\_demand}).

In practice, most countries import and export a variety of products. Hence, the overall price elasticities of their imports and exports reflect a composite of the products they trade. In conjunction with the Marshall-Lerner condition, our review of the factors that determine price elasticities suggests that exchange rate changes will be a more-effective mechanism for trade balance adjustment if a country imports and exports the following:

\begin{itemize}
  \item Goods for which there are good substitutes

  \item Goods that trade in competitive markets

  \item Luxury goods, rather than necessities

  \item Goods that represent a large portion of consumer expenditures or a large portion of input costs for final producers

\end{itemize}

Note that each of these conditions is associated with higher demand elasticities $\left(\varepsilon_{X}\right.$ and $\left.\varepsilon_{M}\right)$.

Even when the Marshall-Lerner condition is satisfied, it is still possible that devaluation (in a fixed parity regime) or depreciation (in a floating regime) of the currency will initially make the trade balance worse before making it better. This effect, called the $J$-curve effect, is illustrated in Exhibit 11.

\section{Exhibit 11: Trade Balance Dynamics: The J-Curve}
Trade Balance

\begin{center}
\includegraphics[max width=\textwidth]{2023_05_04_ccb578c5089fd41615beg-533}
\end{center}

In the very short run, the $J$-curve reflects the order delivery lags that take place in import and export transactions. Imagine a clothing importer in Washington. Orders are placed in January for French spring fashions. Market forces cause the dollar to depreciate in February, but contracts were already signed for payment in euros. When the fashions arrive in March, more dollars have to go out to pay for the order signed in euros. Thus, the trade balance gets worse. However, after the depreciation, the clothing importer has to put in new orders for summer fashions. As a result of the currency depreciation, the French summer fashions are now more expensive, so the clothing store cuts the demand for imported clothes from France. The depreciation eventually improves the trade balance, even though it initially made it worse.

A J-curve pattern may also arise if short-term price elasticities do not satisfy the Marshall-Lerner condition but long-term elasticities do satisfy it. As noted above in the case of oil, significant changes in spending patterns often take time. Thus, the trade balance may worsen initially and then gradually improve following a depreciation of the currency as firms and consumers adapt.

\section{EXCHANGE RATES AND THE TRADE BALANCE: THE ABSORPTION APPROACH}
explain the effects of exchange rates on countries' international trade
and capital flows

The elasticities approach focuses on the expenditure-switching effect of changing the relative prices of imports and exports. It is essentially a microeconomic view of the relationship between exchange rates and the trade balance. The absorption approach adopts an explicitly macroeconomic view of this relationship. Recall from above that the trade balance is equal to the country's saving, including the government fiscal balance, minus its investment in new plants and equipment. Equivalently, it is equal to the difference between income (GDP) and domestic expenditure, or absorption. Thus, in order to move the trade balance toward surplus, a devaluation/depreciation of the domestic currency must increase income relative to expenditure or, equivalently, increase national saving relative to investment in physical capital.

If there is excess capacity in the economy, then by switching demand toward domestically produced goods and services, depreciation of the currency can increase output/ income. Because some of the additional income will be saved, income rises relative to expenditure and the trade balance improves. If the economy is at full employment, however, the trade balance cannot improve unless domestic expenditure declines. If expenditure does not decline, then the depreciation will put upward pressure on domestic prices until the stimulative effect of the exchange rate change is negated by the higher price level and the trade balance reverts to its original level.

How might depreciation of the currency reduce domestic expenditure relative to income? The main mechanism is a wealth effect. A weaker currency reduces the purchasing power of domestic-currency-denominated assets (including the present value of current and future earned income). Households respond by reducing expenditure and increasing saving in order to rebuild their wealth. Of course, as real wealth is rebuilt, the effect on saving is likely to be reversed-resulting in only a temporary improvement in the trade balance. Thus, in the absence of excess capacity in the economy, currency depreciation is likely to provide only a temporary solution for a chronic trade imbalance. Lasting correction of the imbalance requires more fundamental changes in expenditure/saving behavior (e.g., a policy shift that improves the fiscal balance or an increase in saving relative to capital investment induced by an increase in real interest rates).

The absorption approach also reminds us that currency depreciation cannot improve the trade balance unless it also induces a corresponding change in the capital account. Not only must domestic saving increase, but that saving must also be willingly channeled into buying financial assets from foreigners. All else equal, this implies that foreign and domestic asset prices must change such that foreign assets become relatively more attractive and domestic assets relatively less attractive to both foreign and domestic investors.

\section{EXAMPLE 8}
\section{Exchange Rates and the Trade Balance}
An analyst at a foreign exchange dealing bank is examining the exchange rate for the Australian dollar (AUD), which is a freely floating currency. Currently, Australia is running a trade surplus with the rest of the world, primarily reflecting strong demand for Australian resource exports generated by rapid growth in emerging market economies in the Western Pacific region. In turn, Australia imports food and energy from a variety of foreign countries that compete with each other as well as with Australian producers of these products. The analyst uses data in the following table to estimate the effect of changes in the AUD exchange rate on Australia's balance of trade.

\begin{center}
\begin{tabular}{lcc}
\hline
 & $\begin{array}{c}\text { Volume } \\ \text { (AUD billions) }\end{array}$ & Demand Elasticity \\
\hline
Exports & 200 & 0.3 \\
Imports & 180 & 0.6 \\
\hline
\end{tabular}
\end{center}

The analyst's research report on this topic notes that the mix of products that Australia imports and exports seems to be changing and that this will affect the relation between the exchange rates and the trade surplus. The proportion of Australian exports accounted for by fine wines is increasing. These are considered a luxury good and must compete with increased wine exports from comparable-producing regions (such as Chile and New Zealand). At the same time, rising income levels in Australia are allowing the country to increase the proportion of its imports accounted for by luxury goods, and these represent a rising proportion of consumer expenditures. The analyst's report states: "Given the changing export mix, an appreciation of the currency will be more likely to reduce Australia's trade surplus. In contrast, the changing import mix will have the opposite effect."

\begin{enumerate}
  \item Given the data in the table, an appreciation in the AUD will:
\end{enumerate}

A. cause the trade balance to increase.

B. cause the trade balance to decrease

C. have no effect on the trade balance.

\section{Solution to 1:}
A is correct. As the AUD appreciates, the price of exports to offshore buyers goes up and they demand fewer of them; hence, the AUD-denominated revenue from exports decreases. (Although export demand is inelastic, or $\varepsilon_{X}<1$, recall that the Australian price of these exports is assumed not to have changed, so the amount of export revenue received by Australia, in AUD-terms, unambiguously declines as the quantity of exports declines.) Australian expenditure for imports also declines. Although the price of imports declines as the AUD appreciates, the Australians do not increase their import purchases enough to lead to higher expenditures. This is because import demand is also inelastic $\left(\varepsilon_{M}<1\right)$. This effect on import expenditure can be seen from: $\% \Delta R_{M}=\left(1-\varepsilon_{M}\right) \% \Delta P_{M}$, where $\% \Delta P_{M}$ is negative (import prices are declining) and import demand is inelastic $\left(\right.$ so $\left.\left(1-\varepsilon_{M}\right)>0\right)$. With both import expenditures and export revenues declining, the net effect on the trade balance comes down to the relative size of the import and export weights $\left(\omega_{M}\right.$ and $\omega_{X}$, respectively). In this case, $\omega_{X}=0.53$ (i.e., 200/380) and $\omega_{M}=0.47$ (i.e., 180/380). Putting this into the Marshall-Lerner equation leads to:

$$
\omega_{X} \varepsilon_{X}+\omega_{M}\left(\varepsilon_{M}-1\right)=0.53 \times 0.3+0.47(0.6-1)=-0.03
$$

Because the Marshall-Lerner condition is not satisfied, exchange rate movements do not move the trade balance in the expected direction [i.e., appreciation (depreciation) of the currency does not decrease (increase) the trade balance]. However, note that with different import/export weights and the same elasticities, the Marshall-Lerner condition would be met. In particular, the condition would be met for any value of $\omega_{X}$ greater than $4 / 7$ $(\approx 0.571)$

\begin{enumerate}
  \setcounter{enumi}{1}
  \item All else equal, an appreciation in the AUD will be more likely to reduce the trade surplus if the demand:
\end{enumerate}

A. elasticities for imports and exports increase.

B. elasticity for exports and the export share in total trade decrease. C. elasticity for imports decreases and the import share in total trade increases.

\section{Solution to 2:}
A is correct. The basic intuition of the Marshall-Lerner condition is that in order for an exchange rate movement to rebalance trade, the demands for imports and exports must be sufficiently price-sensitive (i.e., they must have sufficiently high elasticities). However, the relative share of imports and exports in total trade must also be considered. The generalized MarshallLerner condition requires:

$$
\omega_{X} \varepsilon_{X}+\omega_{M}\left(\varepsilon_{M}-1\right)>0
$$

An increase in both $\varepsilon_{X}$ and $\varepsilon_{M}$ will clearly make this expression increase (A is correct). In contrast, a decrease in both $\omega_{X}$ and $\varepsilon_{X}$ tends to make the expression smaller (B is incorrect). ${ }^{23}$ If $\varepsilon_{M}$ decreases and $\omega_{M}$ increases, import demand will respond less to an exchange rate movement and will have a larger role in determining the trade balance ( $\mathrm{C}$ is incorrect).

\begin{enumerate}
  \setcounter{enumi}{2}
  \item All else equal, an appreciation in the AUD will be more likely to reduce the trade surplus if it leads to an increase in Australian:
\end{enumerate}

A. tax receipts.

B. private sector investment.

C. government budget surpluses.

\section{Solution to 3:}
B is correct. An Australian trade surplus means that Australia is spending less than it earns and is accumulating claims on foreigners. Equivalently, Australian saving, inclusive of both private saving and the government fiscal balance, is more than sufficient to fund Australian private sector investment. The relationship between the trade balance and expenditure/saving decisions is given by:

$X-M=(S-I)+(T-G)>0$

For Australia's trade balance to decline, it must save less ( $S$ down), invest more ( $I$ up), decrease its fiscal balance ( $T-G$ down), or some combination of these. Increasing tax receipts ( $T$ up) increases rather than decreases the fiscal balance, so answer A is incorrect. Similarly, answer C, increasing the government budget surplus, is incorrect. Increasing private investment ( $I$ up) does decrease the trade balance, so answer B is correct.

\begin{enumerate}
  \setcounter{enumi}{3}
  \item The report's statement about the effect of changing import and export mixes is most likely:
\end{enumerate}

A. correct.

B. incorrect with respect to the import effect.

C. incorrect with respect to the export effect.

23 Because $\omega_{M}=1-\omega_{X}$ and $\varepsilon_{M}<1$ in this example, a decrease in $\omega_{X}$ also decreases the second terms, $\omega_{M}\left(\varepsilon_{M}-1\right)$, in the Marshall-Lerner condition.

\section{Solution to 4:}
B is correct. As Australian exports become more dominated by luxury goods that face highly competitive market conditions, the elasticity of export demand $\left(\varepsilon_{X}\right)$ is likely to be increasing. Increasing export elasticity makes the trade surplus more responsive to an AUD appreciation (the increase in $\varepsilon_{X}$ will tend to increase the computed value for the Marshall-Lerner equation). Similarly, as Australian imports become more dominated by luxury goods that are an increasing proportion of household expenditure, import elasticity $\left(\varepsilon_{M}\right)$ will most likely increase. This will also tend to increase the computed value for the Marshall-Lerner equation.

\begin{enumerate}
  \setcounter{enumi}{4}
  \item Suppose the Australian government imposed capital controls that prohibited the flow of financial capital into or out of the country. What impact would this have on the Australian trade balance?
\end{enumerate}

A. The trade surplus would increase.

B. The trade balance would go to zero.

C. The trade balance would not necessarily be affected.

\section{Solution to 5:}
B is correct. A trade deficit (surplus) must be exactly matched by an offsetting capital account surplus (deficit). Anything that impacts the trade balance must impact the capital account, and vice versa. If capital flows are prohibited, then both the capital account and the trade balance must be zero.

\begin{enumerate}
  \setcounter{enumi}{5}
  \item Suppose the Australian government imposed capital controls that prohibited the flow of financial capital into or out of the country. The impact on the trade balance, if any, would most likely take the form of:
A. a decrease in private saving.
B. a decrease in private investment.
C. an increase in the government fiscal balance.
\end{enumerate}

\section{Solution to 6:}
A is correct. The trade balance must go to zero. An increase in the fiscal balance implies an increase in the existing trade surplus, so answer $C$ is incorrect. A decrease in private investment will also cause an increase in the trade surplus, so answer B is incorrect. A decrease in private saving will decrease the trade surplus as required, so answer A is correct: A decrease in saving will most likely reflect a decline in national income, especially the profit component, as export demand is choked off by the inability to extend credit to foreigners.

\section{SUMMARY}
Foreign exchange markets are crucial for understanding both the functioning of the global economy as well as the performance of investment portfolios. In this reading, we have described the diverse array of FX market participants and have introduced some of the basic concepts necessary to understand the structure and functions of these markets. The reader should be able to understand how exchange rates-both spot and forward-are quoted and be able to calculate cross exchange rates and forward rates. We also have described the array of exchange rate regimes that characterize foreign exchange markets globally and how these regimes determine the flexibility of exchange rates, and hence, the degree of foreign exchange rate risk that international investments are exposed to. Finally, we have discussed how movements in exchange rates affect international trade flows (imports and exports) and capital flows.

The following points, among others, are made in this reading:

\begin{itemize}
  \item Measured by average daily turnover, the foreign exchange market is by far the largest financial market in the world. It has important effects, either directly or indirectly, on the pricing and flows in all other financial markets.

  \item There is a wide diversity of global FX market participants that have a wide variety of motives for entering into foreign exchange transactions.

  \item Individual currencies are usually referred to by standardized three-character codes. These currency codes can also be used to define exchange rates (the price of one currency in terms of another). There are a variety of exchange rate quoting conventions commonly used.

  \item A direct currency quote takes the domestic currency as the price currency and the foreign currency as the base currency (i.e., $S_{d / f}$ ). An indirect quote uses the domestic currency as the base currency (i.e., $S_{f / d}$ ). To convert between direct and indirect quotes, the inverse (reciprocal) is used.

\end{itemize}

Professional FX markets use standardized conventions for how the exchange rate for specific currency pairs will be quoted.

\begin{itemize}
  \item Currencies trade in foreign exchange markets based on nominal exchange rates. An increase (decrease) in the exchange rate, quoted in indirect terms, means that the domestic currency is appreciating (depreciating) versus the foreign currency.

  \item The real exchange rate, defined as the nominal exchange rate multiplied by the ratio of price levels, measures the relative purchasing power of the currencies. An increase in the real exchange rate $\left(R_{d / f}\right)$ implies a reduction in the relative purchasing power of the domestic currency.

  \item Given exchange rates for two currency pairs-A/B and A/C-we can compute the cross-rate $(\mathrm{B} / \mathrm{C})$ between currencies $B$ and C. Depending on how the rates are quoted, this may require inversion of one of the quoted rates.

  \item Spot exchange rates are for immediate settlement (typically, $T+2$ ), while forward exchange rates are for settlement at agreed-upon future dates. Forward rates can be used to manage foreign exchange risk exposures or can be combined with spot transactions to create FX swaps.

  \item The spot exchange rate, the forward exchange rate, and the domestic and foreign interest rates must jointly satisfy an arbitrage relationship that equates the investment return on two alternative but equivalent investments. Given the spot exchange rate and the foreign and domestic interest rates, the forward exchange rate must take the value that prevents riskless arbitrage.

  \item Forward rates are typically quoted in terms of forward (or swap) points. The swap points are added to the spot exchange rate in order to calculate the forward rate. Occasionally, forward rates are presented in terms of percentages relative to the spot rate. - The base currency is said to be trading at a forward premium if the forward rate is above the spot rate (forward points are positive). Conversely, the base currency is said to be trading at a forward discount if the forward rate is below the spot rate (forward points are negative).

  \item The currency with the higher (lower) interest rate will trade at a forward discount (premium).

  \item Swap points are proportional to the spot exchange rate and to the interest rate differential and approximately proportional to the term of the forward contract.

  \item Empirical studies suggest that forward exchange rates may be unbiased predictors of future spot rates, but the margin of error on such forecasts is too large for them to be used in practice as a guide to managing exchange rate exposures. FX markets are too complex and too intertwined with other global financial markets to be adequately characterized by a single variable, such as the interest rate differential.

  \item Virtually every exchange rate is managed to some degree by central banks. The policy framework that each central bank adopts is called an exchange rate regime. These regimes range from using another country's currency (dollarization), to letting the market determine the exchange rate (independent float). In practice, most regimes fall in between these extremes. The type of exchange rate regime used varies widely among countries and over time.

  \item An ideal currency regime would have three properties: (1) the exchange rate between any two currencies would be credibly fixed; (2) all currencies would be fully convertible; and (3) each country would be able to undertake fully independent monetary policy in pursuit of domestic objectives, such as growth and inflation targets. However, these conditions are inconsistent. In particular, a fixed exchange rate and unfettered capital flows severely limit a country's ability to undertake independent monetary policy. Hence, there cannot be an ideal currency regime.

  \item The IMF identifies the following types of regimes: arrangements with no separate legal tender (dollarization, monetary union), currency board, fixed parity, target zone, crawling peg, crawling band, managed float, and independent float. Most major currencies traded in FX markets are freely floating, albeit subject to occasional central bank intervention.

  \item A trade surplus (deficit) must be matched by a corresponding deficit (surplus) in the capital account. Any factor that affects the trade balance must have an equal and opposite impact on the capital account, and vice versa.

  \item A trade surplus reflects an excess of domestic saving (including the government fiscal balance) over investment spending. A trade deficit indicates that the country invests more than it saves and must finance the excess by borrowing from foreigners or selling assets to foreigners.

  \item The impact of the exchange rate on trade and capital flows can be analyzed from two perspectives. The elasticities approach focuses on the effect of changing the relative price of domestic and foreign goods. This approach highlights changes in the composition of spending. The absorption approach focuses on the impact of exchange rates on aggregate expenditure/saving decisions.

  \item The elasticities approach leads to the Marshall-Lerner condition, which describes combinations of export and import demand elasticities such that depreciation (appreciation) of the domestic currency will move the trade balance toward surplus (deficit). - The idea underlying the Marshall-Lerner condition is that demand for imports and exports must be sufficiently price-sensitive so that an increase in the relative price of imports increases the difference between export receipts and import expenditures.

  \item In order to move the trade balance toward surplus (deficit), a change in the exchange rate must decrease (increase) domestic expenditure (also called absorption) relative to income. Equivalently, it must increase (decrease) domestic saving relative to domestic investment.

  \item If there is excess capacity in the economy, then currency depreciation can increase output/income by switching demand toward domestically produced goods and services. Because some of the additional income will be saved, income rises relative to expenditure and the trade balance improves.

  \item If the economy is at full employment, then currency depreciation must reduce domestic expenditure in order to improve the trade balance. The main mechanism is a wealth effect: A weaker currency reduces the purchasing power of domestic-currency-denominated assets (including the present value of current and future earned income), and households respond by reducing expenditure and increasing saving.

\end{itemize}

\section{REFERENCES}
Friedman, Milton. 1953. “The Monetarist Theory of Flexible Exchange Rate Systems." In Essays in Positive Economics. Chicago: University of Chicago Press.

International Monetary Fund (IMF) 2006. "De Facto Classification of Exchange Rate Regimes and Monetary Policy Framework" (\href{http://www.imf.org}{www.imf.org}).

\section{PRACTICE PROBLEMS}
\begin{enumerate}
  \item A decrease in the real exchange rate (quoted in terms of domestic currency per unit of foreign currency) is most likely to be associated with an increase in which of the following?
\end{enumerate}

A. Foreign price level.

B. Domestic price level.

C. Nominal exchange rate.

\begin{enumerate}
  \setcounter{enumi}{1}
  \item In order to minimize the foreign exchange exposure on a euro-denominated receivable due from a German company in 100 days, a British company would most likely initiate a:
A. spot transaction.
B. forward contract.
C. real exchange rate contract.

  \item Which of the following counterparties is most likely to be considered a sell-side foreign-exchange market participant?

\end{enumerate}

A. A large corporation that borrows in foreign currencies.

B. A sovereign wealth fund that influences cross-border capital flows.

C. A multinational bank that trades foreign exchange with its diverse client base.

\begin{enumerate}
  \setcounter{enumi}{3}
  \item What will be the effect on a direct exchange rate quote if the domestic currency appreciates?
A. Increase
B. Decrease
C. No change

  \item An executive from Switzerland checked into a hotel room in Spain and was told by the hotel manager that 1 EUR will buy 1.2983 CHF. From the executive's perspective, an indirect exchange rate quote would be:
A. 0.7702 EUR per CHF.
B. 0.7702 CHF per EUR.
C. 1.2983 EUR per CHF.

  \item Over the past month, the Swiss Franc (CHF) has depreciated 12 percent against pound sterling (GBP). How much has the pound sterling appreciated against the Swiss Franc?

\end{enumerate}

A. $12 \%$

B. Less than $12 \%$ C. More than $12 \%$

\begin{enumerate}
  \setcounter{enumi}{6}
  \item An exchange rate between two currencies has increased to 1.4500 . If the base currency has appreciated by $8 \%$ against the price currency, the initial exchange rate between the two currencies was closest to:
A. 1.3340 .
B. 1.3426 .
C. 1.5660 .
\end{enumerate}

\section{The following information relates to questions}
 8-9A dealer provides the following quotes:

\begin{center}
\begin{tabular}{lc}
\hline
Ratio & Spot rate \\
\hline
CNY/HKD & 0.8422 \\
CNY/ZAR & 0.9149 \\
CNY/SEK & 1.0218 \\
\hline
\end{tabular}
\end{center}

\begin{enumerate}
  \setcounter{enumi}{7}
  \item The spot $\mathrm{ZAR} / \mathrm{HKD}$ cross-rate is closest to:
A. 0.9205 .
B. 1.0864
C. 1.2978 .

  \item Another dealer is quoting the ZAR/SEK cross-rate at 1.1210. The arbitrage profit that can be earned is closest to:
A. ZAR 3671 per million SEK traded.
B. SEK 4200 per million ZAR traded.
C. ZAR 4200 per million SEK traded.

  \item A BRL/MXN spot rate is listed by a dealer at 0.1378 . The 6-month forward rate is 0.14193 . The 6-month forward points are closest to:
A. -41.3 .
B. +41.3 .
C. +299.7 .

  \item A three-month forward exchange rate in CAD/USD is listed by a dealer at 1.0123 . The dealer also quotes 3 -month forward points as a percentage at $6.8 \%$. The CAD/USD spot rate is closest to:
A. 0.9478 .
B. 1.0550 . C. 1.0862 .

  \item If the base currency in a forward exchange rate quote is trading at a forward discount, which of the following statements is most accurate?
A. The forward points will be positive.
B. The forward percentage will be negative.
C. The base currency is expected to appreciate versus the price currency.

  \item A forward premium indicates:
A. an expected increase in demand for the base currency.
B. the interest rate is higher in the base currency than in the price currency.
C. the interest rate is higher in the price currency than in the base currency.

  \item The JPY/AUD spot exchange rate is 82.42 , the JPY interest rate is $0.15 \%$, and the AUD interest rate is $4.95 \%$. If the interest rates are quoted on the basis of a 360-day year, the 90-day forward points in JPY/AUD would be closest to:
A. -377.0 .
B. -97.7.
C. 98.9 .

  \item An exchange rate:
A. is most commonly quoted in real terms.
B. is the price of one currency in terms of another.
C. between two currencies ensures they are fully convertible.

  \item Which of the following is not a condition of an ideal currency regime?

\end{enumerate}

A. Fully convertible currencies.

B. Fully independent monetary policy.

C. Independently floating exchange rates.

\begin{enumerate}
  \setcounter{enumi}{16}
  \item In practice, both a fixed parity regime and a target zone regime allow the exchange rate to float within a band around the parity level. The most likely rationale for the band is that the band allows the monetary authority to:
\end{enumerate}

A. be less active in the currency market.

B. earn a spread on its currency transactions.

C. exercise more discretion in monetary policy.

\begin{enumerate}
  \setcounter{enumi}{17}
  \item A fixed exchange rate regime in which the monetary authority is legally required to hold foreign exchange reserves backing $100 \%$ of its domestic currency issuance is best described as:
\end{enumerate}

A. dollarization. B. a currency board.

C. a monetary union.

\begin{enumerate}
  \setcounter{enumi}{18}
  \item A country with a trade deficit will most likely:
\end{enumerate}

A. have an offsetting capital account surplus.

B. save enough to fund its investment spending.

C. buy assets from foreigners to fund the imbalance.

\begin{enumerate}
  \setcounter{enumi}{19}
  \item A large industrialized country has recently devalued its currency in an attempt to correct a persistent trade deficit. Which of the following domestic industries is most likely to benefit from the devaluation?
A. Luxury cars.
B. Branded prescription drugs.
C. Restaurants and live entertainment venues.

  \item A country with a persistent trade surplus is being pressured to let its currency appreciate. Which of the following best describes the adjustment that must occur if currency appreciation is to be effective in reducing the trade surplus?

\end{enumerate}

A. Domestic investment must decline relative to saving.

B. Foreigners must increase investment relative to saving.

C. Global capital flows must shift toward the domestic market.

\section{SOLUTIONS}
\begin{enumerate}
  \item B is correct. The real exchange rate (quoted in terms of domestic currency per unit of foreign currency) is given by:
\end{enumerate}

\begin{center}
\includegraphics[max width=\textwidth]{2023_05_04_ccb578c5089fd41615beg-546}
\end{center}

An increase in the domestic price level $\left(P_{d}\right)$ decreases the real exchange rate because it implies an increase in the relative purchasing power of the domestic currency.

\begin{enumerate}
  \setcounter{enumi}{1}
  \item B is correct. The receivable is due in 100 days. To reduce the risk of currency exposure, the British company would initiate a forward contract to sell euros/buy pounds at an exchange rate agreed to today. The agreed-upon rate is called the forward exchange rate.

  \item C is correct. The sell side generally consists of large banks that sell foreign exchange and related instruments to buy-side clients. These banks act as market makers, quoting exchange rates at which they will buy (the bid price) or sell (the offer price) the base currency.

  \item B is correct. In the case of a direct exchange rate, the domestic currency is the price currency (the numerator) and the foreign currency is the base currency (the denominator). If the domestic currency appreciates, then fewer units of the domestic currency are required to buy 1 unit of the foreign currency and the exchange rate (domestic per foreign) declines. For example, if sterling (GBP) appreciates against the euro (EUR), then euro-sterling (GBP/EUR) might decline from 0.8650 to 0.8590 .

  \item A is correct. An indirect quote takes the foreign country as the price currency and the domestic country as the base currency. To get CHF-which is the executive's domestic currency-as the base currency, the quote must be stated as EUR/CHF. Using the hotel manager's information, the indirect exchange rate is $(1 / 1.2983)=0.7702$.

  \item $C$ is correct. The appreciation of sterling against the Swiss franc is simply the inverse of the $12 \%$ depreciation of the Swiss franc against Sterling: $[1 /(1-0.12)]-1$ $=(1 / 0.88)-1=0.1364$, or $13.64 \%$.

  \item B is correct. The percentage appreciation of the base currency can be calculated by dividing the appreciated exchange rate by the initial exchange rate. In this case, the unknown is the initial exchange rate. The initial exchange is the value of $X$ that satisfies the formula:

\end{enumerate}

$1.4500 / X=1.08$

Solving for $X$ leads to $1.45 / 1.08=1.3426$.

\begin{enumerate}
  \setcounter{enumi}{7}
  \item A is correct. To get to the ZAR/HKD cross-rate, it is necessary to take the inverse of the CNY/ZAR spot rate and then multiply by the CNY/HKD exchange rate:
\end{enumerate}

$$
\begin{aligned}
& \text { ZAR/HKD }=(\mathrm{CNY} / \mathrm{ZAR})^{-1} \times(\mathrm{CNY} / \mathrm{HKD}) \\
& =(1 / 0.9149) \times 0.8422=0.9205
\end{aligned}
$$

\begin{enumerate}
  \setcounter{enumi}{8}
  \item $\mathrm{C}$ is correct. The ZAR/SEK cross-rate from the original dealer is $(1.0218 / 0.9149)$ $=1.1168$, which is lower than the quote from the second dealer. To earn an arbitrage profit, a currency trader would buy SEK (sell ZAR) from the original dealer and sell SEK (buy ZAR) to the second dealer. On 1 million SEK the profit would be
\end{enumerate}

SEK $1,000,000 \times(1.1210-1.1168)=$ ZAR 4200

\begin{enumerate}
  \setcounter{enumi}{9}
  \item B is correct. The number of forward points equals the forward rate minus the spot rate, or $0.14193-0.1378=0.00413$, multiplied by $10,000: 10,000 \times 0.00413=$ 41.3 points. By convention, forward points are scaled so that \textbackslash pm 1 forward point corresponds to a change of \textbackslash pm 1 in the last decimal place of the spot exchange rate.

  \item A is correct. Given the forward rate and forward points as a percentage, the unknown in the calculation is the spot rate. The calculation is as follows:

\end{enumerate}

Spot rate $\times(1+$ Forward points as a percentage $)=$ Forward rate

Spot rate $\times(1+0.068)=1.0123$

Spot $=1.0123 / 1.068=0.9478$

\begin{enumerate}
  \setcounter{enumi}{11}
  \item B is correct. The base currency trading at a forward discount means that 1 unit of the base currency costs less for forward delivery than for spot delivery; i.e., the forward exchange rate is less than the spot exchange rate. The forward points, expressed either as an absolute number of points or as a percentage, are negative.

  \item $C$ is correct. To eliminate arbitrage opportunities, the spot exchange rate $(S)$, the forward exchange rate $(F)$, the interest rate in the base currency $\left(i_{b}\right)$, and the interest rate in the price currency $\left(i_{p}\right)$ must satisfy:

\end{enumerate}

$\frac{F}{S}=\left(\frac{1+i_{p}}{1+i_{b}}\right)$

According to this formula, the base currency will trade at forward premium $(F>$ $S)$ if, and only if, the interest rate in the price currency is higher than the interest rate in the base currency $\left(i_{p}>i_{b}\right)$.

\begin{enumerate}
  \setcounter{enumi}{13}
  \item B is correct. The forward exchange rate is given by
\end{enumerate}

$$
\begin{aligned}
& F_{J P Y / A U D}=S_{J P Y / A U D}\left(\frac{1+i_{J P Y} \tau}{1+i_{A U D} \tau}\right)=82.42\left(\frac{1+.0015\left(\frac{90}{360}\right)}{1+.0495\left(\frac{90}{360}\right)}\right) \\
= & 82.42 \times .98815=81.443
\end{aligned}
$$

The forward points are $100 \times(F-S)=100 \times(81.443-82.42)=100 \times(-0.977)$ $=-97.7$. Note that because the spot exchange rate is quoted with two decimal places, the forward points are scaled by 100 .

\begin{enumerate}
  \setcounter{enumi}{14}
  \item B is correct. The exchange rate is the number of units of the price currency that 1 unit of the base currency will buy. Equivalently, it is the number of units of the price currency required to buy 1 unit of the base currency.

  \item $\mathrm{C}$ is correct. An ideal currency regime would have credibly fixed exchange rates among all currencies. This would eliminate currency-related uncertainty with respect to the prices of goods and services as well as real and financial assets.

  \item $\mathrm{C}$ is correct. Fixed exchange rates impose severe limitations on the exercise of independent monetary policy. With a rigidly fixed exchange rate, domestic interest rates, monetary aggregates (e.g., money supply), and credit conditions are dictated by the requirement to buy/sell the currency at the rigid parity. Even a narrow band around the parity level allows the monetary authority to exercise some discretionary control over these conditions. In general, the wider the band, the more independent control the monetary authority can exercise.

  \item B is correct. With a currency board, the monetary authority is legally required to exchange domestic currency for a specified foreign currency at a fixed exchange rate. It cannot issue domestic currency without receiving foreign currency in exchange, and it must hold that foreign currency as a $100 \%$ reserve against the domestic currency issued. Thus, the country's monetary base (bank reserves plus notes and coins in circulation) is fully backed by foreign exchange reserves.

  \item A is correct. A trade deficit must be exactly matched by an offsetting capital account surplus to fund the deficit. A capital account surplus reflects borrowing from foreigners (an increase in domestic liabilities) and/or selling assets to foreigners (a decrease in domestic assets). A capital account surplus is often referred to as a "capital inflow" because the net effect is foreign investment in the domestic economy.

  \item A is correct. A devaluation of the domestic currency means domestic producers are cutting the price faced by their foreign customers. The impact on their unit sales and their revenue depends on the elasticity of demand. Expensive luxury goods exhibit high price elasticity. Hence, luxury car producers are likely to experience a sharp increase in sales and revenue due to the devaluation.

  \item $C$ is correct. The trade surplus cannot decline unless the capital account deficit also declines. Regardless of the mix of assets bought and sold, foreigners must buy more assets from (or sell fewer assets to) domestic issuers/investors.

\end{enumerate}

\section{Financial Statement Analysis}
\section*{LEARNING MODULE 
 1 }
\section{Introduction to Financial Statement Analysis}
Elaine Henry, PhD, CFA, is at Stevens Institute of Technology (USA). Thomas R. Robinson, PhD, CFA, CAIA, Robinson Global Investment Management LLC, (USA).

\section{LEARNING OUTCOME}
\begin{center}
\begin{tabular}{c|l}
Mastery & The candidate should be able to: \\
$\square \square$ & $\begin{array}{l}\text { describe the roles of financial reporting and financial statement } \\ \text { analysis } \\ \text { describe the roles of the statement of financial position, statement } \\ \text { of comprehensive income, statement of changes in equity, and } \\ \text { statement of cash flows in evaluating a company's performance and } \\ \text { financial position } \\ \text { describe the importance of financial statement notes and } \\ \text { supplementary information-including disclosures of accounting } \\ \text { policies, methods, and estimates-and management's commentary } \\ \text { describe the objective of audits of financial statements, the types of } \\ \text { audit reports, and the importance of effective internal controls } \\ \text { identify and describe information sources that analysts use in } \\ \text { financial statement analysis besides annual financial statements and } \\ \text { supplementary information } \\ \text { describe the steps in the financial statement analysis framework }\end{array}$ \\
$\square$ &  \\
\end{tabular}
\end{center}

\section{INTRODUCTION}
Financial analysis is the process of examining a company's performance in the context of its industry and economic environment in order to arrive at a decision or recommendation. Often, the decisions and recommendations addressed by financial analysts pertain to providing capital to companies-specifically, whether to invest in the company's debt or equity securities and at what price. An investor in debt securities is concerned about the company's ability to pay interest and to repay the principal lent. An investor in equity securities is an owner with a residual interest in the company and is concerned about the company's ability to pay dividends and the likelihood that its share price will increase. Overall, a central focus of financial analysis is evaluating the company's ability to earn a return on its capital that is at least equal to the cost of that capital, to profitably grow its operations, and to generate enough cash to meet obligations and pursue opportunities.

Fundamental financial analysis starts with the information found in a company's financial reports. These financial reports include audited financial statements, additional disclosures required by regulatory authorities, and any accompanying (unaudited) commentary by management. Basic financial statement analysis-as presented in this reading-provides a foundation that enables the analyst to better understand other information gathered from research beyond the financial reports.

This reading is organized as follows: Section 2 discusses the scope of financial statement analysis. Section 3 describes the sources of information used in financial statement analysis, including the primary financial statements (statement of financial position or balance sheet, statement of comprehensive income, statement of changes in equity, and cash flow statement). Section 4 provides a framework for guiding the financial statement analysis process. A summary of the key points conclude the reading.

\section{SCOPE OF FINANCIAL STATEMENT ANALYSIS}
describe the roles of financial reporting and financial statement
analysis

The role of financial statements issued by companies is to provide information about a company's performance, financial position, and changes in financial position that is useful to a wide range of users in making economic decisions. The role of financial statement analysis is to use financial reports prepared by companies, combined with other information, to evaluate the past, current, and potential performance and financial position of a company for the purpose of making investment, credit, and other economic decisions. Managers within a company perform financial analysis to make operating, investing, and financing decisions but do not necessarily rely on analysis of related financial statements. They have access to additional financial information that can be reported in whatever format is most useful to their decision.

In evaluating financial reports, analysts typically have a specific economic decision in mind. Examples of these decisions include the following:

\begin{itemize}
  \item Evaluating an equity investment for inclusion in a portfolio.

  \item Evaluating a merger or acquisition candidate.

  \item Evaluating a subsidiary or operating division of a parent company.

  \item Deciding whether to make a venture capital or other private equity investment.

  \item Determining the creditworthiness of a company in order to decide whether to extend a loan to the company and if so, what terms to offer.

  \item Extending credit to a customer.

  \item Examining compliance with debt covenants or other contractual arrangements.

  \item Assigning a debt rating to a company or bond issue.

  \item Valuing a security for making an investment recommendation to others. - Forecasting future net income and cash flow.

\end{itemize}

These decisions demonstrate certain themes in financial analysis. In general, analysts seek to examine the past and current performance and financial position of a company in order to form expectations about its future performance and financial position. Analysts are also concerned about factors that affect risks to a company's future performance and financial position. An examination of performance can include an assessment of a company's profitability (the ability to earn a profit from delivering goods and services) and its ability to generate positive cash flows (cash receipts in excess of cash disbursements). Profit and cash flow are not equivalent. Profit (or loss) represents the difference between the prices at which goods or services are provided to customers and the expenses incurred to provide those goods and services.

In addition, profit (or loss) includes other income (such as investing income or income from the sale of items other than goods and services) minus the expenses incurred to earn that income. Overall, profit (or loss) equals income minus expenses, and its recognition is mostly independent from when cash is received or paid. Example 1 illustrates the distinction between profit and cash flow.

\section{EXAMPLE 1}
\section{Profit versus Cash Flow}
Sennett Designs (SD) sells furniture on a retail basis. SD began operations during December 2017 and sold furniture for $€ 250,000$ in cash. The furniture sold by SD was purchased on credit for $€ 150,000$ and delivered by the supplier during December. The credit terms granted by the supplier required SD to pay the $€ 150,000$ in January for the furniture it received during December. In addition to the purchase and sale of furniture, in December, SD paid $€ 20,000$ in cash for rent and salaries.

\begin{enumerate}
  \item How much is SD's profit for December 2017 if no other transactions occurred?
\end{enumerate}

\section{Solution to 1:}
SD's profit for December 2017 is the excess of the sales price $(€ 250,000)$ over the cost of the goods that were sold $(€ 150,000)$ and rent and salaries $(€ 20,000)$, or $€ 80,000$.

\begin{enumerate}
  \setcounter{enumi}{1}
  \item How much is SD's cash flow for December 2017?
\end{enumerate}

\section{Solution to 2:}
The December 2017 cash flow is $€ 230,000$, the amount of cash received from the customer $(€ 250,000)$ less the cash paid for rent and salaries $(€ 20,000)$.

\begin{enumerate}
  \setcounter{enumi}{2}
  \item If SD purchases and sells exactly the same amount in January 2018 as it did in December and under the same terms (receiving cash for the sales and making purchases on credit that will be due in February), how much will the company's profit and cash flow be for the month of January?
\end{enumerate}

\section{Solution to 3:}
SD's profit for January 2018 will be identical to its profit in December: $€ 80,000$, calculated as the sales price $(€ 250,000)$ minus the cost of the goods that were sold $(€ 150,000)$ and minus rent and salaries $(€ 20,000)$. SD's cash flow in January 2018 will also equal $€ 80,000$, calculated as the amount of cash received from the customer $(€ 250,000)$ minus the cash paid for rent and salaries ( $€ 20,000)$ and minus the $€ 150,000$ that SD owes for the goods it had purchased on credit in the prior month.

Although profitability is important, so is a company's ability to generate positive cash flow. Cash flow is important because, ultimately, the company needs cash to pay employees, suppliers, and others in order to continue as a going concern. A company that generates positive cash flow from operations has more flexibility in funding needed for investments and taking advantage of attractive business opportunities than an otherwise comparable company without positive operating cash flow. Additionally, a company needs cash to pay returns (interest and dividends) to providers of debt and equity capital.

The expected magnitude of future cash flows is important in valuing corporate securities and in determining the company's ability to meet its obligations. The ability to meet short-term obligations is generally referred to as liquidity, and the ability to meet long-term obligations is generally referred to as solvency. Cash flow in any given period is not, however, a complete measure of performance for that period because, as shown in Example 1, a company may be obligated to make future cash payments as a result of a transaction that generates positive cash flow in the current period.

Profits may provide useful information about cash flows, past and future. If the transaction of Example 1 were repeated month after month, the long-term average monthly cash flow of SD would equal $€ 80,000$, its monthly profit. Analysts typically not only evaluate past profitability but also forecast future profitability.

Exhibit 1 shows how news coverage of corporate earnings announcements places corporate results in the context of analysts' expectations. Panel A shows the earnings announcement, and Panel B shows a sample of the news coverage of the announcement. Earnings are also frequently used by analysts in valuation. For example, an analyst may value shares of a company by comparing its price-to-earnings ratio (P/E) to the $\mathrm{P} /$ Es of peer companies and/or may use forecasted future earnings as direct or indirect inputs into discounted cash flow models of valuation.

\section{Exhibit 1: An Earnings Release and News Media Comparison with Analysts'}
Expectations

\section{Panel A: Excerpt from Apple Earnings Release}
\section{Apple Reports Second Quarter Results}
Revenue Grows 16 Percent and EPS Grows 30 Percent to New March Quarter Records

New \$100 Billion Share Repurchase Authorization Announced, Dividend Raised by 16 Percent

Cupertino, California-May 1, 2018-Apple today announced financial results for its fiscal 2018 second quarter ended March 31, 2018. The Company posted quarterly revenue of $\$ 61.1$ billion, an increase of 16 percent from the year-ago quarter, and quarterly earnings per diluted share of $\$ 2.73$, up 30 percent. International sales accounted for 65 percent of the quarter's revenue.

"We're thrilled to report our best March quarter ever, with strong revenue growth in iPhone, Services and Wearables," said Tim Cook, Apple's CEO. "Customers chose iPhone $\mathrm{X}$ more than any other iPhone each week in the March quarter, just as they did following its launch in the December quarter. We also grew revenue in all of our geographic segments, with over $20 \%$ growth in Greater China and Japan." "Our business performed extremely well during the March quarter, as we grew earnings per share by 30 percent and generated over $\$ 15$ billion in operating cash flow," said Luca Maestri, Apple's CFO. "With the greater flexibility we now have from access to our global cash, we can more efficiently invest in our US operations and work toward a more optimal capital structure. Given our confidence in Apple's future, we are very happy to announce that our Board has approved a new $\$ 100$ billion share repurchase authorization and a 16 percent increase in our quarterly dividend."

The Company will complete the execution of the previous $\$ 210$ billion share repurchase authorization during the third fiscal quarter.

Reflecting the approved increase, the Board has declared a cash dividend of $\$ 0.73$ per share of Apple's common stock payable on May 17, 2018 to shareholders of record as of the close of business on May 14, 2018.

The Company also expects to continue to net-share-settle vesting restricted stock units.

From the inception of its capital return program in August 2012 through March 2018, Apple has returned $\$ 275$ billion to shareholders, including $\$ 200$ billion in share repurchases. The management team and the Board will continue to review each element of the capital return program regularly and plan to provide an update on the program on an annual basis.

Apple is providing the following guidance for its fiscal 2018 third quarter:

\begin{itemize}
  \item revenue between $\$ 51.5$ billion and $\$ 53.5$ billion

  \item gross margin between 38 percent and 38.5 percent

  \item operating expenses between $\$ 7.7$ billion and $\$ 7.8$ billion

  \item other income/(expense) of $\$ 400$ million

  \item tax rate of approximately 14.5 percent

\end{itemize}

Apple will provide live streaming of its Q2 2018 financial results conference call beginning at 2:00 p.m. PDT on May 1, 2018 at www.apple. com/investor/earnings-call/. This webcast will also be available for replay for approximately two weeks thereafter.

Source: \href{https://www.apple.com/newsroom/2018/05/apple-reports-secon}{https://www.apple.com/newsroom/2018/05/apple-reports-secon} d-quarter-results/ (retrieved 3 November 2018)

\section{Panel B: Excerpt from News Article: Apple Second Quarter 2018 Earnings Release}
Apple reported quarterly earnings and revenue on Tuesday that beat expectations, but sold fewer iPhones than expected.

Shares rose as much 5 percent after hours, as investors digested the company's better-than-expected outlook for the current quarter, and a hefty capital return program.

The soft iPhone sales were still up from a year ago, and Apple CEO Tim Cook said in a statement that customers "chose iPhone $X$ more than any other iPhone each week in the March quarter."

\begin{itemize}
  \item Earnings per share: $\$ 2.73$ vs. $\$ 2.67$, adjusted, expected by a Thomson Reuters consensus estimate

  \item Revenue: $\$ 61.1$ billion vs. $\$ 60.82$ billion expected by Thomson Reuters consensus

  \item iPhone unit sales: 52.2 million vs. 52.54 million expected by a StreetAccount estimate - Fiscal Q3 revenue guidance: $\$ 51.5$ billion to $\$ 53.5$ billion vs. $\$ 51.61$ billion expected by Thomson Reuters consensus

\end{itemize}

Net income was $\$ 13.82$ billion, up from $\$ 11.03$ billion a year ago. A year ago, Apple earned $\$ 2.10$ a share on revenue of $\$ 52.9$ billion.

Source: \href{https://cnbc.com/2018/05/01/apple-earnings-q2-2018.html}{https://cnbc.com/2018/05/01/apple-earnings-q2-2018.html} (retrieved 3 November 2018)

Analysts are also interested in the financial position of a company. The financial position can be measured by comparing the resources controlled by the company (assets) in relation to the claims against those resources (liabilities and equity). The combination of liabilities and equity used to finance its assets represents the capital structure of the company. An example of a resource is cash. In Example 1, if no other transactions occur, the company should have $€ 230,000$ more in cash at 31 December 2017 than at the start of the period. The cash can be used by the company to pay its obligation to the supplier (a claim against the company) and may also be used to make distributions to the owner (who has a residual claim against the company's assets, net of liabilities).

Financial position and capital structure are particularly important in credit analysis, as depicted in Exhibit 2. Panel A of the exhibit is an excerpt from the company's annual earnings release highlighting the cumulative profitability, strong cash flow, strong balance sheet, and strong return on invested capital. Panel B of the exhibit is an excerpt from an August 2017 news article about an increase in the credit rating of Southwest Airlines due to a long history of profitability and a conservative capital structure.

\section{Exhibit 2:}
Panel A: Excerpt from Earnings Announcement by Southwest Airlines

\section{Southwest Airlines Reports Fourth Quarter and Record Annual Profit; 44th Consecutive Year of Profitability}
DALLAS, Jan. 26, 2017 /PRNewswire/ -- Southwest Airlines Co. (NYSE:LUV) (the "Company") today reported its fourth quarter and annual 2016 results:

Fourth quarter net income of $\$ 522$ million, or $\$ .84$ per diluted share, compared with fourth quarter 2015 net income of $\$ 536$ million, or $\$ .82$ per diluted share.

Excluding special items, fourth quarter net income of $\$ 463$ million, or $\$ .75$ per diluted share, compared with fourth quarter 2015 net income of $\$ 591$ million, or $\$ .90$ per diluted share. This exceeded the First Call fourth quarter 2016 consensus estimate of $\$ .70$ per diluted share.

Record annual net income of $\$ 2.24$ billion, or $\$ 3.55$ per diluted share, compared with 2015 net income of $\$ 2.18$ billion, or $\$ 3.27$ per diluted share.

Excluding special items, record annual net income of $\$ 2.37$ billion, or $\$ 3.75$ per diluted share, compared with 2015 net income of $\$ 2.36$ billion, or $\$ 3.52$ per diluted share.

Annual operating income of $\$ 3.76$ billion, resulting in an operating margin of 18.4 percent.

Excluding special items, annual operating income of $\$ 3.96$ billion, resulting in an operating margin of 19.4 percent.

Record annual operating cash flow of $\$ 4.29$ billion, and record annual free cash flow1 of $\$ 2.25$ billion. Returned \$1.97 billion to Shareholders in 2016, through a combination of $\$ 222$ million in dividends and $\$ 1.75$ billion in share repurchases.

Annual return on invested capital (ROIC) 1 of 30.0 percent.

Gary C. Kelly, Chairman of the Board and Chief Executive Officer, stated, "We are delighted to report record annual profits for 2016, our 44th consecutive year of profitability. Our total operating revenues reached a record $\$ 20.4$ billion, with sustained demand for our legendary low fares and superior Customer Service. Our profit margins were very strong, and our ROIC was a near-record 30.0 percent. Our record profits and balance sheet discipline generated record free cash flow, allowing us to return significant value to our Shareholders. Operationally, our performance was also very solid. We carried a record number of Customers while improving our ontime performance, baggage delivery rate, and net promoter score. My thanks and congratulations to the superb People of Southwest for these outstanding results, which earned them $\$ 586$ million in profit sharing during 2016.

"We ended the year with a solid fourth quarter 2016 performance. Total operating revenues grew 2.0 percent, year-over-year, to a fourth quarter record $\$ 5.1$ billion, exceeding our expectations as of the beginning of the fourth quarter. Travel demand and close-in yields improved post-election. In addition, December business travel was stronger than anticipated leading up to the holiday period. Based on current bookings and revenue trends, we estimate first quarter 2017 operating unit revenues will be flat to down one percent, year-over-year. This represents a continued and sequential improvement from the 2.9 percent operating unit revenue year-over-year decline in fourth quarter 2016, which is an encouraging start to the year.

"As expected, our fourth quarter unit costs increased, year-over-year, due to higher fuel costs, pay increases from amended union contracts, and additional depreciation expense associated with the accelerated retirement of our Boeing 737-300 aircraft. While inflationary cost pressures are expected in 2017 due to the union contract pay increases, we are continuing our efforts to drive offsetting cost efficiencies through fleet modernization and ongoing technology investments in our operations.

..."As we close out a year of record results, we begin 2017 with momentum and enthusiasm. We are on track to open a new international terminal in Fort Lauderdale, along with the launch of new service, this June. We are on track to launch the new Boeing 737-8 in the fall. And, we are encouraged by recent revenue trends, as well as the prospects for continued economic growth and moderate fuel prices. We are excited about our current outlook for another strong year with opportunities to win more Customers and reward our People and our Shareholders."

Source: \href{http://www.southwestairlinesinvestorrelations.com/}{http://www.southwestairlinesinvestorrelations.com/} news-and-events/news-releases/2017/01-26-2017-111504198 (retrieved November 3, 2018)

\section{Panel B: Excerpt from News Article About Southwest Airlines}
\section{Southwest Wins Another Credit Rating Upgrade}
Citing its "consistent record of profitability," S\&P Global Ratings upgraded Southwest Airlines (LUV) on Monday to a triple-B-plus rating from triple-B.

This line from the report sums up the rationale for the upgrade nicely:

Southwest is the only U.S. airline that has recorded 44 consecutive years of profitability and remains the largest low-cost airline in the world. The credit rating agency commends the airline for its record despite headwinds including, "multiple industry cycles, the evolution of the large U.S. hub-and-spoke airlines into more efficient and financially secure competitors, the emergence of ultra-low-cost airlines, the acquisition and integration of AirTran Holdings Inc. in 2011, and the company's entry into more congested but lucrative metropolitan airports such as New York's LaGuardia Airport."

In June, Moody's upgraded Southwest to A3, one notch higher than S\&P's new rating on its credit scale. Moody's Senior Credit Officer Jonathan Root wrote then, "the upgrade to A3 reflects Moody's expectation that Southwest will continue to conservatively manage its capital structure, allowing it to sustain credit metrics supportive of the A3 rating category".

Source: \href{https://www.barrons.com/articles/southwest-wins-another-credi}{https://www.barrons.com/articles/southwest-wins-another-credi} t-rating-upgrade-1502747699 (retrieved November 3, 2018)

In conducting financial analysis of a company, the analyst will regularly refer to the company's financial statements, financial notes, and supplementary schedules and a variety of other information sources. The next section introduces the primary financial statements and some commonly used information sources.

\section*{MAJOR FINANCIAL STATEMENTS - BALANCE SHEET }
\begin{abstract}
describe the roles of the statement of financial position, statement of comprehensive income, statement of changes in equity, and statement of cash flows in evaluating a company's performance and financial position
\end{abstract}

In order to perform an equity or credit analysis of a company, an analyst collects a great deal of information. The nature of the information collected will vary on the basis of the individual decision to be made (or the specific purpose of the analysis) but will typically include information about the economy, industry, and company, as well as information about comparable peer companies. Information from outside the company will likely include economic statistics, industry reports, trade publications, and databases containing information on competitors. The company itself provides core information for analysis in its financial reports, press releases, investor conference calls, and webcasts.

Companies prepare financial reports at regular intervals (annually, semiannually, and/or quarterly depending on the applicable regulatory requirements). Financial reports include financial statements along with supplemental disclosures necessary to assess the company's financial position and periodic performance.

Financial statements are the result of an accounting process that records a company's economic activities, following the applicable accounting standards and principles. These statements summarize the accounting information, mainly for users outside the company (such as investors, creditors, analysts, and others) because insiders have direct access to the underlying financial data summarized in the financial statements and to other information that is not included in the financial reporting process.

Financial statements are almost always audited by independent accountants, who provide an opinion on whether the financial statements present fairly the company's performance and financial position, in accordance with a specified, applicable set of accounting standards and principles.

\section{Financial Statements and Supplementary Information}
A complete set of financial statements include a statement of financial position (i.e., a balance sheet), a statement of comprehensive income (i.e., a single statement of comprehensive income or an income statement and a statement of comprehensive income), a statement of changes in equity, and a statement of cash flows. ${ }^{1}$ The balance sheet portrays the company's financial position at a given point in time. The statement of comprehensive income and statement of cash flows present different aspects of a company's performance over a period of time. The statement of changes in equity provides additional information regarding the changes in a company's financial position. In addition, the accompanying required notes, or footnotes, are considered an integral part of a complete set of financial statements.

Along with the required financial statements, a company typically provides additional information in its financial reports. In many jurisdictions, some or all of this additional information is mandated by regulators or accounting standards boards. The additional information provided may include a letter from the chairman of the company, a report from management discussing the results (typically called management discussion and analysis [MD\&A] or management commentary), an external auditor's report providing assurances, a governance report describing the structure of the company's board of directors, and a corporate responsibility report. As part of his or her analysis, the financial analyst should read and assess this additional information along with the financial statements. The following sections describe and illustrate each financial statement and some of the additional information.

\section{Balance Sheet (Statement of Financial Position)}
The balance sheet (also called the statement of financial position or statement of financial condition) presents a company's financial position by disclosing the resources the company controls (assets) and its obligations to lenders and other creditors (liabilities) at a specific point in time. Owners' equity (sometimes called "net assets") represents the excess of assets over liabilities. This amount is attributable to the company's owners or shareholders. Owners' equity is the owners' residual interest in (i.e., residual claim on) the company's assets after deducting its liabilities.

The relationship among the three parts of the balance sheet (assets, liabilities, and owners' equity) can be expressed in the following equation form: Assets = Liabilities + Owners' equity. This equation (sometimes called the accounting equation or the balance sheet equation) shows that the total amount of assets must equal or balance with the combined total amounts of liabilities and owners' equity. Alternatively, the equation may be rearranged as follows: Assets - Liabilities = Owners' equity. This formulation emphasizes the residual claim aspect of owners' equity. Depending on the form of the organization, owners' equity may be referred to as "partners' capital" or "shareholders' equity" or "shareholders' funds."

Exhibit 3 presents the balance sheet of the Volkswagen Group from its Annual Report 2017.

\section{Exhibit 3: Balance Sheet of the Volkswagen Group (Excerpt)}
Assets

Noncurrent assets

1 The names of the financial statements are those in IAS 1 . Commonly used terms for these financial statements are indicated in parentheses. Later readings will elaborate on each of these financial statements.

\begin{center}
\begin{tabular}{|c|c|c|}
\hline
$€$ million & 31 Dec. 2017 & 31 Dec. 2016 \\
\hline
Intangible assets & 63,419 & 62,599 \\
\hline
Property, plant and equipment & 55,243 & 54,033 \\
\hline
Lease assets & 39,254 & 38,439 \\
\hline
Investment property & 468 & 512 \\
\hline
Equity-accounted investments & 8,205 & 8,616 \\
\hline
Other equity investments & 1,318 & 996 \\
\hline
Financial services receivables & 73,249 & 68,402 \\
\hline
Other financial assets & 8,455 & 8,256 \\
\hline
Other receivables & 2,252 & 2,009 \\
\hline
Tax receivables & 407 & 392 \\
\hline
\multirow[t]{2}{*}{Deferred tax assets} & 9,810 & 9,756 \\
\hline
 & 262,081 & 254,010 \\
\hline
\multicolumn{3}{|l|}{Current assets} \\
\hline
Inventories & 40,415 & 38,978 \\
\hline
Trade receivables & 13,357 & 12,187 \\
\hline
Financial services receivables & 53,145 & 49,673 \\
\hline
Other financial assets & 11,998 & 11,844 \\
\hline
Other receivables & 5,346 & 5,130 \\
\hline
Tax receivables & 1,339 & 1,126 \\
\hline
Marketable securities & 15,939 & 17,520 \\
\hline
Cash, cash equivalents and time deposits & 18,457 & 19,265 \\
\hline
\multirow[t]{2}{*}{Assets held for sale} & 115 & - \\
\hline
 & 160,112 & 155,722 \\
\hline
Total assets & 422,193 & 409,732 \\
\hline
\multicolumn{3}{|l|}{Equity and Liabilities} \\
\hline
\multicolumn{3}{|l|}{Equity} \\
\hline
Subscribed capital & 1,283 & 1,283 \\
\hline
Capital reserves & 14,551 & 14,551 \\
\hline
Retained earnings & 81,367 & 70,446 \\
\hline
Other reserves & 560 & $(1,158)$ \\
\hline
$\begin{array}{l}\text { Equity attributable to Volkswagen AG hybrid } \\ \text { capital investors }\end{array}$ & 11,088 & 7,567 \\
\hline
$\begin{array}{l}\text { Equity attributable to Volkswagen AG share- } \\ \text { holders and hybrid capital investors }\end{array}$ & 108,849 & 92,689 \\
\hline
\multirow[t]{2}{*}{Noncontrolling interests} & 229 & 221 \\
\hline
 & 109,077 & 92,910 \\
\hline
\multicolumn{3}{|c|}{Noncurrent liabilities} \\
\hline
Financial liabilities & 81,628 & 66,358 \\
\hline
Other financial liabilities & 2,665 & 4,488 \\
\hline
Other liabilities & 6,199 & 5,664 \\
\hline
Deferred tax liabilities & 5,636 & 4,745 \\
\hline
Provisions for pensions & 32,730 & 33,012 \\
\hline
Provisions for taxes & 3,030 & 3,556 \\
\hline
Other provisions & 20,839 & 21,482 \\
\hline
\end{tabular}
\end{center}

\begin{center}
\begin{tabular}{|c|c|c|}
\hline
$€$ million & 31 Dec. 2017 & 31 Dec. 2016 \\
\hline
 & 152,726 & 139,306 \\
\hline
\multicolumn{3}{|l|}{Current liabilities} \\
\hline
$\begin{array}{l}\text { Put options and compensation rights granted } \\ \text { to noncontrolling interest shareholders }\end{array}$ & 3,795 & 3,849 \\
\hline
Financial liabilities & 81,844 & 88,461 \\
\hline
Trade payables & 23,046 & 22,794 \\
\hline
Tax payables & 430 & 500 \\
\hline
Other financial liabilities & 8,570 & 9,438 \\
\hline
Other liabilities & 15,961 & 15,461 \\
\hline
Provisions for taxes & 1,397 & 1,301 \\
\hline
\multirow[t]{2}{*}{Other provisions} & 25,347 & 35,711 \\
\hline
 & 160,389 & 177,515 \\
\hline
Total equity and liabilities & 422,193 & 409.732 \\
\hline
\end{tabular}
\end{center}

Note: Numbers are as shown in the annual report and may not precisely add because of rounding. Source: Volkswagen 2017 annual report

In Exhibit 3, the balance sheet is presented with the most recent year in the left column and the earlier year in the right column. Although this is a common presentation, analysts should be careful when reading financial statements. In some cases, the ordering may be reversed, with the earlier year(s) on the left and the most recent year on the far right.

At 31 December 2017, Volkswagen's total resources or assets were $€ 422$ billion. This number is the sum of non-current assets of $€ 262$ billion and current assets of $€ 160$ billion. ${ }^{2}$ Total equity was $€ 109$ billion. Although Volkswagen does not give a total amount for all the balance sheet liabilities, it can be determined by adding the non-current and current liabilities, $€ 153$ billion $+€ 160$ billion $=€ 313$ billion. ${ }^{3}$

Referring back to the basic accounting equation, Assets = Liabilities + Equity, we have $€ 422$ billion $=€ 313$ billion $+€ 109$ billion. In other words, Volkswagen has assets of $€ 422$ billion, owes $€ 313$ billion, and thus has equity of $€ 109$ billion. Using the balance sheet and applying financial statement analysis, the analyst can answer such questions as

\begin{itemize}
  \item Has the company's liquidity (ability to meet short-term obligations) improved?

  \item Is the company solvent (does it have sufficient resources to cover its obligations)?

  \item What is the company's financial position relative to the industry?

\end{itemize}

Volkswagen, a German-based automobile manufacturer, prepares its financial statements in accordance with International Financial Reporting Standards (IFRS). IFRS require companies to present balance sheets that show current and non-current assets and current and non-current liabilities as separate classifications. However, IFRS do not prescribe a particular ordering or format, and the order in which companies present their balance sheet items is largely a function of tradition.

2 Current assets are defined, in general, as those assets that are cash or cash equivalents; are held for trading; or are expected to be converted to cash (realized), sold, or consumed within 12 months or the company's normal operating cycle. All other assets are classified as non-current.

3 Current liabilities are defined, in general, as those that are expected to be settled within 12 months or the company's normal operating cycle. All other liabilities are classified as non-current. As shown, Volkswagen presents non-current assets before current assets, owners' equity before liabilities, and non-current liabilities before current liabilities. This method generally reflects a presentation from least liquid to most liquid. In other countries, the typical order of presentation may differ. For example, in the United States, Australia, and Canada, companies usually present their assets and liabilities from most liquid to least liquid. Cash is typically the first asset shown, and equity is presented after liabilities.

As a basis for comparison, Exhibit 4 presents the balance sheet of Walmart, Inc. (Walmart) from its 2018 annual report, with a fiscal year end of 31 January.

Exhibit 4: Walmart Consolidated Balance Sheet

As of 31 January

(Amounts in \$ millions)

ASSETS

Current assets:

Cash and cash equivalents

\begin{center}
\begin{tabular}{ccc}
6,756 & 6,867 &  \\
5,614 & 5,835 &  \\
43,783 & 43,046 &  \\
3,511 &  & 1,941 \\
\hline
59,664 &  & 57,689 \\
\hline
\end{tabular}
\end{center}

Total current assets

Property and equipment:

Property and equipment

Less accumulated depreciation

Property and equipment, net

\begin{center}
\begin{tabular}{|c|c|}
\hline
185,154 & 179,492 \\
\hline
$(77,479)$ & $(71,782)$ \\
\hline
107,67 & 107,710 \\
\hline
\end{tabular}
\end{center}

Property under capital lease and financing obligations:

Property under capital lease and financing obligations

Less accumulated amortization

Property under capital lease and financing obligations, net

\begin{center}
\begin{tabular}{cc}
12,703 & 11,637 \\
$(5,560)$ & $(5,169)$ \\
\hline
7,143 & 6,468 \\
\hline
\end{tabular}
\end{center}

Goodwill

Other assets and deferred charges

Total assets

\begin{center}
\begin{tabular}{ccc}
18,242 &  & 17,037 \\
11,798 &  &  \\
\hline
204,522 &  & 9,921 \\
198,825 &  &  \\
\end{tabular}
\end{center}

LIABILITIES AND EQUITY

Current liabilities:

Short-term borrowings

\begin{center}
\begin{tabular}{|c|c|}
\hline
5,257 & 1,099 \\
\hline
46,092 & 41,433 \\
\hline
22,122 & 20,654 \\
\hline
645 & 921 \\
\hline
3,738 & 2,256 \\
\hline
667 & 565 \\
\hline
78,521 & 66,928 \\
\hline
30,045 & 36,015 \\
\hline
6,780 & 6,003 \\
\hline
\end{tabular}
\end{center}

Long-term capital lease and financing obligations

\begin{center}
\begin{tabular}{llcc}
\hline
 & \multicolumn{2}{c}{As of $\mathbf{3 1}$ January} \\
\hline
(Amounts in \$ millions) & $\mathbf{2 0 1 8}$ & $\mathbf{2 0 1 7}$ \\
\hline
Deferred income taxes and other & 8,354 & 9,344 \\
\hline
\end{tabular}
\end{center}

Commitments and contingencies

Equity:

$\begin{array}{llll}\text { Common stock } & 295 & 305\end{array}$

$\begin{array}{llr}\text { Capital in excess of par value } & 2,648 & 2,371\end{array}$

$\begin{array}{llll}\text { Retained earnings } & 85,107 & 89,354\end{array}$

Accumulated other comprehensive loss

Total Walmart shareholders' equity

$\frac{(10,181)}{77,869} \quad \frac{(14,232)}{77,798}$

$\begin{array}{llll}\text { Noncontrolling interest } & 2,953 & 2,737\end{array}$

Total equity

Total liabilities and equity

$\frac{80,822}{204,522} \quad \frac{80,535}{198,825}$

Source: Walmart 2018 annual report.

As of 31 January 2018, Walmart has total assets of $\$ 205$ billion. Liabilities and other non-equity claims total $\$ 124$ billion, and equity is $\$ 81$ billion.

\section{STATEMENT OF COMPREHENSIVE INCOME}
describe the roles of the statement of financial position, statement of comprehensive income, statement of changes in equity, and statement of cash flows in evaluating a company's performance and financial position

The statement of comprehensive income can be presented as a single statement of comprehensive income or as two statements, an income statement and a statement of comprehensive income that begins with profit or loss from the income statement. The Volkswagen Group chose the latter form of presentation.

\section{Income Statement}
The income statement presents information on the financial performance of a company's business activities over a period of time. It communicates how much revenue and other income the company generated during a period and the expenses it incurred to generate that revenue and other income. Revenue typically refers to amounts charged for the delivery of goods or services in the ordinary activities of a business. Other income may include gains that may or may not arise in the ordinary activities of the business, such as profit on a business disposal. Expenses reflect outflows, depletions of assets, and incurrences of liabilities that decrease equity. Expenses typically include such items as cost of sales (cost of goods sold), administrative expenses, and income tax expenses and may be defined to include losses. Net income (revenue plus other income minus expenses) on the income statement is often referred to as the "bottom line" because of its proximity to the bottom of the income statement. Net income may also be referred to as "net earnings," "net profit," and "profit or loss." In the event that expenses exceed revenues and other income, the result is referred to as "net loss."

Income statements are reported on a consolidated basis, meaning that they include the income and expenses of subsidiary companies under the control of the parent (reporting) company. The income statement is sometimes referred to as a statement of operations or profit and loss (P\&L) statement. The basic equation underlying the income statement is Revenue + Other income - Expenses = Income - Expenses = Net income.

In general terms, when one company (the parent) controls another company (the subsidiary), the parent presents its own financial statement information consolidated with that of the subsidiary. (When a parent company owns more than 50 percent of the voting shares of a subsidiary company, it is presumed to control the subsidiary and thus presents consolidated financial statements.) Each line item of the consolidated income statement includes the entire amount from the relevant line item on the subsidiary's income statement (after removing any intercompany transactions). However, if the parent does not own 100 percent of the subsidiary, it is necessary for the parent to present an allocation of net income to the minority interests. Minority interests, also called non-controlling interests, refer to owners of the remaining shares of the subsidiary that are not owned by the parent. The share of consolidated net income attributable to minority interests is shown at the bottom of the income statement along with the net income attributable to shareholders of the parent company. Exhibit 5 presents the income statements of the Volkswagen Group from its 2017 annual report.

\section{Exhibit 5: Income Statement of the Volkswagen Group (Excerpt)}
\begin{center}
\begin{tabular}{|c|c|c|}
\hline
$€$ million & 2017 & 2016 \\
\hline
Sales revenue & 230,682 & 217,267 \\
\hline
Cost of sales & $(188,140)$ & $(176,270)$ \\
\hline
Gross result & 42,542 & 40,997 \\
\hline
Distribution expenses & $(22,710)$ & $(22,700)$ \\
\hline
Administrative expenses & $(8,254)$ & $(7,336)$ \\
\hline
Other operating income & 14,500 & 13,049 \\
\hline
Other operating expenses & $(12,259)$ & $(16,907)$ \\
\hline
Operating result & 13,818 & 7,103 \\
\hline
Share of the result of equity-accounted investments & 3,482 & 3,497 \\
\hline
Interest income & 951 & 1,285 \\
\hline
Interest expenses & $(2,317)$ & $(2,955)$ \\
\hline
Other financial result & $(2,022)$ & $(1,638)$ \\
\hline
Financial result & 94 & 189 \\
\hline
Earnings before tax & 13,913 & 7,292 \\
\hline
Income tax income/expense & $(2,275)$ & $(1,912)$ \\
\hline
Current & $(3,205)$ & $(3,273)$ \\
\hline
Deferred & 930 & 1,361 \\
\hline
Earnings after tax & 11,638 & 5,379 \\
\hline
\multicolumn{3}{|l|}{of which attributable to} \\
\hline
Noncontrolling interests & 10 & 10 \\
\hline
Volkswagen AG hybrid capital investors & 274 & 225 \\
\hline
\end{tabular}
\end{center}

\begin{center}
\begin{tabular}{llc}
\hline
$€$ million & 2017 & 2016 \\
\hline
Volkswagen AG shareholders & 11,354 & 5,144 \\
 &  & 10.24 \\
Basic earnings per ordinary share in $€$ & 22.63 & 10.24 \\
Diluted earnings per ordinary share in $€$ & 22.63 & 10.30 \\
Basic earnings per preferred share in $€$ & 22.69 & 10.30 \\
Diluted earnings per preferred share in $€$ & 22.69 &  \\
\hline
\end{tabular}
\end{center}

Note: The numbers are as shown in the annual report and may not add because of rounding.

Source: 2017 Volkswagen annual report.

Exhibit 5 shows that Volkswagen's sales revenue for the year ended 31 December 2017 was $€ 231$ billion. Subtracting cost of sales from revenue gives gross profit (called "gross results" by Volkswagen) of $€ 43$ billion. After subtracting operating costs and expenses and adding other operating income, the company's operating profit (called "operating result" by Volkswaen) totals $€ 14$ billion. Operating profit represents the results of the company's usual business activities before deducting interest expense or taxes. Operating profit (also called operating income in addition to operating result) is thus often referred to as earnings before interest and taxes (EBIT).

Next, operating profit is increased by Volkswagen's share of the profits generated by certain of its investments ( $€ 3.5$ billion) plus interest income of $€ 1.0$ billion, and decreased by losses from its other financial activities ( $€ 2.0$ billion) and by interest expense of $€ 2.3$ billion, resulting in profit (earnings) before tax of $€ 13.9$ billion. Total income tax expense for 2017 was $€ 2.3$ billion, resulting in profit after tax (net income) of $€ 11.6$ billion. After allocating the profits attributable to minority interest ownership in Volkswagen subsidiary companies, the profit attributable to shareholders of Volkswagen for 2017 was $€ 11.4$ billion.

Companies present both basic and diluted earnings per share on the face of the income statement. Earnings per share numbers represent net income attributable to the class of shareholders divided by the relevant number of shares outstanding during the period. Basic earnings per share is calculated using the weighted-average number of common (ordinary) shares that were actually outstanding during the period and the profit or loss attributable to the common shareowners. Diluted earnings per share uses diluted shares - the number of shares that would hypothetically be outstanding if potentially dilutive claims on common shares (e.g., stock options or convertible bonds) were exercised or converted by their holders-and an appropriately adjusted profit or loss attributable to the common shareowners.

Volkswagen has two types of shareholders, ordinary and preferred, and presents earnings per share information for both, although there is no requirement to present earnings per share information for preferred shareowners. Volkswagen's basic earnings per ordinary share was $€ 22.63$. A note to the company's financial statements explains that this number was calculated as follows: $€ 11.4$ billion profit attributable to shareholders of Volkswagen, of which $€ 6.8$ billion is attributable to ordinary shareholders and the balance is attributable to preferred shareholders. The $€ 6.8$ billion attributable to ordinary shareholders divided by the weighted-average number of ordinary shares of 0.295 billion shares equals basic earnings per share of $€ 22.63$. Similar detail is provided in the notes for each of the earnings per share numbers. An analyst examining the income statement might note that Volkswagen was profitable in both years. The company's profits increased substantially in 2017, primarily because of higher sales and lower other operating expenses (footnotes reveal this was largely due to $\$ 4$ billion less in litigation expenses in 2017). The analyst might formulate questions related to profitability, such as the following:

\begin{itemize}
  \item Is the change in revenue related to an increase in units sold, an increase in prices, or some combination?

  \item If the company has multiple business segments (for example, Volkswagen's segments include passenger cars, light commercial vehicles, and financial services, among others), how are the segments' revenue and profits changing?

  \item How does the company compare with other companies in the industry?

\end{itemize}

Answering such questions requires the analyst to gather, analyze, and interpret information from a number of sources, including, but not limited to, the income statement.

\section{Other Comprehensive Income}
Comprehensive income includes all items that impact owners' equity but are not the result of transactions with shareowners. Some of these items are included in the calculation of net income, and some are included in other comprehensive income (OCI). When comprehensive income is presented in two statements, the statement of comprehensive income begins with the profit or loss from the income statement and then presents the components of OCI.

Exhibit 6 presents an excerpt from the statement of comprehensive income of the Volkswagen Group from its 2017 annual report.

Exhibit 6: Statement of Comprehensive Income of the Volkswagen Group (Excerpt)

Fiscal Year Ended

31 December

$\epsilon$ million

2017

Earnings after tax

11,638

Pension plan remeasurements recognized in other comprehensive income

Pension plan remeasurements recognized in other comprehensive income, before tax 785

Deferred taxes relating to pension plan remeasurements recognized in other comprehensive income

Pension plan remeasurements recognized in other comprehensive income, net of tax 588

Share of other comprehensive income of equity-accounted investments that will not be reclassified to profit or loss, net of tax

Items that will not be reclassified to profit or loss

Exchange differences on translating foreign operations

Unrealized currency translation gains/losses

Transferred to profit or loss

Exchange differences on translating foreign operations, before tax

Deferred taxes relating to exchange differences on translating foreign operations

Exchange differences on translating foreign operations, net of tax

Cash flow hedges

Fair value changes recognized in other comprehensive income $€$ million

Fiscal Year Ended

31 December

Transferred to profit or loss

2017

Cash flow hedges, before tax

(558)

Deferred taxes relating to cash flow hedges

5,579

Cash flow hedges, net of $\operatorname{tax}$

$(1,597)$

Available-for-sale financial assets

3,982

Fair value changes recognized in other comprehensive income 56

Transferred to profit or loss 62

Available-for-sale financial assets, before tax 118

Deferred taxes relating to available-for-sale financial assets

Available-for-sale financial assets, net of tax 93

Share of other comprehensive income of equity-accounted investments that may be reclassified subsequently to profit or loss, net of tax

Items that may be reclassified subsequently to profit or loss

1,622

Other comprehensive income, before tax

4,133

Deferred taxes relating to other comprehensive income

Other comprehensive income, net of tax

Total comprehensive income

$(1,828)$

Source: Volkswagen 2017 annual report.

Exhibit 6 shows total comprehensive income for 2017 was $€ 13.9$ billion, which is the sum of earnings after tax of $€ 11.6$ billion, reported on the income statement, and other comprehensive income of $€ 2.3$ billion. The items in OCI reflect changes in the company's equity that are not considered to be profit or loss, some of which may be reclassified as such in the future such as unrealized currency translation gains and losses. The statement of comprehensive income will be discussed in greater detail in a later reading.

\section{STATEMENT OF CHANGES IN EQUITY AND CASH FLOW STATEMENT}
describe the roles of the statement of financial position, statement of comprehensive income, statement of changes in equity, and statement of cash flows in evaluating a company's performance and financial position

The statement of changes in equity, sometimes called the "statement of changes in owners' equity" or "statement of changes in shareholders' equity," primarily serves to report changes in the owners' investment in the business over time. The basic components of owners' equity are paid-in capital and retained earnings. Retained earnings include the cumulative amount of the company's profits that have been retained in the company. In addition, non-controlling or minority interests and reserves that represent accumulated OCI items are included in equity. The latter items may be shown separately or included in retained earnings. Volkswagen includes reserves as components of retained earnings.

The statement of changes in equity is organized to present, for each component of equity, the beginning balance, any increases during the period, any decreases during the period, and the ending balance. For paid-in capital, an example of an increase is a new issuance of equity and an example of a decrease is a repurchase of previously issued stock. For retained earnings, income (both net income as reported on the income statement and $\mathrm{OCI})$ is the most common increase and a dividend payment is the most common decrease.

Volkswagen's balance sheet in Exhibit 3 shows that equity at the end of 2017 totaled $€ 109$ billion, compared with $€ 93$ billion at the end of 2016 . The company's statement of changes in equity presents additional detail on the change in each line item. Exhibit 7 presents an excerpt of the statement of changes in equity of the Volkswagen Group from its 2017 annual report. For purposes of brevity, several interim columns were excluded from the presentation.

\section{Exhibit 7: Statement of Changes in Equity of the Volkswagen Group (Excerpt)}
\begin{center}
\begin{tabular}{|c|c|c|c|c|}
\hline
$€$ million & $\begin{array}{c}\text { Subscribed } \\ \text { capital }\end{array}$ & Capital reserves & $\begin{array}{l}\text { Retained } \\ \text { earnings }\end{array}$ & Total equity \\
\hline
Balance at 1 January 2017 & 1,283 & 14,551 & 70,446 & 92,910 \\
\hline
Earnings after tax & - & - & 11,354 & 11,638 \\
\hline
Other comprehensive income, net of tax & - & - & 586 & 2,305 \\
\hline
Total comprehensive income & - & - & 11,940 & 13,943 \\
\hline
Capital increases & - & - & - & 3,481 \\
\hline
Dividends payment & - & - & $-1,015$ & $-1,332$ \\
\hline
$\begin{array}{l}\text { Capital transactions involving a change in } \\ \text { ownership interest }\end{array}$ & - & - & - & - \\
\hline
Other changes & - & - & -4 & 75 \\
\hline
Balance at 31 December 2017 & 1,283 & 14,551 & 81,367 & 109,077 \\
\hline
\end{tabular}
\end{center}

Note: Numbers are as shown in the annual report and may not add and cross-add because of the exclusion of columns and due to rounding.

In Exhibit 7, as shown in the far right column, total equity is increased during the year by total comprehensive income of $€ 13.9$ billion and by capital infusions of $€ 3.5$ billion; it is decreased by dividends of $€ 1.3$ billion; and finally, slightly increased by $€ 75$ million of other changes. Explanatory notes on equity are included in the notes to the consolidated financial statements.

\section{Cash Flow Statement}
Although the income statement and balance sheet provide measures of a company's success in terms of performance and financial position, cash flow is also vital to a company's long-term success. Disclosing the sources and uses of cash helps creditors, investors, and other statement users evaluate the company's liquidity, solvency, and financial flexibility. Financial flexibility is the ability of the company to react and adapt to financial adversity and opportunities. The cash flow statement classifies all cash flows of the company into three categories: operating, investing, and financing. Cash flows from operating activities generally involve the cash effects of transactions involved in the determination of net income and, hence, comprise the day-to-day operations of the company. Cash flows from investing activities are associated with the acquisition and disposal of long-term assets, such as property and equipment. Cash flows from financing activities relate to obtaining or repaying capital to be used in the business. IFRS permit more flexibility than US GAAP in classifying dividend and interest receipts and payments within these categories.

Exhibit 8 presents Volkswagen's statement of cash flows for the fiscal years ended 31 December 2017 and 2016.

\section{Exhibit 8: Cash Flow Statement of the Volkswagen Group: 1 January to 31 December}
\begin{center}
\begin{tabular}{|c|c|c|}
\hline
$\epsilon$ million & 2017 & 2016 \\
\hline
Cash and cash equivalents at beginning of period & 18,833 & 20,462 \\
\hline
Earnings before tax & 13,913 & 7,292 \\
\hline
Income taxes paid & $(3,664)$ & $(3,315)$ \\
\hline
$\begin{array}{l}\text { Depreciation and amortization of, and impairment losses on, intangible assets, property, } \\ \text { plant and equipment, and investment property }\end{array}$ & 10,562 & 10,100 \\
\hline
Amortization of and impairment losses on capitalized development costs & 3,734 & 3,586 \\
\hline
Impairment losses on equity investments & 136 & 130 \\
\hline
Depreciation of and impairment losses on lease assets & 7,734 & 7,107 \\
\hline
Gain/loss on disposal of noncurrent assets and equity investments & $(25)$ & $(222)$ \\
\hline
Share of the result of equity-accounted investments & 274 & 377 \\
\hline
Other noncash expense/income & $(480)$ & 716 \\
\hline
Change in inventories & $(4,198)$ & $(3,637)$ \\
\hline
Change in receivables (excluding financial services) & $(1,660)$ & $(2,155)$ \\
\hline
Change in liabilities (excluding financial liabilities) & 5,302 & 5,048 \\
\hline
Change in provisions & $(9,443)$ & 5,966 \\
\hline
Change in lease assets & $(11,478)$ & $(12,074)$ \\
\hline
Change in financial services receivables & $(11,891)$ & $(9,490)$ \\
\hline
Cash flows from operating activities & $(1,185)$ & 9,430 \\
\hline
$\begin{array}{l}\text { Investments in intangible assets (excluding development costs), property, plant and } \\ \text { equipment, and investment property }\end{array}$ & $(13,052)$ & $(13,152)$ \\
\hline
Additions to capitalized development costs & $(5,260)$ & $(5,750)$ \\
\hline
Acquisition of subsidiaries & $(277)$ & $(119)$ \\
\hline
Acquisition of other equity investments & $(561)$ & $(309)$ \\
\hline
Disposal of subsidiaries & 496 & $(7)$ \\
\hline
Disposal of other equity investments & 24 & 2,190 \\
\hline
$\begin{array}{l}\text { Proceeds from disposal of intangible assets, property, plant and equipment, and invest- } \\ \text { ment property }\end{array}$ & 411 & 351 \\
\hline
Change in investments in securities & 1,376 & $(1,245)$ \\
\hline
Change in loans and time deposits & 335 & $(2,638)$ \\
\hline
Cash flows from investing activities & $(16,508)$ & $(20,679)$ \\
\hline
Capital contributions & 3,473 & - \\
\hline
Dividends paid & $(1,332)$ & $(364)$ \\
\hline
\end{tabular}
\end{center}

\begin{center}
\begin{tabular}{|c|c|c|}
\hline
$€$ million & 2017 & 2016 \\
\hline
Capital transactions with noncontrolling interest shareholders & - & $(3)$ \\
\hline
Proceeds from issuance of bonds & 30,279 & 14,262 \\
\hline
Repayments of bonds & $(17,877)$ & $(23,601)$ \\
\hline
Changes in other financial liabilities & 3,109 & 19,455 \\
\hline
Lease payments & $(28)$ & $(36)$ \\
\hline
Cash flows from financing activities & 17,625 & 9,712 \\
\hline
Effect of exchange rate changes on cash and cash equivalents & $(727)$ & $(91)$ \\
\hline
Net change in cash and cash equivalents & $(796)$ & $(1,628)$ \\
\hline
Cash and cash equivalents at end of period & 18,038 & 18,833 \\
\hline
\end{tabular}
\end{center}

After showing the beginning cash balance, the operating activities section starts with profit before tax,${ }^{4} € 13.9$ billion for 2017, subtracts actual income tax payments, and then adjusts for the effects of non-cash transactions, accruals and deferrals, and transactions of an investing and financing nature to arrive at the amount of cash generated from operating activities: a negative $€ 1.2$ billion (a net use of cash). This approach to reporting cash flow from operating activities is termed the indirect method. The direct method of reporting cash flows from operating activities discloses major classes of gross cash receipts and gross cash payments. Examples are cash received from customers and cash paid to suppliers and employees.

The indirect method emphasizes the different perspectives of the income statement and cash flow statement. On the income statement, income is reported when earned, not necessarily when cash is received, and expenses are reported when incurred, not necessarily when paid. The cash flow statement presents another aspect of performance: the ability of a company to generate cash flow from running its business. Ideally, for an established company, the analyst would like to see that the primary source of cash flow is from operating activities as opposed to investing or financing activities.

The sum of the net cash flows from operating, investing, and financing activities and the effect of exchange rates on cash equals the net change in cash during the fiscal year. For Volkswagen, the sum of these four items was a negative $€ 796$ million in 2017, decreasing the company's cash and cash equivalents from $€ 18.8$ billion at the beginning of the period to $€ 18.0$ billion at the end of the period.

\section{FINANCIAL NOTES, SUPPLEMENTARY SCHEDULES, AND MANAGEMENT COMMENTARY}
describe the importance of financial statement notes and supplementary information-including disclosures of accounting policies, methods, and estimates-and management's commentary

The notes (also sometimes referred to as footnotes) that accompany the four financial statements are required and are an integral part of the complete set of financial statements. The notes provide information that is essential to understanding the information provided in the primary statements. Volkswagen's 2017 financial statements, for example, include more than 100 pages of notes.

\begin{enumerate}
  \setcounter{enumi}{3}
  \item Other companies may choose to begin with net income. The notes disclose the basis of preparation for the financial statements. For example, Volkswagen discloses that its fiscal year corresponds to the calendar year; its financial statements are prepared in accordance with IFRS as adopted by the European Union; the statements are prepared in compliance with German law; the statements are denominated in millions of euros unless otherwise specified; and the figures have been rounded, which might give rise to minor discrepancies when they are added. Volkswagen also states that its financial statements are on a consolidated basis-that is, including Volkswagen AG and all of the subsidiary companies it controls.
\end{enumerate}

The notes also disclose information about the accounting policies, methods, and estimates used to prepare the financial statements. As will be discussed in later readings, both IFRS and US GAAP allow some flexibility in choosing among alternative policies and methods when accounting for certain items. This flexibility aims to meet the divergent needs of many businesses for reporting a variety of economic transactions. In addition to differences in accounting policies and methods, differences arise as a result of estimates needed to record and measure transactions, events, and financial statement line items.

Overall, flexibility in accounting choices is necessary because, ideally, a company will select those policies, methods, and estimates that are allowable and most relevant and that fairly reflect the unique economic environment of the company's business and industry. Flexibility can, however, create challenges for the analyst because the use of different policies, methods, and estimates reduces comparability across different companies' financial statements. Comparability occurs when different companies' information is measured and reported in a similar manner over time. Comparability helps the analyst identify and analyze the real economic differences across companies, rather than differences that arise solely from different accounting choices. Because comparability of financial statements is a critical requirement for objective financial analysis, an analyst should be aware of the potential for differences in accounting choices even when comparing two companies that use the same set of accounting standards.

For example, if a company acquires a piece of equipment to use in its operations, accounting standards require that the cost of the equipment be reported as an expense by allocating its cost, less any residual value, in a systematic manner over the equipment's useful life. This allocation of the cost is known as depreciation. Accounting standards permit flexibility, however, in determining the manner in which each year's expense is determined. Two companies may acquire similar equipment but use different methods and assumptions to record the expense over time. An analyst's ability to compare the companies' performance is hindered by the difference. Analysts must understand reporting choices in order to make appropriate adjustments when comparing companies' financial positions and performance.

A company's significant accounting choices (policies, methods, and estimates) must be discussed in the notes to the financial statements. For example, a note containing a summary of significant accounting policies includes how the company recognizes its revenues and depreciates its non-current tangible assets. Analysts must understand the accounting choices a company makes and determine whether they are similar to those of other companies identified and used as benchmarks or comparables. If the policies of the companies being compared are different, the analyst who understands accounting and financial reporting can often make necessary adjustments so that the financial statement data used are more comparable.

For many companies, the financial notes and supplemental schedules provide explanatory information about every line item (or almost every line item) on the balance sheet and income statement, as illustrated by the note references in Volkswagen's balance sheet and income statement in Exhibits 3 and 5. In addition, note disclosures include information about the following (this is not an exhaustive list):

\begin{itemize}
  \item financial instruments and risks arising from financial instruments, - commitments and contingencies,

  \item legal proceedings,

  \item related-party transactions,

  \item subsequent events (i.e., events that occur after the balance sheet date),

  \item business acquisitions and disposals, and

  \item operating segments' performance.

\end{itemize}

Exhibit 9: Notes to the Consolidated Financial Statements of the Volkswagen Group: Selected Data on Operating Segments (Excerpt)

\begin{center}
\begin{tabular}{|c|c|c|c|c|c|c|c|}
\hline
$€$ million & $\begin{array}{c}\text { Passenger } \\ \text { Cars }\end{array}$ & $\begin{array}{c}\text { Commercial } \\ \text { Vehicles }\end{array}$ & $\begin{array}{c}\text { Power } \\ \text { Engineering }\end{array}$ & $\begin{array}{l}\text { Financial } \\ \text { Services }\end{array}$ & $\begin{array}{c}\text { Total } \\ \text { segments }\end{array}$ & Reconciliation & $\begin{array}{c}\text { Volkswager } \\ \text { Group }\end{array}$ \\
\hline
$\begin{array}{l}\text { Sales revenue from } \\ \text { external customers }\end{array}$ & 169,513 & 27,632 & 3,280 & 30,191 & 230,618 & 64 & 230,682 \\
\hline
$\begin{array}{l}\text { Intersegment sales } \\ \text { revenue }\end{array}$ & 18,892 & 7,568 & 3 & 3,541 & 30,004 & $-30,004$ & - \\
\hline
Total sales revenue & 188,405 & 35,200 & 3,283 & 33,733 & 260,621 & $-29,939$ & 230,682 \\
\hline
$\begin{array}{l}\text { Depreciation and } \\ \text { amortization }\end{array}$ & 11,363 & 2,557 & 371 & 6,797 & 21,089 & -147 & 20,941 \\
\hline
Impairment losses & 704 & 2 & 0 & 574 & 1,280 & 0 & 1,280 \\
\hline
$\begin{array}{l}\text { Reversal of impairment } \\ \text { losses }\end{array}$ & 14 & 1 & - & 41 & 56 & - & 56 \\
\hline
$\begin{array}{l}\text { Segment result (operat- } \\ \text { ing result) }\end{array}$ & 12,644 & 1,892 & -55 & 2,673 & 17,153 & $-3,335$ & 13,818 \\
\hline
$\begin{array}{l}\text { Share of the result } \\ \text { of equity-accounted } \\ \text { investments }\end{array}$ & 3,390 & 83 & 1 & 9 & 3,482 & - & 3,482 \\
\hline
$\begin{array}{l}\text { Net interest result and } \\ \text { other financial result }\end{array}$ & $-1,920$ & -220 & -2 & -180 & $-2,321$ & $-1,067$ & $-3,388$ \\
\hline
$\begin{array}{l}\text { Equity-accounted } \\ \text { investments }\end{array}$ & 6,724 & 753 & 18 & 710 & 8,205 & - & 8,205 \\
\hline
$\begin{array}{l}\text { Investments in intan- } \\ \text { gible assets, property, } \\ \text { plant and equipment, } \\ \text { and investment property }\end{array}$ & 15,713 & 1,915 & 159 & 421 & 18,208 & 104 & 18,313 \\
\hline
\end{tabular}
\end{center}

An analyst uses a significant amount of judgment in deciding how to incorporate information from note disclosures into the analysis. For example, such information as financial instrument risk, contingencies, and legal proceedings can alert an analyst to risks that can affect a company's financial position and performance in the future and that require monitoring over time. As another example, information about a company's operating segments can be useful as a means of quickly understanding what a company does and how and where it earns money. The operating segment data shown in Exhibit 9 appear in the notes to the financial statements for Volkswagen. (The totals of the segment data do not equal the amounts reported in the company's financial statements because the financial statement data are adjusted for intersegment activities and unallocated items. The notes provide a complete reconciliation of the segment data to the reported data.) From the data in Exhibit 9, an analyst can quickly see that most of the company's revenues and operating profits come from the sale of passenger cars. Over 80 percent of the company's revenues was generated by this segment This segment accounted for over 90 percent of the company's total segment operating profits.

Experience using the disclosures made by a company and its competitors typically enhances an analyst's judgment about the relative importance of different disclosures and the ways in which they can be helpful.

\section{Management Commentary or Management's Discussion and Analysis}
Publicly held companies typically include a section in their annual reports where management discusses a variety of issues, including the nature of the business, past results, and future outlook. This section is referred to by a variety of names, including management report(ing), management commentary, operating and financial review, and management's discussion and analysis. Inclusion of a management report is recommended by the International Organization of Securities Commissions and frequently required by regulatory authorities, such as the US Securities and Exchange Commission (SEC) or the UK Financial Reporting Council (FRC). In Germany, management reporting has been required since 1931 and is audited.

The discussion by management is arguably one of the most useful parts of a company's annual report besides the financial statements themselves; however, other than excerpts from the financial statements, information included in the management commentary is typically unaudited. When using information from the management report, an analyst should be aware of whether the information is audited or unaudited.

To help improve the quality of the discussion by management, the International Accounting Standards Board (IASB) issued an IFRS Practice Statement "Management Commentary" includes a framework for the preparation and presentation of management commentary. The framework provides guidance rather than sets forth requirements in a standard. The framework identifies five content elements of a "decision-useful management commentary:" 1) the nature of the business; 2) management's objectives and strategies; 3) the company's significant resources, risks, and relationships; 4) results of operations; and 5) critical performance measures.

In the United States, the SEC requires listed companies to provide an MD\&A and specifies the content. ${ }^{5}$ Management must highlight any favorable or unfavorable trends and identify significant events and uncertainties that affect the company's liquidity, capital resources, and results of operations. The MD\&A must also provide information about the effects of inflation, changing prices, or other material events and uncertainties that may cause the future operating results and financial condition to materially depart from the current reported financial information. In addition, the MD\&A must provide information about off-balance-sheet obligations and about contractual commitments such as purchase obligations. Management should also discuss the critical accounting policies that require them to make subjective judgments and that have a significant impact on reported financial results.

The management commentary, or MD\&A, is a good starting place for understanding information in the financial statements. In particular, the forward-looking disclosures, such as those about planned capital expenditures, new store openings, or divestitures, can be useful in projecting a company's future performance. However, the commentary is only one input for the analyst seeking an objective and independent perspective on a company's performance and prospects.

5 Relevant sections of SEC requirements are included for reference in the FASB ASC. The FASB ASC does not include sections of SEC requirements that deal with matters outside the basic financial statements, such as the MD\&A The management report in Volkswagen's 2017 annual report includes much information of potential interest to an analyst. The lengthy report contains sections such as Goals and Strategies, Internal Management and Key Performance Indicators, Structure and Business Activities, Corporate Governance, Remuneration, Executive Bodies, Disclosures Required under Takeover Law, Diesel Issue, Business Development, Shares and Bonds, Results of Operations, and Report on Risks and Opportunities.

\section{7}
\section{AUDITOR'S REPORTS}
describe the objective of audits of financial statements, the types of audit reports, and the importance of effective internal controls

Financial statements presented in companies' annual reports are generally required to be audited (examined) by an independent accounting firm in accordance with specified auditing standards. The independent auditor then provides a written opinion on the financial statements. This opinion is referred to as the audit report. Audit reports may vary in different jurisdictions, but the minimum components, including a specific statement of the auditor's opinion, are similar. Audits of financial statements may be required by contractual arrangement, law, or regulation.

International standards on auditing (ISAs) have been developed by the International Auditing and Assurance Standards Board (IAASB). This body has emerged from the International Federation of Accountants. ISAs have been adopted by many countries and are referenced in audit reports issued in those countries. Other countries, such as the United States, specify their own auditing standards. With the enactment of the Sarbanes-Oxley Act of 2002 in the United States, auditing standards for public companies are promulgated by the Public Company Accounting Oversight Board.

Under international standards for auditing (ISAs), the overall objectives of an auditor in conducting an audit of financial statements are

A. To obtain reasonable assurance about whether the financial statements as a whole are free from material misstatement, whether due to fraud or error, thereby enabling the auditor to express an opinion on whether the financial statements are prepared, in all material respects, in accordance with an applicable financial reporting framework; and

B. To report on the financial statements, and communicate as required by the ISAs, in accordance with the auditor's findings. ${ }^{6}$

Publicly traded companies may also have requirements set by regulators or stock exchanges, such as appointing an independent audit committee within its board of directors to oversee the audit process. The audit process provides a basis for the independent auditor to express an opinion on whether the information in the audited financial statements presents fairly the financial position, performance, and cash flows of the company in accordance with a specified set of accounting standards.

Audits are designed and conducted using sampling techniques, and financial statement line items may be based on estimates and assumptions. This means that the auditors cannot express an opinion that provides absolute assurance about the accuracy or precision of the financial statements. Instead, the independent audit report provides

6 See the International Auditing and Assurance Standards Board (IAASB) Handbook of International Quality Control, Auditing, Review, Other Assurance, and Related Services Pronouncements. reasonable assurance that the financial statements are fairly presented, meaning that there is a high probability that the audited financial statements are free from material error, fraud, or illegal acts that have a direct effect on the financial statements.

The independent audit report expresses the auditor's opinion on the fairness of the audited financial statements, and specifies which financial statements were audited, the reporting entity, and the date. An unqualified audit opinion states that the financial statements give a "true and fair view" (international) or are "fairly presented" (international and US) in accordance with applicable accounting standards. This is also referred to as an "unmodified" or a "clean" opinion and is the one that analysts would like to see in a financial report. There are several other types of modified opinions. A qualified audit opinion is one in which there is some scope limitation or exception to accounting standards. Exceptions are described in the audit report with additional explanatory paragraphs so that the analyst can determine the importance of the exception. An adverse audit opinion is issued when an auditor determines that the financial statements materially depart from accounting standards and are not fairly presented. Finally, a disclaimer of opinion occurs when, for some reason, such as a scope limitation, the auditors are unable to issue an opinion.

The audit report also describes the basis for the auditor's opinion and, for listed companies, includes a discussion of Key Audit Matters (international) and Critical Audit Matters (US). ${ }^{7}$ Key Audit Matters are defined as issues that the auditor considers to be most important, such as those that have a higher risk of misstatement, involve significant management judgment, or report the effects of significant transactions during the period. Critical Audit Matters are defined as issues that involve "especially challenging, subjective, or complex auditor judgment" and similarly include areas with higher risk of misstatement or involving significant management judgement and estimates.

Exhibit 10 presents the independent auditor's report for Volkswagen. Note that Volkswagen received an unqualified audit opinion (i.e., clean or unmodified opinion) from PricewaterhouseCoopers for the company's fiscal year ended 31 December 2017.

\section{Exhibit 10: Volkswagen's Independent Audit Report}
On completion of our audit, we issued the following unqualified auditor's report dated February 23, 2018. This report was originally prepared in German. In case of ambiguities the German version takes precedence:

To VOLKSWAGEN AKTIENGESELLSCHAFT, Wolfsburg

REPORT ON THE AUDIT OF THE CONSOLIDATED FINANCIAL STATEMENTS AND OF THE GROUP MANAGEMENT REPORT

\section{AUDIT OPINIONS}
We have audited the consolidated financial statements of VOLKSWAGEN AKTIENGESELLSCHAFT, Wolfsburg, and its subsidiaries (the Group), which comprise the income statement and the statement of comprehensive income, the balance sheet, the statement of changes in equity and the cash flow statement for the financial year from January 1 to December 31, 2017, and notes to the consolidated financial statements, including a summary of significant accounting policies. In addition, we have audited the group management report of VOLKSWAGEN

7 Discussion of Key Audit Matters in the auditor's report is required by the International Standard on Auditing (ISA) ISA 701, effective in 2017, issued by the International Audit and Assurance Standards Board. Discussion of Critical Audit Matters in the auditor's report is required by the Auditor Reporting Standard AS 3101, effective for large filers' fiscal years ending on or after June 30,2019, issued by the Public Company Accounting Oversight Board. AKTIENGESELLSCHAFT, which is combined with the Company's management report, for the financial year from January 1 to December 31, 2017. We have not audited the content of those parts of the group management report listed in the "Other Information" section of our auditor's report in accordance with the German legal requirements.

In our opinion, on the basis of the knowledge obtained in the audit,

\begin{itemize}
  \item the accompanying consolidated financial statements comply, in all material respects, with the IFRSs as adopted by the EU, and the additional requirements of German commercial law pursuant to $₫$ [Article] 315e Abs. [paragraph] 1 HGB [Handelsgesetzbuch: German Commercial Code] and, in compliance with these requirements, give a true and fair view of the assets, liabilities, and financial position of the Group as at December 31, 2017, and of its financial performance for the financial year from January 1 to December 31, 2017, and

  \item the accompanying group management report as a whole provides an appropriate view of the Group's position. In all material respects, this group management report is consistent with the consolidated financial statements, complies with German legal requirements and appropriately presents the opportunities and risks of future development. Our audit opinion on the group management report does not cover the content of those parts of the group management report listed in the "Other Information" section of our auditor's report.

\end{itemize}

Pursuant to $\$ 322$ Abs. 3 Satz [sentence] 1 HGB, we declare that our audit has not led to any reservations relating to the legal compliance of the consolidated financial statements and of the group management report.

\section{BASIS FOR THE AUDIT OPINIONS}
We conducted our audit of the consolidated financial statements and of the group management report in accordance with $₫ 317$ HGB and the EU Audit Regulation (No. 537/2014, referred to subsequently as "EU Audit Regulation") and in compliance with German Generally Accepted Standards for Financial Statement Audits promulgated by the Institut der Wirtschaftsprüfer [Institute of Public Auditors in Germany] (IDW). Our responsibilities under those requirements and principles are further described in the "Auditor's Responsibilities for the Audit of the Consolidated Financial Statements and of the Group Management Report" section of our auditor's report. We are independent of the group entities in accordance with the requirements of European law and German commercial and professional law, and we have fulfilled our other German professional responsibilities in accordance with these requirements. In addition, in accordance with Article 10 (2) point (f) of the EU Audit Regulation, we declare that we have not provided non-audit services prohibited under Article 5 (1) of the EU Audit Regulation. We believe that the audit evidence we have obtained is sufficient and appropriate to provide a basis for our audit opinions on the consolidated financial statements and on the group management report.

\section{EMPHASIS OF MATTER - DIESEL ISSUE}
We draw attention to the information provided and statements made in section "Key Events" of the notes to the consolidated financial statements and in section "Diesel Issue" of the group management report with regard to the diesel issue including information about the underlying causes, the noninvolvement of members of the board of management as well as the impact on these financial statements. Based on the results of the various measures taken to investigate the issue presented so far, which underlie the consolidated financial statements and the group management report, there is still no evidence that members of the Company's board of management were aware of the deliberate manipulation of engine management software before summer 2015. Nevertheless, should as a result of the ongoing investigation new solid knowledge be obtained showing that members of the board of management were informed earlier about the diesel issue, this could eventually have an impact on the consolidated financial statements and on the group management report for financial year 2017 and prior years. The provisions for warranties and legal risks recorded so far are based on the presented state of knowledge. Due to the inevitable uncertainties associated with the current and expected litigation it cannot be excluded that a future assessment of the risks may be different.

Our opinions on the consolidated financial statements and on the group management report are not modified in respect of this matter.

Source: Volkswagen 2017 annual report

In the United States, under the Sarbanes-Oxley Act, the auditors must also express an opinion on the company's internal control systems. This information may be provided in a separate opinion or incorporated as a paragraph in the opinion related to the financial statements. The internal control system is the company's internal system that is designed, among other things, to ensure that the company's process for generating financial reports is sound. Although management has always been responsible for maintaining effective internal control, the Sarbanes-Oxley Act greatly increases management's responsibility for demonstrating that the company's internal controls are effective. Management of publicly traded companies in the United States are now required by securities regulators to explicitly accept responsibility for the effectiveness of internal control, evaluate the effectiveness of internal control using suitable control criteria, support the evaluation with sufficient competent evidence, and provide a report on internal control.

Although these reports and attestations provide some assurances to analysts, they are not infallible. The analyst must always use a degree of healthy skepticism when analyzing financial statements.

\section{OTHER SOURCES OF INFORMATION}
identify and describe information sources that analysts use in financial statement analysis besides annual financial statements and supplementary information

The information described in Section 3.1 is generally provided to shareholders at least annually. In addition, companies also provide information on management and director compensation, company stock performance, and any potential conflicts of interest that may exist between management, the board, and shareholders. This information may appear in the company's annual report or other publicly available documents. Public companies often provide this information in proxy statements, which are distributed to shareholders about matters that are to be put to a vote at the annual (or special) meeting of shareholders. Interim reports are also provided by the company either semiannually or quarterly, depending on the applicable regulatory requirements. Interim reports generally present the four primary financial statements and condensed notes but are not audited. These interim reports provide updated information on a company's performance and financial position since the last annual period.

Companies also provide relevant current information on their websites, in press releases, and in conference calls with analysts and investors. One type of press release, which analysts often consider to be particularly important, is the periodic earnings announcement. The earnings announcement often happens well before the company files its formal financial statements. Such earnings announcements are often followed by a conference call in which the company's senior executives describe the company's performance and answer questions posed by conference call participants. Following the earnings conference call, the investor relations portion of the company's website may post a recording of the call accompanied by slides and supplemental information that was discussed.

When performing financial statement analysis, analysts should review all these company sources of information as well as information from external sources regarding the economy, the industry, the company, and peer (comparable) companies. Information on the economy, industry, and peer companies is useful in putting the company's financial performance and position in perspective and in assessing the company's future. In most cases, information from sources apart from the company is crucial to an analyst's effectiveness. For example, an analyst studying a consumer-oriented company will typically seek direct experience with the products (taste the food or drink, use the shampoo or soap, visit the stores or hotels). An analyst following a highly regulated industry will study the existing and expected relevant regulations. An analyst following a highly technical industry will gain relevant expertise personally or seek input from a technical specialist. In sum, thorough research goes beyond financial reports.

The next section presents a framework for using all this information in financial statement analysis.

FINANCIAL STATEMENT ANALYSIS FRAMEWORK

describe the steps in the financial statement analysis framework

Analysts work in a variety of positions within the investment management industry. Some are equity analysts whose main objective is to evaluate potential investments in a company's equity securities (i.e., the shares or stock it issues) as a basis for deciding whether a prospective investment is attractive and what an appropriate purchase price might be. Others are credit analysts who evaluate the credit-worthiness of a company to decide whether (and on what terms) a loan should be made or what credit rating should be assigned. Analysts may also be involved in a variety of other tasks, such as evaluating the performance of a subsidiary company, evaluating a private equity investment, or finding stocks that are overvalued for purposes of taking a short position.

This section presents a generic framework for financial statement analysis that can be used in these various tasks. The framework is summarized in Exhibit 11. ${ }^{8}$

8 Components of this framework have been adapted from van Greuning and Bratanovic (2003, p. 300) and from Benninga and Sarig (1997, pp. 134-156).

\section{Exhibit 11: Financial Statement Analysis Framework}
\begin{center}
\includegraphics[max width=\textwidth]{2023_05_04_ccb578c5089fd41615beg-579}
\end{center}

The following sections discuss the individual phases of financial statement analysis.

\section{Articulate the Purpose and Context of Analysis}
Prior to undertaking any analysis, it is essential to understand the purpose of the analysis. An understanding of the purpose is particularly important in financial statement analysis because of the numerous available techniques and the substantial amount of data.

Some analytical tasks are well defined, in which case articulating the purpose of the analysis requires little decision making by the analyst. For example, a periodic credit review of an investment-grade debt portfolio or an equity analyst's report on a particular company may be guided by institutional norms such that the purpose of the analysis is given. Furthermore, the format, procedures, and/or sources of information may also be given. For other analytical tasks, articulating the purpose of the analysis requires the analyst to make decisions. The purpose of an analysis guides further decisions about the approach, the tools, the data sources, the format in which to report the results of the analysis, and the relative importance of different aspects of the analysis.

When facing a substantial amount of data, a less experienced analyst may be tempted to just start making calculations and generating financial ratios without considering what is relevant for the decision at hand. It is generally advisable to resist this temptation and thus avoid unnecessary or pointless efforts. Consider the questions: If you could have all the calculations and ratios completed instantly, what conclusion would you be able to draw? What question would you be able to answer? What decision would your answer support?

The analyst should also define the context at this stage. Who is the intended audience? What is the end product-for example, a final report explaining conclusions and recommendations? What is the time frame (i.e., when is the report due)? What resources and resource constraints are relevant to completion of the analysis? Again, the context may be predefined (i.e., standard and guided by institutional norms).

Having clarified the purpose and context of the financial statement analysis, the analyst should next compile the specific questions to be answered by the analysis. For example, if the purpose of the financial statement analysis (or, more likely, the particular stage of a larger analysis) is to compare the historical performance of three companies operating in a particular industry, specific questions would include the following: What has been the relative growth rate of the companies, and what has been their relative profitability?

\section{Collect Data}
Next, the analyst obtains the data required to answer the specific questions. A key part of this step is obtaining an understanding of the company's business, financial performance, and financial position (including trends over time and in comparison with peer companies). For historical analyses, financial statement data alone are adequate in some cases. For example, to screen a large number of companies to find those with a minimum level of profitability, financial statement data alone would be adequate. But to address more in-depth questions, such as why and how one company performed better or worse than its competitors, additional information would be required. As another example, to compare the historical performance of two companies in a particular industry, the historical financial statements would be sufficient to determine which had faster-growing sales or earnings and which was more profitable. However, a broader comparison with overall industry growth and profitability would obviously require industry data.

Furthermore, information on the economy and industry is necessary to understand the environment in which the company operates. Analysts often take a top-down approach whereby they 1) gain an understanding of the macroeconomic environment, such as prospects for growth in the economy and inflation, 2) analyze the prospects of the industry in which the company operates, based on the expected macroeconomic environment, and 3) determine the prospects for the company given the expected industry and macroeconomic environments. For example, an analyst may need to forecast future growth in earnings for a company. Past company data provide the platform for statistical forecasting; however, an understanding of economic and industry conditions can improve the analyst's ability to forecast earnings.

\section{Process Data}
After obtaining the requisite financial statement and other information, the analyst processes these data using appropriate analytical tools. For example, processing the data may involve computing ratios or growth rates; preparing common-size financial statements; creating charts; performing statistical analyses, such as regressions or Monte Carlo simulations; performing equity valuation; performing sensitivity analyses; or using any other analytical tools or combination of tools that are available and appropriate for the task. A comprehensive financial analysis at this stage would include the following:

\begin{itemize}
  \item Reading and evaluating financial statements for each company being analyzed. This includes reading the notes and understanding what accounting standards have been used (for example, IFRS or US GAAP), what accounting choices have been made (for example, when to report revenue on the income statement), and what operating decisions have been made that affect reported financial statements (for example, leasing versus purchasing equipment).

  \item Making any needed adjustments to the financial statements to facilitate comparison when the unadjusted statements of the subject companies reflect differences in accounting standards, accounting choices, or operating decisions. Note that commonly used databases do not always make such analyst adjustments.

  \item Preparing or collecting common-size financial statement data (which scale data to directly reflect percentages [e.g., of sales] or changes [e.g., from the prior year]) and financial ratios (which are measures of various aspects of corporate performance based on financial statement elements). On the basis of common-size financial statements and financial ratios, analysts can evaluate a company's relative profitability, liquidity, leverage, efficiency, and valuation in relation to past results and/or peers' results.

\end{itemize}

\section{Analyze/Interpret the Processed Data}
Once the data have been processed, the next step-critical to any analysis-is to interpret the output. The answer to a specific question is seldom the numerical answer alone. Rather, the answer relies on the analyst's interpretation of the output, and the use of this interpreted output to support a conclusion or recommendation. The answers to the specific analytical questions may themselves achieve the underlying purpose of the analysis, but usually, a conclusion or recommendation is required. For example, an equity analysis may require a buy, hold, or sell recommendation or a conclusion about the value of a share of stock. In support of the decision, the analysis would cite such information as target value, relative performance, expected future performance given a company's strategic position, quality of management, and whatever other information influenced the decision.

\section{Develop and Communicate Conclusions/Recommendations}
Communicating the conclusion or recommendation in an appropriate format is the next step. The appropriate format will vary by analytical task, by institution, and/ or by audience. For example, an equity analyst's report would typically include the following components:

\begin{itemize}
  \item summary and investment conclusion;

  \item earnings projections; - valuation;

  \item business summary;

  \item risk, industry, and competitive analysis;

  \item historical performance; and

  \item forecasts.

\end{itemize}

The contents of reports may also be specified by regulatory agencies or professional standards. For example, the CFA Institute Standards of Practice Handbook (Handbook) dictates standards that must be followed in communicating recommendations. According to the Handbook:

Standard V(B) states that members and candidates should communicate in a recommendation the factors that were instrumental in making the investment recommendation. A critical part of this requirement is to distinguish clearly between opinions and facts. In preparing a research report, the member or candidate must present the basic characteristics of the security(ies) being analyzed, which will allow the reader to evaluate the report and incorporate information the reader deems relevant to his or her investment decision making process. ${ }^{9}$

The Handbook requires that limitations to the analysis and any risks inherent to the investment be disclosed. Furthermore, it requires that any report include elements important to the analysis and conclusions so that readers can evaluate the conclusions themselves.

\section{Follow-Up}
The process does not end with the report. If an equity investment is made or a credit rating is assigned, periodic review is required to determine if the original conclusions and recommendations are still valid. In the case of a rejected investment, follow-up may not be necessary but may be useful in determining whether the analytical process is adequate or should be refined (for example, if a rejected investment turns out to be successful in the market, perhaps the rejection was due to inadequate analysis). Follow-up may involve repeating all the previous steps in the process on a periodic basis.

\section{SUMMARY}
The information presented in financial and other reports, including the financial statements, notes, and management's commentary, help the financial analyst to assess a company's performance and financial position. An analyst may be called on to perform a financial analysis for a variety of reasons, including the valuation of equity securities, the assessment of credit risk, the performance of due diligence on an acquisition, and the evaluation of a subsidiary's performance relative to other business units. Major considerations in both equity analysis and credit analysis are evaluating a company's financial position, its ability to generate profits and cash flow, and its potential to generate future growth in profits and cash flow.

9 Standards of Practice Handbook (2014, p. 169). This reading has presented an overview of financial statement analysis. Among the major points covered are the following:

\begin{itemize}
  \item The primary purpose of financial reports is to provide information and data about a company's financial position and performance, including profitability and cash flows. The information presented in the reports -including the financial statements and notes and management's commentary or management's discussion and analysis-allows the financial analyst to assess a company's financial position and performance and trends in that performance.

  \item The primary financial statements are the statement of financial position (i.e., the balance sheet), the statement of comprehensive income (or two statements consisting of an income statement and a statement of comprehensive income), the statement of changes in equity, and the statement of cash flows.

  \item The balance sheet discloses what resources a company controls (assets) and what it owes (liabilities) at a specific point in time. Owners' equity represents the net assets of the company; it is the owners' residual interest in, or residual claim on, the company's assets after deducting its liabilities. The relationship among the three parts of the balance sheet (assets, liabilities, and owners' equity) may be shown in equation form as follows: Assets $=$ Liabilities + Owners' equity.

  \item The income statement presents information on the financial results of a company's business activities over a period of time. The income statement communicates how much revenue and other income the company generated during a period and what expenses, including losses, it incurred in connection with generating that revenue and other income. The basic equation underlying the income statement is Revenue + Other income - Expenses $=$ Net income.

  \item The statement of comprehensive income includes all items that change owners' equity except transactions with owners. Some of these items are included as part of net income, and some are reported as other comprehensive income (OCI).

  \item The statement of changes in equity provides information about increases or decreases in the various components of owners' equity.

  \item Although the income statement and balance sheet provide measures of a company's success, cash and cash flow are also vital to a company's long-term success. Disclosing the sources and uses of cash helps creditors, investors, and other statement users evaluate the company's liquidity, solvency, and financial flexibility.

  \item The notes (also referred to as footnotes) that accompany the financial statements are an integral part of those statements and provide information that is essential to understanding the statements. Analysts should evaluate note disclosures regarding the use of alternative accounting methods, estimates, and assumptions.

  \item In addition to the financial statements, a company provides other sources of information that are useful to the financial analyst. As part of his or her analysis, the financial analyst should read and assess this additional information, particularly that presented in the management commentary (also called management report[ing], operating and financial review, and management's discussion and analysis [MD\&A]). - A publicly traded company must have an independent audit performed on its annual financial statements. The auditor's report expresses an opinion on the financial statements and provides some assurance about whether the financial statements fairly present a company's financial position, performance, and cash flows. In addition, for US publicly traded companies, auditors must also express an opinion on the company's internal control systems.

  \item Information on the economy, industry, and peer companies is useful in putting the company's financial performance and position in perspective and in assessing the company's future. In most cases, information from sources apart from the company are crucial to an analyst's effectiveness.

  \item The financial statement analysis framework provides steps that can be followed in any financial statement analysis project. These steps are:

  \item articulate the purpose and context of the analysis;

  \item collect input data;

  \item process data;

  \item analyze/interpret the processed data;

  \item develop and communicate conclusions and recommendations; and - follow up.

\end{itemize}

\section{REFERENCES}
Benninga, Simon Z., Oded H. Sarig. 1997. Corporate Finance: A Valuation Approach. New York: McGraw-Hill Publishing.

van Greuning, Hennie, Sonja Brajovic Bratanovic. 2003. Analyzing and Managing Banking Risk: A Framework for Assessing Corporate Governance and Financial Risk. Washington, DC: World Bank.

\section{PRACTICE PROBLEMS}
\begin{enumerate}
  \item Which of the following best describes the role of financial statement analysis?
A. To provide information about a company's performance
B. To provide information about a company's changes in financial position
C. To form expectations about a company's future performance and financial position

  \item The role of financial statement analysis is best described as:

\end{enumerate}

A. providing information useful for making investment decisions.

B. evaluating a company for the purpose of making economic decisions.

C. using financial reports prepared by analysts to make economic decisions.

\begin{enumerate}
  \setcounter{enumi}{2}
  \item Providing information about the performance and financial position of companies so that users can make economic decisions best describes the role of:
A. auditing.
B. financial reporting.
C. financial statement analysis.

  \item A company's financial position would best be evaluated using the:

\end{enumerate}

A. balance sheet.

B. income statement.

C. statement of cash flows.

\begin{enumerate}
  \setcounter{enumi}{4}
  \item A company's profitability for a period would best be evaluated using the:
\end{enumerate}

A. balance sheet.

B. income statement.

C. statement of cash flows.

\begin{enumerate}
  \setcounter{enumi}{5}
  \item The financial statement that presents a shareholder's residual claim on assets is the:
A. balance sheet.
B. income statement.
C. cash flow statement.

  \item A company's profitability over a period of time is best evaluated using the:

\end{enumerate}

A. balance sheet.

B. income statement.

C. cash flow statement. 8. The income statement is best used to evaluate a company's:
A. financial position.
B. sources of cash flow.
C. financial results from business activities.

\begin{enumerate}
  \setcounter{enumi}{8}
  \item Which of the following best describes why the notes that accompany the financial statements are required? The notes:
A. permit flexibility in statement preparation.
B. standardize financial reporting across companies.
C. provide information necessary to understand the financial statements.

  \item Accounting policies, methods, and estimates used in preparing financial statements are most likely to be found in the:
A. auditor's report.
B. management commentary.
C. notes to the financial statements.

  \item Information about management and director compensation are least likely to be found in the:
A. auditor's report.
B. proxy statement.
C. notes to the financial statements.

  \item Information about a company's objectives, strategies, and significant risks are most likely to be found in the:
A. auditor's report.
B. management commentary.
C. notes to the financial statements.

  \item What type of audit opinion is preferred when analyzing financial statements?
A. Qualified.
B. Adverse.
C. Unqualified.

  \item An auditor determines that a company's financial statements are prepared in accordance with applicable accounting standards except with respect to inventory reporting. This exception is most likely to result in an audit opinion that is:
A. adverse.
B. qualified.
C. unqualified. 15. An independent audit report is most likely to provide:
A. absolute assurance about the accuracy of the financial statements.
B. reasonable assurance that the financial statements are fairly presented.
C. a qualified opinion with respect to the transparency of the financial statements.

  \item Interim financial reports released by a company are most likely to be:
A. monthly.
B. unaudited.
C. unqualified.

  \item Which of the following sources of information used by analysts is found outside a company's annual report?
A. Auditor's report
B. Peer company analysis
C. Management's discussion and analysis

  \item Ratios are an input into which step in the financial statement analysis framework?
A. Process data.
B. Collect input data.
C. Analyze/interpret the processed data.

  \item Which phase in the financial statement analysis framework is most likely to involve producing updated reports and recommendations?
A. Follow-up
B. Analyze/interpret the processed data
C. Develop and communicate conclusions and recommendations

\end{enumerate}

\section{SOLUTIONS}
\begin{enumerate}
  \item C is correct. In general, analysts seek to examine the past and current performance and financial position of a company in order to form expectations about its future performance and financial position.

  \item B is correct. The primary role of financial statement analysis is to use financial reports prepared by companies to evaluate their past, current, and potential performance and financial position for the purpose of making investment, credit, and other economic decisions.

  \item B is correct. This is the role of financial reporting. The role of financial statement analysis is to evaluate the financial reports.

  \item A is correct. The balance sheet portrays the company's financial position on a specified date. The income statement and statement of cash flows present different aspects of performance during the period.

  \item B is correct. Profitability is the performance aspect measured by the income statement. The balance sheet portrays the financial position. The statement of cash flows presents a different aspect of performance.

  \item A is correct. Owners' equity is the owners' residual interest in (i.e., residual claim on) the company's assets after deducting its liabilities, which is information presented on the balance sheet.

  \item B is correct. A company's profitability is best evaluated using the income statement. The income statement presents information on the financial results of a company's business activities over a period of time by communicating how much revenue was generated and the expenses incurred to generate that revenue.

  \item C is correct. A company's revenues and expenses are presented on the income statement, which is used to evaluate a company's financial results (or profitability) from business activities over a period of time. A company's financial position is best evaluated by using the balance sheet. A company's sources of cash flow are best evaluated using the cash flow statement.

  \item $\mathrm{C}$ is correct. The notes provide information that is essential to understanding the information provided in the primary statements.

  \item $\mathrm{C}$ is correct. The notes disclose choices in accounting policies, methods, and estimates.

  \item A is correct. Information about management and director compensation is not found in the auditor's report. Disclosure of management compensation is required in the proxy statement, and some aspects of management compensation are disclosed in the notes to the financial statements.

  \item B is correct. These are components of management commentary.

  \item $\mathrm{C}$ is correct. An unqualified opinion is a "clean" opinion and indicates that the financial statements present the company's performance and financial position fairly in accordance with a specified set of accounting standards.

  \item B is correct. A qualified audit opinion is one in which there is some scope limitation or exception to accounting standards. Exceptions are described in the audit report with additional explanatory paragraphs so that the analyst can determine the importance of the exception.

  \item B is correct. The independent audit report provides reasonable assurance that the financial statements are fairly presented, meaning that there is a high probability that the audited financial statements are free from material error, fraud, or illegal acts that have a direct effect on the financial statements.

  \item B is correct. Interim reports are typically provided semiannually or quarterly and present the four basic financial statements and condensed notes. They are not audited. Unqualified is a type of audit opinion

  \item B is correct. When performing financial statement analysis, analysts should review all company sources of information as well as information from external sources regarding the economy, the industry, the company, and peer (comparable) companies.

  \item $C$ is correct. Ratios are an output of the process data step but are an input into the analyze/interpret data step.

  \item A is correct. The follow-up phase involves gathering information and repeating the analysis to determine whether it is necessary to update reports and recommendations.

\end{enumerate}

\section*{LEARNING MODULE 2 }
\section{Financial Reporting Standards}
 Thomas R. Robinson, PhD, CFA, CAIA.Elaine Henry, PhD, CFA, is at Stevens Institute of Technology (USA). J. Hennie van Greuning, DCom, CFA, is at BIBD (Brunei). Thomas R. Robinson, PhD, CFA, CAIA,

Robinson Global Investment Management LLC, (USA).

\section{LEARNING OUTCOME}
\begin{center}
\begin{tabular}{|c|c|}
\hline
Mastery & The candidate should be able to: \\
\hline
 & $\begin{array}{l}\text { describe the objective of financial reporting and the importance of } \\ \text { financial reporting standards in security analysis and valuation }\end{array}$ \\
\hline
 & $\begin{array}{l}\text { describe the roles of financial reporting standard-setting bodies } \\ \text { and regulatory authorities in establishing and enforcing reporting } \\ \text { standards }\end{array}$ \\
\hline
 & $\begin{array}{l}\text { describe the International Accounting Standards Board's conceptual } \\ \text { framework, including qualitative characteristics of financial reports, } \\ \text { constraints on financial reports, and required reporting elements }\end{array}$ \\
\hline
 & $\begin{array}{l}\text { describe general requirements for financial statements under } \\ \text { International Financial Reporting Standards (IFRS) }\end{array}$ \\
\hline
 & $\begin{array}{l}\text { describe implications for financial analysis of alternative financial } \\ \text { reporting systems and the importance of monitoring developments } \\ \text { in financial reporting standards }\end{array}$ \\
\hline
\end{tabular}
\end{center}

\section{INTRODUCTION}
Financial reporting standards provide principles for preparing financial reports and determine the types and amounts of information that must be provided to users of financial statements, including investors and creditors, so that they may make informed decisions. This reading focuses on the context within which these standards are created. An understanding of the underlying framework of financial reporting standards, which is broader than knowledge of specific accounting rules, will allow an analyst to assess the valuation implications of financial statement elements and transactions-including transactions, such as those that represent new developments, which are not specifically addressed by the standards. Section 2 of this reading discusses the objective of financial reporting and the importance of financial reporting standards in security analysis and valuation. Section 3 describes the roles of financial reporting standard-setting bodies and regulatory authorities and several of the financial reporting standard-setting bodies and regulatory authorities. Section 4 describes the International Financial Reporting Standards (IFRS) framework $^{1}$ and general requirements for financial statements. Section 5 compares IFRS and alternative reporting systems, and Section 6 discusses the importance of monitoring developments in financial reporting standards. A summary of the key points concludes the reading.

\section{THE OBJECTIVE OF FINANCIAL REPORTING}
describe the objective of financial reporting and the importance of financial reporting standards in security analysis and valuation

The financial reports of a company include financial statements and other supplemental disclosures necessary to assess a company's financial position and periodic financial performance. Financial reporting is based on a simple premise. The International Accounting Standards Board (IASB), which sets financial reporting standards that have been adopted in many countries, said in its Conceptual Framework for Financial Reporting (Conceptual Framework) that the objective of financial reporting is to provide financial information that is useful to users in making decisions about providing resources to the reporting entity, where those decisions relate to equity and debt instruments, or loans or other forms of credit, and in influencing management's actions that affect the use of the entity's economic resources. ${ }^{2}$

A fully articulated framework is an essential first step to guide the development of a set of standards. Previously, financial reporting standards were primarily developed independently by each country's standard-setting body. This independent standard setting process created a wide range of standards, some of which were quite comprehensive and complex (rules-based standards), and others that were more general (principles-based standards). The globalization of capital flows and various accounting scandals increased awareness of the need for more uniform, high quality global financial reporting standards and provided the impetus for stronger coordination among the major standard-setting bodies. Such coordination is also a natural outgrowth of the increased globalization of capital markets.

Developing financial reporting standards is complicated because the underlying economic reality is complicated. The financial transactions and financial position that companies aim to present in their financial reports are also complex. Furthermore, uncertainty about various aspects of transactions often results in the need for accruals and estimates, both of which necessitate judgment. Judgment varies from one preparer to the next. Accordingly, standards are needed to achieve some amount of consistency in these judgments. Even with such standards, there usually will be no single correct answer to the question of how to reflect economic reality in financial reports. Nevertheless, financial reporting standards try to limit the range of acceptable answers to increase consistency in financial reports.

1 The body of standards issued by the International Accounting Standards Board (IASB) is referred to as International Financial Reporting Standards.

2 In March 2018, the IASB updated the Conceptual Framework for Financial Reporting originally adopted in 2010. The IASB and the US-based Financial Accounting Standards Board (FASB) have developed similar financial reporting frameworks which specify the overall objective and qualities of information to be provided. Financial reports are intended to provide information to many users, including investors, creditors, employees, customers, and others. As a result, financial reports are not designed solely with asset valuation in mind. However, financial reports provide important inputs into the process of valuing a company or the securities a company issues. Understanding the financial reporting framework-including how and when judgments and estimates can affect the numbers reported-enables an analyst to evaluate the information reported and to use the information appropriately when assessing a company's financial performance. Clearly, such an understanding is also important in assessing the financial impact of business decisions by, and in making comparisons across, entities.

\section{ACCOUNTING STANDARDS BOARDS}
$$
\begin{aligned}
& \text { describe the objective of financial reporting and the importance of } \\
& \text { financial reporting standards in security analysis and valuation } \\
& \text { describe the roles of financial reporting standard-setting bodies } \\
& \text { and regulatory authorities in establishing and enforcing reporting } \\
& \text { standards }
\end{aligned}
$$

A distinction must be made between standard-setting bodies and regulatory authorities. Standard-setting bodies, such as the IASB and FASB, are typically private sector, self-regulated organizations with board members who are experienced accountants, auditors, users of financial statements, and academics. The requirement to prepare financial reports in accordance with specified accounting standards is the responsibility of regulatory authorities. Regulatory authorities, such as the Accounting and Corporate Regulatory Authority in Singapore, the Securities and Exchange Commission (SEC) in the United States, and the Securities and Exchange Commission of Brazil have the legal authority to enforce financial reporting requirements and exert other controls over entities that participate in the capital markets within their jurisdiction.

In other words, generally, standard-setting bodies set the standards and regulatory authorities recognise and enforce the standards. Without the recognition of the standards by the regulatory authorities, the private sector standard-setting bodies would have no authority. Note, however, that regulators often retain the legal authority to establish financial reporting standards in their jurisdiction and can overrule the private sector standard-setting bodies.

This section provides a brief overview of the IASB and the FASB. The overview is followed by descriptions of the International Organization of Securities Commissions, the US Securities and Exchange Commission, and capital markets regulation in the European Union. The topics covered in these overviews were chosen to serve as examples of standard-setting boards, securities commissions, and capital market regulation. After reading these descriptions, the reader should be able to describe the functioning and roles of standard-setting bodies and regulatory authorities in more detail than is given in the introduction to this section.

\section{Accounting Standards Boards}
Accounting standards boards exist in virtually every national market. These boards are typically independent, private, not-for-profit organizations. There are certain attributes that are typically common to these standard setters-the IASB and the FASB are discussed in this section as primary examples.

\section{International Accounting Standards Board}
The IASB is the independent standard-setting body of the IFRS Foundation, an independent, not-for-profit private sector organization. The Trustees of the IFRS Foundation reflect a diversity of geographical and professional backgrounds. The Trustees appoint the members of the IASB and are accountable to a monitoring board composed of public authorities that include representatives from the European Commission, IOSCO, the Japan Financial Services Agency, and the US SEC, with the chairman of the Basel Committee on Banking Supervision as an observer.

The Trustees of the IFRS Foundation make a commitment to act in the public interest. The principle objectives of the IFRS Foundation are to develop and promote the use and adoption of a single set of high quality financial standards; to ensure the standards result in transparent, comparable, and decision-useful information while taking into account the needs of a range of sizes and types of entities in diverse economic settings; and to promote the convergence of national accounting standards and IFRS. The Trustees are responsible for ensuring that the IASB is and is perceived as independent.

The members of the IASB are appointed by the Trustees on the basis of professional competence and practical experience and reflect a diversity of geographical and professional backgrounds. The members deliberate, develop, and issue international financial reporting standards, ${ }^{3}$ assisted by advice on the standards, and their application, from advisory bodies whose members represent a wide range of organizations and individuals that are affected by and interested in international financial reporting.

The IASB has a basic process that it goes through when deliberating, developing, and issuing international financial reporting standards. A simplified version of the typical process is as follows. An issue is identified as a priority for consideration and placed on the IASB's agenda. After considering an issue, which may include soliciting advice from others including national standard-setters, the IASB may publish an exposure draft for public comment. After reviewing the input of others, the IASB may issue a new or revised financial reporting standard. These standards are authoritative to the extent that they are recognised and adopted by regulatory authorities.

\section{Financial Accounting Standards Board}
The FASB and its predecessor organizations have been issuing financial reporting standards in the United States since the 1930s. The FASB operates within a structure similar to that of the IASB. The Financial Accounting Foundation oversees, administers, and finances the organization. The Foundation ensures the independence of the standard-setting process and appoints members to the FASB and related advisory entities.

The FASB issues new and revised standards to improve standards of financial reporting so that decision-useful information is provided to users of financial reports. This is done through a thorough and independent process that seeks input from stakeholders

3 Although the name of the IASB incorporates "Accounting Standards" and early standards were titled International Accounting Standards (IAS), the term "International Financial Reporting Standards" (IFRS) is being used for new standards. The use of the words "financial reporting" recognizes the importance of disclosures outside of the core financial statements, such as management discussion of the business, risks, and future plans. and is overseen by the Financial Accounting Foundation. The steps in the process are similar to those described for the IASB. The outputs of the standard-setting process are contained in the FASB Accounting Standards Codification ${ }^{\mathrm{mm}}$ (Codification). ${ }^{4}$ The Codification, organized by topic, is the source of authoritative US generally accepted accounting principles to be applied to non-governmental entities.

US GAAP, as established by the FASB, is officially recognized as authoritative by the SEC, however the SEC retains the authority to establish standards. Although it has rarely overruled the FASB, the SEC does issue authoritative financial reporting guidance including Staff Accounting Bulletins. These bulletins reflect the SEC's views regarding accounting-related disclosure practices and can be found on the SEC website. Certain portions-but not all portions-of the SEC regulations, releases, interpretations, and guidance are included for reference in the FASB Codification.

\section*{REGULATORY AUTHORITIES }
\begin{abstract}
describe the objective of financial reporting and the importance of financial reporting standards in security analysis and valuation describe the roles of financial reporting standard-setting bodies and regulatory authorities in establishing and enforcing reporting standards
\end{abstract}

The requirement to prepare financial reports in accordance with specified accounting standards is the responsibility of regulatory authorities. Regulatory authorities are governmental entities that have the legal authority to enforce financial reporting requirements and exert other controls over entities that participate in the capital markets within their jurisdiction. Regulatory authorities may require that financial reports be prepared in accordance with one specific set of accounting standards or may specify acceptable accounting standards. For example in Switzerland, Swiss-based companies listed on the main board of the Swiss Exchange have to prepare their financial statements in accordance with either IFRS or US GAAP if they are multinational. ${ }^{5}$ Other registrants in Switzerland could use IFRS, US GAAP, or Swiss GAAP.

\section{International Organization of Securities Commissions}
While technically not a regulatory authority, the International Organization of Securities Commissions (IOSCO) regulate a significant portion of the world's financial capital markets. This organization has established objectives and principles to guide securities and capital market regulation.

IOSCO was formed in 1983 and consists of ordinary members, associate members, and affiliate members. Ordinary members are the securities commission or similar governmental regulatory authority with primary responsibility for securities regulation

4 The Codification combines literature issued by various standard setters, including the FASB, the Emerging Issues Task Force (EITF), the Derivative Implementation Group (DIG), and the American Institute of Certified Public Accountants (AICPA).

5 \href{https://www.iasplus.com/en/jurisdictions/europe/switzerland}{https://www.iasplus.com/en/jurisdictions/europe/switzerland}. in the member country. ${ }^{6}$ The members regulate more than 95 percent of the world's financial capital markets in more than 115 jurisdictions, and securities regulators in emerging markets account for $75 \%$ of its ordinary membership.

IOSCO's comprehensive set of Objectives and Principles of Securities Regulation is updated as required and is recognized as an international benchmark for all markets. The principles of securities regulation are based upon three core objectives: ${ }^{7}$

\begin{itemize}
  \item protecting investors;

  \item ensuring that markets are fair, efficient, and transparent; and

  \item reducing systemic risk.

\end{itemize}

IOSCO's principles are grouped into ten categories, including principles for regulators, for enforcement, for auditing, and for issuers, among others. Within the category "Principles for Issuers," two principles relate directly to financial reporting:

\begin{itemize}
  \item There should be full, accurate, and timely disclosure of financial results, risk, and other information which is material to investors' decisions.

  \item Accounting standards used by issuers to prepare financial statements should be of a high and internationally acceptable quality.

\end{itemize}

Historically, regulation and related financial reporting standards were developed within individual countries and were often based on the cultural, economic, and political norms of each country. As financial markets have become more global, it has become desirable to establish comparable financial reporting standards internationally. Ultimately, laws and regulations are established by individual jurisdictions, so this also requires cooperation among regulators. Another IOSCO principle deals with the use of self-regulatory organizations (SROs), which exercise some direct oversight for their areas of competence and should be subject to the oversight of the relevant regulator and observe fairness and confidentiality 8 .

To ensure consistent application of international financial standards (such as the Basel Committee on Banking Supervision's standards and IFRS), it is important to have uniform regulation and enforcement across national boundaries. IOSCO assists in attaining this goal of uniform regulation as well as cross-border co-operation in combating violations of securities and derivatives laws.

\section{The Securities and Exchange Commission (US)}
The US SEC has primary responsibility for securities and capital markets regulation in the United States and is an ordinary member of IOSCO. Any company issuing securities within the United States, or otherwise involved in US capital markets, is subject to the rules and regulations of the SEC. The SEC, one of the oldest and most developed regulatory authorities, originated as a result of reform efforts made after the stock market crash of 1929 , sometimes referred to as simply the "Great Crash."

6 The names of the primary securities regulator vary from country to country. For example: China Securities Regulatory Commission, Egyptian Financial Supervisory Authority, Securities and Exchange Board of India, Kingdom of Saudi Arabia Capital Market Authority, and Banco Central del Uruguay.

7 Objectives and Principles of Securities Regulation, IOSCO, May 2017.

8 Objectives and Principles of Securities Regulation, IOSCO, May 2017. A number of laws affect reporting companies, broker/dealers, and other market participants. From a financial reporting and analysis perspective, the most significant pieces of legislation are the Securities Acts of 1933 and 1934 and the Sarbanes-Oxley Act of 2002.

\begin{itemize}
  \item Securities Act of 1933 (The 1933 Act): This act specifies the financial and other significant information that investors must receive when securities are sold, prohibits misrepresentations, and requires initial registration of all public issuances of securities.

  \item Securities Exchange Act of 1934 (The $1934 \mathrm{Act}$ ): This act created the SEC, gave the SEC authority over all aspects of the securities industry, and empowered the SEC to require periodic reporting by companies with publicly traded securities.

  \item Sarbanes-Oxley Act of 2002: This act created the Public Company Accounting Oversight Board (PCAOB) to oversee auditors. The SEC is responsible for carrying out the requirements of the act and overseeing the PCAOB. The act addresses auditor independence (it prohibits auditors from providing certain non-audit services to the companies they audit); strengthens corporate responsibility for financial reports (it requires executive management to certify that the company's financial reports fairly present the company's condition); and requires management to report on the effectiveness of the company's internal control over financial reporting (including obtaining external auditor confirmation of the effectiveness of internal control).

\end{itemize}

Companies comply with these acts principally through the completion and submission (i.e., filing) of standardized forms issued by the SEC. There are more than 50 different types of SEC forms that are used to satisfy reporting requirements; the discussion herein will be limited to those forms most relevant for financial analysts.

Most of the SEC filings are required to be made electronically, so filings that an analyst would be interested in can be retrieved online from one of many websites, including the SEC's own website. Some filings are required on the initial offering of securities, whereas others are required on a periodic basis thereafter. The following are some of the more common information sources used by analysts.

\begin{itemize}
  \item Securities Offerings Registration Statement: The 1933 Act requires companies offering securities to file a registration statement. New issuers as well as previously registered companies that are issuing new securities are required to file these statements. Required information and the precise form vary depending upon the size and nature of the offering. Typically, required information includes: 1) disclosures about the securities being offered for sale, 2) the relationship of these new securities to the issuer's other capital securities, 3) the information typically provided in the annual filings, 4) recent audited financial statements, and 5) risk factors involved in the business.

  \item Forms 10-K, 20-F, and 40-F: These are forms that companies are required to file annually. Form 10-K is for US registrants, Form 40-F is for certain Canadian registrants, and Form 20-F is for all other non-US registrants. These forms require a comprehensive overview, including information concerning a company's business, financial disclosures, legal proceedings, and information related to management. The financial disclosures include a historical summary of financial data (usually 10 years), management's discussion and analysis (MD\&A) of the company's financial condition and results of operations, and audited financial statements. - Annual Report: In addition to the SEC's annual filings (e.g., Form 10-K), most companies prepare an annual report to shareholders. This is not a requirement of the SEC. The annual report is usually viewed as one of the most significant opportunities for a company to present itself to shareholders and other external parties; accordingly, it is often a highly polished marketing document with photographs, an opening letter from the chief executive officer, financial data, market segment information, research and development activities, and future corporate goals. In contrast, the Form $10-\mathrm{K}$ is a more legal type of document with minimal marketing emphasis. Although the perspectives vary, there is considerable overlap between a company's annual report and its Form 10-K. Some companies elect to prepare just the Form $10-\mathrm{K}$ or a document that integrates both the $10-\mathrm{K}$ and annual report.

  \item Proxy Statement/Form DEF-14A: The SEC requires that shareholders of a company receive a proxy statement prior to a shareholder meeting. A proxy is an authorization from the shareholder giving another party the right to cast its vote. Shareholder meetings are held at least once a year, but any special meetings also require a proxy statement. Proxies, especially annual meeting proxies, contain information that is often useful to financial analysts. Such information typically includes proposals that require a shareholder vote, details of security ownership by management and principal owners, biographical information on directors, and disclosure of executive compensation. Proxy statement information is filed with the SEC as Form DEF-14A.

  \item Forms 10- $Q$ and 6-K: These are forms that companies are required to submit for interim periods (quarterly for US companies on Form 10-Q, semiannually for many non-US companies on Form 6-K). The filing requires certain financial information, including unaudited financial statements and a MD\&A for the interim period covered by the report. Additionally, if certain types of non-recurring events-such as the adoption of a significant accounting policy, commencement of significant litigation, or a material limitation on the rights of any holders of any class of registered securitiestake place during the period covered by the report, these events must be included in the Form 10-Q report. Companies may provide the 10-Q report to shareholders or may prepare a separate, abbreviated, quarterly report to shareholders.

\end{itemize}

\section{EXAMPLE 1}
\section{Initial Registration Statement}
\begin{enumerate}
  \item In 2004, Google filed a registration statement with the US SEC to register its initial public offering of securities (Class A Common Stock). In addition to a large amount of financial and business information, the registration statement provided a 20-page discussion Google's business and industry.
\end{enumerate}

Which of the following is most likely to have been included in Google's registration statement?

A. Interim unaudited financial statements.

B. Assessment of risk factors involved in the business.

C. Projected cash flows and earnings for the business.

\section{Solution:}
B is correct. Information provided by companies in registration statements typically includes disclosures about the securities being offered for sale; the relationship of these new securities to the issuer's other capital securities; the information typically provided in the annual filings; recent audited financial statements; and risk factors involved in the business.. Companies provide information useful in developing projected cash flows and earnings but do not typically include these in the registration statement, nor do they provide unaudited interim statements in the initial registration statement.

A company or its officers make other SEC filings-either periodically, or, if significant events or transactions have occurred, in between the periodic reports noted above. By their nature, these forms sometimes contain the most interesting and timely information and may have significant valuation implications.

\begin{itemize}
  \item Form 8-K: In addition to filing annual and interim reports, SEC registrants must report material corporate events on a more current basis. Form 8-K (6-K for non-US registrants) is the "current report" companies must file with the SEC to announce such major events as acquisitions or disposals of corporate assets, changes in securities and trading markets, matters related to accountants and financial statements, corporate governance and management changes, and Regulation FD disclosures. ${ }^{9}$

  \item Forms 3, 4,5 and 144: Forms 3, 4 and 5 are required to report beneficial ownership of securities. These filings are required for any director or officer of a registered company as well as beneficial owners of greater than 10 percent of a class of registered equity securities. Form 3 is the initial statement, Form 4 reports changes, and Form 5 is the annual report. Form 144 is notice of the proposed sale of restricted securities or securities held by an affiliate of the issuer. These forms can be used to examine purchases and sales of securities by officers, directors, and other affiliates of the company, who collectively are regarded as corporate insiders.

  \item Form 11-K: This is the annual report of employee stock purchase, savings, and similar plans. It might be of interest to analysts for companies with significant employee benefit plans because it contains more information than that disclosed in the company's financial statements.

\end{itemize}

In jurisdictions other than the United States, similar legislation exists for the purpose of regulating securities and capital markets. Regulatory authorities are responsible for enforcing regulation, and securities regulation is intended to be consistent with the IOSCO objectives described in the previous section. Within each jurisdiction, regulators will either establish or, more typically, recognize and adopt a specified set or sets of accounting standards. The regulators will also establish reporting and filing requirements. IOSCO members have agreed to cooperate in the development, implementation, and enforcement of internationally recognised and consistent standards of regulation.

9 Regulation Fair Disclosure (FD) provides that when an issuer discloses material non-public information to certain individuals or entities-generally, securities market professionals such as stock analysts or holders of the issuer's securities who may trade on the basis of the information-the issuer must make public disclosure of that information. In this way, the rule aims to promote full and fair disclosure.

\section*{Capital Markets Regulation in Europe }
Each individual member state of the European Union (EU) regulates capital markets in its jurisdiction. There are, however, certain regulations that have been adopted at the EU level. Importantly the EU agreed that from 2005 consolidated accounts of EU listed companies would use International Financial Reporting Standards. The endorsement process by which newly issued IFRS are adopted by the EU reflects the balance between the individual member state's autonomy and the need for cooperation and convergence. When the IASB issues a new standard, the European Financial Reporting Advisory Group advises the European Commission on the standard, and the Standards Advice Review Group provides the Commission an opinion about that advice. Based on the input from these two entities, the Commission prepares a draft endorsement regulation. The Accounting Regulatory Committee votes on the proposal; and if the vote is favorable, the proposal proceeds to the European Parliament and the Council of the European Union for approval. ${ }^{10}$

Two bodies related to securities regulation established by the European Commission are the European Securities Committee (ESC) and the European Securities and Market Authority (ESMA). The ESC consists of high-level representatives of member states and advises the European Commission on securities policy issues. ESMA is an EU cross-border supervisor established to co-ordinate supervision of the EU market. As noted earlier, regulation still rests with the individual member states and, therefore, requirements for registering shares and filing periodic financial reports vary from country to country. ESMA is one of three European supervisory authorities; the two others supervise the banking and insurance industries.

\section{THE INTERNATIONAL FINANCIAL REPORTING STANDARDS FRAMEWORK}
\begin{abstract}
describe the objective of financial reporting and the importance of financial reporting standards in security analysis and valuation describe the International Accounting Standards Board's conceptual framework, including qualitative characteristics of financial reports, constraints on financial reports, and required reporting elements
\end{abstract}

As previously discussed, the IASB Conceptual Framework for Financial Reporting sets forth the concepts that underlie the preparation and presentation of financial statements for external users. The framework is designed to: assist standard setters in developing and reviewing standards; assist preparers of financial statements in applying standards and in dealing with issues not specifically covered by standard; assist auditors in forming an opinion on financial statements; and assist users in interpreting financial statement information. The objective of financial reporting is the provision of financial information that is useful to current and potential providers of resources in making decisions and all other aspects of the framework flow from that central objective.

The providers of resources are considered to be the primary users of financial reports and include investors, lenders, and other creditors. The purpose of providing the financial information is to be useful in making decisions about providing resources.

10 Source: European Commission. \href{http://ec.europa.eu/internal_market/accounting/legal_framework/}{http://ec.europa.eu/internal\_market/accounting/legal\_framework/} ias\_regulation\_en.htm. Other users may find the financial information useful for making economic decisions. The types of economic decisions differ by users, so the specific information needed differs as well. However, although these users may have unique information needs, some information needs are common across all users. Information is needed about the company's financial position: its resources and its financial obligations. Information is needed about a company's financial performance; this information explains how and why the company's financial position changed in the past and can be useful in evaluating potential changes in the future. The third common information need is the need for information about a company's cash. How did the company obtain cash (by selling its products and services, borrowing, other)? How did the company use cash (by paying expenses, investing in new equipment, paying dividends, other)?

Information that is helpful to users in assessing future net cash inflows to the entity includes information about the economic resources of (assets) and claims against (liabilities and equity) the entity, and about how well the management and governing board have utilized the resources of the entity. Users need to consider information from other sources as well in making their decisions. Although financial reports do not show the value of an entity they are useful in estimating the value of an entity.

\section{Qualitative Characteristics of Financial Reports}
Flowing from the central objective of providing information that is $u s e f u l$ to providers of resources, the Conceptual Framework identifies two fundamental qualitative characteristics that make financial information useful: relevance and faithful representation. 11 The concept of materiality is discussed within the context of relevance.

\begin{enumerate}
  \item Relevance: Information is relevant if it would potentially affect or make a difference in users' decisions. The information can have predictive value (useful in making forecasts), confirmatory value (useful to evaluate past decisions or forecasts), or both. In other words, relevant information helps users of financial information to evaluate past, present, and future events, or to confirm or correct their past evaluations in a decision-making context. Materiality: Information is considered to be material if omission or misstatement of the information could influence users' decisions. Materiality is a function of the nature and/or magnitude of the information.

  \item Faithful representation: Information that faithfully represents an economic phenomenon that it purports to represent is ideally complete, neutral, and free from error. Complete means that all information necessary to understand the phenomenon is depicted. Neutral means that information is selected and presented without bias. In other words, the information is not presented in such a manner as to bias the users' decisions. Free from error means that there are no errors of commission or omission in the description of the economic phenomenon, and that an appropriate process to arrive at the reported information was selected and was adhered to without error. Faithful representation maximizes the qualities of complete, neutral, and free from error to the extent possible.

\end{enumerate}

11 Conceptual Framework for Financial Reporting. Relevance and faithful representation are the fundamental, most critical characteristics of useful financial information. In addition the Conceptual Framework identifies four enhancing qualitative characteristics: comparability, verifiability, timeliness, and understandability.

\begin{enumerate}
  \item Comparability: Comparability allows users "to identify and understand similarities and differences of items." Information presented in a consistent manner over time and across entities enables users to make comparisons more easily than information with variations in how similar economic phenomena are represented.

  \item Verifiability: Verifiability means that different knowledgeable and independent observers would agree that the information presented faithfully represents the economic phenomena it purports to represent.

  \item Timeliness: Timely information is available to decision makers prior to their making a decision.

  \item Understandability: Clear and concise presentation of information enhances understandability. Information should be prepared for and be understandable by users who have a reasonable knowledge of business and economic activities, and who are willing to study the information with diligence. Information that is useful should not be excluded simply because it is difficult to understand and it may be necessary for users to seek assistance to understand information about complex economic phenomena.

\end{enumerate}

Financial information exhibiting these qualitative characteristics-fundamental and enhancing-should be useful for making economic decisions.

\section{Constraints on Financial Reports}
Although it would be ideal for financial statements to exhibit all of these qualitative characteristics and thus achieve maximum usefulness, it may be necessary to make tradeoffs across the enhancing characteristics. The application of the enhancing characteristics follows no set order of priority and each enhancing characteristic may take priority over the others. The aim is an appropriate balance among the enhancing characteristics.

A pervasive constraint on useful financial reporting is the cost of providing and using this information. Optimally, benefits derived from information should exceed the costs of providing and using it. Again, the aim is a balance between costs and benefits.

A limitation of financial reporting involves information that is not included. Financial statements, by necessity, omit information that is non-quantifiable. For example, the creativity, innovation, and competence of a company's work force are not directly captured in the financial statements. Similarly, customer loyalty, a positive corporate culture, environmental responsibility, and many other aspects about a company may not be directly reflected in the financial statements. Of course, to the extent that these items result in superior financial performance, a company's financial reports will reflect the results.

\section{EXAMPLE 2}
\section{Balancing Qualitative Characteristics of Useful Information}
\begin{enumerate}
  \item A tradeoff between enhancing qualitative characteristics often occurs. For example, when a company records sales revenue, it is required to simultaneously estimate and record an expense for potential bad debts (uncollectible accounts). Including this estimated expense is considered to represent the economic event faithfully and to provide relevant information about the net profits for the accounting period. The information is timely and understandable; but because bad debts may not be known with certainty until a later period, inclusion of this estimated expense involves a sacrifice of verifiability. The bad debt expense is simply an estimate. It is apparent that it is not always possible to simultaneously fulfill all qualitative characteristics.
\end{enumerate}

Companies are most likely to make tradeoffs between which of the following when preparing financial reports?

A. Relevance and materiality.

B. Timeliness and verifiability.

C. Relevance and faithful representation.

\section{Solution:}
B is correct. Providing timely information implies a shorter time frame between the economic event and the information preparation; however, fully verifying information may require a longer time frame. Relevance and faithful representation are fundamental qualitative characteristics that make financial information useful. Both characteristics are required; there is no tradeoff between these. Materiality is an aspect of relevance.

\section{THE ELEMENTS OF FINANCIAL STATEMENTS}
$$
\begin{aligned}
& \text { describe the International Accounting Standards Board's conceptual } \\
& \text { framework, including qualitative characteristics of financial reports, } \\
& \text { constraints on financial reports, and required reporting elements } \\
& \text { describe general requirements for financial statements under } \\
& \text { International Financial Reporting Standards (IFRS) }
\end{aligned}
$$

Financial statements portray the financial effects of transactions and other events by grouping them into broad classes (elements) according to their economic characteristics. Three elements of financial statements are directly related to the measurement of financial position: assets, liabilities, and equity.

\begin{itemize}
  \item Assets: A present economic resource controlled by the entity as a result of past events. An economic resource is a right that has the potential to produce economic benefits. Assets are what a company owns (e.g., inventory and equipment). - Liabilities:. A present obligation of the entity to transfer an economic resource as a result of past events. An obligation is a duty or responsibility that the entity has no practical ability to avoid. Liabilities are what a company owes (e.g., bank borrowings).

  \item Equity (for public companies, also known as "shareholders' equity" or "stockholders' equity"): Assets less liabilities. Equity is the residual interest in the assets after subtracting the liabilities.

\end{itemize}

The elements of financial statements directly related to the measurement of performance (profit and related measures) are income and expenses.

\begin{itemize}
  \item Income: Increases in assets, or decreases in liabilities, that result in increases in equity, other than those relating to contributions from holders of equity claims. Income includes both revenues and gains. Revenues represent income from the ordinary activities of the enterprise (e.g., the sale of products or provision of services). Gains may result from ordinary activities or other activities (the sale of surplus equipment).

  \item Expenses: Decreases in assets, or increases in liabilities, that result in decreases in equity, other than those relating to distributions to holders of equity claims. Expenses include those related to operating activities such as cost of good sold and operating expenses including wages, rents and other items. Losses are also considered expenses and can result from the sale of assets at less than their carrying values, impairments of asset values and a variety of other items.

\end{itemize}

\section{Underlying Assumptions in Financial Statements}
Two important assumptions underlie financial statements: accrual accounting and going concern. These assumptions determine how financial statement elements are recognized and measured.

The use of "accrual accounting" assumes that financial statements should reflect transactions in the period when they actually occur, not necessarily when cash movements occur. For example, a company reports revenues when they are earned (when the performance obligations have been satisfied), regardless of whether the company received cash before or after delivering the product, at the time of delivery.

"Going concern" refers to the assumption that the company will continue in business for the foreseeable future. To illustrate, consider the value of a company's inventory if it is assumed that the inventory can be sold over a normal period of time versus the value of that same inventory if it is assumed that the inventory must all be sold in a day (or a week). Companies with the intent to liquidate or materially curtail operations would require different information for a fair presentation.

\section{Recognition of Financial Statement Elements}
Recognition means that an item is included in the balance sheet or income statement. Recognition occurs if the item meets the definition of an element and satisfies the criteria for recognition. Recognition is appropriate if it results in both relevant information about assets, liabilities, equity, income and expenses and a faithful representation of those items, because the aim is to provide information that is useful to investors, lenders and other creditors.

\section{Measurement of Financial Statement Elements}
Measurement is the process of determining the monetary amounts at which the elements of the financial statements are to be recognized and carried in the balance sheet and income statement. The following alternative bases of measurement are used to different degrees and in varying combinations to measure assets and liabilities:

\begin{itemize}
  \item Historical cost: Historical cost is simply the amount of cash or cash equivalents paid to purchase an asset, including any costs of acquisition and/or preparation. If the asset was not bought for cash, historical cost is the fair value of whatever was given in order to buy the asset. When referring to liabilities, the historical cost basis of measurement means the amount of proceeds received in exchange for the obligation.

  \item Amortised cost: Historical cost adjusted for amortisation, depreciation, or depletion and/or impairment.

  \item Current cost: In reference to assets, current cost is the amount of cash or cash equivalents that would have to be paid to buy the same or an equivalent asset today. In reference to liabilities, the current cost basis of measurement means the undiscounted amount of cash or cash equivalents that would be required to settle the obligation today.

  \item Realizable (settlement) value: In reference to assets, realizable value is the amount of cash or cash equivalents that could currently be obtained by selling the asset in an orderly disposal. For liabilities, the equivalent to realizable value is called "settlement value" - that is, settlement value is the undiscounted amount of cash or cash equivalents expected to be paid to satisfy the liabilities in the normal course of business.

  \item Present value (PV): For assets, present value is the present discounted value of the future net cash inflows that the asset is expected to generate in the normal course of business. For liabilities, present value is the present discounted value of the future net cash outflows that are expected to be required to settle the liabilities in the normal course of business.

  \item Fair value: is defined as an exit price, the price that would be received to sell an asset or paid to transfer a liability in an orderly transaction between market participants at the measurement date. This may involve either market measures or present value measures depending on the availability of information.

\end{itemize}

\section{GENERAL REQUIREMENTS FOR FINANCIAL STATEMENTS}
describe general requirements for financial statements under International Financial Reporting Standards (IFRS)

The Conceptual Framework provides a basis for establishing standards and the elements of financial statements. However, the framework does not address the general contents of the financial statements, which are addressed in International Accounting Standard (IAS) No. 1, Presentation of Financial Statements. IAS No. 1 specifies the required financial statements, general features of financial statements, and structure and content of financial statements. ${ }^{12}$ These general requirements are illustrated in Exhibit 1 and described in the subsections below.

\section{Exhibit 1: IASB General Requirements for Financial Statements}
\begin{center}
\includegraphics[max width=\textwidth]{2023_05_04_ccb578c5089fd41615beg-606}
\end{center}

In the following sections, we discuss the required financial statements, the general features underlying the preparation of financial statements, and the specified structure and content in greater detail.

\section{Required Financial Statements}
Under IAS No. 1, a complete set of financial statements includes: ${ }^{13}$

\begin{itemize}
  \item a statement of financial position (balance sheet);

  \item a statement of comprehensive income (a single statement of comprehensive income or two statements, an income statement and a statement of comprehensive income that begins with profit or loss from the income statement);

  \item a statement of changes in equity, separately showing changes in equity resulting from profit or loss, each item of other comprehensive income, and transactions with owners in their capacity as owners ${ }^{14}$

  \item a statement of cash flows; and

  \item notes comprising a summary of significant accounting policies and other explanatory notes that disclose information required by IFRS and not presented elsewhere and that provide information relevant to an understanding of the financial statements.

\end{itemize}

12 For US GAAP, financial statement presentation is covered in Sections 205 through 280 of the Accounting Standards Codification.

13 IAS No. 1, Presentation of Financial Statements, paragraph 10.

14 Examples of transactions with owners acting in their capacity as owners include sale of equity securities to investors, distributions of earnings to investors, and repurchases of equity securities from investors. Entities are encouraged to furnish other related financial and non-financial information in addition to that required. Financial statements need to present fairly the financial position, financial performance, and cash flows of an entity.

\section{General Features of Financial Statements}
A company that applies IFRS is required to state explicitly in the notes to its financial statements that it is in compliance with the standards. Such a statement is only made when a company is in compliance with all requirements of IFRS. In extremely rare circumstances, a company may deviate from a requirement of IFRS if management concludes that complying with IFRS would result in misleading financial statements. In this case, management must disclose details of the departure from IFRS.

IAS No. 1 specifies a number of general features underlying the preparation of financial statements. These features clearly reflect the Conceptual Framework.

\begin{itemize}
  \item Fair Presentation: The application of IFRS is presumed to result in financial statements that achieve a fair presentation. The IAS describes fair presentation as follows:
\end{itemize}

Fair presentation requires the faithful representation of the effects of transactions, other events and conditions in accordance with the definitions and recognition criteria for assets, liabilities, income and expenses set out in the Framework. ${ }^{15}$

\begin{itemize}
  \item Going Concern: Financial statements are prepared on a going concern basis unless management either intends to liquidate the entity or to cease trading, or has no realistic alternative but to do so. If not presented on a going concern basis, the fact and rationale should be disclosed.

  \item Accrual Basis: Financial statements (except for cash flow information) are to be prepared using the accrual basis of accounting.

  \item Materiality and Aggregation: Omissions or misstatements of items are material if they could, individually or collectively, influence the economic decisions that users make on the basis of the financial statements. Each material class of similar items is presented separately. Dissimilar items are presented separately unless they are immaterial.

  \item No Offsetting: Assets and liabilities, and income and expenses, are not offset unless required or permitted by an IFRS.

  \item Frequency of Reporting: Financial statements must be prepared at least annually.

  \item Comparative Information: Financial statements must include comparative information from the previous period. The comparative information of prior periods is disclosed for all amounts reported in the financial statements, unless an IFRS requires or permits otherwise.

  \item Consistency: The presentation and classification of items in the financial statements are usually retained from one period to the next.

\end{itemize}

\section{Structure and Content Requirements}
IAS No. 1 also specifies structure and content of financial statements. These requirements include the following:

\begin{itemize}
  \item Classified Statement of Financial Position (Balance Sheet): requires the balance sheet to distinguish between current and non-current assets, and between current and non-current liabilities unless a presentation based on liquidity provides more relevant and reliable information (e.g., in the case of a bank or similar financial institution).

  \item Minimum Information on the Face of the Financial Statements: specifies the minimum line item disclosures on the face of, or in the notes to, the financial statements. For example, companies are specifically required to disclose the amount of their plant, property, and equipment as a line item on the face of the balance sheet. The major line items included in financial statements are covered in other readings.

  \item Minimum Information in the Notes (or on the face of financial statements): specifies disclosures about information to be presented in the financial statements. This information must be provided in a systematic manner and cross-referenced from the face of the financial statements to the notes. The required information is summarized in Exhibit 2.

  \item Comparative Information: For all amounts reported in a financial statement, comparative information should be provided for the previous period unless another standard requires or permits otherwise. Such comparative information allows users to better understand reported amounts.

\end{itemize}

\section{Exhibit 2: Summary of IFRS Required Disclosures in the Notes to the Financial Statements}
Disclosure of Accounting Policies

\begin{itemize}
  \item Measurement bases used in preparing financial statements

  \item Significant accounting policies used

  \item Judgments made in applying accounting policies that have the most significant effect on the amounts recognized in the financial statements

\end{itemize}

Sources of Estimation Uncertainty

\begin{itemize}
  \item Key assumptions about the future and other key sources of estimation uncertainty that have a significant risk of causing material adjustment to the carrying amount of assets and liabilities within the next year Other Disclosures

  \item Information about capital and about certain financial instruments classified as equity

  \item Dividends not recognized as a distribution during the period, including dividends declared before the financial statements were issued and any cumulative preference dividends

  \item Description of the entity, including its domicile, legal form, country of incorporation, and registered office or business address

  \item Nature of operations and principal activities

  \item Name of parent and ultimate parent

\end{itemize}

\section*{COMPARISON OF IFRS WITH ALTERNATIVE REPORTING SYSTEMS }
\begin{abstract}
describe implications for financial analysis of alternative financial reporting systems and the importance of monitoring developments in financial reporting standards
\end{abstract}

The adoption of IFRS as the required financial reporting standard by the EU and other countries has advanced the goal of global convergence. Nevertheless, there are still significant differences in financial reporting in the global capital markets. Arguably, the most critical are the differences that exist between IFRS and US GAAP. Following the EU adoption of IFRS (in 2005), a significant number of the world's listed companies use one of these two reporting standards.

In general, the IASB and FASB work together to coordinate changes to accounting standards and reduce differences between the standards. A joint IASB/FASB project initiated in 2004 aimed to develop an improved, common conceptual framework. In late 2010, convergence of the conceptual frameworks was put on hold. In December 2012, the IASB reactivated the conceptual framework as an IASB project. As of year-end 2018, there are no new projects on the convergence agenda.

As more countries adopt IFRS, the need for analysts to consider other financial reporting systems will be reduced. Analysts are likely to encounter financial statements that are prepared on a basis other than IFRS. Differences between IFRS and US GAAP remain and affect the framework as well as numerous financial reporting standards. Curriculum readings on individual financial statements and specific topics provide a more detailed review of related differences in IFRS and US GAAP.

\section{Exhibit 3: Differences between IFRS and US GAAP}
Basis for

Comparison US GAAP IFRS

\begin{center}
\begin{tabular}{lll}
\hline
Developed by & $\begin{array}{l}\text { Financial Accounting Standard } \\ \text { Board (FASB). }\end{array}$ & $\begin{array}{l}\text { International Accounting } \\ \text { Standard Board (IASB). }\end{array}$ \\
\hline
Based on & Rules & Principles \\
\hline
Inventory valuation & FIFO, LIFO and Weighted & FIFO and Weighted Average \\
 & Average Method. & Method. \\
\hline
\end{tabular}
\end{center}

\begin{center}
\begin{tabular}{|c|c|c|}
\hline
$\begin{array}{l}\text { Basis for } \\ \text { Comparison }\end{array}$ & US GAAP & IFRS \\
\hline
Extraordinary items & $\begin{array}{l}\text { Shown below (at the bottom of } \\ \text { the income statement). }\end{array}$ & $\begin{array}{l}\text { Not segregated in the income } \\ \text { statement. }\end{array}$ \\
\hline
Development cost & Treated as an expense & $\begin{array}{l}\text { Capitalized, only if certain condi- } \\ \text { tions are satisfied. }\end{array}$ \\
\hline
Reversal of Inventory & Prohibited & $\begin{array}{l}\text { Permissible, if specified condi- } \\ \text { tions are met. }\end{array}$ \\
\hline
\end{tabular}
\end{center}

Source: \href{https://keydifferences.com/difference-between-gaap-and-ifrs.html}{https://keydifferences.com/difference-between-gaap-and-ifrs.html}

For analyzing financial statements created under different standards, reconciliation schedules and disclosures regarding the significant differences between the reporting bases-historically available in some jurisdictions-were particularly helpful. For example, the SEC historically required reconciliation for foreign private issuers that did not prepare financial statements in accordance with US GAAP. In 2007, however, the SEC eliminated the reconciliation requirement for companies that prepared their financial statements according to IFRS. Although the disclosures related to any such differences were sometimes dauntingly long, the numerical reconciliations of net income and shareholders' equity appeared in charts that were relatively easy to use. Because reconciliation disclosures are no longer generally available, an analyst comparing two companies that use different reporting standards must be aware of areas where accounting standards have not converged. In many cases, a user of financial statements prepared under different accounting standards does not have enough information to make specific adjustments required to achieve comparability. Instead, an analyst must maintain caution in interpreting comparative financial measures produced under different accounting standards and monitor significant developments in financial reporting standards, as this can have important implications for comparing the performance of companies and security valuation.

\section*{9 }
\section{MONITORING DEVELOPMENTS IN FINANCIAL REPORTING STANDARDS}
\begin{abstract}
describe the objective of financial reporting and the importance of financial reporting standards in security analysis and valuation describe implications for financial analysis of alternative financial reporting systems and the importance of monitoring developments in financial reporting standards
\end{abstract}

In studying financial reporting and financial statement analysis in general, the analyst must be aware that reporting standards are evolving. Analysts need to monitor ongoing developments in financial reporting and assess their implications for security analysis and valuation. The need to monitor developments in financial reporting standards does not mean that analysts should be accountants. An accountant monitors these developments from a preparer's perspective; an analyst needs to monitor from a user's perspective. More specifically, analysts need to know how these developments will affect financial reports. Analysts can remain aware of developments in financial reporting standards by monitoring new products or transactions, actions of standard setters and other groups representing users of financial statements (such as CFA Institute), and company disclosures regarding critical accounting policies and estimates.

\section{New Products or Types of Transactions}
New products and new types of transactions can have unusual or unique elements to them such that no explicit guidance in the financial reporting standards exists. New products or transactions typically arise from economic events, such as new businesses (e.g., fintech), or from a newly developed financial instrument or financial structure. Financial instruments, whether exchange traded or not, are typically designed to enhance a company's business or to mitigate inherent risks. However, at times, financial instruments or structured transactions have been developed primarily for purposes of financial report "window dressing."

Although companies might discuss new products and transactions in their financial reports, the analyst can also monitor business journals and the capital markets to identify such items. Additionally, when one company in an industry develops a new product or transaction, other companies in the industry often do the same. Once new products, financial instruments, or structured transactions are identified, it is helpful to gain an understanding of the business purpose. If necessary, an analyst can obtain further information from a company's management, which should be able to describe the economic purpose, the financial statement reporting, significant estimates, judgments applied in determining the reporting, and future cash flow implications for these items.

\section{Evolving Standards and the Role of CFA Institute}
The actions of standard setters and regulators are unlikely to be helpful in identifying new products and transactions, given the lag between new product development and regulatory action. Monitoring the actions of these authorities is nonetheless important for another reason: Changes in regulations can affect companies' financial reports and, thus, valuations. This is particularly true if the financial reporting standards change to require more explicit identification of matters affecting asset/liability valuation or financial performance. For example, one regulatory change required companies to include a provision for expenses associated with the grant and vesting of employee stock option grants as an expense in the income statement. Prior to the required expensing, an analyst could only assess the dilutive effect to shareholders associated with stock option grants by reviewing information disclosed in the notes to the financial statements.

To the extent that some market participants do not examine financial statement details and thus ignore some items when valuing a company's securities, more explicit identification could affect the value of the company's securities. Additionally, it is plausible to believe that management is more attentive to and rigorous in any calculations/estimates of items that appear in the financial statements, compared to items that are only disclosed in the notes.

The IASB (\href{http://www.iasb.org}{www.iasb.org}) and FASB (\href{http://www.fasb.org}{www.fasb.org}) provide a great deal of information on their websites regarding new standards and proposals for future changes in standards. In addition, the IASB and FASB seek input from the financial analyst community-those who regularly use financial statements in making investment and credit decisions. When a new standard is proposed, an exposure draft is made available and users of financial statements can draft comment letters and position papers for submission to the IASB and FASB in order to evaluate the proposal. CFA Institute is active in supporting improvements to financial reporting. Volunteer members of CFA Institute serve on several liaison committees that meet regularly to make recommendations to the IASB and FASB on proposed standards and to draft comment letters and position papers. The comment letters and position papers of these groups on financial reporting issues are available at \href{http://www.cfainstitute.org/advocacy}{www.cfainstitute.org/advocacy}.

In 2007, CFA Institute issued a position paper titled A Comprehensive Business Reporting Model: Financial Reporting for Investors, which provides a suggested model for significantly improving financial reporting. The position paper remains relevant in stating:

Corporate financial statements and their related disclosures are fundamental to sound investment decision making. The well-being of the world's financial markets, and of the millions of investors who entrust their financial present and future to those markets, depends directly on the information financial statements and disclosures provide. Consequently, the quality of the information drives global financial markets. The quality, in turn, depends directly on the principles and standards managers apply when recognizing and measuring the economic activities and events affecting their companies' operations....

Investors require timeliness, transparency, comparability, and consistency in financial reporting. Investors have a preference for decision relevance over reliability..."analysts need to know economic reality-what is really going on-to the greatest extent it can be depicted by accounting numbers." Corporate financial statements that fail to reflect this economic reality undermine the investment decision-making process. ${ }^{16}$

Among other principles, the proposed model stresses the importance of information regarding the current fair value of assets and liabilities, of neutrality in financial reporting, and of providing detailed information on cash flows to investors through the choice of the so-called direct format for the cash flow statement.

In summary, analysts can improve their investment decision making by keeping current on financial reporting standards. In addition, analysts can contribute to improving financial reporting by sharing their perspective as users with standard-setting bodies, which typically invite comments concerning proposed changes.

\section{SUMMARY}
An awareness of financial reporting and underlying financial reporting standards can assist in security valuation and other financial analysis. This reading describes the conceptual objectives of financial reporting standards, the parties involved in standard-setting processes, and the implication for analysts in monitoring developments in reporting standards.

Some key points of the reading are summarized below:

\begin{itemize}
  \item The objective of financial reporting is to provide financial information about the reporting entity that is useful to existing and potential investors, lenders, and other creditors in making decisions about providing resources to the entity.
\end{itemize}

16 A Comprehensive Business Reporting Model: Financial Reporting for Investors, CFA Institute Centre for Financial Market Integrity, July 2007, p. 1, 2. - Financial reporting requires policy choices and estimates. These choices and estimates require judgment, which can vary from one preparer to the next. Accordingly, standards are needed to ensure increased consistency in these judgments.

\begin{itemize}
  \item Private sector standard setting bodies and regulatory authorities play significant but different roles in the standard setting process. In general, standard setting bodies make the rules, and regulatory authorities enforce the rules. However, regulators typically retain legal authority to establish financial reporting standards in their jurisdiction.

  \item The IFRS framework sets forth the concepts that underlie the preparation and presentation of financial statements for external users.

  \item The objective of fair presentation of useful information is the center of the IASB's Conceptual Framework. The qualitative characteristics of useful information include fundamental and enhancing characteristics. Information must exhibit the fundamental characteristics of relevance and faithful representation to be useful. The enhancing characteristics identified are comparability, verifiability, timeliness, and understandability.

  \item IFRS Financial Statements: IAS No. 1 prescribes that a complete set of financial statements includes a statement of financial position (balance sheet), a statement of comprehensive income (either two statements-one for net income and one for comprehensive income-or a single statement combining both net income and comprehensive income), a statement of changes in equity, a cash flow statement, and notes. The notes include a summary of significant accounting policies and other explanatory information.

  \item Financial statements need to reflect certain basic features: fair presentation, going concern, accrual basis, materiality and aggregation, and no offsetting.

  \item Financial statements must be prepared at least annually, must include comparative information from the previous period, and must be consistent.

  \item Financial statements must follow certain presentation requirements including a classified statement of financial position (balance sheet) and minimum information on both the face of the financial statements and in the notes.

  \item A significant number of the world's listed companies report under either IFRS or US GAAP.

  \item In many cases, a user of financial statements will lack the information necessary to make specific adjustments required to achieve comparability between companies that use IFRS and companies that use US GAAP. Instead, an analyst must maintain general caution in interpreting comparative financial measures produced under different accounting standards and monitor significant developments in financial reporting standards.

  \item Analysts can remain aware of ongoing developments in financial reporting by monitoring new products or types of transactions; actions of standard setters, regulators, and other groups; and company disclosures regarding critical accounting policies and estimates.

\end{itemize}

\section{PRACTICE PROBLEMS}
\begin{enumerate}
  \item Which of the following is most likely not an objective of financial statements?
\end{enumerate}

A. To provide information about the performance of an entity.

B. To provide information about the financial position of an entity.

C. To provide information about the users of an entity's financial statements.

\begin{enumerate}
  \setcounter{enumi}{1}
  \item International financial reporting standards are currently developed by which entity?
\end{enumerate}

A. The IFRS Foundation.

B. The International Accounting Standards Board.

C. The International Organization of Securities Commissions.

\begin{enumerate}
  \setcounter{enumi}{2}
  \item US generally accepted accounting principles are currently developed by which entity?
\end{enumerate}

A. The Securities and Exchange Commission.

B. The Financial Accounting Standards Board.

C. The Public Company Accounting Oversight Board.

\begin{enumerate}
  \setcounter{enumi}{3}
  \item A core objective of the International Organization of Securities Commissions is to:
\end{enumerate}

A. eliminate systemic risk.

B. protect users of financial statements.

C. ensure that markets are fair, efficient, and transparent.

\begin{enumerate}
  \setcounter{enumi}{4}
  \item According to the Conceptual Framework for Financial Reporting, which of the following is not an enhancing qualitative characteristic of information in financial statements?
\end{enumerate}

A. Accuracy.

B. Timeliness.

C. Comparability.

\begin{enumerate}
  \setcounter{enumi}{5}
  \item Which of the following is not a constraint on the financial statements according to the Conceptual Framework?
\end{enumerate}

A. Understandability.

B. Benefit versus cost.

C. Balancing of qualitative characteristics.

\begin{enumerate}
  \setcounter{enumi}{6}
  \item The assumption that an entity will continue to operate for the foreseeable future is called:
\end{enumerate}

A. accrual basis. B. comparability.

C. going concern.

\begin{enumerate}
  \setcounter{enumi}{7}
  \item The assumption that the effects of transactions and other events are recognized when they occur, not when the cash flows occur, is called:
A. relevance.
B. accrual basis.
C. going concern.

  \item Neutrality of information in the financial statements most closely contributes to which qualitative characteristic?
A. Relevance.
B. Understandability.
C. Faithful representation.

  \item Valuing assets at the amount of cash or equivalents paid or the fair value of the consideration given to acquire them at the time of acquisition most closely describes which measurement of financial statement elements?
A. Current cost.
B. Historical cost.
C. Realizable value.

  \item The valuation technique under which assets are recorded at the amount that would be received in an orderly disposal is:
A. current cost.
B. present value.
C. realizable value.

  \item Which of the following elements of financial statements is most closely related to measurement of performance?
A. Assets.
B. Expenses.
C. Liabilities.

  \item Which of the following elements of financial statements is most closely related to measurement of financial position?
A. Equity.
B. Income.
C. Expenses.

  \item Which of the following disclosures regarding new accounting standards provides the most meaningful information to an analyst?

\end{enumerate}

A. The impact of adoption is discussed.

B. The standard will have no material impact.

C. Management is still evaluating the impact.

\begin{enumerate}
  \setcounter{enumi}{14}
  \item Which of the following is not a required financial statement according to IAS No. 1 ?
\end{enumerate}

A. Statement of financial position.

B. Statement of changes in income.

C. Statement of comprehensive income.

\section{SOLUTIONS}
\begin{enumerate}
  \item $\mathrm{C}$ is correct. Financial statements provide information, including information about the entity's financial position, performance, and changes in financial position, to users. They do not typically provide information about users.

  \item B is correct. The IASB is currently charged with developing International Financial Reporting Standards.

  \item B is correct. The FASB is responsible for the Accounting Standards Codification $^{\mathrm{mw}}$, the single source of nongovernmental authoritative US generally accepted accounting principles.

  \item C is correct. A core objective of IOSCO is to ensure that markets are fair, efficient, and transparent. The other core objectives are to reduce, not eliminate, systemic risk and to protect investors, not all users of financial statements.

  \item A is correct. Accuracy is not an enhancing qualitative characteristic. Faithful representation, not accuracy, is a fundamental qualitative characteristic.

  \item A is correct. Understandability is an enhancing qualitative characteristic of financial information-not a constraint.

  \item C is correct. The Conceptual Framework identifies two important underlying assumptions of financial statements: accrual basis and going concern. Going concern is the assumption that the entity will continue to operate for the foreseeable future. Enterprises with the intent to liquidate or materially curtail operations would require different information for a fair presentation.

  \item B is correct. Accrual basis reflects the effects of transactions and other events being recognized when they occur, not when the cash flows. These effects are recorded and reported in the financial statements of the periods to which they relate.

  \item C is correct. The fundamental qualitative characteristic of faithful representation is contributed to by completeness, neutrality, and freedom from error.

  \item B is correct. Historical cost is the consideration paid to acquire an asset.

  \item $\mathrm{C}$ is correct. The amount that would be received in an orderly disposal is realizable value.

  \item B is correct. The elements of financial statements related to the measure of performance are income and expenses.

  \item A is correct. The elements of financial statements related to the measurement of financial position are assets, liabilities, and equity.

  \item A is correct. A discussion of the impact would be the most meaningful, although B would also be useful.

  \item B is correct. There is no statement of changes in income. Under IAS No. 1, a complete set of financial statements includes a statement of financial position, a statement of comprehensive income, a statement of changes in equity, a statement of cash flows, and notes comprising a summary of significant accounting policies and other explanatory information.

\end{enumerate}

\end{document}