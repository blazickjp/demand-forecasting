\documentclass[10pt]{article}
\usepackage[utf8]{inputenc}
\usepackage[T1]{fontenc}
\usepackage{amsmath}
\usepackage{amsfonts}
\usepackage{amssymb}
\usepackage[version=4]{mhchem}
\usepackage{stmaryrd}
\usepackage{hyperref}
\hypersetup{colorlinks=true, linkcolor=blue, filecolor=magenta, urlcolor=cyan,}
\urlstyle{same}
\usepackage{graphicx}
\usepackage[export]{adjustbox}
\graphicspath{ {./images/} }
\usepackage{multirow}

\title{CORPORATE ISSUERS, EQUITY INVESTMENTS, FIXED INCOME }


\author{describe characteristics and risks of commercial mortgage-backed\\
securities}
\date{}


\DeclareUnicodeCharacter{00D7}{$\times$}

\begin{document}
\maketitle
CFA $^{\circledR}$ Program Curriculum 2023 • LEVEL 1 • VOLUME 4 @2022 by CFA Institute. All rights reserved. This copyright covers material written expressly for this volume by the editor/s as well as the compilation itself. It does not cover the individual selections herein that first appeared elsewhere. Permission to reprint these has been obtained by CFA Institute for this edition only. Further reproductions by any means, electronic or mechanical, including photocopying and recording, or by any information storage or retrieval systems, must be arranged with the individual copyright holders noted.

CFA $^{\circ}$, Chartered Financial Analyst ${ }^{\circ}$, AIMR-PPS $^{\circ}$, and GIPS ${ }^{\circ}$ are just a few of the trademarks owned by CFA Institute. To view a list of CFA Institute trademarks and the Guide for Use of CFA Institute Marks, please visit our website at \href{http://www.cfainstitute.org}{www.cfainstitute.org}.

This publication is designed to provide accurate and authoritative information in regard to the subject matter covered. It is sold with the understanding that the publisher is not engaged in rendering legal, accounting, or other professional service. If legal advice or other expert assistance is required, the services of a competent professional should be sought.

All trademarks, service marks, registered trademarks, and registered service marks are the property of their respective owners and are used herein for identification purposes only.

ISBN 978-1-950157-99-0 (paper)

ISBN 978-1-953337-26-9 (ebook)

2022

\section{CONTENTS}
How to Use the CFA Program Curriculum $\quad$ xiii

Errata $\quad$ xiii

Designing Your Personal Study Program $\quad$ xiii

CFA Institute Learning Ecosystem (LES) $\quad$ xiv

Feedback $\quad$ xiv

\section{Corporate Issuers}
\section{Learning Module 1}
Cost of Capital-Foundational Topics

3

Introduction 3

Cost of Capital 4

Taxes and the Cost of Capital $\quad 6$

$\begin{array}{ll}\text { Costs of the Various Sources of Capital } & 7\end{array}$

$\begin{array}{ll}\text { Cost of Debt } & 7\end{array}$

$\begin{array}{lr}\text { Cost of Preferred Stock } & 10\end{array}$

Cost of Common Equity $\quad 12$

Estimating Beta 16

Estimating Beta for Public Companies $\quad 16$

Estimating Beta for Thinly Traded and Nonpublic Companies $\quad 17$

Flotation Costs $\quad 20$

Methods in Use $\quad 23$

Summary $\quad 24$

$\begin{array}{lr}\text { References } & 26\end{array}$

$\begin{array}{ll}\text { Practice Problems } & 27\end{array}$

$\begin{array}{lr}\text { Solutions } & 34\end{array}$

Learning Module $2 \quad$ Capital Structure 39

Introduction 39

$\begin{array}{lr}\text { Factors Affecting Capital Structure } & 40\end{array}$

Internal Factors Affecting Capital Structure $\quad 41$

Existing leverage 44

External Factors Affecting Capital Structure 49

Capital Structure and Company Life Cycle $\quad 51$

Background $\quad 51$

\begin{center}
\begin{tabular}{lr}
Start-Ups & 52 \\
\hline
GrowthBusinesses & 52 \\
\hline
\end{tabular}
\end{center}

Growth Businesses 53

Mature Businesses 54

Unique Situations 56

Modigliani-Miller Propositions $\quad 58$

MM Proposition I without Taxes: Capital Structure Irrelevance 59

MM Proposition II without Taxes: Higher Financial Leverage Raises

the Cost of Equity 59

MM Propositions with Taxes: Firm Value $\quad 62$

MM Propositions with Taxes: Cost of Capital $\quad 63$

Costs of Financial Distress $\quad 65$ Optimal and Target Capital Structures $\quad 66$

Market Value vs. Book Value $\quad 68$

$\begin{array}{lr}\text { Target Weights and WACC } & 69\end{array}$

$\begin{array}{ll}\text { Pecking Order Theory and Agency Costs } & 70\end{array}$

$\begin{array}{ll}\text { Stakeholder Interests } & 72\end{array}$

Debt vs. Equity Conflict $\quad 73$

$\begin{array}{ll}\text { Preferred Shareholders } & 78\end{array}$

$\begin{array}{ll}\text { Management and Directors } & 78\end{array}$

$\begin{array}{lr}\text { Summary } & 80\end{array}$

References 81

Practice Problems $\quad 82$

$\begin{array}{lr}\text { Solutions } & 87\end{array}$

\section{Learning Module 3}
Measures of Leverage 91

$\begin{array}{lr}\text { Introduction } & 91\end{array}$

$\begin{array}{lr}\text { Leverage } & 92\end{array}$

$\begin{array}{ll}\text { Business and Sales Risks } & 94\end{array}$

Business Risk and Its Components $\quad 94$

Sales Risk $\quad 95$

Operating Risk and the Degree of Operating Leverage 96

Financial Risk, the Degree of Financial Leverage and the Leveraging Role

of Debt

103

Total Leverage and the Degree of Total Leverage 107

Breakeven Points and Operating Breakeven Points 109

The Risks of Creditors and Owners $\quad 112$

Summary 113

Practice Problems $\quad 115$

$\begin{array}{ll}\text { Solutions } & 120\end{array}$

\section{Equity Investments}
Learning Module 1

Market Organization and Structure $\quad 125$

Introduction 126

The Functions of the Financial System $\quad 126$

Helping People Achieve Their Purposes in Using the Financial System 127

Determining Rates of Return $\quad 132$

Capital Allocation Efficiency 133

Assets and Contracts 134

Classifications of Assets and Markets $\quad 135$

Securities $\quad 137$

Fixed Income $\quad 137$

Equities 138

Pooled Investments 139

$\begin{array}{lr}\text { Currencies, Commodities, and Real Assets } & 140\end{array}$

$\begin{array}{lr}\text { Commodities } & 141\end{array}$

$\begin{array}{lr}\text { Real Assets } & 141\end{array}$

$\begin{array}{lr}\text { Contracts } & 144\end{array}$

$\begin{array}{lr}\text { Forward Contracts } & 145\end{array}$

$\begin{array}{ll}\text { Futures Contracts } & 146\end{array}$ Swap Contracts $\quad 147$

Option Contracts $\quad 148$

$\begin{array}{ll}\text { Other Contracts } & 149\end{array}$

Financial Intermediaries 149

$\begin{array}{lr}\text { Brokers, Exchanges, and Alternative Trading Systems } & 150\end{array}$

Dealers $\quad 151$

Arbitrageurs 152

Securitizers, Depository Institutions and Insurance Companies $\quad 154$

Depository Institutions and Other Financial Corporations $\quad 156$

Insurance Companies $\quad 157$

Settlement and Custodial Services and Summary 158

$\begin{array}{lr}\text { Summary } & 160\end{array}$

$\begin{array}{ll}\text { Positions and Short Positions } & 161\end{array}$

Short Positions $\quad 162$

Leveraged Positions 163

Orders and Execution Instructions 166

$\begin{array}{ll}\text { Execution Instructions } & 167\end{array}$

$\begin{array}{lr}\text { Validity Instructions and Clearing Instructions } & 171\end{array}$

Stop Orders $\quad 171$

\begin{center}
\begin{tabular}{lr}
Clearing Instructions & 173 \\
\hline
\end{tabular}
\end{center}

Primary Security Markets $\quad 173$

$\begin{array}{lr}\text { Public Offerings } & 173\end{array}$

Private Placements and Other Primary Market Transactions 175

Importance of Secondary Markets to Primary Markets $\quad 177$

Secondary Security Market and Contract Market Structures $\quad 177$

$\begin{array}{lr}\text { Trading Sessions } & 177\end{array}$

$\begin{array}{lr}\text { Execution Mechanisms } & 178\end{array}$

Market Information Systems 181

Well-functioning Financial Systems $\quad 182$

Market Regulation $\quad 184$

\begin{center}
\begin{tabular}{lr}
Summary & 187 \\
\hline
Practice Problems & 184 \\
\hline
\end{tabular}
\end{center}

Practice Problems 190

$\begin{array}{lr}\text { Solutions } & 198\end{array}$

Learning Module $2 \quad$ Security Market Indexes 203

Introduction 203

Index Definition and Calculations of Value and Returns 204

Calculation of Single-Period Returns 205

$\begin{array}{ll}\text { Calculation of Index Values over Multiple Time Periods } & 207\end{array}$

Index Construction 208

$\begin{array}{lr}\text { Target Market and Security Selection } & 209\end{array}$

Index Weighting $\quad 209$

Index Management: Rebalancing and Reconstitution $\quad 218$

$\begin{array}{lr}\text { Rebalancing } & 218\end{array}$

Reconstitution $\quad 218$

Uses of Market Indexes $\quad 220$

$\begin{array}{ll}\text { Gauges of Market Sentiment } & 220\end{array}$

Proxies for Measuring and Modeling Returns, Systematic Risk, and Risk-Adjusted Performance Proxies for Asset Classes in Asset Allocation Models

$\begin{array}{lr}\text { Style Indexes } & 224 \\ -224\end{array}$

Fixed-income indexes 225

Construction $\quad 225$

Types of Fixed-Income Indexes $\quad 225$

Indexes for Alternative Investments $\quad 228$

Commodity Indexes $\quad 228$

Real Estate Investment Trust Indexes $\quad 228$

$\begin{array}{ll}\text { Hedge Fund Indexes } & 2229\end{array}$

Summary $\quad 231$

Practice Problems $\quad 233$

$\begin{array}{ll}\text { Solutions } & 240\end{array}$

\section{Learning Module 3}
Market Efficiency 245

Introduction 245

$\begin{array}{lr}\text { The Concept of Market Efficiency } & 247\end{array}$

The Description of Efficient Markets $\quad 247$

Market Value versus Intrinsic Value 249

Factors Affecting Market Efficiency Including Trading Costs $\quad 250$

$\begin{array}{ll}\text { Market Participants } & 251\end{array}$

Information Availability and Financial Disclosure $\quad 252$

Limits to Trading 253

Transaction Costs and Information-Acquisition Costs $\quad 253$

Forms of Market Efficiency $\quad 255$

Weak Form $\quad 255$

Semi-Strong Form $\quad 256$

$\begin{array}{ll}\text { Strong Form } & 259\end{array}$

Implications of the Efficient Market Hypothesis 259

$\begin{array}{lr}\text { Fundamental Analysis } & 259\end{array}$

$\begin{array}{ll}\text { Technical Analysis } & 260\end{array}$

$\begin{array}{ll}\text { Portfolio Management } & 260\end{array}$

Market Pricing Anomalies - Time Series and Cross-Sectional $\quad 261$

Time-Series Anomalies $\quad 262$

Cross-Sectional Anomalies $\quad 264$

Other Anomalies, Implications of Market Pricing Anomalies $\quad 264$

Closed-End Investment Fund Discounts $\quad 265$

Earnings Surprise $\quad 265$

Initial Public Offerings (IPOs) 266

Predictability of Returns Based on Prior Information $\quad 267$

Implications for Investment Strategies $\quad 267$

Behavioral Finance $\quad 268$

Loss Aversion $\quad 268$

Herding $\quad 268$ Learning Module 4

Overconfidence $\quad 269$

$\begin{array}{lr}\text { Information Cascades } & 269\end{array}$

$\begin{array}{ll}\text { Other Behavioral Biases } & 270\end{array}$

$\begin{array}{ll}\text { Behavioral Finance and Investors } & 270\end{array}$

$\begin{array}{ll}\text { Behavioral Finance and Efficient Markets } & 270\end{array}$

$\begin{array}{ll}\text { Summary } & 271\end{array}$

$\begin{array}{lr}\text { References } & 272\end{array}$

$\begin{array}{lr}\text { Practice Problems } & 274\end{array}$

$\begin{array}{ll}\text { Solutions } & 278\end{array}$

Overview of Equity Securities $\quad 281$

Importance of Equity Securities $\quad 281$

Equity Securities in Global Financial Markets $\quad 282$

$\begin{array}{lr}\text { Characteristics of Equity Securities } & 287\end{array}$

Common Shares $\quad 288$

Preference Shares $\quad 290$

Private Versus Public Equity Securities $\quad 292$

Non-Domestic Equity Securities $\quad 295$

$\begin{array}{lr}\text { Direct Investing } & 296\end{array}$

Depository Receipts $\quad 297$

$\begin{array}{lr}\text { Risk and Return Characteristics } & 300\end{array}$

$\begin{array}{ll}\text { Return Characteristics of Equity Securities } & 300\end{array}$

$\begin{array}{ll}\text { Risk of Equity Securities } & 301\end{array}$

Equity and Company Value $\quad 302$

Accounting Return on Equity $\quad 303$

$\begin{array}{ll}\text { The Cost of Equity and Investors' Required Rates of Return } & 307\end{array}$

$\begin{array}{ll}\text { Summary } & 309\end{array}$

References 311

$\begin{array}{lr}\text { Practice Problems } & 312\end{array}$

Solutions 316

Learning Module $5 \quad$ Introduction to Industry and Company Analysis 319

$\begin{array}{ll}\text { Introduction } & 320\end{array}$

Uses of Industry Analysis $\quad 320$

$\begin{array}{ll}\text { Approaches to Identifying Similar Companies } & 321\end{array}$

Products and/or Services Supplied $\quad 321$

Business-Cycle Sensitivities $\quad 322$

Statistical Similarities 323

$\begin{array}{lr}\text { Industry Classification Systems } & 324\end{array}$

Commercial Industry Classification Systems 325

Constructing a Peer Group 329

Describing and Analyzing an Industry and Principles of Strategic Analysis 334

Principles of Strategic Analysis $\quad 335$

$\begin{array}{ll}\text { Barriers to Entry } & 336\end{array}$

Industry Concentration $\quad 338$

$\begin{array}{ll}\text { Industry Capacity } & 340\end{array}$

Market Share Stability 342

Price Competition 343

Industry Life Cycle 344 External Influences on Industry 348

Macroeconomic Influences $\quad 348$

$\begin{array}{lr}\text { Technological Influences } & 349\end{array}$

Demographic Influences $\quad 350$

Governmental Influences $\quad 350$

Social Influences $\quad 351$

Environmental Influences 352

Industry Comparison 355

Company Analysis $\quad 357$

Elements That Should Be Covered in a Company Analysis 358

Spreadsheet Modeling $\quad 360$

Summary $\quad 361$

References $\quad 365$

Practice Problems $\quad 366$

$\begin{array}{ll}\text { Solutions } & 371\end{array}$

Learning Module $6 \quad$ Equity Valuation: Concepts and Basic Tools 375

Introduction 376

Estimated Value and Market Price $\quad 376$

Categories of Equity Valuation Models $\quad 378$

$\begin{array}{ll}\text { Background for the Dividend Discount Model } & 380\end{array}$

Dividends: Background for the Dividend Discount Model $\quad 380$

Dividend Discount Model (DDM) and Free-Cash-Flow-to-Equity Model

(FCFE) $\quad 383$

Preferred Stock Valuation $\quad 386$

The Gordon Growth Model $\quad 389$

Multistage Dividend Discount Models 394

Multipler Models and Relationship Among Price Multiples, Present Value

Models, and Fundamentals 398

Relationships among Price Multiples, Present Value Models, and

Fundamentals 399

Method of Comparables and Valuation Based on Price Multiples 402

$\begin{array}{ll}\text { Illustration of a Valuation Based on Price Multiples } & 405\end{array}$

$\begin{array}{ll}\text { Enterprise Value } & 407\end{array}$

Asset-Based Valuation $\quad 410$

Summary 414

References 416

Practice Problems 417

Solutions 424

\section{Fixed Income}
Learning Module $1 \quad$ Fixed-Income Securities: Defining Elements 431

Introduction and Overview of a Fixed-Income Security 431

Overview of a Fixed-Income Security 432

Bond Indenture 438

Bond Indenture 439

Legal, Regulatory, and Tax Considerations 447

$\begin{array}{lr}\text { Tax Considerations } & 450\end{array}$ Principal Repayment Structures $\quad 452$

\begin{center}
\begin{tabular}{lr}
Principal Repayment Structures & 452 \\
\hline
\end{tabular}
\end{center}

Coupon Payment Structures $\quad 456$

$\begin{array}{lr}\text { Floating-Rate Notes } & 457\end{array}$

\begin{center}
\begin{tabular}{lr}
Step-Up Coupon Bonds & 457 \\
\hline
\end{tabular}
\end{center}

Credit-Linked Coupon Bonds 458

$\begin{array}{lr}\text { Payment-in-Kind Coupon Bonds } & 458\end{array}$

$\begin{array}{lr}\text { Deferred Coupon Bonds } & 459\end{array}$

Index-Linked Bonds $\quad 459$

$\begin{array}{ll}\text { Callable and Putable Bonds } & 463\end{array}$

$\begin{array}{lr}\text { Callable Bonds } & 463\end{array}$

Putable Bonds $\quad 465$

Convertible Bonds $\quad 466$

$\begin{array}{lr}\text { Summary } & 469\end{array}$

$\begin{array}{lr}\text { Practice Problems } & 473\end{array}$

$\begin{array}{ll}\text { Solutions } & 478\end{array}$

Learning Module $2 \quad$ Fixed-Income Markets: Issuance, Trading, and Funding 483

Introduction 483

Classification of Fixed-Income Markets $\quad 484$

$\begin{array}{lr}\text { Classification of Fixed-Income Markets } & 484\end{array}$

Fixed-Income Indexes 491

$\begin{array}{lr}\text { Investors in Fixed-Income Securities } & 492\end{array}$

Primary Bond Markets 493

Primary Bond Markets 494

Secondary Bond Markets 499

$\begin{array}{ll}\text { Sovereign Bonds } & 501\end{array}$

Characteristics of Sovereign Bonds $\quad 502$

$\begin{array}{lr}\text { Credit Quality of Sovereign Bonds } & 502\end{array}$

Types of Sovereign Bonds 503

Non-Sovereign, Quasi-Government, and Supranational Bonds 505

Non-Sovereign Bonds $\quad 505$

$\begin{array}{ll}\text { Quasi-Government Bonds } & 506\end{array}$

Supranational Bonds $\quad 506$

Corporate Debt: Bank Loans, Syndicated Loans, and Commercial Paper 507

Bank Loans and Syndicated Loans $\quad 508$

$\begin{array}{lr}\text { Commercial Paper } & 509\end{array}$

$\begin{array}{lr}\text { Corporate Debt: Notes and Bonds } & 511\end{array}$

$\begin{array}{ll}\text { Maturities } & 511\end{array}$

$\begin{array}{lr}\text { Coupon Payment Structures } & 512\end{array}$

$\begin{array}{lr}\text { Principal Repayment Structures } & 512\end{array}$

Asset or Collateral Backing $\quad 513$

Contingency Provisions 513

$\begin{array}{lr}\text { Issuance, Trading, and Settlement } & 514\end{array}$

$\begin{array}{lr}\text { Structured Financial Instruments } & 516\end{array}$

$\begin{array}{lr}\text { Capital Protected Instruments } & 516\end{array}$

$\begin{array}{lr}\text { Yield Enhancement Instruments } & 517\end{array}$

$\begin{array}{lr}\text { Participation Instruments } & 517\end{array}$

$\begin{array}{ll}\text { Leveraged Instruments } & 517\end{array}$ Short-Term Bank Funding Alternatives $\quad 519$

$\begin{array}{lr}\text { Retail Deposits } & 519\end{array}$

$\begin{array}{lr}\text { Short-Term Wholesale Funds } & 519\end{array}$

$\begin{array}{lr}\text { Repurchase and Reverse Repurchase Agreements } & 521\end{array}$

Structure of Repurchase and Reverse Repurchase Agreements 521

Credit Risk Associated with Repurchase Agreements 523

$\begin{array}{ll}\text { Summary } & 525\end{array}$

\begin{center}
\begin{tabular}{lr}
Practice Problems & 528 \\
\hline
$o l u t i o n s$ & 533 \\
\hline
\end{tabular}
\end{center}

Solutions 533

\section{Learning Module 3}
Introduction to Fixed-Income Valuation 537

Introduction 537

Bond Prices and the Time Value of Money 538

$\begin{array}{lr}\text { Bond Pricing with a Market Discount Rate } & 538\end{array}$

Yield-to-Maturity $\quad 542$

Relationships between the Bond Price and Bond Characteristics 543

$\begin{array}{ll}\text { Pricing Bonds Using Spot Rates } & 547\end{array}$

Prices and Yields: Conventions For Quotes and Calculations $\quad 549$

Flat Price, Accrued Interest, and the Full Price $\quad 549$

$\begin{array}{ll}\text { Matrix Pricing } & 553\end{array}$

Annual Yields for Varying Compounding Periods in the Year 556

$\begin{array}{lr}\text { Yield Measures for Fixed-Rate Bonds } & 559\end{array}$

Yield Measures for Floating-Rate Notes $\quad 561$

Yield Measures for Money Market Instruments 565

$\begin{array}{ll}\text { The Maturity Structure of Interest Rates } & 569\end{array}$

$\begin{array}{ll}\text { Yield Spreads } & 577\end{array}$

Yield Spreads over Benchmark Rates $\quad 577$

Yield Spreads over the Benchmark Yield Curve $\quad 579$

$\begin{array}{ll}\text { Summary } & 582\end{array}$

$\begin{array}{lr}\text { Practice Problems } & 585\end{array}$

$\begin{array}{ll}\text { Solutions } & 595\end{array}$

\section{Learning Module 4}
\section{Introduction to Asset-Backed Securities}
Parties to a Securitization and Their Roles $\quad 613$

$\begin{array}{lr}\text { Structure of a Securitization } & 615\end{array}$

Key Role of the Special Purpose Entity $\quad 617$

$\begin{array}{lr}\text { Residential Mortgage Loans } & 620\end{array}$

$\begin{array}{lr}\text { Maturity } & 621\end{array}$

$\begin{array}{lr}\text { Interest Rate Determination } & 621\end{array}$

Amortization Schedule $\quad 622$

$\begin{array}{lr}\text { Prepayment Options and Prepayment Penalties } & 622\end{array}$

Rights of the Lender in a Foreclosure $\quad 623$

Mortgage Pass-Through Securities $\quad 625$

Mortgage Pass-Through Securities $\quad 626$

Collateralized Mortgage Obligations and Non-Agency RMBS 632 CMO Structures Including Planned Amortization Class and Support

Tranches

Non-Agency Residential Mortgage-Backed Securities $\quad 639$

$\begin{array}{ll}\text { Commercial Mortgage-Backed Securities } & 640\end{array}$

$\begin{array}{lr}\text { Credit Risk } & 640\end{array}$

$\begin{array}{lr}\text { CMBS Structure } & 640\end{array}$

Non-Mortgage Asset-Backed Securities $\quad 644$

Auto Loan ABS $\quad 645$

Credit Card Receivable ABS $\quad 647$

$\begin{array}{lr}\text { Collateralized Debt Obligations } & 649\end{array}$

$\begin{array}{lr}\text { CDO Structure } & 649\end{array}$

$\begin{array}{lr}\text { An Example of a CDO Transaction } & 650\end{array}$

$\begin{array}{lr}\text { Covered Bonds } & 652\end{array}$

Summary $\quad 653$

Practice Problems $\quad 657$

$\begin{array}{ll}\text { Solutions } & 664\end{array}$

\section{How to Use the CFA Program Curriculum}
The CFA Program exams measure your mastery of the core knowledge, skills, and abilities required to succeed as an investment professional. These core competencies are the basis for the Candidate Body of Knowledge $\left(\mathrm{CBOK}^{\mathrm{m}}\right)$ ). The CBOK consists of four components:

\begin{itemize}
  \item A broad outline that lists the major CFA Program topic areas (www. \href{http://cfainstitute.org/programs/cfa/curriculum/cbok}{cfainstitute.org/programs/cfa/curriculum/cbok})

  \item Topic area weights that indicate the relative exam weightings of the top-level topic areas (\href{http://www.cfainstitute.org/programs/cfa/curriculum}{www.cfainstitute.org/programs/cfa/curriculum})

  \item Learning outcome statements (LOS) that advise candidates about the specific knowledge, skills, and abilities they should acquire from curriculum content covering a topic area: LOS are provided in candidate study sessions and at the beginning of each block of related content and the specific lesson that covers them. We encourage you to review the information about the LOS on our website (\href{http://www.cfainstitute.org/programs/cfa/curriculum/}{www.cfainstitute.org/programs/cfa/curriculum/} study-sessions), including the descriptions of LOS "command words" on the candidate resources page at \href{http://www.cfainstitute.org}{www.cfainstitute.org}.

  \item The CFA Program curriculum that candidates receive upon exam registration

\end{itemize}

Therefore, the key to your success on the CFA exams is studying and understanding the CBOK. You can learn more about the CBOK on our website: www.cfainstitute. org/programs/cfa/curriculum/cbok.

The entire curriculum, including the practice questions, is the basis for all exam questions and is selected or developed specifically to teach the knowledge, skills, and abilities reflected in the $\mathrm{CBOK}$.

\section{ERRATA}
The curriculum development process is rigorous and includes multiple rounds of reviews by content experts. Despite our efforts to produce a curriculum that is free of errors, there are instances where we must make corrections. Curriculum errata are periodically updated and posted by exam level and test date online on the Curriculum Errata webpage (\href{http://www.cfainstitute.org/en/programs/submit-errata}{www.cfainstitute.org/en/programs/submit-errata}). If you believe you have found an error in the curriculum, you can submit your concerns through our curriculum errata reporting process found at the bottom of the Curriculum Errata webpage.

\section{DESIGNING YOUR PERSONAL STUDY PROGRAM}
An orderly, systematic approach to exam preparation is critical. You should dedicate a consistent block of time every week to reading and studying. Review the LOS both before and after you study curriculum content to ensure that you have mastered the applicable content and can demonstrate the knowledge, skills, and abilities described by the LOS and the assigned reading. Use the LOS self-check to track your progress and highlight areas of weakness for later review.

Successful candidates report an average of more than 300 hours preparing for each exam. Your preparation time will vary based on your prior education and experience, and you will likely spend more time on some study sessions than on others.

\section{CFA INSTITUTE LEARNING ECOSYSTEM (LES)}
Your exam registration fee includes access to the CFA Program Learning Ecosystem (LES). This digital learning platform provides access, even offline, to all of the curriculum content and practice questions and is organized as a series of short online lessons with associated practice questions. This tool is your one-stop location for all study materials, including practice questions and mock exams, and the primary method by which CFA Institute delivers your curriculum experience. The LES offers candidates additional practice questions to test their knowledge, and some questions in the LES provide a unique interactive experience.

\section{FEEDBACK}
Please send any comments or feedback to \href{mailto:info@cfainstitute.org}{info@cfainstitute.org}, and we will review your suggestions carefully.

\section{Corporate Issuers}
\section*{LEARNING MODULE 
 1 }
\section{Cost of Capital-Foundational Topics}
Yves Courtois, CMT, MRICS, CFA, is at KPMG (Luxembourg). Gene C. Lai, PhD, is at the University of North Carolina at Charlotte (USA). Pamela Peterson Drake, PhD, CFA, is at James Madison University (USA).

\section{LEARNING OUTCOME}
\begin{center}
\begin{tabular}{|c|c|}
\hline
Mastery & The candidate should be able to: \\
\hline
. & $\begin{array}{l}\text { calculate and interpret the weighted average cost of capital (WACC) } \\ \text { of a company }\end{array}$ \\
\hline
 & $\begin{array}{l}\text { describe how taxes affect the cost of capital from different capital } \\ \text { sources }\end{array}$ \\
\hline
 & $\begin{array}{l}\text { calculate and interpret the cost of debt capital using the } \\ \text { yield-to-maturity approach and the debt-rating approach }\end{array}$ \\
\hline
 & $\begin{array}{l}\text { calculate and interpret the cost of noncallable, nonconvertible } \\ \text { preferred stock }\end{array}$ \\
\hline
 & $\begin{array}{l}\text { calculate and interpret the cost of equity capital using the capital } \\ \text { asset pricing model approach and the bond yield plus risk premium } \\ \text { approach }\end{array}$ \\
\hline
 & $\begin{array}{l}\text { explain and demonstrate beta estimation for public companies, } \\ \text { thinly traded public companies, and nonpublic companies }\end{array}$ \\
\hline
 & explain and demonstrate the correct treatment of flotation costs \\
\hline
\end{tabular}
\end{center}

\section{INTRODUCTION}
A company grows by making investments that are expected to increase revenues and profits. It acquires the capital or funds necessary to make such investments by borrowing (i.e., using debt financing) or by using funds from the owners (i.e., equity financing). By applying this capital to investments with long-term benefits, the company is producing value today. How much value? The answer depends not only on the investments' expected future cash flows but also on the cost of the funds. Borrowing is not costless, nor is using owners' funds. The cost of this capital is an important ingredient in both investment decision making by the company's management and the valuation of the company by investors. If a company invests in projects that produce a return in excess of the cost of capital, the company has created value; in contrast, if the company invests in projects whose returns are less than the cost of capital, the company has destroyed value. Therefore, the estimation of the cost of capital is a central issue in corporate financial management and for an analyst seeking to evaluate a company's investment program and its competitive position.

Cost of capital estimation is a challenging task. As we have already implied, the cost of capital is not observable but, rather, must be estimated. Arriving at a cost of capital estimate requires a multitude of assumptions and estimates. Another challenge is that the cost of capital that is appropriately applied to a specific investment depends on the characteristics of that investment: The riskier the investment's cash flows, the greater its cost of capital. In reality, a company must estimate project-specific costs of capital. What is often done, however, is to estimate the cost of capital for the company as a whole and then adjust this overall corporate cost of capital upward or downward to reflect the risk of the contemplated project relative to the company's average project.

This reading is organized as follows: In Section 2, we introduce the cost of capital and its basic computation. Section 3 presents a selection of methods for estimating the costs of the various sources of capital: debt, preferred stock, and common equity. For the latter, two approaches for estimating the equity risk premium are mentioned. Section 4 discusses beta estimation, a key input in using the CAPM to calculate the cost of equity, and Section 5 examines the correct treatment of flotation, or capital issuance, costs. Section 6 highlights methods used by corporations, and a summary concludes the reading.

\section{COST OF CAPITAL}
calculate and interpret the weighted average cost of capital (WACC) of a company

describe how taxes affect the cost of capital from different capital sources

The cost of capital is the rate of return that the suppliers of capital-lenders and owners-require as compensation for their contribution of capital. Another way of looking at the cost of capital is that it is the opportunity cost of funds for the suppliers of capital: A potential supplier of capital will not voluntarily invest in a company unless its return meets or exceeds what the supplier could earn elsewhere in an investment of comparable risk. In other words, to raise new capital, the issuer must price the security to offer a level of expected return that is competitive with the expected returns being offered by similarly risky securities.

A company typically has several alternatives for raising capital, including issuing equity, debt, and hybrid instruments that share characteristics of both debt and equity, such as preferred stock and convertible debt. Each source selected becomes a component of the company's funding and has a cost (required rate of return) that may be called a component cost of capital. Because we are using the cost of capital in the evaluation of investment opportunities, we are dealing with a marginal cost-what it would cost to raise additional funds for the potential investment project. Therefore, the cost of capital that the investment analyst is concerned with is a marginal cost, and the required return on a security is the issuer's marginal cost for raising additional capital of the same type.

The cost of capital of a company is the required rate of return that investors demand for the average-risk investment of a company. A company with higher-than-average-risk investments must pay investors a higher rate of return, competitive with other securities of similar risk, which corresponds to a higher cost of capital. Similarly, a company with lower-than-average-risk investments will have lower rates of return demanded by investors, resulting in a lower associated cost of capital. The most common way to estimate this required rate of return is to calculate the marginal cost of each of the various sources of capital and then calculate a weighted average of these costs. You will notice that the debt and equity costs of capital and the tax rate are all understood to be "marginal" rates: the cost or tax rate for additional capital.

The weighted average is referred to as the weighted average cost of capital (WACC). The WACC is also referred to as the marginal cost of capital (MCC) because it is the cost that a company incurs for additional capital. Further, this is the current cost: what it would cost the company today.

The weights are the proportions of the various sources of capital that the company uses to support its investment program. It is important to note that the weights should represent the company's target capital structure, not the current capital structure. A company's target capital structure is its chosen (or targeted) proportions of debt and equity, whereas its current capital structure is the company's actual weighting of debt and equity. For example, suppose the current capital structure is one-third debt, one-third preferred stock, and one-third common stock. Now suppose the new investment will be financed by issuing more debt so that capital structure changes to one-half debt, one-fourth preferred stock, and one-fourth common stock. Those new weights (i.e., the target weights) should be used to calculate the WACC.

Taking the sources of capital to be common stock, preferred stock, and debt and allowing for the fact that in some jurisdictions, interest expense may be tax deductible, the expression for WACC is

$\mathrm{WACC}=w_{d} r_{d}(1-t)+w_{p} r_{p}+w_{e} r_{e}$

where

$w_{d}=$ the target proportion of debt in the capital structure when the company raises new funds

$r_{d}=$ the before-tax marginal cost of debt

$t=$ the company's marginal tax rate

$w_{p}=$ the target proportion of preferred stock in the capital structure when the company raises new funds

$r_{p}=$ the marginal cost of preferred stock

$w_{e}=$ the target proportion of common stock in the capital structure when the company raises new funds

$$
r_{e}=\text { the marginal cost of common stock }
$$

Note that preferred stock is also referred to as preferred equity, and common stock is also referred to as common equity, or equity.

\section{EXAMPLE 1}
\section{Computing the Weighted Average Cost of Capital}
\begin{enumerate}
  \item Assume that $\mathrm{ABC}$ Corporation has the following capital structure: $30 \%$ debt, $10 \%$ preferred stock, and $60 \%$ common stock, or equity. Also assume that interest expense is tax deductible. ABC Corporation wishes to maintain these proportions as it raises new funds. Its before-tax cost of debt is $8 \%$, its cost of preferred stock is $10 \%$, and its cost of equity is $15 \%$. If the company's marginal tax rate is $40 \%$, what is $A B C$ 's weighted average cost of capital?
\end{enumerate}

\section{Solution:}
The weighted average cost of capital is

$$
\begin{aligned}
& \text { WACC }=(0.3)(0.08)(1-0.40)+(0.1)(0.1)+(0.6)(0.15) \\
& =11.44 \% .
\end{aligned}
$$

The cost for $\mathrm{ABC}$ Corporation to raise new funds while keeping its current capital structure is $11.44 \%$.

\section{Self reflection:}
Maintaining its current capital structure, what happens to ABC's weighted average cost of capital if component costs increase or decrease? What happens if the company's marginal tax rate increases or decreases?

There are important points concerning the calculation of the WACC as shown in Equation 1 that the analyst must be familiar with. The next section addresses the key issue of taxes.

\section{Taxes and the Cost of Capital}
The marginal cost of debt financing is the cost of debt after considering the allowable deduction for interest on debt based on the country's tax law. If interest cannot be deducted for tax purposes, the tax rate applied is zero, so the effective marginal cost of debt is equal to $r_{d}$ in Equation 1. If interest can be deducted in full, the tax deductibility of debt reduces the effective marginal cost of debt to reflect the income shielded from taxation (often referred to as the tax shield) and the marginal cost of debt is $r_{d}(1-t)$. For example, suppose a company pays $€ 1$ million in interest on its $€ 10$ million of debt. The cost of this debt is not $€ 1$ million, because this interest expense reduces taxable income by $€ 1$ million, resulting in a lower tax. If the company has a marginal tax rate of $40 \%$, this $€ 1$ million of interest costs the company ( $€ 1$ million) $(1-0.4)=€ 0.6$ million because the interest reduces the company's tax bill by $€ 0.4$ million. In this case, the before-tax cost of debt is $10 \%$, whereas the after-tax cost of debt is $(€ 0.6$ million $) /(€ 10$ million $)=6 \%$, which can also be calculated as $10 \%(1-0.4)$.

In jurisdictions in which a tax deduction for a business's interest expense is allowed, there may be reasons why additional interest expense is not tax deductible (e.g., not having sufficient income to offset with interest expense). If the above company with $€ 10$ million in debt were in that position, its effective marginal cost of debt would be $10 \%$ rather than $6 \%$ because any additional interest expense would not be deductible for tax purposes. In other words, if the limit on tax deductibility is reached, the marginal cost of debt is the cost of debt without any adjustment for a tax shield.

\section{EXAMPLE 2}
\section{Incorporating the Effect of Taxes on the Costs of Capital}
Jorge Ricard, a financial analyst, is estimating the costs of capital for the Zeale Corporation. In the process of this estimation, Ricard has estimated the before-tax costs of capital for Zeale's debt and equity as $4 \%$ and $6 \%$, respectively. What are the after-tax costs of debt and equity if there is no limit to the tax deductibility of interest and Zeale's marginal tax rate is:

\begin{enumerate}
  \item $30 \%$ ?

  \item $48 \% ?$

\end{enumerate}

\begin{center}
\begin{tabular}{lcl}
\hline
 & Marginal Tax Rate & After-Tax Cost of Debt \\
\hline
Solution to 1: & $30 \%$ & $0.04(1-0.30)=2.80 \%$. \\
Solution to 2: & $48 \%$ & $0.04(1-0.48)=2.08 \%$. \\
\hline
\end{tabular}
\end{center}

Note: There is no adjustment for taxes in the case of equity; the before-tax cost of equity is equal to the after-tax cost of equity.

\section{COSTS OF THE VARIOUS SOURCES OF CAPITAL}
calculate and interpret the cost of debt capital using the yield-to-maturity approach and the debt-rating approach calculate and interpret the cost of noncallable, nonconvertible preferred stock

calculate and interpret the cost of equity capital using the capital asset pricing model approach and the bond yield plus risk premium approach

Each source of capital has a different cost because of the differences among the sources, such as risk, seniority, contractual commitments, and potential value as a tax shield. We focus on the costs of three primary sources of capital: debt, preferred stock, and common equity.

\section{Cost of Debt}
The cost of debt is the cost of debt financing to a company when it issues a bond or takes out a bank loan. That cost is equal to the risk-free rate plus a premium for risk. A company that is perceived to be very risky would have a higher cost of debt than one that presents little investment risk. Factors that might affect the level of investment risk include profitability, stability of profits, and the degree of financial leverage. In general, the cost of debt would be higher for companies that are unprofitable, whose profits are not stable, or that are already using a lot of debt in their capital structure. We discuss two methods to estimate the before-tax cost of debt, $r_{d}$ : the yield-to-maturity approach and debt-rating approach. Cost of Capital-Foundational Topics

\section{Yield-to-Maturity Approach}
The before-tax required return on debt is typically estimated using the expected yield to maturity (YTM) of the company's debt based on current market values. YTM is the annual return that an investor earns on a bond if the investor purchases the bond today and holds it until maturity. In other words, it is the yield, $r_{d}$, that equates the present value of the bond's promised payments to its market price:

$$
P_{0}=\frac{P M T_{1}}{\left(1+\frac{r_{d}}{2}\right)}+\ldots+\frac{P M T_{n}}{\left(1+\frac{r_{d}}{2}\right)^{n}}+\frac{F V}{\left(1+\frac{r_{d}}{2}\right)^{n}}=\left[\sum_{t=1}^{n} \frac{P M T_{t}}{\left(1+\frac{r_{d}}{2}\right)^{t}}\right]+\frac{F V}{\left(1+\frac{r_{d}}{2}\right)^{n}},
$$

where

$$
\begin{aligned}
P_{0} & =\text { the current market price of the bond } \\
P M T_{t} & =\text { the interest payment in period } t \\
r_{d} & =\text { the yield to maturity } \\
n & =\text { the number of periods remaining to maturity } \\
F V & =\text { the maturity value of the bond }
\end{aligned}
$$

In this valuation equation, the constant 2 reflects the assumption that the bond pays interest semi-annually (which is the case in many but not all countries) and that any intermediate cash flows (i.e., the interest payments prior to maturity) are reinvested at the rate $r_{d} / 2$ semi-annually.

Example 3 illustrates the calculation of the after-tax cost of debt.

\section{EXAMPLE 3}
\section{Calculating the After-Tax Cost of Debt}
\begin{enumerate}
  \item Valence Industries issues a bond to finance a new project. It offers a 10-year, $\$ 1,000$ face value, $5 \%$ semi-annual coupon bond. Upon issue, the bond sells at $\$ 1,025$. What is Valence's before-tax cost of debt? If Valence's marginal tax rate is $35 \%$, what is Valence's after-tax cost of debt?
\end{enumerate}

\section{Solution:}
The following are given:

$$
\begin{aligned}
& P V=\$ 1,025 . \\
& F V=\$ 1,000 . \\
& P M T=5 \% \text { of } 1,000 \div 2=\$ 25 . \\
& n=10 \times 2=20 . \\
& \$ 1,025=\left[\sum_{t=1}^{20} \frac{\$ 25}{(1+i)^{t}}\right]+\frac{\$ 1,000}{(1+i)^{20}} .
\end{aligned}
$$

Before proceeding to solve the problem, we already know that the before-tax cost of debt must be less than $5 \%$ because the present value of the bond is greater than the face value. We can use a financial calculator to solve for $i$, the six-month yield. Because $i=2.342 \%$, the before-tax cost of debt is $r_{d}$ $=2.342 \% \times 2=4.684 \%$, and Valence's after-tax cost of debt is $r_{d}(1-t)=$ $0.04684(1-0.35)=0.03045$, or $3.045 \%$.

\section{Debt-Rating Approach}
When a reliable current market price for a company's debt is not available, the debt-rating approach can be used to estimate the before-tax cost of debt. Based on a company's debt rating, we estimate the before-tax cost of debt by using the yield on comparably rated bonds for maturities that closely match that of the company's existing debt.

Suppose a company's capital structure includes debt with an average maturity of 10 years and the company's marginal tax rate is 35\%. If the company's rating is AAA and the yield on debt with the same debt rating and similar maturity is $4 \%$, the company's after-tax cost of debt is

$r_{d}(1-t)=0.04(1-0.35)=2.6 \%$

\section{EXAMPLE 4}
\section{Calculating the After-Tax Cost of Debt}
\begin{enumerate}
  \item Elttaz Company's capital structure includes debt with an average maturity of 15 years. The company's rating is $A 1$, and it has a marginal tax rate of $18 \%$. If the yield on comparably rated A 1 bonds with similar maturity is $6.1 \%$, what is Elttaz's after-tax cost of debt?
\end{enumerate}

\section{Solution:}
Elttaz's after-tax cost of debt is

$$
r_{d}(1-t)=0.061(1-0.18)=5.0 \%
$$

A consideration when using this approach is that debt ratings are ratings of the debt issue itself, with the issuer being only one of the considerations. Other factors, such as debt seniority and security, also affect ratings and yields, so care must be taken to consider the likely type of debt to be issued by the company in determining the comparable debt rating and yield. The debt-rating approach is a simple example of pricing on the basis of valuation-relevant characteristics, which in bond markets has been known as evaluated pricing or matrix pricing.

\section{Issues in Estimating the Cost of Debt}
There are other issues to consider when estimating the cost of debt. Among these are whether the debt is fixed rate or floating rate, whether it has option-like features, whether it is unrated, and whether the company uses leases instead of typical debt.

\section{Fixed-Rate Debt vs. Floating-Rate Debt}
Up to now, we have assumed that the interest on debt is a fixed amount each period. We can observe market yields of the company's existing debt or market yields of debt of similar risk in estimating the before-tax cost of debt. However, the company may also issue floating-rate debt, in which the interest rate adjusts periodically according to a prescribed index, such as the prime rate, over the life of the instrument.

Estimating the cost of a floating-rate security is difficult because the cost of this form of capital over the long term depends not only on the current yields but also on the future yields. The analyst may use the current term structure of interest rates and term structure theory to assign an average cost to such instruments.

\section{Debt with Optionlike Features}
How should an analyst determine the cost of debt when the company uses debt with option-like features, such as call, conversion, or put provisions? Clearly, options affect the value of debt. For example, a callable bond would have a yield greater than a similar noncallable bond of the same issuer because bondholders want to be compensated for the call risk associated with the bond. In a similar manner, the put feature of a bond, which provides the investor with an option to sell the bond back to the issuer at a predetermined price, has the effect of lowering the yield on a bond below that of a similar nonputable bond. Likewise, convertible bonds, which give investors the option of converting the bonds into common stock, lower the yield on the bonds below that of similar nonconvertible bonds.

If the company already has debt outstanding incorporating optionlike features that the analyst believes are representative of the future debt issuance of the company, the analyst may simply use the yield to maturity on such debt in estimating the cost of debt.

If the analyst believes that the company will add or remove option features in future debt issuance, the analyst can make market value adjustments to the current YTM to reflect the value of such additions or deletions. The technology for such adjustments is an advanced topic that is outside the scope of this coverage.

\section{Nonrated Debt}
If a company does not have any debt outstanding or if the yields on the company's existing debt are not available, the analyst may not always be able to use the yield on similarly rated debt securities. It may be the case that the company does not have rated bonds. Although researchers offer approaches for estimating a company's "synthetic" debt rating based on financial ratios, these methods are imprecise because debt ratings incorporate not only financial ratios but also information about the particular bond issue and the issuer that are not captured in financial ratios. A further discussion of these methods is outside the scope of this reading.

\section{Leases}
A lease is a contractual obligation that can substitute for other forms of borrowing. This is true whether the lease is an operating lease or a finance lease (also called a capital lease). If the company uses leasing as a source of capital, the cost of these leases should be included in the cost of capital. The cost of this form of borrowing is similar to that of the company's other long-term borrowing.

\section{Cost of Preferred Stock}
The cost of preferred stock is the cost that a company has committed to pay preferred stockholders as a preferred dividend when it issues preferred stock. In the case of nonconvertible, noncallable preferred stock that has a fixed dividend rate and no maturity date (fixed-rate perpetual preferred stock), we can use the formula for the value of a preferred stock:

$$
P_{p}=\frac{D_{p}}{r_{p}}
$$

where

$$
\begin{aligned}
& P_{p}=\text { the current preferred stock price per share } \\
& D_{p}=\text { the preferred stock dividend per share } \\
& r_{p}=\text { the cost of preferred stock }
\end{aligned}
$$

We can rearrange this equation to solve for the cost of preferred stock:

$$
r_{p}=\frac{D_{p}}{P_{p}} .
$$

Therefore, the cost of preferred stock is the preferred stock's dividend per share divided by the current preferred stock's price per share. Unlike interest on debt, the dividend on preferred stock is not tax-deductible by the company; therefore, there is no adjustment to the cost for taxes.

A preferred stock may have a number of features that affect its yield and hence its cost. These features include a call option, cumulative dividends, participating dividends, adjustable-rate dividends, and convertibility into common stock. When estimating a yield based on current yields of the company's preferred stock, we must make appropriate adjustments for the effects of these features on the yield of an issue. For example, if the company has callable, convertible preferred stock outstanding yet it is expected that the company will issue only noncallable, nonconvertible preferred stock in the future, we would have to either use the current yields on comparable companies' noncallable, nonconvertible preferred stock or estimate the yield on preferred equity using methods outside the scope of this coverage.

\section{EXAMPLE 5}
\section{Calculating the Cost of Preferred Stock}
\begin{enumerate}
  \item Consider a company that has one issue of preferred stock outstanding with a $\$ 3.75$ cumulative dividend. If the price of this stock is $\$ 80$, what is the estimate of its cost of preferred stock?
\end{enumerate}

\section{Solution:}
Cost of preferred stock $=\$ 3.75 / \$ 80=4.6875 \%$.

\section{EXAMPLE 6}
\section{Choosing the Best Estimate of the Cost of Preferred Stock}
\begin{enumerate}
  \item Wim Vanistendael is finance director of De Gouden Tulip N.V., a leading Dutch flower producer and distributor. He has been asked by the CEO to calculate the cost of preferred stock and has recently obtained the following information:
\end{enumerate}

\begin{itemize}
  \item The issue price of preferred stock was $€ 3.5$ million, and the preferred dividend is $5 \%$.

  \item If the company issued new preferred stock today, the preferred dividend yield would be $6.5 \%$.

  \item The company's marginal tax rate is $30.5 \%$.

\end{itemize}

What is the cost of preferred stock for De Gouden Tulip N.V.?

\section{Solution:}
If De Gouden Tulip were to issue new preferred stock today, the dividend yield would be close to $6.5 \%$. The current terms thus prevail over the past terms when evaluating the actual cost of preferred stock. The cost of preferred stock for De Gouden Tulip is, therefore, 6.5\%. Because preferred dividends offer no tax shield, there is no adjustment made on the basis of the marginal tax rate.

\section{Cost of Common Equity}
The cost of common equity, $r_{e}$, usually referred to simply as the cost of equity, is the rate of return required by a company's common stockholders. A company may increase common equity through the reinvestment of earnings-that is, retained earnings-or through the issuance of new shares of stock.

The estimation of the cost of equity is challenging because of the uncertain nature of the future cash flows in terms of the amount and timing. Commonly used approaches for estimating the cost of equity include the capital asset pricing model (CAPM) method and the bond yield plus risk premium (BYPRP) method. In practice, analysts may use more than one approach to develop the cost of equity. A survey of analysts showed that the CAPM approach is used by $68 \%$ of respondents, whereas a build-up approach (bond yield plus a premium) is used by $43 \%$ of respondents (Pinto, Robinson, and Stowe 2019).

\section{Capital Asset Pricing Model Approach}
In the CAPM approach, we use the basic relationship from the capital asset pricing model theory that the expected return on a stock, $E\left(R_{i}\right)$, is the sum of the risk-free rate of interest, $R_{F}$, and a premium for bearing the stock's market risk, $\beta_{i}\left(R_{M}-R_{F}\right)$. Note that this premium incorporates the stock's return sensitivity to changes in the market return, or market-related risk, known as $\beta_{i}$, or beta:

$$
\mathrm{E}\left(R_{i}\right)=R_{F}+\beta_{i}\left[\mathrm{E}\left(R_{M}\right)-R_{F}\right]
$$

where

$$
\begin{aligned}
& \beta_{i}=\text { the return sensitivity of stock } i \text { to changes in the market return } \\
& \mathrm{E}\left(R_{M}\right)=\text { the expected return on the market } \\
& \mathrm{E}\left(R_{M}\right)-R_{F}=\text { the expected market risk premium }
\end{aligned}
$$

A risk-free asset is defined here as an asset that has no default risk. A common proxy for the risk-free rate is the yield on a default-free government debt instrument. In general, the selection of the appropriate risk-free rate should be guided by the duration of projected cash flows. For example, for the evaluation of a project with an estimated useful life of 10 years, the rate on the 10-year Treasury bond would be an appropriate proxy to use.

\section{Using the CAPM to Estimate the Cost of Equity}
\begin{enumerate}
  \item Valence Industries wants to know its cost of equity. Its chief financial officer (CFO) believes the risk-free rate is $5 \%$, the market risk premium is $7 \%$, and Valence's equity beta is 1.5 . What is Valence's cost of equity using the CAPM approach?
\end{enumerate}

\section{Solution:}
The cost of equity for Valence is $5 \%+1.5(7 \%)=15.5 \%$. 2. Exxon Mobil Corporation, BP p.l.c., and Total S.A. are three "super major" integrated oil and gas companies headquartered, respectively, in the United States, the United Kingdom, and France. An analyst estimates that the market risk premium in the United States, the United Kingdom, and the eurozone are, respectively, 4.4\%, 5.5\%, and 5.9\%. Other information is summarized in Exhibit 1.

\section{Exhibit 1: ExxonMobil, BP, and Total}
\begin{center}
\begin{tabular}{|c|c|c|c|}
\hline
Company & Beta & $\begin{array}{l}\text { Estimated Market } \\ \text { Risk Premium (\%) }\end{array}$ & Risk-Free Rate (\%) \\
\hline
$\begin{array}{l}\text { Exxon Mobil } \\ \text { Corporation }\end{array}$ & 0.90 & 4.4 & 2.8 \\
\hline
BP p.l.c. & 0.78 & 5.5 & 2.0 \\
\hline
Total S.A. & 0.71 & 5.9 & 1.7 \\
\hline
\end{tabular}
\end{center}

Source: Bloomberg; Fernandez, Pershn, and Acin (2018) survey.

Using the capital asset pricing model, calculate the cost of equity for:

\begin{enumerate}
  \item Exxon Mobil Corporation.

  \item BP p.l.c.

  \item Total S.A.

\end{enumerate}

\section{Solution:}
\begin{enumerate}
  \item The cost of equity for ExxonMobil is $2.8 \%+0.90(4.4 \%)=6.76 \%$.

  \item The cost of equity for BP is $2.0 \%+0.78(5.5 \%)=6.29 \%$.

  \item The cost of equity for Total is $1.7 \%+0.71(5.9 \%)=5.89 \%$.

\end{enumerate}

The expected market risk premium, or $\mathrm{E}\left(R_{M}-R_{F}\right)$, is the premium that investors demand for investing in a market portfolio relative to the risk-free rate. When using the CAPM to estimate the cost of equity, in practice we typically estimate beta relative to an equity market index. In that case, the market premium estimate we are using is actually an estimate of the equity risk premium (ERP). Therefore, we are using the terms market risk premium and equity risk premium interchangeably.

An alternative to the CAPM to accommodate risks that may not be captured by the market portfolio alone is a multifactor model that incorporates factors that may be other sources of priced risk (risk for which investors demand compensation for bearing), including macroeconomic factors and company-specific factors. In general,

$\mathrm{E}\left(R_{i}\right)=R_{F}+\beta_{i 1}(\text { Factor risk premium })_{1}$

$+\beta_{i 2}$ (Factor risk premium $)_{2}+\ldots$

$+\beta_{i j}(\text { Factor risk premium })_{j}$,

where

$\beta_{i j}=$ stock $i$ 's sensitivity to changes in the $j$ th factor

(Factor risk premium $)_{j}=$ expected risk premium for the $j$ th factor

The basic idea behind these multifactor models is that the CAPM beta may not capture all the risks, especially in a global context, which include inflation, business-cycle, interest rate, exchange rate, and default risks. There are several ways to estimate the equity risk premium, although there is no general agreement as to the best approach. The two we discuss are the historical equity risk premium approach and the survey approach.

The historical equity risk premium approach is a well-established approach based on the assumption that the realized equity risk premium observed over a long period of time is a good indicator of the expected equity risk premium. This approach requires compiling historical data to find the average rate of return of a country's market portfolio and the average rate of return for the risk-free rate in that country. For example, an analyst might use the historical returns to the TOPIX Index to estimate the risk premium for Japanese equities. The exceptional bull market observed during the second half of the 1990s and the bursting of the technology bubble that followed during 2000-2002 remind us that the time period for such estimates should cover complete market cycles.

Elroy Dimson, Paul Marsh, and Mike Staunton (2018) conducted an analysis of the equity risk premiums observed in markets located in 21 countries, including the United States, over the period 1900-2017. These researchers found that the annualized US equity risk premium relative to US Treasury bills was 5.6\% (geometric mean) and 7.5\% (arithmetic mean). They also found that the annualized US equity risk premium relative to bonds was $4.4 \%$ (geometric mean) and $6.5 \%$ (arithmetic mean). Jeremy Siegel (2005), covering the period from 1802 through 2004, observed an equity return of $6.82 \%$ and an equity risk premium in the range of $3.31 \%-5.36 \%$. Note that the arithmetic mean is greater than the geometric mean as a result of the significant volatility of the observed market rate of return and the observed risk-free rate. Under the assumption of an unchanging distribution of returns over time, the arithmetic mean is the unbiased estimate of the expected single-period equity risk premium, but the geometric mean better reflects the growth rate over multiple periods. In Exhibit 2, we provide historical estimates of the equity risk premium for a few of the developed markets from Dimson et al. (2018).

Exhibit 2: Selected Equity Risk Premiums Relative to Bonds (1900-2017)

\begin{center}
\begin{tabular}{lcc}
\hline
 & Mean &  \\
\hline
 & Geometric & Arithmetic \\
\hline
Australia & $5.0 \%$ & $6.6 \%$ \\
Canada & 3.5 & 5.1 \\
France & 3.1 & 5.4 \\
Germany & 5.1 & 8.4 \\
Japan & 5.1 & 9.1 \\
South Africa & 5.3 & 7.1 \\
Switzerland & 2.2 & 3.7 \\
United Kingdom & 3.7 & 5.0 \\
United States & 4.4 & 6.5 \\
 &  &  \\
\hline
\end{tabular}
\end{center}

Note: Germany excludes 1922-1923.

Source: Dimson, Marsh, and Staunton (2018). To illustrate the historical method as applied in the CAPM, suppose that we use the historical geometric mean for US equity of $4.4 \%$ to value Apple Computer as of early August 2018. According to Yahoo Finance, Apple had a beta of 1.14 at that time. Using a 10-year US Treasury bond yield of 3.0\% to represent the risk-free rate, the estimate of the cost of equity for Apple Computer is $3.0 \%+1.14(4.4 \%)=8.02 \%$.

In general, the equity risk premium can be written as

$\mathrm{ERP}=\bar{R}_{M}-\bar{R}_{F}$

where ERP is the equity risk premium, $\bar{R}_{M}$ is the mean return for equity, and $\bar{R}_{F}$ the risk-free rate.

The historical premium approach has several limitations. One limitation is that the level of risk of the stock index may change over time. Another is that the risk aversion of investors may change over time. A third limitation is that the estimates are sensitive to the method of estimation and the historical period covered.

\section{EXAMPLE 6}
\section{Estimating the Equity Risk Premium Using Historical Rates of Return}
\begin{enumerate}
  \item Suppose that the arithmetic average T-bond rate observed over the last 90 years is an unbiased estimator for the risk-free rate and is $4.88 \%$. Likewise, suppose the arithmetic average of return on the market observed over the last 90 years is an unbiased estimator for the expected return for the market. The average rate of return of the market was $9.65 \%$. Calculate the equity risk premium.
\end{enumerate}

\section{Solution:}
$$
\mathrm{ERP}=\bar{R}_{M}-\bar{R}_{F}=9.65 \%-4.88 \%=4.77 \%
$$

Another approach to estimate the equity risk premium is quite direct: Ask a panel of finance experts for their estimates, and take the mean response. This is the survey approach. For example, a survey of US CFOs in December 2017 found that the average expected US equity risk premium over the next 10 years was $4.42 \%$ and the median was 3.63\% (Graham and Harvey 2018).

Once we have an estimate of the equity risk premium, we fine-tune this estimate for the particular company or project by adjusting it for the specific systematic risk of the project. We adjust for the specific systematic risk by multiplying the market risk premium by beta to arrive at the company's or project's risk premium, which we then add to the risk-free rate to determine the cost of equity within the framework of the CAPM.

\section{Bond Yield plus Risk Premium Approach}
For companies with publicly traded debt, the bond yield plus risk premium approach provides a quick estimate of the cost of equity. The BYPRP approach is based on the fundamental tenet in financial theory that the cost of capital of riskier cash flows is higher than that of less risky cash flows. In this approach, we sum the before-tax cost of debt, $r_{d}$, and a risk premium that captures the additional yield on a company's stock relative to its bonds. The estimate is, therefore,

$$
r_{e}=r_{d}+\text { Risk premium }
$$

The risk premium compensates for the additional risk of the equity issue compared with the debt issue (recognizing that debt has a prior claim on the cash flows of the company). This risk premium is not to be confused with the equity risk premium. The equity risk premium is the difference between the cost of equity and the risk-free rate of interest. The risk premium in the bond yield plus risk premium approach is the difference between the cost of equity and the cost of debt of the company. Ideally, this risk premium is forward looking, representing the additional risk associated with the equity of the company as compared with the company's debt. However, we often estimate this premium using historical spreads between bond yields and stock yields. In developed country markets, a typical risk premium added is in the range of $3 \%-5 \%$.

Looking again at Apple Computer, as of early August 2018, the yield to maturity of Apple's 3.35\% coupon bonds maturing in 2027 was approximately 3.56\%. Adding an arbitrary risk premium of $4.0 \%$ produces an estimate of the cost of equity of $3.56 \%$ $+4.0 \%=7.56 \%$. This estimate contrasts with the higher estimate of $8.026 \%$ from the CAPM approach. Such disparities are not uncommon and reflect the difficulty of cost of equity estimation.

\section{ESTIMATING BETA}
explain and demonstrate beta estimation for public companies, thinly traded public companies, and nonpublic companies

Beta is an estimate of the company's systematic or market-related risk. It is a critical component of the CAPM, and it can be used to calculate a company's WACC. Therefore, it is essential to have a good understanding of how beta is estimated.

\section{Estimating Beta for Public Companies}
The simplest estimate of beta results from an ordinary least squares regression of the return on the stock on the return on the market. The result is often called an unadjusted or "raw" historical beta. The actual values of beta estimates are influenced by several choices:

\begin{itemize}
  \item The choice of the index used to represent the market portfolio: For US equities, the S\&P 500 Index and NYSE Composite have been traditional choices. In Japan, analysts would likely use the Nikkei 225 Index.

  \item The length of the data period and the frequency of observations: The most common choice is five years of monthly data, yielding 60 observations.

\end{itemize}

Researchers have observed that beta tends to regress toward 1.0. In other words, the value of a stock's beta in a future period is likely to be closer to the mean value of 1.0, the beta of an average-systematic-risk security, than to the value of the calculated raw beta. Because valuation is forward looking, it is logical to adjust the raw beta so that it more accurately predicts a future beta. The most commonly used adjustment was introduced by Blume (1971):

Adjusted beta $=(2 / 3)($ Unadjusted beta $)+(1 / 3)(1.0)$.

For example, if the beta from a regression of an asset's returns on the market return is 1.30, adjusted beta is $(2 / 3)(1.30)+(1 / 3)(1.0)=1.20$. Equation 7 acts to "smooth" raw betas by adjusting betas above and below 1.0 toward 1.0. Vendors of financial information, such as Bloomberg, often report both raw and adjusted betas.

\section{EXAMPLE 7}
\section{Estimating the Adjusted Beta for a Public Company}
\begin{enumerate}
  \item Betty Lau is an analyst trying to estimate the cost of equity for Singapore Telecommunications Limited. She begins by running an ordinary least squares regression to estimate the beta. Her estimated value is 0.4 , which she believes needs adjustment. What is the adjusted beta value she should use in her analysis?
\end{enumerate}

\section{Solution:}
Adjusted beta $=(2 / 3)(0.4)+(1 / 3)(1.0)=0.6$.

Arriving at an estimated beta for publicly traded companies is generally not a problem because of the accessibility of stock return data, the ease of use of estimating beta using simple regression, and the availability of estimated betas on publicly traded companies from financial analysis vendors.

The challenge comes in estimating a beta for a company that is thinly traded or nonpublic or for a project that is not the average or typical project of a publicly traded company. Estimating beta in these cases requires proxying for the beta by using information on the project or company combined with the beta of a publicly traded company.

\section{Estimating Beta for Thinly Traded and Nonpublic Companies}
It is not possible to run an ordinary least squares regression to estimate beta if a stock is thinly traded or a company is nonpublic. When a share issue trades infrequently, the most recent transaction price may be stale and may not reflect underlying changes in value. If beta is estimated on the basis of, for example, a monthly data series in which missing values are filled with the most recent transaction price, the estimated beta will be too small. This is because this methodology implicitly assumes that the stock's price is more stable than it really is. As a result, the required return on equity will be understated.

In these cases, a practical alternative is to base the beta estimate on the betas of comparable companies that are publicly traded. A comparable company, also called a peer company, is a company that has similar business risk. A comparable, or peer, company can be identified by using an industry classification system, such as the MSCI/Standard \& Poor's Global Industry Classification Standard (GICS) or the FTSE Industry Classification Benchmark (ICB). The analyst can then indirectly estimate the beta on the basis of the betas of the peer companies.

Because financial leverage can affect beta, an adjustment must be made if the peer company has a substantially different capital structure. First, the peer company's beta must be unlevered to estimate the beta of the assets-reflecting only the systematic risk arising from the fundamentals of the industry. Then, the unlevered beta, often referred to as the asset beta because it reflects the business risk of the assets, must be re-levered to reflect the capital structure of the company in question.

Let $\beta_{E}$ be the equity beta of the peer company before removing the effects of leverage. Assuming the debt of the peer company is of high quality-so that the debt's beta, or $\beta_{D}$, is approximately equal to zero (that is, it is assumed to have no market risk)-analysts can use the following expression to unlever the beta:

$$
\beta_{U}=\beta_{E}\left[\frac{1}{1+(1-t) \frac{D}{E}}\right]
$$

where $\beta_{U}$ is the unlevered beta, $t$ is the marginal tax rate of the peer company, and $D$ and $E$ are the market values of debt and equity, respectively, of the peer company.

Now we can re-lever the unlevered beta by rearranging the equation to reflect the capital structure of the thinly traded or nonpublic company in question:

$$
\beta_{E}=\beta_{U}\left[1+(1-t) \frac{D}{E}\right] \text {. }
$$

where $\beta_{E}$ is now the equity beta of the thinly traded or nonpublic company, $t$ is the marginal tax rate of the thinly traded or nonpublic company, and $D$ and $E$ are the debt-to-equity values, respectively, of the thinly traded or nonpublic company.

\section{EXAMPLE 8}
\section{Estimating the Adjusted Beta for a Nonpublic Company}
\begin{enumerate}
  \item Raffi Azadian wants to determine the cost of equity for Elucida Oncology, a privately held company. Raffi realizes that he needs to estimate Elucida's beta before he can proceed. He determines that Merck \& Co. is an appropriate publicly traded peer company. Merck has a beta of 0.7 , it is $40 \%$ funded by debt, and its marginal tax rate is $21 \%$. If Elucida is only $10 \%$ funded by debt and its marginal tax rate is also $21 \%$, what is Elucida's beta?
\end{enumerate}

\section{Solution:}
Since Merck is $40 \%$ funded by debt, it is $60 \%$ funded by equity. Therefore,

Unlevered beta $=(0.7)\left[\frac{1}{1+(1-0.21)\left(\frac{0.4}{0.6}\right)}\right]=0.46$.

Now we re-lever the unlevered beta using Elucida's tax rate and capital structure:

Elucida's beta $=0.46\left[1+(1-0.21)\left(\frac{0.1}{0.9}\right)\right]=0.50$.

The following table and figure show how Elucida's beta increases as leverage rises.

\begin{center}
\begin{tabular}{|cc|}
\hline
Debt-to-Equity Ratio & Equity Beta \\
\hline
0.00 & 0.46 \\
\hline
0.11 & 0.50 \\
\hline
0.25 & 0.55 \\
0.43 & 0.62 \\
0.67 & 0.70 \\
1.00 & 0.82 \\
1.50 & 1.01 \\
2.33 & 1.31 \\
4.00 & 1.91 \\
9.00 & 3.73 \\
\hline
\end{tabular}
\end{center}

\begin{center}
\includegraphics[max width=\textwidth]{2023_05_04_7b535d0a870224f62e3dg-032}
\end{center}

The beta estimate can then be used to determine the component cost of equity and combined with the cost of debt in a weighted average to provide an estimate of the cost of capital for the company.

\section{EXAMPLE 9}
\section{Estimating the Weighted Average Cost of Capital}
\begin{enumerate}
  \item Georg Schrempp is the CFO of Bayern Chemicals KgaA, a German manufacturer of industrial, commercial, and consumer chemical products. Bayern Chemicals is privately owned, and its shares are not listed on an exchange. The CFO has appointed Markus Meier, CFA, a third-party valuator, to perform a stand-alone valuation of Bayern Chemicals. Meier has access to the following information to calculate Bayern Chemicals' weighted average cost of capital:
\end{enumerate}

\begin{itemize}
  \item The nominal risk-free rate, represented by the yield on the long-term 10-year German bund, was $4.5 \%$ at the valuation date.

  \item The average long-term historical equity risk premium in Germany is assumed to be $5.7 \%$.

  \item Bayern Chemicals' corporate tax rate is $38 \%$.

  \item Bayern Chemicals' target debt-to-equity ratio is 0.7. Its capital structure is $41 \%$ debt.

  \item Bayern Chemicals' cost of debt has an estimated spread of 225 bps over the 10-year bund.

  \item Exhibit 3 supplies additional information on comparables for Bayern Chemicals.

\end{itemize}

\section{Exhibit 3: Information on Comparables}
\begin{center}
\begin{tabular}{|c|c|c|c|c|c|c|}
\hline
$\begin{array}{l}\text { Comparable } \\ \text { Companies }\end{array}$ & Country & $\begin{array}{c}\text { Tax } \\ \text { Rate } \\ \text { (\%) }\end{array}$ & $\begin{array}{c}\text { Market } \\ \text { Capitalization } \\ \text { in Millions }\end{array}$ & $\begin{array}{c}\text { Net } \\ \text { Debt in } \\ \text { Millions }\end{array}$ & D/E & Beta \\
\hline
$\begin{array}{l}\text { British } \\ \text { Chemicals Ltd. }\end{array}$ & $\begin{array}{l}\text { United } \\ \text { Kingdom }\end{array}$ & 30.0 & 4,500 & 6,000 & 1.33 & 1.45 \\
\hline
$\begin{array}{l}\text { Compagnie } \\ \text { Petrochimique } \\ \text { S.A. }\end{array}$ & France & 30.3 & 9,300 & 8,700 & 0.94 & 0.75 \\
\hline
$\begin{array}{l}\text { Rotterdam } \\ \text { Chemie N.V. }\end{array}$ & Netherlands & 30.5 & 7,000 & 7,900 & 1.13 & 1.05 \\
\hline
Average &  &  &  &  & 1.13 & 1.08 \\
\hline
\end{tabular}
\end{center}

Based only on the information given, calculate Bayern Chemicals' WACC.

\section{Solution:}
To calculate the cost of equity, the first step is to "unlever" the betas of the comparable companies and calculate an average for a company with business risk similar to the average of these companies:

Comparable Companies

Unlevered Beta

British Chemicals Ltd.

$0.75=\left[\frac{1.45}{1+(1-0.30)(1.33)}\right]$

Compagnie Petrochimique S.A.

$0.45=\left[\frac{0.75}{1+(1-0.303)(0.94)}\right]$

\begin{center}
\begin{tabular}{lc}
\hline
Comparable Companies & Unlevered Beta \\
\hline
Rotterdam Chemie N.V. & $0.59=\left[\frac{1.05}{1+(1-0.305)(1.13)}\right]$ \\
Average* $^{0} 0.60$ &  \\
\hline
\end{tabular}
\end{center}

"An analyst must apply judgment and experience to determine a representative average for the comparable companies. This example uses a simple average, but in some situations a weighted average based on some factor, such as market capitalization, may be more appropriate.

Levering the average unlevered beta for the peer group average, applying Bayern Chemicals' target debt-to-equity ratio and marginal tax rate, results in a beta of 0.86 :

$$
\beta_{\text {BayernChemicals }}=0.60\{1+[(1-0.38) 0.7]\}=0.86 \text {. }
$$

Using CAPM, the cost of equity of Bayern Chemicals $\left(r_{e}\right)$ can be calculated as follows:

$$
r_{e}=4.5 \%+(0.86)(5.7 \%)=9.4 \% \text {. }
$$

The weights for the cost of debt and cost of equity may be calculated as follows:

$$
w_{d}=0.41, \text { and } w_{e}=(1-0.41)=0.59 \text {. }
$$

The before-tax cost of debt of Bayern Chemicals $\left(r_{d}\right)$ is 6.75\%:

$r_{d}=4.5 \%+2.25 \%=6.75 \%$.

As a result, Bayern Chemicals' WACC is 7.26\%:

$\mathrm{WACC}=(0.41)(0.0675)(1-0.38)+(0.59)(0.094)$

$=0.0726$, or $7.26 \%$.

\section{FLOTATION COSTS}
$$
\text { explain and demonstrate the correct treatment of flotation costs }
$$

When a company raises new capital, it generally seeks the assistance of investment bankers. Investment bankers charge the company a fee based on the size and type of offering. This fee is referred to as the flotation cost. In general, flotation costs are higher in percentage terms for equity issuances than they are for debt. They are also higher for smaller issuance amounts or for issuances that are perceived to be riskier. In the case of debt and preferred stock, we do not usually incorporate flotation costs in the estimated cost of capital because the amount of these costs is quite small, often less than $1 \%$ of the value of the offering.

However, with equity issuance, the flotation costs may be substantial, so we should consider these when estimating the cost of external equity capital. Average flotation costs for new equity have been estimated at $7.11 \%$ of the value of the offering in the United States, ${ }^{1} 1.65 \%$ in Germany, ${ }^{2} 5.78 \%$ in the United Kingdom, ${ }^{3}$ and $4.53 \%$ in Switzerland. ${ }^{4}$ A large part of the differences in costs among these studies is likely attributed to the type of offering; cash underwritten offers, typical in the United States, are generally more expensive than rights offerings, which are common in Europe.

How should flotation costs be accounted for? There are two views on this topic. One view, which you can find often in textbooks, is to directly incorporate the flotation costs into the cost of capital. The other view is that flotation costs should be incorporated into the valuation analysis as an additional cost. We will argue that the second view is preferred.

Consistent with the first view, we can specify flotation costs in monetary terms as an amount per share or as a percentage of the share price. With flotation costs specified in monetary terms on a per share basis, the cost of external equity is

$r_{e}=\left(\frac{D_{1}}{P_{0}-F}\right)+g$

where

$r_{e}$ is the cost of equity

$D_{1}$ is the dividend expected at the end of Period 1

$P_{0}$ is the current stock price

$F$ is the monetary per share flotation cost

$g$ is the growth rate

As a percentage applied against the price per share, the cost of external equity is

$$
r_{e}=\left[\frac{D_{1}}{P_{0}(1-f)}\right]+g
$$

where $f$ is the flotation cost as a percentage of the issue price.

\section{EXAMPLE 11}
\section{Estimating the Cost of Equity with Flotation Costs}
A company has a current dividend of $\$ 2$ per share, a current price of $\$ 40$ per share, and an expected growth rate of $5 \%$.

\begin{enumerate}
  \item What is the cost of internally generated equity (i.e., stock is not issued and flotation costs are not incurred)?
\end{enumerate}

\section{Solution:}
$$
r_{e}=\left[\frac{\$ 2(1+0.05)}{\$ 40}\right]+0.05=0.0525+0.05=0.1025, \text { or } 10.25 \%
$$

1 Inmoo Lee, Scott Lochhead, Jay R. Ritter, and Quanshui Zhao, "The Costs of Raising Capital," Journal of Financial Research 19 (Spring 1996): 59-71.

2 Thomas Bühner and Christoph Kaserer, "External Financing Costs and Economies of Scale in Investment Banking: The Case of Seasoned Equity Offerings in Germany," European Financial Management 9 (June 2002): 249

3 Seth Armitage, "The Direct Costs of UK Rights Issues and Open Offers," European Financial Management $6(2000): 57-68$

\begin{enumerate}
  \setcounter{enumi}{3}
  \item Christoph Kaserer and Fabian Steiner, "The Cost of Raising Capital-New Evidence from Seasoned Equity Offerings in Switzerland," working paper, Technische Universität München (February 2004) 2. What is the cost of external equity (i.e., new shares are issued and flotation costs are incurred) if the flotation costs are $4 \%$ of the issuance?
\end{enumerate}

Solution:

$$
r_{e}=\left[\frac{\$ 2(1+0.05)}{\$ 40(1-0.04)}\right]+0.05=0.05469+0.05=0.1047 \text {, or } 10.47 \% .
$$

Many experts object to this methodology. Flotation costs are a cash flow that occurs at issue and they affect the value of the project only by reducing the initial cash flow. However, by adjusting the cost of capital for flotation costs, we apply a higher cost of capital to determine the present value of the future cash flows. The result is that the calculated net present value of a project is less than its true net present value. As a result, otherwise profitable projects may get rejected when this methodology is used.

An alternative and preferred approach is to make the adjustment for flotation costs to the cash flows in the valuation computation. For example, consider a project that requires a $€ 60,000$ initial cash outlay and is expected to produce cash flows of $€ 10,000$ each year for 10 years. Suppose the company's marginal tax rate is $40 \%$, the before-tax cost of debt is $5 \%$, and the cost of equity is $10 \%$. Assume the company will finance the project with $40 \%$ debt and $60 \%$ equity. Exhibit 4 summarizes the information on the component costs of capital.

\section{Exhibit 4: After-Tax Costs of Debt and Equity}
\begin{center}
\begin{tabular}{lccc}
\hline
Source of Capital & $\begin{array}{c}\text { Amount } \\ \text { Raised ( } \boldsymbol{\epsilon})\end{array}$ & Proportion & Marginal After-Tax Cost \\
\hline
Debt & 24,000 & 0.40 & $0.05(1-0.4)=0.03$ \\
Equity & 36,000 & 0.60 & 0.10 \\
\hline
\end{tabular}
\end{center}

The weighted average cost of capital is $7.2 \%$, calculated as $0.40(3 \%)+0.60(10 \%)$. Ignoring flotation costs for the moment and using a financial calculator, we find that the net present value (NPV) of this project can be expressed as

NPV = Present value of inflows - Present value of the outflows,

or

$$
\mathrm{NPV}=€ 69,591-€ 60,000=€ 9,591
$$

where $€ 69,591$ is the present value of $€ 10,000$ per year for 10 years at $7.2 \%$. Now suppose flotation costs amount to $5 \%$ of the new equity capital: $(0.05)(€ 36,000)=$ $€ 1,800$. If flotation costs are not tax deductible, the net present value considering flotation costs is

$$
\mathrm{NPV}=€ 69,591-€ 60,000-€ 1,800=€ 7,791 .
$$

If flotation costs are tax deductible, the net present value considering flotation costs is

$$
\text { NPV }=€ 69,591-€ 60,000-€ 1,800(1-0.4)=€ 8,511 .
$$

Suppose instead of considering the flotation costs as part of the cash flows, we made an adjustment to the cost of equity. Without showing the calculations, the cost of equity increases from $10 \%$ to $10.2632 \%$, the cost of capital increases from $7.22 \%$ to $7.3579 \%$, and the NPV decreases from $€ 9,591$ to $€ 9,089$. As you can see, we arrive at different assessments of value using these two methods. If the preferred method is to deduct the flotation costs as part of the net present value calculation, why do many textbooks highlight the adjustment to the cost of capital? One reason is that it is often difficult to identify specific financing associated with a project. Making the adjustment for flotation costs to the cost of capital is most useful if specific project financing cannot be identified. A second reason is that by adjusting the cost of capital for the flotation costs, it is easier to demonstrate how the costs of financing a company change as a company exhausts internally generated equity (i.e., retained earnings) and switches to externally generated equity (i.e., a new stock issue).

\section{METHODS IN USE}
We have introduced methods that may be used to estimate the cost of capital for a company or a project, but which methods do companies actually use when making investment decisions? John Graham and Campbell Harvey (2002) surveyed a large number of CFOs to find out which methods they prefer. Their survey revealed the following:

\begin{itemize}
  \item The most popular method for estimating the cost of equity is the capital asset pricing model.

  \item Few companies use the dividend discount model (which we did not cover) to estimate the cost of equity.

  \item Publicly traded companies are more likely to use the capital asset pricing model than are private companies.

  \item In evaluating projects, the majority of CFOs use a single company cost of capital, but a large portion apply some type of risk adjustment for individual projects.

\end{itemize}

Their survey also revealed that the single-factor capital asset pricing model is the most popular method for estimating the cost of equity. The second and third most popular methods, respectively, are average stock returns and multifactor return models. The lack of popularity of the dividend discount model indicates that this approach, although once favored, has lost its appeal in practice.

In a survey of publicly traded multinational European companies, Franck Bancel and Usha Mittoo (2004) provided evidence consistent with the Graham and Harvey (2002) survey. They found that over $70 \%$ of companies use the CAPM to determine the cost of equity; this finding is similar to the $73.5 \%$ of US companies that use the CAPM. In a survey of both publicly traded and private European companies, Dirk Brounen, Abe de Jong, and Kees Koedijk (2004) confirmed the result of Graham and Harvey that larger companies are more likely to use the more sophisticated methods, such as CAPM, in estimating the cost of equity. Brounen, de Jong, and Koedijk found that the use of the CAPM was less popular for their sample (ranging from 34\% to $55.6 \%$, depending on the country) than for the other two surveys, which may reflect the inclusion of smaller, private companies in their sample.

We learn from the survey evidence that the CAPM is a popular method for estimating the cost of equity capital and that it is used less often by smaller, private companies. The latter result is not surprising, because of the difficulty in estimating systematic risk in cases in which the company's equity is not publicly traded.

\section{SUMMARY}
In this reading, we provided an overview of the techniques used to calculate the cost of capital for companies and projects. We examined the weighted average cost of capital, discussing the methods commonly used to estimate the component costs of capital and the weights applied to these components.

\begin{itemize}
  \item The weighted average cost of capital is a weighted average of the after-tax marginal costs of each source of capital: WACC $=w_{d} r_{d}(1-t)+w_{p} r_{p}+w_{e} r_{e}$.

  \item The before-tax cost of debt is generally estimated by either the yield-to-maturity method or the bond rating method.

  \item The yield-to-maturity method of estimating the before-tax cost of debt uses the familiar bond valuation equation. Assuming semi-annual coupon payments, the equation is

\end{itemize}

$$
P_{0}=\frac{P M T_{1}}{\left(1+\frac{r_{d}}{2}\right)}+\ldots+\frac{P M T_{n}}{\left(1+\frac{r_{d}}{2}\right)^{n}}+\frac{F V}{\left(1+\frac{r_{d}}{2}\right)^{n}}=\left[\sum_{t=1}^{n} \frac{P M T_{t}}{\left(1+\frac{r_{d}}{2}\right)^{t}}\right]+\frac{F V}{\left(1+\frac{r_{d}}{2}\right)^{n}} .
$$

We solve for the six-month yield $\left(r_{d} / 2\right)$ and then annualize it to arrive at the before-tax cost of debt, $r_{d}$.

\begin{itemize}
  \item Because interest payments are generally tax deductible, the after-tax cost is the true, effective cost of debt to the company. If a yield to maturity or bond rating is not available, such as in the case of a private company without rated debt or a project, the estimate of the cost of debt becomes more challenging.

  \item The cost of preferred stock is the preferred stock dividend divided by the current preferred stock price:

\end{itemize}

$r_{p}=\frac{D_{p}}{P_{p}}$

\begin{itemize}
  \item The cost of equity is the rate of return required by a company's common stockholders. We estimate this cost using the CAPM (or its variants).

  \item The CAPM is the approach most commonly used to calculate the cost of equity. The three components needed to calculate the cost of equity are the risk-free rate, the equity risk premium, and beta:

\end{itemize}

$\mathrm{E}\left(R_{i}\right)=R_{F}+\beta_{i}\left[\mathrm{E}\left(R_{M}\right)-R_{F}\right]$

\begin{itemize}
  \item In estimating the cost of equity, an alternative to the CAPM is the bond yield plus risk premium approach. In this approach, we estimate the before-tax cost of debt and add a risk premium that reflects the additional risk associated with the company's equity.

  \item When estimating the cost of equity capital using the CAPM, if we do not have publicly traded equity, we may be able to use a comparable company operating in the same business line to estimate the unlevered beta for a company with similar business risk, $\beta_{U}$ :

\end{itemize}

$$
\beta_{U}=\beta_{E}\left[\frac{1}{1+(1-t) \frac{D}{E}}\right] .
$$

Then, we lever this beta to reflect the financial risk of the project or company:

$\beta_{E}=\beta_{U}\left[1+(1-t) \frac{D}{E}\right]$

\begin{itemize}
  \item Flotation costs are costs incurred in the process of raising additional capital. The preferred method of including these costs in the analysis is as an initial cash flow in the valuation analysis.

  \item Survey evidence tells us that the CAPM method is the most popular method used by companies in estimating the cost of equity. The CAPM method is more popular with larger, publicly traded companies, which is understandable considering the additional analyses and assumptions required in estimating systematic risk for a private company or project.

\end{itemize}

\section{REFERENCES}
Bancel, Franck, Usha Mittoo. 2004. "Cross-Country Determinants of Capital Structure Choice: A Survey of European Firms." Financial Management 33 (4): 103-32.

Blume, Marshall. 1971. "On the Assessment of Risk." Journal of Finance 26 (1): 1-10.

Brounen, Dirk, Abe de Jong, Kees Koedijk. 2004. "Corporate Finance in Europe: Confronting Theory with Practice." Financial Management 33 (4): 71-101.

Dimson, E., P. Marsh, M. Staunton. 2018. "Credit Suisse Global Investment Returns Yearbook 2018" (February).

Fernandez, Pablo, Vitaly Pershin, Isabel Fernández Acin. 2018. "Market Risk Premium and Risk-Free Rate Used for 59 Countries in 2018: A Survey." Available at SSRN: \href{https://ssrn.com/}{https://ssrn.com/} abstract $=3155709$.

Graham, John R., Campbell R. Harvey. 2002. "How Do CFOs Make Capital Budgeting and Capital Structure Decisions?" Journal of Applied Corporate Finance 15 (1): 8-23.

Graham, John R., Campbell R. Harvey. 2018. “The Equity Risk Premium in 2018.” Working paper (27 March). Available at \href{https://ssrn.com/abstract}{https://ssrn.com/abstract} $=3151162$.

Pinto, Jerald E., Thomas R. Robinson, John D. Stowe. 2019. "Equity Valuation: A Survey of Professional Practice." Review of Financial Economics 37 (2): 219-33.

Siegel, Jeremy J. 2005. "Perspectives on the Equity Risk Premium." Financial Analysts Journal $61(6): 61-73$.

\section{PRACTICE PROBLEMS}
\section{The following information relates to questions}
 1-5Jurgen Knudsen has been hired to provide industry expertise to Henrik Sandell, CFA, an analyst for a pension plan managing a global large-cap fund internally. Sandell is concerned about one of the fund's larger holdings, auto parts manufacturer Kruspa AB. Kruspa currently operates in 80 countries, with the previous year's global revenues at $€ 5.6$ billion. Recently, Kruspa's CFO announced plans for expansion into Trutan, a country with a developing economy. Sandell worries that this expansion will change the company's risk profile and wonders if he should recommend a sale of the position.

Sandell provides Knudsen with the basic information. Kruspa's global annual free cash flow to the firm is $€ 500$ million, and earnings are $€ 400$ million. Sandell estimates that cash flow will level off at a $2 \%$ rate of growth. Sandell also estimates that Kruspa's after-tax free cash flow to the firm on the Trutan project for next three years is, respectively, $€ 48$ million, $€ 52$ million, and $€ 54.4$ million. Kruspa recently announced a dividend of $€ 4.00$ per share of stock. For the initial analysis, Sandell requests that Knudsen ignore possible currency fluctuations. He expects the Trutanese plant to sell only to customers within Trutan for the first three years. Knudsen is asked to evaluate Kruspa's planned financing of the required $€ 100$ million in Sweden with an $€ 80$ million public offering of 10-year debt and the remainder with an equity offering.

\section{Additional information:}
Equity risk premium, Sweden

$4.82 \%$

Risk-free rate of interest, Sweden

$4.25 \%$

Industry debt-to-equity ratio

0.3

Market value of Kruspa's debt

$€ 900$ million

Market value of Kruspa's equity

$€ 2.4$ billion

Kruspa's equity beta

1.3

Kruspa's before-tax cost of debt

$9.25 \%$

Trutan credit A2 country risk premium

$1.88 \%$

Corporate tax rate

$37.5 \%$

Interest payments each year

Level

\begin{enumerate}
  \item Using the capital asset pricing model, Kruspa's cost of equity capital for its typical project is closest to:
A. $7.62 \%$.
B. $10.52 \%$.
C. $12.40 \%$.

  \item Sandell is interested in the weighted average cost of capital of Kruspa AB prior to its investing in the Trutan project. This weighted average cost of capital is closest to:
A. $7.65 \%$.
B. $9.23 \%$.
C. $10.17 \%$.

  \item In his estimation of the project's cost of capital, Sandell would like to use the asset beta of Kruspa as a base in his calculations. The estimated asset beta of Kruspa prior to the Trutan project is closest to:
A. 1.053 .
B. 1.110 .
C. 1.327 .

  \item Sandell is performing a sensitivity analysis of the effect of the new project on the company's cost of capital. If the Trutan project has the same asset risk as Kruspa, the estimated project beta for the Trutan project, if it is financed $80 \%$ with debt, is closest to:
A. 1.300 .
B. 2.635 .
C. 3.686 .

  \item As part of the sensitivity analysis of the effect of the new project on the company's cost of capital, Sandell is estimating the cost of equity of the Trutan project considering that the Trutan project requires a country equity premium to capture the risk of the project. The cost of equity for the project in this case is closest to:
A. $10.52 \%$.
B. $19.91 \%$.
C. $28.95 \%$.

  \item Which of the following statements is correct?

\end{enumerate}

A. The appropriate tax rate to use in the adjustment of the before-tax cost of debt to determine the after-tax cost of debt is the average tax rate because interest is deductible against the company's entire taxable income.

B. For a given company, the after-tax cost of debt is generally less than both the cost of preferred equity and the cost of common equity.

C. For a given company, the after-tax cost of debt is generally higher than both the cost of preferred equity and the cost of common equity.

\begin{enumerate}
  \setcounter{enumi}{6}
  \item The Gearing Company has an after-tax cost of debt capital of $4 \%$, a cost of preferred stock of $8 \%$, a cost of equity capital of $10 \%$, and a weighted average cost of capital of 7\%. Gearing intends to maintain its current capital structure as it raises additional capital. In making its capital-budgeting decisions for the average-risk project, the relevant cost of capital is:
A. $4 \%$.
B. $7 \%$. C. $8 \%$.

  \item Fran McClure, of Alba Advisers, is estimating the cost of capital of Frontier Corporation as part of her valuation analysis of Frontier. McClure will be using this estimate, along with projected cash flows from Frontier's new projects, to estimate the effect of these new projects on the value of Frontier. McClure has gathered the following information on Frontier Corporation:

\end{enumerate}

\begin{center}
\begin{tabular}{lcc}
\hline
 & Current Year (\$) & $\begin{array}{c}\text { Forecasted for Next } \\ \text { Year (\$) }\end{array}$ \\
\hline
Book value of debt & 50 & 50 \\
Market value of debt & 62 & 63 \\
Book value of equity & 55 & 58 \\
Market value equity & 210 & 220 \\
\hline
\end{tabular}
\end{center}

The weights that McClure should apply in estimating Frontier's cost of capital for debt and equity are, respectively:

A. $w_{d}=0.200$ and $w_{e}=0.800$.

B. $w_{d}=0.185$ and $w_{e}=0.815$.

C. $w_{d}=0.223$ and $w_{e}=0.777$.

\begin{enumerate}
  \setcounter{enumi}{8}
  \item An analyst assembles the following facts concerning a company's component costs of capital and capital structure. Based on the information given, calculate the company's WACC.
\end{enumerate}

\begin{center}
\begin{tabular}{lc}
\hline
Facts & (\%) \\
\hline
Cost of equity based on the CAPM & 15.60 \\
Pretax cost of debt & 8.28 \\
Corporate tax rate & 30.00 \\
Capital structure weight & Equity 80, Debt 20 \\
\hline
\end{tabular}
\end{center}

\begin{enumerate}
  \setcounter{enumi}{9}
  \item The cost of equity is equal to the:
A. expected market return.
B. rate of return required by stockholders.
C. cost of retained earnings plus dividends.

  \item \href{http://Dot.Com}{Dot.Com} has determined that it could issue $\$ 1,000$ face value bonds with an $8 \%$ coupon paid semi-annually and a five-year maturity at $\$ 900$ per bond. If Dot. Com's marginal tax rate is $38 \%$, its after-tax cost of debt is closest to:
A. $6.2 \%$.
B. $6.4 \%$.
C. $6.6 \%$.

  \item The cost of debt can be determined using the yield-to-maturity and bond rating approaches. If the bond rating approach is used, the:

\end{enumerate}

A. coupon is the yield. B. yield is based on the interest coverage ratio.

C. company is rated and the rating can be used to assess the credit default spread of the company's debt.

\begin{enumerate}
  \setcounter{enumi}{12}
  \item Morgan Insurance Ltd. issued a fixed-rate perpetual preferred stock three years ago and placed it privately with institutional investors. The stock was issued at $\$ 25$ per share with a $\$ 1.75$ dividend. If the company were to issue preferred stock today, the yield would be $6.5 \%$. The stock's current value is:
A. $\$ 25.00$.
B. $\$ 26.92$.
C. $\$ 37.31$.

  \item Two years ago, a company issued $\$ 20$ million in long-term bonds at par value with a coupon rate of $9 \%$. The company has decided to issue an additional $\$ 20$ million in bonds and expects the new issue to be priced at par value with a coupon rate of $7 \%$. The company has no other debt outstanding and has a tax rate of $40 \%$. To compute the company's weighted average cost of capital, the appropriate after-tax cost of debt is closest to:
A. $4.2 \%$.
B. $4.8 \%$.
C. $5.4 \%$

  \item At the time of valuation, the estimated betas for JPMorgan Chase \& Co. and the Boeing Company were 1.50 and 0.80 , respectively. The risk-free rate of return was $4.35 \%$, and the equity risk premium was $8.04 \%$. Based on these data, calculate the required rates of return for these two stocks using the CAPM.

  \item An analyst's data source shows that Newmont Mining (NEM) has an estimated beta of -0.2 . The risk-free rate of return is $2.5 \%$, and the equity risk premium is estimated to be $4.5 \%$.

\end{enumerate}

A. Using the CAPM, calculate the required rate of return for investors in NEM.

B. The analyst notes that the current yield to maturity on corporate bonds with a credit rating similar to NEM is approximately $3.9 \%$. How should this information affect the analyst's estimate?

\begin{enumerate}
  \setcounter{enumi}{16}
  \item Wang Securities had a long-term stable debt-to-equity ratio of 0.65 . Recent bank borrowing for expansion into South America raised the ratio to 0.75 . The increased leverage has what effect on the asset beta and equity beta of the company?
\end{enumerate}

A. The asset beta and the equity beta will both rise.

B. The asset beta will remain the same, and the equity beta will rise.

C. The asset beta will remain the same, and the equity beta will decline.

\begin{enumerate}
  \setcounter{enumi}{17}
  \item Brandon Wiene is a financial analyst covering the beverage industry. He is evaluating the impact of DEF Beverage's new product line of flavored waters. DEF currently has a debt-to-equity ratio of 0.6 . The new product line would be financed with $\$ 50$ million of debt and $\$ 100$ million of equity. In estimating the valuation impact of this new product line on DEF's value, Wiene has estimated the equity beta and asset beta of comparable companies. In calculating the equity beta for the product line, Wiene is intending to use DEF's existing capital structure when converting the asset beta into a project beta. Which of the following statements is correct?
\end{enumerate}

A. Using DEF's debt-to-equity ratio of 0.6 is appropriate in calculating the new product line's equity beta.

B. Using DEF's debt-to-equity ratio of 0.6 is not appropriate; rather, the debt-to-equity ratio of the new product, 0.5 , is appropriate to use in calculating the new product line's equity beta.

C. Wiene should use the new debt-to-equity ratio of DEF that would result from the additional $\$ 50$ million debt and $\$ 100$ million equity in calculating the new product line's equity beta.

\begin{enumerate}
  \setcounter{enumi}{18}
  \item Happy Resorts Company currently has 1.2 million common shares of stock outstanding, and the stock has a beta of 2.2. It also has $\$ 10$ million face value of bonds that have five years remaining to maturity and an $8 \%$ coupon with semi-annual payments and are priced to yield 13.65\%. If Happy issues up to $\$ 2.5$ million of new bonds, the bonds will be priced at par and will have a yield of $13.65 \%$; if it issues bonds beyond $\$ 2.5$ million, the expected yield on the entire issuance will be 16\%. Happy has learned that it can issue new common stock at $\$ 10$ a share. The current risk-free rate of interest is $3 \%$, and the expected market return is $10 \%$. Happy's marginal tax rate is $30 \%$. If Happy raises $\$ 7.5$ million of new capital while maintaining the same debt-to-equity ratio, its weighted average cost of capital will be closest to:
A. $14.5 \%$.
B. $15.5 \%$.
C. $16.5 \%$.
\end{enumerate}

\section{The following information relates to questions}
\section{0-23}
Boris Duarte, CFA, covers initial public offerings for Zellweger Analytics, an independent research firm specializing in global small-cap equities. He has been asked to evaluate the upcoming new issue of TagOn, a US-based business intelligence software company. The industry has grown at $26 \%$ per year for the previous three years. Large companies dominate the market, but sizable comparable companies, such as Relevant Ltd., ABJ Inc., and Opus Software Pvt. Ltd., also compete. Each of these competitors is domiciled in a different country, but they all have shares of stock that trade on the US NASDAQ. The debt ratio of the industry has risen slightly in recent years.

\begin{center}
\begin{tabular}{|c|c|c|c|c|c|c|}
\hline
Company & $\begin{array}{c}\text { Sales in } \\ \text { Millions } \\ \text { (\$) }\end{array}$ & $\begin{array}{c}\text { Market } \\ \text { Value } \\ \text { Equity in } \\ \text { Millions (\$) }\end{array}$ & $\begin{array}{c}\text { Market } \\ \text { Value Debt } \\ \text { in Millions } \\ \text { (\$) }\end{array}$ & $\begin{array}{c}\text { Equity } \\ \text { Beta }\end{array}$ & $\begin{array}{l}\text { Tax } \\ \text { Rate } \\ \text { (\%) }\end{array}$ & $\begin{array}{c}\text { Share Price } \\ \text { (\$) }\end{array}$ \\
\hline
Relevant Ltd. & 752 & 3,800 & 0.0 & 1.702 & 23 & 42 \\
\hline
ABI Inc. & 843 & 2,150 & 6.5 & 2.800 & 23 & 24 \\
\hline
\end{tabular}
\end{center}

\begin{center}
\begin{tabular}{|c|c|c|c|c|c|c|}
\hline
Company & $\begin{array}{c}\text { Sales in } \\ \text { Millions } \\ \text { (\$) }\end{array}$ & $\begin{array}{c}\text { Market } \\ \text { Value } \\ \text { Equity in } \\ \text { Millions (\$) }\end{array}$ & $\begin{array}{c}\text { Market } \\ \text { Value Debt } \\ \text { in Millions } \\ \text { (\$) }\end{array}$ & $\begin{array}{c}\text { Equity } \\ \text { Beta }\end{array}$ & $\begin{array}{c}\text { Tax } \\ \text { Rate } \\ \text { (\%) }\end{array}$ & $\begin{array}{l}\text { Share Price } \\ \text { (\$) }\end{array}$ \\
\hline
Opus Software & 211 & 972 & 13.0 & 3.400 & 23 & 13 \\
\hline
\end{tabular}
\end{center}

Duarte uses the information from the preliminary prospectus for TagOn's initial offering. The company intends to issue 1 million new shares. In his conversation with the investment bankers for the deal, he concludes the offering price will be between $\$ 7$ and $\$ 12$. The current capital structure of TagOn consists of a $\$ 2.4$ million five-year noncallable bond issue and 1 million common shares. The following table includes other information that Duarte has gathered:

Currently outstanding bonds $\$ 2.4$ million five-year bonds, coupon of $12.5 \%$ paying semi-annually with a market value of $\$ 2.156$ million

Risk-free rate of interest $5.25 \%$

Estimated equity risk premium $\quad 7 \%$

Tax rate $23 \%$

\begin{enumerate}
  \setcounter{enumi}{19}
  \item The asset betas for Relevant, ABJ, and Opus, respectively, are:
A. $1.70,2.52$, and 2.73 .
B. $1.70,2.79$, and 3.37 .
C. $1.70,2.81$, and 3.44 .

  \item The average asset beta for comparable players in this industry, Relevant, ABJ, and Opus, weighted by market value of equity is closest to:
A. 1.67 .
B. 1.97 .
C. 2.27 .

  \item Using the capital asset pricing model, the cost of equity capital for a company in this industry with a debt-to-equity ratio of 0.01 , an asset beta of 2.27 , and a marginal tax rate of $23 \%$ is closest to:
A. $17 \%$.
B. $21 \%$.
C. $24 \%$.

  \item The marginal cost of capital for TagOn, based on an average asset beta of 2.27 for the industry and assuming that new stock can be issued at $\$ 8$ per share, is closest to:
A. $20.5 \%$.
B. $21.0 \%$. C. $21.5 \%$.

  \item An analyst gathered the following information about a private company and its publicly traded competitor:

\end{enumerate}

\begin{center}
\begin{tabular}{lccc}
\hline
Comparable Companies & Tax Rate (\%) & Debt/Equity & Equity Beta \\
\hline
Private company & 30.0 & 1.00 & na \\
Public company & 35.0 & 0.90 & 1.75 \\
\hline
\end{tabular}
\end{center}

The estimated equity beta for the private company is closest to:
A. 1.029
B. 1.104 .
C. 1.877 .

\begin{enumerate}
  \setcounter{enumi}{24}
  \item Which of the following statements is most accurate? If two equity issues have the same market risk but the first issue has higher leverage, greater liquidity, and a higher required return, the higher required return is most likely the result of the first issue's:
A. greater liquidity.
B. higher leverage.
C. higher leverage and greater liquidity.

  \item SebCoe plc, a British firm, is evaluating an investment in a $\pounds 50$ million project that will be financed with $50 \%$ debt and $50 \%$ equity. Management has already determined that the NPV of this project is $\pounds 5$ million if it uses internally generated equity. However, if the company uses external equity, it will incur flotation costs of $5.8 \%$. Assuming flotation costs are not tax deductible, the NPV using external equity would be:

\end{enumerate}

A. less than $\pounds 5$ million because we would discount the cash flows using a higher weighted average cost of capital that reflects the flotation costs.

B. $\pounds 3.55$ million because flotation costs reduce NPV by $\$ 1.45$ million.

C. $\pounds 5$ million because flotation costs have no impact on NPV.

\section{SOLUTIONS}
\begin{enumerate}
  \item B is correct.
\end{enumerate}

$r_{e}=0.0425+(1.3)(0.0482)=0.1052$, or $10.52 \%$.

\begin{enumerate}
  \setcounter{enumi}{1}
  \item B is correct.
\end{enumerate}

$\mathrm{WACC}=[(€ 900 / € 3300)(0.0925)(1-0.375)]+[(€ 2,400 / € 3,300)(0.1052)]$

$=0.0923$, or $9.23 \%$.

\begin{enumerate}
  \setcounter{enumi}{2}
  \item A is correct.
\end{enumerate}

Asset beta $=$ Unlevered beta $=1.3 /\{1+[(1-0.375)(€ 900 / € 2,400)]\}=1.053$.

\begin{enumerate}
  \setcounter{enumi}{3}
  \item $\mathrm{C}$ is correct.
\end{enumerate}

Project beta $=1.053\{1+[(1-0.375)(€ 80 / € 20)]\}=1.053(3.5)=3.686$.

\begin{enumerate}
  \setcounter{enumi}{4}
  \item $\mathrm{C}$ is correct.
\end{enumerate}

$r_{e}=0.0425+3.686(0.0482+0.0188)=0.2895$, or $28.95 \%$.

\begin{enumerate}
  \setcounter{enumi}{5}
  \item B is correct. Debt is generally less costly than preferred or common stock. The cost of debt is further reduced if interest expense is tax deductible.

  \item B is correct. The weighted average cost of capital, using weights derived from the current capital structure, is the best estimate of the cost of capital for the average-risk project of a company.

  \item $\mathrm{C}$ is correct. McClure should use the forecasted or target market values to calculate the weights.

\end{enumerate}

$w_{d}=\$ 63 /(\$ 220+63)=0.223$.

$w_{e}=\$ 220 /(\$ 220+63)=0.777$.

\begin{enumerate}
  \setcounter{enumi}{8}
  \item The company's WACC is $13.64 \%$, calculated as follows:
\end{enumerate}

\begin{center}
\begin{tabular}{|c|c|c|c|c|}
\hline
 & Equity &  & Debt & WACC \\
\hline
Weight & 0.80 &  & 0.20 &  \\
\hline
After-Tax Cost & $15.6 \%$ &  & $(1-0.30) 8.28 \%$ &  \\
\hline
Weight $\times$ Cost & $12.48 \%$ & + & $1.16 \%$ & $=13.64 \%$ \\
\hline
\end{tabular}
\end{center}

\begin{enumerate}
  \setcounter{enumi}{9}
  \item B is correct. The cost of equity is defined as the rate of return required by stockholders.

  \item $C$ is correct. $F V=\$ 1,000, P M T=\$ 40, N=10$, and $P V=\$ 900$.

\end{enumerate}

Solve for $i$. The six-month yield, $i$, is $5.3149 \%$.

$\mathrm{YTM}=5.3149 \% \times 2=10.62985 \%$

$r_{d}(1-t)=10.62985 \%(1-0.38)=6.5905 \%$. 12. $\mathrm{C}$ is correct. The bond rating approach depends on knowledge of the company's rating and can be compared with yields on bonds in the public market.

\begin{enumerate}
  \setcounter{enumi}{12}
  \item B is correct. The company can issue preferred stock at $6.5 \%$. Therefore, the calculation of the preferred stock's current value is
\end{enumerate}

$P_{p}=\$ 1.75 / 0.065=\$ 26.92$.

\begin{enumerate}
  \setcounter{enumi}{13}
  \item A is correct. The relevant cost is the marginal cost of debt. The before-tax marginal cost of debt can be estimated by the yield to maturity of the company's expected new issue, which is $7 \%$. After adjusting for tax, the after-tax cost is $7 \%(1$ $-0.4)=7 \%(0.6)=4.2 \%$.

  \item For JPMorgan Chase, the required return is

\end{enumerate}

$r=R_{F}+\beta\left[\mathrm{E}\left(R_{M}\right)-R_{F}\right]=4.35 \%+1.50(8.04 \%)=4.35 \%+12.06 \%$ $=16.41 \%$.

For Boeing, the required return is

$r=R_{F}+\beta\left[\mathrm{E}\left(R_{M}\right)-R_{F}\right]=4.35 \%+0.80(8.04 \%)=4.35 \%+6.43 \%$

$=10.78 \%$.

\begin{enumerate}
  \setcounter{enumi}{15}
  \item A. The required return is given by
\end{enumerate}

$r=0.025+(-0.2)(0.045)=2.5 \%-0.9 \%=1.6 \%$.

This example indicates that Newmont Mining has a required return of $1.6 \%$.

When beta is negative, the CAPM calculation yields a required rate of return that is below the risk-free rate, which is arguably not meaningful. Cases of equities with negative betas are relatively rare.

B. The fact that the NEM's cost of debt is higher than the calculated required return on equity is another indicator that the return estimated using CAPM is not useful for valuing the company's equity.

\begin{enumerate}
  \setcounter{enumi}{16}
  \item B is correct. Asset risk does not change with a higher debt-to-equity ratio. Equity risk rises with higher debt.

  \item B is correct. The debt-to-equity ratio of the new product should be used when making the adjustment from the asset beta, derived from the comparables, to the equity beta of the new product.

  \item B is correct. The capital structure is as follows:

\end{enumerate}

Market value of debt: $F V=\$ 10,000,000, P M T=\$ 400,000, N=10$, and $I / Y R=$ 6.825\%. Solving for $P V$ gives $\$ 7,999,688$.

Market value of equity: 1.2 million shares outstanding at $\$ 10=\$ 12,000,000$.

\begin{center}
\begin{tabular}{lrr}
\hline
Market value of debt & $\$ 7,999,688$ & $40 \%$ \\
\hline
Market value of equity & $12,000,000$ & $60 \%$ \\
\cline { 2 - 3 }
Total capital & $\$ 19,999,688$ & $100 \%$ \\
\hline
\end{tabular}
\end{center}

To raise $\$ 7.5$ million of new capital while maintaining the same capital structure, the company would issue $\$ 7.5$ million $\times 40 \%=\$ 3.0$ million in bonds, which results in a before-tax rate of $16 \%$.

$r_{d}(1-t)=0.16(1-0.3)=0.112$, or $11.2 \%$.

$r_{e}=0.03+2.2(0.10-0.03)=0.184$, or $18.4 \%$. $\mathrm{WACC}=0.40(0.112)+0.6(0.184)=0.0448+0.1104=0.1552$, or $15.52 \%$.

\begin{enumerate}
  \setcounter{enumi}{19}
  \item B is correct.
\end{enumerate}

Asset beta: $\beta_{\text {equity }} /[1+(1-t)(D / E)]$

Relevant $=1.702 /[1+(0.77)(0)]=1.702$

$\mathrm{ABJ}=2.8 /[1+(0.77)(0.003)]=2.7918$

Opus $=3.4 /[1+(0.77)(0.013)]=3.3663$.

\begin{enumerate}
  \setcounter{enumi}{20}
  \item C is correct.
\end{enumerate}

Weights are determined on the basis of relative market values:

\begin{center}
\begin{tabular}{lcc}
\hline
Comparables & $\begin{array}{c}\text { Market Value of } \\ \text { Equity in Millions }\end{array}$ & $\begin{array}{c}\text { Proportion of } \\ \text { Total }\end{array}$ \\
\hline
Relevant & $\$ 3,800$ & 0.5490 \\
\hline
ABJ & 2,150 & 0.3106 \\
Opus & 972 & 0.1404 \\
Total & $\$ 6,922$ & 1.0000 \\
\hline
\end{tabular}
\end{center}

Weighted average beta $=(0.5490)(1.702)+(0.3106)(2.7918)+(0.1404)(3.3572)$ $=2.27$.

\begin{enumerate}
  \setcounter{enumi}{21}
  \item B is correct.
\end{enumerate}

Asset beta $=2.27$

Levered beta $=2.27[1+(1-0.23)(0.01)]=2.2875$.

Cost of equity capital $=0.0525+(2.2875)(0.07)=0.2126$, or $21.26 \%$.

\begin{enumerate}
  \setcounter{enumi}{22}
  \item $\mathrm{C}$ is correct.
\end{enumerate}

For debt: $F V=2,400,000 ; P V=2,156,000 ; n=10 ; P M T=150,000$.

Solve for $i: i=0.07748$. $\mathrm{YTM}=15.5 \%$.

Before-tax cost of debt $=15.5 \%$.

Market value of equity $=1$ million shares outstanding +1 million newly issued shares $=2$ million shares at $\$ 8$

$=\$ 16$ million.

Total market capitalization $=\$ 2.156$ million $+\$ 16$ million $=\$ 18.156$ million.

Levered beta $=2.27[1+(1-0.23)(2.156 / 16)]=2.27(1.1038)=2.5055$.

Cost of equity $=0.0525+2.5055(0.07)=0.2279$, or $22.79 \%$.

Debt weight $=\$ 2.156 / \$ 18.156=0.1187$ Equity weight $=\$ 16 / \$ 18.156=0.8813$.

$$
\begin{aligned}
& \text { TagOn's MCC }=(0.1187)(0.155)(1-0.23)+(0.8813)(0.2279) \\
& =0.01417+0.20084 \\
& =0.2150 \text {. or } 21.50 \% .
\end{aligned}
$$

\begin{enumerate}
  \setcounter{enumi}{23}
  \item $\mathrm{C}$ is correct. Inferring the asset beta for the public company: Unlevered beta $=$ $1.75 /[1+(1-0.35)(0.90)]=1.104$. Re-levering to reflect the target debt ratio of the private firm: Levered beta $=1.104 \times[1+(1-0.30)(1.00)]=1.877$.

  \item B is correct. All else equal, the first issue's greater liquidity would tend to make its required return lower than the second issue's. However, the required return on equity increases as leverage increases. The first issue's higher required return must result from its higher leverage, more than offsetting the effect of its greater liquidity, given that both issues have the same market risk.

  \item B is correct. Since the project will be financed with $50 \%$ equity, the company will issue $\pounds 25$ million of new stock. The flotation cost of external equity is $(0.058 \times$ $25,000,000)=1,450,000$. The NPV of the project using external equity is the NPV using internal equity less the flotation cost. Adjusting the cost of capital to reflect the flotation cost is not a preferred way to account for flotation costs.

\end{enumerate}

\section{LEARNING MODULE 2}
\section{Capital Structure}
by Raj Aggarwal, PhD, CFA, Glen D. Campbell, MBA, Lee M. Dunham, PhD, CFA, Pamela Peterson Drake, PhD, CFA, and Adam Kobor, PhD, CFA.

Raj Aggarwal, PhD, CFA, is at the Kent State University Foundation Board (USA). Glen D. Campbell, MBA (Canada). Lee M. Dunham, PhD, CFA, is at Creighton University (USA). Pamela Peterson Drake, PhD, CFA, is at James Madison University (USA). Adam Kobor, PhD, CFA, is at New York University (USA).

\section{LEARNING OUTCOME}
\begin{center}
\begin{tabular}{|c|c|}
\hline
Mastery & The candidate should be able to: \\
\hline
$\square$ & explain factors affecting capital structure \\
\hline
$\square$ & $\begin{array}{l}\text { describe how a company's capital structure may change over its life } \\ \text { cycle }\end{array}$ \\
\hline
$\square$ & $\begin{array}{l}\text { explain the Modigliani-Miller propositions regarding capital } \\ \text { structure }\end{array}$ \\
\hline
$\square$ & $\begin{array}{l}\text { describe the use of target capital structure in estimating WACC, and } \\ \text { calculate and interpret target capital structure weights }\end{array}$ \\
\hline
$\square$ & $\begin{array}{l}\text { describe competing stakeholder interests in capital structure } \\ \text { decisions }\end{array}$ \\
\hline
\end{tabular}
\end{center}

\section{INTRODUCTION}
Capital structure refers to the specific mix of debt and equity used to finance a company's assets and operations. From a corporate perspective, equity represents a more expensive, permanent source of capital with greater financial flexibility. Financial flexibility allows a company to raise capital on reasonable terms when capital is needed. Debt, on the other hand, represents a cheaper, finite-to-maturity capital source that legally obligates a company to make promised cash outflows on a fixed schedule with the need to refinance at some future date at an unknown cost.

As we will show, debt is an important component in the "optimal" capital structure. The trade-off theory of capital structure tells us that managers should seek an optimal mix of equity and debt that minimizes the firm's weighted average cost of capital, which in turn maximizes company value. That optimal capital structure represents a trade-off between the cost-effectiveness of borrowing relative to the higher cost of equity and the costs of financial distress. In reality, many practical considerations affect capital structure and the use of leverage by companies, leading to wide variation in capital structures even among otherwise similar companies. Practical considerations impacting capital structure include:

\begin{itemize}
  \item business characteristics: features associated with a company's business model, operations, or maturity;

  \item capital structure policies and leverage targets: guidelines set by management and the board that seek to establish sensible borrowing limits for the company based on the company's risk appetite and ability to support debt; and

  \item market conditions: current share price levels and market interest rates for a company's debt. The prevalence of low interest rates increases the debt-carrying capacity of businesses and the use of debt by companies.

\end{itemize}

Since we are considering how a company minimizes its overall cost of capital, the focus here is on the market values of debt and equity. Therefore, capital structure is also affected by changes in the market value of a company's securities over time.

While we tend to think of capital structure as the result of a conscious decision by management, it is not that simple. For example, unmanageable debt, or financial distress, can arise because a company's capital structure policy was too aggressive, but it can also occur because operating results or prospects deteriorate unexpectedly.

Finally, in seeking to maximize shareholder value, company management may make capital structure decisions that are not in the interests of other stakeholders, such as debtholders, suppliers, customers, or employees.

\section{FACTORS AFFECTING CAPITAL STRUCTURE}
$$
\text { explain factors affecting capital structure }
$$

Many factors influence a company's capital structure and its ability to support debt, including the nature and stability of its business, its maturity, capital intensity, and the strength of its market position. In addition, there is significant variation in capital structures across industries. What causes the capital structures of companies to differ, in some cases, so significantly from one another?

The primary factors, both internal and external, that affect a company's capital structure and differences in capital structures across companies and industries are summarized in Exhibit 1 . Note that the discussion on company maturity, or life-cycle stage, is covered in a following section.

\section{Exhibit 1: Key Determinants of Capital Structure}
\begin{center}
\includegraphics[max width=\textwidth]{2023_05_04_7b535d0a870224f62e3dg-055}
\end{center}

\section{Internal Factors Affecting Capital Structure}
\section{Business model characteristics}
The risk inherent in a company's business model can greatly impact the company's capital structure by influencing its ability to service debt. Key factors that differ among business models include differences in:

\begin{itemize}
  \item revenue, earnings, and cash flow sensitivity,

  \item asset type, and

  \item asset ownership.

\end{itemize}

\section{Revenue, earnings, and cash flow sensitivity}
Some companies, such as Vodafone in the telecom industry or Microsoft in the software industry, generally have very stable revenue streams resulting from a large proportion of their revenues being subscription-like, recurring revenues. A high proportion of recurring revenues for a company is generally viewed as a positive for its ability to support debt, because the company's revenue stream is likely to be more predictable and less sensitive to the ups and downs of the macro economy.

In contrast, companies in cyclical industries, such as Toyota in the automobile industry and Komatsu in the construction equipment industry, typically have more volatile revenue streams that are highly sensitive to the macroeconomic environment. Revenue streams subject to relatively high volatility, and consequently less predictability, are less favorable for supporting debt in the capital structure. Further, companies with pay-per-use business models, rather than subscription-based models, are likely to have a lower degree of revenue predictability and a lower ability to support debt in the capital structure.

Stable revenue streams can also lead to earnings and cash flow streams being relatively more stable. However, the company's cost structure-the proportions of fixed costs and variable costs-will impact the degree of stability and predictability in earnings and cash flows.

One measure of business risk is a company's degree of operating leverage, measured as the proportion of fixed costs to total costs:

Operating leverage $=$ Fixed costs $/$ Total costs

Companies with higher operating leverage experience a greater change in earnings and cash flows for a given change in revenue than firms with low operating leverage. Companies with low volatility in their revenue, earnings, and cash flow streams can support higher levels of debt in their capital structures than firms with high volatility in these streams. For a given level of debt, firms with low revenue, earnings, and cash flow volatility are likely to have a lower probability of default and be able to access debt at lower cost than firms with high volatility in these streams.

Exhibit 2 presents a summary of the relationships between these business model characteristics and a firm's ability to support debt.

Exhibit 2: Relationships between Business Model Factors and Ability to Support Debt

\begin{center}
\begin{tabular}{ll}
\hline
Business Model Factor & Ability to Support Debt \\
\hline
High (low) revenue, cash flow volatility & Reduced (increased) \\
High (low) earnings predictability & Increased (reduced) \\
High (low) operating leverage & Reduced (increased) \\
\hline
\end{tabular}
\end{center}

\section{Asset type}
A company's assets can be broadly categorized as tangible or intangible, fungible or non-fungible, and liquid or illiquid. Tangible assets are identifiable, physical assets like property, plant and equipment, inventory, cash, and marketable securities, whereas intangible assets do not exist in physical form, such as goodwill and patents/intellectual property rights. It should be noted that under most accounting standards, intangible assets on the balance sheet reflect only acquired intangible assets, such as goodwill, and not internally generated intangible assets.

Generally speaking, assets supporting the use of debt include those that are typically considered strong collateral by creditors, cash generative, and relatively easy to market (i.e., liquid). From a creditor's perspective, tangible assets are deemed safer than intangible assets and can better serve as debt collateral. Therefore, companies with mostly tangible assets, such as those in the oil and gas, infrastructure, and real estate industries, are likely to be able to operate with greater amounts of debt in their capital structures than companies with a high proportion of intangible assets, such as those in the software industry, due to their ability to pledge such assets as collateral for debt.

Fungible assets are those assets that are interchangeable with, or substitutable for, another of similar identity. Money, for example, represents a fungible asset. In contrast, non-fungible assets are unique assets, such as art pieces, that are not mutually interchangeable or substitutable. Asset liquidity refers to the ability to convert an asset into cash without losing a substantial amount of its value. Marketable securities are very liquid, given the ease with which they can be converted into cash. On the other hand, real estate and property as well as plant and equipment are fairly illiquid assets that are not easily convertible into cash at their market values. Companies with mostly fungible and highly liquid assets are likely to support greater debt capacity than firms with mostly non-fungible, illiquid assets. That said, illiquid assets can serve as debt collateral given their tangible nature. Exhibit 3 presents a summary of the relationships between a firm's asset type and its cost of capital ability/ability to support debt.

\section{Exhibit 3: Relationships between Asset Type and Ability to Support Debt}
\begin{center}
\begin{tabular}{lc}
\hline
Asset Type & Ability to Support Debt \\
\hline
$\begin{array}{l}\text { Greater (fewer) fungible, tangible, liquid, or market- } \\ \text { able assets }\end{array}$ & Increased (reduced) \\
\hline
\end{tabular}
\end{center}

\section{Asset ownership}
For some companies, management may choose not to own assets but instead "outsource" asset ownership to other parties, thereby reducing balance sheet assets. Outsourcing of assets can allow the company to move to a variable cost structure, resulting in lower business risk as measured by lower operating leverage. By minimizing the costs and risks associated with owning and operating a significant amount of fixed assets, these "asset-light" companies can maximize their flexibility and ability to scale quickly.

For example, Uber is a global taxi company that does not actually own its fleet of automobiles, and Airbnb is a global accommodations company that does not actually own the underlying real estate. Parent companies of major restaurant chains like McDonald's that choose to sell some franchises to individuals rather than own all of their physical locations also fit into this classification of asset-light companies. Companies like Uber, Airbnb, and McDonald's still benefit from these tangible assets by having control over them - they just do not own them.

In one respect, the lower asset base allows these asset-light companies to operate with lower operating leverage, which, all else equal, can support high debt capacities. On the other hand, asset-light companies often have a lower proportion of tangible assets that better serve as collateral for debt. This characteristic of asset-light companies suggests a lower ability to support debt in the capital structure than their asset-heavy counterparts, such as real estate and hotel companies, that own the underlying tangible assets.

\section{EXAMPLE 1}
\section{Business Model Characteristics and Capital Structure}
An analyst is reviewing the business characteristics of two companies in different industries to assess their ability to support debt in their capital structures. The analyst gathers the following common-size balance sheet information and other information on the two companies:

\begin{center}
\begin{tabular}{lcc}
\hline
 & Company A & Company B \\
\hline
Common-size balance sheet &  &  \\
Cash and equivalents & $7 \%$ & $2 \%$ \\
Accounts receivable & $8 \%$ & $2 \%$ \\
Inventory & $1 \%$ & $1 \%$ \\
Other current assets & $2 \%$ & $0 \%$ \\
 &  &  \\
Property, plant, and equipment (net) & $30 \%$ & $72 \%$ \\
Operating leases-right to use assets & $7 \%$ & $18 \%$ \\
Intangible assets and goodwill & $41 \%$ & $2 \%$ \\
Other assets & $4 \%$ & $43 \%$ \\
 &  &  \\
\end{tabular}
\end{center}

\begin{center}
\begin{tabular}{lcc}
\hline
 & Company A & Company B \\
\hline
Other Selected Information: &  &  \\
\% of total revenue that is recurring & $10 \%$ & $90 \%$ \\
Operating leverage & Low & High \\
\hline
\end{tabular}
\end{center}

\begin{enumerate}
  \item Discuss two characteristics of either company that would support a relatively low proportion of debt in its capital structure. Discuss one characteristic of either company that would support a relatively high proportion of debt in its capital structure.
\end{enumerate}

\section{Solution}
Two characteristics of Company A that would support a relatively low proportion of debt are (1) its large proportion of intangible assets and (2) its low proportion of recurring revenues.

First, from a creditor's perspective, tangible assets are often deemed safer than intangible assets and can better serve as debt collateral. Therefore, companies like Company A with a high proportion of intangible assets (41\%) have less ability to pledge such assets as collateral for debt. Another $7 \%$ of assets are leased, and these assets are also unlikely to be usable as debt collateral. Second, companies with a low percentage of recurring revenues (10\% for Company A) are likely to have a lower degree of revenue predictability and a relatively lower ability to support debt in the capital structure. In contrast, a large majority of Company B's assets (72\%) are tangible property, plant, and equipment and most of the company's revenues are recurring (90\%), suggesting a subscription-based business model with more predictable revenues, earnings streams, and cash flow streams. Companies like Company B are likely to be able to support greater amounts of debt. One characteristic of Company A that would support a relatively high proportion of debt is its low operating leverage. Companies with lower variability in earnings and cash flows can achieve greater results with a higher ability to support debt in their capital structures. In contrast, Company B operates with much higher operating leverage and would likely experience greater earnings and cash flow volatility for a given change in revenue.

\section{Existing leverage}
When a company elects to raise capital, whether it be debt or equity, the cost of that capital is highly dependent on the firm's existing financial leverage (debt level relative to total assets or total equity) and capital structure. A company's debt capacity is the total amount of debt that the company can take on and repay without causing insolvency. In general, firms with higher proportions of debt in their capital structures face a higher probability of default and have less ability to service additional debt than underleveraged firms.

If a company wants to raise debt capital, an important question is whether the company can service the additional debt. An analysis of a few financial ratios can help answer this question. Commonly used ratios for this exercise are presented in Exhibit 4 .

\section{Exhibit 4: Metrics Used to Assess a Company's Ability to Service Debt}
\begin{center}
\begin{tabular}{|c|c|c|c|}
\hline
Liquidity & Profitability & Leverage & Interest Coverage \\
\hline
\multirow[t]{3}{*}{$\begin{array}{l}\text { Current assets/ } \\
\text { Current liabilities }\end{array}$} & EBIT/Revenue & $\begin{array}{c}\text { Total debt (or net } \\ \text { debt)/ EBITDA }\end{array}$ & EBIT/Interest expense \\
\hline
 & EBITDA/Revenue & $\begin{array}{l}\text { Total debt (or net } \\ \text { debt)/ Total assets } \\ \text { (or total equity) }\end{array}$ & $\begin{array}{c}(\text { EBIT }+ \text { lease payments) } \\ \text { (Interest expense }+ \text { lease } \\ \text { payments) }\end{array}$ \\
\hline
 & $\begin{array}{c}\text { Operating income/ } \\ \text { Revenue }\end{array}$ &  & $\begin{array}{c}\text { EBITDA/Interest } \\ \text { expense }\end{array}$ \\
\hline
\end{tabular}
\end{center}

Notes:

Net debt = Interest-bearing debt - cash and cash equivalents

$\mathrm{EBIT}=$ Net income + interest expense + taxes $=$ EBITDA - depreciation and amortization expenses

Operating income $=$ Operating revenue - operating expenses $=$ EBIT non-operating profit + non-operating expenses

A starting point for determining a company's debt capacity is its current ratio, equal to current assets divided by current liabilities. The current ratio provides an indication of the ability of the firm to meet its short-term debt obligations. The higher the ratio, the greater the ability of the company to repay its debt. Similarly, a higher level of profitability-usually measured in terms of operating income or EBIT or EBITDA divided by revenue-implies greater amounts of earnings available to service the company's debt obligations. In general, companies with higher liquidity and profitability have a greater ability to support greater use of debt in their capital structures.

A common leverage ratio used by analysts to assess a firm's debt capacity is the ratio of total (or net) debt to EBITDA. This ratio provides an estimate of how many years it would take to pay off the company's debt. In general, ratio values of 3 or less are considered acceptable; values higher than 3 start to raise concerns about the company's ability to service its debt obligations. Finally, interest coverage ratios are also commonly used to assess companies' debt capacities. Generally, these ratios provide an estimate of how many times a company can cover its interest expense (or interest expense plus lease payments) with current earnings (usually measured as EBIT or EBITDA). In other words, interest coverage ratios provide an indication of a company's financial cushion in meeting its debt service obligations. The larger the interest coverage ratio, the larger the financial cushion and the greater the company's ability to service its debt obligations.

For example, an interest coverage ratio of 4 would indicate that the company's earnings could fall by as much as $75 \%$ before the company would be unable to meet interest payments and/or lease obligations. In general, an interest coverage ratio of 2 is deemed the minimum acceptable level, as values less than 2 imply little financial cushion and cast doubt on the company's ability to service its debt obligations.

Exhibit 5 presents a summary of these relationships.

Exhibit 5: Relationships between Selected Financial Ratio Types and Ability to Support Debt

Financial Ratio Type

Higher (lower) liquidity Ability to Support Debt

Increased (reduced)

\begin{center}
\begin{tabular}{ll}
\hline
Financial Ratio Type & Ability to Support Debt \\
\hline
Higher (lower) profitability & Increased (reduced) \\
Higher (lower) leverage & Reduced (increased) \\
Higher (lower) interest coverage & Increased (reduced) \\
\hline
\end{tabular}
\end{center}

\section{Corporate tax rate}
Another important factor in the determination of a firm's capital structure is its marginal income tax rate. In many countries and jurisdictions, interest expense is a tax-deductible expense in the income statement. Therefore, a company's after-tax cost of debt will be lower than the actual cost due to this tax savings. The higher (lower) the firm's marginal income tax rate, the greater (lower) the tax benefit of using debt in the firm's capital structure.

\section{EXAMPLE 2}
\section{Changes in the US Corporate Tax Rate}
In 2017, the US corporate income tax system changed from a progressive tax rate ranging from $15 \%$ to $35 \%$ (depending on taxable income) to a flat rate of $21 \%$.

\begin{enumerate}
  \item Discuss the implications of this change in the tax system on the capital structures of companies subject to corporate tax rates well above $21 \%$ before the change.
\end{enumerate}

\section{Solution}
Companies subject to income tax rates well above $21 \%$ before the change benefited from a higher tax savings per dollar of interest expense after the change. For firms facing the top tax rate of $35 \%$, the change in the tax system reduced the tax savings associated with debt capital from USD0.35 per dollar of interest expense to USD0.21. The reduction in the tax rate for these companies had the effect of raising the after-tax cost of debt, thereby making debt capital less attractive than before the change. Despite this increase in the after-tax cost of debt making debt capital more costly, it is important to note that these companies would have benefited from the tax rate reduction by paying less income tax on their pre-tax earnings.

\section{Capital structure policies/guidelines}
The capital structures of some companies are often also guided and influenced by firm-specific policies relating to financing decisions. Given the difficulty of estimating a company's cost of equity capital or potential costs of financial distress, it is common for companies to establish capital structure policies that are debt-oriented, defining acceptable levels of debt in the capital structure. Such policies might contain directives such as "debt/equity less than 0.5 times"; "debt to operating cash flow less than 2.0 times"; or "debt as a maximum of $\mathrm{X} \%$ of total capital."

Companies differ in the measures and thresholds used, but they frequently start with debt covenants or rating agency thresholds to which they add a cushion, or "margin of safety," to ensure that the debt limits or covenants are not breached or to avoid a rating downgrade.

Having explicit capital structure policies is important for companies that are significant and frequent borrowers, since the risks associated with too much leverage can be severe while the risks associated with too little leverage generally are not. Companies with little to no debt, such as start-ups or some technology companies, may not have capital structure policies or targets. In certain industries such as banking and utilities, capital structures are evaluated by regulators, and thus capital structure policies are determined accordingly.

Another important consideration for corporate issuers is whether their debt or equity issuances meet index provider requirements that allow inclusion in a benchmark index, as such inclusion affects the demand for their securities by major investors. Companies reaching a certain market capitalization size in addition to meeting specific index provider requirements may qualify for index membership. Similarly, whether a specific debt issue meets the requirements-including issuance size, in particular-for inclusion in an index is a key factor in debt-financing decisions.

\section{EXAMPLE 3}
\section{Capital Structure Policy: Abha Software}
\begin{enumerate}
  \item Abha Software's management believes the company can obtain attractive borrowing terms if debt/EBITDA (earnings before interest, taxes, depreciation, and amortization) is at 2 times or less, which for Abha is the primary criterion used by lenders to assess borrowing capacity and terms. Abha's EBITDA for the latest year is USD50 million, and it has a total capitalization of USD1.0 billion. Management and the board have agreed that Abha should use leverage if the company can borrow easily at attractive rates. They are seeking a capital structure policy that reflects this objective. Which of the following is most likely to be an appropriate capital structure policy for Abha?
\end{enumerate}

A. Debt should be a maximum of $25 \%$ of total capital in book value terms.

B. Debt should be a maximum of $10 \%$ of total capital in market value terms.

C. The debt-to-EBITDA ratio should be no more than 2 times.

\section{Solution}
$\mathrm{C}$ is correct. It is the most appropriate capital structure policy, since the debt-to-EBITDA ratio is the primary criterion that lenders and management are using to determine Abha's debt capacity. Note that lending criteria vary by sector, by market, and over time, including the ratios used (as well as the threshold ratios for those ratios).

B is least likely to be appropriate. While useful in calculating the cost of capital for Abha, market value weights are less practical and less commonly used in capital structure targets. Market values of equity can fluctuate a great deal without a corresponding change in the company's borrowing capacity. Increases or decreases in the company's share price would imply increases or decreases in maximum debt that are unlikely to be appropriate. A's policy, with 25\% debt/capital, implies up to USD250 million in debt. This ratio, unlike B's policy, is unaffected by fluctuations in the share price. However, the implied debt capacity of USD250 million and 5.0 times debt/ EBITDA (USD250 million/USD50 million) is greater than the cap of 2 times established by management and identified as the primary criterion by Abha's lenders.

Finally, a company's cash flow generation is almost never entirely predictable and is likely to deviate from management's plan, which will affect the company's borrowing capacity and its financing plans. As management consider their financing plans, they will continually refine their views on the company's cash flow generation as well as its investment needs.

\section{Third-Party Debt Ratings}
Debt ratings, also called credit ratings, are independent, third-party measures of the quality and safety of a company's debt based on an analysis of the company's ability to pay the promised cash flows. Debt ratings are an important consideration in the practical management of leverage and a primary factor underlying a company's capital structure policies. For example, maintaining the company's rating at a certain level, such as investment-grade or above, may be an explicit policy target for management.

Most large companies pay one or more rating services to provide debt ratings for their bonds. Debt issues are rated for creditworthiness by these credit-rating agencies after they perform a financial analysis of the company's ability to pay the promised cash flows as well as an analysis of the bond's indenture (i.e., the set of complex legal documents associated with the issuance of debt instruments). These agencies evaluate a wealth of information about the issuer and the bond, including the bond's characteristics and indenture agreement, and provide investors with an assessment of the company's ability to pay the interest and principal on the bond as promised. Both the issuer, or company, and the issuance, or security, are rated by these agencies.

As leverage rises, rating agencies tend to lower the ratings of the company's debt to reflect the higher credit risk and probability of default arising from the increase in leverage. Lower ratings signify higher risk to both equity and debt capital providers, who demand higher returns for supplying capital to the company.

Most managers consider the company's debt rating in their capital structure policies because the cost of capital is tied so closely to bond ratings. Some companies explicitly target a certain debt rating in their capital structure policies. For example, a company might target an S\&P debt rating of A or higher, which is one level above BBB, the minimum investment-grade rating on S\&P's scale. The cost of debt increases significantly when a bond's rating drops below investment-grade. In economic recessions, these spreads may widen significantly, and borrowing can become more difficult for companies with non-investment-grade ratings.

Firms that have more stable revenues, earnings, and cash flows as well as greater profitability, a higher proportion of tangible assets, shorter asset conversion cycles, and/ or more liquid assets can support more debt and tend to receive higher debt ratings and have lower borrowing costs. These companies are often in stable and defensive sectors, such as utilities and consumer staples, and enjoy strong market positions.

Companies with a high proportion of intangible assets require relatively heavy investment in inventory or specialized equipment-with long asset conversion cycles and cyclical or less stable cash flows and/or profitability-have a reduced ability to consistently meet fixed debt payments over time, and thus face lower ratings and higher borrowing costs.

Highly rated companies have greater borrowing flexibility, better terms, and lower borrowing costs. They can borrow in debt markets for "general corporate purposes" on an unsecured basis, with very few debt covenants. Similarly, lower-rated, non-investment-grade companies have higher borrowing costs, more restrictive debt covenants, liens on assets, and differing subordination levels.

\section{External Factors Affecting Capital Structure}
\section{Market conditions/business cycle}
A company's capital structure is highly influenced by market conditions-namely, interest rates and the current macroeconomic environment. A company's use of debt capital is driven by its costs, which are driven by interest rates and credit spreads. A company's cost of debt is equal to a benchmark risk-free rate $\left({ }^{f}\right)$ plus a credit spread specific to the company.

The credit spread above the benchmark risk-free rate reflects issuer-specific risks that influence the company's probability of default, such as business model characteristics, earnings predictability, and the company's existing debt level. The greater these company-specific risks, the higher the credit spread and the cost of debt capital.

Macroeconomic and country-specific factors are also reflected in the benchmark rate and in the overall level of credit spreads, which widen during recessions and tighten during expansionary times. When borrowing is less costly due to low interest rates and/or tight credit spreads, companies may increase their use of debt and vice versa. The macroeconomic conditions over the longer term-as measured by business cycles-may also affect firms' capital structures. Some firms tend to borrow more during expansionary times as credit spreads tighten and borrow less during recessionary times as credit spreads widen.

In cyclical sectors, such as mining and materials and many industrials, revenues and cash flows vary widely through the economic cycle, which limits debt capacity. As a result, businesses in cyclical sectors may have less debt in their capital structures than companies in other, less cyclical industries.

\section{EXAMPLE 4}
\section{Changes in Capital Structure during the COVID-19 Pandemic}
The airline industry was one of the hardest-hit industries during the COVID-19 pandemic. Exhibit 6 presents selected data on leverage-related ratios for American Airlines, Southwest Airlines, and Lufthansa AG for 2018-2020.

\section{Exhibit 6: Selected Leverage Ratios for American Airlines,}
Southwest Airlines, and Lufthansa AG

\begin{center}
\begin{tabular}{llll}
\hline
 & $\mathbf{2 0 1 8}$ & $\mathbf{2 0 1 9}$ & $\mathbf{2 0 2 0}$ \\
\hline
Total debt to total assets &  &  &  \\
American Airlines & 0.43 & 0.43 & 0.55 \\
Southwest Airlines & 0.13 & 0.12 & 0.31 \\
Lufthansa AG & 0.04 & 0.04 & 0.08 \\
Total debt to equity &  &  &  \\
American Airlines & 1.80 & 2.21 &  \\
Southwest Airlines & 0.13 & 0.11 & 0.27 \\
Lufthansa AG & 0.16 & 0.19 & 0.40 \\
EBITDA to interest expense &  &  &  \\
American Airlines & 3.66 & 4.61 & $(7.94)$ \\
\end{tabular}
\end{center}

\begin{center}
\begin{tabular}{llll}
\hline
 & $\mathbf{2 0 1 8}$ & $\mathbf{2 0 1 9}$ & $\mathbf{2 0 2 0}$ \\
\hline
Southwest Airlines & 33.64 & 35.39 & $(7.34)$ \\
Lufthansa AG & 23.61 & 11.33 & $(6.47)$ \\
\hline
\end{tabular}
\end{center}

Notes: Equity is measured as calendar year-end market capitalization. Total debt includes the current portion of long-term debt, operating lease liabilities, and long-term debt.

\begin{enumerate}
  \item Compare and contrast the capital structures of these three companies over 2018-2020.
\end{enumerate}

\section{Solution}
In the two years before the COVID-19 pandemic, which started in early 2020, Southwest Airlines and Lufthansa AG had very little debt in their capital structures, as indicated by very low ratios of total debt to total assets and total debt to equity. As a result, the interest coverage ratio (EBITDA to interest expense) was very high for both companies before the pandemic. In contrast, American Airlines had significantly more debt in its capital structure heading into the pandemic, as indicated by much higher ratios of debt to total assets and total debt to equity. American's interest coverage of less than $5 \times$ heading into the pandemic provided little financial cushion in meeting its debt service obligations.

During the pandemic year of 2020, all three companies experienced a sharp decrease in profitability. EBIT and EBITDA were negative for all three firms in 2020, resulting in negative interest coverage ratios, and leverage ratios significantly worsened for all three companies.

\section{Regulatory constraints}
The capital structures of some firms are regulated by government or other regulators. Some key financial decisions of financial firms, utility firms, and property developers, such as those relating to capital structure, payout policy, and pricing, are often subject to guidelines set by regulatory bodies. For example, financial institutions must generally maintain certain levels of solvency or capital adequacy, as defined by regulators. Similarly, regulatory oversight of public utility companies by local governments can often influence their capital structures through rules and regulations relating to setting pricing/rates.

\section{Industry/peer firm leverage}
The industry in which a firm operates is likely to have a significant effect on its capital structure. It is not uncommon for companies in the same industry to have fairly similar capital structures. One explanation for this result is that companies in the same industry are likely to have common asset types and business model characteristics, among other commonalities. For instance, companies in the automobile industry tend to own large proportions of tangible, non-fungible fixed assets in the form of property, plant, and equipment, with significant proportions of debt in their capital structures.

\section{CAPITAL STRUCTURE AND COMPANY LIFE CYCLE $\square$ $\begin{aligned} & \text { describe how a company's capital structure may change over its life } 
 & \text { cycle }\end{aligned}$}
This section discusses the typical changes in a company's capital structure as it evolves over time, from a start-up to a growth business to a mature business. It will also highlight key variations.

\section{Background}
Many financial transactions can be seen in simple terms as the trading of money across time: A party with surplus cash today provides it to another party requiring cash in exchange for payments in the future. This statement is true of not only everyday personal banking activities but also companies accessing the capital markets over their life cycle, from start-up through maturity. Typically, companies begin life as consumers of capital. As they mature, cash flows turn from negative to positive. Capital markets connect them to investors with the opposite requirement: those with surplus cash to invest today in exchange for future payments.

The framework in Exhibit 7 describes the relationship between a company's life-cycle stage and its cash flow characteristics and ability to support debt. Life-cycle stage is a principal factor in determining capital structure. Capital not sourced through borrowing must come from equity, either from retained earnings or from issuing/selling shares. The framework in Exhibit 7 categorizes the stages in life-cycle development as start-up, growth, and mature and shows typical revenue and cash flow characteristics of companies in each stage. As companies mature, business risk typically declines, and their cash flows turn positive and become increasingly predictable, allowing for greater use of leverage on more attractive (less costly) financing terms. Debt then becomes a larger component of their capital structures.

Note that the framework references cash flow, which is net of investment. Profitable, high-growth businesses may have negative cash flow once investment is taken into account. Investment includes spending on property, plant, and equipment and other fixed assets as well as the expansion of working capital to support and grow the business over time.

Note also that the stages of a company's life cycle-start-up, growth, and maturityare similar to the evolutionary stages of an industry. However, start-ups and growth companies can often be found in mature industries. Examples include restaurants and apparel (with new concepts, formats, and fashions appearing regularly) and technology-driven disruption of established industries, such as advertising (e.g., Google and Facebook) and even automobiles (e.g., Tesla and BYD). This framework is a very general one, and there are wide variations between and within sectors.

\section{Exhibit 7: Capital Structure and Company Life Cycle}
\begin{center}
\includegraphics[max width=\textwidth]{2023_05_04_7b535d0a870224f62e3dg-066}
\end{center}

\begin{center}
\begin{tabular}{|c|c|c|c|}
\hline
Stage in life cycle & Start-up & Growth & Mature \\
\hline
\multicolumn{4}{|c|}{Financial management} \\
\hline
Revenue growth & Beginning & Rising & Slowing \\
\hline
Cash flow & Negative & Improving & Positive/Predictable \\
\hline
Business risk & High & Medium & Low \\
\hline
\multicolumn{4}{|c|}{Debt capital/leverage} \\
\hline
Availability & Very limited & Limited/improving & High \\
\hline
Cost & High & Medium & Low \\
\hline
Typical cases & $\mathrm{N} / \mathrm{A}$ & $\begin{array}{l}\text { Secured (by receivables, } \\ \text { fixed assets) }\end{array}$ & $\begin{array}{l}\text { Unsecured (bank } \\ \text { and public debt) }\end{array}$ \\
\hline
$\begin{array}{l}\text { Typical \% of capital } \\ \text { structure }^{1}\end{array}$ & Close to $0 \%$ & $0 \%-20 \%$ & $20 \%+$ \\
\hline
\end{tabular}
\end{center}

${ }^{1}$ These ratios are calculated based on the market values of equity and debt.

\section{Start-Ups}
Early in its life, a company is typically a cash consumer. Investment is required to advance concepts through the prototype stage and into commercial production. Revenues are zero or minimal, and risk of business failure is high. The company must raise capital, and since the timing and potential for cash flow generation are highly uncertain, it will generally raise equity rather than debt. Equity for start-ups is usually sourced privately (e.g., through venture capital rather than in public markets through an IPO), in part because many stock exchange listing requirements include minimum profitability levels, which most start-ups have yet to achieve.

At this early stage, debt capital is typically not available or available only at a high cost. Most lenders require stable and positive cash flow to service debt and/or collateral to secure it. An early-stage company often has neither, making it a high-risk prospect to lenders.

From the perspective of an early-stage issuer, debt may be an attractive way to reduce or avoid the dilution associated with equity issuance. However, with cash flows that are negative and unpredictable, a typical start-up would have difficulty making regular debt payments, or "servicing the debt." Even if debt is available, the cost, inflexibility, and risk associated with borrowing are often unattractive to a start-up. As a result, debt is a negligible component of the capital structures of most start-up companies.

\section{EXAMPLE 5}
\section{Start-Up Financing}
\begin{enumerate}
  \item Which of the following are limiting factors in the ability of a start-up company to take on debt?
A. Lender requirement for positive cash flows
B. Few to no assets that can be used as collateral
C. Exchange listing requirements for minimum levels of profitability
D. A and B only
E. All of the above
\end{enumerate}

\section{Solution}
D is correct. Most lenders require positive and stable cash flows and/or collateral for servicing debt payments. Start-ups often have neither, given their early stage in development. With expenditures outpacing revenues, a start-up company's cash flow is negative and unpredictable. For equity financing, most start-ups fail to meet minimum profitability requirements for an exchange listing in public markets and thus source their capital needs through private equity markets. Therefore, $\mathrm{C}$ is incorrect, since an exchange listing relates to equity and is not a debt-financing limiting factor.

\section{Growth Businesses}
As a company exits the start-up stage, it is typically generating revenues, providing confirmation of the product concept and evidence of demand. Revenue growth may be rising and/or high, but investment is needed to achieve this growth and scale. This spending includes operating expenses, such as sales and marketing expenses, growth-related capital expenditures, and working capital investment. Operating cash flow is still likely to be negative but may be improving and becoming more predictable. The company begins to establish a customer and supplier base.

As the business progresses through the growth stage, execution and competitive risks decline. Operating cash flow typically turns positive and then becomes more stable and predictable as the business grows. As a result, the business becomes more attractive to lenders. Depending on the business model, there may also be assets that can be used to secure debt, such as receivables, inventory, or fixed assets. Both the availability and the terms of debt financing improve for the company during this stage.

Many growth companies use debt conservatively in order to preserve operational and financial flexibility and minimize the risk of financial distress. Equity remains the predominant source of capital. Some business models are inherently capital light; that is, they require little incremental investment in fixed assets or working capital to enable revenue growth. As a result, they have minimal need to borrow or otherwise raise capital to grow, even though they could easily support debt. Software businesses often fit this description.

\section{EXAMPLE 6}
\section{Growth Company Financing}
\begin{enumerate}
  \item A growth retail business that is expected to generate positive operating cash flows in the next $3-5$ years would normally be financed with:
\end{enumerate}

A. little or no debt.

B. significant debt to minimize equity dilution.

C. significant debt to minimize its weighted average cost of capital.

\section{Solution}
A is correct. A growth retail business expected to generate positive operating cash flows in the next 3-5 years would normally be financed with little or no debt.

While one can often find exceptions, the standard approach to financing a business in such a highly competitive sector as retail, with negative and/or unpredictable cash flows, is to rely primarily on equity. Using predominantly equity financing to meet capital needs allows management to preserve operational and financial flexibility while minimizing the risk of financial distress associated with debt.

\section{Mature Businesses}
At the maturity stage, a company's revenue growth may slow or even begin to decline. At this stage, however, a successful business generates reliable and positive cash flow and likely has an established customer and supplier base. There is typically a decline or a deceleration in growth-related investment spending. The company becomes able to support low-cost debt, often on an unsecured basis. From the company's perspective, debt financing is likely to be more attractive than higher-cost equity financing.

In practice, large, mature public companies commonly use significant debt in their capital structures, although many seek to maintain an investment-grade rating in order to preserve maximum financial flexibility. An investment-grade rating enables a company to access a very wide pool of potential investors on an unsecured basis without onerous debt covenants, and it minimizes the risk that financing might become unavailable.

Mature businesses often de-leverage over time, experiencing a reduction in debt as a proportion of total capital. De-leveraging occurs due to continuing cash flow generation and because equity values commonly rise over time from share price appreciation. To offset this de-leveraging, companies may elect to pay more cash dividends or conduct share buybacks. By using cash to buy back shares, a company reduces its outstanding share count, thereby reducing equity in its capital structure.

Share buybacks, also called stock repurchases, are attractive to companies because, compared with dividends, they offer greater flexibility and are generally a more tax-efficient means of distributing cash to taxable investors. Particularly in North America, investors typically expect dividend levels to be maintained once established, but that is not their expectation for share buybacks. Buybacks can, therefore, be conducted when cash is available and when the share price is seen by management to be undervalued. Investors generally respond favorably to share buyback announcements, which may lead to increases in share price and in share option values. The share count reduction caused by buybacks can enhance per-share metrics, such as EPS or EPS growth, which in turn can enhance management compensation and also lead to potential conflicts of interest with other stakeholders.

\section{EXAMPLE 7}
\section{Mature Business Capital Structure Considerations}
Sleepy Mattress Company went public 10 years ago. Since then, the company has seen its revenues and cash flows increase every year. This growth is expected to continue for the foreseeable future. No dividends have ever been paid. All cash flow has gone toward debt reduction (strongly preferred by the prior CEO), and the company has recently become debt-free. The shares are trading at $10 \times$ trailing cash flow.

There is now a lively debate within the board about the company's capital structure. Board members agree that the company's capital structure should include debt in order to minimize its overall cost of capital, but they do not agree on how to achieve this goal. We consider each board member's recommendation, its associated scenario, the likely impact on Sleepy's capital structure, and what is most likely the best course of action:

A. Director A has argued that the company should pay a one-time special dividend equal to last year's after-tax cash flow, drawing down its unused line of credit to do so.

B. Director B recommends that the company borrow an amount equal to two years' cash flow and then seek an appropriate acquisition.

C. Director $C$ advocates making a tender offer to buy back $50 \%$ of the company's shares at a $50 \%$ premium to market price and issuing debt to finance the offer. Since this would be a very large tender offer (equal to $75 \%$ of the company's equity market capitalization), it would be financed using a combination of senior and high-yield debt, carrying a relatively high $6 \%$ average interest rate. The company's cash flow will cover the projected interest payments, with operating cash flow equal to 1.1 times interest, and the surplus available for debt reduction. Projected debt/capital would be above $75 \%$.

D. Director D favors a smaller share buyback: $20 \%$ of the company's shares at a $20 \%$ premium to market price, financed by issuing investment-grade debt with a 3\% coupon. The company's cash flow will cover the projected interest payments, with operating cash flow equal to 6.9 times interest, and the surplus available for share repurchases. Projected debt/capital would be $20 \%$ to $25 \%$.

\section{Discussion}
Sleepy Mattress is a mature and stable business. It should therefore be able to support significant debt. The optimal capital structure for such a company would normally include debt, with the firm's value enhanced by the tax-deductibility of interest expense. But the company is now debt-free, generating cash flow and paying no dividend. If no action is taken, cash will accumulate on the company's balance sheet. We look at each recommendation in turn:

A. A's recommendation to pay a special cash dividend is reasonable. However, the size of the proposed dividend and associated borrowing (1× after-tax cash flow) is small, leaving Sleepy with a balance sheet that has very little debt (and likely no debt within a year).

B. B's recommendation is not appropriate unless the company has already identified an acquisition that would create value for the business. Even then, it would be prudent to look at the capital structure of the target company (and the pro forma combined capital structures) before determining how to finance the acquisition.

C. C's recommendation to buy back shares with debt would increase the debt and reduce the equity in the capital structure. This recapitalization plan is a step in the right direction; however, the resulting interest coverage of 1.1 times is extremely low, so taking this action would likely be too aggressive for the company. Any profit deterioration could easily cause the company to fail to meet its interest obligations. While we are assuming that Sleepy Mattress could raise the debt, the company could find it challenging to do so. At over $75 \%$, pro forma debt as a percentage of total capital is very high. For companies with more stable and predictable cash flows, such as regulated utilities or REITs, higher percentages (i.e., above $60 \%$ ) are common, but they are not common in consumer or industrial companies. Percentages above 75\% often indicate that a company may be in financial distress.

D. D's recommendation is a superior course of action. It is a more balanced version of $\mathrm{C}$, resulting in a manageable level of debt in the capital structure (under 25\%), with a strong projected interest coverage ratio (6.9 times). It would be reasonable for the board to recommend this recapitalization to address the all-equity capital structure and, in addition, to initiate a cash dividend. Conducting a regular share buyback combined with a cash dividend would increase debt relative to equity in the capital structure and help counteract the de-leveraging in the business over time.

\section{Unique Situations}
In our discussion of the general relationship between company maturity and capital structure, there are important exceptions to note.

\section{Capital-intensive businesses with marketable assets}
Some businesses use high levels of leverage regardless of their development stage. For example, in real estate, utilities, shipping, airlines, and certain other highly capital-intensive businesses, the underlying assets can be bought and sold fairly easily, tend to retain their value regardless of who owns them, and can therefore support substantial debt secured by those assets. This would be true of a downtown office building, for example, but would not be true of a highly specialized factory in a remote location.

Note that in some cases, business models have evolved to take advantage of this fact. What were once highly capital-intensive and vertically integrated businesses (e.g., trucking, hotels, retailers, restaurants) have evolved to the point that many major brands are now owned by marketing or service businesses, which in turn have contractual relationships with the owners of real estate or other fixed assets used in the business. For example, Hilton Worldwide, one of the world's largest hotel companies, operates almost all its hotel rooms through long-term franchise or management agreements; the hotels themselves are owned by others. Conversely, some relatively large and mature businesses use little debt.

\section{"Capital-light" businesses}
Some business models-notably, software-based technology businesses-regardless of their development stage, have minimal fixed investments or working capital needs. They tend to have little debt in their capital structures and in many cases have substantial net cash, reflecting several factors:

\begin{itemize}
  \item With little or no fixed assets or capital expenditure required to support growth, these businesses are often cash flow positive from an early stage, never needing to raise large amounts of capital.

  \item Many companies in rapidly evolving industries see the need to accumulate cash for potential acquisitions.

  \item If they are rapidly growing and successful, companies may not face typical pressure to pay dividends or repurchase shares.

  \item A rapidly rising share price can cause the market value of a company's equity to significantly outpace the value of any debt that might have been raised.

\end{itemize}

\section{EXAMPLE 8}
\section{Debt Use: Industry Variations}
\begin{enumerate}
  \item Green Company and Black Company are each achieving $15 \%$ annual revenue growth and have recently started to generate positive cash flow. Green Company owns and acquires renewable energy generation projects. Black Company is a cloud-based software company with a dominant market position, serving auto dealers. Which company is more likely to have greater debt in its capital structure, and why?
\end{enumerate}

A. Black Company, because it serves a cyclical business

B. Black Company, because of the strength of its market position

C. Green Company, because its underlying assets can be financed with debt

\section{Solution}
$\mathrm{C}$ is correct. Green Company has fixed assets, for which there is likely to be a ready and liquid market, and stable cash flows, which are supportive of debt financing. Black Company, a cloud-based software technology company, is a "capital-light" business, with few fixed assets. Its assets are likely to consist of mostly human capital. Additionally, servicing a cyclical industry is also likely to lead to Black Company having low debt. 4

\section{MODIGLIANI-MILLER PROPOSITIONS}
explain the Modigliani-Miller propositions regarding capital structure

In a classic paper, Nobel Prize-winning economists Franco Modigliani and Merton Miller (1958) argued that, given certain assumptions, a company's choice of capital structure does not affect (or is "irrelevant" in determining) its value, where firm value is equal to the present value of the firm's expected future cash flows, discounted by the firm's weighted average cost of capital. In short, managers cannot change firm value simply by changing the company's capital structure.

Let's begin by imagining a company's capital structure as a pie, with each slice of the pie representing a specific type of capital (e.g., common equity or debt) and the size of the pie representing the company's total value. We can slice the pie in any number of ways, yet the total size remains the same. This is equivalent to saying that a company's value, or present value of expected cash flows, remains the same no matter how the capital slices are allocated, so long as the future cash flow stream is expected to remain the same and the risk of that cash flow stream, as reflected by the company's weighted average cost of capital, remains the same.

In their work, Modigliani and Miller (MM) used simplifying assumptions to show the irrelevance of capital structure to firm value, and then they relaxed the assumptions to show the impact of taxes and financial distress costs on capital structure. In their theoretical framework, MM assumed that all investors have homogeneous expectations about expected future corporate earnings and the riskiness of those earnings. Their assumptions, including a world of perfect capital markets-in which there are no taxes, no transaction costs, and no bankruptcy costs, and all investors have equal ("symmetric") information-are shown in Exhibit 8.

Exhibit 8: Modigliani-Miller Assumptions

\begin{center}
\includegraphics[max width=\textwidth]{2023_05_04_7b535d0a870224f62e3dg-072(1)}
\end{center}

\begin{center}
\includegraphics[max width=\textwidth]{2023_05_04_7b535d0a870224f62e3dg-072}
\end{center}

borrow and lend at the risk-free rate. Modigliani and Miller's seminal work provides us with a starting point and allows us to examine what happens when the assumptions are relaxed to reflect real-world considerations. While these assumptions do not hold in practice-which ultimately does alter MM's original conclusion of capital structure irrelevance-the MM theoretical framework remains a popular starting point for thinking about the strategic use of debt in a company's capital structure.

\section{Proposition I without Taxes: Capital Structure Irrelevance}
Given their assumptions, Modigliani and Miller proved that changing the capital structure does not affect firm value, in part because individual investors can create any capital structure they prefer for the company by borrowing and lending in their own accounts in addition to holding shares in the company. This "homemade" leverage argument relies on the MM assumption that investors can lend and borrow at the risk-free rate.

Example: Suppose that a company has a capital structure consisting of $50 \%$ debt and $50 \%$ equity and that an individual investor would prefer that the company's capital structure be $70 \%$ debt and $30 \%$ equity. The investor could borrow money to finance their share purchases so that their ownership of company assets would reflect their preferred $70 \%$ debt financing. This action would be equivalent to buying stock on margin and would have no effect on either the company's expected operating cash flows or company value.

Modigliani and Miller used the concept of arbitrage to demonstrate their point: If the value of an unlevered company (i.e., a company without any debt) is not equal to that of a levered company, investors could make a riskless arbitrage profit at no cost by selling shares of the overvalued company and using the proceeds to buy shares of the undervalued company, forcing their values to become equal. The value of a firm is thus determined not by the securities it issues but, rather, by its expected future cash flows. Their conclusion is summarized below.

\section{PROPOSITION I WITHOUT TAXES}
The market value of a company is not affected by the company's capital structure.

\section{Implications:}
\begin{enumerate}
  \item The value of the levered company $\left(V_{L}\right)=$ the value of the unlevered company $\left(V_{U}\right)$, or: $V_{L}=V_{U}$

  \item The value of a company is determined solely by its expected future cash flows (not its relative reliance on debt and equity capital).

  \item In the absence of taxes, the weighted average cost of capital (WACC), or

\end{enumerate}

$\mathrm{WACC}=w_{d} r_{d}+w_{p} r_{p}+w_{e} r_{e}$

is unaffected by capital structure.

\section{Proposition II without Taxes: Higher Financial Leverage Raises the Cost of Equity}
Debt capital is cheaper than equity capital because debtholders have a priority in claims. Therefore, one might expect a company's weighted average cost of capital to decline by adding debt capital to its capital structure. However, adding leverage to a company's capital structure increases the risk to equityholders because greater debt increases the probability of bankruptcy. As a result, equityholders will demand a higher return on their equity investment as leverage increases in order to offset the increase in risk.

MM Proposition II without taxes tells us that adding any amount of lower-cost debt capital to the capital structure is always perfectly offset by an increase in the cost of equity, resulting in no change to the company's overall weighted average cost of capital. MM Proposition II explains why investors require higher returns on levered equity; their required returns should match the increased risk from leverage. Specifically, MM Proposition II without taxes tells us that a company's cost of equity is a linear function of its debt-to-equity ratio:

$$
r_{e}=r_{0}+\left(r_{0}-r_{d}\right) \frac{D}{E}
$$

where $r_{e}$ is the cost of equity, $r_{o}$ is the cost of capital for a company financed only with equity (i.e., an all-equity company), $r_{d}$ is the cost of debt, $D$ is the market value of debt, and $E$ is the market value of equity. Equation 1 is a linear function with the intercept equal to $r_{0}$ and the slope equal to the quantity $\left(r_{0}-r_{d}\right)$.

Given that capital structure changes do not affect the company's future cash flow stream and the company's weighted average cost of capital remains unchanged for any chosen capital structure, there is no change in the value of the company. Note that Modigliani and Miller did not assume away the possibility of bankruptcy. They simply assumed there was no cost to bankruptcy.

\section{PROPOSITION II WITHOUT TAXES}
The cost of equity is a linear function of the company's debt-to-equity ratio.

\section{Implications:}
\begin{enumerate}
  \item Higher leverage raises the cost of equity but does not change firm value or WACC.

  \item The increase in the cost of equity must exactly offset the greater use of lower-cost debt.

\end{enumerate}

Example: Consider the Leverkin Company, which currently has an all-equity capital structure. Leverkin has expected annual cash flows to equityholders (which we denote "CF $\mathrm{e}_{\mathrm{e}}$ ") of USD5,000 and a cost of equity of $10 \%$, which is also its WACC since equity is the firm's only source of capital. For simplicity, we assume that all cash flows are perpetual. Therefore, Leverkin's value is equal to:

$$
V=\frac{\mathrm{CFe}}{r_{\text {wacc }}}=\frac{\$ 5,000}{0.10}=\$ 50,000
$$

Now suppose that Leverkin plans to issue USD15,000 in debt at a cost of 5\% and use the proceeds to buy back and reduce its outstanding equity by USD15,000. This action leaves the total invested capital unchanged at USD50,000.

Under MM Proposition I, $V_{L}=V_{U}$, the value of Leverkin must remain the same at USD50,000 after the change in capital structure. Under MM Proposition II, after the change in capital structure, the cost of equity for Leverkin-now with USD15,000 in debt capital and USD35,000 in equity capital-increases to $12.143 \%$ :

$$
r_{e}=0.10+(0.10-0.05) \frac{\$ 15,000}{\$ 35,000} \approx 0.12143=12.143 \%
$$

To prove that Leverkin's firm value does not change after the change in capital structure, we need to show that its WACC remains unchanged at $10 \%$. With the new cost of equity, Leverkin's WACC is now calculated as:

$$
r_{\text {wacc }}=\left(\frac{\$ 15,000}{\$ 50,000}\right) 0.05+\left(\frac{\$ 35,000}{\$ 50,000}\right) 0.12143=0.10=10 \%
$$

Leverkin's WACC is still 10\%, as the move to cheaper debt was perfectly offset by an increase in the cost of equity. Thus, consistent with MM Proposition I, the value of the firm remains unchanged at USD50,000. Furthermore, the value of Leverkin must equal the sum of the present values of cash flows to debtholders (which we denote "CF $\mathrm{d}_{\mathrm{d}}$ ") and equityholders. With USD15,000 debt at a cost of $5 \%\left(\mathrm{r}_{\mathrm{d}}\right)$, Leverkin makes annual interest payments of USD750 to debtholders, leaving USD5,000 - USD750 = USD4,250 remaining for equityholders. Therefore, the total value of the company can also be expressed as:

$$
\begin{aligned}
V & =D+E \\
V & =\frac{C F_{d}}{r_{d}}+\frac{C F_{e}}{r_{e}} \\
V & =\frac{\mathrm{USD} 750}{0.05}+\frac{\text { USD4,250 }}{0.12143}=\text { USD50, } 000
\end{aligned}
$$

As the level of debt rises, the risk of the company defaulting on its debt increases. This risk is borne by the equity holders. So, as the proportionate use of debt rises, the equity's beta, $\beta_{e}$, also rises.

Now let's consider a company's systematic risk. Portfolio managers know that the beta of an investment portfolio is a market-value-weighted average of the betas of the investments in that portfolio. Similarly, the beta (or systematic risk) of a company's assets is a weighted average of the systematic risks of its sources of capital. This concept is explained by the Hamada equation, which does not account for default risk (see Hamada 1972):

$$
\beta_{a}=\left(\frac{D}{V}\right) \beta_{d}+\left(\frac{E}{V}\right) \beta_{e}
$$

where $\beta_{a}$ is the asset's systematic risk, or asset beta, $\beta_{d}$ is the beta of debt, and $\beta_{e}$ is the equity beta. According to Modigliani and Miller, the company's cost of capital depends not on its capital structure but, rather, on its business risk as reflected in the firm's WACC. As the level of debt rises, however, the risk of the company defaulting on its debt increases. This risk is borne by the equityholders. So, as the proportionate use of debt rises, the equity's beta, $\beta_{e}$, also rises. By reordering the terms of Equation 2 , we get

$$
\beta_{e}=\beta_{a}+\left(\beta_{a}-\beta_{d}\right)\left(\frac{D}{E}\right)
$$

Conclusion: We have learned that under Modigliani and Miller's restrictive assumptions, when there are no taxes, leverage does not affect the value of the company or its weighted average cost of capital. However, leverage does cause the risk of equity to increase and thus the cost of equity to increase as well.

\section{EXAMPLE 9}
\section{Value of Firm in Perfect Capital Markets}
\begin{enumerate}
  \item Company A has $25 \%$ debt and $75 \%$ equity in its capital structure. Management decides to increase leverage, so it issues more debt and buys back company stock. As a result, the new capital structure is $50 \%$ debt and $50 \%$ equity. Which of the following statements is true in perfect capital markets?
\end{enumerate}

A. After refinancing, the company is worth more because leverage has increased.

B. After refinancing, the company is worth less because there is a greater chance of bankruptcy.

\section{Neither is true.}
\section{Solution}
Neither answer is correct. In perfect capital markets, a change in capital structure has no impact on the value of the company.

\section{EXAMPLE 10}
\section{Effect of Leverage on Equity Beta}
\begin{enumerate}
  \item The Chang Shou Noodle Company is financed with $10 \%$ debt and $90 \%$ equity.
\end{enumerate}

If the asset beta is 0.8 and the debt beta is 0.2 , what is the equity beta?

\section{Solution}
We use Equation 3 to solve this problem.

Prior to the change in capital structure, the equity beta is

$$
\beta_{e}=0.8+(0.8-0.2)(10 / 90)=0.87
$$

\begin{enumerate}
  \setcounter{enumi}{1}
  \item How does the equity beta change if management increases leverage to $30 \%$ debt?
\end{enumerate}

\section{Solution}
After the capital structure change, the equity beta is higher because more debt increases the risk to equityholders. The new equity beta is

$$
\beta_{e}=0.8+(0.8-0.2)(30 / 70)=1.06 \text {. }
$$

\section{Propositions with Taxes: Firm Value}
Now let's explore what happens to the two MM propositions when we relax the assumption of no corporate taxes.

In most countries, interest expense is deductible from income for tax purposes. In other words, debt provides a tax shield for companies that are earning profits, and the money saved in taxes enhances the value of the company. If we ignore other realities for now, such as the costs of financial distress and bankruptcy, the value of the company increases with increasing levels of debt. The actual cost of debt is reduced by the amount of the tax benefit:

After-tax cost of debt $=$ Before-tax cost of debt $\times(1-$ marginal tax rate $)$

Modigliani and Miller's Proposition I with corporate taxes states that in the presence of corporate taxes (but not personal taxes), the value of the levered company is greater than that of the all-equity company by an amount equal to the tax rate multiplied by the value of the debt, defined as the present value of the debt tax shield:

$$
V_{L}=V_{U}+t D
$$

where $t$ is the marginal tax rate and $t D$ is the present value of the debt tax shield. When there are corporate taxes, a profitable company can increase its value by using debt financing. Taken to the extreme, Equation 4 predicts a value-maximizing capital structure of $100 \%$ debt.

\section{PROPOSITION I WITH CORPORATE TAXES}
The market value of a levered company is equal to the value of an unlevered company plus the value of the debt tax shield.

\section{Implications:}
\begin{enumerate}
  \item In the presence of taxes, a profitable company can increase its value (V) by using debt.

  \item The higher the tax rate, the greater the benefit of using debt in the capital structure.

\end{enumerate}

\section{Propositions with Taxes: Cost of Capital}
If the value of the company increases as it uses more debt, the company's weighted average cost of capital must decrease as it uses more debt. That is, in the earlier propositions without corporate taxes, the lower cost of debt was perfectly offset by an increase in the cost of equity. Now, in the presence of corporate taxes, the cost of debt is further lowered by the tax benefit such that the lower debt cost outweighs the increase in the cost of equity and results in a lower WACC.

To demonstrate this idea, let's begin with the revised cost of equity under MM Proposition II with corporate taxes:

$$
r_{e}=r_{0}+\left(r_{0}-r_{d}\right)(1-t) \frac{D}{E}
$$

Notice that the only difference between Equation 5 and Equation 1 (MM Proposition II with no taxes) is the presence of the term $(1-t)$. When $t$ is zero, the two equations are identical. When $t$ is not zero, the term $(1-t)$ is less than 1 and serves to reduce the cost of levered equity. The cost of equity still rises as the company increases the amount of debt in its capital structure, but it rises at a slower rate than in the no-tax case.

Consequently, as debt increases, the company's WACC decreases and the company's value increases. This result implies that when there are taxes (and no financial distress or bankruptcy costs), debt financing is highly advantageous. Taken to an extreme, this result also suggests that a company's optimal capital structure is all debt-a conclusion that is at odds with reality and a direct result of Modigliani and Miller's restrictive assumptions.

\section{PROPOSITION II WITH CORPORATE TAXES}
The cost of equity is a linear function of the company's debt-to-equity ratio with an adjustment for the tax rate.

\section{Implications:}
\begin{enumerate}
  \item In the presence of taxes, the cost of equity rises as the company uses more debt but at a slower rate than in the no-tax case.

  \item As the company's use of debt increases, its WACC decreases and its value increases.

  \item In the presence of taxes (but no financial distress or bankruptcy costs), the use of debt is value enhancing and, at the extreme, $100 \%$ debt is optimal. Example: Referring back to the previous example involving the Leverkin Company, recall that annual cash flows to equityholders were USD5,000 and the cost of equity (and WACC) was 10\%. As before, Leverkin is planning to issue USD15,000 of 5\% debt in order to buy back an equivalent amount of equity. Now, however, assume that Leverkin pays corporate taxes at a rate of $25 \%$.

\end{enumerate}

Since the company does not currently have debt, the after-tax cash flows are now USD5,000(1 - 0.25), or USD3,750. Because the cash flows are assumed to be perpetual, the value of the company is USD37,500 (= USD3,750/0.10), considerably less than it was when there were no corporate taxes.

Now, suppose Leverkin proceeds to issue USD15,000 of debt and uses the proceeds to repurchase equity. According to MM Proposition I with corporate taxes (i.e., Equation 4), the value of the company is

$$
V_{L}=V_{U}+t D=\mathrm{USD} 37,500+0.25(\mathrm{USD} 15,000)=\mathrm{USD} 41,250 .
$$

The total company value is now USD41,250, consisting of debt of USD15,000 and equity of USD26,250 (= USD41,250 - USD15,000).

According to MM Proposition II with corporate taxes (Equation 5), the new cost of equity for Leverkin is

$$
r_{e}=0.10+(0.10-0.05)(1-0.25) \frac{\$ 15,000}{\$ 26,250}=0.12143=12.143 \% .
$$

Using this new cost of equity, we can compute the new WACC:

$$
\begin{aligned}
& r_{\text {wacc }}=\frac{\$ 15,000}{\$ 41,250}(0.05)(1-0.25)+\frac{\$ 26,250}{\$ 41,250}(0.12143) \\
& =0.09091=9.091 \%
\end{aligned}
$$

Unlike the previous example where Leverkin's WACC did not change with the change in capital structure, Leverkin's WACC decreased in this example from 10\% to $9.091 \%$. This reduction in WACC resulted in an increase in company value to USD41,250:

$$
V_{L}=\frac{\mathrm{CFe}(1-t)}{\mathrm{WACC}}=\frac{\$ 5,000(1-0.25)}{0.09091} \approx \$ 41,250
$$

As shown in the previous example, the value of the company must also equal the present value of cash flows to debtholders and equityholders:

$$
\begin{aligned}
& V_{L}=D+E=\frac{r_{d} D}{r_{d}}+\frac{\left(\mathrm{CFe}-r_{d} D\right)(1-t)}{r_{e}} \\
= & \frac{\$ 750}{0.05}+\frac{(\$ 5,000-\$ 750)(1-0.25)}{0.12143} \approx \$ 41,250
\end{aligned}
$$

Exhibit 9 summarizes the results using formulae.

\section{Exhibit 9: Modigliani and Miller Propositions}
\begin{center}
\begin{tabular}{lcc}
 & Without Taxes & With Taxes \\
\hline
Proposition I & $V_{L}=V_{U}$ & $V_{L}=V_{U}+t D$ \\
Proposition II & $r_{e}=r_{0}+\left(r_{0}-r_{d}\right) \frac{D}{E}$ & $r_{e}=r_{0}+\left(r_{0}-r_{d}\right)(1-t) \frac{D}{E}$ \\
\hline
\end{tabular}
\end{center}

Of course, in the real world, taxes are not the only factor that affects the value of a levered company. The analysis gets more complicated when we allow for such things as the costs of financial distress and other real-world considerations. We consider costs of financial distress next.

\section{Costs of Financial Distress}
Operating and financial leverage can magnify profits and losses. Losses can put companies into financial distress, the costs of which can be explicit or implicit. Financial distress refers to the heightened uncertainty regarding a company's ability to meet its various obligations because of diminished earnings power or actual current losses. Even before filing for bankruptcy, companies under financial distress may lose customers, creditors, suppliers, and valuable employees.

\section{EXAMPLE 11}
\section{Costs of Financial Distress}
The Carillion PLC was a UK-based entity with revenues of almost GBP4.4 billion in 2016. It had a number of contracts with the UK government for the maintenance of roads and for catering and cleaning at hundreds of schools. Its share price was GBP308 at the end of 2015 and GBP236 at the end of 2016.

In mid-2017, Carillion warned that profits would fall short of expectations due to problems collecting on several construction contracts. Soon after, its share price fell by $30 \%$. Even as it faced collection problems, the company was awarded more contracts, including contracts for railway and military projects.

Subsequent profit warnings sent the share price further downward. The second profit warning of that year was followed by new credit facilities and deferrals of debt repayments. The third profit warning, which included concerns over violating debt covenants, caused the share price to fall to around GBP21.

Discussions with creditors in December 2017 and January 2018 did not result in an agreement. Carillion was put in compulsory liquidation on 15 January 2018, and share trading was suspended.

The primary causes of Carillion's problems were its many overreaching, unprofitable projects and its inability to collect payments quickly.

The costs of financial distress included:

\begin{itemize}
  \item the loss of all shareholder value;

  \item GBP44.2 million paid to an accounting firm to manage the insolvency;

  \item costs to the UK government as it grappled with nationalizing contracts; and

  \item creditors getting paid only a fraction of what they were owed.

\end{itemize}

The direct costs of financial distress include actual cash expenses associated with the bankruptcy process, such as legal and administrative fees. Indirect costs include forgone investment opportunities, reputational risk, impaired ability to conduct business, and costs arising from conflicts of interest between managers and debtholders, known as agency costs of debt, during periods in which the company is near or in bankruptcy.

The costs associated with financial distress are lower for companies whose assets have a ready secondary market. Airlines, shipping companies, and steel manufacturers typically have tangible assets that can be easily sold. High-tech growth companies, pharmaceutical companies, information technology companies, and companies in the services industry typically have fewer tangible assets that can be sold. These companies have higher costs associated with financial distress. The probability of financial distress and bankruptcy increases when there is a greater amount of debt in the capital structure, a greater amount of business risk, and fewer reserves available to delay bankruptcy. Other factors that affect the probability of bankruptcy include diminished earnings power, unprofitable capital investments, the company's corporate governance structure, and the quality of the management team.

\section{OPTIMAL AND TARGET CAPITAL STRUCTURES
describe the use of target capital structure in estimating WACC, and calculate and interpret target capital structure weights}
Thus far in our discussion of the MM propositions, we have relaxed only the assumption of no corporate taxes. We now consider a scenario with both corporate taxes and bankruptcy/financial distress costs. The value-enhancing effects of leverage from the tax-deductibility of interest must now be weighed against the value-reducing impact of the present value of expected (or probability-weighted) costs of financial distress or bankruptcy, debt agency costs, debt restructuring and issuance fees, etc. Formally, we can modify MM Proposition I with corporate taxes to incorporate a deduction in firm value for the present value of financial distress costs:

$$
V_{L}=V_{U}+t D-P V \text { (Costs of financial distress) }
$$

Equation 7 represents the static trade-off theory of capital structure, which is illustrated in Exhibit 10. The point $\mathrm{V}_{\mathrm{U}}$ represents the value of an unlevered or all-equity firm. Moving to the right from this starting point, debt is added to the capital structure, and the new levered firm value shown by the green line is derived from Equation 7. At low levels of debt ( $x$-axis), the tax benefit of debt typically outweighs the present value of financial distress costs, resulting in a higher firm value.

However, as more and more debt is added to the capital structure (moving farther to the right on the graph), the company's financial distress costs begin to rise substantially and eventually equal the tax benefit of debt at the point $\mathrm{D}^{*}$. Taking on debt beyond this point begins to reduce firm value because the substantial present value of financial distress costs now outweighs the tax benefit of debt. The theoretical point $\mathrm{D}^{*}$, the point at which the value of the company is maximized, is referred to as the optimal capital structure. Exhibit 10: Static Trade-Off Theory with Taxes and Costs of Financial

Distress: Firm Value and the Debt-to-Equity Ratio

\begin{center}
\includegraphics[max width=\textwidth]{2023_05_04_7b535d0a870224f62e3dg-081}
\end{center}

The static trade-off theory of capital structure suggests that managers should seek the optimal capital structure $\left(\mathrm{D}^{*}\right)$ mix, the mix at which the marginal increase in the present value of interest tax shields is perfectly offset by the increase in the present value of financial distress costs. However, in practice, managers cannot precisely identify $\mathrm{D}^{*}$. But, by considering the company's business risk, tax situation, corporate governance, financial accounting information, probable bankruptcy costs, and other factors, management should be able to identify a range, or target capital structure, to strive for that is fairly close to the optimal capital structure. In addition, defining the company's optimal capital structure may vary according to economic conditions, industry sector, line of business, firm maturity, and regulatory environment.

There are a few reasons why a company's actual capital structure may differ from its target capital structure. For example, management may be able to exploit short-term opportunities in a particular financing source. A company issuing high-grade debt at attractive rates, for instance, may choose to increase deal size in response to strong investor demand. Or, fluctuations in the market values of the company's debt and equity securities can cause the company's actual capital structure to deviate from the target. In addition, flotation costs, which are the costs incurred by a publicly traded company when it issues new debt or equity securities, may make it impractical for a company to continually adjust its capital structure to keep it on target. In any case, while optimal capital structure may exist as a specific point in theory, it cannot be exactly determined because it is difficult to precisely estimate some costs, such as financial distress costs. In practice, the optimal capital structure should be thought of as being within a range rather than a precise ratio. For example, it would be impossible to determine with certainty that the optimal amount of debt in the capital structure should be exactly $40 \%$. Yet, the company should be able to determine that its target capital structure should be in the range of $30 \%$ to $50 \%$ debt.

The starting MM capital structure framework, even with its unrealistic assumptions, is appealing for understanding the strategic use of debt by companies. While its original predictions of capital structure irrelevance and a firm-value-maximizing capital structure of $100 \%$ debt do not hold when the assumptions are relaxed, the resulting static trade-off theory of capital structure still provides a thoughtful framework for using debt in the capital structure from a cost-benefit perspective.

One prediction of the trade-off theory is that highly profitable firms will use more debt than firms with lower profits to increase the tax-deductibility benefit of using debt. That is, the marginal benefit of using debt for highly profitable firms is often quite high as reflected in a high marginal corporate income tax rate, and these high-profit firms also tend to benefit from relatively lower costs of financial distress.

However, in practice, numerous studies have documented a negative relationship between debt use and profit levels. Even though this empirical finding does suggest that firm managers do not subscribe to the trade-off theory in setting their firms' capital structures, surveys have shown that managers do indeed manage their capital structures with a target capital structure in mind. Regardless of the empirical shortcomings of the trade-off theory, it is still a common starting point for understanding the strategic use of debt and has led to other possible capital structure theories, such as the pecking order theory and the market timing theory. A detailed discussion of these competing theories of capital structure is beyond the scope of this coverage.

\section{Market Value vs. Book Value}
While an optimal capital structure is calculated using the market value of equity and debt, a company's capital structure targets often use book value instead for the following reasons:

\begin{enumerate}
  \item Market values can fluctuate substantially and seldom impact the appropriate level of borrowing. On the contrary, a company that has seen a rapid increase in its share price might opportunistically decide to raise equity capital rather than debt to maintain a certain debt-to-equity ratio.

  \item For management, the primary concern is the amount and types of capital invested by the company, not in the company. This perspective includes how future investments will be funded. It is different, for example, from that of a shareholder who has purchased shares at the prevailing market price and is concerned with generating a return on that investment.

  \item Capital structure policy ensures management's ability to borrow easily and at low cost. Since lenders, debt investors, and rating agencies generally focus on the book value of debt and equity for their calculation measures, companies and their managers will take this fact into account in determining their capital structure policies.

\end{enumerate}

Although it is common for target capital structures to be expressed in terms of book values, managers pay close attention to both the price of their companies' shares and the market interest rates for their debt in deciding when, how much, and what type of capital to raise. For this reason, financings are often opportunistic to some degree, explaining why capital structures often deviate from target levels defined by management.

\section{Target Weights and WACC}
When conducting an analysis, how do we determine what WACC weights to use? Ideally, we want to use the proportion of each source of capital that the company would use in the project or company. If we assume that a company has a target capital structure and raises capital consistent with this target, we should use this target capital structure-the capital structure that the company is striving to obtain. If we know the company's target capital structure, then of course we should use it in our analysis. Analysts outside the company, however, would not know the target capital structure if not disclosed by the company and must estimate it using one of several methods:

\begin{enumerate}
  \item Assume the company's current capital structure, at market value weights for the components, represents the company's target capital structure.

  \item Examine trends in the company's capital structure or statements by management regarding capital structure policy to infer the target capital structure.

  \item Use averages of comparable companies' capital structures as the target capital structure.

\end{enumerate}

In the absence of knowledge of a company's target capital structure, it may be best to rely on the first method as the baseline. Note that in applying the third method, it is common, for simplicity, to use an unweighted, arithmetic average. An alternative is to calculate a weighted average, which would give more weight to larger companies.

Suppose we are using a company's current capital structure as a proxy for the target capital structure. In this case, we use the market values of the different capital sources in calculating the proportions. For example, suppose a company has the following capital structure based on market values:

\begin{center}
\begin{tabular}{lc}
\hline
Bonds outstanding & USD5 million \\
\hline
Preferred stock & 1 million \\
Common stock & 14 million \\
Total capital & USD20 million \\
\hline
\end{tabular}
\end{center}

The weights that we use would be as follows:

\begin{center}
\begin{tabular}{lc}
\hline
 & Weights \\
\hline
Debt & 0.25 \\
Preferred equity & 0.05 \\
Common equity & 0.70 \\
\hline
\end{tabular}
\end{center}

Example 12 illustrates the estimation of weights. Note that a simple way of transforming a debt-to-equity ratio (D/E) into a weight-that is, $D /(D+E)$-is to divide $D / E$ by $1+D / E$.

\section{EXAMPLE 12}
\section{Estimating the Proportions of Capital}
Fin Anziell is a financial analyst with Analytiker Firma. Anziell is in the process of estimating the cost of capital of Gewicht $\mathrm{GmbH}$. The following information is provided:

Market value of debt: EUR50 million

Market value of equity: EUR60 million

Primary competitors and their capital structures (in millions):

\begin{center}
\begin{tabular}{lcc}
\hline
Competitor & $\begin{array}{c}\text { Market Value of } \\ \text { Debt }\end{array}$ & $\begin{array}{c}\text { Market Value } \\ \text { of Equity }\end{array}$ \\
\hline
A & EUR25 & EUR50 \\
B & EUR101 & EUR190 \\
C & GBP40 & GBP60 \\
\hline
\end{tabular}
\end{center}

What are Gewicht's proportions of debt and equity that Anziell would use in his analysis if he estimates these proportions using the company's:

\begin{enumerate}
  \item Current capital structure?
\end{enumerate}

\section{Solution}
Current capital structure:

$$
\begin{aligned}
& w_{d}=\frac{€ 50 \text { million }}{€ 50 \text { million }+€ 60 \text { million }}=0.4545 \\
& w_{e}=\frac{€ 60 \text { million }}{€ 50 \text { million }+€ 60 \text { million }}=0.5454
\end{aligned}
$$

\section{Competitors' capital structure?}
\section{Solution}
Competitors' capital structure:

$$
\begin{gathered}
w_{d}=\frac{\left(\frac{€ € 25}{€ 25+€ 50}\right)+\left(\frac{€ 101}{€ 101+€ 190}\right)+\left(\frac{\pounds 40}{€ 40+\pounds 60}\right)}{3}=0.3601 \\
w_{e}=\frac{\left(\frac{€ 50}{€ 25+€ 50}\right)+\left(\frac{€ 190}{€ 101+€ 190}\right)+\left(\frac{\pounds 60}{\pounds 40+\pounds 60}\right)}{3}=0.6399
\end{gathered}
$$

\begin{enumerate}
  \setcounter{enumi}{2}
  \item Suppose Gewicht announces that a debt-to-equity ratio of 0.7 reflects its target capital structure. What weights should Anziell use in the cost of capital calculations?
\end{enumerate}

\section{Solution}
A debt-to-equity ratio of 0.7 represents a weight on debt of $0.7 / 1.7=0.4118$, so $w_{d}=0.4118$ and $w_{e}=1-0.4118=0.5882$. These would be the preferred weights to use in a cost of capital calculation.

\section{Pecking Order Theory and Agency Costs}
Investors understand that management has access to more information about a business and its prospects than they do-known as asymmetric information (or information asymmetry)-and that management will take this information into account when raising capital. Providers of both equity and debt capital demand higher returns from companies with higher information asymmetry and will be concerned either that a company will issue shares when its shares are expensive or that it will issue debt when its creditworthiness is about to deteriorate-for example, due to higher investment or acquisitions.

Some degree of asymmetric information always exists because investors never know as much as managers and other insiders do. Consequently, investors often closely watch manager behavior for insight into insider opinions on the company's prospects. Aware of this scrutiny, managers consider how their actions might be interpreted by outsiders. The signaling model of capital structure suggests a hierarchy ("pecking order") to managers' selection of methods for financing new investments. The pecking order theory, developed by Myers and Majluf (1984), suggests that managers choose methods of financing according to a hierarchy that gives first preference to methods with the least potential information content (internally generated funds) and lowest preference to methods with the most potential information content (public equity offerings). Public equity offerings are often closely scrutinized because investors are typically skeptical that existing owners would share ownership of a company that has a great future with other investors. In brief, managers prefer internal financing. If internal financing is insufficient, managers next prefer debt, then equity.

Another implication of the work of Myers and Majluf is that financial managers tend to issue equity when they believe the stock is overvalued, but they are reluctant to issue equity if they believe the stock is undervalued, potentially choosing instead to buy back shares. Thus, additional issuance of equity is often interpreted by investors as a negative signal.

We can read the signals that managers provide in their choice of financing method. Managers can signal to the market their confidence in a company's prospects through the issuance of debt, which commits the company to future obligations (so-called debt signaling). For example, commitments to fixed payments, such as dividends and debt service payments, may be interpreted as the company's management having confidence in the company's future ability to make payments. Such signals are considered too costly for poorly performing companies to afford.

Alternatively, the signal of raising money at the top of the pecking order and issuing equity at the bottom of the pecking order holds other clues. If, for instance, the company's cost of capital increases after an equity issuance, we can interpret this effect as an indication that management needs capital beyond what comes cheaply; in other words, this is a negative signal regarding the company's prospects. Managers may hesitate to issue new equity when they believe the company's shares are underpriced, because they wish to avoid signaling that they believe the shares are overpriced and also to avoid the cost and effort involved with new equity issuance.

Information asymmetry and "signaling" about the value of a company's shares can also become important factors when there are share sales in a company that has a controlling shareholder. Investors will look for signals about the controlling shareholder's confidence in the value of the shares and reasons for selling shares. For instance, if the company is issuing new equity, is the controlling shareholder participating? If there is no new equity being issued, what is the stated reason for the sale and how convincing is that reason?

When a private equity owner takes a portfolio company public, it is typically assumed that it will eventually sell, often in stages over time. With companies controlled by founders or their families, this is less likely to be the case, although it is common for founders to seek to reduce their stake in order to decrease their exposure to a single business or for succession management reasons. In any case, public investors will assume that controlling shareholders will sell their shares at opportune times, taking advantage of favorable market conditions and/or their superior knowledge of the business.

Agency costs are the incremental costs that arise from conflicts of interest between managers, shareholders, and bondholders. Additionally, agency theory predicts that a reduction in net agency costs of equity results from an increase in the use of debt versus equity. That is, there are savings in the agency costs of equity associated with the use of debt. Similarly, the more financially leveraged a company, the less freedom for managers to either take on more debt or spend cash unwisely. This is the foundation of Michael Jensen's (1986)free cash flow hypothesis: higher debt levels discipline managers by forcing them to manage the company efficiently so the company can make its interest and principal payments and by reducing the company's free cash flow and thus managers' opportunities to misuse cash.

\section{EXAMPLE 13}
\section{Agency Costs, Asymmetric Information, and Signaling}
Cloudy Prospects PLC, a small pharmaceutical company, is testing the potential for its primary drug, which is already in commercial production, to be used in a new application for which it was not originally intended: treatment of patients with a widespread virus. The company is managed by a team of medical scientists. Management hopes this new application for the drug will result in a big increase in sales and the value of the company. The company has been consistently profitable and debt-free.

Over the course of three months, the following events occur:

In June, management monitors test results closely. Results are promising; the share price is stable.

In July, test results continue to look promising. Anticipating that it might have to scale up production quickly, the company negotiates and announces a large increase in its credit line.

In August, Cloudy announces successful test results. The CEO receives a call from a headhunter, who thinks the CEO might be a good candidate to lead a larger competitor. The CEO takes the call and decides to meet the headhunter.

Which, if any, of the following are represented here, and by which event and when?

\section{Agency costs}
\section{Solution}
Agency costs. The CEO's decision in August to take the headhunter meeting about a possible opportunity with a competitor was unwise. If the $\mathrm{CEO}$ were acting solely in the interests of the shareholders of Cloudy Prospects, he would not take the meeting.

\section{Asymmetric information}
\section{Solution}
Asymmetric information. Management had knowledge of the positive test results in June and July, before they were publicly announced in August.

\section{Signaling}
\section{Solution}
Signaling. The company's announcement in July that it was increasing its credit line could reasonably be taken as a signal of its confidence that test results will be positive and that it will need to expand production capacity. Note: This is true whether or not the signaling is intentional.

\section{STAKEHOLDER INTERESTS}
describe competing stakeholder interests in capital structure decisions We now look at how and why stakeholder interests are affected when specific capital structure decisions are made by management. Stakeholder groups that we will consider include providers of capital, such as shareholders and debtholders, as well as others, such as management and the board. Key stakeholder groups are shown in Exhibit 11.

\section{Exhibit 11: Key Stakeholder Groups}
Board of directors
\includegraphics[max width=\textwidth, center]{2023_05_04_7b535d0a870224f62e3dg-087}

Capital structure decisions impact stakeholder groups differently. In general, higher financial leverage increases risk for all stakeholders to varying degrees. However, the benefits of increased leverage, primarily through higher returns, accrue almost entirely to shareholders, often including senior management and the board. As a result, shareholders seeking higher returns from the company-and managers acting on their behalf-might prefer decisions that increase the company's financial leverage, while debtholders and other stakeholders in the company generally will not.

\section{Debt vs. Equity Conflict}
High leverage increases the potential conflict of interest between debtholders and shareholders. At high levels of leverage, the risk-return profile of shareholders becomes more asymmetric. The downside for equity investors is never more than $100 \%$, or the full value of their investment, assuming no leverage or shorting. For highly leveraged companies, the upside reward for shareholders can be multiples of their initial investment. Meanwhile, for a holder of debt to maturity, the return, or upside, remains the face value of the debt plus the coupon. However, downside risk for a debtholder increases with higher leverage and the increased probability that the company will be unable to meet its outstanding debt obligations. Similarly, lower leverage levels decrease downside risk for a debtholder as the probability that the company can meet its debt obligations increases.

Let's look at a case study.

\section{EXAMPLE 14}
\section{Acme Holdings Case Study}
Acme Holdings (Acme), a hypothetical company, is planning to make an investment. Details are below.

Acme business valuation

USD1 billion

New investment

USD500 million

New investment financing

Debt: 50\% = USD250 million

Equity: 50\% = USD250 million

Ending value of new investment, T1

USD1.2 billion (50\% probability)

USD0 ( $50 \%$ probability)

Expected value of new investment, $E(V)$

USD600 million or

\begin{center}
\includegraphics[max width=\textwidth]{2023_05_04_7b535d0a870224f62e3dg-088}
\end{center}

NPV of new investment

USD100 million = (USD600 million - USD500 million $)$

Post-investment Acme valuation

USD2.2 billion (50\% probability) = (USD1.0 billion + USD1.2 billion)

USD1.0 billion (50\% probability) = (USD1.0 billion + USD0)

Three capital structure scenarios, each with a different starting leverage for Acme, are compared for their impact on Acme's shareholders and debtholders. The scenarios are as follows:

\section{Scenario 1. Normal leverage}
Acme has a modest amount of debt (USD250 million), consistent with an investment-grade rating and composing a minority (25\%) of its capital structure, with equity worth USD750 million.

\section{Scenario 2. High leverage}
Acme is highly leveraged, with debt composing the majority (USD900 million, or $90 \%$ ) of its capital structure and equity of USD100 million (10\%).

\section{Scenario 3. Financial distress}
Acme has debt of USD1.5 billion, exceeding the value of its business; the intrinsic value of its equity is therefore zero.

\section{Discussion}
This case illustrates how the outcomes for shareholders and debtholders are different in each scenario and how management decisions may or may not compromise the interests of debtholders.

It is not necessary that you memorize or be able to replicate the calculations in this example. The purpose is simply to illustrate a debt-equity conflict of interest numerically.

All values are shown in millions

\section{Scenario 1 (normal leverage)}
\begin{center}
\begin{tabular}{lcc}
\hline
 & Debt & Equity \\
\hline
Value $T_{0}$ & 250 & 750 \\
+ New Financing & 250 & 250 \\
$=$ Value $T_{1}$ & 500 & 1,000 \\
\hline
\end{tabular}
\end{center}

$\begin{array}{lllll}\text { 1. Normal leverage } & \text { Outcome } & \text { ACME } \boldsymbol{T}_{1} & \text { Debtholder } \boldsymbol{T}_{1} & \text { Equityholder } \boldsymbol{T}_{1} \\ \text { Debt: } .25 \text { billion } & \text { Upside } & 2,200 & 500 & 1,700 \text { (or } 2,200-500 \text { ) } \\ \text { Equity: } .75 \text { billion } & \text { Downside } & 1,000 & 500 & 500 \text { (or } 1,000-500) \\ & \boldsymbol{E}(\boldsymbol{V}) & & 500 & 1,100 \text { (or } .5(1,700)+.5(500) \text { ) } \\ & \boldsymbol{E}(\text { Impact on value) } & & 0 & 100 \text { (or } 1,100-750-250)\end{array}$

In Scenario 1, the value of the business in both the upside outcome (2.2 billion) and the downside outcome (1.0 billion) is sufficient to easily repay debtholder value, or post-investment debt, of 500 million. Debtholders are not disadvantaged by the investment since they will receive proceeds regardless of whether the investment fails or succeeds. The expected impact on value is equal to $(E(V)$ - Debt value $\left.\mathrm{T}_{1}\right)$, or $(500-500)=0$, or neutral for debtholders.

For equityholders, the expected return is positive. The expected value remaining for equityholders is 1.1 billion. Equityholders benefit because this expected value is higher than it would be, by 100 million, if the investment were not made $(750$ million) plus the 250 million of new investment (1.1 billion - 1.0 billion). In this scenario, there is no debt-equity conflict.

\section{Scenario 2 (high leverage)}
\begin{center}
\begin{tabular}{lcc}
\hline
 & Debt & Equity \\
\hline
Value $T_{0}$ & 900 & 100 \\
+ New Financing & 250 & 250 \\
$=$ Value $T_{1}$ & 1,150 & 350 \\
\hline
\end{tabular}
\end{center}

\begin{center}
\begin{tabular}{|c|c|c|c|c|}
\hline
$\begin{array}{l}\text { 2. High leverage } \\ \text { Debt: } .9 \text { billion }\end{array}$ & $\begin{array}{l}\text { Outcome } \\ \text { Upside }\end{array}$ & $\begin{array}{l}\text { ACME } T_{1} \\ 2,200\end{array}$ & $\begin{array}{l}\text { Debtholder } \boldsymbol{T}_{1} \\ 1,150\end{array}$ & $\begin{array}{l}\text { Equityholder } \boldsymbol{T}_{1} \\ 1,050 \text { (or } 2,200-1,150 \text { ) }\end{array}$ \\
\hline
\multirow[t]{3}{*}{Equity: .1 billion} & Downside & 1,000 & 1,000 (loss) & 0 (loss) \\
\hline
 & $E(V)$ &  & 1075 & $525($ or $.5(1,050)+0)$ \\
\hline
 & $\boldsymbol{E}$ (Impact on value) &  & -75 & 175 or $525-100-250$ \\
\hline
\end{tabular}
\end{center}

For Scenario 2 (high leverage), we calculate the results the same way, but a conflict arises. In this scenario, post-investment debt equals 1,150 million. In the upside outcome, debtholders are fully covered; however, in the downside outcome, there is more debt to repay (1,150 million) than the business is worth (1.0 billion). This potential for loss makes the investment negative from a debtholder perspective, given that debtholders do not participate in any upside. Reducing the expected value of 1,075 by the post-investment value of 1,150 , or Debt value $T_{1}$, results in an expected impact on value of -75 .

From an equityholder perspective, the investment is positive. Although there is a $50 \%$ chance equityholders will receive nothing, given that debt exceeds the value of the business in the downside outcome, they have a $50 \%$ chance that the business will be worth 2.2 billion, making their equity worth 800 million (2.2 billion -1.15 billion debt -250 million equity investment), which compares very favorably with the 100 million starting equity value without the investment. Expected value is positive at 525 million, and after reducing this amount by the starting equity value and the investment, the expected impact on value is 175 . If management and the board do what maximizes value for equity, they would make the investment-to the detriment of debtholders. In this case, there is a debt-equity conflict.

Scenario 3 (financial distress)

\begin{center}
\begin{tabular}{|c|c|c|c|c|}
\hline
 &  & Debt &  & Equity \\
\hline
Value $T_{0}$ &  & 1,500 &  & 0 \\
\hline
+ New Financing &  & 250 &  & 250 \\
\hline
$=$ Value $T_{1}$ &  & 1,750 &  & 250 \\
\hline
$\begin{array}{l}\text { 3. Financial distress } \\ \text { Debt: } 1.5 \text { billion }\end{array}$ & $\begin{array}{l}\text { Outcome } \\ \text { Upside }\end{array}$ & $\begin{array}{l}\text { ACME } \boldsymbol{T}_{1} \\ 2,200\end{array}$ & $\begin{array}{l}\text { Debtholder } \boldsymbol{T}_{1} \\ 1,750\end{array}$ & $\begin{array}{l}\text { Equityholder } \boldsymbol{T}_{1} \\ 450 \text { (or 2,200-1,750) }\end{array}$ \\
\hline
\multirow[t]{3}{*}{Equity: 0} & Downside & 1,000 & 1,000 (loss) & 0 (loss) \\
\hline
 & $E(V)$ &  & 1,375 & $225($ or $.5(450)+0))$ \\
\hline
 & $\boldsymbol{E}$ (lmpact on value) &  & 125 & -25 (or $225-250$ ) \\
\hline
\end{tabular}
\end{center}

For Scenario 3, the impacts are again different. The value of the business before the proposed investment (1.0 billion) is so much less than the amount of debt outstanding (1.5 billion) that equityholders would not see an increase in value, despite the fact that the investment would have a positive 100 million NPV. However, netting the expected value of 225 for the equity investment results in a negative impact on value of -25 . For debtholders, the expected impact on value is positive at 125 (or 1,375 - 1,000 - 250).In this case, making the investment would create positive value for Acme, but in seeking to maximize equity value, management would not make the investment because the return to equityholders is negative. Management's decision compromises the interests of its bondholders for the interests of its equityholders. In this case, there is a debt-equity conflict. In summary, management is likely to make the investment when equityholders are expected to benefit, even if the impact on debtholders is negative, resulting in a debt-equity conflict. Management is unlikely to make the investment when the impact on equityholders is negative, even if the impact on bondholders and the project's NPV is positive.

\section{Distressed debt: observations}
The Acme Holdings example shows a distress scenario: The value of debt outstanding exceeds the value of the enterprise, and the company's equity has negative intrinsic value. Given this scenario, it would be difficult for Acme to raise new equity capital, and management might decline to make investments that would increase enterprise value. In this situation, a financial restructuring is often a logical course of action. Restructuring can be involuntary, such as when the company is in default, or voluntary. In either case, by exchanging debt for equity, the company's capital structure can be repaired, new investments can be financed, and the company can avoid liquidationwhich is a very costly type of financial distress.

When leverage and financial risk are high, the perspective of debtholders becomes more like that of equityholders. Normally, debt has low default risk and is priced to provide a modest yield to maturity. Return, or upside, is limited, unless the general level of interest rates declines. A holder of debt to maturity will be very focused on minimizing downside risk. However, with higher-risk debt and particularly with distressed debt, the potential upside can be substantial if the company does not default or if recoveries in default are higher than expected. In these circumstances, the return profile (and thus the perspective) of the debtholder becomes closer to that of an equity investor.

\section{Other debt considerations}
\section{Seniority and security}
Debtholder returns are influenced not only by the probability of default but also by the likely recoveries in the event of default. In a default scenario, secured lenders are likely to recover more of their principal than unsecured lenders; likewise, senior debt is likely to recover more than subordinated debt. Recall that senior debt is often secured, but not always. As a result, secured and senior lenders can better tolerate actions by management that increase the probability of default, such as new investment by the company or increasing the level of unsecured or subordinated debt.

\section{Safeguards for debtholders}
Management generally seeks to maximize equity value and thus may take actions that are unfavorable to debtholders. However, debtholders have the following safeguards to protect their interests:

\begin{itemize}
  \item Debt covenants can be used to put contractual limits on future borrowing by the company and can limit other management actions that would increase risk to debtholders. Positive covenants state what the borrower must do, such as maintaining financial ratios within certain thresholds. Ratios commonly used in debt covenants are drawn from the income and cash flow statements (cash coverage of interest or fixed charges), the balance sheet (minimum tangible net worth, debt-to-equity ratio), or both (debt-to-EBITDA ratio). Negative covenants state what the borrower cannot do, such as additional secured borrowing or dividend payments.

  \item Companies generally want to preserve access to credit on favorable terms and so will act accordingly to maintain or improve their creditworthiness, since they may need to borrow in the future. Management's track recordits statements and actions with respect to capital structure, investment, M\&A, dividends, and strategy-will impact the company's future credit rating, banking relationships, and cost of borrowing.

  \item Financial distress costs can be substantial. Even with a primary focus on maximizing common share value, management will generally seek to keep default risk-and other stakeholder impacts from high leverage-at very low levels.

\end{itemize}

\section{Preferred Shareholders}
As with common shareholders, preferred shareholders provide long-term capital, with no maturity (repayment) date. The issuance of preferred equity, therefore, creates less risk for the company's debtholders and common shareholders than if the company were to issue debt. From a common shareholder perspective, whether issuing preferred equity is more attractive than issuing debt depends on the relative after-tax borrowing cost for the company.

Since they are providers of permanent capital and lack covenant protection, preferred shareholders are vulnerable to decisions that increase financial leverage and risk over the long term. For example, a company in a mature business might commit to a long-term policy of dividend growth and share repurchases. This strategy would result in higher financial leverage over the long term. Existing senior secured debtholders might be unconcerned, particularly for debt maturing relatively soon, since there is minimal default risk. However, preferred shareholders might be concerned that the new policy could gradually erode the company's capacity to pay preferred dividends.

\section{Management and Directors}
It is critical to consider the interests of management and the board since they make capital structure and business strategy decisions. In practice, compensation is the principal tool used to create alignment of interests between management and directors and shareholders. Equity compensation can be a substantial portion of management's total compensation.

In principle, management compensation is intended to motivate managers to work hard to maximize long-term shareholder value. However, the alignment of interests between managers and shareholders is never perfect. For example, the interests of owners are not necessarily identical to the interests of managers, who happen to receive shares and options as part of their compensation.

The following highlights this aspect of management compensation.

\section{EXAMPLE 15}
\section{Management Stock Options and Leverage}
The company Binary Outcomes Ltd. is debt-free and has a share price of USD100, with 1 million shares outstanding. Its intrinsic value is USD100 million (USD100 $\times 1$ million), which is based on a $50 \%$ probability of a USD110/share value and a $50 \%$ probability of a USD $90 /$ share value. If management has options exercisable at USD100, their current expected value is USD5.

Is it in management's interest to increase leverage at the company, such as by issuing USD50 million of debt to buy back shares?

\section{Discussion}
As shown in Exhibit 12, it would be in management's interest to increase leverage, such as by issuing debt to buy back 50\% of the company's shares at USD100 per share. This action would increase debt by USD50 million (purchase of 0.5 million shares at USD100 per share) and reduce equity value by an equivalent amount, creating a $50 \%$ probability of a USD120/share value and a $50 \%$ probability of an USD80/share value. Although there is no change in shareholder value, the leverage results in an increase in share value (USD120) and corresponding option value (USD20) in the upside scenario, while the downside option value remains USD0, with an average expected option value of USD10. The use of leverage doubles the average expected value of management's options from USD5 to USD10.

\section{Exhibit 12: Binary Outcomes Leverage Scenario}
\begin{center}
\begin{tabular}{|c|c|c|c|}
\hline
Scenario & Upside & Downside & Expected Value \\
\hline
Probability & $50 \%$ & $50 \%$ &  \\
\hline
Firm value & USD110,000,000 & USD90,000,000 & USD $100,000,000$ \\
\hline
\multicolumn{4}{|l|}{No leverage} \\
\hline
Debt & - & - &  \\
\hline
Equity value & USD110,000,000 & USD90,000,000 &  \\
\hline
Shares outstanding & $1,000,000$ & $1,000,000$ &  \\
\hline
Share price & USD110 & USD90 & USD100 \\
\hline
Option exercise price & USD100 & USD100 &  \\
\hline
Option value & USD10 & - & USD5 \\
\hline
\multicolumn{4}{|l|}{With leverage} \\
\hline
Debt & USD50,000,000 & USD50,000,000 &  \\
\hline
Equity value & USD60,000,000 & USD40,000,000 &  \\
\hline
Shares outstanding & 500,000 & 500,000 &  \\
\hline
Share price & USD120 & USD80 & USD100 \\
\hline
Option exercise price & USD100 & USD100 &  \\
\hline
Option value & USD20 & - & USD10 \\
\hline
\end{tabular}
\end{center}

With typical management share option schemes, rather than paying dividends, management will be motivated to buy back shares with excess cash. This strategy benefits them by increasing the value of their share options. Similarly, management would be incentivized to retain profits rather than distribute them as a dividend to shareholders.

Continuing this example, if Binary Outcomes earns a profit of USD10/share and adds USD10/share in cash, retaining that profit would increase its intrinsic value from USD100/share to USD110/share, thus increasing the intrinsic value of management's options by USD5 (from USD5 to USD10). If the entire profit is instead distributed as a dividend, however, there would be no increase in intrinsic value for the shares or management's options.

To mitigate this issue, some companies pay executive performance bonuses in shares rather than in options or have minimum share ownership requirements for senior managers and, in some cases, directors.

In mature or stable businesses, compensation can skew heavily toward salaries and bonuses tied to "easy" targets, and management complacency can be a risk. The business may not have enough upside to attract highly talented managers or enough influence to make management aggressively pursue incremental sales or cost efficiencies. In these cases, adding financial leverage is a common strategy, increasing both the incentive for and the pressure on management to maximize value.

When managers stay in their positions for only a few years, they may be motivated to pursue short-term rather than long-term objectives: They want to demonstrate success and keep their jobs, and they are unsure whether they will be with the company long enough to benefit from longer-term initiatives. From a shareholder perspective, the solution is not to avoid management turnover but, rather, to ensure that it is based on performance (i.e., evaluated objectively and carefully) instead of on non-performance-related reasons.

Some of these misalignments offset each other. For example, stock options motivate risk-taking, while high cash compensation can create excessive caution and entrenchment.

\section{SUMMARY}
\begin{itemize}
  \item Financing decisions are typically tied to investment spending and are based on the company's ability to support debt given the nature of its business model, assets, and operating cash flows.

  \item A company's stage in the life cycle, its cash flow characteristics, and its ability to support debt largely dictate its capital structure, since capital not sourced through borrowing must come from equity (including retained earnings).

  \item Generally speaking, as companies mature and move from start-up through growth to maturity, their business risk declines as operating cash flows turn positive with increasing predictability, allowing for greater use of leverage on more attractive terms.

  \item Modigliani and Miller's work, with its simplifying assumptions, provides a starting point for thinking about the strategic use of debt and shows us that managers cannot change firm value simply by changing the firm's capital structure. Firm value is independent of capital structure decisions.

  \item Given the tax-deductibility of interest, adding leverage increases firm value up to a point but also increases the risk of default for capital providers who demand higher returns in compensation.

  \item To maximize firm value, management should target the optimal capital structure that minimizes the company's weighted average cost of capital.

  \item "Optimal capital structure" involves a trade-off between the benefits of higher leverage, which include the tax-deductibility of interest and the lower cost of debt relative to equity, and the costs of higher leverage, which include higher risk for all capital providers and the potential costs of financial distress.

  \item Managers may provide investors with information ("signaling") through their choice of financing method. For example, commitments to fixed payments may signal management's confidence in the company's prospects.

  \item Managers' capital structure decisions impact various stakeholder groups differently. In seeking to maximize shareholder wealth or their own, managers may create conflicts of interest in which one or more groups are favored at the expense of others, such as a debt-equity conflict.

\end{itemize}

\section{REFERENCES}
Hamada, Robert. 1972. “The Effect of the Firm's Capital Structure on the Systematic Risk of Common Stocks." Journal of Finance 27 (2): 435-52. doi:10.1111/j.1540-6261.1972.tb00971.x

Jensen, Michael C. 1986. "Agency Costs of Free Cash Flow, Corporate Finance, and Takeovers." American Economic Review 76 (2): 323-29.

Modigliani, Franco, and Merton H. Miller. 1958. “The Cost of Capital, Corporation Finance, and the Theory of Investment." American Economic Review 48 (3): 261-97.

Myers, Stewart, and Nicholas S. Majluf. 1984. "Corporate Financing and Investment Decisions When Firms Have Information That Investors Do Not Have." Journal of Financial Economics 13:187-221. doi:10.1016/0304-405X(84)90023-0

\section{PRACTICE PROBLEMS}
\begin{enumerate}
  \item Which of the following is least likely to affect the capital structure of Longdrive Trucking Company? Longdrive has moderate leverage today.
\end{enumerate}

A. The acquisition of a major competitor for shares

B. A substantial increase in share price

C. The payment of a stock dividend

\begin{enumerate}
  \setcounter{enumi}{1}
  \item Which of these statements is most accurate with respect to the use of debt by a start-up fashion retailer with negative cash flow and uncertain revenue prospects?
\end{enumerate}

A. Debt financing will be unavailable or very costly.

B. The company will prefer to use equity rather than debt given its uncertain cash flow outlook.

C. Both $\mathrm{A}$ and $\mathrm{B}$.

\begin{enumerate}
  \setcounter{enumi}{2}
  \item Which of the following is true of the growth stage in a company's development?
\end{enumerate}

A. Cash flow is negative, by definition, with investment outlays exceeding cash flow from operations.

B. Cash flow may be negative or positive.

C. Cash flow is positive and growing quickly.

\begin{enumerate}
  \setcounter{enumi}{3}
  \item Which of the following mature companies is most likely to use a high proportion of debt in its capital structure?
\end{enumerate}

A. A mining company with a large, fixed asset base

B. A software company with very stable and predictable revenues and an asset-light business model

C. An electric utility

\begin{enumerate}
  \setcounter{enumi}{4}
  \item Which of the following is most likely to occur as a company evolves from growth stage to maturity and seeks to optimize its capital structure?
\end{enumerate}

A. The company relies on equity to finance its growth.

B. Leverage increases as the company needs more capital to support organic expansion.

C. Leverage increases as the company is able to support more debt.

\begin{enumerate}
  \setcounter{enumi}{5}
  \item If investors have homogeneous expectations, the market is efficient, and there are no taxes, no transaction costs, and no bankruptcy costs, Modigliani and Miller's Proposition I states that:
\end{enumerate}

A. bankruptcy risk rises with more leverage.

B. managers cannot change the value of the company by changing the amount of debt. C. managers cannot increase the value of the company by using tax-saving strategies.

\begin{enumerate}
  \setcounter{enumi}{6}
  \item According to Modigliani and Miller's Proposition II without taxes:
A. the capital structure decision has no effect on the cost of equity.
B. investment and capital structure decisions are interdependent.
C. the cost of equity increases as the use of debt in the capital structure increases.

  \item The weighted average cost of capital (WACC) for Van der Welde is $10 \%$. The company announces a debt offering that raises the WACC to $13 \%$. The most likely conclusion is that for Van der Welde:

\end{enumerate}

A. the company's prospects are improving.

B. equity financing is cheaper than debt financing.

C. the company's debt/equity has moved beyond the optimal range.

\begin{enumerate}
  \setcounter{enumi}{8}
  \item According to the static trade-off theory:
A. debt should be used only as a last resort.
B. companies have an optimal level of debt.
C. the capital structure decision is irrelevant.
\end{enumerate}

\section{The following information relates to questions}
\section{0-12}
Nailah Mablevi is an equity analyst who covers the entertainment industry for Kwame Capital Partners, a major global asset manager. Kwame owns a significant position, with a large unrealized capital gain, in Mosi Broadcast Group (MBG). On a recent conference call, MBG's management stated that they plan to increase the proportion of debt in the company's capital structure. Mablevi is concerned that any changes in MBG's capital structure will negatively affect the value of Kwame's investment.

To evaluate the potential impact of such a capital structure change on Kwame's investment, she gathers the information about MBG given in Exhibit 1.

\section{Exhibit 1: Current Selected Financial Information on MBG}
Yield to maturity on debt

Market value of debt

Number of shares of common stock

Market price per share of common stock

Cost of capital if all equity-financed $8.00 \%$

USD100 million

10 million

USD30

$10.3 \%$

\begin{center}
\begin{tabular}{ll}
\hline
Yield to maturity on debt & $\mathbf{8 . 0 0 \%}$ \\
\hline
Marginal tax rate & $35 \%$ \\
\hline
\end{tabular}
\end{center}

\begin{enumerate}
  \setcounter{enumi}{9}
  \item MBG is best described as currently:
A. $25 \%$ debt-financed and $75 \%$ equity-financed.
B. $33 \%$ debt-financed and $66 \%$ equity-financed.
C. $75 \%$ debt-financed and $25 \%$ equity-financed.

  \item Holding operating earnings constant, an increase in the marginal tax rate to $40 \%$ would:

\end{enumerate}

A. result in a lower cost of debt capital.

B. result in a higher cost of debt capital.

C. not affect the company's cost of capital.

\begin{enumerate}
  \setcounter{enumi}{11}
  \item Which of the following is least likely to be true with respect to optimal capital structure?
\end{enumerate}

A. The optimal capital structure minimizes WACC.

B. The optimal capital structure is generally close to the target capital structure.

C. Debt can be a significant portion of the optimal capital structure because of the tax-deductibility of interest.

\begin{enumerate}
  \setcounter{enumi}{12}
  \item Other factors being equal, in which of the following situations are debt-equity conflicts likely to arise?
\end{enumerate}

A. Financial leverage is low.

B. The company's debt is secured.

C. The company's debt is long-term.

\begin{enumerate}
  \setcounter{enumi}{13}
  \item Which of the following is an example of agency costs? In each case, management is advocating a substantial acquisition and management compensation is heavily composed of stock options.
\end{enumerate}

A. Management believes the acquisition will be positive for shareholder value but negative for the value and interests of the company's debtholders.

B. Management's stock options are worthless at the current share price. The acquisition has a high (50\%) risk of failure (with zero value) but substantial (30\%) upside if it works out.

C. The acquisition is positive for equityholders and does not significantly impair the position of debtholders. However, the acquisition puts the company into a new business where labor practices are harsh and the production process is environmentally damaging. 15. Which of the following is least accurate with respect to debt-equity conflicts?

A. Equityholders focus on potential upside and downside outcomes, while debtholders focus primarily on downside risk.

B. Management attempts to balance the interests of equityholders and debtholders.

C. Debt covenants can mitigate the conflict between debtholders and equityholders.

\begin{enumerate}
  \setcounter{enumi}{15}
  \item Which of the following is least likely to be true with respect to agency costs and senior management compensation?
\end{enumerate}

A. Equity-based incentive compensation is the primary method to address the problem of agency costs.

B. A well-designed compensation scheme should eliminate agency costs.

C. High cash compensation for senior management, without significant equity-based performance incentives, can lead to excessive caution and complacency.

\begin{enumerate}
  \setcounter{enumi}{16}
  \item Integrated Systems Solutions Inc. (ISS) is a technology company that sells software to companies in the building construction industry. The company's assets consist mostly of intangible assets. Although the company is profitable, revenue growth and earnings growth have been slowing in recent years. The company's business model is a pay-per-use model, and given the cyclical nature of the construction industry, the company's revenues and earnings vary considerably over the business cycle.
\end{enumerate}

Describe two factors that would point to ISS having a relatively high cost of borrowing and low proportion of debt in its capital structure.

\begin{enumerate}
  \setcounter{enumi}{17}
  \item Tillett Technologies is a manufacturer of high-end audio and video (AV) equipment. The company, with no debt in its capital structure, has experienced rapid growth in revenues and improved profitability in recent years. About half of the company's revenues come from subscription-based service agreements. The company's assets consist mostly of inventory and property, plant, and equipment, representing its production facilities. Now, the company seeks to raise new capital to finance additional growth.
\end{enumerate}

Describe two factors that would support Tillett being able to access debt capital at a reasonable cost to finance the additional growth. Justify your response.

\begin{enumerate}
  \setcounter{enumi}{18}
  \item Discuss two financial metrics that can be used to assess a company's ability to service additional debt in its capital structure.

  \item Identify two market conditions that can be characterized as favorable for companies wishing to add debt to their capital structures.

  \item Which of the following is least accurate with respect to the market value and book value of a company's equity?

\end{enumerate}

A. Market value is more relevant than book value when measuring a company's cost of capital.

B. Book value is often used by lenders and in financial ratio calculations.

C. Both market value and book value fluctuate with changes in the company's share price. 22. Fran McClure of Alba Advisers is estimating the cost of capital of Frontier Corporation as part of her valuation analysis of Frontier. McClure will be using this estimate, along with projected cash flows from Frontier's new projects, to estimate the effect of these new projects on the value of Frontier. McClure has gathered the following information on Frontier Corporation:

\begin{center}
\begin{tabular}{lcc}
\hline
 & Current Year (USD) & $\begin{array}{c}\text { Forecasted for Next } \\ \text { Year (USD) }\end{array}$ \\
\hline
Book value of debt & 50 & 50 \\
Market value of debt & 62 & 63 \\
Book value of shareholders' equity & 55 & 58 \\
Market value of shareholders' equity & 210 & 220 \\
\hline
\end{tabular}
\end{center}

The weights that McClure should apply in estimating Frontier's cost of capital for debt and equity are, respectively:
A. $w_{d}=0.200 ; w_{e}=0.800$.
B. $w_{d}=0.185 ; w_{e}=0.815$.
C. $w_{d}=0.223 ; w_{e}=0.777$.

\begin{enumerate}
  \setcounter{enumi}{22}
  \item Which of the following is not a reason why target capital structure and actual capital structure tend to differ?
\end{enumerate}

A. Financing is often tied to a specific investment.

B. Companies raise capital when the terms are attractive.

C. Target capital structure is set for a particular project, while actual capital structure is measured at the consolidated company level.

\begin{enumerate}
  \setcounter{enumi}{23}
  \item According to the pecking order theory:
A. new debt is preferable to new equity.
B. new debt is preferable to internally generated funds.
C. new equity is always preferable to other sources of capital.

  \item Vega Company has announced that it intends to raise capital next year, but it is unsure as to the appropriate method of raising capital. White, the CFO, has concluded that Vega should apply the pecking order theory to determine the appropriate method of raising capital. Based on White's conclusion, Vega should raise capital in the following order:

\end{enumerate}

A. debt, internal financing, equity.

B. equity, debt, internal financing.

C. internal financing, debt, equity.

\section{SOLUTIONS}
\begin{enumerate}
  \item C is correct. Stock dividends, like stock splits, have no impact on the value of a company's equity. Issuing shares to acquire a competitor would increase equity relative to debt in the capital structure. Share price appreciation would also increase the market value of equity, thus increasing equity relative to debt.

  \item C is correct. For a start-up company of this nature, debt financing is likely to be unattractive to lenders-and therefore very expensive or difficult to obtain. Debt financing is also unappealing to the company, because it commits the company to interest and principal payments that might be difficult to manage given the company's uncertain cash flow outlook.

  \item B is correct. Cash flow typically turns positive during the growth stage, but it may be negative, particularly at the beginning of this stage.

  \item C is correct. An electric utility has the capacity to support substantial debt, with very stable and predictable revenues and cash flows. The software company also has these attributes, but it would have been much less likely to have raised debt during its development and may have raised equity. The mining company has fixed assets, which it would have needed to finance, but the cyclical nature of its business would limit its debt capacity.

  \item C is correct. As cash flows become more predictable, the company is able to support more debt in its capital structure; the optimal capital structure includes a higher proportion of debt. While mature companies do borrow to support growth, this action would typically not occur because the company is optimizing its capital structure. Likewise, while a mature company might issue equity to finance growth, this action would not be the typical approach for a company optimizing its capital structure.

  \item B is correct. Proposition I, or the capital structure irrelevance theorem, states that in perfect markets the level of debt versus equity in the capital structure has no effect on company value.

  \item $C$ is correct. The cost of equity rises with the use of debt in the capital structure (e.g., with increasing financial leverage).

  \item C is correct. If the company's WACC increases as a result of taking on additional debt, the company has moved beyond the optimal capital range. The costs of financial distress may outweigh any tax benefits from the use of debt.

  \item B is correct. The static trade-off theory indicates that there is a trade-off between the tax shield for interest on debt and the costs of financial distress, leading to an optimal amount of debt in a company's capital structure.

  \item A is correct. The market value of equity is $\operatorname{USD} 30)(10,000,000)=$ USD300,000,000. With the market value of debt equal to USD100,000,000, the market value of the company is USD100,000,000 + USD300,000,000 = USD400,000,000. Therefore, the company is USD100,000,000/USD400,000,000 $=$ 0.25 , or $25 \%$ debt-financed.

  \item A is correct. The after-tax cost of debt decreases as the marginal tax rate increases.

  \item B is correct. A company's optimal and target capital structures may be different from each other.

  \item $\mathrm{C}$ is correct. Long-term debt is more exposed than short-term debt to the risk of a management decision that is not debtholder-friendly. Secured debt is less exposed than unsecured debt to such a risk, and with low leverage, the risk of a debt-equity conflict is reduced, not increased, relative to high leverage.

  \item B is correct. Management is advocating an acquisition that is likely to be positive for the value of the company's options but negative for equityholders, given the substantial risk. A is an example of a debt-equity conflict. $\mathrm{C}$ is an example of stakeholder interests that are not being considered by management.

  \item B is correct. Management is generally focused on maximizing the value of equity.

  \item B is correct. A well-designed management compensation scheme can reduce, but not eliminate, agency costs.

  \item The cyclical nature of ISS's revenues, which cause the company's earnings and cash flows to vary considerably over the business cycle, would point to a relatively high cost of borrowing and low proportion of debt in the capital structure. Revenue and earnings streams subject to relatively high volatility, and consequently less predictability, are less favorable for supporting debt in the capital structure. Further, companies with pay-per-use business models, rather than subscription-based models, are likely to have a lower degree of revenue predictability and a lower ability to support debt in the capital structure.

\end{enumerate}

Another factor pointing to a relatively high cost of borrowing and low proportion of debt in the capital structure is the fact that most of the company's assets are intangible and thus less likely to be accepted by lenders as collateral for secured financing. Asset-light companies with a lower proportion of tangible assets will have a lower ability to support debt in the capital structure.

\begin{enumerate}
  \setcounter{enumi}{17}
  \item The fact that Tillett earns about half of its revenues from subscription-based service agreements would suggest that the company's revenue stream is likely somewhat predictable. A high proportion of recurring revenues for a company is generally viewed as a positive for its ability to support debt, because the company's revenue stream is likely to be more predictable and less sensitive to the ups and downs of the macro economy. Further, Tillett's assets consist mostly of inventory and property, plant, and equipment, representing its production facilities. Tangible assets, such as inventory and property, plant, and equipment, are often deemed safer than intangible assets and can better serve as debt collateral. Finally, the fact that Tillett currently has no debt in its capital structure and has experienced improved profitability in recent years would also suggest that Tillett might be able to access debt capital at a reasonable cost to finance the additional growth.

  \item Leverage ratios and interest coverage ratios are commonly used to determine whether a company can service additional debt. Regarding leverage ratios, a company's ratio of total debt to total assets measures the proportion of total assets funded by debt capital, and its ratio of total debt to EBITDA provides an estimate of how many years it would take to repay its total debt based on EBITDA (a proxy for operating cash flow). The interest coverage ratio (EBIT to interest expense) measures the number of times a company's EBIT could cover its interest payments.

  \item A company's cost of debt is equal to a risk-free rate plus a credit spread specific to the company. Lower interest rates and tighter credit spreads would make borrowing less costly and make debt financing relatively more attractive than when interest rates are high or credit spreads are wide.

  \item $C$ is correct. Share price changes will cause the market value of the company's equity to change; book value is unaffected.

\end{enumerate}

Statements A and B are accurate.

\begin{enumerate}
  \setcounter{enumi}{21}
  \item $\mathrm{C}$ is correct.
\end{enumerate}

$w_{d}=\mathrm{USD} 63 /(\mathrm{USD} 220+\mathrm{USD} 63)=0.223$.

$w_{e}=\mathrm{USD} 220 /(\mathrm{USD} 220+\mathrm{USD} 63)=0.777$

Market values should be used in cost of capital calculations, and forecasted market values should be used in this case given that the cost of capital will be applied to projected cash flows in McClure's analysis.

\begin{enumerate}
  \setcounter{enumi}{22}
  \item $\mathrm{C}$ is correct. Companies generally raise capital when it is needed, such as for investment spending or when market pricing and terms are favorable for debt or equity issuance.

  \item A is correct. According to the pecking order theory, internally generated funds are preferable to both new equity and new debt. If internal financing is insufficient, managers next prefer new debt, then new equity. Managers prefer forms of financing with the least amount of visibility to outsiders.

  \item $\mathrm{C}$ is correct. According to the pecking order theory, managers prefer internal financing. If internal financing is insufficient, managers next prefer debt, then equity-in order of increasing visibility to outsiders. LEARNING MODULE

\end{enumerate}

\begin{center}
\includegraphics[max width=\textwidth]{2023_05_04_7b535d0a870224f62e3dg-105}
\end{center}

\section*{Measures of Leverage }
 Harrington, CFA, and Adam Kobor, PhD, CFA.Pamela Peterson Drake, PhD, CFA, is at James Madison University (USA). Raj

Aggarwal, PhD, CFA, is at the Kent State University Foundation Board (USA). Cynthia

Harrington, CFA, is at \href{http://teamyou.co}{teamyou.co} (USA). Adam Kobor, PhD, CFA, is at New York

University (USA).

\section{LEARNING OUTCOME}
\begin{center}
\begin{tabular}{c|l}
Mastery & The candidate should be able to: \\
$\square \square$ & $\begin{array}{l}\text { define and explain leverage, business risk, sales risk, operating risk, } \\ \text { and financial risk and classify a risk } \\ \text { calculate and interpret the degree of operating leverage, the degree } \\ \text { of financial leverage, and the degree of total leverage } \\ \text { analyze the effect of financial leverage on a company's net income } \\ \text { and return on equity } \\ \text { calculate the breakeven quantity of sales and determine the } \\ \text { company's net income at various sales levels } \\ \text { calculate and interpret the operating breakeven quantity of sales }\end{array}$ \\
$\square$ &  \\
\end{tabular}
\end{center}

\section{INTRODUCTION}
This reading presents elementary topics in leverage. Leverage is the use of fixed costs in a company's cost structure. Fixed costs that are operating costs (such as depreciation or rent) create operating leverage. Fixed costs that are financial costs (such as interest expense) create financial leverage.

Analysts refer to the use of fixed costs as leverage because fixed costs act as a fulcrum for the company's earnings. Leverage can magnify earnings both up and down. The profits of highly leveraged companies might soar with small upturns in revenue. But the reverse is also true: Small downturns in revenue may lead to losses.

Analysts need to understand a company's use of leverage for three main reasons. First, the degree of leverage is an important component in assessing a company's risk and return characteristics. Second, analysts may be able to discern information about a company's business and future prospects from management's decisions about the use of operating and financial leverage. Knowing how to interpret these signals also helps the analyst evaluate the quality of management's decisions. Third, the valuation of a company requires forecasting future cash flows and assessing the risk associated with those cash flows. Understanding a company's use of leverage should help in forecasting cash flows and in selecting an appropriate discount rate for finding their present value.

The reading is organized as follows: Section 2 introduces leverage and defines important terms. Section 3 illustrates and discusses measures of operating leverage and financial leverage, which combine to define a measure of total leverage that gauges the sensitivity of net income to a given percent change in units sold. This section also covers breakeven points in using leverage and corporate reorganization (a possible consequence of using leverage inappropriately). A summary and practice problems conclude this reading.

\section{LEVERAGE}
define and explain leverage, business risk, sales risk, operating risk, and financial risk and classify a risk

Leverage increases the volatility of a company's earnings and cash flows and increases the risk of lending to or owning a company. Additionally, the valuation of a company and its equity is affected by the degree of leverage: The greater a company's leverage, the greater its risk and, hence, the greater the discount rate that should be applied in its valuation. Further, highly leveraged (levered) companies have a greater chance of incurring significant losses during downturns, thus accelerating conditions that lead to financial distress and bankruptcy.

Consider the simple example of two companies, Impulse Robotics, Inc., and Malvey Aerospace, Inc. These companies have the following performance for the period of study: 1

\section{Exhibit 1: Impulse Robotics and Malvey Aerospace}
\begin{center}
\begin{tabular}{lrc}
\hline
 & Impulse Robotics & Malvey Aerospace \\
\hline
Revenues & $\$ 1,000,000$ & $\$ 1,000,000$ \\
Operating costs & 700,000 & 750,000 \\
Operating income & $\$ 300,000$ & $\$ 250,000$ \\
Financing expense & 100,000 & 50,000 \\
Net income & $\$ 200,000$ & $\$ 200,000$ \\
\hline
\end{tabular}
\end{center}

These companies have the same net income, but are they identical in terms of operating and financial characteristics? Would we appraise these two companies at the same value? Not necessarily.

The risk associated with future earnings and cash flows of a company are affected by the company's cost structure. The cost structure of a company is the mix of variable and fixed costs. Variable costs fluctuate with the level of production and sales. Some examples of variable costs are the cost of goods purchased for resale, costs of materials or supplies, shipping charges, delivery charges, wages for hourly employees,

1 We are ignoring taxes for this example, but when taxes are included, the general conclusions remain the same. sales commissions, and sales or production bonuses. Fixed costs are expenses that are the same regardless of the production and sales of the company. These costs include depreciation, rent, interest on debt, insurance, and wages for salaried employees.

Suppose that the cost structures of the companies differ in the manner shown in Exhibit 2.

\section{Exhibit 2: Impulse Robotics and Malvey Aerospace}
\begin{center}
\begin{tabular}{lcc}
\hline
 & Impulse Robotics & Malvey Aerospace \\
\hline
Number of units produced and sold & 100,000 & 100,000 \\
Sales price per unit & $\$ 10$ & $\$ 10$ \\
Variable cost per unit & $\$ 2$ & $\$ 6$ \\
Fixed operating cost & $\$ 500,000$ & $\$ 150,000$ \\
Fixed financing expense & $\$ 100,000$ & $\$ 50,000$ \\
\hline
\end{tabular}
\end{center}

The risk associated with these companies is different, although, as we saw in Exhibit 1 , they have the same net income. They have different operating and financing cost structures, resulting in differing volatility of net income.

For example, if the number of units produced and sold is different from 100,000, the net income of the two companies diverges. If 50,000 units are produced and sold, Impulse Robotics has a loss of $\$ 200,000$ and Malvey Aerospace has $\$ 0$ earnings. If, on the other hand, the number of units produced and sold is 200,000, Impulse Robotics earns $\$ 1$ million whereas Malvey Aerospace earns $\$ 600,000$. In other words, the variability in net income is greater for Impulse Robotics, which has higher fixed costs in terms of both fixed operating costs and fixed financing costs.

Impulse Robotics' cost structure results in more leverage than that of Malvey Aerospace. We can see this effect when we plot the net income of each company against the number of units produced and sold, as in Exhibit 3. The greater leverage of Impulse Robotics is reflected in the greater slope of the line representing net income. This means that as the number of units sold changes, Impulse Robotics experiences a greater change in net income than does Malvey Aerospace for the same change in units sold. Exhibit 3: Net Income for Different Numbers of Units Produced and Sold

\begin{center}
\includegraphics[max width=\textwidth]{2023_05_04_7b535d0a870224f62e3dg-108}
\end{center}

Companies that have more fixed costs relative to variable costs in their cost structures have greater variation in net income as revenues fluctuate and, hence, more risk.

\section{BUSINESS AND SALES RISKS}
define and explain leverage, business risk, sales risk, operating risk, and financial risk and classify a risk

Risk arises from both the operating and financing activities of a company. In the following, we address how that happens and the measures available to the analyst to gauge the risk in each case.

\section{Business Risk and Its Components}
Business risk is the risk associated with operating earnings. Operating earnings are risky because total revenues are risky, as are the costs of producing revenues. Revenues are affected by a large number of factors, including economic conditions, industry dynamics (including the actions of competitors), government regulation, and demographics. Therefore, prices of the company's goods or services or the quantity of sales may be different from what is expected. We refer to the uncertainty with respect to the price and quantity of goods and services as sales risk. Operating risk is the risk attributed to the operating cost structure, in particular the use of fixed costs in operations. The greater the fixed operating costs relative to variable operating costs, the greater the operating risk. Business risk is therefore the combination of sales risk and operating risk. Companies that operate in the same line of business generally have similar business risk.

\section{Sales Risk}
Consider Impulse Robotics once again. Suppose that the forecasted number of units produced and sold in the next period is 100,000 but that the standard deviation of the number of units sold is 20,000. And suppose the price that the units sell for is expected to be $\$ 10$ per unit but the standard deviation is $\$ 2$. Contrast this situation with that of a company named Tolley Aerospace, Inc., which has the same cost structure but a standard deviation of units sold of 40,000 and a price standard deviation of $\$ 4$.

If we assume, for simplicity's sake, that the fixed operating costs are known with certainty and that the units sold and price per unit follow a normal distribution, we can see the impact of the different risks on the operating income of the two companies through a simulation; the results are shown in Exhibit 4. Here, we see the differing distributions of operating income that result from the distributions of units sold and price per unit. So, even if the companies have the same cost structure, differing sales risk affects the potential variability of the company's profitability. In our example, Tolley Aerospace has a wider distribution of likely outcomes in terms of operating profit. This greater volatility in operating earnings means that Tolley Aerospace has more sales risk than Impulse Robotics.

Exhibit 4: Operating Income Simulations for Impulse Robotics and Tolley Aerospace

\section{Panel A: Impulse Robotics}
\begin{center}
\includegraphics[max width=\textwidth]{2023_05_04_7b535d0a870224f62e3dg-110(1)}
\end{center}

\section{Panel B: Tolley Aerospace}
\begin{center}
\includegraphics[max width=\textwidth]{2023_05_04_7b535d0a870224f62e3dg-110}
\end{center}

Operating Profit

\section{OPERATING RISK AND THE DEGREE OF OPERATING LEVERAGE}
define and explain leverage, business risk, sales risk, operating risk, and financial risk and classify a risk

calculate and interpret the degree of operating leverage, the degree of financial leverage, and the degree of total leverage

The greater the fixed component of costs, the more difficult it is for a company to adjust its operating costs to changes in sales. The mixture of fixed and variable costs depends largely on the type of business. Even within the same line of business, companies can vary their fixed and variable costs to some degree. We refer to the risk arising from the mix of fixed and variable costs as operating risk. The greater the fixed operating costs relative to variable operating costs, the greater the operating risk.

Next, we look at how operating risk affects the variability of cash flows. A concept taught in microeconomics is elasticity, which is simply a measure of the sensitivity of changes in one item to changes in another. We can apply this concept to examine how sensitive a company's operating income is to changes in demand, as measured by unit sales. We will calculate the operating income elasticity, which we refer to as the degree of operating leverage (DOL). DOL is a quantitative measure of operating risk as it was defined earlier.

The degree of operating leverage is the ratio of the percentage change in operating income to the percentage change in units sold. We will simplify things and assume that the company sells all that it produces in the same period. Then,

$\mathrm{DOL}=\frac{\text { Percentage change in operating income }}{\text { Percentage change in units sold }}$

For example, if DOL at a given level of unit sales is 2.0 , a 5 percent increase in unit sales from that level would be expected to result in a (2.0)(5\%) 10 percent increase in operating income. As illustrated later in relation to Exhibit 6, a company's DOL is dependent on the level of unit sales being considered.

Returning to Impulse Robotics, the price per unit is $\$ 10$, the variable cost per unit is $\$ 2$, and the total fixed operating costs are $\$ 500,000$. If Impulse Robotics' output changes from 100,000 units to 110,000 units-an increase of 10 percent in the number of units sold-operating income changes from $\$ 300,000$ to $\$ 380,000:^{2}$

Exhibit 5: Operating Leverage of Impulse Robotics

\begin{center}
\begin{tabular}{lccc}
\hline
Item & $\begin{array}{c}\text { Selling 100,000 } \\ \text { Units }\end{array}$ & $\begin{array}{c}\text { Selling 110,000 } \\ \text { Units }\end{array}$ & $\begin{array}{c}\text { Percentage } \\ \text { Change }\end{array}$ \\
\hline
Revenues & $\$ 1,000,000$ & $\$ 1,100,000$ & +10.00 \\
Less variable costs & 200,000 & 220,000 & +10.00 \\
Less fixed costs & 500,000 & 500,000 & 0.00 \\
Operating income & $\$ 300,000$ & $\$ 380,000$ & +26.67 \\
\hline
 &  &  &  \\
\hline
\end{tabular}
\end{center}

Operating income increases by 26.67 percent when units sold increases by 10 percent. What if the number of units decreases by 10 percent, from 100,000 to 90,000? Operating income is $\$ 220,000$, representing a decline of 26.67 percent.

What is happening is that for a 1 percent change in units sold, the operating income changes by 2.67 times that percentage, in the same direction. If units sold increases by 10 percent, operating income increases by 26.7 percent; if units sold decreased by 20 percent, operating income would decrease by 53.3 percent.

We can represent the degree of operating leverage as given in Equation 1 in terms of the basic elements of the price per unit, variable cost per unit, number of units sold, and fixed operating costs. Operating income is revenue minus total operating costs (with variable and fixed cost components):

2 We provide the variable and fixed operating costs for our sample companies used in this reading to illustrate the leverage and breakeven concepts. In reality, however, the financial analyst does not have these breakdowns but rather is faced with interpreting reported account values that often combine variable and fixed costs and costs for different product lines. Operating
income $\left[\left(\begin{array}{l}\text { Price } \\ \text { per unit }\end{array}\right)\left(\begin{array}{l}\text { Number of } \\ \text { units sold }\end{array}\right)\right]$

$-\left[\left(\begin{array}{l}\text { Variable cost } \\ \text { per unit }\end{array}\right)\left(\begin{array}{l}\text { Number of } \\ \text { units sold }\end{array}\right)\right]-\left[\begin{array}{l}\text { Fixed operating } \\ \text { costs }\end{array}\right]$

or

\begin{center}
\includegraphics[max width=\textwidth]{2023_05_04_7b535d0a870224f62e3dg-112}
\end{center}

The per unit contribution margin is the amount that each unit sold contributes to covering fixed costs-that is, the difference between the price per unit and the variable cost per unit. That difference multiplied by the quantity sold is the contribution margin, which equals revenue minus variable costs.

How much does operating income change when the number of units sold changes? Fixed costs do not change; therefore, operating income changes by the contribution margin. The percentage change in operating income for a given change in units sold simplifies to

$\mathrm{DOL}=\frac{Q(P-V)}{Q(P-V)-F}$

where $Q$ is the number of units, $P$ is the price per unit, $V$ is the variable operating cost per unit, and $F$ is the fixed operating cost. Therefore, $P-V$ is the per unit contribution margin and $Q(P-V)$ is the contribution margin.

Applying the formula for DOL using the data for Impulse Robotics, we can calculate the sensitivity to change in units sold from 100,000 units:

$\begin{aligned} & \mathrm{DOL} @ \\ & 100,000 \text { units }\end{aligned}=\frac{100,000(\$ 10-\$ 2)}{100,000(\$ 10-\$ 2)-\$ 500,000}=2.67$

A DOL of 2.67 means that a 1 percent change in units sold results in a $1 \% \times 2.67$ $=2.67 \%$ change in operating income; a DOL of 5 means that a 1 percent change in units sold results in a 5 percent change in operating income, and so on.

Why do we specify that the DOL is at a particular quantity sold (in this case, 100,000 units)? Because the DOL is different at different numbers of units produced and sold. For example, at 200,000 units,

$\begin{aligned} & \mathrm{DOL} @ \\ & 200,000 \text { units }\end{aligned}=\frac{200,000(\$ 10-\$ 2)}{200,000(\$ 10-\$ 2)-\$ 500,000}=1.45$

We can see the sensitivity of the DOL for different numbers of units produced and sold in Exhibit 6. When operating profit is negative, the DOL is negative. At positions just below and just above the point where operating income is $\$ 0$, operating income is at its most sensitive on a percentage basis to changes in units produced and sold. At the point at which operating income is $\$ 0$ (at 62,500 units produced and sold in this example), the DOL is undefined because the denominator in the DOL calculation is $\$ 0$. After this point, the DOL gradually declines as more units are produced and sold. Exhibit 6: Impulse Robotics' Degree of Operating Leverage for Different

Number of Units Produced and Sold

\begin{center}
\includegraphics[max width=\textwidth]{2023_05_04_7b535d0a870224f62e3dg-113}
\end{center}

We will now look at a similar situation in which the company has shifted some of the operating costs away from fixed costs and into variable costs. Malvey Aerospace has a unit sales price of $\$ 10$, a variable cost of $\$ 6$ a unit, and $\$ 150,000$ in fixed costs. $A$ change in units sold from 100,000 to 110,000 (a 10 percent change) changes operating profit from $\$ 250,000$ to $\$ 290,000$, or 16 percent. The DOL in this case is 1.6 :

$\begin{aligned} & \text { DOL } @ \\ & 100,000 \text { units }\end{aligned}=\frac{100,000(\$ 10-\$ 6)}{100,000(\$ 10-\$ 6)-\$ 150,000}=1.6$

and the change in operating income is 16 percent:

$\begin{aligned} & \text { Percentage change } \\ & \text { in operating income }\end{aligned}(\mathrm{DOL})\left(\begin{array}{l}\text { Percentage change } \\ \text { in units sold }\end{array}\right)=(1.6)(10 \%)=16 \%$

We can see the difference in leverage in the case of Impulse Robotics and Malvey Aerospace companies in Exhibit 7. In Panel $\mathrm{A}$, we see that Impulse Robotics has higher operating income than Malvey Aerospace when both companies produce and sell more than 87,500 units, but lower operating income than Malvey when both companies produce and sell less than 87,500 units. $^{3}$

\section{Exhibit 7: Profitability and the DOL for Impulse Robotics and Malvey}
Aerospace

Impulse Robotics: $P=\$ 10 ; V=\$ 2 ; F=\$ 500,000$

Malvey Aerospace: $P=\$ 10 ; V=\$ 6 ; F=\$ 150,000$

3 We can calculate the number of units that produce the same operating income for these two companies by equating the operating incomes and solving for the number of units. Let $X$ be the number of units. The $X$ at which Malvey Aerospace and Impulse Robotics generate the same operating income is the $X$ that solves the following: $10 X-2 X-500,000=10 X-6 X-150,000$; that is, $X=87,500$.

\section{Panel A: Operating Income and Number of Units Produced and Sold}
Operating Income

\begin{center}
\includegraphics[max width=\textwidth]{2023_05_04_7b535d0a870224f62e3dg-114(1)}
\end{center}

\begin{center}
\includegraphics[max width=\textwidth]{2023_05_04_7b535d0a870224f62e3dg-114}
\end{center}

Units Produced and Sold

\section{Panel B: Degree of Operating Leverage (DOL)}
DOL

\begin{center}
\includegraphics[max width=\textwidth]{2023_05_04_7b535d0a870224f62e3dg-114(2)}
\end{center}

Units Produced and Sold

This example confirms what we saw earlier in our reasoning of fixed and variable costs: The greater the use of fixed, relative to variable, operating costs, the more sensitive operating income is to changes in units sold and, therefore, the more operating risk. Impulse Robotics has more operating risk because it has more operating leverage. However, as Panel B of Exhibit 7 shows, the degrees of operating leverage are similar for the two companies for larger numbers of units produced and sold.

Both sales risk and operating risk influence a company's business risk. And both sales risk and operating risk are determined in large part by the type of business the company is in. But management has more opportunity to manage and control operating risk than sales risk. Suppose a company is deciding which equipment to buy to produce a particular product. The sales risk is the same no matter what equipment is chosen to produce the product. But the available equipment may differ in terms of the fixed and variable operating costs of producing the product. Financial analysts need to consider how the operating cost structure of a company affects the company's risk.

\section{EXAMPLE 1}
\section{Calculating the Degree of Operating Leverage}
\begin{enumerate}
  \item Arnaud Kenigswald is analyzing the potential impact of an improving economy on earnings at Global Auto, one of the world's largest car manufacturers. Global is headquartered in Berlin. Global Auto manufactures passenger cars and produces revenues of $€ 168$ billion. Kenigswald projects that sales will improve by 10 percent due to increased demand for cars. He wants to see how Global's earnings might respond given that level of increase in sales. He first looks at the degree of leverage at Global, starting with operating leverage.
\end{enumerate}

Global sold 6 million passenger cars in 2017. The average price per car was $€ 28,000$, fixed costs associated with passenger car production total $€ 15$ billion per year, and variable costs per car are $€ 20,500$. What is the degree of operating leverage of Global Auto?

\section{Solution:}
$\begin{aligned} & \mathrm{DOL} @ \\ & 6 \text { million units }\end{aligned}=\frac{6 \text { million }(€ 28,000-€ 20,500)}{6 \text { million }(€ 28,000-€ 20,500)-€ 15 \text { billion }}=1.5$

Operating income is $[6$ million $\times(€ 28,000-€ 20,500)]-€ 15$ billion

$=€ 30$ billion

For a 10 percent increase in cars sold, operating income increases by $1.50 \times$ $10 \%=15.0 \%$.

Industries that tend to have high operating leverage are those that invest up front to produce a product but spend relatively little on making and distributing it. Software developers and pharmaceutical companies fit this description. Alternatively, retailers have low operating leverage because much of the cost of goods sold is variable.

Because most companies produce more than one product, the ratio of variable to fixed costs is difficult to obtain. We can get an idea of the operating leverage of a company by looking at changes in operating income in relation to changes in sales for the entire company. This relation can be estimated by regressing changes in operating income (the variable to be explained) on changes in sales (the explanatory variable) over a recent time period. ${ }^{4}$ Although this approach does not provide a precise measure of operating risk, it can help provide a general idea of the amount of operating leverage present. For example, compare the relation between operating earnings and revenues for Delta Air Lines, a transportation company, and Wal-Mart Stores, a discount retailer, as shown in Exhibit 8 . Note that the slope of the least-squares regression line is greater for Delta Air Lines (with a slope coefficient of 0.1702) than for Wal-Mart (with a slope coefficient of 0.0493). (A visual comparison of slopes should not be

4 A least-squares regression is a procedure for finding the best-fitting line (called the least squares regression line) through a set of data points by minimizing the squared deviations from the line. relied upon because the scales of the $x$ - and $y$-axes are different in diagrams for the two regressions.) We can see that operating earnings are more sensitive to changes in revenues for the higher-operating-leveraged Delta Air Lines as compared to the lower-operating-leveraged Wal-Mart Stores.

\section{Exhibit 8: Relation between Operating Earnings and Revenues}
Panel A: Delta Airlines Operating Earnings and Revenues, 1990-2017

Estimated regression: Operating earnings $=-\$ 2,249+0.1702$ Revenues $\mathrm{R}^{2}=64.73 \%$

\begin{center}
\includegraphics[max width=\textwidth]{2023_05_04_7b535d0a870224f62e3dg-116(1)}
\end{center}

\section{Panel B: Wal-Mart Stores Operating Earnings and Revenues, 1990-2017}
Estimated regression: Operating earnings $=\$ 253.16+0.0493$ Revenues $\mathrm{R}^{2}=94.89 \%$

Operating Earnings in Millions

\begin{center}
\includegraphics[max width=\textwidth]{2023_05_04_7b535d0a870224f62e3dg-116}
\end{center}

Sources: Delta Air Lines 10-K filings and Wal-Mart Stores 10-K filings, various years.

\section{FINANCIAL RISK, THE DEGREE OF FINANCIAL LEVERAGE AND THE LEVERAGING ROLE OF DEBT}
\begin{center}
\includegraphics[max width=\textwidth]{2023_05_04_7b535d0a870224f62e3dg-117}
\end{center}

We can expand on the concept of risk to accommodate the perspective of owning a security. A security represents a claim on the income and assets of a business; therefore, the risk of the security goes beyond the variability of operating earnings to include how the cash flows from those earnings are distributed among the claimants-the creditors and owners of the business. The risk of a security is therefore affected by both business risk and financial risk.

Financial risk is the risk associated with how a company finances its operations. If a company finances with debt, it is legally obligated to pay the amounts that make up its debts when due. By taking on fixed obligations, such as debt and long-term leases, the company increases its financial risk. If a company finances its business with common equity, generated either from operations (retained earnings) or from issuing new common shares, it does not incur fixed obligations. The more fixed-cost financial obligations (e.g., debt) incurred by the company, the greater its financial risk.

We can quantify this risk in the same way we did for operating risk, looking at the sensitivity of the cash flows available to owners when operating income changes. This sensitivity, which we refer to as the degree of financial leverage (DFL), is

$\mathrm{DFL}=\frac{\text { Percentage change in net income }}{\text { Percentage change in operating income }}$

For example, if DFL at a given level of operating income is 1.1 , a 5 percent increase in operating income would be expected to result in a $(1.1)(5 \%)=5.5$ percent increase in net income. A company's DFL is dependent on the level of operating income being considered.

Net income is equal to operating income, less interest and taxes. ${ }^{5}$ If operating income changes, how does net income change? Consider Impulse Robotics. Suppose the interest payments are $\$ 100,000$ and, for simplicity, the tax rate is 0 percent: If operating income changes from $\$ 300,000$ to $\$ 360,000$, net income changes from $\$ 200,000$ to $\$ 260,000$ :

\section{Exhibit 9: Financial Risk of Impulse Robotics (1)}
\begin{center}
\begin{tabular}{lccc}
\hline
 & $\begin{array}{c}\text { Operating Income of } \\ \mathbf{\$ 3 0 0 , 0 0 0}\end{array}$ & $\begin{array}{c}\text { Operating Income of } \\ \mathbf{\$ 3 6 0 , 0 0 0}\end{array}$ & $\begin{array}{c}\text { Percentage } \\ \text { Change }\end{array}$ \\
\hline
Operating income & $\$ 300,000$ & $\$ 360,000$ & +20 \\
Less interest & 100,000 & 100,000 & 0 \\
\end{tabular}
\end{center}

5 More complex entities than we have been using for our examples may also need to account for other income (losses) and extraordinary income (losses) together with operating income as the basis for earnings before interest and taxes.

\begin{center}
\begin{tabular}{lccc}
\hline
 & $\begin{array}{c}\text { Operating Income of } \\ \mathbf{\$ 3 0 0 , 0 0 0}\end{array}$ & $\begin{array}{c}\text { Operating Income of } \\ \mathbf{\$ 3 6 0 , 0 0 0}\end{array}$ & $\begin{array}{c}\text { Percentage } \\ \text { Change }\end{array}$ \\
\hline
Net income & $\$ 200,000$ & $\$ 260,000$ & +30 \\
\hline
\end{tabular}
\end{center}

A 20 percent increase in operating income increases net income by $\$ 60,000$, or 30 percent. What if the fixed financial costs are $\$ 150,000$ ? A 20 percent change in operating income results in a 40 percent change in the net income, from $\$ 150,000$ to $\$ 210,000$ :

Exhibit 10: Financial Risk of Impulse Robotics (2)

\begin{center}
\begin{tabular}{lccc}
\hline
 & $\begin{array}{c}\text { Operating Income of } \\ \mathbf{\$ 3 0 0 , 0 0 0}\end{array}$ & $\begin{array}{c}\text { Operating Income of } \\ \mathbf{\$ 3 6 0 , 0 0 0}\end{array}$ & $\begin{array}{c}\text { Percentage } \\ \text { Change }\end{array}$ \\
\hline
Operating income & $\$ 300,000$ & $\$ 360,000$ & +20 \\
Less interest & 150,000 & 150,000 & 0 \\
Net income & $\$ 150,000$ & $\$ 210,000$ & +40 \\
\hline
\end{tabular}
\end{center}

Using more debt financing, which results in higher fixed costs, increases the sensitivity of net income to changes in operating income. We can represent the sensitivity of net income to a change in operating income, continuing the notation from before and including the fixed financial cost, $C$, and the tax rate, $t$, as

$$
\mathrm{DFL}=\frac{[Q(P-V)-F](1-t)}{[Q(P-V)-F-C](1-t)}=\frac{[Q(P-V)-F]}{[Q(P-V)-F-C]}
$$

As you can see in Equation 4, the factor that adjusts for taxes, $(1-t)$, cancels out of the equation. In other words, the DFL is not affected by the tax rate.

In the case in which operating income is $\$ 300,000$ and fixed financing costs are $\$ 100,000$, the degree of financial leverage is

\begin{center}
\includegraphics[max width=\textwidth]{2023_05_04_7b535d0a870224f62e3dg-118}
\end{center}

If, instead, fixed financial costs are $\$ 150,000$, the DFL is equal to 2.0 :

\begin{center}
\includegraphics[max width=\textwidth]{2023_05_04_7b535d0a870224f62e3dg-118(1)}
\end{center}

Again, we need to qualify our degree of leverage by the level of operating income because DFL is different at different levels of operating income.

The greater the use of financing sources that require fixed obligations, such as interest, the greater the sensitivity of net income to changes in operating income.

\section{EXAMPLE 2}
\section{Calculating the Degree of Financial Leverage}
\begin{enumerate}
  \item Global Auto also employs debt financing. If Global can borrow at 8 percent, the interest cost is $€ 18$ billion. What is the degree of financial leverage of Global Auto if 6 million cars are produced and sold?
\end{enumerate}

\section{Solution:}
At 6 million cars produced and sold, operating income $=€ 30$ billion. Therefore: $€ 30$ billion operating income $=\frac{€ 30 \text { billion }}{€ 30 \text { billion }-€ 18 \text { billion }}=2.5$

For every 1 percent change in operating income, net income changes 2.5 percent due to financial leverage.

Unlike operating leverage, the degree of financial leverage is most often a choice by the company's management. Whereas operating costs are very similar among companies in the same industry, competitors may decide on differing capital structures.

Companies with relatively high ratios of tangible assets to total assets may be able to use higher degrees of financial leverage than companies with relatively low ratios because the claim on the tangible assets that lenders would have in the event of a default may make lenders more confident in extending larger amounts of credit. In general, businesses with plants, land, and equipment that can be used to collateralize borrowings and businesses whose revenues have below-average business cycle sensitivity may be able to use more financial leverage than businesses without such assets and with relatively high business cycle sensitivity.

Using financial leverage generally increases the variability of return on equity (net income divided by shareholders' equity). In addition, its use by a profitable company may increase the level of return on equity. Example 3 illustrates both effects.

\section{EXAMPLE 3}
\section{The Leveraging Role of Debt}
Consider the Capital Company, which is expected to generate $\$ 1,500,000$ in revenues and $\$ 500,000$ in operating earnings next year. Currently, the Capital Company does not use debt financing and has assets of $\$ 2,000,000$.

Suppose Capital were to change its capital structure, buying back $\$ 1,000,000$ of stock and issuing $\$ 1,000,000$ in debt. If we assume that interest on debt is 5 percent and income is taxed at a rate of 30 percent, what is the effect of debt financing on Capital's net income and return on equity if operating earnings may vary as much as 40 percent from expected earnings?

\section{Exhibit 11: Return on Equity of Capital Company}
\begin{center}
\begin{tabular}{|c|c|c|c|}
\hline
$\begin{array}{l}\text { No Debt } \\ \text { (Shareholders' } \\ \text { Equity }=\$ 2 \text { million) }\end{array}$ & $\begin{array}{l}\text { Expected } \\ \text { Operating } \\ \text { Earnings, } \\ \text { Less } 40 \%\end{array}$ & $\begin{array}{l}\text { Expected } \\ \text { Operating } \\ \text { Earnings }\end{array}$ & $\begin{array}{l}\text { Expected } \\ \text { Operating } \\ \text { Earnings, } \\ \text { Plus } 40 \%\end{array}$ \\
\hline
$\begin{array}{l}\text { Earnings before } \\ \text { interest and taxes }\end{array}$ & $\$ 300,000$ & $\$ 500,000$ & $\$ 700,000$ \\
\hline
Interest expense & 0 & 0 & 0 \\
\hline
$\begin{array}{l}\text { Earnings before } \\ \text { taxes }\end{array}$ & $\$ 300,000$ & $\$ 500,000$ & $\$ 700,000$ \\
\hline
Taxes & 90,000 & 150,000 & 210,000 \\
\hline
Net income & $\$ 210,000$ & $\$ 350,000$ & $\$ 490,000$ \\
\hline
Return on equity $^{1}$ & $10.5 \%$ & $17.5 \%$ & $24.5 \%$ \\
\hline
\end{tabular}
\end{center}

\begin{center}
\begin{tabular}{|c|c|c|c|}
\hline
$\begin{array}{l}\text { Debt to Total } \\ \text { Assets = 50\%; } \\ \text { (Shareholders' } \\ \text { Equity = } \$ 1 \text { million) }\end{array}$ & $\begin{array}{l}\text { Expected } \\ \text { Operating } \\ \text { Earnings, } \\ \text { Less } 40 \%\end{array}$ & $\begin{array}{l}\text { Expected } \\ \text { Operating } \\ \text { Earnings }\end{array}$ & $\begin{array}{l}\text { Expected } \\ \text { Operating } \\ \text { Earnings, } \\ \text { Plus } 40 \%\end{array}$ \\
\hline
$\begin{array}{l}\text { Earnings before } \\ \text { interest and taxes }\end{array}$ & $\$ 300,000$ & $\$ 500,000$ & $\$ 700,000$ \\
\hline
Interest expense & 50,000 & 50,000 & 50,000 \\
\hline
$\begin{array}{l}\text { Earnings before } \\ \text { taxes }\end{array}$ & $\$ 250,000$ & $\$ 450,000$ & $\$ 650,000$ \\
\hline
Taxes & 75,000 & 135,000 & 195,000 \\
\hline
Net income & $\$ 175,000$ & $\$ 315,000$ & $\$ 455,000$ \\
\hline
Return on equity & $17.5 \%$ & $31.5 \%$ & $45.5 \%$ \\
\hline
\end{tabular}
\end{center}

${ }^{1}$ Recall that ROE is calculated as net income/shareholders' equity.

Depicting a broader array of capital structures and operating earnings, ranging from an operating loss of $\$ 500,000$ to operating earnings of $\$ 2,000,000$, Exhibit 12 shows the effect of leverage on the return on equity for Capital Company:

Exhibit 12: Return on Equity of Capital Company for Different Levels of Operating Earnings and Different Financing Choices

\begin{center}
\includegraphics[max width=\textwidth]{2023_05_04_7b535d0a870224f62e3dg-120}
\end{center}

Business is generally an uncertain venture. Changes in the macroeconomic and competitive environments that influence sales and profitability are typically difficult to discern and forecast. The larger the proportion of debt in the financing mix of a business, the greater is the chance that it will face default. Similarly, the greater the proportion of debt in the capital structure, the more earnings are magnified upward in improving economic times. The bottom line? Financial leverage tends to increase the risk of ownership for shareholders.

\section{TOTAL LEVERAGE AND THE DEGREE OF TOTAL LEVERAGE}
define and explain leverage, business risk, sales risk, operating risk, and financial risk and classify a risk

calculate and interpret the degree of operating leverage, the degree of financial leverage, and the degree of total leverage

The degree of operating leverage gives us an idea of the sensitivity of operating income to changes in revenues. And the degree of financial leverage gives us an idea of the sensitivity of net income to changes in operating income. But often we are concerned about the combined effect of both operating leverage and financial leverage. Owners are concerned about the combined effect because both factors contribute to the risk associated with their future cash flows. And financial managers, making decisions intended to maximize owners' wealth, need to be concerned with how investment decisions (which affect the operating cost structure) and financing decisions (which affect the capital structure) affect lenders' and owners' risk.

Look back at the example of Impulse Robotics. The sensitivity of owners' cash flow to a given change in units sold is affected by both operating and financial leverage. Consider using 100,000 units as the base number produced and sold. A 10 percent increase in units sold results in a 27 percent increase in operating income and a 40 percent increase in net income; a like decrease in units sold results in a similar decrease in operating income and net income.

\section{Exhibit 13: Total Leverage of Impulse Robotics}
Units Produced and Sold:

\begin{center}
\begin{tabular}{|c|c|c|c|}
\hline
 & 90,000 & 100,000 & 110,000 \\
\hline
Revenues & $\$ 900,000$ & $\$ 1,000,000$ & $\$ 1,100,000$ \\
\hline
Less variable costs & 180,000 & 200,000 & 220,000 \\
\hline
Less fixed costs & 500,000 & 500,000 & 500,000 \\
\hline
Operating income & $\$ 220,000$ & $\$ 300,000$ & $\$ 380,000$ \\
\hline
Less interest & 100,000 & 100,000 & 100,000 \\
\hline
Net income & $\$ 120,000$ & $\$ 200,000$ & $\$ 280,000$ \\
\hline
\multicolumn{4}{|l|}{Relative to 100,000 units produced and sold} \\
\hline
Percentage change in units sold & $-10 \%$ &  & $+10 \%$ \\
\hline
Percentage change in operating profit & $-27 \%$ &  & $+27 \%$ \\
\hline
Percentage change in net income & $-40 \%$ &  & $+40 \%$ \\
\hline
\end{tabular}
\end{center}

Combining a company's degree of operating leverage with its degree of financial leverage results in the degree of total leverage (DTL), a measure of the sensitivity of net income to changes in the number of units produced and sold. We again make the simplifying assumption that a company sells all that it produces in the same period:

$$
\text { DTL }=\frac{\text { Percentage change in net income }}{\text { Percentage change in the number of units sold }}
$$

$$
\begin{array}{rlr} 
& \mathrm{DTL}=\frac{Q(P-V)}{Q(P-V)-F} \times \frac{[Q(P-V)-F]}{[Q(P-V-F-C]} & \\
& \text { DOL } & \\
= & & \\
Q(P-V-V) &
\end{array}
$$

Suppose

$\begin{array}{ll}\text { Number of units sold } & =Q=100,000 \\ \text { Price per unit } & =P=\$ 10 \\ \text { Variable cost per unit } & =V=\$ 2 \\ \text { Fixed operating cost } & =F=\$ 500,000 \\ \text { Fixed financing cost } & =C=\$ 100,000\end{array}$

Then,

$$
\mathrm{DTL}=\frac{100,000(\$ 10-\$ 2)}{100,000(\$ 10-\$ 2)-\$ 500,000-\$ 100,000}=4.0
$$

which we could also have determined by multiplying the DOL, 2.67, by the DFL, 1.5 . This means that a 1 percent increase in units sold will result in a 4 percent increase in net income; a 50 percent increase in units produced and sold results in a 200 percent increase in net income; a 5 percent decline in units sold results in a 20 percent decline in income to owners; and so on.

Because the DOL is relative to the base number of units produced and sold and the DFL is relative to the base level of operating earnings, DTL is different depending on the number of units produced and sold. We can see the DOL, DFL, and DTL for Impulse Robotics for different numbers of units produced and sold, beginning at the number of units for which the degrees are positive, in Exhibit 14.

Exhibit 14: DOL, DFL, and DTL for Different Numbers of Units Produced and Sold

$$
P=\$ 10, V=\$ 2, F=\$ 500,000, C=\$ 100,000
$$

Degree of Leverage

\begin{center}
\includegraphics[max width=\textwidth]{2023_05_04_7b535d0a870224f62e3dg-122}
\end{center}

Number of Units Produced and Sold In the case of operating leverage, the fixed operating costs act as a fulcrum. The greater the proportion of operating costs that are fixed, the more sensitive operating income is to changes in sales. In the case of financial leverage, the fixed financial costs, such as interest, act as a fulcrum. The greater the proportion of financing with fixed cost sources, such as debt, the more sensitive cash flows available to owners are to changes in operating income. Combining the effects of both types of leverage, we see that fixed operating and financial costs together increase the sensitivity of earnings to owners.

\section{EXAMPLE 4}
\section{Calculating the Degree of Total Leverage}
\begin{enumerate}
  \item Continuing from Examples 1 and 2, Global Auto's total leverage is
\end{enumerate}

DTL@@.DOL@@.DFL@

6 million units $=6$ million units ${ }^{\times} € 30$ billion

\begin{center}
\includegraphics[max width=\textwidth]{2023_05_04_7b535d0a870224f62e3dg-123(1)}
\end{center}

\begin{center}
\includegraphics[max width=\textwidth]{2023_05_04_7b535d0a870224f62e3dg-123}
\end{center}

$=\frac{€ 45 \text { billion }}{€ 12 \text { billion }}=3.75$

DTL $@=1.5 \times 2.5=3.75$

on units

Given Global Auto's operating and financial leverage, a 1 percent change in unit sales changes net income by 3.75 percent.

\section{BREAKEVEN POINTS AND OPERATING BREAKEVEN POINTS}
calculate the breakeven quantity of sales and determine the company's net income at various sales levels

calculate and interpret the operating breakeven quantity of sales

Looking back at Exhibit 3, we see that there is a number of units at which the company goes from being unprofitable to being profitable-that is, the number of units at which the net income is zero. This number is referred to as the breakeven point. The breakeven point, $Q_{\mathrm{BE}}$, is the number of units produced and sold at which the company's net income is zero-the point at which revenues are equal to costs.

Plotting revenues and total costs against the number of units produced and sold, as in Exhibit 15, indicates that the breakeven is at 75,000 units. At this number of units produced and sold, revenues are equal to costs and, hence, profit is zero.

\section{Exhibit 15: Impulse Robotics Breakeven}
\begin{center}
\includegraphics[max width=\textwidth]{2023_05_04_7b535d0a870224f62e3dg-124}
\end{center}

We can calculate this breakeven point for Impulse Robotics and Malvey Aerospace. Consider that net income is zero when the revenues are equal to the expenses. We can represent this equality of revenues and costs (summing variable operating costs, fixed operating costs, and fixed financing costs) by the following equation:

$$
P Q=V Q+F+C
$$

where

$$
\begin{aligned}
& P=\text { the price per unit } \\
& Q=\text { the number of units produced and sold } \\
& V=\text { the variable cost per unit } \\
& F=\text { the fixed operating costs } \\
& C=\text { the fixed financial cost }
\end{aligned}
$$

Therefore,

$$
P Q_{\mathrm{BE}}=V Q_{\mathrm{BE}}+F+C
$$

and the breakeven number of units, $Q_{\mathrm{BE}}$, is ${ }^{6}$

$$
Q_{\mathrm{BE}}=\frac{F+C}{P-V}
$$

In the case of Impulse Robotics and Malvey Aerospace, Impulse Robotics has a higher breakeven point. Using numbers taken from Exhibit 2:

$$
\begin{aligned}
& \text { Impulse Robotics: } Q_{\mathrm{BE}}=\frac{\$ 500,000+\$ 100,000}{\$ 10-\$ 2}=75,000 \text { units } \\
& \text { Malvey Aerospace: } Q_{\mathrm{BE}}=\frac{\$ 150,000+\$ 50,000}{\$ 10-\$ 6}=50,000 \text { units }
\end{aligned}
$$

6 You will notice that we did not consider taxes in our calculation of the breakeven point. This is because at the point of breakeven, taxable income is zero. This means that Impulse Robotics must produce and sell more units to achieve a profit. So, while the higher-leveraged Impulse Robotics has a greater breakeven point relative to Malvey Aerospace, the profit that Impulse Robotics generates beyond this breakeven point is greater than that of Malvey Aerospace. Therefore, leverage has its rewards in terms of potentially greater profit, but it also increases risk.

In addition to the breakeven point specified in terms of net income, $Q_{\mathrm{BE}}$, we can also specify the breakeven point in terms of operating profit, which we refer to as the operating breakeven point, $Q_{\mathrm{OBE}}$. Revenues at the operating breakeven point are set equal to operating costs at the operating breakeven point to solve for the operating breakeven number of units, $Q_{\mathrm{OBE}}$. The expression shows $Q_{\mathrm{OBE}}$ as equal to fixed operating costs divided by the difference between price per unit and variable cost per unit:

$$
\begin{aligned}
& P Q_{\mathrm{OBE}}=V Q_{\mathrm{OBE}}+F \\
& Q_{\mathrm{OBE}}=\frac{F}{P-V}
\end{aligned}
$$

For the two companies in our example, Impulse Robotics and Malvey Aerospace, the operating breakevens are 62,500 and 37,500 units, respectively:

Impulse Robotics: $Q_{\mathrm{OBE}}=\frac{\$ 500,000}{\$ 10-\$ 2}=62,500$ units

Malvey Aerospace: $Q_{\mathrm{OBE}}=\frac{\$ 150,000}{\$ 10-\$ 6}=37,500$ units

Impulse Robotics has a higher operating breakeven point in terms of the number of units produced and sold.

\section{EXAMPLE 5}
\section{Calculating Operating Breakeven and Breakeven Points}
Continuing with his analysis, Kenigswald considers the effect of a possible downturn on Global Auto's earnings. He divides the fixed operating costs of $€ 15$ billion by the per unit contribution margin:

$$
Q_{\mathrm{OBE}}=\frac{€ 15 \text { billion }}{€ 28,000-€ 20,500}=2 \text { million cars }
$$

The operating breakeven for Global is $2,000,000$ cars, or $€ 56$ billion in revenues. We calculate the breakeven point by dividing fixed operating costs, plus interest costs, by the contribution margin:

$$
Q_{\mathrm{BE}}=\frac{€ 15 \text { billion }+€ 18 \text { billion }}{€ 28,000-€ 20,500}=4,400,000
$$

Considering the degree of total leverage, Global's breakeven is 4.4 million cars, or revenues of $€ 123.2$ billion.

We can verify these calculations by constructing an income statement for the breakeven sales (in $€$ billions):

$2,000,000$ Cars

$4,400,000$ Cars

Revenues $(=P \times Q)$

\begin{center}
\begin{tabular}{cc}
$€ 56.0$ & $€ 123.2$ \\
41.0 & 90.2 \\
15.0 & 15.0 \\
\hline
$€ 0$ & $€ 18.0$ \\
18.0 & 18.0 \\
\hline
$-€ 18.0$ & $€ 0$ \\
\hline
\end{tabular}
\end{center}

As business expands or contracts beyond or below breakeven points, fixed costs do not change. The breakeven points for companies with low operating and financial leverage are less important than those for companies with high leverage. Companies with greater total leverage must generate more revenue to cover fixed operating and financing costs. The farther unit sales are from the breakeven point for high-leverage companies, the greater the magnifying effect of this leverage.

\section{THE RISKS OF CREDITORS AND OWNERS}
As we discussed earlier, business risk refers to the effect of economic conditions as well as the level of operating leverage. Uncertainty about demand, output prices, and costs are among the many factors that affect business risk. When conditions change for any of these factors, companies with higher business risk experience more volatile earnings. Financial risk is the additional risk that results from the use of debt and preferred stock. The degree of financial risk grows with greater use of debt. Who bears this risk?

The risk for providers of equity and debt capital differs because of the relative rights and responsibilities associated with the use of borrowed money in a business. Lenders have a prior claim on assets relative to shareholders, so they have greater security. In return for lending money to a business, lenders require the payment of interest and principal when due. These contractual payments to lenders must be made regardless of the profitability of the business. A business must satisfy these claims in a timely fashion or face the pain of bankruptcy should it default. In return for their higher priority in claims, lenders get predefined yet limited returns.

In contrast, equity providers claim whatever is left over after all expenses, including debt service, have been paid. So, unlike the fixed and known commitments to the lenders, what is left over for the owners may be a great deal or may be nothing. In exchange for this risk, providers of equity capital exercise the decision-making power over the business, including the right to hire, guide, and if necessary, fire managers. In public companies, ownership rights are usually exercised through an elected board of directors. They undertake the decisions over what portion of the business's earnings should be paid out as dividends for common shareholders.

Legal codes in most countries provide for these rights, as well as conditions for companies to file for bankruptcy (with reference to businesses, often called insolvency). A number of bankruptcy codes provide in some form for two categories of bankruptcies. One form provides for a temporary protection from creditors so that a viable business may reorganize. In the United States, the US Bankruptcy Code sets the terms for the form of negotiated reorganization of a company's capital structure that allows it to remain a going concern in Chapter 11.7 For businesses that are not viable, the second form of bankruptcy process allows for the orderly satisfaction of the creditors' claims. In the United States, this form of bankruptcy is referred to as liquidation. ${ }^{8}$ Whereas both types of bankruptcy lead to major dislocations in the rights and privileges of owners, lenders, employees, and managers, it is in this latter category of bankruptcy that the original business ceases to exist.

The difference between a company that reorganizes and emerges from bankruptcy and one that is liquidated is often the difference between operating and financial leverage. Companies with high operating leverage have less flexibility in making changes,

7 US Code, Title 11-Bankruptcy, Chapter 11-Reorganization. Companies filing for bankruptcy under this code are referred to as having filed for Chapter 11 bankruptcy.

8 US Code, Title 11-Bankruptcy, Chapter 7-Liquidation. and bankruptcy protection does little to help reduce operating costs. Companies with high financial leverage use bankruptcy laws and protection to change their capital structure and, once the restructuring is complete, can emerge as ongoing concerns.

\section{EXAMPLE 6}
\section{Dow Corning: In and Out of Chapter 11 Bankruptcy}
Dow Corning, a leading silicone producer, was a joint venture of Dow Chemical and Corning Inc. Dow Chemical filed for bankruptcy protection using Chapter 11 in 1995 as a result of the lawsuits related to silicone implants. The company was profitable at the time of filing, with more than $\$ 500$ million in net income in 1994, but the potential liability from lawsuits, initially estimated around $\$ 2$ billion, was significant when compared to its $\$ 4$ billion in total assets. In 2004, Dow Corning emerged from bankruptcy, and in 2016 Dow Chemical completed the acquisition of 100 percent of Dow Corning, which currently operates as a wholly owned subsidiary of Dow Chemical Company.

\section{EXAMPLE 7}
\section{Retailers Do Not Deliver}
Traditional brick-and-mortar retail stores have been challenged with competition from online retailers. Whereas some retailers successfully added online access, others have struggled to compete effectively. A number of US retail stores filed for bankruptcy protection in 2017 and 2018, including Radio Shack, Nine West, Toys R Us, Brookstone, Payless, and hhgregg. Though some of these retailers were "reborn" online (e.g., \href{http://hhgregg.com}{hhgregg.com}), others liquidated (e.g., Toys R Us), and many are closing stores and working closely with creditors to stave off liquidation.

Whereas the ability to file for bankruptcy is important to the economy, the goal of most investors is to avoid ownership of companies that are heading toward this extreme step, as well as to be able to evaluate opportunities among companies already in bankruptcy. Under both Chapter 7 and Chapter 11, providers of equity capital generally lose all value during the bankruptcy. On the other hand, debtholders typically receive at least a portion of their capital, but the payments of principal and interest are delayed during the period of bankruptcy protection.

\section{SUMMARY}
In this reading, we have reviewed the fundamentals of business risk, financial risk, and measures of leverage.

\begin{itemize}
  \item Leverage is the use of fixed costs in a company's cost structure. Business risk is the risk associated with operating earnings and reflects both sales risk (uncertainty with respect to the price and quantity of sales) and operating risk (the risk related to the use of fixed costs in operations). Financial risk is the risk associated with how a company finances its operations (i.e., the split between equity and debt financing of the business).

  \item The degree of operating leverage (DOL) is the ratio of the percentage change in operating income to the percentage change in units sold. We can use the following formula to measure the degree of operating leverage: $\mathrm{DOL}=\frac{Q(P-V)}{Q(P-V)-F}$

  \item The degree of financial leverage (DFL) is the percentage change in net income for a one percent change in operating income. We can use the following formula to measure the degree of financial leverage:

\end{itemize}

$\mathrm{DFL}=\frac{[Q(P-V)-F](1-t)}{[Q(P-V)-F-C](1-t)}=\frac{[Q(P-V-F]}{[Q(P-V)-F-C]}$

\begin{itemize}
  \item The degree of total leverage (DTL) is a measure of the sensitivity of net income to changes in unit sales, which is equivalent to DTL $=\mathrm{DOL} \times \mathrm{DFL}$.

  \item The breakeven point, $Q_{\mathrm{BE}}$, is the number of units produced and sold at which the company's net income is zero, which we calculate as

\end{itemize}

$Q_{\mathrm{BE}}=\frac{F+C}{P-V}$

\begin{itemize}
  \item The operating breakeven point, $Q_{\mathrm{OBE}}$, is the number of units produced and sold at which the company's operating income is zero, which we calculate as
\end{itemize}

$Q_{\mathrm{OBE}}=\frac{F}{P-V}$

\section{PRACTICE PROBLEMS}
\section{The following information relates to questions}
1-9

Mary Benn, CFA, is a financial analyst for Twin Fields Investments, located in Storrs, Connecticut, USA. She has been asked by her supervisor, Bill Cho, to examine two small Japanese cell phone component manufacturers: 4G, Inc. and Qphone Corp. Cho indicates that his clients are most interested in the use of leverage by $4 \mathrm{G}$ and Qphone. Benn states, "I will have to specifically analyze each company's respective business risk, sales risk, operating risk, and financial risk." "Fine, I'll check back with you shortly," Cho, answers.

Benn begins her analysis by examining the sales prospects of the two firms. The results of her sales analysis appear in Exhibit 1 . She also expects very little price variability for these cell phones. She next gathers more data on these two companies to assist her analysis of their operating and financial risk.

When Cho inquires as to her progress Benn responds, "I have calculated Qphone's degree of operating leverage (DOL) and degree of financial leverage (DFL) at Qphone's 2009 level of unit sales. I have also calculated Qphone's breakeven level for unit sales. I will have 4G's leverage results shortly."

Cho responds, "Good, I will call a meeting of some potential investors for tomorrow. Please help me explain these concepts to them, and the differences in use of leverage by these two companies. In preparation for the meeting, I have a number of questions":

\begin{itemize}
  \item "You mentioned business risk; what is included in that?"

  \item "How would you classify the risk due to the varying mix of variable and fixed costs?"

  \item "Could you conduct an analysis and tell me how the two companies will fare relative to each other in terms of net income if their unit sales increased by 10 percent above their 2009 unit sales levels?"

  \item "Finally, what would be an accurate verbal description of the degree of total leverage?"

\end{itemize}

The relevant data for analysis of 4G is contained in Exhibit 2, and Benn's analysis of the Qphone data appears in Exhibit 3:

Exhibit 1: Benn's Unit Sales Estimates for 4G, Inc. and Qphone Corp.

\begin{center}
\begin{tabular}{|c|c|c|c|}
\hline
Company & 2009 Unit Sales & $\begin{array}{c}\text { Standard Deviation } \\ \text { of Unit Sales }\end{array}$ & $\begin{array}{c}2010 \text { Expected Unit } \\ \text { Sales Growth Rate (\%) }\end{array}$ \\
\hline
4G, Inc. & $1,000,000$ & 25,000 & 15 \\
\hline
Ophone Corp. & $1,500,000$ & 10,000 & 15 \\
\hline
\end{tabular}
\end{center}

Exhibit 2: Sales, Cost, and Expense Data for 4G, Inc. (At Unit Sales of $1,000,000)$

Number of units produced and sold

$1,000,000$

$\begin{array}{lr}\text { Sales price per unit } & ¥ 108\end{array}$

Variable cost per unit $\quad ¥ 72$

Fixed operating cost $\quad ¥ 22,500,000$

Fixed financing expense $\quad ¥ 9,000,000$

Exhibit 3: Benn's Analysis of Qphone (At Unit Sales of 1,500,000)

$\begin{array}{lc}\text { Degree of operating leverage } & 1.40 \\ \text { Degree of financial leverage } & 1.15 \\ \text { Breakeven quantity (units) } & 571,429\end{array}$

\begin{enumerate}
  \item Based on Benn's analysis, 4G's sales risk relative to Qphone's is most likely to be:
A. lower.
B. equal.
C. higher.

  \item What is the most appropriate response to Cho's question regarding the components of business risk?

\end{enumerate}

A. Sales risk and financial risk

B. Operating risk and sales risk.

C. Financial risk and operating risk.

\begin{enumerate}
  \setcounter{enumi}{2}
  \item The most appropriate response to Cho's question regarding the classification of risk arising from the mixture of variable and fixed costs is:
A. sales risk.
B. financial risk.
C. operating risk.

  \item Based on the information in Exhibit 17, the degree of operating leverage (DOL) of 4G, Inc., at unit sales of $1,000,000$, is closest to:
A. 1.60 .
B. 2.67 .
C. 3.20 .

  \item Based on the information in Exhibit 17, 4G, Inc.'s degree of financial leverage (DFL), at unit sales of $1,000,000$, is closest to:

\end{enumerate}

A. 1.33 .
B. 2.67 .
C. 3.00 .

\begin{enumerate}
  \setcounter{enumi}{5}
  \item Based on the information in Exhibit 16 and Exhibit 18, Qphone's expected percentage change in operating income for 2010 is closest to:
A. $17.25 \%$.
B. $21.00 \%$.
C. $24.30 \%$.

  \item 4G's breakeven quantity of unit sales is closest to:
A. 437,500 units.
B. 625,000 units.
C. 875,000 units.

  \item In response to Cho's question regarding an increase in unit sales above 2009 unit sales levels, it is most likely that $4 \mathrm{G}$ 's net income will increase at:
A. a slower rate than Qphone's.
B. the same rate as Qphone's.
C. a faster rate than Qphone's.

  \item The most appropriate response to Cho's question regarding a description of the degree of total leverage is that degree of total leverage is:

\end{enumerate}

A. the percentage change in net income divided by the percentage change in units sold.

B. the percentage change in operating income divided by the percentage change in units sold.

C. the percentage change in net income divided by the percentage change in operating income.

\begin{enumerate}
  \setcounter{enumi}{9}
  \item If two companies have identical unit sales volume and operating risk, they are most likely to also have identical:
A. sales risk.
B. business risk.
C. sensitivity of operating earnings to changes in the number of units produced and sold.

  \item Degree of operating leverage is best described as a measure of the sensitivity of:
A. net earnings to changes in sales.
B. fixed operating costs to changes in variable costs.
C. operating earnings to changes in the number of units produced and sold.

  \item The Fulcrum Company produces decorative swivel platforms for home televi- sions. If Fulcrum produces 40 million units, it estimates that it can sell them for $\$ 100$ each. Variable production costs are $\$ 65$ per unit and fixed production costs are $\$ 1.05$ billion. Which of the following statements is most accurate? Holding all else constant, the Fulcrum Company would:

\end{enumerate}

A. generate positive operating income if unit sales were 25 million.

B. have less operating leverage if fixed production costs were 10 percent greater than $\$ 1.05$ billion.

C. generate 20 percent more operating income if unit sales were 5 percent greater than 40 million.

\begin{enumerate}
  \setcounter{enumi}{12}
  \item The business risk of a particular company is most accurately measured by the company's:
\end{enumerate}

A. debt-to-equity ratio.

B. efficiency in using assets to generate sales.

C. operating leverage and level of uncertainty about demand, output prices, and competition.

\begin{enumerate}
  \setcounter{enumi}{13}
  \item Consider two companies that operate in the same line of business and have the same degree of operating leverage: the Basic Company and the Grundlegend Company. The Basic Company and the Grundlegend Company have, respectively, no debt and 50 percent debt in their capital structure. Which of the following statements is most accurate? Compared to the Basic Company, the Grundlegend Company has:
\end{enumerate}

A. a lower sensitivity of net income to changes in unit sales.

B. the same sensitivity of operating income to changes in unit sales.

C. the same sensitivity of net income to changes in operating income.

\begin{enumerate}
  \setcounter{enumi}{14}
  \item Myundia Motors now sells 1 million units at $¥ 3,529$ per unit. Fixed operating costs are $¥ 1,290$ million and variable operating costs are $¥ 1,500$ per unit. If the company pays $¥ 410$ million in interest, the levels of sales at the operating breakeven and breakeven points are, respectively:
\end{enumerate}

A. $¥ 1,500,000,000$ and $¥ 2,257,612,900$.

B. $¥ 2,243,671,760$ and $¥ 2,956,776,737$.

C. $¥ 2,975,148,800$ and $¥ 3,529,000,000$.

\begin{enumerate}
  \setcounter{enumi}{15}
  \item Juan Alavanca is evaluating the risk of two companies in the machinery industry: The Gearing Company and Hebelkraft, Inc. Alavanca used the latest fiscal year's financial statements and interviews with managers of the respective companies to gather the following information:
\end{enumerate}

\begin{center}
\begin{tabular}{lll}
\hline
 & The Gearing Company & Hebelkraft, Inc. \\
\hline
Number of units produced and sold & 1 million & 1.5 million \\
Sales price per unit & $\$ 200$ & $\$ 200$ \\
Variable cost per unit & $\$ 120$ & $\$ 100$ \\
Fixed operating cost & $\$ 40$ million & $\$ 90$ million \\
\end{tabular}
\end{center}

\begin{center}
\begin{tabular}{lll}
\hline
 & The Gearing Company & Hebelkraft, Inc. \\
\hline
Fixed financing expense & $\$ 20$ million & $\$ 20$ million \\
\hline
\end{tabular}
\end{center}

Based on this information, the breakeven points for The Gearing Company and Hebelkraft, Inc. are:
A. 0.75 million and 1.1 million units, respectively.
B. 1 million and 1.5 million units, respectively.
C. 1.5 million and 0.75 million units, respectively.

\section{SOLUTIONS}
\begin{enumerate}
  \item $C$ is correct. Sales risk is defined as uncertainty with respect to the price or quantity of goods and services sold. $4 \mathrm{G}$ has a higher standard deviation of unit sales than Qphone; in addition, 4G's standard deviation of unit sales stated as a fraction of its level of unit sales, at 25,000/1,000,000 $=0.025$, is greater than the comparable ratio for Qphone, 10,000/1,500,000 $=0.0067$.

  \item B is correct. Business risk is associated with operating earnings. Operating earnings are affected by sales risk (uncertainty with respect to price and quantity), and operating risk (the operating cost structure and the level of fixed costs).

  \item $\mathrm{C}$ is correct. Operating risk refers to the risk arising from the mix of fixed and variable costs.

  \item $\mathrm{B}$ is correct. $\mathrm{DOL}=\frac{Q(P-V)}{Q(P-V)-F}$

\end{enumerate}

\begin{center}
\includegraphics[max width=\textwidth]{2023_05_04_7b535d0a870224f62e3dg-134}
\end{center}

\begin{enumerate}
  \setcounter{enumi}{4}
  \item C is correct. Degree of financial leverage is
\end{enumerate}

$$
\begin{aligned}
& \text { DFL }=\frac{[Q(P-V)-F]}{[Q(P-V)-F-C]} \\
& =\frac{1,000,000(¥ 108-¥ 72)-¥ 22,500,000}{1,000,000(¥ 108-¥ 72)-¥ 22,500,000-¥ 9,000,000}=3.00
\end{aligned}
$$

\begin{enumerate}
  \setcounter{enumi}{5}
  \item B is correct. The degree of operating leverage of Qphone is 1.4. The percentage change in operating income is equal to the DOL times the percentage change in units sold, therefore:
\end{enumerate}

$\begin{aligned} & \text { Percentage change } \\ & \text { in operating income }\end{aligned}=(\mathrm{DOL})\left(\begin{array}{l}\text { Percentage change } \\ \text { in units sold }\end{array}\right)=(1.4)(15 \%)=21 \%$

\begin{enumerate}
  \setcounter{enumi}{6}
  \item $\mathrm{C}$ is correct. The breakeven quantity is computed
\end{enumerate}

$Q_{\mathrm{BE}}=\frac{F+C}{P-V}=\frac{(¥ 22,500,000+¥ 9,000,000)}{(¥ 108-¥ 72)}=875,000$

\begin{enumerate}
  \setcounter{enumi}{7}
  \item $\mathrm{C}$ is correct. 4G, Inc's degree of total leverage can be shown to equal 8, whereas Qphone Corp.'s degree of total leverage is only DOL $\times \mathrm{DFL}=1.4 \times 1.15=1.61$. Therefore, a 10 percent increase in unit sales will mean an 80 percent increase in net income for 4G, but only a 16.1 percent increase in net income for Qphone Corp. The calculation for 4G, Inc.'s DTL is
\end{enumerate}

$$
\begin{aligned}
& \text { DTL }=\frac{Q(P-V)}{Q(P-V)-F-C} \\
& =\frac{1,000,000(¥ 108-¥ 72)}{1,000,000(¥ 108-¥ 72)-¥ 22,500,000-¥ 9,000,000}=8.00
\end{aligned}
$$

\begin{enumerate}
  \setcounter{enumi}{8}
  \item A is correct. Degree of total leverage is defined as the percentage change in net income divided by the percentage change in units sold.

  \item $\mathrm{C}$ is correct. The companies' degree of operating leverage should be the same, consistent with C. Sales risk refers to the uncertainty of the number of units produced and sold and the price at which units are sold. Business risk is the joint effect of sales risk and operating risk. 11. $\mathrm{C}$ is correct. The degree of operating leverage is the elasticity of operating earnings with respect to the number of units produced and sold. As an elasticity, the degree of operating leverage measures the sensitivity of operating earnings to a change in the number of units produced and sold.

  \item C is correct. Because DOL is 4 , if unit sales increase by 5 percent, Fulcrum's operating earnings are expected to increase by $4 \times 5 \%=20 \%$. The calculation for DOL is:

\end{enumerate}

$$
\begin{aligned}
& \mathrm{DOL}=\frac{(40 \text { million })(\$ 100-\$ 65)}{[(40 \text { million) }(\$ 100-\$ 65)]-\$ 1.05 \text { billion }} \\
= & \frac{\$ 1.400 \text { billion }}{\$ 1.400 \text { billion }-\$ 1.05 \text { billion }}=\frac{\$ 1.4}{\$ 0.35}=4
\end{aligned}
$$

\begin{enumerate}
  \setcounter{enumi}{12}
  \item $C$ is correct. Business risk reflects operating leverage and factors that affect sales (such as those given).

  \item B is correct. Grundlegend's degree of operating leverage is the same as Basic Company's, whereas Grundlegend's degree of total leverage and degree of financial leverage are higher.

  \item B is correct.

\end{enumerate}

Operating breakeven units $=\frac{¥ 1,290 \text { million }}{(¥ 3,529-¥ 1,500)}=635,781.173$ units

Operating breakeven sales $=¥ 3,529 \times 635,781.173$ units $=¥ 2,243,671,760$

or

Operating breakeven sales $=\frac{¥ 1,290 \text { million }}{1-(¥ 1,500 / ¥ 3,529)}=¥ 2,243,671,760$

Total breakeven $=\frac{¥ 1,290 \text { million }+¥ 410 \text { million }}{(¥ 3,529-¥ 1,500)}=\frac{¥ 1,700 \text { million }}{¥ 2,029}$

$=837,851.1582$ units

Breakeven sales $=¥ 3,529 \times 837,851.1582$ units $=¥ 2,956,776,737$

or

Breakeven sales $=\frac{¥ 1,700 \text { million }}{1-(¥ 1,500 / ¥ 3,529)}=¥ 2,956,776,737$

\begin{enumerate}
  \setcounter{enumi}{15}
  \item A is correct. For The Gearing Company,
\end{enumerate}

$Q_{\mathrm{BE}}=\frac{F+C}{P-V}=\frac{\$ 40 \text { million }+\$ 20 \text { million }}{\$ 200-\$ 120}=750,000$

For Hebelkraft, Inc.,

$Q_{\mathrm{BE}}=\frac{F+C}{P-V}=\frac{\$ 90 \text { million }+\$ 20 \text { million }}{\$ 200-\$ 100}=1,100,000$

\section{Equity Investments}
\section*{LEARNING MODULE 
 1 }
\section{Market Organization and Structure}
Larry Harris, PhD, CFA, is at the USC Marshall School of Business (USA).

\section{LEARNING OUTCOME}
\begin{center}
\begin{tabular}{c|l}
Mastery & The candidate should be able to: \\
\hline
$\square$ & explain the main functions of the financial system \\
describe classifications of assets and markets &  \\
describe the major types of securities, currencies, contracts, &  \\
commodities, and real assets that trade in organized markets, &  \\
including their distinguishing characteristics and major subtypes &  \\
describe types of financial intermediaries and services that they &  \\
provide &  \\
compare positions an investor can take in an asset &  \\
$\square$ & $\begin{array}{l}\text { calculate and interpret the leverage ratio, the rate of return on a } \\ \text { margin transaction, and the security price at which the investor } \\ \text { would receive a margin call } \\ \text { compare execution, validity, and clearing instructions } \\ \text { compare market orders with limit orders } \\ \text { define primary and secondary markets and explain how secondary } \\ \text { markets support primary markets } \\ \text { describe how securities, contracts, and currencies are traded in } \\ \text { quote-driven, order-driven, and brokered markets } \\ \text { describe characteristics of a well-functioning financial system } \\ \text { describe objectives of market regulation }\end{array}$ \\
$\square$ &  \\
\end{tabular}
\end{center}

\section{INTRODUCTION}
Financial analysts gather and process information to make investment decisions, including those related to buying and selling assets. Generally, the decisions involve trading securities, currencies, contracts, commodities, and real assets such as real estate. Consider several examples:

\begin{itemize}
  \item Fixed income analysts evaluate issuer credit-worthiness and macroeconomic prospects to determine which bonds and notes to buy or sell to preserve capital while obtaining a fair rate of return.

  \item Stock analysts study corporate values to determine which stocks to buy or sell to maximize the value of their stock portfolios.

  \item Corporate treasurers analyze exchange rates, interest rates, and credit conditions to determine which currencies to trade and which notes to buy or sell to have funds available in a needed currency.

  \item Risk managers work for producers or users of commodities to calculate how many commodity futures contracts to buy or sell to manage inventory risks.

\end{itemize}

Financial analysts must understand the characteristics of the markets in which their decisions will be executed. This reading, by examining those markets from the analyst's perspective, provides that understanding.

This reading is organized as follows. Section 2 examines the functions of the financial system. Section 3 introduces assets that investors, information-motivated traders, and risk managers use to advance their financial objectives and presents ways practitioners classify these assets into markets. These assets include such financial instruments as securities, currencies, and some contracts; certain commodities; and real assets. Financial analysts must know the distinctive characteristics of these trading assets.

Section 4 is an overview of financial intermediaries (entities that facilitate the functioning of the financial system). Section 5 discusses the positions that can be obtained while trading assets. You will learn about the benefits and risks of long and short positions, how these positions can be financed, and how the financing affects their risks. Section 6 discusses how market participants order trades and how markets process those orders. These processes must be understood to achieve trading objectives while controlling transaction costs.

Section 7 focuses on describing primary markets. Section 8 describes the structures of secondary markets in securities. Sections 9 and 10 close the reading with discussions of the characteristics of a well-functioning financial system and of how regulation helps make financial markets function better. A summary reviews the reading's major ideas and points, and practice problems conclude.

\section{THE FUNCTIONS OF THE FINANCIAL SYSTEM}
explain the main functions of the financial system

The financial system includes markets and various financial intermediaries that help transfer financial assets, real assets, and financial risks in various forms from one entity to another, from one place to another, and from one point in time to another. These transfers take place whenever someone exchanges one asset or financial contract for another. The assets and contracts that people (people act on behalf of themselves, companies, charities, governments, etc., so the term "people" has a broad definition in this reading) trade include notes, bonds, stocks, exchange-traded funds, currencies, forward contracts, futures contracts, option contracts, swap contracts, and certain commodities. When the buyer and seller voluntarily arrange their trades, as is usually the case, the buyer and the seller both expect to be better off.

People use the financial system for six main purposes:

\begin{enumerate}
  \item to save money for the future;

  \item to borrow money for current use;

  \item to raise equity capital;

  \item to manage risks;

  \item to exchange assets for immediate and future deliveries; and

  \item to trade on information.

\end{enumerate}

The main functions of the financial system are to facilitate:

\begin{enumerate}
  \item the achievement of the purposes for which people use the financial system;

  \item the discovery of the rates of return that equate aggregate savings with aggregate borrowings; and

  \item the allocation of capital to the best uses.

\end{enumerate}

These functions are extremely important to economic welfare. In a well-functioning financial system, transaction costs are low, analysts can value savings and investments, and scarce capital resources are used well.

Sections 2.1 through 2.3 expand on these three functions. The six subsections of Section 2.1 cover the six main purposes for which people use the financial system and how the financial system facilitates the achievement of those purposes. Sections 2.2 and 2.3 discuss determining rates of return and capital allocation efficiency, respectively.

\section{Helping People Achieve Their Purposes in Using the Financial System}
People often arrange transactions to achieve more than one purpose when using the financial system. For example, an investor who buys the stock of an oil producer may do so to move her wealth from the present to the future, to hedge the risk that she will have to pay more for energy in the future, and to exploit insightful research that she conducted that suggests the company's stock is undervalued in the marketplace. If the investment proves to be successful, she will have saved money for the future, managed her energy risk exposure, and obtained a return on her research.

The separate discussions of each of the six main uses of the financial system by people will help you better identify the reasons why people trade. Your ability to identify the various uses of the financial system will help you avoid confusion that often leads to poor financial decisions. The financial intermediaries that are mentioned in these discussions are explained further in Section 4.

\section{Saving}
People often have money that they choose not to spend now and that they want available in the future. For example, workers who save for their retirements need to move some of their current earnings into the future. When they retire, they will use their savings to replace the wages that they will no longer be earning. Similarly, companies save money from their sales revenue so that they can pay vendors when their bills come due, repay debt, or acquire assets (for example, other companies or machinery) in the future. To move money from the present to the future, savers buy notes, certificates of deposit, bonds, stocks, mutual funds, or real assets such as real estate. These alternatives generally provide a better expected rate of return than simply storing money. Savers then sell these assets in the future to fund their future expenditures. When savers commit money to earn a financial return, they commonly are called investors. They invest when they purchase assets, and they divest when they sell them.

Investors require a fair rate of return while their money is invested. The required fair rate of return compensates them for the use of their money and for the risk that they may lose money if the investment fails or if inflation reduces the real value of their investments.

The financial system facilitates savings when institutions create investment vehicles, such as bank deposits, notes, stocks, and mutual funds, that investors can acquire and sell without paying substantial transaction costs. When these instruments are fairly priced and easy to trade, investors will use them to save more.

\section{Borrowing}
People, companies, and governments often want to spend money now that they do not have. They can obtain money to fund projects that they wish to undertake now by borrowing it. Companies can also obtain funds by selling ownership or equity interests (covered in Section 2.1.3). Banks and other investors provide those requiring funds with money because they expect to be repaid with interest or because they expect to be compensated with future disbursements, such as dividends and capital gains, as the ownership interest appreciates in value.

People may borrow to pay for such items as vacations, homes, cars, or education. They generally borrow through mortgages and personal loans, or by using credit cards. People typically repay these loans with money they earn later.

Companies often require money to fund current operations or to engage in new capital projects. They may borrow the needed funds in a variety of ways, such as arranging a loan or a line of credit with a bank, or selling fixed income securities to investors. Companies typically repay their borrowing with income generated in the future. In addition to borrowing, companies may raise funds by selling ownership interests.

Governments may borrow money to pay salaries and other expenses, to fund projects, to provide welfare benefits to their citizens and residents, and to subsidize various activities. Governments borrow by selling bills, notes, or bonds. Governments repay their debt using future revenues from taxes and in some instances from the projects funded by these debts.

Borrowers can borrow from lenders only if the lenders believe that they will be repaid. If the lenders believe, however, that repayment in full with interest may not occur, they will demand higher rates of interest to cover their expected losses and to compensate them for the discomfit they experience wondering whether they will lose their money. To lower the costs of borrowing, borrowers often pledge assets as collateral for their loans. The assets pledged as collateral often include those that will be purchased by the proceeds of the loan. If the borrowers do not repay their loans, the lenders can sell the collateral and use the proceeds to settle the loans.

Lenders often will not loan to borrowers who intend to invest in risky projects, especially if the borrowers cannot pledge other collateral. Investors may still be willing to supply capital for these risky projects if they believe that the projects will likely produce valuable future cash flows. Rather than lending money, however, they will contribute capital in exchange for equity in the projects.

The financial system facilitates borrowing. Lenders aggregate from savers the funds that borrowers require. Borrowers must convince lenders that they can repay their loans, and that, in the event they cannot, lenders can recover most of the funds lent. Credit bureaus, credit rating agencies, and governments promote borrowing; credit bureaus and credit rating agencies do so by collecting and disseminating information that lenders need to analyze credit prospects and governments do so by establishing bankruptcy codes and courts that define and enforce the rights of borrowers and lenders. When the transaction costs of loans (i.e., the costs of arranging, monitoring, and collecting them) are low, borrowers can borrow more to fund current expenditures with credible promises to return the money in the future.

\section{Raising Equity Capital}
Companies often raise money for projects by selling (issuing) ownership interests (e.g., corporate common stock or partnership interests). Although these equity instruments legally represent ownership in companies rather than loans to the companies, selling equity to raise capital is simply another mechanism for moving money from the future to the present. When shareholders or partners contribute capital to a company, the company obtains money in the present in exchange for equity instruments that will be entitled to distributions in the future. Although the repayment of the money is not scheduled as it would be for loans, equity instruments also represent potential claims on money in the future.

The financial system facilitates raising equity capital. Investment banks help companies issue equities, analysts value the securities that companies sell, and regulatory reporting requirements and accounting standards attempt to ensure the production of meaningful financial disclosures. The financial system helps promote capital formation by producing the financial information needed to determine fair prices for equity. Liquid markets help companies raise capital. In these markets, shareholders can easily divest their equities as desired. When investors can easily value and trade equities, they are more willing to fund reasonable projects that companies wish to undertake.

\section{EXAMPLE 1}
\section{Financing Capital Projects}
\begin{enumerate}
  \item As a chief financial officer (CFO) of a large industrial firm, you need to raise cash within a few months to pay for a project to expand existing and acquire new manufacturing facilities. What are the primary options available to you?
\end{enumerate}

\section{Solution:}
Your primary options are to borrow the funds or to raise the funds by selling ownership interests. If the company borrows the funds, you may have the company pledge some or all of the project as collateral to reduce the cost of borrowing.

\section{Managing Risks}
Many people, companies, and governments face financial risks that concern them. These risks include default risk and the risk of changes in interest rates, exchange rates, raw material prices, and sale prices, among many other risks. These risks are often managed by trading contracts that serve as hedges for the risks.

For example, a farmer and a food processor both face risks related to the price of grain. The farmer fears that prices will be lower than expected when his grain is ready for sale whereas the food processor fears that prices will be higher than expected when she has to buy grain in the future. They both can eliminate their exposures to these risks if they enter into a binding forward contract for the farmer to sell a specified quantity of grain to the food processor at a future date at a mutually agreed upon price. By entering into a forward contract that sets the future trade price, they both eliminate their exposure to changing grain prices.

In general, hedgers trade to offset or insure against risks that concern them. In addition to forward contracts, they may use futures contracts, option contracts, or insurance contracts to transfer risk to other entities more willing to bear the risks (these contracts will be covered in Section 3.4). Often the hedger and the other entity face exactly the opposite risks, so the transfer makes both more secure, as in the grain example.

The financial system facilitates risk management when liquid markets exist in which risk managers can trade instruments that are correlated (or inversely correlated) with the risks that concern them without incurring substantial transaction costs. Investment banks, exchanges, and insurance companies devote substantial resources to designing such contracts and to ensuring that they will trade in liquid markets. When such markets exist, people are better able to manage the risks that they face and often are more willing to undertake risky activities that they expect will be profitable.

\section{Exchanging Assets for Immediate Delivery (Spot Market Trading)}
People and companies often trade one asset for another that they rate more highly or, equivalently, that is more useful to them. They may trade one currency for another currency, or money for a needed commodity or right. Following are some examples that illustrate these trades:

\begin{itemize}
  \item Volkswagen pays its German workers in euros, but the company receives dollars when it sells cars in the United States. To convert money from dollars to euros, Volkswagen trades in the foreign exchange markets.

  \item A Mexican investor who is worried about the prospects for peso inflation or a potential devaluation of the peso may buy gold in the spot gold market. (This transaction may hedge against the risk of devaluation of the peso because the value of gold may increase with inflation.)

  \item A plastic producer must buy carbon credits to emit carbon dioxide when burning fuel to comply with environmental regulations. The carbon credit is a legal right that the producer must have to engage in activities that emit carbon dioxide.

\end{itemize}

In each of these cases, the trades are considered spot market trades because the instruments trade for immediate delivery. The financial system facilitates these exchanges when liquid spot markets exist in which people can arrange and settle trades without substantial transaction costs.

\section{Information-Motivated Trading}
Information-motivated traders trade to profit from information that they believe allows them to predict future prices. Like all other traders, they hope to buy at low prices and sell at higher prices. Unlike pure investors, however, they expect to earn a return on their information in addition to the normal return expected for bearing risk through time.

Active investment managers are information-motivated traders who collect and analyze information to identify securities, contracts, and other assets that their analyses indicate are under- or overvalued. They then buy those that they consider undervalued and sell those that they consider overvalued. If successful, they obtain a greater return than the unconditional return that would be expected for bearing the risk in their positions. The return that they expect to obtain is a conditional return earned on the basis of the information in their analyses. Practitioners often call this process active portfolio management. Note that the distinction between pure investors and information-motivated traders depends on their motives for trading and not on the risks that they take or their expected holding periods. Investors trade to move wealth from the present to the future whereas information-motivated traders trade to profit from superior information about future values. When trading to move wealth forward, the time period may be short or long. For example, a bank treasurer may only need to move money overnight and might use money market instruments trading in an interbank funds market to accomplish that. A pension fund, however, may need to move money 30 years forward and might do that by using shares trading in a stock market. Both are investors although their expected holding periods and the risks in the instruments that they trade are vastly different.

In contrast, information-motivated traders trade because their information-based analyses suggest to them that prices of various instruments will increase or decrease in the future at a rate faster than others without their information or analytical models would expect. After establishing their positions, they hope that prices will change quickly in their favor so that they can close their positions, realize their profits, and redeploy their capital. These price changes may occur almost instantaneously, or they may take years to occur if information about the mispricing is difficult to obtain or understand.

The two categories of traders are not mutually exclusive. Investors also are often information-motivated traders. Many investors who want to move wealth forward through time collect and analyze information to select securities that will allow them to obtain conditional returns that are greater than the unconditional returns expected for securities in their asset classes. If they have rational reasons to expect that their efforts will indeed produce superior returns, they are information-motivated traders. If they consistently fail to produce such returns, their efforts will be futile, and they would have been better off simply buying and holding well-diversified portfolios.

\section{EXAMPLE 2}
\section{Investing versus Information-Motivated Trading}
\begin{enumerate}
  \item The head of a large labor union with a pension fund asks you, a pension consultant, to distinguish between investing and information-motivated trading. You are expected to provide an explanation that addresses the financial problems that she faces. How would you respond?
\end{enumerate}

\section{Solution:}
The object of investing for the pension fund is to move the union's pension assets from the present to the future when they will be needed to pay the union's retired pensioners. The pension fund managers will typically do this by buying stocks, bonds, and perhaps other assets. The pension fund managers expect to receive a fair rate of return on the pension fund's assets without paying excessive transaction costs and management fees. The return should compensate the fund for the risks that it bears and for the time that other people are using the fund's money.

The object of information-motivated trading is to earn a return in excess of the fair rate of return. Information-motivated traders analyze information that they collect with the hope that their analyses will allow them to predict better than others where prices will be in the future. They then buy assets that they think will produce excess returns and sell those that they think will underperform. Active investment managers are information-motivated traders. The characteristic that most distinguishes investors from information-motivated traders is the return that they expect. Although both types of traders hope to obtain extraordinary returns, investors rationally expect to receive only fair returns during the periods of their investments. In contrast, information-motivated traders expect to make returns in excess of required fair rates of return. Of course, not all investing or information-motivated trading is successful (in other words, the actual returns may not equal or exceed the expected returns).

The financial system facilitates information-motivated trading when liquid markets allow active managers to trade without significant transaction costs. Accounting standards and reporting requirements that produce meaningful financial disclosures reduce the costs of being well informed, but do not necessarily help informed traders profit because they often compete with each other. The most profitable well-informed traders are often those that have the most unique insights into future values.

\section{Summary}
People use the financial system for many purposes, the most important of which are saving, borrowing, raising equity capital, managing risk, exchanging assets in spot markets, and information-motivated trading. The financial system best facilitates these uses when people can trade instruments that interest them in liquid markets, when institutions provide financial services at low cost, when information about assets and about credit risks is readily available, and when regulation helps ensure that everyone faithfully honors their contracts.

\section{Determining Rates of Return}
Saving, borrowing, and selling equity are all means of moving money through time. Savers move money from the present to the future whereas borrowers and equity issuers move money from the future to the present.

Because time machines do not exist, money can travel forward in time only if an equal amount of money is travelling in the other direction. This equality always occurs because borrowers and equity sellers create the securities in which savers invest. For example, the bond sold by a company that needs to move money from the future to the present is the same bond bought by a saver who needs to move money from the present to the future.

The aggregate amount of money that savers will move from the present to the future is related to the expected rate of return on their investments. If the expected return is high, they will forgo current consumption and move more money to the future. Similarly, the aggregate amount of money that borrowers and equity sellers will move from the future to the present depends on the costs of borrowing funds or of giving up ownership. These costs can be expressed as the rate of return that borrowers and equity sellers are expected to deliver in exchange for obtaining current funds. It is the same rate that savers expect to receive when delivering current funds. If this rate is low, borrowers and equity sellers will want to move more money to the present from the future. In other words, they will want to raise more funds.

Because the total money saved must equal the total money borrowed and received in exchange for equity, the expected rate of return depends on the aggregate supply of funds through savings and the aggregate demand for funds. If the rate is too high, savers will want to move more money to the future than borrowers and equity issuers will want to move to the present. The expected rate will have to be lower to discourage the savers and to encourage the borrowers and equity issuers. Conversely, if the rate is too low, savers will want to move less money forward than borrowers and equity issuers will want to move to the present. The expected rate will have to be higher to encourage the savers and to discourage the borrowers and equity issuers. Between rates too high and too low, an expected rate of return exists, in theory, in which the aggregate supply of funds for investing (supply of funds saved) and the aggregate demand for funds through borrowing and equity issuing are equal.

Economists call this rate the equilibrium interest rate. It is the price for moving money through time. Determining this rate is one of the most important functions of the financial system. The equilibrium interest rate is the only interest rate that would exist if all securities were equally risky, had equal terms, and were equally liquid. In fact, the required rates of return for securities vary by their risk characteristics, terms, and liquidity. For a given issuer, investors generally require higher rates of return for equity than for debt, for long-term securities than for short-term securities, and for illiquid securities than for liquid ones. Financial analysts recognize that all required rates of return depend on a common equilibrium interest rate plus adjustments for risk.

\section{EXAMPLE 3}
\section{Interest Rates}
\begin{enumerate}
  \item For a presentation to private wealth clients by your firm's chief economist, you are asked to prepare the audience by explaining the most fundamental facts concerning the role of interest rates in the economy. You agree. What main points should you try to convey?
\end{enumerate}

\section{Solution:}
Savers have money now that they will want to use in the future. Borrowers want to use money now that they do not have, but they expect that they will have money in the future. Borrowers are loaned money by savers and promise to repay it in the future.

The interest rate is the return that lenders, the savers, expect to receive from borrowers for allowing borrowers to use the savers' money. The interest rate is the price of using money.

Interest rates depend on the total amount of money that people want to borrow and the total amount of money that people are willing to lend. Interest rates are high when, in aggregate, people value having money now substantially more than they value having money in the future. In contrast, if many people with money want to use it in the future and few people presently need more money than they have, interest rates will be low.

\section{Capital Allocation Efficiency}
Primary capital markets (primary markets) are the markets in which companies and governments raise capital (funds). Companies may raise funds by borrowing money or by issuing equity. Governments may raise funds by borrowing money.

Economies are said to be allocationally efficient when their financial systems allocate capital (funds) to those uses that are most productive. Although companies may be interested in getting funding for many potential projects, not all projects are worth funding. One of the most important functions of the financial system is to ensure that only the best projects obtain scarce capital funds; the funds available from savers should be allocated to the most productive uses.

In market-based economies, savers determine, directly or indirectly, which projects obtain capital. Savers determine capital allocations directly by choosing which securities they will invest in. Savers determine capital allocations indirectly by giving funds to financial intermediaries that then invest the funds. Because investors fear the loss of their money, they will lend at lower interest rates to borrowers with the best credit prospects or the best collateral, and they will lend at higher rates to other borrowers with less secure prospects. Similarly, they will buy only those equities that they believe have the best prospects relative to their prices and risks.

To avoid losses, investors carefully study the prospects of the various investment opportunities available to them. The decisions that they make tend to be well informed, which helps ensure that capital is allocated efficiently. The fear of losses by investors and by those raising funds to invest in projects ensures that only the best projects tend to be funded. The process works best when investors are well informed about the prospects of the various projects.

In general, investors will fund an equity project if they expect that the value of the project is greater than its cost, and they will not fund projects otherwise. If the investor expectations are accurate, only projects that should be undertaken will be funded and all such projects will be funded. Accurate market information thus leads to efficient capital allocation.

\section{EXAMPLE 4}
\section{Primary Market Capital Allocation}
\begin{enumerate}
  \item How can poor information about the value of a project result in poor capital allocation decisions?
\end{enumerate}

\section{Solution:}
Projects should be undertaken only if their value is greater than their cost. If investors have poor information and overestimate the value of a project in which its true value is less than its cost, a wealth-diminishing project may be undertaken. Alternatively, if investors have poor information and underestimate the value of a project in which its true value is greater than its cost, a wealth-enhancing project may not be undertaken.

\section{ASSETS AND CONTRACTS}
describe classifications of assets and markets

People, companies, and governments use many different assets and contracts to further their financial goals and to manage their risks. The most common assets include financial assets (such as bank deposits, certificates of deposit, loans, mortgages, corporate and government bonds and notes, common and preferred stocks, real estate investment trusts, master limited partnership interests, pooled investment products, and exchange-traded funds), currencies, certain commodities (such as gold and oil), and real assets (such as real estate). The most common contracts are option, futures, forward, swap, and insurance contracts. People, companies, and governments use these assets and contracts to raise funds, to invest, to profit from information-motivated trading, to hedge risks, and/or to transfer money from one form to another.

\section{Classifications of Assets and Markets}
Practitioners often classify assets and the markets in which they trade by various common characteristics to facilitate communications with their clients, with each other, and with regulators.

The most actively traded assets are securities, currencies, contracts, and commodities. In addition, real assets are traded. Securities generally include debt instruments, equities, and shares in pooled investment vehicles. Currencies are monies issued by national monetary authorities. Contracts are agreements to exchange securities, currencies, commodities or other contracts in the future. Commodities include precious metals, energy products, industrial metals, and agricultural products. Real assets are tangible properties such as real estate, airplanes, or machinery. Securities, currencies, and contracts are classified as financial assets whereas commodities and real assets are classified as physical assets.

Securities are further classified as debt or equity. Debt instruments (also called fixed-income instruments) are promises to repay borrowed money. Equities represent ownership in companies. Pooled investment vehicle shares represent ownership of an undivided interest in an investment portfolio. The portfolio may include securities, currencies, contracts, commodities, or real assets. Pooled investment vehicles, such as exchange-traded funds, which exclusively own shares in other companies, generally are also considered equities.

Securities are also classified by whether they are public or private securities. Public securities are those registered to trade in public markets, such as on exchanges or through dealers. In most jurisdictions, issuers must meet stringent minimum regulatory standards, including reporting and corporate governance standards, to issue publicly traded securities.

Private securities are all other securities. Often, only specially qualified investors can purchase private equities and private debt instruments. Investors may purchase them directly from the issuer or indirectly through an investment vehicle specifically formed to hold such securities. Issuers often issue private securities when they find public reporting standards too burdensome or when they do not want to conform to the regulatory standards associated with public equity. Venture capital is private equity that investors supply to companies when or shortly after they are founded. Private securities generally are illiquid. In contrast, many public securities trade in liquid markets in which sellers can easily find buyers for their securities.

Contracts are derivative contracts if their values depend on the prices of other underlying assets. Derivative contracts may be classified as physical or financial depending on whether the underlying instruments are physical products or financial securities. Equity derivatives are contracts whose values depend on equities or indexes of equities. Fixed-income derivatives are contracts whose values depend on debt securities or indexes of debt securities.

Practitioners classify markets by whether the markets trade instruments for immediate delivery or for future delivery. Markets that trade contracts that call for delivery in the future are forward or futures markets. Those that trade for immediate delivery are called spot markets to distinguish them from forward markets that trade contracts on the same underlying instruments. Options markets trade contracts that deliver in the future, but delivery takes place only if the holders of the options choose to exercise them.

When issuers sell securities to investors, practitioners say that they trade in the primary market. When investors sell those securities to others, they trade in the secondary market. In the primary market, funds flow to the issuer of the security from the purchaser. In the secondary market, funds flow between traders. Practitioners classify financial markets as money markets or capital markets. Money markets trade debt instruments maturing in one year or less. The most common such instruments are repurchase agreements (defined in Section 3.2.1), negotiable certificates of deposit, government bills, and commercial paper. In contrast, capital markets trade instruments of longer duration, such as bonds and equities, whose values depend on the credit-worthiness of the issuers and on payments of interest or dividends that will be made in the future and may be uncertain. Corporations generally finance their operations in the capital markets, but some also finance a portion of their operations by issuing short-term securities, such as commercial paper.

Finally, practitioners distinguish between traditional investment markets and alternative investment markets. Traditional investments include all publicly traded debts and equities and shares in pooled investment vehicles that hold publicly traded debts and/or equities. Alternative investments include hedge funds, private equities (including venture capital), commodities, real estate securities and real estate properties, securitized debts, operating leases, machinery, collectibles, and precious gems. Because these investments are often hard to trade and hard to value, they may sometimes trade at substantial deviations from their intrinsic values. The discounts compensate investors for the research that they must do to value these assets and for their inability to easily sell the assets if they need to liquidate a portion of their portfolios.

The remainder of this section describes the most common assets and contracts that people, companies, and governments trade.

\section{EXAMPLE 5}
\section{Asset and Market Classification}
The investment policy of a mutual fund only permits the fund to invest in public equities traded in secondary markets. Would the fund be able to purchase:

\begin{enumerate}
  \item Common stock of a company that trades on a large stock exchange?
\end{enumerate}

\section{Solution:}
Yes. Common stock is equity. Those common stocks that trade on large exchanges invariably are public equities that trade in secondary markets.

\begin{enumerate}
  \setcounter{enumi}{1}
  \item Common stock of a public company that trades only through dealers?
\end{enumerate}

\section{Solution:}
Yes. Dealer markets are secondary markets and the security is a public equity.

\begin{enumerate}
  \setcounter{enumi}{2}
  \item A government bond?
\end{enumerate}

\section{Solution:}
No. Although government bonds are public securities, they are not equities. They are debt securities. 4. A single stock futures contract?

\section{Solution:}
No. Although the underlying instruments for single stock futures are invariably public equities, single stock futures are derivative contracts, not equities.

\begin{enumerate}
  \setcounter{enumi}{4}
  \item Common stock sold for the first time by a properly registered public company?
\end{enumerate}

\section{Solution:}
No. The fund would not be able to buy these shares because a purchase from the issuer would be in the primary market. The fund would have to wait until it could buy the shares from someone other than the issuer.

\begin{enumerate}
  \setcounter{enumi}{5}
  \item Shares in a privately held bank with $€ 10$ billion of capital?
\end{enumerate}

\section{Solution:}
No. These shares are private equities, not public equities. The public prominence of the company does not make its securities public securities unless they have been properly registered as public securities.

\section{SECURITIES}
describe the major types of securities, currencies, contracts, commodities, and real assets that trade in organized markets, including their distinguishing characteristics and major subtypes

People, companies, and governments sell securities to raise money. Securities include bonds, notes, commercial paper, mortgages, common stocks, preferred stocks, warrants, mutual fund shares, unit trusts, and depository receipts. These can be classified broadly as fixed-income instruments, equities, and shares in pooled investment vehicles. Note that the legal definition of a security varies by country and may or may not coincide with the usage here. Securities that are sold to the public or that can be resold to the public are called issues. Companies and governments are the most common issuers.

\section{Fixed Income}
Fixed-income instruments contractually include predetermined payment schedules that usually include interest and principal payments. Fixed-income instruments generally are promises to repay borrowed money but may include other instruments with payment schedules, such as settlements of legal cases or prizes from lotteries. The payment amounts may be pre-specified or they may vary according to a fixed formula that depends on the future values of an interest rate or a commodity price. Bonds, notes, bills, certificates of deposit, commercial paper, repurchase agreements, loan agreements, and mortgages are examples of promises to repay money in the future. People, companies, and governments create fixed-income instruments when they borrow money. Corporations and governments issue bonds and notes. Fixed-income securities with shorter maturities are called "notes," those with longer maturities are called "bonds." The cutoff is usually at 10 years. In practice, however, the terms are generally used interchangeably. Both become short-term instruments when the remaining time until maturity is short, usually taken to be one year or less.

Some corporations issue convertible bonds, which are typically convertible into stock, usually at the option of the holder after some period. If stock prices are high so that conversion is likely, convertibles are valued like stock. Conversely, if stock prices are low so that conversion is unlikely, convertibles are valued like bonds.

Bills, certificates of deposit, and commercial paper are respectively issued by governments, banks, and corporations. They usually mature within a year of being issued; certificates of deposit sometimes have longer initial maturities.

Repurchase agreements (repos) are short-term lending instruments. The term can be as short as overnight. A borrower seeking funds will sell an instrument-typically a high-quality bond-to a lender with an agreement to repurchase it later at a slightly higher price based on an agreed upon interest rate.

Practitioners distinguish between short-term, intermediate-term, and long-term fixed-income securities. No general consensus exists about the definition of short-term, intermediate-term, and long-term. Instruments that mature in less than one to two years are considered short-term instruments whereas those that mature in more than five to ten years are considered long-term instruments. In the middle are intermediate-term instruments.

Instruments trading in money markets are called money market instruments. Such instruments are traded debt instruments maturing in one year or less. Money market funds and corporations seeking a return on their short-term cash balances typically hold money market instruments.

\section{Equities}
Equities represent ownership rights in companies. These include common and preferred shares. Common shareholders own residual rights to the assets of the company. They have the right to receive any dividends declared by the boards of directors, and in the event of liquidation, any assets remaining after all other claims are paid. Acting through the boards of directors that they elect, common shareholders usually can select the managers who run the corporations.

Preferred shares are equities that have preferred rights (relative to common shares) to the cash flows and assets of the company. Preferred shareholders generally have the right to receive a specific dividend on a regular basis. If the preferred share is a cumulative preferred equity, the company must pay the preferred shareholders any previously omitted dividends before it can pay dividends to the common shareholders. Preferred shareholders also have higher claims to assets relative to common shareholders in the event of corporate liquidation. For valuation purposes, financial analysts generally treat preferred stocks as fixed-income securities when the issuers will clearly be able to pay their promised dividends in the foreseeable future.

Warrants are securities issued by a corporation that allow the warrant holders to buy a security issued by that corporation, if they so desire, usually at any time before the warrants expire or, if not, upon expiration. The security that warrant holders can buy usually is the issuer's common stock, in which case the warrants are considered equities because the warrant holders can obtain equity in the company by exercising their warrants. The warrant exercise price is the price that the warrant holder must pay to buy the security.

\section{EXAMPLE 6}
\section{Securities}
\begin{enumerate}
  \item What factors distinguish fixed-income securities from equities?
\end{enumerate}

\section{Solution:}
Fixed-income securities generate income on a regular schedule. They derive their value from the promise to pay a scheduled cash flow. The most common fixed-income securities are promises made by people, companies, and governments to repay loans.

Equities represent residual ownership in companies after all other claimsincluding any fixed-income liabilities of the company-have been satisfied. For corporations, the claims of preferred equities typically have priority over the claims of common equities. Common equities have the residual ownership in corporations.

\section{Pooled Investments}
Pooled investment vehicles are mutual funds, trusts, depositories, and hedge funds, that issue securities that represent shared ownership in the assets that these entities hold. The securities created by mutual funds, trusts, depositories, and hedge fund are respectively called shares, units, depository receipts, and limited partnership interests but practitioners often use these terms interchangeably. People invest in pooled investment vehicles to benefit from the investment management services of their managers and from diversification opportunities that are not readily available to them on an individual basis.

Mutual funds are investment vehicles that pool money from many investors for investment in a portfolio of securities. They are often legally organized as investment trusts or as corporate investment companies. Pooled investment vehicles may be open-ended or closed-ended. Open-ended funds issue new shares and redeem existing shares on demand, usually on a daily basis. The price at which a fund redeems and sells the fund's shares is based on the net asset value of the fund's portfolio, which is the difference between the fund's assets and liabilities, expressed on a per share basis. Investors generally buy and sell open-ended mutual funds by trading with the mutual fund.

In contrast, closed-end funds issue shares in primary market offerings that the fund or its investment bankers arrange. Once issued, investors cannot sell their shares of the fund back to the fund by demanding redemption. Instead, investors in closed-end funds must sell their shares to other investors in the secondary market. The secondary market prices of closed-end funds may differ-sometimes quite significantly-from their net asset values. Closed-end funds generally trade at a discount to their net asset values. The discount reflects the expenses of running the fund and sometimes investor concerns about the quality of the management. Closed-end funds may also trade at a discount or a premium to net asset value when investors believe that the portfolio securities are overvalued or undervalued. Many financial analysts thus believe that discounts and premiums on closed-end funds measure market sentiment.

Exchange-traded funds (ETFs) and exchange-traded notes (ETNs) are open-ended funds that investors can trade among themselves in secondary markets. The prices at which ETFs trade rarely differ much from net asset values because a class of investors, known as authorized participants (APs), has the option of trading directly with the ETF. If the market price of an equity ETF is sufficiently below its net asset value, APs will buy shares in the secondary market at market price and redeem shares at net asset value with the fund. Conversely, if the price of an ETF is sufficiently above its net asset value, APs will buy shares from the fund at net asset value and sell shares in the secondary market at market price. As a result, the market price and net asset values of ETFs tend to converge.

Many ETFs permit only in-kind deposits and redemptions. Buyers who buy directly from such a fund pay for their shares with a portfolio of securities rather than with cash. Similarly, sellers receive a portfolio of securities. The transaction portfolio generally is very similar-often essentially identical-to the portfolio held by the fund. Practitioners sometimes call such funds "depositories" because they issue depository receipts for the portfolios that traders deposit with them. The traders then trade the receipts in the secondary market. Some warehouses holding industrial materials and precious metals also issue tradable warehouse receipts.

Asset-backed securities are securities whose values and income payments are derived from a pool of assets, such as mortgage bonds, credit card debt, or car loans. These securities typically pass interest and principal payments received from the pool of assets through to their holders on a monthly basis. These payments may depend on formulas that give some classes of securities-called tranches-backed by the pool more value than other classes.

Hedge funds are investment funds that generally organize as limited partnerships. The hedge fund managers are the general partners. The limited partners are qualified investors who are wealthy enough and well informed enough to tolerate and accept substantial losses, should they occur. The regulatory requirements to participate in a hedge fund and the regulatory restrictions on hedge funds vary by jurisdiction. Most hedge funds follow only one investment strategy, but no single investment strategy characterizes hedge funds as a group. Hedge funds exist that follow almost every imaginable strategy ranging from long-short arbitrage in the stock markets to direct investments in exotic alternative assets.

The primary distinguishing characteristic of hedge funds is their management compensation scheme. Almost all funds pay their managers with an annual fee that is proportional to their assets and with an additional performance fee that depends on the wealth that the funds generate for their shareholders. A secondary distinguishing characteristic of many hedge funds is the use of leverage to increase risk exposure and to hopefully increase returns.

\section{CURRENCIES, COMMODITIES, AND REAL ASSETS}
describe the major types of securities, currencies, contracts, commodities, and real assets that trade in organized markets, including their distinguishing characteristics and major subtypes

Currencies are monies issued by national monetary authorities. Approximately 180 currencies are currently in use throughout the world. Some of these currencies are regarded as reserve currencies. Reserve currencies are currencies that national central banks and other monetary authorities hold in significant quantities. The primary reserve currencies are the US dollar and the euro. Secondary reserve currencies include the British pound, the Japanese yen, and the Swiss franc. Currencies trade in foreign exchange markets. In spot currency transactions, one currency is immediately or almost immediately exchanged for another. The rate of exchange is called the spot exchange rate. Traders typically negotiate institutional trades in multiples of large quantities, such as US 1 million or $¥ 100$ million. Institutional trades generally settle in two business days.

Retail currency trades most commonly take place through commercial banks when their customers exchange currencies at a location of the bank, use ATM machines when travelling to withdraw a different currency than the currency in which their bank accounts are denominated, or use credit cards to buy items priced in different currencies. Retail currency trades also take place at airport kiosks, at store front currency exchanges, or on the street.

\section{Commodities}
Commodities include precious metals, energy products, industrial metals, agricultural products, and carbon credits. Spot commodity markets trade commodities for immediate delivery whereas the forward and futures markets trade commodities for future delivery. Managers seeking positions in commodities can acquire them directly by trading in the spot markets or indirectly by trading forward and futures contracts.

The producers and processors of industrial metals and agricultural products are the primary users of the spot commodity markets because they generally are best able to take and make delivery and to store physical products. They undertake these activities in the normal course of operating their businesses. Their ability to handle physical products and the information that they gather operating businesses also gives them substantial advantages as information-motivated traders in these markets. Many producers employ financial analysts to help them analyze commodity market conditions so that they can best manage their inventories to hedge their operational risks and to speculate on future price changes.

Commodities also interest information-motivated traders and investment managers because they can use them as hedges against risks that they hold in their portfolios or as vehicles to speculate on future price changes. Most such traders take positions in the futures markets because they usually do not have facilities to handle most physical products nor can they easily obtain them. They also cannot easily cope with the normal variation in qualities that characterizes many commodities. Information-motivated traders and investment managers also prefer to trade in futures markets because most futures markets are more liquid than their associated spot markets and forward markets. The liquidity allows them to easily close their positions before delivery so that they can avoid handling physical products.

Some information-motivated traders and investment managers, however, trade in the spot commodity markets, especially when they can easily contract for low-cost storage. Commodities for which delivery and storage costs are lowest are nonperishable products for which the ratio of value to weight is high and variation in quality is low. These generally include precious metals, industrial diamonds, such high-value industrial metals as copper, aluminum, and mercury, and carbon credits.

\section{Real Assets}
Real assets include such tangible properties as real estate, airplanes, machinery, or lumber stands. These assets normally are held by operating companies, such as real estate developers, airplane leasing companies, manufacturers, or loggers. Many institutional investment managers, however, have been adding real assets to their portfolios as direct investments (involving direct ownership of the real assets) and indirect investments (involving indirect ownership, for example, purchase of securities of companies that invest in real assets or real estate investment trusts). Investments in real assets are attractive to them because of the income and tax benefits that they often generate, and because changes in their values may have a low correlation with other investments that the managers hold.

Direct investments in real assets generally require substantial management to ensure that the assets are maintained and used efficiently. Investment managers investing in such assets must either hire personnel to manage them or hire outside management companies. Either way, management of real assets is quite costly.

Real assets are unique properties in the sense that no two assets are alike. An example of a unique property is a real estate parcel. No two parcels are the same because, if nothing else, they are located in different places. Real assets generally differ in their conditions, remaining useful lives, locations, and suitability for various purposes. These differences are very important to the people who use them, so the market for a given real asset may be very limited. Thus, real assets tend to trade in very illiquid markets.

The heterogeneity of real assets, their illiquidity, and the substantial costs of managing them are all factors that complicate the valuation of real assets and generally make them unsuitable for most investment portfolios. These same problems, however, often cause real assets to be misvalued in the market, so astute information-motivated traders may occasionally identify significantly undervalued assets. The benefits from purchasing such assets, however, are often offset by the substantial costs of searching for them and by the substantial costs of managing them.

Many financial intermediaries create entities, such as real estate investment trusts (REITs) and master limited partnerships (MLPs), to securitize real assets and to facilitate indirect investment in real assets. The financial intermediaries manage the assets and pass through the net benefits after management costs to the investors who hold these securities. Because these securities are much more homogenous and divisible than the real assets that they represent, they tend to trade in much more liquid markets. Thus, they are much more suitable as investments than the real assets themselves.

Of course, investors seeking exposure to real assets can also buy shares in corporations that hold and operate real assets. Although almost all corporations hold and operate real assets, many specialize in assets that particularly interest investors seeking exposure to specific real asset classes. For example, investors interested in owning aircraft can buy an aircraft leasing company such as Waha Capital (Abu Dhabi Securities Exchange) and Aircastle Limited (NYSE).

\section{EXAMPLE 7}
\section{Assets and Contracts}
Consider the following assets and contracts:

Bank deposits

Certificates of deposit

Common stocks

Corporate bonds

Currencies

Exchange-traded funds

Lumber forward contracts

Crude oil futures contracts

Gold Hedge funds

Master limited partnership interests

Mortgages

Mutual funds

Stock option contracts

Preferred stocks

Real estate parcels

Interest rate swaps

Treasury notes 1. Which of these represent ownership in corporations?

\section{Solution:}
Common and preferred stocks represent ownership in corporations.

\begin{enumerate}
  \setcounter{enumi}{1}
  \item Which of these are debt instruments?
\end{enumerate}

\section{Solution:}
Bank deposits, certificates of deposit, corporate bonds, mortgages, and Treasury notes are all debt instruments. They respectively represent loans made to banks, corporations, mortgagees (typically real estate owners), and the Treasury.

\begin{enumerate}
  \setcounter{enumi}{2}
  \item Which of these are created by traders rather than by issuers?
\end{enumerate}

\section{Solution:}
Lumber forward contracts, crude oil futures contracts, stock option contracts, and interest rate swaps are created when the seller sells them to a buyer.

\begin{enumerate}
  \setcounter{enumi}{3}
  \item Which of these are pooled investment vehicles?
\end{enumerate}

\section{Solution:}
Exchange-traded funds, hedge funds, and mutual funds are pooled investment vehicles. They represent shared ownership in a portfolio of other assets.

\begin{enumerate}
  \setcounter{enumi}{4}
  \item Which of these are real assets?
\end{enumerate}

\section{Solution:}
Real estate parcels are real assets.

\begin{enumerate}
  \setcounter{enumi}{5}
  \item Which of these would a home builder most likely use to hedge construction costs?
\end{enumerate}

\section{Solution:}
A builder would buy lumber forward contracts to lock in the price of lumber needed to build homes.

\begin{enumerate}
  \setcounter{enumi}{6}
  \item Which of these would a corporation trade when moving cash balances among various countries?
\end{enumerate}

\section{Solution:}
Corporations often trade currencies when moving cash from one country to another.

\section{CONTRACTS}
$$
\begin{aligned}
& \text { describe the major types of securities, currencies, contracts, } \\
& \text { commodities, and real assets that trade in organized markets, } \\
& \text { including their distinguishing characteristics and major subtypes }
\end{aligned}
$$

A contract is an agreement among traders to do something in the future. Contracts include forward, futures, swap, option, and insurance contracts. The values of most contracts depend on the value of an underlying asset. The underlying asset may be a commodity, a security, an index representing the values of other instruments, a currency pair or basket, or other contracts.

Contracts provide for some physical or cash settlement in the future. In a physically settled contract, settlement occurs when the parties to the contract physically exchange some item, such as avocados, pork bellies, or gold bars. Physical settlement also includes the delivery of such financial instruments as bonds, equities, or futures contracts even though the delivery is electronic. In contrast, cash settled contracts settle through cash payments. The amount of the payment depends on formulas specified in the contracts.

Financial analysts classify contracts by whether they are physical or financial based on the nature of the underlying asset. If the underlying asset is a physical product, the contract is a physical; otherwise, the contract is a financial. Examples of assets classified as physical include contracts for the delivery of petroleum, lumber, and gold. Examples of assets classified as financial include option contracts, and contracts on interest rates, stock indexes, currencies, and credit default swaps.

Contracts that call for immediate delivery are called spot contracts, and they trade in spot markets. Immediate delivery generally is three days or less, but depends on each market. All other contracts involve what practitioners call futurity. They derive their values from events that will take place in the future.

\section{EXAMPLE 8}
\section{Contracts for Difference}
Contracts for difference (CFD) allow people to speculate on price changes for an underlying asset, such as a common stock or an index. Dealers generally sell CFDs to their clients. When the clients sell the CFDs back to their dealer, they receive any appreciation in the underlying asset's price between the time of purchase and sale (open and close) of the contract. If the underlying asset's price drops over this interval, the client pays the dealer the difference.

\begin{enumerate}
  \item Are contracts for difference derivative contracts?
\end{enumerate}

\section{Solution:}
Contracts for difference are derivative contracts because their values are derived from changes in the prices of the underlying asset on which they are based. 2. Are contracts for difference based on copper prices cash settled or physically settled?

\section{Solution:}
All contracts for difference are cash settled contracts regardless of the underlying asset on which they are based because they settle in cash and not in the underlying asset.

\section{Forward Contracts}
A forward contract is an agreement to trade the underlying asset in the future at a price agreed upon today. For example, a contract for the sale of wheat after the harvest is a forward contract. People often use forward contracts to reduce risk. Before planting wheat, farmers like to know the price at which they will sell their crop. Similarly, before committing to sell flour to bakers in the future, millers like to know the prices that they will pay for wheat. The farmer and the miller both reduce their operating risks by agreeing to trade wheat forward.

Practitioners call such traders hedgers because they use their contractual commitments to hedge their risks. If the price of wheat falls, the wheat farmer's crop will drop in value on the spot market but he has a contract to sell wheat in the future at a higher fixed price. The forward contract has become more valuable to the farmer. Conversely, if the price of wheat rises, the miller's future obligation to sell flour will become more burdensome because of the high price he would have to pay for wheat on the spot market, but the miller has a contract to buy wheat at a lower fixed price. The forward contract has become more valuable to the miller. In both cases, fluctuations in the spot price are hedged by the forward contract. The forward contract offsets the operating risks that the hedgers face.

Consider a simple example of hedging. An avocado farmer in Mexico expects to harvest 15,000 kilograms of avocados and that the price at harvest will be 60 pesos per kilogram. That price, however, could fluctuate significantly before the harvest. If the price of avocados drops to 50 pesos, the farmer would lose 10 pesos per kilogram (60 pesos - 50 pesos) relative to his expectations, or a total of 150,000 pesos. Now, suppose that the farmer can sell avocados forward to Del Rey Avocado at 58 pesos for delivery at the harvest. If the farmer sells 15,000 kilograms forward, and the price of avocados drops to 50 pesos, the farmer would still be able to sell his avocados for 58 pesos, and thus would not suffer from the drop in the price of avocados below this level.

\section{EXAMPLE 9}
\section{Hedging Gold Production}
A Zimbabwean gold producer invests in a mine expansion project on the expectation that gold prices will remain at or above $\$ 1,200$ USD per ounce when the new project starts producing ore.

\begin{enumerate}
  \item What risks does the gold producer face with respect to the price of gold?
\end{enumerate}

\section{Solution:}
The gold producer faces the risk that the price of gold could fall below $\$ 1,200$ USD before it can sell its new production. If so, the investment in the expansion project will be less profitable than expected, and may even generate losses for the mine. 2. How might the gold producer hedge its gold price risk?

Solution:

The gold producer could hedge the gold price risk by selling gold forward, hopefully at a price near $\$ 1,200$ USD. Even if the price of gold falls, the gold producer would get paid the contract price.

Forward contracts are very common, but two problems limit their usefulness for many market participants. The first problem is counterparty risk. Counterparty risk is the risk that the other party to a contract will fail to honor the terms of the contract. Concerns about counterparty risk ensure that generally only parties who have long-standing relationships with each other execute forward contracts. Trustworthiness is critical when prices are volatile because, after a large price change, one side or the other may prefer not to settle the contract.

The second problem is liquidity. Trading out of a forward contract is very difficult because it can only be done with the consent of the other party. The liquidity problem ensures that forward contracts tend to be executed only among participants for whom delivery is economically efficient and quite certain at the time of contracting so that both parties will want to arrange for delivery.

The counterparty risk problem and the liquidity problem often make it difficult for market participants to obtain the hedging benefits associated with forward contracting. Fortunately, futures contracts have been developed to mitigate these problems.

\section{Futures Contracts}
A futures contract is a standardized forward contract for which a clearinghouse guarantees the performance of all traders. The buyer of a futures contract is the side that will take physical delivery or its cash equivalent. The seller of a futures contract is the side that is liable for the delivery or its cash equivalent. A clearinghouse is an organization that ensures that no trader is harmed if another trader fails to honor the contract. In effect, the clearinghouse acts as the buyer for every seller and as the seller for every buyer. Buyers and sellers, therefore, can trade futures without worrying whether their counterparties are creditworthy. Because futures contracts are standardized, a buyer can eliminate his obligation to buy by selling his contract to anyone. A seller similarly can eliminate her obligation to deliver by buying a contract from anyone. In either case, the clearinghouse will release the trader from all future obligations if his or her long and short positions exactly offset each other.

To protect against defaults, futures clearinghouses require that all participants post with the clearinghouse an amount of money known as initial margin when they enter a contract. The clearinghouse then settles the margin accounts on a daily basis. All participants who have lost on their contracts that day will have the amount of their losses deducted from their margin by the clearinghouse. The clearinghouse similarly increases margins for all participants who gained on that day. Participants whose margins drop below the required maintenance margin must replenish their accounts. If a participant does not provide sufficient additional margin when required, the participant's broker will immediately trade to offset the participant's position. These variation margin payments ensure that the liabilities associated with futures contracts do not grow large.

\section{EXAMPLE 10}
\section{Futures Margin}
\begin{enumerate}
  \item NYMEX's Light Sweet Crude Oil futures contract specifies the delivery of 1,000 barrels of West Texas Intermediate Crude Oil when the contract finally settles. A broker requires that its clients post an initial overnight margin of $\$ 7,763$ per contract and an overnight maintenance margin of $\$ 5,750$ per contract. A client buys ten contracts at $\$ 75$ per barrel through this broker. On the next day, the contract settles for $\$ 72$ per barrel. How much additional margin will the client have to provide to his broker?
\end{enumerate}

\section{Solution:}
The client lost three dollars per barrel (he is the side committed to take delivery or its cash equivalent at $\$ 75$ per barrel). This results in a $\$ 3,000$ loss on each of his 10 contracts, and a total loss of $\$ 30,000$. His initial margin of $\$ 77,630$ is reduced by $\$ 30,000$ leaving $\$ 47,630$ in his margin account. Because his account has dropped below the maintenance margin requirement of $\$ 57,500$, the client will get a margin call. The client must provide an additional $\$ 30,000=\$ 77,630-\$ 47,630$ to replenish his margin account; the account is replenished to the amount of the initial margin. The client will only receive another margin call if his account drops to below $\$ 57,500$ again.

Futures contracts have vastly improved the efficiency of forward contracting markets. Traders can trade standardized futures contracts with anyone without worrying about counterparty risk, and they can close their positions by arranging offsetting trades. Hedgers for whom the terms of the standard contract are not ideal generally still use the futures markets because the contracts embody most of the price risk that concerns them. They simply offset (close out) their futures positions, at the same time they enter spot contracts on which they make or take ultimate delivery.

\section{EXAMPLE 11}
\section{Forward and Futures Contracts}
\begin{enumerate}
  \item What feature most distinguishes futures contracts from forward contracts?
\end{enumerate}

\section{Solution:}
A futures contract is a standardized forward contract for which a clearinghouse guarantees the performance of all buyers and sellers. The clearinghouse reduces the counterparty risk problem. The clearinghouse allows a buyer who has bought a contract from one person and sold the same contract to another person to net out the two obligations so that she is no longer liable for either side of the contract; the positions are closed. The ability to trade futures contracts provides liquidity in futures contracts compared with forward contracts.

\section{Swap Contracts}
A swap contract is an agreement to exchange payments of periodic cash flows that depend on future asset prices or interest rates. For example, in a typical interest rate swap, at periodic intervals, one party makes fixed cash payments to the counterparty in exchange for variable cash payments from the counterparty. The variable payments are based on a pre-specified variable interest rate such as the London Interbank Offered Rate (Libor). This swap effectively exchanges fixed interest payments for variable interest payments. Because the variable rate is set in the future, the cash flows for this contract are uncertain when the parties enter the contract.

Investment managers often enter interest rate swaps when they own a fixed long-term income stream that they want to convert to a cash flow that varies with current short-term interest rates, or vice versa. The conversion may allow them to substantially reduce the total interest rate risk to which they are exposed. Hedgers often use swap contracts to manage risks.

In a commodity swap, one party typically makes fixed payments in exchange for payments that depend on future prices of a commodity such as oil. In a currency swap, the parties exchange payments denominated in different currencies. The payments may be fixed, or they may vary depending on future interest rates in the two countries. In an equity swap, the parties exchange fixed cash payments for payments that depend on the returns to a stock or a stock index.

\section{EXAMPLE 12}
\section{Swap and Forward Contracts}
\begin{enumerate}
  \item What feature most distinguishes a swap contract from a cash-settled forward contract?
\end{enumerate}

\section{Solution:}
Both contracts provide for the exchange of cash payments in the future. A forward contract only has a single cash payment at the end that depends on an underlying price or index at the end. In contrast, a swap contract has several scheduled periodic payments, each of which depends on an underlying price or index at the time of the payment.

\section{Option Contracts}
An option contract allows the holder (the purchaser) of the option to buy or sell, depending on the type of option, an underlying instrument at a specified price at or before a specified date in the future. Those that do buy or sell are said to exercise their contracts. An option to buy is a call option, and an option to sell is a put option. The specified price is called the strike price (exercise price). If the holders can exercise their contracts only when they mature, they are European-style contracts. If they can exercise the contracts earlier, they are American-style contracts. Many exchanges list standardized option contracts on individual stocks, stock indexes, futures contracts, currencies, swaps, and precious metals. Institutions also trade many customized option contracts with dealers in the over-the-counter derivative market.

Option holders generally will exercise call options if the strike price is below the market price of the underlying instrument, in which case, they will be able to buy at a lower price than the market price. Similarly, they will exercise put options if the strike price is above the underlying instrument price so that they sell at a higher price than the market price. Otherwise, option holders allow their options to expire as worthless.

The price that traders pay for an option is the option premium. Options can be quite expensive because, unlike forward and futures contracts, they do not impose any liability on the holder. The premium compensates the sellers of options-called option writers-for giving the call option holders the right to potentially buy below market prices and put option holders the right to potentially sell above market prices. Because the writers must trade if the holders exercise their options, option contracts may impose substantial liabilities on the writers.

\section{EXAMPLE 13}
\section{Option and Forward Contracts}
\begin{enumerate}
  \item What feature most distinguishes option contracts from forward contracts?
\end{enumerate}

\section{Solution:}
The holder of an option contract has the right, but not the obligation, to buy (for a call option) or sell (for a put option) the underlying instrument at some time in the future. The writer of an option contract must trade the underlying instrument if the holder exercises the option.

In contrast, the two parties to a forward contract must trade the underlying instrument (or its equivalent value for a cash-settled contract) at some time in the future if either party wants to settle the contract.

\section{Other Contracts}
Insurance contracts pay their beneficiaries a cash benefit if some event occurs. Life, liability, and automobile insurance are examples of insurance contracts sold to retail clients. People generally use insurance contracts to compensate for losses that they will experience if bad things happen unexpectedly. Insurance contracts allow them to hedge risks that they face.

Credit default swaps (CDS) are insurance contracts that promise payment of principal in the event that a company defaults on its bonds. Bondholders use credit default swaps to convert risky bonds into more secure investments. Other creditors of the company may also buy them to hedge against the risk they will not be paid if the company goes bankrupt.

Well-informed traders who believe that a corporation will default on its bonds may buy credit default swaps written on the corporation's bonds if the swap prices are sufficiently low. If they are correct, the traders will profit if the payoff to the swap is more than the cost of buying and maintaining the swap position.

People sometimes also buy insurance contracts as investments, especially in jurisdictions where payouts from insurance contracts are not subject to as much taxation as are payouts to other investment vehicles. They may buy these contracts directly from insurance companies, or they may buy already issued contracts from their owners. For example, the life settlements market trades life insurance contracts that people sell to investors when they need cash.

\section{FINANCIAL INTERMEDIARIES}
describe types of financial intermediaries and services that they provide Financial intermediaries help entities achieve their financial goals. These intermediaries include commercial, mortgage, and investment banks; credit unions, credit card companies, and various other finance corporations; brokers and exchanges; dealers and arbitrageurs; clearinghouses and depositories; mutual funds and hedge funds; and insurance companies. The services and products that financial intermediaries provide allow their clients to solve the financial problems that they face more efficiently than they could do so by themselves. Financial intermediaries are essential to well-functioning financial systems.

Financial intermediaries are called intermediaries because the services and products that they provide help connect buyers to sellers in various ways. Whether the connections are easy to identify or involve complex financial structures, financial intermediaries stand between one or more buyers and one or more sellers and help them transfer capital and risk between them. Financial intermediaries' activities allow buyers and sellers to benefit from trading, often without any knowledge of the other.

This section introduces the main financial intermediaries that provide services and products in well-developed financial markets. The discussion starts with those intermediaries whose services most obviously connect buyers to sellers and then proceeds to those intermediaries whose services create more subtle connections. Because many financial intermediaries provide many different types of services, some are mentioned more than once. The section concludes with a general characterization of the various ways in which financial intermediaries add value to the financial system.

\section{Brokers, Exchanges, and Alternative Trading Systems}
Brokers are agents who fill orders for their clients. They do not trade with their clients. Instead, they search for traders who are willing to take the other side of their clients' orders. Individual brokers may work for large brokerage firms, the brokerage arm of banks, or at exchanges. Some brokers match clients to clients personally. Others use specialized computer systems to identify potential trades and help their clients fill their orders. Brokers help their clients trade by reducing the costs of finding counterparties for their trades.

Block brokers provide brokerage service to large traders. Large orders are hard to fill because finding a counterparty willing to do a large trade is often quite difficult. A large buy order generally will trade at a premium to the current market price, and a large sell order generally will trade at a discount to the current market price. These price concessions encourage other traders to trade with the large traders. They also make large traders reluctant, however, to expose their orders to the public before their trades are arranged because they do not want to move the market. Block brokers, therefore, carefully manage the exposure of the orders entrusted to them, which makes filling them difficult.

Investment banks provide advice to their mostly corporate clients and help them arrange transactions such as initial and seasoned securities offerings. Their corporate finance divisions help corporations finance their business by issuing securities, such as common and preferred shares, notes, and bonds. Another function of corporate finance divisions is to help companies identify and acquire other companies (i.e., in mergers and acquisitions).

Exchanges provide places where traders can meet to arrange their trades. Historically, brokers and dealers met on an exchange floor to negotiate trades. Increasingly, exchanges arrange trades for traders based on orders that brokers and dealers submit to them. Such exchanges essentially act as brokers. The distinction between exchanges and brokers has become quite blurred. Exchanges and brokers that use electronic order matching systems to arrange trades among their clients are functionally indistinguishable in this respect. Examples of exchanges include the NYSE, Eurex, Frankfurt Stock Exchange, the Chicago Mercantile Exchange, the Tokyo Stock Exchange, and the Singapore Exchange.

Exchanges are easily distinguished from brokers by their regulatory operations. Most exchanges regulate their members' behavior when trading on the exchange, and sometimes away from the exchange.

Many securities exchanges regulate the issuers that list their securities on the exchange. These regulations generally require timely financial disclosure. Financial analysts use this information to value the securities traded at the exchange. Without such disclosure, valuing securities could be very difficult and market prices might not reflect the fundamental values of the securities. In such situations, well-informed participants may profit from less-informed participants. To avoid such losses, the less-informed participants may withdraw from the market, which can greatly increase corporate costs of capital.

Some exchanges also prohibit issuers from creating capital structures that would concentrate voting rights in the hands of a few owners who do not own a commensurate share of the equity. These regulations attempt to ensure that corporations are run for the benefit of all shareholders and not to promote the interests of controlling shareholders who do not have significant economic stakes in the company.

Exchanges derive their regulatory authority from their national or regional governments, or through the voluntary agreements of their members and issuers to subject themselves to the exchange regulations. In most countries, government regulators oversee the exchange rules and the regulatory operations. Most countries also impose financial disclosure standards on public issuers. Examples of government regulatory bodies include the Japanese Financial Services Agency, the British Financial Conduct Authority, the German Federal Financial Supervisory Authority (BaFin), the US Securities and Exchange Commission, the Ontario Securities Commission, and the Argentine National Securities Commission (CNV).

Alternative trading systems (ATSs), also known as electronic communications networks (ECNs) or multilateral trading facilities (MTFs) are trading venues that function like exchanges but that do not exercise regulatory authority over their subscribers except with respect to the conduct of their trading in their trading systems. Some ATSs operate electronic trading systems that are otherwise indistinguishable from the trading systems operated by exchanges. Others operate innovative trading systems that suggest trades to their customers based on information that their customers share with them or that they obtain through research into their customers' preferences. Many ATSs are known as dark pools because they do not display the orders that their clients send to them. Large investment managers especially like these systems because market prices often move to their disadvantage when other traders know about their large orders. ATSs may be owned and operated by broker-dealers, exchanges, banks, or by companies organized solely for this purpose, many of which may be owned by a consortia of brokers-dealers and banks. Examples of ATSs include MATCHNow (Canada), BATS (United States), POSIT (United States), Liquidnet (United States), Baxter-FX (Ireland), and Turquoise (Europe). Many of these ATSs provide services in many markets besides the ones in which they are domiciled.

\section{Dealers}
Dealers fill their clients' orders by trading with them. When their clients want to sell securities or contracts, dealers buy the instruments for their own accounts. If their clients want to buy securities, dealers sell securities that they own or have borrowed. After completing a transaction, dealers hope to reverse the transaction by trading with another client on the other side of the market. When they are successful, they effectively connect a buyer who arrived at one point in time with a seller who arrived at another point in time.

The service that dealers provide is liquidity. Liquidity is the ability to buy or sell with low transactions costs when you want to trade. By allowing their clients to trade when they want to trade, dealers provide liquidity to them. In over-the-counter markets, dealers offer liquidity when their clients ask them to trade with them. In exchange markets, dealers offer liquidity to anyone who is willing to trade at the prices that the dealers offer at the exchange. Dealers profit when they can buy at prices that on average are lower than the prices at which they sell.

Dealers may organize their operations within proprietary trading houses, investment banks, and hedge funds, or as sole proprietorships. Some dealers are traditional dealers in the sense that individuals make trading decisions. Others use computerized trading to make all trading decisions. Examples of companies with large dealing operations include Deutsche Bank (Germany), RBC Capital Markets (Canada), Nomura Securities (Japan), Timber Hill (United States), Goldman Sachs (United States), and IG Group (United Kingdom). Almost all investment banks have large dealing operations.

Most dealers also broker orders, and many brokers deal to their customers. Accordingly, practitioners often use the term broker-dealer to refer to dealers and brokers. Broker-dealers have a conflict of interest with respect to how they fill their customers' orders. When acting as a broker, they must seek the best price for their customers' orders. When acting as dealers, however, they profit most when they sell to their customers at high prices or buy from their customers at low prices. The problem is most serious when the customer allows the broker-dealer to decide whether to trade the order with another trader or to fill it as a dealer. Consequently, when trading with a broker-dealer, some customers specify how they want their orders filled. They may also trade only with pure agency brokers who do not also deal.

Primary dealers are dealers with whom central banks trade when conducting monetary policy. They buy bills, notes, and bonds when the central banks sell them to decrease the money supply. The dealers then sell these instruments to their clients. Similarly, when the central banks want to increase the money supply, the primary dealers buy these instruments from their clients and sell them to the central banks.

\section{EXAMPLE 14}
\section{Brokers and Dealers}
\begin{enumerate}
  \item What characteristic most likely distinguishes brokers from dealers?
\end{enumerate}

\section{Solution:}
Brokers are agents that arrange trades on behalf of their clients. They do not trade with their clients. In contrast, dealers are proprietary traders who trade with their clients.

\section{Arbitrageurs}
Arbitrageurs trade when they can identify opportunities to buy and sell identical or essentially similar instruments at different prices in different markets. They profit when they can buy in one market for less than they sell in another market. Arbitrageurs are financial intermediaries because they connect buyers in one market to sellers in another market. The purest form of arbitrage involves buying and selling the same instrument in two different markets. Arbitrageurs who do such trades sell to buyers in one market and buy from sellers in the other market. They provide liquidity to the markets because they make it easier for buyers and sellers to trade when and where they want to trade.

Because dealers and arbitrageurs both provide liquidity to other traders, they compete with each other. The dealers connect buyers and sellers who arrive in the same market at different times whereas the arbitrageurs connect buyers and sellers who arrive at the same time in different markets. In practice, traders who profit from offering liquidity rarely are purely dealers or purely arbitrageurs. Instead, most traders attempt to identify and exploit every opportunity they can to manage their inventories profitably.

If information about prices is readily available to market participants, pure arbitrages involving the same instrument will be quite rare. Traders who are well informed about market conditions usually route their orders to the market offering the best price so that arbitrageurs will have few opportunities to match traders across markets when they want to trade the exact same instrument.

Arbitrageurs often trade securities or contracts whose values depend on the same underlying factors. For example, dealers in equity option contracts often sell call options in the contract market and buy the underlying shares in the stock market. Because the values of the call options and of the underlying shares are closely correlated (the value of the call increases with the value of the shares), the long stock position hedges the risk in the short call position so that the dealer's net position is not too risky.

Similar to the pure arbitrage that involves the same instrument in different markets, these arbitrage trades connect buyers in one market to sellers in another market. In this case, however, the buyers and sellers are interested in different instruments whose values are closely related. In the example, the buyer is interested in buying a call options contract, the value of which is a nonlinear function of the value of the underlying stock; the seller is interested in selling the underlying stock.

Options dealers buy stock and sell calls when calls are overpriced relative to the underlying stocks. They use complicated financial models to value options in relation to underlying stock values, and they use financial engineering techniques to control the risk of their portfolios. Successful arbitrageurs must know valuation relations well and they must manage the risk in their portfolios well to trade profitably. They profit by buying the relatively undervalued instrument and selling the relatively overvalued instrument.

Buying a risk in one form and selling it another form involves a process called replication. Arbitrageurs use various trading strategies to replicate the returns to securities and contracts. If they can substantially replicate those returns, they can use the replication trading strategy to offset the risk of buying or selling the actual securities and contracts. The combined effect of their trading is to transform risk from one form to another. This process allows them to create or eliminate contracts in response to the excess demand for, and supply of, contracts.

For example, when traders want to buy more call contracts than are presently available, they push the call contract prices up so that calls become overvalued relative to the underlying stock. The arbitrageurs replicate calls by using a particular financial engineering strategy to buy the underlying stock, and then create the desired call option contracts by selling them short. In contrast, if more calls have been created than traders want to hold, call prices will fall so that calls become undervalued relative to the underlying stock. The arbitrageurs will trade stocks and contracts to absorb the excess contracts. Arbitrageurs who use these strategies are financial intermediaries because they connect buyers and sellers who want to trade the same underlying risks but in different forms.

\section{EXAMPLE 15}
\section{Dealers and Arbitrageurs}
\begin{enumerate}
  \item With respect to providing liquidity to market participants, what characteristics most clearly distinguish dealers from arbitrageurs?
\end{enumerate}

\section{Solution:}
Dealers provide liquidity to buyers and sellers who arrive at the same market at different times. They move liquidity through time. Arbitrageurs provide liquidity to buyers and sellers who arrive at different markets at the same time. They move liquidity across markets.

\section{SECURITIZERS, DEPOSITORY INSTITUTIONS AND INSURANCE COMPANIES}
describe types of financial intermediaries and services that they
provide

Banks and investment companies create new financial products when they buy and repackage securities or other assets. For example, mortgage banks commonly originate hundreds or thousands of residential mortgages by lending money to homeowners. They then place the mortgages in a pool and sell shares of the pool to investors as mortgage pass-through securities, which are also known as mortgage-backed securities. All payments of principal and interest are passed through to the investors each month, after deducting the costs of servicing the mortgages. Investors who purchase these pass-through securities obtain securities that in aggregate have the same net cash flows and associated risks as the pool of mortgages.

The process of buying assets, placing them in a pool, and then selling securities that represent ownership of the pool is called securitization.

Mortgage-backed securities have the advantage that default losses and early repayments are much more predictable for a diversified portfolio of mortgages than they are for individual mortgages. They are also attractive to investors who cannot efficiently service mortgages but wish to invest in mortgages. By securitizing mortgage pools, the mortgage banks allow investors who are not large enough to buy hundreds of mortgages to obtain the benefits of diversification and economies of scale in loan servicing.

Securitization greatly improves liquidity in the mortgage markets because it allows investors in the pass-through securities to buy mortgages indirectly that they otherwise would not buy. Because the financial risks associated with mortgage-backed securities (debt securities with specified claims on the cash flows of a portfolio of mortgages) are much more predictable than those of individual mortgages, mortgage-backed securities are easier to price and thus easier to sell when investors need to raise cash. These characteristics make the market for mortgage-backed securities much more liquid than the market for individual mortgages. Because investors value liquidity-the ability to sell when they want to-they will pay more for securitized mortgages than for individual mortgages. The homeowners benefit because higher mortgage prices imply lower interest rates. The mortgage bank is a financial intermediary because it connects investors who want to buy mortgages to homeowners who want to borrow money. The homeowners sell mortgages to the bank when the bank lends them money.

Some mortgage banks form mortgage pools from mortgages that they buy from other banks that originate the loans. These mortgage banks are also financial intermediaries because they connect sellers of mortgages to buyers of mortgage-backed securities. Although the sellers of the mortgages are the originating lenders and not the borrowers, the benefits of creating liquid mortgage-backed securities ultimately flow back to the borrowers.

The creation of the pass-through securities generally takes place on the accounts of the mortgage bank. The bank buys mortgages and sells pass-through securities whose values depend on the mortgage pool. The mortgages appear on the bank's accounts as assets and the mortgage-backed securities appear as liabilities.

In many securitizations, the financial intermediary avoids placing the assets and liabilities on its balance sheet by setting up a special corporation or trust that buys the assets and issues the securities. That corporation or trust is called a special purpose vehicle (SPV) or alternatively a special purpose entity (SPE). Conducting a securitization through a special purpose vehicle is advantageous to investors because their interests in the asset pool are better protected in an SPV than they would be on the balance sheet of the financial intermediary if the financial intermediary were to go bankrupt.

Financial intermediaries securitize many assets. Besides mortgages, banks securitize car loans, credit card receivables, bank loans, and airplane leases, to name just a few assets. As a class, these securities are called asset-backed securities.

When financial intermediaries securitize assets, they often create several classes of securities, called tranches, that have different rights to the cash flows from the asset pool. The tranches are structured so that some produce more predictable cash flows than do others. The senior tranches have first rights to the cash flow from the asset pool. Because the overall risk of a given asset pool cannot be changed, the more junior tranches bear a disproportionate share of the risk of the pool. Practitioners often call the most junior tranche toxic waste because it is so risky. The complexity associated with slicing asset pools into tranches can make the resulting securities difficult to value. Mistakes in valuing these securities contributed to the financial crisis that started in 2007.

Investment companies also create pass-through securities based on investment pools. For example, an exchange-traded fund is an asset-backed security that represents ownership in the securities and contracts held by the fund. The shareholders benefit from the securitization because they can buy or sell an entire portfolio in a single transaction. Because the transaction cost savings are quite substantial, exchange-traded funds often trade in very liquid markets. The investment companies (and sometimes the arbitrageurs) that create exchange-traded funds are financial intermediaries because they connect the buyers of the funds to the sellers of the assets that make up the fund portfolios.

More generally, the creators of all pooled investment vehicles are financial intermediaries that transform portfolios of securities and contracts into securities that represent undivided ownership of the portfolios. The investors in these funds thus indirectly invest in the securities held by the fund. They benefit from the expertise of the investment manager and from obtaining a portfolio that may be more diversified than one they might otherwise be able to hold.

\section{Depository Institutions and Other Financial Corporations}
Depository institutions include commercial banks, savings and loan banks, credit unions, and similar institutions that raise funds from depositors and other investors and lend it to borrowers. The banks give their depositors interest and transaction services, such as check writing and check cashing, in exchange for using their money. They may also raise funds by selling bonds or equity in the bank.

These banks are financial intermediaries because they transfer funds from their depositors and investors to their borrowers. The depositors and investors benefit because they obtain a return (in interest, transaction services, dividends, or capital appreciation) on their funds without having to contract with the borrowers and manage their loans. The borrowers benefit because they obtain the funds that they need without having to search for investors who will trust them to repay their loans.

Many other financial corporations provide credit services. For example, acceptance corporations, discount corporations, payday advance corporations, and factors provide credit to borrowers by lending them money secured by such assets as consumer loans, machinery, future paychecks, or accounts receivables. They finance these loans by selling commercial paper, bonds, and shares to investors. These corporations are intermediaries because they connect investors to borrowers. The investors obtain investments secured by a diversified portfolio of loans while the borrowers obtain funds without having to search for investors.

Brokers also act as financial intermediaries when they lend funds to clients who want to buy securities on margin. They generally obtain the funds from other clients who deposit them in their accounts. Brokers who provide these services to hedge funds and other similar institutions are called prime brokers.

Banks, financial corporations, and brokers can only raise money from depositors and other lenders because their equity owners retain residual interests in the performance of the loans that they make. If the borrowers default, the depositors and other lenders have priority claims over the equity owners. If insufficient money is collected from the borrowers, shareholders' equity is used to pay their depositors and other lenders. The risk of losing capital focuses the equity owners' and management's attention so that credit is not offered foolishly.

Because the ability of these companies to cover their credit losses is limited by the capital that their owners invest in them, the depositors and other investors who lend them money pay close attention to how much money the owners have at risk. For example, if a finance corporation is poorly capitalized, its shareholders will lose little if its clients default on the loans that the finance corporation makes to them. In that case, the finance corporation will have little incentive to lend only to creditworthy borrowers and to effectively manage collection on those loans once they have been made. Worse, it may even choose to lend to borrowers with poor credit because the interest rates that they can charge such borrowers are higher. Until those loans default, the higher income will make the corporation appear to be more profitable than it actually is. Depositors and other investors are aware of these problems and generally pay close attention to them. Accordingly, poorly capitalized financial institutions cannot easily borrow money to finance their operations at favorable rates.

Depository banks and financial corporations are similar to securitized asset pools that issue pass-through securities. Their depositors and investors own securities that ultimately are backed by an asset pool consisting of their loan portfolios. The depositors generally hold the most senior tranche, followed by the other creditors. The shareholders hold the most junior tranche. In the event of bankruptcy, they are paid only if everyone else is paid.

\section{EXAMPLE 16}
\section{Commercial Banks}
\begin{enumerate}
  \item What services do commercial banks provide that make them financial intermediaries?
\end{enumerate}

\section{Solution:}
Commercial banks collect deposits from investors and lend them to borrowers. They are intermediaries because they connect lenders to borrowers. Commercial banks also provide transaction services that make it easier for the banks' depository customers to pay bills and collect funds from their own customers.

\section{Insurance Companies}
Insurance companies help people and companies offset risks that concern them. They do this by creating insurance contracts (policies) that provide a payment in the event that some loss occurs. The insured buy these contracts to hedge against potential losses. Common examples of insurance contracts include auto, fire, life, liability, medical, theft, and disaster insurance contracts.

Credit default swaps are also insurance contracts, but historically they have not been subject to the same reserve requirements that most governments apply to more traditional insurance contracts. They may be sold by insurance companies or by other financial entities, such as investment banks or hedge funds.

Insurance contracts transfer risk from those who buy the contracts to those who sell them. Although insurance companies occasionally broker trades between the insured and the insurer, they more commonly provide the insurance themselves. In that case, the insurance company's owners and creditors become the indirect insurers of the risks that the insurance company assumes. Insurance companies also often transfer risks that they do not wish to bear by buying reinsurance policies from reinsurers.

Insurers are financial intermediaries because they connect the buyers of their insurance contracts with investors, creditors, and reinsurers who are willing to bear the insured risks. The buyers benefit because they can easily obtain the risk transfers that they seek without searching for entities that would be willing to assume those risks.

The owners, creditors, and reinsurers of the insurance company benefit because the company allows them to sell their tolerance for risk easily without having to manage the insurance contracts. Instead, the company manages the relationships with the insured-primarily collections and claims-and hopefully controls the various problems-fraud, moral hazard, and adverse selection - that often plague insurance markets. Fraud occurs when people deliberately cause or falsely report losses to collect on insurance. Moral hazard occurs when people are less careful about avoiding insured losses than they would be if they were not insured so that losses occur more often than they would otherwise. Adverse selection occurs when only those who are most at risk buy insurance so that insured losses tend to be greater than average.

Everyone benefits because insurance companies hold large diversified portfolios of policies. Loss rates for well-diversified portfolios of insurance contracts are much more predictable than for single contracts. For such contracts as auto insurance in which losses are almost uncorrelated across policies, diversification ensures that the financial performance of a large portfolio of contracts will be quite predictable and so holding the portfolio will not be very risky. The insured benefit because they do not have to pay the insurers much to compensate them for bearing risk (the expected loss is quite predictable so the risk is relatively low). Instead, their insurance premiums primarily reflect the expected loss rate in the portfolio plus the costs of running and financing the company.

\section{SETTLEMENT AND CUSTODIAL SERVICES AND SUMMARY}
$$
\begin{aligned}
& \text { describe types of financial intermediaries and services that they } \\
& \text { provide }
\end{aligned}
$$

In addition to connecting buyers to sellers through a variety of direct and indirect means, financial intermediaries also help their customers settle their trades and ensure that the resulting positions are not stolen or pledged more than once as collateral.

Clearinghouses arrange for final settlement of trades. In futures markets, they guarantee contract performance. In other markets, they may act only as escrow agents, transferring money from the buyer to the seller while transferring securities from the seller to the buyer.

The members of a clearinghouse are the only traders for whom the clearinghouse will settle trades. To ensure that their members settle the trades that they present to the clearinghouse, clearinghouses require that their members have adequate capital and post-performance bonds (margins). Clearinghouses also limit the aggregate net (buy minus sell) quantities that their members can settle.

Brokers and dealers who are not members of the clearinghouse must arrange to have a clearinghouse member settle their trades. To ensure that the non-member brokers and dealers can settle their trades, clearinghouse members require that their customers (the non-member brokers and dealers) have adequate capital and post-margins. They also limit the aggregate net quantities that their customers can settle and they monitor their customers' trading to ensure that they do not arrange trades that they cannot settle.

Brokers and dealers similarly monitor the trades made by their retail and institutional customers, and regulate their customers to ensure that they do not arrange trades that they cannot settle.

This hierarchical system of responsibility generally ensures that traders settle their trades. The brokers and dealers guarantee settlement of the trades they arrange for their retail and institutional customers. The clearinghouse members guarantee settlement of the trades that their customers present to them, and clearinghouses guarantee settlement of all trades presented to them by their members. If a clearinghouse member fails to settle a trade, the clearinghouse settles the trade using its own capital or capital drafted from the other members.

Reliable settlement of all trades is extremely important to a well-functioning financial system because it allows strangers to confidently contract with each other without worrying too much about counterparty risk, the risk that their counterparties will not settle their trades. A secure clearinghouse system thus greatly increases liquidity because it greatly increases the number of counterparties with whom a trader can safely arrange a trade.

In many national markets, clearinghouses clear all securities trades so that traders can trade securities through any exchange, broker, alternative trading system, or dealer. These clearinghouse systems promote competition among these exchange service providers. In contrast, most futures exchanges have their own clearinghouses. These clearinghouses usually will not accept trades arranged away from their exchanges so that a competing exchange cannot trade another exchange's contracts. Competing exchanges may create similar contracts, but moving traders from one established market to a new market is extraordinarily difficult because traders prefer to trade where other traders trade.

Depositories or custodians hold securities on behalf of their clients. These services, which are often offered by banks, help prevent the loss of securities through fraud, oversight, or natural disaster. Broker-dealers also often hold securities on behalf of their customers so that the customers do not have to hold the securities in certificate form. To avoid problems with lost certificates, securities increasingly are issued only in electronic form.

\section{EXAMPLE 17}
\section{Financial Intermediaries}
\begin{enumerate}
  \item As a relatively new member of the business community, you decide it would be advantageous to join the local lunch club to network with businessmen. Upon learning that you are a financial analyst, club members soon enlist you to give a lunch speech. During the question and answer session afterwards, a member of the audience asks, "I keep reading in the newspaper about the need to regulate 'financial intermediaries', but really don't understand exactly what they are. Can you tell me?" How do you answer?
\end{enumerate}

\section{Solution:}
Financial intermediaries are companies that help their clients achieve their financial goals. They are called intermediaries because, in some way or another, they stand between two or more people who would like to trade with each other, but for various reasons find it difficult to do so directly. The intermediary arranges the trade for them, or more often, trades with both sides.

For example, a commercial bank is an intermediary that connects investors with money to borrowers who need money. The investors buy certificates of deposit from the bank, buy bonds or stock issued by the bank, or simply are depositors in the bank. The borrowers borrow this money from the bank when they arrange loans. Without the bank's intermediation, the investors would have to find trustworthy borrowers themselves, which would be difficult, and the borrowers would have to find trusting lenders, which would also be difficult.

Similarly, an insurance company is an intermediary because it connects customers who want to insure risks with investors who are willing to bear those risks. The investors own shares or bonds issued by the insurance company, or they have sold reinsurance contracts to the insurance company. The insured benefit because they can more easily buy a policy from an insurance company than they can find counterparties who would be willing to bear their risks. The investors benefit because the insurance company creates a diversified portfolio of risks by selling insurance to thousands or millions of customers. Diversification ensures that the net risk borne by the insurance company and its investors will be predictable and thus financially manageable.

In both cases, the financial intermediary also manages the relationships with its customers and investors so that neither side has to worry about the credit-worthiness or trust-worthiness of its counterparties. For example, the bank manages credit quality and collections on its loans and the insurance company manages risk exposure and collections on its policies. These services benefit both sides by reducing the costs of connecting investors to borrowers or of insured to insurers.

These are only two examples of financial intermediation. Many others involve firms engaged in brokerage, dealing, arbitrage, securitization, investment management, and the clearing and settlement of trades. In all cases, the financial intermediary stands between a buyer and a seller, offering them services that allow them to better achieve their financial goals in a cost effective and efficient manner.

\section{Summary}
By facilitating transactions among buyers and sellers, financial intermediaries provide services essential to a well-functioning financial system. They facilitate transactions the following ways:

\begin{enumerate}
  \item Brokers, exchanges, and various alternative trading systems match buyers and sellers interested in trading the same instrument at the same place and time. These financial intermediaries specialize in discovering and organizing information about who wants to trade.

  \item Dealers and arbitrageurs connect buyers to sellers interested in trading the same instrument but who are not present at the same place and time. Dealers connect buyers to sellers who are present at the same place but at different times whereas arbitrageurs connect buyers to sellers who are present at the same time but in different places. These financial intermediaries trade for their own accounts when providing these services. Dealers buy or sell with one client and hope to do the offsetting transaction later with another client. Arbitrageurs buy from a seller in one market while simultaneously selling to a buyer in another market.

  \item Many financial intermediaries create new instruments that depend on the cash flows and associated financial risks of other instruments. The intermediaries provide these services when they securitize assets, manage investment funds, operate banks and other finance corporations that offer investments to investors and loans to borrowers, and operate insurance companies that pool risks. The instruments that they create generally are more attractive to their clients than the instruments on which they are based. The new instruments also may be differentiated to appeal to diverse clienteles. Their efforts connect buyers of one or more instruments to sellers of other instruments, all of which in aggregate provide the same cash flows and risk exposures. Financial intermediaries thus effectively arrange trades among traders who otherwise would not trade with each other.

  \item Arbitrageurs who conduct arbitrage among securities and contracts whose values depend on common factors convert risk from one form to another. Their trading connects buyers and sellers who want to trade similar risks expressed in different forms.

  \item Banks, clearinghouses, and depositories provide services that ensure traders settle their trades and that the resulting positions are not stolen or pledged more than once as collateral.

\end{enumerate}

\section{POSITIONS AND SHORT POSITIONS}
compare positions an investor can take in an asset

People generally solve their financial and risk management problems by taking positions in various assets or contracts. A position in an asset is the quantity of the instrument that an entity owns or owes. A portfolio consists of a set of positions.

People have long positions when they own assets or contracts. Examples of long positions include ownership of stocks, bonds, currencies, contracts, commodities, or real assets. Long positions benefit from an appreciation in the prices of the assets or contracts owned.

People have short positions when they have sold assets that they do not own, or when they write and sell contracts. Short positions benefit from a decrease in the prices of the assets or contracts sold. Short sellers profit by selling at high prices and repurchasing at lower prices. Information-motivated traders sell assets and contracts short positions when they believe that prices will fall.

Hedgers also often sell instruments short. They short securities and contracts when the financial risks inherent in the instruments are positively correlated with the risks to which they are exposed. For example, to hedge the risk associated with holding copper inventories, a wire manufacturer would sell short copper futures. If the price of copper falls, the manufacturer will lose on his copper inventories but gain on his short futures position. (If the risk in an instrument is inversely correlated with a risk to which hedgers are exposed, the hedgers will hedge with long positions.)

Contracts have long sides and short sides. The long side of a forward or futures contract is the side that will take physical delivery or its cash equivalent. The short side of such contracts is the side that is liable for the delivery. The long side of a futures contract increases in value when the value of the underlying asset increases in value.

The identification of the two sides can be confusing for option contracts. The long side of an option contract is the side that holds the right to exercise the option. The short side is the side that must satisfy the obligation. Practitioners say that that the long side holds the option and the short side writes the option, so the long side is the holder and the short side is the writer. The put contracts are the source of the potential confusion. The put contract holder has the right to sell the underlying to the writer. The holder will benefit if the price of the underlying falls, in which case the price of the put contract will rise. The holder is long the put contract and has an indirect short position in the underlying instrument. Analysts call the indirect short position short exposure to the underlying. The put contract holders have long exposure to their option contract and short exposure to the underlying instrument.

Exhibit 1: Option Positions and Their Associated Underlying Risk Exposures

\begin{center}
\begin{tabular}{lcc}
\hline
Type of Option & Option Position & Exposure to Underlying Risk \\
\hline
Call & Long & Long \\
Call & Short & Short \\
Put & Long & Short \\
Put & Short & Long \\
\hline
\end{tabular}
\end{center}

The identification of the long side in a swap contract is often arbitrary because swap contracts call for the exchange of contractually determined cash flows rather than for the purchase (or the cash equivalent) of some underlying instrument. In general, the side that benefits from an increase in the quoted price is the long side.

The identification of the long side in currency contracts also may be confusing. In this case, the confusion stems from symmetry in the contracts. The buyer of one currency is the seller of the other currency, and vice versa for the seller. Thus, a long forward position in one currency is a short forward position in the other currency. When practitioners describe a position, they generally will say, "I'm long the dollar against the yen," which means they have bought dollars and sold yen.

\section{Short Positions}
Short sellers create short positions in contracts by selling contracts that they do not own. In a sense, they become the issuers of the contract when they create the liabilities associated with their contracts. This analogy will also help you better understand risk when you study corporate finance: Corporations create short positions in their bonds when they issue bonds in exchange for cash. Although bonds are generally considered to be securities, they are also contracts between the issuer and the bondholder.

Short sellers create short positions in securities by borrowing securities from security lenders who are long holders. The short sellers then sell the borrowed securities to other traders. Short sellers close their positions by repurchasing the securities and returning them to the security lenders. If the securities drop in value, the short sellers profit because they repurchase the securities at lower prices than the prices at which they sold the securities. If the securities rise in value, they will lose. Short sellers who buy to close their positions are said to cover their positions.

The potential gains in a long position generally are unbounded. For example, the stock prices of such highly successful companies as Yahoo! have increased more than 50-fold since they were first publicly traded. The potential losses on long positions, however, are limited to no more than 100 percent-a complete loss-for long positions without any associated liabilities.

In contrast, the potential gains on a short position are limited to no more than 100 percent whereas the potential losses are unbounded. The unbounded potential losses on short positions make short positions very risky in volatile instruments. As an extreme example of this, if you had shorted 100 shares of Yahoo! in July 1996 at $\$ 20$ and kept the position open for four years, you would have lost $\$ 148,000$ on your $\$ 2,000$ initial short position. During this period, Yahoo! rose 75 -fold to $\$ 1,500$ on a split-adjusted equivalent basis.

Although security lenders generally believe that they are long the securities that they lend, in fact, they do not actually own the securities during the periods of their loans. Instead, they own promises made by the short sellers to return the securities. These promises are memorialized in security lending agreements. These agreements specify that the short sellers will pay the long sellers all dividends or interest that they otherwise would have received had they not lent their securities. These payments are called payments-in-lieu of dividends (or of interest), and they may have different tax treatments than actual dividends and interest. The security lending agreements also protect the lenders in the event of a stock split.

To secure the security loans, lenders require that the short seller leave the proceeds of the short sale on deposit with them as collateral for the stock loan. They invest the collateral in short-term securities, and they rebate the interest to the short sellers at rates called short rebate rates. The short rebate rates are determined in the market and generally are available only to institutional short-sellers and some large retail traders. If a security is hard to borrow, the rebate rate may be very small or even negative. Such securities are said to be "on special". Most security lending agreements require various margin payments to keep the credit risk among the parties from growing when prices change.

Securities lenders lend their securities because the short rebate rates they pay on the collateral are lower than the interest rates they receive from investing the collateral. The difference is because of the implicit loan fees that they receive from the borrowers for borrowing the stock. The difference also compensates lenders for risks that the lenders take when investing the collateral and for the risk that the borrowers will default if prices rise significantly.

\section{EXAMPLE 18}
\section{Short Positions in Securities and Contracts}
\begin{enumerate}
  \item How is the process of short selling shares of Siemens different from that of short selling a Siemens equity call option contract?
\end{enumerate}

\section{Solution:}
To short sell shares of Siemens, the seller (or his broker) must borrow the shares from a long holder so that he can deliver them to the buyer. To short sell a Siemens equity call option contract, the seller simply creates the contract when he sells it to the buyer.

\section{LEVERAGED POSITIONS}
calculate and interpret the leverage ratio, the rate of return on a margin transaction, and the security price at which the investor would receive a margin call

In many markets, traders can buy securities by borrowing some of the purchase price. They usually borrow the money from their brokers. The borrowed money is called the margin loan, and they are said to buy on margin. The interest rate that the buyers pay for their margin loan is called the call money rate. The call money rate is above the government bill rate and is negotiable. Large buyers generally obtain more favorable rates than do retail buyers. For institutional-size buyers, the call money rate is quite low because the loans are generally well secured by securities held as collateral by the lender.

Trader's equity is that portion of the security price that the buyer must supply. Traders who buy securities on margin are subject to minimum margin requirements. The initial margin requirement is the minimum fraction of the purchase price that must be trader's equity. This requirement may be set by the government, the exchange, or the exchange clearinghouse. For example, in the United States, the Federal Reserve Board sets the initial margin requirement through Regulation T. In Hong Kong SAR, the Securities and Futures Commission sets the margin requirements. In all markets, brokers often require more equity than the government-required minimum from their clients when lending to them.

Many markets allow brokers to lend their clients more money if the brokers use risk models to measure and control the overall risk of their clients' portfolios. This system is called portfolio margining. Buying securities on margin can greatly increase the potential gains or losses for a given amount of equity in a position because the trader can buy more securities on margin than he could otherwise. The buyer thus earns greater profits when prices rise and suffers greater losses when prices fall. The relation between risk and borrowing is called financial leverage (often simply called leverage). Traders leverage their positions when they borrow to buy more securities. A highly leveraged position is large relative to the equity that supports it.

The leverage ratio is the ratio of the value of the position to the value of the equity investment in it. The leverage ratio indicates how many times larger a position is than the equity that supports it. The maximum leverage ratio associated with a position financed by the minimum margin requirement is one divided by the minimum margin requirement. If the requirement is 40 percent, then the maximum leverage ratio is $2.5=100 \%$ position $\div 40 \%$ equity.

The leverage ratio indicates how much more risky a leveraged position is relative to an unleveraged position. For example, if a stock bought on 40 percent margin rises 10 percent, the buyer will experience a 25 percent $(2.5 \times 10 \%)$ return on the equity investment in her leveraged position. But if the stock falls by 10 percent, the return on the equity investment will be -25 percent (before the interest on the margin loan and before payment of commissions).

Financial analysts must be able to compute the total return to the equity investment in a leveraged position. The total return depends on the price change of the purchased security, the dividends or interest paid by the security, the interest paid on the margin loan, and the commissions paid to buy and sell the security. The following example illustrates the computation of the total return to a leveraged purchase of stock that pays a dividend.

\section{EXAMPLE 19}
\section{Computing Total Return to a Leveraged Stock Purchase}
A buyer buys stock on margin and holds the position for exactly one year, during which time the stock pays a dividend. For simplicity, assume that the interest on the loan and the dividend are both paid at the end of the year.

$\begin{array}{lc}\text { Purchase price } & \$ 20 / \text { share } \\ \text { Sale price } & \$ 15 / \text { share } \\ \text { Shares purchased } & 1,000 \\ \text { Leverage ratio } & 2.5 \\ \text { Call money rate } & 5 \% \\ \text { Dividend } & \$ 0.10 / \text { share } \\ \text { Commission } & \$ 0.01 / \text { share }\end{array}$

\begin{enumerate}
  \item What is the total return on this investment?
\end{enumerate}

\section{Solution:}
To find the return on this investment, first determine the initial equity and then determine the equity remaining after the sale. The total purchase price is $\$ 20,000$. The leverage ratio of 2.5 indicates that the buyer's equity financed 40 percent $=(1 \div 2.5)$ of the purchase price. Thus, the equity investment is $\$ 8,000=40 \%$ of $\$ 20,000$. The $\$ 12,000$ remainder is borrowed. The actual investment is slightly higher because the buyer must pay a commission of $\$ 10$ $=\$ 0.01 /$ share $\times 1,000$ shares to buy the stock. The total initial investment is $\$ 8,010$.

At the end of the year, the stock price has declined by $\$ 5 /$ share. The buyer lost $\$ 5,000=\$ 5 /$ share $\times 1,000$ shares as a result of the price change. In addition, the buyer has to pay interest at 5 percent on the $\$ 12,000$ loan, or $\$ 600$. The buyer also receives a dividend of $\$ 0.10$ /share, or $\$ 100$. The trader's equity remaining after the sale is computed from the initial equity investment as follows:

Initial investment

$\$ 8,010$

Purchase commission

$-10$

Trading gains/losses

$-5,000$

Margin interest paid

$-600$

Dividends received

\begin{center}
\begin{tabular}{c}
100 \\
-10 \\
\hline
$\$ 2,490$ \\
\hline
\end{tabular}
\end{center}

Remaining equity

or

\begin{center}
\begin{tabular}{lc}
Proceeds on sale & $\$ 15,000$ \\
Payoff loan & $-12,000$ \\
Margin interest paid & -600 \\
Dividends received & 100 \\
Sales commission paid & -10 \\
\hline
Remaining equity & $\$ 2,490$ \\
\hline
\end{tabular}
\end{center}

so that the return on the initial investment of $\$ 8,010$ is $(2,490-8,010) / 8,010$

$=-68.9 \%$.

\begin{enumerate}
  \setcounter{enumi}{1}
  \item Why is the loss greater than the 25 percent decrease in the market price?
\end{enumerate}

\section{Solution:}
The realized loss is substantially greater than the stock price return of $(\$ 15-$ $\$ 20) / \$ 20=-25 \%$. Most of the difference is because of the leverage with the remainder primarily the result of the interest paid on the loan. Based on the leverage alone and ignoring the other cash flows, we would expect that the return on the equity would be $-62.5 \%=2.5$ leverage times the $-25 \%$ stock price return.

In the above example, if the stock dropped more than the buyer's original 40 percent margin (ignoring commissions, interest, and dividends), the trader's equity would have become negative. In that case, the investor would owe his broker more than the stock is worth. Brokers often lose money in such situations if the buyer does not repay the loan out of other funds.

To prevent such losses, brokers require that margin buyers always have a minimum amount of equity in their positions. This minimum is called the maintenance margin requirement. It is usually 25 percent of the current value of the position, but it may be higher or lower depending on the volatility of the instrument and the policies of the broker. If the value of the equity falls below the maintenance margin requirement, the buyer will receive a margin call, or request for additional equity. If the buyer does not deposit additional equity with the broker in a timely manner, the broker will close the position to prevent further losses and thereby secure repayment of the margin loan.

When you buy securities on margin, you must know the price at which you will receive a margin call if prices drop. The answer to this question depends on your initial equity and on the maintenance margin requirement.

\section{EXAMPLE 20}
\section{Margin Call Price}
\begin{enumerate}
  \item A trader buys stock on margin posting 40 percent of the initial stock price of $\$ 20$ as equity. The maintenance margin requirement for the position is 25 percent. Below what price will a margin call occur?
\end{enumerate}

\section{Solution:}
The trader's initial equity is 40 percent of the initial stock price of $\$ 20$, or $\$ 8$ per share. Subsequent changes in equity per share are equal to the share price change so that equity per share is equal to $\$ 8+(P-20)$ where $P$ is the current share price. The margin call takes place when equity drops below the 25 percent maintenance margin requirement. The price below which a margin call will take place is the solution to the following equation:

$$
\frac{\text { Equity/share }}{\text { Price } / \text { share }}=\frac{\$ 8+P-20}{P}=25 \%
$$

which occurs at $P=16$. When the price drops below $\$ 16$, the equity will be under $\$ 4 /$ share, which is less than 25 percent of the price.

Traders who sell securities short are also subject to margin requirements because they have borrowed securities. Initially, the trader's equity supporting the short position must be at least equal to the margin requirement times the initial value of the short position. If prices rise, equity will be lost. At some point, the short seller will have to contribute additional equity to meet the maintenance margin requirement. Otherwise, the broker will buy the security back to cover the short position to prevent further losses and thereby secure repayment of the stock loan.

\section{2}
\section{ORDERS AND EXECUTION INSTRUCTIONS}
compare execution, validity, and clearing instructions compare market orders with limit orders

Buyers and sellers communicate with the brokers, exchanges, and dealers that arrange their trades by issuing orders. All orders specify what instrument to trade, how much to trade, and whether to buy or sell. Most orders also have other instructions attached to them. These additional instructions may include execution instructions, validity instructions, and clearing instructions. Execution instructions indicate how to fill the order, validity instructions indicate when the order may be filled, and clearing instructions indicate how to arrange the final settlement of the trade. In this section, we introduce various order instructions and explain how traders use them to achieve their objectives. We discuss execution mechanisms-how exchanges, brokers and dealers fill orders-in the next section. To understand the concepts in this section, however, you need to know a little about order execution mechanisms.

In most markets, dealers and various other proprietary traders often are willing to buy from, or sell to, other traders seeking to sell or buy. The prices at which they are willing to buy are called bid prices and those at which they are willing to sell are called ask prices, or sometimes offer prices. The ask prices are invariably higher than the bid prices.

The traders who are willing to trade at various prices may also indicate the quantities that they will trade at those prices. These quantities are called bid sizes and ask sizes depending on whether they are attached to bids or offers.

Practitioners say that the traders who offer to trade make a market. Those who trade with them take the market.

The highest bid in the market is the best bid, and the lowest ask in the market is the best offer. The difference between the best bid and the best offer is the market bid-ask spread. When traders ask, "What's the market?" they want to know the best bid and ask prices and their associated sizes. Bid-ask spreads are an implicit cost of trading. Markets with small bid-ask spreads are markets in which the costs of trading are small, at least for the sizes quoted. Dealers often quote both bid and ask prices, and in that case, practitioners say that they quote a two-sided market. The market spread is never more than any dealer spread.

\section{Execution Instructions}
Market and limit orders convey the most common execution instructions. A market order instructs the broker or exchange to obtain the best price immediately available when filling the order. A limit order conveys almost the same instruction: Obtain the best price immediately available, but in no event accept a price higher than a specified limit price when buying or accept a price lower than a specified limit price when selling.

Many people mistakenly believe that limit orders specify the prices at which the orders will trade. Although limit orders do often trade at their limit prices, remember that the first instruction is to obtain the best price available. If better prices are available than the limit price, brokers and exchanges should obtain those prices for their clients.

Market orders generally execute immediately if other traders are willing to take the other side of the trade. The main drawback with market orders is that they can be expensive to execute, especially when the order is placed in a market for a thinly traded security, or more generally, when the order is large relative to the normal trading activity in the market. In that case, a market buy order may fill at a high price, or a market sell order may fill at a low price if no traders are willing to trade at better prices. High purchase prices and low sale prices represent price concessions given to other traders to encourage them to take the other side of the trade. Because the sizes of price concessions can be difficult to predict, and because prices often change between when a trader submits an order and when the order finally fills, the execution prices for market orders are often uncertain.

Buyers and sellers who are concerned about the possibility of trading at unacceptable prices add limit price instructions to their orders. The main problem with limit orders is that they may not execute. Limit orders do not execute if the limit price on a buy order is too low, or if the limit price on a sell order is too high. For example, if an investment manager submits a limit order to buy at the limit price of 20 (buy limit 20) and nobody is willing to sell at or below 20 , the order will not trade. If prices never drop to 20 , the manager will never buy. If the price subsequently rises, the manager will have lost the opportunity to profit from the price rise. Whether traders use market orders or limit orders when trying to arrange trades depends on their concerns about price, trading quickly, and failing to trade. On average, limit orders trade at better prices than do market orders, but they often do not trade. Traders generally regret when their limit orders fail to trade because they usually would have profited if they had traded. Limit buy orders do not fill when prices are rising, and limit sell orders do not fill when prices are falling. In both cases, traders would be better off if their orders had filled.

The probability that a limit order will execute depends on where the order is placed relative to market prices. An aggressively priced order is more likely to trade than is a less aggressively priced order. A limit buy order is aggressively priced when the limit price is high relative to the market bid and ask prices. If the limit price is placed above the best offer, the buy order generally will partially or completely fill at the best offer price, depending on the size available at the best offer. Such limit orders are called marketable limit orders because at least part of the order can trade immediately. A limit buy order with a very high price relative to the market is essentially a market order.

If the buy order is placed above the best bid but below the best offer, traders say the order makes a new market because it becomes the new best bid. Such orders generally will not immediately trade, but they may attract sellers who are interested in trading. A buy order placed at the best bid is said to make market. It may have to wait until all other buy orders at that price trade first. Finally, a buy order placed below the best bid is behind the market. It will not execute unless market prices drop. Traders call limit orders that are waiting to trade standing limit orders.

Sell limit orders are aggressively priced if the limit price is low relative to market prices. The limit price of a marketable sell limit order is below the best bid. A limit sell order placed between the best bid and the best offer makes a new market on the sell side, one placed at the best offer makes market, and one placed above the best offer is behind the market.

Exhibit 2 presents a simplified limit order book in which orders are presented ranked by their limit prices for a hypothetical market. The market is "26 bid, offered at 28 " because the best bid is 26 and the best offer (ask) is 28 .

\section{Exhibit 2: Terms Traders Use to Describe Standing Limit Orders}
\section{Order Prices}
\begin{center}
\begin{tabular}{l|l|l}
\multicolumn{1}{c}{Oidsers (Asks)} &  \\
\hline
$\begin{array}{l}\text { The best bid and } \\ \text { best offer make } \\ \text { the market. }\end{array}$ & $\begin{array}{l}\text { The least aggressively priced sell orders are far } \\ \text { from the market. }\end{array}$ \\
These sell orders are behind the market. We also say &  \\
that they are away from the market. &  \\
\end{tabular}
\end{center}

Source: Trading and Exchanges. ${ }^{1}$

\section{EXAMPLE 21}
\section{Making and Taking}
\begin{enumerate}
  \item What is the difference between making a market and taking a market?
\end{enumerate}

\section{Solution:}
A trader makes a market when the trader offers to trade. A trader takes a market when the trader accepts an offer to trade.

\begin{enumerate}
  \setcounter{enumi}{1}
  \item What order types are most likely associated with making a market and taking a market?
\end{enumerate}

\section{Solution:}
Traders place standing limit orders to give other traders opportunities to trade. Standing limit orders thus make markets. In contrast, traders use market orders or marketable limit orders to take offers to trade. These marketable orders take the market.

1 Harris, Larry. 2003. Trading and Exchanges: Market Microstructure for Practitioners. New York: Oxford University Press. A trade-off exists between how aggressively priced an order is and the ultimate trade price. Although aggressively priced orders fill faster and with more certainty then do less aggressively priced limit orders, the prices at which they execute are inferior. Buyers seeking to trade quickly must pay higher prices to increase the probability of trading quickly. Similarly, sellers seeking to trade quickly must accept lower prices to increase the probability of trading quickly.

Some order execution instructions specify conditions on size. For example, all-or-nothing (AON) orders can only trade if their entire sizes can be traded. Traders can similarly specify minimum fill sizes. This specification is common when settlement costs depend on the number of trades made to fill an order and not on the aggregate size of the order.

Exposure instructions indicate whether, how, and perhaps to whom orders should be exposed. Hidden orders are exposed only to the brokers or exchanges that receive them. These agencies cannot disclose hidden orders to other traders until they can fill them. Traders use hidden orders when they are afraid that other traders might behave strategically if they knew that a large order was in the market. Traders can discover hidden size only by submitting orders that will trade with that size. Thus, traders can only learn about hidden size after they have committed to trading with it.

Traders also often indicate a specific display size for their orders. Brokers and exchanges then expose only the display size for these orders. Any additional size is hidden from the public but can be filled if a suitably large order arrives. Traders sometimes call such orders iceberg orders because most of the order is hidden. Traders specify display sizes when they do not want to display their full sizes, but still want other traders to know that someone is willing to trade at the displayed price. Traders on the opposite side who wish to trade additional size at that price can discover the hidden size only if they trade the displayed size, at which point the broker or exchange will display any remaining size up to the display size. They also can discover the hidden size by submitting large orders that will trade with that size.

\section{EXAMPLE 22}
\section{Market versus Limit and Hidden versus Displayed Orders}
You are the buy-side trader for a very clever investment manager. The manager has hired a commercial satellite firm to take regular pictures of the parking lots in which new car dealers store their inventories. It has also hired some part-time workers to count the cars on the lots. With this information and some econometric analyses, the manager can predict weekly new car sale announcements more accurately than can most analysts. The manager typically makes a quarter percent each week on this strategy. Once a week, a day before the announcements are made, the manager gives you large orders to buy or sell car manufacturers based on his insights into their dealers' sales. What primary issues should you consider when deciding whether to:

\begin{enumerate}
  \item use market or limit orders to fill his orders?
\end{enumerate}

\section{Solution:}
The manager's information is quite perishable. If his orders are not filled before the weekly sales are reported to the public, the manager will lose the opportunity to profit from the information as prices immediately adjust to the news. The manager, therefore, needs to get the orders filled quickly. This consideration suggests that the orders should be submitted as market or- ders. If submitted as limit orders, the orders might not execute and the firm would lose the opportunity to profit.

Large market orders, however, can be very expensive to execute, especially if few people are willing to trade significant size on the other side of the market. Because transaction costs can easily exceed the expected quarter percent return, you should submit limit orders to limit the execution prices that you are willing to accept. It is better to fail to trade than to trade at losing prices.

\begin{enumerate}
  \setcounter{enumi}{1}
  \item display the orders or hide them?
\end{enumerate}

\section{Solution:}
Your large orders could easily move the market if many people were aware of them, and even more so if others were aware that you are trading on behalf of a successful information-motivated trader. You thus should consider submitting hidden orders. The disadvantage of hidden orders is that they do not let people know that they can trade the other side if they want to.

\section{VALIDITY INSTRUCTIONS AND CLEARING INSTRUCTIONS}
compare execution, validity, and clearing instructions

Validity instructions indicate when an order may be filled. The most common validity instruction is the day order. A day order is good for the day on which it is submitted. If it has not been filled by the close of business, the order expires unfilled.

Good-till-cancelled orders (GTC) are just that. In practice, most brokers limit how long they will manage an order to ensure that they do not fill orders that their clients have forgotten. Such brokers may limit their GTC orders to a few months.

Immediate or cancel orders (IOC) are good only upon receipt by the broker or exchange. If they cannot be filled in part or in whole, they cancel immediately. In some markets these orders are also known as fill or kill orders. When searching for hidden liquidity, electronic algorithmic trading systems often submit thousands of these IOC orders for every order that they fill.

Good-on-close orders can only be filled at the close of trading. These orders often are market orders, so traders call them market-on-close orders. Traders often use on-close orders when they want to trade at the same prices that will be published as the closing prices of the day. Mutual funds often like to trade at such prices because they value their portfolios at closing prices. Many traders also use good-on-open orders.

\section{Stop Orders}
A stop order is an order in which a trader has specified a stop price condition. The stop order may not be filled until the stop price condition has been satisfied. For a sell order, the stop price condition suspends execution of the order until a trade occurs at or below the stop price. After that trade, the stop condition is satisfied and the order becomes valid for execution, subject to all other execution instructions attached to it. If the market price subsequently rises above the sell order's stop price before the order trades, the order remains valid. Similarly, a buy order with a stop condition becomes valid only after a price rises above the specified stop price.

Traders often call stop orders stop-loss orders because many traders use them with the hope of stopping losses on positions that they have established. For example, a trader who has bought stock at 40 may want to sell the stock if the price falls below 30. In that case, the trader might submit a "GTC, stop 30, market sell" order. If the price falls to or below 30 , the market order becomes valid and it should immediately execute at the best price then available in the market. That price may be substantially lower than 30 if the market is falling quickly. The stop-loss order thus does not guarantee a stop to losses at the stop price. If potential sellers are worried about trading at too low of a price, they can attach stop instructions to limit orders instead of market orders. In this example, if the trader is unwilling to sell below 25 , the trader would submit a "GTC, stop 30, limit 25 sell" order.

If a trader wants to guarantee that he can sell at 30 , the trader would buy a put option contract struck at 30 . The purchase price of the option would include a premium for the insurance that the trader is buying. Option contracts can be viewed as limit orders for which execution is guaranteed at the strike price. A trader similarly might use a stop-buy order or a call option to limit losses on a short position.

A portfolio manager might use a stop-buy order when the manager believes that a security is undervalued but is unwilling to trade without market confirmation. For example, suppose that a stock currently trades for $50 \mathrm{RMB}$ and a manager believes that it should be worth 100 RMB. Further, the manager believes that the stock will much more likely be worth 100 RMB if other traders are willing to buy it above 65 RMB. To best take advantage of this information, the manager would consider issuing a "GTC, stop 65 RMB, limit 100 RMB buy" order. Note that if the manager relies too much on the market when making this trading decision, however, he may violate CFA Standard of Professional Conduct V.A.2, which requires that all investment actions have a reasonable and adequate basis supported by appropriate research and investigation.

Because stop-sell orders become valid when prices are falling and stop-buy orders become valid when prices are rising, traders using stop orders contribute to market momentum as their sell orders push prices down further and their buy orders push prices up. Execution prices for stop orders thus are often quite poor.

\section{EXAMPLE 23}
\section{Limit and Stop Instructions}
\begin{enumerate}
  \item In what ways do limit and stop instructions differ?
\end{enumerate}

\section{Solution:}
Although both limit and stop instructions specify prices, the role that these prices play in the arrangement of a trade are completely different. A limit price places a limit on what trade prices will be acceptable to the trader. A buyer will accept prices only at or lower than the limit price whereas a seller will accept prices only at or above the limit price.

In contrast, a stop price indicates when an order can be filled. A buy order can only be filled once the market has traded at a price at or above the stop price. A sell order can only be filled once the market has traded at a price at or below the stop price.

Both order instructions may delay or prevent the execution of an order. A buy limit order will not execute until someone is willing to sell at or below the limit price. Similarly, a sell limit order will not execute until someone is willing to buy at or above the limit sell price. In contrast, a stop-buy order will not execute if the market price never rises to the stop price. Similarly, a stop-sell order will not execute if the market price never falls to the stop price.

\section{Clearing Instructions}
Clearing instructions tell brokers and exchanges how to arrange final settlement of trades. Traders generally do not attach these instructions to each order-instead they provide them as standing instructions. These instructions indicate what entity is responsible for clearing and settling the trade. For retail trades, that entity is the customer's broker. For institutional trades, that entity may be a custodian or another broker. When a client uses one broker to arrange trades and another broker to settle trades, traders say that the first broker gives up the trade to the other broker, who is often known as the prime broker. Institutional traders provide these instructions so they can obtain specialized execution services from different brokers while maintaining a single account for custodial services and other prime brokerage services, such as margin loans.

An important clearing instruction that must appear on security sale orders is an indication of whether the sale is a long sale or a short sale. In either case, the broker representing the sell order must ensure that the trader can deliver securities for settlement. For a long sale, the broker must confirm that the securities held are available for delivery. For a short sale, the broker must either borrow the security on behalf of the client or confirm that the client can borrow the security.

\section{PRIMARY SECURITY MARKETS}
$$
\mid \begin{aligned}
& \text { define primary and secondary markets and explain how secondary } \\
& \text { markets support primary markets }
\end{aligned}
$$

When issuers first sell their securities to investors, practitioners say that the trades take place in the primary markets. An issuer makes an initial public offering (IPO)\_sometimes called a placing-of a security issue when it sells the security to the public for the first time. A seasoned security is a security that an issuer has already issued. If the issuer wants to sell additional units of a previously issued security, it makes a seasoned offering (sometimes called a secondary offering). Both types of offerings occur in the primary market where issuers sell their securities to investors. Later, if investors trade these securities among themselves, they trade in secondary markets. This section discusses primary markets and the procedures that issuers use to offer their securities to the public.

\section{Public Offerings}
Corporations generally contract with an investment bank to help them sell their securities to the public. The investment bank then lines up subscribers who will buy the security. Investment bankers call this process book building. In London, the book builder is called the book runner. The bank tries to build a book of orders to which they can sell the offering. Investment banks often support their book building by providing investment information and opinion about the issuer to their clients and to the public. Before the offering, the issuer generally makes a very detailed disclosure of its business, of the risks inherent in it, and of the uses to which the new funds will be placed.

When time is of the essence, issuers in Europe may issue securities through an accelerated book build, in which the investment bank arranges the offering in only one or two days. Such sales often occur at discounted prices.

The first public offering of common stock in a company consists of newly issued shares to be sold by the company. It may also include shares that the founders and other early investors in the company seek to sell. The initial public offering provides these investors with a means of liquidating their investments.

In an underwritten offering - the most common type of offering-the investment bank guarantees the sale of the issue at an offering price that it negotiates with the issuer. If the issue is undersubscribed, the bank will buy whatever securities it cannot sell at the offering price. In the case of an IPO, the underwriter usually also promises to make a market in the security for about a month to ensure that the secondary market will be liquid and to provide price support, if necessary. For large issues, a syndicate of investment banks and broker-dealers helps the lead underwriter build the book. The issuer usually pays an underwriting fee of about 7 percent for these various services. The underwriting fee is a placement cost of the offering.

In a best effort offering, the investment bank acts only as broker. If the offering is undersubscribed, the issuer will not sell as much as it hoped to sell.

For both types of offerings, the issuer and the bank usually jointly set the offering price following a negotiation. If they set a price that buyers consider too high, the offering will be undersubscribed, and they will fail to sell the entire issue. If they set the price too low, the offering will be oversubscribed, in which case the securities are often allocated to preferred clients or on a pro-rata basis.

(Note that CFA Standard of Professional Conduct III.B-fair dealing-requires that the allocation be based on a written policy disclosed to clients and suggests that the securities be offered on a pro-rata basis among all clients who have comparable relationships with their broker-dealers.)

Investment banks have a conflict of interest with respect to the offering price in underwritten offerings. As agents for the issuers, they generally are supposed to select the offering price that will raise the most money. But as underwriters, they have strong incentives to choose a low price. If the price is low, the banks can allocate valuable shares to benefit their clients and thereby indirectly benefit the banks. If the price is too high, the underwriters will have to buy overvalued shares in the offering and perhaps also during the following month if they must support the price in the secondary market, which directly costs the banks. These considerations tend to lower initial offering prices so that prices in the secondary market often rise immediately following an IPO. They are less important in a seasoned offering because trading in the secondary market helps identify the proper price for the offering.

First time issuers generally accept lower offering prices because they and many others believe that an undersubscribed IPO conveys very unfavorable information to the market about the company's prospects at a time when it is most vulnerable to public opinion about its prospects. They fear that an undersubscribed initial public offering will make it substantially harder to raise additional capital in subsequent seasoned offerings.

\section{EXAMPLE 24}
\section{The Healthybots Initial Public Offering}
Healthybots is a health care company that treats diseases using artificial intelligence-based solutions. Healthybots raised approximately $\pounds 265$ million gross through an initial public offering of $103,142,466$ ordinary shares at $\pounds 2.57$ per ordinary share. After the initial public offering, Healthybots had 213,333,333 ordinary shares issued and outstanding.

Healthybots received gross proceeds of approximately $\pounds 34.3$ million and net proceeds of $\pounds 31.8$ million. The ordinary shares that were sold to the public represented approximately 48 percent of Healthybots' total issued ordinary shares.

The shares commenced trading at 8:00 a.m. on the AIM market of the London Stock Exchange, where Healthybots opened at $\pounds 2.74$, traded 37 million shares between $\pounds 2.68$ and $\pounds 2.74$, and closed at $\pounds 2.73$.

\begin{enumerate}
  \item Approximately how many new shares were issued by the company and how many shares were sold by the company's founders? What fraction of their holdings in the company did the founders sell?
\end{enumerate}

\section{Solution:}
Healthybots received gross proceeds of $\pounds 34.3$ million at $\pounds 2.57$ per share, so the company issued and sold 13,346,304 shares $(=\pounds 34.3$ million/£2.57 per share). The total placement was for 103,142,466 shares, so the founders sold $89,796,162$ shares (= 103,142,466 shares $-13,346,304$ shares). Because approximately 200 million shares (= 213.3 million shares -13.3 million shares) were outstanding before the placement, the founders sold approximately 45 percent (= 90 million shares/200 million shares) of the company.

\begin{enumerate}
  \setcounter{enumi}{1}
  \item Approximately what return did the subscribers who participated in the IPO make on the first day it traded?
\end{enumerate}

\section{Solution:}
The subscribers bought the stock for $\pounds 2.57$ per share, and it closed at $\pounds 2.73$. The first day return thus was $6.2 \%=\frac{2.73-2.57}{2.57} \times 100$.

\begin{enumerate}
  \setcounter{enumi}{2}
  \item Approximately how much did Healthybots pay in placement costs as a percentage of the new funds raised?
\end{enumerate}

\section{Solution:}
Healthybots obtained gross proceeds of $\pounds 34.3$ million but only raised net proceeds of $€ 31.8$ million. The $\pounds 2.5$ million difference was the total cost of the placement to the firm, which is 7.9 percent of net proceeds, or new funds raised ( $\pounds 2.5$ million/ $\pounds 31.8$ million).

\section{Private Placements and Other Primary Market Transactions}
Corporations sometimes issue their securities in private placements. In a private placement, corporations sell securities directly to a small group of qualified investors, usually with the assistance of an investment bank. Qualified investors have sufficient knowledge and experience to recognize the risks that they assume, and sufficient wealth to assume those risks responsibly. Most countries allow corporations to do private placements without nearly as much public disclosure as is required for public offerings. Private placements, therefore, may be cheaper than public offerings, but the buyers generally require higher returns (lower purchase prices) because they cannot subsequently trade the securities in an organized secondary market.

Corporations sometimes sell new issues of seasoned securities directly to the public on a piecemeal basis via a shelf registration. In a shelf registration, the corporation makes all public disclosures that it would for a regular offering, but it does not sell the shares in a single transaction. Instead, it sells the shares directly into the secondary market over time, generally when it needs additional capital. Shelf registrations provide corporations with flexibility in the timing of their capital transactions, and they can alleviate the downward price pressures often associated with large secondary offerings.

Many corporations may also issue shares via dividend reinvestment plans (DRPs or DRIPs, for short) that allow their shareholders to reinvest their dividends in newly issued shares of the corporation (in particular, DRPs specify that the corporation issue new shares for the plan rather than purchase them on the open market). These plans sometimes also allow existing shareholders and other investors to buy additional stock at a slight discount to current prices.

Finally, corporations can issue new stock via a rights offering. In a rights offering, the corporation distributes rights to buy stock at a fixed price to existing shareholders in proportion to their holdings. Because the rights need not be exercised, they are options. The exercise price, however, is set below the current market price of the stock so that buying stock with the rights is immediately profitable. Consequently, shareholders will experience dilution in the value of their existing shares. They can offset the dilution loss by exercising their rights or by selling the rights to others who will exercise them. Shareholders generally do not like rights offerings because they must provide additional capital (or sell their rights) to avoid losses through dilution. Financial analysts recognize that these securities, although called rights, are actually short-term stock warrants and value them accordingly.

The national governments of financially strong countries generally issue their bonds, notes, and bills in public auctions organized by a government agency (usually associated with the finance ministry). They may also sell them directly to dealers.

Smaller and less financially secure national governments and most regional governments often contract with investment banks to help them sell and distribute their securities. The laws of many governments, however, require that they auction their securities.

\section{EXAMPLE 25}
\section{Private and Public Placements}
\begin{enumerate}
  \item In what ways do private placements differ from public placements?
\end{enumerate}

\section{Solution:}
Issuers make private placements to a limited number of investors that generally are financially sophisticated and well informed about risk. The investors generally have some relationship to the issuer. Issuers make public placements when they sell securities to the general public. Public placements generally require substantially more financial disclosure than do private placements.

\section{Importance of Secondary Markets to Primary Markets}
Corporations and governments can raise money in the primary markets at lower cost when their securities will trade in liquid secondary markets. In a liquid market, traders can buy or sell with low transaction costs and small price concessions when they want to trade. Buyers value liquidity because they may need to sell their securities to meet liquidity needs. Investors thus will pay more for securities that they can easily sell than for those that they cannot easily sell. Higher prices translate into lower costs of capital for the issuers.

\section{SECONDARY SECURITY MARKET AND CONTRACT MARKET STRUCTURES}
\begin{center}
\includegraphics[max width=\textwidth]{2023_05_04_7b535d0a870224f62e3dg-191}
\end{center}

Trading is the successful outcome to a bilateral search in which buyers look for sellers and sellers look for buyers. Many market structures have developed to reduce the costs of this search. Markets are liquid when the costs of finding a suitable counterparty to a trade are low.

Trading in securities and contracts takes place in a variety of market structures. The structures differ by when trades can be arranged, who arranges the trades, how they do so, and how traders learn about possible trading opportunities and executed trades. This section introduces the various market structures used to trade securities and contracts. We first consider trading sessions, then execution mechanisms, and finally market information systems.

\section{Trading Sessions}
Markets are organized as call markets or as continuous trading markets. In a call market, trades can be arranged only when the market is called at a particular time and place. In contrast in a continuous trading market, trades can be arranged and executed anytime the market is open.

Buyers can easily find sellers and vice versa in call markets because all traders interested in trading (or orders representing their interests) are present at the same time and place. Call markets thus have the potential to be very liquid when they are called. But they are completely illiquid between trading sessions. In contrast, traders can arrange and execute their trades at any time in continuous trading markets, but doing so can be difficult if the buyers and sellers (or their orders) are not both present at the same time.

Most call markets use single price auctions to match buyers to sellers. In these auctions, the market constructs order books representing all buy orders and all seller orders. The market then chooses a single trade price that will maximize the total volume of trade. The order books are supply and demand schedules, and the point at which they cross determines the trade price.

Call markets usually are organized just once a day, but some markets organize calls at more frequent intervals. Many continuous trading markets start their trading with a call market auction. During a pre-opening period, traders submit their orders for the market call. At the opening, any possible trades are arranged and then trading continues in the continuous trading session. Some continuous trading markets also close their trading with a call. In these markets, traders who are only interested in trading in the closing call submit market- or limit-on-close orders.

\section{EXAMPLE 26}
\section{Call Markets and Continuous Trading Markets}
\begin{enumerate}
  \item What is the main advantage of a call market compared with a continuous trading market?
\end{enumerate}

\section{Solution:}
By gathering all traders to the same place at the same time, a call market makes it easier for buyers to find sellers and vice versa. In contrast, if buyers and sellers (or their orders) are not present at the same time in a continuous market, they cannot trade.

\begin{enumerate}
  \setcounter{enumi}{1}
  \item What is the main advantage of a continuous trading market compared with a call market?
\end{enumerate}

\section{Solution:}
In a continuous trading market, a willing buyer and seller can trade at any time the market is open. In contrast, in a call market trading can take place only when the market is called.

\section{Execution Mechanisms}
The three main types of market structures are quote-driven markets (sometimes called price-driven or dealer markets), order-driven markets, and brokered markets. In quote-driven markets, customers trade with dealers. In order-driven markets, an order matching system run by an exchange, a broker, or an alternative trading system uses rules to arrange trades based on the orders that traders submit. Most exchanges and ECNs organize order-driven markets. In brokered markets, brokers arrange trades between their customers. Brokered markets are common for transactions of unique instruments, such as real estate properties, intellectual properties, or large blocks of securities. Many trading systems use more than one type of market structure.

\section{Quote-Driven Markets}
Worldwide, most trading, other than in stocks, takes place in quote-driven markets. Almost all bonds and currencies and most spot commodities trade in quote-driven markets. Traders call them quote-driven (or price-driven or dealer) because customers trade at the prices quoted by dealers. Depending on the instrument traded, the dealers work for commercial banks, for investment banks, for broker-dealers, or for proprietary trading houses.

Quote-driven markets also often are called over-the-counter (OTC) markets because securities used to be literally traded over the dealer's counter in the dealer's office. Now, most trades in OTC markets are conducted over proprietary computer communications networks, by telephone, or sometimes over instant messaging systems.

\section{Order-Driven Markets}
Order-driven markets arrange trades using rules to match buy orders to sell orders. The orders may be submitted by customers or by dealers. Almost all exchanges use order-driven trading systems, and every automated trading system is an order-driven system.

Because rules match buyers to sellers, traders often trade with complete strangers. Order-driven markets thus must have procedures to ensure that buyers and sellers perform on their trade contracts. Otherwise, dishonest traders would enter contracts that they would not settle if a change in market conditions made settlement unprofitable.

Two sets of rules characterize order-driven market mechanisms: Order matching rules and trade pricing rules. The order matching rules match buy orders to sell orders. The trade pricing rules determine the prices at which the matched trades take place.

\section{Order Matching Rules}
Order-driven trading systems match buyers to sellers using rules that rank the buy orders and the sell orders based on price, and often along with other secondary criteria. The systems then match the highest-ranking buy order with the highest-ranking sell order. If the buyer is willing to pay at least as much as the seller is willing to receive, the system will arrange a trade for the minimum of the buy and sell quantities. The remaining size, if any, is then matched with the next order on the other side and the process continues until no further trades can be arranged.

The order precedence hierarchy determines which orders go first. The first rule is price priority: The highest priced buy orders and the lowest priced sell orders go first. They are the most aggressively priced orders. Secondary precedence rules determine how to rank orders at the same price. Most trading systems use time precedence to rank orders at the same price. The first order to arrive has precedence over other orders. In trading systems that permit hidden and partially hidden orders, displayed quantities at a given price generally have precedence over the undisplayed quantities. So the complete precedence hierarchy is given by price priority, display precedence at a given price, and finally time precedence among all orders with the same display status at a given price. These rules give traders incentives to improve price, display their orders, and arrive early if they want to trade quickly. These incentives increase market liquidity.

\section{Trade Pricing Rules}
After the orders are matched, the trading system then uses its trade pricing rule to determine the trade price. The three rules that various order-driven markets use to price their trades are the uniform pricing rule, the discriminatory pricing rule, and the derivative pricing rule.

Call markets commonly use the uniform pricing rule. Under this rule, all trades execute at the same price. The market chooses the price that maximizes the total quantity traded.

Continuous trading markets use the discriminatory pricing rule. Under this rule, the limit price of the order or quote that first arrived-the standing order-determines the trade price. This rule allows a large arriving trader to discriminate among standing limit orders by filling the most aggressively priced orders first at their limit prices and then filling less aggressively priced orders at their less favorable (from the point of view of the arriving trader) limit prices. If trading systems did not use this pricing rule, large traders would break their orders into pieces to price discriminate on their own.

\section{EXAMPLE 27}
\section{Filling a Large Order in a Continuous Trading Market}
\begin{enumerate}
  \item Before the arrival of a large order, the Tokyo Stock Exchange has the following limit orders standing on its book:
\end{enumerate}

\begin{center}
\begin{tabular}{lcccc}
\hline
Buyer & Bid Size & Limit Price(¥) & Offer Size & Seller \\
\hline
Takumi & 15 & 100.1 &  &  \\
Hiroto & 8 & 100.2 &  &  \\
Shou & 10 & 100.3 & 4 & Hina \\
 & 100.4 & 6 & Sakura &  \\
 & 100.5 & 12 & Miku &  \\
\hline
\end{tabular}
\end{center}

Tsubasa submits a day order to buy 15 contracts, limit $¥ 100.5$. With whom does he trade, what is his average trade price, and what does the limit order book look like afterward?

\section{Solution:}
Tsubasa's buy order first fills with the most aggressively priced sell order, which is Hina’s order for four contracts. A trade takes place at $¥ 100.4$ for four contracts, Hina's order fills completely, and Tsubasa still has 11 more contracts remaining.

The next most aggressively priced sell order is Sakura's order for six contracts. A second trade takes place at $¥ 100.5$ for six contracts, Sakura’s order fills completely, and Tsubasa still has five more contracts remaining. The next most aggressively priced sell order is Miku’s order at $¥ 100.6$. No further trade is possible, however, because her limit sell price is above Tsubasa’s limit buy price. Tsubasa’s average trade price is $¥ 100.46$ $=\frac{4 \times ¥ 100.4+6 \times ¥ 100.5}{4+6}$.

Because Tsubasa issued a day order, the remainder of his order is placed on the book on the buy side at $¥ 100.5$. The following orders are then on the book:

\begin{center}
\begin{tabular}{lcccc}
\hline
Buyer & Bid Size & Limit Price (¥) & Offer Size & Seller \\
\hline
Takumi & 15 & 100.1 &  &  \\
Hiroto & 8 & 100.2 &  &  \\
Shou & 10 & 100.3 &  &  \\
 &  & 100.4 & 12 & Miku \\
Tsubasa & 100.5 & 100.6 &  &  \\
\end{tabular}
\end{center}

If Tsubasa had issued an immediate-or-cancel order, the remaining five contracts would have been cancelled.

Crossing networks use the derivative pricing rule. Crossing networks are trading systems that match buyers and sellers who are willing to trade at prices obtained from other markets. Most systems cross their trades at the midpoint of the best bid and ask quotes published by the exchange at which the security primarily trades. This pricing rule is called a derivative pricing rule because the price is derived from another market. In particular, the price does not depend on the orders submitted to the crossing network. Some crossing networks are organized as call markets and others as continuously trading markets. The most important crossing market is the equity trading system POSIT.

\section{Brokered Markets}
The third execution mechanism is the brokered market, in which brokers arrange trades among their clients. Brokers organize markets for instruments for which finding a buyer or a seller willing to trade is difficult because the instruments are unique and thus of interest only to a limited number of people or institutions. These instruments generally are also infrequently traded and expensive to carry in inventory. Examples of such instruments include very large blocks of stock, real estate properties, fine art masterpieces, intellectual properties, operating companies, liquor licenses, and taxi medallions. Because dealers generally are unable or unwilling to hold these assets in their inventories, they will not make markets in them. Organizing order-driven markets for these instruments is not sensible because too few traders would submit orders to them.

Successful brokers in these markets try to know everyone who might now or in the future be willing to trade. They spend most of their time on the telephone and in meetings building their networks.

\section{EXAMPLE 28}
\section{Quote-Driven, Order-Driven, and Brokered Markets}
\begin{enumerate}
  \item What are the primary advantages of quote-driven, order-driven, and brokered markets?
\end{enumerate}

\section{Solution:}
In a quote-driven market, dealers generally are available to supply liquidity. In an order-driven market, traders can supply liquidity to each other. In a brokered market, brokers help find traders who are willing to trade when dealers would not be willing to make markets and when traders would not be willing to post orders.

\section{Market Information Systems}
Markets vary in the type and quantity of data that they disseminate to the public. Traders say that a market is pre-trade transparent if the market publishes real-time data about quotes and orders. Markets are post-trade transparent if the market publishes trade prices and sizes soon after trades occur.

Buy-side traders value transparency because it allows them to better manage their trading, understand market values, and estimate their prospective and actual transaction costs. In contrast, dealers prefer to trade in opaque markets because, as frequent traders, they have an information advantage over those who know less than they do. Bid-ask spreads tend to be wider and transaction costs tend to be higher in opaque markets because finding the best available price is harder for traders in such markets.

\section{6}
\section{WELL-FUNCTIONING FINANCIAL SYSTEMS}
describe characteristics of a well-functioning financial system

The financial system allows traders to solve financing and risk management problems. In a well-functioning financial system:

\begin{itemize}
  \item investors can easily move money from the present to the future while obtaining a fair rate of return for the risks that they bear;

  \item borrowers can easily obtain funds that they need to undertake current projects if they can credibly promise to repay the funds in the future;

  \item hedgers can easily trade away or offset the risks that concern them; and

  \item traders can easily trade currencies for other currencies or commodities that they need.

\end{itemize}

If the assets or contracts needed to solve these problems are available to trade, the financial system has complete markets. If the costs of arranging these trades are low, the financial system is operationally efficient. If the prices of the assets and contracts reflect all available information related to fundamental values, the financial system is informationally efficient.

Well-functioning financial systems are characterized by:

\begin{itemize}
  \item the existence of well-developed markets that trade instruments that help people solve their financial problems (complete markets);

  \item liquid markets in which the costs of trading-commissions, bid-ask spreads, and order price impacts-are low (operationally efficient markets);

  \item timely financial disclosures by corporations and governments that allow market participants to estimate the fundamental values of securities (support informationally efficient markets); and

  \item prices that reflect fundamental values so that prices vary primarily in response to changes in fundamental values and not to demands for liquidity made by uninformed traders (informationally efficient markets).

\end{itemize}

Such complete and operationally efficient markets are produced by financial intermediaries who:

\begin{itemize}
  \item organize exchanges, brokerages, and alternative trading systems that match buyers to sellers;

  \item provide liquidity on demand to traders;

  \item securitize assets to produce investment instruments that are attractive to investors and thereby lower the costs of funds for borrowers;

  \item run banks that match investors to borrowers by taking deposits and making loans;

  \item run insurance companies that pool uncorrelated risks;

  \item provide investment advisory services that help investors manage and grow their assets at low cost;

  \item organize clearinghouses that ensure everyone settles their trades and contracts; and

  \item organize depositories that ensure nobody loses their assets. The benefits of a well-functioning financial system are huge. In such systems, investors who need to move money to the future can easily connect with entrepreneurs who need money now to develop new products and services. Similarly, producers who would otherwise avoid valuable projects because they are too risky can easily transfer those risks to others who can better bear them. Most importantly, these transactions generally can take place among strangers so that the benefits from trading can be derived from an enormous number of potential matches.

\end{itemize}

In contrast, economies that have poorly functioning financial systems have great difficulties allocating capital among the many companies who could use it. Financial transactions tend to be limited to arrangements within families when people cannot easily find trustworthy counterparties who will honor their contracts. In such economies, capital is allocated inefficiently, risks are not easily shared, and production is inefficient.

An extraordinarily important byproduct of an operationally efficient financial system is the production of informationally efficient prices. Prices are informationally efficient when they reflect all available information about fundamental values. Informative prices are crucially important to the welfare of an economy because they help ensure that resources go where they are most valuable. Economies that use resources where they are most valuable are allocationally efficient. Economies that do not use resources where they are most valuable waste their resources and consequently often are quite poor.

Well-informed traders make prices informationally efficient. When they buy assets and contracts that they think are undervalued, they tend to push the assets' prices up. Similarly, when they sell assets and contracts that they think are overvalued, they tend to push the assets' prices down. The effect of their trading thus causes prices to reflect their information about values.

How accurately prices reflect fundamental information depends on the costs of obtaining fundamental information and on the liquidity available to well-informed traders. Accounting standards and reporting requirements that produce meaningful and timely financial disclosures reduce the costs of obtaining fundamental information and thereby allow analysts to form more accurate estimates of fundamental values. Liquid markets allow well-informed traders to fill their orders at low cost. If filling orders is very costly, informed trading may not be profitable. In that case, information-motivated traders will not commit resources to collect and analyze data and they will not trade. Without their research and their associated trading, prices would be less informative.

\section{EXAMPLE 29}
\section{Well-Functioning Financial Systems}
\begin{enumerate}
  \item As a financial analyst specializing in emerging market equities, you understand that a well-functioning financial system contributes to the economic prosperity of a country. You are asked to start covering a new small market country. What factors will you consider when characterizing the quality of its financial markets?
\end{enumerate}

\section{Solution:}
In general, you will consider whether:

\begin{itemize}
  \item the country has markets that allow its companies and residents to finance projects, save for the future, and exchange risk;

  \item the costs of trading in those markets is low; and - prices reflect fundamental values.

\end{itemize}

You may specifically check to see whether:

\begin{itemize}
  \item fixed income and stock markets allow borrowers to easily obtain capital from investors;

  \item corporations disclose financial and operating data on a timely basis in conformity to widely respected reporting standards, such as IFRS;

  \item forward, futures, and options markets trade instruments that companies need to hedge their risks;

  \item dealers and arbitrageurs allow traders to trade when they want to;

  \item bid-ask spreads are small;

  \item trades and contracts invariably settle as expected;

  \item investment managers provide high-quality management services for reasonable fees;

  \item banks and other financing companies are well capitalized and thus able to help investors provide capital to borrowers;

  \item securitized assets are available and represent reasonable credit risks;

  \item insurance companies are well capitalized and thus able to help those exposed to risks insure against them; and

  \item price volatility appears consistent with changes in fundamental values.

\end{itemize}

\section{MARKET REGULATION}
describe objectives of market regulation

Government agencies and practitioner organizations regulate many markets and the financial intermediaries that participate in them. The regulators generally seek to promote fair and orderly markets in which traders can trade at prices that accurately reflect fundamental values without incurring excessive transaction costs. This section identifies the problems that financial regulators hope to solve and the objectives of their regulations.

Regrettably, some people will steal from each other if given a chance, especially if the probability of detection is low or if the penalty for being caught is low. The number of ways that people can steal or misappropriate wealth generally increases with the complexity of their relationships and with asymmetries in their knowledge. Because financial markets tend to be complex, and because customers are often much less sophisticated than the professionals that serve them, the potential for losses through various frauds can be unacceptably high in unregulated markets.

Regulators thus ensure that systems are in place to protect customers from fraud. In principle, the customers themselves would demand such systems as a condition of doing business. When customers are unsophisticated or poorly informed, however, they may not know how to protect themselves. When the costs of learning are large-as they often are in complex financial markets-having regulators look out for the public interest can be economically efficient.

More customer money is probably lost in financial markets through negligence than through outright fraud. Most customers in financial markets use various agents to help them solve problems that they do not understand well. These agents include securities brokers, financial advisers, investment managers, and insurance agents. Because customers generally do not have much information about market conditions, they find it extremely difficult to measure the added value they obtain from their agents. This problem is especially challenging when performance has a strong random component. In that case, determining whether agents are skilled or lucky is very difficult. Moreover, if the agent is a good salesman, the customer may not critically evaluate their agent's performance. These conditions, which characterize most financial markets, ensure that customers cannot easily determine whether their agents are working faithfully for them. They tend to lose if their agents are unqualified or lazy, or if they unconsciously favor themselves and their friends over their clients, as is natural for even the most honest people.

Regulators help solve these agency problems by setting minimum standards of competence for agents and by defining and enforcing minimum standards of practice. CFA Institute provides significant standard setting leadership in the areas of investment management and investment performance reporting through its Chartered Financial Analyst Program, in which you are studying, and its Global Investment Performance Standards. In principle, regulation would not be necessary if customers could identify competent agents and effectively measure their performance. In the financial markets, doing so is very difficult.

Regulators often act to level the playing field for market participants. For example, in many jurisdictions, insider trading in securities is illegal. The rule prevents corporate insiders and others with access to corporate information from trading on material information that has not been released to the public. The purpose of the rule is to reduce the profits that insiders could extract from the markets. These profits would come from other traders who would lose when they trade with well-informed insiders. Because traders tend to withdraw from markets when they lose, rules against insider trading help keep markets liquid. They also keep corporate insiders from hoarding information.

Many situations arise in financial markets in which common standards benefit everyone involved. For example, having all companies report financial results on a common basis allows financial analysts to easily compare companies. Accordingly, the International Accounting Standards Board (IASB) and the US-based Financial Accounting Standards Board (FASB), among many others, promulgate common financial standards to which all companies must report. The benefits of having common reporting standards has led to a very successful and continuing effort to converge all accounting standards to a single worldwide standard. Without such regulations, investors might eventually refuse to invest in companies that do not report to a common standard, but such market-based discipline is a very slow regulator of behavior, and it would have little effect on companies that do not need to raise new capital.

Regulators generally require that financial firms maintain minimum levels of capital. These capital requirements serve two purposes. First, they ensure that the companies will be able to honor their contractual commitments when unexpected market movements or poor decisions cause them to lose money. Second, they ensure that the owners of financial firms have substantial interest in the decisions that they make. Without a substantial financial interest in the decisions that they make, companies often take too many risks and exercise poor judgment about extending credit to others. When such companies fail, they impose significant costs on others. Minimum capital requirements reduce the probability that financial firms will fail and they reduce the disruptions associated with those failures that do occur. In principle, a firm's customers and counterparties could require minimum capital levels as a condition of doing business with the firm, but they have more difficulty enforcing their contracts than do governments who can imprison people. Regulators similarly regulate insurance companies and pension funds that make long-term promises to their clients. Such entities need to maintain adequate reserves to ensure that they can fund their liabilities. Unfortunately, their managers have a tendency to underestimate these reserves if they will not be around when the liabilities come due. Again, in principle, policyholders and employees could regulate the behavior of their insurance funds and their employers by refusing to contract with them if they do not promise to adequately fund their liabilities. In practice, however, the sophistication, information, and time necessary to write and enforce contracts that control these problems are beyond the reach of most people. The government thus is a sensible regulator of such problems.

Many regulators are self-regulating organizations (SROs) that regulate their members. Exchanges, clearinghouses, and dealer trade organizations are examples of self-regulating organizations. In some cases, the members of these organizations voluntarily subject themselves to the SRO's regulations to promote the common good. In other cases, governments delegate regulatory and enforcement authorities to SROs, usually subject to the supervision of a government agency, such as a national securities and exchange authority. Exchanges, dealer associations, and clearing agencies often regulate their members with these delegated powers.

By setting high standards of behavior, SROs help their members obtain the confidence of their customers. They also reduce the chance that members of the SRO will incur losses when dealing with other members of the SRO.

When regulators fail to solve the problems discussed here, the financial system does not function well. People who lose money stop saving and borrowers with good ideas cannot fund their projects. Similarly, hedgers withdraw from markets when the costs of hedging are high. Without the ability to hedge, producers become reluctant to specialize because specialization generally increases risk. Because specialization also decreases costs, however, production becomes less efficient as producers chose safer technologies. Economies that cannot solve the regulatory problems described in this section tend to operate less efficiently than do better regulated economies, and they tend to be less wealthy.

To summarize, the objectives of market regulation are to:

\begin{enumerate}
  \item control fraud;

  \item control agency problems;

  \item promote fairness;

  \item set mutually beneficial standards;

  \item prevent undercapitalized financial firms from exploiting their investors by making excessively risky investments; and

  \item ensure that long-term liabilities are funded.

\end{enumerate}

Regulation is necessary because regulating certain behaviors through market-based mechanisms is too costly for people who are unsophisticated and uninformed. Effectively regulated markets allow people to better achieve their financial goals.

\section{EXAMPLE 30}
\section{Bankrupt Traders}
You are the chief executive officer of a brokerage that is a member of a clearinghouse. A trader who clears through your firm is bankrupt at midday, but you do not yet know it even though your clearing agreement with him explicitly requires that he immediately report significant losses. The trader knows that if he takes a large position, prices might move in his favor so that he will no longer be bankrupt. The trader attempts to do so and succeeds. You find out about this later in the evening.

\begin{enumerate}
  \item Why does the clearinghouse regulate its members?
\end{enumerate}

\section{Solution:}
The clearinghouse regulates its members to ensure that no member imposes costs on another member by failing to settle a trade.

\begin{enumerate}
  \setcounter{enumi}{1}
  \item What should you do about the trader?
\end{enumerate}

\section{Solution:}
You should immediately end your clearing relationship with the trader and confiscate his trading profits. The trader was trading with your firm's capital after he became bankrupt. Had he lost, your firm would have borne the loss.

\begin{enumerate}
  \setcounter{enumi}{2}
  \item Why would the clearinghouse allow you to keep his trading profits?
\end{enumerate}

\section{Solution:}
If the clearinghouse did not permit you to keep his trading profits, other traders similarly situated might attempt the same strategy.

\section{SUMMARY}
This reading introduces how the financial system operates and explains how well-functioning financial systems lead to wealthy economies. Financial analysts need to understand how the financial system works because their analyses often lead to trading decisions.

The financial system consists of markets and the financial intermediaries that operate in them. These institutions allow buyers to connect with sellers. They may trade directly with each other when they trade the same instrument or they only may trade indirectly when a financial intermediary connects the buyer to the seller through transactions with each that appear on the intermediary's balance sheet. The buyer and seller may exchange instruments, cash flows, or risks.

The following points, among others, were made in this reading:

\begin{itemize}
  \item The financial system consists of mechanisms that allow strangers to contract with each other to move money through time, to hedge risks, and to exchange assets that they value less for those that they value more.

  \item Investors move money from the present to the future when they save. They expect a normal rate of return for bearing risk through time. Borrowers move money from the future to the present to fund current projects and expenditures. Hedgers trade to reduce their exposure to risks they prefer not to take. Information-motivated traders are active investment managers who try to identify under- and overvalued instruments.

  \item Securities are first sold in primary markets by their issuers. They then trade in secondary markets. - People invest in pooled investment vehicles to benefit from the investment management services of their managers.

  \item Forward contracts allow buyers and sellers to arrange for future sales at predetermined prices. Futures contracts are forward contracts guaranteed by clearinghouses. The guarantee ensures that strangers are willing to trade with each other and that traders can offset their positions by trading with anybody. These features of futures contract markets make them highly attractive to hedgers and information-motivated traders.

  \item Many financial intermediaries connect buyers to sellers in a given instrument, acting directly as brokers and exchanges or indirectly as dealers and arbitrageurs.

  \item Financial intermediaries create instruments when they conduct arbitrage, securitize assets, borrow to lend, manage investment funds, or pool insurance contracts. These activities all transform cash flows and risks from one form to another. Their services allow buyers and sellers to connect with each other through instruments that meet their specific needs.

  \item Financial markets work best when strangers can contract with each other without worrying about whether their counterparts are able and willing to honor their contract. Clearinghouses, variation margins, maintenance margins, and settlement guarantees made by creditworthy brokers on behalf of their clients help manage credit risk and ultimately allow strangers to contract with each other.

  \item Information-motivated traders short sell when they expect that prices will fall. Hedgers short sell to reduce the risks of a long position in a related contract or commodity.

  \item Margin loans allow people to buy more securities than their equity would otherwise permit them to buy. The larger positions expose them to more risk so that gains and losses for a given amount of equity will be larger. The leverage ratio is the value of a position divided by the value of the equity supporting it. The returns to the equity in a position are equal to the leverage ratio times the returns to the unleveraged position.

  \item To protect against credit losses, brokers demand maintenance margin payments from their customers who have borrowed cash or securities when adverse price changes cause their customer's equity to drop below the maintenance margin ratio. Brokers close positions for customers who do not satisfy these margin calls.

  \item Orders are instructions to trade. They always specify instrument, side (buy or sell), and quantity. They usually also provide several other instructions.

  \item Market orders tend to fill quickly but often at inferior prices. Limit orders generally fill at better prices if they fill, but they may not fill. Traders choose order submission strategies on the basis of how quickly they want to trade, the prices they are willing to accept, and the consequences of failing to trade.

  \item Stop instructions are attached to other orders to delay efforts to fill them until the stop condition is satisfied. Although stop orders are often used to stop losses, they are not always effective.

  \item Issuers sell their securities using underwritten public offerings, best efforts public offerings, private placements, shelf registrations, dividend reinvestment programs, and rights offerings. Investment banks have a conflict of interests when setting the initial offering price in an IPO. - Well-functioning secondary markets are essential to raising capital in the primary markets because investors value the ability to sell their securities if they no longer want to hold them or if they need to disinvest to raise cash. If they cannot trade their securities in a liquid market, they will not pay as much for them.

  \item Matching buyers and sellers in call markets is easy because the traders (or their orders) come together at the same time and place.

  \item Dealers provide liquidity in quote-driven markets. Public traders as well as dealers provide liquidity in order-driven markets.

  \item Order-driven markets arrange trades by ranking orders using precedence rules. The rules generally ensure that traders who provide the best prices, display the most size, and arrive early trade first. Continuous order-driven markets price orders using the discriminatory pricing rule. Under this rule, standing limit orders determine trade prices.

  \item Brokers help people trade unique instruments or positions for which finding a buyer or a seller is difficult.

  \item Transaction costs are lower in transparent markets than in opaque markets because traders can more easily determine market value and more easily manage their trading in transparent markets.

  \item A well-functioning financial system allows people to trade instruments that best solve their wealth and risk management problems with low transaction costs. Complete and liquid markets characterize a well-functioning financial system. Complete markets are markets in which the instruments needed to solve investment and risk management problems are available to trade. Liquid markets are markets in which traders can trade when they want to trade at low cost.

  \item The financial system is operationally efficient when its markets are liquid. Liquid markets lower the costs of raising capital.

  \item A well-functioning financial system promotes wealth by ensuring that capital allocation decisions are well made. A well-functioning financial system also promotes wealth by allowing people to share the risks associated with valuable products that would otherwise not be undertaken.

  \item Prices are informationally efficient when they reflect all available information about fundamental values. Information-motivated traders make prices informationally efficient. Prices will be most informative in liquid markets because information-motivated traders will not invest in information and research if establishing positions based on their analyses is too costly.

  \item Regulators generally seek to promote fair and orderly markets in which traders can trade at prices that accurately reflect fundamental values without incurring excessive transaction costs. Governmental agencies and self-regulating organizations of practitioners provide regulatory services that attempt to make markets safer and more efficient.

  \item Mandated financial disclosure programs for the issuers of publicly traded securities ensure that information necessary to estimate security values is available to financial analysts on a consistent basis.

\end{itemize}

\section{PRACTICE PROBLEMS}
\begin{enumerate}
  \item Akihiko Takabe has designed a sophisticated forecasting model, which predicts the movements in the overall stock market, in the hope of earning a return in excess of a fair return for the risk involved. He uses the predictions of the model to decide whether to buy, hold, or sell the shares of an index fund that aims to replicate the movements of the stock market. Takabe would best be characterized as $\mathrm{a}(\mathrm{n})$ :
A. hedger.
B. investor.
C. information-motivated trader.

  \item James Beach is young and has substantial wealth. A significant proportion of his stock portfolio consists of emerging market stocks that offer relatively high expected returns at the cost of relatively high risk. Beach believes that investment in emerging market stocks is appropriate for him given his ability and willingness to take risk. Which of the following labels most appropriately describes Beach?

\end{enumerate}

A. Hedger.

B. Investor.

C. Information-motivated trader.

\begin{enumerate}
  \setcounter{enumi}{2}
  \item Lisa Smith owns a manufacturing company in the United States. Her company has sold goods to a customer in Brazil and will be paid in Brazilian real (BRL) in three months. Smith is concerned about the possibility of the BRL depreciating more than expected against the US dollar (USD). Therefore, she is planning to sell three-month futures contracts on the BRL. The seller of such contracts generally gains when the BRL depreciates against the USD. If Smith were to sell these future contracts, she would most appropriately be described as a(n):
A. hedger.
B. investor.
C. information-motivated trader.

  \item Which of the following is not a function of the financial system?
A. To regulate arbitrageurs' profits (excess returns).
B. To help the economy achieve allocational efficiency.
C. To facilitate borrowing by businesses to fund current operations.

  \item An investor primarily invests in stocks of publicly traded companies. The investor wants to increase the diversification of his portfolio. A friend has recommended investing in real estate properties. The purchase of real estate would best be characterized as a transaction in the:

\end{enumerate}

A. derivative investment market.

B. traditional investment market. C. alternative investment market.

\begin{enumerate}
  \setcounter{enumi}{5}
  \item A hedge fund holds its excess cash in 90-day commercial paper and negotiable certificates of deposit. The cash management policy of the hedge fund is best described as using:
A. capital market instruments.
B. money market instruments.
C. intermediate-term debt instruments.

  \item An oil and gas exploration and production company announces that it is offering 30 million shares to the public at $\$ 45.50$ each. This transaction is most likely a sale in the:

\end{enumerate}

A. futures market.

B. primary market.

C. secondary market.

\begin{enumerate}
  \setcounter{enumi}{7}
  \item Consider a mutual fund that invests primarily in fixed-income securities that have been determined to be appropriate given the fund's investment goal. Which of the following is least likely to be a part of this fund?
A. Warrants.
B. Commercial paper.
C. Repurchase agreements.

  \item A friend has asked you to explain the differences between open-end and closed-end funds. Which of the following will you most likely include in your explanation?

\end{enumerate}

A. Closed-end funds are unavailable to new investors.

B. When investors sell the shares of an open-end fund, they can receive a discount or a premium to the fund's net asset value.

C. When selling shares, investors in an open-end fund sell the shares back to the fund whereas investors in a closed-end fund sell the shares to others in the secondary market.

\begin{enumerate}
  \setcounter{enumi}{9}
  \item The Standard \& Poor's Depositary Receipts (SPDRs) is an investment that tracks the S\&P 500 stock market index. Purchases and sales of SPDRs during an average trading day are best described as:
\end{enumerate}

A. primary market transactions in a pooled investment.

B. secondary market transactions in a pooled investment.

C. secondary market transactions in an actively managed investment.

\begin{enumerate}
  \setcounter{enumi}{10}
  \item Which of the following statements about exchange-traded funds is most correct?
\end{enumerate}

A. Exchange-traded funds are not backed by any assets.

B. The investment companies that create exchange-traded funds are financial intermediaries. C. The transaction costs of trading shares of exchange-traded funds are substantially greater than the combined costs of trading the underlying assets of the fund.

\begin{enumerate}
  \setcounter{enumi}{11}
  \item The usefulness of a forward contract is limited by some problems. Which of the following is most likely one of those problems?
\end{enumerate}

A. Once you have entered into a forward contract, it is difficult to exit from the contract.

B. Entering into a forward contract requires the long party to deposit an initial amount with the short party.

C. If the price of the underlying asset moves adversely from the perspective of the long party, periodic payments must be made to the short party.

\begin{enumerate}
  \setcounter{enumi}{12}
  \item Tony Harris is planning to start trading in commodities. He has heard about the use of futures contracts on commodities and is learning more about them. Which of the following is Harris least likely to find associated with a futures contract?
\end{enumerate}

A. Existence of counterparty risk.

B. Standardized contractual terms.

C. Payment of an initial margin to enter into a contract.

\begin{enumerate}
  \setcounter{enumi}{13}
  \item A German company that exports machinery is expecting to receive $\$ 10$ million in three months. The firm converts all its foreign currency receipts into euros. The chief financial officer of the company wishes to lock in a minimum fixed rate for converting the $\$ 10$ million to euro but also wants to keep the flexibility to use the future spot rate if it is favorable. What hedging transaction is most likely to achieve this objective?
\end{enumerate}

A. Selling dollars forward.

B. Buying put options on the dollar.

C. Selling futures contracts on dollars.

\begin{enumerate}
  \setcounter{enumi}{14}
  \item A book publisher requires substantial quantities of paper. The publisher and a paper producer have entered into an agreement for the publisher to buy and the producer to supply a given quantity of paper four months later at a price agreed upon today. This agreement is a:
\end{enumerate}

A. futures contract.

B. forward contract.

C. commodity swap.

\begin{enumerate}
  \setcounter{enumi}{15}
  \item The Standard \& Poor's Depositary Receipts (SPDRs) is an exchange-traded fund in the United States that is designed to track the S\&P 500 stock market index. The latest price of a share of SPDRs is $\$ 290$. A trader has just bought call options on shares of SPDRs for a premium of $\$ 3$ per share. The call options expire in six months and have an exercise price of $\$ 305$ per share. On the expiration date, the trader will exercise the call options (ignore any transaction costs) if and only if the shares of SPDRs are trading:
\end{enumerate}

A. below $\$ 305$ per share. B. above $\$ 305$ per share.

C. above $\$ 308$ per share.

\begin{enumerate}
  \setcounter{enumi}{16}
  \item Jason Schmidt works for a hedge fund and he specializes in finding profit opportunities that are the result of inefficiencies in the market for convertible bondsbonds that can be converted into a predetermined amount of a company's common stock. Schmidt tries to find convertibles that are priced inefficiently relative to the underlying stock. The trading strategy involves the simultaneous purchase of the convertible bond and the short sale of the underlying common stock. The above process could best be described as:
A. hedging.
B. arbitrage.
C. securitization.

  \item Pierre-Louis Robert just purchased a call option on shares of the Michelin Group. A few days ago he wrote a put option on Michelin shares. The call and put options have the same exercise price, expiration date, and number of shares underlying. Considering both positions, Robert's exposure to the risk of the stock of the Michelin Group is:
A. long.
B. short.
C. neutral.

  \item An online brokerage firm has set the minimum margin requirement at 55 percent. What is the maximum leverage ratio associated with a position financed by this minimum margin requirement?
A. 1.55 .
B. 1.82
C. 2.22

  \item A trader has purchased 200 shares of a non-dividend-paying firm on margin at a price of $\$ 50$ per share. The leverage ratio is 2.5 . Six months later, the trader sells these shares at $\$ 60$ per share. Ignoring the interest paid on the borrowed amount and the transaction costs, what was the return to the trader during the six-month period?
A. 20 percent.
B. 33.33 percent.
C. 50 percent.

  \item Jason Williams purchased 500 shares of a company at $\$ 32$ per share. The stock was bought on 75 percent margin. One month later, Williams had to pay interest on the amount borrowed at a rate of 2 percent per month. At that time, Williams received a dividend of $\$ 0.50$ per share. Immediately after that he sold the shares at $\$ 28$ per share. He paid commissions of $\$ 10$ on the purchase and $\$ 10$ on the sale of the stock. What was the rate of return on this investment for the one-month period?
A. -12.5 percent.
B. -15.4 percent.
C. -50.1 percent.

  \item Caroline Rogers believes the price of Gamma Corp. stock will go down in the near future. She has decided to sell short 200 shares of Gamma Corp. at the current market price of $€ 47$. The initial margin requirement is 40 percent. Which of the following is an appropriate statement regarding the margin requirement that Rogers is subject to on this short sale?

\end{enumerate}

A. She will need to contribute $€ 3,760$ as margin.

B. She will need to contribute $€ 5,640$ as margin.

C. She will only need to leave the proceeds from the short sale as deposit and does not need to contribute any additional funds.

\begin{enumerate}
  \setcounter{enumi}{22}
  \item The current price of a stock is $\$ 25$ per share. You have $\$ 10,000$ to invest. You borrow an additional $\$ 10,000$ from your broker and invest $\$ 20,000$ in the stock. If the maintenance margin is 30 percent, at what price will a margin call first occur?
A. $\$ 9.62$.
B. $\$ 17.86$.
C. $\$ 19.71$.

  \item A market has the following limit orders standing on its book for a particular stock. The bid and ask sizes are number of shares in hundreds.

\end{enumerate}

\begin{center}
\begin{tabular}{ccc}
\hline
Bid Size & Limit Price (€) & Offer Size \\
\hline
5 & 9.73 &  \\
12 & 9.81 &  \\
4 & 9.84 &  \\
6 & 9.95 &  \\
 & 10.02 & 5 \\
 & 10.10 & 12 \\
 & 10.14 & 8 \\
\hline
\end{tabular}
\end{center}

What is the market?
A. 9.73 bid, offered at 10.14 .
B. 9.81 bid, offered at 10.10 .
C. 9.95 bid, offered at 10.02 .

\begin{enumerate}
  \setcounter{enumi}{24}
  \item Consider the following limit order book for a stock. The bid and ask sizes are number of shares in hundreds.
\end{enumerate}

\begin{center}
\begin{tabular}{lcc}
\hline
Bid Size & Limit Price $(¥)$ & Offer Size \\
\hline
3 & 122.80 &  \\
\hline
\end{tabular}
\end{center}

\begin{center}
\begin{tabular}{lcc}
\hline
Bid Size & Limit Price ( $¥)$ & Offer Size \\
\hline
8 & 123.00 &  \\
4 & 123.35 &  \\
 & 123.80 & 7 \\
 & 124.10 & 6 \\
 & 124.50 & 7 \\
\hline
\end{tabular}
\end{center}

A new buy limit order is placed for 300 shares at $¥ 123.40$. This limit order is said to:
A. take the market.
B. make the market.
C. make a new market.

\begin{enumerate}
  \setcounter{enumi}{25}
  \item Currently, the market in a stock is "\$54.62 bid, offered at $\$ 54.71$." A new sell limit order is placed at $\$ 54.62$. This limit order is said to:
A. take the market.
B. make the market.
C. make a new market.

  \item You have placed a sell market-on-open order-a market order that would automatically be submitted at the market's open tomorrow and would fill at the market price. Your instruction, to sell the shares at the market open, is a(n):
A. execution instruction.
B. validity instruction.
C. clearing instruction.

  \item Jim White has sold short 100 shares of Super Stores at a price of $\$ 42$ per share. He has also simultaneously placed a "good-till-cancelled, stop 50, limit 55 buy" order. Assume that if the stop condition specified by White is satisfied and the order becomes valid, it will get executed. Excluding transaction costs, what is the maximum possible loss that White can have?
A. $\$ 800$.
B. $\$ 1,300$.
C. Unlimited.

  \item You own shares of a company that are currently trading at $\$ 30$ a share. Your technical analysis of the shares indicates a support level of $\$ 27.50$. That is, if the price of the shares is going down, it is more likely to stay above this level rather than fall below it. If the price does fall below this level, however, you believe that the price may continue to decline. You have no immediate intent to sell the shares but are concerned about the possibility of a huge loss if the share price declines below the support level. Which of the following types of orders could you place to most appropriately address your concern?

\end{enumerate}

A. Short sell order. B. Good-till-cancelled stop sell order.

C. Good-till-cancelled stop buy order.

\begin{enumerate}
  \setcounter{enumi}{29}
  \item In an underwritten offering, the risk that the entire issue may not be sold to the public at the stipulated offering price is borne by the:
\end{enumerate}

A. issuer.

B. investment bank.

C. buyers of the part of the issue that is sold.

\begin{enumerate}
  \setcounter{enumi}{30}
  \item A British company listed on AIM (formerly the Alternative Investment Market) of the London Stock Exchange announced the sale of 6,686,665 shares to a small group of qualified investors at $\pounds 0.025$ per share. Which of the following best describes this sale?
\end{enumerate}

A. Shelf registration.

B. Private placement.

C. Initial public offering.

\begin{enumerate}
  \setcounter{enumi}{31}
  \item A German publicly traded company, to raise new capital, gave its existing shareholders the opportunity to subscribe for new shares. The existing shareholders could purchase two new shares at a subscription price of $€ 4.58$ per share for every 15 shares held. This is an example of a(n):
\end{enumerate}

A. rights offering.

B. private placement.

C. initial public offering.

\begin{enumerate}
  \setcounter{enumi}{32}
  \item Consider an order-driven system that allows hidden orders. The following four sell orders on a particular stock are currently in the system's limit order book. Based on the commonly used order precedence hierarchy, which of these orders will have precedence over others?
\end{enumerate}

\begin{center}
\begin{tabular}{lccc}
\hline
Order & $\begin{array}{c}\text { Time of Arrival } \\ \text { (HH:MM:SS) }\end{array}$ & $\begin{array}{c}\text { Limit Price } \\ \text { ( } \boldsymbol{)}\end{array}$ & $\begin{array}{c}\text { Special Instruction } \\ \text { (If any) }\end{array}$ \\
\hline
I & $9: 52: 01$ & 20.33 &  \\
II & $9: 52: 08$ & 20.29 & Hidden order \\
III & $9: 53: 04$ & 20.29 &  \\
IV & $9: 53: 49$ & 20.29 &  \\
\hline
\end{tabular}
\end{center}

A. Order I (time of arrival of 9:52:01).

B. Order II (time of arrival of 9:52:08).

C. Order III (time of arrival of 9:53:04)

\begin{enumerate}
  \setcounter{enumi}{33}
  \item Zhenhu Li has submitted an immediate-or-cancel buy order for 500 shares of a company at a limit price of CNY 74.25. There are two sell limit orders standing in that stock's order book at that time. One is for 300 shares at a limit price of CNY 74.30 and the other is for 400 shares at a limit price of CNY 74.35. How many shares in Li's order would get cancelled?
\end{enumerate}

A. None (the order would remain open but unfilled).

B. 200 (300 shares would get filled).

C. 500 (there would be no fill).

\begin{enumerate}
  \setcounter{enumi}{34}
  \item A market has the following limit orders standing on its book for a particular stock:
\end{enumerate}

\begin{center}
\begin{tabular}{lcccl}
\hline
Buyer & $\begin{array}{c}\text { Bid Size } \\ \text { (Number of Shares) }\end{array}$ & Limit Price ( $\boldsymbol{)}$ ) & $\begin{array}{c}\text { Offer Size } \\ \text { (Number of Shares) }\end{array}$ & Seller \\
\hline
Keith & 1,000 & 19.70 &  &  \\
Paul & 200 & 19.84 &  &  \\
Ann & 400 & 19.89 &  &  \\
Mary & 300 & 20.02 &  &  \\
 &  & 20.03 & 800 & Jack \\
 &  & 20.11 & 1,100 & Margaret \\
\hline
\end{tabular}
\end{center}

Ian submits a day order to sell 1,000 shares, limit $\pounds 19.83$. Assuming that no more buy orders are submitted on that day after Ian submits his order, what would be Ian's average trade price?
A. $\pounds 19.70$.
B. $\pounds 19.92$.
C. $\pounds 20.05$.

\begin{enumerate}
  \setcounter{enumi}{35}
  \item A financial analyst is examining whether a country's financial market is well functioning. She finds that the transaction costs in this market are low and trading volumes are high. She concludes that the market is quite liquid. In such a market:
\end{enumerate}

A. traders will find it hard to make use of their information.

B. traders will find it easy to trade and their trading will make the market less informationally efficient.

C. traders will find it easy to trade and their trading will make the market more informationally efficient.

\begin{enumerate}
  \setcounter{enumi}{36}
  \item The government of a country whose financial markets are in an early stage of development has hired you as a consultant on financial market regulation. Your first task is to prepare a list of the objectives of market regulation. Which of the following is least likely to be included in this list of objectives?
\end{enumerate}

A. Minimize agency problems in the financial markets.

B. Ensure that financial markets are fair and orderly.

C. Ensure that investors in the stock market achieve a rate of return that is at least equal to the risk-free rate of return.

\section{SOLUTIONS}
\begin{enumerate}
  \item C is correct. Takabe is best characterized as an information-motivated trader. Takabe believes that his model provides him superior information about the movements in the stock market and his motive for trading is to profit from this information.

  \item B is correct. Beach is an investor. He is simply investing in risky assets consistent with his level of risk aversion. Beach is not hedging any existing risk or using information to identify and trade mispriced securities. Therefore, he is not a hedger or an information-motivated trader.

  \item A is correct. Smith is a hedger. The short position on the BRL futures contract offsets the BRL long position in three months. She is hedging the risk of the BRL depreciating against the USD. If the BRL depreciates, the value of the cash inflow goes down in USD terms but there is a gain on the futures contracts.

  \item A is correct. Regulation of arbitrageurs' profits is not a function of the financial system. The financial system facilitates the allocation of capital to the best uses and the purposes for which people use the financial system, including borrowing money.

  \item $\mathrm{C}$ is correct. The purchase of real estate properties is a transaction in the alternative investment market.

  \item B is correct. The 90-day commercial paper and negotiable certificates of deposit are money market instruments.

  \item B is correct. This transaction is a sale in the primary market. It is a sale of shares from the issuer to the investor and funds flow to the issuer of the security from the purchaser.

  \item A is correct. Warrants are least likely to be part of the fund. Warrant holders have the right to buy the issuer's common stock. Thus, warrants are typically classified as equity and are least likely to be a part of a fixed-income mutual fund. Commercial paper and repurchase agreements are short-term fixed-income securities.

  \item C is correct. When investors want to sell their shares, investors of an open-end fund sell the shares back to the fund whereas investors of a closed-end fund sell the shares to others in the secondary market. Closed-end funds are available to new investors but they must purchase shares in the fund in the secondary market. The shares of a closed-end fund trade at a premium or discount to net asset value.

  \item B is correct. SPDRs trade in the secondary market and are a pooled investment vehicle.

  \item B is correct. The investment companies that create exchange-traded funds (ETFs) are financial intermediaries. ETFs are securities that represent ownership in the assets held by the fund. The transaction costs of trading shares of ETFs are substantially lower than the combined costs of trading the underlying assets of the ETF.

  \item A is correct. Once you have entered into a forward contract, it is difficult to exit from the contract. As opposed to a futures contract, trading out of a forward contract is quite difficult. There is no exchange of cash at the origination of a forward contract. There is no exchange on a forward contract until the maturity of the contract.

  \item A is correct. Harris is least likely to find counterparty risk associated with a futures contract. There is limited counterparty risk in a futures contract because the clearinghouse is on the other side of every contract.

  \item B is correct. Buying a put option on the dollar will ensure a minimum exchange rate but does not have to be exercised if the exchange rate moves in a favorable direction. Forward and futures contracts would lock in a fixed rate but would not allow for the possibility to profit in case the value of the dollar three months later in the spot market turns out to be greater than the value in the forward or futures contract.

  \item B is correct. The agreement between the publisher and the paper supplier to respectively buy and supply paper in the future at a price agreed upon today is a forward contract.

  \item B is correct. The holder of the call option will exercise the call options if the price is above the exercise price of $\$ 305$ per share. Note that if the stock price is above $\$ 305$ but less than $\$ 308$, the option would be exercised even though the net result for the option buyer after considering the premium is a loss. For example, if the stock price is $\$ 307$, the option buyer would exercise the option to make $\$ 2=\$ 307$ - $\$ 305$ per share, resulting in a loss of $\$ 1=\$ 3$ - $\$ 2$ after considering the premium. It is better to exercise and have a loss of only $\$ 1$, however, rather than not exercise and lose the entire $\$ 3$ premium.

  \item B is correct. The process can best be described as arbitrage because it involves buying and selling instruments, whose values are closely related, at different prices in different markets.

  \item A is correct. Robert's exposure to the risk of the stock of the Michelin Group is long. The exposure as a result of the long call position is long. The exposure as a result of the short put position is also long. Therefore, the combined exposure is long.

  \item B is correct. The maximum leverage ratio is $1.82=100 \%$ position $\div 55 \%$ equity. The maximum leverage ratio associated with a position financed by the minimum margin requirement is one divided by the minimum margin requirement.

  \item $C$ is correct. The return is 50 percent. If the position had been unleveraged, the return would be $20 \%=(60-50) / 50$. Because of leverage, the return is $50 \%=2.5$ $\times 20 \%$

\end{enumerate}

Another way to look at this problem is that the equity contributed by the trader (the minimum margin requirement) is $40 \%=100 \% \div 2.5$. The trader contributed $\$ 20=40 \%$ of $\$ 50$ per share. The gain is $\$ 10$ per share, resulting in a return of $50 \%$ $=10 / 20$.

\begin{enumerate}
  \setcounter{enumi}{20}
  \item B is correct. The return is -15.4 percent.
\end{enumerate}

Total cost of the purchase $=\$ 16,000=500 \times \$ 32$

Equity invested $=\$ 12,000=0.75 \times \$ 16,000$

Amount borrowed $=\$ 4,000=16,000-12,000$

Interest paid at month end $=\$ 80=0.02 \times \$ 4,000$

Dividend received at month end $=\$ 250=500 \times \$ 0.50$

Proceeds on stock sale $=\$ 14,000=500 \times \$ 28$ Total commissions paid $=\$ 20=\$ 10+\$ 10$

Net gain/loss $=-\$ 1,850=-16,000-80+250+14,000-20$

Initial investment including commission on purchase $=\$ 12,010$

Return $=-15.4 \%=-\$ 1,850 / \$ 12,010$

\begin{enumerate}
  \setcounter{enumi}{21}
  \item A is correct. She will need to contribute $€ 3,760$ as margin. In view of the possibility of a loss, if the stock price goes up, she will need to contribute $€ 3,760=40 \%$ of $€ 9,400$ as the initial margin. Rogers will need to leave the proceeds from the short sale $(€ 9,400=200 \times € 47)$ on deposit.

  \item B is correct. A margin call will first occur at a price of $\$ 17.86$. Because you have contributed half and borrowed the remaining half, your initial equity is 50 percent of the initial stock price, or $\$ 12.50=0.50 \times \$ 25$. If $P$ is the subsequent price, your equity would change by an amount equal to the change in price. So, your equity at price $P$ would be $12.50+(P-25)$. A margin call will occur when the percentage margin drops to 30 percent. So, the price at which a margin call will occur is the solution to the following equation.

\end{enumerate}

$\frac{\text { Equity/Share }}{\text { Price/Share }}=\frac{12.50+P-25}{P}=30 \%$

The solution is $P=\$ 17.86$.

\begin{enumerate}
  \setcounter{enumi}{23}
  \item $\mathrm{C}$ is correct. The market is 9.95 bid, offered at 10.02. The best bid is at $€ 9.95$ and the best offer is $€ 10.02$.

  \item $\mathrm{C}$ is correct. This order is said to make a new market. The new buy order is at $¥ 123.40$, which is better than the current best bid of $¥ 123.35$. Therefore, the buy order is making a new market. Had the new order been at $¥ 123.35$, it would be said to make the market. Because the new buy limit order is at a price less than the best offer of $¥ 123.80$, it will not immediately execute and is not taking the market.

  \item A is correct. This order is said to take the market. The new sell order is at $\$ 54.62$, which is at the current best bid. Therefore, the new sell order will immediately trade with the current best bid and is taking the market.

  \item B is correct. An instruction regarding when to fill an order is considered a validity instruction.

  \item B is correct. The maximum possible loss is $\$ 1,300$. If the stock price crosses $\$ 50$, the stop buy order will become valid and will get executed at a maximum limit price of $\$ 55$. The maximum loss per share is $\$ 13=\$ 55-\$ 42$, or $\$ 1,300$ for 100 shares.

  \item B is correct. The most appropriate order is a good-till-cancelled stop sell order. This order will be acted on if the stock price declines below a specified price (in this case, \$27.50). This order is sometimes referred to as a good-till-cancelled stop loss sell order. You are generally bullish about the stock, as indicated by no immediate intent to sell, and would expect a loss on short selling the stock. A stop buy order is placed to buy a stock when the stock is going up.

  \item B is correct. The investment bank bears the risk that the issue may be undersubscribed at the offering price. If the entire issue is not sold, the investment bank underwriting the issue will buy the unsold securities at the offering price.

  \item B is correct. This sale is a private placement. As the company is already publicly traded, the share sale is clearly not an initial public offering. The sale also does not involve a shelf registration because the company is not selling shares to the public on a piecemeal basis.

  \item A is correct. This offering is a rights offering. The company is distributing rights to buy stock at a fixed price to existing shareholders in proportion to their holdings.

  \item $C$ is correct. Order III (time of arrival of 9:53:04) has precedence. In the order precedence hierarchy, the first rule is price priority. Based on this rule, sell orders II, III, and IV get precedence over order I. The next rule is display precedence at a given price. Because order II is a hidden order, orders III and IV get precedence. Finally, order III gets precedence over order IV based on time priority at same price and same display status.

  \item C is correct. The order for 500 shares would get cancelled; there would be no fill. Li is willing to buy at CNY 74.25 or less but the minimum offer price in the book is CNY 74.30; therefore, no part of the order would be filled. Because Li's order is immediate-or-cancel, it would be cancelled.

  \item B is correct. Ian's average trade price is:

\end{enumerate}

$\pounds 19.92=\frac{300 \times \pounds 20.02+400 \times \pounds 19.89+200 \times \pounds 19.84}{300+400+200}$

Ian's sell order first fills with the most aggressively priced buy order, which is Mary's order for 300 shares at $\pounds 20.02$. Ian still has 700 shares for sale. The next most aggressively priced buy order is Ann's order for 400 shares at $\pounds 19.89$. This order is filled. Ian still has 300 shares for sale. The next most aggressively priced buy order is Paul's order for 200 shares at $\pounds 19.84$. A third trade takes place. Ian still has 100 shares for sale.

The next buy order is Keith's order for 1,000 shares at $\pounds 19.70$. However, this price is below Ian's limit price of $€ 19.83$. Therefore, no more trade is possible.

\begin{enumerate}
  \setcounter{enumi}{35}
  \item $\mathrm{C}$ is correct. In such a market, well-informed traders will find it easy to trade and their trading will make the market more informationally efficient. In a liquid market, it is easier for informed traders to fill their orders. Their trading will cause prices to incorporate their information and the prices will be more in line with the fundamental values.

  \item $C$ is correct. Ensure that investors in the stock market achieve a rate of return that is at least equal to the risk-free rate of return is least likely to be included as an objective of market regulation. Stocks are risky investments and there would be occasions when a stock market investment would not only have a return less than the risk-free rate but also a negative return. Minimizing agency costs and ensuring that financial markets are fair and orderly are objectives of market regulation.

\end{enumerate}

\section*{LEARNING MODULE 2 }
\section{Security Market Indexes}
Paul D. Kaplan, PhD, CFA, is at Morningstar (Canada). Dorothy C. Kelly, CFA, is at McIntire School of Commerce, University of Virginia (USA).

\section{LEARNING OUTCOME}
\begin{center}
\begin{tabular}{|c|c|}
\hline
Mastery & The candidate should be able to: \\
\hline
$\square$ & describe a security market index \\
\hline
$\square$ & $\begin{array}{l}\text { calculate and interpret the value, price return, and total return of an } \\ \text { index }\end{array}$ \\
\hline
$\square$ & $\begin{array}{l}\text { describe the choices and issues in index construction and } \\ \text { management }\end{array}$ \\
\hline
$\square$ & compare the different weighting methods used in index construction \\
\hline
$\square$ & $\begin{array}{l}\text { calculate and analyze the value and return of an index given its } \\ \text { weighting method }\end{array}$ \\
\hline
$\square$ & describe rebalancing and reconstitution of an index \\
\hline
$\square$ & describe uses of security market indexes \\
\hline
$\square$ & describe types of equity indexes \\
\hline
\includegraphics[max width=\textwidth]{2023_05_04_7b535d0a870224f62e3dg-217}
 & compare types of security market indexes \\
\hline
\includegraphics[max width=\textwidth]{2023_05_04_7b535d0a870224f62e3dg-217(1)}
 & describe types of fixed-income indexes \\
\hline
. & describe indexes representing alternative investments \\
\hline
\end{tabular}
\end{center}

\section{INTRODUCTION}
Investors gather and analyze vast amounts of information about security markets on a continual basis. Because this work can be both time consuming and data intensive, investors often use a single measure that consolidates this information and reflects the performance of an entire security market. Security market indexes were first introduced as a simple measure to reflect the performance of the US stock market. Since then, security market indexes have evolved into important multi-purpose tools that help investors track the performance of various security markets, estimate risk, and evaluate the performance of investment managers. They also form the basis for new investment products.

\includegraphics[max width=\textwidth, center]{2023_05_04_7b535d0a870224f62e3dg-218}
indicator, sign, or measure of something.

\section{ORIGIN OF MARKET INDEXES}
Investors had access to regularly published data on individual security prices in London as early as 1698 , but nearly 200 years passed before they had access to a simple indicator to reflect security market information. To give readers a sense of how the US stock market in general performed on a given day, publishers Charles $\mathrm{H}$. Dow and Edward $\mathrm{D}$. Jones introduced the Dow Jones Average, the world's first security market index, in 1884. The index, which appeared in The Customers' Afternoon Letter, consisted of the stocks of nine railroads and two industrial companies. It eventually became the Dow Jones Transportation Average. Convinced that industrial companies, rather than railroads, would be "the great speculative market" of the future, Dow and Jones introduced a second index in May 1896-the Dow Jones Industrial Average (DJIA). It had an initial value of 40.94 and consisted of 12 stocks from major US industries. Today, investors can choose from among thousands of indexes to measure and monitor different security markets and asset classes.

This reading is organized as follows. Section 2 defines a security market index and explains how to calculate the price return and total return of an index for a single period and over multiple periods. Section 3 describes how indexes are constructed and managed. Section 4 discusses the use of market indexes. Sections 5, 6, and 7 discuss various types of indexes, and the final section summarizes the reading. Practice problems follow the conclusions and summary.

\section{INDEX DEFINITION AND CALCULATIONS OF VALUE AND RETURNS}
 describe a security market indexcalculate and interpret the value, price return, and total return of an
index

A security market index represents a given security market, market segment, or asset class. Most indexes are constructed as portfolios of marketable securities.

The value of an index is calculated on a regular basis using either the actual or estimated market prices of the individual securities, known as constituent securities, within the index. For each security market index, investors may encounter two versions of the same index (i.e., an index with identical constituent securities and weights): one version based on price return and one version based on total return. As the name suggests, a price return index, also known as a price index, reflects only the prices of the constituent securities within the index. A total return index, in contrast, reflects not only the prices of the constituent securities but also the reinvestment of all income received since inception.

At inception, the values of the price and total return versions of an index are equal. As time passes, however, the value of the total return index, which includes the reinvestment of all dividends and/or interest received, will exceed the value of the price return index by an increasing amount. A look at how the values of each version are calculated over multiple periods illustrates why.

The value of a price return index is calculated as:

$V_{P R I}=\frac{\sum_{i=1}^{N} n_{i} P_{i}}{D}$

where

$$
\begin{aligned}
V_{P R I} & =\text { the value of the price return index } \\
n_{i} & =\text { the number of units of constituent security } i \text { held in the index portfolio } \\
N & =\text { the number of constituent securities in the index } \\
P_{i} & =\text { the unit price of constituent security } i \\
D & =\text { the value of the divisor }
\end{aligned}
$$

The divisor is a number initially chosen at inception. It is frequently chosen so that the price index has a convenient initial value, such as 1,000. The index provider then adjusts the value of the divisor as necessary to avoid changes in the index value that are unrelated to changes in the prices of its constituent securities. For example, when changing index constituents, the index provider may adjust the divisor so that the value of the index with the new constituents equals the value of the index prior to the changes.

Index return calculations, like calculations of investment portfolio returns, may measure price return or total return. Price return measures only price appreciation or percentage change in price. Total return measures price appreciation plus interest, dividends, and other distributions.

\section{Calculation of Single-Period Returns}
For a security market index, price return can be calculated in two ways: either the percentage change in value of the price return index, or the weighted average of price returns of the constituent securities. The price return of an index can be expressed as:

$$
\mathrm{PR}_{I}=\frac{V_{P R I 1}-V_{P R I 0}}{V_{P R I 0}}
$$

where

$\mathrm{PR}_{I}=$ the price return of the index portfolio (as a decimal number, i.e., 12 percent is 0.12 )

$V_{P R I 1}=$ the value of the price return index at the end of the period

$V_{P R I O}=$ the value of the price return index at the beginning of the period

Similarly, the price return of each constituent security can be expressed as:

$$
\mathrm{PR}_{i}=\frac{P_{i 1}-P_{i 0}}{P_{i 0}}
$$

where

$$
\begin{aligned}
\mathrm{PR}_{i} & =\text { the price return of constituent security } i \text { (as a decimal number) } \\
P_{i 1} & =\text { the price of constituent security } i \text { at the end of the period } \\
P_{i 0} & =\text { the price of constituent security } i \text { at the beginning of the period }
\end{aligned}
$$

Because the price return of the index equals the weighted average of price returns of the individual securities, we can write:

$$
\mathrm{PR}_{I}=\sum_{i=1}^{N} \mathrm{w}_{i} \mathrm{PR}_{i}=\sum_{i=1}^{N} \mathrm{w}_{i}\left(\frac{P_{i 1}-P_{i 0}}{P_{i 0}}\right)
$$

where:

$\mathrm{PR}_{I}=$ the price return of index portfolio (as a decimal number)

$\mathrm{PR}_{i}=$ the price return of constituent security $i$ (as a decimal number)

$N=$ the number of individual securities in the index

$\mathrm{w}_{i}=$ the weight of security $i$ (the fraction of the index portfolio allocated to security $i$ )

$P_{i 1}=$ the price of constituent security $i$ at the end of the period

$P_{i 0}=$ the price of constituent security $i$ at the beginning of the period

Equation 4 can be rewritten simply as:

$$
\mathrm{PR}_{I}=\mathrm{w}_{1} \mathrm{PR}_{1}+\mathrm{w}_{2} \mathrm{PR}_{2}+\ldots+\mathrm{w}_{N} \mathrm{PR}_{N}
$$

where

$\mathrm{PR}_{I}=$ the price return of index portfolio (as a decimal number)

$\mathrm{PR}_{i}=$ the price return of constituent security $i$ (as a decimal number)

$\mathrm{w}_{i}=$ the weight of security $i$ (the fraction of the index portfolio allocated to security $i$ )

$N=$ the number of securities in the index

Total return measures price appreciation plus interest, dividends, and other distributions. Thus, the total return of an index is the price appreciation, or change in the value of the price return index, plus income (dividends and/or interest) over the period, expressed as a percentage of the beginning value of the price return index. The total return of an index can be expressed as:

$$
\mathrm{TR}_{I}=\frac{V_{P R I 1}-V_{P R I 0}+I n c_{I}}{V_{P R I 0}}
$$

where

$\mathrm{TR}_{I}=$ the total return of the index portfolio (as a decimal number)

$V_{P R I 1}=$ the value of the price return index at the end of the period

$V_{P R I O}=$ the value of the price return index at the beginning of the period

$\operatorname{Inc}_{I}=$ the total income (dividends and/or interest) from all securities in the index held over the period The total return of an index can also be calculated as the weighted average of total returns of the constituent securities. The total return of each constituent security in the index is calculated as:

$\mathrm{TR}_{i}=\frac{P_{1 i}-P_{0 i}+\operatorname{In} c_{i}}{P_{0 i}}$

where

$\mathrm{TR}_{i}=$ the total return of constituent security $i$ (as a decimal number)

$P_{1 i}=$ the price of constituent security $i$ at the end of the period

$P_{0 i}=$ the price of constituent security $i$ at the beginning of the period

Inc $_{i}=$ the total income (dividends and/or interest) from security $i$ over the period

Because the total return of an index can be calculated as the weighted average of total returns of the constituent securities, we can express total return as:

$\mathrm{TR}_{I}=\sum_{i=1}^{N} \mathrm{w}_{i} \mathrm{TR}_{i}=\sum_{i=1}^{N} \mathrm{w}_{i}\left(\frac{P_{1 i}-P_{0 i}+\operatorname{In} c_{i}}{P_{0 i}}\right)$

Equation 8 can be rewritten simply as

$\mathrm{TR}_{I}=\mathrm{w}_{1} \mathrm{TR}_{1}+\mathrm{w}_{2} \mathrm{TR}_{2}+\ldots+\mathrm{w}_{N} \mathrm{TR}_{N}$

where

$\mathrm{TR}_{I}=$ the total return of the index portfolio (as a decimal number)

$\mathrm{TR}_{i}=$ the total return of constituent security $i$ (as a decimal number)

$\mathrm{w}_{i}=$ the weight of security $i$ (the fraction of the index portfolio allocated to

security $i$ )

$N=$ the number of securities in the index

\section{Calculation of Index Values over Multiple Time Periods}
The calculation of index values over multiple time periods requires geometrically linking the series of index returns. With a series of price returns for an index, we can calculate the value of the price return index with the following equation:

$$
\mathrm{V}_{\text {PRIT }}=\mathrm{V}_{P R I 0}\left(1+\mathrm{PR}_{I 1}\right)\left(1+\mathrm{PR}_{I 2}\right) \ldots\left(1+\mathrm{PR}_{I T}\right)
$$

where

$\mathrm{V}_{P R I 0}=$ the value of the price return index at inception

$\mathrm{V}_{P R I T}=$ the value of the price return index at time $t$

$\mathrm{PR}_{I T}=$ the price return (as a decimal number) on the index over period $t, t=1$,

$2, \ldots, T$

For an index with an inception value set to 1,000 and price returns of 5 percent and 3 percent for Periods 1 and 2 respectively, the values of the price return index would be calculated as follows:

\begin{center}
\begin{tabular}{cccc}
\hline
Period & Return (\%) & Calculation & Ending Value \\
\hline
0 &  & $1,000(1.00)$ & $1,000.00$ \\
1 & 5.00 & $1,000(1.05)$ & $1,050.00$ \\
\end{tabular}
\end{center}

Security Market Indexes

\begin{center}
\begin{tabular}{cccc}
\hline
Period & Return (\%) & Calculation & Ending Value \\
\hline
2 & 3.00 & $1,000(1.05)(1.03)$ & $1,081.50$ \\
\hline
\end{tabular}
\end{center}

Similarly, the series of total returns for an index is used to calculate the value of the total return index with the following equation:

$$
\mathrm{V}_{T R I T}=\mathrm{V}_{T R I 0}\left(1+\mathrm{TR}_{I 1}\right)\left(1+\mathrm{TR}_{I 2}\right) \ldots\left(1+\mathrm{TR}_{I T}\right)
$$

where

$\mathrm{V}_{T R I 0}=$ the value of the index at inception

$\mathrm{V}_{\text {TRIT }}=$ the value of the total return index at time $t$

$\mathrm{TR}_{I T}=$ the total return (as a decimal number) on the index over period $t, t=1$, $2, \ldots, T$

Suppose that the same index yields an additional 1.5 percent return from income in Period 1 and an additional 2.0 percent return from income in Period 2, bringing the total returns for Periods 1 and 2, respectively, to 6.5 percent and 5 percent. The values of the total return index would be calculated as follows:

\begin{center}
\begin{tabular}{cccc}
\hline
Period & Return (\%) & Calculation & Ending Value \\
\hline
0 &  & $1,000(1.00)$ & $1,000.00$ \\
1 & 6.50 & $1,000(1.065)$ & $1,065.00$ \\
2 & 5.00 & $1,000(1.065)(1.05)$ & $1,118.25$ \\
\hline
\end{tabular}
\end{center}

As illustrated above, as time passes, the value of the total return index, which includes the reinvestment of all dividends and/or interest received, exceeds the value of the price return index by an increasing amount.

\section{INDEX CONSTRUCTION}
\begin{center}
\includegraphics[max width=\textwidth]{2023_05_04_7b535d0a870224f62e3dg-222}
\end{center}

Constructing and managing a security market index is similar to constructing and managing a portfolio of securities. Index providers must decide the following:

\begin{enumerate}
  \item Which target market should the index represent?

  \item Which securities should be selected from that target market?

  \item How much weight should be allocated to each security in the index?

  \item When should the index be rebalanced?

  \item When should the security selection and weighting decision be re-examined?

\end{enumerate}

\section{Target Market and Security Selection}
The first decision in index construction is identifying the target market, market segment, or asset class that the index is intended to represent. The target market may be defined very broadly or narrowly. It may be based on asset class (e.g., equities, fixed income, real estate, commodities, hedge funds); geographic region (e.g., Japan, South Africa, Latin America, Europe); the exchange on which the securities are traded (e.g., Shanghai, Toronto, Tokyo), and/or other characteristics (e.g., economic sector, company size, investment style, duration, or credit quality).

The target market determines the investment universe and the securities available for inclusion in the index. Once the investment universe is identified, the number of securities and the specific securities to include in the index must be determined. The constituent securities could be nearly all those in the target market or a representative sample of the target market. Some equity indexes, such as the S\&P 500 Index and the FTSE 100, fix the number of securities included in the index and indicate this number in the name of the index. Other indexes allow the number of securities to vary to reflect changes in the target market or to maintain a certain percentage of the target market. For example, the Tokyo Stock Price Index (TOPIX) represents and includes all of the largest stocks, known as the First Section, listed on the Tokyo Stock Exchange. To be included in the First Section-and thus the TOPIX-stocks must meet certain criteria, such as the number of shares outstanding, the number of shareholders, and market capitalization. Stocks that no longer meet the criteria are removed from the First Section and also the TOPIX. Objective or mechanical rules determine the constituent securities of most, but not all, indexes. The S\&P Bombay Stock Exchange Sensitive Index, also called the S\&P BSE SENSEX and the S\&P 500, for example, use a selection committee and more subjective decision-making rules to determine constituent securities.

\section{Index Weighting}
The weighting decision determines how much of each security to include in the index and has a substantial impact on an index's value. Index providers use a number of methods to weight the constituent securities in an index. Indexes can be price weighted, equal weighted, market-capitalization weighted, or fundamentally weighted. Each weighting method has its advantages and disadvantages.

\section{Price Weighting}
The simplest method to weight an index and the one used by Charles Dow to construct the Dow Jones Industrial Average is price weighting. In price weighting, the weight on each constituent security is determined by dividing its price by the sum of all the prices of the constituent securities. The weight is calculated using the following formula:

$$
w_{i}^{P}=\frac{P_{i}}{\sum_{i=1}^{N} P_{i}}
$$

Exhibit 1 illustrates the values, weights, and single-period returns following inception of a price-weighted equity index with five constituent securities. The value of the price-weighted index is determined by dividing the sum of the security values (101.50) by the divisor, which is typically set at inception to equal the initial number of securities in the index. Thus, in our example, the divisor is 5 and the initial value of the index is calculated as $101.50 \div 5=20.30$.

\begin{center}
\includegraphics[max width=\textwidth]{2023_05_04_7b535d0a870224f62e3dg-224}
\end{center}

As illustrated in this exhibit, Security A, which has the highest price, also has the highest weighting and thus will have the greatest impact on the return of the index. Note how both the price return and the total return of the index are calculated on the basis of the corresponding returns on the constituent securities.

A property unique to price-weighted indexes is that a stock split on one constituent security changes the weights on all the securities in the index. ${ }^{1}$ To prevent the stock split and the resulting new weights from changing the value of the index, the index provider must adjust the value of the divisor as illustrated in Exhibit 2. Given a 2-for-1 split in Security A, the divisor is adjusted by dividing the sum of the constituent prices after the split (77.50) by the value of the index before the split (21.00). This adjustment results in changing the divisor from 5 to 3.69 so that the index value is maintained at 21.00 .

The primary advantage of price weighting is its simplicity. The main disadvantage of price weighting is that it results in arbitrary weights for each security. In particular, a stock split in any one security causes arbitrary changes in the weights of all the constituents' securities.

\section{Exhibit 2: Impact of 2-for-1 Split in Security A}
\begin{center}
\begin{tabular}{lcccc}
\hline
Security & Price before Split & Weight before Split (\%) & Price after Split & Weight after Split (\%) \\
\hline
A & 55.00 & 52.38 & 27.50 & 35.48 \\
B & 22.00 & 20.95 & 22.00 & 28.39 \\
C & 8.00 & 7.62 & 8.00 & 10.32 \\
D & 14.00 & 13.33 & 14.00 & 18.07 \\
E & 6.00 & 5.72 & 6.00 & 7.74 \\
Total & 105.00 & 100.00 & 77.50 & 100.00 \\
Divisor & 5.00 & 3.69 &  &  \\
Index Value & 21.00 & 21.00 &  &  \\
\hline
\end{tabular}
\end{center}

\section{Equal Weighting}
Another simple index weighting method is equal weighting. This method assigns an equal weight to each constituent security at inception. The weights are calculated as:

$$
\mathrm{w}_{i}^{\mathrm{E}}=\frac{1}{N}
$$

where

$\mathrm{w}_{i}=$ fraction of the portfolio that is allocated to security $i$ or weight of security $i$

$N=$ number of securities in the index

To construct an equal-weighted index from the five securities in Exhibit 1, the index provider allocates one-fifth ( 20 percent) of the value of the index (at the beginning of the period) to each security. Dividing the value allocated to each security by each security's individual share price determines the number of shares of each security to

1 A stock split is an increase in the number of shares outstanding and a proportionate decrease in the price per share such that the total market value of equity, as well as investors' proportionate ownership in the company, does not change. include in the index. Unlike a price-weighted index, where the weights are arbitrarily determined by the market prices, the weights in an equal-weighted index are arbitrarily assigned by the index provider.

Exhibit 3 illustrates the values, weights, and single-period returns following inception of an equal-weighted index with the same constituent securities as those in Exhibit 1. This example assumes a beginning index portfolio value of 10,000 (i.e., an investment of 2,000 in each security). To set the initial value of the index to 1,000 , the divisor is set to $10(10,000 \div 10=1,000)$.

Exhibit 1 and Exhibit 3 demonstrate how different weighting methods result in different returns. The 10.4 percent price return of the equal-weighted index shown in Exhibit 3 differs significantly from the 3.45 percent price return of the price-weighted index in Exhibit 1.

Like price weighting, the primary advantage of equal weighting is its simplicity. Equal weighting, however, has a number of disadvantages. First, securities that constitute the largest fraction of the target market value are underrepresented, and securities that constitute a small fraction of the target market value are overrepresented. Second, after the index is constructed and the prices of constituent securities change, the index is no longer equally weighted. Therefore, maintaining equal weights requires frequent adjustments (rebalancing) to the index.

\section{Market-Capitalization Weighting}
In market-capitalization weighting, or value weighting, the weight on each constituent security is determined by dividing its market capitalization by the total market capitalization (the sum of the market capitalization) of all the securities in the index. Market capitalization or value is calculated by multiplying the number of shares outstanding by the market price per share.

The market-capitalization weight of security $i$ is:

$$
\mathrm{w}_{i}^{\mathrm{M}}=\frac{\mathrm{Q}_{i} \mathrm{P}_{i}}{\sum_{j=1}^{N} \mathrm{Q}_{j} \mathrm{P}_{j}}
$$

where

$\mathrm{w}_{i}=$ fraction of the portfolio that is allocated to security $i$ or weight of security $i$

$\mathrm{Q}_{i}=$ number of shares outstanding of security $i$

$\mathrm{P}_{i}=$ share price of security $i$

$N=$ number of securities in the index

Exhibit 4 illustrates the values, weights, and single-period returns following inception of a market-capitalization-weighted index for the same five-security market. Security A, with 3,000 shares outstanding and a price of 50 per share, has a market capitalization of 150,000 or 26.29 percent $(150,000 / 570,500)$ of the entire index portfolio. The resulting index weights in the exhibit reflect the relative value of each security as measured by its market capitalization.

As shown in Exhibit 1, Exhibit 3, and Exhibit 4, the weighting method affects the index's returns. The price and total returns of the market-capitalization index in Exhibit 4 (1.49 percent and 2.13 percent, respectively) differ significantly from those of the price-weighted ( 3.45 percent and 4.33 percent, respectively) and equal-weighted (10.40 percent and 10.88 percent respectively) indexes. To understand the source and magnitude of the difference, compare the weights and returns of each security under each of the weighting methods. The weight of Security A, for example, ranges from 49.26 percent in the price-weighted index to 20 percent in the equal-weighted index. With
\includegraphics[max width=\textwidth, center]{2023_05_04_7b535d0a870224f62e3dg-227}

\begin{center}
\includegraphics[max width=\textwidth]{2023_05_04_7b535d0a870224f62e3dg-228}
\end{center}

a price return of 10 percent, Security A contributes 4.93 percent to the price return of the price-weighted index, 2.00 percent to the price return of the equal-weighted index, and 2.63 percent to the price return of the market-capitalization-weighted index. With a total return of 11.50 percent, Security A contributes 5.66 percent to the total return of the price-weighted index, 2.30 percent to the total return of the equal-weighted index, and 3.02 percent to the total return of the market-capitalization-weighted index.

\section{Float-Adjusted Market-Capitalization Weighting}
In float-adjusted market-capitalization weighting, the weight on each constituent security is determined by adjusting its market capitalization for its market float. Typically, market float is the number of shares of the constituent security that are available to the investing public. For companies that are closely held, only a portion of the shares outstanding are available to the investing public (the rest are held by a small group of controlling investors). In addition to excluding shares held by controlling shareholders, most float-adjusted market-capitalization-weighted indexes also exclude shares held by other corporations and governments. Some providers of indexes that are designed to represent the investment opportunities of global investors further reduce the number of shares included in the index by excluding shares that are not available to foreigner investors. The index providers may refer to these indexes as "free-float-adjusted market-capitalization-weighted indexes."

Float-adjusted market-capitalization-weighted indexes reflect the shares available for public trading by multiplying the market price per share by the number of shares available to the investing public (i.e., the float-adjusted market capitalization) rather than the total number of shares outstanding (total market capitalization). Currently, most market-capitalization-weighted indexes are float adjusted. Therefore, unless otherwise indicated, for the remainder of this reading, "market-capitalization" weighting refers to float-adjusted market-capitalization weighting.

The float-adjusted market-capitalization weight of security $i$ is calculated as:

$\mathrm{w}_{i}^{\mathrm{M}}=\frac{\mathrm{f}_{i} \mathrm{Q}_{i} \mathrm{P}_{i}}{\sum_{j=1}^{N} \mathrm{f}_{j} \mathrm{Q}_{j} \mathrm{P}_{j}}$

where

$\mathrm{f}_{i}=$ fraction of shares outstanding in the market float

$\mathrm{w}_{i}=$ fraction of the portfolio that is allocated to security $i$ or weight of security $i$

$\mathrm{Q}_{i}=$ number of shares outstanding of security $i$

$\mathrm{P}_{i}=$ share price of security $i$

$N=$ number of securities in the index

Exhibit 5 illustrates the values, weights, and single-period returns following inception of a float-adjusted market-capitalization-weighted equity index using the same five securities as before. The low percentage of shares of Security D in the market float compared with the number of shares outstanding indicates that the security is closely held.

The primary advantage of market-capitalization weighting (including float adjusted) is that constituent securities are held in proportion to their value in the target market. The primary disadvantage is that constituent securities whose prices have risen the most (or fallen the most) have a greater (or lower) weight in the index (i.e., as a security's price rises relative to other securities in the index, its weight increases; and as its price decreases in value relative to other securities in the index, its weight decreases). This weighting method leads to overweighting stocks that have risen in price

\begin{center}
\includegraphics[max width=\textwidth]{2023_05_04_7b535d0a870224f62e3dg-230}
\end{center}

(and may be overvalued) and underweighting stocks that have declined in price (and may be undervalued). The effect of this weighting method is similar to a momentum investment strategy in that over time, the securities that have risen in price the most will have the largest weights in the index.

\section{Fundamental Weighting}
Fundamental weighting attempts to address the disadvantages of market-capitalization weighting by using measures of a company's size that are independent of its security price to determine the weight on each constituent security. These measures include book value, cash flow, revenues, earnings, dividends, and number of employees.

Some fundamental indexes use a single measure, such as total dividends, to weight the constituent securities, whereas others combine the weights from several measures to form a composite value that is used for weighting.

Letting $\mathrm{F}_{i}$ denote a given fundamental size measure of company $i$, the fundamental weight on security $i$ is:

$$
\mathrm{w}_{i}^{\mathrm{F}}=\frac{\mathrm{F}_{i}}{\sum_{j=1}^{N} \mathrm{~F}_{j}}
$$

Relative to a market-capitalization-weighted index, a fundamental index with weights based on such an item as earnings will result in greater weights on constituent securities with earnings yields (earnings divided by price) that are higher than the earnings yield of the overall market-weighted portfolio. Similarly, stocks with earnings yields less than the yield on the overall market-weighted portfolio will have lower weights. For example, suppose there are two stocks in an index. Stock A has a market capitalization of $€ 200$ million, Stock B has a market capitalization of $€ 800$ million, and their aggregate market capitalization is $€ 1$ billion ( $€ 1,000$ million). Both companies have earnings of $€ 20$ million and aggregate earnings of $€ 40$ million. Thus, Stock A has an earnings yield of 10 percent (20/200) and Stock B has an earnings yield of 2.5 percent (20/800). The earnings weight of Stock A is 50 percent (20/40), which is higher than its market-capitalization weight of 20 percent $(200 / 1,000)$. The earnings weight of Stock B is 50 percent (20/40), which is less than its market-capitalization weight of 80 percent $(800 / 1,000)$. Relative to the market-cap-weighted index, the earnings-weighted index over-weights the high-yield Stock A and under-weights the low-yield Stock B.

The most important property of fundamental weighting is that it leads to indexes that have a "value" tilt. That is, a fundamentally weighted index has ratios of book value, earnings, dividends, etc. to market value that are higher than its market-capitalization-weighted counterpart. Also, in contrast to the momentum "effect" of market-capitalization-weighted indexes, fundamentally weighted indexes generally will have a contrarian "effect" in that the portfolio weights will shift away from securities that have increased in relative value and toward securities that have fallen in relative value whenever the portfolio is rebalanced.

\section*{INDEX MANAGEMENT: REBALANCING AND RECONSTITUTION }
So far, we have discussed index construction. Index management entails the two remaining questions:

\begin{itemize}
  \item When should the index be rebalanced?

  \item When should the security selection and weighting decisions be re-examined?

\end{itemize}

\section{Rebalancing}
Rebalancing refers to adjusting the weights of the constituent securities in the index. To maintain the weight of each security consistent with the index's weighting method, the index provider rebalances the index by adjusting the weights of the constituent securities on a regularly scheduled basis (rebalancing dates)-usually quarterly. Rebalancing is necessary because the weights of the constituent securities change as their market prices change. Note, for example, that the weights of the securities in the equal-weighted index (Exhibit 3) at the end of the period are no longer equal (i.e., 20 percent):

$\begin{array}{ll}\text { Security A } & 19.93 \% \\ \text { Security B } & 15.94 \\ \text { Security C } & 11.60 \\ \text { Security D } & 25.36 \\ \text { Security E } & 27.17\end{array}$

In rebalancing the index, the weights of Securities $D$ and $E$ (which had the highest returns) would be decreased and the weights of Securities A, B, and C (which had the lowest returns) would be increased. Thus, rebalancing creates turnover within an index.

Price-weighted indexes are not rebalanced because the weight of each constituent security is determined by its price. For market-capitalization-weighted indexes, rebalancing is less of a concern because the indexes largely rebalance themselves. In our market-capitalization index, for example, the weight of Security $\mathrm{C}$ automatically declined from 10.96 percent to 6.91 percent, reflecting the 36 percent decline in its market price. Market-capitalization weights are only adjusted to reflect mergers, acquisitions, liquidations, and other corporate actions between rebalancing dates.

\section{Reconstitution}
Reconstitution is the process of changing the constituent securities in an index. It is similar to a portfolio manager deciding to change the securities in his or her portfolio. Reconstitution is part of the rebalancing cycle. The reconstitution date is the date on which index providers review the constituent securities, re-apply the initial criteria for inclusion in the index, and select which securities to retain, remove, or add. Constituent securities that no longer meet the criteria are replaced with securities that do meet the criteria. Once the revised list of constituent securities is determined, the weighting method is re-applied. Indexes are reconstituted to reflect changes in the target market (bankruptcies, de-listings, mergers, acquisitions, etc.) and/or to reflect the judgment of the selection committee.

Reconstitution creates turnover in a number of different ways, particularly for market-capitalization-weighted indexes. When one security is removed and another is added, the index provider has to change the weights of the other securities in order to maintain the market-capitalization weighting of the index.

The frequency of reconstitution is a major issue for widely used indexes and their constituent securities. The Russell 2000 Index, for example, reconstitutes annually. It is used as a benchmark by numerous investment funds, and each year, prior to the index's reconstitution, the managers of these funds buy stocks they think will be added to the index-driving those stocks' prices up-and sell stocks they think will be deleted from the index-driving those stocks' prices down. Exhibit 6 illustrates a historical example of the potential impact of these decisions. Beginning in late April 2009, some managers began acquiring and bidding up the price of Uranium Energy Corporation (UEC) because they believed that it would be included in the reconstituted Russell 2000 Index. On 12 June, Russell listed UEC as a preliminary addition to the Russell 2000 Index and the Russell 3000 Index. ${ }^{2}$ By that time, the stock value had increased by more than 300 percent. Investors continued to bid up the stock price in the weeks following the announcement, and the stock closed on the reconstitution date of 30 June at USD2.90, up nearly 400 percent for the quarter.

Exhibit 6: Three-Month Performance of Uranium Energy Corporation and NASDAQ April through June 2009

\begin{center}
\includegraphics[max width=\textwidth]{2023_05_04_7b535d0a870224f62e3dg-233}
\end{center}

Source: Yahoo! Finance and Capital IQ.

2 According to the press release, final membership in the index would be published after market close on Friday, 26 June.

\section*{USES OF MARKET INDEXES }
Indexes were initially created to give a sense of how a particular security market performed on a given day. With the development of modern financial theory, their uses in investment management have expanded significantly. Some of the major uses of indexes include:

\begin{itemize}
  \item gauges of market sentiment;

  \item proxies for measuring and modeling returns, systematic risk, and risk-adjusted performance;

  \item proxies for asset classes in asset allocation models;

  \item benchmarks for actively managed portfolios; and

  \item model portfolios for such investment products as index funds and exchange-traded funds (ETFs).

\end{itemize}

Investors using security market indexes must be familiar with how various indexes are constructed in order to select the index or indexes most appropriate for their needs.

\section{Gauges of Market Sentiment}
The original purpose of stock market indexes was to provide a gauge of investor confidence or market sentiment. As indicators of the collective opinion of market participants, indexes reflect investor attitudes and behavior. The Dow Jones Industrial Average has a long history, is frequently quoted in the media, and remains a popular gauge of market sentiment. It may not accurately reflect the overall attitude of investors or the "market," however, because the index consists of only 30 of the thousands of US stocks traded each day.

\section{Proxies for Measuring and Modeling Returns, Systematic Risk, and Risk-Adjusted Performance}
The capital asset pricing model (CAPM) defines beta as the systematic risk of a security with respect to the entire market. The market portfolio in the CAPM consists of all risky securities. To represent the performance of the market portfolio, investors use a broad index. For example, the Tokyo Price Index (TOPIX) and the S\&P 500 often serve as proxies for the market portfolio in Japan and the United States, respectively, and are used for measuring and modeling systematic risk and market returns.

Security market indexes also serve as market proxies when measuring risk-adjusted performance. The beta of an actively managed portfolio allows investors to form a passive alternative with the same level of systematic risk. For example, if the beta of an actively managed portfolio of global stocks is 0.95 with respect to the MSCI World Index, investors can create a passive portfolio with the same systematic risk by investing 95 percent of their portfolio in a MSCI World Index fund and holding the remaining 5 percent in cash. Alpha, the difference between the return of the actively managed portfolio and the return of the passive portfolio, is a measure of risk-adjusted return or investment performance. Alpha can be the result of manager skill (or lack thereof), transaction costs, and fees.

\section{Proxies for Asset Classes in Asset Allocation Models}
Because indexes exhibit the risk and return profiles of select groups of securities, they play a critical role as proxies for asset classes in asset allocation models. They provide the historical data used to model the risks and returns of different asset classes.

\section{Benchmarks for Actively Managed Portfolios}
Investors often use indexes as benchmarks to evaluate the performance of active portfolio managers. The index selected as the benchmark should reflect the investment strategy used by the manager. For example, an active manager investing in global small-capitalization stocks should be evaluated using a benchmark index, such as the FTSE Global Small Cap Index, which includes approximately 4,400 liquid small-capitalization stocks across 47 countries as of August 2018.

The choice of an index to use as a benchmark is important because an inappropriate index could lead to incorrect conclusions regarding an active manager's investment performance. Suppose that the small-cap manager underperformed the small-cap index but outperformed a broad equity market index. If investors use the broad market index as a benchmark, they might conclude that the small-cap manager is earning his or her fees and should be retained or given additional assets to invest. Using the small-cap index as a benchmark might lead to a very different conclusion.

\section{Model Portfolios for Investment Products}
Indexes also serve as the basis for the development of new investment products. Using indexes as benchmarks for actively managed portfolios has led some investors to conclude that they should invest in the benchmarks instead. Based on the CAPM's conclusion that investors should hold the market portfolio, broad market index funds have been developed to function as proxies for the market portfolio.

Investment management firms initially developed and managed index portfolios for institutional investors. Eventually, mutual fund companies introduced index funds for individual investors. Subsequently, investment management firms introduced exchange-traded funds, which are managed the same way as index mutual funds but trade like stocks.

The first ETFs were based on existing indexes. As the popularity of ETFs increased, index providers created new indexes for the specific purpose of forming ETFs, leading to the creation of numerous narrowly defined indexes with corresponding ETFs. The VanEck Vectors Vietnam ETF, for example, allows investors to invest in the equity market of Vietnam.

The choice of indexes to meet the needs of investors is extensive. Index providers are constantly looking for opportunities to develop indexes to meet the needs of investors.

\section*{EQUITY INDEXES }
A wide variety of equity indexes exist, including broad market, multi-market, sector, and style indexes.

\section{Broad Market Indexes}
A broad equity market index, as its name suggests, represents an entire given equity market and typically includes securities representing more than 90 percent of the selected market. For example, the Shanghai Stock Exchange Composite Index (SSE) is a market-capitalization-weighted index of all shares that trade on the Shanghai Stock Exchange. In the United States, the Wilshire 5000 Total Market Index is a market-capitalization-weighted index that includes all US equities with readily available prices and is designed to represent the entire US equity market. ${ }^{3}$ The Russell 3000, consisting of the largest 3,000 stocks by market capitalization, represents approximately 98 percent of the US equity market.

\section{Multi-Market Indexes}
Multi-market indexes usually comprise indexes from different countries and regions and are designed to represent multiple security markets. Multi-market indexes may represent multiple national markets, geographic regions, economic development groups, and, in some cases, the entire world. World indexes are of importance to investors who take a global approach to equity investing without any particular bias toward a particular country or region. A number of index providers publish families of multi-market equity indexes.

MSCI offers a number of multi-market indexes. As shown in Exhibit 7, MSCI classifies countries and regions along two dimensions: level of economic development and geographic region. Developmental groups, which MSCI refers to as market classifications, include developed markets, emerging markets, and frontier markets. The geographic regions are largely divided by longitudinal lines of the globe: the Americas, Europe with Africa, and Asia with the Pacific. MSCI provides country- and region-specific indexes for each of the developed and emerging markets within its multi-market indexes. MSCI periodically reviews the classifications of markets in its indexes for movement from frontier markets to emerging markets and from emerging markets to developed markets and reconstitutes the indexes accordingly.

3 Despite its name, the Wilshire 5000 has no constraint on the number of securities that can be included. It included approximately 5,000 securities at inception.

\section{Exhibit 7: MSCI Global Investable Market Indexes (as of October 2018)}
Developed Markets

\begin{center}
\begin{tabular}{lll}
\hline
Americas & Europe and Middle East & Pacific \\
\hline
Canada, United States & Austria, Belgium, Denmark, Finland, & Australia, Hong Kong SAR, Japan, New \\
 & France, Germany, Ireland, Israel, Italy, $\quad$ Zealand, Singapore &  \\
 & Netherlands, Norway, Portugal, Spain, &  \\
 & Sweden, Switzerland, United Kingdom &  \\
\end{tabular}
\end{center}

Emerging Markets

\begin{center}
\begin{tabular}{|c|c|c|}
\hline
Americas & Europe, Middle East, Africa & Asia \\
\hline
Brazil, Chile, Colombia, Mexico, Peru & $\begin{array}{l}\text { Czech Republic, Egypt, Greece, } \\ \text { Hungary, Poland, Qatar, Russia, South } \\ \text { Africa, Turkev, United Arab Emirates }\end{array}$ & $\begin{array}{l}\text { Chinese mainland, India, Indonesia, } \\ \text { South Korea, Malaysia, Pakistan, } \\ \text { Philippines, Taiwan region, Thailand }\end{array}$ \\
\hline
\end{tabular}
\end{center}

Frontier Markets

\begin{center}
\begin{tabular}{lllll}
\hline
Americas & Europe \& CIS & Africa & Middle East & Asia \\
\hline
Argentina & Croatia, Estonia, & Kenya, Mauritius, & Bahrain, Jordan, & Bangladesh, Sri Lanka, \\
 & Lithuania, Kazakhstan, & Morocco, Nigeria, & Kuwait, Lebanon, & Vietnam \\
 & Romania, Serbia, & Tunisia, WAEMU 1 & Oman &  \\
 & Slovenia &  &  &  \\
\hline
\end{tabular}
\end{center}

MSCI Standalone Market Indexes ${ }^{2}$

\begin{center}
\begin{tabular}{lllll}
$\begin{array}{l}\text { Europe, Middle East, } \\ \text { and Africa }\end{array}$ & Americas & Europe and CIS & Africa \\
\hline
Saudi Arabia & Jamaica, Panama, ${ }^{3}$ & Bosnia Herzegovina, & Botswana, Ghana, \\
 & Trinidad \& Tobago & Bulgaria, Ukraine & Zimbabwe \\
\end{tabular}
\end{center}

${ }^{1}$ The West African Economic and Monetary Union (WAEMU) consists of the following countries: Benin, Burkina Faso, Ivory Coast, Guinea-Bissau, Mali, Niger, Senegal, and Togo. Currently the MSCI WAEMU Indexes include securities classified in Senegal, Ivory Coast, and Burkina Faso.

${ }^{2}$ The MSCI Standalone Market Indexes are not included in the MSCI Emerging Markets Index or MSCI Frontier Markets Index. However, these indexes use either the Emerging Markets or the Frontier Markets methodological criteria concerning size and liquidity.

${ }^{3}$ MSCI Panama Index has been launched as a Standalone Market Index.

Source: adapted from MSCI (\href{https://www.msci.com/en/market-cap-weighted-indexes}{https://www.msci.com/en/market-cap-weighted-indexes}), October 2018.

\section{Fundamental Weighting in Multi-Market Indexes}
Some index providers weight the securities within each country/region by market capitalization and then weight each country/region in the overall index in proportion to its relative GDP, effectively creating fundamental weighting in multi-market indexes. GDP-weighted indexes were some of the first fundamentally weighted indexes created. Introduced in 1987 by MSCI to address the 60 percent weight of Japanese equities in the market-capitalization-weighted MSCI EAFE Index at the time, GDP-weighted indexes reduced the allocation to Japanese equities by half. ${ }^{4}$

\section{Sector Indexes}
Sector indexes represent and track different economic sectors-such as consumer goods, energy, finance, health care, and technology-on either a national, regional, or global basis. Because different sectors of the economy behave differently over the course of the business cycle, some investors may seek to overweight or underweight their exposure to particular sectors.

Sector indexes are organized as families; each index within the family represents an economic sector. Typically, the aggregation of a sector index family is equivalent to a broad market index. Economic sector classification can be applied on a global, regional, or country-specific basis, but no universally agreed upon sector classification method exists.

Sector indexes play an important role in performance analysis because they provide a means to determine whether a portfolio manager is more successful at stock selection or sector allocation. Sector indexes also serve as model portfolios for sector-specific ETFs and other investment products.

\section{Style Indexes}
Style indexes represent groups of securities classified according to market capitalization, value, growth, or a combination of these characteristics. They are intended to reflect the investing styles of certain investors, such as the growth investor, value investor, and small-cap investor.

\section{Market Capitalization}
Market-capitalization indexes represent securities categorized according to the major capitalization categories: large cap, midcap, and small cap. With no universal definition of these categories, the indexes differ on the distinctions between large cap and midcap and between midcap and small cap, as well as the minimum market-capitalization size required to be included in a small-cap index. Classification into categories can be based on absolute market capitalization (e.g., below $€ 100$ million) or relative market capitalization (e.g., the smallest 2,500 stocks).

\section{Value/Growth Classification}
Some indexes represent categories of stocks based on their classifications as either value or growth stocks. Different index providers use different factors and valuation ratios (low price-to-book ratios, low price-to-earnings ratios, high dividend yields, etc.) to distinguish between value and growth equities.

\section{Market Capitalization and Value/Growth Classification}
Combining the three market-capitalization groups with value and growth classifications results in six basic style index categories:

\begin{itemize}
  \item Large-Cap Value

  \item Mid-Cap Value

  \item Small-Cap Value - Large-Cap Growth

  \item Mid-Cap Growth

  \item Small-Cap Growth

\end{itemize}

Because indexes use different size and valuation classifications, the constituents of indexes designed to represent a given style, such as small-cap value, may differsometimes substantially.

Because valuation ratios and market capitalizations change over time, stocks frequently migrate from one style index category to another on reconstitution dates. As a result, style indexes generally have much higher turnover than do broad market indexes.

\section*{FIXED-INCOME INDEXES }
A wide variety of fixed-income indexes exists, but the nature of the fixed-income markets and fixed-income securities leads to some very important challenges to fixed-income index construction and replication. These challenges are the number of securities in the fixed-income universe, the availability of pricing data, and the liquidity of the securities.

\section{Construction}
The fixed-income universe includes securities issued by governments, government agencies, and corporations. Each of these entities may issue a variety of fixed-income securities with different characteristics. As a result, the number of fixed-income securities is many times larger than the number of equity securities. To represent a specific fixed-income market or segment, indexes may include thousands of different securities. Over time, these fixed-income securities mature, and issuers offer new securities to meet their financing needs, leading to turnover in fixed-income indexes.

Another challenge in index construction is that fixed-income markets are predominantly dealer markets. This means that firms (dealers) are assigned to specific securities and are responsible for creating liquid markets for those securities by purchasing and selling them from their inventory. In addition, many securities do not trade frequently and, as a result, are relatively illiquid. As a result, index providers must contact dealers to obtain current prices on constituent securities to update the index or they must estimate the prices of constituent securities using the prices of traded fixed-income securities with similar characteristics.

These challenges can result in indexes with dissimilar numbers of bonds representing the same markets. The large number of fixed-income securities-combined with the lack of liquidity of some securities-has made it more costly and difficult, compared with equity indexes, for investors to replicate fixed-income indexes and duplicate their performance.

\section{Types of Fixed-Income Indexes}
The wide variety of fixed-income securities, ranging from zero-coupon bonds to bonds with embedded options (i.e., callable or putable bonds), results in a number of different types of fixed-income indexes. Similar to equities, fixed-income securities can be categorized according to the issuer's economic sector, the issuer's geographic region, or the economic development of the issuer's geographic region. Fixed-income securities can also be classified along the following dimensions:

\begin{itemize}
  \item type of issuer (government, government agency, corporation);

  \item type of financing (general obligation, collateralized);

  \item currency of payments;

  \item maturity;

  \item credit quality (investment grade, high yield, credit agency ratings); and

  \item absence or presence of inflation protection. Fixed-income indexes are based on these various dimensions and can be categorized as follows:

  \item aggregate or broad market indexes;

  \item market sector indexes;

  \item style indexes;

  \item economic sector indexes; and

  \item specialized indexes such as high-yield, inflation-linked, and emerging market indexes.

\end{itemize}

The first fixed-income index created, the Bloomberg Barclays US Aggregate Bond Index (formerly the Barclays Capital Aggregate Bond Index), is an example of a single-country aggregate index. Designed to represent the broad market of US fixed-income securities, it comprises approximately 8,000 securities, including US Treasury, government-related, corporate, mortgage-backed, asset-backed, and commercial mortgage-backed securities.

Aggregate indexes can be subdivided by market sector (government, government agency, collateralized, corporate); style (maturity, credit quality); economic sector, or some other characteristic to create more narrowly defined indexes. A common distinction reflected in indexes is between investment grade (e.g., those with a Standard \& Poor's credit rating of BBB- or better) and high-yield securities. Investment-grade indexes are typically further subdivided by maturity (i.e., short, intermediate, or long) and by credit rating (e.g., AAA, BBB, etc.). ${ }^{5}$ The wide variety of fixed-income indexes reflects the partitioning of fixed-income securities on the basis of a variety of dimensions.

Exhibit 8 illustrates how the major types of fixed-income indexes can be organized on the basis of various dimensions.

\section{Exhibit 8: Dimensions of Fixed-Income Indexes}
\begin{center}
\begin{tabular}{l|l|c|c|c}
\hline
\multirow{2}{*}{Market} & \multicolumn{3}{|c}{Global} \\
\cline { 2 - 4 }
\multirow{2}{*}{} & \multicolumn{3}{|c}{Regional} \\
\cline { 2 - 4 }
 & \multicolumn{3}{|c}{Country or currency zone} \\
\cline { 2 - 4 }
Maturity & $\begin{array}{c}\text { Collateralized } \\ \text { Cecuritized } \\ \text { Mortgage-backed }\end{array}$ & \multicolumn{1}{c}{$\begin{array}{c}\text { Government } \\
\text { agency }\end{array}$} & Government \\
For example, 1-3, 3-5, 5-7, 7-10, 10+ years; short-term, medium-term, &  &  &  \\
or long-term &  &  &  \\
\end{tabular}
\end{center}

All aggregate indexes include a variety of market sectors and credit ratings. The breakdown of the Bloomberg Barclays Global Aggregate Bond Index by market sectors and by credit rating is shown in Exhibit 9 and Exhibit 10, respectively. Exhibit 9: Market Sector Breakdown of the Bloomberg Barclays Global Aggregate Bond Index

\section{Sector Breakdown as of June 30, 2016}
\begin{center}
\includegraphics[max width=\textwidth]{2023_05_04_7b535d0a870224f62e3dg-241(1)}
\end{center}

Source: Bloomberg Barclays Indices, Global Aggregate Index Factsheet, August 24, 2016. https:// \href{http://data.bloomberglp.com/indices/sites/2/2016/08/Factsheet-Global-Aggregate.pdf}{data.bloomberglp.com/indices/sites/2/2016/08/Factsheet-Global-Aggregate.pdf}.

Exhibit 10: Credit Breakdown of the Bloomberg Barclays Global Aggregate Bond Index

Quality Breakdown as of June 30, 2016

\begin{center}
\includegraphics[max width=\textwidth]{2023_05_04_7b535d0a870224f62e3dg-241}
\end{center}

Source: Bloomberg Barclays Indices, Global Aggregate Index Factsheet, August 24, 2016. https:// \href{http://data.bloomberglp.com/indices/sites/2/2016/08/Factsheet-Global-Aggregate.pdf}{data.bloomberglp.com/indices/sites/2/2016/08/Factsheet-Global-Aggregate.pdf}.

\section*{INDEXES FOR ALTERNATIVE INVESTMENTS }
Many investors seek to lower the risk or enhance the performance of their portfolios by investing in assets classes other than equities and fixed income. Interest in alternative assets and investment strategies has led to the creation of indexes designed to represent broad classes of alternative investments. Three of the most widely followed alternative investment classes are commodities, real estate, and hedge funds.

\section{Commodity Indexes}
Commodity indexes consist of futures contracts on one or more commodities, such as agricultural products (rice, wheat, sugar), livestock (cattle, hogs), precious and common metals (gold, silver, copper), and energy commodities (crude oil, natural gas).

Although some commodity indexes may include the same commodities, the returns of these indexes may differ because each index may use a different weighting method. Because commodity indexes do not have an obvious weighting mechanism, such as market capitalization, commodity index providers create their own weighting methods. Some indexes, such as the Thomson Reuters/Core Commodity CRB Index (TR/CC CRB Index), formerly known as the Commodity Research Bureau (CRB) Index, contain a fixed number of commodities that are weighted equally. The S\&P GSCI uses a combination of liquidity measures and world production values in its weighting scheme and allocates more weight to commodities that have risen in price. Other indexes have fixed weights that are determined by a committee.

The different weighting methods can also lead to large differences in exposure to specific commodities. The S\&P GSCI in 2018, for example, weights the energy-sector approximately $50 \%$ higher and the agriculture sector $40 \%$ lower than the CRB Index. These differences result in indexes with very different risk and return profiles. Unlike commodity indexes, broad equity and fixed-income indexes that target the same markets share similar risk and return profiles.

The performance of commodity indexes can also be quite different from their underlying commodities because the indexes consist of futures contracts on the commodities rather than the actual commodities. Index returns are affected by factors other than changes in the prices of the underlying commodities because futures contracts must be continually "rolled over" (i.e., replacing a contract nearing expiration with a new contract). Commodity index returns reflect the risk-free interest rate, the changes in future prices, and the roll yield. Therefore, a commodity index return can be quite different from the return based on changes in the prices of the underlying commodities.

\section{Real Estate Investment Trust Indexes}
Real estate indexes represent not only the market for real estate securities but also the market for real estate-a highly illiquid market and asset class with infrequent transactions and pricing information. Real estate indexes can be categorized as appraisal indexes, repeat sales indexes, and real estate investment trust (REIT) indexes.

REIT indexes consist of shares of publicly traded REITs. REITS are public or private corporations organized specifically to invest in real estate, either through ownership of properties or investment in mortgages. Shares of public REITs are traded on the world's various stock exchanges and are a popular choice for investing in commercial real estate properties. Because REIT indexes are based on publicly traded REITs with continuous market pricing, the value of REIT indexes is calculated continuously.

The FTSE EPRA/NAREIT global family of REIT indexes shown in Exhibit 11 seeks to represent trends in real estate stocks worldwide and includes representation from the European Public Real Estate Association (EPRA) and the National Association of Real Estate Investment Trusts (NAREIT).

\section{Exhibit 11: The FTSE EPRA/NAREIT Global REIT Index Family}
FTSE EPRA/NAREIT Global Real Estate Index Series

\begin{center}
\includegraphics[max width=\textwidth]{2023_05_04_7b535d0a870224f62e3dg-243}
\end{center}

Source: FTSE International, "FTSE EPRA/NAREIT Global \& Global Ex US Indices" Factsheet 2009). "FTSE" is a trade mark of the London Stock Exchange Plc, "NAREIT" is a trade mark of the National Association of Real Estate Investment Trusts ("NAREIT") and "EPRA"" is a trade mark of the European Public Real Estate Association ("EPRA") and all are used by FTSE International Limited (“FTSE") under license.

\section{Hedge Fund Indexes}
Hedge fund indexes reflect the returns on hedge funds. Hedge funds are private investment vehicles that typically use leverage and long and short investment strategies.

A number of research organizations maintain databases of hedge fund returns and summarize these returns into indexes. These database indexes are designed to represent the performance of the hedge funds on a very broad global level (hedge funds in general) or the strategy level. Most of these indexes are equal weighted and represent the performance of the hedge funds within a particular database.

Most research organizations rely on the voluntary cooperation of hedge funds to compile performance data. As unregulated entities, however, hedge funds are not required to report their performance to any party other than their investors. Therefore, each hedge fund decides to which database(s) it will report its performance. As a result, rather than index providers determining the constituents, the constituents determine the index. Frequently, a hedge fund reports its performance to only one database. The result is little overlap of funds covered by the different indexes. With little overlap between their constituents, different global hedge fund indexes may reflect very different performance for the hedge fund industry over the same period of time.

Another consequence of the voluntary performance reporting is the potential for survivorship bias and, therefore, inaccurate performance representation. This means that hedge funds with poor performance may be less likely to report their performance to the database or may stop reporting to the database, so their returns may be excluded when measuring the return of the index. As a result, the index may not accurately reflect actual hedge fund performance so much as the performance of hedge funds that are performing well.

\section{REPRESENTATIVE INDEXES WORLDWIDE}
As indicated in this reading, the choice of indexes to meet the needs of investors is extensive. Investors using security market indexes must be careful in their selection of the index or indexes most appropriate for their needs. The following table illustrates the variety of indexes reflecting different asset classes, markets, and weighting methods.

\begin{center}
\begin{tabular}{|c|c|c|c|c|}
\hline
Index & Representing & $\begin{array}{l}\text { Number of } \\ \text { Securities }\end{array}$ & Weighting Method & Comments \\
\hline
$\begin{array}{l}\text { Dow Jones } \\ \text { Industrial Average }\end{array}$ & US blue chip companies & 30 & Price & $\begin{array}{l}\text { The oldest and most widely known } \\ \text { US equity index. } \\ \text { Wall Street Journal editors choose } \\ 30 \text { stocks from among large, mature } \\ \text { blue-chip companies. }\end{array}$ \\
\hline
$\begin{array}{l}\text { Nikkei Stock } \\ \text { Average }\end{array}$ & $\begin{array}{l}\text { Japanese blue chip } \\ \text { companies }\end{array}$ & 225 & Modified price & $\begin{array}{l}\text { Known as the Nikkei } 225 \text { and } \\ \text { originally formulated by Dow Jones } \\ \text { \& Company. Because of extreme } \\ \text { variation in price levels of compo- } \\ \text { nent securities, some high-priced } \\ \text { shares are weighted as a fraction of } \\ \text { share price. }\end{array}$ \\
\hline
TOPIX & $\begin{array}{l}\text { All companies listed on } \\ \text { the Tokyo Stock Exchange } \\ \text { First Section }\end{array}$ & Varies & $\begin{array}{l}\text { Float-adjusted mar- } \\ \text { ket cap }\end{array}$ & $\begin{array}{l}\text { Represents about } 93 \text { percent of the } \\ \text { market value of all Japanese equities. } \\ \text { Contains a large number of very } \\ \text { small, illiquid stocks, making exact } \\ \text { replication difficult. }\end{array}$ \\
\hline
$\begin{array}{l}\text { MSCI All Country } \\ \text { World Index }\end{array}$ & $\begin{array}{l}\text { Stocks of } 23 \text { developed } \\ \text { and } 24 \text { emerging markets }\end{array}$ & Varies & $\begin{array}{l}\text { Free-float-adjusted } \\ \text { market cap }\end{array}$ & $\begin{array}{l}\text { Composed of companies represen- } \\ \text { tative of the market structure of } \\ \text { developed and emerging market } \\ \text { countries in the Americas, Europe/ } \\ \text { Middle East, and Asia/Pacific } \\ \text { regions. Price return and total return } \\ \text { versions available in both USD and } \\ \text { local currencies. }\end{array}$ \\
\hline
$\begin{array}{l}\text { S\&P Developed } \\ \text { Ex-US BMI Energy } \\ \text { Sector Index }\end{array}$ & $\begin{array}{l}\text { Energy sector of devel- } \\ \text { oped global markets } \\ \text { outside the United States }\end{array}$ & Varies & $\begin{array}{l}\text { Float-adjusted mar- } \\ \text { ket cap }\end{array}$ & $\begin{array}{l}\text { Serves as a model portfolio for } \\ \text { the SPDR S\&P Energy Sector } \\ \text { Exchange-Traded Fund (ETF). }\end{array}$ \\
\hline
$\begin{array}{l}\text { Bloomberg } \\ \text { Barclays Global } \\ \text { Aggregate Bond } \\ \text { Index }\end{array}$ & $\begin{array}{l}\text { Investment-grade bonds } \\ \text { in the North American, } \\ \text { European, and Asian } \\ \text { markets }\end{array}$ & Varies & Market cap & $\begin{array}{l}\text { Formerly known as Lehman Brothers } \\ \text { Global Aggregate Bond Index. }\end{array}$ \\
\hline
\end{tabular}
\end{center}

\begin{center}
\begin{tabular}{|c|c|c|c|c|}
\hline
Index & Representing & $\begin{array}{c}\text { Number of } \\ \text { Securities }\end{array}$ & Weighting Method & Comments \\
\hline
$\begin{array}{l}\text { Markit iBoxx Euro } \\ \text { High-Yield Bond } \\ \text { Indexes }\end{array}$ & $\begin{array}{l}\text { Sub-investment-grade } \\ \text { euro-denominated corpo- } \\ \text { rate bonds }\end{array}$ & Varies & $\begin{array}{l}\text { Market cap and } \\ \text { variations }\end{array}$ & $\begin{array}{l}\text { Rebalanced monthly. Represents } \\ \text { tradable part of market. Price and } \\ \text { total return versions available with } \\ \text { such analytical values as yield, dura- } \\ \text { tion, modified duration, and convex- } \\ \text { ity. Provides platform for research } \\ \text { and structured products. }\end{array}$ \\
\hline
\end{tabular}
\end{center}

\begin{center}
\begin{tabular}{|c|c|c|c|c|}
\hline
$\begin{array}{l}\text { FTSE } \\ \text { EPRA/NAREIT } \\ \text { Global Real Estate } \\ \text { Index }\end{array}$ & $\begin{array}{l}\text { Real estate securities in } \\ \text { the North American, } \\ \text { European, and Asian } \\ \text { markets }\end{array}$ & Varies & $\begin{array}{l}\text { Float-adjusted mar- } \\ \text { ket cap }\end{array}$ & $\begin{array}{l}\text { The stocks of REITs that constitute } \\ \text { the index trade on public stock } \\ \text { exchanges and may be constituents } \\ \text { of equity market indexes. }\end{array}$ \\
\hline
$\begin{array}{l}\text { HFRX Global } \\ \text { Hedge Fund Index }\end{array}$ & $\begin{array}{l}\text { Overall composition of } \\ \text { the HFR database }\end{array}$ & Varies & Asset weighted & $\begin{array}{l}\text { Comprises all eligible hedge fund } \\ \text { strategies. Examples include } \\ \text { convertible arbitrage, distressed } \\ \text { securities, market neutral, event } \\ \text { driven, macro, and relative value } \\ \text { arbitrage. Constituent strategies are } \\ \text { asset weighted on the basis of asset } \\ \text { distribution within the hedge fund } \\ \text { industry. }\end{array}$ \\
\hline
\end{tabular}
\end{center}

\begin{center}
\begin{tabular}{|c|c|c|c|c|}
\hline
$\begin{array}{l}\text { HFRX Equal } \\ \text { Weighted } \\ \text { Strategies EUR } \\ \text { Index }\end{array}$ & $\begin{array}{l}\text { Overall composition of } \\ \text { the HFR database }\end{array}$ & Varies & Equal weighted & $\begin{array}{l}\text { Denominated in euros and is con- } \\ \text { structed from the same strategies as } \\ \text { the HFRX Global Hedge Fund Index }\end{array}$ \\
\hline
$\begin{array}{l}\text { Morningstar Style } \\ \text { Indexes }\end{array}$ & $\begin{array}{l}\text { US stocks classified by } \\ \text { market cap and value/ } \\ \text { growth orientation }\end{array}$ & Varies & $\begin{array}{l}\text { Float-adjusted mar- } \\ \text { ket cap }\end{array}$ & $\begin{array}{l}\text { The nine indexes defined by com- } \\ \text { binations of market cap (large, mid, } \\ \text { and small) and value/growth ori- } \\ \text { entation (value, core, growth) have } \\ \text { mutually exclusive constituents and } \\ \text { are exhaustive with respect to the } \\ \text { Morningstar US Market Index. Each } \\ \text { is a model portfolio for one of the } \\ \text { iShares Morningstar ETFs. }\end{array}$ \\
\hline
\end{tabular}
\end{center}

\section{SUMMARY}
This reading explains and illustrates the construction, management, and uses of security market indexes. It also discusses various types of indexes. Security market indexes are invaluable tools for investors, who can select from among thousands of indexes representing a variety of security markets, market segments, and asset classes. These indexes range from those representing the global market for major asset classes to those representing alternative investments in specific geographic markets. To benefit from the use of security market indexes, investors must understand their construction and determine whether the selected index is appropriate for their purposes. Frequently, an index that is well suited for one purpose may not be well suited for other purposes. Users of indexes must be familiar with how various indexes are constructed in order to select the index or indexes most appropriate for their needs. Among the key points made in this reading are the following:

\begin{itemize}
  \item Security market indexes are intended to measure the values of different target markets (security markets, market segments, or asset classes).

  \item The constituent securities selected for inclusion in the security market index are intended to represent the target market.

  \item A price return index reflects only the prices of the constituent securities.

  \item A total return index reflects not only the prices of the constituent securities but also the reinvestment of all income received since the inception of the index.

  \item Methods used to weight the constituents of an index range from the very simple, such as price and equal weightings, to the more complex, such as market-capitalization and fundamental weightings.

  \item Choices in index construction-in particular, the choice of weighting method-affect index valuation and returns.

  \item Index management includes 1) periodic rebalancing to ensure that the index maintains appropriate weightings and 2) reconstitution to ensure the index represents the desired target market.

  \item Rebalancing and reconstitution create turnover in an index. Reconstitution can dramatically affect prices of current and prospective constituents.

  \item Indexes serve a variety of purposes. They gauge market sentiment and serve as benchmarks for actively managed portfolios. They act as proxies for measuring systematic risk and risk-adjusted performance. They also serve as proxies for asset classes in asset allocation models and as model portfolios for investment products.

  \item Investors can choose from security market indexes representing various asset classes, including equity, fixed-income, commodity, real estate, and hedge fund indexes.

  \item Within most asset classes, index providers offer a wide variety of indexes, ranging from broad market indexes to highly specialized indexes based on the issuer's geographic region, economic development group, or economic sector or other factors.

  \item Proper use of security market indexes depends on understanding their construction and management.

\end{itemize}

\section{PRACTICE PROBLEMS}
\begin{enumerate}
  \item A security market index represents the:
A. risk of a security market.
B. security market as a whole.
C. security market, market segment, or asset class.

  \item One month after inception, the price return version and total return version of a single index (consisting of identical securities and weights) will be equal if:
A. market prices have not changed.
B. capital gains are offset by capital losses.
C. the securities do not pay dividends or interest.

  \item The values of a price return index and a total return index consisting of identical equal-weighted dividend-paying equities will be equal:
A. only at inception.
B. at inception and on rebalancing dates.
C. at inception and on reconstitution dates.

  \item Security market indexes are:

\end{enumerate}

A. constructed and managed like a portfolio of securities.

B. simple interchangeable tools for measuring the returns of different asset classes.

C. valued on a regular basis using the actual market prices of the constituent securities.

\begin{enumerate}
  \setcounter{enumi}{4}
  \item When creating a security market index, an index provider must first determine the:
A. target market.
B. appropriate weighting method.
C. number of constituent securities.

  \item An analyst gathers the following information for an equal-weighted index comprised of assets Able, Baker, and Charlie:

\end{enumerate}

\begin{center}
\begin{tabular}{|c|c|c|c|}
\hline
Security & $\begin{array}{l}\text { Beginning of } \\ \text { Period Price }(€)\end{array}$ & $\begin{array}{l}\text { End of Period } \\ \text { Price }(€)\end{array}$ & $\begin{array}{c}\text { Total } \\ \text { Dividends ( } \epsilon)\end{array}$ \\
\hline
Able & 10.00 & 12.00 & 0.75 \\
\hline
Baker & 20.00 & 19.00 & 1.00 \\
\hline
Charlie & 30.00 & 30.00 & 2.00 \\
\hline
\end{tabular}
\end{center}

The price return of the index is: A. $1.7 \%$.

B. $5.0 \%$

C. $11.4 \%$.

\begin{enumerate}
  \setcounter{enumi}{6}
  \item An analyst gathers the following information for an equal-weighted index comprised of assets Able, Baker, and Charlie:
\end{enumerate}

\begin{center}
\begin{tabular}{lccc}
\hline
Security & $\begin{array}{c}\text { Beginning of } \\ \text { Period Price }(\boldsymbol{\epsilon})\end{array}$ & $\begin{array}{c}\text { End of Period } \\ \text { Price }(\boldsymbol{\epsilon})\end{array}$ & $\begin{array}{c}\text { Total } \\ \text { Dividends }(\boldsymbol{\epsilon})\end{array}$ \\
\hline
Able & 10.00 & 12.00 & 0.75 \\
Baker & 20.00 & 19.00 & 1.00 \\
Charlie & 30.00 & 30.00 & 2.00 \\
\hline
\end{tabular}
\end{center}

The total return of the index is:
A. $5.0 \%$.
B. $7.9 \%$.
C. $11.4 \%$.

\begin{enumerate}
  \setcounter{enumi}{7}
  \item An analyst gathers the following information for a price-weighted index comprised of securities ABC, DEF, and GHI:
\end{enumerate}

\begin{center}
\begin{tabular}{lccc}
\hline
Security & $\begin{array}{c}\text { Beginning of } \\ \text { Period Price }(\boldsymbol{\varepsilon})\end{array}$ & $\begin{array}{c}\text { End of Period } \\ \text { Price }(\boldsymbol{\varepsilon})\end{array}$ & $\begin{array}{c}\text { Total } \\ \text { Dividends }(\boldsymbol{f})\end{array}$ \\
\hline
ABC & 25.00 & 27.00 & 1.00 \\
DEF & 35.00 & 25.00 & 1.50 \\
GHI & 15.00 & 16.00 & 1.00 \\
\hline
\end{tabular}
\end{center}

The price return of the index is:
A. $-4.6 \%$
B. $-9.3 \%$.
C. $-13.9 \%$.

\begin{enumerate}
  \setcounter{enumi}{8}
  \item An analyst gathers the following information for a market-capitalization-weighted index comprised of securities MNO, QRS, and $\mathrm{XYZ}$
\end{enumerate}

\begin{center}
\begin{tabular}{|c|c|c|c|c|}
\hline
Security & $\begin{array}{l}\text { Beginning of } \\ \text { Period Price (¥) }\end{array}$ & $\begin{array}{c}\text { End of Period } \\ \text { Price ( } ¥ \text { ) }\end{array}$ & $\begin{array}{c}\text { Dividends } \\ \text { per Share (¥) }\end{array}$ & $\begin{array}{c}\text { Shares } \\ \text { Outstanding }\end{array}$ \\
\hline
$\mathrm{MNO}$ & 2,500 & 2,700 & 100 & 5,000 \\
\hline
QRS & 3,500 & 2,500 & 150 & 7,500 \\
\hline
$X Y Z$ & 1,500 & 1,600 & 100 & 10,000 \\
\hline
\end{tabular}
\end{center}

The price return of the index is:
A. $-9.33 \%$.
B. $-10.23 \%$. C. $-13.90 \%$.

\begin{enumerate}
  \setcounter{enumi}{9}
  \item An analyst gathers the following information for a market-capitalization-weighted index comprised of securities MNO, QRS, and XYZ:
\end{enumerate}

\begin{center}
\begin{tabular}{lcccc}
\hline
Security & $\begin{array}{c}\text { Beginning of } \\ \text { Period Price (¥) }\end{array}$ & $\begin{array}{c}\text { End of Period } \\ \text { Price (¥) }\end{array}$ & $\begin{array}{c}\text { Dividends } \\ \text { Per Share (¥) }\end{array}$ & $\begin{array}{c}\text { Shares } \\ \text { Outstanding }\end{array}$ \\
\hline
MNO & 2,500 & 2,700 & 100 & 5,000 \\
QRS & 3,500 & 2,500 & 150 & 7,500 \\
XYZ & 1,500 & 1,600 & 100 & 10,000 \\
\hline
\end{tabular}
\end{center}

The total return of the index is:
A. $1.04 \%$.
B. $-5.35 \%$.
C. $-10.23 \%$.

\begin{enumerate}
  \setcounter{enumi}{10}
  \item When creating a security market index, the target market:
A. determines the investment universe.
B. is usually a broadly defined asset class.
C. determines the number of securities to be included in the index.

  \item An analyst gathers the following data for a price-weighted index:

\end{enumerate}

\begin{center}
\begin{tabular}{|c|c|c|c|c|}
\hline
\multirow[b]{2}{*}{Security} & \multicolumn{2}{|c|}{Beginning of Period} & \multicolumn{2}{|c|}{End of Period} \\
\hline
 & Price (€) & $\begin{array}{c}\text { Shares } \\ \text { Outstanding }\end{array}$ & Price $(\epsilon)$ & $\begin{array}{c}\text { Shares } \\ \text { Outstanding }\end{array}$ \\
\hline
A & 20.00 & 300 & 22.00 & 300 \\
\hline
B & 50.00 & 300 & 48.00 & 300 \\
\hline
C & 26.00 & 2,000 & 30.00 & 2,000 \\
\hline
\end{tabular}
\end{center}

The price return of the index over the period is:
A. $4.2 \%$.
B. $7.1 \%$.
C. $21.4 \%$.

\begin{enumerate}
  \setcounter{enumi}{12}
  \item An analyst gathers the following data for a value-weighted index:
\end{enumerate}

\begin{center}
\begin{tabular}{lcccc}
\hline
 & \multicolumn{2}{c}{Beginning of Period} & \multicolumn{2}{c}{End of Period} \\
\hline
Security & Shares & Shares &  &  \\
\hline
A & Price $(\boldsymbol{\pounds})$ & Outstanding & Price (⿷) & Outstanding \\
B & 20.00 & 300 & 22.00 & 300 \\
 & 50.00 & 300 & 48.00 & 300 \\
\end{tabular}
\end{center}

Security Market Indexes

\begin{center}
\begin{tabular}{lcccc}
\hline
 & \multicolumn{2}{c}{Beginning of Period} & \multicolumn{2}{c}{End of Period} \\
\hline
Security & Shares & Shares &  &  \\
\hline
$\mathrm{C}$ & 26.00 & 2,000 & Price $(\boldsymbol{E})$ & Outstanding \\
\hline
\end{tabular}
\end{center}

The return on the value-weighted index over the period is:
A. $7.1 \%$.
B. $11.0 \%$.
C. $21.4 \%$.

\begin{enumerate}
  \setcounter{enumi}{13}
  \item An analyst gathers the following data for an equally-weighted index:
\end{enumerate}

\begin{center}
\begin{tabular}{lcccc}
\hline
 & \multicolumn{2}{c}{Beginning of Period} & \multicolumn{2}{c}{End of Period} \\
\hline
Security & Price (¥) & Outstanding & Price ( $¥$ ) & $\begin{array}{c}\text { Shares } \\ \text { Outstanding }\end{array}$ \\
\hline
A & 20.00 & 300 & 22.00 & 300 \\
B & 50.00 & 300 & 48.00 & 300 \\
C & 26.00 & 2,000 & 30.00 & 2,000 \\
\hline
\end{tabular}
\end{center}

The return on the index over the period is:
A. $4.2 \%$.
B. $6.8 \%$
C. $7.1 \%$.

\begin{enumerate}
  \setcounter{enumi}{14}
  \item Which of the following index weighting methods requires an adjustment to the divisor after a stock split?
\end{enumerate}

A. Price weighting.

B. Fundamental weighting.

C. Market-capitalization weighting.

\begin{enumerate}
  \setcounter{enumi}{15}
  \item If the price return of an equal-weighted index exceeds that of a market-capitalization-weighted index comprised of the same securities, the most likely explanation is:
\end{enumerate}

A. stock splits.

B. dividend distributions.

C. outperformance of small-market-capitalization stocks.

\begin{enumerate}
  \setcounter{enumi}{16}
  \item A float-adjusted market-capitalization-weighted index weights each of its constituent securities by its price and:
A. its trading volume.
B. the number of its shares outstanding.
C. the number of its shares available to the investing public. 18. Which of the following index weighting methods is most likely subject to a value tilt?
A. Equal weighting.
B. Fundamental weighting.
C. Market-capitalization weighting.

  \item Rebalancing an index is the process of periodically adjusting the constituent:
A. securities' weights to optimize investment performance.
B. securities to maintain consistency with the target market.
C. securities' weights to maintain consistency with the index's weighting method.

  \item Which of the following index weighting methods requires the most frequent rebalancing?
A. Price weighting.
B. Equal weighting.
C. Market-capitalization weighting.

  \item Reconstitution of a security market index reduces:
A. portfolio turnover.
B. the need for rebalancing.
C. the likelihood that the index includes securities that are not representative of the target market.

  \item Security market indexes are used as:
A. measures of investment returns.
B. proxies to measure unsystematic risk.
C. proxies for specific asset classes in asset allocation models.

  \item Uses of market indexes do not include serving as a:
A. measure of systemic risk.
B. basis for new investment products.
C. benchmark for evaluating portfolio performance.

  \item Which of the following statements regarding sector indexes is most accurate? Sector indexes:

\end{enumerate}

A. track different economic sectors and cannot be aggregated to represent the equivalent of a broad market index.

B. provide a means to determine whether an active investment manager is more successful at stock selection or sector allocation. C. apply a universally agreed upon sector classification system to identify the constituent securities of specific economic sectors, such as consumer goods, energy, finance, health care.

\begin{enumerate}
  \setcounter{enumi}{24}
  \item Which of the following is an example of a style index? An index based on:
A. geography.
B. economic sector.
C. market capitalization.

  \item Which of the following statements regarding fixed-income indexes is most accurate?

\end{enumerate}

A. Liquidity issues make it difficult for investors to easily replicate fixed-income indexes.

B. Rebalancing and reconstitution are the only sources of turnover in fixed-income indexes.

C. Fixed-income indexes representing the same target market hold similar numbers of bonds.

\begin{enumerate}
  \setcounter{enumi}{26}
  \item An aggregate fixed-income index:
\end{enumerate}

A. comprises corporate and asset-backed securities.

B. represents the market of government-issued securities.

C. can be subdivided by market or economic sector to create more narrowly defined indexes.

\begin{enumerate}
  \setcounter{enumi}{27}
  \item Fixed-income indexes are least likely constructed on the basis of:
A. maturity.
B. type of issuer.
C. coupon frequency.

  \item In comparison to equity indexes, the constituent securities of fixed-income indexes are:
A. more liquid.
B. easier to price.
C. drawn from a larger investment universe.

  \item Commodity index values are based on:
A. futures contract prices.
B. the market price of the specific commodity.
C. the average market price of a basket of similar commodities.

  \item Which of the following statements is most accurate?

\end{enumerate}

A. Commodity indexes all share similar weighting methods. B. Commodity indexes containing the same underlying commodities offer similar returns.

C. The performance of commodity indexes can be quite different from that of the underlying commodities.

\begin{enumerate}
  \setcounter{enumi}{31}
  \item Which of the following is not a real estate index category?
A. Appraisal index.
B. Initial sales index.
C. Repeat sales index.

  \item A unique feature of hedge fund indexes is that they:

\end{enumerate}

A. are frequently equal weighted.

B. are determined by the constituents of the index.

C. reflect the value of private rather than public investments.

\begin{enumerate}
  \setcounter{enumi}{33}
  \item The returns of hedge fund indexes are most likely:
A. biased upward.
B. biased downward.
C. similar across different index providers.
\end{enumerate}

\section{SOLUTIONS}
\begin{enumerate}
  \item C is correct. A security market index represents the value of a given security market, market segment, or asset class.

  \item $\mathrm{C}$ is correct. The difference between a price return index and a total return index consisting of identical securities and weights is the income generated over time by the underlying securities. If the securities in the index do not generate income, both indexes will be identical in value.

  \item A is correct. At inception, the values of the price return and total return versions of an index are equal.

  \item A is correct. Security market indexes are constructed and managed like a portfolio of securities.

  \item A is correct. The first decision is identifying the target market that the index is intended to represent because the target market determines the investment universe and the securities available for inclusion in the index.

  \item B is correct. The price return is the sum of the weighted returns of each security. The return of Able is 20 percent [(12 - 10)/10]; of Baker is -5 percent [(19 20)/20]; and of Charlie is 0 percent $[(30-30) / 30]$. The price return index assigns a weight of $1 / 3$ to each asset; therefore, the price return is $1 / 3 \times[20 \%+(-5 \%)+$ $0 \%]=5 \%$.

  \item $\mathrm{C}$ is correct. The total return of an index is calculated on the basis of the change in price of the underlying securities plus the sum of income received or the sum of the weighted total returns of each security. The total return of Able is 27.5 percent; of Baker is 0 percent; and of Charlie is 6.7 percent:

\end{enumerate}

Able: $(12-10+0.75) / 10=27.5 \%$

Baker: $(19-20+1) / 20=0 \%$

Charlie: $(30-30+2) / 30=6.7 \%$

An equal-weighted index applies the same weight $(1 / 3)$ to each security's return; therefore, the total return $=1 / 3 \times(27.5 \%+0 \%+6.7 \%)=11.4 \%$.

\begin{enumerate}
  \setcounter{enumi}{7}
  \item B is correct. The price return of the price-weighted index is the percentage change in price of the index: $(68-75) / 75=-9.33 \%$.
\end{enumerate}

\begin{center}
\begin{tabular}{|c|c|c|}
\hline
Security & $\begin{array}{c}\text { Beginning of Period } \\ \text { Price }(\boldsymbol{\varepsilon})\end{array}$ & $\begin{array}{c}\text { End of Perioc } \\ \text { Price }(\pounds)\end{array}$ \\
\hline
$\mathrm{ABC}$ & 25.00 & 27.00 \\
\hline
DEF & 35.00 & 25.00 \\
\hline
GHI & 15.00 & 16.00 \\
\hline
TOTAL & 75.00 & 68.00 \\
\hline
\end{tabular}
\end{center}

\begin{enumerate}
  \setcounter{enumi}{8}
  \item B is correct. The price return of the index is $(48,250,000-53,750,000) / 53,750,000$ $=-10.23 \%$.
\end{enumerate}

\begin{center}
\begin{tabular}{lccccc}
\hline
 & $\begin{array}{c}\text { Beginning } \\ \text { of Period } \\ \text { Price ( } ¥ \text { ) }\end{array}$ & $\begin{array}{c}\text { Shares } \\ \text { Outstanding }\end{array}$ & $\begin{array}{c}\text { Beginning } \\ \text { of Period } \\ \text { Value ( } ¥ \text { ) }\end{array}$ & $\begin{array}{c}\text { End of } \\ \text { Period } \\ \text { Price ( } ¥ \text { ) }\end{array}$ & $\begin{array}{c}\text { End of Period } \\ \text { Value ( } ¥ \text { ) }\end{array}$ \\
\hline
Security & 2,500 & 5,000 & $12,500,000$ & 2,700 & $13,500,000$ \\
QRS & 3,500 & 7,500 & $26,250,000$ & 2,500 & $18,750,000$ \\
XYZ & 1,500 & 10,000 & $\frac{15,000,000}{}$ & 1,600 & $16,000,000$ \\
Total &  & $53,750,000$ &  & $48,250,000$ &  \\
\hline
\end{tabular}
\end{center}

\begin{enumerate}
  \setcounter{enumi}{9}
  \item B is correct. The total return of the market-capitalization-weighted index is calculated below:
\end{enumerate}

\begin{center}
\begin{tabular}{|c|c|c|c|c|}
\hline
Security & $\begin{array}{c}\text { Beginning of } \\ \text { Period Value }(¥) \text { ) }\end{array}$ & $\begin{array}{c}\text { End of Period } \\ \text { Value ( } ¥)\end{array}$ & $\begin{array}{c}\text { Total } \\ \text { Dividends ( } ¥ \text { ) }\end{array}$ & $\begin{array}{l}\text { Total Return } \\ \text { (\%) }\end{array}$ \\
\hline
$\mathrm{MNO}$ & $12,500,000$ & $13,500,000$ & 500,000 & 12.00 \\
\hline
QRS & $26,250,000$ & $18,750,000$ & $1,125,000$ & -24.29 \\
\hline
XYZ & $15,000,000$ & $16,000,000$ & $1,000,000$ & 13.33 \\
\hline
Total & $53,750,000$ & $48,250,000$ & $2,625,000$ & -5.35 \\
\hline
\end{tabular}
\end{center}

\begin{enumerate}
  \setcounter{enumi}{10}
  \item A is correct. The target market determines the investment universe and the securities available for inclusion in the index.

  \item A is correct. The sum of prices at the beginning of the period is 96 ; the sum at the end of the period is 100. Regardless of the divisor, the price return is 100/96 $-1=$ 0.042 or 4.2 percent.

  \item B is correct. It is the percentage change in the market value over the period:

\end{enumerate}

Market value at beginning of period: $(20 \times 300)+(50 \times 300)+(26 \times 2,000)$ $=73,000$

Market value at end of period: $(22 \times 300)+(48 \times 300)+(30 \times 2,000)=$ 81,000

Percentage change is $81,000 / 73,000-1=0.1096$ or 11.0 percent with rounding.

\begin{enumerate}
  \setcounter{enumi}{13}
  \item $\mathrm{C}$ is correct. With an equal-weighted index, the same amount is invested in each security. Assuming $\$ 1,000$ is invested in each of the three stocks, the index value is $\$ 3,000$ at the beginning of the period and the following number of shares is purchased for each stock:
\end{enumerate}

Security A: 50 shares

Security B: 20 shares

Security C: 38.46 shares.

Using the prices at the beginning of the period for each security, the index value at the end of the period is $\$ 3,213.8$ : $(\$ 22 \times 50)+(\$ 48 \times 20)+(\$ 30 \times 38.46)$. The price return is $\$ 3,213.8 / \$ 3,000-1=7.1 \%$.

\begin{enumerate}
  \setcounter{enumi}{14}
  \item A is correct. In the price weighting method, the divisor must be adjusted so the index value immediately after the split is the same as the index value immediately prior to the split.

  \item $\mathrm{C}$ is correct. The main source of return differences arises from outperformance of small-cap securities or underperformance of large-cap securities. In an equal-weighted index, securities that constitute the largest fraction of the market are underrepresented and securities that constitute only a small fraction of the market are overrepresented. Thus, higher equal-weighted index returns will occur if the smaller-cap equities outperform the larger-cap equities.

  \item $C$ is correct. "Float" is the number of shares available for public trading.

  \item B is correct. Fundamental weighting leads to indexes that have a value tilt.

  \item $C$ is correct. Rebalancing refers to adjusting the weights of constituent securities in an index to maintain consistency with the index's weighting method.

  \item B is correct. Changing market prices will cause weights that were initially equal to become unequal, thus requiring rebalancing.

  \item $\mathrm{C}$ is correct. Reconstitution is the process by which index providers review the constituent securities, re-apply the initial criteria for inclusion in the index, and select which securities to retain, remove, or add. Constituent securities that no longer meet the criteria are replaced with securities that do. Thus, reconstitution reduces the likelihood that the index includes securities that are not representative of the target market.

  \item C is correct. Security market indexes play a critical role as proxies for asset classes in asset allocation models.

  \item A is correct. Security market indexes are used as proxies for measuring market or systematic risk, not as measures of systemic risk.

  \item B is correct. Sector indexes provide a means to determine whether a portfolio manager is more successful at stock selection or sector allocation.

  \item C is correct. Style indexes represent groups of securities classified according to market capitalization, value, growth, or a combination of these characteristics.

  \item A is correct. The large number of fixed-income securities-combined with the lack of liquidity of some securities-makes it costly and difficult for investors to replicate fixed-income indexes.

  \item $C$ is correct. An aggregate fixed-income index can be subdivided by market sector (government, government agency, collateralized, corporate), style (maturity, credit quality), economic sector, or some other characteristic to create more narrowly defined indexes.

  \item $\mathrm{C}$ is correct. Coupon frequency is not a dimension on which fixed-income indexes are based.

  \item $\mathrm{C}$ is correct. The fixed-income market has more issuers and securities than the equity market.

  \item A is correct. Commodity indexes consist of futures contracts on one or more commodities.

  \item C is correct. The performance of commodity indexes can be quite different from that of the underlying commodities because the indexes consist of futures contracts on the commodities rather than the actual commodities.

  \item B is correct. It is not a real estate index category. 33. B is correct. Hedge funds are not required to report their performance to any party other than their investors. Therefore, each hedge fund decides to which database(s) it will report its performance. Thus, for a hedge fund index, constituents determine the index rather than index providers determining the constituents.

  \item A is correct. Voluntary performance reporting may lead to survivorship bias, and poorer performing hedge funds will be less likely to report their performance.

\end{enumerate}

\textbackslash section\{LEARNING MODULE

\begin{center}
\includegraphics[max width=\textwidth]{2023_05_04_7b535d0a870224f62e3dg-259(1)}
\end{center}

\section{Market Efficiency}
by Sean Cleary, PhD, CFA, Howard J. Atkinson, CIMA, ICD.D, CFA, and Pamela Peterson Drake, PhD, CFA.

Sean Cleary, PhD, CFA, is at Queen's University (Canada). Howard J. Atkinson, CIMA, ICD.D, CFA, is at Horizons ETF Management (Canada) Inc. (Canada). Pamela Peterson Drake, PhD, CFA, is at James Madison University (USA).

\section{LEARNING OUTCOME}
\begin{center}
\begin{tabular}{|c|c|}
\hline
Mastery & The candidate should be able to: \\
\hline
$\square$ & $\begin{array}{l}\text { describe market efficiency and related concepts, including their } \\ \text { importance to investment practitioners }\end{array}$ \\
\hline
$\square$ & contrast market value and intrinsic value \\
\hline
$\square$ & explain factors that affect a market's efficiency \\
\hline
$\square$ & $\begin{array}{l}\text { contrast weak-form, semi-strong-form, and strong-form market } \\ \text { efficiency }\end{array}$ \\
\hline
[ & $\begin{array}{l}\text { explain the implications of each form of market efficiency for } \\ \text { fundamental analysis, technical analysis, and the choice between } \\ \text { active and passive portfolio management }\end{array}$ \\
\hline
\includegraphics[max width=\textwidth]{2023_05_04_7b535d0a870224f62e3dg-259}
 & describe market anomalies \\
\hline
 & $\begin{array}{l}\text { describe behavioral finance and its potential relevance to } \\ \text { understanding market anomalies }\end{array}$ \\
\hline
\end{tabular}
\end{center}

\section{INTRODUCTION}
Market efficiency concerns the extent to which market prices incorporate available information. If market prices do not fully incorporate information, then opportunities may exist to make a profit from the gathering and processing of information. The subject of market efficiency is, therefore, of great interest to investment managers, as illustrated in Example 1.

\section{EXAMPLE 1}
\section{Market Efficiency and Active Manager Selection}
\begin{enumerate}
  \item The chief investment officer $(\mathrm{CIO})$ of a major university endowment fund has listed eight steps in the active manager selection process that can be applied both to traditional investments (e.g., common equity and fixed-income securities) and to alternative investments (e.g., private equity, hedge funds, and real assets). The first step specified is the evaluation of market opportunity:
\end{enumerate}

What is the opportunity and why is it there? To answer this question, we start by studying capital markets and the types of managers operating within those markets. We identify market inefficiencies and try to understand their causes, such as regulatory structures or behavioral biases. We can rule out many broad groups of managers and strategies by simply determining that the degree of market inefficiency necessary to support a strategy is implausible. Importantly, we consider the past history of active returns meaningless unless we understand why markets will allow those active returns to continue into the future. ${ }^{1}$

The CIO's description underscores the importance of not assuming that past active returns that might be found in a historical dataset will repeat themselves in the future. Active returns refer to returns earned by strategies that do not assume that all information is fully reflected in market prices.

Governments and market regulators also care about the extent to which market prices incorporate information. Efficient markets imply informative prices-prices that accurately reflect available information about fundamental values. In market-based economies, market prices help determine which companies (and which projects) obtain capital. If these prices do not efficiently incorporate information about a company's prospects, then it is possible that funds will be misdirected. By contrast, prices that are informative help direct scarce resources and funds available for investment to their highest-valued uses. ${ }^{2}$ Informative prices thus promote economic growth. The efficiency of a country's capital markets (in which businesses raise financing) is an important characteristic of a well-functioning financial system.

The remainder of this reading is organized as follows. Section 2 provides specifics on how the efficiency of an asset market is described and discusses the factors affecting (i.e., contributing to and impeding) market efficiency. Section 3 presents an influential three-way classification of the efficiency of security markets and discusses its implications for fundamental analysis, technical analysis, and portfolio management. Section 4 presents several market anomalies (apparent market inefficiencies that have received enough attention to be individually identified and named) and describes how these anomalies relate to investment strategies. Section 5 introduces behavioral finance and how that field of study relates to market efficiency. A summary concludes the reading.

1 The CIO is Christopher J. Brightman, CFA, of the University of Virginia Investment Management Company, as reported in Yau, Schneeweis, Robinson, and Weiss (2007, pp. 481-482).

2 This concept is known as allocative efficiency.

\section*{THE CONCEPT OF MARKET EFFICIENCY }


\section{The Description of Efficient Markets}
An informationally efficient market (an efficient market) is a market in which asset prices reflect new information quickly and rationally. An efficient market is thus a market in which asset prices reflect all past and present information. ${ }^{3}$

In this section we expand on this definition by clarifying the time frame required for an asset's price to incorporate information as well as describing the elements of information releases assumed under market efficiency. We discuss the difference between market value and intrinsic value and illustrate how inefficiencies or discrepancies between these values can provide profitable opportunities for active investment. As financial markets are generally not considered being either completely efficient or inefficient, but rather falling within a range between the two extremes, we describe a number of factors that contribute to and impede the degree of efficiency of a financial market. Finally, we conclude our overview of market efficiency by illustrating how the costs incurred by traders in identifying and exploiting possible market inefficiencies affect how we interpret market efficiency.

Investment managers and analysts, as noted, are interested in market efficiency because the extent to which a market is efficient affects how many profitable trading opportunities (market inefficiencies) exist. Consistent, superior, risk-adjusted returns (net of all expenses) are not achievable in an efficient market. ${ }^{4}$ In an efficient market, a passive investment strategy (i.e., buying and holding a broad market portfolio) that does not seek superior risk-adjusted returns can be preferred to an active investment strategy because of lower costs (for example, transaction and information-seeking costs). By contrast, in a very inefficient market, opportunities may exist for an active investment strategy to achieve superior risk-adjusted returns (net of all expenses in executing the strategy) as compared with a passive investment strategy. In inefficient markets, an active investment strategy may outperform a passive investment strategy on a risk-adjusted basis. Understanding the characteristics of an efficient market and being able to evaluate the efficiency of a particular market are important topics for investment analysts and portfolio managers.

An efficient market is a market in which asset prices reflect information quickly. But what is the time frame of "quickly"? Trades are the mechanism by which information can be incorporated into asset transaction prices. The time needed to execute trades to exploit an inefficiency may provide a baseline for judging speed of adjustment. ${ }^{5}$

3 This definition is convenient for making several instructional points. The definition that most simply explains the sense of the word efficient in this context can be found in Fama (1976): "An efficient capital market is a market that is efficient in processing information" (p. 134).

\begin{enumerate}
  \setcounter{enumi}{3}
  \item The technical term for superior in this context is positive abnormal in the sense of higher than expected given the asset's risk (as measured, according to capital market theory, by the asset's contribution to the risk of a well-diversified portfolio).
\end{enumerate}

5 Although the original theory of market efficiency does not quantify this speed, the basic idea is that it is sufficiently swift to make it impossible to consistently earn abnormal profits. Chordia, Roll, and Subrahmanyam (2005) suggest that the adjustment to information on the New York Stock Exchange (NYSE) is between 5 and 60 minutes The time frame for an asset's price to incorporate information must be at least as long as the shortest time a trader needs to execute a transaction in the asset. In certain markets, such as foreign exchange and developed equity markets, market efficiency relative to certain types of information has been studied using time frames as short as one minute or less. If the time frame of price adjustment allows many traders to earn profits with little risk, then the market is relatively inefficient. These considerations lead to the observation that market efficiency can be viewed as falling on a continuum.

Finally, an important point is that in an efficient market, prices should be expected to react only to the elements of information releases that are not anticipated fully by investors-that is, to the "unexpected" or "surprise" element of such releases. Investors process the unexpected information and revise expectations (for example, about an asset's future cash flows, risk, or required rate of return) accordingly. The revised expectations enter or get incorporated in the asset price through trades in the asset. Market participants who process the news and believe that at the current market price an asset does not offer sufficient compensation for its perceived risk will tend to sell it or even sell it short. Market participants with opposite views should be buyers. In this way the market establishes the price that balances the various opinions after expectations are revised.

\section{EXAMPLE 2}
\section{Price Reaction to the Default on a Bond Issue}
Suppose that a speculative-grade bond issuer announces, just before bond markets open, that it will default on an upcoming interest payment. In the announcement, the issuer confirms various reports made in the financial media in the period leading up to the announcement. Prior to the issuer's announcement, the financial news media reported the following: 1) suppliers of the company were making deliveries only for cash payment, reducing the company's liquidity; 2) the issuer's financial condition had probably deteriorated to the point that it lacked the cash to meet an upcoming interest payment; and 3) although public capital markets were closed to the company, it was negotiating with a bank for a private loan that would permit it to meet its interest payment and continue operations for at least nine months. If the issuer defaults on the bond, the consensus opinion of analysts is that bondholders will recover approximately $\$ 0.36$ to $\$ 0.38$ per dollar face value.

\begin{enumerate}
  \item If the market for the bond is efficient, the bond's market price is most likely to fully reflect the bond's value after default:
\end{enumerate}

A. in the period leading up to the announcement.

B. in the first trade prices after the market opens on the announcement day.

C. when the issuer actually misses the payment on the interest payment date.

\section{Solution:}
$\mathrm{B}$ is correct. The announcement removed any uncertainty about default. In the period leading up to the announcement, the bond's market price incorporated a probability of default, but the price would not have fully reflected the bond's value after default. The possibility that a bank loan might permit the company to avoid default was not eliminated until the announcement. 2. If the market for the bond is efficient, the piece of information that bond investors most likely focus on in the issuer's announcement is that the issuer had:

A. failed in its negotiations for a bank loan.

B. lacked the cash to meet the upcoming interest payment.

C. been required to make cash payments for supplier deliveries.

\section{Solution:}
A is correct. The failure of the loan negotiations first becomes known in this announcement. The failure implies default.

\section{Market Value versus Intrinsic Value}
Market value is the price at which an asset can currently be bought or sold. Intrinsic value (sometimes called fundamental value) is, broadly speaking, the value that would be placed on it by investors if they had a complete understanding of the asset's investment characteristics. ${ }^{6}$ For a bond, for example, such information would include its interest (coupon) rate, principal value, the timing of its interest and principal payments, the other terms of the bond contract (indenture), a precise understanding of its default risk, the liquidity of its market, and other issue-specific items. In addition, market variables such as the term structure of interest rates and the size of various market premiums applying to the issue (for default risk, etc.) would enter into a discounted cash flow estimate of the bond's intrinsic value (discounted cash flow models are often used for such estimates). The word estimate is used because in practice, intrinsic value can be estimated but is not known for certain.

If investors believe a market is highly efficient, they will usually accept market prices as accurately reflecting intrinsic values. Discrepancies between market price and intrinsic value are the basis for profitable active investment. Active investors seek to own assets selling below perceived intrinsic value in the marketplace and to sell or sell short assets selling above perceived intrinsic value.

If investors believe an asset market is relatively inefficient, they may try to develop an independent estimate of intrinsic value. The challenge for investors and analysts is estimating an asset's intrinsic value. Numerous theories and models, including the dividend discount model, can be used to estimate an asset's intrinsic value, but they all require some form of judgment regarding the size, timing, and riskiness of the future cash flows associated with the asset. The more complex an asset's future cash flows, the more difficult it is to estimate its intrinsic value. These complexities and the estimates of an asset's market value are reflected in the market through the buying and selling of assets. The market value of an asset represents the intersection of supply and demand - the point that is low enough to induce at least one investor to buy while being high enough to induce at least one investor to sell. Because information relevant to valuation flows continually to investors, estimates of intrinsic value change, and hence, market values change.

6 Intrinsic value is often defined as the present value of all expected future cash flows of the asset.

\section{EXAMPLE 3}
\section{Intrinsic Value}
\begin{enumerate}
  \item An analyst estimates that a security's intrinsic value is lower than its market value. The security appears to be:
A. undervalued.
B. fairly valued.
C. overvalued.
\end{enumerate}

Solution:

$\mathrm{C}$ is correct. The market is valuing the asset at more than its true worth.

\begin{enumerate}
  \setcounter{enumi}{1}
  \item A market in which assets' market values are, on average, equal to or nearly equal to intrinsic values is best described as a market that is attractive for:
A. active investment.
B. passive investment.
C. both active and passive investment.
\end{enumerate}

Solution:

$B$ is correct because an active investment is not expected to earn superior risk-adjusted returns if the market is efficient. The additional costs of active investment are not justified in such a market.

\begin{enumerate}
  \setcounter{enumi}{2}
  \item Suppose that the future cash flows of an asset are accurately estimated. The asset trades in a market that you believe is efficient based on most evidence, but your estimate of the asset's intrinsic value exceeds the asset's market value by a moderate amount. The most likely conclusion is that you have:
\end{enumerate}

A. overestimated the asset's risk.

B. underestimated the asset's risk.

C. identified a market inefficiency.

Solution:

$\mathrm{B}$ is correct. If risk is underestimated, the discount rate being applied to find the present value of the expected cash flows (estimated intrinsic value) will be too low and the intrinsic value estimate will be too high.

FACTORS AFFECTING MARKET EFFICIENCY INCLUDING TRADING COSTS

describe market efficiency and related concepts, including their importance to investment practitioners

explain factors that affect a market's efficiency For markets to be efficient, prices should adjust quickly and rationally to the release of new information. In other words, prices of assets in an efficient market should "fully reflect" all information. Financial markets, however, are generally not classified at the two extremes as either completely inefficient or completely efficient but, rather, as exhibiting various degrees of efficiency. In other words, market efficiency should be viewed as falling on a continuum between extremes of completely efficient, at one end, and completely inefficient, at the other. Asset prices in a highly efficient market, by definition, reflect information more quickly and more accurately than in a less-efficient market. These degrees of efficiency also vary through time, across geographical markets, and by type of market. A number of factors contribute to and impede the degree of efficiency in a financial market.

\section{Market Participants}
One of the most critical factors contributing to the degree of efficiency in a market is the number of market participants. Consider the following example that illustrates the relationship between the number of market participants and market efficiency.

\section{EXAMPLE 4}
\section{Illustration of Market Efficiency}
Assume that the shares of a small market capitalization (cap) company trade on a public stock exchange. Because of its size, it is not considered "blue-chip" and not many professional investors follow the activities of the company. ${ }^{7} \mathrm{~A}$ small-cap fund analyst reports that the most recent annual operating performance of the company has been surprisingly good, considering the recent slump in its industry. The company's share price, however, has been slow to react to the positive financial results because the company is not being recommended by the majority of research analysts. This mispricing implies that the market for this company's shares is less than fully efficient. The small-cap fund analyst recognizes the opportunity and immediately recommends the purchase of the company's shares. The share price gradually increases as more investors purchase the shares once the news of the mispricing spreads through the market. As a result, it takes a few days for the share price to fully reflect the information.

Six months later, the company reports another solid set of interim financial results. But because the previous mispricing and subsequent profit opportunities have become known in the market, the number of analysts following the company's shares has increased substantially. As a result, as soon as unexpected information about the positive interim results are released to the public, a large number of buy orders quickly drive up the stock price, thereby making the market for these shares more efficient than before.

A large number of investors (individual and institutional) follow the major financial markets closely on a daily basis, and if mispricings exist in these markets, as illustrated by the example, investors will act so that these mispricings disappear quickly. Besides the number of investors, the number of financial analysts who follow or analyze a security or asset should be positively related to market efficiency. The number of market participants and resulting trading activity can vary significantly through time. A lack of trading activity can cause or accentuate other market imperfections that impede market efficiency. In fact, in many of these markets, trading in many of

7 A "blue-chip" share is one from a well-recognized company that is considered to be high quality but low risk. This term generally refers to a company that has a long history of earnings and paying dividends the listed stocks is restricted for foreigners. By nature, this limitation reduces the number of market participants, restricts the potential for trading activity, and hence reduces market efficiency.

\section{EXAMPLE 5}
\section{Factors Affecting Market Efficiency}
\begin{enumerate}
  \item The expected effect on market efficiency of opening a securities market to trading by foreigners would most likely be to:
\end{enumerate}

A. decrease market efficiency.

B. leave market efficiency unchanged.

C. increase market efficiency.

\section{Solution:}
$\mathrm{C}$ is correct. The opening of markets as described should increase market efficiency by increasing the number of market participants.

\section{Information Availability and Financial Disclosure}
Information availability (e.g., an active financial news media) and financial disclosure should promote market efficiency. Information regarding trading activity and traded companies in such markets as the New York Stock Exchange, the London Stock Exchange, and the Tokyo Stock Exchange is readily available. Many investors and analysts participate in these markets, and analyst coverage of listed companies is typically substantial. As a result, these markets are quite efficient. In contrast, trading activity and material information availability may be lacking in smaller securities markets, such as those operating in some emerging markets.

Similarly, significant differences may exist in the efficiency of different types of markets. For example, many securities trade primarily or exclusively in dealer or over-the-counter (OTC) markets, including bonds, money market instruments, currencies, mortgage-backed securities, swaps, and forward contracts. The information provided by the dealers that serve as market makers for these markets can vary significantly in quality and quantity, both through time and across different product markets.

Treating all market participants fairly is critical for the integrity of the market and explains why regulators place such an emphasis on "fair, orderly, and efficient markets." ${ }^{8}$ A key element of this fairness is that all investors have access to the information necessary to value securities that trade in the market. Rules and regulations that promote fairness and efficiency in a market include those pertaining to the disclosure of information and illegal insider trading.

For example, US Securities and Exchange Commission's (SEC's) Regulation FD (Fair Disclosure) requires that if security issuers provide nonpublic information to some market professionals or investors, they must also disclose this information to the public. ${ }^{9}$ This requirement helps provide equal and fair opportunities, which is important in encouraging participation in the market. A related issue deals with illegal insider trading. The SEC's rules, along with court cases, define illegal insider trading as trading in securities by market participants who are considered insiders "while in

8 "The Investor's Advocate: How the SEC Protects Investors, Maintains Market Integrity, and Facilitates Capital Formation," US Securities and Exchange Commission (\href{http://www.sec.gov/about/whatwedo.shtml}{www.sec.gov/about/whatwedo.shtml}).

9 Regulation FD, "Selective Disclosure and Insider Trading," 17 CFR Parts 240, 243, and 249, effective 23 October 2000 possession of material, nonpublic information about the security. ${ }^{10}$ Although these rules cannot guarantee that some participants will not have an advantage over others and that insiders will not trade on the basis of inside information, the civil and criminal penalties associated with breaking these rules are intended to discourage illegal insider trading and promote fairness. In the European Union, insider trading laws are generally enshrined in legislation and enforced by regulatory and judicial authorities. ${ }^{11}$

\section{Limits to Trading}
Arbitrage is a set of transactions that produces riskless profits. Arbitrageurs are traders who engage in such trades to benefit from pricing discrepancies (inefficiencies) in markets. Such trading activity contributes to market efficiency. For example, if an asset is traded in two markets but at different prices, the actions of buying the asset in the market in which it is underpriced and selling the asset in the market in which it is overpriced will eventually bring these two prices together. The presence of these arbitrageurs helps pricing discrepancies disappear quickly. Obviously, market efficiency is impeded by any limitation on arbitrage resulting from operating inefficiencies, such as difficulties in executing trades in a timely manner, prohibitively high trading costs, and a lack of transparency in market prices.

Some market experts argue that restrictions on short selling limit arbitrage trading, which impedes market efficiency. Short selling is the transaction whereby an investor sells shares that he or she does not own by borrowing them from a broker and agreeing to replace them at a future date. Short selling allows investors to sell securities they believe to be overvalued, much in the same way they can buy those they believe to be undervalued. In theory, such activities promote more efficient pricing. Regulators and others, however, have argued that short selling may exaggerate downward market movements, leading to crashes in affected securities. In contrast, some researchers report evidence indicating that when investors are unable to borrow securities, that is to short the security, or when costs to borrow shares are high, market prices may deviate from intrinsic values. ${ }^{12}$ Furthermore, research suggests that short selling is helpful in price discovery (that is, it facilitates supply and demand in determining prices). ${ }^{13}$

\section{Transaction Costs and Information-Acquisition Costs}
The costs incurred by traders in identifying and exploiting possible market inefficiencies affect the interpretation of market efficiency. The two types of costs to consider are transaction costs and information-acquisition costs.

\begin{itemize}
  \item Transaction costs: Practically, transaction costs are incurred in trading to exploit any perceived market inefficiency. Thus, "efficient" should be viewed as efficient within the bounds of transaction costs. For example, consider a violation of the principle that two identical assets should sell for the same price in different markets. Such a violation can be considered to be a rather simple possible exception to market efficiency because prices appear to
\end{itemize}

10 Although not the focus of this particular reading, it is important to note that a party is considered an insider not only when the individual is a corporate insider, such as an officer or director, but also when the individual is aware that the information is nonpublic information [Securities and Exchange Commission, Rules 10b5-1 ("Trading on the Basis of Material Nonpublic Information in Insider Trading Cases") and Rule 10b5-2 "Duties of Trust or Confidence in Misappropriation Insider Trading Cases")].

11 See the European Union's Market Abuse Regulation (Regulation (EU) no. 596/2014 of the European Parliament and of the Council of 16 April 2014 on market abuse) and Directive for Criminal Sanctions for Market Abuse (Directive 2014/57/EU of the European Parliament and of the Council of 16 April 2014 on criminal sanctions for market abuse).

12 See Deng, Mortal, and Gupta (2017) and references therein."

13 See Bris, Goetzmann, and Zhu (2009) be inconsistently processing information. To exploit the violation, a trader could arbitrage by simultaneously shorting the asset in the higher-price market and buying the asset in the lower-price market. If the price discrepancy between the two markets is smaller than the transaction costs involved in the arbitrage for the lowest cost traders, the arbitrage will not occur, and both prices are in effect efficient within the bounds of arbitrage. These bounds of arbitrage are relatively narrow in highly liquid markets, such as the market for US Treasury bills, but could be wide in illiquid markets.

\begin{itemize}
  \item Information-acquisition costs: Practically, expenses are always associated with gathering and analyzing information. New information is incorporated in transaction prices by traders placing trades based on their analysis of information. Active investors who place trades based on information they have gathered and analyzed play a key role in market prices adjusting to reflect new information. The classic view of market efficiency is that active investors incur information acquisition costs but that money is wasted because prices already reflect all relevant information. This view of efficiency is very strict in the sense of viewing a market as inefficient if active investing can recapture any part of the costs, such as research costs and active asset selection. Grossman and Stiglitz (1980) argue that prices must offer a return to information acquisition; in equilibrium, if markets are efficient, returns net of such expenses are just fair returns for the risk incurred. The modern perspective views a market as inefficient if, after deducting such costs, active investing can earn superior returns. Gross of expenses, a return should accrue to information acquisition in an efficient market.
\end{itemize}

In summary, a modern perspective calls for the investor to consider transaction costs and information-acquisition costs when evaluating the efficiency of a market. A price discrepancy must be sufficiently large to leave the investor with a profit (adjusted for risk) after taking account of the transaction costs and information-acquisition costs to reach the conclusion that the discrepancy may represent a market inefficiency. Prices may somewhat less than fully reflect available information without there being a true market opportunity for active investors.

\section{FORMS OF MARKET EFFICIENCY}
contrast weak-form, semi-strong-form, and strong-form market efficiency

Eugene Fama developed a framework for describing the degree to which markets are efficient. ${ }^{14}$ In his efficient market hypothesis, markets are efficient when prices reflect all relevant information at any point in time. This means that the market prices observed for securities, for example, reflect the information available at the time.

In his framework, Fama defines three forms of efficiency: weak, semi-strong, and strong. Each form is defined with respect to the available information that is reflected in prices.

\begin{center}
\begin{tabular}{lccc}
\hline
 & \multicolumn{2}{c}{Market Prices Reflect:} &  \\
\hline
Forms of Market Efficiency & $\begin{array}{c}\text { Past } \\ \text { Market Data }\end{array}$ & $\begin{array}{c}\text { Public } \\ \text { Information }\end{array}$ & $\begin{array}{c}\text { Private } \\ \text { Information }\end{array}$ \\
\hline
$\begin{array}{l}\text { Weak form of market efficiency } \\ \text { Semi-strong form of market } \\ \text { efficiency }\end{array}$ & $\checkmark$ & $\checkmark$ & $\checkmark$ \\
Strong form of market efficiency & $\checkmark$ & $\checkmark$ &  \\
\hline
\end{tabular}
\end{center}

A finding that investors can consistently earn abnormal returns by trading on the basis of information is evidence contrary to market efficiency. In general, abnormal returns are returns in excess of those expected given a security's risk and the market's return. In other words, abnormal return equals actual return less expected return.

\section{Weak Form}
In the weak-form efficient market hypothesis, security prices fully reflect all past market data, which refers to all historical price and trading volume information. If markets are weak-form efficient, past trading data are already reflected in current prices and investors cannot predict future price changes by extrapolating prices or patterns of prices from the past. ${ }^{15}$

Tests of whether securities markets are weak-form efficient require looking at patterns of prices. One approach is to see whether there is any serial correlation in security returns, which would imply a predictable pattern. ${ }^{16}$ Although there is some weak correlation in daily security returns, there is not enough correlation to make this a profitable trading rule after considering transaction costs.

An alternative approach to test weak-form efficiency is to examine specific trading rules that attempt to exploit historical trading data. If any such trading rule consistently generates abnormal risk-adjusted returns after trading costs, this evidence will contradict weak-form efficiency. This approach is commonly associated with technical analysis, which involves the analysis of historical trading information (primarily pricing and volume data) in an attempt to identify recurring patterns in the trading data

14 Fama (1970).

15 Market efficiency should not be confused with the random walk hypothesis, in which price changes over time are independent of one another. A random walk model is one of many alternative expected return generating models. Market efficiency does not require that returns follow a random walk.

16 Serial correlation is a statistical measure of the degree to which the returns in one period are related to the returns in another period. that can be used to guide investment decisions. Many technical analysts, also referred to as "technicians," argue that many movements in stock prices are based, in large part, on psychology. Many technicians attempt to predict how market participants will behave, based on analyses of past behavior, and then trade on those predictions. Technicians often argue that simple statistical tests of trading rules are not conclusive because they are not applied to the more sophisticated trading strategies that can be used and that the research excludes the technician's subjective judgment. Thus, it is difficult to definitively refute this assertion because there are an unlimited number of possible technical trading rules.

Can technical analysts profit from trading on past trends? Overall, the evidence indicates that investors cannot consistently earn abnormal profits using past prices or other technical analysis strategies in developed markets. ${ }^{17}$ Some evidence suggests, however, that there are opportunities to profit on technical analysis in countries with developing markets, including Hungary, Bangladesh, and Turkey, among others. ${ }^{18}$

\section{Semi-Strong Form}
In a semi-strong-form efficient market, prices reflect all publicly known and available information. Publicly available information includes financial statement data (such as earnings, dividends, corporate investments, changes in management, etc.) and financial market data (such as closing prices, shares traded, etc.). Therefore, the semi-strong form of market efficiency encompasses the weak form. In other words, if a market is semi-strong efficient, then it must also be weak-form efficient. A market that quickly incorporates all publicly available information into its prices is semi-strong efficient.

In a semi-strong market, efforts to analyze publicly available information are futile. That is, analyzing earnings announcements of companies to identify underpriced or overpriced securities is pointless because the prices of these securities already reflect all publicly available information. If markets are semi-strong efficient, no single investor has access to information that is not already available to other market participants, and as a consequence, no single investor can gain an advantage in predicting future security prices. In a semi-strong efficient market, prices adjust quickly and accurately to new information. Suppose a company announces earnings that are higher than expected. In a semi-strong efficient market, investors would not be able to act on this announcement and earn abnormal returns.

A common empirical test of investors' reaction to information releases is the event study. Suppose a researcher wants to test whether investors react to the announcement that the company is paying a special dividend. The researcher identifies a sample period and then those companies that paid a special dividend in the period and the date of the announcement. Then, for each company's stock, the researcher calculates the expected return on the share for the event date. This expected return may be based on many different models, including the capital asset pricing model, a simple market model, or a market index return. The researcher calculates the excess return as the difference between the actual return and the expected return. Once the researcher has calculated the event's excess return for each share, statistical tests are conducted to see whether the abnormal returns are statistically different from zero. The process of an event study is outlined in Exhibit 1.

17 Bessembinder and Chan (1998) and Fifield, Power, and Sinclair (2005).

18 Fifield, Power, and Sinclair (2005), Chen and Li (2006), and Mobarek, Mollah, and Bhuyan (2008).

\section{Exhibit 1: The Event Study Process}
Identify the period of study

\begin{center}
\includegraphics[max width=\textwidth]{2023_05_04_7b535d0a870224f62e3dg-271}
\end{center}

How do event studies relate to efficient markets? In a semi-strong efficient market, share prices react quickly and accurately to public information. Therefore, if the information is good news, such as better-than-expected earnings, one would expect the company's shares to increase immediately at the time of the announcement; if it is bad news, one would expect a swift, negative reaction. If actual returns exceed what is expected in absence of the announcement and these returns are confined to the announcement period, then they are consistent with the idea that market prices react quickly to new information. In other words, the finding of excess returns at the time of the announcement does not necessarily indicate market inefficiency. In contrast, the finding of consistent excess returns following the announcement would suggest a trading opportunity. Trading on the basis of the announcement-that is, once the announcement is made-would not, on average, yield abnormal returns.

\section{EXAMPLE 6}
\section{Information Arrival and Market Reaction}
Consider an example of a news item and its effect on a share's price. The following events related to Tesla, Inc. in August of 2018 :

1 August $2018 \quad$ After the market closes, Tesla, Inc., publicly reports that there was a smaller-than expected cash burn for the most recent quarter.

2 August $2018 \quad$ Elon Musk, Chairman and CEO of Tesla, Inc., notifies Tesla's board of directors that he wants to take the company private. This is not public information at this point.

7 August $2018 \quad$ Before the market opens, the Financial Times reports that a Saudi fund has a $\$ 2$ billion investment in Tesla.

During market trading, Musk announces on Twitter "Am considering taking Tesla private at $\$ 420$. Funding secured." [Twitter, Elon Musk @elonmusk, 9:48 a.m., 7 August 2018] 24 August 2018 After the market closed, Musk announces that he no longer intends on taking Tesla private.

Exhibit 2: Price of Tesla, Inc. Stock: 31 July 2018-31 August 2018

Note: Open-High-Low-Close graph of Tesla's stock price, with white rectangles indicating upward movement in the day and black rectangles indicating downward movement during the day.

\begin{center}
\includegraphics[max width=\textwidth]{2023_05_04_7b535d0a870224f62e3dg-272}
\end{center}

Source of data: Yahoo! Finance.

\begin{enumerate}
  \item Is the fact that the price of Tesla moves up immediately on the day after the Q2 earnings (the first day of trading with this information) indicative of efficiency regarding information?
\end{enumerate}

Most likely.

\begin{enumerate}
  \setcounter{enumi}{1}
  \item Does the fact that the price of Tesla moves up but does not reach $\$ 420$ on the day the going-private Twitter announcement is made mean that investors underreacted?
\end{enumerate}

Not necessarily. There was confusion and uncertainty about the going-private transaction at the time, so the price did not close in on the proposed $\$ 420$ per share for going private.

\begin{enumerate}
  \setcounter{enumi}{2}
  \item Does the fact that the market price of the stock declined well before the issue of going-private was laid to rest by Musk mean that the market is inefficient?
\end{enumerate}

Not necessarily. There were numerous analyses, discussions, and other news regarding the likelihood of the transaction, all of which was incorporated in the price of the stock before the going-private transaction was dismissed by Musk.

Researchers have examined many different company-specific information events, including stock splits, dividend changes, and merger announcements, as well as economy-wide events, such as regulation changes and tax rate changes. The results of most research are consistent with the view that developed securities markets might be semi-strong efficient. But some evidence suggests that the markets in developing countries may not be semi-strong efficient. ${ }^{19}$

19 See Gan, Lee, Hwa, and Zhang (2005) and Raja, Sudhahar, and Selvam (2009).

\section{Strong Form}
In a strong-form efficient market, security prices fully reflect both public and private information. A market that is strong-form efficient is, by definition, also semi-strong- and weak-form efficient. In the case of a strong-form efficient market, insiders would not be able to earn abnormal returns from trading on the basis of private information. A strong-form efficient market also means that prices reflect all private information, which means that prices reflect everything that the management of a company knows about the financial condition of the company that has not been publicly released. However, this is not likely because of the strong prohibitions against insider trading that are found in most countries. If a market is strong-form efficient, those with insider information cannot earn abnormal returns.

Researchers test whether a market is strong-form efficient by testing whether investors can earn abnormal profits by trading on nonpublic information. The results of these tests are consistent with the view that securities markets are not strong-form efficient; many studies have found that abnormal profits can be earned when nonpublic information is used. ${ }^{20}$

\section{IMPLICATIONS OF THE EFFICIENT MARKET HYPOTHESIS}
explain the implications of each form of market efficiency for fundamental analysis, technical analysis, and the choice between active and passive portfolio management

The implications of efficient markets to investment managers and analysts are important because they affect the value of securities and how these securities are managed. Several implications can be drawn from the evidence on efficient markets for developed markets:

\begin{itemize}
  \item Securities markets are weak-form efficient, and therefore, investors cannot earn abnormal returns by trading on the basis of past trends in price.

  \item Securities markets are semi-strong efficient, and therefore, analysts who collect and analyze information must consider whether that information is already reflected in security prices and how any new information affects a security's value.

  \item Securities markets are not strong-form efficient because securities laws are intended to prevent exploitation of private information.

\end{itemize}

\section{Fundamental Analysis}
Fundamental analysis is the examination of publicly available information and the formulation of forecasts to estimate the intrinsic value of assets. Fundamental analysis involves the estimation of an asset's value using company data, such as earnings and

20 Evidence that finds that markets are not strong-form efficient include Jaffe (1974) and Rozeff and Zaman (1988). sales forecasts, and risk estimates as well as industry and economic data, such as economic growth, inflation, and interest rates. Buy and sell decisions depend on whether the current market price is less than or greater than the estimated intrinsic value.

The semi-strong form of market efficiency says that all available public information is reflected in current prices. So, what good is fundamental analysis? Fundamental analysis is necessary in a well-functioning market because this analysis helps the market participants understand the value implications of information. In other words, fundamental analysis facilitates a semi-strong efficient market by disseminating value-relevant information. And, although fundamental analysis requires costly information, this analysis can be profitable in terms of generating abnormal returns if the analyst creates a comparative advantage with respect to this information. ${ }^{21}$

\section{Technical Analysis}
Investors using technical analysis attempt to profit by looking at patterns of prices and trading volume. Although some price patterns persist, exploiting these patterns may be too costly and, hence, would not produce abnormal returns.

Consider a situation in which a pattern of prices exists. With so many investors examining prices, this pattern will be detected. If profitable, exploiting this pattern will eventually affect prices such that this pattern will no longer exist; it will be arbitraged away. In other words, by detecting and exploiting patterns in prices, technical analysts assist markets in maintaining weak-form efficiency. Does this mean that technical analysts cannot earn abnormal profits? Not necessarily, because there may be a possibility of earning abnormal profits from a pricing inefficiency. But would it be possible to earn abnormal returns on a consistent basis from exploiting such a pattern? No, because the actions of market participants will arbitrage this opportunity quickly, and the inefficiency will no longer exist.

\section{Portfolio Management}
If securities markets are weak-form and semi-strong-form efficient, the implication is that active trading, whether attempting to exploit price patterns or public information, is not likely to generate abnormal returns. In other words, portfolio managers cannot beat the market on a consistent basis, so therefore, passive portfolio management should outperform active portfolio management. Researchers have observed that mutual funds do not, on average, outperform the market on a risk-adjusted basis. ${ }^{22}$ Mutual funds perform, on average, similar to the market before considering fees and expenses and perform worse than the market, on average, once fees and expenses are considered. Even if a mutual fund is not actively managed, there are costs to managing these funds, which reduces net returns.

So, what good are portfolio managers? The role of a portfolio manager is not necessarily to beat the market but, rather, to establish and manage a portfolio consistent with the portfolio's objectives, with appropriate diversification and asset allocation, while taking into consideration the risk preferences and tax situation of the investor.

21 Brealey $(1983)$

22 See Malkiel (1995). One of the challenges to evaluating mutual fund performance is that the researcher must control for survivorship bias.

\section{MARKET PRICING ANOMALIES - TIME SERIES AND CROSS-SECTIONAL}
describe market anomalies

Although considerable evidence shows that markets are efficient, researchers have identified a number of apparent market inefficiencies or anomalies. These market anomalies, if persistent, are exceptions to the notion of market efficiency. Researchers conclude that a market anomaly may be present if a change in the price of an asset or security cannot directly be linked to current relevant information known in the market or to the release of new information into the market.

The validity of any evidence supporting the potential existence of a market inefficiency or anomaly must be consistent over reasonably long periods. Otherwise, a detected market anomaly may largely be an artifact of the sample period chosen. In the widespread search for discovering profitable anomalies, many findings could simply be the product of a process called data mining, also known as data snooping. In generally accepted research practice, an initial hypothesis is developed which is based on economic rationale. Tests are then conducted on objectively selected data to either confirm or reject the original hypothesis. However, with data mining the process is typically reversed: data are examined with the intent to develop a hypothesis, instead of developing a hypothesis first. This is done by analyzing data in various manners, and even utilizing different empirical approaches until you find support for a desired result, in this case a profitable anomaly.

Can researchers look back on data and find a trading strategy that would have yielded abnormal returns? Absolutely. Enough data snooping often can detect a trading strategy that would have worked in the past by chance alone. But in an efficient market, such a strategy is unlikely to generate abnormal returns on a consistent basis in the future. Also, although identified anomalies may appear to produce excess returns, it is generally difficult to profitably exploit the anomalies after accounting for risk, trading costs, and so on.

Several well-known anomalies are listed in Exhibit 3. This list is by no means exhaustive, but it provides information on the breadth of the anomalies. A few of these anomalies are discussed in more detail in the following sections. The anomalies are placed into categories based on the research method that identified the anomaly. Time-series anomalies were identified using time series of data. Cross-sectional anomalies were identified based on analyzing a cross section of companies that differ on some key characteristics. Other anomalies were identified by a variety of means, including event studies.

\section{Exhibit 3: Sampling of Observed Pricing Anomalies}
\begin{center}
\begin{tabular}{lll}
\hline
Time Series & Cross-Sectional & Other \\
\hline
January effect & Size effect & Closed-end fund discount \\
Day-of-the-week effect & Value effect & Earnings surprise \\
Weekend effect & Book-to-market ratios & Initial public offerings \\
Turn-of-the-month effect & P/E ratio effect & Distressed securities effect \\
Holiday effect & Value Line enigma & Stock splits \\
Time-of-day effect &  & Super Bowl \\
\end{tabular}
\end{center}

\begin{center}
\begin{tabular}{lll}
\hline
Time Series & Cross-Sectional & Other \\
\hline
Momentum &  &  \\
Overreaction &  &  \\
\hline
\end{tabular}
\end{center}

\section{Time-Series Anomalies}
Two of the major categories of time-series anomalies that have been documented are 1) calendar anomalies and 2) momentum and overreaction anomalies.

\section{Calendar Anomalies}
In the 1980s, a number of researchers reported that stock market returns in January were significantly higher compared to the rest of the months of the year, with most of the abnormal returns reported during the first five trading days in January. Since its first documentation in the 1980s, this pattern, known as the January effect, has been observed in most equity markets around the world. This anomaly is also known as the turn-of-the-year effect, or even often referred to as the "small firm in January effect" because it is most frequently observed for the returns of small market capitalization stocks. ${ }^{23}$

The January effect contradicts the efficient market hypothesis because excess returns in January are not attributed to any new and relevant information or news. A number of reasons have been suggested for this anomaly, including tax-loss selling. Researchers have speculated that, in order to reduce their tax liabilities, investors sell their "loser" securities in December for the purpose of creating capital losses, which can then be used to offset any capital gains. A related explanation is that these losers tend to be small-cap stocks with high volatility. ${ }^{24}$ This increased supply of equities in December depresses their prices, and then these shares are bought in early January at relatively attractive prices. This demand then drives their prices up again. Overall, the evidence indicates that tax-loss selling may account for a portion of January abnormal returns, but it does not explain all of it.

Another possible explanation for the anomaly is so-called "window dressing", a practice in which portfolio managers sell their riskier securities prior to 31 December. The explanation is as follows: many portfolio managers prepare the annual reports of their portfolio holdings as of 31 December. Selling riskier securities is an attempt to make their portfolios appear less risky. After 31 December, a portfolio manager would then simply purchase riskier securities in an attempt to earn higher returns. However, similar to the tax-loss selling hypothesis, the research evidence in support of the window dressing hypothesis explains some, but not all, of the anomaly.

Recent evidence for both stock and bond returns suggests that the January effect is not persistent and, therefore, is not a pricing anomaly. Once an appropriate adjustment for risk is made, the January "effect" does not produce abnormal returns. ${ }^{25}$

Several other calendar effects, including the day-of-the-week and the weekend effects, ${ }^{26}$ have been found. These anomalies are summarized in Exhibit $4 .{ }^{27}$ But like the size effect, which will be described later, most of these anomalies have been eliminated

23 There is also evidence of a January effect in bond returns that is more prevalent in high-yield corporate bonds, similar to the small-company effect for stocks.

\begin{enumerate}
  \setcounter{enumi}{23}
  \item See Roll (1983).
\end{enumerate}

25 See, for example, Kim (2006).

26 For a discussion of several of these anomalous patterns, see Jacobs and Levy (1988).

27 The weekend effect consists of a pattern of returns around the weekend: abnormal positive returns on Fridays followed by abnormally negative returns on Mondays. This is a day-of-the-week effect that specifically links Friday and Monday returns. It is interesting to note that in 2009 , the weekend effect in the United States was inverted, with 80 percent of the gains from March 2009 onward coming from the over time. One view is that the anomalies have been exploited such that the effect has been arbitraged away. Another view, however, is that increasingly sophisticated statistical methodologies fail to detect pricing inefficiencies.

\section{Exhibit 4: Calendar-Based Anomalies}
Anomaly Observation

Turn-of-the-month effect

Returns tend to be higher on the last trading day of the month and the first three trading days of the next month.

Day-of-the-week effect The average Monday return is negative and lower than the average returns for the other four days, which are all positive.

Weekend effect

Returns on weekends tend to be lower than returns on weekdays.

Holiday effect Returns on stocks in the day prior to market holidays tend to be higher than other days.

\section{Momentum and Overreaction Anomalies}
Momentum anomalies relate to short-term share price patterns. One of the earliest studies to identify this type of anomaly was conducted by Werner DeBondt and Richard Thaler, who argued that investors overreact to the release of unexpected public information. ${ }^{28}$ Therefore, stock prices will be inflated (depressed) for those companies releasing good (bad) information. This anomaly has become known as the overreaction effect. Using the overreaction effect, they proposed a strategy that involved buying "loser" portfolios and selling "winner" portfolios. They defined stocks as winners or losers based on their total returns over the previous three- to five-year period. They found that in a subsequent period, the loser portfolios outperformed the market, while the winner portfolios underperformed the market. Similar patterns have been documented in many, but not all, global stock markets as well as in bond markets. One criticism is that the observed anomaly may be the result of statistical problems in the analysis.

A contradiction to weak-form efficiency occurs when securities that have experienced high returns in the short term tend to continue to generate higher returns in subsequent periods. ${ }^{29}$ Empirical support for the existence of momentum in stock returns in most stock markets around the world is well documented. If investors can trade on the basis of momentum and earn abnormal profits, then this anomaly contradicts the weak form of the efficient market hypothesis because it represents a pattern in prices that can be exploited by simply using historical price information. ${ }^{30}$

Researchers have argued that the existence of momentum is rational and not contrary to market efficiency because it is plausible that there are shocks to the expected growth rates of cash flows to shareholders and that these shocks induce a

first trading day of the week.

28 DeBondt and Thaler (1985).

29 Notice that this pattern lies in sharp contrast to DeBondt and Thaler's reversal pattern that is displayed over longer periods of time. In theory, the two patterns could be related. In other words, it is feasible that prices are bid up extremely high, perhaps too high, in the short term for companies that are doing well. In the longer term (three-to-five years), the prices of these short-term winners correct themselves and they do poorly.

30 Jegadeesh and Titman (2001). serial correlation that is rational and short lived. ${ }^{31}$ In other words, having stocks with some degree of momentum in their security returns may not imply irrationality but, rather, may reflect prices adjusting to a shock in growth rates.

\section*{Cross-Sectional Anomalies }
Two of the most researched cross-sectional anomalies in financial markets are the size effect and the value effect.

\section{Size Effect}
The size effect results from the observation that equities of small-cap companies tend to outperform equities of large-cap companies on a risk-adjusted basis. Many researchers documented a small-company effect soon after the initial research was published in 1981. This effect, however, was not apparent in subsequent studies. ${ }^{32}$ Part of the reason that the size effect was not confirmed by subsequent studies may be because of the fact that if it were truly an anomaly, investors acting on this effect would reduce any potential returns. But some of the explanation may simply be that the effect as originally observed was a chance outcome and, therefore, not actually an inefficiency.

\section{Value Effect}
A number of global empirical studies have shown that value stocks, which are generally referred to as stocks that have below-average price-to-earnings (P/E) and market-to-book (M/B) ratios, and above-average dividend yields, have consistently outperformed growth stocks over long periods of time. ${ }^{33}$ If the effect persists, the value stock anomaly contradicts semi-strong market efficiency because all the information used to categorize stocks in this manner is publicly available.

Fama and French developed a three-factor model to predict stock returns. ${ }^{34} \mathrm{In}$ addition to the use of market returns as specified by the capital asset pricing model (CAPM), the Fama and French model also includes the size of the company as measured by the market value of its equity and the company's book value of equity divided by its market value of equity, which is a value measure. The Fama and French model captures risk dimensions related to stock returns that the CAPM model does not consider. Fama and French find that when they apply the three-factor model instead of the CAPM, the value stock anomaly disappears.

\section{OTHER ANOMALIES, IMPLICATIONS OF MARKET PRICING ANOMALIES}
31 Johnson (2002).

32 Although a large number of studies documents a small-company effect, these studies are concentrated in a period similar to that of the original research and, therefore, use a similar data set. The key to whether something is a true anomaly is persistence in out-of-sample tests. Fama and French (2008) document that the size effect is apparent only in microcap stocks but not in small- and large-cap stocks and these microcap stocks may have a significant influence in studies that document a size effect.

33 For example, see Capaul, Rowley, and Sharpe (1993) and Fama and French (1998).

\begin{enumerate}
  \setcounter{enumi}{33}
  \item Fama and French (1995). A number of additional anomalies has been documented in the financial markets, including the existence of closed-end investment fund discounts, price reactions to the release of earnings information, returns of initial public offerings, and the predictability of returns based on prior information.
\end{enumerate}

\section{Closed-End Investment Fund Discounts}
A closed-end investment fund issues a fixed number of shares at inception and does not sell any additional shares after the initial offering. Therefore, the fund capitalization is fixed unless a secondary public offering is made. The shares of closed-end funds trade on stock markets like any other shares in the equity market (i.e., their prices are determined by supply and demand).

Theoretically, these shares should trade at a price approximately equal to their net asset value (NAV) per share, which is simply the total market value of the fund's security holdings less any liabilities divided by the number of shares outstanding. An abundance of research, however, has documented that, on average, closed-end funds trade at a discount from NAV. Most studies have documented average discounts in the 4-10 percent range, although individual funds have traded at discounts exceeding 50 percent and others have traded at large premiums. ${ }^{35}$

The closed-end fund discount presents a puzzle because conceptually, an investor could purchase all the shares in the fund, liquidate the fund, and end up making a profit. Some researchers have suggested that these discounts are attributed to management fees or expectations of the managers' performance, but these explanations are not supported by the evidence. ${ }^{36}$ An alternative explanation for the discount is that tax liabilities are associated with unrealized capital gains and losses that exist prior to when the investor bought the shares, and hence, the investor does not have complete control over the timing of the realization of gains and losses. ${ }^{37}$ Although the evidence supports this hypothesis to a certain extent, the tax effect is not large enough to explain the entire discount. Finally, it has often been argued that the discounts exist because of liquidity problems and errors in calculating NAV. The illiquidity explanation is plausible if shares are recorded at the same price as more liquid, publicly traded stocks; some evidence supports this assertion. But as with tax reasons, liquidity issues explain only a portion of the discount effect.

Can these discounts be exploited to earn abnormal returns if transaction costs are taken into account? No. First, the transaction costs involved in exploiting the discount-buying all the shares and liquidating the fund-would eliminate any profit. ${ }^{38}$ Second, these discounts tend to revert to zero over time. Hence, a strategy to trade on the basis of these discounts would not likely be profitable. ${ }^{39}$

\section{Earnings Surprise}
Although most event studies have supported semi-strong market efficiency, some researchers have provided evidence that questions semi-strong market efficiency. One of these studies relates to the extensively examined adjustment of stock prices

35 See Dimson and Minio-Kozerski (1999) for a review of this literature.

36 See Lee, Sheifer, and Thaler (1990).

37 The return to owners of closed-end fund shares has three parts: 1 ) the price appreciation or depreciation of the shares themselves, 2) the dividends earned and distributed to owners by the fund, and 3) the capital gains and losses earned by the fund that are distributed by the fund. The explanation of the anomalous pricing has to do with the timing of the distribution of capital gains.

38 See, for example, the study by Pontiff (1996), which shows how the cost of arbitraging these discounts eliminates the profit.

39 See Pontiff (1995) to earnings announcements. ${ }^{40}$ The unexpected part of the earnings announcement, or earnings surprise, is the portion of earnings that is unanticipated by investors and, according to the efficient market hypothesis, merits a price adjustment. Positive (negative) surprises should cause appropriate and rapid price increases (decreases). Several studies have been conducted using data from numerous markets around the world. Most of the results indicate that earnings surprises are reflected quickly in stock prices, but the adjustment process is not always efficient. In particular, although a substantial adjustment occurs prior to and at the announcement date, an adjustment also occurs after the announcement. ${ }^{41}$

As a result of these slow price adjustments, companies that display the largest positive earnings surprises subsequently display superior stock return performance, whereas poor subsequent performance is displayed by companies with low or negative earnings surprises. ${ }^{42}$ This finding implies that investors could earn abnormal returns using publicly available information by buying stocks of companies that had positive earnings surprises and selling those with negative surprises.

Although there is support for abnormal returns associated with earnings surprises, and some support for such returns beyond the announcement period, there is also evidence indicating that these observed abnormal returns are an artifact of studies that do not sufficiently control for transaction costs and risk. ${ }^{43}$

\section{Initial Public Offerings (IPOs)}
When a company offers shares of its stock to the public for the first time, it does so through an initial public offering (or IPO). This offering involves working with an investment bank that helps price and market the newly issued shares. After the offering is complete, the new shares trade on a stock market for the first time. Given the risk that investment bankers face in trying to sell a new issue for which the true price is unknown, it is perhaps not surprising to find that, on average, the initial selling price is set too low and that the price increases dramatically on the first trading day. The percentage difference between the issue price and the closing price at the end of the first day of trading is often referred to as the degree of underpricing.

The evidence suggests that, on average, investors who are able to buy the shares of an IPO at their offering price may be able to earn abnormal profits. For example, during the internet bubble of 1995-2000, many IPOs ended their first day of trading up by more than 100 percent. Such performance, however, is not always the case. Sometimes the issues are priced too high, which means that share prices drop on their first day of trading. In addition, the evidence also suggests that investors buying after the initial offering are not able to earn abnormal profits because prices adjust quickly to the "true" values, which supports semi-strong market efficiency. In fact, the subsequent long-term performance of IPOs is generally found to be below average. Taken together, the IPO underpricing and the subsequent poor performance suggests that the markets are overly optimistic initially (i.e., investors overreact).

40 See Jones, Rendleman, and Latané (1984).

41 Not surprisingly, it is often argued that this slow reaction contributes to a momentum pattern.

42 A similar pattern has been documented in the corporate bond market, where bond prices react too slowly to new company earnings announcements as well as to changes in company debt ratings.

43 See Brown (1997) for a summary of evidence supporting the existence of this anomaly. See Zarowin (1989) for evidence regarding the role of size in explaining abnormal returns to surprises; Alexander, Goff, and Peterson (1989) for evidence regarding transaction costs and unexpected earnings strategies; and Kim and Kim (2003) for evidence indicating that the anomalous returns can be explained by risk factors. Some researchers have examined closely why IPOs may appear to have anomalous returns. Because of the small size of the IPO companies and the method of equally weighting the samples, what appears to be an anomaly may simply be an artifact of the methodology ${ }^{44}$

\section{Predictability of Returns Based on Prior Information}
A number of researchers have documented that equity returns are related to prior information on such factors as interest rates, inflation rates, stock volatility, and dividend yields. ${ }^{45}$ But finding that equity returns are affected by changes in economic fundamentals is not evidence of market inefficiency and would not result in abnormal trading returns. ${ }^{46}$

Furthermore, the relationship between stock returns and the prior information is not consistent over time. For example, in one study, the relationship between stock prices and dividend yields changed from positive to negative in different periods. ${ }^{47}$ Hence, a trading strategy based on dividend yields would not yield consistent abnormal returns.

\section{Implications for Investment Strategies}
Although it is interesting to consider the anomalies just described, attempting to benefit from them in practice is not easy. In fact, most researchers conclude that observed anomalies are not violations of market efficiency but, rather, are the result of statistical methodologies used to detect the anomalies. As a result, if the methodologies are corrected, most of these anomalies disappear. ${ }^{48}$ Another point to consider is that in an efficient market, overreactions may occur, but then so do under-reactions. ${ }^{49}$ Therefore, on average, the markets are efficient. In other words, investors face challenges when they attempt to translate statistical anomalies into economic profits. Consider the following quote regarding anomalies from the Economist ("Frontiers of Finance Survey," 9 October 1993):

Many can be explained away. When transactions costs are taken into account, the fact that stock prices tend to over-react to news, falling back the day after good news and bouncing up the day after bad news, proves unexploitable: price reversals are always within the bid-ask spread. Others such as the small-firm effect, work for a few years and then fail for a few years. Others prove to be merely proxies for the reward for risk taking. Many have disappeared since (and because) attention has been drawn to them.

It is difficult to envision entrusting your retirement savings to a manager whose strategy is based on buying securities on Mondays, which tends to have negative returns on average, and selling them on Fridays. For one thing, the negative Monday returns are merely an average, so on any given week, they could be positive. In addition, such a strategy would generate large trading costs. Even more importantly, investors would likely be uncomfortable investing their funds in a strategy that has no compelling underlying economic rationale.

\begin{enumerate}
  \setcounter{enumi}{43}
  \item See Brav, Geczy, and Gompers (1995).
\end{enumerate}

45 See, for example, Fama and Schwert (1977) and Fama and French (1988).

46 See Fama and French (2008).

47 Schwert (2003, Chapter 15).

48 Fama (1998)

49 This point is made by Fama (1998)

\section{BEHAVIORAL FINANCE}
describe behavioral finance and its potential relevance to understanding market anomalies

Behavioral finance examines investor behavior to understand how people make decisions, individually and collectively. Behavioral finance does not assume that people consider all available information in decision-making and act rationally by maximizing utility within budget constraints and updating expectations consistent with Bayes' formula. The resulting behaviors may affect what is observed in the financial markets.

In a broader sense, behavioral finance attempts to explain why individuals make the decisions that they do, whether these decisions are rational or irrational. The focus of much of the work in this area is on the behavioral biases that affect investment decisions. The behavior of individuals, in particular their behavioral biases, has been offered as a possible explanation for a number of pricing anomalies.

Most asset-pricing models assume that markets are rational and that the intrinsic value of a security reflects this rationality. But market efficiency and asset-pricing models do not require that each individual is rational-rather, only that the market is rational. If individuals deviate from rationality, other individuals are assumed to observe this deviation and respond accordingly. These responses move the market toward efficiency. If this does not occur in practice, it may be possible to explain some market anomalies referencing observed behaviors and behavioral biases.

\section{Loss Aversion}
In most financial models, the assumption is that investors are risk averse. Risk aversion refers to the tendency of people to dislike risk and to require higher expected returns to compensate for exposure to additional risk. Behavioral finance allows for the possibility that the dissatisfaction resulting from a loss exceeds the satisfaction resulting from a gain of the same magnitude. Loss aversion refers to the tendency of people to dislike losses more than they like comparable gains. This results in a strong preference for avoiding losses as opposed to achieving gains. ${ }^{50}$ Some argue that behavioral theories of loss aversion can explain observed overreaction in markets. If loss aversion is more important than risk aversion, researchers should observe that investors overreact. ${ }^{51}$ Although loss aversion can explain the overreaction anomaly, evidence also suggests that under reaction is just as prevalent as overreaction, which counters these arguments.

\section{Herding}
Herding behavior has been advanced as a possible explanation of under reaction and overreaction in financial markets. Herding occurs when investors trade on the same side of the market in the same securities, or when investors ignore their own private information and/or analysis and act as other investors do. Herding is clustered trading that may or may not be based on information. ${ }^{52}$ Herding may result in under- or over-reaction to information depending upon the direction of the herd.

50 See DeBondt and Thaler (1985) and Tversky and Kahneman (1981).

51 See Fama (1998).

52 The term used when there is herding without information is "spurious herding."

\section{Overconfidence}
A behavioral bias offered to explain pricing anomalies is overconfidence. If investors are overconfident, they overestimate their ability to process and interpret information about a security. Overconfident investors may not process information appropriately, and if there is a sufficient number of these investors, stocks will be mispriced. ${ }^{53}$ But most researchers argue that this mispricing is temporary, with prices correcting eventually. If it takes a sufficiently long time for prices to become correctly priced and the mispricing is predictable, it may be possible for investors to earn abnormal profits.

Evidence has suggested that overconfidence results in mispricing for US, UK, German, French, and Japanese markets. ${ }^{54}$ This overconfidence, however, is predominantly in higher-growth companies, whose prices react slowly to new information. ${ }^{55}$

\section{Information Cascades}
An application of behavioral theories to markets and pricing focuses on the role of personal learning in markets. Personal learning is what investors learn by observing outcomes of trades and what they learn from "conversations"-ideas shared among investors about specific assets and the markets. ${ }^{56}$ Social interaction and the resultant contagion is important in pricing and may explain such phenomena as price changes without accompanying news and mistakes in valuation.

Biases that investors possess can lead to herding behavior or information cascades. Herding and information cascades are related but not identical concepts. An information cascade is the transmission of information from those participants who act first and whose decisions influence the decisions of others. Those who are acting on the choices of others may be ignoring their own preferences in favor of imitating the choices of others. In particular, information cascades may occur with respect to the release of accounting information because accounting information may be difficult to interpret and may be noisy. For example, the release of earnings is difficult to interpret because it is necessary to understand how the number was arrived at and noisy because it is uncertain what the current earnings imply about future earnings.

Information cascades may result in serial correlation of stock returns, which is consistent with overreaction anomalies. Do information cascades result in correct pricing? Some argue that if a cascade is leading toward an incorrect value, this cascade is "fragile" and will be corrected because investors will ultimately give more weight to public information or the trading of a recognized informed trader. ${ }^{57}$ Information cascades, although documented in markets, do not necessarily mean that investors can exploit knowledge of them as profitable trading opportunities.

Are information cascades rational? If the informed traders act first and uninformed traders imitate the informed traders, this behavior is consistent with rationality. The imitation trading by the uninformed traders may help the market incorporate relevant information and improve market efficiency. ${ }^{58}$ However, the imitation trading may lead

53 Another aspect to overconfidence is that investors who are overconfident in their ability to select investments and manage a portfolio tend to use less diversification, investing in what is most familiar. Therefore, investor behavior may affect investment results-returns and risk-without implications for the efficiency of markets.

54 Scott, Stumpp, and Xu (2003) and Boujelbene Abbes, Boujelbene, and Bouri (2009).

55 Scott, Stumpp, and Xu (2003).

56 Hirshleifer and Teoh (2009).

57 Avery and Zemsky (1999).

58 Another alternative is that the uninformed traders are the majority of the market participants and the imitators are imitating not because they agree with the actions of the majority but because they are looking to act on the actions of the uninformed traders. to an overreaction to information. The empirical evidence indicates that information cascades are greater for a stock when the information quality regarding the company is poor. ${ }^{59}$ Information cascades may enhance the information available to investors.

\section{Other Behavioral Biases}
Other behavioral biases that have been put forth to explain observed investor behavior include the following:

\begin{itemize}
  \item representativeness-investors assess new information and probabilities of outcomes based on similarity to the current state or to a familiar classification;

  \item mental accounting-investors keep track of the gains and losses for different investments in separate mental accounts and treat those accounts differently;

  \item conservatism-investors tend to be slow to react to new information and continue to maintain their prior views or forecasts; and

  \item narrow framing-investors focus on issues in isolation and respond to the issues based on how the issues are posed. ${ }^{60}$

\end{itemize}

The basic idea behind behavioral finance is that investors are humans and, therefore, imperfect. These observed less than rational behaviors may help explain observed pricing anomalies. The beliefs investors have about a given asset's value may not be homogeneous. But an issue, which is controversial, is whether these insights can help someone identify and exploit any mispricing. In other words, can investors use knowledge of behavioral biases to predict how asset prices will be affected and act based on the predictions to earn abnormal profits?

\section{Behavioral Finance and Investors}
Behavior biases can affect all market participants, from the novice investor to the most experienced investment manager. An understanding of behavioral finance can help market participants recognize their own and others' behavioral biases. As a result of this recognition, they may be able to respond and make improved decisions, individually and collectively.

\section{Behavioral Finance and Efficient Markets}
The use of behavioral finance to explain observed pricing is an important part of the understanding of how markets function and how prices are determined. Whether there is a behavioral explanation for market anomalies remains a debate. Pricing anomalies are continually being uncovered, and then statistical and behavioral explanations are offered to explain these anomalies.

On the one hand, if investors must be rational for efficient markets to exist, then all the imperfections of human investors suggest that markets cannot be efficient. On the other hand, if all that is required for markets to be efficient is that investors cannot consistently beat the market on a risk-adjusted basis, then the evidence does support market efficiency.

59 Avery and Zemsky (1999) and Bikhchandani, Hirshleifer, and Welch (1992).

60 For a review of these behavioral issues, see Hirshleifer (2001).

\section{SUMMARY}
This reading has provided an overview of the theory and evidence regarding market efficiency and has discussed the different forms of market efficiency as well as the implications for fundamental analysis, technical analysis, and portfolio management. The general conclusion drawn from the efficient market hypothesis is that it is not possible to beat the market on a consistent basis by generating returns in excess of those expected for the level of risk of the investment.

Additional key points include the following:

\begin{itemize}
  \item The efficiency of a market is affected by the number of market participants and depth of analyst coverage, information availability, and limits to trading.

  \item There are three forms of efficient markets, each based on what is considered to be the information used in determining asset prices. In the weak form, asset prices fully reflect all market data, which refers to all past price and trading volume information. In the semi-strong form, asset prices reflect all publicly known and available information. In the strong form, asset prices fully reflect all information, which includes both public and private information.

  \item Intrinsic value refers to the true value of an asset, whereas market value refers to the price at which an asset can be bought or sold. When markets are efficient, the two should be the same or very close. But when markets are not efficient, the two can diverge significantly.

  \item Most empirical evidence supports the idea that securities markets in developed countries are semi-strong-form efficient; however, empirical evidence does not support the strong form of the efficient market hypothesis.

  \item A number of anomalies have been documented that contradict the notion of market efficiency, including the size anomaly, the January anomaly, and the winners-losers anomalies. In most cases, however, contradictory evidence both supports and refutes the anomaly.

  \item Behavioral finance uses human psychology, such as behavioral biases, in an attempt to explain investment decisions. Whereas behavioral finance is helpful in understanding observed decisions, a market can still be considered efficient even if market participants exhibit seemingly irrational behaviors, such as herding.

\end{itemize}

\section{REFERENCES}
Alexander, John C., Delbert Goff, Pamela P. Peterson. 1989. "Profitability of a Trading Strategy Based on Unexpected Earnings." Financial Analysts Journal, vol. 45, no. 4:65-71. 10.2469/faj. v45.n4.65

Bessembinder, Hendrik, Kalok Chan. 1998. "Market Efficiency and the Returns to Technical Analysis." Financial Management, vol. 27, no. 2:5-17. 10.2307/3666289

Bikhchandani, Sushil, David Hirshleifer, Ivo Welch. 1992. "A Theory of Fads, Fashion, Custom, and Cultural Change as Informational Cascades." Journal of Political Economy, vol. 100, no. 5:992-1026. 10.1086/261849

Brav, Alon, Christopher Geczy, Paul A. Gompers. 1995. “The Long-Run Underperformance of Seasoned Equity Offerings Revisited." Working paper, Harvard University.

Brealey, Richard. 1983. "Can Professional Investors Beat the Market?" An Introduction to Risk and Return from Common Stocks, 2nd edition. Cambridge, MA: MIT Press.

Bris, Arturo, William N. Goetzmann, Ning Zhu. 2009. "Efficiency and the Bear: Short Sales and Markets around the World." Journal of Finance, vol. 62, no. 3:1029-1079. 10.1111/j.1540-6 261.2007.01230.x

Brown, Laurence D. 1997. "Earning Surprise Research: Synthesis and Perspectives." Financial Analysts Journal, vol. 53, no. 2:13-19. 10.2469/faj.v53.n2.2067

Capaul, Carlo, Ian Rowley, William Sharpe. 1993. "International Value and Growth Stock Returns." Financial Analysts Journal, vol. 49:27-36. 10.2469/faj.v49.n1.27

Chen, Kong-Jun, Xiao-Ming Li. 2006. "Is Technical Analysis Useful for Stock Traders in China? Evidence from the Szse Component A-Share Index." Pacific Economic Review, vol. 11, no. 4:477-488. 10.1111/j.1468-0106.2006.00329.x

Chordia, Tarun, Richard Roll, Avanidhar Subrahmanyam. 2005. "Evidence on the Speed of Convergence to Market Efficiency." Journal of Financial Economics, vol. 76, no. 2:271-292. $10.1016 /$ j.jfineco.2004.06.004

DeBondt, Werner, Richard Thaler. 1985. "Does the Stock Market Overreact?" Journal of Finance, vol. 40, no. 3:793-808. 10.2307/2327804

Deng, Xiaohu, Sandra Mortal, Vishal Gupta. 2017. “The Real Effects of Short Selling Constraints: Cross-Country Evidence." Working paper.

Dimson, Elroy, Carolina Minio-Kozerski. 1999. "Closed-End Funds: A Survey." Financial Markets, Institutions \& Instruments, vol. 8, no. 2:1-41. 10.1111/1468-0416.00027

Fama, Eugene F. 1970. "Efficient Capital Markets: A Review of Theory and Empirical Work." Journal of Finance, vol. 25, no. 2:383-417. 10.2307/2325486

Fama, Eugene F. 1976. Foundations of Finance. New York: Basic Books.

Fama, Eugene F. 1998. “Market Efficiency, Long-Term Returns, and Behavioral Finance." Journal of Financial Economics, vol. 50, no. 3:283-306. 10.1016/S0304-405X(98)00026-9

Fama, Eugene F., Kenneth R. French. 1988. "Dividend Yields and Expected Stock Returns." Journal of Financial Economics, vol. 22, no. 1:3-25. 10.1016/0304-405X(88)90020-7

Fama, Eugene F., Kenneth R. French. 1995. "Size and Book-to-Market Factors in Earnings and Returns." Journal of Finance, vol. 50, no. 1:131-155. 10.2307/2329241

Fama, Eugene F., Kenneth R. French. 1998. "Value versus Growth: The International Evidence." Journal of Finance, vol. 53:1975-1999. 10.1111/0022-1082.00080

Fama, Eugene F., Kenneth R. French. 2008. "Dissecting Anomalies." Journal of Finance, vol. 63, no. 4:1653-1678. 10.1111/j.1540-6261.2008.01371.x

Grossman, Sanford J., Joseph E. Stiglitz. 1980. “On the Impossibility of Informationally Efficient Markets." American Economic Review, vol. 70, no. 3:393-408.

Hirshleifer, David. 2001. "Investor Psychology and Asset Pricing." Journal of Finance, vol. 56, no. 4:1533-1597. 10.1111/0022-1082.00379

Hirshleifer, David, Siew Hong Teoh. 2009. "Thought and Behavior Contagion in Capital Markets." In Handbook of Financial Markets: Dynamics and Evolution. Edited by Klaus Reiner Schenk-Hoppe and Thorstein Hens. Amsterdam: North Holland.

Jacobs, Bruce I., Kenneth N. Levy. 1988. "Calendar Anomalies: Abnormal Returns at Calendar Turning Points." Financial Analysts Journal, vol. 44, no. 6:28-39. 10.2469/faj.v44.n6.28 Jaffe, Jeffrey. 1974. "Special Information and Insider Trading." Journal of Business, vol. 47, no. 3:410-428. 10.1086/295655

Jegadeesh, Narayan, Sheridan Titman. 2001. "Profitability of Momentum Strategies: An Evaluation of Alternative Explanations." Journal of Finance, vol. 56:699-720. 10.1111/0022-1082.00342

Johnson, Timothy C. 2002. "Rational Momentum Effects." Journal of Finance, vol. 57, no. 2:585-608. 10.1111/1540-6261.00435

Kim, Donchoi, Myungsun Kim. 2003. "A Multifactor Explanation of Post-Earnings Announcement Drift." Journal of Financial and Quantitative Analysis, vol. 38, no. 2:383-398. $10.2307 / 4126756$

Kim, Dongcheol. 2006. “On the Information Uncertainty Risk and the January Effect." Journal of Business, vol. 79, no. 4:2127-2162. 10.1086/503659

Lee, Charles M.C., Andrei Sheifer, Richard H. Thaler. 1990. "Anomalies: Closed-End Mutual Funds." Journal of Economic Perspectives, vol. 4, no. 4:153-164.

Malkiel, Burton G. 1995. “Returns from Investing in Equity Mutual Funds 1971 to 1991.” Journal of Finance, vol. 50:549-572. 10.2307/2329419

Mobarek, Asma, A. Sabur Mollah, Rafiqul Bhuyan. 2008. "Market Efficiency in Emerging Stock Market.” JournalofEmergingMarketFinance,vol.7, no.1:17-41.10.1177/097265270700700102

Pontiff, Jeffrey. 1995. "Closed-End Fund Premia and Returns: Implications for Financial Market Equilibrium." JournalofFinancialEconomics,vol.37:341-370.10.1016/0304-405X(94)00800-G

Pontiff, Jeffrey. 1996. "Costly Arbitrage: Evidence from Closed-End Funds." Quarterly Journal of Economics, vol. 111, no. 4:1135-1151. 10.2307/2946710

Roll, Richard. 1983. "On Computing Mean Returns and the Small Firm Premium." Journal of Financial Economics, vol. 12:371-386. 10.1016/0304-405X(83)90055-7

Rozeff, Michael S., Mir A. Zaman. 1988. "Market Efficiency and Insider Trading: New Evidence." Journal of Business, vol. 61:25-44. 10.1086/296418

Schwert, G. William. 2003. "Anomalies and Market Efficiency" Handbook of the Economics of Finance. Edited by George M. Constantinides, M. Harris, and Rene Stulz. Amsterdam: Elsevier Science, B. V.

Scott, James, Margaret Stumpp, Peter Xu. 2003. "Overconfidence Bias in International Stock Prices." Journal of Portfolio Management, vol. 29, no. 2:80-89. 10.3905/jpm.2003.319875

Tversky, Amos, Daniel Kahneman. 1981. “The Framing of Decisions and the Psychology of Choice." Science, vol. 211, no. 30:453-458. 10.1126/science.7455683

Yau, Jot, Thomas Schneeweis, Thomas Robinson, Lisa Weiss. 2007. "Alternative Investments Portfolio Management." Managing Investment Portfolios: A Dynamic Process. Hoboken, NJ: John Wiley \& Sons.

Zarowin, P. 1989. "Does the Stock Market Overreact to Corporate Earnings Information?" Journal of Finance, vol. 44:1385-1399. 10.2307/2328649

\section{PRACTICE PROBLEMS}
\begin{enumerate}
  \item If markets are efficient, the difference between the intrinsic value and market value of a company's security is:
A. negative.
B. zero.
C. positive.

  \item The intrinsic value of an undervalued asset is:

\end{enumerate}

A. less than the asset's market value.

B. greater than the asset's market value.

C. the value at which the asset can currently be bought or sold.

\begin{enumerate}
  \setcounter{enumi}{2}
  \item The market value of an undervalued asset is:
A. greater than the asset's intrinsic value.
B. the value at which the asset can currently be bought or sold.
C. equal to the present value of all the asset's expected cash flows.

  \item In an efficient market, the change in a company's share price is most likely the result of:

\end{enumerate}

A. insiders' private information.

B. the previous day's change in stock price.

C. new information coming into the market.

\begin{enumerate}
  \setcounter{enumi}{4}
  \item Regulation that restricts some investors from participating in a market will most likely:
\end{enumerate}

A. impede market efficiency.

B. not affect market efficiency.

C. contribute to market efficiency.

\begin{enumerate}
  \setcounter{enumi}{5}
  \item With respect to efficient market theory, when a market allows short selling, the efficiency of the market is most likely to:
A. increase.
B. decrease.
C. remain the same.

  \item Which of the following regulations will most likely contribute to market efficiency? Regulatory restrictions on:
A. short selling.
B. foreign traders. C. insiders trading with nonpublic information.

  \item Which of the following market regulations will most likely impede market efficiency?
A. Restricting traders' ability to short sell.
B. Allowing unrestricted foreign investor trading.
C. Penalizing investors who trade with nonpublic information.

  \item An increase in the time between when an order to trade a security is placed and when the order is executed most likely indicates that market efficiency has:
A. decreased.
B. remained the same.
C. increased.

  \item With respect to the efficient market hypothesis, if security prices reflect only past prices and trading volume information, then the market is:
A. weak-form efficient.
B. strong-form efficient.
C. semi-strong-form efficient.

  \item Which one of the following statements best describes the semi-strong form of market efficiency?
A. Empirical tests examine the historical patterns in security prices.
B. Security prices reflect all publicly known and available information.
C. Semi-strong-form efficient markets are not necessarily weak-form efficient.

  \item If prices reflect all public and private information, the market is best described as:
A. weak-form efficient.
B. strong-form efficient.
C. semi-strong-form efficient.

  \item If markets are semi-strong efficient, standard fundamental analysis will yield abnormal trading profits that are:
A. negative.
B. equal to zero.
C. positive.

  \item If markets are semi-strong-form efficient, then passive portfolio management strategies are most likely to:
A. earn abnormal returns.
B. outperform active trading strategies. C. underperform active trading strategies.

  \item If a market is semi-strong-form efficient, the risk-adjusted returns of a passively managed portfolio relative to an actively managed portfolio are most likely:
A. lower.
B. higher.
C. the same.

  \item Technical analysts assume that markets are:
A. weak-form efficient.
B. weak-form inefficient.
C. semi-strong-form efficient.

  \item Fundamental analysts assume that markets are:
A. weak-form inefficient.
B. semi-strong-form efficient.
C. semi-strong-form inefficient.

  \item If a market is weak-form efficient but semi-strong-form inefficient, then which of the following types of portfolio management is most likely to produce abnormal returns?

\end{enumerate}

A. Passive portfolio management.

B. Active portfolio management based on technical analysis.

C. Active portfolio management based on fundamental analysis.

\begin{enumerate}
  \setcounter{enumi}{18}
  \item Which of the following is least likely to explain the January effect anomaly?
\end{enumerate}

A. Tax-loss selling.

B. Release of new information in January.

C. Window dressing of portfolio holdings.

\begin{enumerate}
  \setcounter{enumi}{19}
  \item If a researcher conducting empirical tests of a trading strategy using time series of returns finds statistically significant abnormal returns, then the researcher has most likely found:
\end{enumerate}

A. a market anomaly.

B. evidence of market inefficiency.

C. a strategy to produce future abnormal returns.

\begin{enumerate}
  \setcounter{enumi}{20}
  \item Researchers have found that value stocks have consistently outperformed growth stocks. An investor wishing to exploit the value effect should purchase the stock of companies with above-average:
\end{enumerate}

A. dividend yields. B. market-to-book ratios.

C. price-to-earnings ratios.

\begin{enumerate}
  \setcounter{enumi}{21}
  \item Which of the following market anomalies is inconsistent with weak-form market efficiency?
A. Earnings surprise.
B. Momentum pattern.
C. Closed-end fund discount.

  \item With respect to efficient markets, a company whose share price changes gradually after the public release of its annual report most likely indicates that the market where the company trades is:
A. semi-strong-form efficient.
B. subject to behavioral biases.
C. receiving additional information about the company.

  \item With respect to rational and irrational investment decisions, the efficient market hypothesis requires:
A. only that the market is rational.
B. that all investors make rational decisions.
C. that some investors make irrational decisions.

  \item Observed overreactions in markets can be explained by an investor's degree of:
A. risk aversion.
B. loss aversion.
C. confidence in the market.

  \item Like traditional finance models, the behavioral theory of loss aversion assumes that investors dislike risk; however, the dislike of risk in behavioral theory is assumed to be:
A. leptokurtic.
B. symmetrical.
C. asymmetrical.

\end{enumerate}

\section{SOLUTIONS}
\begin{enumerate}
  \item B is correct. A security's intrinsic value and market value should be equal when markets are efficient.

  \item B is correct. The intrinsic value of an undervalued asset is greater than the market value of the asset, where the market value is the transaction price at which an asset can be currently bought or sold.

  \item B is correct. The market value is the transaction price at which an asset can be currently bought or sold.

  \item C is correct. Today's price change is independent of the one from yesterday, and in an efficient market, investors will react to new, independent information as it is made public.

  \item A is correct. Reducing the number of market participants can accentuate market imperfections and impede market efficiency (e.g., restrictions on foreign investor trading).

  \item A is correct. According to theory, reducing the restrictions on trading will allow for more arbitrage trading, thereby promoting more efficient pricing. Although regulators argue that short selling exaggerates downward price movements, empirical research indicates that short selling is helpful in price discovery.

  \item C is correct. Regulation to restrict unfair use of nonpublic information encourages greater participation in the market, which increases market efficiency. Regulators (e.g., US SEC) discourage illegal insider trading by issuing penalties to violators of their insider trading rules.

  \item A is correct. Restricting short selling will reduce arbitrage trading, which promotes market efficiency. Permitting foreign investor trading increases market participation, which makes markets more efficient. Penalizing insider trading encourages greater market participation, which increases market efficiency.

  \item A is correct. Operating inefficiencies reduce market efficiency.

  \item A is correct. The weak-form efficient market hypothesis is defined as a market where security prices fully reflect all market data, which refers to all past price and trading volume information.

  \item B is correct. In semi-strong-form efficient markets, security prices reflect all publicly available information.

  \item B is correct. The strong-form efficient market hypothesis assumes all information, public or private, has already been reflected in the prices.

  \item $B$ is correct. If all public information should already be reflected in the market price, then the abnormal trading profit will be equal to zero when fundamental analysis is used.

  \item B is correct. Costs associated with active trading strategies would be difficult to recover; thus, such active trading strategies would have difficulty outperforming passive strategies on a consistent after-cost basis.

  \item B is correct. In a semi-strong-form efficient market, passive portfolio strategies should outperform active portfolio strategies on a risk-adjusted basis. 16. B is correct. Technical analysts use past prices and volume to predict future prices, which is inconsistent with the weakest form of market efficiency (i.e., weak-form market efficiency). Weak-form market efficiency states that investors cannot earn abnormal returns by trading on the basis of past trends in price and volume.

  \item C is correct. Fundamental analysts use publicly available information to estimate a security's intrinsic value to determine if the security is mispriced, which is inconsistent with the semi-strong form of market efficiency. Semi-strong-form market efficiency states that investors cannot earn abnormal returns by trading based on publicly available information.

  \item $\mathrm{C}$ is correct. If markets are not semi-strong-form efficient, then fundamental analysts are able to use publicly available information to estimate a security's intrinsic value and identify misvalued securities. Technical analysis is not able to earn abnormal returns if markets are weak-form efficient. Passive portfolio managers outperform fundamental analysis if markets are semi-strong-form efficient.

  \item B is correct. The excess returns in January are not attributed to any new information or news; however, research has found that part of the seasonal pattern can be explained by tax-loss selling and portfolio window dressing.

  \item A is correct. Finding significant abnormal returns does not necessarily indicate that markets are inefficient or that abnormal returns can be realized by applying the strategy to future time periods. Abnormal returns are considered market anomalies because they may be the result of the model used to estimate the expected returns or may be the result of underestimating transaction costs or other expenses associated with implementing the strategy, rather than because of market inefficiency.

  \item A is correct. Higher than average dividend yield is a characteristic of a value stock, along with low price-to-earnings and low market-to-book ratios. Growth stocks are characterized by low dividend yields and high price-to-earnings and high market-to-book ratios.

  \item B is correct. Trading based on historical momentum indicates that price patterns exist and can be exploited by using historical price information. A momentum trading strategy that produces abnormal returns contradicts the weak form of the efficient market hypothesis, which states that investors cannot earn abnormal returns on the basis of past trends in prices.

  \item $\mathrm{C}$ is correct. If markets are efficient, the information from the annual report is reflected in the stock prices; therefore, the gradual changes must be from the release of additional information.

  \item A is correct. The efficient market hypothesis and asset-pricing models only require that the market is rational. Behavioral finance is used to explain some of the market anomalies as irrational decisions.

  \item B is correct. Behavioral theories of loss aversion can explain observed overreaction in markets, such that investors dislike losses more than comparable gains (i.e., risk is not symmetrical).

  \item C is correct. Behavioral theories of loss aversion allow for the possibility that the dislike for risk is not symmetrical, which allows for loss aversion to explain observed overreaction in markets such that investors dislike losses more than they like comparable gains.

\end{enumerate}

\section*{LEARNING MODULE 
 4 }
\section{Overview of Equity Securities}
Ryan C. Fuhrmann, CFA, is at Fuhrmann Capital LLC (USA). Asjeet S. Lamba, PhD, CFA, is at the University of Melbourne (Australia).

\section{LEARNING OUTCOME}
\begin{center}
\begin{tabular}{|c|c|}
\hline
Mastery & The candidate should be able to: \\
\hline
$\square$ & describe characteristics of types of equity securities \\
\hline
$\square$ & $\begin{array}{l}\text { describe differences in voting rights and other ownership } \\ \text { characteristics among different equity classes }\end{array}$ \\
\hline
$\square$ & compare and contrast public and private equity securities \\
\hline
$\square$ & describe methods for investing in non-domestic equity securities \\
\hline
$\square$ & $\begin{array}{l}\text { compare the risk and return characteristics of different types of } \\ \text { equity securities }\end{array}$ \\
\hline
$\square$ & $\begin{array}{l}\text { explain the role of equity securities in the financing of a company's } \\ \text { assets }\end{array}$ \\
\hline
$\square$ & contrast the market value and book value of equity securities \\
\hline
[ & $\begin{array}{l}\text { compare a company's cost of equity, its (accounting) return on } \\ \text { equity, and investors' required rates of return }\end{array}$ \\
\hline
\end{tabular}
\end{center}

\section{IMPORTANCE OF EQUITY SECURITIES}
Equity securities represent ownership claims on a company's net assets. As an asset class, equity plays a fundamental role in investment analysis and portfolio management because it represents a significant portion of many individual and institutional investment portfolios.

The study of equity securities is important for many reasons. First, the decision on how much of a client's portfolio to allocate to equities affects the risk and return characteristics of the entire portfolio. Second, different types of equity securities have different ownership claims on a company's net assets, which affect their risk and return characteristics in different ways. Finally, variations in the features of equity securities are reflected in their market prices, so it is important to understand the valuation implications of these features. This reading provides an overview of equity securities and their different features and establishes the background required to analyze and value equity securities in a global context. It addresses the following questions:

\begin{itemize}
  \item What distinguishes common shares from preference shares, and what purposes do these securities serve in financing a company's operations?

  \item What are convertible preference shares, and why are they often used to raise equity for unseasoned or highly risky companies?

  \item What are private equity securities, and how do they differ from public equity securities?

  \item What are depository receipts and their various types, and what is the rationale for investing in them?

  \item What are the risk factors involved in investing in equity securities?

  \item How do equity securities create company value?

  \item What is the relationship between a company's cost of equity, its return on equity, and investors' required rate of return?

\end{itemize}

The remainder of this reading is organized as follows. Section 2 provides an overview of global equity markets and their historical performance. Section 3 examines the different types and characteristics of equity securities, and Section 4 outlines the differences between public and private equity securities. Section 5 provides an overview of the various types of equity securities listed and traded in global markets. Section 6 discusses the risk and return characteristics of equity securities. Section 7 examines the role of equity securities in creating company value and the relationship between a company's cost of equity, its return on equity, and investors' required rate of return. The final section summarizes the reading.

\section{Equity Securities in Global Financial Markets}
This section highlights the relative importance and performance of equity securities as an asset class. We examine the total market capitalization and trading volume of global equity markets and the prevalence of equity ownership across various geographic regions. We also examine historical returns on equities and compare them to the returns on government bonds and bills.

Exhibit 1 summarizes the contributions of selected countries and geographic regions to global gross domestic product (GDP) and global equity market capitalization. Analysts may examine the relationship between equity market capitalization and GDP as a rough indicator of whether the global equity market (or a specific country's or region's equity market) is under, over, or fairly valued, particularly compared to its long-run average.

Exhibit 1 illustrates the significant value that investors attach to publicly traded equities relative to the sum of goods and services produced globally every year. It shows the continued significance, and the potential over-representation, of US equity markets relative to their contribution to global GDP. That is, while US equity markets contribute around 51 percent to the total capitalization of global equity markets, their contribution to the global GDP is only around 25 percent. Following the stock market turmoil in 2008, however, the market capitalization to GDP ratio of the United States fell to 59 percent, which is significantly lower than its long-run average of 79 percent.

As equity markets outside the United States develop and become increasingly global, their total capitalization levels are expected to grow closer to their respective world GDP contributions. Therefore, it is important to understand and analyze equity securities from a global perspective.

\section{Exhibit 1: Country and Regional Contributions to Global GDP and Equity Market Capitalization (2017)}
Contribution to GDP 2017

\begin{center}
\includegraphics[max width=\textwidth]{2023_05_04_7b535d0a870224f62e3dg-297}
\end{center}

Relative Sizes of World Stock Market 2017

\begin{center}
\includegraphics[max width=\textwidth]{2023_05_04_7b535d0a870224f62e3dg-297(1)}
\end{center}

Source: The WorldBank Databank 2017, and Dimson, Marsh, and Staunton (2018).

Exhibit 2 lists the top 10 equity markets at the end of 2017 based on total market capitalization (in billions of US dollars), trading volume, and the number of listed companies. ${ }^{1}$ Note that the rankings differ based on the criteria used. For example, the top three markets based on total market capitalization are the NYSE Euronext (US), NASDAQ OMX, and the Japan Exchange Group; however, the top three markets based on total US dollar trading volume are the Nasdaq OMX, NYSE Euronext (US), and the Shenzhen Stock Exchange, respectively. ${ }^{2}$

Exhibit 2: Equity Markets Ranked by Total Market Capitalization at the End of 2017 (Billions of US Dollars)

\begin{center}
\begin{tabular}{llccc}
\hline
 & \multicolumn{1}{c}{Total US} & $\begin{array}{c}\text { Total US } \\ \text { Dollar } \\ \text { Dollar Market } \\ \text { Trading } \\ \text { Capitalization }\end{array}$ & $\begin{array}{c}\text { Volume }\end{array}$ & $\begin{array}{c}\text { Number } \\ \text { of Listed } \\ \text { Companies }\end{array}$ \\
\hline
1 & NYSE Euronext (US) & $\$ 22,081.4$ & $\$ 16,140.1$ & 2,286 \\
2 & NASDAQ OMX & $\$ 10,039.4$ & $\$ 33,407.1$ & 2,949 \\
3 & Japan Exchange Group ${ }^{a}$ & $\$ 6,220.0$ & $\$ 6,612.1$ & 3,604 \\
4 & Shanghai Stock Exchange & $\$ 5,084.4$ & $\$ 7,589.3$ & 1,396 \\
5 & Euronext ${ }^{\mathrm{b}}$ & $\$ 4,393.0$ & $\$ 1,981.6$ & 1,255 \\
6 & Hong Kong Exchanges & $\$ 4,350.5$ & $\$ 1,958.8$ & 2,118 \\
7 & Shenzhen Stock Exchanges & $\$ 3,617.9$ & $\$ 9,219.7$ & 2,089 \\
\end{tabular}
\end{center}

1 The market capitalization of an individual stock is computed as the share price multiplied by the number of shares outstanding. The total market capitalization of an equity market is the sum of the market capitalizations of each individual stock listed on that market. Similarly, the total trading volume of an equity market is computed by value weighting the total trading volume of each individual stock listed on that market. Total dollar trading volume is computed as the average share price multiplied by the number of shares traded. 2 NASDAQ is the acronym for the National Association of Securities Dealers Automated Quotations.

\begin{center}
\begin{tabular}{|c|c|c|c|c|}
\hline
Rank & Name of Market & $\begin{array}{c}\text { Total US } \\ \text { Dollar Market } \\ \text { Capitalization }\end{array}$ & $\begin{array}{c}\text { Total US } \\ \text { Dollar } \\ \text { Trading } \\ \text { Volume }\end{array}$ & $\begin{array}{l}\text { Number } \\ \text { of Listed } \\ \text { Companies }\end{array}$ \\
\hline
8 & $\begin{array}{l}\text { National Stock Exchange of } \\ \text { India }\end{array}$ & $\$ 2,351.5$ & $\$ 1,013.3$ & 1,897 \\
\hline
9 & BSE Limited $^{\mathrm{c}}$ & $\$ 2,331.6$ & $\$ 183.0$ & 5,616 \\
\hline
10 & Deutsche Börse & $\$ 2,262.2$ & $\$ 1,497.9$ & 499 \\
\hline
\end{tabular}
\end{center}

Notes:

a Japan Exchange Group is the merged entity containing the Tokyo Stock Exchange and Osaka Securities Exchange.

b From 2001, includes Netherlands, France, England, Belgium, and Portugal.

c Bombay Stock Exchange.

Source: Adapted from the World Federation of Exchanges 2017 Report (see \href{http://www.world-exchanges}{http://www.world-exchanges}. org). Note that market capitalization by company is calculated by multiplying its stock price by the number of shares outstanding. The market's overall capitalization is the aggregate of the market capitalizations of all companies traded on that market. The number of listed companies includes both domestic and foreign companies whose shares trade on these markets.

Exhibit 3 compares the real (or inflation-adjusted) compounded returns on government bonds, government bills, and equity securities in 21 countries plus the world index ("Wld"), the world ex-US ("WxU"), and Europe ("Eur") during the 118 years 1900-2017. ${ }^{3}$ In real terms, government bonds and bills have essentially kept pace with the inflation rate, earning annualized real returns of less than 2 percent in most countries. ${ }^{4}$ By comparison, real returns in equity markets have generally been around 3.5 percent per year in most markets-with a world average return of around 5.2 percent and a world average return excluding the United States just under 5 percent. During this period, South Africa and Australia were the best performing markets followed by the United States, New Zealand, and Sweden.

3 The real return for a security is approximated by taking the nominal return and subtracting the observed inflation rate in that country.

4 The exceptions are Austria, Belgium, Finland, France, Germany, Portugal, and Italy-where the average real returns on government bonds and/or bills have been negative. In general, that performance reflects the very high inflation rates in these countries during the World War years. Exhibit 3: Real Returns on Global Equity Securities, Bonds, and Bills During 1900-2017

\begin{center}
\includegraphics[max width=\textwidth]{2023_05_04_7b535d0a870224f62e3dg-299}
\end{center}

Source: Dimson, Marsh, and Staunton (2018).

Exhibit 4 shows the annualized real returns on major asset classes for the world index over 1900-2017.

Exhibit 4: Annualized Real Returns on Asset Classes for the World Index, 1900-2017

6

\begin{center}
\includegraphics[max width=\textwidth]{2023_05_04_7b535d0a870224f62e3dg-299(1)}
\end{center}

$-2$

2000-2017

$1968-2017$

$1900-2017$

$\square$ Equities $\square$ Bonds $\square$ Bills

Source: Dimson, Marsh, and Staunton (2018). The volatility in asset market returns is further highlighted in Exhibit 5, which shows the annualized risk premia for equity relative to bonds (EP Bonds), and equity relative to treasury bills (EP Bills). Maturity premium for government bond returns relative to treasury bill returns (Mat Prem) is also shown.

These observations and historical data are consistent with the concept that the return on securities is directly related to risk level. That is, equity securities have higher risk levels when compared with government bonds and bills, they earn higher rates of return to compensate investors for these higher risk levels, and they also tend to be more volatile over time.

Exhibit 5: Annualized Real Returns on Asset Classes and Risk Premiums for the World Index since 1900-2017
\includegraphics[max width=\textwidth, center]{2023_05_04_7b535d0a870224f62e3dg-300}

Notes: Equities are total returns, including reinvested dividend income. Bonds are total return, including reinvested coupons, on long-term government bonds. Bills denotes the total return, including any income, from Treasury bills. All returns are adjusted for inflation and are expressed as geometric mean returns. EP bonds denotes the equity risk premium relative to long-term government bonds. EP Bills denotes the equity premium relative to Treasury bills. MatPrem denotes the maturity premium for government bond returns relative to bill returns. RealXRate denotes the real (inflation-adjusted) change in the exchange rate against the US dollar.

Source: Dimson, Marsh, and Staunton (2018).

Given the high risk levels associated with equity securities, it is reasonable to expect that investors' tolerance for risk will tend to differ across equity markets. This is illustrated in Exhibit 6, which shows the results of a series of studies conducted by the Australian Securities Exchange on international differences in equity ownership. During the 2004-2014 period, equity ownership as a percentage of the population was lowest in South Korea (averaging 9.0 percent), followed by Germany (14.5 percent) and Sweden (17.7 percent). In contrast, Australia and New Zealand had the highest equity ownership as a percentage of the population (averaging more than 20 percent). In addition, there has been a relative decline in share ownership in several countries over recent years, which is not surprising given the recent overall uncertainty in global economies and the volatility in equity markets that this uncertainty has created. Exhibit 6: International Comparisons of Stock Ownership: 2004-2014

\begin{center}
\begin{tabular}{lcccccc}
\hline
 & $\mathbf{2 0 0 4}$ & $\mathbf{2 0 0 6}$ & $\mathbf{2 0 0 8}$ & $\mathbf{2 0 1 0}$ & $\mathbf{2 0 1 2}$ & $\mathbf{2 0 1 4}$ \\
\hline
Australia - Direct/Indirect & $55 \%$ & $46 \%$ & $41 \%$ & $43 \%$ & $38 \%$ & $36 \%$ \\
South Korea - Shares & 8 & 7 & 10 & 10 & 10 & N/A \\
Germany - Shares/Funds & 16 & 16 & 14 & 13 & 15 & 13 \\
Sweden - Shares & 22 & 20 & 18 & 17 & 15 & 14 \\
United Kingdom - Shares/ & 22 & 20 & 18 & N/A & 17 & N/A \\
Funds &  &  &  &  &  &  \\
New Zealand - Direct & 23 & 26 & N/A & 22 & 23 & 26 \\
\hline
\end{tabular}
\end{center}

Source: Adapted from the 2014 Australian Share Ownership Study conducted by the Australian Securities Exchange (see \href{http://www.asx.com.au}{http://www.asx.com.au}). For Australia and the United States, the data pertain to direct and indirect ownership in equity markets; for other countries, the data pertain to direct ownership in shares and share funds. Data not available in specific years are shown as "N/A."

An important implication from the above discussion is that equity securities represent a key asset class for global investors because of their unique return and risk characteristics. We next examine the various types of equity securities traded on global markets and their salient characteristics.

\section{CHARACTERISTICS OF EQUITY SECURITIES}
$$
\mid \begin{aligned}
& \text { describe characteristics of types of equity securities } \\
& \text { describe differences in voting rights and other ownership } \\
& \text { characteristics among different equity classes }
\end{aligned}
$$

Companies finance their operations by issuing either debt or equity securities. A key difference between these securities is that debt is a liability of the issuing company, whereas equity is not. This means that when a company issues debt, it is contractually obligated to repay the amount it borrows (i.e., the principal or face value of the debt) at a specified future date. The cost of using these funds is called interest, which the company is contractually obligated to pay until the debt matures or is retired.

When the company issues equity securities, it is not contractually obligated to repay the amount it receives from shareholders, nor is it contractually obligated to make periodic payments to shareholders for the use of their funds. Instead, shareholders have a claim on the company's assets after all liabilities have been paid. Because of this residual claim, equity shareholders are considered to be owners of the company. Investors who purchase equity securities are seeking total return (i.e., capital or price appreciation and dividend income), whereas investors who purchase debt securities (and hold until maturity) are seeking interest income. As a result, equity investors expect the company's management to act in their best interest by making operating decisions that will maximize the market price of their shares (i.e., shareholder wealth).

5 The percentages reported in the exhibit are based on samples of the adult population in each country who own equity securities either directly or indirectly through investment or retirement funds. For example, 36 percent of the adult population of Australia in 2014 (approximately 6.5 million people) owned equity securities either directly or indirectly. As noted in the study, it is not appropriate to make absolute comparisons across countries given the differences in methodology, sampling, timing, and definitions that have been used in different countries. However, trends across different countries can be identified. In addition to common shares (also known as ordinary shares or common stock), companies may also issue preference shares (also known as preferred stock), the other type of equity security. The following sections discuss the different types and characteristics of common and preference securities.

\section{Common Shares}
Common shares represent an ownership interest in a company and are the predominant type of equity security. As a result, investors share in the operating performance of the company, participate in the governance process through voting rights, and have a claim on the company's net assets in the case of liquidation. Companies may choose to pay out some, or all, of their net income in the form of cash dividends to common shareholders, but they are not contractually obligated to do so. ${ }^{6}$

Voting rights provide shareholders with the opportunity to participate in major corporate governance decisions, including the election of its board of directors, the decision to merge with or take over another company, and the selection of outside auditors. Shareholder voting generally takes place during a company's annual meeting. As a result of geographic limitations and the large number of shareholders, it is often not feasible for shareholders to attend the annual meeting in person. For this reason, shareholders may vote by proxy, which allows a designated party-such as another shareholder, a shareholder representative, or management-to vote on the shareholders' behalf.

Regular shareholder voting, where each share represents one vote, is referred to as statutory voting. Although it is the common method of voting, it is not always the most appropriate one to use to elect a board of directors. To better serve shareholders who own a small number of shares, cumulative voting is often used. Cumulative voting allows shareholders to direct their total voting rights to specific candidates, as opposed to having to allocate their voting rights evenly among all candidates. Total voting rights are based on the number of shares owned multiplied by the number of board directors being elected. For example, under cumulative voting, if four board directors are to be elected, a shareholder who owns 100 shares is entitled to 400 votes and can either cast all 400 votes in favor of a single candidate or spread them across the candidates in any proportion. In contrast, under statutory voting, a shareholder would be able to cast only a maximum of 100 votes for each candidate.

The key benefit to cumulative voting is that it allows shareholders with a small number of shares to apply all of their votes to one candidate, thus providing the opportunity for a higher level of representation on the board than would be allowed under statutory voting.

Exhibit 7 describes the rights of Viacom Corporation's shareholders. In this case, a dual-share arrangement allows the founding chairman and his family to control more than 70 percent of the voting rights through the ownership of Class A shares. This arrangement gives them the ability to exert control over the board of director election process, corporate decision making, and other important aspects of managing the company. A cumulative voting arrangement for any minority shareholders of Class A shares would improve their board representation.

6 It is also possible for companies to pay more than the current period's net income as dividends. Such payout policies are, however, generally not sustainable in the long run.

\section{Exhibit 7: Share Class Arrangements at Viacom Corporation ${ }^{7}$}
Viacom has two classes of common stock: Class A, which is the voting stock, and Class B, which is the non-voting stock. There is no difference between the two classes except for voting rights; they generally trade within a close price range of each other. There are, however, far more shares of Class B outstanding, so most of the trading occurs in that class.

\begin{itemize}
  \item Voting Rights-Holders of Class A common stock are entitled to one vote per share. Holders of Class B common stock do not have any voting rights, except as required by Delaware law. Generally, all matters to be voted on by Viacom stockholders must be approved by a majority of the aggregate voting power of the shares of Class A common stock present in person or represented by proxy, except as required by Delaware law.

  \item Dividends-Stockholders of Class A common stock and Class B common stock will share ratably in any cash dividend declared by the Board of Directors, subject to any preferential rights of any outstanding preferred stock. Viacom does not currently pay a cash dividend, and any decision to pay a cash dividend in the future will be at the discretion of the Board of Directors and will depend on many factors.

  \item Conversion-So long as there are 5,000 shares of Class A common stock outstanding, each share of Class A common stock will be convertible at the option of the holder of such share into one share of Class B common stock.

  \item Liquidation Rights-In the event of liquidation, dissolution, or winding-up of Viacom, all stockholders of common stock, regardless of class, will be entitled to share ratably in any assets available for distributions to stockholders of shares of Viacom common stock subject to the preferential rights of any outstanding preferred stock.

  \item Split, Subdivision, or Combination-In the event of a split, subdivision, or combination of the outstanding shares of Class A common stock or Class B common stock, the outstanding shares of the other class of common stock will be divided proportionally.

  \item Preemptive Rights-Shares of Class A common stock and Class B common stock do not entitle a stockholder to any preemptive rights enabling a stockholder to subscribe for or receive shares of stock of any class or any other securities convertible into shares of stock of any class of Viacom.

\end{itemize}

As seen in Exhibit 7, companies can issue different classes of common shares (Class A and Class B shares), with each class offering different ownership rights. ${ }^{8}$ For example, as shown in Exhibit 8, the Ford Motor Company has Class A shares ("Common Stock"), which are owned by the investing public. It also has Class B shares, which are owned only by the Ford family. The exhibit contains an excerpt from Ford's 2017 Аnnual Report (p. 144). Class A shareholders have 60 percent voting rights, whereas Class B shareholders have 40 percent. In the case of liquidation, however, Class B shareholders will not only receive the first US $\$ 0.50$ per share that is available for dis-

7 This information has been adapted from Viacom's investor relations website and its 10-K filing with the US Securities and Exchange Commission; see \href{http://www.viacom.com}{www.viacom.com}.

8 In some countries, including the United States, companies can issue different classes of shares, with Class A shares being the most common. The role and function of different classes of shares is described in more detail in Exhibit 8. tribution (as will Class A shareholders), but they will also receive the next US\$1.00 per share that is available for distribution before Class A shareholders receive anything else. Thus, Class B shareholders have an opportunity to receive a larger proportion of distributions upon liquidation than do Class A shareholders. ${ }^{9}$

Exhibit 8: Share Class Arrangements at Ford Motor Company ${ }^{10}$

\section{NOTE 21. CAPITAL STOCK AND AMOUNTS PER SHARE}
All general voting power is vested in the holders of Common Stock and Class B Stock. Holders of our Common Stock have $60 \%$ of the general voting power and holders of our Class B Stock are entitled to such number of votes per share as will give them the remaining $40 \%$. Shares of Common Stock and Class B Stock share equally in dividends when and as paid, with stock dividends payable in shares of stock of the class held.

If liquidated, each share of Common Stock is entitled to the first $\$ 0.50$ available for distribution to holders of Common Stock and Class B Stock, each share of Class B Stock is entitled to the next $\$ 1.00$ so available, each share of Common Stock is entitled to the next $\$ 0.50$ so available, and each share of Common and Class B Stock is entitled to an equal amount thereafter.

\section{Preference Shares}
Preference shares (or preferred stock) rank above common shares with respect to the payment of dividends and the distribution of the company's net assets upon liquidation. ${ }^{11}$ However, preference shareholders generally do not share in the operating performance of the company and do not have any voting rights, unless explicitly allowed for at issuance. Preference shares have characteristics of both debt securities and common shares. Similar to the interest payments on debt securities, the dividends on preference shares are fixed and are generally higher than the dividends on common shares. However, unlike interest payments, preference dividends are not contractual obligations of the company. Similar to common shares, preference shares can be perpetual (i.e., no fixed maturity date), can pay dividends indefinitely, and can be callable or putable.

Exhibit 9 provides an example of callable preference shares issued by the GDL Fund to raise capital to redeem the remaining outstanding Series B Preferred shares. In this case, the purchaser of the shares will receive an ongoing dividend from the GDL Fund. If the GDL Fund chooses to buy back the shares, it must do so at the $\$ 50$ a share liquidation preference price. The purchasers of the shares also have the right to put back the shares to GDL at the $\$ 50$ a share price.

9 For example, if US $\$ 2.00$ per share is available for distribution, the Common Stock (Class A) shareholders will receive US $\$ 0.50$ per share, while the Class B shareholders will receive US $\$ 1.50$ per share. However, if there is US $\$ 3.50$ per share available for distribution, the Common Stock shareholders will receive a total of US\$1.50 per share and the Class B shareholders will receive a total of US $\$ 2.00$ per share.

10 Extracted from Ford Motor Company's 2017 Annual Report (\href{http://s22.q4cdn.com/857684434/files/}{http://s22.q4cdn.com/857684434/files/} doc\_financials/2017/annual/Final-Annual-Report-2017.pdf).

11 Preference shares have a lower priority than debt in the case of liquidation. That is, debt holders have a higher claim on a firm's assets in the event of liquidation and will receive what is owed to them first, followed by preference shareholders and then common shareholders.

\section{Exhibit 9: Callable Stock offering by the GDL Fund ${ }^{12}$}
RYE, NY-March 26, 2018-The GDL Fund (NYSE:GDL) (the "Fund") is pleased to announce the completion of a rights offering (the "Offering") in which the Fund issued 2,624,025 Series C Cumulative Puttable and Callable Preferred Shares (the "Series C Preferred"), totaling $\$ 131,201,250$. Pursuant to the Offering, the Fund issued one non-transferable right (a "Right") for each outstanding Series B Cumulative Puttable and Callable Preferred Share (the "Series B Preferred") of the Fund to Series B Preferred shareholders of record as of February 14, 2018. Holders of Rights were entitled to purchase the Series C Preferred with any combination of cash or surrender of the Series B Preferred at liquidation preference. Therefore, one Right plus $\$ 50.00$, or one Right plus one share of Series B Preferred with a liquidation value of $\$ 50.00$ per share, was required to purchase each share of the Series C Preferred. The Offering expired at 5:00 PM Eastern Time on March 20, 2018.

Dividends on preference shares can be cumulative, non-cumulative, participating, non-participating, or some combination thereof (i.e., cumulative participating, cumulative non-participating, non-cumulative participating, non-cumulative non-participating).

Dividends on cumulative preference shares accrue so that if the company decides not to pay a dividend in one or more periods, the unpaid dividends accrue and must be paid in full before dividends on common shares can be paid. In contrast, non-cumulative preference shares have no such provision. This means that any dividends that are not paid in the current or subsequent periods are forfeited permanently and are not accrued over time to be paid at a later date. However, the company is still not permitted to pay any dividends to common shareholders in the current period unless preferred dividends have been paid first.

Participating preference shares entitle the shareholders to receive the standard preferred dividend plus the opportunity to receive an additional dividend if the company's profits exceed a pre-specified level. In addition, participating preference shares can also contain provisions that entitle shareholders to an additional distribution of the company's assets upon liquidation, above the par (or face) value of the preference shares. Non-participating preference shares do not allow shareholders to share in the profits of the company. Instead, shareholders are entitled to receive only a fixed dividend payment and the par value of the shares in the event of liquidation. The use of participating preference shares is much more common for smaller, riskier companies where the possibility of future liquidation is more of a concern to investors.

Preference shares can also be convertible. Convertible preference shares entitle shareholders to convert their shares into a specified number of common shares. This conversion ratio is determined at issuance. Convertible preference shares have the following advantages:

\begin{itemize}
  \item They allow investors to earn a higher dividend than if they invested in the company's common shares.

  \item They allow investors the opportunity to share in the profits of the company.

  \item They allow investors to benefit from a rise in the price of the common shares through the conversion option.

  \item Their price is less volatile than the underlying common shares because the dividend payments are known and more stable.

\end{itemize}

12 \href{https://www.businesswire.com/news/home/20180326005609/en/GDL-Fund-Successfully-Complete}{https://www.businesswire.com/news/home/20180326005609/en/GDL-Fund-Successfully-Complete} s-Offering-Issues-131 As a result, the use of convertible preference shares is a popular financing option in venture capital and private equity transactions in which the issuing companies are considered to be of higher risk and when it may be years before the issuing company "goes public" (i.e., issues common shares to the public).

Exhibit 10 provides examples of the types and characteristics of preference shares as issued by Tsakos Energy Navigation Ltd (TNP.PRE).

\section{Exhibit 10: Examples of Preference Shares Issued by TEN Ltd ${ }^{13}$}
Athens, Greece, June 21, 2018-TEN Ltd. (“TEN”) (NYSE: TNP), a leading diversified crude, product and LNG tanker operator, today announced the pricing of its public offering of its Series F Fixed-to-Floating Rate Cumulative Redeemable Perpetual Preferred Shares, par value $\$ 1.00$ per share, liquidation preference \$25.00 per share ("Series F Preferred Shares"). TEN will issue 5,400,000 Series F Preferred Shares at a price to the public of $\$ 25.00$ per share. Dividends will be payable on the Series F Preferred Shares to July 30, 2028 at a fixed rate equal to $9.50 \%$ per annum and from July 30,2028 , if not redeemed, at a floating rate. In connection with the offering, TEN has granted the underwriters a 30-day option to purchase 810,000 additional Series F Preferred Shares, which, if exercised in full, would result in total gross proceeds of $\$ 155,250,000$. TEN intends to use the net proceeds from the offering for general corporate purposes, which may include making vessel acquisitions and/or strategic investments and preferred share redemptions. Following the offering, TEN intends to file an application to list the Series F Preferred Shares on the New York Stock Exchange. The offering is expected to close on or about June $28,2018$.

\section{PRIVATE VERSUS PUBLIC EQUITY SECURITIES}
compare and contrast public and private equity securities

Our discussion so far has focused on equity securities that are issued and traded in public markets and on exchanges. Equity securities can also be issued and traded in private equity markets. Private equity securities are issued primarily to institutional investors via non-public offerings, such as private placements. Because they are not listed on public exchanges, there is no active secondary market for these securities. As a result, private equity securities do not have "market determined" quoted prices, are highly illiquid, and require negotiations between investors in order to be traded. In addition, financial statements and other important information needed to determine the fair value of private equity securities may be difficult to obtain because the issuing companies are typically not required by regulatory authorities to publish this information.

There are three primary types of private equity investments: venture capital, leveraged buyouts, and private investment in public equity (or PIPE). Venture capital investments provide "seed" or start-up capital, early-stage financing, or mezzanine financing to companies that are in the early stages of development and require additional capital for expansion. These funds are then used to finance the company's product development and growth. Venture capitalists range from family and friends to wealthy

13 \href{https://www.tenn.gr/wp-content/uploads/2018/06/tenn062118.pdf}{https://www.tenn.gr/wp-content/uploads/2018/06/tenn062118.pdf} individuals and private equity funds. Because the equity securities issued to venture capitalists are not publicly traded, they generally require a commitment of funds for a relatively long period of time; the opportunity to "exit" the investment is typically within 3 to 10 years from the initial start-up. The exit return earned by these private equity investors is based on the price that the securities can be sold for if and when the start-up company first goes public, either via an initial public offering (IPO) on the stock market or by being sold to other investors.

A leveraged buyout (LBO) occurs when a group of investors (such as the company's management or a private equity partnership) uses a large amount of debt to purchase all of the outstanding common shares of a publicly traded company. In cases where the group of investors acquiring the company is primarily comprised of the company's existing management, the transaction is referred to as a management buyout (MBO). After the shares are purchased, they cease to trade on an exchange and the investor group takes full control of the company. In other words, the company is taken "private" or has been privatized. Companies that are candidates for these types of transactions generally have large amounts of undervalued assets (which can be sold to reduce debt) and generate high levels of cash flows (which are used to make interest and principal payments on the debt). The ultimate objective of a buyout (LBO or MBO) is to restructure the acquired company and later take it "public" again by issuing new shares to the public in the primary market.

The third type of private investment is a private investment in public equity, or PIPE. ${ }^{14}$ This type of investment is generally sought by a public company that is in need of additional capital quickly and is willing to sell a sizeable ownership position to a private investor or investor group. For example, a company may require a large investment of new equity funds in a short period of time because it has significant expansion opportunities, is facing high levels of indebtedness, or is experiencing a rapid deterioration in its operations. Depending on how urgent the need is and the size of the capital requirement, the private investor may be able to purchase shares in the company at a significant discount to the publicly-quoted market price. Exhibit 11 contains a recent PIPE transaction for the health care company TapImmune, which also included the proposed merger with Maker Therapeutics.

\section{Exhibit 11: Example of a PIPE Transaction ${ }^{15}$}
JACKSONVILLE, Florida, June 8, 2018-TapImmune Inc. (NASDAQ: TPIV), a clinical-stage immuno-oncology company, today announced that it has entered into security purchase agreements with certain institutional and accredited investors in connection with a private placement of its equity securities. The private placement will be led by New Enterprise Associates (NEA) with participation from Aisling Capital and Perceptive Advisors, among other new and existing investors. The private placement is expected to be completed concurrently with the closing of the proposed merger between TapImmune Inc. and Marker Therapeutics, Inc., which was previously announced on May 15, 2018.

Upon closing the private placement, TapImmune will issue 17,500,000 shares of its common stock at a price of $\$ 4.00$ per share. The aggregate offering size, before deducting the placement agent fees and other offering expenses, is expected to be $\$ 70$ million. Additionally, TapImmune will issue warrants to purchase $13,125,000$ shares of TapImmune common stock at an exercise price of $\$ 5.00$ per share that will be exercisable for a period of five years from the date of issuance. The closing of the transaction, which is subject to the closing of the merger with

14 The term PIPE is widely used in the United States and is also used internationally, including in emerging markets.

15 \href{https://tapimmune.com/2018/06/tapimmune-announces-pricing-of-70-million-private-placement/}{https://tapimmune.com/2018/06/tapimmune-announces-pricing-of-70-million-private-placement/} Marker, the approval by TapImmune's stockholders as required by NASDAQ Stock Market Rules, and other customary closing conditions, is anticipated to occur by the end of the third quarter of 2018.

While the global private equity market is relatively small in comparison to the global public equity market, it has experienced considerable growth over the past three decades. According to a study of the private equity market sponsored by the World Economic Forum and spanning the period 1970-2007, approximately US $\$ 3.6$ trillion in debt and equity were acquired in leveraged buyouts. Of this amount, approximately 75 percent or US\$2.7 trillion worth of transactions occurred during 2001-2007. ${ }^{16}$ This pace continued with a further US $\$ 2.9$ trillion in transactions occurring during 2008-2017. ${ }^{17}$ While the US and the UK markets were the focus of most private equity investments during the 1980s and 1990s, private equity investments outside of these markets have grown substantially in recent years. In addition, the number of companies operating under private equity ownership has also grown. For example, during the mid-1990s, fewer than 2,000 companies were under LBO ownership compared to more than 20,000 companies that were under LBO ownership globally at the beginning of 2017. The holding period for private equity investments has also increased during this time period from 3 to 5 years (1980s and 1990s) to approximately 10 years. ${ }^{18}$

The move to longer holding periods has given private equity investors the opportunity to more effectively and patiently address any underlying operational issues facing the company and to better manage it for long-term value creation. Because of the longer holding periods, more private equity firms are issuing convertible preference shares because they provide investors with greater total return potential through their dividend payments and the ability to convert their shares into common shares during an IPO.

In operating a publicly traded company, management often feels pressured to focus on short-term results ${ }^{19}$ (e.g., meeting quarterly sales and earnings targets from analysts biased toward near-term price performance) instead of operating the company to obtain long-term sustainable revenue and earnings growth. By "going private," management can adopt a more long-term focus and can eliminate certain costs that are necessary to operate a publicly traded company-such as the cost of meeting regulatory and stock exchange filing requirements, the cost of maintaining investor relations departments to communicate with shareholders and the media, and the cost of holding quarterly analyst conference calls.

As described above, public equity markets are much larger than private equity networks and allow companies more opportunities to raise capital that is subsequently actively traded in secondary markets. By operating under public scrutiny, companies are incentivized to be more open in terms of corporate governance and executive compensation to ensure that they are acting for the benefit of shareholders. In fact, some studies have shown that private equity firms score lower in terms of corporate governance effectiveness, which may be attributed to the fact that shareholders, analysts, and other stakeholders are able to influence management when corporate governance and other policies are public.

16 Stromberg (2008).

17 \href{https://www.statista.com/statistics/270195/global-private-equity-deal-value/}{https://www.statista.com/statistics/270195/global-private-equity-deal-value/}

18 See, for example, Bailey, Wirth, and Zapol (2005).

19 See, for example, Graham, Harvey, and Rajgopal (2005).

\section{NON-DOMESTIC EQUITY SECURITIES}
describe methods for investing in non-domestic equity securities

Technological innovations and the growth of electronic information exchanges (electronic trading networks, the internet, etc.) have accelerated the integration and growth of global financial markets. As detailed previously, global capital markets have expanded at a much more rapid rate than global GDP in recent years; both primary and secondary international markets have benefited from the enhanced ability to rapidly and openly exchange information. Increased integration of equity markets has made it easier and less expensive for companies to raise capital and to expand their shareholder base beyond their local market. Integration has also made it easier for investors to invest in companies that are located outside of their domestic markets. This has enabled investors to further diversify and improve the risk and return characteristics of their portfolios by adding a class of assets with lower correlations to local country assets.

One barrier to investing globally is that many countries still impose "foreign restrictions" on individuals and companies from other countries that want to invest in their domestic companies. There are three primary reasons for these restrictions. The first is to limit the amount of control that foreign investors can exert on domestic companies. For example, some countries prevent foreign investors from acquiring a majority interest in domestic companies. The second is to give domestic investors the opportunity to own shares in the foreign companies that are conducting business in their country. For example, the Swedish home furnishings retailer IKEA abandoned efforts to invest in parts of the Asia/Pacific region because local governments did not want IKEA to maintain complete ownership of its stores. The third reason is to reduce the volatility of capital flows into and out of domestic equity markets. For example, one of the main consequences of the Asian Financial Crisis in 1997-98 was the large outflow of capital from such emerging market countries as Thailand, Indonesia, and South Korea. These outflows led to dramatic declines in the equity markets of these countries and significant currency devaluations and resulted in many governments placing restrictions on capital flows. Today, many of these same markets have built up currency reserves to better withstand capital outflows inherent in economic contractions and periods of financial market turmoil.

Studies have shown that reducing restrictions on foreign ownership has led to improved equity market performance over the long term. ${ }^{20}$ Although restrictions vary widely, more countries are allowing increasing levels of foreign ownership. For example, Australia has sought tax reforms as a means to encourage international demand for its managed funds in order to increase its role as an international financial center.

Over the past two decades, three trends have emerged: a) an increasing number of companies have issued shares in markets outside of their home country; b) the number of companies whose shares are traded in markets outside of their home has increased; and c) an increasing number of companies are dual listed, which means that their shares are simultaneously issued and traded in two or more markets. Companies located in emerging markets have particularly benefited from these trends because they no longer have to be concerned with capital constraints or lack of liquidity in their domestic markets. These companies have found it easier to raise capital in the markets of developed countries because these markets generally have higher levels of liquidity and more stringent financial reporting requirements and accounting standards.

20 See, for example, Henry and Chari (2004). Being listed on an international exchange has a number of benefits. It can increase investor awareness about the company's products and services, enhance the liquidity of the company's shares, and increase corporate transparency because of the additional market exposure and the need to meet a greater number of filing requirements.

Technological advancements have made it easier for investors to trade shares in foreign markets. The German insurance company Allianz SE recently delisted its shares from the NYSE and certain European markets because international investors increasingly traded its shares on the Frankfurt Stock Exchange. Exhibit 12 illustrates the extent to which the institutional shareholder base at BASF, a large German chemical corporation, has become increasingly global in nature.

\section{Exhibit 12: Example of Increased Globalization of Share Ownership 21}
BASF is one of the largest publicly owned companies with over 500,000 shareholders and a high free float. An analysis of the shareholder structure carried out in March 2018 showed that, at $21 \%$ of share capital, the United States and Canada made up the largest regional group of institutional investors. Institutional investors from Germany made up 12\%. Shareholders from United Kingdom and Ireland held $12 \%$ of BASF shares, while a further $17 \%$ are held by institutional investors from the rest of Europe. Around 28\% of the company's share capital is held by private investors, most of whom are resident in Germany.

\begin{center}
\includegraphics[max width=\textwidth]{2023_05_04_7b535d0a870224f62e3dg-310}
\end{center}

\section{Direct Investing}
Investors can use a variety of methods to invest in the equity of companies outside of their local market. The most obvious is to buy and sell securities directly in foreign markets. However, this means that all transactions-including the purchase and sale of shares, dividend payments, and capital gains-are in the company's, not the investor's, domestic currency. In addition, investors must be familiar with the trading, clearing, and settlement regulations and procedures of that market. Investing directly often results in less transparency and more volatility because audited financial information may not be provided on a regular basis and the market may be less liquid. Alternatively, investors can use such securities as depository receipts and global registered shares,

21 Adapted from BASF's investor relations website (\href{http://www.basf.com}{www.basf.com}). Free float refers to the extent that shares are readily and freely tradable in the secondary market. which represent the equity of international companies and are traded on local exchanges and in the local currencies. With these securities, investors have to worry less about currency conversions (price quotations and dividend payments are in the investor's local currency), unfamiliar market practices, and differences in accounting standards. The sections that follow discuss various securities that investors can invest in outside of their home market.

\section{Depository Receipts}
A depository receipt ${ }^{22}$ (DR) is a security that trades like an ordinary share on a local exchange and represents an economic interest in a foreign company. It allows the publicly listed shares of a foreign company to be traded on an exchange outside its domestic market. A depository receipt is created when the equity shares of a foreign company are deposited in a bank (i.e., the depository) in the country on whose exchange the shares will trade. The depository then issues receipts that represent the shares that were deposited. The number of receipts issued and the price of each DR is based on a ratio, which specifies the number of depository receipts to the underlying shares. Consequently, a DR may represent one share of the underlying stock, many shares of the underlying stock, or a fractional share of the underlying stock. The price of each DR will be affected by factors that affect the price of the underlying shares, such as company fundamentals, market conditions, analysts' recommendations, and exchange rate movements. In addition, any short-term valuation discrepancies between shares traded on multiple exchanges represent a quick arbitrage profit opportunity for astute traders to exploit. The responsibilities of the depository bank that issues the receipts include acting as custodian and as a registrar. This entails handling dividend payments, other taxable events, stock splits, and serving as the transfer agent for the foreign company whose securities the DR represents. The Bank of New York Mellon is the largest depository bank; however, Deutsche Bank, JPMorgan, and Citibank also offer depository services. ${ }^{23}$

A DR can be sponsored or unsponsored. A sponsored DR is when the foreign company whose shares are held by the depository has a direct involvement in the issuance of the receipts. Investors in sponsored DRs have the same rights as the direct owners of the common shares (e.g., the right to vote and the right to receive dividends). In contrast, with an unsponsored DR, the underlying foreign company has no involvement with the issuance of the receipts. Instead, the depository purchases the foreign company's shares in its domestic market and then issues the receipts through brokerage firms in the depository's local market. In this case, the depository bank, not the investors in the DR, retains the voting rights. Sponsored DRs are generally subject to greater reporting requirements than unsponsored DRs. In the United States, for example, sponsored DRs must be registered (meet the reporting requirements) with the US Securities and Exchange Commission (SEC). Exhibit 13 contains an example of a sponsored DR issued by Alibaba in September 2014.

\section{Exhibit 13: Sponsored Depository Receipts ${ }^{24}$}
NEW YORK-(BUSINESS WIRE)-Citi today announced that Alibaba Group Holding Limited (“Alibaba Group") has appointed Citi’s Issuer Services business, acting through Citibank, N.A., as the depositary bank for its American Depositary

22 Note that the spellings depositary and depository are used interchangeably in financial markets. In this reading, we use the spelling depository throughout

23 Boubakri, Cosset, and Samet (2010)

24 \href{https://www.businesswire.com/news/home/20140924005984/en/Citi-Appointed-Depositary-Ban}{https://www.businesswire.com/news/home/20140924005984/en/Citi-Appointed-Depositary-Ban} k-Alibaba-Group-Holding Receipt ("ADR") program. Alibaba Group's ADRs, which began trading on September 19, 2014, represent the largest Depositary Receipt program in initial public offering market history.

Alibaba Group's ADR program was established through a $\$ 25.03$ billion initial public offering of 368,122,000 American Depositary Shares (“ADSs"), representing ordinary shares of Alibaba Group, which was priced at $\$ 68$ per ADS on September 18, 2014. The IPO ranks as the largest in history. The ADRs are listed on the New York Stock Exchange (the "NYSE") under the trading symbol BABA. Each ADS represents one ordinary share of the Company. In its role as depositary bank, Citibank will hold the underlying ordinary shares through its local custodian and issue ADSs representing such shares. Alibaba Group's ADSs trade on the NYSE in ADR form.

There are two types of depository receipts: Global depository receipts (GDRs) and American depository receipts (ADRs), which are described below.

\section{Global Depository Receipts}
A global depository receipt (GDR) is issued outside of the company's home country and outside of the United States. The depository bank that issues GDRs is generally located (or has branches) in the countries on whose exchanges the shares are traded. A key advantage of GDRs is that they are not subject to the foreign ownership and capital flow restrictions that may be imposed by the issuing company's home country because they are sold outside of that country. The issuing company selects the exchange where the GDR is to be traded based on such factors as investors' familiarity with the company or the existence of a large international investor base. The London and Luxembourg exchanges were the first ones to trade GDRs. Some other stock exchanges trading GDRs are the Dubai International Financial Exchange and the Singapore Stock Exchange. Currently, the London and Luxembourg exchanges are where most GDRs are traded because they can be issued in a more timely manner and at a lower cost. Regardless of the exchange they are traded on, the majority of GDRs are denominated in US dollars, although the number of GDRs denominated in pound sterling and euros is increasing. Note that although GDRs cannot be listed on US exchanges, they can be privately placed with institutional investors based in the United States.

\section{American Depository Receipts}
An American depository receipt (ADR) is a US dollar-denominated security that trades like a common share on US exchanges. First created in 1927, ADRs are the oldest type of depository receipts and are currently the most commonly traded depository receipts. They enable foreign companies to raise capital from US investors. Note that an ADR is one form of a GDR; however, not all GDRs are ADRs because GDRs cannot be publicly traded in the United States. The term American depository share (ADS) is often used in tandem with the term ADR. A depository share is a security that is actually traded in the issuing company's domestic market. That is, while American depository receipts are the certificates that are traded on US markets, American depository shares are the underlying shares on which these receipts are based.

There are four primary types of ADRs, with each type having different levels of corporate governance and filing requirements. Level I Sponsored ADRs trade in the over-the-counter (OTC) market and do not require full registration with the Securities and Exchange Commission (SEC). Level II and Level III Sponsored ADRs can trade on the New York Stock Exchange (NYSE), NASDAQ, and American Stock Exchange (AMEX). Level II and III ADRs allow companies to raise capital and make acquisitions using these securities. However, the issuing companies must fulfill all SEC requirements. The fourth type of ADR, an SEC Rule 144A or a Regulation S depository receipt, does not require SEC registration. Instead, foreign companies are able to raise capital by privately placing these depository receipts with qualified institutional investors or to offshore non-US investors. Exhibit 14 summarizes the main features of ADRs.

Exhibit 14: Summary of the Main Features of American Depository Receipts

\begin{center}
\begin{tabular}{|c|c|c|c|c|}
\hline
 & $\begin{array}{l}\text { Level I } \\ \text { (Unlisted) }\end{array}$ & $\begin{array}{l}\text { Level II } \\ \text { (Listed) }\end{array}$ & $\begin{array}{l}\text { Level III } \\ \text { (Listed) }\end{array}$ & $\begin{array}{l}\text { Rule 144A } \\ \text { (Unlisted) }\end{array}$ \\
\hline
Objectives & $\begin{array}{l}\text { Develop and broaden } \\ \text { US investor base with } \\ \text { existing shares }\end{array}$ & $\begin{array}{l}\text { Develop and broaden } \\ \text { US investor base with } \\ \text { existing shares }\end{array}$ & $\begin{array}{l}\text { Develop and broaden } \\ \text { US investor base with } \\ \text { existing/new shares }\end{array}$ & $\begin{array}{l}\text { Access qualified institu- } \\ \text { tional buyers (QIBs) }\end{array}$ \\
\hline
$\begin{array}{l}\text { Raising capital on US } \\ \text { markets? }\end{array}$ & No & No & $\begin{array}{l}\text { Yes, through public } \\ \text { offerings }\end{array}$ & $\begin{array}{l}\text { Yes, through private } \\ \text { placements to QIBs }\end{array}$ \\
\hline
SEC registration & Form F-6 & Form F-6 & Forms F-1 and F-6 & None \\
\hline
Trading & $\begin{array}{l}\text { Over the counter } \\ \text { (OTC) }\end{array}$ & $\begin{array}{l}\text { NYSE, NASDAQ, or } \\ \text { AMEX }\end{array}$ & $\begin{array}{l}\text { NYSE, NASDAQ, or } \\ \text { AMEX }\end{array}$ & $\begin{array}{l}\text { Private offerings, } \\ \text { resales, and trading } \\ \text { through automated } \\ \text { linkages such as } \\ \text { PORTAL }\end{array}$ \\
\hline
Listing fees & Low & High & High & Low \\
\hline
$\begin{array}{l}\text { Size and earnings } \\ \text { requirements }\end{array}$ & None & Yes & Yes & None \\
\hline
\end{tabular}
\end{center}

Source: Adapted from Boubakri, Cosset, and Samet (2010): Table 1.

More than 2,000 DRs, from over 80 countries, currently trade on US exchanges. Based on current statistics, the total market value of DRs issued and traded is estimated at approximately US 2 trillion, or 15 percent of the total dollar value of equities traded in US markets. ${ }^{25}$

\section{Global Registered Share}
A global registered share (GRS) is a common share that is traded on different stock exchanges around the world in different currencies. Currency conversions are not needed to purchase or sell them, because identical shares are quoted and traded in different currencies. Thus, the same share purchased on the Swiss exchange in Swiss francs can be sold on the Tokyo exchange for Japanese yen. As a result, GRSs offer more flexibility than depository receipts because the shares represent an actual ownership interest in the company that can be traded anywhere and currency conversions are not needed to purchase or sell them. GRSs were created and issued by Daimler Chrysler in 1998 and by UBS AG in 2011.

\section{Basket of Listed Depository Receipts}
Another type of global security is a basket of listed depository receipts (BLDR), which is an exchange-traded fund (ETF) that represents a portfolio of depository receipts. An ETF is a security that tracks an index but trades like an individual share on an exchange. An equity-ETF is a security that contains a portfolio of equities that tracks an index. It trades throughout the day and can be bought, sold, or sold short, just like an individual share. Like ordinary shares, ETFs can also be purchased on margin and used in hedging or arbitrage strategies. The BLDR is a specific class of

25 JPMorgan Depositary Receipt Guide (2005):4. ETF security that consists of an underlying portfolio of DRs and is designed to track the price performance of an underlying DR index. For example, the Invesco BLDRS Asia 50 ADR Index Fund is a capitalization-weighted ETF designed to track the performance of 50 Asian market-based ADRs.

\section{RISK AND RETURN CHARACTERISTICS}
compare the risk and return characteristics of different types of equity securities

Different types of equity securities have different ownership claims on a company's net assets. The type of equity security and its features affect its risk and return characteristics. The following sections discuss the different return and risk characteristics of equity securities.

\section{Return Characteristics of Equity Securities}
There are two main sources of equity securities' total return: price change (or capital gain) and dividend income. The price change represents the difference between the purchase price $\left(P_{t-1}\right)$ and the sale price $\left(P_{t}\right)$ of a share at the end of time $t-1$ and $t$, respectively. Cash or stock dividends $\left(D_{t}\right)$ represent distributions that the company makes to its shareholders during period $t$. Therefore, an equity security's total return is calculated as:

$$
\text { Total return, } R_{t}=\left(P_{t}-P_{t-1}+D_{t}\right) / P_{t-1}
$$

For non-dividend-paying stocks, the total return consists of price appreciation only. Companies that are in the early stages of their life cycle generally do not pay dividends because earnings and cash flows are reinvested to finance the company's growth. In contrast, companies that are in the mature phase of their life cycle may not have as many profitable growth opportunities; therefore, excess cash flows are often returned to investors via the payment of regular dividends or through share repurchases.

For investors who purchase depository receipts or foreign shares directly, there is a third source of return: foreign exchange gains (or losses). Foreign exchange gains arise because of the change in the exchange rate between the investor's currency and the currency that the foreign shares are denominated in. For example, US investors who purchase the ADRs of a Japanese company will earn an additional return if the yen appreciates relative to the US dollar. Conversely, these investors will earn a lower total return if the yen depreciates relative to the US dollar. For example, if the total return for a Japanese company was 10 percent in Japan and the yen depreciated by 10 percent against the US dollar, the total return of the ADR would be (approximately) 0 percent. If the yen had instead appreciated by 10 percent against the US dollar, the total return of the ADR would be (approximately) 20 percent.

Investors that only consider price appreciation overlook an important source of return: the compounding that results from reinvested dividends. Reinvested dividends are cash dividends that the investor receives and uses to purchase additional shares. As Exhibit 15 shows, in the long run total returns on equity securities are dramatically influenced by the compounding effect of reinvested dividends. Between 1900 and 2016, US\$1 invested in US equities in 1900 would have grown in real terms to US $\$ 1,402$ with dividends reinvested, but to just US\$11.9 when taking only the price appreciation or capital gain into account. This corresponds to a real compounded return of 6.4 percent per year with dividends reinvested, versus only 2.1 percent per year without dividends reinvested. The comparable ending real wealth for bonds and bills are US $\$ 9.8$ and US $\$ 2.60$, respectively. These ending real wealth figures correspond to annualized real compounded returns of 2.0 percent on bonds and 0.8 percent on bills.

Exhibit 15: Impact of Reinvested Dividends on Cumulative Real Returns in the US and UK Equity Market: 1900-2016

\begin{center}
\includegraphics[max width=\textwidth]{2023_05_04_7b535d0a870224f62e3dg-315}
\end{center}

Source: Dimson, Marsh, and Staunton (2017). This chart is updated annually and can be found at \href{http://publications.credit-suisse.com/index.cfm/publikationen-shop/research-institute/}{http://publications.credit-suisse.com/index.cfm/publikationen-shop/research-institute/}.

\section{Risk of Equity Securities}
The risk of any security is based on the uncertainty of its future cash flows. The greater the uncertainty of its future cash flows, the greater the risk and the more variable or volatile the security's price. As discussed above, an equity security's total return is determined by its price change and dividends. Therefore, the risk of an equity security can be defined as the uncertainty of its expected (or future) total return. Risk is most often measured by calculating the standard deviation of the equity's expected total return.

A variety of different methods can be used to estimate an equity's expected total return and risk. One method uses the equity's average historical return and the standard deviation of this return as proxies for its expected future return and risk. Another method involves estimating a range of future returns over a specified period of time, assigning probabilities to those returns, and then calculating an expected return and a standard deviation of return based on this information.

The type of equity security, as well as its characteristics, affects the uncertainty of its future cash flows and therefore its risk. In general, preference shares are less risky than common shares for three main reasons:

\begin{enumerate}
  \item Dividends on preference shares are known and fixed, and they account for a large portion of the preference shares' total return. Therefore, there is less uncertainty about future cash flows.

  \item Preference shareholders receive dividends and other distributions before common shareholders. 3. The amount preference shareholders will receive if the company is liquidated is known and fixed as the par (or face) value of their shares. However, there is no guarantee that investors will receive that amount if the company experiences financial difficulty.

\end{enumerate}

With common shares, however, a larger portion of shareholders' total return (or all of their total return for non-dividend shares) is based on future price appreciation and future dividends are unknown. If the company is liquidated, common shareholders will receive whatever amount (if any) is remaining after the company's creditors and preference shareholders have been paid. In summary, because the uncertainty surrounding the total return of preference shares is less than common shares, preference shares have lower risk and lower expected return than common shares.

It is important to note that some preference shares and common shares can be riskier than others because of their associated characteristics. For example, from an investor's point of view, putable common or preference shares are less risky than their callable or non-callable counterparts because they give the investor the option to sell the shares to the issuer at a pre-determined price. This pre-determined price establishes a minimum price that investors will receive and reduces the uncertainty associated with the security's future cash flow. As a result, putable shares generally pay a lower dividend than non-putable shares.

Because the major source of total return for preference shares is dividend income, the primary risk affecting all preference shares is the uncertainty of future dividend payments. Regardless of the preference shares' features (callable, putable, cumulative, etc.), the greater the uncertainty surrounding the issuer's ability to pay dividends, the greater risk. Because the ability of a company to pay dividends is based on its future cash flows and net income, investors try to estimate these amounts by examining past trends or forecasting future amounts. The more earnings and the greater amount of cash flow that the company has had, or is expected to have, the lower the uncertainty and risk associated with its ability to pay future dividends.

Callable common or preference shares are riskier than their non-callable counterparts because the issuer has the option to redeem the shares at a pre-determined price. Because the call price limits investors' potential future total return, callable shares generally pay a higher dividend to compensate investors for the risk that the shares could be called in the future. Similarly, putable preference shares have lower risk than non-putable preference shares. Cumulative preference shares have lower risk than non-cumulative preference shares because the cumulative feature gives investors the right to receive any unpaid dividends before any dividends can be paid to common shareholders.

\section{EQUITY AND COMPANY VALUE}
\begin{center}
\includegraphics[max width=\textwidth]{2023_05_04_7b535d0a870224f62e3dg-316}
\end{center}

Companies issue equity securities on primary markets to raise capital and increase liquidity. This additional liquidity also provides the corporation an additional "currency" (its equity), which it can use to make acquisitions and provide stock option-based incentives to employees. The primary goal of raising capital is to finance the company's revenue-generating activities in order to increase its net income and maximize the wealth of its shareholders. In most cases, the capital that is raised is used to finance the purchase of long-lived assets, capital expansion projects, research and development, the entry into new product or geographic regions, and the acquisition of other companies. Alternatively, a company may be forced to raise capital to ensure that it continues to operate as a going concern. In these cases, capital is raised to fulfill regulatory requirements, improve capital adequacy ratios, or to ensure that debt covenants are met.

The ultimate goal of management is to increase the book value (shareholders' equity on a company's balance sheet) of the company and maximize the market value of its equity. Although management actions can directly affect the book value of the company (by increasing net income or by selling or purchasing its own shares), they can only indirectly affect the market value of its equity. The book value of a company's equity-the difference between its total assets and total liabilities-increases when the company retains its net income. The more net income that is earned and retained, the greater the company's book value of equity. Because management's decisions directly influence a company's net income, they also directly influence its book value of equity.

The market value of the company's equity, however, reflects the collective and differing expectations of investors concerning the amount, timing, and uncertainty of the company's future cash flows. Rarely will book value and market value be equal. Although management may be accomplishing its objective of increasing the company's book value, this increase may not be reflected in the market value of the company's equity because it does not affect investors' expectations about the company's future cash flows. A key measure that investors use to evaluate the effectiveness of management in increasing the company's book value is the accounting return on equity.

\section{Accounting Return on Equity}
Return on equity (ROE) is the primary measure that equity investors use to determine whether the management of a company is effectively and efficiently using the capital they have provided to generate profits. It measures the total amount of net income available to common shareholders generated by the total equity capital invested in the company. It is computed as net income available to ordinary shareholders (i.e., after preferred dividends have been deducted) divided by the average total book value of equity (BVE). That is:

\begin{center}
\includegraphics[max width=\textwidth]{2023_05_04_7b535d0a870224f62e3dg-317}
\end{center}

where $\mathrm{NI}_{t}$ is the net income in year $t$ and the average book value of equity is computed as the book values at the beginning and end of year $t$ divided by 2. Return on equity assumes that the net income produced in the current year is generated by the equity existing at the beginning of the year and any new equity that was invested during the year. Note that some formulas only use shareholders' equity at the beginning of year $t$ (that is, the end of year $t-1$ ) in the denominator. This assumes that only the equity existing at the beginning of the year was used to generate the company's net income during the year. That is:

$$
\mathrm{ROE}_{t}=\frac{\mathrm{NI}_{t}}{\mathrm{BVE}_{t-1}}
$$

Both formulas are appropriate to use as long as they are applied consistently. For example, using beginning of the year book value is appropriate when book values are relatively stable over time or when computing ROE for a company annually over a period of time. Average book value is more appropriate if a company experiences more volatile year-end book values or if the industry convention is to use average book values in calculating ROE.

One caveat to be aware of when computing and analyzing ROE is that net income and the book value of equity are directly affected by management's choice of accounting methods, such as those relating to depreciation (straight line versus accelerated methods) or inventories (first in, first out versus weighted average cost). Different accounting methods can make it difficult to compare the return on equity of companies even if they operate in the same industry. It may also be difficult to compare the ROE of the same company over time if its accounting methods have changed during that time.

Exhibit 16 contains information on the net income and total book value of shareholders' equity for three blue chip (widely held large market capitalization companies that are considered financially sound and are leaders in their respective industry or local stock market) pharmaceutical companies: Pfizer, Novartis AG, and GlaxoSmithKline. The data are for their financial years ending December 2015 through December 2017. ${ }^{26}$

Exhibit 16: Net Income and Book Value of Equity for Pfizer, Novartis AG, and GlaxoSmithKline (in Thousands of US Dollars)

Financial Year Ending

31 Dec 2015

31 Dec 2016

31 Dec 2017

Pfizer

Net income

\begin{center}
\begin{tabular}{lcc}
$\$ 6,960,000$ & $\$ 7,215,000$ & $\$ 21,308,000$ \\
$\$ 64,998,000$ & $\$ 59,840,000$ & $\$ 71,287,000$ \\
$\$ 17,783,000$ & $\$ 6,712,000$ & $\$ 7,703,000$ \\
$\$ 77,122,000$ & $\$ 74,891,000$ & $\$ 74,227,000$ \\
$\$ 12,420,000$ & $\$ 1,126,000$ & $\$ 2,070,700$ \\
$\$ 11,309,250$ & $\$ 6,127,800$ & $\$ 4,715,800$ \\
\hline
\end{tabular}
\end{center}

Using the average book value of equity, the return on equity for Pfizer for the years ending December 2016 and 2017 can be calculated as:

Return on equity for the year ending December 2016

$$
\mathrm{ROE}_{2016}=\frac{\mathrm{NI}_{2016}}{\left(\mathrm{BVE}_{2015}+\mathrm{BVE}_{2016}\right) / 2}=\frac{7,215,000}{(64,998,000+59,840,000) / 2}=11.6 \%
$$

Return on equity for the year ending December 2017

$$
\operatorname{ROE}_{2017}=\frac{\mathrm{NI}_{2017}}{\left(\mathrm{BVE}_{2016}+\mathrm{BVE}_{2017}\right) / 2}=\frac{21,308,000}{(59,840,000+71,287,000) / 2}=32.5 \%
$$

Exhibit 17 summarizes the return on equity for Novartis and GlaxoSmithKline in addition to Pfizer for 2016 and 2017.

26 Pfizer uses US GAAP to prepare its financial statements; Novartis and GlaxoSmithKline use International Financial Reporting Standards. Therefore, it would be inappropriate to compare the ROE of Pfizer to that of Novartis or GlaxoSmithKline.

\section{Exhibit 17: Return on Equity for Pfizer, Novartis AG, and}
 GlaxoSmithKline\begin{center}
\begin{tabular}{lcc}
\hline
 & 31 Dec 2016 (\%) & 31 Dec 2017 (\%) \\
\hline
Pfizer & 11.6 & 32.5 \\
Novartis AG & 8.8 & 10.3 \\
GlaxoSmithKline & 12.9 & 38.2 \\
\hline
\end{tabular}
\end{center}

In the case of Pfizer, the ROE of 32.5 percent in 2017 indicates that the company was able to generate a return (profit) of US $\$ 0.325$ on every US $\$ 1.00$ of capital invested by shareholders. GlaxoSmithKline almost tripled its return on equity over this period, from 12.9 percent to 38.2 percent. Novartis's ROE remained relatively unchanged.

ROE can increase if net income increases at a faster rate than shareholders' equity or if net income decreases at a slower rate than shareholders' equity. In the case of GlaxoSmithKline, ROE almost tripled between 2016 and 2017 due to its net income almost doubling during the period and due to its average shareholder's fund decreasing by almost 45 percent during the period. Stated differently, in 2017 compared to 2016, GlaxoSmithKline was significantly more effective in using its equity capital to generate profits. In the case of Pfizer, its ROE increased dramatically from 11.6 percent to 32.5 percent in 2017 versus 2016 even though its average shareholder equity increased by around 5 percent due to a nearly tripling of net income during the period.

An important question to ask is whether an increasing ROE is always good. The short answer is, "it depends." One reason ROE can increase is if net income decreases at a slower rate than shareholders' equity, which is not a positive sign. In addition, ROE can increase if the company issues debt and then uses the proceeds to repurchase some of its outstanding shares. This action will increase the company's leverage and make its equity riskier. Therefore, it is important to examine the source of changes in the company's net income and shareholders' equity over time. The DuPont formula, which is discussed in a separate reading, can be used to analyze the sources of changes in a company's ROE.

The book value of a company's equity reflects the historical operating and financing decisions of its management. The market value of the company's equity reflects these decisions as well as investors' collective assessment and expectations about the company's future cash flows generated by its positive net present value investment opportunities. If investors believe that the company has a large number of these future cash flow-generating investment opportunities, the market value of the company's equity will exceed its book value. Exhibit 18 shows the market price per share, the total number of shares outstanding, and the total book value of shareholders' equity for Pfizer, Novartis AG, and GlaxoSmithKline at the end of December 2017. This exhibit also shows the total market value of equity (or market capitalization) computed as the number of shares outstanding multiplied by the market price per share.

Exhibit 18: Market Information for Pfizer, Novartis AG, and GlaxoSmithKline (in Thousands of US Dollars except market price)

\begin{center}
\begin{tabular}{lccc}
\hline
 & Pfizer & Novartis AG & GlaxoSmithKline \\
\hline
Market price & $\$ 35.74$ & $\$ 90.99$ & $\$ 18.39$ \\
Total shares outstanding & $5,952,900$ & $2,317,500$ & $4,892,200$ \\
Total shareholders' equity & $\$ 71,287,000$ & $\$ 74,227,000$ & $\$ 4,715,800$ \\
\end{tabular}
\end{center}

\begin{center}
\begin{tabular}{lccc}
\hline
 & Pfizer & Novartis AG & GlaxoSmithKline \\
\hline
Total market value of equity & $\$ 212,756,646$ & $\$ 210,869,325$ & $\$ 89,967,558$ \\
\hline
\end{tabular}
\end{center}

Note that in Exhibit 18, the total market value of equity for Pfizer is computed as:

Market value of equity $=$ Market price per share $\times$ Shares outstanding

Market value of equity $=$ US $\$ 35.74 \times 5,952,900=$ US $\$ 212,756,646$.

The book value of equity per share for Pfizer can be computed as:

Book value of equity per share $=$ Total shareholders' equity $/$ Shares outstanding

Book value of equity per share $=$ US $\$ 71,287,000 / 5,952,900=$ US $\$ 11.98$.

A useful ratio to compute is a company's price-to-book ratio, which is also referred to as the market-to-book ratio. This ratio provides an indication of investors' expectations about a company's future investment and cash flow-generating opportunities. The larger the price-to-book ratio (i.e., the greater the divergence between market value per share and book value per share), the more favorably investors will view the company's future investment opportunities. For Pfizer the price-to-book ratio is:

Price-to-book ratio = Market price per share/Book value of equity per share

Price-to-book ratio $=$ US $\$ 35.74 / \mathrm{US} \$ 11.98=2.98$

Exhibit 19 contains the market price per share, book value of equity per share, and price-to-book ratios for Novartis and GlaxoSmithKline in addition to Pfizer.

\section{Exhibit 19: Pfizer, Novartis AG, and GlaxoSmithKline}
\begin{center}
\begin{tabular}{lccc}
\hline
 & Pfizer & Novartis AG & GlaxoSmithKline \\
\hline
Market price per share & $\$ 35.74$ & $\$ 90.99$ & $\$ 18.39$ \\
Book value of equity per share & $\$ 11.98$ & $\$ 32.03$ & $\$ 0.96$ \\
Price-to-book ratio & 2.98 & 2.84 & 19.16 \\
\hline
\end{tabular}
\end{center}

The market price per share of all three companies exceeds their respective book values, so their price-to-book ratios are all greater than 1.00 . However, there are significant differences in the sizes of their price-to-book ratios. GlaxoSmithKline has the largest price-to-book ratio, while the price-to-book ratios of Pfizer and Novartis are similar to each other. This suggests that investors believe that GlaxoSmithKline has substantially higher future growth opportunities than either Pfizer or Novartis.

It is not appropriate to compare the price-to-book ratios of companies in different industries because their price-to-book ratios also reflect investors' outlook for the industry. Companies in high growth industries, such as technology, will generally have higher price-to-book ratios than companies in slower growth (i.e., mature) industries, such as heavy equipment. Therefore, it is more appropriate to compare the price-to-book ratios of companies in the same industry. A company with relatively high growth opportunities compared to its industry peers would likely have a higher price-to-book ratio than the average price-to-book ratio of the industry.

Book value and return on equity are useful in helping analysts determine value but can be limited as a primary means to estimate a company's true or intrinsic value, which is the present value of its future projected cash flows. In Exhibit 20, Warren Buffett, one of the most successful investors in the world and CEO of Berkshire Hathaway, provides an explanation of the differences between the book value of a company and its intrinsic value in a letter to shareholders. As discussed above, market value reflects the collective and differing expectations of investors concerning the amount, timing, and uncertainty of a company's future cash flows. A company's intrinsic value can only be estimated because it is impossible to predict the amount and timing of its future cash flows. However, astute investors-such as Buffett-have been able to profit from discrepancies between their estimates of a company's intrinsic value and the market value of its equity.

\section{Exhibit 20: Book Value versus Intrinsic Value 27}
We regularly report our per-share book value, an easily calculable number, though one of limited use. Just as regularly, we tell you that what counts is intrinsic value, a number that is impossible to pinpoint but essential to estimate.

For example, in 1964, we could state with certitude that Berkshire's per-share book value was $\$ 19.46$. However, that figure considerably overstated the stock's intrinsic value since all of the company's resources were tied up in a sub-profitable textile business. Our textile assets had neither going-concern nor liquidation values equal to their carrying values. In 1964, then, anyone inquiring into the soundness of Berkshire's balance sheet might well have deserved the answer once offered up by a Hollywood mogul of dubious reputation: "Don't worry, the liabilities are solid."

Today, Berkshire's situation has reversed: Many of the businesses we control are worth far more than their carrying value. (Those we don't control, such as Coca-Cola or Gillette, are carried at current market values.) We continue to give you book value figures, however, because they serve as a rough, understated, tracking measure for Berkshire's intrinsic value.

We define intrinsic value as the discounted value of the cash that can be taken out of a business during its remaining life. Anyone calculating intrinsic value necessarily comes up with a highly subjective figure that will change both as estimates of future cash flows are revised and as interest rates move. Despite its fuzziness, however, intrinsic value is all-important and is the only logical way to evaluate the relative attractiveness of investments and businesses.

To see how historical input (book value) and future output (intrinsic value) can diverge, let's look at another form of investment, a college education. Think of the education's cost as its "book value." If it is to be accurate, the cost should include the earnings that were foregone by the student because he chose college rather than a job.

For this exercise, we will ignore the important non-economic benefits of an education and focus strictly on its economic value. First, we must estimate the earnings that the graduate will receive over his lifetime and subtract from that figure an estimate of what he would have earned had he lacked his education. That gives us an excess earnings figure, which must then be discounted, at an appropriate interest rate, back to graduation day. The dollar result equals the intrinsic economic value of the education.

\section{The Cost of Equity and Investors' Required Rates of Return}
When companies issue debt (or borrow from a bank) or equity securities, there is a cost associated with the capital that is raised. In order to maximize profitability and shareholder wealth, companies attempt to raise capital efficiently so as to minimize these costs.

27 Extracts from Berkshire Hathaway's 2008 Annual Report (\href{http://www.berkshirehathaway.com}{www.berkshirehathaway.com}). When a company issues debt, the cost it incurs for the use of these funds is called the cost of debt. The cost of debt is relatively easy to estimate because it reflects the periodic interest (or coupon) rate that the company is contractually obligated to pay to its bondholders (lenders). When a company raises capital by issuing equity, the cost it incurs is called the cost of equity. Unlike debt, however, the company is not contractually obligated to make any payments to its shareholders for the use of their funds. As a result, the cost of equity is more difficult to estimate.

Investors require a return on the funds they provide to the company. This return is called the investor's minimum required rate of return. When investors purchase the company's debt securities, their minimum required rate of return is the periodic rate of interest they charge the company for the use of their funds. Because all of the bondholders receive the same periodic rate of interest, their required rate of return is the same. Therefore, the company's cost of debt and the investors' minimum required rate of return on the debt are the same.

When investors purchase the company's equity securities, their minimum required rate of return is based on the future cash flows they expect to receive. Because these future cash flows are both uncertain and unknown, the investors' minimum required rate of return must be estimated. In addition, the minimum required return may differ across investors based on their expectations about the company's future cash flows. As a result, the company's cost of equity may be different from the investors' minimum required rate of return on equity. ${ }^{28}$ Because companies try to raise capital at the lowest possible cost, the company's cost of equity is often used as a proxy for the investors' minimum required rate of return.

In other words, the cost of equity can be thought of as the minimum expected rate of return that a company must offer its investors to purchase its shares in the primary market and to maintain its share price in the secondary market. If this expected rate of return is not maintained in the secondary market, then the share price will adjust so that it meets the minimum required rate of return demanded by investors. For example, if investors require a higher rate of return on equity than the company's cost of equity, they would sell their shares and invest their funds elsewhere resulting in a decline in the company's share price. As the share price declined, the cost of equity would increase to reach the higher rate of return that investors require.

Two models commonly used to estimate a company's cost of equity (or investors' minimum required rate of return) are the dividend discount model (DDM) and the capital asset pricing model (CAPM). These models are discussed in detail in other curriculum readings.

The cost of debt (after tax) and the cost of equity (i.e., the minimum required rates of return on debt and equity) are integral components of the capital budgeting process because they are used to estimate a company's weighted average cost of capital (WACC). Capital budgeting is the decision-making process that companies use to evaluate potential long-term investments. The WACC represents the minimum required rate of return that the company must earn on its long-term investments to satisfy all providers of capital. The company then chooses among those long-term investments with expected returns that are greater than its WACC.

28 Another important factor that can cause a firm's cost of equity to differ from investors' required rate of return on equity is the flotation cost associated with equity

\section{SUMMARY}
Equity securities play a fundamental role in investment analysis and portfolio management. The importance of this asset class continues to grow on a global scale because of the need for equity capital in developed and emerging markets, technological innovation, and the growing sophistication of electronic information exchange. Given their absolute return potential and ability to impact the risk and return characteristics of portfolios, equity securities are of importance to both individual and institutional investors.

This reading introduces equity securities and provides an overview of global equity markets. A detailed analysis of their historical performance shows that equity securities have offered average real annual returns superior to government bills and bonds, which have provided average real annual returns that have only kept pace with inflation. The different types and characteristics of common and preference equity securities are examined, and the primary differences between public and private equity securities are outlined. An overview of the various types of equity securities listed and traded in global markets is provided, including a discussion of their risk and return characteristics. Finally, the role of equity securities in creating company value is examined as well as the relationship between a company's cost of equity, its accounting return on equity, investors' required rate of return, and the company's intrinsic value.

We conclude with a summary of the key components of this reading:

\begin{itemize}
  \item Common shares represent an ownership interest in a company and give investors a claim on its operating performance, the opportunity to participate in the corporate decision-making process, and a claim on the company's net assets in the case of liquidation.

  \item Callable common shares give the issuer the right to buy back the shares from shareholders at a price determined when the shares are originally issued.

  \item Putable common shares give shareholders the right to sell the shares back to the issuer at a price specified when the shares are originally issued.

  \item Preference shares are a form of equity in which payments made to preference shareholders take precedence over any payments made to common stockholders.

  \item Cumulative preference shares are preference shares on which dividend payments are accrued so that any payments omitted by the company must be paid before another dividend can be paid to common shareholders. Non-cumulative preference shares have no such provisions, implying that the dividend payments are at the company's discretion and are thus similar to payments made to common shareholders.

  \item Participating preference shares allow investors to receive the standard preferred dividend plus the opportunity to receive a share of corporate profits above a pre-specified amount. Non-participating preference shares allow investors to simply receive the initial investment plus any accrued dividends in the event of liquidation.

  \item Callable and putable preference shares provide issuers and investors with the same rights and obligations as their common share counterparts.

  \item Private equity securities are issued primarily to institutional investors in private placements and do not trade in secondary equity markets. There are three types of private equity investments: venture capital, leveraged buyouts, and private investments in public equity (PIPE). - The objective of private equity investing is to increase the ability of the company's management to focus on its operating activities for long-term value creation. The strategy is to take the "private" company "public" after certain profit and other benchmarks have been met.

  \item Depository receipts are securities that trade like ordinary shares on a local exchange but which represent an economic interest in a foreign company. They allow the publicly listed shares of foreign companies to be traded on an exchange outside their domestic market.

  \item American depository receipts are US dollar-denominated securities trading much like standard US securities on US markets. Global depository receipts are similar to ADRs but contain certain restrictions in terms of their ability to be resold among investors.

  \item Underlying characteristics of equity securities can greatly affect their risk and return.

  \item A company's accounting return on equity is the total return that it earns on shareholders' book equity.

  \item A company's cost of equity is the minimum rate of return that stockholders require the company to pay them for investing in its equity.

\end{itemize}

\section{REFERENCES}
Bailey, Elizabeth, Meg Wirth, David Zapol. 2005. "Venture Capital and Global Health." Financing Global Health Ventures, Discussion Paper (September 2005): \href{http://www.commonscapital}{http://www.commonscapital}. com/downloads/Venture\_Capital\_and\_Global\_Health.pdf

Boubakri, Narjess, Jean-Claude Cosset, Anis Samet. 2010. “The Choice of ADRs." Journal of Banking and Finance, vol. 34, no. 9:2077-2095.

Dimson, Elroy, Paul Marsh, Mike Staunton. 2018. Credit Suisse Global Investment Returns Sourcebook 2017. Credit Suisse Research Institute.

Graham, John R., Campbell R. Harvey, Shiva Rajgopal. 2005. “The Economic Implications of Corporate Financial Reporting." Journal of Accounting and Economics, vol. 40, no. 1-3:3-73.

Henry, Peter Blair, Anusha Chari. 2004. "Risk Sharing and Asset Prices: Evidence from a Natural Experiment." Journal of Finance, vol. 59, no. 3:1295-1324.

Strömberg, Per. 2008. “The New Demography of Private Equity." TheGlobal Economic Impact of Private Equity Report 2008, World Economic Forum.

\section{PRACTICE PROBLEMS}
\begin{enumerate}
  \item Which of the following is not a characteristic of common equity?
\end{enumerate}

A. It represents an ownership interest in the company.

B. Shareholders participate in the decision-making process.

C. The company is obligated to make periodic dividend payments.

\begin{enumerate}
  \setcounter{enumi}{1}
  \item The type of equity voting right that grants one vote for each share of equity owned is referred to as:
A. proxy voting.
B. statutory voting.
C. cumulative voting.

  \item All of the following are characteristics of preference shares except:
A. They are either callable or putable.
B. They generally do not have voting rights.
C. They do not share in the operating performance of the company.

  \item Participating preference shares entitle shareholders to:

\end{enumerate}

A. participate in the decision-making process of the company.

B. convert their shares into a specified number of common shares.

C. receive an additional dividend if the company's profits exceed a pre-determined level.

\begin{enumerate}
  \setcounter{enumi}{4}
  \item Which of the following statements about private equity securities is incorrect?
A. They cannot be sold on secondary markets.
B. They have market-determined quoted prices.
C. They are primarily issued to institutional investors.

  \item Venture capital investments:

\end{enumerate}

A. can be publicly traded.

B. do not require a long-term commitment of funds.

C. provide mezzanine financing to early-stage companies.

\begin{enumerate}
  \setcounter{enumi}{6}
  \item Which of the following statements most accurately describes one difference between private and public equity firms?
\end{enumerate}

A. Private equity firms are focused more on short-term results than public firms.

B. Private equity firms' regulatory and investor relations operations are less costly than those of public firms. C. Private equity firms are incentivized to be more open with investors about governance and compensation than public firms.

\begin{enumerate}
  \setcounter{enumi}{7}
  \item Emerging markets have benefited from recent trends in international markets. Which of the following has not been a benefit of these trends?
\end{enumerate}

A. Emerging market companies do not have to worry about a lack of liquidity in their home equity markets.

B. Emerging market companies have found it easier to raise capital in the markets of developed countries.

C. Emerging market companies have benefited from the stability of foreign exchange markets.

\begin{enumerate}
  \setcounter{enumi}{8}
  \item When investing in unsponsored depository receipts, the voting rights to the shares in the trust belong to:
A. the depository bank.
B. the investors in the depository receipts.
C. the issuer of the shares held in the trust.

  \item With respect to Level III sponsored ADRs, which of the following is least likely to be accurate? They:

\end{enumerate}

A. have low listing fees.

B. are traded on the NYSE, NASDAQ, and AMEX.

C. are used to raise equity capital in US markets.

\begin{enumerate}
  \setcounter{enumi}{10}
  \item A basket of listed depository receipts, or an exchange-traded fund, would most likely be used for:
A. gaining exposure to a single equity.
B. hedging exposure to a single equity.
C. gaining exposure to multiple equities.

  \item Calculate the total return on a share of equity using the following data:

\end{enumerate}

Purchase price: $\$ 50$

Sale price: $\$ 42$

Dividend paid during holding period: $\$ 2$
A. $-12.0 \%$
B. $-14.3 \%$
C. $-16.0 \%$

\begin{enumerate}
  \setcounter{enumi}{12}
  \item If a US-based investor purchases a euro-denominated ETF and the euro subsequently depreciates in value relative to the dollar, the investor will have a total return that is:
\end{enumerate}

A. lower than the ETF's total return.

B. higher than the ETF's total return. C. the same as the ETF's total return.

\begin{enumerate}
  \setcounter{enumi}{13}
  \item Which of the following is incorrect about the risk of an equity security? The risk of an equity security is:
\end{enumerate}

A. based on the uncertainty of its cash flows.

B. based on the uncertainty of its future price.

C. measured using the standard deviation of its dividends.

\begin{enumerate}
  \setcounter{enumi}{14}
  \item From an investor's point of view, which of the following equity securities is the least risky?
\end{enumerate}

A. Putable preference shares.

B. Callable preference shares.

C. Non-callable preference shares.

\begin{enumerate}
  \setcounter{enumi}{15}
  \item Which of the following is least likely to be a reason for a company to issue equity securities on the primary market?
A. To raise capital.
B. To increase liquidity.
C. To increase return on equity.

  \item Which of the following is not a primary goal of raising equity capital?

\end{enumerate}

A. To finance the purchase of long-lived assets.

B. To finance the company's revenue-generating activities.

C. To ensure that the company continues as a going concern.

\begin{enumerate}
  \setcounter{enumi}{17}
  \item Which of the following statements is most accurate in describing a company's book value?
\end{enumerate}

A. Book value increases when a company retains its net income.

B. Book value is usually equal to the company's market value.

C. The ultimate goal of management is to maximize book value.

\begin{enumerate}
  \setcounter{enumi}{18}
  \item Calculate the book value of a company using the following information:
\end{enumerate}

\begin{center}
\begin{tabular}{lc}
\hline
Number of shares outstanding & 100,000 \\
Price per share & $€ 52$ \\
Total assets & $€ 12,000,000$ \\
Total liabilities & $€ 7,500,000$ \\
Net Income & $€ 2,000,000$ \\
\hline
\end{tabular}
\end{center}

A. $€ 4,500,000$.

B. $€ 5,200,000$.

C. $€ 6,500,000$. 20. Which of the following statements is least accurate in describing a company's market value?
A. Management's decisions do not influence the company's market value.
B. Increases in book value may not be reflected in the company's market value.
C. Market value reflects the collective and differing expectations of investors.

\begin{enumerate}
  \setcounter{enumi}{20}
  \item Calculate the return on equity (ROE) of a stable company using the following data:
\end{enumerate}

\begin{center}
\begin{tabular}{lc}
\hline
Total sales & $\pounds 2,500,000$ \\
Net income & $\pounds 2,000,000$ \\
Beginning of year total assets & $\pounds 50,000,000$ \\
Beginning of year total liabilities & $\pounds 35,000,000$ \\
Number of shares outstanding at the end of the year & $1,000,000$ \\
Price per share at the end of the year & $\pounds 20$ \\
\hline
\end{tabular}
\end{center}

A. $10.0 \%$.
B. $13.3 \%$.
C. $16.7 \%$.

\begin{enumerate}
  \setcounter{enumi}{21}
  \item Holding all other factors constant, which of the following situations will most likely lead to an increase in a company's return on equity?
A. The market price of the company's shares increases.
B. Net income increases at a slower rate than shareholders' equity.
C. The company issues debt to repurchase outstanding shares of equity.

  \item Which of the following measures is the most difficult to estimate?
A. The cost of debt.
B. The cost of equity.
C. Investors' required rate of return on debt.

  \item A company's cost of equity is often used as a proxy for investors':
A. average required rate of return.
B. minimum required rate of return.
C. maximum required rate of return.

\end{enumerate}

\section{SOLUTIONS}
\begin{enumerate}
  \item C is correct. The company is not obligated to make dividend payments. It is at the discretion of the company whether or not it chooses to pay dividends.

  \item B is correct. Statutory voting is the type of equity voting right that grants one vote per share owned.

  \item A is correct. Preference shares do not have to be either callable or putable.

  \item $\mathrm{C}$ is correct. Participating preference shares entitle shareholders to receive an additional dividend if the company's profits exceed a pre-determined level.

  \item B is correct. Private equity securities do not have market-determined quoted prices.

  \item $\mathrm{C}$ is correct. Venture capital investments can be used to provide mezzanine financing to companies in their early stage of development.

  \item B is correct. Regulatory and investor relations costs are lower for private equity firms than for public firms. There are no stock exchange, regulatory, or shareholder involvements with private equity, whereas for public firms these costs can be high.

  \item $\mathrm{C}$ is correct. The trends in emerging markets have not led to the stability of foreign exchange markets.

  \item A is correct. In an unsponsored DR, the depository bank owns the voting rights to the shares. The bank purchases the shares, places them into a trust, and then sells shares in the trust-not the underlying shares-in other markets.

  \item A is correct. The listing fees on Level III sponsored ADRs are high.

  \item C is correct. An ETF is used to gain exposure to a basket of securities (equity, fixed income, commodity futures, etc.).

  \item A is correct. The formula states $R_{t}=\left(P_{t}-P_{t-1}+D_{t}\right) / P_{t-1}$. Therefore, total return $=(42-50+2) / 50=-12.0 \%$.

  \item A is correct. The depreciated value of the euro will create an additional loss in the form of currency return that is lower than the ETF's return.

  \item $\mathrm{C}$ is correct. Some equity securities do not pay dividends, and therefore the standard deviation of dividends cannot be used to measure the risk of all equity securities.

  \item A is correct. Putable shares, whether common or preference, give the investor the option to sell the shares back to the issuer at a pre-determined price. This pre-determined price creates a floor for the share's price that reduces the uncertainty of future cash flows for the investor (i.e., lowers risk relative to the other two types of shares listed).

  \item $\mathrm{C}$ is correct. Issuing shares in the primary (and secondary) market reduces a company's return on equity because it increases the total amount of equity capital invested in the company (i.e., the denominator in the ROE formula).

  \item $C$ is correct. Capital is raised to ensure the company's existence only when it is required. It is not a typical goal of raising capital.

  \item A is correct. A company's book value increases when a company retains its net income.

  \item A is correct. The book value of the company is equal to total assets minus total liabilities, which is $€ 12,000,000-€ 7,500,000=€ 4,500,000$.

  \item A is correct. A company's market value is affected by management's decisions. Management's decisions can directly affect the company's book value, which can then affect its market value.

  \item B is correct. A company's $\mathrm{ROE}$ is calculated as $\left(\mathrm{NI}_{t} / \mathrm{BVE}_{t-1}\right)$. The $\mathrm{BVE}_{t-1}$ is equal to the beginning total assets minus the beginning total liabilities, which equals $\pounds 50,000,000-\pounds 35,000,000=\pounds 15,000,000$. Therefore, $\mathrm{ROE}=$ $\pounds 2,000,000 / \pounds 15,000,000=13.3 \%$.

  \item C is correct. A company's ROE will increase if it issues debt to repurchase outstanding shares of equity.

  \item B is correct. The cost of equity is not easily determined. It is dependent on investors' required rate of return on equity, which reflects the different risk levels of investors and their expectations about the company's future cash flows.

  \item B is correct. Companies try to raise funds at the lowest possible cost. Therefore, cost of equity is used as a proxy for the minimum required rate of return.

\end{enumerate}

\section{LEARNINGMODULE}
\begin{center}
\includegraphics[max width=\textwidth]{2023_05_04_7b535d0a870224f62e3dg-333}
\end{center}

\section{Introduction to Industry and Company Analysis}
by Patrick W. Dorsey, CFA, Anthony M. Fiore, CFA, and Ian Rossa O'Reilly, CFA.

Patrick W. Dorsey, CFA, is at Dorsey Asset Management (USA). Anthony M. Fiore, CFA, is at Silvercrest Asset Management (USA). Ian Rossa O'Reilly, CFA (Canada).

\section{LEARNING OUTCOME}
\begin{center}
\begin{tabular}{c|l}
Mastery & The candidate should be able to: \\
\hline
$\square$ & $\begin{array}{l}\text { explain uses of industry analysis and the relation of industry analysis } \\ \text { to company analysis } \\ \text { compare methods by which companies can be grouped } \\ \text { explain the factors that affect the sensitivity of a company to the } \\ \text { business cycle and the uses and limitations of industry and company } \\ \text { descriptors such as "growth," “defensive," and “cyclical" } \\ \text { describe current industry classification systems, and identify how a } \\ \text { company should be classified, given a description of its activities and } \\ \text { the classification system } \\ \text { explain how a company's industry classification can be used to } \\ \text { identify a potential "peer group" for equity valuation } \\ \text { describe the elements that need to be covered in a thorough industry } \\ \text { analysis } \\ \text { describe the principles of strategic analysis of an industry } \\ \square\end{array} \square$ \\
$\square$ & $\begin{array}{l}\text { explain the effects of barriers to entry, industry concentration, } \\ \text { industry capacity, and market share stability on pricing power and } \\ \text { price competition } \\ \text { describe industry life-cycle models, classify an industry as to } \\ \text { life-cycle stage, and describe limitations of the life-cycle concept in } \\ \text { forecasting industry performance } \\ \text { describe macroeconomic, technological, demographic, } \\ \text { governmental, social, and environmental influences on industry } \\ \text { growth, profitability, and risk } \\ \text { compare characteristics of representative industries from the various } \\ \text { economic sectors } \\ \text { describe the elements that should be covered in a thorough company } \\ \text { analysis }\end{array}$ \\
$\square$ &  \\
\end{tabular}
\end{center}

\section{INTRODUCTION}
Understanding how industries and companies operate, together with an analysis of financial statements, provides a basis for forecasting company performance and allows analysts to determine the value of an investment in a company or its securities. Industry analysis is the analysis of a specific branch of manufacturing, service, or trade. Understanding the industry in which a company operates provides an essential framework for the analysis of the individual company-that is, company analysis. Equity analysis and credit analysis are often conducted by analysts who concentrate on one or several industries, which results in synergies and efficiencies in gathering and interpreting information.

In this reading, we will address the following questions:

\begin{itemize}
  \item What are the similarities and differences among industry classification systems?

  \item How does an analyst go about choosing a peer group of companies?

  \item What are the key factors to consider when analyzing an industry?

  \item What advantages are enjoyed by companies in strategically well-positioned industries?

  \item How should an analyst approach research and analysis of new industries?

  \item What are the factors that influence individual companies?

\end{itemize}

\section{USES OF INDUSTRY ANALYSIS}
explain uses of industry analysis and the relation of industry analysis to company analysis

Industry analysis is useful in a number of investment applications that make use of fundamental analysis. Its uses include the following:

\begin{itemize}
  \item Understanding a company's business and business environment. Industry analysis is often a critical early step in stock selection and valuation because it provides insights into the issuer's growth opportunities, competitive dynamics, and business risks. For a credit analyst, industry analysis provides insights into the appropriateness of a company's use of debt financing and into its ability to meet its promised payments.

  \item Identifying active equity investment opportunities. Investors taking a top-down investing approach use industry analysis to identify industries with positive, neutral, or negative outlooks for profitability and growth. Generally, investors will then position their portfolios accordingly-by overweighting, market weighting, or underweighting those industries relative to the investor's benchmark if the investor judges that the industry's perceived prospects are not fully incorporated in market prices. In fact, some investors base their investment process on attempting to outperform their benchmarks or deliver absolute returns by industry rotation -that is, timing investments in industries in relation to an analysis of industry fundamentals and/or business-cycle conditions (technical analysis may also play a role in such strategies). - Portfolio performance attribution. The purpose of performance attribution is to identify and understand the sources of portfolio return. Portfolio attribution output will show any positive or negative contribution to returns from the fund manager's choice of industries and/or sectors in the portfolio. Industry classification schemes play a role in such performance attribution.

\end{itemize}

\section{EXAMPLE 1}
\begin{enumerate}
  \item Which of the following information about a company would most likely depend on an industry analysis? The company's:
\end{enumerate}

A. dividend distribution policy.

B. competitive environment.

C. trends in personnel expenses.

\section{Solution}
$B$ is correct. To understand the competitive environment, analysts need to understand the industry as a whole. A and $\mathrm{C}$ are incorrect because both are specific to the company.

We will next explore how companies may be grouped into industries.

\section{APPROACHES TO IDENTIFYING SIMILAR COMPANIES}
compare methods by which companies can be grouped

explain the factors that affect the sensitivity of a company to the business cycle and the uses and limitations of industry and company descriptors such as "growth," "defensive," and "cyclical"

Industry classification attempts to place companies into groups on the basis of commonalities. In the following sections, we discuss the three major approaches to industry classification:

\begin{itemize}
  \item products and/or services supplied,

  \item business-cycle sensitivities, and

  \item statistical similarities

\end{itemize}

\section{Products and/or Services Supplied}
Modern classification schemes are most commonly based on grouping companies by similar products and/or services. According to this perspective, an industry is defined as a group of companies offering similar products and/or services. For example, major companies in the global technology industry include Microsoft, Apple, Alphabet (owner of Google), Tencent, and Samsung Electronics. Although those companies are known to belong to the technology industry, they differ from one another in numerous ways. This is an issue that analysts should be aware of and take into consideration when making recommendations and investment decisions. Industry classification schemes typically provide multiple levels of aggregation. The term sector is often used to refer to a group of related industries. The health care sector, for example, consists of a number of related industries, including the pharmaceutical, biotechnology, medical device, medical supply, hospital, and managed care industries.

These classification schemes typically place a company in an industry on the basis of a determination of its principal business activity. A company's principal business activity is the source from which the company derives a majority of its revenues and/or earnings. For example, companies that derive a majority of their revenues from the sale of pharmaceuticals include Novartis AG, Pfizer Inc., Roche Holding AG, GSK plc, and Sanofi S.A., all of which could be grouped together as part of the global pharmaceutical industry. Companies that engage in more than one significant business activity usually report the revenues (and, in many cases, operating profits) of the different business segments in their financial statements.

Examples of classification systems based on products and/or services include the commercial classification systems that will be discussed later-namely, the Global Industry Classification Standard (GICS) and the Industry Classification Benchmark (ICB). In addition to grouping companies by products and/or services, some of the major classification systems group consumer-related companies into cyclical and non-cyclical categories depending on a company's sensitivity to the business cycle. The next section addresses how companies can be categorized on the basis of economic sensitivity.

\section{Business-Cycle Sensitivities}
Companies are sometimes grouped on the basis of their relative sensitivity to the business cycle. This method often results in two broad groupings of companies:

\begin{itemize}
  \item A cyclical company is one whose profits are strongly correlated with the strength of the overall economy. Such companies experience wider-than-average fluctuations in demand-high demand during periods of economic expansion and low demand during periods of economic contraction-and/or are subject to greater-than-average profit variability related to high operating leverage (i.e., high fixed costs). Concerning demand, cyclical products and services are often relatively expensive and represent purchases that can be delayed if necessary (e.g., because of declining disposable income). Examples of cyclical industries and broader sectors are autos, housing, basic materials, industrials, and technology.

  \item A non-cyclical company is one whose performance is largely independent of the business cycle. Non-cyclical companies produce goods or services for which demand remains relatively stable throughout the business cycle. Examples of non-cyclical industries are food and beverage, household and personal care products, health care, and utilities.

\end{itemize}

Although the classification systems we will discuss do not label their categories as cyclical or non-cyclical, certain sectors tend to experience greater economic sensitivity than others. Sectors that tend to exhibit a relatively high degree of economic sensitivity include consumer discretionary, energy, financials, industrials, technology, and materials. In contrast, sectors that exhibit relatively less economic sensitivity include consumer staples, health care, telecommunications, and utilities.

\section{DESCRIPTIONS RELATED TO THE CYCLICAL/NON-CYCLICAL DISTINCTION}
Analysts commonly encounter a number of labels related to the cyclical/non-cyclical distinction. For example, non-cyclical industries have sometimes been sorted into defensive (or stable) versus growth. Defensive industries and companies are those whose revenues and profits are least affected by fluctuations in overall economic activity. These industries/companies tend to produce staple consumer goods (e.g., bread), provide basic services (grocery stores, drug stores, fast food outlets), or have their rates and revenues determined by contracts or government regulation (e.g., cost-of-service, rate-of-return regulated public utilities). Growth industries include industries with specific demand dynamics that are so strong that they override the significance of broad economic or other external factors and generate growth regardless of overall economic conditions, although their rates of growth may slow during an economic downturn.

The usefulness of industry and company labels, such as cyclical, growth, and defensive, is limited. Cyclical industries often have growth companies within them. A cyclical industry itself, although exposed to the effects of fluctuations in overall economic activity, may grow at an above-average rate for periods spanning multiple business cycles. The labe "growth cyclical" is sometimes used to describe companies that are growing rapidly on a long-term basis but that still experience above-average fluctuation in their revenues and profits over the course of a business cycle.

Furthermore, when fluctuations in economic activity are large, as in the deep recession of 2008-2009, few companies escape the effects of the cyclical weakness in overall economic activity.

The defensive label is also problematic. Industries may include both companies that are growth and companies that are defensive in character, making the choice between a "growth" and a "defensive" label difficult. Moreover, "defensive" cannot be understood as necessarily being descriptive of investment characteristics. Food supermarkets, for example, would typically be described as defensive but can be subject to profit-damaging price wars. So-called defensive industries/companies may sometimes face industry dynamics that make them far from defensive in the sense of preserving shareholders' capital.

One limitation of the cyclical/non-cyclical classification is that business-cycle sensitivity is a continuous spectrum rather than an "either/or" issue, so placement of companies in one of the two major groups is somewhat arbitrary. The impact of severe recessions usually reaches all parts of the economy, so non-cyclical is better understood as a relative term.

Another limitation of a business-cycle classification for global investing is that different countries and regions of the world frequently progress through the various stages of the business cycle at different times. While one region of the world may be experiencing economic expansion, other regions or countries may be in recession, which complicates the application of a business-cycle approach to industry analysis. For example, a jewelry retailer (i.e., a cyclical company) that is selling domestically in a weak economy will exhibit markedly different fundamental performance relative to a jewelry company operating in an environment where demand is robust. Comparing these two companies-that is, similar companies that are currently exposed to different demand environments-could suggest investment opportunities. Combining fundamental data from such companies, however, to establish industry benchmark values would be misleading.

\section{Statistical Similarities}
Statistical approaches to grouping companies are typically based on the correlations of past securities' returns. For example, using the technique known as cluster analysis, companies are separated (on the basis of historical correlations of stock returns) into groups in which correlations are relatively high but between which correlations are relatively low. Clustering algorithms can be based on a range of financial characteristics (e.g., total assets, total revenue, profitability, leverage) and operating characteristics (e.g., R\&D intensity, employee headcount). This method of aggregation often results in non-intuitive groups of companies, and the composition of the groups may vary significantly by time period and region of the world. Moreover, statistical approaches rely on historical data, but analysts have no guarantee that past correlation values will continue in the future. In addition, such approaches carry the inherent dangers of all statistical methods-namely, (1) falsely indicating a relationship that arose by chance and is not supported by plausible economic basis or (2) falsely excluding a relationship that actually is significant.

\section{EXAMPLE 2}
\begin{enumerate}
  \item Which of the following is the least likely to be a method for grouping companies?
\end{enumerate}

A. Statistical approaches that assess past correlations of securities' returns

B. Similarity of products and services

C. Similarity of operating profit margins

Solution:

$\mathrm{C}$ is correct. Profit margins are not used as a method for grouping companies. Companies with similar profit margins may have little in common. A is incorrect because it describes the statistical similarities approach, which is a method for grouping companies. B is incorrect because similarity of products and services is one of the most common ways of grouping companies.

\begin{enumerate}
  \setcounter{enumi}{1}
  \item Companies are classified as cyclical or non-cyclical (defensive) on the basis of the exposure of their business to:
\end{enumerate}

A. their industry life cycle.

B. the business cycle of the economy.

C. the credit cycle, which affects their ability to borrow.

Solution:

$\mathrm{B}$ is correct. The cyclical/non-cyclical grouping reflects companies' sensitivity to the economy's business cycle. A is incorrect because the industry life cycle describes how the industry evolves over time, not how it relates to economic fluctuations. $C$ is incorrect because the credit cycle explains the availability and cost of credit, not the fluctuations of output of the economy to which cyclical or non-cyclical companies are exposed. company should be classified, given a description of its activities and the classification system explain how a company's industry classification can be used to identify a potential "peer group" for equity valuation

A well-designed classification system often serves as a useful starting point for industry analysis. It allows analysts to compare industry trends and relative valuations among companies in a group. Classification systems that take a global perspective enable portfolio managers and research analysts to make global comparisons of companies in the same industry. For example, given the global nature of the automobile industry, a thorough analysis of the industry would include auto companies from many different countries and regions of the world. Classification systems are developed and used by both commercial entities and various governmental agencies. Most government and commercial classification systems are reviewed and, if necessary, updated from time to time. Generally, commercial classification systems are adjusted more frequently than government classification systems, which may be updated only every five years or so. We focus on the commercial systems because they are more commonly used in the investment industry.

\section{Commercial Industry Classification Systems}
Major index providers, including Standard \& Poor's, MSCI, Russell Investments, Dow Jones, and FTSE Russell, classify companies in their equity indexes into industry groupings. Most classification schemes used by these index providers contain multiple levels of classification that start at the broadest level with a general sector grouping and then, in several further steps, subdivide or disaggregate the sectors into more granular, narrowly defined sub-industry groups.

\section{Global Industry Classification Standard}
GICS, jointly developed by Standard \& Poor's and MSCI, was designed to facilitate global comparisons of industries. It classifies companies in both developed and developing economies. In June 2020, the GICS classification structure comprised four levels of detail consisting of 11 sectors, 24 industry groups, 69 industries, and 158 sub-industries. The composition of GICS historically has been adjusted over time to reflect changes in the global equity markets. One of the most significant changes happened in 2018 when GICS launched the new Communication Services sector to expand beyond simply the Telecommunication Services sector to incorporate many information technology and internet-related companies that offer communication platforms, content, and information.

\section{Industry Classification Benchmark}
The Industry Classification Benchmark (ICB), which was jointly developed by Dow Jones and FTSE, uses a four-tier structure to categorize companies globally on the basis of the source from which a company derives the majority of its revenue. In June 2020, the ICB classification system consisted of 11 industries, 20 supersectors, 45 sectors, and 173 subsectors. Although the ICB is similar to GICS in the number of tiers and the method by which companies are assigned to particular groups, the two systems use significantly different nomenclature. For example, whereas GICS uses the term "sector" to describe its broadest grouping of companies, ICB uses the term "industry" to describe the broadest category. The ICB is used by many global stock exchanges to categorize listed companies, including the London Stock Exchange, Euronext, and NASDAQ OMX. In addition, all Russell Indexes are set to adopt a newly enhanced ICB structure after the market close on 18 September 2020.

Exhibit 1 provides examples of how several companies are classified. Note how differently the systems classify PayPal, Uber, and Activision Blizzard.

\section{Exhibit 1: Industry Classification Examples}
\begin{center}
\begin{tabular}{|c|c|c|}
\hline
Stock Examples & $\begin{array}{c}\text { Global Industry Classification } \\ \text { Standard }\end{array}$ & $\begin{array}{l}\text { Industry Classification } \\ \text { Benchmark }\end{array}$ \\
\hline
PayPal & $\begin{array}{l}\text { Information Technology }> \\ \text { Software \& Services }>\text { IT } \\ \text { Services }>\text { Data Processing \& } \\ \text { Outsourced Services }\end{array}$ & $\begin{array}{l}\text { Industrials > Industrial Goods } \\ \text { \& Services > Support Services > } \\ \text { Financial Administration }\end{array}$ \\
\hline
Peloton Interactive & $\begin{array}{c}\text { Consumer Discretionary }> \\ \text { Consumer Durables \& Apparel } \\ >\text { Leisure Products > Leisure } \\ \text { Products }\end{array}$ & $\begin{array}{c}\text { Consumer Services }>\text { Travel \& } \\ \text { Leisure }>\text { Travel \& Leisure }> \\ \text { Recreational Services }\end{array}$ \\
\hline
Uber & $\begin{array}{c}\text { Industrials }>\text { Transportation }> \\ \text { Road \& Rail > Trucking }\end{array}$ & $\begin{array}{c}\text { Consumer Services }>\text { Retail }> \\ \text { General Retailers }>\text { Specialized } \\ \text { Consumer Services }\end{array}$ \\
\hline
Activision Blizzard & $\begin{array}{c}\text { Communication Services }> \\ \text { Media \& Entertainment > } \\ \text { Entertainment > Interactive } \\ \text { Home Entertainment }\end{array}$ & $\begin{array}{c}\text { Consumer Goods }>\text { Personal \& } \\ \text { Household Goods }>\text { Leisure Goods } \\ >\text { Toys }\end{array}$ \\
\hline
\end{tabular}
\end{center}

Despite the fact that some companies are assigned to different broader sectors and industries by the different systems, the two commercial classification systems use common methodologies for assigning companies to groups. Also, the broadest level of grouping for GICS and the ICB is quite similar. Specifically, GICS and the ICB each identify 11 broad groupings below which all other categories reside. Next, we describe sectors that are fairly representative of how the broadest level of classification is viewed by GICS (into "sectors") and the ICB (into "industries").

\section{Description of Representative Sectors/Industries}
Exhibit 2 provides a description of each of the 11 broad sectors.

\section{Exhibit 2: Sector Descriptions}
Materials

Consumer

Discretionary

Consumer Staples

Energy

Financials Companies engaged in the production of building materials, chemicals, paper and forest products, and containers and packaging, as well as metal, mineral, and mining companies.

Companies that derive a majority of revenue from the sale of consumer-related products or services for which demand tends to exhibit a relatively high degree of economic sensitivity. Examples of business activities that frequently fall into this category are automotive, apparel, hotel, and restaurant businesses.

Consumer-related companies whose business tends to exhibit less economic sensitivity than other companies-for example, manufacturers of food, beverages, tobacco, and personal care products. Companies whose primary line of business involves exploring for, production of, or refining of natural resources used to produce energy; companies that derive a majority of revenue from the sale of equipment or through the provision of services to energy companies also fall into this category.

Companies whose primary line of business involves banking, finance, insurance, asset management, and/or brokerage services. Health Care Manufacturers of pharmaceutical and biotech products, medical devices, health care equipment, and medical supplies and providers of health care services.

Industrials Manufacturers of capital goods and providers of commercial services; for example, business activities would include heavy machinery and equipment manufacture, aerospace and defense, transportation services, and commercial services and supplies.

Real Estate Companies engaged in the development and operation of real estate. This includes companies offering real estate-related services and equity real estate investment trusts (REITs).

Information Technology Manufacture or sale of computers, software, semiconductors, and communications equipment; other business activities that frequently fall into this category are electronic entertainment, internet services, and technology consulting and services.

Communication

Services/

This sector includes traditional telecommunication companies that provide fixed line and wireless communication services, with

Telecommunications media, entertainment, and interactive media and services, such as video gaming companies.

Utilities

Electric, gas, and water utilities; telecommunication companies are sometimes included in this category.

To classify a company accurately in a particular classification scheme requires definitions of the classification categories, a statement about the criteria used in classification, and detailed information about the subject company. Example 3 introduces an exercise in such classification. In addressing the question, the reader can make use of the widely applicable sector descriptions just given and familiarity with available business products and services.

\section{EXAMPLE 3}
\section{Classifying Companies into Industries}
\begin{enumerate}
  \item Exhibit 2 defines 11 representative sectors, repeated here in Exhibit 3. Assume the classification system is based on the criterion of a company's principal business activity as judged primarily by source of revenue.
\end{enumerate}

\section{Exhibit 3: List of Sectors}
Sector

Materials

Consumer Discretionary

Consumer Staples

Energy

Financials

Health Care

Industrials

Real Estate

Information Technology

Communication Services/

Telecommunications

Utilities Based on the information given, determine an appropriate sector membership for each of the following hypothetical companies:

\begin{enumerate}
  \item An operator of shopping malls

  \item A natural gas transporter and marketer

  \item A manufacturer of heavy construction equipment

  \item A provider of regional telephone services

  \item A semiconductor company

  \item A manufacturer of medical devices

  \item A video conference provider

  \item A manufacturer of chemicals and plastics

  \item A manufacturer of automobiles

  \item A cloud computing service provider

  \item A food delivery company

  \item A regulated supplier of electricity

  \item A provider of wireless broadband services

  \item A manufacturer of soaps and detergents

  \item A software development company

  \item An insurer

  \item A regulated provider of water/wastewater services

  \item A robotic-assisted surgery company

  \item A manufacturer of pharmaceuticals

  \item A provider of rail transportation services

  \item A data center real estate investment trust

  \item A developer of residential housing

\end{enumerate}

Solution

Sector

Company Number

Materials

7

Consumer Discretionary

8,10

Consumer Staples

Energy

1

Financials

15

Health Care

$5,17,18$

Industrials

2,19

Real Estate

0, 20, 21

Information Technology

$4,9,14$

Communication Services/

$3,6,12$

Telecommunications

Utilities

11,16

\section{EXAMPLE 4}
\section{Industry Classification Schemes}
\begin{enumerate}
  \item The GICS classification system classifies companies on the basis of a company's primary business activity as measured primarily by:
A. assets.
B. income.
C. revenue.
\end{enumerate}

\section{Solution:}
$\mathrm{C}$ is correct.

\begin{enumerate}
  \setcounter{enumi}{1}
  \item Which of the following is least likely to be accurately described as a cyclical company?
A. An automobile manufacturer
B. A producer of breakfast cereals
C. An apparel company producing new, trendy clothes for teenage girls
\end{enumerate}

\section{Solution:}
B is correct. A producer of staple foods, such as cereals, is a classic example of a non-cyclical company. Demand for automobiles is cyclical-that is, relatively high during economic expansions and relatively low during economic contractions. Also, demand for teenage fashions is likely to be more sensitive to the business cycle than demand for standard food items, such as breakfast cereals. When budgets have been reduced, families may try to avoid expensive clothing or may try to extend the life of existing clothing.

\begin{enumerate}
  \setcounter{enumi}{2}
  \item Which of the following is the most accurate statement? A statistical approach to grouping companies into industries:
\end{enumerate}

A. is based on historical correlations of the securities' returns.

B. frequently produces industry groups whose composition is similar worldwide.

C. emphasizes the descriptive statistics of industries consisting of companies producing similar products and/or services.

\section{Solution:}
A is correct.

\section{Constructing a Peer Group}
A peer group is a group of companies engaged in similar business activities whose economics and valuation are influenced by closely related factors. Comparisons of a company in relation to a well-defined peer group can provide valuable insights into the company's performance and its relative valuation.

The construction of a peer group is a subjective process; the result often differs significantly from even the most narrowly defined categories given by the commercial classification systems. However, commercial classification systems do provide a starting point for the construction of a relevant peer group because by using such systems, an analyst can quickly discover the public companies operating in the chosen industry. In fact, one approach to constructing a peer group is to start by identifying other companies operating in the same industry. Analysts who subscribe to one or more of the commercial classification systems can quickly generate a list of other companies in the industry in which the company operates according to that particular service provider's definition of the industry. An analyst can then investigate the business activities of these companies through a range of sources, such as companies' public disclosures or industry trade publications, and confirm that each comparable company derives a significant portion of its revenue and operating profit from a business activity similar to the primary business of the subject company.

The following questions may improve the list of peer companies:

\begin{itemize}
  \item What proportion of revenue and operating profit is derived from business activities similar to those of the subject company? In general, a higher percentage results in a more meaningful comparison.

  \item Does a potential peer company face a demand environment similar to that of the subject company? For example, a comparison of growth rates, margins, and valuations may be of limited value when comparing companies that are exposed to different stages of the business cycle. (As mentioned, such differences may be the result of conducting business in geographically different markets.)

  \item Does the peer company have a significant business segment(s) that is comparable to other companies even if they are not in the same commercial classification system? For example, Amazon Web Services' business (cloud computing) is more comparable to that of such peers as Microsoft and Cisco than to that of its e-commerce peers.

\end{itemize}

Although companies with limited lines of business may be neatly categorized into a single peer group, companies with multiple divisions may be included in more than one category. For example, US-based Hewlett-Packard Company (HP), a global provider of technology and software solutions, used to be included in more than one peer group before it separated its business into HP Inc., which sells hardware, and Hewlett-Packard Enterprise (HPE), which sells software and services. Investors interested in the personal computer (PC) industry, for example, used to include HP in their peer group, but investors constructing a peer group of providers of information technology services also included HP in that group.

Example 5 illustrates the process of identifying a peer group of companies and shows some of the practical hurdles to determining a peer group.

\section{EXAMPLE 5}
\section{An Analyst Researches the Peer Group of Novartis}
\begin{enumerate}
  \item Suppose that an analyst needs to identify the peer group of companies for Novartis for use in the valuation section of a company report. Novartis develops and produces branded pharmaceuticals and operates globally. The analyst starts by looking at Novartis's industry classification according to GICS. The most narrowly defined category that GICS uses is the sub-industry level, and in July 2020, Novartis was in the GICS sub-industry called Pharmaceuticals, together with the companies listed in Exhibit 4 (a sample of companies is shown).
\end{enumerate}

\section{Exhibit 4: GICS Industry Classification: Pharmaceuticals}
\begin{center}
\includegraphics[max width=\textwidth]{2023_05_04_7b535d0a870224f62e3dg-345}
\end{center}

After looking over the list of companies, the analyst quickly realizes that some adjustments need to be made to the list to end up with a peer group of companies that are comparable to Novartis.

Research of the companies' disclosures, websites, and company descriptions provided by electronic data/information vendors reveals that Novartis focuses on research, development, and production of branded, patent-protected original products. Such products account for $79 \%$ of the company's revenues (source: "Novartis Annual Report 2019"). The business model is, therefore, distinct from that of companies that focus on the production of generic pharmaceuticals that do not benefit from patent protection the way branded original products do. For example, Teva and Dr. Reddy's main business is in the generic pharmaceuticals space, so those companies would not fit Novartis's peer group. In contrast, Novartis's business is subject to similar drivers and external influences that apply to a narrow range of companies, such as Roche, Pfizer, GSK, and Sanofi. The peer group that the analyst should use is, therefore, narrower than what the GICS Pharmaceuticals sub-industry would suggest.

In summary, analysts must distinguish between a company's industry-as defined by one or more of the various classification systems-and its peer group. A company's peer group should consist of companies with similar business activities whose economic activity depends on similar drivers of demand and similar factors related to cost structure and access to financial capital. In practice, these necessities frequently result in a smaller group (even a different group) of companies than the most narrowly defined categories used by the common commercial classification systems. Example 6 illustrates various aspects of developing and using peer groups.

\section{EXAMPLE 6}
\section{The Semiconductor Industry: Business-Cycle Sensitivity and Peer-Group Determination}
The GICS semiconductor and semiconductor equipment industry (453010) has two sub-industries-the semiconductor equipment sub-industry (45301010) and the semiconductors sub-industry (45301020). Members of the semiconductor equipment sub-industry include equipment suppliers, such as Lam Research Corporation and ASML Holdings NV; the semiconductors sub-industry includes integrated circuit manufacturers Intel Corporation and Taiwan Semiconductor Manufacturing Company Ltd.

Lam Research is a leading supplier of wafer fabrication equipment and services to the world's semiconductor industry. Lam also offers wafer-cleaning equipment that is used after many of the individual steps required to manufacture a finished wafer. Often, the technical advances that Lam introduces in its wafer-etching and wafer-cleaning products are also available as upgrades to its installed base. This benefit provides customers with a cost-effective way to extend the performance and capabilities of their existing wafer fabrication lines.

ASML describes itself as the world's leading provider of lithography systems (etching and printing on wafers) for the semiconductor industry. ASML manufactures complex machines that are critical to the production of integrated circuits or microchips. ASML designs, develops, integrates, markets, and services these advanced systems, which help chip makers reduce the size and increase the functionality of microchips and consumer electronic equipment. The machines are costly and thus represent a substantial capital investment for a purchaser.

Based on revenue, Intel is the world's largest semiconductor chip maker and has a dominant share of microprocessors for the personal computer market. Intel designs and produces its own proprietary semiconductors for direct sale to customers, such as personal computer makers. Intel has made significant investments in research and development (R\&D) to introduce and produce new chips for new applications.

Established in 1987, Taiwan Semiconductor Manufacturing (TSMC) is the world's largest dedicated semiconductor foundry company (dedicated semiconductor foundries are semiconductor fabrication plants that execute the designs of other companies). TSMC describes itself as offering cutting-edge process technologies, pioneering design services, manufacturing efficiency, and product quality. TSMC provides design and production services to a diverse group of integrated circuit suppliers that generally do not have their own in-house manufacturing capabilities. The company's revenues represented more than $50 \%$ of the dedicated foundry segment in the semiconductor industry.

The questions that follow take the perspective of a recession similar to that of early 2009, when many economies around the world were in a recession. Based only on the information given, answer the following questions:

\begin{enumerate}
  \item If the weak economy, similar to that of early 2009, were to recover within the next 12-18 months, which of the two sub-industries of the semiconductor and semiconductor equipment industry would most likely be the first to experience a positive improvement in business?
\end{enumerate}

\section{Solution:}
In the most likely scenario, improvement in the business of the equipment makers (Lam and ASML) would lag that of semiconductor companies (Intel and TSMC). Because of the weak economy, excess manufacturing capacity should be available to meet increased demand for integrated circuits in the near term without additional equipment, which is a major capital investment. When semiconductor manufacturers believe the longer-term outlook has improved, they should begin to place orders for additional equipment.

\begin{enumerate}
  \setcounter{enumi}{1}
  \item Explain whether Intel and TSMC should be considered members of the same peer group.
\end{enumerate}

\section{Solution:}
Intel and TSMC are not likely to be considered comparable members of the same peer group because they have different sets of customers and different business models. Intel sells its proprietary semiconductors directly to customers, whereas TSMC provides design and production services to circuit suppliers that do not have their own in-house manufacturing capabilities. Standard \& Poor's does not group Intel and TSMC in the same peer group; Intel was in the Semiconductors, Logic, Larger Companies group, and TSMC was in the Semiconductors, Foundry Services group.

\begin{enumerate}
  \setcounter{enumi}{2}
  \item Explain whether Lam Research and ASML should be considered members of the same peer group.
\end{enumerate}

\section{Solution:}
Both Lam Research and ASML are leading companies that design and manufacture equipment to produce semiconductor chips. The companies are comparable because they both depend on the same economic factors that drive demand for their products. Their major customers are the semiconductor chip companies. Standard \& Poor's grouped both companies in the same peer group-Semiconductor Equipment, Larger Front End.

\section{DESCRIBING AND ANALYZING AN INDUSTRY AND PRINCIPLES OF STRATEGIC ANALYSIS}
\begin{center}
\includegraphics[max width=\textwidth]{2023_05_04_7b535d0a870224f62e3dg-348}
\end{center}

In their work, analysts with superior knowledge about an industry's characteristics, conditions, and trends have a competitive edge in evaluating the investment merits of the companies in the industry. Analysts attempt to develop practical, reliable industry forecasts by using various approaches to forecasting. They often estimate a range of projections for a variable reflecting various possible scenarios. In order to conduct this analysis, analysts can study a wide range of factors, including but not limited to the following:

\begin{enumerate}
  \item Statistical relationships between industry trends

  \item Economic and business variables

  \item Information from industry associations, from the individual subject companies they are analyzing, and from these companies' competitors, suppliers, and customers

  \item Relevant industry trends and metrics to help understand and forecast trends

\end{enumerate}

Investment managers and analysts also examine industry performance (1) in relation to other industries to identify industries with superior/inferior returns and (2) over time to determine the degree of consistency, stability, and risk in the returns in the industry over time. The objective of this analysis is to identify industries that offer the highest potential for investment returns on a risk-adjusted basis. The investment time horizon can be either long or short, as is the case for a rotation strategy in which portfolios are rotated into the industry groups that are expected to benefit from the next stage in the business cycle.

Analysts may also seek to compare their assumptions and projections with those made by other analysts. Doing so enables them to identify and understand any differences in methodology and any differences between their forecasts and consensus forecasts.

Exhibit 5 provides a framework designed to help analysts check that they have considered the range of forces that may affect the evolution of an industry. It shows, at the macro level, macroeconomic, demographic, environmental, governmental, social, and technological influences affecting the industry. It also depicts how an industry is affected by the forces driving industry competition (threat of new entrants, substitution threats, customer and supplier bargaining forces), the competitive forces in the industry (rivalry), life-cycle issues, and business-cycle considerations. Exhibit 5 summarizes and brings together visually the topics and concepts discussed in this section.

\section{Exhibit 5: A Framework for Industry Analysis}
\begin{center}
\includegraphics[max width=\textwidth]{2023_05_04_7b535d0a870224f62e3dg-349}
\end{center}

\section{Principles of Strategic Analysis}
When analyzing an industry, the analyst must recognize that the economic fundamentals can vary markedly among industries. Some industries are highly competitive, with most players struggling to earn adequate returns on capital, whereas other industries have attractive characteristics that allow almost all industry participants to generate healthy profits.

Differing competitive environments are often tied to the structural attributes of an industry, which is one reason industry analysis is a vital complement to company analysis. To thoroughly analyze a company itself, the analyst needs to understand the context in which the company operates. As analysts examine the competitive structure of an industry, they should always be thinking about what attributes could change in the future. Needless to say, industry analysis must be forward looking and could ultimately help determine an overweight or underweight to sector positioning.

Analysis of the competitive environment with an emphasis on the implications of the environment for corporate strategy is known as strategic analysis. Michael Porter's "five forces" framework is the classic starting point for strategic analysis; although it was originally aimed more at internal managers of businesses than at external security analysts, the framework is useful to both.

Porter (2008) identified five determinants of the intensity of competition in an industry: The threat of entry, the power of suppliers, the power of buyers, the threat of substitutes, and rivalry among existing competitors. Exhibit 6 illustrates the Porter framework. Exhibit 6: Porter's Five Forces

\section{Threat of New Entrants}
Depends on the existence and extent of barriers to enter the industry.

Q: How difficult would it be for new competitors to enter the industry? Industries that are easy to

Bargaining Power

of Suppliers enter will generally be more competitive than

Affected by:

\begin{itemize}
  \item concentration of industry is, industries with high barriers to entry.

  \item switching costs of suppliers' customers,

  \item supply substitutes. Q: Are suppliers able to increase prices, restrict supply, or pass on rising input costs?

\end{itemize}

Suppliers of scarce or limited parts or elements often possess significan pricing power.

\begin{center}
\includegraphics[max width=\textwidth]{2023_05_04_7b535d0a870224f62e3dg-350}
\end{center}

Threat of Substitutes (products or services) Affected by the availability of (alternative) product categories that can satisfy customer needs Q: Are lower-priced brands close substitutes for premium brands and vice versa ? Bargaining Power

of Customers

Affected by:

\begin{itemize}
  \item size and concentration of customers,

  \item costs of switching to other suppliers,

  \item customers' ability to produce the product or service themselves. Are customers able to force price reductions or better payment terms ?

\end{itemize}

Addressing the following questions should help the analyst evaluate the threat of new entrants and the level of competition in an industry and thereby provide an effective base for describing and analyzing the industry:

\begin{itemize}
  \item What are the barriers to entry? Is it difficult or easy for a new competitor to challenge incumbents? Relatively high (low) barriers to entry imply that the threat of new entrants is relatively low (high).

  \item How concentrated is the industry? Do a small number of companies control a relatively large share of the market, or does the industry have many players, each with a small market share?

  \item What are capacity levels? That is, based on existing investment, how much of the goods or services can be delivered in a given time frame? Does the industry suffer chronic over- or undercapacity, or do supply and demand tend to come into balance reasonably quickly in the industry?

  \item How stable are market shares? Do companies tend to rapidly gain or lose share, or is the industry stable?

  \item Where is the industry in its life cycle? Does it have meaningful growth prospects, or is demand stagnant/declining?

  \item How important is price to the customer's purchase decision?

\end{itemize}

The answers to these questions are elements of any thorough industry analysis. They are explored each in turn in the sections that follow.

\section{Barriers to Entry}
When a company is earning economic profits, the chances that it will be able to sustain them over time are greater, all else being equal, if the industry has high barriers to entry. The ease with which new competitors can challenge incumbents is often an important factor in determining the competitive landscape of an industry. If new competitors can easily enter the industry, the industry is likely to be highly competitive because high returns on invested capital will quickly be competed away by new entrants eager to grab their share of economic profits. As a result, industries with low barriers to entry often have little pricing power because price increases that raise companies' returns on capital will eventually attract new competitors to the industry.

If incumbents are protected by barriers to entry, the threat of new entrants is lower and incumbents may enjoy a more benign competitive environment. Often, these barriers to entry can lead to greater pricing power, because potential competitors would find it difficult to enter the industry and undercut incumbents' prices. Of course, high barriers to entry do not guarantee pricing power, because incumbents may compete fiercely among each other.

A classic example of an industry with low barriers to entry is restaurants. Anyone with a modest amount of capital and some culinary skill can open a restaurant, and popular restaurants quickly attract competition. As a result, the industry is very competitive, and many restaurants fail in their first few years of business.

At the other end of the spectrum of barriers to entry are the global credit card networks, such as MasterCard and Visa, both of which often post operating margins greater than $30 \%$. Such high profits should attract competition, but the barriers to entry are extremely high. Capital costs are one hurdle; also, building a massive data-processing network would not be cheap. Imagine for a moment that a venture capitalist were willing to fund the construction of a network that would replicate the physical infrastructure of the incumbents; the new card-processing company would have to convince millions of consumers to use the new card and convince thousands of merchants to accept the card. Consumers would not want to use a card that merchants did not accept, and merchants would not want to accept a card that few consumers carried. This problem would be difficult to solve, which is why the barriers to entering this industry are quite high. The barriers help preserve the profitability of the incumbent players.

One way of understanding barriers to entry is simply by thinking about what it would take for new players to compete in an industry. How much money would they need to spend? What kind of intellectual capital would they need to acquire? How easy would it be to attract enough customers to become successful?

Another way to investigate the issue is by looking at historical data. How often have new companies tried to enter the industry? Is a list of industry participants today markedly different from what it was 5 or 10 years ago? These kinds of data can be very helpful because the information is based on the real-world experience of many entrepreneurs and businesses making capital allocation decisions. If an industry has seen a flood of new entrants over the past several years, the odds that the barriers are low are good; conversely, if the same 10 companies that dominate an industry today dominated it 10 years ago, barriers to entry are probably fairly high.

Do not confuse barriers to entry, however, with barriers to success. In some industries, entering may be easy but becoming successful enough to threaten the incumbents might be quite hard.

Also, high barriers to entry do not automatically lead to good pricing power and attractive industry economics. Consider the cases of the auto manufacturing, commercial aircraft manufacturing, and oil refining industries. Starting up a new company in any of these industries is very difficult. Aside from the massive capital costs, there would be significant other barriers to entry: A new automaker would need manufacturing expertise and a dealer network, an aircraft manufacturer would need a significant amount of intellectual capital, and a refiner would need process expertise and regulatory approvals. Despite the high barriers to entry, the industries are all quite competitive with limited pricing power. Very few industry participants reliably generate returns on capital in excess of their costs of capital. Exiting the industries and re-deploying capacity to other uses when demand conditions worsen are also difficult, meaning that the industries can be prone to overcapacity. A final consideration when analyzing barriers to entry is that they can change over time as new technologies or ways of conducting business become available. For example outsourcing of some activities means that companies can focus on product or service design, without the need to build and own production capacity.

Exhibit 7 provides an overview of barriers to entry for three representative industries.

Exhibit 7: Elements of a Strategic Analysis for Three Industries: Barriers to Entry

\begin{center}
\begin{tabular}{|c|c|c|c|}
\hline
 & Branded Pharmaceuticals & Oil Services & Confections/Candy \\
\hline
Major Companies & $\begin{array}{l}\text { Pfizer, Novartis, Merck, GSK } \\ \text { plc }\end{array}$ & $\begin{array}{l}\text { Schlumberger, Baker Hughes, } \\ \text { Halliburton }\end{array}$ & Hershey, Mars/Wrigley, Nestl \\
\hline
Barriers to Entry & $\begin{array}{l}\text { Very High: Substantial financial } \\ \text { and intellectual capital required } \\ \text { to compete effectively. A poten- } \\ \text { tial new entrant would need to } \\ \text { create a sizable R\&D operation, } \\ \text { a global distribution network, } \\ \text { and large-scale manufacturing } \\ \text { capacity. }\end{array}$ & $\begin{array}{l}\text { High: Technological expertise } \\ \text { is required, but a high level of } \\ \text { innovation allows niche compa- } \\ \text { nies to enter the industry and } \\ \text { compete in specific areas. }\end{array}$ & $\begin{array}{l}\text { Medium/Low: Low financial } \\ \text { or technological hurdles, but } \\ \text { new players would lack the } \\ \text { established brands that drive } \\ \text { consumer purchase decisions } \\ \text { and potential supply chain } \\ \text { operations. }\end{array}$ \\
\hline
\end{tabular}
\end{center}

\section{Industry Concentration}
Much like industries with barriers to entry, industries that are concentrated among a relatively small number of players often experience relatively less price competition. Again, there are important exceptions, so the reader should not automatically assume that concentrated industries always have pricing power or that fragmented industries do not. The degree of concentration is often measured using the Herfindahl-Hirschman Index.

An analysis of industry concentration should start with market share: What percentage of the market does each of the largest players have, and how large are those shares relative to each other and relative to the remainder of the market? Often, the relative market shares of competitors matter as much as their absolute market shares.

For example, the global market for commercial aircraft (for 100 or more passengers) is extremely concentrated; only Boeing and Airbus manufacture this type of plane on significant scale. The two companies tend to have broadly similar market shares, however, and control essentially the entire market. Because neither enjoys a scale advantage relative to its competitor and because any business gained by one is lost by the other, competition tends to be fierce.

This situation contrasts with the market for home improvement products in the United States, which is dominated by Home Depot and Lowe's. These two companies had about $11 \%$ and $7 \%$ market shares for the decade up to 2020 , respectively, which does not sound very large. However, the next largest competitor had only $2 \%$ of the market, and most market participants are tiny, with miniscule market shares. Both Home Depot and Lowe's have historically posted high returns on invested capital, in part because they could profitably grow by squeezing out smaller competitors rather than engaging in fierce competition with each other.

Fragmented industries tend to be highly price competitive for several reasons. First, the large number of companies makes coordination difficult because there are too many competitors for each industry member to monitor effectively. Second, each player has such a small piece of the market that even a small gain in market share can make a meaningful difference to its fortunes, which increases the incentive of each company to undercut prices and attempt to steal market share. Finally, the large number of players encourages industry members to think of themselves individualistically rather than as members of a larger group, which can lead to fierce competitive behavior.

In concentrated industries, in contrast, each player can relatively easily keep track of what its competitors are doing, which makes tacit coordination much more feasible. Also, leading industry members are large, which means they have more to lose-and proportionately less to gain-by destructive price behavior. Large companies are also more tied to the fortunes of the industry as a whole, making them more likely to consider the long-run effects of a price war on overall industry economics.

As with barriers to entry, the level of industry concentration is a guideline rather than a hard and fast rule when thinking about the level of pricing power in an industry. For example, Exhibit 8 shows a rough classification of industries based on analysis by Morningstar after asking its equity analysts whether industries were characterized by strong or weak pricing power and whether those industries were concentrated or fragmented. Examples of companies in industries are included in parentheses. In the upper right quadrant ("concentrated with weak pricing power"), those industries that are capital intensive and/or sell commodity-like products are shown in boldface.

\section{Exhibit 8: A Two-Factor Analysis of Industries}
Concentrated with Strong Pricing Power

Orthopedic Devices (Zimmer, Smith \& Nephew)

Biotech (Amgen, Genzyme)

Pharmaceuticals (Merck \& Co., Novartis)

Industrial Gases (Praxair, Air Products and Chemicals)

Enterprise Networking (Cisco Systems)

US Defense (General Dynamics)

Heavy Construction Equipment (Caterpillar, Komatsu)

Credit Card Networks (MasterCard, Visa)

Investment Banking/Mergers \& Acquisitions (Goldman Sachs, UBS)

Futures Exchanges (Chicago Mercantile Exchange, Intercontinental

Exchange)

Tobacco (Philip Morris, British American Tobacco)

Alcoholic Beverages (Diageo, Pernod Ricard)

Fragmented with Strong Pricing Power

For-Profit Education (Apollo Group, DeVry University)

Analog Chips (Texas Instruments, STMicroelectronics)

Industrial Distribution (Fastenal, W.W. Grainger)

Propane Distribution (AmeriGas, Ferrellgas)

Private Banking (Northern Trust, Credit Suisse) Concentrated with Weak Pricing Power

Commercial Aircraft (Boeing, Airbus)

Automobiles (General Motors, Toyota, Daimler)

Memory (DRAM \& Flash Product, Samsung, Hynix)

Semiconductor Equipment (Applied Materials, Tokyo

Electron)

Generic Drugs (Teva Pharmaceutical Industries,

Sandoz)

Printers/Office Machines (HP, Lexmark)

Refiners (Valero, Marathon Oil)

Major Integrated Oil (BP, ExxonMobil)

Airlines

Restaurants

Life Insurance

Source: Adapted from Morningstar Equity Research.

The industries in the top right quadrant defy the "concentration is good for pricing" guideline. When we examine these concentrated-yet-competitive industries, a clear theme emerges: Many industries in this quadrant (the boldface ones) are highly capital intensive and sell commodity-like products. Exiting the industries and re-deploying capacity to other uses when demand conditions worsen are also difficult, meaning that the industries can be prone to overcapacity. Also, if the industry sells a commodity product that is difficult-or impossible-to differentiate, the incentive to compete on price increases because a lower price frequently results in greater market share.

Generally, industry concentration is a good indicator that an industry has pricing power and rational competition, but other factors may override the importance of concentration. Industry fragmentation is a much stronger signal that the industry is competitive with limited pricing power.

The industry characteristics discussed here are guidelines meant to steer the analyst in a particular direction, not rules that should cause the analyst to ignore other relevant analytical factors.

\section{Exhibit 9: Elements of a Strategic Analysis for Three Industries: Levels of Concentration}
\begin{center}
\begin{tabular}{|c|c|c|c|}
\hline
 & Branded Pharmaceuticals & Oil Services & Confections/Candy \\
\hline
Major Companies & $\begin{array}{l}\text { Pfizer, Novartis, Merck, GSK } \\ \text { plc }\end{array}$ & $\begin{array}{l}\text { Schlumberger, Baker Hughes, } \\ \text { Halliburton }\end{array}$ & Hershey, Mars/Wrigley, Nestle \\
\hline
Level of Concentration & $\begin{array}{l}\text { Concentrated: A small num- } \\ \text { ber of companies control the } \\ \text { bulk of the global market for } \\ \text { branded drugs. Recent merg- } \\ \text { ers have increased the level of } \\ \text { concentration. }\end{array}$ & $\begin{array}{l}\text { Fragmented: Although only a } \\ \text { small number of companies } \\ \text { provide a full range of services, } \\ \text { many smaller players compete } \\ \text { effectively in specific areas. } \\ \text { Service arms of national oil } \\ \text { companies may control signifi- } \\ \text { cant market share in their own } \\ \text { countries, and some product } \\ \text { lines are concentrated in the } \\ \text { mature US market. }\end{array}$ & $\begin{array}{l}\text { Very Concentrated: The top four } \\ \text { companies have a large pro- } \\ \text { portion of global market share. } \\ \text { Recent mergers have increased } \\ \text { the level of concentration. }\end{array}$ \\
\hline
\end{tabular}
\end{center}

\section{Industry Capacity}
The effect of industry capacity (the maximum amount of a good or service that can be supplied in a given time period) on pricing is clear: Tight, or limited, capacity gives participants more pricing power because demand for the product or service exceeds supply, whereas overcapacity leads to price cutting and a very competitive environment because excess supply chases demand. An analyst should think about not only current capacity conditions but future changes in capacity levels. How quickly can companies in the industry adjust to fluctuations in demand? How flexible is the industry in bringing supply and demand into balance? What will be the effect of that process on industry pricing power or on industry margins?

\section{Capacity: Short term vs. Long Term}
Generally, capacity is fixed in the short term and variable in the long term because capacity can be increased (e.g., new factories can be built) if time is sufficient. What is considered "sufficient" time-and, therefore, the duration of the short term, in which capacity cannot be increased-may vary dramatically among industries. Sometimes, adding capacity takes years to complete, as in the case of the construction of a new manufacturing plant for pharmaceuticals, which is complex and subject to regulatory requirements. In other situations, capacity may be added or reduced relatively quickly, as in the case of service industries, such as advertising. In cyclical markets, such as commercial paper and paperboard, capacity conditions can change rapidly. Strong demand in the early stages of an economic recovery can result in the addition of supply. Given the long lead times to build manufacturing plants, new supply may reach the market just as demand slows, rapidly changing capacity conditions from tight to loose. Such considerations underscore the importance of forecasting long-term industry demand in evaluating industry investments in capacity.

\section{Physical Capacity}
Generally, if new capacity is physical-for example, an auto manufacturing plant or a massive cargo ship-it will take longer for new capacity to come on line to meet an increase in demand, resulting in a longer period of tight conditions. Unfortunately, capacity additions frequently overshoot long-run demand, and because physical capital is often hard to re-deploy, industries reliant on physical capacity may get stuck in conditions of excess capacity and diminished pricing power for an extended period.

Note that capacity need not be physical. After Hurricane Katrina caused enormous damage to the southeastern United States in 2005, reinsurance rates quickly spiked as customers sought to increase their financial protection from future hurricanes. However, these high reinsurance rates enticed a flood of fresh capital into the reinsurance market, and a number of new reinsurance companies were founded, which brought rates back down.

Financial and human capital, in contrast, can be quickly shifted to new uses. In the reinsurance example, for instance, financial capital was quick to enter the reinsurance market and take advantage of tight capacity conditions, but if too much capital had entered the market, some portion of that capital could easily have left to seek higher returns elsewhere. Money can be used for many things, but massive bulk cargo vessels are not useful for much more than transporting heavy goods across oceans. Exhibit 10 shows the elements of a strategic analysis relating to industry capacity.

Exhibit 10: Elements of a Strategic Analysis: Industry Capacity

\begin{center}
\begin{tabular}{|c|c|c|c|}
\hline
 & Branded Pharmaceuticals & Oil Services & Confections/Candy \\
\hline
Major Companies & $\begin{array}{l}\text { Pfizer, Novartis, Merck, GSK } \\ \text { plc }\end{array}$ & $\begin{array}{l}\text { Schlumberger, Baker Hughes, } \\ \text { Halliburton }\end{array}$ & Hershey, Mars/Wrigley, Nestle \\
\hline
$\begin{array}{l}\text { Impact of Industry } \\ \text { Capacity }\end{array}$ & $\begin{array}{l}\text { Not Applicable: Pharmaceutical } \\ \text { pricing is primarily deter- } \\ \text { mined by patent protection } \\ \text { and regulatory issues, including } \\ \text { government approvals of drugs } \\ \text { and of manufacturing facilities. } \\ \text { Manufacturing capacity is of } \\ \text { little importance. }\end{array}$ & $\begin{array}{l}\text { Medium/High: Demand can } \\ \text { fluctuate quickly depending on } \\ \text { commodity prices, and industry } \\ \text { players often find themselves } \\ \text { with too few (or too many) } \\ \text { employees on the payroll. }\end{array}$ & $\begin{array}{l}\text { Not Applicable: Pricing is driven } \\ \text { primarily by brand strength. } \\ \text { Manufacturing capacity has little } \\ \text { effect. }\end{array}$ \\
\hline
\end{tabular}
\end{center}

\section{EXAMPLE 7}
\begin{enumerate}
  \item Which of the following companies is most likely to have the greatest ability to quickly increase capacity?
A. Legal services provider
B. Manufacturing company producing heavy machinery
C. Company operating cargo ships
\end{enumerate}

\section{Solution}
A is correct. Capacity increases in providing legal services would not require significant fixed capital investments. $\mathrm{B}$ and $\mathrm{C}$ are incorrect because the companies would require capital investments and capacity expansion would take time to implement.

\section{Market Share Stability}
Examining the stability of industry market shares over time is similar to thinking about barriers to entry and the frequency with which new players enter an industry. In fact, barriers to entry and the frequency of new product introductions, together with such factors as product differentiation, affect market shares. Stable market shares typically indicate less competitive industries; unstable market shares often indicate highly competitive industries that have limited pricing power. Exhibit 11 illustrates the development of market shares in two medical devices markets.

\section{Exhibit 11: Market Shares in Orthopedic and Metal Mesh Devices}
Over 2010-2020, the orthopedic device industry-mainly artificial hips and knees-has been a relatively stable global oligopoly led by Stryker, Zimmer Biomet, Smith \& Nephew, and Johnson \& Johnson, which jointly control about three quarters of the global market.

In contrast, the market for stents-small metal mesh devices used to prop open blocked arteries-while controlled by a handful of companies, has seen market shares change from being very stable to being marked by rapid change. Johnson \& Johnson, which together with Boston Scientific, dominated the US stent market for many years, went from having about half the market in 2007 to having only 15\% in early 2009 and exited the market in 2011; over the same period, Abbott Laboratories increased its market share from $0 \%$ to around $30 \%$. The reason for this change was the launch of new stents by Abbott and Medtronic, which took market share from Johnson \& Johnson's and Boston Scientific's established stents. Orthopedic device companies have experienced more stability in their market shares for two reasons. First, artificial hips and knees are complicated to implant, and each manufacturer's products are slightly different. As a result, orthopedic surgeons become proficient at using one or several companies' devices and may be reluctant to incur the time and cost of learning how to implant products from a competing company. The second reason is the relatively slow pace of innovation in the orthopedic device industry, making the benefit of switching among product lines relatively low. In addition, the number of orthopedic device companies has remained fairly static over many years.

In contrast, the US stent market has experienced rapid shifts in market shares because of several factors. First, interventional cardiologists seem to be more open to implanting stents from different manufacturers; that tendency may reflect lower switching costs for stents relative to orthopedic devices and the interchangeability of stents. More importantly, however, the pace of innovation in the stent market has become quite rapid, giving cardiologists added incentive to switch to newer stents, with potentially better patient outcomes, as they became available.

Low switching costs plus a relatively high benefit from switching caused market shares to change quickly in the stent market. High switching costs for orthopedic devices coupled with slow innovation resulted in a lower benefit from switching, which led to greater market share stability in orthopedic devices. Exhibit 12 briefly describes market share stability in the three representative industries.

Exhibit 12: Elements of a Strategic Analysis: Market Share

\begin{center}
\begin{tabular}{|c|c|c|c|}
\hline
 & Branded Pharmaceuticals & Oil Services & Confections/Candy \\
\hline
Major Companies & $\begin{array}{l}\text { Pfizer, Novartis, Merck, GSK } \\ \text { plc }\end{array}$ & $\begin{array}{l}\text { Schlumberger, Baker Hughes, } \\ \text { Halliburton }\end{array}$ & Hershey, Mars/Wrigley, Nestle \\
\hline
Industry Stability & $\begin{array}{l}\text { Stable: The branded pharma- } \\ \text { ceutical market is dominated by } \\ \text { major companies and consolida- } \\ \text { tion via mega-mergers. Market } \\ \text { shares shift quickly, however, } \\ \text { as new drugs are approved and } \\ \text { gain acceptance or lose patent } \\ \text { protection. }\end{array}$ & $\begin{array}{l}\text { Unstable: Market shares may } \\ \text { shift frequently depending } \\ \text { on technology offerings and } \\ \text { demand levels. }\end{array}$ & $\begin{array}{l}\text { Very Stable: Market shares } \\ \text { change glacially slowly. }\end{array}$ \\
\hline
\end{tabular}
\end{center}

\section{Price Competition}
A useful tool for analyzing an industry is attempting to think like a customer of the industry. Whatever factor most influences customer purchase decisions is likely to also be the focus of competitive rivalry in the industry. In general, industries for which price is a large factor in customer purchase decisions tend to be more competitive than industries in which customers value other attributes more highly.

Although this depiction may sound like the description of a commodity industry versus a non-commodity industry, it is, in fact, a bit more subtle. Commercial aircraft and passenger cars are certainly more differentiated than lumps of coal or gallons of gasoline, but price nonetheless weighs heavily in the purchase decisions of buyers of aircraft and cars, because fairly good substitutes are easily available. If Airbus charges too much for an aircraft, such as the A350, an airline can buy as an alternative a Boeing 777 (a small amount of "path dependence" characterizes the airline industry, in that an airline with a large fleet of a particular Airbus model will be marginally more likely to stick with that model for a new purchase than it will be to buy a Boeing). If BMW's price for a four-door premium sedan rises too high, customers can switch to an alternative premium brand with similar features. Similar switching can be expected as a result of a unilateral price increase in the case of most industries in the "Weak Pricing Power" column of the two-factor analysis in Exhibit 8 .

Contrast industries in software services that can be characterized by strong pricing power. Such software companies as Adobe, for example, sell a suite of products for creative professionals. Although Adobe's individual products do face competition, its successful transition to selling most of its products via its Creative Cloud has enabled the company to enjoy gross margins above $80 \%$. In addition, the growth of digital companies and content bodes well for Adobe's pricing positioning in the industry.

Returning to a more capital-intensive industry, consider heavy-equipment manufacturers, such as Caterpillar, JCB, and Komatsu. A large wheel loader or combine harvester requires a large capital outlay, so price certainly plays a part in the buyers' decisions. However, other factors are important enough to customers to allow these companies a small amount of pricing power. Construction equipment is typically used as a complement to other gear on a large project, which means that downtime for repairs increases costs because, for example, hourly laborers must wait for a bulldozer to be fixed. Broken equipment is also expensive for agricultural users, who may have only a few days in which to harvest a season's crop. Because of the importance to users of their products' reliability and their large service networks-which are important "differentiators," or factors bestowing a competitive advantage-Caterpillar, Komatsu, and Deere have historically been able to price their equipment at levels that have generated solid returns on invested capital.

\section{Industry Life Cycle}
An industry's life-cycle position often has a large impact on its competitive dynamics, making this position an important component of the strategic analysis of an industry.

\section{Description of an Industry Life-Cycle Model}
Industries, like individual companies, tend to evolve over time and usually experience significant changes in the rate of growth and levels of profitability along the way. Just as an investment in an individual company requires careful monitoring, industry analysis is a continuous process to identify changes that may be occurring or likely to occur. A useful framework for analyzing the evolution of an industry is an industry life-cycle model, which identifies the sequential stages that an industry typically goes through. The five stages of an industry life-cycle model are embryonic, growth, shakeout, mature, and decline. Each stage is characterized by different opportunities and threats. Exhibit 13 shows the model as a curve illustrating the level and growth rate of demand at each stage.

\section{Exhibit 13: An Industry Life-Cycle Model}
\begin{center}
\includegraphics[max width=\textwidth]{2023_05_04_7b535d0a870224f62e3dg-358}
\end{center}

Source: Based on Figure 2.4 in Hill and Jones (2008).

\section{Embryonic}
An embryonic industry is one that is just beginning to develop. For example, in 2014, the mobile food delivery industry was in the embryonic stage (it has grown to become a nearly US\$15 billion industry in 2020), and in 1997, the global social media industry was just getting started. Characteristics of the embryonic stage include slow growth and high prices because customers tend to be unfamiliar with the industry's product and volumes are not yet sufficient to achieve meaningful economies of scale. Increasing product awareness and developing distribution channels are key strategic initiatives of companies during this stage. Substantial investment is generally required, and the risk of failure is high. A majority of startup companies do not succeed.

\section{Growth}
A growth industry tends to be characterized by rapidly increasing demand, improving profitability, falling prices, and relativity low competition among companies in the industry. Demand is fueled by new customers entering the market, and prices fall as economies of scale are achieved and as distribution channels develop. The threat of new competitors entering the industry is usually highest during the growth stage, when barriers to entry are relatively low. Competition tends to be relatively limited, however, because rapidly expanding demand provides companies with an opportunity to grow without needing to capture market share from competitors. Industry profitability improves as volumes rise and economies of scale are attained.

\section{Shakeout}
The shakeout stage is usually characterized by slowing growth, intense competition, and declining profitability. During the shakeout stage, demand approaches market saturation levels because few new customers are left to enter the market. Competition is intense as growth becomes increasingly dependent on market share gains. Excess industry capacity begins to develop as the rate at which companies continue to invest exceeds the overall growth of industry demand. In an effort to boost volumes to fill excess capacity, companies often cut prices, so industry profitability begins to decline. During the shakeout stage, companies increasingly focus on reducing their cost structure (restructuring) and building brand loyalty. Marginal companies may fail or merge with others.

\section{Mature}
Characteristics of a mature industry include little or no growth, industry consolidation, and relatively high barriers to entry. Industry growth tends to be limited to replacement demand and population expansion because the market at this stage is completely saturated. As a result of the shakeout, mature industries often consolidate and become oligopolies. The surviving companies tend to have brand loyalty and relatively efficient cost structures, both of which are significant barriers to entry. During periods of stable demand, companies in mature industries tend to recognize their interdependence and try to avoid price wars. Periodic price wars do occur, however, most notably during periods of declining demand (such as during economic downturns). Companies with superior products or services are likely to gain market share and experience above-industry-average growth and profitability.

\section{Decline}
During the decline stage, industry growth turns negative, excess capacity develops, and competition increases. Industry demand at this stage may decline for a variety of reasons, including technological substitution (for example, the newspaper industry has been declining for many years as more people turn to video platforms, the internet, and 24-hour cable news networks for information), social changes, and global competition. As demand falls, excess capacity in the industry forms and companies respond by cutting prices, which often leads to price wars. The weaker companies often exit the industry at this point, merge, or redeploy capital into different products and services.

When overall demand for an industry's products or services is declining, the opportunity for individual companies to earn above-average returns on invested capital tends to be less than when demand is stable or increasing, because of price cutting and higher per-unit costs as production is cut back. Exhibit 14 shows the life cycle stages for three representative industries.

\section{Exhibit 14: Elements of a Strategic Analysis: Industry Life Cycle}
\begin{center}
\begin{tabular}{llll}
\hline
 & Branded Pharmaceuticals & Oil Services & Confections/Candy \\
\hline
Major Companies & Pfizer, Novartis, Merck, GSK & $\begin{array}{l}\text { Schlumberger, Baker Hughes, } \\ \text { plc }\end{array}$ & Hershey, Mars/Wrigley, Nestle \\
\hline
Life Cycle & $\begin{array}{l}\text { Mature: Overall demand does } \\ \text { not change greatly from year to }\end{array}$ & $\begin{array}{l}\text { Mature: Demand does fluctuate } \\ \text { with energy prices, but normal- } \\ \text { ized revenue growth is only in } \\ \text { the population trends and pricing. }\end{array}$ &  \\
\hline
\end{tabular}
\end{center}

\section{Using an Industry Life-Cycle Model}
In general, new industries tend to be more competitive (with lots of players entering and exiting) than mature industries, which often have stable competitive environments and players that are more interested in protecting what they have than in gaining lots of market share. However, as industries move from maturity to decline, competitive pressures may increase again as industry participants perceive a zero-sum environment and fight over pieces of an ever-shrinking pie.

\section{Relating Management Behavior to Industry Life Cycle}
An important point for the analyst to think about is where a company is relative to where its industry sits in the life cycle. Companies in growth industries should be building customer loyalty as they introduce consumers to new products or services, building scale, and reinvesting heavily in their operations to capitalize on increasing demand. They are probably not focusing strongly on internal efficiency. Growth companies typically reinvest their cash flows in new products and product platforms rather than return cash flows to shareholders because these companies still have many opportunities to deploy their capital to make higher returns. Some may be able to sustain or even accelerate growth through innovation or by expansion into new markets.

Companies in mature industries are likely to be pursuing replacement demand or incremental demand rather than new buyers and are probably focused on extending successful product lines rather than introducing revolutionary new products. They are also probably focusing on cost rationalization and efficiency gains rather than on taking lots of market share. Importantly, these companies have fewer growth opportunities than in the previous stage and thus more limited avenues for profitably reinvesting capital, but they often have strong cash flows. Given their strong cash flows and relatively limited reinvestment opportunities, such companies should be, according to a common perspective, returning capital to shareholders via share repurchases or dividends.

\section{Spotting Warning Signs}
What can be a concern is a middle-aged company acting like a young growth company and pouring capital into projects with low return prospects in an effort to pursue size for its own sake. Many companies have a difficult time managing the transition from growth to maturity, and their returns on capital-and shareholder returns-may suffer until management decides to allocate capital in a manner more appropriate to the company's life-cycle stage.

\section{Limitations of Industry Life-Cycle Analysis}
Although models can provide a useful framework for thinking about an industry, the evolution of an industry does not always follow a predictable pattern. Various external factors that we will discuss in the subsequent sections influence life cycles. These include technological, demographic, social, environmental, or regulatory factors that may significantly affect the shape of the pattern, causing some stages to be longer or shorter than expected and, in certain cases, causing some stages to be skipped altogether. Technological changes may cause an industry to experience an abrupt shift from growth to decline, thus skipping the shakeout and mature stages. For example, the movie rental industry experienced rapid change as consumers switched from renting physical movies (on VHS tapes and DVDs) to on-demand services, such as downloading movies from the internet or through their cable providers.

Regulatory changes can also have a profound impact on the structure of an industry. A prime example is the deregulation of the telecommunications industry in the United States and Europe in the 1990s, which transformed a monopolistic industry into an intensely competitive one.

Social changes also have the ability to affect the profile of an industry. The casual dining industry has benefited over the past 30 years from the increase in the number of dual-income families in many large markets, who often have more income but less time to cook meals to eat at home. Thus, life-cycle models tend to be most useful for analyzing industries during periods of relative stability. They are less practical when the industry is experiencing rapid change over a short time period because of external or other special circumstances.

Another limiting factor of life-cycle models is that not all companies in an industry experience similar performance. The key objective for the analyst is to identify the potential winners while avoiding potential losers. Highly profitable companies can exist in competitive industries with below-average profitability-and vice versa, as shown by the cellular phone manufacturer Nokia for a number of years starting in the 1990s. It has been able to use its scale to generate levels of profitability that were well above average despite operating in a highly competitive industry. In contrast, despite the historically above-average growth and profitability of the software industry, countless examples exist of software companies that failed to ever generate a profit and eventually went out of business.

\section{EXAMPLE 8}
\section{Industry Life Cycle}
\begin{enumerate}
  \item An industry experiencing slow growth and high prices is best characterized as being in the:
A. mature stage.
B. shakeout stage.
C. embryonic stage.
\end{enumerate}

\section*{Solution: }
$\mathrm{C}$ is correct. Both slow growth and high prices are associated with the embryonic stage. High price is not a characteristic of the mature or shakeout stage.

\begin{enumerate}
  \setcounter{enumi}{1}
  \item Which of the following statements about the industry life-cycle model is least accurate?
\end{enumerate}

A. The model is more appropriately used during a period of rapid change than during a period of relative stability.

B. External factors may cause some stages of the model to be longer or shorter than expected, and in certain cases, a stage may be skipped entirely.

C. Not all companies in an industry will experience similar performance, and very profitable companies can exist in an industry with below-average profitability.

\section{Solution:}
A is correct. This statement is the least accurate. The model is best used during a period of relative stability rather than during a period of rapid change.

\section{EXTERNAL INFLUENCES ON INDUSTRY}
We now turn our attention to external factors that affect industries. We will explain macroeconomic, technological, demographic, governmental, social, and environmental influences.

\section{Macroeconomic Influences}
Trends in overall economic activity generally have significant effects on the demand for an industry's products or services. These trends can be cyclical (i.e., related to the changes in economic activity caused by the business cycle) or structural (i.e., related to enduring changes in the composition or magnitude of economic activity). Among the economic variables that usually affect an industry's revenues and profits are the following:

\begin{itemize}
  \item GDP, either in current or constant currency (inflation-adjusted) terms;

  \item interest rates, which represent the cost of debt to consumers and businesses and are important ingredients in financial institutions' revenues and costs;

  \item the availability of credit, which affects business and consumer spending and financial solvency; and - inflation, which reflects the changes in prices of goods and services and influences costs, interest rates, and consumer and business confidence.

\end{itemize}

\section{Technological Influences}
New technologies create new or improved products that can radically change an industry and can also change how other industries that use the products conduct their operations. Exhibit 15 provides an illustration of how innovation bringing new technologies changed the compute hardware and software industry.

\section{Exhibit 15: Technological Influences}
The computer hardware industry provides one of the best examples of how technological change can affect industries. The 1958 invention of the microchip (also known as an "integrated circuit") enabled the computer hardware industry to eventually create a new market of personal computing for the general public and radically extended the use of computers in business, government, and educational institutions.

Moore's law states that the number of transistors that can be inexpensively placed on an integrated circuit doubles approximately every two years. Several other measures of digital technology have improved at exponential rates related to Moore's law, including the size, cost, density, and speed of components. As a result of these trends, the computer hardware industry came to dominate the fields of hardware for word processing and many forms of electronic communication and home entertainment. The computing industry's integrated circuit innovation increased economies of scale and erected large barriers to new entrants because the capital costs of innovation and production became very high. Intel capitalized on both factors, which allowed it to garner an industry market leadership position and to become the dominant supplier of the PC industry's highest-value component (the microprocessor). Thus, Intel became dominant because of its cost advantage, brand power, and access to capital. Along the way, the computer hardware industry was supported and greatly assisted by the complementary industries of computer software and telecommunications (particularly in regard to development of the internet); also important were other industries-entertainment (television, movies, games), retailing, and print media. Ever more powerful integrated circuits, advances in wireless technology, and the convergence of media, which the internet and new wireless technology have facilitated, continue to reshape the uses and the roles of PC hardware in business and personal life. In the middle of the 20th century, few people in the world would have imagined they would ever have any use for a home computer. Today, an estimated 4.3 billion people, or over half the world's population, have access to connected computing. For the United States, the estimate is at least $95 \%$ of the population; it is much less in emerging and underdeveloped countries. More than 8 billion mobile cellular telephone subscriptions exist in the world today.

Exhibit 16 explains how three representative industries are affected by technological change.

\section{Exhibit 16: Technological Influences: Representative Industries}
\begin{center}
\begin{tabular}{|c|c|c|c|}
\hline
 & $\begin{array}{l}\text { Branded } \\ \text { Pharmaceuticals }\end{array}$ & Oil Services & Confections/Candy \\
\hline
$\begin{array}{l}\text { Technological } \\ \text { Influences }\end{array}$ & $\begin{array}{l}\text { Medium/ } \\ \text { High: Biologic } \\ \text { (large-molecule) drugs } \\ \text { are pushing new ther- } \\ \text { apeutic boundaries, } \\ \text { and many large phar- } \\ \text { maceutical companies } \\ \text { have a relatively small } \\ \text { presence in biotech. }\end{array}$ & $\begin{array}{l}\text { Medium/High: } \\ \text { Industry is reasonably } \\ \text { innovative, and play- } \\ \text { ers must reinvest in } \\ \text { R\&D to remain com- } \\ \text { petitive. Temporary } \\ \text { competitive advan- } \\ \text { tages are possible via } \\ \text { commercialization } \\ \text { of new processes or } \\ \text { exploitation of accu- } \\ \text { mulated expertise. }\end{array}$ & $\begin{array}{l}\text { Very Low: Innovation } \\ \text { does not play a major } \\ \text { role in the industry. }\end{array}$ \\
\hline
\end{tabular}
\end{center}

\section{Demographic Influences}
Changes in population size, in the distributions of age and gender, and in other demographic characteristics may have significant effects on economic growth and on the amounts and types of goods and services consumed.

The effects of demographics on industries are exemplified by the impact of Japan's aging population, which has one of the highest percentages of elderly residents ( $26 \%$ over the age of 65) and a very low birth rate. Japan's ministry of health estimates that by 2055 , the percentage of the population over 65 will rise to $40 \%$ and the total population will fall by $25 \%$. These demographic changes are expected by some observers to have negative effects on the overall economy because, essentially, they imply a declining workforce. However, some sectors of the economy stand to benefit from these trends-for example, the health care industry. Another example of demographic influence on an industry is the experience of "baby boomers" (those born in the years after World War II), who reached adulthood in significant numbers in the early 1970s, contributing to a strong demand for new housing throughout the 1970s and during 1995-2005 when their children reached adulthood.

\section{Governmental Influences}
Governmental influence on industries' revenues and profits is pervasive and important. In setting tax rates for corporations and individuals, governments affect profits and incomes, which, in turn, affect corporate and personal spending. Governments also set rules and regulations to protect consumers, employees, and the environment, affecting how companies operate, with implications for profitability. In addition, governments are also major purchasers of goods and services from a range of industries.

Often, governments exert their influence indirectly by empowering other regulatory or self-regulatory organizations (e.g., stock exchanges, medical associations, utility rate setters, and other regulatory commissions) to govern the affairs of an industry. By setting the terms of entry into various sectors, such as financial services and health care, and the rules that companies and individuals must adhere to in these fields, governments control the supply, quality, and nature of many products and services and the public's access to them. For example, in the financial industry, the acceptance of savings deposits from and the issuance of securities to the investing public are usually tightly controlled by governments and their agencies. This control is imposed through rules designed to protect investors from fraudulent operators and to ensure that investors receive adequate disclosure about the nature and risks of their investments. Exhibit 17 illustrates government influences on three representative industries.

\section{Exhibit 17: Government Influences: Representative Industries}
\begin{center}
\begin{tabular}{|c|c|c|c|}
\hline
 & $\begin{array}{l}\text { Branded } \\ \text { Pharmaceuticals }\end{array}$ & Oil Services & Confections/Candy \\
\hline
$\begin{array}{l}\text { Government } \\ \text { \& Regulatory } \\ \text { Influences }\end{array}$ & $\begin{array}{l}\text { Very High: All drugs } \\ \text { must be approved for } \\ \text { sale by national safety } \\ \text { regulators. Patent } \\ \text { regimes may differ } \\ \text { among countries. } \\ \text { Also, health care is } \\ \text { heavily regulated in } \\ \text { most countries. }\end{array}$ & $\begin{array}{l}\text { Medium: Regulatory } \\ \text { frameworks can affect } \\ \text { energy demand at } \\ \text { the margin. Also, } \\ \text { governments play } \\ \text { an important role in } \\ \text { allocating exploration } \\ \text { opportunities to E\&P } \\ \text { companies, which can } \\ \text { indirectly affect the } \\ \text { amount of work flow- } \\ \text { ing down to service } \\ \text { companies. }\end{array}$ & $\begin{array}{l}\text { Low: Industry is not } \\ \text { regulated, but child- } \\ \text { hood obesity concerns } \\ \text { in developed mar- } \\ \text { kets are a low-level } \\ \text { potential threat. Also, } \\ \text { high-growth emerging } \\ \text { markets may block } \\ \text { entry of established } \\ \text { players into their mar- } \\ \text { kets, possibly limiting } \\ \text { growth. }\end{array}$ \\
\hline
\end{tabular}
\end{center}

\section{EXAMPLE 9}
\section{The Effects of Purchases by Government-Related Entities on the Aerospace Industry}
The aerospace, construction, and firearms industries are prime examples of industries for which governments are major customers and whose revenues and profits are significantly-in some cases, predominantly-affected by their sales to governments. An example is Airbus, a major player in aerospace, defense, and related services with head offices in Paris and Ottobrunn, Germany. In 2017, Airbus generated revenues of $€ 66.8$ billion and employed an international workforce of about 130,000. Besides being a leading manufacturer of commercial aircraft, a business whose customers include government-owned airlines, Airbus also includes Airbus Military, providing tanker, transport, and mission aircraft; Airbus Helicopters SAS, the world's largest helicopter supplier; and EADS Astrium, the European leader in space programs, including Ariane and Galileo. Its Defence \& Security Division is a provider of comprehensive systems solutions and makes Airbus the major partner in the Eurofighter consortium and a stakeholder in missile systems provider MBDA. These divisions are highly dependent on orders from governments.

\section{Social Influences}
Societal changes involving how people work, spend their money, enjoy their leisure time, and conduct other aspects of their lives can have significant effects on the sales of various industries.

Tobacco consumption in the United Kingdom provides a good example of the effects of social influences on an industry. Although the role of government in curbing tobacco advertising, legislating health warnings on the purchases of tobacco products, and banning smoking in public places (such as restaurants, bars, public houses, and transportation vehicles) probably has been the most powerful apparent instrument of changes in tobacco consumption, the forces underlying that change have really been social in nature. These forces are increasing consciousness on the part of the population of the damage to the health of tobacco users and those in their vicinity from smoking, the increasing cost to individuals and governments of the chronic illnesses caused by tobacco consumption, and the accompanying shift in public perception of smokers from socially correct to socially incorrect or even inconsiderate or reckless. As a result of these changes in society's views of smoking, cigarette consumption in the United Kingdom declined from 102.5 billion cigarettes in 1990 to less than 40.0 billion three decades later, placing downward pressure on tobacco companies' unit sales.

\section{Environmental Influences}
As industries continue to adapt to new technologies and strategies in order to compete and grow, the need to evaluate and mitigate environmental impact is an important influence. For example, climate change is shifting the way entire industries are perceived and evolving. With limited resources on our planet as a whole, climate change poses a real threat to many industries' growth and profitability that should be taken into consideration when evaluating external influences. These influences can be seen through three channels:

\begin{itemize}
  \item consumer perception for certain brands, products, and services;

  \item increased government regulations and protections;

  \item potential disruptions to supply chains and the ability to operate, such as an increase in natural disasters or resource shortages in water or energy.

\end{itemize}

Take, for example, the agriculture industry. Global greenhouse gas emissions from agriculture, forestry, and other land use account for $25 \%$ of total carbon emissions globally. This is more than two times the pollution generated from all the cars on the planet. This factor and the rise of organic, natural, cruelty-free, and vegan foods will cause the agriculture industry to face strong environmental influences throughout the 21st century. The shift toward healthier and plant-based diets has become a major trend in the last 10 years. Organic food sales in the United States rose almost 6\% in 2018 to reach $\$ 47.9$ billion (according to the 2019 Organic Industry Survey).

Because of consumer awareness of climate change and a growing movement to protect animal rights, companies in the livestock industry could face significant reputational risk and disruption to their brands along with potential government intervention/fines. Public awareness of the environmental impact of raising livestock has been increasing-for example, that cattle are a significant source of methane (a more potent contributor to climate change than $\mathrm{CO}_{2}$ ) and that cattle require very large quantities of land, fertilizer, water, and grain. Additionally, livestock treatment has been a huge concern and subject of debate among many organizations, including animal rights groups. As the trend toward ESG (environmental, social, and governance) and sustainable investing continues to grow, there is an increasing awareness among investors of the responsibility to work with food businesses to better ensure the mitigation of potential headline risk of animal cruelty and supply chain disruption from contaminated meat. Innovation in this space has evolved to include plant-based meat substitutes (e.g. Beyond Meat and Impossible Foods), which has become a multi-billion-dollar food category over the last decade.

\section{EXAMPLE 10}
\begin{enumerate}
  \item Company A is a mining company with operations in several countries.
\end{enumerate}

Company B is a technology company providing video conferencing services. Company $\mathrm{C}$ is a consumer company, an apparel producer, with facilities in a number of Latin American and Asian countries. Which of the companies belongs to an industry most likely to be affected by environmental factors that analysts should evaluate?
A. Company A
B. Company B
C. Company C

\section{Solution:}
A is correct. Environmental considerations should be incorporated in the industry analysis of Company A. It operates in a sector that has high exposure to greenhouse gas emissions, as well as water management and waste and hazardous materials management, which could incur additional costs to running businesses in this sector.

\begin{enumerate}
  \setcounter{enumi}{1}
  \item Which of the following statements best describes social influences that should be considered in evaluating the Consumer Goods industry?
\end{enumerate}

Statement 1 Supply chain management, selling practices, and product labeling

Statement $2 \quad$ Access and affordability, pricing policy

Statement 3 Impact of the industry on the environment

\section{Solution:}
Statement 1 is correct. Statements 2 and 3 describe characteristics that, although important, are not considered to be of social nature.

\begin{enumerate}
  \setcounter{enumi}{2}
  \item Which of the following describes one of the ways governments influence large companies that produce and offer services relating to heating, air conditioning, and lighting systems?
\end{enumerate}

A. Purchasing goods and services

B. Determining availability of credit

C. Providing raw materials

\section{Solution:}
A is correct. Governments are a major buyer of goods and services. B is incorrect because governments do not normally provide credit to the industry. $\mathrm{C}$ is incorrect because the private sector provides raw materials to customers.

\section{EXAMPLE 11}
\section{The Airline Industry: A Case Study of Many Influences}
The global airline industry exemplifies many of the concepts and influences we have discussed.

\section{Life-Cycle Stage}
The industry can be described as having some mature characteristics because average annual growth in global passenger traffic has remained relatively stable, growing at a single-digit rate for the first two decades of the 21st century, with the Asian and Middle Eastern markets growing faster than the more mature markets of North America and Europe.

\section{Sensitivity to the Business Cycle}
The airline industry is a cyclical industry; global economic activity produces swings in revenues and, especially, profitability, because of the industry's high fixed costs and operating leverage. In 2009, for example, global passenger traffic declined by approximately $3.5 \%$ and airlines lost close to US $\$ 11.0$ billion, which was down from a global industry profit of US $\$ 12.9$ billion in 2007 . The industry tends to respond early to upward and downward moves in economic cycles; depending on the region, air travel changes at 1.5 times to 2.0 times GDP growth.

\section{Regulation}
The industry is highly regulated, with governments and airport authorities playing a large role in allocating routes and airport take-off and landing slots. Government agencies and the International Airline Transport Association set rules for aircraft and flight safety.

\section{Product Differentiation and Customer Behavior}
Airline customers tend to have low brand loyalty (except at the extremes of high and low prices and service); leisure travelers focus mainly on price, and business travelers focus mostly on schedules and service. Product and service differentiation at particular price points is low because aircraft, cabin configuration, and catering tend to be quite similar in most cases. For leisure travelers, the price competition is intense and is led by low-cost discount carriers, including Southwest Airlines in the United States, Ryanair in Europe, and Air Asia in Asia. For business travelers, the major scheduled airlines and a few service-quality specialists, such as Singapore Airlines, are the main contenders.

\section{Technology and Costs}
Fuel costs (typically more than $25 \%$ of total costs and highly volatile) and labor costs (around 10\% of total costs) have been the focus of management cost-reduction efforts. The airline industry is highly unionized, and labor strikes have frequently been a source of costly disruptions to the industry. Technology and innovation have always played a major role, impacting fuel efficiency and operating costs. Technology also poses a threat to the growth of business air travel in the form of improved and widely adopted telecommunications-notably, videoconferencing and webcasting.

\section{Social and Environmental Influences}
Arguably, the airline industry has been a great force in shaping demography by permitting difficult-to-access geographical areas to be settled with large populations. At the same time, large numbers of post-World War II baby boomers have been a factor in generating growth in demand for air travel in the past half century. Environmental issues now play a role in the airline industry; carbon emissions, for example, have come under scrutiny by environmentalists and governments.

\section{Industry Comparison}
To illustrate how these elements might be applied, Exhibit 18 reproduces the factors discussed in stages and illustrated in the context of three representative industries.

Exhibit 18: Elements of a Strategic Analysis for Three Industries

\begin{center}
\begin{tabular}{llll}
\hline
 & Branded Pharmaceuticals & Oil Services & Confections/Candy \\
\hline
Major Companies & $\begin{array}{l}\text { Pfizer, Novartis, Merck, GSK } \\ \text { plc }\end{array}$ & $\begin{array}{l}\text { Schlumberger, Baker Hughes, } \\ \text { Halliburton }\end{array}$ & Hershey, Mars/Wrigley, Nestle \\
\hline
\end{tabular}
\end{center}

Barriers to Entry

Very High: Substantial financial and intellectual capital required to compete effectively. A potential new entrant would need to create a sizable R\&D operation, a global distribution network, and large-scale manufacturing capacity.

Level of Concentration

Concentrated: A small number of companies control the bulk of the global market for branded drugs. Recent mergers have increased the level of concentration.

Impact of Industry

Not Applicable: Pharmaceutical Capacity pricing is primarily determined by patent protection and regulatory issues, including government approvals of drugs and of manufacturing facilities. Manufacturing capacity is of little importance.

Industry Stability

Stable: The branded pharmaceutical market is dominated by major companies and consolida tion via mega-mergers. Market shares shift quickly, however, as new drugs are approved and gain acceptance or lose patent protection.

Life Cycle

Mature: Overall demand does not change greatly from year to year. Medium: Technological expertise is required, but a high level of innovation allows niche companies to enter the industry and compete in specific areas.

Fragmented: Although only a small number of companies provide a full range of services, many smaller players compete effectively in specific areas. Service arms of national oil companies may control significant market share in their own countries, and some product lines are concentrated in the mature US market.

Medium/High: Demand can fluctuate quickly depending on commodity prices, and industry players often find themselves with too few (or too many) employees on the payroll.

Unstable: Market shares may shift frequently depending on technology offerings and demand levels.

Mature: Demand does fluctuate with energy prices, but normalized revenue growth is only in the mid-single digits. Very High: Low financial or technological hurdles, but new players would lack the established brands that drive consumer purchase decisions.

Very Concentrated: The top four companies have a large proportion of global market share. Recent mergers have increased the level of concentration.

Not Applicable: Pricing is driven primarily by brand strength.

Manufacturing capacity has little effect.

Very Stable: Market shares change glacially slowly.

Mature: Growth is driven by population trends and pricing.

\begin{center}
\begin{tabular}{|c|c|c|c|}
\hline
 & Branded Pharmaceuticals & Oil Services & Confections/Candy \\
\hline
Price Competition & $\begin{array}{l}\text { Low/Medium: In the United } \\ \text { States, price is a minimal factor } \\ \text { because of a consumer- and } \\ \text { provider-driven, deregulated } \\ \text { health care system. Price is } \\ \text { a larger part of the decision } \\ \text { process in single-payer systems, } \\ \text { where efficacy hurdles are } \\ \text { higher. }\end{array}$ & $\begin{array}{l}\text { High: Price is a major factor in } \\ \text { purchasers' decisions. Some } \\ \text { companies have modest pricing } \\ \text { power because of a wide range } \\ \text { of services or best-in-class tech- } \\ \text { nology, but primary customers } \\ \text { (major oil companies) can } \\ \text { usually substitute with in-house } \\ \text { services if prices are too high. } \\ \text { Also, innovation tends to diffuse } \\ \text { quickly throughout the industry. }\end{array}$ & $\begin{array}{l}\text { Low: A lack of private-label } \\ \text { competition keeps pricing stable } \\ \text { among established players, and } \\ \text { brand familiarity plays a much } \\ \text { larger role in consumer purchase } \\ \text { decisions than price. }\end{array}$ \\
\hline
Demographic Influences & $\begin{array}{l}\text { Positive: Populations of devel- } \\ \text { oped markets are aging, which } \\ \text { slightly increases demand. }\end{array}$ & Not Applicable & Not Applicable \\
\hline
$\begin{array}{l}\text { Government \& } \\ \text { Regulatory Influences }\end{array}$ & $\begin{array}{l}\text { Very High: All drugs must be } \\ \text { approved for sale by national } \\ \text { safety regulators. Patent regimes } \\ \text { may differ among countries. } \\ \text { Also, health care is heavily reg- } \\ \text { ulated in most countries. }\end{array}$ & $\begin{array}{l}\text { Medium: Regulatory frame- } \\ \text { works can affect energy demand } \\ \text { at the margin. Also, govern- } \\ \text { ments play an important role in } \\ \text { allocating exploration opportu- } \\ \text { nities to E\&P companies, which } \\ \text { can indirectly affect the amount } \\ \text { of work flowing down to service } \\ \text { companies. }\end{array}$ & $\begin{array}{l}\text { Low: Industry is not regulated, } \\ \text { but childhood obesity concerns } \\ \text { in developed markets are a } \\ \text { low-level potential threat. Also, } \\ \text { high-growth emerging markets } \\ \text { may block entry of established } \\ \text { players into their markets, possi- } \\ \text { bly limiting growth. }\end{array}$ \\
\hline
$\begin{array}{l}\text { Technological } \\ \text { Influences }\end{array}$ & $\begin{array}{l}\text { Medium/High: Biologic } \\ \text { (large-molecule) drugs are } \\ \text { pushing new therapeutic } \\ \text { boundaries, and many large } \\ \text { pharmaceutical companies have } \\ \text { a relatively small presence in } \\ \text { biotech. }\end{array}$ & $\begin{array}{l}\text { Medium/High: Industry is } \\ \text { reasonably innovative, and } \\ \text { players must reinvest in R\&D to } \\ \text { remain competitive. Temporary } \\ \text { competitive advantages are } \\ \text { possible via commercialization } \\ \text { of new processes or exploitation } \\ \text { of accumulated expertise. }\end{array}$ & $\begin{array}{l}\text { Very Low: Innovation does not } \\ \text { play a major role in the industry. }\end{array}$ \\
\hline
$\begin{array}{l}\text { Growth vs. Defensive } \\ \text { vs. Cyclical }\end{array}$ & $\begin{array}{l}\text { Defensive: Demand for most } \\ \text { health care services does not } \\ \text { fluctuate with the economic } \\ \text { cycle, but demand is not strong } \\ \text { enough to be considered } \\ \text { "growth." }\end{array}$ & $\begin{array}{l}\text { Cyclical: Demand is highly vari- } \\ \text { able and depends on oil prices, } \\ \text { exploration budgets, and the } \\ \text { economic cycle. }\end{array}$ & $\begin{array}{l}\text { Defensive: Demand for candy anc } \\ \text { gum is extremely stable. }\end{array}$ \\
\hline
\end{tabular}
\end{center}

Example 12 reviews some of the information presented in Exhibit 18.

\section{EXAMPLE 12}
\section{External Influences}
\begin{enumerate}
  \item Which of the following industries is most affected by government regulation?
A. Oil services
B. Pharmaceuticals
C. Confections and candy
\end{enumerate}

\section{Solution:}
B is correct. Exhibit 18 states that the pharmaceutical industry has a high amount of government and regulatory influences.

\begin{enumerate}
  \setcounter{enumi}{1}
  \item Which of the following industries is least affected by technological innovation?
A. Oil services
B. Pharmaceuticals
C. Confections and candy
\end{enumerate}

\section{Solution:}
$\mathrm{C}$ is correct. Exhibit 18 states that innovation does not play a large role in the candy industry.

\begin{enumerate}
  \setcounter{enumi}{2}
  \item Which of the following statements about industry characteristics is least accurate?
\end{enumerate}

A. Manufacturing capacity has little effect on pricing in the confections/ candy industry.

B. The branded pharmaceutical industry is considered to be defensive rather than a growth industry.

C. With respect to the worldwide market, the oil services industry has a high level of concentration with a limited number of service providers.

\section{Solution:}
$\mathrm{C}$ is correct; it is a false statement. From a worldwide perspective, the industry is considered fragmented. Although a small number of companies provide the full range of services, competition by many smaller players occurs in niche areas. In addition, national oil service companies control significant market share in their home countries.

\section{COMPANY ANALYSIS}
$$
\begin{array}{|l}
\text { describe the elements that should be covered in a thorough company } \\
\text { analysis }
\end{array}
$$

Company analysis includes an analysis of the company's financial position, products and/or services, and competitive strategy (its plans for responding to the threats and opportunities presented by the external environment). Company analysis takes place after the analyst has gained an understanding of a company's external environmentthe macroeconomic, demographic, governmental, technological, environmental, and social forces influencing the industry's competitive structure. The analyst should seek to determine whether the strategy is primarily defensive or offensive in its nature and how the company intends to implement the strategy. Porter (2008) identified two chief competitive strategies: a low-cost strategy (cost leadership) and a product/service differentiation strategy:

\begin{itemize}
  \item Low-cost strategy: Companies strive to become the low-cost producers and to gain market share by offering their products and services at lower prices than their competition while still making a profit margin sufficient to generate a superior rate of return based on the higher revenues achieved. Low-cost strategies may be pursued defensively to protect market positions and returns or offensively to gain market share and increase returns. Pricing also can be defensive (when the competitive environment is one of low rivalry) or aggressive (when rivalry is intense). In the case of intense rivalry, pricing may even become predatory -that is, aimed at rapidly driving competitors out of business at the expense of near-term profitability. The hope in such a strategy is that having achieved a larger market share, the company can later increase prices to generate higher returns than before. For example, the ride-sharing industry has produced fierce competition among companies that seek to capture market share by offering incentives and discount pricing for rides. Companies seeking to follow low-cost strategies must have tight cost controls, efficient operating and reporting systems, and appropriate managerial incentives. In addition, they must commit themselves to painstaking scrutiny of production systems and their labor forces and to low-cost designs and product distribution. In some cases, they must be able to invest in productivity-improving capital equipment and to finance that investment at a low cost of capital.

  \item Differentiation strategies: Companies attempt to establish themselves as the suppliers or producers of products and services that are unique either in quality, type, or means of distribution. To be successful, their price premiums must be above their costs of differentiation and the differentiation must be appealing to customers and sustainable over time. Corporate managers who successfully pursue differentiation strategies tend to have strong market research teams to identify and match customer needs with product development and marketing. Such a strategy puts a premium on employing creative and inventive people.

\end{itemize}

\section{Elements That Should Be Covered in a Company Analysis}
A thorough company analysis, particularly as presented in a research report, should

\begin{itemize}
  \item provide an overview of the company (corporate profile), including a basic understanding of its businesses, investment activities, corporate governance, and perceived strengths and weaknesses;

  \item explain relevant industry characteristics;

  \item analyze the demand for the company's products and services;

  \item analyze the supply of products and services, which includes an analysis of costs;

  \item explain the company's pricing environment; and

  \item present and interpret relevant financial ratios, including comparisons over time and comparisons with competitors, using multi-year spreadsheets with historical and forecast data

\end{itemize}

Company analysis often includes forecasting the company's financial statements, particularly when the purpose of the analysis is to use a discounted cash flow method to value the company's common equity. Exhibit 19 provides a checklist of points to cover in a company analysis. The list is not exhaustive and may need to be adapted to serve the needs of a particular company analysis.

\section{Exhibit 19: A Checklist for Company Analysis}
\section{Corporate Profile}
Company's major products \& services \& current position in industry \& history

Company's product life-cycle stages

Research \& development \& capital expenditure activities past, present \& planned

Governance Arrangements

\begin{itemize}
  \item Board structure, composition, electoral system, anti-takeover provisions \& other corporate
\end{itemize}

governance issues plus Insider ownership levels \& changes

\begin{itemize}
  \item Management strengths, weaknesses, compensation, turnover, \& corporate culture

  \item Labor relations, business ethics, anti-discrimination policy/diversity \& inclusion initiatives

  \item Legal actions \& the company's state of preparedness

\end{itemize}

\section{Industry Characteristics}
\begin{itemize}
  \item Stage in its life cycle

  \item Business-cycle sensitivity or economic characteristics

  \item Typical product life cycles

  \item Brand loyalty, customer switching costs \& intensity of competition

  \item Entry \& exit barriers

  \item Industry supplier considerations

  \item Industry structure \& concentration

  \item Opportunity to differentiate product/service \& relative product/service price, cost \& quality

\end{itemize}

advantages/disadvantages

\begin{itemize}
  \item Technologies used \& resource management (energy, water \& waste)

  \item Government regulation

  \item Ecological impacts \& evaluation

\end{itemize}

\begin{center}
\begin{tabular}{|c|c|c|c|c|}
\hline
$\begin{array}{l}\text { Demand Analysis } \\ \text { - Sources of demand } \\ \text { - Product differentiation } \\ \text { - Past record, sensitivities, } \\ \text { \& correlations with social, } \\ \text { demographic, economic } \\ \text { \& other variables } \\ \text { - Outlook-short, medium } \\ \text { \& long term, including new } \\ \text { product \& business opportunities } \\ \text { - Selling practices } \\ \text { \& product labeling, } \\ \text { access \& affordability }\end{array}$ & $\begin{array}{l} \\ \end{array}$ & $\begin{array}{l}\text { Pricing analysis } \\ \text { - Past relationships among } \\ \text { demand, supply \& prices } \\ \text { - Significance of raw material \& } \\ \text { labor costs \& the outlook for } \\ \text { their cost \& availability } \\ \text { - } \quad \text { Outlook for selling prices, } \\ \text { demand, \& profitability } \\ \text { (current \& anticipated trends) }\end{array}$ & \$ & $\begin{array}{l}\text { Supply Analysis } \\ \text { - Sources (concentration, } \\ \text { competition \& substitutes) } \\ \text { - Industry capacity } \\ \text { outlook-short, medium } \\ \text { \& long term } \\ \text { - Ability to switch supplies } \\ \text { - Company's capacity \& } \\ \text { cost structure } \\ \text { - Import/export } \\ \text { considerations } \\ \text { - Proprietary products or } \\ \text { trademarks } \\ \text { - Labor practices }\end{array}$ \\
\hline
\end{tabular}
\end{center}

\section{Financial Ratios and Measures}
I. Activity Ratios (measuring efficiently):

\begin{itemize}
  \item Days of sales outstanding (DSO), inventory on hand (DOH) \& Days of payables outstanding (DPO)
\end{itemize}

II. Liquidity Ratios (ability to meet its short-term obligations)

\begin{itemize}
  \item Current, quick \& cash ratios

  \item Cash conversion cycle (DOH + DSO - DPO)

\end{itemize}

III. Solvency Ratios (the ability to meet obligations)

\begin{itemize}
  \item Net debt to EBITDA, debt to capital or assets

  \item Financial leverage \& interest cover ratio

  \item Contingent liabilities \& non-arm's-length transactions

\end{itemize}

IV. Profitability Ratios

\begin{itemize}
  \item Margins: gross, operating, pre-tax \& net

  \item Returns: ROA, ROE \& ROIC (net operating profits after tax/average invested capital)

\end{itemize}

V. Financial Statistics and Related Considerations (quantities \& facts about a company's finances)

\begin{itemize}
  \item Growth rates of net sales \& gross profit

  \item EBITDA, net income, operating cash flow, EPS \& operating cash flow per share \& how cash

\end{itemize}

flow relates to capital expenditures

\begin{itemize}
  \item Expected rate of return on retained cash flow

  \item Debt maturities, ability to refinance or repay debt

  \item Dividend payout ratio

  \item Off-balance-sheet liabilities \& contingent liabilities Evaluation of past and current company performance and forecasting future cash flows and a company's prospects is a major topic that is the subject of extensive coverage elsewhere in the CFA Program curriculum. Here, we provide only a brief introduction.

\end{itemize}

To evaluate a company's performance, the key measures presented in the "checklist" in Exhibit 19 should be compared over time and between companies (particularly peer companies). The following formula can be used to analyze how and why a company's return on equity (ROE) differs from that of other companies or its own ROE in other periods by tracing the differences to changes in its profit margin, the productivity of its assets, or its financial leverage:

ROE $=($ Net profit margin: Net earnings $/$ Net sales $) \times($ Asset turn-

over: Net sales/Average total assets $) \times($ Financial leverage: Average total assets/

Average common equity).

The financial statements of a company over time provide numerous insights into the effects of industry conditions on its performance and the success or failure of its strategies. They also provide a framework for forecasting the company's operating performance when given the analyst's assumptions for numerous variables in the future. The financial ratios listed in Exhibit 19 are applicable to a wide range of companies and industries, but other statistics and ratios are often also used.

\section{Spreadsheet Modeling}
Spreadsheet modeling of financial statements to analyze and forecast revenues, operating and net income, and cash flows has become one of the most widely used tools in company analysis. Although spreadsheet models are a valuable tool for understanding past financial performance and forecasting future performance, the complexity of such models can at times be a problem. Because modeling requires the analyst to predict and input numerous items in financial statements, there is a risk of errors-either in assumptions made or in formulas in the model-which can compound, leading to erroneous forecasts. Yet, those forecasts may seem precise because of the sheer complexity of the model. The result is often a false sense of understanding and security on the part of those who rely on the models. To guard against this, before or after a model is completed, a "reality check" of the model is useful.

Such testing for reasonableness can be done by, first, asking what the few most important changes in income statement items are likely to be from last year to this year and the next year and, second, attempting to quantify the effects of these significant changes or "swing factors" on the bottom line. If an analyst cannot summarize in a few points what factors are realistically expected to change income from year to year and is not convinced that these assumptions are correct, then he or she does not really understand the output of the modeling efforts. In general, financial models should be in a format that matches the company's reporting of its financial results or supplementary disclosures or that can be accurately derived from these reports. Otherwise, there will be no natural reality check when the company issues its financial results and the analyst will not be able to compare his or her estimates with actual reported results.

\section{EXAMPLE 13}
\begin{enumerate}
  \item An analyst makes the following statement:
\end{enumerate}

"Analysis of a company's supply includes examination of labor relations; sources of and access to raw materials, including concentration of suppliers; and analysis of the company's production capacity."

The statement is:

A. correct.

B. incorrect because the analysis of supply should also include analysis of the stage in the product life cycle.

C. incorrect because the analysis of supply should consider the product's differentiating characteristics.

\section{Solution:}
A is correct. The statement accurately describes the analysis of a company's supply.

\begin{enumerate}
  \setcounter{enumi}{1}
  \item The challenge with using spreadsheet forecasting models stems from the fact that
\end{enumerate}

A. the analyst may have a false sense of understanding the business.

B. spreadsheet models cannot be used to compare companies with their peer groups.

C. spreadsheet models require precise inputs.

\section{Solution:}
A is correct. B is incorrect because spreadsheet models are frequently used to compare companies. $\mathrm{C}$ is incorrect because although spreadsheet models require precise inputs, this requirement does not mean that they are not helpful.

\section{SUMMARY}
In this reading, we have provided an overview of industry analysis and illustrated approaches that are widely used by analysts to examine an industry.

\begin{itemize}
  \item Company analysis and industry analysis are closely interrelated. Company and industry analysis together can provide insight into sources of industry revenue growth and competitors' market shares and thus the future of an individual company's top-line growth and bottom-line profitability.

  \item Industry analysis is useful for

  \item understanding a company's business and business environment,

  \item identifying active equity investment opportunities,

  \item formulating an industry or sector rotation strategy, and

  \item portfolio performance attribution.

  \item The three main approaches to classifying companies are - products and/or services supplied,

  \item business-cycle sensitivities, and

  \item statistical similarities.

  \item A cyclical company is one whose profits are strongly correlated with the strength of the overall economy.

  \item A non-cyclical company is one whose performance is largely independent of the business cycle.

  \item Commercial industry classification systems include

  \item The Global Industry Classification Standard (GICS)

  \item The Industry Classification Benchmark (ICB)

  \item A limitation of current classification systems is that the narrowest classification unit assigned to a company generally cannot be assumed to constitute its peer group for the purposes of detailed fundamental comparisons or valuation.

  \item A peer group is a group of companies engaged in similar business activities whose economics and valuation are influenced by closely related factors.

  \item The steps in constructing a preliminary list of peer companies are as follows:

  \item Examine commercial classification systems if available. These systems often provide a useful starting point for identifying companies operating in the same industry.

  \item Review the subject company's annual report for a discussion of the competitive environment. Companies frequently cite specific competitors.

  \item Review competitors' annual reports to identify other potential comparables.

  \item Review industry trade publications to identify additional peer companies.

  \item Confirm that each comparable or peer company derives a significant portion of its revenue and operating profit from a business activity similar to that of the subject company.

  \item Not all industries are created equal. Some are highly competitive, with many companies struggling to earn returns in excess of their cost of capital, and other industries have attractive characteristics that enable a majority of industry participants to generate healthy profits.

  \item Differing competitive environments are determined by the structural attributes of the industry. For this important reason, industry analysis is a vital complement to company analysis. The analyst needs to understand the context in which a company operates to fully understand the opportunities and threats that a company faces.

  \item The framework for strategic analysis known as "Porter's five forces" can provide a useful starting point. Porter maintained that the profitability of companies in an industry is determined by five forces: (1) the threat of new entrants, which, in turn, is determined by economies of scale, brand loyalty, absolute cost advantages, customer switching costs, and government regulation; (2) the bargaining power of suppliers, which is a function of the feasibility of product substitution, the concentration of the buyer and supplier groups, and switching costs and entry costs in each case; (3) the bargaining power of buyers, which is a function of switching costs among customers and the ability of customers to produce their own product; (4) the threat of substitutes; and (5) the intensity of rivalry among existing competitors, which, in turn, is a function of industry competitive structure, demand conditions, cost conditions, and the height of exit barriers.

  \item The concept of barriers to entry refers to the ease with which new competitors can challenge incumbents and can be an important factor in determining the competitive environment of an industry. If new competitors can easily enter the industry, the industry is likely to be highly competitive because incumbents that attempt to raise prices will be undercut by newcomers. As a result, industries with low barriers to entry tend to have low pricing power. Conversely, if incumbents are protected by barriers to entry, they may enjoy a more benign competitive environment that gives them greater pricing power over their customers because they do not have to worry about being undercut by startups.

  \item Industry concentration is often, although not always, a sign that an industry may have pricing power and rational competition. Industry fragmentation is a much stronger signal, however, that the industry is competitive and pricing power is limited.

  \item The effect of industry capacity on pricing is clear: Tight capacity gives participants more pricing power because demand for products or services exceeds supply; overcapacity leads to price cutting and a highly competitive environment as excess supply chases demand. The analyst should think about not only current capacity conditions but also future changes in capacity levels-how long it takes for supply and demand to come into balance and what effect that process has on industry pricing power and returns.

  \item Examining the market share stability of an industry over time is similar to thinking about barriers to entry and the frequency with which new players enter an industry. Stable market shares typically indicate less competitive industries, whereas unstable market shares often indicate highly competitive industries with limited pricing power.

  \item An industry's position in its life cycle often has a large impact on its competitive dynamics, so it is important to keep this positioning in mind when performing strategic analysis of an industry. Industries, like individual companies, tend to evolve over time and usually experience significant changes in the rate of growth and levels of profitability along the way. Just as an investment in an individual company requires careful monitoring, industry analysis is a continuous process that must be repeated over time to identify changes that may be occurring.

  \item A useful framework for analyzing the evolution of an industry is an industry life-cycle model, which identifies the sequential stages that an industry typically goes through. The five stages of an industry life cycle according to the Hill and Jones model are

  \item embryonic,

  \item growth,

  \item shakeout

  \item mature, and

  \item decline.

  \item Price competition and thinking like a customer are important factors that are often overlooked when analyzing an industry. Whatever factors most influence customer purchasing decisions are also likely to be the focus of competitive rivalry in the industry. Broadly, industries for which price is a large factor in customer purchase decisions tend to be more competitive than industries in which customers value other attributes more highly.

  \item External influences on industry growth, profitability, and risk include

  \item technology,

  \item demographics,

  \item government,

  \item social factors, and

  \item environmental factors.

  \item Company analysis takes place after the analyst has gained an understanding of the company's external environment and includes answering questions about how the company will respond to the threats and opportunities presented by the external environment. This intended response is the individual company's competitive strategy. The analyst should seek to determine whether the strategy is primarily defensive or offensive in its nature and how the company intends to implement it.

  \item Porter identified two chief competitive strategies:

  \item A low-cost strategy (cost leadership) is one in which companies strive to become the low-cost producers and to gain market share by offering their products and services at lower prices than their competition while still making a profit margin sufficient to generate a superior rate of return based on the higher revenues achieved.

  \item A product/service differentiation strategy is one in which companies attempt to establish themselves as the suppliers or producers of products and services that are unique either in quality, type, or means of distribution. To be successful, the companies' price premiums must be above their costs of differentiation and the differentiation must be appealing to customers and sustainable over time.

  \item A checklist for company analysis includes a thorough investigation of

  \item the corporate profile,

  \item industry characteristics,

  \item demand for products/services,

  \item supply of products/services,

  \item pricing,

  \item financial ratios, and

  \item sustainability metrics.

  \item Spreadsheet modeling of financial statements to analyze and forecast revenues, operating and net income, and cash flows is one of the most widely used tools in company analysis. Spreadsheet modeling can be used to quantify the effects of the changes in certain swing factors on the various financial statements. The analyst should be aware that the output of the model will depend significantly on the assumptions that are made.

\end{itemize}

\section{REFERENCES}
Hill, Charles, and Gareth Jones. 2008. "External Analysis: The Identification of Opportunities and Threats." In Strategic Management: An Integrated Approach. Boston: Houghton Mifflin Co.

Porter, Michael E. 2008. “The Five Competitive Forces That Shape Strategy" Harvard Business Review 86 (1): 78-93.

\section{PRACTICE PROBLEMS}
\begin{enumerate}
  \item Which of the following is least likely to involve industry analysis?
A. Sector rotation strategy
B. Top-down fundamental investing
C. Tactical asset allocation strategy

  \item A sector rotation strategy involves investing in a sector by:

\end{enumerate}

A. making regular investments in it.

B. investing in a pre-selected group of sectors on a rotating basis.

C. timing investment to take advantage of business-cycle conditions.

\begin{enumerate}
  \setcounter{enumi}{2}
  \item Which of the following information about a company would most likely depend on an industry analysis? The company's:
\end{enumerate}

A. treatment of long-lived assets on its financial statements.

B. competitive environment.

C. trends in corporate expenses.

\begin{enumerate}
  \setcounter{enumi}{3}
  \item Which of the following is not a limitation of the cyclical/non-cyclical descriptive approach to classifying companies?
\end{enumerate}

A. A cyclical company may have a growth component in it.

B. Business-cycle sensitivity is a discrete phenomenon rather than a continuous spectrum.

C. A global company can experience economic expansion in one part of the world while experiencing recession in another part.

\begin{enumerate}
  \setcounter{enumi}{4}
  \item A cyclical company is most likely to:
\end{enumerate}

A. have low operating leverage.

B. sell relatively inexpensive products.

C. experience wider-than-average fluctuations in demand.

\begin{enumerate}
  \setcounter{enumi}{5}
  \item A company that is sensitive to the business cycle would most likely:
\end{enumerate}

A. not have growth opportunities.

B. experience below-average fluctuation in demand.

C. sell products that customers can purchase at a later date if necessary.

\begin{enumerate}
  \setcounter{enumi}{6}
  \item Which of the following factors would most likely be a limitation of applying business-cycle analysis to global industry analysis?
\end{enumerate}

A. Some industries are relatively insensitive to the business cycle. B. Correlations of security returns between different world markets are relatively low.

C. One region or country of the world may experience recession while another region experiences expansion.

\begin{enumerate}
  \setcounter{enumi}{7}
  \item In which sector would a manufacturer of personal care products be classified?
A. Health care
B. Consumer staples
C. Consumer discretionary

  \item An automobile manufacturer is most likely classified in which of the following industry sectors?
A. Consumer staples
B. Industrial durables
C. Consumer discretionary

  \item Which of the following statements about commercial and government industry classification systems is most accurate?

\end{enumerate}

A. Many commercial classification systems include private for-profit companies.

B. Both commercial and government classification systems exclude not-for-profit companies.

C. Commercial classification systems are generally updated more frequently than government classification systems.

\begin{enumerate}
  \setcounter{enumi}{10}
  \item Which of the following statements about peer groups is most accurate?
A. Constructing a peer group for a company follows a standardized process.
B. Commercial industry classification systems often provide a starting point for constructing a peer group.
C. A peer group is generally composed of all the companies in the most narrowly defined category used by the commercial industry classification system.

  \item With regard to forming a company's peer group, which of the following statements is not correct?

\end{enumerate}

A. Comments from the management of the company about competitors are generally not used when selecting the peer group.

B. The higher the proportion of revenue and operating profit of the peer company derived from business activities similar to those of the subject company, the more meaningful the comparison.

C. Comparing the company's performance measures with those for a potential peer-group company is of limited value when the companies are exposed to different stages of the business cycle. 13. When selecting companies for inclusion in a peer group, a company operating in three different business segments would:

A. be in only one peer group.

B. possibly be in more than one peer group.

C. not be included in any peer group.

\begin{enumerate}
  \setcounter{enumi}{13}
  \item An industry that most likely has both high barriers to entry and high barriers to exit is the:
\end{enumerate}

A. restaurant industry.

B. advertising industry.

C. automobile industry.

\begin{enumerate}
  \setcounter{enumi}{14}
  \item Which factor is most likely associated with stable market share?
\end{enumerate}

A. Low switching costs

B. Low barriers to entry

C. Slow pace of product innovation

\begin{enumerate}
  \setcounter{enumi}{15}
  \item Which of the following companies most likely has the greatest ability to quickly increase its capacity to offer goods or services?
\end{enumerate}

A. A restaurant

B. A steel producer

C. An insurance company

\begin{enumerate}
  \setcounter{enumi}{16}
  \item Which of the following life-cycle phases is typically characterized by high prices?
A. Mature
B. Growth
C. Embryonic

  \item In which of the following life-cycle phases are price wars most likely to be absent?
A. Mature
B. Decline
C. Growth

  \item When graphically depicting the life-cycle model for an industry as a curve, the variables on the axes are:
A. price and time.
B. demand and time.
C. demand and stage of the life cycle.

  \item Industry consolidation and high barriers to entry most likely characterize which life-cycle stage?
A. Mature
B. Growth
C. Embryonic

  \item Which of the following is most likely a characteristic of a concentrated industry?
A. Infrequent, tacit coordination
B. Difficulty in monitoring other industry members
C. Industry members attempting to avoid competition on price

  \item Which of the following industry characteristics is generally least likely to produce high returns on capital?

\end{enumerate}

A. High barriers to entry

B. High degree of concentration

C. Short lead time to build new plants

\begin{enumerate}
  \setcounter{enumi}{22}
  \item An industry with high barriers to entry and weak pricing power most likely has:
A. high barriers to exit.
B. stable market shares.
C. significant numbers of issued patents.

  \item Economic value is created for an industry's shareholders when the industry earns a return:
A. below the cost of capital.
B. equal to the cost of capital.
C. above the cost of capital.

  \item Which of the following industries is most likely to be characterized as concentrated with strong pricing power?
A. Asset management
B. Alcoholic beverages
C. Household and personal products

  \item A population that is rapidly aging would most likely cause the growth rate of the industry producing eye glasses and contact lenses to:
A. decrease.
B. increase.
C. not change.

  \item If over a long period of time a country's average level of educational accomplishment increases, this development would most likely lead to the country's amount of income spent on consumer discretionary goods to:
A. decrease.
B. increase.
C. not change.

  \item If the technology for an industry involves high fixed capital investment, then one way to seek higher profit growth is by pursuing:
A. economies of scale.
B. diseconomies of scale.
C. removal of features that differentiate the product or service provided.

  \item With respect to competitive strategy, a company with a successful cost leadership strategy is most likely characterized by:
A. a low cost of capital.
B. reduced market share.
C. the ability to offer products at higher prices than those of its competitors.

  \item When conducting a company analysis, the analysis of demand for a company's product is least likely to consider the:

\end{enumerate}

A. company's cost structure.

B. motivations of the customer base.

C. product's differentiating characteristics.

\begin{enumerate}
  \setcounter{enumi}{30}
  \item Which of the following statements about company analysis is most accurate?
\end{enumerate}

A. The complexity of spreadsheet modeling ensures precise forecasts of financial statements.

B. The interpretation of financial ratios should focus on comparing the company's results over time but not with competitors.

C. The corporate profile would include a description of the company's business, investment activities, governance, and strengths and weaknesses.

\section{SOLUTIONS}
\begin{enumerate}
  \item C is correct. Tactical asset allocation involves timing investments in asset classes and does not make use of industry analysis.

  \item $\mathrm{C}$ is correct. A sector rotation strategy is conducted by investors wishing to time investment in industries through an analysis of fundamentals and/or business-cycle conditions.

  \item B is correct. Determination of a company's competitive environment depends on understanding its industry.

  \item B is correct. Business-cycle sensitivity falls on a continuum and is not a discrete "either/or" phenomenon.

  \item C is correct. Cyclical companies are sensitive to the business cycle, with low product demand during periods of economic contraction and high product demand during periods of economic expansion. They, therefore, experience wider-than-average fluctuations in product demand.

  \item C is correct. Customers' flexibility as to when they purchase a product makes the product more sensitive to the business cycle.

  \item C is correct. Varying conditions of recession or expansion around the world would affect the comparisons of companies with sales in different regions of the world.

  \item B is correct. Personal care products are classified as consumer staples in the "Description of Representative Sectors."

  \item C is correct. Automobile manufacturers are classified as consumer discretionary. Consumer discretionary companies derive a majority of revenue from the sale of consumer-related products for which demand tends to exhibit a high degree of economic sensitivity - that is, high demand during periods of economic expansion and low demand during periods of contraction.

  \item C is correct. Commercial systems are generally updated more frequently than government systems and include only publicly traded for-profit companies.

  \item B is correct. Constructing a peer group is a subjective process, and a logical starting point is to begin with a commercially available classification system. This system will identify a group of companies that may have properties comparable to the business activity of interest.

  \item A is correct because it is a false statement. Reviewing the annual report to find management's discussion about the competitive environment and specific competitors is a suggested step in the process of constructing a peer group.

  \item $\mathrm{B}$ is correct. The company could be in more than one peer group depending on the demand drivers for the business segments, although the multiple business segments may make it difficult to classify the company.

  \item $\mathrm{C}$ is correct. For the automobile industry, the high capital requirements and other elements mentioned in the reading provide high barriers to entry, and recognition that auto factories are generally only of use for manufacturing cars implies a high barrier to exit. 15. C is correct. A slow pace of product innovation often means that customers prefer to stay with suppliers they know, implying stable market shares.

  \item C is correct. Capacity increases in providing insurance services would not involve several factors that would be important to the other two industries, including the need for substantial fixed capital investments or, in the case of a restaurant, outfitting rental or purchased space. These requirements would tend to slow down, respectively, steel production and restaurant expansion.

  \item $\mathrm{C}$ is correct. The embryonic stage is characterized by slow growth and high prices.

  \item $\mathrm{C}$ is correct. The growth phase is not likely to experience price wars because expanding industry demand provides companies the opportunity to grow even without increasing market share. When industry growth is stagnant, companies may only be able to grow by increasing market share-for example, by engaging in price competition.

  \item B is correct. The industry life-cycle model shows how demand evolves over time as an industry passes from the embryonic stage through the stage of decline.

  \item A is correct. Industry consolidation and relatively high barriers to entry are two characteristics of a mature-stage industry.

  \item $\mathrm{C}$ is correct. The relatively few members of the industry generally try to avoid price competition.

  \item C is correct. With short lead times, industry capacity can be rapidly increased to satisfy demand, but it may also lead to overcapacity and lower profits.

  \item A is correct. An industry that has high barriers to entry generally requires substantial physical capital and/or financial investment. With weak pricing power in the industry, finding a buyer for excess capacity (i.e., to exit the industry) may be difficult.

  \item C is correct. Economic profit is earned and value is created for shareholders when the industry earns returns above the company's cost of capital.

  \item B is correct. The alcoholic beverage industry is concentrated and possesses strong pricing power.

  \item B is correct. Vision typically deteriorates at advanced ages. An increased number of older adults implies more eyewear products will be purchased.

  \item B is correct. As their educational level increases, workers are able to perform more skilled tasks, earn higher wages, and as a result, have more income left for discretionary expenditures.

  \item A is correct. Seeking economies of scale would tend to reduce per-unit costs and increase profit.

  \item A is correct. Companies with low-cost strategies must be able to invest in productivity-improving equipment and finance that investment at a low cost of capital. Market share and pricing depend on whether the strategy is pursued defensively or offensively.

  \item A is correct. The cost structure is an appropriate element when analyzing the supply of the product, but analysis of demand relies on the product's differentiating characteristics and the customers' needs and wants. 31. $C$ is correct. The corporate profile would provide an understanding of these elements.

\end{enumerate}

\section{LEARNING MODULE}
6

\section{Equity Valuation: Concepts and Basic Tools}
by John J. Nagorniak, CFA, and Stephen E. Wilcox, PhD, CFA.

John J. Nagorniak, CFA (USA). Stephen E. Wilcox, PhD, CFA, is at Minnesota State University, Mankato (USA).

\section{LEARNING OUTCOME}
\begin{center}
\begin{tabular}{|c|c|}
\hline
Mastery & The candidate should be able to: \\
\hline
 & $\begin{array}{l}\text { evaluate whether a security, given its current market price and a } \\ \text { value estimate, is overvalued, fairly valued, or undervalued by the } \\ \text { market }\end{array}$ \\
\hline
 & describe major categories of equity valuation models \\
\hline
 & $\begin{array}{l}\text { describe regular cash dividends, extra dividends, stock dividends, } \\ \text { stock splits, reverse stock splits, and share repurchases }\end{array}$ \\
\hline
 & describe dividend payment chronology \\
\hline
 & $\begin{array}{l}\text { explain the rationale for using present value models to value equity } \\ \text { and describe the dividend discount and free-cash-flow-to-equity } \\ \text { models }\end{array}$ \\
\hline
 & $\begin{array}{l}\text { explain advantages and disadvantages of each category of valuation } \\ \text { model }\end{array}$ \\
\hline
 & $\begin{array}{l}\text { calculate the intrinsic value of a non-callable, non-convertible } \\ \text { preferred stock }\end{array}$ \\
\hline
 & $\begin{array}{l}\text { calculate and interpret the intrinsic value of an equity security based } \\ \text { on the Gordon (constant) growth dividend discount model or a } \\ \text { two-stage dividend discount model, as appropriate }\end{array}$ \\
\hline
 & $\begin{array}{l}\text { identify characteristics of companies for which the constant growth } \\ \text { or a multistage dividend discount model is appropriate }\end{array}$ \\
\hline
 & $\begin{array}{l}\text { explain the rationale for using price multiples to value equity, how } \\ \text { the price to earnings multiple relates to fundamentals, and the use of } \\ \text { multiples based on comparables }\end{array}$ \\
\hline
 & $\begin{array}{l}\text { calculate and interpret the following multiples: price to earnings, } \\ \text { price to an estimate of operating cash flow, price to sales, and price } \\ \text { to book value }\end{array}$ \\
\hline
 & $\begin{array}{l}\text { describe enterprise value multiples and their use in estimating equity } \\ \text { value }\end{array}$ \\
\hline
 & $\begin{array}{l}\text { describe asset-based valuation models and their use in estimating } \\ \text { equity value }\end{array}$ \\
\hline
\end{tabular}
\end{center}

\section{INTRODUCTION}
Analysts gather and process information to make investment decisions, including buy and sell recommendations. What information is gathered and how it is processed depend on the analyst and the purpose of the analysis. Technical analysis uses such information as stock price and trading volume as the basis for investment decisions. Fundamental analysis uses information about the economy, industry, and company as the basis for investment decisions. Examples of fundamentals are unemployment rates, gross domestic product (GDP) growth, industry growth, and quality of and growth in company earnings. Whereas technical analysts use information to predict price movements and base investment decisions on the direction of predicted change in prices, fundamental analysts use information to estimate the value of a security and to compare the estimated value to the market price and then base investment decisions on that comparison.

This reading introduces equity valuation models used to estimate the intrinsic value (synonym: fundamental value) of a security; intrinsic value is based on an analysis of investment fundamentals and characteristics. The fundamentals to be considered depend on the analyst's approach to valuation. In a top-down approach, an analyst examines the economic environment, identifies sectors that are expected to prosper in that environment, and analyzes securities of companies from previously identified attractive sectors. In a bottom-up approach, an analyst typically follows an industry or industries and forecasts fundamentals for the companies in those industries in order to determine valuation. Whatever the approach, an analyst who estimates the intrinsic value of an equity security is implicitly questioning the accuracy of the market price as an estimate of value. Valuation is particularly important in active equity portfolio management, which aims to improve on the return-risk trade-off of a portfolio's benchmark by identifying mispriced securities.

This reading is organized as follows. Section 2 discusses the implications of differences between estimated value and market price. Section 3 introduces three major categories of valuation model. Section 4 presents an overview of present value models with a focus on the dividend discount model. Section 5 describes and examines the use of multiples in valuation. Section 6 explains asset-based valuation and demonstrates how these models can be used to estimate value. Section 7 states conclusions and summarizes the reading.

\section{ESTIMATED VALUE AND MARKET PRICE}
evaluate whether a security, given its current market price and a value estimate, is overvalued, fairly valued, or undervalued by the market

By comparing estimates of value and market price, an analyst can arrive at one of three conclusions: The security is undervalued, overvalued, or fairly valued in the marketplace. For example, if the market price of an asset is $\$ 10$ and the analyst estimates intrinsic value at $\$ 10$, a logical conclusion is that the security is fairly valued. If the security is selling for $\$ 20$, the security would be considered overvalued. If the security is selling for $\$ 5$, the security would be considered undervalued. Basically, by estimating value, the analyst is assuming that the market price is not necessarily the best estimate of intrinsic value. If the estimated value exceeds the market price, the analyst infers the security is undervalued. If the estimated value equals the market price, the analyst infers the security is fairly valued. If the estimated value is less than the market price, the analyst infers the security is overvalued.

In practice, the conclusion is not so straightforward. Analysts must cope with uncertainties related to model appropriateness and the correct value of inputs. An analyst's final conclusion depends not only on the comparison of the estimated value and the market price but also on the analyst's confidence in the estimated value (i.e., in the model selected and the inputs used in it). One can envision a spectrum running from relatively high confidence in the valuation model and the inputs to relatively low confidence in the valuation model and/or the inputs. When confidence is relatively low, the analyst might demand a substantial divergence between his or her own value estimate and the market price before acting on an apparent mispricing. For instance, if the estimate of intrinsic value is $\$ 10$ and the market price is $\$ 10.05$, the analyst might reasonably conclude that the security is fairly valued and that the $1 / 2$ of 1 percent market price difference from the estimated value is within the analyst's confidence interval.

Confidence in the convergence of the market price to the intrinsic value over the investment time horizon relevant to the objectives of the portfolio must also be taken into account before an analyst acts on an apparent mispricing or makes a buy, sell, or hold recommendation: The ability to benefit from identifying a mispriced security depends on the market price converging to the estimated intrinsic value.

In seeking to identify mispricing and attractive investments, analysts are treating market prices with skepticism, but they are also treating market prices with respect. For example, an analyst who finds that many securities examined appear to be overvalued will typically recheck models and inputs before acting on a conclusion of overvaluation. Analysts also often recognize and factor into recommendations that different market segments-such as securities closely followed by analysts versus securities relatively neglected by analysts-may differ in how common or persistent mispricing is. Mispricing may be more likely in securities neglected by analysts.

\section{EXAMPLE 1}
\section{Valuation and Analyst Response}
\begin{enumerate}
  \item An analyst finds that all the securities analyzed have estimated values higher than their market prices. The securities all appear to be:
A. overvalued.
B. undervalued.
C. fairly valued.
\end{enumerate}

\section{Solution:}
B is correct. The estimated intrinsic value for each security is greater than the market price. The securities all appear to be undervalued in the market. Note, however, that the analyst may wish to reexamine the model and inputs to check that the conclusion is valid.

\begin{enumerate}
  \setcounter{enumi}{1}
  \item An analyst finds that nearly all companies in a market segment have common shares which are trading at market prices above the analyst's estimate of the shares' values. This market segment is widely followed by analysts. Which of the following statements describes the analyst's most appropriate first action?
\end{enumerate}

A. Issue a sell recommendation for each share issue. B. Issue a buy recommendation for each share issue.

C. Reexamine the models and inputs used for the valuations.

Solution:

$\mathrm{C}$ is correct. It seems improbable that all the share issues analyzed are overvalued, as indicated by market prices in excess of estimated valueparticularly because the market segment is widely followed by analysts. Thus, the analyst will not issue a sell recommendation for each issue. The analyst will most appropriately reexamine the models and inputs prior to issuing any recommendations. A buy recommendation is not an appropriate response to an overvalued security.

\begin{enumerate}
  \setcounter{enumi}{2}
  \item An analyst, using a number of models and a range of inputs, estimates a security’s value to be between $¥ 250$ and $¥ 270$. The security is trading at $¥ 265$. The security appears to be:
A. overvalued.
B. undervalued.
C. fairly valued.
\end{enumerate}

Solution:

$\mathrm{C}$ is correct. The security’s market price of $¥ 265$ is within the range estimated by the analyst. The security appears to be fairly valued.

Analysts often use a variety of models and inputs to achieve greater confidence in their estimates of intrinsic value. The use of more than one model and a range of inputs also helps the analyst understand the sensitivity of value estimates to different models and inputs.

\section{CATEGORIES OF EQUITY VALUATION MODELS}
describe major categories of equity valuation models

Three major categories of equity valuation models are as follows:

\begin{itemize}
  \item Present value models (synonym: discounted cash flow models). These models estimate the intrinsic value of a security as the present value of the future benefits expected to be received from the security. In present value models, benefits are often defined in terms of cash expected to be distributed to shareholders (dividend discount models) or in terms of cash flows available to be distributed to shareholders after meeting capital expenditure and working capital needs (free-cash-flow-to-equity models). Many models fall within this category, ranging from the relatively simple to the very complex. In Section 4, we discuss in detail two of the simpler models, the Gordon (constant) growth model and the two-stage dividend discount models.

  \item Multiplier models (synonym: market multiple models). These models are based chiefly on share price multiples or enterprise value multiples. The former model estimates intrinsic value of a common share from a price multiple for some fundamental variable, such as revenues, earnings, cash flows, or book value. Examples of the multiples include price to earnings ( $\mathrm{P} / \mathrm{E}$, share price divided by earnings per share) and price to sales (P/S, share price divided by sales per share). The fundamental variable may be stated on a forward basis (e.g., forecasted EPS for the next year) or a trailing basis (e.g., EPS for the past year), as long as the usage is consistent across companies being examined. Price multiples are also used to compare relative values. The use of the ratio of share price to $\mathrm{EPS}$ - that is, the $\mathrm{P} / \mathrm{E}$ multiple-to judge relative value is an example of this approach to equity valuation.

\end{itemize}

Enterprise value (EV) multiples have the form (Enterprise value)/(Value of a fundamental variable). Two possible choices for the denominator are earnings before interest, taxes, depreciation, and amortization (EBITDA) and total revenue. Enterprise value, the numerator, is a measure of a company's total market value from which cash and short-term investments have been subtracted (because an acquirer could use those assets to pay for acquiring the company). An estimate of common share value can be calculated indirectly from the EV multiple; the value of liabilities and preferred shares can be subtracted from the EV to arrive at the value of common equity.

\begin{itemize}
  \item Asset-based valuation models. These models estimate intrinsic value of a common share from the estimated value of the assets of a corporation minus the estimated value of its liabilities and preferred shares. The estimated market value of the assets is often determined by making adjustments to the book value (synonym: carrying value) of assets and liabilities. The theory underlying the asset-based approach is that the value of a business is equal to the sum of the value of the business's assets.
\end{itemize}

As already mentioned, many analysts use more than one type of model to estimate value. Analysts recognize that each model is a simplification of the real world and that there are uncertainties related to model appropriateness and the inputs to the models. The choice of model(s) will depend on the availability of information to input into the model(s) and the analyst's confidence in the information and in the appropriateness of the model(s).

\section{EXAMPLE 2}
\section{Categories of Equity Valuation Models}
\begin{enumerate}
  \item An analyst is estimating the intrinsic value of a new company. The analyst has one year of financial statements for the company and has calculated the average values of a variety of price multiples for the industry in which the company operates. The analyst plans to use at least one model from each of the three categories of valuation models. The analyst is least likely to rely on the estimate(s) from the:
\end{enumerate}

A. multiplier model(s).

B. present value model(s).

C. asset-based valuation $\operatorname{model}(\mathrm{s})$.

\section{Solution:}
$B$ is correct. Because the company has only one year of data available, the analyst is least likely to be confident in the inputs for a present value model. The values on the balance sheet, even before adjustment, are likely to be close to market values because the assets are all relatively new. The multiplier models are based on average multiples from the industry. 2. Based on a company's EPS of $€ 1.35$, an analyst estimates the intrinsic value of a security to be $€ 16.60$. Which type of model is the analyst most likely to be using to estimate intrinsic value?

A. Multiplier model.

B. Present value model.

C. Asset-based valuation model.

\section{Solution:}
A is correct. The analyst is using a multiplier model based on the P/E multiple. The P/E multiple used was $16.60 / 1.35=12.3$.

As you begin the study of specific equity valuation models in the next section, you must bear in mind that any model of value is, by necessity, a simplification of the real world. Never forget this simple fact! You may encounter models much more complicated than the ones discussed here, but even those models will be simplifications of reality.

\section{BACKGROUND FOR THE DIVIDEND DISCOUNT MODEL}
describe regular cash dividends, extra dividends, stock dividends, stock splits, reverse stock splits, and share repurchases describe dividend payment chronology

Present value models follow a fundamental tenet of economics which states that individuals defer consumption-that is, they invest-for the future benefits expected. Individuals and companies make an investment because they expect thereby to earn a rate of return over the investment period. Logically, the value of an investment should be equal to the present value of the expected future benefits. For common shares, an analyst can equate benefits to the cash flows to be generated by the investment. The simplest present value model of equity valuation is the dividend discount model (DDM), which specifies cash flows from a common stock investment to be dividends.

The next section describes aspects of dividends that users of dividend discount models should understand.

\section{Dividends: Background for the Dividend Discount Model}
Generally, there are two sources of return from investing in equities: (1) cash dividends received by an investor over his or her holding period and (2) the change in the market price of equities over that holding period.

A dividend is a distribution paid to shareholders based on the number of shares owned, and a cash dividend is a cash distribution made to a company's shareholders. Cash dividends are typically paid out regularly at known intervals; such dividends are known as regular cash dividends. By contrast, an extra dividend or special dividend is a dividend paid by a company that does not pay dividends on a regular schedule or a dividend that supplements regular cash dividends with an extra payment. Companies in cyclical industries and companies undergoing corporate and/or financial restructuring are among those observed to use extra dividends. ${ }^{1}$

The payment of dividends is not a legal obligation: dividends must be declared (i.e., authorized) by a company's board of directors; in some jurisdictions, they must also be approved by shareholders. Regular cash dividends are customarily declared and paid out quarterly in the United States and Canada; semiannually in Europe and Japan; and annually in some other countries, including China.

Dividend discount models address discounting expected cash dividends. A stock dividend (also known as a bonus issue of shares) is a type of dividend in which a company distributes additional shares of its common stock (typically, 2\%-10\% of the shares then outstanding) to shareholders instead of cash. A stock dividend divides the "pie" (the market value of shareholders' equity) into smaller pieces without affecting the value of the pie or any shareholder's proportional ownership in the company. Thus, stock dividends are not relevant for valuation. Stock splits and reverse stock splits are similar to stock dividends in that they have no economic effect on the company or shareholders. A stock split involves an increase in the number of shares outstanding with a consequent decrease in share price. An example of a stock split is a two-for-one stock split in which each shareholder is issued an additional share for each share currently owned. A reverse stock split involves a reduction in the number of shares outstanding with a corresponding increase in share price. In a one-for-two reverse stock split, each shareholder would receive one new share for every two old shares held, thereby reducing the number of shares outstanding by half.

In contrast to stock dividends and stock splits, share repurchases are an alternative to cash dividend payments. A share repurchase (or buyback) is a transaction in which a company uses cash to buy back its own shares. Shares that have been repurchased are not considered for dividends, voting, or computing earnings per share. A share repurchase is viewed as equivalent to the payment of cash dividends of equal value in terms of the effect on shareholders' wealth, all other things being equal. Company managements have expressed several key reasons for engaging in share repurchasesnamely, (1) signaling a belief that their shares are undervalued (or, more generally, to support share prices), (2) flexibility in the amount and timing of distributing cash to shareholders, (3) tax efficiency in markets where tax rates on dividends exceed tax rates on capital gains, and (4) the ability to absorb increases in outstanding shares because of the exercise of employee stock options.

The payout of regular cash dividends to common shareholders follows a fairly standard chronology that is set in motion once the company's board of directors votes to pay the dividend. First is the declaration date, the day that the company issues a statement declaring a specific dividend. Next comes the ex-dividend date (or ex-date), the first date that a share trades without (i.e., "ex") the dividend. This is followed closely (one or two business days later) by the holder-of-record date (also called the owner-of-record date, shareholder-of-record date, record date, date of record, or date of book closure), the date that a shareholder listed on the company's books will be deemed to have ownership of the shares for purposes of receiving the upcoming dividend; the amount of time between the ex-date and the holder-of-record date is linked to the trade settlement cycle in force. The final milestone is the payment date (or payable date), which is the day that the company actually mails out (or electronically transfers) a dividend payment to shareholders.

1 Another type of dividend is a liquidating dividend, which is a return of capital rather than a distribution from earnings or retained earnings. Liquidating dividends are used when a company goes out of business and distributes its net assets, sells a portion of its business for cash and distributes the sale's proceeds, or pays a dividend that exceeds its accumulated retained earnings.

\section{EXAMPLE 3}
\section{Total S.A. Dividend Payment Time Line}
On 26 May 2017, Total S.A., one of the world's largest integrated energy companies, declared an annual dividend of $€ 2.48$ per share, payable on a quarterly basis. The first quarterly dividend of $€ 2.48 / 4=€ 0.62$ was payable on 12 October 2017. The holder-of-record date was 26 September, and the ex-dividend date was 25 September. A timeline for the upcoming Total S.A. quarterly dividend is shown in Exhibit 1.

\section{Exhibit 1: Timeline for Total S.A. Quarterly Dividend}
\begin{center}
\includegraphics[max width=\textwidth]{2023_05_04_7b535d0a870224f62e3dg-396}
\end{center}

Source: Total S.A. website: \href{http://www.total.com}{www.total.com}.

Because buyers of a company's shares on the ex-dividend date are no longer eligible to receive the upcoming dividend, all else being equal, on that day the company's share price immediately decreases by the amount of the foregone dividend. Exhibit 2 illustrates the decrease in share price that occurs for a hypothetical company that has declared a $\$ 1.00$ per share dividend as trading begins on its ex-dividend date.

Exhibit 2: Stock Price Change for Hypothetical Company on Ex-Dividend Date

\begin{center}
\includegraphics[max width=\textwidth]{2023_05_04_7b535d0a870224f62e3dg-396(1)}
\end{center}

Note: Assumes dividend declared is $\$ 1$ per share and convention for stock trade settlement is $T+3$.

\textbackslash section\{DIVIDEND DISCOUNT MODEL (DDM) AND FREE-CASH-FLOW-TO-EQUITY MODEL (FCFE)

\begin{center}
\includegraphics[max width=\textwidth]{2023_05_04_7b535d0a870224f62e3dg-397}
\end{center}

If the issuing company is assumed to be a going concern, the intrinsic value of a share is the present value of expected future dividends. If a constant required rate of return is also assumed, then the DDM expression for the intrinsic value of a share is Equation 1:

$$
V_{0}=\sum_{t=1}^{\infty} \frac{D_{t}}{(1+r)^{t}}
$$

where

$$
\begin{aligned}
V_{0} & =\text { value of a share of stock today, at } t=0 \\
D_{t} & =\text { expected dividend in year } t, \text { assumed to be paid at the end of the year } \\
r & =\text { required rate of return on the stock }
\end{aligned}
$$

At the shareholder level, cash received from a common stock investment includes any dividends received and the proceeds when shares are sold. If an investor intends to buy and hold a share for one year, the value of the share today is the present value of two cash flows-namely, the expected dividend plus the expected selling price in one year:

$$
V_{0}=\frac{D_{1}+P_{1}}{(1+r)^{1}}=\frac{D_{1}}{(1+r)^{1}}+\frac{P_{1}}{(1+r)^{1}}
$$

where $P_{1}=$ the expected price per share at $t=1$.

To estimate the expected selling price, $P_{1}$, the analyst could estimate the price another investor with a one-year holding period would pay for the share in one year. If $V_{0}$ is based on $D_{1}$ and $P_{1}$, it follows that $P_{1}$ could be estimated from $D_{2}$ and $P_{2}$ :

$$
P_{1}=\frac{D_{2}+P_{2}}{(1+r)}
$$

Substituting the right side of this equation for $P_{1}$ in Equation 2 results in $V_{0}$ estimated as

$$
V_{0}=\frac{D_{1}}{(1+r)}+\frac{D_{2}+P_{2}}{(1+r)^{2}}=\frac{D_{1}}{(1+r)}+\frac{D_{2}}{(1+r)^{2}}+\frac{P_{2}}{(1+r)^{2}}
$$

Repeating this process, we find the value for $n$ holding periods is the present value of the expected dividends for the $n$ periods plus the present value of the expected price in $n$ periods:

$$
V_{0}=\frac{D_{1}}{(1+r)^{1}}+\cdots+\frac{D_{n}}{(1+r)^{n}}+\frac{P_{n}}{(1+r)^{n}}
$$

Using summation notation to represent the present value of the $n$ expected dividends, we arrive at the general expression for an $n$-period holding period or investment horizon:

$$
V_{0}=\sum_{t=1}^{n} \frac{D_{t}}{(1+r)^{t}}+\frac{P_{n}}{(1+r)^{n}}
$$

The expected value of a share at the end of the investment horizon-in effect, the expected selling price-is often referred to as the terminal stock value (or terminal value).

\section{EXAMPLE 4}
\section{Estimating Share Value for a Three-Year Investment Horizon}
\begin{enumerate}
  \item For the next three years, the annual dividends of a stock are expected to be $€ 2.00, € 2.10$, and $€ 2.20$. The stock price is expected to be $€ 20.00$ at the end of three years. If the required rate of return on the shares is 10 percent, what is the estimated value of a share?
\end{enumerate}

\section{Solution:}
The present values of the expected future cash flows can be written as follows:

$$
V_{0}=\frac{2.00}{(1.10)^{1}}+\frac{2.10}{(1.10)^{2}}+\frac{2.20}{(1.10)^{3}}+\frac{20.00}{(1.10)^{3}}
$$

Calculating and summing these present values gives an estimated share value of $V_{0}=1.818+1.736+1.653+15.026=€ 20.23$.

The three dividends have a total present value of $€ 5.207$, and the terminal stock value has a present value of $€ 15.026$, for a total estimated value of $€ 20.23$.

Extending the holding period into the indefinite future, we can say that a stock's estimated value is the present value of all expected future dividends as shown in Equation 1.

Consideration of an indefinite future is valid because businesses established as corporations are generally set up to operate indefinitely. This general form of the DDM applies even in the case in which the investor has a finite investment horizon. For that investor, stock value today depends directly on the dividends the investor expects to receive before the stock is sold and depends indirectly on the expected dividends for periods subsequent to that sale, because those expected future dividends determine the expected selling price. Thus, the general expression given by Equation 1 holds irrespective of the investor's holding period.

In practice, many analysts prefer to use a free-cash-flow-to-equity (FCFE) valuation model. These analysts assume that dividend-paying capacity should be reflected in the cash flow estimates rather than expected dividends. FCFE is a measure of dividend-paying capacity. Analysts may also use FCFE valuation models for a non-dividend-paying stock. To use a DDM, the analyst needs to predict the timing and amount of the first dividend and all the dividends or dividend growth thereafter. Making these predictions for non-dividend-paying stock accurately is typically difficult, so in such cases, analysts often resort to FCFE models.

The calculation of FCFE starts with the calculation of cash flow from operations (CFO). CFO is simply defined as net income plus non-cash expenses minus investment in working capital. FCFE is a measure of cash flow generated in a period that is available for distribution to common shareholders. What does "available for distribution" mean? The entire CFO is not available for distribution; the portion of the CFO needed for fixed capital investment (FCInv) during the period to maintain the value of the company as a going concern is not viewed as available for distribution to common shareholders. Net amounts borrowed (borrowings minus repayments) are considered to be available for distribution to common shareholders. Thus, FCFE can be expressed as

$$
\mathrm{FCFE}=\mathrm{CFO}-\mathrm{FCInv}+\text { Net borrowing }
$$

The information needed to calculate historical FCFE is available from a company's statement of cash flows and financial disclosures. Frequently, under the assumption that management is acting in the interest of maintaining the value of the company as a going concern, reported capital expenditure is taken to represent FCInv. Analysts must make projections of financials to forecast future FCFE. Valuation obtained by using FCFE involves discounting expected future FCFE by the required rate of return on equity; the expression parallels Equation 1:

$$
V_{0}=\sum_{t=1}^{\infty} \frac{\mathrm{FCFE}_{t}}{(1+r)^{t}}
$$

\section{EXAMPLE 5}
\section{Present Value Models}
\begin{enumerate}
  \item An investor expects a share to pay dividends of $\$ 3.00$ and $\$ 3.15$ at the end of Years 1 and 2, respectively. At the end of the second year, the investor expects the shares to trade at $\$ 40.00$. The required rate of return on the shares is 8 percent. If the investor's forecasts are accurate and the market price of the shares is currently $\$ 30$, the most likely conclusion is that the shares are:
A. overvalued.
B. undervalued.
C. fairly valued.
\end{enumerate}

\section{Solution:}
$\mathrm{B}$ is correct.

$$
V_{0}=\frac{3.00}{(1.08)^{1}}+\frac{3.15}{(1.08)^{2}}+\frac{40.00}{(1.08)^{2}}=39.77
$$

The value estimate of $\$ 39.77$ exceeds the market price of $\$ 30$, so the conclusion is that the shares are undervalued.

\begin{enumerate}
  \setcounter{enumi}{1}
  \item Two investors with different holding periods but the same expectations and required rate of return for a company are estimating the intrinsic value of a common share of the company. The investor with the shorter holding period will most likely estimate a:
\end{enumerate}

A. lower intrinsic value.

B. higher intrinsic value.

C. similar intrinsic value.

\section{Solution:}
$\mathrm{C}$ is correct. The intrinsic value of a security is independent of the investor's holding period.

\begin{enumerate}
  \setcounter{enumi}{2}
  \item An equity valuation model that focuses on expected dividends rather than the capacity to pay dividends is the:
\end{enumerate}

A. dividend discount model. B. free cash flow to equity model.

C. cash flow return on investment model.

Solution:

A is correct. Dividend discount models focus on expected dividends.

How is the required rate of return for use in present value models estimated? To estimate the required rate of return on a share, analysts frequently use the capital asset pricing model (CAPM):

Required rate of return on share $i=$ Current expected risk

\begin{itemize}
  \item free rate of return
\end{itemize}

$+\operatorname{Beta}_{i}[$ Market (equity) risk premium]

Equation 5 states that the required rate of return on a share is the sum of the current expected risk-free rate plus a risk premium that equals the product of the stock's beta (a measure of non-diversifiable risk) and the market risk premium (the expected return of the market in excess of the risk-free return, where in practice, the "market" is often represented by a broad stock market index). However, even if analysts agree that the CAPM is an appropriate model, their inputs into the CAPM may differ. Thus, there is no uniquely correct answer to the question: What is the required rate of return?

Other common methods for estimating the required rate of return for the stock of a company include adding a risk premium that is based on economic judgments, rather than the CAPM, to an appropriate risk-free rate (usually a government bond) and adding a risk premium to the yield on the company's bonds. Good business and economic judgment is paramount in estimating the required rate of return. In many investment firms, required rates of return are determined by firm policy.

\section{PREFERRED STOCK VALUATION}
\begin{center}
\includegraphics[max width=\textwidth]{2023_05_04_7b535d0a870224f62e3dg-400}
\end{center}

General dividend discount models are relatively easy to apply to preferred shares. In its simplest form, preferred stock is a form of equity (generally, non-voting) that has priority over common stock in the receipt of dividends and on the issuer's assets in the event of a company's liquidation. It may have a stated maturity date at which time payment of the stock's par (face) value is made or it may be perpetual with no maturity date; additionally, it may be callable or convertible.

For a non-callable, non-convertible perpetual preferred share paying a level dividend $D$ and assuming a constant required rate of return over time, Equation 1 reduces to the formula for the present value of a perpetuity. Its value is:

$$
V_{0}=\frac{D_{0}}{r}
$$

For example, a $\$ 100$ par value non-callable perpetual preferred stock offers an annual dividend of $\$ 5.50$. If its required rate of return is 6 percent, the value estimate would be $\$ 5.50 / 0.06=\$ 91.67$. For a non-callable, non-convertible preferred stock with maturity at time $n$, the estimated intrinsic value can be estimated by using Equation 3 but using the preferred stock's par value, $F$, instead of $P_{n}$ :

$$
V_{0}=\sum_{t=1}^{n} \frac{D_{t}}{(1+r)^{t}}+\frac{F}{(1+r)^{n}}
$$

When Equation 7 is used, the most precise approach is to use values for $n, r$, and $D$ that reflect the payment schedule of the dividends. This method is similar to the practice of fixed-income analysts in valuing a bond. For example, a non-convertible preferred stock with a par value of $\pounds 20.00$, maturity in six years, a nominal required rate of return of 8.20 percent, and semiannual dividends of $\pounds 2.00$ would be valued by using an $n$ of 12 , an $r$ of 4.10 percent, a $D$ of $\pounds 2.00$, and an $F$ of $\pounds 20.00$. The result would be an estimated value of $\pounds 31.01$. Assuming payments are annual rather than semiannual (i.e., assuming that $n=6, r=8.20$ percent, and $D=\pounds 4.00$ ) would result in an estimated value of $\pounds 30.84$.

Preferred stock issues are frequently callable (redeemable) by the issuer at some point prior to maturity, often at par value or at prices in excess of par value that decline to par value as the maturity date approaches. Such call options tend to reduce the value of a preferred issue to an investor because the option to redeem will be exercised by the issuer when it is in the issuer's favor and ignored when it is not. For example, if an issuer can redeem shares at par value that would otherwise trade (on the basis of dividends, maturity, and required rate of return) above par value, the issuer has motivation to redeem the shares.

Preferred stock issues can also include a retraction option that enables the holder of the preferred stock to sell the shares back to the issuer prior to maturity on prespecified terms. Essentially, the holder of the shares has a put option. Such put options tend to increase the value of a preferred issue to an investor because the option to retract will be exercised by the investor when it is in the investor's favor and ignored when it is not. Although the precise valuation of issues with such embedded options is beyond the scope of this reading, Example 6 includes a case in which Equation 7 can be used to approximate the value of a callable, retractable preferred share.

\section{EXAMPLE 6}
\section{Preferred Share Valuation: Two Cases}
Case 1: Non-callable, Non-convertible, Perpetual Preferred Shares

The following facts concerning the Union Electric Company 4.75 percent perpetual preferred shares are as follows:

\begin{itemize}
  \item Issuer: Union Electric Co. (owned by Ameren)

  \item Par value: US $\$ 100$

  \item Dividend: US $\$ 4.75$ per year

  \item Maturity: perpetual

  \item Embedded options: none

  \item Credit rating: Moody's Investors Service/Standard \& Poor's Ba1/BB

  \item Required rate of return on Ba1/BB rated preferred shares as of valuation date: 7.5 percent. 1. Estimate the intrinsic value of this preferred share.

\end{itemize}

\section{Solution}
Basing the discount rate on the required rate of return on Ba1/BB rated preferred shares of 7.5 percent gives an intrinsic value estimate of US $\$ 4.75 / 0.075=$ US $\$ 63.33$.

\begin{enumerate}
  \setcounter{enumi}{1}
  \item Explain whether the intrinsic value of this issue would be higher or lower if the issue were callable (with all other facts remaining unchanged).
\end{enumerate}

\section{Solution}
The intrinsic value would be lower if the issue were callable. The option to redeem or call the issue is valuable to the issuer because the call will be exercised when doing so is in the issuer's interest. The intrinsic value of the shares to the investor will typically be lower if the issue is callable. In this case, because the intrinsic value without the call is much less than the par value, the issuer would be unlikely to redeem the issue if it were callable; thus, callability would reduce intrinsic value, but only slightly.

\section{Case 2: Retractable Term Preferred Shares}
Retractable term preferred shares are a type of preferred share that has been previously issued by Canadian companies, and have now began to be offered by companies in other jurisdictions, including Japan. This type of issue specifies a "retraction date" when the preferred shareholders have the option to sell back their shares to the issuer at par value (i.e., the shares are "retractable" or "putable" at that date). ${ }^{2}$ At predetermined dates prior to the retraction date, the issuer has the option to redeem the preferred issue at predetermined prices (which are always at or above par value).

An example of a retractable term preferred share currently outstanding is TMC (Toyota Motor Corporation), First Series Model AA class shares, with a 0.5 percent dividend rate, increasing by 0.5 percent every year until 2020 and thereafter, becoming fixed at 2.5 percent. TMC is leading global automobile manufacturer, with headquarters in Japan and global operations. The issue is in Japanese Yen. The shares have a $¥ 10,598$ par value and pay a semiannual dividend of $¥ 26.5$ [ $=(0.5$ percent $x ¥ 10,598) / 2$ ] on 31 March 2016 . The semiannual dividend is expected to increase to $¥ 132.5$ [= ( 2.5 percent $x ¥ 10,598) / 2$ ] on 31 March 2020 and beyond. As of 31 December 2017 the company carried ratings from Moody's and Standard \& Poor's of Aa3 and AA-, respectively. Thus, the shares are viewed by Moody's and Standard \& Poor's as having "adequate" credit quality, qualified by "Aa3 and AA-," which means relatively high quality within that group. Beginning from 2 April 2021, the shares are redeemable at the option of TMC at $¥ 10,598$ (par value). The retraction date is the last day of March, June, September, and December of each year, starting from 1 September 2020, with the shares retractable at par value. The Series AA shares have voting rights and may exercise their voting rights and other rights held by holders of common shares of TMC in the same manner. The Series AA shares were issued at a $20 \%$ premium to the common shares price in 2015, and since then the share price has decreased to $¥ 7,243$ as at 31 December 2017, and with a current required rate of 3.05 per year (1.525 percent semiannual). Because the issue's market price is so far below the prices at which TMC could redeem or call the issue, redemption is considered to be unlikely by TMC, whereas the retraction option for the Series

2 "Retraction" refers to this option, which is a put option. The terminology is not completely settled: The type of share being called "retractable term preferred" is also known as "hard retractable preferred," with "hard" referring to payment in cash rather than common shares at the retraction date. AA holders appears to have a significant value since they will potentially be able to put back the shares to TMC at approximately 45 percent over the current market value ( $¥ 10,598$ compared to $¥ 7,243$ ).

\begin{enumerate}
  \item Assume that the issue will be retracted in December 2020; the holders of the shares will put the shares to the company in December 2020. Based on the information given, estimate the intrinsic value of a share. Assume it is December 2017.
\end{enumerate}

\section{Solution}
An intrinsic value estimate of a share of this preferred issue is $¥ 10,279$.

Expected semiannual dividends:

Year ended March 31, 2018: $¥ 79.5$ [= $(1.5$ percent $\times ¥ 10,598) / 2]$

Year ended March 31, 2019: $¥ 106$ [= (2.0 percent $\times ¥ 10,598) / 2]$

Year ended March 31, 2020: $¥ 132.5$ [= $(2.5$ percent $\times ¥ 10,598) / 2]$

$$
V_{0}=\left[\frac{¥ 79.5}{1.01525}+\frac{¥ 79.5}{1.01525^{2}}+\frac{¥ 106}{1.01525^{3}}+\frac{¥ 106}{1.01525^{4}}+\frac{¥ 132.5}{1.01525^{5}}+\frac{¥ 132.5}{1.01525^{6}}+\frac{¥ 10,598}{1.01525^{6.5}}\right]
$$\$\$

\textbackslash approx ¥ 10,205

\$\$

The difference between the current market price of $¥ 7,243$ and the intrinsic value of $¥ 10,205$ is the implied value of retractable option given to the holders of the Series AA shares.

\section{THE GORDON GROWTH MODEL}
$\square \quad \mid \begin{aligned} & \text { calculate and interpret the intrinsic value of an equity security based } \\ & \text { on the Gordon (constant) growth dividend discount model or a } \\ & \text { two-stage dividend discount model, as appropriate } \\ & \text { identify characteristics of companies for which the constant growth } \\ & \text { or a multistage dividend discount model is appropriate } \\ & \text { explain advantages and disadvantages of each category of valuation } \\ & \text { model }\end{aligned}$

A rather obvious problem when one is trying to implement Equation 1 for common equity is that it requires the analyst to estimate an infinite series of expected dividends. To simplify this process, analysts frequently make assumptions about how dividends will grow or change over time. The Gordon (constant) growth model (Gordon, 1962) is a simple and well-recognized DDM. The model assumes dividends grow indefinitely at a constant rate.

Because of its assumption of a constant growth rate, the Gordon growth model is particularly appropriate for valuing the equity of dividend-paying companies that are relatively insensitive to the business cycle and in a mature growth phase. Examples might include an electric utility serving a slowly growing area or a producer of a staple food product (e.g., bread). A history of increasing the dividend at a stable growth rate is another practical criterion if the analyst believes that pattern will hold in the future.

With a constant growth assumption, Equation 1 can be written as Equation 8, where $g$ is the constant growth rate:

$$
V_{0}=\sum_{t=1}^{\infty} \frac{D_{0}(1+g)^{t}}{(1+r)^{t}}=D_{0}\left[\frac{(1+g)}{(1+r)}+\frac{(1+g)^{2}}{(1+r)^{2}}+\ldots+\frac{(1+g)^{\infty}}{(1+r)^{\infty}}\right]
$$

If required return $r$ is assumed to be strictly greater than growth rate $g$, then the square-bracketed term in Equation 8 is an infinite geometric series and sums to [(1 $+g) /(r-g)]$. Substituting into Equation 8 produces the Gordon growth model as presented in Equation 9:

$$
V_{0}=\frac{D_{0}(1+g)}{r-g}=\frac{D_{1}}{r-g}
$$

For an illustration of the expression, suppose the current (most recent) annual dividend on a share is $€ 5.00$ and dividends are expected to grow at 4 percent per year. The required rate of return on equity is 8 percent. The Gordon growth model estimate of intrinsic value is, therefore, $€ 5.00(1.04) /(0.08-0.04)=€ 5.20 / 0.04=€ 130$ per share. Note that the numerator is $D_{1}$ not $D_{0}$. (Using the wrong numerator is a common error.)

The Gordon growth model estimates intrinsic value as the present value of a growing perpetuity. If the growth rate, $g$, is assumed to be zero, Equation 8 reduces to the expression for the present value of a perpetuity, given earlier as Equation 6 .

In estimating a long-term growth rate, analysts use a variety of methods, including assessing the growth in dividends or earnings over time, using the industry median growth rate, and using the relationship shown in Equation 10 to estimate the sustainable growth rate:

$$
g=b \times \mathrm{ROE}
$$

where

$$
\begin{aligned}
g & =\text { dividend growth rate } \\
b & =\text { earnings retention rate }=(1-\text { Dividend payout ratio }) \\
\text { ROE } & =\text { return on equity }
\end{aligned}
$$

Example 7 illustrates the application of the Gordon growth model to the shares of a large industrial manufacturing company. The analyst believes it will continue to grow at a rate that it achieved in the previous three years and remain stable in the future. The example asks how much the dividend growth assumption adds to the intrinsic value estimate. The question is relevant to valuation because if the amount is high on a percentage basis, a large part of the value of the share depends on the realization of the growth estimate. One can answer the question by subtracting from the intrinsic value estimate determined by Equation 9 the value determined by Equation 6, which assumes no dividend growth. ${ }^{3}$

\section{EXAMPLE 7}
\section{Applying the Gordon Growth Model}
Siemens AG operates in the capital goods and technology space. It is involved in the engineering, manufacturing, automation, power, and transportation sectors. It operates globally and is one of the largest companies in the sectors in which it operates. It is a substantial employer in both its original, domestic German market, as well as dozens of countries around the world. Selected financial information for Siemens appears in Exhibit 3.

3 A related concept, the present value of growth opportunities (PVGO), is discussed in more advanced readings.

\section{Exhibit 3: Selected Financial Information for Siemens AG}
\begin{center}
\begin{tabular}{lccccc}
\hline
Year & $\mathbf{2 0 1 7}$ & $\mathbf{2 0 1 6}$ & $\mathbf{2 0 1 5}$ & $\mathbf{2 0 1 4}$ & $\mathbf{2 0 1 3}$ \\
\hline
EPS & $€ 7.45$ & $€ 6.74$ & $€ 8.85$ & $€ 6.37$ & $€ 5.08$ \\
DPS & $€ 3.7$ & $€ 3.6$ & $€ 3.5$ & $€ 3.3$ & $€ 3.0$ \\
Payout ratio & $50 \%$ & $53 \%$ & $40 \%$ & $52 \%$ & $59 \%$ \\
ROE & $15.6 \%$ & $15.9 \%$ & $22.3 \%$ & $18.2 \%$ & $14.6 \%$ \\
Share price &  &  &  &  &  \\
(XETRA & $€ 119.2$ & $€ 104.2$ & $€ 79.94$ & $€ 94.37$ & $€ 89.06$ \\
- Frankfurt) &  &  &  &  &  \\
\end{tabular}
\end{center}

Note: DPS stands for "dividends per share."

Source: Morningstar, \href{http://www.siemens.com}{www.siemens.com}.

The analyst estimates the growth rate to be approximately 5.4 percent based on the dividend growth rate over the period 2013 to $2017\left[3(1+g)^{4}=3.7\right.$, so $g$ $=5.4 \%]$. To verify that the estimated growth rate of 5.4 percent is feasible in the future, the analyst also uses the average of Siemens's retention rate and ROE for the previous five years ( $g \approx 0.49 \times 17.3 \% \approx 8.5 \%$ ) to estimate the sustainable growth rate.

Using a number of approaches, including adding a risk premium to a long-term German government bond and using the CAPM, the analyst estimates a required return of 7.5 percent. The most recent dividend of $€ 3.70$ is used for $D_{0}$.

\begin{enumerate}
  \item Use the Gordon growth model to estimate Siemens's intrinsic value.
\end{enumerate}

\section{Solution:}
$$
V_{0}=\frac{€ 3.70(1+0.054)}{0.075-0.054}=€ 185.70
$$

\begin{enumerate}
  \setcounter{enumi}{1}
  \item How much does the dividend growth assumption add to the intrinsic value estimate?
\end{enumerate}

\section{Solution:}
$€ 184.20-\frac{€ 3.70}{0.075}=€ 136.37$

\begin{enumerate}
  \setcounter{enumi}{2}
  \item Based on the estimated intrinsic value, is a share of Siemens undervalued, overvalued, or fairly valued?
\end{enumerate}

\section{Solution:}
A share of Siemens appears to be undervalued. The analyst, before making a recommendation, might consider how realistic the estimated inputs are and check the sensitivity of the estimated value to changes in the inputs.

\begin{enumerate}
  \setcounter{enumi}{3}
  \item What is the intrinsic value if the growth rate estimate is lowered to 4.4 percent?
\end{enumerate}

\section{Solution:}
$$
V_{0}=\frac{€ 3.70(1+0.044)}{0.075-0.044}=€ 124.61
$$

\begin{enumerate}
  \setcounter{enumi}{4}
  \item What is the intrinsic value if the growth rate estimate is lowered to 4.4 percent and the required rate of return estimate is increased to 8.5 percent?
\end{enumerate}

Solution:

$V_{0}=\frac{€ 3.70(1+0.044)}{0.085-0.044}=€ 94.21$

The Gordon growth model estimate of intrinsic value is extremely sensitive to the choice of required rate of return $r$ and growth rate $g$. It is possible that the growth rate assumption and the required return assumption used initially were too high. Worldwide economic growth is typically in the low single digits, which may mean that a large company such as Siemens may struggle to grow dividends at 5.4 percent into perpetuity. Exhibit 4 presents a further sensitivity analysis of Siemens's intrinsic value to the required return and growth estimates.

Exhibit 4: Sensitivity Analysis of the Intrinsic-Value Estimate for Siemens AG

\begin{center}
\begin{tabular}{lccccc}
\hline
 & $\boldsymbol{g}=\mathbf{2 . 5 \%}$ & $\boldsymbol{g}=\mathbf{3 . 5 \%}$ & $\boldsymbol{g}=\mathbf{4 . 5 \%}$ & $\boldsymbol{g}=\mathbf{5 . 5 \%}$ & $\boldsymbol{g}=\mathbf{6 . 5 \%}$ \\
\hline
$r=6 \%$ & $€ 108.4$ & $€ 153.2$ & $€ 257.8$ & $€ 780.7$ & - \\
$r=7 \%$ & $€ 84.3$ & $€ 109.4$ & $€ 154.7$ & $€ 260.2$ & $€ 788.1$ \\
$r=8 \%$ & $€ 69.0$ & $€ 85.1$ & $€ 110.5$ & $€ 156.1$ & $€ 262.7$ \\
$r=9 \%$ & $€ 58.3$ & $€ 69.6$ & $€ 85.9$ & $€ 111.5$ & $€ 157.6$ \\
$r=10 \%$ & $€ 50.6$ & $€ 58.9$ & $€ 70.3$ & $€ 86.7$ & $€ 112.6$ \\
\hline
\end{tabular}
\end{center}

Note that no value is shown when the growth rate exceeds the required rate of return. The Gordon growth model assumes that the growth rate cannot be greater than the required rate of return.

The assumptions of the Gordon model are as follows:

\begin{itemize}
  \item Dividends are the correct metric to use for valuation purposes.

  \item The dividend growth rate is forever: It is perpetual and never changes.

  \item The required rate of return is also constant over time.

  \item The dividend growth rate is strictly less than the required rate of return.

\end{itemize}

An analyst might be dissatisfied with these assumptions for many reasons. The equities being examined might not currently pay a dividend. The Gordon assumptions might be too simplistic to reflect the characteristics of the companies being evaluated. Some alternatives to using the Gordon model are as follows:

\begin{itemize}
  \item Use a more robust DDM that allows for varying patterns of growth.

  \item Use a cash flow measure other than dividends for valuation purposes.

  \item Use some other approach (such as a multiplier method) to valuation.

\end{itemize}

Applying a DDM is difficult if the company being analyzed is not currently paying a dividend. A company may not be paying a dividend if 1 ) the investment opportunities the company has are all so attractive that the retention and reinvestment of funds is preferable, from a return perspective, to the distribution of a dividend to shareholders or 2) the company is in such shaky financial condition that it cannot afford to pay a dividend. An analyst might still use a DDM to value such companies by assuming that dividends will begin at some future point in time. The analyst might further assume that constant growth occurs after that date and use the Gordon growth model for valuation. Extrapolating from no current dividend, however, generally yields highly uncertain forecasts. Analysts typically choose to use one or more of the alternatives instead of or as a supplement to the Gordon growth model.

\section{EXAMPLE 8}
\section{Gordon Growth Model in the Case of No Current Dividend}
\begin{enumerate}
  \item A company does not currently pay a dividend but is expected to begin to do so in five years (at $t=5$ ). The first dividend is expected to be $\$ 4.00$ and to be received five years from today. That dividend is expected to grow at 6 percent into perpetuity. The required return is 10 percent. What is the estimated current intrinsic value?
\end{enumerate}

\section{Solution:}
The analyst can value the share in two pieces:

\begin{enumerate}
  \item The analyst uses the Gordon growth model to estimate the value at $t=$ 5 ; in the model, the year-ahead dividend is $\$ 4(1.06)$. Then the analyst finds the present value of this value as of $t=0$.

  \item The analyst finds the present value of the $\$ 4$ dividend not "counted" in the estimate in Piece 1 (which values dividends from $t=6$ onward). Note that the statement of the problem implies that $D_{0}, D_{1}, D_{2}, D_{3}$, and $D_{4}$ are zero.

\end{enumerate}

Piece 1: The value of this piece is $\$ 65.818$ :

$V_{n}=\frac{D_{n}(1+g)}{r-g}=\frac{D_{n+1}}{r-g}$

$V_{5}=\frac{\$ 4(1+0.06)}{0.10-0.06}=\frac{\$ 4.24}{0.04}=\$ 106$

$V_{0}=\frac{\$ 106}{(1+0.10)^{5}}=\$ 65.818$

Piece 2: The value of this piece is $\$ 2.484$ :

$$
V_{0}=\frac{\$ 4}{(1+0.10)^{5}}=\$ 2.484
$$

The sum of the two pieces is $\$ 65.818+\$ 2.484=\$ 68.30$.

Alternatively, the analyst could value the share at $t=4$, the point at which dividends are expected to be paid in the following year and from which point they are expected to grow at a constant rate.

$$
\begin{aligned}
& V_{4}=\frac{\$ 4.00}{0.10-0.06}=\frac{\$ 4.00}{0.04}=\$ 100 \\
& V_{0}=\frac{\$ 100}{(1+0.10)^{4}}=\$ 68.30
\end{aligned}
$$

The next section addresses the application of the DDM with more flexible assumptions as to the dividend growth rate.

\section{MULTISTAGE DIVIDEND DISCOUNT MODELS}
calculate and interpret the intrinsic value of an equity security based on the Gordon (constant) growth dividend discount model or a two-stage dividend discount model, as appropriate

identify characteristics of companies for which the constant growth or a multistage dividend discount model is appropriate explain advantages and disadvantages of each category of valuation model

Multistage growth models are often used to model rapidly growing companies. The two-stage $D D M$ assumes that at some point the company will begin to pay dividends that grow at a constant rate, but prior to that time the company will pay dividends that are growing at a higher rate than can be sustained in the long run. That is, the company is assumed to experience an initial, finite period of high growth, perhaps prior to the entry of competitors, followed by an infinite period of sustainable growth. The two-stage DDM thus makes use of two growth rates: a high growth rate for an initial, finite period followed by a lower, sustainable growth rate into perpetuity. The Gordon growth model is used to estimate a terminal value at time $n$ that reflects the present value at time $n$ of the dividends received during the sustainable growth period.

Equation 11 will be used here as the starting point for a two-stage valuation model. The two-stage valuation model is similar to Example 8 except that instead of assuming zero dividends for the initial period, the analyst assumes that dividends will exhibit a high rate of growth during the initial period. Equation 11 values the dividends over the short-term period of high growth and the terminal value at the end of the period of high growth. The short-term growth rate, $g_{S}$, lasts for $n$ years. The intrinsic value per share in year $n, V_{n}$, represents the year $n$ value of the dividends received during the sustainable growth period or the terminal value at time $n$. $V_{n}$ can be estimated by using the Gordon growth model as shown in Equation 12, where $g_{L}$ is the long-term or sustainable growth rate. The dividend in year $n+1, D_{n+1}$, can be determined by using Equation 13:

$$
\begin{aligned}
& V_{0}=\sum_{t=1}^{n} \frac{D_{0}\left(1+g_{S}\right)^{t}}{(1+r)^{t}}+\frac{V_{n}}{(1+r)^{n}} \\
& V_{n}=\frac{D_{n+1}}{r-g_{L}} \\
& D_{n+1}=D_{0}\left(1+g_{S}\right)^{n}\left(1+g_{L}\right)
\end{aligned}
$$

\section{EXAMPLE 9}
\section{Applying the Two-Stage Dividend Discount Model}
\begin{enumerate}
  \item The current dividend, $D_{0}$, is $\$ 5.00$. Growth is expected to be 10 percent a year for three years and then 5 percent thereafter. The required rate of return is 15 percent. Estimate the intrinsic value.
\end{enumerate}

Solution:

$$
D_{1}=\$ 5.00(1+0.10)=\$ 5.50
$$

$$
\begin{aligned}
& D_{2}=\$ 5.00(1+0.10)^{2}=\$ 6.05 \\
& D_{3}=\$ 5.00(1+0.10)^{3}=\$ 6.655 \\
& D_{4}=\$ 5.00(1+0.10)^{3}(1+0.05)=\$ 6.98775 \\
& V_{3}=\frac{\$ 6.98775}{0.15-0.05}=\$ 69.8775 \\
& V_{0}=\frac{\$ 5.50}{(1+0.15)}+\frac{\$ 6.05}{(1+0.15)^{2}}+\frac{\$ 6.655}{(1+0.15)^{3}}+\frac{\$ 69.8775}{(1+0.15)^{3}} \approx \$ 59.68
\end{aligned}
$$

The DDM can be extended to as many stages as deemed appropriate. For most publicly traded companies (that is, companies beyond the start-up stage), practitioners assume growth will ultimately fall into three stages: 1) growth, 2) transition, and 3) maturity. This assumption supports the use of a three-stage $D D M$, which makes use of three growth rates: a high growth rate for an initial finite period, followed by a lower growth rate for a finite second period, followed by a lower, sustainable growth rate into perpetuity.

One can make the case that a three-stage DDM would be most appropriate for a fairly young company, one that is just entering the growth phase. The two-stage DDM would be appropriate to estimate the value of an older company that has already moved through its growth phase and is currently in the transition phase (a period with a higher growth rate than the sustainable growth rate) prior to moving to the maturity phase (the period with a lower, sustainable growth rate).

However, the choice of a two-stage DDM need not rely solely on the age of a company. Long-established companies sometimes manage to restart above-average growth through, for example, innovation, expansion to new markets, or acquisitions. Or a company's long-run growth rate may be interrupted by a period of subnormal performance. If growth is expected to moderate (in the first case) or improve (in the second case) toward some long-term growth rate, a two-stage DDM may be appropriate. Thus, we chose a two-stage DDM to value International Business Machines Corporation in Example 10.

\section{EXAMPLE 10}
\section{Two-Stage Dividend Discount Model: International Business Machines Corporation}
International Business Machines Corporation (IBM) is a US based leading technology company. IBM was founded in 1911, initially as a company that manufactured machinery for sale and lease, ranging from commercial scales and industrial time recorders, meat and cheese slicers, to punched cards. IBM introduced the personal computer in 1981; however, by the 1990s, it began to suffer losses in its core computer manufacturing business and by the $2000 \mathrm{~s}$, it had begun to diversify into business consulting, which was finalized in 2005 when it sold its personal computer business to Chinese company Lenovo. IBM now operates through five segments: Cognitive Solutions, Global Business Services (GBS), Technology Services \& Cloud Platforms, Systems, and Global Financing. The Cognitive Solutions segment delivers a spectrum of capabilities from descriptive, predictive, and prescriptive analytics to cognitive systems. Cognitive Solutions includes Watson, a cognitive computing platform that has the ability to interact in natural language, process big data, and learn from interactions with people and computers. The GBS segment provides clients with consulting, application management services, and global process services. The Technology Services \& Cloud Platforms segment provides information technology infrastructure services. The Systems segment provides clients with infrastructure technologies. The Global Financing segment includes client financing, commercial financing, and remanufacturing and remarketing.

The 30 July 2018 Value Line report on IBM appears in Exhibit 5. IBM has increased its dividends every year over the past 15 years. Information from Value Line shows that the dividend growth is around 17.0 percent for the past 10 years, 13.5 percent for the past 5 years, and estimated to be 4.5 percent for 2015 to 2023. After a period of growth through acquisition and merger, the pattern suggests that IBM may be transitioning to a mature growth phase.

The two-stage DDM is arguably a good choice for valuing IBM because the company appears to be transitioning from a high-growth phase (note the 13.5 percent dividend growth for the past 5 years) to a lower-growth phase (note the forecast of 4.5 percent dividend growth to 2015-2023).

The CAPM can be used to estimate the required return, $r$, for IBM. The Value Line report (in the top left corner) estimates beta to be 0.95 . Using the yield of about 2.0 percent on 10-year US Treasury notes as a proxy for the risk-free rate and assuming an equity risk premium of 5.0 percent, we find the estimate for $r$ would be 6.75 percent $[2.0 \%+0.95(5.0 \%)]$

To estimate the intrinsic value at the end of 2018 , we use the 2018 dividend of US\$6.21 from the Value Line report. The dividend is assumed to grow at a rate of 5.0 percent for one year and then 2.33 percent thereafter. The growth rate assumption for the first stage is consistent with the Value Line forecast for the 2018 and 2019 dividends. Our assumption of a 2.33 percent perpetual growth rate produces an 8-year growth rate assumption near $4.5 \%,{ }^{4}$ which is consistent with the Value Line forecast of 4.5 percent growth from 2015-2022. Thus:

$$
\begin{aligned}
& D_{2019}=\operatorname{US} \$ 6.21(1+0.05)=\mathrm{US} \$ 6.5205 \\
& D_{2020}=\mathrm{US} \$ 6.21(1+0.05)(1+0.0233)=\mathrm{US} \$ 6.6724 \\
& D_{2021}=\mathrm{US} \$ 6.21(1+0.05)(1+0.0233)^{2}=\mathrm{US} \$ 6.8279 \\
& V_{2020}=\frac{\mathrm{US} \$ 6.8279}{0.0675-0.0233}=\mathrm{US} \$ 154.4774 \\
& V_{2018}=\frac{\mathrm{US} \$ 6.5205}{(1+0.0675)}+\frac{\mathrm{US} \$ 6.6724}{(1+0.0675)^{2}}+\frac{\mathrm{US} \$ 154.4774}{(1+0.0675)^{2}} \approx \mathrm{US} \$ 147.523
\end{aligned}
$$

Given a recent price of US\$148.56, as noted at the top left corner of the Value Line report, the intrinsic-value estimate of US\$147.523 suggests that IBM is approximately fairly valued.

\section{Exhibit 5: Value Line Report on IBM}
\begin{center}
\includegraphics[max width=\textwidth]{2023_05_04_7b535d0a870224f62e3dg-411}
\end{center}

Source: @ 2018 Value Line, Inc. All Rights Reserved Worldwide. "Value Line" is a registered trademark of Value Line, Inc. Value Line Geometric and Arithmetic Indices calculated by Thomson Reuters. Information supplied by Thomson Reuters.

\section{MULTIPLER MODELS AND RELATIONSHIP AMONG PRICE MULTIPLES, PRESENT VALUE MODELS, AND FUNDAMENTALS}
\begin{center}
\includegraphics[max width=\textwidth]{2023_05_04_7b535d0a870224f62e3dg-412}
\end{center}

The term price multiple refers to a ratio that compares the share price with some sort of monetary flow or value to allow evaluation of the relative worth of a company's stock. Some practitioners use price ratios as a screening mechanism. If the ratio falls below a specified value, the shares are identified as candidates for purchase, and if the ratio exceeds a specified value, the shares are identified as candidates for sale. Many practitioners use ratios when examining a group or sector of stocks and consider the shares for which the ratio is relatively low to be attractively valued securities.

Price multiples that are used by security analysts include the following:

\begin{itemize}
  \item Price-to-earnings ratio $(\mathrm{P} / \mathrm{E})$. This measure is the ratio of the stock price to earnings per share. $\mathrm{P} / \mathrm{E}$ is arguably the price multiple most frequently cited by the media and used by analysts and investors (Block 1999). The seminal works of McWilliams (1966), Miller and Widmann (1966), Nicholson (1968), Dreman (1977), and Basu (1977) presented evidence of a return advantage to low-P/E stocks.

  \item Price-to-book ratio (P/B). The ratio of the stock price to book value per share. Considerable evidence suggests that P/B multiples are inversely related to future rates of return (Fama and French 1995).

  \item Price-to-sales ratio $(\mathrm{P} / \mathrm{S})$. This measure is the ratio of stock price to sales per share. O'Shaughnessy (2005) provided evidence that a low P/S multiple is the most useful multiple for predicting future returns.

  \item Price-to-cash-flow ratio $(\mathrm{P} / \mathrm{CF})$. This measure is the ratio of stock price to some per-share measure of cash flow. The measures of cash flow include free cash flow (FCF) and operating cash flow (OCF).

\end{itemize}

A common criticism of all of these multiples is that they do not consider the future. This criticism is true if the multiple is calculated from trailing or current values of the divisor. Practitioners seek to counter this criticism by a variety of techniques, including forecasting fundamental values (the divisors) one or more years into the future. The resulting forward (leading or prospective) price multiples may differ markedly from the trailing price multiples. In the absence of an explicit forecast of fundamental values, the analyst is making an implicit forecast of the future when implementing such models. The choice of price multiple-trailing or forward-should be used consistently for companies being compared.

Besides the traditional price multiples used in valuation, just presented, analysts need to know how to calculate and interpret other ratios. Such ratios include those used to analyze business performance and financial condition based on data reported in financial statements. In addition, many industries have specialized measures of business performance that analysts covering those industries should be familiar with. In analyzing cable television companies, for example, the ratio of total market value of the company to the total number of subscribers is commonly used. Another common measure is revenue per subscriber. In the oil industry, a commonly cited ratio is proved reserves per common share. Industry-specific or sector-specific ratios such as these can be used to understand the key business variables in an industry or sector as well as to highlight attractively valued securities.

\section{Relationships among Price Multiples, Present Value Models, and Fundamentals}
Price multiples are frequently used independently of present value models. One price multiple valuation approach, the method of comparables, does not involve cash flow forecasts or discounting to present value. A price multiple is often related to fundamentals through a discounted cash flow model, however, such as the Gordon growth model. Understanding such connections can deepen the analyst's appreciation of the factors that affect the value of a multiple and often can help explain reasons for differences in multiples that do not involve mispricing. The expressions that are developed can be interpreted as the justified value of a multiple - that is, the value justified by (based on) fundamentals or a set of cash flow predictions. These expressions are an alternative way of presenting intrinsic-value estimates.

As an example, using the Gordon growth model identified previously in Equation 9 and assuming that price equals intrinsic value $\left(P_{0}=V_{0}\right)$, we can restate Equation 9 as follows:

$$
P_{0}=\frac{D_{1}}{r-g}
$$

To arrive at the model for the justified forward P/E given in Equation 15, we divide both sides of Equation 14 by a forecast for next year's earnings, $E_{1}$. In Equation 15, the dividend payout ratio, $p$, is the ratio of dividends to earnings:

$$
\frac{P_{0}}{E_{1}}=\frac{D_{1} / E_{1}}{r-g}=\frac{p}{r-g}
$$

Equation 15 indicates that the $\mathrm{P} / \mathrm{E}$ is inversely related to the required rate of return and positively related to the growth rate; that is, as the required rate of return increases, the P/E declines, and as the growth rate increases, the P/E increases. The P/E and the payout ratio appear to be positively related. This relationship may not be true, however, because a higher payout ratio may imply a slower growth rate as a result of the company retaining a lower proportion of earnings for reinvestment. This phenomenon is referred to as the dividend displacement of earnings.

\section{EXAMPLE 11}
\section{Value Estimate Based on Fundamentals}
\begin{enumerate}
  \item Petroleo Brasileiro SA, commonly known as Petrobras, was once labeled "the most expensive oil company" by \href{http://Bloomberg.com}{Bloomberg.com}. Data for Petrobras and the oil industry, including the trailing twelve-month (TTM) P/E and payout ratios, appear below.
\end{enumerate}

\begin{center}
\begin{tabular}{ccc}
\hline
 & Petrobras & Industry \\
\hline
P/E ratio (TTM) & 39.61 & 13.0 \\
\hline
\end{tabular}
\end{center}

\begin{center}
\begin{tabular}{lcc}
\hline
 & Petrobras & Industry \\
\hline
Return on assets (TTM) (\%) & 3.0 & 3.2 \\
EPS 3-year growth rate (\%) & $\mathrm{NM}$ & 66.00 \\
EPS (MRQ) vs. Qtr. 1 yr. ago (\% change) & 138.96 & -12.0 \\
\hline
\end{tabular}
\end{center}

Note: NM stands for non-quantifiable. Petrobras EPS has decreased from a loss of BRL 1.14 per share to a profit of BRL 0.16 per share in the most recent period. MRQ stands for "most recent quarter."

\section{Source: Reuters.}
Explain how the information shown supports a higher P/E for Petrobras than for the industry.

\section{Solution:}
The data support a higher P/E for Petrobras because its (MRQ) EPS growth rate exceed those of the industry. Equation 15 implies a positive relationship between the payout ratio and the P/E multiple. Petrobras has had a negative EPS for the period 2014 to 2017, and has paid no dividend during that period. A higher payout ratio supports a higher P/E. Furthermore, to the extent that higher EPS growth implies a high growth rate in dividends, the high EPS growth rate supports a high P/E. The higher P/E ratio is due to an improvement in the underlying financial performance of the company and the expected higher growth potential of the stock compared to the median firms in the industry.

\section{EXAMPLE 12}
\section{Determining Justified Forward P/E}
Heinrich Gladisch, CFA, is estimating the justified forward P/E for Nestlé, one of the world's leading nutrition and health companies. Gladisch notes that sales for 2017 were SFr89.78 billion (US\$90.3 billion) and that net income was SFr7.18 billion (US\$7.25 billion). He organizes the data for EPS, dividends per share, and the dividend payout ratio for the years 2013-2017 in the following table:

\begin{center}
\begin{tabular}{cccccc}
\hline
 & $\mathbf{2 0 1 3}$ & $\mathbf{2 0 1 4}$ & $\mathbf{2 0 1 5}$ & $\mathbf{2 0 1 6}$ & $\mathbf{2 0 1 7}$ \\
\hline
Earnings per share & SFr3.24 & SFr4.54 & SFr2.90 & SFr2.76 & SFr2.32 \\
Year over year \% change &  & $44.6 \%$ & $-36.1 \%$ & $-4.8 \%$ & $-15.9 \%$ \\
Dividend per share & SFr2.15 & SFr2.2 & SFr2.25 & SFr2.3 & SFr2.35 \\
Year over year \% change &  & $2.3 \%$ & $2.3 \%$ & $2.2 \%$ & $2.2 \%$ \\
Dividend payout ratio & $68.5 \%$ & $48.5 \%$ & $77.6 \%$ & $83.3 \%$ & $99.2 \%$ \\
\hline
\end{tabular}
\end{center}

Gladisch calculates that ROE averaged 15.5 percent in the period 2013-2017 but was below that level at 11.7 percent in 2017. In that year, however, Nestlé's reported net income included a large nonrecurring component. The company reported 2017 "underlying earnings," which it defined as net income "from continuing operations before impairments, restructuring costs, results on disposals and significant one-off items," to be SFr2.93, giving an adjusted 14.8\% ROE. Predicting increasing improvement in Nestlés profit margins from growth in its product markets, Gladisch estimates a long-run ROE of 21.5 percent. Gladisch decides that the dividend payout ratios of the 2013-2016 periodaveraging 67.7 percent-are more representative of Nestlé's future payout ratio than is the high 2017 dividend payout ratio (when based on reported earnings). The dividend payout ratio in 2017 was higher because management apparently based the 2017 dividend on the components of net income that were expected to continue into the future. But basing a dividend on net income including non-recurring items creates the potential need to increase dividends in the future. Rounding up the 2013-2017 average, Gladisch settles on an estimate of 68 percent for the dividend payout ratio for use in calculating a justified forward P/E using Equation 15.

Gladisch's firm estimates that the required rate of return for Nestlé's shares is 9 percent per year. Gladisch also finds the following data at the opposite ends of the spectrum of external research analyst forecasts:

\begin{center}
\begin{tabular}{lcc}
\hline
 & $\mathbf{2 0 1 8 E}$ & 2019E \\
\hline
Most optimistic analyst forecast: &  &  \\
EPS & SFr3.99 & SFr4.33 \\
Year over year \% change & $71.9 \%$ & $8.5 \%$ \\
P/E (based on a target price of SFr105) & 26.3 & 24.2 \\
Least optimistic analyst forecast: &  &  \\
EPS & SFr3.52 & SFr3.59 \\
Year over year \% change & $51.7 \%$ & $2.0 \%$ \\
P/E (based on a target price of SFr68) & 19.3 & 18.9 \\
\hline
\end{tabular}
\end{center}

\begin{enumerate}
  \item Based only on information and estimates developed by Gladisch and his firm, estimate Nestlé's justified forward P/E.
\end{enumerate}

\section{Solution:}
The estimate of the justified forward $\mathrm{P} / \mathrm{E}$ is 32.38 . The dividend growth rate can be estimated by using Equation 10 as (1 - Dividend payout ratio $) \times \mathrm{ROE}$ $=(1-0.68) \times 0.21 .5=0.069$, or 6.9 percent. Therefore ,

$$
\frac{P_{0}}{E_{1}}=\frac{\text { Payout }}{r-g}=\frac{0.68}{0.09-0.069}=32.38
$$

\begin{enumerate}
  \setcounter{enumi}{1}
  \item Compare and contrast the justified forward $\mathrm{P} / \mathrm{E}$ estimate from Question 1 to the estimates from each end of the spectrum of external research analysts forecasts.
\end{enumerate}

\section{Solution:}
The estimated justified forward P/E of 32.38 is higher than the justified 2018 P/E estimates of 26.3 and 19.3 of the two analysts. Using a required rate of return of 9.5 percent rather than 9 percent results in a justified forward $P / E$ estimate of $26.2=0.68 /(0.095-0.069)$. Using an ROE of 16.5 percent (the average ROE of the 2013-2016 period) rather than 21.5 percent results in a justified forward P/E estimate of $18.4=0.68 /[0.09-(0.32)(0.165)]=0.68 /$ $(0.09-0.053)$. The justified forward $P / E$ is very sensitive to changes in the inputs.

\section{Equity Valuation: Concepts and Basic Tools}
Justified forward P/E estimates can be sensitive to small changes in assumptions. Therefore, analysts can benefit from carrying out a sensitivity analysis, as shown in Exhibit 6, which is based on Example 12. Exhibit 6 shows how the justified forward P/E varies with changes in the estimates for the dividend payout ratio (columns) and return on equity. The dividend growth rate (rows) changes because of changes in the retention rate $(1-$ Payout rate) and ROE. Recall $g=$ ROE times retention rate.

Exhibit 6: Estimates for Nestlé's Justified Forward P/E (Required Rate of Return $=9$ Percent)

\begin{center}
\begin{tabular}{|c|c|c|c|c|c|}
\hline
\multirow{2}{*}{$\begin{array}{l}\text { Constant Dividend } \\
\text { Growth Rate (\%) }\end{array}$} & \multicolumn{5}{|c|}{Dividend Payout Ratio} \\
\hline
 & $55 \%$ & $60 \%$ & $65 \%$ & $70 \%$ & $75 \%$ \\
\hline
4.0 & 11.0 & 12.0 & 13.0 & 14.0 & 15.0 \\
\hline
4.5 & 12.2 & 13.3 & 14.4 & 15.6 & 16.7 \\
\hline
5.0 & 13.8 & 15.0 & 16.3 & 17.5 & 18.8 \\
\hline
5.5 & 15.7 & 17.1 & 18.6 & 20.0 & 21.4 \\
\hline
6.0 & 18.3 & 20.0 & 21.7 & 23.3 & 25.0 \\
\hline
6.5 & 22.0 & 24.0 & 26.0 & 28.0 & 30.0 \\
\hline
7.0 & 27.5 & 30.0 & 32.5 & 35.0 & 37.5 \\
\hline
7.5 & 36.7 & 40.0 & 43.3 & 46.7 & 50.0 \\
\hline
\end{tabular}
\end{center}

METHOD OF COMPARABLES AND VALUATION BASED ON PRICE MULTIPLES

\begin{center}
\includegraphics[max width=\textwidth]{2023_05_04_7b535d0a870224f62e3dg-416}
\end{center}

The method of comparables is the most widely used approach for analysts reporting valuation judgments on the basis of price multiples. This method essentially compares relative values estimated using multiples or the relative values of multiples. The economic rationale underlying the method of comparables is the law of one price: Identical assets should sell for the same price. The methodology involves using a price multiple to evaluate whether an asset is fairly valued, undervalued, or overvalued in relation to a benchmark value of the multiple. Choices for the benchmark multiple include the multiple of a closely matched individual stock or the average or median value of the multiple for the stock's industry. Some analysts perform trend or time-series analyses and use past or average values of a price multiple as a benchmark. Identifying individual companies or even an industry as the "comparable" may present a challenge. Many large corporations operate in several lines of business, so the scale and scope of their operations can vary significantly. When identifying comparables (sometimes referred to as "comps"), the analyst should be careful to identify companies that are most similar according to a number of dimensions. These dimensions include (but are not limited to) overall size, product lines, and growth rate. The type of analysis shown in Section 5.1 relating multiples to fundamentals is a productive way to identify the fundamental variables that should be taken into account in identifying comparables.

\section{EXAMPLE 13}
\section{Method of Comparables (1)}
\begin{enumerate}
  \item As noted previously, $\mathrm{P} / \mathrm{E}$ is a price multiple frequently used by analysts. Using P/E in the method of comparables can be problematic, however, as a result of business cycle effects on EPS. An alternative valuation tool that is useful during periods of economic slowdown or extraordinary growth is the P/S multiple. Although sales will decline during a recession and increase during a period of economic growth, the change in sales will be less than the change in earnings in percentage terms because earnings are heavily influenced by fixed operating and financing costs (operating and financial leverage).
\end{enumerate}

The following data provide the $\mathrm{P} / \mathrm{S}$ for most of the major automobile manufacturers as at December 2017:

\begin{center}
\begin{tabular}{ll}
\hline
Company & P/S \\
\hline
Peugeot & 0.28 \\
Ford Motor & 0.33 \\
General Motors & 0.36 \\
Nissan Motor & 0.38 \\
Honda Motor & 0.46 \\
Tata Motors & 0.49 \\
Daimler & 0.55 \\
BMW & 0.57 \\
Toyota Motor & 0.80 \\
\hline
\end{tabular}
\end{center}

Sources: Morningstar and company websites.

Based on the data presented, which stock appears to be undervalued when compared with the others?

\section{Solution:}
The P/S analysis suggests that Peugeot shares offer the best value. An analyst must be alert for a range of potential explanations of apparently low or high multiples when performing comparables analysis, rather than just assuming a relative mispricing.

\section{EXAMPLE 14}
\section{Method of Comparables ( $(2)$}
\begin{enumerate}
  \item Incorporated in the Netherlands, Airbus is active in the aerospace and defense industry. It is a dominant aerospace company in Europe. Its largest business, Airbus Commercial Aircraft, is a manufacturing company with bases in several European countries and accounts for the majority of Airbus SE profits. Airbus and its primary competitor, Boeing, control most of the global commercial airplane industry.
\end{enumerate}

Comparisons are frequently made between Airbus and Boeing. As noted in Exhibit 7, the companies are broadly similar in size as measured by total revenues. Converting total forecast revenues from euros to US dollars using the average exchange rate for 2017 of US $\$ 1.13 / €$ results in a value of $\$ 75.5$ billion for Airbus's total revenues. Thus, total revenues for Boeing are expected to be about a fifth higher than those for Boeing.

The companies do differ, however, in several important areas. Airbus derives a greater share of its revenue from commercial aircraft production than does Boeing, and the order backlog for Airbus is much higher than that for Boeing. Converting the Airbus order backlog from euros to US dollars using the quarter-end rate for September 2017 of $\$ 1.1813 / €$ results in a value of $\$ 1.12$ billion for Airbus's order backlog. Thus, the order backlog for Airbus is more than twice as high as the backlog for Boeing. ${ }^{5}$

\section{Exhibit 7: Data for EADS and Boeing}
\begin{center}
\begin{tabular}{lcc}
\hline
 & Airbus & Boeing \\
\hline
Total revenues (billions, 2017) & $€ 66.8$ & $\$ 92.2$ \\
Annual revenue growth (2015-2017 &  &  \\
average) & $1.8 \%$ & $-2.1 \%$ \\
Percent of revenues from commercial &  &  \\
aircraft & $75 \%$ & $\$ 9 \%$ \\
Order backlog (billions) & $€ 945$ & $\$ 283.73$ \\
Share price, 12/Dec/17 & $€ 86.96$ & $\$ 10.18$ \\
EPS (basic) & $€ 3.33$ & $\$ 5.7$ \\
DPS & $€ 1.48$ & $56 \%$ \\
Dividend payout ratio & $44 \%$ & 27.9 \\
P/E ratio & 26.1 &  \\
\hline
\end{tabular}
\end{center}

Note: 2017 forecast data and YTD average exchange rate as of 12 December 2017. Order backlog as of 30 September 2017

Sources: Company websites: \href{http://www.airbus.com}{www.airbus.com} and \href{http://www.boeing.com}{www.boeing.com}, Financial Times.

What data shown in Exhibit 7 support a higher P/E for Boeing than for Airbus?

5 Exchange rate data are available from FRED (Federal Reserve Economic Data) at \href{http://research.stlouisfed}{http://research.stlouisfed}. org/fred2/. Each company uses slightly different methodology for calculating order backlog.

\section{Solution:}
Recall from Equation 14 and the discussion that followed it that P/E is directly related to the payout ratio and the dividend growth rate. The $P / E$ is inversely related to the required rate of return. The only data presented in Exhibit 7 that support a higher P/E for Boeing is the company's higher dividend payout ratio (expected at 56 percent versus 44 percent for Airbus).

The following implicitly supports a higher P/E for Airbus: Airbus has higher revenue growth (as reported for 2016 and expected for 2017) and a higher backlog of orders, suggesting that it may have a higher future growth rate.

\section{EXAMPLE 15}
\section{Method of Comparables (3)}
\begin{enumerate}
  \item Canon Inc. is a leading worldwide manufacturer of business machines, cameras, and optical products. Canon was founded in 1937 as a camera manufacturer and is incorporated in Tokyo. The corporate philosophy of Canon is kyosei or "living and working together for the common good." The following data can be used to determine a P/E for Canon over the time period 2013-2017. Analyze the P/E of Canon over time and discuss the valuation of Canon.
\end{enumerate}

\begin{center}
\begin{tabular}{lccc}
\hline
Year & $\begin{array}{c}\text { Price } \\ (\mathbf{a})\end{array}$ & $\begin{array}{c}\text { EPS } \\ \text { (b) }\end{array}$ & $\begin{array}{c}\text { P/E } \\ \text { (a) } \div \text { (b) }\end{array}$ \\
\hline
2013 & $¥ 3,330$ & $¥ 200.8$ & 16.6 \\
2014 & $¥ 3,840.5$ & $¥ 229.0$ & 16.8 \\
2015 & $¥ 3,675$ & $¥ 201.7$ & 18.2 \\
2016 & $¥ 3,295$ & $¥ 138$ & 23.9 \\
2017 & $¥ 4,200$ & $¥ 222.88$ & 18.8 \\
\hline
\end{tabular}
\end{center}

Sources: EPS, year-end prices, and P/E data are from Capital IQ and the Financial Times.

\section{Solution:}
Trend analysis of Canon's P/E reveals a peak of 23.9 at the end of 2016. The $2013 \mathrm{P} / \mathrm{E}$ of 16.6 is the lowest of the five years reported. This finding suggests that Canon's share price may be fairly price as of year-end 2017. A bearish case for Canon's stock can be made if an analyst believes that P/E will return to its historical low (16.6 over this five-year period) or be lower. Such a bearish prediction requires that a decrease in P/E not be offset by an increase in EPS. A bullish case can be made if the analyst believes the stock deserves re-rating and an even higher than trend P/E.

\section{Illustration of a Valuation Based on Price Multiples}
Telefónica S.A., a world leader in the telecommunication sector, provides communication, information, and entertainment products and services in Europe, Africa, and Latin America. It has operated in its home country of Spain since 1924, but as of 2017, more than 75 percent of its business was outside its home market. Deutsche Telekom AG provides network access, communication services, and value-added services via fixed and mobile networks. It generates more than half of its revenues outside its home country, Germany.

Exhibit 8 provides comparable data for these two communication giants for 2015-2017.

\section{Exhibit 8: Data for Telefónica and Deutsche Telekom}
\begin{center}
\begin{tabular}{|c|c|c|c|c|c|c|}
\hline
 & \multicolumn{3}{|c|}{Telefónica} & \multicolumn{3}{|c|}{Deutsche Telekom} \\
\hline
 & 2017 & 2016 & 2015 & 2017 & 2016 & 2015 \\
\hline
(1) Total assets ( $€$ billions) & 115.0 & 123.6 & 120.3 & 141.3 & 148.5 & 143.9 \\
\hline
Asset growth & $-6.9 \%$ & $2.7 \%$ & - & $-4.9 \%$ & $3.2 \%$ & - \\
\hline
(2) Net revenues ( $€$ billions) & 52.0 & 52.0 & 54.9 & 77.3 & 75.2 & 71.3 \\
\hline
Revenue growth & $0 \%$ & $-5.2 \%$ & - & $2.8 \%$ & $5.5 \%$ & - \\
\hline
$\begin{array}{l}\text { (3) Net cash flow from operating activities } \\ \text { ( } € \text { billions) }\end{array}$ & 13.8 & 13.3 & 13.6 & 17.2 & 15.5 & 15.0 \\
\hline
Cash flow growth & $3.4 \%$ & $-2.0 \%$ & - & $11.0 \%$ & $3.3 \%$ & - \\
\hline
$\begin{array}{l}\text { (4) Book value of common shareholders' equity } \\ \text { ( } € \text { billions) }\end{array}$ & 16.9 & 18.2 & 15.8 & 42.5 & 38.8 & 38.2 \\
\hline
$\begin{array}{l}\text { Debt ratio: } \\ \quad 1-[(4) \div(1)]\end{array}$ & $85.3 \%$ & $85.3 \%$ & $86.9 \%$ & $70.0 \%$ & $73.9 \%$ & $73.5 \%$ \\
\hline
(5) Net profit ( $€$ billions) & 2.9 & 2.1 & 0.4 & 3.5 & 2.7 & 3.3 \\
\hline
Earnings growth & $38.1 \%$ & $425.0 \%$ & - & $29.6 \%$ & $-18.2 \%$ & - \\
\hline
$\begin{array}{l}\text { (6) Weighted average number of shares outstand- } \\ \text { ing (millions) }\end{array}$ & $5,122.9$ & $4,896.6$ & $4,833.6$ & $4,740.2$ & $4,654.9$ & $4,584.8$ \\
\hline
(7) Price per share $(€)$ & 7.93 & 9.52 & 9.13 & 13.42 & 16.21 & 15.44 \\
\hline
Price-to-revenue ratio $(P / R)$ : &  &  &  &  &  &  \\
\hline
$(7) \div[(2) \div(6)]$ & 0.8 & 0.9 & 0.8 & 0.8 & 1.0 & 1.0 \\
\hline
$P / C F$ &  &  &  &  &  &  \\
\hline
$(7) \div[(3) \div(6)]$ & 2.9 & 3.5 & 3.2 & 3.7 & 4.9 & 4.7 \\
\hline
$P / B$ &  &  &  &  &  &  \\
\hline
$(7) \div[(4) \div(6)]$ & 2.4 & 2.6 & 2.8 & 1.5 & 1.9 & 1.9 \\
\hline
$P / E$ &  &  &  &  &  &  \\
\hline
(7) $\div[(5) \div(6)]$ & 14.0 & 22.2 & 110.3 & 18.2 & 27.9 & 21.5 \\
\hline
\end{tabular}
\end{center}

Sources: Company websites: \href{http://www.telefonica.es}{www.telefonica.es} and \href{http://www.deutschetelekom.com}{www.deutschetelekom.com}.

Time-series analysis of all price multiples in Exhibit 8 suggests that both companies are currently attractively valued. For example, the 2017 price-to-revenue ratio (P/R) of 0.78 for Telefónica is below the 2015-2017 average for this ratio of approximately 0.83. The 2017 P/CF of 3.7 for Deutsche Telekom is below the 2015-2017 average for this ratio of approximately 4.4 .

A comparative analysis produces somewhat mixed results. The 2017 values for Deutsche Telekom for the P/R, P/CF, P/E multiples are higher than those for Telefónica. This result suggests that Telefónica is attractively valued when compared with Deutsche Telekom. The 2017 P/B for Telefónica, however, is higher than for Deutsche Telekom.

An analyst investigating these contradictory results would look for information not reported in Exhibit 8. For example, the earnings before interest, taxes, depreciation, and amortization (EBITDA) for Telefónica was $€ 16.4$ billion in 2017. The EBITDA value for Deutsche Telekom was $€ 20.7$ billion in 2017. The 2017 price-to-EBITDA ratio for Telefónica is $[(7.93 \times 5,123) / 16,400]$ or $[7.92 /(16,400 / 5,123]=2.5$, whereas the 2017 price-to-EBITDA ratio for Deutsche Telekom is 3.1. Thus, the higher P/E for Deutsche Telekom can-not be explained by higher depreciation charges, higher interest costs, and/or a greater tax burden, but appears to be due to a better quality of earnings

In summary, the major advantage of using price multiples is that they allow for relative comparisons, both cross-sectional (versus the market or another comparable) and in time series. The approach can be especially beneficial for analysts who are assigned to a particular industry or sector and need to identify the expected best performing stocks within that sector. Price multiples are popular with investors because the multiples can be calculated easily and many multiples are readily available from financial websites and newspapers.

Caution is necessary. A stock may be relatively undervalued when compared with its benchmarks but overvalued when compared with an estimate of intrinsic value as determined by one of the discounted cash flow methodologies. Furthermore, differences in reporting rules among different markets and in chosen accounting methods can result in revenues, earnings, book values, and cash flows that are not easily comparable. These differences can, in turn, result in multiples that are not easily comparable. Finally, the multiples for cyclical companies may be highly influenced by current economic conditions.

\section{ENTERPRISE VALUE}
\begin{center}
\includegraphics[max width=\textwidth]{2023_05_04_7b535d0a870224f62e3dg-421}
\end{center}

An alternative to estimating the value of equity is to estimate the value of the enterprise. Enterprise value is most frequently determined as market capitalization plus market value of preferred stock plus market value of debt minus cash and investments (cash equivalents and short-term investments). Enterprise value is often viewed as the cost of a takeover: In the event of a buyout, the acquiring company assumes the acquired company's debt but also receives its cash. Enterprise value is most useful when comparing companies with significant differences in capital structure.

Enterprise value (EV) multiples are widely used in Europe, with EV/EBITDA arguably the most common. EBITDA is a proxy for operating cash flow because it excludes depreciation and amortization. EBITDA may include other non-cash expenses, however, and non-cash revenues. EBITDA can be viewed as a source of funds to pay interest, dividends, and taxes. Because EBITDA is calculated prior to payment to any of the company's financial stakeholders, using it to estimate enterprise value is logically appropriate.

Using enterprise value instead of market capitalization to determine a multiple can be useful to analysts. Even where the $\mathrm{P} / \mathrm{E}$ is problematic because of negative earnings, the EV/EBITDA multiple can generally be computed because EBITDA is usually positive. An alternative to using EBITDA in EV multiples is to use operating income.

In practice, analysts may have difficulty accurately assessing enterprise value if they do not have access to market quotations for the company's debt. When current market quotations are not available, bond values may be estimated from current quotations for bonds with similar maturity, sector, and credit characteristics. Substituting the book value of debt for the market value of debt provides only a rough estimate of the debt's market value. This is because market interest rates change and investors' perception of the issuer's credit risk may have changed since the debt was issued.

\section{EXAMPLE 16}
\section{Estimating the Market Value of Debt and Enterprise Value}
\begin{enumerate}
  \item Cameco Corporation is one of the world's largest uranium producers; it accounts for 16 percent of world production from its mines in Canada and the United States. Cameco estimates it has about 458 million kilograms of proven and probable reserves and holds premier land positions in the world's most promising areas for new uranium discoveries in Canada and Australia. Cameco is also a leading provider of processing services required to produce fuel for nuclear power plants. It generates 1,000 megawatts of electricity through a partnership in North America's largest nuclear generating station located in Ontario, Canada.
\end{enumerate}

For simplicity of exposition in this example, we will present share counts in thousands and all dollar amounts in thousands of Canadian dollars. In 2017, Cameco had 395,793 shares outstanding. Its 2017 year-end share price was $\$ 14.11$. Therefore, Cameco's 2017 year-end market capitalization was $\$ 5,584,640$.

In its 2017 Annual Report (available at \href{http://www.cameco.com}{www.cameco.com}), Cameco reported total debt and other liabilities of $\$ 2,919,100$. The company presented the following schedule for long-term debt payments:

\begin{center}
\begin{tabular}{lc}
\hline
Year & Payment \\
\hline
2018 & $\$ 69,000$ \\
2019 and &  \\
2020 & 610,000 \\
2021 and &  \\
2022 & 482,000 \\
Thereafter & 744,000 \\
\cline { 2 - 3 }
Total & $\$ 1,905,000$ \\
\hline
\end{tabular}
\end{center}

Cameco's longest maturity debt matures in 2042. We will assume that the amounts paid in 2019 and 2020, and in 2021 and 2022, will be paid equally during the two years. The "thereafter" period includes two debenture tranches, the first one maturing in 2024 for a total value of $\$ 620,000$ and the second tranche maturing in 2042 for the remaining $\$ 124,000$. A yield curve for zero-coupon Canadian government securities was available from the Bank of Canada. The yield-curve data and assumed risk premiums in Exhibit 9 were used to estimate the market value of Cameco's long-term debt:

\section{Exhibit 9: Estimated Market Value}
\begin{center}
\begin{tabular}{|c|c|c|c|c|c|}
\hline
Year & $\begin{array}{c}\text { Yield on } \\ \text { Zero-Coupon } \\ \text { Government } \\ \text { Security (\%) }\end{array}$ & $\begin{array}{c}\text { Assumed } \\ \text { Risk } \\ \text { Premium } \\ \text { (\%) }\end{array}$ & $\begin{array}{l}\text { Discount } \\ \text { Rate (\%) }\end{array}$ & $\begin{array}{l}\text { Book } \\ \text { Value }\end{array}$ & $\begin{array}{c}\text { Market } \\ \text { Value }\end{array}$ \\
\hline
2018 & 0.89 & 0.50 & 1.39 & $\$ 69,000$ & $\$ 68,054$ \\
\hline
2019 & 1.11 & 1.00 & 2.11 & $\$ 305,000$ & $\$ 292,525$ \\
\hline
2020 & 1.39 & 1.50 & 2.89 & $\$ 305,000$ & $\$ 280,014$ \\
\hline
2021 & 1.65 & 2.00 & 3.65 & $\$ 241,000$ & $\$ 208,804$ \\
\hline
2022 & 1.88 & 2.50 & 4.38 & $\$ 241,000$ & $\$ 194,505$ \\
\hline
2023 & 2.10 & 3.00 & 5.10 & $\$ 0$ & $\$ 0$ \\
\hline
2024 & 2.30 & 3.50 & 5.80 & $\$ 620,000$ & $\$ 417,823$ \\
\hline
2025 & 2.50 & 4.00 & 6.50 & $\$ 0$ & $\$ 0$ \\
\hline
$\ldots$ & $\ldots$ & $\ldots$ & $\ldots$ & $\$ \ldots$ & $\$ .$. \\
\hline
\multirow[t]{2}{*}{2042} & 2.92 & 5.00 & 7.92 & $\$ 124,000$ & $\$ 18,445$ \\
\hline
 &  &  &  & $\$ 1,905,000$ & $\$ 1,480,170$ \\
\hline
\end{tabular}
\end{center}

Note from Exhibit 9 that the book value of long-term debt is $\$ 1,905,000$ and its estimated market value is $\$ 1,480,170$. The book value of total debt and liabilities of $\$ 2,919,100$ minus the book value of long-term debt of $\$ 1,905,000$ is $\$ 1,014,100$. If we assume that the market value of that remaining debt is equal to its book value of $\$ 1,014,100$ an estimate of the market value of total debt and liabilities is that amount plus the estimated market value of long-term debt of $\$ 1,480,170$ or $\$ 2,494,270$.

At the end of 2017, Cameco had cash and equivalents of $\$ 591,600$. Enterprise value can be estimated as the $\$ 5,584,640$ market value of stock plus the $\$ 2,494,270$ market value of debt minus the $\$ 591,600$ cash and equivalents, or $\$ 7,487,310$. Cameco's 2017 EBITDA was $\$ 606,000$; an estimate of EV/ EBITDA is, therefore, $\$ 7,487,310$ divided by $\$ 606,000$, or 12.4 .

\section{EXAMPLE 17}
\section{EV/Operating Income}
\begin{enumerate}
  \item Exhibit 10 presents data for twelve major mining companies. Based only on the information in Exhibit 10, which two mining companies seem to be the most undervalued?
\end{enumerate}

\section{Exhibit 10: Data for Twelve Major Mining Companies}
\begin{center}
\begin{tabular}{lccc}
\hline
 & EV & $\begin{array}{c}\text { Operating } \\ \text { Income (OI) }\end{array}$ &  \\
Company & $119,712.3$ & 11,753 & 10.19 \\
\hline
BHP Billiton & $93,856.1$ & 6,471 & 14.5 \\
Rio Tinto & $82,051.2$ & 6,366 & 12.89 \\
\end{tabular}
\end{center}

\begin{center}
\begin{tabular}{lccc}
\hline
 &  &  &  \\
 & EV & $\begin{array}{c}\text { Operating } \\ \text { Income (OI) }\end{array}$ &  \\
Company & (US\$ millions) & $\begin{array}{c}\text { (US\$ millions) }\end{array}$ & EV/OI \\
\hline
Glencore & $80,772.0$ & -549 & -147.13 \\
Southern Copper & $37,817.0$ & 1,564 & 24.18 \\
Freeport-McMoRan & $33,452.0$ & $-2,766$ & -12.09 \\
Anglo American & $32,870.3$ & 2,562 & 12.83 \\
Norilsk Nickel & $22,483.0$ & 3,377 & 6.66 \\
Coal India & $21,652.1$ & 1,382 & 15.67 \\
Barrick Gold & $21,549.8$ & 2,424 & 8.89 \\
Newmont Mining & $20,683.0$ & -65 & -318.20 \\
Goldcorp & $12,986.7$ & 369 & 35.19 \\
\hline
\end{tabular}
\end{center}

Source: \href{http://www.miningfeeds.com}{www.miningfeeds.com}, Morningstar.

\section{Solution:}
Norilsk Nickel and Barrick Gold have the lowest EV/OI and thus appear to be the most undervalued or favorably priced on the basis of the EV/OI. Note the negative ratio for Glencore, Freeport-McMoRan, and Newmont Mining. Negative ratios are difficult to interpret, so other means are used to evaluate companies with negative ratios.

An asset-based valuation of a company uses estimates of the market or fair value of the company's assets and liabilities. Thus, asset-based valuations work well for companies that do not have a high proportion of intangible or "off the books" assets and that do have a high proportion of current assets and current liabilities. The analyst may be able to value these companies' assets and liabilities in a reasonable fashion by starting with balance sheet items. For most companies, however, balance sheet values are different from market (fair) values, and the market (fair) values can be difficult to determine.

Asset-based valuation models are frequently used together with multiplier models to value private companies. As public companies increase reporting or disclosure of fair values, asset-based valuation may be increasingly used to supplement present value and multiplier models of valuation. Important facts that the practitioner should realize are as follows:

\begin{itemize}
  \item Companies with assets that do not have easily determinable market (fair) values-such as those with significant property, plant, and equipment-are very difficult to analyze using asset valuation methods.

  \item Asset and liability fair values can be very different from the values at which they are carried on the balance sheet of a company. - Some assets that are "intangible" are shown on the books of the company. Other intangible assets, such as the value from synergies or the value of a good business reputation, may not be shown on the books. Because asset-based valuation may not consider some intangibles, it can give a "floor" value for a situation involving a significant amount of intangibles. When a company has significant intangibles, the analyst should prefer a forward-looking cash flow valuation.

  \item Asset values may be more difficult to estimate in a hyper-inflationary environment.

\end{itemize}

We begin by discussing asset-based valuation for hypothetical nonpublic companies and then move on to a public company example. Analysts should consider the difficulties and rewards of using asset-based valuation for companies that are suited to this measure. Owners of small privately held businesses are familiar with valuations arrived at by valuing the assets of the company and then subtracting any relevant liabilities.

\section{EXAMPLE 18}
\section{An Asset-Based Valuation of a Family-Owned Laundry}
\begin{enumerate}
  \item A family owns a laundry and the real estate on which the laundry stands. The real estate is collateral for an outstanding loan of $\$ 100,000$. How can asset-based valuation be used to value this business?
\end{enumerate}

\section{Solution:}
The analyst should get at least two market appraisals for the real estate (building and land) and estimate the cost to extinguish the $\$ 100,000$ loan. This information would provide estimated values for everything except the laundry as a going concern. That is, the analyst has market values for the building and land and the loan but needs to value the laundry business. The analyst can value the assets of the laundry: the equipment and inventory. The equipment can be valued at depreciated value, inflation-adjusted depreciated value, or replacement cost. Replacement cost in this case means the amount that would have to be spent to buy equivalent used machines. This amount is the market value of the used machines. The analyst will recognize that any intangible value of the laundry (prime location, clever marketing, etc.) is being excluded, which will result in an inaccurate asset-based valuation.

Example 18 shows some of the subtleties present in applying asset-based valuation to determine company value. It also shows how asset-based valuation does not deal with intangibles. Example 19 emphasizes this point.

\section{EXAMPLE 19}
\section{An Asset-Based Valuation of a Restaurant}
\begin{enumerate}
  \item The business being valued is a restaurant that serves breakfast and lunch. The owner/proprietor wants to sell the business and retire. The restaurant space is rented, not owned. This particular restaurant is hugely popular because of the proprietor's cooking skills and secret recipes. How can the analyst value this business?
\end{enumerate}

Solution:

Because of the intangibles, setting a value on this business is challenging. A multiple of income or revenue might be considered. But even those approaches overlook the fact that the proprietor may not be selling his secret recipes and, furthermore, does not intend to continue cooking. Some (or all) of the intangible assets may vanish when the business is sold. Asset-based valuation for this restaurant would begin with estimating the value of the restaurant equipment and inventory and subtracting the value of any liabilities. This approach will provide only a good baseline, however, for a minimum valuation.

For public companies, the assets will typically be so extensive that a piece-by-piece analysis will be impossible, and the transition from book value to market value is a nontrivial task. The asset-based valuation approach is most applicable when the market value of the corporate assets is readily determinable and the intangible assets, which are typically difficult to value, are a relatively small proportion of corporate assets. Asset-based valuation has also been applied to financial companies, natural resource companies, and formerly going-concerns that are being liquidated. Even for other types of companies, however, asset-based valuation of tangible assets may provide a baseline for a minimal valuation.

\section{EXAMPLE 20}
\section{An Asset-Based Valuation of an Airline}
\begin{enumerate}
  \item Consider the value of an airline company that has few routes, high labor and other operating costs, has stopped paying dividends, and is losing millions of dollars each year. Using most valuation approaches, the company will have a negative value. Why might an asset-based valuation approach be appropriate for use by one of the company's competitors that is considering acquisition of this airline?
\end{enumerate}

\section{Solution:}
The airline's routes, landing rights, leases of airport facilities, and ground equipment and airplanes may have substantial value to a competitor. An asset-based approach to valuing this company would value the company's assets separately and aside from the money-losing business in which they are presently being utilized.

Analysts recognizing the uncertainties related to model appropriateness and the inputs to the models frequently use more than one model or type of model in valuation to increase their confidence in their estimates of intrinsic value. The choice of models will depend on the availability of information to put into the models. Example 21 illustrates the use of three valuation methods.

\section{EXAMPLE 21}
\section{A Simple Example of the Use of Three Major Equity Valuation Models}
Company data for dividend per share (DPS), earnings per share (EPS), share price, and price-to-earnings ratio (P/E) for the most recent five years are presented in Exhibit 11. In addition, estimates (indicated by an "E" after the amount) of DPS and EPS for the next five years are shown. The valuation date is at the end of Year 5. The company has 1,000 shares outstanding.

Exhibit 11: Company DPS, EPS, Share Price, and P/E Data

\begin{center}
\begin{tabular}{|c|c|c|c|c|}
\hline
Year & DPS & EPS & Share Price & TTM P/E \\
\hline
10 & $\$ 3.10 \mathrm{E}$ & $\$ 5.20 \mathrm{E}$ & - & - \\
\hline
9 & $\$ 2.91 \mathrm{E}$ & $\$ 4.85 \mathrm{E}$ & - & - \\
\hline
8 & $\$ 2.79 \mathrm{E}$ & $\$ 4.65 \mathrm{E}$ & - & - \\
\hline
7 & $\$ 2.65 \mathrm{E}$ & $\$ 4.37 \mathrm{E}$ & - & - \\
\hline
6 & $\$ 2.55 \mathrm{E}$ & $\$ 4.30 \mathrm{E}$ & - & - \\
\hline
5 & $\$ 2.43$ & $\$ 4.00$ & $\$ 50.80$ & 12.7 \\
\hline
4 & $\$ 2.32$ & $\$ 3.90$ & $\$ 51.48$ & 13.2 \\
\hline
3 & $\$ 2.19$ & $\$ 3.65$ & $\$ 59.86$ & 16.4 \\
\hline
2 & $\$ 2.14$ & $\$ 3.60$ & $\$ 54.72$ & 15.2 \\
\hline
1 & $\$ 2.00$ & $\$ 3.30$ & $\$ 46.20$ & 14.0 \\
\hline
\end{tabular}
\end{center}

The company's balance sheet at the end of Year 5 is given in Exhibit 12.

\section{Exhibit 12: Balance Sheet as of End of Year 5}
\begin{center}
\begin{tabular}{lc}
Cash & $\$ 5,000$ \\
Accounts receivable & 15,000 \\
Inventories & 30,000 \\
Net fixed assets & 50,000 \\
\cline { 2 - 2 }
Total assets & $\$ 100,000$ \\
Accounts payable &  \\
Notes payable & $\$ 3,000$ \\
Term loans & 17,000 \\
Common shareholders' equity & 25,000 \\
Total liabilities and equity & $\$ 5,000$ \\
\hline
\end{tabular}
\end{center}

\begin{enumerate}
  \item Using a Gordon growth model, estimate intrinsic value. Use a discount rate of 10 percent and an estimate of growth based on growth in dividends over the next five years.
\end{enumerate}

\section{Solution:}
$D_{5}(1+g)^{5}=D_{10} 2.43(1+g)^{5}=3.10$

$g \approx 5.0 \%$ Estimate of value $=V_{5}=2.55 /(0.10-0.05)=\$ 51.00$

\begin{enumerate}
  \setcounter{enumi}{1}
  \item Using a multiplier approach, estimate intrinsic value. Assume that a reasonable estimate of $\mathrm{P} / \mathrm{E}$ is the average trailing twelve-month (TTM) $P / E$ ratio over Years 1 through 4.
\end{enumerate}

Solution:

Average P/E $=(14.0+15.2+16.4+13.2) / 4=14.7$

Estimate of value $=\$ 4.00 \times 14.7=\$ 58.80$

\begin{enumerate}
  \setcounter{enumi}{2}
  \item Using an asset-based valuation approach, estimate value per share from adjusted book values. Assume that the market values of accounts receivable and inventories are as reported, the market value of net fixed assets is 110 percent of reported book value, and the reported book values of liabilities reflect their market values.
\end{enumerate}

Solution:

Market value of assets $=5,000+15,000+30,000+1.1(50,000)=\$ 105,000$

Market value of liabilities $=\$ 3,000+17,000+25,000=\$ 45,000$

Adjusted book value $=\$ 105,000-45,000=\$ 60,000$

Estimated value (adjusted book value per share) $=\$ 60,000 \div 1,000$ shares $=\$ 60.00$

Given the current share price of $\$ 50.80$, the multiplier and the asset-based valuation approaches indicate that the stock is undervalued. Given the intrinsic value estimated using the Gordon growth model, the analyst is likely to conclude that the stock is fairly priced. The analyst might examine the assumptions in the multiplier and the asset-based valuation approaches to determine why their estimated values differ from the estimated value provided by the Gordon growth model and the market price.

\section{SUMMARY}
The equity valuation models used to estimate intrinsic value-present value models, multiplier models, and asset-based valuation-are widely used and serve an important purpose. The valuation models presented here are a foundation on which to base analysis and research but must be applied wisely. Valuation is not simply a numerical analysis. The choice of model and the derivation of inputs require skill and judgment.

When valuing a company or group of companies, the analyst wants to choose a valuation model that is appropriate for the information available to be used as inputs. The available data will, in most instances, restrict the choice of model and influence the way it is used. Complex models exist that may improve on the simple valuation models described in this reading; but before using those models and assuming that complexity increases accuracy, the analyst would do well to consider the "law of parsimony:" A model should be kept as simple as possible in light of the available inputs. Valuation is a fallible discipline, and any method will result in an inaccurate forecast at some time. The goal is to minimize the inaccuracy of the forecast.

Among the points made in this reading are the following:

\begin{itemize}
  \item An analyst estimating intrinsic value is implicitly questioning the market's estimate of value.

  \item If the estimated value exceeds the market price, the analyst infers the security is undervalued. If the estimated value equals the market price, the analyst infers the security is fairly valued. If the estimated value is less than the market price, the analyst infers the security is overvalued. Because of the uncertainties involved in valuation, an analyst may require that value estimates differ markedly from market price before concluding that a misvaluation exists.

  \item Analysts often use more than one valuation model because of concerns about the applicability of any particular model and the variability in estimates that result from changes in inputs.

  \item Three major categories of equity valuation models are present value, multiplier, and asset-based valuation models.

  \item Present value models estimate value as the present value of expected future benefits.

  \item Multiplier models estimate intrinsic value based on a multiple of some fundamental variable.

  \item Asset-based valuation models estimate value based on the estimated value of assets and liabilities.

  \item The choice of model will depend upon the availability of information to input into the model and the analyst's confidence in both the information and the appropriateness of the model.

  \item Companies distribute cash to shareholders using dividend payments and share repurchases.

  \item Regular cash dividends are a key input to dividend valuation models.

  \item Key dates in dividend chronology are the declaration date, ex-dividend date, holder-of-record date, and payment date.

  \item In the dividend discount model, value is estimated as the present value of expected future dividends.

  \item In the free cash flow to equity model, value is estimated as the present value of expected future free cash flow to equity.

  \item The Gordon growth model, a simple DDM, estimates value as $D_{1} /(r-g)$.

  \item The two stage dividend discount model estimates value as the sum of the present values of dividends over a short-term period of high growth and the present value of the terminal value at the end of the period of high growth. The terminal value is estimated using the Gordon growth model.

  \item The choice of dividend model is based upon the patterns assumed with respect to future dividends.

  \item Multiplier models typically use multiples of the form: $\mathrm{P} /$ measure of fundamental variable or $\mathrm{EV} /$ measure of fundamental variable.

  \item Multiples can be based upon fundamentals or comparables.

  \item Asset-based valuations models estimate value of equity as the value of the assets less the value of liabilities.

\end{itemize}

\section{REFERENCES}
Basu, S. 1977. "Investment Performance of Common Stocks in Relation to Their Price-Earnings Ratios: A Test of the Efficient Market Hypothesis." Journal of Finance, vol. 32, no. 3:663-682. $10.2307 / 2326304$

Block, S. 1999. "A Study of Financial Analysts: Practice and Theory. Financial Analysts Journal, vol. 55, no. 4:86-95. 10.2469/faj.v55.n4.2288

Dreman, D. 1977. Psychology of the Stock Market. New York: AMACOM.

Fama, E., K. French. 1995. "Size and Book-to-Market Factors in Earnings and Returns." Journal of Finance, vol. 50, no. 1:131-155. 10.2307/2329241

McWilliams, J. 1966. "Prices, Earnings and P-E Ratios." Financial Analysts Journal, vol. 22, no. $3: 137.10 .2469 /$ faj.v22.n3.137

Miller, P., E. Widmann. 1966. "Price Performance Outlook for High \& Low P/E Stocks." 1966 Stock \& Bond Issue, Commercial \& Financial Chronicle: 26-28.

Nicholson, S. 1968. "Price Ratios in Relation to Investment Results." Financial Analysts Journal, vol. 24, no. 1:105-109. 10.2469/faj.v24.n1.105

O'Shaughnessy, J. 2005. What Works on Wall Street. New York: McGraw-Hill.

\section{PRACTICE PROBLEMS}
\begin{enumerate}
  \item An analyst estimates the intrinsic value of a stock to be in the range of $€ 17.85$ to $€ 21.45$. The current market price of the stock is $€ 24.35$. This stock is most likely:
A. overvalued.
B. undervalued.
C. fairly valued.

  \item An analyst determines the intrinsic value of an equity security to be equal to $\$ 55$. If the current price is $\$ 47$, the equity is most likely:
A. undervalued.
B. fairly valued.
C. overvalued.

  \item In asset-based valuation models, the intrinsic value of a common share of stock is based on the:
A. estimated market value of the company's assets.
B. estimated market value of the company's assets plus liabilities.
C. estimated market value of the company's assets minus liabilities.

  \item Which of the following is most likely used in a present value model?
A. Enterprise value.
B. Price to free cash flow.
C. Free cash flow to equity.

  \item Book value is least likely to be considered when using:
A. a multiplier model.
B. an asset-based valuation model.
C. a present value model.

  \item An analyst is attempting to calculate the intrinsic value of a company and has gathered the following company data: EBITDA, total market value, and market value of cash and short-term investments, liabilities, and preferred shares. The analyst is least likely to use:
A. a multiplier model.
B. a discounted cash flow model.
C. an asset-based valuation model.

  \item An analyst who bases the calculation of intrinsic value on dividend-paying capacity rather than expected dividends will most likely use the:

\end{enumerate}

A. dividend discount model. B. free cash flow to equity model.

C. cash flow from operations model.

\begin{enumerate}
  \setcounter{enumi}{7}
  \item An investor expects to purchase shares of common stock today and sell them after two years. The investor has estimated dividends for the next two years, $D_{1}$ and $D_{2}$, and the selling price of the stock two years from now, $P_{2}$. According to the dividend discount model, the intrinsic value of the stock today is the present value of:
\end{enumerate}

A. next year's dividend, $D_{1}$.

B. future expected dividends, $D_{1}$ and $D_{2}$.

C. future expected dividends and price $-D_{1}, D_{2}$ and $P_{2}$.

\begin{enumerate}
  \setcounter{enumi}{8}
  \item In the free cash flow to equity (FCFE) model, the intrinsic value of a share of stock is calculated as:
\end{enumerate}

A. the present value of future expected FCFE.

B. the present value of future expected FCFE plus net borrowing.

C. the present value of future expected FCFE minus fixed capital investment.

\begin{enumerate}
  \setcounter{enumi}{9}
  \item With respect to present value models, which of the following statements is most accurate?
\end{enumerate}

A. Present value models can be used only if a stock pays a dividend.

B. Present value models can be used only if a stock pays a dividend or is expected to pay a dividend.

C. Present value models can be used for stocks that currently pay a dividend, are expected to pay a dividend, or are not expected to pay a dividend.

\begin{enumerate}
  \setcounter{enumi}{10}
  \item A Canadian life insurance company has an issue of 4.80 percent, $\$ 25$ par value, perpetual, non-convertible, non-callable preferred shares outstanding. The required rate of return on similar issues is 4.49 percent. The intrinsic value of a preferred share is closest to:
\end{enumerate}

A. $\$ 25.00$.

B. $\$ 26.75$.

C. $\$ 28.50$.

\begin{enumerate}
  \setcounter{enumi}{11}
  \item Two analysts estimating the value of a non-convertible, non-callable, perpetual preferred stock with a constant dividend arrive at different estimated values. The most likely reason for the difference is that the analysts used different:
\end{enumerate}

A. time horizons.

B. required rates of return.

C. estimated dividend growth rates.

\begin{enumerate}
  \setcounter{enumi}{12}
  \item The Beasley Corporation has just paid a dividend of $\$ 1.75$ per share. If the required rate of return is 12.3 percent per year and dividends are expected to grow indefinitely at a constant rate of 9.2 percent per year, the intrinsic value of Beasley Corporation stock is closest to:
A. $\$ 15.54$.
B. $\$ 56.45$.
C. $\$ 61.65$.

  \item An investor is considering the purchase of a common stock with a $\$ 2.00$ annual dividend. The dividend is expected to grow at a rate of 4 percent annually. If the investor's required rate of return is 7 percent, the intrinsic value of the stock is closest to:
A. $\$ 50.00$
B. $\$ 66.67$.
C. $\$ 69.33$

  \item The Gordon growth model can be used to value dividend-paying companies that are:
A. expected to grow very fast.
B. in a mature phase of growth.
C. very sensitive to the business cycle.

  \item Which of the following is most likely considered a weakness of present value models?

\end{enumerate}

A. Present value models cannot be used for companies that do not pay dividends

B. Small changes in model assumptions and inputs can result in large changes in the computed intrinsic value of the security.

C. The value of the security depends on the investor's holding period; thus, comparing valuations of different companies for different investors is difficult.

\begin{enumerate}
  \setcounter{enumi}{16}
  \item An analyst gathers or estimates the following information about a stock:
\end{enumerate}

\begin{center}
\begin{tabular}{lc}
Current price per share & $€ 22.56$ \\
Current annual dividend per share & $€ 1.60$ \\
Annual dividend growth rate for Years 1-4 & $9.00 \%$ \\
Annual dividend growth rate for Years 5+ & $4.00 \%$ \\
Required rate of return & $12 \%$ \\
\hline
\end{tabular}
\end{center}

Based on a dividend discount model, the stock is most likely:

A. undervalued.

B. fairly valued.

C. overvalued.

\begin{enumerate}
  \setcounter{enumi}{17}
  \item An analyst is attempting to value shares of the Dominion Company. The company has just paid a dividend of $\$ 0.58$ per share. Dividends are expected to grow by 20 percent next year and 15 percent the year after that. From the third year onward, dividends are expected to grow at 5.6 percent per year indefinitely. If the required rate of return is 8.3 percent, the intrinsic value of the stock is closest to:
A. $\$ 26.00$.
B. $\$ 27.00$.
C. $\$ 28.00$.

  \item Hideki Corporation has just paid a dividend of $¥ 450$ per share. Annual dividends are expected to grow at the rate of 4 percent per year over the next four years. At the end of four years, shares of Hideki Corporation are expected to sell for $¥ 9000$. If the required rate of return is 12 percent, the intrinsic value of a share of Hideki Corporation is closest to:
A. $¥ 5,850$.
B. $¥ 7,220$.
C. $¥ 7,670$.

  \item The best model to use when valuing a young dividend-paying company that is just entering the growth phase is most likely the:
A. Gordon growth model.
B. two-stage dividend discount model.
C. three-stage dividend discount model.

  \item An equity analyst has been asked to estimate the intrinsic value of the common stock of Omega Corporation, a leading manufacturer of automobile seats. Omega is in a mature industry, and both its earnings and dividends are expected to grow at a rate of 3 percent annually. Which of the following is most likely to be the best model for determining the intrinsic value of an Omega share?

\end{enumerate}

A. Gordon growth model.

B. Free cash flow to equity model.

C. Multistage dividend discount model.

\begin{enumerate}
  \setcounter{enumi}{21}
  \item A price earnings ratio that is derived from the Gordon growth model is inversely related to the:
A. growth rate.
B. dividend payout ratio.
C. required rate of return.

  \item The primary difference between $P / E$ multiples based on comparables and $P / E$ multiples based on fundamentals is that fundamentals-based P/Es take into account:

\end{enumerate}

A. future expectations.

B. the law of one price.

C. historical information. 24. An analyst makes the following statement: "Use of $\mathrm{P} / \mathrm{E}$ and other multiples for analysis is not effective because the multiples are based on historical data and because not all companies have positive accounting earnings.' The analyst's statement is most likely:

A. inaccurate with respect to both historical data and earnings.

B. accurate with respect to historical data and inaccurate with respect to earnings.

C. inaccurate with respect to historical data and accurate with respect to earnings.

\begin{enumerate}
  \setcounter{enumi}{24}
  \item An analyst has gathered the following information for the Oudin Corporation:
\end{enumerate}

Expected earnings per share $=€ 5.70$

Expected dividends per share $=€ 2.70$

Dividends are expected to grow at 2.75 percent per year indefinitely

The required rate of return is 8.35 percent

Based on the information provided, the price/earnings multiple for Oudin is closest to:
A. 5.7 .
B. 8.5 .
C. 9.4 .

\begin{enumerate}
  \setcounter{enumi}{25}
  \item An analyst has prepared a table of the average trailing twelve-month price-to-earning $(\mathrm{P} / \mathrm{E})$, price-to-cash flow $(\mathrm{P} / \mathrm{CF})$, and price-to-sales $(\mathrm{P} / \mathrm{S})$ for the Tanaka Corporation for the years 2014 to 2017.
\end{enumerate}

\begin{center}
\begin{tabular}{cccc}
\hline
Year & P/E & P/CF & P/S \\
\hline
2014 & 4.9 & 5.4 & 1.2 \\
2015 & 6.1 & 8.6 & 1.5 \\
2016 & 8.3 & 7.3 & 1.9 \\
2017 & 9.2 & 7.9 & 2.3 \\
\hline
\end{tabular}
\end{center}

As of the date of the valuation in 2018, the trailing twelve-month P/E, P/CF, and $\mathrm{P} / \mathrm{S}$ are, respectively, 9.2, 8.0, and 2.5. Based on the information provided, the analyst may reasonably conclude that Tanaka shares are most likely:

A. overvalued.

B. undervalued.

C. fairly valued.

\begin{enumerate}
  \setcounter{enumi}{26}
  \item An analyst gathers the following information about two companies:
\end{enumerate}

\begin{center}
\begin{tabular}{lcc}
\hline
 & Alpha Corp. & Delta Co. \\
\hline
Current price per share & $\$ 57.32$ & $\$ 18.93$ \\
Last year's EPS & $\$ 3.82$ & $\$ 1.35$ \\
Current year's estimated EPS & $\$ 4.75$ & $\$ 1.40$ \\
\hline
\end{tabular}
\end{center}

Which of the following statements is most accurate? A. Delta has the higher trailing P/E multiple and lower current estimated P/E multiple.

B. Alpha has the higher trailing P/E multiple and lower current estimated P/E multiple.

C. Alpha has the higher trailing P/E multiple and higher current estimated P/E multiple.

\begin{enumerate}
  \setcounter{enumi}{27}
  \item An analyst gathers the following information about similar companies in the banking sector:
\end{enumerate}

\begin{center}
\begin{tabular}{lccc}
\hline
 & First Bank & Prime Bank & Pioneer Trust \\
\hline
P/B & 1.10 & 0.60 & 0.60 \\
P/E & 8.40 & 11.10 & 8.30 \\
\hline
\end{tabular}
\end{center}

Which of the companies is most likely to be undervalued?

A. First Bank.

B. Prime Bank.

C. Pioneer Trust.

\begin{enumerate}
  \setcounter{enumi}{28}
  \item The market value of equity for a company can be calculated as enterprise value:
\end{enumerate}

A. minus market value of debt, preferred stock, and short-term investments.

B. plus market value of debt and preferred stock minus short-term investments.

C. minus market value of debt and preferred stock plus short-term investments.

\begin{enumerate}
  \setcounter{enumi}{29}
  \item Which of the following statements regarding the calculation of the enterprise value multiple is most likely correct?
\end{enumerate}

A. Operating income may be used instead of EBITDA.

B. EBITDA may not be used if company earnings are negative.

C. Book value of debt may be used instead of market value of debt.

\begin{enumerate}
  \setcounter{enumi}{30}
  \item An analyst has determined that the appropriate EV/EBITDA for Rainbow Company is 10.2 . The analyst has also collected the following forecasted information for Rainbow Company:
\end{enumerate}

EBITDA $=\$ 22,000,000$

Market value of debt $=\$ 56,000,000$

Cash $=\$ 1,500,000$

The value of equity for Rainbow Company is closest to:

A. $\$ 169$ million.

B. $\$ 224$ million.

C. $\$ 281$ million. 32. Enterprise value is most often determined as market capitalization of common equity and preferred stock minus the value of cash equivalents plus the:
A. book value of debt.
B. market value of debt.
C. market value of long-term debt.

\begin{enumerate}
  \setcounter{enumi}{32}
  \item A disadvantage of the EV method for valuing equity is that the following information may be difficult to obtain:
A. Operating income.
B. Market value of debt.
C. Market value of equity.

  \item Asset-based valuation models are best suited to companies where the capital structure does not have a high proportion of:
A. debt.
B. intangible assets.
C. current assets and liabilities.

  \item Which of the following is most likely a reason for using asset-based valuation?
A. The analyst is valuing a privately held company.
B. The company has a relatively high level of intangible assets.
C. The market values of assets and liabilities are different from the balance sheet values.

  \item Which type of equity valuation model is most likely to be preferable when one is comparing similar companies?
A. A multiplier model.
B. A present value model.
C. An asset-based valuation model.

\end{enumerate}

\section{SOLUTIONS}
\begin{enumerate}
  \item A is correct. The current market price of the stock exceeds the upper bound of the analyst's estimate of the intrinsic value of the stock.

  \item A is correct. The market price is less than the estimated intrinsic, or fundamental, value.

  \item $C$ is correct. Asset-based valuation models calculate the intrinsic value of equity by subtracting liabilities from the market value of assets.

  \item C is correct. FCFE can be used in a form of present value, or discounted cash flow, model. Both EV and price to free cash flow are forms of multiplier models.

  \item $\mathrm{C}$ is correct. Multiplier valuation models (in the form of $P / B$ ) and asset-based valuation models (in the form of adjustments to book value) use book value, whereas present value models typically discount future expected cash flows.

  \item B is correct. To use a discounted cash flow model, the analyst will require FCFE or dividend data. In addition, the analyst will need data to calculate an appropriate discount rate.

  \item B is correct. The FCFE model assumes that dividend-paying capacity is reflected in FCFE.

  \item C is correct. According to the dividend discount model, the intrinsic value of a stock today is the present value of all future dividends. In this case, the intrinsic value is the present value of $D_{1}, D_{2}$, and $P_{2}$. Note that $P_{2}$ is the present value at Period 2 of all future dividends from Period 3 to infinity.

  \item A is correct. In the FCFE model, the intrinsic value of stock is calculated by discounting expected future FCFE to present value. No further adjustments are required.

  \item $\mathrm{C}$ is correct. Dividend discount models can be used for a stock that pays a current dividend or a stock that is expected to pay a dividend. FCFE can be used for both of those stocks and for stocks that do not, or are not expected to, pay dividends in the near future. Both of these models are forms of present value models.

  \item B is correct. The expected annual dividend is $4.80 \% \times \$ 25=\$ 1.20$. The value of a preferred share is $\$ 1.20 / 0.0449=\$ 26.73$.

  \item B is correct. The required rate of return, $r$, can vary widely depending on the inputs and is not unique. A preferred stock with a constant dividend would not have a growth rate to estimate, and the investor's time horizon would have no effect on the calculation of intrinsic value.

  \item $C$ is correct. $P_{0}=D_{1} /(r-g)=1.75(1.092) /(0.123-0.092)=\$ 61.65$.

  \item $C$ is correct. According to the Gordon growth model, $V_{0}=D_{1} /(r-g)$. In this case, $D_{1}=\$ 2.00 \times 1.04=\$ 2.08$, so $V_{0}=\$ 2.08 /(0.07-0.04)=\$ 69.3333=\$ 69.33$.

  \item B is correct. The Gordon growth model (also known as the constant growth model) can be used to value dividend-paying companies in a mature phase of growth. A stable dividend growth rate is often a plausible assumption for such companies.

  \item B is correct. Very small changes in inputs, such as required rate of return or dividend growth rate, can result in large changes to the valuation model output. Some present value models, such as FCFE models, can be used to value companies without dividends. Also, the intrinsic value of a security is independent of the investor's holding period.

  \item A is correct. The current price of $€ 22.56$ is less than the intrinsic value $\left(V_{0}\right)$ of $€ 24.64$; therefore, the stock appears to be currently undervalued. According to the two-stage dividend discount model:

\end{enumerate}

$$
\begin{aligned}
& V_{0}=\sum_{t=1}^{n} \frac{D_{0}\left(1+g_{S}\right)^{t}}{(1+r)^{t}}+\frac{V_{n}}{(1+r)^{n}} \text { and } V_{n}=\frac{D_{n+1}}{r-g_{L}} \\
& D_{n+1}=D_{0}\left(1+g_{S}\right)^{n}\left(1+g_{L}\right) \\
& D_{1}=€ 1.60 \times 1.09=€ 1.744 \\
& D_{2}=€ 1.60 \times(1.09)^{2}=€ 1.901 \\
& D_{3}=€ 1.60 \times(1.09)^{3}=€ 2.072 \\
& D_{4}=€ 1.60 \times(1.09)^{4}=€ 2.259 \\
& D_{5}=\left[€ 1.60 \times(1.09)^{4}\right](1.04)=€ 2.349 \\
& V_{4}=€ 2.349 /(0.12-0.04)=€ 29.363 \\
& V_{0}=\frac{1.744}{(1.12)^{1}}+\frac{1.901}{(1.12)^{2}}+\frac{2.072}{(1.12)^{3}}+\frac{2.259}{(1.12)^{4}}+\frac{29.363}{(1.12)^{4}} \\
& =1.557+1.515+1.475+1.436+18.661 \\
& =€ 24.64(\text { which is greater than the current price of } € 22.56)
\end{aligned}
$$

\begin{enumerate}
  \setcounter{enumi}{17}
  \item $\mathrm{C}$ is correct.
\end{enumerate}

$$
\begin{aligned}
& V_{0}=\frac{D_{1}}{(1+r)}+\frac{D_{2}}{(1+r)^{2}}+\frac{P_{2}}{(1+r)^{2}} \\
= & \frac{0.70}{(1.083)}+\frac{0.80}{(1.083)^{2}}+\frac{31.29}{(1.083)^{2}} \\
= & \$ 28.01
\end{aligned}
$$

Note that $D_{1}=0.58(1.20)=0.70, D_{2}=0.58(1.20)(1.15)=0.80$, and $P_{2}=D_{3} /(k-g)$ $=0.80(1.056) /(0.083-0.056)=31.29$

\begin{enumerate}
  \setcounter{enumi}{18}
  \item B is correct.
\end{enumerate}

$$
\begin{aligned}
& V_{0}=\frac{D_{1}}{(1+r)}+\frac{D_{2}}{(1+r)^{2}}+\frac{D_{3}}{(1+r)^{3}}+\frac{D_{4}}{(1+r)^{4}}+\frac{P_{4}}{(1+r)^{4}} \\
= & \frac{468}{(1.12)}+\frac{486.72}{(1.12)^{2}}+\frac{506.19}{(1.12)^{3}}+\frac{526.44}{(1.12)^{4}}+\frac{9000}{(1.12)^{4}} \\
= & ¥ 7,220
\end{aligned}
$$

\begin{enumerate}
  \setcounter{enumi}{19}
  \item C is correct. The Gordon growth model is best suited to valuing mature companies. The two-stage model is best for companies that are transitioning from a growth stage to a mature stage. The three-stage model is appropriate for young companies just entering the growth phase.

  \item A is correct. The company is a mature company with a steadily growing dividend rate. The two-stage (or multistage) model is unnecessary because the dividend growth rate is expected to remain stable. Although an FCFE model could be used, that model is more often chosen for companies that currently pay no dividends.

  \item $C$ is correct. The justified forward $P / E$ is calculated as follows:

\end{enumerate}

\section{Equity Valuation: Concepts and Basic Tools}
$\frac{P_{0}}{E_{1}}=\frac{\frac{D_{1}}{E_{1}}}{r-g}$

$\mathrm{P} / \mathrm{E}$ is inversely related to the required rate of return, $r$, and directly related to the growth rate, $g$, and the dividend payout ratio, $D / E$.

\begin{enumerate}
  \setcounter{enumi}{22}
  \item A is correct. Multiples based on comparables are grounded in the law of one price and take into account historical multiple values. In contrast, P/E multiples based on fundamentals can be based on the Gordon growth model, which takes into account future expected dividends.

  \item A is correct. The statement is inaccurate in both respects. Although multiples can be calculated from historical data, forecasted values can be used as well. For companies without accounting earnings, several other multiples can be used. These multiples are often specific to a company's industry or sector and include price-to-sales and price-to-cash flow.

  \item B is correct.

\end{enumerate}

$$
\frac{P_{0}}{E_{1}}=\frac{\frac{D_{1}}{E_{1}}}{r-g}=\frac{\frac{2.7}{5.7}}{0.0835-0.0275}=8.5
$$

\begin{enumerate}
  \setcounter{enumi}{25}
  \item A is correct. Tanaka shares are most likely overvalued. As the table below shows, all the 2018 multiples are currently above their 2014-2017 averages.
\end{enumerate}

\begin{center}
\begin{tabular}{llll}
\hline
Year & P/E & P/CF & P/R \\
\hline
2014 & 4.9 & 5.4 & 1.2 \\
2015 & 6.1 & 8.6 & 1.5 \\
2016 & 8.3 & 7.3 & 1.9 \\
2017 & 9.2 & 7.9 & 2.3 \\
\hline
Average & 7.1 & 7.3 & 1.7 \\
\hline
\end{tabular}
\end{center}

\begin{enumerate}
  \setcounter{enumi}{26}
  \item $B$ is correct. $P / E=$ Current price/EPS, and Estimated $P / E=$ Current price/Estimated EPS.
\end{enumerate}

Alpha P/E $=\$ 57.32 / \$ 3.82=15.01$

Alpha estimated $\mathrm{P} / \mathrm{E}=\$ 57.32 / 4.75=12.07$

Delta $\mathrm{P} / \mathrm{E}=\$ 18.93 / \$ 1.35=14.02$

Delta estimated $\mathrm{P} / \mathrm{E}=\$ 18.93 / \$ 1.40=13.52$

\begin{enumerate}
  \setcounter{enumi}{27}
  \item $\mathrm{C}$ is correct. Relative to the others, Pioneer Trust has the lowest P/E multiple and the P/B multiple is tied for the lowest with Prime Bank. Given the law of one price, similar companies should trade at similar P/B and P/E levels. Thus, based on the information presented, Pioneer is most likely to be undervalued.

  \item C is correct. Enterprise value is calculated as the market value of equity plus the market value of debt and preferred stock minus short-term investments. Therefore, the market value of equity is enterprise value minus the market value of debt and preferred stock plus short-term investments.

  \item A is correct. Operating income may be used in place of EBITDA when calculating the enterprise value multiple. EBITDA may be used when company earnings are negative because EBITDA is usually positive. The book value of debt cannot be used in place of market value of debt. 31. A is correct.

\end{enumerate}

$\mathrm{EV}=10.2 \times 22,000,000=\$ 224,400,000$

Equity value $=\mathrm{EV}-$ Debt + Cash

$$
\begin{aligned}
& =224,400,000-56,000,000+1,500,000 \\
& =\$ 169,900,000
\end{aligned}
$$

\begin{enumerate}
  \setcounter{enumi}{31}
  \item B is correct. The market value of debt must be calculated and taken out of the enterprise value. Enterprise value, sometimes known as the cost of a takeover, is the cost of the purchase of the company, which would include the assumption of the company's debts at market value.

  \item B is correct. According to the reading, analysts may have not have access to market quotations for company debt.

  \item B is correct. Intangible assets are hard to value. Therefore, asset-based valuation models work best for companies that do not have a high proportion of intangible assets.

  \item A is correct. Asset-based valuations are most often used when an analyst is valuing private enterprises. Both $\mathrm{B}$ and $\mathrm{C}$ are considerations in asset-based valuations but are more likely to be reasons to avoid that valuation model rather than reasons to use it.

  \item A is correct. Although all models can be used to compare various companies, multiplier models have the advantage of reducing varying fundamental data points into a format that allows direct comparisons. As long as the analyst applies the data in a consistent manner for all the companies, this approach provides useful comparative data.

\end{enumerate}

\section{Fixed Income}
\section{LEARNING MODULE 1}
\section{Fixed-Income Securities: Defining Elements}
by Moorad Choudhry, PhD, FRM, FCSI, and Stephen E. Wilcox, PhD, CFA.

Moorad Choudhry PhD, FRM, FCSI is at Recognise Bank (England). Stephen E. Wilcox, PhD, CFA, is at Minnesota State University, Mankato (USA).

\section{LEARNING OUTCOME}
\begin{center}
\begin{tabular}{c|l}
Mastery & The candidate should be able to: \\
\hline
$\square$ & describe basic features of a fixed-income security \\
describe content of a bond indenture &  \\
$\square$ & $\begin{array}{l}\text { compare affirmative and negative covenants and identify examples of } \\ \text { each } \\ \text { describe how legal, regulatory, and tax considerations affect the } \\ \text { issuance and trading of fixed-income securities } \\ \text { describe how cash flows of fixed-income securities are structured } \\ \text { describe contingency provisions affecting the timing and/or nature } \\ \text { of cash flows of fixed-income securities and whether such provisions } \\ \text { benefit the borrower or the lender }\end{array}$ \\
\end{tabular}
\end{center}

\section{INTRODUCTION AND OVERVIEW OF A FIXED-INCOME}
 SECURITYdescribe basic features of a fixed-income security

Fixed-income securities constitute the most prevalent means of raising capital globally based on total market value. These instruments allow governments, companies, and other issuers to borrow from investors, promising future interest payments and the return of principal, which are contractual (legal) obligations of the issuer. Fixed-income securities are the largest source of capital for government, not-for-profit, and other entities that do not issue equity. For private companies, fixed-income investors differ from shareholders in not having ownership rights. Payments of interest and repayment of principal (amount borrowed) are a higher priority claim on the company's earnings and assets compared with the claim of common shareholders. Since fixed-income claims rank above shareholder claims in the capital structure, a company's fixed-income securities have, in theory, lower risk than their common shares.

Financial analysts who master these and other fixed-income concepts have a distinct edge over their peers for several reasons. First, given the nature of fixed-income cash flows and their preponderance across issuers and regions, these instruments form the basis for risk versus return comparisons both across and within specific jurisdictions. For example, as bonds issued by the US Treasury and other developed market central governments are viewed as having little to no default risk, they serve as building blocks in determining the time value of money for less certain cash flows. Fixed-income securities also fulfill an important role in portfolio management as a prime means by which individual and institutional investors can fund known future obligations, such as tuition payments or retirement obligations. Finally, while the correlation of fixed-income returns with common share returns varies, adding fixed-income securities to portfolios that include common shares can be an effective way of obtaining diversification benefits.

Among the questions to be addressed are the following:

\begin{itemize}
  \item Which features define a fixed-income security, and how do they determine the scheduled cash flows?

  \item What are the legal, regulatory, and tax considerations associated with a fixed-income security, and why are they important for investors?

  \item What are the common interest and principal payment structures?

  \item What types of provisions may affect the disposal or redemption of fixed-income securities?

\end{itemize}

Note that the terms "fixed-income securities," "debt securities," and "bonds" are often used interchangeably by experts and non-experts alike. We will also follow this convention, and where any nuance of meaning is intended, it will be made clear. Moreover, the term "fixed income" is not to be understood literally: Some fixed-income securities have interest payments that change over time.

\section{Overview of a Fixed-Income Security}
A bond is a contractual agreement between the issuer and the bondholders. Three important elements that an investor needs to know about when considering a fixed-income security are:

\begin{itemize}
  \item The bond's features, including the issuer, maturity, par value, coupon rate and frequency, and currency denomination. These features determine the bond's scheduled cash flows and, therefore, are key determinants of the investor's expected and actual return.

  \item The legal, regulatory, and tax considerations that apply to the contractual agreement between the issuer and the bondholders.

  \item The contingency provisions that may affect the bond's scheduled cash flows. These contingency provisions are options providing either issuers or bondholders certain rights affecting the bond's disposal or redemption.

\end{itemize}

This section describes a bond's basic features and introduces yield measures. The legal, regulatory, and tax considerations and contingency provisions are discussed in subsequent sections.

\section{Basic Features of a Bond}
All bonds, regardless of issuer, are characterized by the same basic features, which include maturity, par or principal amount, coupon size, frequency, and currency.

\section{Issuer}
Many entities issue bonds: private individuals, such as the musician David Bowie; national governments, such as Singapore or Italy; and companies, such as BP, General Electric, or Tata Group.

Bond issuers are classified into categories based on the similarities of these issuers and their characteristics. Major types of issuers include the following:

\begin{itemize}
  \item Supranational organizations, such as the World Bank or the European Investment Bank;

  \item Sovereign (national) governments, such as the United States or Japan. As sovereign bonds are backed by the full faith and credit of each respective government, they usually represent the lowest risk and most secure bonds in each market;

  \item Non-sovereign (local) governments, such as the State of Minnesota in the United States, the Catalonia region in Spain, or the City of Edmonton in Alberta, Canada;

  \item Quasi-government entities (i.e., agencies that are owned or sponsored by governments), such as postal services in many countries-for example, Correios in Brazil, La Poste in France, or Pos in Indonesia;

  \item Companies (i.e., corporate issuers). A distinction is often made between financial issuers (e.g., banks and insurance companies) and non-financial issuers; and

  \item Special legal entities (i.e., special purpose entities) that use specific assets, such as auto loans and credit card debt obligations, to guarantee (or secure) a bond issue known as an asset-backed security that is then sold to investors.

\end{itemize}

Market participants often classify fixed-income markets by the type of issuer, which leads to the identification of three bond market sectors: the government and government-related sector (i.e., the first four types of issuers just listed), the corporate sector (the fifth type listed), and the structured finance sector (the last type listed).

While several major local fixed-income markets, such as China and Japan, are dominated by government issuers, a significant portion of the US bond market consists of corporate issuance and asset-backed securities in addition to US treasury securities issued by the federal government.

Asset-backed securities (ABS) are created from a process called securitization, which involves moving assets from the owner of the assets into a special legal entity. This special legal entity then uses the securitized assets as guarantees to back (secure) a bond issue, leading to the creation of ABS. Assets that are typically used to create ABS include residential and commercial mortgage loans (mortgages), automobile (auto) loans, student loans, bank loans, and credit card debt. Many elements discussed in this reading apply to both traditional bonds and ABS. Considerations specific to ABS are discussed in a separate reading on asset-backed securities.

Bondholders are exposed to credit risk-that is, the risk of loss resulting from the issuer failing to make full and timely payments of interest and/or repayments of principal. Credit risk is inherent to all debt investments. Bond markets are sometimes classified into sectors based on the issuer's creditworthiness as judged by credit rating agencies. The three largest credit rating agencies are Moody's Investors Service, Standard \& Poor's, and Fitch Ratings. One major distinction is between investment-grade and non-investment-grade bonds, also called high-yield or speculative bonds. For example, bonds rated Baa3 or higher by Moody's and BBB- or higher by Standard \& Poor's and Fitch are considered investment grade. Although a variety of considerations enter into distinguishing the two sectors, the promised payments of investment-grade bonds are perceived as less risky than those of non-investment-grade bonds because of profitability and liquidity considerations. Some regulated financial intermediaries, such as banks and life insurance companies, may face explicit or implicit limitations of holdings of non-investment-grade bonds. The investment policy statements of some investors may also include constraints or limits on such holdings. From the issuer's perspective, an investment-grade credit rating generally allows easier access to bond markets and at lower interest rates than does a non-investment-grade credit rating.

\section{Maturity}
The maturity date of a bond refers to the date when the issuer is obligated to redeem the bond by paying the outstanding principal amount. The tenor is the time remaining until the bond's maturity date. Tenor is an important consideration in analyzing a bond's risk and return, as it indicates the period over which the bondholder can expect to receive interest payments and the length of time until the principal is repaid in full.

Maturities typically range from overnight to 30 years or longer. Fixed-income securities with maturities at issuance (original maturity) of one year or less are known as money market securities. Issuers of money market securities include governments and companies. Commercial paper and certificates of deposit are examples of money market securities. Fixed-income securities with original maturities that are longer than one year are called capital market securities. Although very rare, perpetual bonds, such as the consols issued by the sovereign government in the United Kingdom, have no stated maturity date.

\section{Par Value}
The principal amount, principal value, or simply principal of a bond is the amount that the issuer agrees to repay the bondholders on the maturity date. This amount is also referred to as the par value, or simply par, face value, nominal value, redemption value, or maturity value. Bonds can have any par value.

In practice, bond prices are quoted as a percentage of their par value. For example, assume that a bond's par value is $\$ 1,000$. A quote of 95 means that the bond price is $\$ 950(95 \% \times \$ 1,000)$. When the bond is priced at $100 \%$ of par, the bond is said to be trading at par. If the bond's price is below $100 \%$ of par, such as in the previous example, the bond is trading at a discount. Alternatively, if the bond's price is above $100 \%$ of par, the bond is trading at a premium.

\section{Coupon Rate and Frequency}
The coupon rate or nominal rate of a bond is the interest rate that the issuer agrees to pay each year until the maturity date. The annual amount of interest payments made is called the coupon. A bond's coupon is determined by multiplying its coupon rate by its par value. For example, a bond with a coupon rate of $6 \%$ and a par value of $\$ 1,000$ will pay annual interest of $\$ 60(6 \% \times \$ 1,000)$.

Coupon payments may be made annually, such as those for German government bonds or Bunds. Many bonds, such as government and corporate bonds issued in the United States or government gilts issued in the United Kingdom, pay interest semi-annually. Some bonds make quarterly or monthly interest payments. The acronyms QUIBS (quarterly interest bonds) and QUIDS (quarterly income debt securities) are used by Morgan Stanley and Goldman Sachs, respectively, for bonds that make quarterly interest payments. Many mortgage-backed securities (MBS), which are ABS backed by residential or commercial mortgages, pay interest monthly to match the cash flows of the underlying assets. If a bond has a coupon rate of $6 \%$ and a par value of $\$ 1,000$, the periodic interest payments will be $\$ 60$ if coupon payments are made annually, $\$ 30$ if they are made semi-annually, $\$ 15$ if they are made quarterly, and $\$ 5$ if they are made monthly.

A plain vanilla bond or conventional bond pays a fixed rate of interest. In this case, the coupon payment does not change during the bond's life. However, there are bonds that pay a floating rate of interest; such bonds are called floating-rate notes (FRNs) or floaters. The coupon rate of an FRN includes two components: a market reference rate (MRR) plus a spread. The spread, also called margin, is typically constant and is expressed in basis points (bps). A basis point is equal to $0.01 \%$ (i.e., 100 basis points equals 1\%). The spread is set when the bond is issued based on the issuer's creditworthiness at issuance: The higher the issuer's credit quality, the lower the spread. The MRR, however, resets periodically. Thus, as the MRR changes, the coupon rate and coupon payment change accordingly.

The market reference rate is a collective name for a set of rates covering different currencies for different maturities, ranging from overnight to one year. These rates have historically included the London Interbank Offered Rate (Libor), the Euro Interbank Offered Rate (Euribor), the Hong Kong Interbank Offered Rate (Hibor), or the Singapore Interbank Offered Rate (Sibor) for issues denominated in US dollars, euros, Hong Kong dollars, and Singapore dollars, respectively. The process of phasing out Libor and moving to new market reference rates is discussed in a subsequent CFA Level I reading.

For example, assume that the coupon rate of an FRN that makes semi-annual interest payments in June and December is expressed as the six-month MRR +150 bps. Suppose that in December 20X0, the six-month MRR is 3.25\%. The interest rate that will apply to the payment due in June $20 \mathrm{X} 1$ will be $4.75 \%(3.25 \%+1.50 \%)$. Now suppose that in June 20X1, the six-month MRR has decreased to 3.15\%. The interest rate that will apply to the payment due in December $20 \mathrm{X} 1$ will decrease to $4.65 \%$ $(3.15 \%+1.50 \%)$

All bonds, whether they pay a fixed or floating rate of interest, make periodic coupon payments except for zero-coupon bonds. Zero-coupon, or pure discount bonds, are issued at a discount to par value and are redeemed at par. The interest earned on a zero-coupon bond is implied and equal to the difference between the par value and the purchase price. For example, if the par value is $\$ 1,000$ and the purchase price is $\$ 950$, the implied interest is $\$ 50$.

\section{Currency Denomination}
Bonds can be issued in any currency, although a large number of bond issues are made in either euros or US dollars. The currency of issue may affect a bond's attractiveness. Borrowers in emerging markets often elect to issue bonds in euros or US dollars because doing so makes the bonds more attractive to international investors than bonds in a domestic currency that may be illiquid or not freely traded. Issuers may also choose to borrow in a foreign currency if they expect cash flows in that currency to offset interest payments and principal repayments. If a bond is aimed solely at a country's domestic investors, it is more likely that the borrower will issue in the local currency.

Dual-currency bonds make coupon payments in one currency and pay the par value at maturity in another currency. For example, assume that a Japanese company needs to finance a long-term project in the United States that will take several years to become profitable. The Japanese company could issue a yen/US dollar dual-currency bond. The coupon payments in yen can be made from the cash flows generated in Japan, and the principal can be repaid in dollars once the project becomes profitable.

Currency option bonds can be viewed as a combination of a single-currency bond plus a foreign currency option. They give bondholders the right to choose the currency in which they want to receive interest payments and principal repayments. Bondholders can select one of two currencies for each payment. Exhibit 1 brings all the basic features of a bond together and illustrates how these features determine the cash flow pattern for a plain vanilla bond. The bond is a five-year Japanese government bond (JGB) with a coupon rate of $0.4 \%$ and a par value of $¥ 10,000$. Interest payments are made semi-annually. The bond is priced at par when it is issued and is redeemed at par.

\section{Exhibit 1: Cash Flows for a Plain Vanilla Bond}
\begin{center}
\includegraphics[max width=\textwidth]{2023_05_04_7b535d0a870224f62e3dg-450}
\end{center}

The downward-pointing arrow in Exhibit 1 represents the cash flow paid by the bond investor (received by the issuer) on the day of the bond issue-that is, $¥ 10,000$. The upward-pointing arrows are the cash flows received by the bondholder (paid by the issuer) during the bond's life. As interest is paid semi-annually, the coupon payment is $¥ 20[(0.004 \times ¥ 10,000) \div 2]$ every six months for five years-that is, 10 coupon payments of $¥ 20$. The last payment is equal to $¥ 10,020$ because it includes both the last coupon payment and the payment of the par value.

\section{EXAMPLE 1}
\begin{enumerate}
  \item An example of a sovereign bond is a bond issued by:
\end{enumerate}

A. the World Bank.

B. the city of New York.

C. the federal German government.

Solution:

$\mathrm{C}$ is correct. A sovereign bond is a bond issued by a national government, such as the federal German government. $\mathrm{A}$ is incorrect because a bond issued by the World Bank is a supranational bond. B is incorrect because a bond issued by a local government, such as the city of New York, is a non-sovereign bond.

\begin{enumerate}
  \setcounter{enumi}{1}
  \item The risk of loss resulting from the issuer failing to make full and timely payment of interest is called:
\end{enumerate}

A. credit risk.

B. systemic risk.

C. interest rate risk.

\section{Solution:}
A is correct. Credit risk is the risk of loss resulting from the issuer failing to make full and timely payments of interest and/or repayments of principal. B is incorrect because systemic risk is the risk of failure of the financial system. $\mathrm{C}$ is incorrect because interest rate risk is the risk that a change in market interest rate affects a bond's value.

\begin{enumerate}
  \setcounter{enumi}{2}
  \item A money market security most likely matures in:
A. one year or less.
B. between 1 and 10 years.
C. over 10 years.
\end{enumerate}

\section{Solution:}
A is correct. The primary difference between a money market security and a capital market security is the maturity at issuance. Money market securities mature in one year or less, whereas capital market securities mature in more than one year.

\begin{enumerate}
  \setcounter{enumi}{3}
  \item If the bond's price is higher than its par value, the bond is trading at:
A. par.
B. a discount.
C. a premium.
\end{enumerate}

\section{Solution:}
$\mathrm{C}$ is correct. If a bond's price is higher than its par value, the bond is trading at a premium. A is incorrect because a bond is trading at par if its price is equal to its par value. $B$ is incorrect because a bond is trading at a discount if its price is lower than its par value.

\begin{enumerate}
  \setcounter{enumi}{4}
  \item A bond has a par value of $\pounds 100$ and a coupon rate of $5 \%$. Coupon payments are made semi-annually. The periodic interest payment is:
A. $\pounds 2.50$, paid twice a year.
B. $\pounds 5.00$, paid once a year.
C. $\pounds 5.00$, paid twice a year.
\end{enumerate}

\section{Solution:}
A is correct. The annual coupon payment is $5 \% \times \pounds 100=\pounds 5.00$. The coupon payments are made semi-annually, so $\pounds 2.50$ paid twice a year.

\begin{enumerate}
  \setcounter{enumi}{5}
  \item The coupon rate of a floating-rate note that makes payments in June and December is expressed as six-month MRR + $25 \mathrm{bps}$. Assuming that the six-month MRR is $3.00 \%$ at the end of June $20 \mathrm{XX}$ and $3.50 \%$ at the end of December 20XX, the interest rate that applies to the payment due in December $20 \mathrm{XX}$ is:
A. $3.25 \%$.
B. $3.50 \%$
C. $3.75 \%$
\end{enumerate}

\section{Solution:}
$\mathrm{A}$ is correct. The interest rate that applies to the payment due in December $20 \mathrm{XX}$ is the six-month MRR at the end of June $20 \mathrm{XX}$ plus $25 \mathrm{bps}$. Thus, it is $3.25 \%(3.00 \%+0.25 \%)$.

\begin{enumerate}
  \setcounter{enumi}{6}
  \item The type of bond that allows bondholders to choose the currency in which they receive each interest payment and principal repayment is a:
\end{enumerate}

A. pure discount bond.

B. dual-currency bond.

C. currency option bond.

Solution:

$\mathrm{C}$ is correct. A currency option bond gives bondholders the right to choose the currency in which they want to receive each interest payment and principal repayment. A is incorrect because a pure discount bond is issued at a discount to par value and redeemed at par. $B$ is incorrect because a dual-currency bond makes coupon payments in one currency and pays the par value at maturity in another currency.

\section{Yield Measures}
Several yield measures are commonly used by market participants. The current yield or running yield is equal to the bond's annual coupon divided by the bond's price, expressed as a percentage. For example, if a bond has a coupon rate of $6 \%$, a par value of $\$ 1,000$, and a price of $\$ 1,010$, the current yield is $5.94 \%(\$ 60 \div \$ 1,010)$. The current yield is a measure of income that is analogous to the dividend yield for a common share.

The most common yield measure is known as the yield-to-maturity, also called the yield-to-redemption or redemption yield. The yield-to-maturity is the internal rate of return on a bond's expected cash flows-that is, the discount rate that equates the present value of the bond's expected cash flows until maturity with the bond's price. The yield-to-maturity can be considered an estimate of the bond's expected return; it reflects the annual return earned by an investor who purchases the bond today and holds it until maturity, provided they receive all promised cash flows and are able to reinvest all coupons at this same yield. There is an inverse relationship between the bond's price and its yield-to-maturity, all else being equal. That is, the higher the bond's yield-to-maturity, the lower its price. Alternatively, the higher the bond's price, the lower its yield-to-maturity. Thus, investors expecting interest rates to fall (and investors to demand a lower yield-to-maturity) anticipate a positive return from price appreciation. The topic of risk and return of fixed-income securities is covered in more detail in a subsequent reading.

\section{BOND INDENTURE}
describe content of a bond indenture

compare affirmative and negative covenants and identify examples of each As contractual agreements between the issuer and the bondholders, bonds are subject to legal, regulatory, and tax considerations that are important to fixed-income investors.

\section{Bond Indenture}
The trust deed is the legal contract that describes the form of the bond, the obligations of the issuer, and the rights of the bondholders. This legal contract is often referred to as the bond indenture, particularly in the United States and Canada. The indenture references both the issuer and the features of the bond issue, such as the principal value, the interest or coupon rate, the interest payment dates, the maturity date, and any contingency provisions. The indenture also identifies the funding sources for interest and principal payments and any collateral, credit enhancements, or covenants. Collateral is a specific asset or assets or financial guarantees securing the debt obligation above and beyond the issuer's promise to pay. Credit enhancements are provisions used to reduce the credit risk of the bond issue. Covenants are clauses that specify the rights of the bondholders and any actions that the issuer is obligated to perform or prohibited from performing.

Because it would be impractical for the issuer to enter into a direct agreement with each bondholder, the indenture is usually held by a trustee. The trustee is typically a financial institution acting as a fiduciary (or representative protecting investor interests) to ensure the issuer complies with the obligations specified in the indenture and to take action on behalf of the bondholders, when necessary. The trustee's duties are administrative and include the maintenance of required documentation and records; holding beneficial title to, safeguarding, and appraising collateral (if any); invoicing the issuer for interest payments and principal repayments; and holding funds until they are paid, although the cash flow movements from issuers to the trustee are typically handled by the principal paying agent. In the event of default, the discretionary powers of the trustee increase considerably. For example, the trustee is responsible for calling bondholder meetings to discuss actions as well as bringing legal action against an issuer on behalf of bondholders, if necessary.

For a plain vanilla bond, the indenture is often a standard template that is updated for a bond's specific terms and conditions. For exotic bonds, the document is tailored and can often be several hundred pages.

The indenture identifies basic bond features (e.g., investor rights in the event of default) that impact the bond's risk-reward profile, such as:

\begin{itemize}
  \item the legal identity of the bond issuer and its legal form;

  \item the source of repayment proceeds;

  \item the asset or collateral backing (if any);

  \item the credit enhancements (if any); and

  \item the covenants (if any).

\end{itemize}

We consider each of these areas in the following sections.

\section{Legal Identity of the Bond Issuer and Its Legal Form}
The bond issuer, identified in the indenture by its legal name, must make all contractual payments assigned to the bond issuer. For a sovereign bond, the legal issuer is usually the office responsible for managing the national budget, such as HM Treasury (Her Majesty's Treasury) in the United Kingdom. The legal issuer may be different from the body that administers the bond issue process. Using the UK example, the legal obligation to repay gilts lies with HM Treasury, but the bonds are issued by the UK Debt Management Office, an executive agency of HM Treasury. For corporate bonds, the issuer is usually the corporate legal entity-for example, Wal-Mart Stores Inc., Samsung Electronics Co. Ltd., or Volkswagen AG. However, bonds are sometimes issued by a subsidiary of a parent legal entity. In this case, investors should consider the credit quality of the subsidiary, unless the indenture specifies that the bond liabilities are guaranteed by the parent. When rated, subsidiaries often receive a credit rating below that of the parent.

Bonds are sometimes issued by a holding company, or parent legal entity for a group of companies, rather than by one of the operating companies in the group. A holding company may be rated differently from its operating companies, and investors may lack recourse to assets held by those companies. If the bonds are issued by a holding company with few assets to call on in the event of default, investors face more credit risk than if the bonds were issued by an operating company in the group.

For ABS, the legal obligation to repay the bondholders often lies with the special legal entity created by the sponsor or originator, a financial institution in charge of the securitization process. The special legal entity is most frequently referred to as a special purpose entity (SPE) in the United States and a special purpose vehicle (SPV) in Europe, also called a special purpose company (SPC). The legal form for the special legal entity may be a limited partnership, a limited liability company, or a trust. Typically, special legal entities are thinly capitalized, have no independent management or employees, and have no purpose other than the transactions for which they were created.

The sponsor transfers the assets to the special legal entity to carry out a securitization or series of securitizations. A key reason for forming a special legal entity is bankruptcy remoteness. The transfer of assets by the sponsor is considered a legal sale; once the assets have been securitized, the sponsor no longer has ownership rights. Any party making claims following the bankruptcy of the sponsor would be unable to recover the assets or their proceeds. As a result, the special legal entity's ability to pay interest and repay the principal should remain intact even if the sponsor were to fail-hence the reason why the special legal entity is also called a bankruptcy-remote vehicle.

\section{Source of Repayment Proceeds}
The indenture usually describes how the issuer intends to service the debt (make interest payments) and repay the principal. Generally, the source of repayment for bonds issued by supranational organizations is either the repayment of previous loans made by the organization or the paid-in capital from its members. National governments may also act as guarantors for certain bond issues. If additional sources of repayment are needed, the supranational organization may call on its members to provide funds.

Sovereign bonds are backed by the "full faith and credit" of the national government. As national governments have unique powers to ensure the ability to repay debt-such as the authority to tax economic activity, print money, and control foreign currency reserves-sovereign bonds denominated in local currency are generally considered the safest of all investments. It is therefore highly probable that interest and principal will be paid fully and on time, resulting in lower yields on sovereign bonds than for similar bonds from other local issuers. Government yields are often used as a "risk-free rate" for time value of money calculations as well as a benchmark reference rate for pricing fixed-income securities from other issuers of the same tenor.

There are three major sources for repayment of non-sovereign government debt issues, and bonds are usually classified according to these sources. The first source is through the general taxing authority of the issuer. The second source is from the cash flows of the project the bond issue is financing. The third source is from special taxes or fees established specifically for the purpose of funding interest payments and principal repayments. The source of payment for corporate bonds is the issuer's ability to generate cash flows, primarily through its operations. These cash flows depend on the issuer's financial strength and integrity. Because corporate bonds carry a higher level of credit risk than otherwise similar sovereign and non-sovereign government bonds, they typically offer a higher yield.

In contrast to corporate bonds, the source of payment for ABS depends on the cash flows generated by one or more of the underlying financial assets, such as mortgages or auto loans, rather than an operating entity. Thus, investors in ABS must pay special attention to the quality of these assets.

\section{Asset or Collateral Backing}
Collateral backing is a way to alleviate credit risk. Investors should review where they rank compared with other creditors in the event of default and analyze the quality of the collateral backing the bond issue.

\section{Seniority Ranking}
Secured bonds are backed by assets or financial guarantees pledged to ensure debt repayment in the case of default. In contrast, unsecured bonds have no collateral; bondholders have only a general claim on the issuer's assets and cash flows. Thus, unsecured bonds are paid after secured bonds in the event of default. By lowering credit risk, collateral backing increases the bond issue's credit quality and decreases its yield.

A bond's collateral backing might not specify an identifiable asset but instead may be described as the "general plant and infrastructure" of the issuer. In such cases, investors rely on seniority ranking-that is, the systematic way in which lenders are repaid in case of bankruptcy or liquidation. What matters to investors is where they rank compared with other creditors rather than whether an asset of sufficient quality and value is in place to cover their claims. Senior debt has a priority claim over subordinated debt or junior debt. Financial institutions issue a large volume of both senior unsecured and subordinated bonds globally; it is not uncommon to see large and small banks issue such bonds. For example, banks as diverse as Royal Bank of Scotland in the United Kingdom and Prime Bank in Bangladesh issue senior unsecured bonds to institutional investors.

Debentures are a type of bond that can be secured or unsecured. In many jurisdictions, debentures are unsecured bonds, with no collateral backing assigned to the bondholders. In contrast, in the United Kingdom and in other Commonwealth countries, such as India, debentures are usually backed by an asset or pool of assets assigned as collateral support for the bond obligations and segregated from other creditor claims. Thus, it is important for investors to review the indenture to determine whether a debenture is secured or unsecured. If the debenture is secured, debenture holders rank above unsecured creditors of the company; they have a specific asset or pool of assets that the trustee can call on to realize the debt in the event of default.

\section{Types of Collateral Backing}
Many types of bonds are secured by a form of collateral. Some companies issue collateral trust bonds and equipment trust certificates. Collateral trust bonds are backed by securities, such as common shares, other bonds, or other financial assets. These securities are pledged by the issuer and typically held by the trustee. Equipment trust certificates are bonds secured by specific types of equipment or physical assets, such as aircraft, railroad cars, shipping containers, or oil rigs. They are often issued to take advantage of the tax benefits of leasing. For example, suppose an airline finances the purchase of new aircraft with equipment trust certificates. The legal title to the aircraft is held by the trustee, which issues equipment trust certificates to investors in the amount of the aircraft purchase price. The trustee leases the aircraft to the airline and collects lease payments from the airline to pay the interest on the certificates. When the certificates mature, the trustee sells the aircraft to the airline, uses the proceeds to retire the principal, and cancels the lease.

Mortgaged property is one of the most common forms of ABS collateral. MBS are debt obligations that represent claims to the cash flows from pools of mortgage loans, most commonly on residential property. Mortgage loans are purchased from banks, mortgage companies, and other originators and then assembled into pools by a governmental, quasi-governmental, or private entity.

Financial institutions, particularly in Europe, issue covered bonds. Covered bonds are debt obligations backed by a segregated pool of assets. They are similar to ABS but offer bondholders recourse against both the financial institution and the underlying asset pool. Covered bonds will be addressed in detail in a later reading.

\section{Credit Enhancements}
Credit enhancements refer to a variety of provisions that can be used to reduce the credit risk of a bond issue. Thus, they increase the issue's credit quality and decrease the bond's yield. Credit enhancements are very often used when creating ABS.

The two primary types of credit enhancements are internal and external. Internal credit enhancement relies on structural features regarding the bond issue. External credit enhancement refers to financial guarantees received from a third party, often called a financial guarantor. We describe each type in the following sections.

\section{Internal Credit Enhancement}
The most common forms of internal credit enhancement are subordination, overcollateralization, and reserve accounts.

Subordination, also known as credit tranching, is the most popular internal credit enhancement technique. It relies on creating more than one bond class or tranche and ordering the claim priorities for ownership or interest in an asset between the tranches. The cash flows generated by the assets are allocated with different priority to tranches of different seniority. The ordering of the claim priorities is called a senior/ subordinated structure, where the tranches of highest seniority are senior followed by subordinated or junior tranches. The subordinated tranches function as credit protection for the more senior tranches, in the sense that the most senior tranche has the first claim on available cash flows. This type of protection is also commonly referred to as a waterfall structure because in the event of default, the proceeds from liquidating assets will first be used to repay the most senior creditors. Thus, if the issuer defaults, losses are allocated from the bottom up-that is, from the most junior to the most senior tranche. The most senior tranche is typically unaffected unless losses exceed the amount of the subordinated tranches, which is why the most senior tranche is usually rated Aaa/AAA.

Overcollateralization refers to the process of posting more collateral than is needed to obtain or secure financing. It represents a form of internal credit enhancement because the additional collateral can be used to absorb losses. For example, a bond issue of $\$ 100$ million with collateral value of $\$ 110$ million has excess collateral of $\$ 10$ million. Over time, the amount of overcollateralization changes. This can happen because of amortization, prepayments, or defaults as well as changes in collateral values. For example, one of the most significant contributors to the 2007-2009 global financial crisis was a valuation problem with the residential housing assets backing MBS. While many properties were originally valued above the value of securities outstanding, as property prices fell and the number of defaults rose, the credit quality of these MBS declined sharply. This resulted in a rapid rise in yields and losses among investors in these securities. Reserve accounts or reserve funds are another internal credit enhancement in the form of either a cash reserve fund or an excess spread account. A cash reserve fund is a cash deposit used to absorb losses. An excess spread account is an allocation of any asset cash flows remaining after paying interest to bondholders. The excess spread (or excess interest cash flow) can be retained and deposited into a reserve account as a first line of protection against losses. In a process called "turboing," the excess spread can be used to retire the principal, with the most senior tranche having the first claim on these funds.

\section{External Credit Enhancement}
The most common forms of external credit enhancement are bank guarantees and surety bonds, letters of credit, and cash collateral accounts.

Bank guarantees and surety bonds are very similar in nature because they both reimburse bondholders for any losses incurred if the issuer defaults. However, there is usually a maximum amount that is guaranteed, called the penal sum. The major difference between a bank guarantee and a surety bond is that the former is issued by a bank, whereas the latter is issued by a rated and regulated insurance company. Insurance companies that specialize in providing financial guarantees are typically called monoline insurance companies or monoline insurers. Monoline insurers played an important role in securitization until the 2007-2009 global financial crisis, but they are now a less common form of credit enhancement due to the credit downgrades of these insurers during the crisis.

A letter of credit from a financial institution is another form of external credit enhancement for a bond issue. The financial institution provides the issuer with a credit line to reimburse any cash flow shortfalls from the assets backing the issue. Letters of credit have also become a less common form of credit enhancement since the credit downgrades of financial institutions during the financial crisis.

Bank guarantees, surety bonds, and letters of credit expose the investor to third-party (or counterparty) risk-that is, the possibility that a guarantor cannot meet its obligations. A cash collateral account mitigates this concern because the issuer immediately borrows the credit enhancement amount and usually invests it in highly rated short-term commercial paper. Because a cash collateral account is an actual deposit rather than a pledge of cash, a cash collateral account provider downgrade will not necessarily result in a downgrade of the bond issue backed by that provider.

\section{Covenants}
Bond covenants are legally enforceable rules that borrowers and lenders agree on at the time of a new bond issue. An indenture will frequently include affirmative (or positive) and negative covenants. Affirmative covenants enumerate what issuers are required to do, whereas negative covenants specify what issuers are prohibited from doing.

Affirmative covenants are typically administrative in nature. For example, frequently used affirmative covenants include what the issuer will do with the proceeds from the bond issue and the promise of making the contractual payments. The issuer may also promise to comply with all laws and regulations, maintain its current lines of business, insure and maintain its assets, and pay taxes as they come due. Other examples include a pari passu (or "equal footing") clause, which ensures that a debt obligation is treated the same as the borrower's other senior debt instruments, or a cross-default clause, which specifies that a borrower is considered in default if they default on another debt obligation. These types of covenants typically do not impose additional costs to the issuer and do not materially constrain the issuer's discretion regarding how to operate its business. In contrast, negative covenants are frequently costly and materially constrain the issuer's potential business decisions. They protect bondholders from the dilution of their claims, asset withdrawals or substitutions, and suboptimal investments by the issuer. Examples of negative covenants include the following:

\begin{itemize}
  \item Restrictions on debt regulate the issue of additional debt. Maximum acceptable debt usage ratios (sometimes called leverage ratios or gearing ratios) and minimum acceptable interest coverage ratios are frequently specified, permitting new debt to be issued only when justified by the issuer's financial condition.

  \item Negative pledges prevent the issuance of debt that would be senior to or rank in priority ahead of the existing bondholders' debt.

  \item Restrictions on prior claims protect unsecured bondholders by preventing the issuer from using assets that are not collateralized (called unencumbered assets) to become collateralized.

  \item Restrictions on distributions to shareholders restrict dividends and other payments to shareholders, such as share buybacks (repurchases). The restriction typically operates by reference to the borrower's profitability; that is, the covenant sets a base date, usually at or near the time of the issue, limiting dividends and share buybacks to a percentage of earnings or cumulative earnings after that date.

  \item Restrictions on asset disposals limit the amount of assets that can be disposed by the issuer during the bond's life. This limit on cumulative disposals as a percentage of a company's gross assets is intended to protect bondholder claims by preventing a break-up of the company.

  \item Restrictions on investments constrain risky investments by blocking speculative investments and ensuring an issuer devotes its capital to its going-concern business. A companion covenant may require the issuer to stay in its present line of business.

  \item Restrictions on mergers and acquisitions prevent these actions unless the company is the surviving company or the acquirer delivers a supplemental indenture to the trustee expressly assuming the old bonds and terms of the old indenture. These requirements effectively prevent a company from avoiding its obligations to bondholders by selling to another company.

\end{itemize}

These are only a few examples of negative covenants. The common characteristic of all negative covenants is ensuring that the issuer will not take any actions that would significantly reduce its ability to make interest payments and repay the principal. Bondholders, however, rarely wish to be too specific about how an issuer should run its business because doing so would imply a degree of control that bondholders legally want to avoid. In addition, very restrictive covenants may not be in the bondholders' best interest if they force the issuer to default when default is avoidable. For example, strict restrictions on debt may prevent the issuer from raising new funds that are necessary to meet its contractual obligations; strict restrictions on asset disposals may prohibit the issuer from selling assets or business units and obtaining the necessary liquidity to make interest payments or principal repayments; and strict restrictions on mergers and acquisitions may prevent the issuer from being taken over by a stronger company that would be able to honor the issuer's contractual obligations.

\section{EXAMPLE 2}
\begin{enumerate}
  \item The term most likely used to refer to the legal contract under which a bond is issued is:
A. indenture.
B. debenture
C. letter of credit.
\end{enumerate}

\section{Solution:}
A is correct. The contract between a bond issuer and the bondholders is very often called an indenture or deed trust. The indenture documents the terms of the issue, including the principal amount, the coupon rate, and the payments schedule. It also provides information about the funding sources for the contractual payments and specifies whether there are any collateral, credit enhancement, or covenants. $\mathrm{B}$ is incorrect because a debenture is a type of bond. $\mathrm{C}$ is incorrect because a letter of credit is an external credit enhancement.

\begin{enumerate}
  \setcounter{enumi}{1}
  \item The individual or entity that most likely assumes the role of trustee for a bond issue is:
A. a financial institution appointed by the issuer.
B. the treasurer or chief financial officer of the issuer.
C. a financial institution appointed by a regulatory authority.
\end{enumerate}

\section{Solution:}
A is correct. The issuer chooses a financial institution with trust powers, such as the trust department of a bank or a trust company, to act as a trustee for the bond issue.

\begin{enumerate}
  \setcounter{enumi}{2}
  \item The individual or entity most likely responsible for the timely payment of interest and repayment of principal to bondholders is the:
A. trustee.
B. primary or lead bank of the issuer.
C. treasurer or chief financial officer of the issuer.
\end{enumerate}

\section{Solution:}
A is correct. Although the issuer is ultimately the source of the contractual payments, it is the trustee that ensures timely payments. Doing so is accomplished by invoicing the issuer for interest payments and principal repayments and holding the funds until they are paid.

\begin{enumerate}
  \setcounter{enumi}{3}
  \item The major advantage of issuing bonds through a special legal entity is:
A. bankruptcy remoteness.
B. beneficial tax treatments.
C. greater liquidity and lower issuing costs.
\end{enumerate}

\section{Solution:}
A is correct. A special legal entity is a bankruptcy-remote vehicle. Bankruptcy remoteness is achieved by transferring the assets from the sponsor to the special legal entity. Once this transfer is completed, the sponsor no longer has ownership rights. If the sponsor defaults, no claims can be made to recover the assets that were transferred or the proceeds from the transfer to the special legal entity.

\begin{enumerate}
  \setcounter{enumi}{4}
  \item The category of bond most likely repaid from the repayment of previous loans made by the issuer is:
\end{enumerate}

A. sovereign bonds.

B. supranational bonds.

C. non-sovereign bonds.

Solution:

B is correct. The source of payment for bonds issued by supranational organizations is either the repayment of previous loans made by the organization or the paid-in capital of its member states. A is incorrect because national governments rely on their taxing authority and money creation to repay their debt. $\mathrm{C}$ is incorrect because non-sovereign bonds are typically repaid from the issuer's taxing authority or the cash flows of the project being financed.

\begin{enumerate}
  \setcounter{enumi}{5}
  \item The type of collateral used to secure collateral trust bonds is most likely:
\end{enumerate}

A. securities.

B. mortgages.

C. physical assets.

Solution:

A is correct. Collateral trust bonds are secured by securities, such as common shares, other bonds, or other financial assets. B is incorrect because MBS are secured by mortgages. $C$ is incorrect because equipment trust certificates are backed by physical assets, such as aircraft, railroad cars, shipping containers, or oil rigs.

\begin{enumerate}
  \setcounter{enumi}{6}
  \item The external credit enhancement that has the least amount of third-party risk is a:
\end{enumerate}

A. surety bond.

B. letter of credit.

C. cash collateral account.

Solution:

$\mathrm{C}$ is correct. The third-party (or counterparty) risk for a surety bond and a letter of credit arises from both being future promises to pay. In contrast, a cash collateral account allows the issuer to immediately borrow the credit-enhancement amount and then invest it.

\begin{enumerate}
  \setcounter{enumi}{7}
  \item An example of an affirmative covenant is the requirement:
\end{enumerate}

A. that dividends will not exceed $60 \%$ of earnings.

B. to insure and perform periodic maintenance on financed assets.

C. that the debt-to-equity ratio will not exceed 0.4 and times interest earned will not fall below 8.0.

\section{Solution:}
B is correct. Affirmative covenants indicate what the issuer "must do" and are administrative in nature. A covenant requiring the issuer to insure and perform periodic maintenance on financed assets is an example of an affirmative covenant. A and $\mathrm{C}$ are incorrect because they are negative covenants; they indicate what the issuer cannot do.

\begin{enumerate}
  \setcounter{enumi}{8}
  \item An example of a covenant that protects bondholders against the dilution of their claims is a restriction on:
A. debt.
B. investments.
C. mergers and acquisitions.
\end{enumerate}

\section{Solution:}
A is correct. A restriction on debt typically takes the form of a maximum acceptable debt usage ratio or a minimum acceptable interest coverage ratio. Thus, it limits the issuer's ability to issue new debt that would dilute the bondholders' claims. B and C are incorrect because they are covenants that restrict the issuer's business activities by preventing the company from making investments or being taken over, respectively.

\section{LEGAL, REGULATORY, AND TAX CONSIDERATIONS}
describe how legal, regulatory, and tax considerations affect the issuance and trading of fixed-income securities

Fixed-income securities are subject to different legal and regulatory requirements across jurisdictions, depending on where they are issued and traded as well as who holds them.

An important consideration for investors is where the bonds are issued and traded because it affects the laws and regulations that apply. The global bond markets consist of national bond markets and the Eurobond market. A national bond market includes all the bonds that are issued and traded in a specific country and are denominated in the currency of that country. Bonds issued by entities that are incorporated in that country are called domestic bonds, whereas bonds issued by entities that are incorporated in another country are called foreign bonds. If Ford Motor Company issues bonds denominated in US dollars in the United States, for example, these bonds will be classified as domestic. If Volkswagen Group and Toyota Motor Corporation (or their German or Japanese subsidiaries) issue bonds denominated in US dollars in the United States, these bonds will be classified as foreign. Foreign bonds very often receive nicknames. For example, foreign bonds are called "kangaroo bonds" in Australia, "maple bonds" in Canada, "panda bonds" in China, "samurai bonds" in Japan, "kimchi bonds" in South Korea, "matryoshka bonds" in Russia, "matador bonds" in Spain, "bulldog bonds" in the United Kingdom, and "Yankee bonds" in the United States. National regulators may make distinctions both between and among resident and non-resident issuers, and they may have different requirements regarding the issuance process, the level of disclosures, or the restrictions imposed on the bond issuer and/or the investors as to who can purchase the bonds. Governments and companies have issued foreign bonds in London since the 19th century, and foreign bond issues expanded in such countries as the United States, Japan, and Switzerland during the 1980s. But the 1960s saw the emergence of another bond market: the Eurobond market. The Eurobond market was created primarily to bypass the legal, regulatory, and tax constraints imposed on bond issuers and investors, particularly in the United States. Bonds issued and traded on the Eurobond market are called Eurobonds, and they are named after the currency in which they are denominated. For example, Eurodollar and Euroyen bonds are denominated in US dollars and Japanese yens, respectively. Bonds that are denominated in euros are called euro-denominated Eurobonds.

Eurobonds are issued outside the jurisdiction of any single country, are usually unsecured, and may be denominated in any currency, including the issuer's domestic currency. They are underwritten by an international syndicate-that is, a group of financial institutions from different jurisdictions-and mostly sold to investors in Europe, the Middle East, and Asia. Eurobonds denominated in US dollars cannot be sold to US investors at the time of issue because they are not registered with the US Securities and Exchange Commission (SEC). In the past, Eurobonds typically were bearer bonds, meaning that the trustee did not keep records of who owned the bonds; only the clearing system knew who the bond owners were. Eurobonds, domestic, and foreign bonds are now registered bonds for which ownership is recorded by either name or serial number.

A global bond is one issued simultaneously in the Eurobond market and in at least one domestic bond market, ensuring sufficient demand for large bond issues and access to all fixed-income investors regardless of location. For example, the World Bank is a regular issuer of global bonds. Many market participants refer to foreign bonds, Eurobonds, and global bonds as international bonds as opposed to domestic bonds.

Domestic bonds, foreign bonds, Eurobonds, and global bonds are subject to different legal, regulatory, and tax requirements. They also have different interest payment frequencies and interest payment calculation methods that affect bond cash flows and prices. Note, however, that the currency in which a bond is denominated has a stronger effect on its price than where the bond is issued or traded. This is because market interest rates have a strong influence on a bond's price, and the market interest rates that affect a bond are those associated with the currency in which the bond is denominated.

As the emergence and growth of the Eurobond market illustrates, legal and regulatory considerations affect the dynamics of the global fixed-income markets. Exhibit 2 compares the amounts of total and international debt outstanding for the 15 countries that were the largest debt issuers at the end of December 2019. The reported amounts are based on the residence of the issuer.

Exhibit 2: Total and International Debt Securities by Residence of Issuer at the End of December 2019

\begin{center}
\includegraphics[max width=\textwidth]{2023_05_04_7b535d0a870224f62e3dg-462}
\end{center}

United States

China

Japan

United Kingdom

France

Germany Total Debt Securities (US\$ billions) International Debt Securities (US\$ billions)

$\begin{array}{cc}41,232 & 2,356 \\ 14,726 & 217 \\ 12,825 & 480 \\ 6,288 & 3,288 \\ 4,670 & 1,433 \\ 3,548 & 1,290\end{array}$

\begin{center}
\begin{tabular}{lcc}
\hline
Issuers & $\begin{array}{c}\text { Total Debt Securities } \\ \text { (US\$ billions) }\end{array}$ & $\begin{array}{c}\text { International Debt Securities } \\ \text { (US\$ billions) }\end{array}$ \\
\hline
Canada & 3,361 & 971 \\
Italy & 3,194 & 840 \\
Netherlands & 2,193 & 2,109 \\
Spain & 1,972 & 532 \\
Australia & 1,959 & 595 \\
Luxembourg & 1,084 & 936 \\
Ireland & 824 & 893 \\
Denmark & 793 & 124 \\
Belgium & 729 & 167 \\
\hline
\end{tabular}
\end{center}

Source: Based on data from the Bank for International Settlements, Table C1, available at \href{http://www.bis.org/}{www.bis.org/} statistics/secstats.htm (updated 3 June 2020).

\section{EXAMPLE 3}
\begin{enumerate}
  \item An example of a domestic bond is a bond issued by:
\end{enumerate}

A. LG Group from South Korea, denominated in British pounds, and sold in the United Kingdom

B. the UK Debt Management Office, denominated in British pounds, and sold in the United Kingdom.

C. Wal-Mart from the United States, denominated in US dollars, and sold in various countries in North America, Europe, the Middle East, and Asia Pacific.

\section{Solution:}
$\mathrm{B}$ is correct. A domestic bond is issued by a local issuer, denominated in local currency, and sold in the domestic market. Gilts are British pound-denominated bonds issued by the UK Debt Management Office in the United Kingdom. Thus, they are UK domestic bonds. A is incorrect because a bond issued by LG Group from South Korea, denominated in British pounds, and sold in the United Kingdom is an example of a foreign bond. $\mathrm{C}$ is incorrect because a bond issued by Wal-Mart from the United States, denominated in US dollars, and sold in various countries in North America, Europe, the Middle East, and Asia Pacific is most likely an example of a global bond, particularly if it is also sold in the Eurobond market.

\begin{enumerate}
  \setcounter{enumi}{1}
  \item A bond issued by Sony in Japan, denominated in US dollars but not registered with the SEC, and sold to an institutional investor in the Middle East, is most likely an example of a:
\end{enumerate}

A. Eurobond.

B. global bond.

C. foreign bond.

\section{Solution:}
A is correct. A Eurobond is a bond that is issued internationally, outside the jurisdiction of any single country. Thus, a bond issued by Sony from Japan, denominated in US dollars but not registered with the SEC, is an example of a Eurobond. B is incorrect because global bonds are bonds that are issued simultaneously in the Eurobond market and in at least one domestic bond market. $\mathrm{C}$ is incorrect because if Sony's bond issue were a foreign bond, it would be registered with the SEC.

\section{Tax Considerations}
Bond interest is usually taxed at the ordinary income tax rate, which is typically the same tax rate that an individual would pay on wage or salary income. Tax-exempt securities are the exception to this rule. For example, interest income received by holders of local government bonds, called municipal bonds in the United States, is often exempt from federal income tax and from the income tax of the state in which the bonds are issued. The tax status of bond income may also depend on where the bond is issued and traded. For example, some domestic bonds pay their interest net of income tax. Other bonds, including some Eurobonds, make gross interest payments.

In addition to earnings from interest, a bond investment will generate a capital gain or loss if sold prior to maturity at a price different from the purchase price. This change will generate a capital gain if the bond price has increased or a capital loss if the bond price has decreased. Capital gains or losses usually face different tax treatment from taxable income, which often varies for long-term and short-term capital gains. For example, capital gains recognized over a year after the original bond purchase may be taxed at a lower long-term capital gains tax rate, whereas capital gains recognized within a year of bond purchase may be taxed as a short-term capital gain that equals the ordinary income tax rate. Exceptions exist, and not all countries have a separate capital gains tax. Differences in national and local tax legislation can add further complexity to aggregate country capital gains tax treatment.

For bonds issued at a discount, the tax status of the original issue discount is an additional tax consideration. The original issue discount is the difference between the par value and the original issue price. For example, the United States includes a prorated portion of the discount in interest income every tax year, while Japan does not. Original issue discount tax treatment varies by country, and Exhibit 3 illustrates the potential importance of this tax consideration.

\section{Exhibit 3: Original Issue Discount Tax Provision}
Assume a hypothetical country, Zinland, where the local currency is the zini (Z). The market interest rate in Zinland is $10 \%$, and both interest income and capital gains are taxed. Companies $A$ and $B$ issue 20-year bonds with a par value of $Z 1,000$. Company $A$ issues a coupon bond with an annual coupon rate of $10 \%$. Investors buy Company A's bonds for Z1,000. Every year, they receive and pay $\operatorname{tax}$ on their Z100 annual interest payments. When Company A's bonds mature,

\includegraphics[max width=\textwidth, center]{2023_05_04_7b535d0a870224f62e3dg-464}
bond at a discount. Investors buy Company B's bonds for Z148.64. They do not receive any cash flows until Company B pays the par value of $Z 1,000$ when the bonds mature.

Company A's bonds and Company B's bonds are economically identical in the sense that they have the same maturity (20 years) and the same yield to maturity (10\%). Company A's bonds make periodic payments, however, whereas Company B's bonds defer payment until maturity. Investors in Company A's bonds must include the annual interest payments in taxable income. When they receive their original Z1,000 investment back at maturity, they face no capital gain or loss. Without an original issue discount tax provision, investors in Company B's bonds do not have any taxable income until the bonds mature. When they receive the par value at maturity, they face a capital gain on the original issue discount-that is, on $\mathrm{Z} 851.36$ ( $\mathrm{Z} 1,000$ - $\mathrm{Z} 148.64$ ). The purpose of an original issue discount tax provision is to tax investors in Company B's bonds the same way as investors in Company A's bonds. Thus, a prorated portion of the Z851.36 original issue discount is included in taxable income every tax year until maturity. This allows investors in Company B's bonds to increase their cost basis in the bonds so that at maturity, they face no capital gain or loss.

Some jurisdictions also have tax provisions for bonds bought at a premium. They may allow investors to deduct a prorated portion of the amount paid in excess of the bond's par value from their taxable income every tax year until maturity. For example, if an investor pays $\$ 1,005$ for a bond that has a par value of $\$ 1,000$ and matures five years later, she can deduct $\$ 1$ from her taxable income every tax year for five years. This deduction may be a choice rather than a requirement, with an investor able to decide whether to deduct a prorated portion of the premium each year or to deduct nothing and declare a capital loss when the bond is redeemed at maturity.

\section{EXAMPLE 4}
\begin{enumerate}
  \item The coupon payment is most likely to be taxed as:
A. ordinary income.
B. short-term capital gain.
C. long-term capital gain.
\end{enumerate}

\section{Solution:}
A is correct. Interest income is typically taxed at the ordinary income tax rate, which may be the same tax rate that individuals pay on wage and salary income.

\begin{enumerate}
  \setcounter{enumi}{1}
  \item Assume that a company issues bonds in the hypothetical country of Zinland, where the local currency is the zini $(\mathrm{Z})$. There is an original issue discount tax provision in Zinland's tax code. The company issues a 10-year zero-coupon bond with a par value of $\mathrm{Z1,000}$ and sells it for Z800. An investor who buys the zero-coupon bond at issuance and holds it until maturity most likely:
\end{enumerate}

A. has to include $\mathrm{Z} 20$ in his taxable income every tax year for 10 years and has to declare a capital gain of Z200 at maturity.

B. has to include Z20 in his taxable income every tax year for 10 years and does not have to declare a capital gain at maturity.

C. does not have to include anything in his taxable income every tax year for 10 years but has to declare a capital gain of Z200 at maturity.

\section{Solution:}
$B$ is correct. The original issue discount tax provision requires the investor to include a prorated portion of the original issue discount in his taxable income every tax year until maturity. The original issue discount is the difference between the par value and the original issue price-that is, $\mathrm{Z1,000}$ $-\mathrm{Z} 800=$ Z200. The bond's maturity is 10 years. Thus, the prorated portion that must be included each year is $\mathrm{Z} 200 \div 10=\mathrm{Z} 20$. The original issue discount tax provision allows the investor to increase his cost basis in the bond so that when the bond matures, the investor faces no capital gain or loss.

\section{PRINCIPAL REPAYMENT STRUCTURES}
describe how cash flows of fixed-income securities are structured

The most common payment structure by far is that of a plain vanilla bond, as depicted in Exhibit 1. These bonds make periodic, fixed coupon payments and a lump-sum payment of principal at maturity. But there are other structures regarding both the principal repayment and the interest payments. This section discusses the major schedules observed in the global fixed-income markets. Schedules for principal repayments and interest payments are typically similar for a particular type of bond, such as 30-year US Treasury bonds. However, payment schedules vary considerably between types of bonds, such as government bonds versus corporate bonds.

\section{Principal Repayment Structures}
How the amount borrowed is repaid is an important consideration for investors as it affects credit risk. Any provision that periodically retires some of the principal amount outstanding is a way to reduce credit risk.

\section{Bullet, Fully Amortized, and Partially Amortized Bonds}
The payment structure of a plain vanilla bond has been used for nearly every government bond ever issued as well as for most corporate bonds. Such a bond is also known as a bullet bond because the entire payment of principal occurs at maturity.

In contrast, an amortizing bond has a payment schedule that calls for periodic payments of interest and repayments of principal. A fully amortized bond is characterized by a fixed periodic payment schedule that reduces the bond's outstanding principal amount to zero by the maturity date. A partially amortized bond also makes fixed periodic payments until maturity, but only a portion of the principal is repaid by the maturity date. Thus, a balloon payment is required at maturity to retire the bond's outstanding principal amount.

Exhibit 4 illustrates the differences in the payment schedules for a bullet bond, a fully amortized bond, and a partially amortized bond. For the three bonds, the principal amount is $\$ 1,000$, the maturity is five years, the coupon rate is $6 \%$, and interest payments are made annually. The market interest rate used to discount the bonds' expected cash flows until maturity is assumed to be constant at $6 \%$. The bonds are issued and redeemed at par. For the partially amortized bond, the balloon payment is $\$ 200$ at maturity.

Exhibit 4: Example of Payment Schedules for Bullet, Fully Amortized, and Partially Amortized Bonds

Bullet Bond

\begin{center}
\begin{tabular}{lcccc}
\hline
Year & $\begin{array}{c}\text { Investor Cash } \\ \text { Flows }\end{array}$ & $\begin{array}{c}\text { Interest } \\ \text { Payment }\end{array}$ & $\begin{array}{c}\text { Principal } \\ \text { Repayment }\end{array}$ & $\begin{array}{c}\text { Outstanding Principal at } \\ \text { the End of the Year }\end{array}$ \\
\hline
0 & $-\$ 1,000.00$ & $\$ 60.00$ & $\$ 0.00$ & $\$ 1,000.00$ \\
1 & 60.00 & 60.00 & 0.00 & $1,000.00$ \\
2 & 60.00 & 60.00 & 0.00 & $1,000.00$ \\
3 & 60.00 &  & $1,000.00$ &  \\
\end{tabular}
\end{center}

Bullet Bond

\begin{center}
\begin{tabular}{lcccc}
\hline
Year & $\begin{array}{c}\text { Investor Cash } \\ \text { Flows }\end{array}$ & $\begin{array}{c}\text { Interest } \\ \text { Payment }\end{array}$ & $\begin{array}{c}\text { Principal } \\ \text { Repayment }\end{array}$ & $\begin{array}{c}\text { Outstanding Principal at } \\ \text { the End of the Year }\end{array}$ \\
\hline
4 & 60.00 & 60.00 & 0.00 & $1,000.00$ \\
5 & $1,060.00$ & 60.00 & $1,000.00$ & 0.00 \\
\hline
\end{tabular}
\end{center}

\begin{center}
\begin{tabular}{lcccc}
\hline
 & \multicolumn{4}{c}{Fully Amortized Bond} \\
\hline
Year & $\begin{array}{c}\text { Investor Cash } \\ \text { Flows }\end{array}$ & $\begin{array}{c}\text { Interest } \\ \text { Payment }\end{array}$ & $\begin{array}{c}\text { Principal } \\ \text { Repayment }\end{array}$ & $\begin{array}{c}\text { Outstanding Principal at } \\ \text { the End of the Year }\end{array}$ \\
\hline
0 & $-\$ 1,000.00$ &  &  &  \\
1 & 237.40 & $\$ 60.00$ & $\$ 177.40$ & $\$ 822.60$ \\
2 & 237.40 & 49.36 & 188.04 & 634.56 \\
3 & 237.40 & 38.07 & 199.32 & 435.24 \\
4 & 237.40 & 26.11 & 211.28 & 223.96 \\
5 & 237.40 & 13.44 & 223.96 & 0.00 \\
\hline
\end{tabular}
\end{center}

\begin{center}
\begin{tabular}{lcccc}
 & \multicolumn{4}{c}{Partially Amortized Bond} \\
\hline
Year & $\begin{array}{c}\text { Investor Cash } \\ \text { Flows }\end{array}$ & $\begin{array}{c}\text { Interest } \\ \text { Payment }\end{array}$ & $\begin{array}{c}\text { Principal } \\ \text { Repayment }\end{array}$ & $\begin{array}{c}\text { Outstanding Principal at } \\ \text { the End of the Year }\end{array}$ \\
\hline
0 & $-\$ 1,000.00$ & $\$ 60.00$ & $\$ 141.92$ & $\$ 858.08$ \\
1 & 201.92 & 51.48 & 150.43 & 707.65 \\
2 & 201.92 & 42.46 & 159.46 & 548.19 \\
3 & 201.92 & 32.89 & 169.03 & 379.17 \\
4 & 201.92 & 22.75 & 379.17 & 0.00 \\
5 & 401.92 &  &  &  \\
\hline
\end{tabular}
\end{center}

Investors pay $\$ 1,000$ now to purchase any of the three bonds. For the bullet bond, they receive the coupon payment of $\$ 60(6 \% \times \$ 1,000)$ every year for five years. The last payment is $\$ 1,060$ because it includes both the last coupon payment and the principal amount.

For the fully amortized bond, the annual payment, which includes both the coupon payment and the principal repayment, is constant. Thus, this annual payment can be viewed as an annuity. This annuity lasts for five years; its present value, discounted at the market interest rate of $6 \%$, is equal to the bond price of $\$ 1,000$. Therefore, the annual payment is $\$ 237.40$. The first year, the interest part of the payment is $\$ 60(6 \%$ $\times \$ 1,000)$, which implies that the principal repayment part is $\$ 177.40(\$ 237.40-\$ 60)$. This repayment leaves an outstanding principal amount, which becomes the basis for the calculation of the interest the following year, of $\$ 822.60(\$ 1,000-\$ 177.40)$. The second year, the interest part of the payment is $\$ 49.36(6 \% \times \$ 822.60)$, the principal repayment part is $\$ 188.04$ (\$237.40 - $\$ 49.36)$, and the outstanding principal amount is $\$ 634.56$ (\$822.60 - \$188.04). The fifth year, the outstanding principal amount is fully repaid. Note that the annual payment is constant, but over time the interest payment decreases while the principal repayment increases.

The partially amortized bond can be viewed as the combination of two elements: a five-year annuity plus the balloon payment at maturity. The sum of the present values of these two elements is equal to the bond price of $\$ 1,000$. As for the fully amortized bond, the discount rate is the market interest rate of $6 \%$, making the constant amount for the annuity $\$ 201.92$. This amount represents the annual payment for the first four years. For Years 1 through 4, the split between interest and principal is done the same way as for the fully amortized bond. The interest part of the payment is equal to $6 \%$ multiplied by the outstanding principal at the end of the previous year; the principal repayment part is equal to $\$ 201.92$ minus the interest part of the payment for the year; and the outstanding principal amount at the end of the year is equal to the outstanding principal amount at the end of the previous year minus the principal repayment for the year. In Year 5, investors receive $\$ 401.92$; this amount is calculated either as the sum of the interest payment (\$22.75) and the outstanding principal amount (\$379.17) or as the constant amount of the annuity (\$201.92) plus the balloon payment (\$200). As for the fully amortized bond, the interest payment decreases and the principal repayment increases over time. Because the principal amount is not fully amortized, interest payments are higher for the partially amortized bond than for the fully amortized bond, except the first year when they are equal.

Exhibit 4 does not address the complexity of the repayment structure for some bonds, such as many ABS. For example, MBS face prepayment risk, which is the possible early repayment of mortgage principal. Borrowers usually have the right to prepay mortgages, which typically occurs when a current homeowner purchases a new home or when homeowners refinance their mortgages because market interest rates have fallen.

\section{EXAMPLE 5}
\begin{enumerate}
  \item The structure that requires the largest repayment of principal at maturity is that of a:
\end{enumerate}

A. bullet bond.

B. fully amortized bond.

C. partially amortized bond.

Solution:

A is correct. The entire repayment of principal occurs at maturity for a bullet (or plain vanilla) bond, whereas it occurs over time for fully and partially amortized bonds. Thus, the largest repayment of principal at maturity is that of a bullet bond.

\begin{enumerate}
  \setcounter{enumi}{1}
  \item A plain vanilla bond has a maturity of 10 years, a par value of $\pounds 100$, and a coupon rate of $9 \%$. Interest payments are made annually. The market interest rate is assumed to be constant at $9 \%$. The bond is issued and redeemed at par. The principal repayment the first year is closest to:
\end{enumerate}

A. $\pounds 0.00$.

B. $\pounds 6.58$.

C. $\pounds 10.00$.

Solution:

A is correct. A plain vanilla (or bullet) bond does not make any principal repayment until the maturity date. $B$ is incorrect because $\pounds 6.58$ would be the principal repayment for a fully amortized bond.

\begin{enumerate}
  \setcounter{enumi}{2}
  \item Relative to a fully amortized bond, the coupon payments of an otherwise similar partially amortized bond are:
\end{enumerate}

A. lower or equal.

B. equal. C. higher or equal.

\section{Solution:}
$\mathrm{C}$ is correct. Except at maturity, the principal repayments are lower for a partially amortized bond than for an otherwise similar fully amortized bond. Consequently, the principal amounts outstanding-and, therefore, the amounts of interest payments-are higher for a partially amortized bond than for a fully amortized bond, all else equal. The only exception is the first interest payment, which is the same for both repayment structures. This is because no principal repayment has been made by the time the first coupon is paid.

\section{Sinking Fund Arrangements}
A sinking fund arrangement is another approach that can be used to achieve the same goal of periodically retiring the bond's principal outstanding. The term sinking fund refers to an issuer's plans to set aside funds over time to retire the bond. Originally, a sinking fund was a specified cash reserve that was segregated from the rest of the issuer's business for the purpose of repaying the principal. More generally today, a sinking fund arrangement specifies the portion of the bond's principal outstanding, perhaps 5\%, that must be repaid each year throughout the bond's life or after a specified date. This repayment occurs whether or not an actual segregated cash reserve has been created.

Typically, the issuer will forward repayment proceeds to the bond's trustee. The trustee will then either redeem bonds to this value or select by lottery the serial numbers of bonds to be paid off. The bonds for repayment may be listed in business newspapers, such as the Wall Street Journal or the Financial Times.

Another type of sinking fund arrangement operates by redeeming a steadily increasing amount of the bond's notional principal (total amount) each year. Any remaining principal is then redeemed at maturity. It is common to find utility and energy companies in the United States, the United Kingdom, and the Commonwealth countries that issue bonds with sinking fund arrangements that incorporate such a provision.

Another common variation is for the bond issue to include a call provision, which gives the issuer the option to repurchase the bonds before maturity. The issuer can usually repurchase the bonds at the market price, at par, or at a specified sinking fund price, whichever is the lowest. To allocate the burden of the call provision fairly among bondholders, the bonds to be retired are selected at random based on serial number. Usually, the issuer can repurchase only a small portion of the bond issue. Some indentures, however, allow issuers to use a doubling option to repurchase double the required number of bonds.

Sinking fund arrangements have several benefits and drawbacks. The benefits include a structure that ensures a formal plan exists for debt retirement and also reduces credit risk due to a reduction of default risk at maturity. Sinking fund drawbacks include investor reinvestment risk, which is the risk associated with having to reinvest cash flows at an interest rate below the current yield-to-maturity. If the serial number of an investor's bonds is selected, the bonds will be repaid and the investor will have to reinvest the proceeds. If market interest rates have fallen since the investor purchased the bonds, the investor probably will not be able to purchase a bond offering the same return. Another potential disadvantage for investors occurs if the issuer has the option to repurchase bonds at below market prices. For example, an issuer could exercise a call option to buy back bonds at par on bonds priced above par. In this case, investors would suffer a loss.

Exhibit 5 illustrates an example of a sinking fund arrangement.

\section{Exhibit 5: Example of a Sinking Fund Arrangement}
The notional principal of the bond issue is $\pounds 200$ million. The sinking fund arrangement calls for $5 \%$ of the outstanding principal amount to be retired in Years 10 through 19, with the outstanding balance paid off at maturity in 20 years.

\begin{center}
\begin{tabular}{lcccc}
\hline
 & $\begin{array}{c}\text { Outstanding Principal } \\ \text { at the Beginning } \\ \text { of the Year } \\ \text { (⿷ millions) }\end{array}$ & $\begin{array}{c}\text { Sinking Fund } \\ \text { Payment } \\ \text { (⿷ millions) }\end{array}$ & $\begin{array}{c}\text { Outstanding } \\ \text { Principal at the } \\ \text { End of the Year } \\ \text { (⿷ millions) }\end{array}$ & $\begin{array}{c}\text { Final Principal } \\ \text { Repayment } \\ \text { (⿷ millions) }\end{array}$ \\
\hline
Year &  & 200.00 &  &  \\
1 to 9 & 200.00 & 0.00 & 200.00 &  \\
10 & 200.00 & 10.00 & 190.00 &  \\
11 & 190.00 & 9.50 & 180.50 &  \\
12 & 180.50 & 9.03 & 171.48 &  \\
13 & 171.48 & 8.57 & 162.90 &  \\
14 & 162.90 & 8.15 & 154.76 &  \\
15 & 154.76 & 7.74 & 147.02 &  \\
16 & 147.02 & 7.35 & 139.67 &  \\
17 & 139.67 & 6.98 & 132.68 &  \\
18 & 132.68 & 6.63 & 126.05 &  \\
19 & 126.05 & 6.30 & 119.75 &  \\
20 & 119.75 &  &  & 119.75 \\
\hline
\end{tabular}
\end{center}

There is no repayment of the principal during the first nine years. Starting the 10th year, the sinking fund arrangement calls for $5 \%$ of the outstanding principal amount to be retired each year. In Year 10, $\pounds 10$ million ( $5 \% \times \pounds 200$ million) are paid off, which leaves an outstanding principal balance of $\pounds 190$ million. In Year 11 , the principal amount repaid is $\pounds 9.50$ million ( $5 \% \times \pounds 190$ million). The final repayment of the remaining balance ( $\pounds 119.75$ million) is a balloon payment at maturity.

\section{COUPON PAYMENT STRUCTURES}
describe how cash flows of fixed-income securities are structured

A coupon is the interest payment that the bond issuer makes to the bondholder. A conventional bond pays a fixed periodic coupon over a specified time to maturity. Most frequently, the coupon is paid semi-annually for sovereign and corporate bonds; this is the case in the United States, the United Kingdom, and such Commonwealth countries as Bangladesh, India, and New Zealand. Eurobonds usually pay an annual coupon, although some Eurobonds make quarterly coupon payments. Most bonds issued in the eurozone have an annual coupon.

Fixed-rate coupons are not the only coupon payment structure, however. A wide range of coupon types is offered in the global fixed-income markets to meet the differing needs of both issuers and investors.

\section{Floating-Rate Notes}
Floating-rate notes do not have a fixed coupon; instead, their coupon rate is linked to an external reference rate, such as Euribor. Thus, an FRN's interest rate will fluctuate periodically during the bond's life, following the changes in the reference rate. Therefore, the FRN's cash flows are not known with certainty. Large issuers of FRNs include government-sponsored enterprises (GSEs), such as the Federal Home Loan Banks (FHLB), the Federal National Mortgage Association ("Fannie Mae"), and the Federal Home Loan Mortgage Corporation ("Freddie Mac") in the United States, as well as banks and financial institutions in Europe and Asia Pacific. It is rare for national governments to issue FRNs because investors in sovereign bonds generally prefer fixed coupon bonds.

Almost all FRNs have quarterly coupons, although counterexamples do exist. FRNs usually pay a fixed spread over the specified reference rate. A typical coupon rate may be the three-month US dollar MRR + 20 bps (i.e., MRR + $0.20 \%$ ) for a US dollar-denominated bond or the three-month Euribor $+20 \mathrm{bps}$ for a euro-denominated FRN.

Contrary to plain vanilla, which are fixed-rate securities that decline in value in a rising interest rate environment, FRNs are less affected when interest rates increase because their coupon rates vary with market interest rates and are reset at regular, short-term intervals. Thus, FRNs have little interest rate risk-that is, the risk that a change in market interest rate affects a bond's value. FRNs are frequently favored by investors who expect that interest rates will rise. That said, investors still face credit risk when investing in FRNs. If an issuer's credit risk does not change from one coupon reset date to the next, the FRN's price generally will stay close to the par value. However, if there is a change in the issuer's credit quality that affects the perceived credit risk associated with the bond, the price of the FRN will deviate from its par value. A higher level of credit risk will lead to a lower price.

Additional features observed in FRNs may include a floor or a cap. A floor (floored FRN) prevents the coupon from falling below a specified minimum rate. This feature benefits the bondholders, who are guaranteed that the interest rate will not fall below the specified rate during a time of falling interest rates. In contrast, a cap (capped FRN) prevents the coupon from rising above a specified maximum rate. This feature benefits the issuer because it sets a limit to the interest rate paid on the debt during a time of rising interest rates. It is also possible to have a collared FRN, which includes both a cap and a floor.

An inverse or reverse FRN, or simply an inverse floater, is a bond whose coupon rate has an inverse relationship to the reference rate. The basic structure is the same as an ordinary FRN except for the direction in which the coupon rate is adjusted. When interest rates fall, the coupon rate on an ordinary FRN decreases; in contrast, the coupon rate on a reverse FRN increases. Thus, inverse FRNs are typically favored by investors who expect interest rates to decline.

\section{Step-Up Coupon Bonds}
The coupon of a step-up coupon bond, which may be fixed or floating, increases by specified margins at specified dates. An example of a bond with a step-up coupon is a ten-year callable bond issued by the Federal Home Loan Bank on 3 August 2016. The initial coupon rate was $1.25 \%$ and steps up to $1.50 \%$ on 3 August 2018, to $2.00 \%$ on 3 August 2020, to $2.50 \%$ on 3 August 2022, to $3.00 \%$ on 3 August 2023, to $4.00 \%$ on 3 August 2024, and finally to $6.00 \%$ on 3 August 2025 for the final year. The bond was first callable at par on 3 August 2018, at the time of the first step up. Bonds with step-up coupons offer bondholders some protection against rising interest rates, and they may be an important feature for callable bonds. When interest rates increase, there is a higher likelihood that the issuer will not call the bonds, particularly if the bonds have a fixed rate of interest. The step-up coupon allows bondholders to receive a higher coupon, in line with the higher market interest rates. Alternatively, when interest rates decrease or remain stable, the step-up feature acts as an incentive for the issuer to call the bond before the coupon rate increases and the interest expense rises.

Redeeming the bond when the coupon increases is not automatic, however; the issuer may choose to keep the bond despite its increasing cost. This may happen if refinancing the bond is necessary and alternatives are less advantageous for this issuer. For example, a financial crisis may make it difficult for the issuer to refinance. Alternatively, the issuer's credit quality may have deteriorated, which would lead to a higher yield, potentially making the coupon rate on the new bond more expensive than that on the existing bond despite the stepped-up coupon. Although the issuer does not have to call the bond, there is an implicit expectation from investors that it will do so if the market price for the bond is above the call price. Failure to do so may be viewed negatively by market participants and reduce investors' appetite for that issuer's bonds in the future.

\section{Credit-Linked Coupon Bonds}
A credit-linked coupon bond has a coupon that changes when the bond's credit rating changes. An example of a bond with a credit-linked coupon is one of British Telecom's bonds maturing in 2030. It was issued with a coupon rate of $8.625 \%$, with a 25 bp increase in coupon for each credit rating downgrade below the bond's credit rating at the time of issuance and a $25 \mathrm{bp}$ decrease for every credit rating upgrade above the bond's credit rating at the time of issuance.

Bonds with credit-linked coupons are attractive to investors who are concerned about the future creditworthiness of the issuer. They may also provide some general protection against a poor economy because credit ratings tend to decline the most during recessions. A potential problem associated with these bonds is that increases in the coupon payments resulting from a downgrade may ultimately result in further deteriorations of the credit rating or even contribute to the issuer's default.

\section{Payment-in-Kind Coupon Bonds}
A payment-in-kind (PIK) coupon bond typically allows the issuer to pay interest in the form of additional amounts of the bond issue rather than as a cash payment. Such bonds are favored by issuers who are concerned that the issuer may face potential cash flow problems in the future. They are used, for example, in financing companies that have a high debt burden, such as companies going through a leveraged buyout (a form of acquisition in which the financing consists primarily of debt). Investors usually demand a higher yield for holding bonds with PIK coupons in exchange for assuming their additional credit risk.

Other forms of PIK arrangements can also be found, such as paying the bondholders with common shares worth the amount of coupon due. With a PIK "toggle" note, the borrower has the option, for each interest period, to pay interest in cash, to make the interest payment in kind, or some mix of the two. Cash payments or payments in kind are frequently at the discretion of the borrower, but whether the payment is made in cash or in kind can be determined by an earnings or cash flow trigger identified in the indenture.

\section{Deferred Coupon Bonds}
A deferred coupon bond, sometimes called a split coupon bond, pays no coupons for its first few years but then pays a higher coupon than it otherwise normally would for the remainder of its life. Issuers of deferred coupon bonds are usually seeking ways to conserve cash in the years immediately following the bond issue, which may indicate poorer credit quality. Deferred coupon bonds are also common in project financing when the assets being developed do not generate any income during the development phase. A deferred coupon bond allows the issuer to delay interest payments until the project is completed, and the cash flows generated by the assets being financed can be used to service the debt.

One of the main advantages of investing in a deferred coupon bond is that these bonds are typically priced at significant discounts to par. Investors may also find the deferred coupon structure to be very helpful in managing taxes. If taxes due on the interest income can be delayed, investors may be able to minimize taxes. This tax advantage, however, depends on the jurisdiction concerned and how its tax rules apply to deferred coupon payments.

A zero-coupon bond can be thought of as an extreme form of deferred coupon bond. These securities pay no interest to the investor and thus are issued at a deep discount to par value. At maturity, the bondholder receives the par value of the bond as payment. Effectively, a zero-coupon bond defers all interest payments until maturity.

\section{Index-Linked Bonds}
An index-linked bond has its coupon payments and/or principal repayment linked to a specified index. In theory, a bond can be indexed to any published variable, including an index reflecting prices, earnings, economic output, commodities, or foreign currencies. Inflation-linked bonds are an example of index-linked bonds. They offer investors protection against inflation by linking a bond's coupon payments and/or the principal repayment to an index of consumer prices, such as the UK Retail Price Index (RPI) or the US Consumer Price Index (CPI). The advantage of using the RPI or CPI is that these indexes are well-known, transparent, and published regularly.

Governments are large issuers of inflation-linked bonds, also called linkers. The United Kingdom was one of the first developed countries to issue inflation-linked bonds in 1981, offering gilts linked to the UK RPI, its main measure of the rate of inflation. In 1997, the US Treasury introduced Treasury Inflation-Protected Securities (TIPS) linked to the US Consumer Price Index (CPI). Inflation-linked bonds are now more frequently being offered by corporate issuers, including both financial and non-financial companies.

A bond's stated coupon rate represents the nominal interest rate received by the bondholders. But inflation reduces the actual value of the interest received. The interest rate that bondholders actually receive, net of inflation, is the real interest rate; it is approximately equal to the nominal interest rate minus the rate of inflation. By increasing the coupon payments and/or the principal repayment in line with increases in the price index, inflation-linked bonds reduce inflation risk. An example of an inflation-linked bond is the 1\% US TIPS that matures on 15 February 2048. The coupon rate remains fixed at $1 \%$, but the principal is adjusted every six months for changes in the CPI.

Exhibit 6 shows some of the national governments that have issued large amounts of inflation-linked bonds. It displays the amount of the linkers and the percentage of the total debt for each country. Sovereign issuers can be grouped into three categories. The first category includes such countries as Brazil and Argentina, who have issued inflation-linked bonds because they were experiencing extremely high rates of inflation when borrowing. Thus, offering inflation-linked bonds was their only available alternative to raise funds. The second category includes the United Kingdom and Australia. These countries have issued inflation-linked bonds to not only add credibility to the government's commitment to disinflationary policies but also to capitalize on investor demand. The third category, which includes the United States, Canada, and Germany, consists of national governments that are most concerned about the social welfare benefits associated with inflation-linked securities. Theoretically, inflation-linked bonds provide investors the benefit of a long-term asset with a fixed real return that is free from inflation risk.

Exhibit 6: Government Inflation-Linked Debt in Billions of USD as of Q4 2019 and the Percentage of Total Debt

\begin{center}
\includegraphics[max width=\textwidth]{2023_05_04_7b535d0a870224f62e3dg-474}
\end{center}

\begin{itemize}
  \item Denotes latest data from 2017.
\end{itemize}

** Denotes latest data from 2018

Source: Based on data from the Bank for International Settlements, Table C2, available at www.bis. org/statistics/secstats.htm (updated 3 June 2020).

Different methods have been used for linking the cash flows of an index-linked bond to a specified index; the link can be made via the interest payments, the principal repayment, or both. The following examples describe how the link between the cash flows and the index is established, using inflation-linked bonds as an illustration.

\begin{itemize}
  \item Zero-coupon-indexed bonds pay no coupon, so the inflation adjustment is made via the principal repayment only: The principal amount to be repaid at maturity increases in line with increases in the price index during the bond's life. This type of bond has been issued in Sweden.

  \item Interest-indexed bonds pay a fixed nominal principal amount at maturity but an index-linked coupon during the bond's life. Thus, the inflation adjustment applies to the interest payments only. This is essentially a floating-rate note in which the reference rate is the inflation rate instead of an MRR, such as Euribor. These have been issued by insurance companies and major commercial banks but not typically by governments.

  \item Capital-indexed bonds pay a fixed coupon rate, but it is applied to a principal amount that increases in line with increases in the index during the bond's life. Thus, both the interest payments and the principal repayment are adjusted for inflation. Such bonds have been issued by governments in Australia, Canada, New Zealand, the United Kingdom, and the United States. - Indexed-annuity bonds are fully amortized bonds, in contrast to interest-indexed and capital-indexed bonds that are non-amortizing coupon bonds. The annuity payment, which includes both payment of interest and repayment of the principal, increases in line with inflation during the bond's life. Indexed-annuity bonds linked to a price index have been issued by local governments in Australia but not by the national government.

\end{itemize}

Exhibit 7 illustrates the different methods used for inflation-linked bonds.

\section{Exhibit 7: Examples of Inflation-Linked Bonds}
Assume a hypothetical country, Lemuria, where the currency is the lemming (L). The country issued 20-year bonds linked to the domestic Consumer Price Index (CPI). The bonds have a par value of L1,000. Lemuria's economy has been free of inflation until the most recent six months, when the CPI increased by $5 \%$. Suppose that the bonds are zero-coupon-indexed bonds. There will never be any coupon payments. Following the $5 \%$ increase in the CPI, the principal amount to be repaid increases to $\mathrm{L} 1,050$ [L1,000 $\times(1+0.05)]$ and will continue increasing in line with inflation until maturity.

Now, suppose that the bonds are coupon bonds that make semi-annual interest payments based on an annual coupon rate of $4 \%$. If the bonds are interest-indexed bonds, the principal amount at maturity will remain L1,000 regardless of the CPI level during the bond's life and at maturity. The coupon payments, however, will be adjusted for inflation. Prior to the increase in inflation, the semi-annual coupon payment was L20 $[(0.04 \times \mathrm{L} 1,000) \div 2]$. Following the $5 \%$ increase in the CPI, the semi-annual coupon payment increases to L21 [L20 $\times(1+0.05)]$. Future coupon payments will also be adjusted for inflation.

If the bonds are capital-indexed bonds, the annual coupon rate remains $4 \%$, but the principal amount is adjusted for inflation and the coupon payment is based on the inflation-adjusted principal amount. Following the $5 \%$ increase in the CPI, the inflation-adjusted principal amount increases to L1,050 [L1,000 $\times(1+$ $0.05)]$, and the new semi-annual coupon payment is L21 $[(0.04 \times \mathrm{L} 1,050) \div 2]$. The principal amount will continue increasing in line with increases in the CPI until maturity, and so will the coupon payments.

If the bonds are indexed-annuity bonds, they are fully amortized. Prior to the increase in inflation, the semi-annual payment was L36.56-the annuity payment based on a principal amount of L1,000 paid back in 40 semi-annual payments with an annual discount rate of $4 \%$. Following the $5 \%$ increase in the CPI, the annuity payment increases to L38.38 [L36.56 $\times(1+0.05)]$. Future annuity payments will also be adjusted for inflation in a similar manner.

\section{EXAMPLE 6}
\begin{enumerate}
  \item Floating-rate notes most likely pay:
A. annual coupons.
B. quarterly coupons.
C. semi-annual coupons.
\end{enumerate}

\section{Solution:}
B is correct. Most FRNs pay interest quarterly and are tied to a three-month MRR. 2. A zero-coupon bond can best be considered a:

A. step-up bond.

B. credit-linked bond.

C. deferred coupon bond.

Solution:

$\mathrm{C}$ is correct. Because interest is effectively deferred until maturity, a zero-coupon bond can be thought of as a deferred coupon bond. A and B are incorrect because both step-up bonds and credit-linked bonds pay regular coupons. For a step-up bond, the coupon increases by specified margins at specified dates. For a credit-linked bond, the coupon changes when the bond's credit rating changes.

\begin{enumerate}
  \setcounter{enumi}{2}
  \item The bonds that do not offer protection to the investor against increases in market interest rates are:
\end{enumerate}

A. step-up bonds.

B. floating-rate notes.

C. inverse floating-rate notes.

Solution:

$\mathrm{C}$ is correct. The coupon rate on an inverse FRN has an inverse relationship to the market reference rate. Thus, an inverse FRN does not offer protection to the investor when market interest rates increase but rather when they decrease. A and B are incorrect because step-up bonds and FRNs both offer protection against increases in market interest rates.

\begin{enumerate}
  \setcounter{enumi}{3}
  \item The US Treasury offers Treasury Inflation-Protected Securities (TIPS). The principal of TIPS increases with inflation and decreases with deflation based on changes in the US Consumer Price Index. When TIPS mature, an investor is paid the original principal or inflation-adjusted principal, whichever is greater. TIPS pay interest twice a year based on a fixed real coupon rate that is applied to the inflation-adjusted principal. TIPS are most likely:
\end{enumerate}

A. capital-indexed bonds.

B. interest-indexed bonds.

C. indexed-annuity bonds.

\section{Solution:}
A is correct. TIPS have a fixed coupon rate, and the principal is adjusted based on changes in the CPI. Thus, TIPS are an example of capital-indexed bonds. B is incorrect because with an interest-indexed bond, it is the principal repayment at maturity that is fixed and the coupon that is linked to an index. $\mathrm{C}$ is incorrect because indexed-annuity bonds are fully amortized bonds, not bullet bonds. The annuity payment (interest payment and principal repayment) is adjusted based on changes in an index.

\begin{enumerate}
  \setcounter{enumi}{4}
  \item Assume a hypothetical country, Lemuria, where the national government has issued 20-year capital-indexed bonds linked to the domestic Consumer Price Index (CPI). Lemuria's economy has been free of inflation until the most recent six months, when the CPI increased. Following the increase in inflation:
A. the principal amount remains unchanged but the coupon rate increases.
B. the coupon rate remains unchanged, but the principal amount increases.
C. the coupon payment remains unchanged, but the principal amount increases.
\end{enumerate}

\section{Solution:}
$\mathrm{B}$ is correct. Following an increase in inflation, the coupon rate of a capital-indexed bond remains unchanged, but the principal amount is adjusted upward for inflation. Thus, the coupon payment, which is equal to the fixed coupon rate multiplied by the inflation-adjusted principal amount, increases.

\section{CALLABLE AND PUTABLE BONDS}
describe contingency provisions affecting the timing and/or nature of cash flows of fixed-income securities and whether such provisions benefit the borrower or the lender

A contingency refers to some future event or circumstance that is possible but not certain. A contingency provision is a clause in a legal document that allows for some action if the event or circumstance does occur. For bonds, the term embedded option refers to various contingency provisions found in the indenture. These contingency provisions provide the issuer or the bondholders the right, but not the obligation, to take some action. These rights are called "options." These options are not independent of the bond and cannot be traded separately-hence the term "embedded." Some common types of bonds with embedded options include callable bonds, putable bonds, and convertible bonds. The options embedded in these bonds grant either the issuer or the bondholders certain rights affecting the disposal or redemption of the bond.

\section{Callable Bonds}
The most widely used embedded option is the call provision. A callable bond gives the issuer the right to redeem all or part of the bond before the specified maturity date. The primary reason why issuers choose to issue callable bonds rather than non-callable bonds is to protect themselves against a decline in interest rates. This decline can come either from market interest rates falling or from the issuer's credit quality improving. If market interest rates fall or credit quality improves, the issuer of a callable bond has the right to replace an old, expensive bond issue with a new, cheaper bond issue. In other words, the issuer can benefit from a decline in interest rates by being able to refinance its debt at a lower interest rate. For example, assume that the market interest rate was $6 \%$ at the time of issuance and that a company issued a bond with a coupon rate of $7 \%$-the market interest rate plus a spread of $100 \mathrm{bps}$. Now assume that the market interest rate has fallen to $4 \%$ and that the company's creditworthiness has not changed; it can still issue at the market interest rate plus $100 \mathrm{bps}$. If the original bond is callable, the company can redeem it and replace it with a new bond paying $5 \%$ annually. If the original bond is non-callable, the company must carry on paying 7\% annually and cannot benefit from the decline in market interest rates.

As illustrated in this example, callable bonds are advantageous to the issuer of the security. Put another way, the call option has value to the issuer. Callable bonds present investors with a higher level of reinvestment risk than non-callable bonds; that is, if the bonds are called, bondholders must reinvest funds in a lower interest rate environment. For this reason, callable bonds have to offer a higher yield and sell at a lower price than otherwise similar non-callable bonds. The higher yield and lower price compensate the bondholders for the value of the call option to the issuer.

Callable bonds have a long tradition and are commonly issued by corporate issuers. Although first issued in the US market, they are now frequently issued in every major bond market and in a variety of forms.

The details about the call provision are specified in the indenture. These details include the call price, which represents the price paid to bondholders when the bond is called. The call premium is the amount over par paid by the issuer if the bond is called. There may be restrictions on when the bond can be called, or the bond may have different call prices depending on when it is called. The call schedule specifies the dates and prices at which a bond may be called. Some callable bonds are issued with a call protection period, also called lockout period, cushion, or deferment period. The call protection period prohibits the issuer from calling a bond early in its life and is often added as an incentive for investors to buy the bond. The earliest time that a bond might be called is known as the call date.

Make-whole calls first appeared in the US corporate bond market in the mid-1990s and have become more commonplace ever since. A typical make-whole call requires the issuer to make a lump-sum payment to the bondholders based on the present value of the future coupon payments and outstanding principal due to early bond redemption. The discount rate used is usually a pre-determined spread over the yield-to-maturity of an appropriate sovereign bond. The typical result is a redemption value that is significantly greater than the bond's current market price. A make-whole call provision is less detrimental to bondholders than a regular call provision because it allows them to be compensated if the issuer calls the bond. Issuers, however, rarely invoke this provision because redeeming a bond that includes a make-whole provision before the maturity date is costly. Issuers tend to include a make-whole provision as a "sweetener" to make the bond issue more attractive to potential buyers and allow them to pay a lower coupon rate.

Available exercise styles on callable bonds include the following:

\begin{itemize}
  \item American-style call, sometimes referred to as continuously callable, for which the issuer has the right to call a bond at any time starting on the first call date.

  \item European-style call, for which the issuer has the right to call a bond only once on the call date.

  \item Bermuda-style call, for which the issuer has the right to call bonds on specified dates following the call protection period. These dates frequently correspond to coupon payment dates.

\end{itemize}

\section{EXAMPLE 7}
Assume a hypothetical 30-year bond is issued on 15 August 2019 at a price of 98.195 (as a percentage of par). Each bond has a par value of $\$ 1,000$. The bond is callable in whole or in part every 15 August from 2029 at the option of the issuer. The call prices are shown below.

\begin{center}
\begin{tabular}{llll}
\hline
Year & Call Price & \multicolumn{1}{c}{Year} & Call Price \\
\hline
2029 & 103.870 & 2035 & 101.548 \\
2030 & 103.485 & 2036 & 101.161 \\
2031 & 103.000 & 2037 & 100.774 \\
2032 & 102.709 & 2038 & 100.387 \\
2033 & 102.322 & 2039 and & 100.000 \\
2034 & 101.955 & thereafter &  \\
\hline
\end{tabular}
\end{center}

\begin{enumerate}
  \item The call protection period is:
A. 10 years.
B. 11 years.
C. 20 years.
\end{enumerate}

\section{Solution:}
A is correct. The bonds were issued in 2019 and are first callable in 2029. The call protection period is $2029-2019=10$ years.

\begin{enumerate}
  \setcounter{enumi}{1}
  \item The call premium (per $\$ 1,000$ in par value) in 2033 is closest to:
A. $\$ 2.32$.
B. $\$ 23.22$.
C. $\$ 45.14$.
\end{enumerate}

\section{Solution:}
B is correct. The call prices are stated as a percentage of par. The call price in 2033 is $\$ 1,023.22(102.322 \% \times \$ 1,000)$. The call premium is the amount paid above par by the issuer. The call premium in 2033 is $\$ 23.22(\$ 1,023.22$ $-\$ 1,000)$.

\begin{enumerate}
  \setcounter{enumi}{2}
  \item The call provision is most likely:
A. a Bermuda call.
B. a European call.
C. an American call.
\end{enumerate}

\section{Solution:}
A is correct. The bond is callable every 15 August from 2029-that is, on specified dates following the call protection period. Thus, the embedded option is a Bermuda call.

\section{Putable Bonds}
A put provision gives the bondholders the right to sell the bond back to the issuer at a pre-determined price on specified dates. Putable bonds are beneficial for the bondholder by guaranteeing a pre-specified selling price at the redemption dates. If interest rates rise after issuance and bond prices fall, the bondholders can put the bond back to the issuer and reinvest the proceeds in bonds that offer higher yields, in line with higher market interest rates. Because a put provision has value to the bondholders, the price of a putable bond will be higher than the price of an otherwise similar bond issued without the put provision. Similarly, the yield on a bond with a put provision will be lower than the yield on an otherwise similar non-putable bond. The lower yield compensates the issuer for the value of the put option to the investor.

The indenture lists the redemption dates and the prices applicable to the sale of the bond back to the issuer. The selling price is usually the par value of the bond. Depending on the terms set out in the indenture, putable bonds may allow buyers to force a sellback only once or multiple times during the bond's life. Putable bonds that incorporate a single sellback opportunity include a European-style put and are often referred to as one-time put bonds. Putable bonds that allow multiple sellback opportunities include a Bermuda-style put and are known as multiple put bonds. Multiple put bonds offer more flexibility for investors, so they are generally more expensive than one-time put bonds.

Typically, putable bonds incorporate one- to five-year put provisions. Their increasing popularity has often been motivated by investors wanting to protect themselves against major declines in bond prices. One benefit of this rising popularity has been an improvement in liquidity in some markets, because the put protection attracts more conservative classes of investors. The global financial crisis that started in 2008 showed that these securities can often exacerbate liquidity problems, however, because they provide a first claim on the issuer's assets. The put provision gives bondholders the opportunity to convert their claim into cash before other creditors.

\section{CONVERTIBLE BONDS}
describe contingency provisions affecting the timing and/or nature of cash flows of fixed-income securities and whether such provisions benefit the borrower or the lender

A convertible bond is a hybrid security with both debt and equity features. It gives the bondholder the right to exchange the bond for a specified number of common shares in the issuing company. Thus, a convertible bond can be viewed as the combination of a straight bond (option-free bond) plus an embedded equity call option. Convertible bonds can also include additional provisions, the most common being a call provision.

From the investor's perspective, a convertible bond offers several advantages relative to a non-convertible bond. First, it gives the bondholder the ability to convert into equity in case of share price appreciation, and thus participate in the equity upside. At the same time, the bondholder receives downside protection; if the share price does not appreciate, the convertible bond offers the comfort of regular coupon payments and the promise of principal repayment at maturity. Even if the share price and thus the value of the equity call option decline, the price of a convertible bond cannot fall below the price of the straight bond. Consequently, the value of the straight bond acts as a floor for the price of the convertible bond.

Because the conversion provision is valuable to bondholders, the price of a convertible bond is higher than the price of an otherwise similar bond without the conversion provision. Similarly, the yield on a convertible bond is lower than the yield on an otherwise similar non-convertible bond. However, most convertible bonds offer investors a yield advantage; the coupon rate on the convertible bond is typically higher than the dividend yield on the underlying common share. From the issuer's perspective, convertible bonds offer two main advantages. The first is reduced interest expense. Issuers are usually able to offer below-market coupon rates due to the conversion feature's value. The second advantage is the elimination of debt if the conversion option is exercised, but this is dilutive to existing shareholders.

Key terms regarding the conversion provision include the following:

\begin{itemize}
  \item The conversion price is the price per share at which the convertible bond can be converted into shares.

  \item The conversion ratio is the number of common shares that each bond can be converted into. The indenture sometimes does not stipulate the conversion ratio but only mentions the conversion price. The conversion ratio is equal to the par value divided by the conversion price. For example, if the par value is $€ 1,000$ and the conversion price is $€ 20$, the conversion ratio is $€ 1,000 \div € 20=50: 1$, or 50 common shares per bond

  \item The conversion value, sometimes called the parity value, is the current share price multiplied by the conversion ratio. For example, if the current share price is $€ 33$ and the conversion ratio is $30: 1$, the conversion value is $€ 33 \times 30=€ 990$

  \item The conversion premium is the difference between the convertible bond's price and its conversion value. For example, if the convertible bond's price is $€ 1,020$ and the conversion value is $€ 990$, the conversion premium is $€ 1,020$

\end{itemize}

$-€ 990=€ 30$.

\begin{itemize}
  \item Conversion parity occurs if the conversion value is equal to the convertible bond's price. Using the previous two examples, if the current share price is $€ 34$ instead of $€ 33$, then both the convertible bond's price and the conversion value are equal to $€ 1,020$ (i.e., a conversion premium equal to 0 ). This condition is referred to as parity. If the common share is selling for less than $€ 34$, the condition is below parity. In contrast, if the common share is selling for more than $€ 34$, the condition is above parity.
\end{itemize}

Generally, convertible bonds have maturities of five to ten years. First-time or newer issuers are usually able to issue convertible bonds of up to three years in maturity only. Although it is common for convertible bonds to reach conversion parity before maturity, bondholders rarely exercise the conversion option before that time. Early conversion would eliminate the yield advantage of continuing to hold the convertible bond; investors would typically receive in dividends less than they would receive in coupon payments. For this reason, it is common to find convertible bonds that are also callable by the issuer on a set of specified dates. If the convertible bond includes a call provision and the conversion value is above the convertible bond price, the issuer may force the bondholders to convert their bonds into common shares before maturity. For this reason, callable convertible bonds have to offer a higher yield and sell at a lower price than otherwise similar non-callable convertible bonds. Some indentures specify that the bonds can be called only if the share price exceeds a specified price, giving investors more predictability about the share price at which the issuer may force conversion.

A warrant is an "attached" rather than embedded option entitling the holder to buy the underlying stock of the issuing company at a fixed exercise price until the expiration date. Warrants are considered yield enhancements; they are frequently attached to bond issues as a "sweetener." Warrants are actively traded in some financial markets, such as the Deutsche Börse and the Hong Kong Stock Exchange.

Several European banks have issued a type of convertible bond called contingent convertible bonds. Contingent convertible bonds, nicknamed "CoCos," are bonds with contingent write-down provisions. Two main features distinguish bonds with contingent write-down provisions from the traditional convertible bonds just described. A traditional convertible bond is convertible at the option of the bondholder, and conversion occurs on the upside-that is, if the issuer's share price increases. In contrast, bonds with contingent write-down provisions are convertible on the downside. In the case of CoCos, conversion is automatic if a specified event occurs-for example, if the bank's core Tier 1 capital ratio (a measure of the bank's proportion of core equity capital available to absorb losses) falls below the minimum requirement set by the regulators. Thus, if the bank experiences losses that reduce its equity capital below the minimum requirement, CoCos are a way to reduce the bank's likelihood of default and, therefore, systemic risk-that is, the risk of failure of the financial system. When the bank's core Tier 1 capital falls below the minimum requirement, the CoCos immediately convert into equity, automatically recapitalizing the bank, lightening the debt burden, and reducing the risk of default. Because the conversion is not at the option of the bondholders but automatic, CoCos force bondholders to take losses. For this reason, CoCos must offer a higher yield than otherwise similar bonds.

\section{EXAMPLE 8}
\begin{enumerate}
  \item Which of the following is not an example of an embedded option?
\end{enumerate}

A. Warrant

B. Call provision

C. Conversion provision

Solution:

A is correct. A warrant is a separate, tradable security that entitles the holder to buy the underlying common share of the issuing company. B and $\mathrm{C}$ are incorrect because the call provision and the conversion provision are embedded options.

\begin{enumerate}
  \setcounter{enumi}{1}
  \item The type of bond with an embedded option that would most likely sell at a lower price than an otherwise similar bond without the embedded option is a:
\end{enumerate}

A. putable bond.

B. callable bond.

C. convertible bond.

Solution:

$\mathrm{B}$ is correct. The call provision is an option that benefits the issuer. Because of this, callable bonds sell at lower prices and higher yields relative to otherwise similar non-callable bonds. A and $\mathrm{C}$ are incorrect because the put provision and the conversion provision are options that benefit the investor. Thus, putable bonds and convertible bonds sell at higher prices and lower yields relative to otherwise similar bonds that lack those provisions.

\begin{enumerate}
  \setcounter{enumi}{2}
  \item The additional risk inherent to a callable bond is best described as:
\end{enumerate}

A. credit risk.

B. interest rate risk.

C. reinvestment risk.

\section{Solution:}
$\mathrm{C}$ is correct. Reinvestment risk refers to the effect that lower interest rates have on available rates of return when reinvesting the cash flows received from an earlier investment. Because bonds are typically called following a decline in market interest rates, reinvestment risk is particularly relevant for the holder of a callable bond. A is incorrect because credit risk refers to the risk of loss resulting from the issuer failing to make full and timely payments of interest and/or repayments of principal. $B$ is incorrect because interest rate risk is the risk that a change in market interest rate affects a bond's value.

\begin{enumerate}
  \setcounter{enumi}{3}
  \item The put provision of a putable bond:
\end{enumerate}

A. limits the risk to the issuer.

B. limits the risk to the bondholder.

C. does not materially affect the risk of either the issuer or the bondholder.

\section{Solution:}
B is correct. A putable bond limits the risk to the bondholder by guaranteeing a pre-specified selling price at the redemption dates.

\begin{enumerate}
  \setcounter{enumi}{4}
  \item Assume that a convertible bond issued in South Korea has a par value of $\# 1,000,000$ and is currently priced at $\# 1,100,000$. The underlying share price is $¥ 40,000$, and the conversion ratio is $25: 1$. The conversion condition for this bond is:
\end{enumerate}

A. parity.

B. above parity.

C. below parity.

\section{Solution:}
$\mathrm{C}$ is correct. The conversion value of the bond is $\# 40,000 \times 25=$ $\# 1,000,000$. The price of the convertible bond is $\# 1,100,000$. Thus, the conversion value of the bond is less than the bond's price, and this condition is referred to as below parity.

\section{SUMMARY}
This reading introduces the salient features of fixed-income securities while noting how these features vary among different types of securities. Important points include the following:

\begin{itemize}
  \item The three important elements that an investor needs to know when investing in a fixed-income security are: (1) the bond's features, which determine its scheduled cash flows and thus the bondholder's expected and actual return; (2) the legal, regulatory, and tax considerations that apply to the contractual agreement between the issuer and the bondholders; and (3) the contingency provisions that may affect the bond's scheduled cash flows. - The basic features of a bond include the issuer, maturity, par value (or principal), coupon rate and frequency, and currency denomination.

  \item Issuers of bonds include supranational organizations, sovereign governments, non-sovereign governments, quasi-government entities, and corporate issuers.

  \item Bondholders are exposed to credit risk and may use bond credit ratings to assess the credit quality of a bond.

  \item A bond's principal is the amount the issuer agrees to pay the bondholder when the bond matures.

  \item The coupon rate is the interest rate that the issuer agrees to pay to the bondholder each year. The coupon rate can be a fixed rate or a floating rate. Bonds may offer annual, semi-annual, quarterly, or monthly coupon payments depending on the type of bond and where the bond is issued.

  \item Bonds can be issued in any currency. Such bonds as dual-currency bonds and currency option bonds are connected to two currencies.

  \item The yield-to-maturity is the discount rate that equates the present value of the bond's future cash flows until maturity to its price. Yield-to-maturity can be considered an estimate of the market's expectation for the bond's return.

  \item A plain vanilla bond has a known cash flow pattern. It has a fixed maturity date and pays a fixed rate of interest over the bond's life.

  \item The bond indenture or trust deed is the legal contract that describes the form of the bond, the issuer's obligations, and the investor's rights. The indenture is usually held by a financial institution called a trustee, which performs various duties specified in the indenture.

  \item The issuer is identified in the indenture by its legal name and is obligated to make timely payments of interest and repayment of principal.

  \item For asset-backed securities, the legal obligation to repay bondholders often lies with a separate legal entity-that is, a bankruptcy-remote vehicle that uses the assets as guarantees to back a bond issue.

  \item How the issuer intends to service the debt and repay the principal should be described in the indenture. The source of repayment proceeds varies depending on the type of bond.

  \item Collateral backing is a way to alleviate credit risk. Secured bonds are backed by assets or financial guarantees pledged to ensure debt payment. Examples of collateral-backed bonds include collateral trust bonds, equipment trust certificates, mortgage-backed securities, and covered bonds.

  \item Credit enhancement can be internal or external. Examples of internal credit enhancement include subordination, overcollateralization, and reserve accounts. A bank guarantee, a surety bond, a letter of credit, and a cash collateral account are examples of external credit enhancement.

  \item Bond covenants are legally enforceable rules that borrowers and lenders agree on at the time of a new bond issue. Affirmative covenants enumerate what issuers are required to do, whereas negative covenants enumerate what issuers are prohibited from doing.

  \item An important consideration for investors is where the bonds are issued and traded, because it affects the laws, regulation, and tax status that apply. Bonds issued in a country in local currency are domestic bonds if they are issued by entities incorporated in the country and foreign bonds if they are issued by entities incorporated in another country. Eurobonds are issued internationally, outside the jurisdiction of any single country and are subject to a lower level of listing, disclosure, and regulatory requirements than domestic or foreign bonds. Global bonds are issued in the Eurobond market and at least one domestic market at the same time.

  \item Although some bonds may offer special tax advantages, as a general rule, interest is taxed at the ordinary income tax rate. Some countries also implement a capital gains tax. There may be specific tax provisions for bonds issued at a discount or bought at a premium.

  \item An amortizing bond is a bond whose payment schedule requires periodic payment of interest and repayment of principal. This differs from a bullet bond, whose entire payment of principal occurs at maturity. The amortizing bond's outstanding principal amount is reduced to zero by the maturity date for a fully amortized bond, but a balloon payment is required at maturity to retire the bond's outstanding principal amount for a partially amortized bond.

  \item Sinking fund agreements provide another approach to the periodic retirement of principal, in which an amount of the bond's principal outstanding amount is usually repaid each year throughout the bond's life or after a specified date.

  \item A floating-rate note, or floater, is a bond whose coupon is set based on a market reference rate (MRR) plus a spread. FRNs can be floored, capped, or collared. An inverse FRN is a bond whose coupon has an inverse relationship to the reference rate.

  \item Other coupon payment structures include bonds with step-up coupons, which pay coupons that increase by specified amounts on specified dates; bonds with credit-linked coupons, which change when the issuer's credit rating changes; bonds with payment-in-kind coupons, which allow the issuer to pay coupons with additional amounts of the bond issue rather than in cash; and bonds with deferred coupons, which pay no coupons in the early years following the issue but higher coupons thereafter.

  \item The payment structures for index-linked bonds vary considerably among countries. A common index-linked bond is an inflation-linked bond, or linker, whose coupon payments and/or principal repayments are linked to a price index. Index-linked payment structures include zero-coupon-indexed bonds, interest-indexed bonds, capital-indexed bonds, and indexed-annuity bonds.

  \item Common types of bonds with embedded options include callable bonds, putable bonds, and convertible bonds. These options are "embedded" in the sense that there are provisions provided in the indenture that grant either the issuer or the bondholder certain rights affecting the disposal or redemption of the bond. They are not separately traded securities.

  \item Callable bonds give the issuer the right to buy bonds back prior to maturity, thereby raising the reinvestment risk for the bondholder. For this reason, callable bonds have to offer a higher yield and sell at a lower price than otherwise similar non-callable bonds to compensate the bondholders for the value of the call option to the issuer.

  \item Putable bonds give the bondholder the right to sell bonds back to the issuer prior to maturity. Putable bonds offer a lower yield and sell at a higher price than otherwise similar non-putable bonds to compensate the issuer for the value of the put option to the bondholders. - A convertible bond gives the bondholder the right to convert the bond into common shares of the issuing company. Because this option favors the bondholder, convertible bonds offer a lower yield and sell at a higher price than otherwise similar non-convertible bonds.

\end{itemize}

\section{PRACTICE PROBLEMS}
\begin{enumerate}
  \item A 10-year bond was issued four years ago. The bond is denominated in US dollars, offers a coupon rate of $10 \%$ with interest paid semi-annually, and is currently priced at $102 \%$ of par. The bond's:
A. tenor is six years.
B. nominal rate is $5 \%$.
C. redemption value is $102 \%$ of the par value.

  \item A sovereign bond has a maturity of 15 years. The bond is best described as a:
A. perpetual bond.
B. pure discount bond.
C. capital market security.

  \item A company has issued a floating-rate note with a coupon rate equal to the three-month MRR + 65 bps. Interest payments are made quarterly on 31 March, 30 June, 30 September, and 31 December. On 31 March and 30 June, the three-month MRR is $1.55 \%$ and $1.35 \%$, respectively. The coupon rate for the interest payment made on 30 June is:
A. $2.00 \%$.
B. $2.10 \%$.
C. $2.20 \%$.

  \item The legal contract that describes the form of the bond, the obligations of the issuer, and the rights of the bondholders can be best described as a bond's:
A. covenant.
B. indenture.
C. debenture.

  \item Which of the following is a type of external credit enhancement?
A. Covenants
B. A surety bond
C. Overcollateralization

  \item An affirmative covenant is most likely tostipulate:

\end{enumerate}

A. limits on the issuer's leverage ratio.

B. how the proceeds of the bond issue will be used.

C. the maximum percentage of the issuer's gross assets that can be sold.

\begin{enumerate}
  \setcounter{enumi}{6}
  \item Which of the following best describes a negative bond covenant? The issuer is:
\end{enumerate}

A. required to pay taxes as they come due. B. prohibited from investing in risky projects.

C. required to maintain its current lines of business.

\begin{enumerate}
  \setcounter{enumi}{7}
  \item Clauses that specify the rights of the bondholders and any actions that the issuer is obligated to perform or is prohibited from performing are:
\end{enumerate}

A. covenants.

B. collaterals.

C. credit enhancements.

\begin{enumerate}
  \setcounter{enumi}{8}
  \item Which of the following type of debt obligation most likely protects bondholders when the assets serving as collateral are non-performing?
\end{enumerate}

A. Covered bonds

B. Collateral trust bonds

C. Mortgage-backed securities

\begin{enumerate}
  \setcounter{enumi}{9}
  \item Which of the following best describes a negative bond covenant? The requirement to:
A. insure and maintain assets.
B. comply with all laws and regulations.
C. maintain a minimum interest coverage ratio.

  \item Contrary to positive bond covenants, negative covenants are most likely:
A. costlier.
B. legally enforceable.
C. enacted at time of issue.

  \item A South African company issues bonds denominated in pound sterling that are sold to investors in the United Kingdom. These bonds can be best described as:

\end{enumerate}

A. Eurobonds.

B. global bonds.

C. foreign bonds.

\begin{enumerate}
  \setcounter{enumi}{12}
  \item Relative to domestic and foreign bonds, Eurobonds are most likely to be:
A. bearer bonds.
B. registered bonds.
C. subject to greater regulation.

  \item An investor in a country with an original issue discount tax provision purchases a 20-year zero-coupon bond at a deep discount to par value. The investor plans to hold the bond until the maturity date. The investor will most likely report:

\end{enumerate}

A. a capital gain at maturity. B. a tax deduction in the year the bond is purchased.

C. taxable income from the bond every year until maturity.

\begin{enumerate}
  \setcounter{enumi}{14}
  \item A bond that is characterized by a fixed periodic payment schedule that reduces the bond's outstanding principal amount to zero by the maturity date is best described as a:
A. bullet bond.
B. plain vanilla bond.
C. fully amortized bond.

  \item A five-year bond has the following cash flows:

\end{enumerate}

\begin{center}
\includegraphics[max width=\textwidth]{2023_05_04_7b535d0a870224f62e3dg-489}
\end{center}

The bond can best be described as a:
A. bullet bond.
B. fully amortized bond.
C. partially amortized bond.

\begin{enumerate}
  \setcounter{enumi}{16}
  \item If interest rates are expected to increase, the coupon payment structure most likely to benefit the issuer is a:
A. step-up coupon.
B. inflation-linked coupon.
C. cap in a floating-rate note.

  \item Investors who believe that interest rates will rise most likely prefer to invest in:
A. inverse floaters.
B. fixed-rate bonds.
C. floating-rate notes.

  \item A 10-year, capital-indexed bond linked to the Consumer Price Index (CPI) is issued with a coupon rate of $6 \%$ and a par value of 1,000 . The bond pays interest semi-annually. During the first six months after the bond's issuance, the CPI increases by $2 \%$. On the first coupon payment date, the bond's:
A. coupon rate increases to $8 \%$.
B. coupon payment is equal to 40 .
C. principal amount increases to 1,020 .

  \item Which type of bond most likely earns interest on an implied basis?
A. Floater
B. Conventional bond C. Pure discount bond

  \item Investors seeking some general protection against a poor economy are most likely to select a:
A. deferred coupon bond.
B. credit-linked coupon bond.
C. payment-in-kind coupon bond.

  \item The benefit to the issuer of a deferred coupon bond is most likely related to:
A. tax management.
B. cash flow management.
C. original issue discount price.

  \item The provision that provides bondholders the right to sell the bond back to the issuer at a predetermined price prior to the bond's maturity date is referred to as:
A. a put provision.
B. a make-whole call provision.
C. an original issue discount provision.

  \item Which of the following bond types provides the most benefit to a bondholder when bond prices are declining?
A. Callable
B. Plain vanilla
C. Multiple put

  \item Which type of call bond option offers the greatest flexibility as to when the issuer can exercise the option?
A. Bermuda call
B. European call
C. American call

  \item Which of the following provisions is a benefit to the issuer?
A. Put provision
B. Call provision
C. Conversion provision

  \item Relative to an otherwise similar option-free bond, a:
A. putable bond will trade at a higher price.
B. callable bond will trade at a higher price.
C. convertible bond will trade at a lower price. 28. Which of the following best describes a convertible bond's conversion premium?
A. Bond price minus conversion value
B. Par value divided by conversion price
C. Current share price multiplied by conversion ratio

\end{enumerate}

\section{SOLUTIONS}
\begin{enumerate}
  \item A is correct. The tenor of the bond is the time remaining until the bond's maturity date. Although the bond had a maturity of ten years at issuance (original maturity), it was issued four years ago. Thus, there are six years remaining until the maturity date.
\end{enumerate}

$\mathrm{B}$ is incorrect because the nominal rate is the coupon rate (i.e., the interest rate that the issuer agrees to pay each year until the maturity date). Although interest is paid semi-annually, the nominal rate is $10 \%$, not $5 \%$. $\mathrm{C}$ is incorrect because it is the bond's price, not its redemption value (also called principal amount, principal value, par value, face value, nominal value, or maturity value), that is equal to $102 \%$ of the par value.

\begin{enumerate}
  \setcounter{enumi}{1}
  \item C is correct. A capital market security has an original maturity longer than one year.
\end{enumerate}

A is incorrect because a perpetual bond does not have a stated maturity date. Thus, the sovereign bond, which has a maturity of 15 years, cannot be a perpetual bond. $\mathrm{B}$ is incorrect because a pure discount bond is a bond issued at a discount to par value and redeemed at par. Some sovereign bonds (e.g., Treasury bills) are pure discount bonds, but others are not.

\begin{enumerate}
  \setcounter{enumi}{2}
  \item $\mathrm{C}$ is correct. The coupon rate that applies to the interest payment due on 30 June is based on the three-month MRR rate prevailing on 31 March. Thus, the coupon rate is $1.55 \%+0.65 \%=2.20 \%$.

  \item B is correct. The indenture, also referred to as trust deed, is the legal contract that describes the form of the bond, the obligations of the issuer, and the rights of the bondholders.

\end{enumerate}

A is incorrect because covenants are only one element of a bond's indenture. Covenants are clauses that specify the rights of the bondholders and any actions that the issuer is obligated to perform or prohibited from performing. $C$ is incorrect because a debenture is a type of bond.

\begin{enumerate}
  \setcounter{enumi}{4}
  \item B is correct. A surety bond is an external credit enhancement (i.e., a guarantee received from a third party). If the issuer defaults, the guarantor who provided the surety bond will reimburse investors for any losses, usually up to a maximum amount called the penal sum.
\end{enumerate}

$\mathrm{A}$ is incorrect because covenants are legally enforceable rules that borrowers and lenders agree upon when the bond is issued. $C$ is incorrect because overcollateralization is an internal, not external, credit enhancement. Collateral is a guarantee underlying the debt above and beyond the issuer's promise to pay, and overcollateralization refers to the process of posting more collateral than is needed to obtain or secure financing. Collateral, such as assets or securities pledged to ensure debt payments, is not provided by a third party. Thus, overcollateralization is not an external credit enhancement.

\begin{enumerate}
  \setcounter{enumi}{5}
  \item B is correct. Affirmative (or positive) covenants enumerate what issuers are required to do and are typically administrative in nature. A common affirmative covenant describes what the issuer intends to do with the proceeds from the bond issue.
\end{enumerate}

A and $C$ are incorrect because imposing a limit on the issuer's leverage ratio or on the percentage of the issuer's gross assets that can be sold are negative covenants. Negative covenants prevent the issuer from taking actions that could reduce its ability to make interest payments and repay the principal. 7. B is correct. Prohibiting the issuer from investing in risky projects restricts the issuer's potential business decisions. These restrictions are referred to as negative bond covenants.

$\mathrm{A}$ and $\mathrm{C}$ are incorrect because paying taxes as they come due and maintaining the current lines of business are positive covenants.

\begin{enumerate}
  \setcounter{enumi}{7}
  \item A is correct. Covenants specify the rights of the bondholders and any actions that the issuer is obligated to perform or is prohibited from performing.

  \item A is correct. A covered bond is a debt obligation backed by a segregated pool of assets called a "cover pool." When the assets that are included in the cover pool become non-performing (i.e., the assets are not generating the promised cash flows), the issuer must replace them with performing assets.

  \item C is correct. Negative covenants enumerate what issuers are prohibited from doing. Restrictions on debt, including maintaining a minimum interest coverage ratio or a maximum debt usage ratio, are typical examples of negative covenants.

  \item A is correct. Affirmative covenants typically do not impose additional costs to the issuer, while negative covenants are frequently costly. B is incorrect because all bond covenants are legally enforceable rules, so there is no difference in this regard between positive and negative bond covenants. $C$ is incorrect because borrowers and lenders agree on all bond covenants at the time of a new bond issue, so there is no difference in this regard between positive and negative bond covenants.

  \item C is correct. Bonds sold in a country and denominated in that country's currency by an entity from another country are referred to as foreign bonds.

\end{enumerate}

A is incorrect because Eurobonds are bonds issued outside the jurisdiction of any single country. B is incorrect because global bonds are bonds issued in the Eurobond market and at least one domestic country simultaneously.

\begin{enumerate}
  \setcounter{enumi}{12}
  \item A is correct. Eurobonds are typically issued as bearer bonds (i.e., bonds for which the trustee does not keep records of ownership). In contrast, domestic and foreign bonds are typically registered bonds for which ownership is recorded by either name or serial number.
\end{enumerate}

B is incorrect because Eurobonds are typically issued as bearer bonds, not registered bonds. $\mathrm{C}$ is incorrect because Eurobonds are typically subject to lower, not greater, regulation than domestic and foreign bonds.

\begin{enumerate}
  \setcounter{enumi}{13}
  \item $\mathrm{C}$ is correct. The original issue discount tax provision requires the investor to include a prorated portion of the original issue discount in his taxable income every tax year until maturity. The original issue discount is equal to the difference between the bond's par value and its original issue price.
\end{enumerate}

A is incorrect because the original issue discount tax provision allows the investor to increase his cost basis in the bond so that when the bond matures, he faces no capital gain or loss. $B$ is incorrect because the original issue discount tax provision does not require any tax deduction in the year the bond is purchased or afterwards.

\begin{enumerate}
  \setcounter{enumi}{14}
  \item C is correct. A fully amortized bond calls for equal cash payments by the bond's issuer prior to maturity. Each fixed payment includes both an interest payment component and a principal repayment component such that the bond's outstanding principal amount is reduced to zero by the maturity date.
\end{enumerate}

A and B are incorrect because a bullet bond or plain vanilla bond only make interest payments prior to maturity. The entire principal repayment occurs at maturity. 16. B is correct. A bond that is fully amortized is characterized by a fixed periodic payment schedule that reduces the bond's outstanding principal amount to zero by the maturity date. The stream of $\pounds 230.97$ payments reflects the cash flows of a fully amortized bond with a coupon rate of $5 \%$ and annual interest payments.

\begin{enumerate}
  \setcounter{enumi}{16}
  \item $\mathrm{C}$ is correct. A cap in a floating-rate note (capped FRN) prevents the coupon rate from increasing above a specified maximum rate. This feature benefits the issuer in a rising interest rate environment because it sets a limit to the interest rate paid on the debt.
\end{enumerate}

A is incorrect because a bond with a step-up coupon is one in which the coupon, which may be fixed or floating, increases by specified margins at specified dates. This feature benefits the bondholders, not the issuer, in a rising interest rate environment because it allows bondholders to receive a higher coupon in line with the higher market interest rates. B is incorrect because inflation-linked bonds have their coupon payments and/or principal repayment linked to an index of consumer prices. If interest rates increase as a result of inflation, this feature is a benefit for the bondholders, not the issuer.

\begin{enumerate}
  \setcounter{enumi}{17}
  \item $\mathrm{C}$ is correct. In contrast to fixed-rate bonds that decline in value in a rising interest rate environment, floating-rate notes (FRNs) are less affected when interest rates increase because their coupon rates vary with market interest rates and are reset at regular, short-term intervals. Consequently, FRNs are favored by investors who believe that interest rates will rise.
\end{enumerate}

A is incorrect because an inverse floater is a bond whose coupon rate has an inverse relationship to the reference rate, so when interest rates rise, the coupon rate on an inverse floater decreases. Thus, inverse floaters are favored by investors who believe that interest rates will decline, not rise. B is incorrect because fixed rate-bonds decline in value in a rising interest rate environment. Consequently, investors who expect interest rates to rise will likely avoid investing in fixed-rate bonds.

\begin{enumerate}
  \setcounter{enumi}{18}
  \item C is correct. Capital-indexed bonds pay a fixed coupon rate that is applied to a principal amount that increases in line with increases in the index during the bond's life. If the consumer price index increases by $2 \%$, the coupon rate remains unchanged at $6 \%$, but the principal amount increases by $2 \%$ and the coupon payment is based on the inflation-adjusted principal amount. On the first coupon payment date, the inflation-adjusted principal amount is $1,000 \times(1+0.02)=$ 1,020 and the semi-annual coupon payment is equal to $(0.06 \times 1,020) \div 2=30.60$.

  \item C is correct. A zero-coupon, or pure discount, bond pays no interest; instead, it is issued at a discount to par value and redeemed at par. As a result, the interest earned is implied and equal to the difference between the par value and the purchase price.

  \item B is correct. A credit-linked coupon bond has a coupon that changes when the bond's credit rating changes. Because credit ratings tend to decline the most during recessions, credit-linked coupon bonds may thus provide some general protection against a poor economy by offering increased coupon payments when credit ratings decline.

  \item B is correct. Deferred coupon bonds pay no coupon for their first few years but then pay higher coupons than they otherwise normally would for the remainder of their life. Deferred coupon bonds are common in project financing when the assets being developed may not generate any income during the development phase, thus not providing cash flows to make interest payments. A deferred coupon bond allows the issuer to delay interest payments until the project is completed and the cash flows generated by the assets can be used to service the debt. 23. A is correct. A put provision provides bondholders the right to sell the bond back to the issuer at a predetermined price prior to the bond's maturity date.

\end{enumerate}

$B$ is incorrect because a make-whole call provision is a form of call provision (i.e., a provision that provides the issuer the right to redeem all or part of the bond before its maturity date). A make-whole call provision requires the issuer to make a lump sum payment to the bondholders based on the present value of the future coupon payments and principal repayments not paid because of the bond being redeemed early by the issuer. $C$ is incorrect because an original issue discount provision is a tax provision relating to bonds issued at a discount to par value. The original issue discount tax provision typically requires the bondholders to include a prorated portion of the original issue discount (i.e., the difference between the par value and the original issue price) in their taxable income every tax year until the bond's maturity date.

\begin{enumerate}
  \setcounter{enumi}{23}
  \item $\mathrm{C}$ is correct. A putable bond is beneficial for the bondholder by guaranteeing a prespecified selling price at the redemption date, thus offering protection when interest rates rise and bond prices decline. Relative to a one-time put bond that incorporates a single sellback opportunity, a multiple put bond offers more frequent sellback opportunities, thus providing the most benefit to bondholders.

  \item $C$ is correct. An American call option gives the issuer the right to call the bond at any time starting on the first call date.

  \item B is correct. A call provision (callable bond) gives the issuer the right to redeem all or part of the bond before the specified maturity date. If market interest rates decline or the issuer's credit quality improves, the issuer of a callable bond can redeem it and replace it by a cheaper bond. Thus, the call provision is beneficial to the issuer.

\end{enumerate}

$\mathrm{A}$ is incorrect because a put provision (putable bond) is beneficial to the bondholders. If interest rates rise, thus lowering the bond's price, the bondholders have the right to sell the bond back to the issuer at a predetermined price on specified dates. $C$ is incorrect because a conversion provision (convertible bond) is beneficial to the bondholders. If the issuing company's share price increases, the bondholders have the right to exchange the bond for a specified number of common shares in the issuing company.

\begin{enumerate}
  \setcounter{enumi}{26}
  \item A is correct. A put feature is beneficial to the bondholders. Thus, the price of a putable bond will typically be higher than the price of an otherwise similar non-putable bond.
\end{enumerate}

$\mathrm{B}$ is incorrect because a call feature is beneficial to the issuer. Thus, the price of a callable bond will typically be lower, not higher, than the price of an otherwise similar non-callable bond. $C$ is incorrect because a conversion feature is beneficial to the bondholders. Thus, the price of a convertible bond will typically be higher, not lower, than the price of an otherwise similar non-convertible bond.

\begin{enumerate}
  \setcounter{enumi}{27}
  \item A is correct. The conversion premium is the difference between the convertible bond's price and its conversion value.
\end{enumerate}

\section{LEARNING MODULE 2}
\section{Fixed-Income Markets: Issuance, Trading, and Funding}
by Moorad Choudhry, PhD, FRM, FCSI, Steven V. Mann, PhD, and Lavone F. Whitmer, CFA.

Moorad Choudhry PhD, FRM, FCSI is at Recognise Bank Limited (England). Steven V. Mann, PhD, is at the University of South Carolina (USA). Lavone F. Whitmer, CFA, is at Federal Home Loan Bank of Indianapolis (USA).

\section{LEARNING OUTCOME}
\begin{center}
\begin{tabular}{|c|c|}
\hline
Mastery & The candidate should be able to: \\
\hline
$\square$ & describe classifications of global fixed-income markets \\
\hline
$\square$ & $\begin{array}{l}\text { describe the use of interbank offered rates as reference rates in } \\ \text { floating-rate debt }\end{array}$ \\
\hline
$\square$ & describe mechanisms available for issuing bonds in primary markets \\
\hline
$\square$ & describe secondary markets for bonds \\
\hline
$\square$ & describe securities issued by sovereign governments \\
\hline
$\square$ & $\begin{array}{l}\text { describe securities issued by non-sovereign governments, } \\ \text { quasi-government entities, and supranational agencies }\end{array}$ \\
\hline
$\square$ & describe types of debt issued by corporations \\
\hline
$\square$ & describe structured financial instruments \\
\hline
$\square$ & describe short-term funding alternatives available to banks \\
\hline
 & $\begin{array}{l}\text { describe repurchase agreements (repos) and the risks associated with } \\ \text { them }\end{array}$ \\
\hline
\end{tabular}
\end{center}

\section{INTRODUCTION}
Global fixed-income markets represent the largest subset of financial markets in terms of number of issuances and market capitalization, they bring borrowers and lenders together to allocate capital globally to its most efficient uses. Fixed-income markets include not only publicly traded securities, such as commercial paper, notes, and bonds, but also non-publicly traded loans. Although they usually attract less attention than equity markets, fixed-income markets are more than three times the size of global equity markets. The Institute of International Finance reports that the size of the global debt market surpassed USD 253 trillion in the third quarter of 2019, representing a 322\% global debt-to-GDP ratio (1 Institute of International Finance Global Debt Monitor, January 13, 2020).

Debt investors and issuers need to understand how fixed-income markets are structured and how they operate. Debt issuers have financing needs that must be met. For example, a government may need to finance an infrastructure project, a new hospital, or a new school. A start-up technology company may require funds beyond its initial seed funding from investors to expand its business and achieve scale. Financial institutions also have funding needs, and they are among the largest issuers of fixed-income securities.

Among the questions this reading addresses are the following:

\begin{itemize}
  \item What are the key bond market sectors?

  \item How are bonds sold in primary markets and traded in secondary markets?

  \item What types of bonds are issued by governments, government-related entities, financial companies, and non-financial companies?

  \item What additional sources of funds are available to banks?

\end{itemize}

We first present an overview of global fixed-income markets and how these markets are classified. We identify the major issuers of and investors in fixed-income securities, present fixed-income indexes, discuss how fixed-income securities are issued in primary markets, and explore how these securities are then traded in secondary markets. We also examine different bond market sectors and discuss additional short-term funding alternatives available to banks.

\section*{CLASSIFICATION OF FIXED-INCOME MARKETS }
Although there is no standard classification of fixed-income markets, many investors and market participants use criteria to structure fixed-income markets and identify bond market sectors. This section starts by describing the most widely used ways of classifying fixed-income markets.

\section{Classification of Fixed-Income Markets}
Common criteria used to classify fixed-income markets include the type of issuer; the bonds' credit quality, maturity, currency denomination, and type of coupon; and where the bonds are issued and traded.

\section{Classification by Type of Issuer}
One way of classifying fixed-income markets is by type of issuer, which leads to the identification of four bond market sectors: households, non-financial corporates, government, and financial institutions. Exhibit 1 presents data on global debt markets at the end of the third quarter of 2019 by these sectors. Each sector is broken down into mature markets (United States, Euro area, Japan, and United Kingdom) and emerging markets. Although the sectors are of roughly comparable sizes overall, in emerging markets the debt of non-financial corporates is proportionately higher and the financial sector lower than in the mature markets.

\section{Exhibit 1: Global Debt by Sector at End of Q3 2019 in USD}
\begin{center}
\begin{tabular}{|c|c|c|c|c|c|}
\hline
USD Trillion (\% of total) & Households & $\begin{array}{c}\text { Non-financial } \\ \text { Corporates }\end{array}$ & Government & Financial Sector & Total \\
\hline
\multirow[t]{2}{*}{Mature markets} & 34.3 & 43.1 & 52.5 & 50.2 & 180.1 \\
\hline
 & $(19.0 \%)$ & $(23.9 \%)$ & $(29.2 \%)$ & $(27.9 \%)$ & $(100.0 \%)$ \\
\hline
\multirow[t]{2}{*}{Emerging markets} & 13.2 & 31.3 & 16.7 & 11.3 & 72.5 \\
\hline
 & $(18.2 \%)$ & $(43.2 \%)$ & $(23.0 \%)$ & $(15.6 \%)$ & $(100.0 \%)$ \\
\hline
\multirow[t]{2}{*}{Global debt} & 47.5 & 74.4 & 69.2 & 61.5 & 252.6 \\
\hline
 & $(18.8 \%)$ & $(29.5 \%)$ & $(27.4 \%)$ & $(24.3 \%)$ & $(100.0 \%)$ \\
\hline
\end{tabular}
\end{center}

Source: Institute of International Finance Global Debt Monitor, January 2020.

Exhibit 2 displays global debt by sector as a percentage of GDP for various mature and emerging market countries at the end of the third quarter of 2019. In general, debt in proportion to GDP is higher in developed countries. However, debt takes different forms in different countries.

Exhibit 2: Global Debt by Sector as a Percentage of GDP, at End of Q3 2019

\begin{center}
\begin{tabular}{|c|c|c|c|c|c|}
\hline
 & Households & $\begin{array}{c}\text { Non-financial } \\ \text { Corporates }\end{array}$ & Government & Financial Sector & Total \\
\hline
United States & $74.2 \%$ & $74.2 \%$ & $101.8 \%$ & $77.1 \%$ & $327.3 \%$ \\
\hline
Japan & 55.3 & 101.9 & 226.3 & 157.0 & 540.5 \\
\hline
United Kingdom & 83.8 & 81.5 & 110.3 & 178.0 & 453.6 \\
\hline
China & 55.4 & 156.7 & 53.6 & 42.8 & 308.5 \\
\hline
South Korea & 95.1 & 101.6 & 40.2 & 88.8 & 325.7 \\
\hline
Brazil & 28.7 & 42.9 & 87.9 & 40.4 & 199.9 \\
\hline
Mexico & 16.7 & 26.4 & 35.3 & 16.6 & 95.0 \\
\hline
Israel & 41.6 & 69.6 & 60.4 & 10.6 & 182.2 \\
\hline
Nigeria & 15.6 & 8.3 & 29.2 & 4.3 & 57.4 \\
\hline
\end{tabular}
\end{center}

Source: Institute of International Finance Global Debt Monitor, 13 January 2020.

\section{Classification by Credit Quality}
Investors who hold bonds are exposed to credit risk, which is the risk of loss resulting from the issuer failing to make full and timely payments of interest and/or principal. Bond markets can be classified based on the issuer's creditworthiness as judged by credit rating agencies. Ratings of Baa3 or above by Moody's Investors Service or BBBor above by Standard \& Poor's (S\&P) and Fitch Ratings are considered investment grade. In contrast, ratings below these levels are referred to as non-investment grade, high yield, speculative, or "junk." An important point to understand is that credit ratings assess the issuer's creditworthiness at a certain point in time; they are not a recommendation to buy or sell the issuer's securities. Credit ratings are not static, however; they will change if a credit rating agency perceives that the probability of default for an issuer has changed.

One of the reasons why the distinction between investment-grade and high-yield bond markets matters is because institutional investors may be prohibited from investing in, or restricted in their exposure to, lower-quality or lower-rated securities. Prohibition or restriction in high-yield bond holdings generally arise because of a more restrictive risk-reward profile that forms part of the investor's investment objectives and constraints. For example, regulated banks, life insurance companies, and pension schemes are usually limited to investing in very highly rated securities. In contrast, hedge funds or some sovereign wealth funds (Qatar and Kuwait) have no formal restrictions on what type of assets they can hold or on the percentage split between bond market sectors. Globally, investment-grade bond markets tend to be more liquid than high-yield bond markets.

\section{Classification by Maturity}
Fixed-income securities can also be classified by the original maturity of the bonds when they are issued. Securities may be issued with maturities at issuance (original maturity) ranging from overnight to $30,40,50$, and up to 100 years. For example, large companies including The Walt Disney Company and The Coca-Cola Company have issued 100-year bonds in the past. Some governments have done the same. These bonds are attractive to the issuers because they secure long-term funding. They are also attractive for many investors, such as insurance companies, that have long-term liabilities and need to buy long-duration assets.

\section{Classification by Currency Denomination}
One of the critical ways to distinguish among fixed-income securities is by currency denomination. The currency denomination of the bond's cash flows influences the yield that must be offered to investors to compensate for the potential impact of currency movements on bond investment returns. For example, if a bond is denominated in Brazilian real, its price will be primarily driven by the credit quality of the issuer, the general level of interest rates in Brazil, and the economic fundamentals of Brazil to the extent that they will affect the exchange rate, and thus ex post bond returns in the investor's domestic currency.

Exhibit 3 presents data on debt as a percentage of GDP for several emerging market countries for local currency (LC) and foreign currency (FC) issuances at the end of the third quarter of 2019. Government debt is almost entirely in the domestic currency, whereas corporate and bank debt differs considerably by country.

Exhibit 3: Debt as a Percentage of GDP by Local and Foreign Currency for Several Emerging Markets

\begin{center}
\begin{tabular}{lcccccc}
\hline
 & \multicolumn{2}{c}{$\begin{array}{l}\text { Non-financial } \\
\text { Corporates }\end{array}$} & Government & Financial Sector &  &  \\
\hline
LC & FC & LC & FC & LC & FC &  \\
\hline
Brazil & $28.1 \%$ & $14.9 \%$ & $87.2 \%$ & $0.6 \%$ & $31.9 \%$ & $8.5 \%$ \\
India & 37.9 & 6.3 & 66.3 & 1.9 & 0.9 & 3.1 \\
Mexico & 9.0 & 17.3 & 28.3 & 7.0 & 13.9 & 2.7 \\
Turkey & 30.8 & 36.5 & 17.9 & 17.1 & 5.3 & 22.2 \\
\hline
\end{tabular}
\end{center}

Note: LC stands for "local currency"; FC stands for "foreign currency". Source: Institute of International Finance Global Debt Monitor, 13 January 2020.

\section{Classification by Type of Coupon}
Another way of classifying fixed-income markets is by type of coupon. Some bonds pay a fixed rate of interest; others, called floating-rate bonds, floating-rate notes (FRNs), or floaters, pay a rate of interest that adjusts to market interest rates at regular, short-term intervals (e.g., quarterly).

\section{Demand and Supply of Fixed-Rate vs. Floating-Rate Debt}
Balance sheet risk management considerations explain much of the demand and supply of floating-rate debt. For instance, the funding of banks - that is, the money banks raise to make loans to companies and individuals-is often short term and issued at rates that change or reset frequently. When a mismatch arises between the interest paid on the liabilities (money the bank borrowed) and the interest received on the assets (money the bank lent or invested), banks are exposed to interest rate risk-that is, the risk associated with a change in interest rate. In an effort to limit the volatility of their net worth resulting from interest rate risk, banks that issue floating-rate debt often prefer to make floating-rate loans and invest in floating-rate bonds or in other adjustable-rate assets. In addition to institutions with short-term funding needs, demand for floating-rate bonds comes from investors who believe that interest rates will rise. In that case, investors will benefit from holding floating-rate investments compared with fixed-rate ones.

On the supply side, issuance of floating-rate debt comes from institutions needing to finance short-term loans, such as consumer finance companies. Corporate borrowers also view floating-rate bonds as an alternative to using bank liquidity facilities (e.g., lines of credit), particularly when they are a lower-cost option, and as an alternative to borrowing long term at fixed rates when they expect interest rates will fall.

\section{Reference Rates}
The coupon rate of a floating-rate bond is typically expressed as a reference rate plus a constant spread or margin. It is primarily a function of the issuer's credit risk at issuance: The lower the issuer's credit quality (the higher its credit risk), the higher the spread. The reference rate, however, resets periodically. Therefore, the coupon rate adjusts to the level of market interest rates each time the reference rate is reset.

Different reference rates are used depending on where the bonds are issued and their currency denomination. In the past, the London interbank offered rate (Libor) has long been the reference rate for many floating-rate bonds. Libor reflected the rate at which a panel of banks believed they could borrow unsecured funds from other banks. It was gathered by way of survey, and subjectivity is believed to have made its way into the process. The sidebar describes how Libor is being phased out and set to be replaced by new money market reference rates.

\section{THE PHASEOUT OF LIBOR}
Starting in 1986, daily Libor quotations for various currencies and maturities were set by the British Bankers' Association (BBA). Every business day, a select group of banks would submit to the BBA the rates at which they believed they could borrow from other banks in the London interbank market. Over time, the coverage grew to include 10 currencies and 15 borrowing periods. The submitted rates would be ranked from highest to lowest, and the upper and lower four submissions would be discarded. The arithmetic mean of the remaining rates became the Libor setting for a particular combination of currency and maturity. The 150 Libor quotations would then be communicated to market participants for use as reference rates in many different types of debt, including floating-rate bonds.

\section{Fixed-Income Markets: Issuance, Trading, and Funding}
Problems with Libor first arose during the 2007-2009 global financial crisis, when the perceived default and liquidity risks of major international banks rose significantly. Some banks were allegedly submitting lowered rates to influence the market perception of their credit quality. In addition, some banks were allegedly altering their submitted rates to improve the valuation of their derivatives positions in contracts tied to Libor. In 2014, UK regulatory authorities transferred the administration of Libor from BBA to the International Currency Exchange (ICE). Under ICE, the size for Libor quotations was reduced to five currencies and seven maturities.

By 2017, it became apparent that the lack of activity in interbank borrowing and lending meant that Libor submissions were based more on judgment of market conditions than actual trades. Therefore, the decision was made that the panel of banks would no longer be required to submit quotations after 2021. In anticipation of the eventual demise of Libor, market participants and regulators have been working to develop new alternative money market reference rates. In the United States, it appears that the new rate will be the secured overnight financing rate (SOFR). SOFR, which has begun to be reported daily by the US Federal Reserve, is based on actual transactions in the sale and repurchase agreement ("repo") market. (Repos are described later.) In July 2018, Fannie Mae, a major institution in the US secondary mortgage market, issued USD6 billion in floating-rate notes at SOFR plus a small spread. (Note: based on "Libor: Its Astonishing Ride and How to Plan for Its End," by Liang Wu, a Numerix white paper, February 2018.)

The use of these money market reference rates extends beyond setting coupon rates for floating-rate debt. These rates are also used as reference rates for other debt instruments, including mortgages, derivatives such as interest rate and cross-currency swaps, and many other financial contracts and products. In 2018, nearly USD350 trillion in financial instruments were estimated to be tied to Libor. A major task for the future will be the transition of the terms on these contracts to the new Market Reference Rate (MRR) as Libor is phased out.

\section{Classification by Geography}
A distinction is very often made among the domestic bond, foreign bond, and Eurobond markets. Bonds issued in a specific country, denominated in the currency of that country, and sold in that country are classified as domestic bonds if they are issued by an issuer domiciled in that country and foreign bonds if they are issued by an issuer domiciled in another country. Domestic and foreign bonds are subject to the legal, regulatory, and tax requirements that apply in that particular country. In contrast, a Eurobond is issued internationally, outside the jurisdiction of the country in whose currency the bond is denominated. The Eurobond market has traditionally been characterized by fewer reporting, regulatory, and tax constraints than domestic and foreign bond markets. This less constrained environment explains why approximately $80 \%$ of entities that issue bonds outside their country of origin choose to do so in the Eurobond market rather than in a foreign bond market. In addition, Eurobonds are attractive for issuers because it enables them to reach out to more investors globally. Access to a wider pool of investors often allows issuers to raise more capital, usually at a lower cost. Along these lines, Eurobond markets are attractive for investors because they offer investors the ability to take on sovereign credit risk without being exposed to risks that are germane to the individual jurisdiction of the issuer countries, such as potential changes to capital control rules, which would restrict the return of capital, and other country risks.

Investors make a distinction between countries with established capital markets (developed markets) and countries where the capital markets are in earlier stages of development (emerging markets). For emerging bond markets, a further distinction is made between bonds issued in local currency and bonds issued in a foreign currency, such as the euro or the US dollar. All else equal, the yield offered by an individual emerging market country for a bond issued denominated in its domestic currency will be higher than the yield offered if the bond were denominated in euros or US dollars. Emerging market currencies are perceived to be riskier, and thus higher yields need to be offered to investors as compensation.

Emerging bond markets are much smaller than developed bond markets. But as demand from local and international investors has increased, issuance and trading of emerging market bonds have risen in volume and value. International investors' interest in emerging market bonds has been triggered by a desire to diversify risk across several jurisdictions in the belief that investment returns across markets are not closely correlated. In addition, emerging market bonds usually offer higher yields (return) than developed market bonds because of the higher perceived risk. Emerging countries typically lag developed countries in the areas of political stability, property rights, and contract enforcement, which often leads to a higher credit risk and higher yields. Many emerging countries, however, are less indebted than their developed counterparts and benefit from higher growth prospects, which appeals to many investors.

Composition of the bond markets varies widely by geography, with many developed and particularly emerging fixed-income markets dominated by sovereign fixed-income securities, with few bonds issued by the private sector and a predominance of bank lending. Potential bond investors may have access to private sector debt securities only indirectly by investing in a bank that holds such loans on its balance sheet. This scenario stands in contrast to markets such as the United States.

\section{Other Classifications of Fixed-Income Markets}
There are various other ways of classifying fixed-income markets. Specific market sectors that are of interest to some investors are inflation-linked bonds and, in some jurisdictions, tax-exempt bonds. Issuance of either type of bond tends to be limited to certain types of issuers. Inflation-linked bonds or linkers are typically issued by governments, government-related entities, and corporate issuers that have an investment-grade rating. They offer investors protection against inflation by linking the coupon payment and/or the principal repayment to an index of consumer prices.

Tax-exempt bonds can be issued only in those jurisdictions that recognize such tax exemption. In the United States for example, there is an income tax exemption for some of the bonds issued by governments or by some non-profit organizations. In particular, local governments can issue municipal bonds (or munis) that are tax exempt (they can also issue taxable municipal bonds, although tax-exempt munis are more frequently issued than taxable munis). Tax-exempt bonds also exist in other jurisdictions. For example, the National Highways Authority of India issues tax-exempt bonds. In countries that implement a capital gains tax, there may be tax exemptions for some types of bonds. In the United Kingdom, for example, government gilts are not subject to capital gains tax.

\section{EXAMPLE 1}
\section{Classification of Fixed-Income Markets}
\begin{enumerate}
  \item Which of the following is most likely an issuer of bonds?
A. Hedge fund
B. Pension fund
C. Local government
\end{enumerate}

\section{Solution:}
$\mathrm{C}$ is correct. Major issuers of bonds include sovereign (national) governments, non-sovereign (local) governments, quasi-government agencies, supranational organizations, and financial and non-financial companies. $A$ and $B$ are incorrect because hedge funds and pension funds are typically investors in, not issuers of, bonds.

\begin{enumerate}
  \setcounter{enumi}{1}
  \item A bond issued by a city would most likely be classified as a:
\end{enumerate}

A. supranational bond.

B. quasi-government bond.

C. non-sovereign government bond.

Solution:

$\mathrm{C}$ is correct. Non-sovereign (local) government bond issuers include provinces, regions, states, and cities. A is incorrect because supranational bonds are issued by international organizations. $B$ is incorrect because quasi-government bonds are issued by agencies that are either owned or sponsored by governments.

\begin{enumerate}
  \setcounter{enumi}{2}
  \item A fixed-income security issued with a maturity at issuance of nine months is most likely classified as a:
\end{enumerate}

A. capital market security.

B. money market security.

C. securitized debt instrument.

Solution:

B is correct. Money market securities are issued with a maturity at issuance (original maturity) ranging from overnight to one year. A is incorrect because capital market securities are issued with an original maturity longer than one year. $C$ is incorrect because securitization, which leads to the creation of securitized debt instruments, does not relate to a bond's maturity but to the process that transforms private transactions between borrowers and lenders into securities traded in public markets.

\begin{enumerate}
  \setcounter{enumi}{3}
  \item The price of a bond issued in the United States by a British company and denominated in US dollars is most likely to:
\end{enumerate}

A. change as US interest rates change.

B. change as British interest rates change.

C. be unaffected by changes in US and British interest rates.

Solution:

A is correct. The currency denomination of a bond's cash flows influences which country's interest rates affect a bond's price. The price of a bond issued by a British company and denominated in US dollars will be affected by US interest rates.

\begin{enumerate}
  \setcounter{enumi}{4}
  \item Interbank offered rates are best described as the rates at which a panel of banks can:
\end{enumerate}

A. issue short-term debt. B. borrow unsecured funds from other major banks.

C. borrow from other major banks against some form of collateral.

\section{Solution:}
$B$ is correct. Interbank offered rates represent a set of interest rates at which major banks believe they could borrow unsecured funds from other major banks in the interbank money market for different currencies and different borrowing periods ranging from overnight to one year.

\begin{enumerate}
  \setcounter{enumi}{5}
  \item A company issues floating-rate bonds. The coupon rate is expressed as the three-month Libor plus a spread. The coupon payments are most likely to increase as:
A. Libor increases.
B. the spread increases.
C. the company's credit quality decreases.
\end{enumerate}

\section{Solution:}
A is correct. The coupon payments of a floating-rate bond that is tied to three-month Libor will reset every three months, based on changes in Libor. Thus, as Libor increases, so will the coupon payments. B is incorrect because the spread on a floating-rate bond is typically constant; it is set when the bond is issued and does not change afterward. $\mathrm{C}$ is incorrect because the issuer's credit quality affects the spread and thus the coupon rate that serves as the basis for the calculation of the coupon payments, but only when the spread is set-that is, at issuance.

\section{Fixed-Income Indexes}
A fixed-income index is a multi-purpose tool used by investors and investment managers to describe a given bond market or sector, as well as to track and quantify the performance of investments and investment managers. Most fixed-income indexes are constructed as portfolios of securities that reflect a particular bond market or sector. The index construction-that is, the security selection and the index weighting-varies among indexes. Index weighting may be based on price, value (market capitalization), or total return.

There are dozens of fixed-income indexes globally, capturing different aspects of the fixed-income markets discussed earlier. One of the most popular set of indexes is the Bloomberg Barclays Global Aggregate Bond Index, which represents a broad-based measure of the global investment-grade fixed-rate bond market. It has an index history beginning on 1 January 1990 and contains three important components: the US Aggregate Bond Index (formerly Lehman Aggregate Bond Index), the Pan-European Aggregate Bond Index, and the Asian-Pacific Aggregate Bond Index. These indexes reflect the investment-grade sectors of the US, European, and Asian-Pacific bond markets, respectively.

With respect to emerging markets, one of the most widely followed indexes is the J.P. Morgan Emerging Market Bond Index (EMBI) Global, which includes US dollar-denominated Brady bonds (bonds issued primarily by Latin American countries in the late 1980s under a debt restructuring plan aimed at converting bank loans into tradable securities), Eurobonds, and loans issued by sovereign and quasi-sovereign entities in several emerging markets. Another popular set of indexes is the FTSE Global Bond Index Series, set up to provide coverage of different classes of securities related to the government and corporate bond markets. It includes indexes of global government bonds, euro-denominated government bonds from emerging markets, sterling- and euro-denominated investment-grade corporate bonds, and covered bonds from Germany and other European Union issuers. Covered bonds are debt obligations issued by banks and backed (secured) by a segregated pool of assets.

Many other fixed-income indexes are available to investors and investment managers to measure and report performance.

It is important to note that constructing a fixed-income index is more challenging than constructing an equity index, because fixed-income markets feature far more securities of different types and more frequent changes in the index as new bonds are issued while others reach maturity.

\section{Investors in Fixed-Income Securities}
The overview of fixed-income markets has so far focused on the supply side. Before discussing bond issuers in greater detail, it is important to consider the demand side because demand for a particular type of bond or issuer may affect supply. After all, market prices are the result of the interaction between demand and supply; neither one can be considered in isolation. For example, an increase in demand for inflation-linked bonds as a result of investors' desire to protect the value of their portfolios against inflation risk may lead governments to issue a greater quantity of this type of bond. By issuing relatively more inflation-linked bonds for which there is demand, a government not only manages to sell its bond issue and obtain the funds required, but it may also benefit from a lower cost of financing.

There are different types of investors in fixed-income securities. Major categories of bond investors include central banks, institutional investors, and retail investors. The first two typically invest directly in fixed-income securities. In contrast, retail investors often invest indirectly, traditionally through fixed-income mutual funds and, during the last two decades, through exchange-traded funds (ETFs), described later.

Central banks use open market operations to implement monetary policy. Open market operations refer to the purchase or sale of bonds, usually but not always sovereign bonds issued by the national government. By purchasing (selling) domestic bonds, central banks increase (decrease) the monetary base in the economy. Central banks may also purchase and sell bonds denominated in foreign currencies as part of their efforts to manage the relative value of the domestic currency and their country's foreign reserves.

Institutional investors, including pension funds, hedge funds, charitable foundations and endowments, insurance companies, and banks, represent the largest groups of investors in fixed-income securities. Another major group of investors is sovereign wealth funds, which are state-owned investment funds that tend to have very long investment horizons and aim to preserve or create wealth for future generations.

Finally, retail investors often invest heavily in fixed-income securities because of the attractiveness of relatively stable prices and steady income production.

Fixed-income markets are dominated by institutional investors, in part because of the high informational barriers to entry and high minimum transaction sizes. Fixed-income securities are far more diverse than equity securities because of the variety of types of issuers and securities. In addition, unlike common shares that are primarily issued and traded in organized markets, the issuance and trading of bonds very often occurs in over-the-counter (OTC) markets. Thus, fixed-income securities are more difficult to access than equity securities. For these reasons, institutional investors tend to invest directly in bonds, whereas most retail investors prefer to use investment vehicles, such as mutual funds and ETFs.

\section{EXAMPLE 2}
\section{Investors in Fixed-Income Securities}
\begin{enumerate}
  \item Open market operations describe the process used by central banks to buy and sell bonds to:
A. implement fiscal policy.
B. control the monetary base.
C. issue and repay government debt.
\end{enumerate}

\section{Solution:}
B is correct. Open market operations refer to the purchase or sale of bonds, usually sovereign bonds issued by the national government, as a means of implementing monetary policy. By purchasing (selling) bonds, central banks increase (decrease) the monetary base in the economy, thus controlling the money supply. A is incorrect because open market operations help facilitate monetary policy, not fiscal policy (which is the taxing and spending by the national government). $C$ is incorrect because although Treasury departments and some central banks may facilitate the issuance and repayment of government debt, open market operations specifically refer to the implementation of monetary policy.

\begin{enumerate}
  \setcounter{enumi}{1}
  \item Retail investors most often:
A. do not invest in fixed-income securities.
B. invest directly in fixed-income securities.
C. invest indirectly in fixed-income securities through mutual funds or exchange-traded funds.
\end{enumerate}

\section{Solution:}
$\mathrm{C}$ is correct. Retail investors often invest in fixed-income securities because of the attractiveness of relatively stable prices and steady income production. Because most retail investors lack the expertise to value fixed-income securities and are not large enough investors to buy and sell them directly, however, they usually invest in fixed income indirectly through mutual funds and exchange-traded funds.

\section{PRIMARY BOND MARKETS}
$\square \quad$ describe mechanisms available for issuing bonds in primary markets

Primary bond markets are markets in which issuers initially sell bonds to investors to raise capital. In contrast, secondary bond markets are markets in which existing bonds are subsequently traded among investors. As with all financial markets, primary and secondary bond markets are regulated within the framework of the overall financial system. An established independent regulatory authority is usually responsible for overseeing both the structure of the markets and the credentials of market participants.

\section{Primary Bond Markets}
Issuances in primary bond markets are frequent. Different bond issuing mechanisms are used depending on the type of issuer and the type of bond issued. A bond issue can be sold via a public offering (or public offer), in which any member of the public may buy the bonds, or via a private placement, in which only a selected investor, or group of investors, may buy the bonds.

\section{Public Offerings}
Investment banks play a critical role in bond issuance by assisting the issuer in accessing the primary market and by providing an array of financial services. The most common bond issuing mechanisms are underwritten offerings, best-efforts offerings, and auctions. In an underwritten offering, also called a firm commitment offering, the investment bank guarantees the sale of the bond issue at an offering price that is negotiated with the issuer. Thus, the investment bank, called the underwriter, takes the risk associated with selling the bonds. In contrast, in a best-efforts offering, the investment bank serves only as a broker. It tries to sell the bond issue at the negotiated offering price only if it is able to for a commission. Thus, the investment bank has less risk and correspondingly less incentive to sell the bonds in a best-efforts offering than in an underwritten offering. An auction is a bond issuing mechanism that involves bidding.

\section{Underwritten Offerings}
Underwritten offerings are typical bond-issuing mechanisms for corporate bonds, some local government bonds (such as municipal bonds in the United States), and some asset-backed securities (such as mortgage-backed securities). The underwriting process typically includes six phases.

The underwriting process starts with the determination of the funding needs. Often with the help of an adviser or advisers, the issuer must determine how much money must be raised, the type of bond offering, and whether the bond issue should be underwritten.

Once the issuer has decided that the bond issue should be underwritten, it must select the underwriter, which is typically an investment bank. The underwriter of a bond issue takes the risk of buying the newly issued bonds from the issuer, and it then resells them to investors or to dealers that then sell them to investors. The difference between the purchase price of the new bond issue and the reselling price to investors is the underwriter's revenue. A relatively small-size bond issue may be underwritten by a single investment bank. It is more common for larger bond issues, however, to be underwritten by a group, or syndicate, of investment banks. In this case, the bond issue is referred to as a syndicated offering. A lead underwriter invites other investment banks to join the syndicate and coordinates the effort. The syndicate is collectively responsible for determining the pricing of the bond issue and for placing (selling) the bonds with investors.

The third phase of an underwritten offering is to structure the transaction. Before the bond issue is announced, the issuer and the lead underwriter discuss the terms of the bond issue, such as the bond's notional principal (total amount), the coupon rate, and the expected offering price. The underwriter or the syndicate typically organizes the necessary regulatory filings and prepares the offering circular or prospectus that provides information about the terms of the bond issue. The issuer must also choose a trustee, which is typically a trust company or the trust department of a bank, to oversee the master bond agreement. The bond offering is formally launched the day the transaction is announced, usually in the form of a press release. The announcement specifies the new bond issue's terms and conditions, including the bond's features, such as the maturity date, the currency denomination, and the expected coupon range, as well as the expected offering price. The issuer also releases the offering circular or prospectus. The final terms may differ from these terms because of changes in market conditions between the announcement day and the pricing day.

The success of the bond issue depends on the underwriter or syndicate's discernment in assessing market conditions and in pricing the bond issue accordingly. The pricing of the bond issue is, therefore, an important phase of an underwritten offering. Ideally, the bond issue should be priced so that the amount of bonds available equals investors' demand for the bonds. If the offering price is set too high, the offering will be undersubscribed-that is, there will be insufficient demand for the bond issue. As a consequence, the underwriter or syndicate will fail to sell the entire bond issue. Alternatively, if the offering price is set too low, the offering will be oversubscribed. Underwriters may aim at a small oversubscription because it reduces the risk of being unable to sell the entire bond issue. But a large oversubscription indicates that the offering terms were probably unfavorable to the issuer, in that the issuer might have raised the desired amount of capital at a lower coupon rate.

Between the announcement of a bond issue and the end of the subscription period, the underwriter or syndicate must gauge the demand for the bond issue and the price at which the bond should be offered to ensure that the entire bond issue is placed without running the risk of a large oversubscription. There are different ways for underwriters to do so. The bond issue is usually marketed to potential investors, either through an indirect approach-such as an advertisement in a newspaper, a commonly used approach for bond issued by household names-or through direct marketing and road shows, aimed at institutional investors such as pension funds and insurance companies. The underwriter or syndicate may also approach large institutional investors and discuss with them the kind of bond issues they are willing to buy. These buyers are known as the "anchor." For some, but not all, bond issues, the grey market is another way for underwriters to gauge investors' interest. The grey market, also called the "when issued" market, is a forward market for bonds about to be issued. Trading in the grey market helps underwriters determine what the final offering price should be.

The pricing day is the last day when investors can commit to buy the bond issue, and it is also the day when the final terms of the bond issue are agreed on. The following day, called the "offering day," the underwriting agreement that includes the bond issue's final terms is signed. The underwriting process then enters the issuing phase. The underwriter or the syndicate purchases the entire bond issue from the issuer, delivers the proceeds, and starts reselling the bonds through its sales network.

The underwriting process comes to an end about 14 days later, on the closing day, when the bonds are delivered to investors. Investors no longer receive a paper settlement; instead, the bond itself is represented by a global note that is typically held by the paying agent.

\section{Shelf Registrations}
A shelf registration allows certain authorized issuers to offer additional bonds to the general public without having to prepare a new and separate offering circular for each bond issue. Rather, the issuer prepares a single, all-encompassing offering circular that describes a range of future bond issuances, all under the same document. This master prospectus may be in place for years before it is replaced or updated, and it may be used to cover multiple bond issuances.

Under a shelf registration, each individual offering is prefaced with a short issue announcement document. This document must confirm that there has been no change to material elements of the issuer's business, or otherwise describe any changes to the issuer's financial condition since the master prospectus was filed. Because shelf issuances are subject to a lower level of scrutiny compared with standard public offerings, they are an option only for well-established issuers that have convinced the regulatory authorities of their financial strength. Additionally, certain jurisdictions may allow shelf registrations to be purchased only by "qualified" institutional investors-that is, institutional investors that meet a set of criteria set forth by the regulators.

\section{Auctions}
An auction is a method that involves bidding. It is helpful in providing price discovery (i.e., it facilitates supply and demand in determining prices) and in allocating securities. In many countries, most sovereign bonds are sold to the public via a public auction. For example, in 2017, the United States conducted 277 public auctions and issued approximately $\$ 8.5$ trillion of new securities such as Treasury bills, notes, bonds, and Treasury Inflation-Protected Securities (TIPS). The public auction process used in the United States is a single-price auction through which all the winning bidders pay the same price and receive the same coupon rate for the bonds. In contrast, the public auction process used in Canada and Germany is a multiple-price auction process, which generates multiple prices and yields for the same bond issue.

The US sovereign bond market is one of the largest and most liquid bond markets globally, so we will illustrate the US single-price auction process. This process includes three phases: announcement, bidding, and issuance. First, the US Treasury announces the auction and provides information about the bond issue, such as the amount of securities being offered, the auction date, the issue date, the maturity date, bidding close times, and other pertinent information.

After the auction announcement is made, dealers, institutional investors, and individual investors may enter competitive or non-competitive bids. With competitive bids, a bidder specifies the rate (yield) that is considered acceptable; if the rate determined at auction is lower than the rate specified in the competitive bid, the investor will not be offered any securities. In contrast, with non-competitive bids, a bidder agrees to accept the rate determined at auction; non-competitive bidders always receive their securities. At the close of the auction, the US Treasury accepts all non-competitive bids and competitive bids in ascending order of their rates (lowest to highest) until the amount of bids is equal to the amount the issuer requires. All bidders receive the same rate, based on the highest accepted bid. This single-price auction process encourages aggressive bidding and potentially results in a lower cost of funds (i.e., lower coupon rate) for the US Treasury because all the winning bidders pay the same price.

On the issue day, the US Treasury delivers the securities to the winning bidders and collects the proceeds from investors. After the auction process is complete, the securities are traded in secondary markets like other securities.

Exhibit 4 shows the results of a US Treasury public auction. Exhibit 4: Results of a US Treasury Public Auction on 23 July 2018

\section{TREASURY NEWS}
Department of the Treasury Bureau of tbe Fiscal service

\begin{center}
\includegraphics[max width=\textwidth]{2023_05_04_7b535d0a870224f62e3dg-511(4)}
\end{center}

TRE ASURY AUCTION RE SULTS

\begin{center}
\includegraphics[max width=\textwidth]{2023_05_04_7b535d0a870224f62e3dg-511(5)}
\end{center}

\begin{center}
\includegraphics[max width=\textwidth]{2023_05_04_7b535d0a870224f62e3dg-511}
\end{center}

'Equivilest cupon ives rield

$390 \%$ of the amovet of acoeptad competive tonders was tecdered at or bel ow that ras

49\% of the a movet of accepsad compeitive tenders was tecdersd a or bel ow that ras.

Bid-to-Cover Ratio: $\$ 130,722,409,3001545,000,169,300=2,90$

aA wads to TresyryDirect $=\$ 394,0,42,100$

\begin{center}
\includegraphics[max width=\textwidth]{2023_05_04_7b535d0a870224f62e3dg-511(2)}
\end{center}

\begin{center}
\includegraphics[max width=\textwidth]{2023_05_04_7b535d0a870224f62e3dg-511(3)}
\end{center}

\begin{center}
\includegraphics[max width=\textwidth]{2023_05_04_7b535d0a870224f62e3dg-511(1)}
\end{center}

Bent of New Yod.

Source: Based on information from \href{http://www.treasurydirect.gov}{www.treasurydirect.gov}.

The rate determined at auction was $2.140 \%$. T-bills are pure discount bonds; they are issued at a discount to par value and redeemed at par. Investors paid $98.918111 \%$ of par-that is, $\$ 9,891.81$ per $\$ 10,000$ in par value. The US Treasury received bids for $\$ 130.7$ billion but raised only $\$ 45.0$ billion. All the non-competitive bids ( $\$ 793.5$ million) were accepted, but only a third ( $\$ 43.2$ of the $\$ 128.9$ billion) of competitive bids were accepted. Note that half the competitive bids were submitted with a rate lower than $2.110 \%$ (the median rate). All successful bidders, however, received the rate of $2.140 \%$, which is the essential feature of a single-price auction.

Exhibit 4 also identifies the types of bidders. Most US Treasury securities are bought at auction by primary dealers. Primary dealers are financial institutions that are authorized to deal in new issues of US Treasury securities. They have established business relationships with the Federal Reserve Bank of New York (New York Fed), which implements US monetary policy. Primary dealers serve primarily as trading counterparties of the New York Fed and are required to participate meaningfully in open market operations and in all auctions of US Treasury securities. They also provide the New York Fed with market information. Institutional investors and central banks are the largest investors in US Treasury securities; only a very small amount of these bonds is purchased directly by individual investors.

\section{Private Placements}
A private placement is typically a non-underwritten, unregistered offering of bonds that are sold only to an investor or a small group of investors. Typical investors in privately placed bonds are large institutional investors. A private placement can be accomplished directly between the issuer and the investor(s) or through an investment bank. Because privately placed bonds are unregistered and may be restricted securities that can be purchased by only some types of investors, there is usually no active secondary market to trade them. However, trading may be possible under certain conditions. For example, restricted securities issued under Rule 144A in the United States cannot be sold to the public, but they can be traded among qualified institutional investors. Even if trading is possible, privately placed bonds typically exhibit lower liquidity than publicly issued bonds. Insurance companies and pension funds are major buyers of privately placed bonds because they do not need every security in their portfolios to be liquid and they often value the additional yield offered by these bonds.

Private placements sometimes represent a step in the company's financing evolution between syndicated loans (loans from a group of lenders to a single borrower further discussed later) and public offerings. Privately placed bonds are often issued in small aggregate amounts, at times by unknown issuers. Many investors may be unwilling to undertake the credit analysis that is required for a new name, in particular if the offering amount is small. Unlike in a public offering in which the bonds are often sold to investors on a take-it-or-leave-it basis, investors in a private placement can influence the structure of the bond issue, including such considerations as asset and collateral backing, credit enhancements, and covenants. It is common for privately placed bonds to have more customized and restrictive covenants than publicly issued ones. In addition to being able to negotiate the terms of the bonds and align those terms with their needs, investors in private placements are rewarded by receiving the bonds, which is not always the case in public offerings in which investors cannot know for sure when the issue will become available and how many securities they will be allocated.

Private placements are also offered by regular bond issuers, in particular for smaller amounts of capital raised in major currencies, such as US dollars, euros, or sterling. Private placements are usually more flexible than public offerings and allow regular issuers to tailor the bond issue to their own needs.

\section{SECONDARY BOND MARKETS}
describe secondary markets for bonds

Secondary markets, also called the "aftermarket," are where existing securities are traded among investors. Securities can be traded directly from investor to investor, through a broker or dealer, or directly from issuer to investor by way of bond tender offer. The major participants in secondary bond markets globally are large institutional investors and central banks. The direct presence of retail investors in secondary bonds markets is limited, unlike in secondary equity markets. Retail investors do participate in secondary bond markets, albeit indirectly, through mutual and exchange-traded funds (ETFs) offered by fund management companies.

There are three main ways for secondary markets to be structured: as an organized exchange, as an over-the-counter market, or as a bond tender offer. An organized exchange provides a place where buyers and sellers can meet to arrange their trades. Although buy or sell orders may come from anywhere, the transaction must take place at the exchange according to the rules imposed by the exchange. In contrast, with over-the-counter (OTC) markets, buy and sell orders initiated from various locations are matched through a communications network. Thus, OTC markets need electronic trading platforms over which users submit buy and sell orders. Bloomberg Fixed Income Electronic Trading platform is an example of such a platform through which dealers stand ready to trade in multiple bond markets globally. Although there is some trading of government bonds and very active corporate bonds on many stock exchanges around the world, the vast majority of bonds are traded in OTC markets. A bond tender offer involves an issuer making an offer to existing bondholders of the company to repurchase a specified number of bonds at a particular price and a specified time. Companies use such offers to restructure or refinance their current capital structure. Investors generally receive a price that is above the market value of the bond, thereby enticing them to participate; yet this price is likely below par, and hence also a win for the issuer. Note, however, that the price at which the bonds are repurchased is not determined in advance at the time of the original issuance of the bond. It merely reflects the market conditions at the time of the repurchase.

The liquidity demands of fixed-income investors have evolved since the early 1990s. The type of investors that would buy and hold a bond to maturity, which once dominated the fixed-income markets, has been supplanted by institutional investors who trade actively. The dynamics of global fixed-income markets reflect this change in the relative demand for liquidity.

We will illustrate how secondary markets work by using the example of Eurobonds. The most important Eurobond trading center by volume is in London, although a large number of market participants are also based in Brussels, Frankfurt, Zurich, and Singapore. Liquidity is supplied by Eurobond market makers, of which approximately 35 are registered with the International Capital Market Association (ICMA). ICMA is an association of banks and other financial institutions that provides a regulatory framework for international bond markets and that is behind much of the established uniform practices observed by all market participants in the Eurobond market.

The key to understanding how secondary bond markets are structured and function is to understand liquidity. Liquidity refers to the ability to trade (buy or sell) securities quickly and easily at prices close to their fair market value. Liquidity involves much more than "how quickly one can turn a bond into cash." This statement implicitly assumes a long position, but some market participants need to buy quickly when covering a short position. The other aspect of liquidity that is often ignored is that speed of trading alone does not constitute a liquid market. One can always buy something quickly by offering a very high price or sell something quickly by accepting a very low price. In a liquid market, trading takes place quickly at prices close to the security's fair market value.

A key indicator and measurement of liquidity is the bid-offer spread or bid-ask spread, which reflects the prices at which dealers will buy from a customer (bid) and sell to a customer (offer or ask). It can be as low as 5 bps for very liquid bond issues, such as issues of the World Bank, to no price quoted for illiquid issues. A reasonable spread is of the order of $10 \mathrm{bps}$ to $12 \mathrm{bps}$, whereas an illiquid spread may be in excess of $50 \mathrm{bps}$. When there is no bid or offer price, the issue is completely illiquid for trading purposes.

Settlement is the process that occurs after the trade is made. The bonds are passed to the buyer, and payment is received by the seller. Secondary market settlement for government and quasi-government bonds typically takes place either on a cash basis or on a $T+1$ basis. With cash settlement, trading and settlement occur on the same day. With $T+1$ settlement, settlement takes place the day after the trade date. In contrast, corporate bonds usually settle on a $T+2$ or $T+3$ basis, although settlement can extend to $T+7$ in some jurisdictions. Trades clear within either or both of the two main clearing systems, Euroclear and Clearstream. Settlement occurs by means of a simultaneous exchange of bonds for cash on the books of the clearing system. An electronic bridge connecting Euroclear and Clearstream allows transfer of bonds from one system to the other, so it is not necessary to have accounts at both systems. Both systems operate on a paperless, computerized book-entry basis, although a bond issue is still represented by a physical document, the global note mentioned earlier. All participants in either system will have their own internal account set up, and they may also act as agent for buyers or sellers who do not possess an account.

\section{EXAMPLE 3}
\section{Bond Markets}
\begin{enumerate}
  \item Which of the following best describes a primary market for bonds? A market:
\end{enumerate}

A. in which bonds are issued for the first time to raise capital.

B. that has a specific location where the trading of bonds takes place.

C. in which existing bonds are traded among individuals and institutions.

\section{Solution:}
A is correct. Primary bond markets are markets in which bonds are issued for the first time to raise capital. B is incorrect because having a specific location where the trading of bonds takes place is not a requirement for a primary bond market. $\mathrm{C}$ is incorrect because a market in which existing bonds are traded among individuals and institutions is the definition of a secondary, not primary, market.

\begin{enumerate}
  \setcounter{enumi}{1}
  \item US Treasury bonds are typically sold to the public via a(n):
A. auction.
B. primary dealer.
C. secondary bond market.
\end{enumerate}

\section{Solution:}
A is correct. US Treasury bonds are typically sold to the public via an auction. B is incorrect because primary dealers are often bidders in the auction; they are financial institutions that are active in trading US Treasury bonds. $\mathrm{C}$ is incorrect because any bond issue coming directly to the market is considered to be in the primary, not the secondary, market.

\begin{enumerate}
  \setcounter{enumi}{2}
  \item In a single-price bond auction, an investor who places a competitive bid and specifies a rate that is above the rate determined at auction will most likely:
\end{enumerate}

A. not receive any bonds.

B. receive the bonds at the rate determined at auction.

C. receive the bonds at the rate specified in the investor's competitive bid.

\section{Solution:}
A is correct. In a single-price bond auction, a bidder that enters a competitive bid specifies the rate (yield) that is considered acceptable. If the rate specified in the competitive bid is above the rate (yield) determined at auction, the investor will not be offered any securities.

\begin{enumerate}
  \setcounter{enumi}{3}
  \item A bond purchased in a secondary market is most likely purchased from:
\end{enumerate}

A. the bond's issuer.

B. the bond's lead underwriter.

C. another investor in the bond.

\section{Solution:}
$\mathrm{C}$ is correct. Secondary bond markets are where bonds are traded among investors. A and B are incorrect because a bond purchased from the bond's issuer or from the bond's lead underwriter would happen in the primary, not secondary, market.

\begin{enumerate}
  \setcounter{enumi}{4}
  \item Corporate bonds will most likely settle:
\end{enumerate}

A. on the trade date

B. on the trade date plus one day.

C. by the trade date plus three days.

\section{Solution:}
C is correct. Corporate bonds typically settle on a $T+2$ or $T+3$ basis-that is, two or three days after the trade date-although settlement can extend to $T+7$ in some jurisdictions. A and B are incorrect because it is government and quasi-government bonds, not corporate bonds, that typically settle either on a cash basis or on a $T+1$ basis

\section{SOVEREIGN BONDS}
National governments issue bonds primarily for fiscal reasons-to fund spending when tax revenues are insufficient to cover expenditures. To meet their spending goals, national governments issue bonds in various types and amounts. This section discusses bonds issued by national governments, often referred to as sovereign bonds or sovereigns.

\section{Characteristics of Sovereign Bonds}
Sovereign bonds denominated in local currency have different names in different countries. For example, they are named US Treasuries in the United States, Japanese government bonds (JGBs) in Japan, gilts in the United Kingdom, Bunds in Germany, and obligations assimilables du Trésor (OATs) in France. Some investors or market participants may refer to sovereign bonds as Treasury securities or Treasuries for short, on the principle that the national Treasury department is often in charge of managing a national government's funding needs.

Names may also vary depending on the original maturity of the sovereign bond. For example, US government bonds are named Treasury bills (T-bills) when the original maturity is one year or shorter, Treasury notes (T-notes) when the original maturity is longer than one year and up to 10 years, and Treasury bonds (T-bonds) when the original maturity is longer than 10 years. In Spain, the sovereigns issued by Tesoro Público are named letras del Tesoro, bonos del Estado, and obligaciones del Estado, depending on the sovereign's original maturity: one year or shorter, longer than one year and up to five years, or longer than five years, respectively. Although very rare, some bonds, such as the consols in the United Kingdom, have no stated maturity date.

The majority of the trading in secondary markets is of sovereign securities that were most recently issued, called on-the-run securities. The latest sovereign bond issue for a given maturity is also referred to as a benchmark issue because it serves as a benchmark against which to compare bonds that have the same features (i.e., maturity, coupon type and frequency, and currency denomination) but that are issued by another type of issuer (e.g., non-sovereign, corporate). As a general rule, as sovereign securities age, they trade less frequently.

One salient difference between money market securities, such as T-bills, and capital market securities, such as T-notes and T-bonds, is the interest provision. T-bills are pure discount bonds; they are issued at a discount to par value and redeemed at par. The difference between the par value and the issue price is the interest paid on the borrowing. In contrast, capital market securities are typically coupon (or coupon-bearing) bonds; these securities make regular coupon payments and repay the par value at maturity. Bunds pay coupons annually, whereas US Treasuries, JGBs, gilts, and OATs make semi-annual coupon payments.

\section{Credit Quality of Sovereign Bonds}
Sovereign bonds are usually unsecured obligations of the sovereign issuer-that is, they are backed not by collateral but by the taxing authority of the national government. When a national government runs a budget surplus, excess tax revenues over expenditures are the primary source of funds for making interest payments and repaying the principal. In contrast, when a country runs a budget deficit, the source of the funds used for the payment of interest and repayment of principal comes either from tax revenues and/or by "rolling over" (refinancing) existing debt into new debt.

Highly rated sovereign bonds denominated in local currency are virtually free of credit risk. Credit rating agencies assign ratings to sovereign bonds, and these ratings are called sovereign ratings. The highest rating (i.e., highest credit quality and lowest credit risk) is AAA by S\&P and Fitch and Aaa by Moody's. In 2018, only a handful of sovereign issuers were rated at this (theoretically) risk-free level by these three credit rating agencies, including Germany, Singapore, Canada, Sweden, Norway, Denmark, Luxembourg, Australia, Switzerland, and the Netherlands. Notably, S\&P downgraded the United States from AAA to AA+ in 2011 and downgraded the United Kingdom from AAA to AA, two notches, in 2016 following the referendum that approved the United Kingdom's exit from the European Union.

Credit rating agencies distinguish between bonds issued in the sovereign's local currency and bonds issued in a foreign currency. In theory, a government can make interest payments and repay the principal by generating cash flows from its unlimited power (in the short run at least) to tax its citizens. A national government also has the ability to print its own currency, whereas it is restricted in being able to pay in a foreign currency only what it earns in exports or can exchange in financial markets. Thus, it is common to observe a higher credit rating for sovereign bonds issued in local currency than for those issued in a foreign currency. But there are limits to a government's ability to reduce the debt burden. As the sovereign debt crisis that followed the global financial crisis has shown, taxing citizens can only go so far in paying down debt before the taxation becomes an economic burden. Additionally, printing money only serves to weaken a country's currency relative to other currencies over time.

The national government of a country that has a strong domestic savings base has the luxury of being able to issue bonds in its local currency and sell them to domestic investors. If the local currency is liquid and freely traded, the sovereign issuer may also attract international investors who may want to hold that sovereign issuer's bonds and have exposure to that country's local currency. A national government may also issue debt in a foreign currency when there is demand for the sovereign issuer's bonds but not necessarily in the sovereign's local currency. For example, demand from overseas investors has caused national governments such as Switzerland and Sweden to issue sovereign bonds in US dollars and euros. Emerging market countries may also have to issue in major currencies because international investors may be willing to accept the credit risk but not the foreign exchange (currency) risk associated with emerging market bonds. The yield offered for bonds issued in major currencies will be lower because of the lower risk profile. When a sovereign issuer raises debt in a foreign currency, it usually swaps the proceeds into its local currency for use and deployment.

\section{Types of Sovereign Bonds}
National governments issue different types of bonds, some of them paying a fixed rate of interest and others paying a floating rate, including inflation-linked bonds.

\section{Fixed-Rate Bonds}
Fixed-rate bonds (i.e., bonds that pay a fixed rate of interest) are by far the most common type of sovereign bond. National governments routinely issue two types of fixed-rate bonds: zero-coupon bonds (or pure discount bonds) and coupon bonds. A zero-coupon bond does not pay interest. Instead, it is issued at a discount to par value and redeemed at par at maturity. Coupon bonds are issued with a stated rate of interest and make interest payments periodically, such as semi-annually or annually. They have a terminal cash flow equal to the final interest payment plus the par value. As mentioned earlier, most sovereign bonds with an original maturity of one year or less are zero-coupon bonds, whereas bonds with an original maturity longer than one year are typically issued as coupon bonds.

\section{Floating-Rate Bonds}
The price of a bond changes in the opposite direction from the change in interest rates, a relationship that is fully explained later when discussing the risk and return of fixed-income securities. Thus, investors who hold fixed-rate bonds are exposed to interest rate risk: As interest rates increase, bond prices decrease, which lowers the value of their portfolio. In response to public demand for less interest rate risk, some national governments around the world issue bonds with a floating rate of interest that resets periodically based on changes in the level of a reference rate such as Libor. Although interest rate risk still exists on floating-rate bonds, it is far less pronounced than that on fixed-rate bonds.

Examples of countries where the national government issues floating-rate bonds include Germany, Spain, and Belgium in developed markets and Brazil, Turkey, Mexico, Indonesia, and Poland in emerging markets. The largest sovereign issuer, the United States, began issuing floating-rate bonds in January 2014. Two other large sovereign issuers, Japan and the United Kingdom, have never issued bonds whose coupon rate is tied to a reference rate.

\section{Inflation-Linked Bonds}
Fixed-income investors are exposed to inflation risk. The cash flows of fixed-rate bonds are fixed by contract. If a particular country experiences an inflationary episode, the purchasing power of the fixed cash flows erodes over time. Thus, to respond to the demand for less inflation risk, many national governments issue inflation-linked bonds, or linkers, whose cash flows are adjusted for inflation. First issuers of inflation-linked bonds were the governments of Argentina, Brazil, and Israel. The United States introduced inflation-linked securities in January 1997, calling them Treasury Inflation-Protected Securities (TIPS). Other countries where the national government has issued inflation-linked bonds include the United Kingdom, Sweden, Australia, and Canada in developed markets and Brazil, South Africa, and Chile in emerging markets.

As explained in the earlier coverage of the defining elements of fixed-income securities, the index to which the coupon payments and/or principal repayments are linked is typically an index of consumer prices. Inflation-linked bonds can be structured a variety of ways: The inflation adjustment can be made via the coupon payments, the principal repayment, or both. In the United States, the index used is the Consumer Price Index for All Urban Consumers (CPI-U). In the United Kingdom, it is the Retail Price Index (RPI) (All Items). In France, there are two inflation-linked bonds with two different indexes: the French consumer price index (CPI) (excluding tobacco) and the Eurozone's Harmonized Index of Consumer Prices (HICP) (excluding tobacco). Although linking the cash flow payments to a consumer price index reduces inflation risk, it does not necessarily eliminate the effect of inflation completely because the consumer price index may be an imperfect proxy for inflation.

\section{EXAMPLE 4}
\section{Sovereign Bonds}
\begin{enumerate}
  \item Sovereign debt with a maturity at issuance shorter than one year most likely consists of:
\end{enumerate}

A. floating-rate instruments.

B. zero-coupon instruments.

C. coupon-bearing instruments.

Solution:

B is correct. Most debt issued by national governments with a maturity at issuance (original maturity) shorter than one year takes the form of zero-coupon instruments. A and $\mathrm{C}$ are incorrect because floating-rate and coupon-bearing instruments are typically types of sovereign debt with maturities longer than one year.

\begin{enumerate}
  \setcounter{enumi}{1}
  \item Floating-rate bonds are issued by national governments as the best way to reduce:
A. credit risk.
B. inflation risk.
C. interest rate risk.
\end{enumerate}

\section{Solution:}
$\mathrm{C}$ is correct. The coupon rates of floating-rate bonds are reset periodically based on changes in the level of a reference rate such as Libor, which reduces interest rate risk. A is incorrect because credit risk, although low for sovereign bonds, cannot be reduced by linking the coupon rate to a reference rate. $\mathrm{B}$ is incorrect because although inflation risk is lower for floating-rate bonds than for fixed-rate bonds, floating-rate bonds are not as good as inflation-linked bonds to reduce inflation risk.

\begin{enumerate}
  \setcounter{enumi}{2}
  \item Sovereign bonds whose coupon payments and/or principal repayments are adjusted by a consumer price index are most likely known as:
A. linkers.
B. floaters.
C. consols.
\end{enumerate}

\section{Solution:}
A is correct because sovereign bonds whose coupon payments and/or principal repayment are adjusted by a consumer price index are known as inflation-linked bonds or linkers. B is incorrect because floaters describe floating-rate bonds that have a coupon rate tied to a reference rate such as Libor. C is incorrect because consols are sovereign bonds with no stated maturity date issued by the UK government.

\section{NON-SOVEREIGN, QUASI-GOVERNMENT, AND SUPRANATIONAL BONDS}
 describe securities issued by non-sovereign governments,quasi-government entities, and supranational agencies

This section covers the bonds issued by local governments and by government-related entities.

\section{Non-Sovereign Bonds}
Levels of government below the national level, such as provinces, regions, states, and cities, issue bonds called non-sovereign government bonds or non-sovereign bonds. These bonds are typically issued to finance public projects, such as schools, motorways, hospitals, bridges, and airports. The sources for paying interest and repaying the principal include the taxing authority of the local government, the cash flows of the project the bond issue is financing, or special taxes and fees established specifically for making interest payments and principal repayments. Non-sovereign bonds are typically not guaranteed by the national government.

As mentioned earlier, bonds issued by state and local governments in the United States are known as municipal bonds, and they often offer income tax exemptions. In the United Kingdom, non-sovereign bonds are known as local authority bonds. Other non-sovereign bonds include those issued by state authorities such as the 16 Länder in Germany.

Credit ratings for non-sovereign bonds vary widely because of the differences in credit and collateral quality. Because default rates of non-sovereign bonds are historically low, they very often receive high credit ratings. However, non-sovereign bonds usually trade at a higher yield and lower price than sovereign bonds with similar characteristics. The additional yield depends on the credit quality, the liquidity of the bond issue, and the implicit or explicit level of guarantee or funding commitment from the national government. The additional yield is the lowest for non-sovereign bonds that have high credit quality, are liquid, and are guaranteed by the national government.

\section{Quasi-Government Bonds}
National governments establish organizations that perform various functions for them. These organizations often have both public and private sector characteristics, but they are not actual governmental entities. They are referred to as quasi-government entities, although they take different names in different countries. These quasi-government entities often issue bonds to fund specific financing needs. These bonds are known as quasi-government bonds or agency bonds.

Examples of quasi-government entities include government-sponsored enterprises (GSEs) in the United States, such as the Federal National Mortgage Association ("Fannie Mae"), the Federal Home Loan Mortgage Corporation ("Freddie Mac"), and the Federal Home Loan Bank (FHLB). Other examples of quasi-government entities that issue bonds include Hydro Quebec in Canada or the Japan Bank for International Cooperation (JBIC). In the case of JBIC's bonds, timely payments of interest and repayment of principal are guaranteed by the Japanese government. Most quasi-government bonds, however, do not offer an explicit guarantee by the national government, although investors often perceive an implicit guarantee.

Because a quasi-government entity typically does not have direct taxing authority, bonds are repaid from the cash flows generated by the entity or from the project the bond issue is financing. Quasi-government bonds may be backed by collateral, but this is not always the case. Quasi-government bonds are usually rated very high by the credit rating agencies because historical default rates are extremely low. Bonds that are guaranteed by the national government receive the highest ratings and trade at a lower yield and higher price than otherwise similar bonds that are not backed by the sovereign government's guarantee.

\section{Supranational Bonds}
A form of often highly rated bonds is issued by supranational agencies, also referred to as multilateral agencies. The most well-known supranational agencies are the International Bank for Reconstruction and Development (the World Bank), the International Monetary Fund (IMF), the European Investment Bank (EIB), the Asian Development Bank (ADB), and the African Development Bank (AFDB). Bonds issued by supranational agencies are called supranational bonds. Supranational bonds are typically plain-vanilla bonds, although floating-rate bonds and callable bonds are sometimes issued. Highly rated supranational agencies, such as the World Bank, frequently issue large-size bond issues that are often used as benchmarks issues when there is no liquid sovereign bond available.

\section{EXAMPLE 5}
\section{Non-Sovereign Government, Quasi-Government, and Supranational Bonds}
\begin{enumerate}
  \item Relative to sovereign bonds, non-sovereign bonds with similar characteristics most likely trade at a yield that is:
A. lower.
B. the same.
C. higher.
\end{enumerate}

\section{Solution:}
$C$ is correct. Non-sovereign bonds usually trade at a higher yield and lower price than sovereign bonds with similar characteristics. The higher yield is because of the higher credit risk associated with non-sovereign issuers relative to sovereign issuers, although default rates of local governments are historically low and their credit quality is usually high. The higher yield may also be a consequence of non-sovereign bonds being less liquid than sovereign bonds with similar characteristics.

\begin{enumerate}
  \setcounter{enumi}{1}
  \item Bonds issued by a governmental agency are most likely:
\end{enumerate}

A. repaid from the cash flows generated by the agency.

B. guaranteed by the national government that sponsored the agency.

C. backed by the taxing power of the national government that sponsored the agency.

\section{Solution:}
A is correct. Most bonds issued by a governmental agency are repaid from the cash flows generated by the agency or from the project the bond issue is financing. $\mathrm{B}$ and $\mathrm{C}$ are incorrect because although some bonds issued by governmental agencies are guaranteed by the national government or are backed by the taxing power of the national government that sponsored the agency, bonds are most likely repaid first from the cash flows generated by the agency.

CORPORATE DEBT: BANK LOANS, SYNDICATED LOANS, AND COMMERCIAL PAPER

describe types of debt issued by corporations Companies differ from governments and government-related entities in that their primary goal is profit; profitability is an important consideration when companies make decisions, including financing decisions. Companies routinely raise debt as part of their overall capital structure, both to fund short-term spending needs (e.g., working capital) as well as long-term capital investments. We have so far focused on publicly issued debt, but loans from banks and other financial institutions are a significant part of the debt raised by companies. For example, it is estimated that European companies meet $75 \%$ of their borrowing needs from banks and only $25 \%$ from financial markets. In Japan, these percentages are $80 \%$ and $20 \%$, respectively. In the United States, however, debt capital is much more significant: $80 \%$ is from financial markets and just 20\% from bank lending (SIFMA 2018 Outlook: Through the Looking Glass).

\section{Bank Loans and Syndicated Loans}
A bilateral loan is a loan from a single lender to a single borrower. Companies routinely use bilateral loans from their banks, and these bank loans are governed by the bank loan documents. Bank loans are the primary source of debt financing for small and medium-size companies as well as for large companies in countries where bond markets are either underdeveloped or where most bond issuances are from government, government-related entities, and financial institutions. Access to bank loans depends not only on the characteristics and financial health of the company but also on market conditions and bank capital availability.

A syndicated loan is a loan from a group of lenders, called the "syndicate," to a single borrower. A syndicated loan is a hybrid between relational lending and publicly traded debt. Syndicated loans are primarily originated by banks, and the loans are extended to companies but also to governments and government-related entities. The coordinator, or lead bank, originates the loan, forms the syndicate, and processes the payments. In addition to banks, a variety of lenders participate in the syndicate, such as pension funds, insurance companies, and hedge funds. Syndicated loans are a way for these institutional investors to participate in corporate lending while diversifying the credit risk among a group of lenders.

There is also a secondary market in syndicated loans. These loans are often packaged and securitized, and the securities created are then sold in secondary markets to investors.

Most bilateral and syndicated loans are floating-rate loans, and the interest rate is based on a reference rate plus a spread. The reference rate may be MRR, formerly Libor, a sovereign rate (e.g., the T-bill rate), or the prime lending rate, also called the "prime rate." The prime rate reflects the rate at which banks lend to their most creditworthy customers, which tends to vary based on the overnight rate at which banks lend to each other plus a spread. Bank loans can be customized to the borrower's needs. They can have different maturities, as well as different interest payment and principal repayment structures. The frequency of interest payments varies among bank loans. Some loans are bullet loans, in which the entire payment of principal occurs at maturity, and others are amortizing loans, in which the principal is repaid over time.

For highly rated companies, both bilateral and syndicated loans can be more expensive than bonds issued in financial markets. Thus, companies often turn to money and capital markets to raise funds, which allows them to diversify their sources of financing.

\section{Commercial Paper}
Commercial paper is a short-term, unsecured promissory note issued in the public market or via a private placement that represents a debt obligation of the issuer. Commercial paper was first issued in the United States more than a century ago. It later appeared in the United Kingdom, in other European countries, and then in the rest of the world.

\section{Characteristics of Commercial Paper}
Commercial paper is a valuable source of flexible, readily available, and relatively low-cost short-term financing. It is a source of funding for working capital and seasonal demands for cash. It is also a source of bridge financing-that is, interim financing that provides funds until permanent financing can be arranged. Suppose a company wants to build a new distribution center in southeast China and wants to finance this investment with an issuance of long-term bonds. The market conditions for issuing long-term bonds may currently be volatile, which would translate into a higher cost of borrowing. Rather than issuing long-term bonds immediately, the company may opt to raise funds with commercial paper and wait for a more favorable environment in which to sell long-term bonds.

The largest issuers of commercial paper are financial institutions, but some non-financial companies are also regular issuers of commercial paper. Although the focus of this section is on corporate borrowers, sovereign governments and supranational agencies routinely issue commercial paper as well.

The maturity of commercial paper can range from overnight to one year, but a typical issue matures in less than three months.

\section{Credit Quality of Commercial Paper}
Traditionally, only the largest, most stable companies issued commercial paper. Although only the strongest, highest-rated companies issue low-cost commercial paper, issuers from across the risk spectrum can issue commercial paper with higher yields than higher-rated companies. Thus, investors in commercial paper are exposed to various levels of credit risk depending on the issuer's creditworthiness. Many investors perform their own credit analysis, but most investors also assess a commercial paper's credit quality by using the ratings provided by credit rating agencies. Exhibit 5 presents the range of commercial paper ratings from the main credit rating agencies. Commercial paper rated adequate or above (the shaded area of Exhibit 5) is called "prime paper," and it is typically considered investment grade by investors.

\section{Exhibit 5: Commercial Paper Ratings}
\begin{center}
\begin{tabular}{lccc}
\hline
Credit Quality & Moody's & S\&P & Fitch \\
\hline
Superior & $\mathrm{P} 1$ & $\mathrm{~A} 1+/ \mathrm{A} 1$ & $\mathrm{~F} 1+/ \mathrm{F} 1$ \\
Satisfactory & $\mathrm{P} 2$ & $\mathrm{~A} 2$ & $\mathrm{~F} 2$ \\
Adequate & $\mathrm{P} 3$ & $\mathrm{~A} 3$ & $\mathrm{~F} 3$ \\
Speculative & $\mathrm{NP}$ & $\mathrm{B} / \mathrm{C}$ & $\mathrm{F} 4$ \\
Defaulted & $\mathrm{NP}$ & $\mathrm{D}$ & $\mathrm{F} 5$ \\
\hline
\end{tabular}
\end{center}

In most cases, maturing commercial paper is paid with the proceeds of new issuances of commercial paper, a practice referred to as "rolling over the paper." This practice creates a risk that the issuer will be unable to issue new paper at maturity, referred to as rollover risk. As a safeguard against rollover risk, credit rating agencies often require that commercial paper issuers secure a backup line of credit from banks. The purpose of the backup lines of credit is to ensure that the issuer will have access to sufficient liquidity to repay maturing commercial paper if rolling over the paper is not a viable option. Therefore, backup lines of credit are sometimes called "liquidity enhancement" or "backup liquidity lines." Issuers of commercial paper may be unable to roll over the paper because of either market-wide or company-specific events. For example, financial markets could be in the midst of a financial crisis that would make it difficult to roll over the paper. A company could also experience some sort of financial distress such that it could only issue new commercial paper at significantly higher rates. In this case, the company could draw on its credit lines instead of rolling over its paper. Most commercial paper issuers maintain 100\% backing, although some large, high-credit-quality issues carry less than 100\% backing. Backup lines of credit typically contain a "material adverse change" provision that allows the bank to cancel the backup line of credit if the financial condition of the issuer deteriorates substantially.

Historically, defaults on commercial paper have been relatively rare, primarily because commercial paper has a short maturity. Each time existing paper matures, investors have another opportunity to assess the issuer's financial position, and they can refuse to buy the new paper if they estimate that the issuer's credit risk is too high. Thus, the commercial paper markets adapt more quickly to a change in an issuer's credit quality than do the markets for longer-term securities. This flexibility reduces the exposure of the commercial paper market to defaults. In addition, corporate managers realize that defaulting on commercial paper would likely prevent any future issuance of this valuable financing alternative.

The combination of short-dated maturity, relatively low credit risk, and a large number of issuers makes commercial paper attractive to a diverse range of investors, including money market mutual funds, bank liquidity desks, corporate treasury departments, and other institutional investors that have liquidity constraints. Most commercial paper investors hold their position to maturity. The result is little secondary market trading except for the largest issues. Investors who wish to sell commercial paper prior to maturity can sell it either back to the dealer, to another investor, or in some cases, directly back to the issuer.

The yield on commercial paper is typically higher than that on short-term sovereign bonds of the same maturity for two main reasons. First, commercial paper is exposed to credit risk unlike most highly rated sovereign bonds. Second, commercial paper markets are generally less liquid than short-term sovereign bond markets. Thus, investors require higher yields to compensate for the lower liquidity. In the United States, the yield on commercial paper also tends to be higher than that on short-term municipal bonds for tax reasons. Income generated by investments in commercial paper is usually subject to income taxes, whereas income from many municipal bonds is tax exempt. Thus, to attract taxable investors, bonds that are subject to income taxes must offer higher yields than those that are tax exempt.

\section{US Commercial Paper vs. Eurocommercial Paper}
The US commercial paper (USCP) market is the largest commercial paper market in the world, although there are other active commercial paper markets in other countries. Commercial paper issued in the international market is known as Eurocommercial paper (ECP). Although ECP is a similar instrument to USCP, there are some differences between the two. These differences are shown in Exhibit 6. Exhibit 6: USCP vS. ECP

\begin{center}
\begin{tabular}{|c|c|c|}
\hline
Feature & US Commercial Paper & Eurocommercial Paper \\
\hline
Currency & US dollar & Any currency \\
\hline
Maturity & Overnight to 270 days $^{\mathrm{a}}$ & Overnight to 364 days \\
\hline
Interest & $\begin{array}{l}\text { Discount basis (instrument is } \\ \text { issued at a discount to par value) }\end{array}$ & $\begin{array}{l}\text { Interest-bearing (instrument issued } \\ \text { at par and pays interest) or discount } \\ \text { basis }\end{array}$ \\
\hline
Settlement & $T+0$ (trade date) & $T+2$ (trade date plus two days) \\
\hline
Negotiable & Can be sold to another party & Can be sold to another party \\
\hline
\end{tabular}
\end{center}

a In the United States, securities with an original maturity greater than 270 days must be registered with the Securities and Exchange Commission (SEC). To avoid the time and expense associated with a SEC registration, issuers of US commercial paper rarely offer maturities longer than 270 days.

A difference between USCP and ECP is related to the interest provision. USCP is typically issued on a discount basis-that is, USCP is issued at a discount to par value and pays full par value at maturity. The difference between the par value and the issue price is the interest paid on the borrowing. In contrast, ECP may be issued at, and trade on, an interest-bearing or yield basis or a discount basis. The distinction between the discount and the interest-bearing basis is explored later in the discussion on fixed-income valuation.

Typical transaction sizes in ECP are also much smaller than in USCP, and it is difficult to place longer-term ECP with investors. The ECP market also exhibits less liquidity than the USCP market.

\section{CORPORATE DEBT: NOTES AND BONDS}
Companies are active participants in global capital markets and regularly issue corporate notes and bonds. These securities can be placed directly with specific investors via private placements or sold in public securities markets. This section discusses various characteristics of corporate notes and bonds.

\section{Maturities}
There is no universally accepted taxonomy as to what constitutes short-, medium-, and long-term maturities. For our purposes, short term refers to original maturities of 5 years or less; intermediate term to original maturities longer than 5 years and up to 12 years; and long term to original maturities longer than 12 years. Those securities with maturities between 1 and 12 years are often considered notes, whereas securities with maturities greater than 12 years are considered bonds. It is not uncommon, however, to refer to bonds for all securities, irrespective of their original maturity.

In practice, most corporate bonds range in term to maturity between 1 and 30 years. In Europe, however, some bond issues have maturities of 40 or 50 years. In addition, companies and sovereigns have issued 100-year bonds; these are called "century bonds." For example, in 2017 Austria issued EUR 3.5 billion in bonds that will mature in 2117.

The first century bond was issued by the Walt Disney Company in 1993 as part of its medium-term note program. Medium-term note (MTN) is a misnomer because, as the Disney example illustrates, MTNs can have very long maturities. From the issuer's perspective, the initial purpose of MTNs was to fill the funding gap between commercial paper and long-term bonds. It is for this reason that they are referred to as "medium term." The MTN market can be broken into three segments: short-term securities that carry floating or fixed rates, medium- to long-term securities that primarily bear a fixed rate of interest, and structured notes.

MTNs have the unique characteristic of being securities that are offered continuously to investors by an agent of the issuer. This feature gives the borrower maximum flexibility for issuing securities on a continuous basis. Financial institutions are the primary issuers of MTNs, in particular short-term ones. Life insurance companies, pension funds, and banks are among the largest buyers of MTNs because they can customize the bond issue to their needs and stipulate the amount and characteristics of the securities they want to purchase. These investors are often willing to accept less liquidity than they would receive with a comparable publicly issued bond because the yield is slightly higher. The cost savings in registration and underwriting often makes MTNs a lower-cost option for the issuer.

\section{Coupon Payment Structures}
Corporate notes and bonds have a range of coupon payment structures. Financial and non-financial companies issue conventional coupon bonds that pay a fixed periodic coupon during the bond's life. They also issue bonds for which the periodic coupon payments adjust to changes in market conditions and/or changes to the issuer's credit quality. Such bonds typically offer investors the opportunity to reduce their exposure to a particular type of risk. For example, FRNs, whose coupon payments adjust to changes in the level of market interest rates, are a way to limit interest rate risk; some of the inflation-linked bonds whose coupon payments adjust to changes in the level of a consumer price index offer a protection against inflation risk; credit-linked coupon bonds, whose coupon payments adjust to changes in the issuer's credit quality, are a way to reduce credit risk. Whether the periodic coupon is fixed or not, coupon payments can be made quarterly, semi-annually, or annually depending on the type of bond and where the bonds are issued and traded.

Other coupon payment structures exist. Zero-coupon bonds pay no coupon. Deferred coupon bonds pay no coupon initially but then offer a higher coupon. Payment-in-kind (PIK) coupon bonds make periodic coupon payments, but not necessarily in cash; the issuer may pay interest in the form of securities, such as bonds or common shares. These types of coupon payment structures increase issuers' flexibility regarding the servicing of their debt.

\section{Principal Repayment Structures}
Corporate note or bond issues have either a serial or a term maturity structure. With a serial maturity structure, the maturity dates are spread out during the bond's life; a stated number of bonds mature and are paid off each year before final maturity. With a term maturity structure, the bond's notional principal is paid off in a lump sum at maturity. Because there is no regular repayment of the principal outstanding throughout the bond's life, a term maturity structure carries more credit risk than a serial maturity structure.

A sinking fund arrangement is a way to reduce credit risk by making the issuer set aside funds over time to retire the bond issue. For example, a corporate bond issue may require a specified percentage of the bond's outstanding principal amount to be retired each year. The issuer may satisfy this requirement in one of two ways. The most common approach is for the issuer to make a random call for the specified percentage of bonds that must be retired and to pay the bondholders whose bonds are called the sinking fund price, which is typically par. Alternatively, the issuer can deliver bonds to the trustee with a total amount equal to the amount that must be retired. To do so, the issuer may purchase the bonds in the open market. The sinking fund arrangement on a term maturity structure accomplishes the same goal as the serial maturity structure-that is, both result in a portion of the bond issue being paid off each year. With a serial maturity structure, however, the bondholders know which bonds will mature and will thus be paid off each year. In contrast, the bonds retired annually with a sinking fund arrangement are designated by a random drawing.

\section{Asset or Collateral Backing}
Unlike most highly rated sovereign bonds, all corporate debt is exposed to varying degrees of credit risk. Thus, corporate debt is structured with this risk in mind. An important consideration for investors is seniority ranking-that is, the systematic way in which lenders are repaid if the issuer defaults. In the case of secured debt, there is some form of collateral pledged to ensure payment of the debt. In contrast, in the case of unsecured debt, claims are settled by the general assets of the company in accordance with the priority of payments that applies either legally or contractually and as described in the bond indenture. Within each category of debt (secured and unsecured), there are finer gradations of rankings, which are discussed in the later in the curriculum coverage of credit analysis.

There is a wide range of bonds that are secured by some form of collateral. Companies that need to finance equipment or physical assets may issue equipment trust certificates. Corporate issuers also sell collateral trust bonds that are secured by securities, such as common shares, bonds, or other financial assets. Banks, particularly in Europe, may issue covered bonds, which are a type of debt obligation that is secured by a segregated pool of assets. Asset-backed securities are also secured forms of debt.

Companies can and do default on their debt. Debt secured by collateral may still experience losses, but during a company's bankruptcy proceedings, investors in secured debt usually fare better than investors in unsecured debt. Investors who face a higher level of credit risk typically require a higher yield than investors exposed to very little credit risk.

\section{Contingency Provisions}
Contingency provisions are clauses in the indenture that provide the issuer or the bondholders rights that affect the disposal or redemption of the bond. The three commonly used contingency provisions are call, put, and conversion provisions.

Callable bonds give issuers the ability to retire debt prior to maturity. The most compelling reason for them to do so is to take advantage of lower borrowing rates. By calling the bonds before their maturity date, the issuer can substitute a new, lower-cost bond issue for an older, higher-cost one. In addition, companies may also retire debt to eliminate restrictive covenants or to alter their capital structure to improve flexibility. Because the call provision is a valuable option for the issuer, investors demand compensation ex ante (before investing in the bond). Thus, other things equal, investors require a higher yield (and thus pay a lower price) for a callable bond than for an otherwise similar non-callable bond.

Companies also issue putable bonds, which give the bondholders the right to sell the bond back to the issuer at a pre-determined price on specified dates before maturity. Most putable bonds pay a fixed rate of interest, although some bonds may have step-up coupons that increase by specified margins at specified dates. Because the put provision is a valuable option for the bondholders, putable bonds offer a lower yield (and thus have a higher price) than otherwise similar non-putable bonds. The main corporate issuers of putable bonds are investment-grade companies. Putable bonds may offer them a cheaper way of raising capital, especially if the company estimates that the benefit of a lower coupon outweighs the risk associated with the put provision. A convertible bond is a hybrid security that lies on a continuum between debt and equity. It consists of a long position in an option-free bond and a conversion option that gives the bondholder the right to convert the bond into a specified number of shares of the issuer's common shares. From the issuer's point of view, convertible bonds make it possible to raise funds that may not be possible without the incentive associated with the conversion option. The more common issuers of convertibles bonds are newer companies that have not established a presence in debt capital markets but who are able to present a more attractive package to institutional investors by including an equity upside potential. Established issuers of bonds may also prefer to issue convertible bonds because they are usually sold at a lower coupon rate than otherwise similar non-convertible bonds as a result of investors' attraction to the conversion provision. There is a potential equity dilution effect, however, if the bonds are converted. From the investor's point of view, convertible bonds represent a means of accessing the equity upside potential of the issuer but at a lower risk-reward profile, because there is the floor of the coupon payments in the meantime.

\section{Issuance, Trading, and Settlement}
In the era before electronic settlement, there were some differences in the processes of issuing and settling corporate bonds depending on where the securities were registered. This is no longer the case; the processes of issuing and settling bonds are now essentially the same globally. New corporate bond issues are usually sold to investors by investment banks acting as underwriters in the case of underwritten offerings or brokers in the case of best-efforts offerings. They are then settled via the local settlement system. These local systems typically possess a "bridge" to the two Eurobond systems, Euroclear and Clearstream. As for Eurobonds from the corporate sector, they are all issued, traded, and settled in the same way, irrespective of the issuer and its local jurisdiction.

Most bond prices are quoted in basis points. The vast majority of corporate bonds are traded in OTC markets through dealers that "make a market" in bonds and sell from their inventory. Dealers do not typically charge a commission or a transaction fee. Instead, they earn a profit from the bid-offer spread.

For corporate bonds, settlement differences exist primarily between new bond issues and the secondary trading of bonds. The issuing phase for an underwritten offering usually takes several days. Thus, settlement takes longer for new bond issued than for the secondary trading of bonds, for which settlement is typically on a $T+2$ or $T+3$ basis.

\section{EXAMPLE 6}
\section{Corporate Debt}
\begin{enumerate}
  \item A loan made by a group of banks to a private company is most likely:
\end{enumerate}

A. a bilateral loan.

B. a syndicated loan.

C. a securitized loan.

Solution:

$\mathrm{B}$ is correct. A loan from a group of lenders to a single borrower is a syndicated loan. A is incorrect because a bilateral loan is a loan from a single lender to a single borrower. $C$ is incorrect because securitization involves moving assets, such as loans, from the owner of the assets into a special legal entity.

\begin{enumerate}
  \setcounter{enumi}{1}
  \item Which of the following statements relating to commercial paper is most accurate? Companies issue commercial paper:
\end{enumerate}

A. only for funding working capital.

B. only as an interim source of financing.

C. both for funding working capital and as an interim source of funding.

\section{Solution:}
$C$ is correct. Companies use commercial paper as a source of funding working capital and seasonal demand for cash, as well as an interim source of financing until permanent financing can be arranged.

\begin{enumerate}
  \setcounter{enumi}{2}
  \item Maturities of Eurocommercial paper range from:
\end{enumerate}

A. overnight to three months.

B. overnight to one year.

C. three months to one year.

\section{Solution:}
B is correct. Eurocommercial paper ranges in maturity from overnight to 364 days.

\begin{enumerate}
  \setcounter{enumi}{3}
  \item A bond issue that has a stated number of bonds that mature and are paid off each year before final maturity most likely has a:
A. term maturity.
B. serial maturity.
C. sinking fund arrangement.
\end{enumerate}

\section{Solution:}
B is correct. With a serial maturity structure, a stated number of bonds mature and are paid off each year before final maturity. A is incorrect because a bond issue with a term maturity structure is paid off in one lump sum at maturity. $\mathrm{C}$ is incorrect because a sinking fund arrangement, such as a serial maturity structure, results in a portion of the bond issue being paid off every year. With a serial maturity structure, however, the bonds are paid off because the maturity dates are spread out during the life of the bond and the bonds that are retired are maturing; the bondholders know in advance which bonds will be retired. In contrast, the bonds retired annually with a sinking fund arrangement are designated by a random drawing.

\section*{STRUCTURED FINANCIAL INSTRUMENTS }
Structured financial instruments represent a broad sector of financial instruments. This sector includes asset-backed securities (ABS) and collateralized debt obligations (CDOs). CDOs are securities backed by a diversified pool of one or more debt obligations, and like ABS, they are discussed later in curriculum coverage on asset-backed securities. A common attribute of all these financial instruments is that they repackage and redistribute risks.

Our focus in this section is on structured financial instruments apart from ABS and CDOs. These instruments typically have customized structures that often combine a bond and at least one derivative. Some of these instruments are called structured products. The use of derivatives gives the holder of the structured financial instrument exposure to one or more underlying assets, such as equities, bonds, and commodities. The redemption value and often the coupons of structured financial instruments are linked via a formula to the performance of the underlying asset(s). Thus, the bond's payment features are replaced with non-traditional payoffs that are derived not from the issuer's cash flows but from the performance of the underlying asset(s). Although no universally accepted taxonomy exists to categorize structured financial instruments, we will present four broad categories of instruments: capital protected, yield enhancement, participation, and leveraged instruments.

\section{Capital Protected Instruments}
Suppose an investor has $\$ 100,000$ to invest. The investor buys zero-coupon bonds issued by a sovereign issuer that will pay off $\$ 100,000$ one year from now. Also suppose the cost of buying the zero-coupon bonds is $\$ 99,000$. The investor can use the $\$ 1,000$ left over from the purchase of the zero-coupon bond to buy a call or put option on some underlying asset that expires one year from now. In this context, buying a call (put) option gives the investor the right to buy (sell) the underlying asset in one year at a pre-determined price. The investor will receive $\$ 100,000$ when the zero-coupon bond matures and may also gain from the upside potential of the option, if any. This combination of the zero-coupon bond and the call option can be prepackaged as a structured financial instrument called a guarantee certificate. The zero-coupon bond provides the investor capital protection; at maturity, the investor will receive $100 \%$ of the capital invested even if the option expires worthless. The call (put) option provides upside potential if the price of the underlying asset rises (falls) and a limited downside if the price of the underlying asset falls (rises) or remains unchanged. The downside is limited to the price, often called the premium, paid for the call or put option. In our example, the maximum loss the investor faces is $\$ 1,000$, which is the price paid for the option.

Capital protected instruments offer different levels of capital protection. A guarantee certificate offers full capital protection. Other structured financial instruments may offer only partial capital protection. Note that the capital protection is only as good as the issuer of the instrument. Should the issuer of guarantee certificates go bankrupt, investors may lose their entire capital. These instruments are naturally more prevalent when the general level of interest rates is high and/or implied volatilities (the key driver of option prices, as we will explain later) are low. The cost of the options in the structure must be absorbed by the forfeited interest.

\section{Yield Enhancement Instruments}
Yield enhancement refers to increasing risk exposure in the hope of realizing a higher expected return. A credit-linked note (CLN) is an example of a yield enhancement instrument. Specifically, it is a type of bond that pays regular coupons but whose redemption value depends on the occurrence of a well-defined credit event, such as a rating downgrade or the default of an underlying asset, called the reference asset. If the specified credit event does not occur, the investor receives the par value of the CLN at maturity. But if the specified credit event occurs, the investor receives the par value of the CLN minus the nominal value of the reference asset to which the CLN is linked.

A CLN allows the issuer to transfer the effect of a credit event to investors. Thus, the issuer is the protection buyer and the investor is the protection seller. Investors are willing to buy CLNs because these securities offer higher coupons than otherwise similar bonds. In addition, CLNs are usually issued at a discount. Thus, if the specified credit event does not occur, investors will realize a significant capital gain on the purchase of the CLN.

\section{Participation Instruments}
As the name suggests, a participation instrument is one that allows investors to participate in the return of an underlying asset. Floating-rate bonds can be viewed as a type of participation instrument. As discussed earlier, floaters differ from fixed-rate bonds in that their coupon rate adjusts periodically according to a pre-specified formula. The coupon formula is usually expressed as a reference rate adjusted for a spread. A floater has almost zero interest rate risk because changes in the cash flows limit the effect of changes in interest rates on the floater's price. Thus, floaters give investors the opportunity to participate in movements of interest rates. For example, the Italian government issued in June 2005 floaters set to mature in June 2020. The coupon payments are delivered annually and determined by the formula of $85 \%$ of the 10-year constant maturity swap rate, a widely used type of interest rate. Thus, investors who hold these floaters participate partially in movements of the 10-year constant maturity swap rate.

Most participation instruments are designed to give investors indirect exposure to a specific index or asset price. For example, investors who are precluded from investing in equity directly may gain indirect equity exposure by investing in participation instruments that are linked via a formula to the performance of equity indexes. Many structured products sold to individuals are participation instruments linked to an equity index. In contrast to capital protected instruments that offer equity exposure, these participation instruments usually do not offer capital protection.

\section{Leveraged Instruments}
Leveraged instruments are structured financial instruments created to magnify returns and offer the possibility of high payoffs from small investments. An inverse floater is an example of a leveraged instrument. As the name suggests, an inverse floater is the opposite of a traditional floater. The cash flows are adjusted periodically and move in the opposite direction of changes in the reference rate. So, when the reference rate decreases, the coupon payment of an inverse floater increases.

A general formula for an inverse floater's coupon rate is as follows:

Inverse floater coupon rate $=\mathrm{C}-(\mathrm{L} \times \mathrm{R})$

where $C$ is the maximum coupon rate reached if the reference rate is equal to zero, $\mathrm{L}$ is the coupon leverage, and $\mathrm{R}$ is the reference rate on the reset date. Note that the coupon leverage indicates the multiple that the coupon rate will change in response to a 100-bp change in the reference rate. For example, if the coupon leverage is three, the inverse floater's coupon rate will decrease by $300 \mathrm{bps}$ when the reference rate increases by $100 \mathrm{bps}$.

Inverse floaters with a coupon leverage greater than zero but lower than one are called deleveraged inverse floaters. Inverse floaters with a coupon leverage greater than one are called leveraged inverse floaters. For example, the Barclays Bank PLC issued a 15-year bond in January 2010 having quarterly payments. The coupon rate was fixed at 7.50\% for the first three years and then in January 2013 transformed into a leveraged inverse floater paying 7.50\% minus the euro three-month Libor. In this case, the coupon leverage is one. Thus, for a 100 bps increase in the euro three-month Libor, the coupon rate of the leveraged inverse floater will decrease by $100 \mathrm{bps}$. Inverse floaters often have a floor that specifies a minimum coupon rate; for example, a floor may be set at zero to avoid the possibility of a negative interest rate. This inverse floater does not have a maximum coupon rate. At the July 2018 reset, euro three-month Libor was $-0.32 \%$, so the coupon rate for the following quarter was $7.82 \%$. Note that the example involves a bond issued in the past, and thus Libor was the reference rate. Future bonds issued will reference a different rate, including prime or MRR.

\section{EXAMPLE 7}
\section{Structured Financial Instruments}
\begin{enumerate}
  \item If an investor holds a credit-linked note and the credit event does not occur, the investor receives:
\end{enumerate}

A. all promised cash flows as scheduled.

B. all coupon payments as scheduled but not the par value at maturity.

C. all coupon payments as scheduled and the par value minus the nominal value of the reference asset to which the credit-linked note is linked at maturity.

\section{Solution:}
A is correct. If the credit event does not occur, the issuer must make all promised cash flows as scheduled-that is, the regular coupon payments and the par value at maturity.

\begin{enumerate}
  \setcounter{enumi}{1}
  \item A structured financial instrument whose coupon rate is determined by the formula $5 \%-(0.5 \times$ Libor) is most likely:
\end{enumerate}

A. a leveraged inverse floater.

B. a participation instrument.

C. a deleveraged inverse floater.

\section{Solution:}
$\mathrm{C}$ is correct. A structured financial instrument whose coupon rate moves in the opposite direction of the reference rate is called an inverse floater. Because the coupon leverage (0.5) is greater than zero but lower than one, the structured financial instrument is a deleveraged inverse floater. In this example, if the reference rate increases by $100 \mathrm{bps}$, the coupon rate decreases by $50 \mathrm{bps}$. A is incorrect because the coupon leverage would have to be higher than one for the structured financial instrument to be a leveraged inverse floater. $\mathrm{B}$ is incorrect because a participation instrument is designed to give investors indirect exposure to a particular underlying asset.

\section{SHORT-TERM BANK FUNDING ALTERNATIVES}
describe short-term funding alternatives available to banks

Funding refers to the amount of money or resources necessary to finance some specific project or enterprise. Accordingly, funding markets are markets in which debt issuers borrow to meet their financial needs. Companies have a range of funding alternatives, including bank loans, commercial paper, notes, and bonds. Financial institutions such as banks have larger financing needs than non-financial companies because of the nature of their operations. This section discusses the additional funding alternatives available to them. The majority of these funding alternatives have short maturities.

Banks, such as deposit-taking (or depository) institutions, typically have access to funds obtained from the retail market - that is, deposit accounts from their customers. It is quite common, however, for banks to originate more loans than they have retail deposits. Thus, whenever the amount of retail deposits is insufficient to meet their financial needs, banks also need to raise funds from the wholesale market. Wholesale funds include central bank funds, interbank deposits, and certificates of deposit. In addition to filling the gaps between loans and deposits, banks raise wholesale funds to minimize their funding cost. At the margin, wholesale funds may be less expensive (in terms of interest expense) than deposit funding. Finally, financial institutions may raise wholesale funds as a balance sheet risk management tool to reduce interest rate risk, as discussed earlier.

\section{Retail Deposits}
One of the primary sources of funding for deposit-taking banks is their retail deposit base, which includes funds from both individual and commercial depositors. There are several types of retail deposit accounts. Demand deposits, also known as checking accounts, are available to customers "on demand." Depositors have immediate access to the funds in their deposit accounts and use the funds as a form of payment for transactions. Because the funds are available immediately, deposit accounts typically pay no interest. In contrast, savings accounts pay interest and allow depositors to accumulate wealth in a very liquid form, but they do not offer the same transactional convenience as demand deposits. Money market accounts were originally designed to compete with money market mutual funds. They offer money market rates of return and depositors can access funds at short or no notice. Thus, money market accounts are, for depositors, an intermediate between demand deposit and savings accounts.

\section{Short-Term Wholesale Funds}
Wholesale funds available for banks include reserve funds, interbank funds, and certificates of deposit.

\section{Reserve Funds}
Many countries require deposit-taking banks to place a reserve balance with the national central bank. The reserve funds help to ensure sufficient liquidity should depositors require withdrawal of funds. When a bank cannot obtain short-term funding, most countries allow that bank to borrow from the central bank. In aggregate, the reserve funds act as a liquidity buffer, providing comfort to depositors and investors that the central bank can act as lender of last resort. Treatment of interest on reserve funds varies among countries, from a low interest payment, to no interest payment, to charges for keeping reserve funds. Additionally, there is an opportunity cost to the banks for holding reserves with the central bank, in that these funds cannot be invested with higher interest or loaned out to consumers or commercial enterprises. Some banks have an excess over the minimum required funds to be held in reserve. At the same time, other banks run short of required reserves. This imbalance is solved through the central bank funds market, which allows banks that have a surplus of funds to lend money to banks that need funds for maturities of up to one year. These funds are known as central bank funds and are called "overnight funds" when the maturity is one day and "term funds" when the maturity ranges from two days to one year. The interest rates at which central bank funds are bought (i.e., borrowed) and sold (i.e., lent) are short-term interest rates determined by the markets but influenced by the central bank's open market operations. These rates are termed the central bank funds rates.

In the United States, the central bank is the Federal Reserve (Fed). The central bank funds and funds rate are called Fed funds and the Fed funds rate, respectively. Other short-term interest rates, such as the yields on Treasury bills, are highly correlated with the Fed funds rate. The most widely followed rate is known as the Fed funds effective rate, which is the volume-weighted average of rates for Fed fund trades arranged throughout the day by the major New York City brokers. Fed funds are traded between banks and other financial institutions globally and may be transacted directly or through money market brokers.

\section{Interbank Funds}
The interbank market is the market of loans and deposits between banks. The term to maturity of an interbank loan or deposit ranges from overnight to one year. The rate on an interbank loan or deposit can be quoted relative to a reference rate, such as an interbank offered rate, or as a fixed interest rate. An interbank deposit is unsecured, so banks placing deposits with another bank need to have an interbank line of credit in place for that institution. Usually, a large bank will make a two-way price, indicating the rate at which it will lend funds and the rate at which it will borrow funds for a specific maturity, on demand. Interest on the deposit is payable at maturity. Much interbank dealing takes place on the Reuters electronic dealing system, so that the transaction is done without either party speaking to the other.

Because the market is unsecured, it is essentially based on confidence in the banking system. At times of stress, such as in the aftermath of the Lehman Brothers' bankruptcy in 2008, the market is prone to "dry up" as banks withdraw from funding other banks.

\section{Large-Denomination Negotiable Certificates of Deposit}
A certificate of deposit (CD) is an instrument that represents a specified amount of funds on deposit for a specified maturity and interest rate. CDs are an important source of funds for financial institutions. A CD may take one of two forms: non-negotiable or negotiable. If the $\mathrm{CD}$ is non-negotiable, the deposit plus the interest are paid to the initial depositor at maturity. A withdrawal penalty is imposed if the depositor withdraws funds prior to the maturity date.

Alternatively, a negotiable CD allows any depositor (initial or subsequent) to sell the CD in the open market prior to the maturity date. Negotiable CDs were introduced in the United States in the early 1960s when various types of deposits were constrained by interest rate ceilings. At the time, bank deposits were not an attractive investment because investors earned a below-market interest rate unless they were prepared to commit their capital for an extended period. The introduction of negotiable CDs enabled bank customers to buy a three-month or longer negotiable instrument yielding a market interest rate and to recover their investment by selling it in the market. This innovation helped banks increase the amount of funds raised in the money markets. It also fostered competition among deposit-taking institutions.

There are two types of negotiable CDs: large-denomination CDs and small-denomination CDs. Thresholds between small- and large-denomination CDs vary among countries. For example, in the United States, large-denomination CDs are usually issued in denominations of $\$ 1$ million or more. Small-denomination CDs are a retail-oriented product, and they are of secondary importance as a funding alternative. Large-denomination CDs, in contrast, are an important source of wholesale funds and are typically traded among institutional investors.

Like other money market securities, CDs are available in domestic bond markets as well as in the Eurobond market. Most CDs have maturities shorter than one year and pay interest at maturity. CDs with longer maturities are called "term CDs."

Yields on CDs are driven primarily by the credit risk of the issuing bank and to a lesser extent by the term to maturity. The spread attributable to credit risk will vary with economic conditions and confidence in the banking system in general and in the issuing bank in particular. As with all debt instruments, spreads widen during times of financial turmoil in response to increased risk aversion.

\section{REPURCHASE AND REVERSE REPURCHASE AGREEMENTS}
describe repurchase agreements (repos) and the risks associated with
them

Repurchase agreements are another important source of funding not only for banks but also for other market participants. A repurchase agreement or repo is the sale of a security with a simultaneous agreement by the seller to buy the same security back from the purchaser at an agreed-on price and future date. In practical terms, a repurchase agreement can be viewed as a collateralized loan in which the security sold and subsequently repurchased represents the collateral posted. One party is borrowing money and providing collateral for the loan at an interest rate that is typically lower than on an otherwise similar bank loan. The other party is lending money while accepting a security as collateral for the loan.

Repurchase agreements are a common source of money market funding for dealer firms in many countries. An active market in repurchase agreements underpins every liquid bond market. Financial and non-financial companies participate actively in the market as both sellers and buyers of collateral depending on their circumstances. Central banks are also active users of repurchase agreements in their daily open market operations; they either lend to the market to increase the supply of funds or withdraw surplus funds from the market.

\section{Structure of Repurchase and Reverse Repurchase Agreements}
Suppose a government securities dealer purchases a 2.25\% UK gilt that matures in three years. The dealer wants to fund the position overnight through the end of the next business day. The dealer could finance the transaction with its own funds, which is what other market participants, such as insurance companies or pension funds, may do in similar circumstances. But a securities dealer typically uses leverage (debt) to fund the position. Rather than borrowing from a bank, the dealer uses a repurchase agreement to obtain financing by using the gilt as collateral for the loan.

A repurchase agreement may be constructed as follows: The dealer sells the 2.25\% UK gilt that matures in three years to a counterparty for cash today. At the same time, the dealer makes a promise to buy the same gilt the next business day for an agreed-on price. The price at which the dealer repurchases the gilt is known as the repurchase price. The date when the gilt is repurchased, the next business day in this example, is called the repurchase date. When the term of a repurchase agreement is one day, it is called an "overnight repo." When the agreement is for more than one day, it is called a "term repo." An agreement lasting until the final maturity date is known as a "repo to maturity."

As in any borrowing or lending transaction, the interest rate of the loan must be negotiated in the agreement. The interest rate on a repurchase agreement is called the repo rate. Several factors affect the repo rate:

\begin{itemize}
  \item The risk associated with the collateral. Repo rates are typically lower for highly rated collaterals, such as highly rated sovereign bonds. They increase with the level of credit risk associated with the collateral underlying the transaction.

  \item The term of the repurchase agreement. Repo rates generally increase with maturity because long-term rates are typically higher than short-term rates in normal circumstances.

  \item The delivery requirement for the collateral. Repo rates are usually lower when delivery to the lender is required.

  \item The supply and demand conditions of the collateral. The more in demand a specific piece of collateral is, the lower the repo rate against it because the borrower has a security that lenders of cash want for specific reasons, perhaps because the underlying issue is in great demand. The demand for such collateral means that it considered to be "on special." Collateral that is not special is known as "general collateral." The party that has a need for collateral that is on special is typically required to lend funds at a below-market repo rate to obtain the collateral.

  \item The interest rates of alternative financing in the money market.

\end{itemize}

The interest on a repurchase agreement is paid on the repurchase date-that is, at the termination of the agreement. Note that any coupon paid by the security during the repurchase agreement belongs to the seller of the security (i.e., the borrower of cash).

When a repurchase agreement is viewed through the lens of the cash lending counterparty, the transaction is referred to as a reverse repurchase agreement or reverse repo. In the foregoing example, the counterparty agrees to buy the 2.25\% UK gilt that matures in three years and promises to sell it back the next business day at the agreed-on price. The counterparty is making a collateralized loan to the dealer. Reverse repurchase agreements are very often used to borrow securities to cover short positions.

The question of whether a particular transaction is labeled a repurchase agreement or a reverse repurchase agreement depends on one's point of view. Standard practice is to view the transaction from the dealer's perspective. If the dealer is borrowing cash from a counterparty and providing securities as collateral, the transaction is termed a repurchase agreement. If the dealer is borrowing securities and lending cash to the counterparty, the transaction is termed a reverse repurchase agreement.

\section{Credit Risk Associated with Repurchase Agreements}
Each market participant in a repurchase agreement is exposed to the risk that the counterparty defaults, regardless of the collateral exchanged. Credit risk is present even if the collateral is a highly rated sovereign bond. Suppose that a dealer (i.e., the borrower of cash) defaults and is not in a position to repurchase the collateral on the specified repurchase date. The lender of funds takes possession of the collateral and retains any income owed to the borrower. The risk is that the price of the collateral has fallen following the inception of the repurchase agreement, causing the market value of the collateral to be lower than the unpaid repurchase price. Conversely, suppose the investor (i.e., the lender of cash) defaults and is unable to deliver the collateral on the repurchase date. The risk is that the price of the collateral has risen since the inception of the repurchase agreement, resulting in the dealer now holding an amount of cash lower than the market value of the collateral. In this case, the investor is liable for any excess of the price paid by the dealer for replacement of the securities over the repurchase price.

Although both parties to a repurchase agreement are subject to credit risk, the agreement is structured as if the lender of funds is the most vulnerable party. Specifically, the amount lent is lower than the collateral's market value. The difference between the market value of the security used as collateral and the value of the loan is known as the repo margin, although the term haircut is more commonly used, particularly in the United States. The repo margin allows for some worsening in market value and thus provides the cash lender a margin of safety if the collateral's market value declines. Repo margins vary by transaction and are negotiated bilaterally between the counterparties. The level of margin is a function of the following factors:

\begin{itemize}
  \item The length of the repurchase agreement. The longer the repurchase agreement, the higher the repo margin.

  \item The quality of the collateral. The higher the quality of the collateral, the lower the repo margin.

  \item The credit quality of the counterparty. The higher the creditworthiness of the counterparty, the lower the repo margin.

  \item The supply and demandconditions of the collateral. Repo margins are lower if the collateral is in short supply or if there is a high demand for it.

\end{itemize}

\section{EXAMPLE 8}
\section{Short-Term Funding Alternatives Available to Banks}
\begin{enumerate}
  \item Which of the following are not considered wholesale funds?
A. Interbank funds
B. Central bank funds
C. Repurchase agreements
\end{enumerate}

\section{Solution:}
$\mathrm{C}$ is correct. Wholesale funds refer to the funds that financial institutions lend to and borrow from each other. They include central bank funds, interbank funds, and certificates of deposit. Although repurchase agreements are an important source of funding for banks, they are not considered wholesale funds. 2. A large-denomination negotiable certificate of deposit most likely:

A. is traded in the open market.

B. is purchased by retail investors.

C. has a penalty for early withdrawal of funds.

\section{Solution:}
A is correct. Large-denomination negotiable CDs can be traded in the open market. B is incorrect because it is small-denomination, not large-denomination, negotiable CDs that are primarily purchased by retail investors. $\mathrm{C}$ is incorrect because it is non-negotiable, not negotiable, CDs that have a penalty for early withdrawal of funds.

\begin{enumerate}
  \setcounter{enumi}{2}
  \item From the dealer's viewpoint, a repurchase agreement is best described as a type of:
\end{enumerate}

A. collateralized short-term lending.

B. collateralized short-term borrowing.

C. uncollateralized short-term borrowing.

Solution:

B is correct. In a repurchase agreement, a security is sold with a simultaneous agreement by the seller to buy the same security back from the purchaser later at a higher price. Thus, a repurchase agreement is similar to a collateralized short-term borrowing in which the security sold and subsequently repurchased represents the collateral posted. A is incorrect because collateralized short-term lending is a description of a reverse repurchase agreement. $\mathrm{C}$ is incorrect because a repurchase agreement involves collateral. Thus, it is a collateralized, not uncollateralized, short-term borrowing.

\begin{enumerate}
  \setcounter{enumi}{3}
  \item The interest on a repurchase agreement is known as the:
A. repo rate.
B. repo yield.
C. repo margin.
\end{enumerate}

Solution:

$\mathrm{A}$ is correct. The repo rate is the interest rate on a repurchase agreement. $\mathrm{B}$ is incorrect because the interest on a repurchase agreement is known as the repo rate, not repo yield. $C$ is incorrect because the repo margin refers to the difference between the market value of the security used as collateral and the value of the loan.

\begin{enumerate}
  \setcounter{enumi}{4}
  \item The level of repo margin is higher:
\end{enumerate}

A. the higher the quality of the collateral.

B. the higher the credit quality of the counterparty.

C. the longer the length of the repurchase agreement.

\section{Solution:}
$\mathrm{C}$ is correct. The longer the length of the repurchase agreement, the higher the repo margin (haircut). A is incorrect because the higher the quality of the collateral, the lower the repo margin. B is incorrect because the higher the credit quality of the counterparty, the lower the repo margin.

\section{SUMMARY}
Debt financing is an important source of funds for households, governments, government-related entities, financial institutions, and non-financial companies. Well-functioning fixed-income markets help ensure that capital is allocated efficiently to its highest and best use globally. Important points include the following:

\begin{itemize}
  \item The most widely used ways of classifying fixed-income markets include the type of issuer; the bonds' credit quality, maturity, currency denomination, and type of coupon; and where the bonds are issued and traded.

  \item Based on the type of issuer, the four major bond market sectors are the household, non-financial corporate, government, and financial institution sectors.

  \item Investors distinguish between investment-grade and high-yield bond markets based on the issuer's credit quality.

  \item Money markets are where securities with original maturities ranging from overnight to one year are issued and traded, whereas capital markets are where securities with original maturities longer than one year are issued and traded.

  \item The majority of bonds are denominated in either euros or US dollars.

  \item Investors distinguish between bonds that pay a fixed rate versus a floating rate of interest. The coupon rate of floating-rate bonds is often expressed as a reference rate plus a spread. Interbank offered rates, such as Libor, historically have been the most commonly used reference rates for floating-rate debt and other financial instruments but are being phased out to be replaced by alternative reference rates.

  \item Based on where the bonds are issued and traded, investors distinguish between domestic and international bond markets. The latter includes the Eurobond market, which falls outside the jurisdiction of any single country and is characterized by less reporting, regulatory, and tax constraints. Investors also distinguish between developed and emerging bond markets.

  \item Investors and investment managers use fixed-income indexes to describe bond markets or sectors and to evaluate performance of investments and investment managers.

  \item The largest investors in bonds include central banks; institutional investors, such as pension funds, hedge funds, charitable foundations and endowments, insurance companies, mutual funds and ETFs, and banks; and retail investors, typically by means of indirect investments.

  \item Primary markets are markets in which issuers first sell bonds to investors to raise capital. Secondary markets are markets in which existing bonds are subsequently traded among investors.

  \item There are two mechanisms for issuing a bond in primary markets: a public offering, in which any member of the public may buy the bonds, or a private placement, in which only an investor or small group of investors may buy the bonds either directly from the issuer or through an investment bank.

  \item Public bond issuing mechanisms include underwritten offerings, best-efforts offerings, shelf registrations, and auctions. - When an investment bank underwrites a bond issue, it buys the entire issue and takes the risk of reselling it to investors or dealers. In contrast, in a best-efforts offering, the investment bank serves only as a broker and sells the bond issue only if it is able to do so. Underwritten and best-efforts offerings are frequently used in the issuance of corporate bonds.

  \item The underwriting process typically includes six phases: the determination of the funding needs, the selection of the underwriter, the structuring and announcement of the bond offering, pricing, issuance, and closing.

  \item A shelf registration is a method for issuing securities in which the issuer files a single document with regulators that describes and allows for a range of future issuances.

  \item An auction is a public offering method that involves bidding and is helpful both in providing price discovery and in allocating securities. Auctions are frequently used in the issuance of sovereign bonds.

  \item Most bonds are traded in OTC markets, and institutional investors are the major buyers and sellers of bonds in secondary markets.

  \item Sovereign bonds are issued by national governments primarily for fiscal reasons. These bonds take different names and forms depending on where they are issued, their maturities, and their coupon types. Most sovereign bonds are fixed-rate bonds, although some national governments also issue floating-rate bonds and inflation-linked bonds.

  \item Local governments, quasi-government entities, and supranational agencies issue bonds, which are named non-sovereign, quasi-government, and supranational bonds, respectively.

  \item Companies raise debt in the form of bilateral loans, syndicated loans, commercial paper, notes, and bonds.

  \item Commercial paper is a short-term unsecured security that companies use as a source of short-term and bridge financing. Investors in commercial paper are exposed to credit risk, although defaults are rare. Many issuers roll over their commercial paper on a regular basis.

  \item Corporate bonds and notes take different forms depending on the maturities, coupon payment, and principal repayment structures. Important considerations also include collateral backing and contingency provisions.

  \item Medium-term notes are securities that are offered continuously to investors by an agent of the issuer. They can have short-term or long-term maturities.

  \item The structured finance sector includes asset-backed securities, collateralized debt obligations, and other structured financial instruments. All of these seemingly disparate financial instruments share the common attribute of repackaging risks.

  \item Many structured financial instruments are customized instruments that often combine a bond and at least one derivative. The redemption and often the coupons of these structured financial instruments are linked via a formula to the performance of the underlying asset(s). Thus, the bond's payment features are replaced with non-traditional payoffs derived not from the issuer's cash flows but from the performance of the underlying asset(s). Capital protected, yield enhancement, participation and leveraged instruments are typical examples of structured financial instruments.

  \item Financial institutions have access to additional sources of funds, such as retail deposits, central bank funds, interbank funds, large-denomination negotiable certificates of deposit, and repurchase agreements. - A repurchase agreement is similar to a collateralized loan. It involves the sale of a security (the collateral) with a simultaneous agreement by the seller (the borrower) to buy back the same security from the purchaser (the lender) at an agreed-on price in the future. Repurchase agreements are a common source of funding for dealer firms and are also used to borrow securities to implement short positions.

\end{itemize}

\section{PRACTICE PROBLEMS}
\begin{enumerate}
  \item The distinction between investment-grade debt and non-investment-grade debt is best described by differences in:
A. tax status.
B. credit quality.
C. maturity dates.

  \item A bond issued internationally, outside the jurisdiction of the country in whose currency the bond is denominated, is best described as a:
A. Eurobond.
B. foreign bond.
C. municipal bond.

  \item When classified by type of issuer, asset-backed securities are part of the:
A. corporate sector.
B. structured finance sector.
C. government and government-related sector.

  \item Compared with developed market bonds, emerging market bonds most likely:
A. offer lower yields.
B. exhibit higher risk.
C. benefit from lower growth prospects.

  \item With respect to floating-rate bonds, a reference rate (such as MRR) is most likely used to determine the bond's:
A. spread.
B. coupon rate.
C. frequency of coupon payments.

  \item The variability of the coupon rate on a Libor-based floating-rate bond is most likely caused by:
A. periodic resets of the reference rate.
B. market-based reassessments of the issuer's creditworthiness.
C. changing estimates by the Libor administrator of borrowing capacity.

  \item Which of the following statements is most accurate? An interbank offered rate:

\end{enumerate}

A. is a single reference rate.

B. applies to borrowing periods of up to 10 years. C. is used as a reference rate for interest rate swaps.

\begin{enumerate}
  \setcounter{enumi}{7}
  \item An investment bank that underwrites a bond issue most likely:
A. buys and resells the newly issued bonds to investors or dealers.
B. acts as a broker and receives a commission for selling the bonds to investors.
C. incurs less risk associated with selling the bonds than in a best-efforts offering.

  \item In major developed bond markets, newly issued sovereign bonds are most often sold to the public via $a(n)$ :
A. auction.
B. private placement.
C. best-efforts offering.

  \item Which of the following describes privately placed bonds?
A. They are non-underwritten and unregistered.
B. They usually have active secondary markets.
C. They are less customized than publicly offered bonds.

  \item A mechanism by which an issuer may be able to offer additional bonds to the general public without preparing a new and separate offering circular best describes:

\end{enumerate}

A. the grey market.

B. a shelf registration.

C. a private placement.

\begin{enumerate}
  \setcounter{enumi}{11}
  \item Which of the following statements related to secondary bond markets is most accurate?
\end{enumerate}

A. Newly issued corporate bonds are issued in secondary bond markets.

B. Secondary bond markets are where bonds are traded between investors.

C. The major participants in secondary bond markets globally are retail investors.

\begin{enumerate}
  \setcounter{enumi}{12}
  \item A bond market in which a communications network matches buy and sell orders initiated from various locations is best described as an:
A. organized exchange.
B. open market operation.
C. over-the-counter market.

  \item A liquid secondary bond market allows an investor to sell a bond at:

\end{enumerate}

A. the desired price. B. a price at least equal to the purchase price.

C. a price close to the bond's fair market value.

\begin{enumerate}
  \setcounter{enumi}{14}
  \item Corporate bond secondary market trading most often occurs:
A. on a book-entry basis.
B. on organized exchanges.
C. prior to settlement at $T+1$.

  \item Sovereign bonds are best described as:

\end{enumerate}

A. bonds issued by local governments.

B. secured obligations of a national government.

C. bonds backed by the taxing authority of a national government.

\begin{enumerate}
  \setcounter{enumi}{16}
  \item Which factor is associated with a more favorable quality sovereign bond credit rating?
\end{enumerate}

A. Issued in local currency, only

B. Strong domestic savings base, only

C. Issued in local currency of country with strong domestic savings base

\begin{enumerate}
  \setcounter{enumi}{17}
  \item Which type of sovereign bond has the lowest interest rate risk for an investor?
A. Floaters
B. Coupon bonds
C. Discount bonds

  \item Agency bonds are issued by:
A. local governments.
B. national governments.
C. quasi-government entities.

  \item The type of bond issued by a multilateral agency such as the International Monetary Fund (IMF) is best described as a:

\end{enumerate}

A. sovereign bond.

B. supranational bond.

C. quasi-government bond.

\begin{enumerate}
  \setcounter{enumi}{20}
  \item A bond issued by a local government authority, typically without an explicit funding commitment from the national government, is most likely classified as a:
\end{enumerate}

A. sovereign bond.

B. quasi-government bond

C. non-sovereign government bond. 22. Which of the following statements relating to commercial paper is most accurate?
A. There is no secondary market for trading commercial paper.
B. Only the strongest, highly rated companies issue commercial paper.
C. Commercial paper is a source of interim financing for long-term projects.

\begin{enumerate}
  \setcounter{enumi}{22}
  \item Eurocommercial paper is most likely:
A. negotiable.
B. denominated in euros.
C. issued on a discount basis.

  \item For the issuer, a sinking fund arrangement is most similar to a:
A. term maturity structure.
B. serial maturity structure.
C. bondholder put provision.

  \item When issuing debt, a company may use a sinking fund arrangement as a means of reducing:
A. credit risk.
B. inflation risk.
C. interest rate risk.

  \item Which of the following is a source of wholesale funds for banks?
A. Demand deposits
B. Money market accounts
C. Negotiable certificates of deposit

  \item A characteristic of negotiable certificates of deposit is:

\end{enumerate}

A. they are mostly available in small denominations.

B. they can be sold in the open market prior to maturity.

C. a penalty is imposed if the depositor withdraws funds prior to maturity.

\begin{enumerate}
  \setcounter{enumi}{27}
  \item A repurchase agreement is most comparable to $\mathrm{a}(\mathrm{n})$ :
A. interbank deposit.
B. collateralized loan.
C. negotiable certificate of deposit.

  \item The repo margin is:
A. negotiated between counterparties.
B. established independently of market-related conditions. C. structured on an agreement assuming equal credit risks to all counterparties.

  \item The repo margin on a repurchase agreement is most likely to be lower when:
A. the underlying collateral is in short supply.
B. the maturity of the repurchase agreement is long.
C. the credit risk associated with the underlying collateral is high.

\end{enumerate}

\section{SOLUTIONS}
\begin{enumerate}
  \item B is correct. The distinction between investment-grade and non-investment-grade debt relates to differences in credit quality, not tax status or maturity dates. Debt markets are classified based on the issuer's creditworthiness as judged by the credit ratings agencies. Ratings of Baa3 or above by Moody's Investors Service or BBB- or above by Standard \& Poor's and Fitch Ratings are considered investment grade, whereas ratings below these levels are referred to as non-investment grade (also called high yield, speculative, or junk).

  \item A is correct. Eurobonds are issued internationally, outside the jurisdiction of any single country. B is incorrect because foreign bonds are considered international bonds, but they are issued in a specific country, in the currency of that country, by an issuer domiciled in another country. $C$ is incorrect because municipal bonds are US domestic bonds issued by a state or local government.

  \item B is correct. Asset-backed securities are securitized debt instruments created by securitization, a process that involves transferring ownership of assets from the original owners to a special legal entity. The special legal entity then issues securities backed by the transferred assets. The assets' cash flows are used to pay interest and repay the principal owed to the holders of the securities. Assets that are typically used to create securitized debt instruments include loans (such as mortgage loans) and receivables (such as credit card receivables). The structured finance sector includes such securitized debt instruments (also called asset-backed securities).

  \item B is correct. Many emerging countries lag developed countries in the areas of political stability, property rights, and contract enforcement. Consequently, emerging market bonds usually exhibit higher risk than developed market bonds. A is incorrect because emerging market bonds typically offer higher (not lower) yields than developed market bonds to compensate investors for the higher risk. $\mathrm{C}$ is incorrect because emerging market bonds usually benefit from higher (not lower) growth prospects than developed market bonds.

  \item B is correct. The coupon rate of a floating-rate bond is expressed as a reference rate plus a spread. Different reference rates are used depending on where the bond is issued and its currency denomination, but one of the most widely used set of reference rates is Libor. A and $C$ are incorrect because a bond's spread and frequency of coupon payments are typically set when the bond is issued and do not change during the bond's life.

  \item A is correct. Changes in the coupon rate of interest on a floating-rate bond that uses a Libor reference rate result from changes in the reference rate (for example, 90-day Libor), which resets periodically. Therefore, the coupon rate adjusts to the level of market interest rates (plus the spread) each time the reference rate is reset.

  \item C is correct. Interbank offered rates are used as reference rates not only for floating-rate bonds but also for other debt instruments, including mortgages, derivatives such as interest rate and currency swaps, and many other financial contracts and products. A and B are incorrect because an interbank offered rate such as Libor or Euribor is a set of reference rates (not a single reference rate) for different borrowing periods of up to one year (not 10 years).

  \item A is correct. In an underwritten offering (also called firm commitment offering), the investment bank (called the underwriter) guarantees the sale of the bond issue at an offering price that is negotiated with the issuer. Thus, the underwriter takes the risk of buying the newly issued bonds from the issuer and then reselling them to investors or to dealers, which then sell them to investors. B and $\mathrm{C}$ are incorrect because the bond issuing mechanism in which an investment bank acts as a broker and receives a commission for selling the bonds to investors, and incurs less risk associated with selling the bonds, is a best-efforts offering (not an underwritten offering).

  \item A is correct. In major developed bond markets, newly issued sovereign bonds are sold to the public via an auction. $B$ and $C$ are incorrect because sovereign bonds are rarely issued via private placements or best-efforts offerings.

  \item A is correct. Private placements are typically non-underwritten, unregistered bond offerings that are sold only to a single investor or a small group of investors.

  \item B is correct. A shelf registration allows certain authorized issuers to offer additional bonds to the general public without having to prepare a new and separate offering circular. The issuer can offer multiple bond issuances under the same master prospectus and only has to prepare a short document when additional bonds are issued. A is incorrect because the grey market is a forward market for bonds about to be issued. $\mathrm{C}$ is incorrect because a private placement is a non-underwritten, unregistered offering of bonds that are not sold to the general public but directly to an investor or a small group of investors.

  \item B is correct. In secondary bond markets, bonds are traded between investors. A is incorrect because newly issued bonds (whether from corporate issuers or other types of issuers) are issued in primary (not secondary) bond markets. $C$ is incorrect because the major participants in secondary bond markets globally are large institutional investors and central banks (not retail investors).

  \item $\mathrm{C}$ is correct. In over-the-counter (OTC) markets, buy and sell orders are initiated from various locations and then matched through a communications network. Most bonds are traded in OTC markets. A is incorrect because on organized exchanges, buy and sell orders may come from anywhere, but the transactions must take place at the exchange according to the rules imposed by the exchange. $B$ is incorrect because open market operations refer to central bank activities in secondary bond markets. Central banks buy and sell bonds, usually sovereign bonds issued by the national government, as a means to implement monetary policy.

  \item $\mathrm{C}$ is correct. Liquidity in secondary bond markets refers to the ability to buy or sell bonds quickly at prices close to their fair market value. A and B are incorrect because a liquid secondary bond market does not guarantee that a bond will sell at the price sought by the investor, or that the investor will not face a loss on his or her investment.

  \item A is correct. The vast majority of corporate bonds are traded in over-the-counter (OTC) markets that use electronic trading platforms through which users submit buy and sell orders. Settlement of trades in the OTC markets occurs by means of a simultaneous exchange of bonds for cash on the books of the clearing system "on a paperless, computerized book-entry basis."

  \item $\mathrm{C}$ is correct. Sovereign bonds are usually unsecured obligations of the national government issuing the bonds; they are backed not by collateral but by the taxing authority of the national government. A is incorrect because bonds issued by local governments are non-sovereign (not sovereign) bonds. $B$ is incorrect because sovereign bonds are typically unsecured (not secured) obligations of a national government.

  \item $\mathrm{C}$ is correct. Bonds issued in the sovereign's currency and a strong domestic savings base are both favorable sovereign rating factors. It is common to observe a higher credit rating for sovereign bonds issued in local currency because of the sovereign's ability to tax its citizens and print its own currency. Although there are practical limits to the sovereign's taxing and currency-printing capacities, each tends to support a sovereign's ability to repay debt. A strong domestic savings base is advantageous because it supports the sovereign's ability to issue debt in local currency to domestic investors.

  \item A is correct. Floaters are bonds with a floating rate of interest that resets periodically based on changes in the level of a reference rate, such as Libor. Because changes in the reference rate reflect changes in market interest rates, price changes of floaters are far less pronounced than those of fixed-rate bonds, such as coupon bonds and discount bonds. Thus, investors holding floaters are less exposed to interest rate risk than investors holding fixed-rate discount or coupon bonds.

  \item $\mathrm{C}$ is correct. Agency bonds are issued by quasi-government entities. These entities are agencies and organizations usually established by national governments to perform various functions for them. A and $\mathrm{B}$ are incorrect because local and national governments issue non-sovereign and sovereign bonds, respectively.

  \item B is correct. The IMF is a multilateral agency that issues supranational bonds. A and $\mathrm{C}$ are incorrect because sovereign bonds and quasi-government bonds are issued by national governments and by entities that perform various functions for national governments, respectively.

  \item C is correct. Bonds issued by levels of government below the national level-such as provinces, regions, states, cities, and local government authorities-are classified as non-sovereign government bonds. These bonds are typically not guaranteed by the national government.

  \item C is correct. Companies use commercial paper not only as a source of funding working capital and seasonal demand for cash but also as a source of interim financing for long-term projects until permanent financing can be arranged. A is incorrect because there is a secondary market for trading commercial paper, although trading is limited except for the largest issues. B is incorrect because commercial paper is issued by companies across the risk spectrum, although only the strongest, highly rated companies issue low-cost commercial paper.

  \item A is correct. Commercial paper, whether US commercial paper or Eurocommercial paper, is negotiable -that is, investors can buy and sell commercial paper on secondary markets. B is incorrect because Eurocommercial paper can be denominated in any currency. $\mathrm{C}$ is incorrect because Eurocommercial paper may be issued on an interest-bearing (or yield) basis or a discount basis.

  \item B is correct. With a serial maturity structure, a stated number of bonds mature and are paid off on a pre-determined schedule before final maturity. With a sinking fund arrangement, the issuer is required to set aside funds over time to retire the bond issue. Both result in a pre-determined portion of the issue being paid off according to a pre-determined schedule.

  \item A is correct. A sinking fund arrangement is a way to reduce credit risk by making the issuer set aside funds over time to retire the bond issue. $\mathrm{B}$ and $\mathrm{C}$ are incorrect because a sinking fund arrangement has no effect on inflation risk or interest rate risk. 26. C is correct. Wholesale funds available for banks include central bank funds, interbank funds, and negotiable certificates of deposit. A and B are incorrect because demand deposits (also known as checking accounts) and money market accounts are retail deposits, not wholesale funds.

  \item B is correct. A negotiable certificate of deposit (CD) allows any depositor (initial or subsequent) to sell the $\mathrm{CD}$ in the open market prior to maturity. A is incorrect because negotiable CDs are mostly available in large (not small) denominations. Large-denomination negotiable CDs are an important source of wholesale funds for banks, whereas small-denomination CDs are not. C is incorrect because a penalty is imposed if the depositor withdraws funds prior to maturity for non-negotiable (instead of negotiable) CDs.

  \item B is correct. A repurchase agreement (repo) can be viewed as a collateralized loan in which the security sold and subsequently repurchased represents the collateral posted. A and C are incorrect because interbank deposits and negotiable certificates of deposit are unsecured deposits-that is, there is no collateral backing the deposit.

  \item A is correct. Repo margins vary by transaction and are negotiated bilaterally between the counterparties.

  \item A is correct. The repo margin (the difference between the market value of the underlying collateral and the value of the loan) is a function of the supply and demand conditions of the collateral. The repo margin is typically lower if the underlying collateral is in short supply or if there is a high demand for it. B and $\mathrm{C}$ are incorrect because the repo margin is usually higher (not lower) when the maturity of the repurchase agreement is long and when the credit risk associated with the underlying collateral is high. LEARNING MODULE

\end{enumerate}

\begin{center}
\includegraphics[max width=\textwidth]{2023_05_04_7b535d0a870224f62e3dg-551}
\end{center}

\section*{Introduction to Fixed-Income Valuation }
James F. Adams, PhD, CFA, is at New York University (USA). Donald J. Smith, PhD, is at Boston University Questrom School of Business (USA).

\section{LEARNING OUTCOME}
\begin{center}
\begin{tabular}{c|l}
Mastery & The candidate should be able to: \\
\hline
$\square$ & calculate a bond's price given a market discount rate \\
identify the relationships among a bond's price, coupon rate, &  \\
maturity, and market discount rate (yield-to-maturity) &  \\
define spot rates and calculate the price of a bond using spot rates &  \\
$\square$ & $\begin{array}{l}\text { describe and calculate the flat price, accrued interest, and the full } \\ \text { price of a bond } \\ \text { describe matrix pricing } \\ \text { calculate annual yield on a bond for varying compounding periods in } \\ \text { a year } \\ \text { calculate and interpret yield measures for fixed-rate bonds and } \\ \text { floating-rate notes } \\ \text { calculate and interpret yield measures for money market instruments } \\ \text { define and compare the spot curve, yield curve on coupon bonds, } \\ \text { par curve, and forward curve } \\ \text { define forward rates and calculate spot rates from forward rates, } \\ \text { forward rates from spot rates, and the price of a bond using forward } \\ \text { rates } \\ \text { compare, calculate, and interpret yield spread measures }\end{array}$ \\
$\square$ &  \\
\end{tabular}
\end{center}

\section{INTRODUCTION}
Globally, the fixed-income market is a key source of financing for businesses and governments. In fact, the total market value outstanding of corporate and government bonds is significantly larger than that of equity securities. Similarly, the fixed-income market, which is also called the debt market or bond market, represents a significant investing opportunity for institutions as well as individuals. Pension funds, mutual funds, insurance companies, and sovereign wealth funds, among others, are major fixed-income investors. Retirees who desire a relatively stable income stream often hold fixed-income securities. Clearly, understanding how to value fixed-income securities is important to investors, issuers, and financial analysts. We focus on the valuation of traditional (option-free) fixed-rate bonds, although other debt securities, such as floating-rate notes and money market instruments, are also covered.

We first describe and illustrate basic bond valuation, which includes pricing a bond using a market discount rate for each of the future cash flows and pricing a bond using a series of spot rates. Valuation using spot rates allows for each future cash flow to be discounted at a rate associated with its timing. This valuation methodology for future cash flows has applications well beyond the fixed-income market. Relationships among a bond's price, coupon rate, maturity, and market discount rate (yield-to-maturity) are also described and illustrated.

We then turn our attention to how bond prices and yields are quoted and calculated in practice. When bonds are actively traded, investors can observe the price and calculate various yield measures. However, these yield measures differ by the type of bond. In practice, different measures are used for fixed-rate bonds, floating-rate notes, and money market instruments.

We then discuss the maturity or term structure of interest rates, involving an analysis of yield curves, which illustrates the relationship between yields-to-maturity and times-to-maturity on bonds with otherwise similar characteristics. Lastly, we describe yield spreads, measures of how much additional yield over the benchmark security (usually a government bond) investors expect for bearing additional risk.

\section{BOND PRICES AND THE TIME VALUE OF MONEY}
calculate a bond's price given a market discount rate

identify the relationships among a bond's price, coupon rate, maturity, and market discount rate (yield-to-maturity)

define spot rates and calculate the price of a bond using spot rates

Bond pricing is an application of discounted cash flow analysis. The complexity of the pricing depends on the particular bond's features and the rate (or rates) used to do the discounting. This section starts with using a single discount factor for all future cash flows and concludes with the most general approach to bond valuation. The general approach to bond valuation is to use a series of spot rates that correspond to the timing of the future cash flows.

\section{Bond Pricing with a Market Discount Rate}
On a traditional (option-free) fixed-rate bond, the promised future cash flows are a series of coupon interest payments and repayment of the full principal at maturity. The coupon payments occur on regularly scheduled dates; for example, an annual payment bond might pay interest on 15 June of each year for five years. The final coupon typically is paid together with the full principal on the maturity date. The price of the bond at issuance is the present value of the promised cash flows. The market discount rate is used in the time-value-of-money calculation to obtain the present value. The market discount rate is the rate of return required by investors given the risk of the investment in the bond. It is also called the required yield, or the required rate of return.

For example, suppose the coupon rate on a bond is $4 \%$ and the payment is made once a year. If the time-to-maturity is five years and the market discount rate is 6\%, the price of the bond is 91.575 per 100 of par value. The par value is the amount of principal on the bond.

$$
\begin{gathered}
\frac{4}{(1.06)^{1}}+\frac{4}{(1.06)^{2}}+\frac{4}{(1.06)^{3}}+\frac{4}{(1.06)^{4}}+\frac{104}{(1.06)^{5}}= \\
3.774+3.560+3.358+3.168+77.715=91.575 .
\end{gathered}
$$

The final cash flow of 104 is the redemption of principal (100) plus the coupon payment for that date (4). The price of the bond is the sum of the present values of the five cash flows. The price per 100 of par value may be interpreted as the percentage of par value. If the par value is USD100,000, the coupon payments are USD4,000 each year and the price of the bond is USD91,575. Its price is $91.575 \%$ of par value. This bond is described as trading at a discount because the price is below par value.

Suppose that another five-year bond has a coupon rate of $8 \%$ paid annually. If the market discount rate is again 6\%, the price of the bond is 108.425 .

$$
\begin{gathered}
\frac{8}{(1.06)^{1}}+\frac{8}{(1.06)^{2}}+\frac{8}{(1.06)^{3}}+\frac{8}{(1.06)^{4}}+\frac{108}{(1.06)^{5}}= \\
7.547+7.120+6.717+6.337+80.704=108.425 .
\end{gathered}
$$

This bond is trading at a premium because its price is above par value.

If another five-year bond pays a $6 \%$ annual coupon and the market discount rate still is $6 \%$, the bond would trade at par value.

$$
\begin{gathered}
\frac{6}{(1.06)^{1}}+\frac{6}{(1.06)^{2}}+\frac{6}{(1.06)^{3}}+\frac{6}{(1.06)^{4}}+\frac{106}{(1.06)^{5}}= \\
5.660+5.340+5.038+4.753+79.209=100.000 .
\end{gathered}
$$

The coupon rate indicates the amount the issuer promises to pay the bondholders each year in interest. The market discount rate reflects the amount investors need to receive in interest each year in order to pay full par value for the bond. Therefore, assuming that these three bonds have the same risk, which is consistent with them having the same market discount rate, the $4 \%$ bond offers a "deficient" coupon rate. The amount of the discount below par value is the present value of the deficiency, which is $2 \%$ of par value each year. The present value of the deficiency, discounted using the market discount rate, is -8.425 .

$$
\frac{-2}{(1.06)^{1}}+\frac{-2}{(1.06)^{2}}+\frac{-2}{(1.06)^{3}}+\frac{-2}{(1.06)^{4}}+\frac{-2}{(1.06)^{5}}=-8.425
$$

The price of the $4 \%$ coupon bond is $91.575(=100-8.425)$. In the same manner, the $8 \%$ bond offers an "excessive" coupon rate given the risk because investors require only $6 \%$. The amount of the premium is the present value of the excess cash flows, which is +8.425 . The price of the $8 \%$ bond is $108.425(=100+8.425)$.

These examples demonstrate that the price of a fixed-rate bond, relative to par value, depends on the relationship of the coupon rate to the market discount rate. Here is a summary of the relationships:

\begin{itemize}
  \item When the coupon rate is less than the market discount rate, the bond is priced at a discount below par value.

  \item When the coupon rate is greater than the market discount rate, the bond is priced at a premium above par value.

  \item When the coupon rate is equal to the market discount rate, the bond is priced at par value. At this point, it is assumed that the bond is priced on a coupon payment date. If the bond is between coupon payment dates, the price paid will include accrued interest, which is interest that has been earned but not yet paid. Accrued interest is discussed in detail later.

\end{itemize}

Equation 1 is a general formula for calculating a bond price given the market discount rate:

$$
P V=\frac{P M T}{(1+r)^{1}}+\frac{P M T}{(1+r)^{2}}+\cdots+\frac{P M T+F V}{(1+r)^{N}}
$$

where

$$
\begin{aligned}
P V & =\text { present value, or the price of the bond } \\
P M T & =\text { coupon payment per period } \\
F V & =\text { future value paid at maturity, or the par value of the bond } \\
r & =\text { market discount rate, or required rate of return per period } \\
N & =\text { number of evenly spaced periods to maturity }
\end{aligned}
$$

The examples so far have been for an annual payment bond, which is the convention for most European bonds. Asian and North American bonds generally make semiannual payments, and the stated rate is the annual coupon rate. Suppose the coupon rate on a bond is stated to be $8 \%$ and the payments are made twice a year (semiannually) on 15 June and 15 December. For each 100 in par value ( $F V=100)$, the coupon payment per period is $4(P M T=4)$. If there are three years to maturity, there are six evenly spaced semiannual periods $(N=6)$. If the market discount rate is $3 \%$ per semiannual period $(r=0.03)$, the price of the bond is 105.417 per 100 of par value.

$$
\frac{4}{(1.03)^{1}}+\frac{4}{(1.03)^{2}}+\frac{4}{(1.03)^{3}}+\frac{4}{(1.03)^{4}}+\frac{4}{(1.03)^{5}}+\frac{104}{(1.03)^{6}}=105.417
$$

This bond is trading at a premium above par value because the coupon rate of $4 \% \mathrm{per}$ period is greater than the market discount rate of $3 \%$ per period. Usually, those interest rates are annualized by multiplying the rate per period by the number of periods in a year. Therefore, an equivalent statement is that the bond is priced at a premium because its stated annual coupon rate of $8 \%$ is greater than the stated annual market discount rate of 6\%. Interest rates, unless stated otherwise, are typically quoted as annual rates.

\section{EXAMPLE 1}
\section{Bonds Trading at a Discount, at a Premium, and at Par}
\begin{enumerate}
  \item Identify whether each of the following bonds is trading at a discount, at par value, or at a premium. Calculate the prices of the bonds per 100 in par value using Equation 1. If the coupon rate is deficient or excessive compared with the market discount rate, calculate the amount of the deficiency or excess per 100 of par value.
\end{enumerate}

\begin{center}
\begin{tabular}{lccc}
\hline
Bond & $\begin{array}{c}\text { Coupon Payment } \\ \text { per Period }\end{array}$ & $\begin{array}{c}\text { Number of Periods } \\ \text { to Maturity }\end{array}$ & $\begin{array}{c}\text { Market Discount Rate } \\ \text { per Period }\end{array}$ \\
\hline
A & 2 & 6 & $3 \%$ \\
B & 6 & 4 & $4 \%$ \\
C & 5 & 5 & $5 \%$ \\
D & 0 & 10 & $2 \%$ \\
\hline
\end{tabular}
\end{center}

\section{Solution:}
\section{Bond A}
$\frac{2}{(1.03)^{1}}+\frac{2}{(1.03)^{2}}+\frac{2}{(1.03)^{3}}+\frac{2}{(1.03)^{4}}+\frac{2}{(1.03)^{5}}+\frac{102}{(1.03)^{6}}=94.583$.

Bond $\mathrm{A}$ is trading at a discount. Its price is below par value because the coupon rate per period ( $2 \%)$ is less than the required yield per period $(3 \%)$. The deficiency per period is the coupon rate minus the market discount rate, times the par value: $(0.02-0.03) \times 100=-1$. The present value of deficiency is -5.417 , discounted using the required yield (market discount rate) per period.

$\frac{-1}{(1.03)^{1}}+\frac{-1}{(1.03)^{2}}+\frac{-1}{(1.03)^{3}}+\frac{-1}{(1.03)^{4}}+\frac{-1}{(1.03)^{5}}+\frac{-1}{(1.03)^{6}}=-5.417$

The amount of the deficiency can be used to calculate the price of the bond; the price is $94.583(=100-5.417)$

\section{Bond B}
$\frac{6}{(1.04)^{1}}+\frac{6}{(1.04)^{2}}+\frac{6}{(1.04)^{3}}+\frac{106}{(1.04)^{4}}=107.260$

Bond $B$ is trading at a premium because the coupon rate per period (6\%) is greater than the market discount rate per period (4\%). The excess per period is the coupon rate minus the market discount rate, times the par value: $(0.06$ $-0.04) \times 100=+2$. The present value of excess is +7.260 , discounted using the required yield per period.

$\frac{2}{(1.04)^{1}}+\frac{2}{(1.04)^{2}}+\frac{2}{(1.04)^{3}}+\frac{2}{(1.04)^{4}}=7.260$

The price of the bond is $107.260(=100+7.260)$.

\section{Bond C}
$\frac{5}{(1.05)^{1}}+\frac{5}{(1.05)^{2}}+\frac{5}{(1.05)^{3}}+\frac{5}{(1.05)^{4}}+\frac{105}{(1.05)^{5}}=100.000$

Bond $C$ is trading at par value because the coupon rate is equal to the market discount rate. The coupon payments are neither excessive nor deficient given the risk of the bond.

\section{Bond D}
$$
\frac{100}{(1.02)^{10}}=82.035
$$

Bond $\mathrm{D}$ is a zero-coupon bond, which always will trade at a discount below par value (as long as the required yield is greater than zero). The deficiency in the coupon payments is -2 per period: $(0-0.02) \times 100=-2$.

$\frac{-2}{(1.02)^{1}}+\frac{-2}{(1.02)^{2}}+\frac{-2}{(1.02)^{3}}+\frac{-2}{(1.02)^{4}}+\frac{-2}{(1.02)^{5}}+$

$\frac{-2}{(1.02)^{6}}+\frac{-2}{(1.02)^{7}}+\frac{-2}{(1.02)^{8}}+\frac{-2}{(1.02)^{9}}+\frac{-2}{(1.02)^{10}}=-17.965$.

The price of the bond is $82.035(=100-17.965)$.

\section{Yield-to-Maturity}
If the market price of a bond is known, Equation 1 can be used to calculate its yield-to-maturity (YTM). The yield-to-maturity is the internal rate of return on the cash flows-the uniform interest rate such that when the future cash flows are discounted at that rate, the sum of the present values equals the price of the bond. It is the implied market discount rate.

The yield-to-maturity is the rate of return on the bond to an investor given three critical assumptions:

\begin{enumerate}
  \item The investor holds the bond to maturity.

  \item The issuer makes all the coupon and principal payments in the full amount on the scheduled dates. Therefore, the yield-to-maturity is the promised yield-the yield assuming the issuer does not default on any of the payments.

  \item The investor is able to reinvest coupon payments at that same yield. This is a characteristic of an internal rate of return.

\end{enumerate}

For example, suppose that a four-year, $5 \%$ annual coupon payment bond is priced at 105 per 100 of par value. The yield-to-maturity is the solution for the rate, $r$, in this equation:

$$
105=\frac{5}{(1+r)^{1}}+\frac{5}{(1+r)^{2}}+\frac{5}{(1+r)^{3}}+\frac{105}{(1+r)^{4}}
$$

Solving by trial-and-error search or using the time-value-of-money keys on a financial calculator obtains the result that $r=0.03634$. The bond trades at a premium because its coupon rate $(5 \%)$ is greater than the yield that is required by investors (3.634\%). The yield is implied by the market price.

The yield-to-maturity on a bond may be positive or negative. In fact, following the launch of unconventional monetary policies in many regions after the global financial crisis of 2008-2009, yields on many government bonds fell into the negative territory. The market value of negative-yielding bonds reached $\$ 17$ trillion in August 2019, amounting to roughly 1 in 3 of all investment-grade bonds, according to the Bloomberg Barclays Global-Aggregate Index. Bonds with a negative yield-to-maturity include those issued at higher yields in the past that have experienced significant price appreciation, as well as newly issued zero-coupon sovereign government bonds priced at a premium to par value. Suppose the four-year bond in the previous example were trading at a price of 122.50 per 100 of par value, instead of 105 . Using Equation 1 , the yield-to-maturity, $r$, for this bond would be $-0.549 \%$.

Yield-to-maturity does not depend on the actual amount of par value in a fixed-income portfolio. For example, suppose a Japanese institutional investor owns a three-year, 2.5\% semiannual payment bond having a par value of JPY100 million. The bond currently is priced at JPY98,175,677. The yield per semiannual period can be obtained by solving this equation for $r$ :

$$
98.175677=\frac{1.25}{(1+r)^{1}}+\frac{1.25}{(1+r)^{2}}+\frac{1.25}{(1+r)^{3}}+\frac{1.25}{(1+r)^{4}}+\frac{1.25}{(1+r)^{5}}+\frac{101.25}{(1+r)^{6}} .
$$

The yield per semiannual period turns out to be $1.571 \%(r=0.01571)$, which can be annualized to be $3.142 \%(0.01571 \times 2=0.03142)$. In general, a three-year, $2.5 \%$ semiannual bond for any amount of par value has an annualized yield-to-maturity of $3.142 \%$ if it is priced at $98.175677 \%$ of par value.

\section{EXAMPLE 2}
\section{Yields-to-Maturity for a Premium, Discount, and Zero-Coupon Bond}
\begin{enumerate}
  \item Calculate the yields-to-maturity for the following bonds. The prices are stated per 100 of par value.
\end{enumerate}

\begin{center}
\begin{tabular}{|c|c|c|c|}
\hline
Bond & $\begin{array}{c}\text { Coupon } \\ \text { Payment } \\ \text { per Period }\end{array}$ & $\begin{array}{c}\text { Number of Periods } \\ \text { to Maturity }\end{array}$ & Price \\
\hline
A & 3.5 & 4 & 103.75 \\
\hline
B & 2.25 & 6 & 96.50 \\
\hline
C & 0 & 60 & 22.375 \\
\hline
\end{tabular}
\end{center}

\section{Solution:}
\section{Bond A}
$103.75=\frac{3.5}{(1+r)^{1}}+\frac{3.5}{(1+r)^{2}}+\frac{3.5}{(1+r)^{3}}+\frac{103.5}{(1+r)^{4}} ; r=0.02503$.

Bond A is trading at a premium, so its yield-to-maturity per period (2.503\%) must be lower than its coupon rate per period (3.5\%).

\section{Bond B}
$96.50=\frac{2.25}{(1+r)^{1}}+\frac{2.25}{(1+r)^{2}}+\frac{2.25}{(1+r)^{3}}+\frac{2.25}{(1+r)^{4}}+$

$\frac{2.25}{(1+r)^{5}}+\frac{102.25}{(1+r)^{6}} ; r=0.02894$.

Bond $\mathrm{B}$ is trading at a discount, so the yield-to-maturity per period (2.894\%) must be higher than the coupon rate per period (2.25\%).

\section{Bond C}
$22.375=\frac{100}{(1+r)^{60}} ; r=0.02527$.

Bond $\mathrm{C}$ is a zero-coupon bond trading at a significant discount below par value. Its yield-to-maturity is $2.527 \%$ per period.

\section{Relationships between the Bond Price and Bond Characteristics}
The price of a fixed-rate bond will change whenever the market discount rate changes. The following relationships pertain to the change in the bond price given the market discount rate:

\begin{enumerate}
  \item The bond price is inversely related to the market discount rate. When the market discount rate increases, the bond price decreases (the inverse effect).

  \item For the same coupon rate and time-to-maturity, the percentage price change is greater (in absolute value, meaning without regard to the sign of the change) when the market discount rate goes down than when it goes up (the convexity effect). 3. For the same time-to-maturity, a lower-coupon bond has a greater percentage price change than a higher-coupon bond when their market discount rates change by the same amount (the coupon effect).

  \item Generally, for the same coupon rate, a longer-term bond has a greater percentage price change than a shorter-term bond when their market discount rates change by the same amount (the maturity effect).

\end{enumerate}

Exhibit 1 illustrates these relationships using nine annual coupon payment bonds. The bonds have different coupon rates and times-to-maturity but otherwise are the same in terms of risk. The coupon rates are $10 \%, 20 \%$, and $30 \%$ for bonds having 10 , 20 , and 30 years to maturity. At first, the bonds are all priced at a market discount rate of $20 \%$. Equation 1 is used to determine the prices. Going across columns, the market discount rate is decreased by 1 percentage point, from $20 \%$ to $19 \%$, and next, it is increased from $20 \%$ to $21 \%$.

Exhibit 1: Relationships between Bond Prices and Bond Characteristics

\begin{center}
\begin{tabular}{|c|c|c|c|c|c|c|c|}
\hline
\multirow[b]{2}{*}{Bond} & \multirow[b]{2}{*}{$\begin{array}{c}\text { Coupon } \\
\text { Rate }\end{array}$} & \multirow[b]{2}{*}{Maturity} & \multirow[b]{2}{*}{$\begin{array}{c}\text { Price at } \\
20 \%\end{array}$} & \multicolumn{2}{|c|}{$\begin{array}{c}\text { Discount Rates Go } \\
\text { Down }\end{array}$} & \multicolumn{2}{|c|}{$\begin{array}{c}\text { Discount Rates Go } \\
\text { Up }\end{array}$} \\
\hline
 &  &  &  & $\begin{array}{c}\text { Price at } \\ 19 \%\end{array}$ & $\begin{array}{c}\% \\ \text { Change }\end{array}$ & $\begin{array}{c}\text { Price at } \\ 21 \%\end{array}$ & $\begin{array}{c}\% \\ \text { Change }\end{array}$ \\
\hline
A & $10.00 \%$ & 10 & 58.075 & 60.950 & $4.95 \%$ & 55.405 & $-4.60 \%$ \\
\hline
B & $20.00 \%$ & 10 & 100.000 & 104.339 & $4.34 \%$ & 95.946 & $-4.05 \%$ \\
\hline
C & $30.00 \%$ & 10 & 141.925 & 147.728 & $4.09 \%$ & 136.487 & $-3.83 \%$ \\
\hline
D & $10.00 \%$ & 20 & 51.304 & 54.092 & $5.43 \%$ & 48.776 & $-4.93 \%$ \\
\hline
E & $20.00 \%$ & 20 & 100.000 & 105.101 & $5.10 \%$ & 95.343 & $-4.66 \%$ \\
\hline
F & $30.00 \%$ & 20 & 148.696 & 156.109 & $4.99 \%$ & 141.910 & $-4.56 \%$ \\
\hline
$G$ & $10.00 \%$ & 30 & 50.211 & 52.888 & $5.33 \%$ & 47.791 & $-4.82 \%$ \\
\hline
$\mathrm{H}$ & $20.00 \%$ & 30 & 100.000 & 105.235 & $5.23 \%$ & 95.254 & $-4.75 \%$ \\
\hline
I & $30.00 \%$ & 30 & 149.789 & 157.581 & $5.20 \%$ & 142.716 & $-4.72 \%$ \\
\hline
\end{tabular}
\end{center}

The first relationship is that the bond price and the market discount rate move inversely. All bond prices in Exhibit 1 go up when the rates go down from $20 \%$ to 19\%, and all prices go down when the rates go up from $20 \%$ to $21 \%$. This happens because of the fixed cash flows on a fixed-rate bond. The numerators in Equation 1 do not change when the market discount rate in the denominators rises or falls. Therefore, the price $(P V)$ moves inversely with the market discount rate $(r)$.

The second relationship reflects the convexity effect. In Exhibit 1, the percentage price changes are calculated using this equation:

$$
\% \text { Change }=\frac{\text { New price }- \text { Old price }}{\text { Old price }} \text {. }
$$

For example, when the market discount rate on Bond A falls, the price rises from 58.075 to 60.950 . The percentage price increase is $4.95 \%$.

$$
\% \text { Change }=\frac{60.950-58.075}{58.075}=0.0495 \text {. }
$$

For each bond, the percentage price increases are greater in absolute value than the percentage price decreases. This implies that the relationship between bond prices and the market discount rate is not linear; instead, it is curved. It is described as being "convex." The convexity effect is shown in Exhibit 2 for a 10\%, 10-year bond.

\section{Exhibit 2: The Convex Relationship between the Market Discount Rate and}
 the Price of a 10-Year, 10\% Annual Coupon Payment Bond\begin{center}
\includegraphics[max width=\textwidth]{2023_05_04_7b535d0a870224f62e3dg-559}
\end{center}

The third relationship is the coupon effect. Consider Bonds $A, B$, and $C$, which have 10 years to maturity. For both the decrease and increase in the yield-to-maturity, Bond A has a larger percentage price change than Bond $B$ and Bond $B$ has a larger change than C. The same pattern holds for the 20-year and 30-year bonds. Therefore, lower-coupon bonds have more price volatility than higher-coupon bonds, other things being equal.

The fourth relationship is the maturity effect. Compare the results for Bonds $A$ and D, for Bonds $B$ and $E$, and for Bonds $C$ and $F$. The 20-year bonds have greater percentage price changes than the 10-year bonds for either an increase or a decrease in the market discount rate. In general, longer-term bonds have more price volatility than shorter-term bonds, other things being equal.

There are exceptions to the maturity effect. That is why the word "generally" appears in the statement of the relationship at the beginning of this section. Compare the results in Exhibit 1 for Bonds $D$ and $G$, for Bonds $E$ and $\mathrm{H}$, and for Bonds $\mathrm{F}$ and $\mathrm{I}$. For the higher-coupon bonds trading at a premium, Bonds $F$ and I, the usual property holds: The 30-year bonds have greater percentage price changes than the 20-year bonds. The same pattern holds for Bonds $\mathrm{E}$ and $\mathrm{H}$, which are priced initially at par value. The exception is illustrated in the results for Bonds $D$ and $G$, which are priced at a discount because the coupon rate is lower than the market discount rate. The 20-year, 10\% bond has a greater percentage price change than the 30-year, 10\% bond. Exceptions to the maturity effect are rare in practice. They occur only for low-coupon (but not zero-coupon), long-term bonds trading at a discount. The maturity effect always holds on zero-coupon bonds, as it does for bonds priced at par value or at a premium above par value. One final point to note in Exhibit 1 is that Bonds B, E, and $\mathrm{H}$, which have coupon rates of $20 \%$, all trade at par value when the market discount rate is $20 \%$. A bond having a coupon rate equal to the market discount rate is priced at par value on a coupon payment date, regardless of the number of years to maturity.

\section{EXAMPLE 3}
\section{Bond Percentage Price Changes Based on Coupon and Time-to-Maturity}
An investor is considering the following six annual coupon payment government bonds:

\begin{center}
\begin{tabular}{lccc}
\hline
Bond & Coupon Rate & Time-to-Maturity & Yield-to-Maturity \\
\hline
A & $0 \%$ & 2 years & $5.00 \%$ \\
B & $5 \%$ & 2 years & $5.00 \%$ \\
C & $8 \%$ & 2 years & $5.00 \%$ \\
D & $0 \%$ & 4 years & $5.00 \%$ \\
E & $5 \%$ & 4 years & $5.00 \%$ \\
F & $8 \%$ & 4 years & $5.00 \%$ \\
\hline
\end{tabular}
\end{center}

\begin{enumerate}
  \item Based on the relationships between bond prices and bond characteristics, which bond will go up in price the most on a percentage basis if all yields go down from $5.00 \%$ to $4.90 \%$ ?
\end{enumerate}

\section{Solution:}
Bond D will go up in price the most on a percentage basis because it has the lowest coupon rate (the coupon effect) and the longer time-to-maturity (the maturity effect). There is no exception to the maturity effect in these bonds because there are no low-coupon bonds trading at a discount.

\begin{enumerate}
  \setcounter{enumi}{1}
  \item Based on the relationships between the bond prices and bond characteristics, which bond will go down in price the least on a percentage basis if all yields go up from $5.00 \%$ to $5.10 \%$ ?
\end{enumerate}

\section{Solution:}
Bond $\mathrm{C}$ will go down in price the least on a percentage basis because it has the highest coupon rate (the coupon effect) and the shorter time-to-maturity (the maturity effect). There is no exception to the maturity effect because Bonds $\mathrm{C}$ and $\mathrm{F}$ are priced at a premium above par value.

Exhibit 2 demonstrates the impact on a bond price assuming the time-to-maturity does not change. It shows an instantaneous change in the market discount rate from one moment to the next.

In practice, bond prices change even if the market discount rate remains the same. As time passes, the bondholder comes closer to receiving the par value at maturity. The constant-yield price trajectory illustrates the change in the price of a fixed-income bond over time. This trajectory shows the "pull to par" effect on the price of a bond trading at a premium or a discount to par value. If the issuer does not default, the price of a bond approaches par value as its time-to-maturity approaches zero. Exhibit 3 shows the constant-yield price trajectories for $4 \%$ and $12 \%$ annual coupon payment 10-year bonds. Both bonds have a market discount rate of $8 \%$. The $4 \%$ bond's initial price is 73.160 per 100 of par value. The price increases each year and approaches par value as the maturity date nears. The $12 \%$ bond's initial price is 126.840, and it decreases each year, approaching par value as the maturity date nears. Both prices are "pulled to par."

Exhibit 3: Constant-Yield Price Trajectories for $4 \%$ and $12 \%$ Annual Coupon Payment 10-Year Bonds at a Market Discount Rate of $8 \%$

\begin{center}
\includegraphics[max width=\textwidth]{2023_05_04_7b535d0a870224f62e3dg-561(1)}
\end{center}

\includegraphics[max width=\textwidth, center]{2023_05_04_7b535d0a870224f62e3dg-561}
$\begin{array}{lllllllllllll}\text { Premium Bond } & 126.84 & 124.98 & 122.98 & 120.82 & 118.49 & 115.97 & 113.24 & 110.30 & 107.13 & 103.70 & 100.00\end{array}$

\section{Pricing Bonds Using Spot Rates}
When a fixed-rate bond is priced using the market discount rate, the same discount rate is used for each cash flow. A more fundamental approach to calculate the price of a bond is to use a sequence of market discount rates that correspond to the cash flow dates. These market discount rates are called spot rates. Spot rates are yields-to-maturity on zero-coupon bonds maturing at the date of each cash flow. Sometimes these are called "zero rates." Bond price (or value) determined using the spot rates is sometimes referred to as the bond's "no-arbitrage value." If a bond's price differs from its no-arbitrage value, an arbitrage opportunity exists in the absence of transaction costs.

Suppose that the one-year spot rate is $2 \%$, the two-year spot rate is $3 \%$, and the three-year spot rate is $4 \%$. Then, the price of a three-year bond that makes a $5 \%$ annual coupon payment is 102.960 .

$$
\begin{gathered}
\frac{5}{(1.02)^{1}}+\frac{5}{(1.03)^{2}}+\frac{105}{(1.04)^{3}}= \\
4.902+4.713+93.345=102.960 .
\end{gathered}
$$

This three-year bond is priced at a premium above par value, so its yield-to-maturity must be less than 5\%. Using Equation 1, the yield-to-maturity is $3.935 \%$.

$102.960=\frac{5}{(1+r)^{1}}+\frac{5}{(1+r)^{2}}+\frac{105}{(1+r)^{3}} ; r=0.03935$.

When the coupon and principal cash flows are discounted using the yield-to-maturity, the same price is obtained.

$$
\begin{gathered}
\frac{5}{(1.03935)^{1}}+\frac{5}{(1.03935)^{2}}+\frac{105}{(1.03935)^{3}}= \\
4.811+4.629+93.520=102.960 .
\end{gathered}
$$

Notice that the present values of the individual cash flows discounted using spot rates differ from those using the yield-to-maturity. The present value of the first coupon payment is 4.902 when discounted at $2 \%$, but it is 4.811 when discounted at $3.935 \%$. The present value of the final cash flow, which includes the redemption of principal, is 93.345 at $4 \%$ and 93.520 at $3.935 \%$. Nevertheless, the sum of the present values using either approach is 102.960 .

Equation 2 is a general formula for calculating a bond price given the sequence of spot rates:

$$
P V=\frac{P M T}{\left(1+Z_{1}\right)^{1}}+\frac{P M T}{\left(1+Z_{2}\right)^{2}}+\cdots+\frac{P M T+F V}{\left(1+Z_{N}\right)^{N}},
$$

where

$Z_{1}=$ spot rate, or zero-coupon yield, or zero rate, for Period 1

$Z_{2}=$ spot rate, or zero-coupon yield, or zero rate, for Period 2

$Z_{N}=$ spot rate, or zero-coupon yield, or zero rate, for Period $N$

\section{EXAMPLE 4}
\section{Bond Prices and Yields-to-Maturity Based on Spot Rates}
\begin{enumerate}
  \item Calculate the price (per 100 of par value) and the yield-to-maturity for a four-year, $3 \%$ annual coupon payment bond given the following two sequences of spot rates.
\end{enumerate}

\begin{center}
\begin{tabular}{lcc}
\hline
Time-to-Maturity & Spot Rates A & Spot Rates B \\
\hline
1 year & $0.39 \%$ & $4.08 \%$ \\
2 years & $1.40 \%$ & $4.01 \%$ \\
3 years & $2.50 \%$ & $3.70 \%$ \\
4 years & $3.60 \%$ & $3.50 \%$ \\
\hline
\end{tabular}
\end{center}

Solution:

Spot Rates A

$\frac{3}{(1.0039)^{1}}+\frac{3}{(1.0140)^{2}}+\frac{3}{(1.0250)^{3}}+\frac{103}{(1.0360)^{4}}=$

$2.988+2.918+2.786+89.412=98.104$

Given Spot Rates A, the four-year, 3\% bond is priced at 98.104 . $98.104=\frac{3}{(1+r)^{1}}+\frac{3}{(1+r)^{2}}+\frac{3}{(1+r)^{3}}+\frac{103}{(1+r)^{4}} ; r=0.03516$

The yield-to-maturity is $3.516 \%$.

\section{Spot Rates B}
$\frac{3}{(1.0408)^{1}}+\frac{3}{(1.0401)^{2}}+\frac{3}{(1.0370)^{3}}+\frac{103}{(1.0350)^{4}}=$

$2.882+2.773+2.690+89.759=98.104$.

$98.104=\frac{3}{(1+r)^{1}}+\frac{3}{(1+r)^{2}}+\frac{3}{(1+r)^{3}}+\frac{103}{(1+r)^{4}} ; r=0.03516$

Given Spot Rates B, the four-year, 3\% bond is again priced at 98.104 to yield $3.516 \%$.

This example demonstrates that two very different sequences of spot rates can result in the same bond price and yield-to-maturity. Spot Rates A are increasing for longer maturities, whereas Spot Rates B are decreasing.

\section{PRICES AND YIELDS: CONVENTIONS FOR QUOTES AND CALCULATIONS}
\begin{center}
\includegraphics[max width=\textwidth]{2023_05_04_7b535d0a870224f62e3dg-563}
\end{center}

When investors purchase shares, they pay the quoted price. For bonds, however, there can be a difference between the quoted price and the price paid. This section explains why this difference occurs and how to calculate the quoted price and the price that will be paid. It also describes how prices are estimated for bonds that are not actively traded and demonstrates how yield measures are calculated for fixed-rate bonds, floating-rate notes, and money market instruments.

\section{Flat Price, Accrued Interest, and the Full Price}
When a bond is between coupon payment dates, its price has two parts: the flat price $\left(P V^{\text {Flat }}\right)$ and the accrued interest $(A I)$. The sum of the parts is the full price $\left(P V^{\text {Full })}\right.$, which also is called the invoice or "dirty" price. The flat price, which is the full price minus the accrued interest, is also called the quoted or "clean" price.

$$
P V^{F u l l}=P V^{F l a t}+A I
$$

The flat price usually is quoted by bond dealers. If a trade takes place, the accrued interest is added to the flat price to obtain the full price paid by the buyer and received by the seller on the settlement date. The settlement date is when the bond buyer makes cash payment and the seller delivers the security.

The reason for using the flat price for quotation is to avoid misleading investors about the market price trend for the bond. If the full price were to be quoted by dealers, investors would see the price rise day after day even if the yield-to-maturity did not change. That is because the amount of accrued interest increases each day. Then, after the coupon payment is made, the quoted price would drop dramatically. Using the flat price for quotation avoids that misrepresentation. It is the flat price that is "pulled to par" along the constant-yield price trajectory shown in Exhibit 3.

Accrued interest is the proportional share of the next coupon payment. Assume that the coupon period has " $T$ " days between payment dates and that " $t$ " days have gone by since the last payment. The accrued interest is calculated using Equation 4:

$$
A I=\frac{t}{T} \times P M T
$$

where

$$
\begin{aligned}
t & =\text { number of days from the last coupon payment to the settlement date } \\
T & =\text { number of days in the coupon period } \\
t / T & =\text { fraction of the coupon period that has gone by since the last payment } \\
P M T & =\text { coupon payment per period }
\end{aligned}
$$

Notice that the accrued interest part of the full price does not depend on the yield-to-maturity. Therefore, it is the flat price that is affected by a market discount rate change.

There are different conventions used in bond markets to count days. The two most common day-count conventions are actual/actual and 30/360. For the actual/ actual method, the actual number of days is used, including weekends, holidays, and leap days. For example, a semiannual payment bond pays interest on 15 May and 15 November of each year. The accrued interest for settlement on 27 June would be the actual number of days between 15 May and 27 June ( $t=43$ days) divided by the actual number of days between 15 May and 15 November ( $T=184$ days), times the coupon payment. If the stated coupon rate is $4.375 \%$, the accrued interest is 0.511209 per 100 of par value.

$$
A I=\frac{43}{184} \times \frac{4.375}{2}=0.511209
$$

Day-count conventions vary from market to market. However, actual/actual is most common for government bonds.

The 30/360 day-count convention often is used for corporate bonds. It assumes that each month has 30 days and that a full year has 360 days. Therefore, for this method, there are assumed to be 42 days between 15 May and 27 June: 15 days between 15 May and 30 May and 27 days between 1 June and 27 June. There are assumed to be 180 days in the six-month period between 15 May and 15 November. The accrued interest on a $4.375 \%$ semiannual payment corporate bond is 0.510417 per 100 of par value.

$$
A I=\frac{42}{180} \times \frac{4.375}{2}=0.510417
$$

The full price of a fixed-rate bond between coupon payments given the market discount rate per period $(r)$ can be calculated with Equation 5:

$$
P V^{F u l l}=\frac{P M T}{(1+r)^{1-t / T}}+\frac{P M T}{(1+r)^{2-t / T}}+\cdots+\frac{P M T+F V}{(1+r)^{N-t / T}}
$$

This is very similar to Equation 1. The difference is that the next coupon payment $(P M T)$ is discounted for the remainder of the coupon period, which is $1-t / T$. The second coupon payment is discounted for that fraction plus another full period, $2-t / T$.

Equation 5 is simplified by multiplying the numerator and denominator by the expression $(1+r)^{t / T}$. The result is Equation 6:

$$
\begin{aligned}
& P V^{F u l l}=\left[\frac{P M T}{(1+r)^{1}}+\frac{P M T}{(1+r)^{2}}+\cdots+\frac{P M T+F V}{(1+r)^{N}}\right] \times(1+r)^{t / T} \\
& =P V \times(1+r)^{t / T} .
\end{aligned}
$$

An advantage to Equation 6 is that $P V$, the expression in the brackets, is easily obtained using the time-value-of-money keys on a financial calculator because there are $N$ evenly spaced periods. $P V$ here is identical to Equation 1 and is not the same as $P V^{\text {Flat }}$

For example, consider a $5 \%$ semiannual coupon payment government bond that matures on 15 February 2028. Accrued interest on this bond uses the actual/actual day-count convention. The coupon payments are made on 15 February and 15 August of each year. The bond is to be priced for settlement on 14 May 2019. That date is 88 days into the 181 -day period. There are actually 88 days from the last coupon on 15 February to 14 May and 181 days between 15 February and the next coupon on 15 August. The annual yield-to-maturity is stated to be $4.80 \%$. That corresponds to a market discount rate of $2.40 \%$ per semiannual period. As of the beginning of the coupon period on 15 February 2019, there would be 18 evenly spaced semiannual periods until maturity. The first step is to solve for $P V$ using Equation 1, whereby $P M T=2.5, N=18, F V=100$, and $r=0.0240$.

$$
P V=\frac{2.5}{(1.0240)^{1}}+\frac{2.5}{(1.0240)^{2}}+\cdots+\frac{102.5}{(1.0240)^{18}}=101.447790
$$

The price of the bond would be 101.447790 per 100 of par value if its yield-to-maturity is $2.40 \%$ per period on the last coupon payment date. This is not the actual price for the bond on that date. It is a "what-if" price using the required yield that corresponds to the settlement date of 14 May 2019.

Equation 6 can be used to get the full price for the bond.

$P V^{F u l l}=101.447790 \times(1.0240)^{88 / 181}=102.624323$

The full price is 102.624323 per 100 of par value. The accrued interest is 1.215470 per 100 of par value.

$$
A I=\frac{88}{181} \times 2.5=1.215470
$$

The flat price is 101.408853 per 100 of par value.

$P V^{\text {Flat }}=P V^{\text {Full }}-A I=102.624323-1.215470=101.408853$.

Microsoft Excel users can obtain the flat price using the PRICE financial function: PRICE(DATE(2019,5,14),DATE(2028,2,15),0.05,0.048,100,2,1). The inputs are the settlement date, maturity date, annual coupon rate as a decimal, annual yield-to-maturity as a decimal, par value, number of periods in the year, and code for the day count (0 for 30/360, 1 for actual/actual).

\section{EXAMPLE 5}
\section{Calculating the Full Price, Accrued Interest, and Flat Price for a Bond}
\begin{enumerate}
  \item A 6\% German corporate bond is priced for settlement on 18 June 2019. The bond makes semiannual coupon payments on 19 March and 19 September of each year and matures on 19 September 2030. The corporate bond uses the 30/360 day-count convention for accrued interest. Calculate the full price, the accrued interest, and the flat price per EUR100 of par value for three stated annual yields-to-maturity: (A) $5.80 \%$, (B) $6.00 \%$, and (C) $6.20 \%$.
\end{enumerate}

\section{Solution:}
Given the 30/360 day-count convention assumption, there are 89 days between the last coupon on 19 March 2015 and the settlement date on 18 June 2019 (11 days between 19 March and 30 March, plus 60 days for the full months of April and May, plus 18 days in June). Therefore, the fraction of the coupon period that has gone by is assumed to be 89/180. At the beginning of the period, there are 11.5 years (and 23 semiannual periods) to maturity.

\section{(A) Stated annual yield-to-maturity of $5.80 \%$, or $2.90 \%$ per semiannual period:}
The price at the beginning of the period is 101.661589 per 100 of par value.

$$
P V=\frac{3}{(1.0290)^{1}}+\frac{3}{(1.0290)^{2}}+\cdots+\frac{103}{(1.0290)^{23}}=101.661589 .
$$

The full price on 18 June is EUR103.108770.

$$
P V^{\text {Full }}=101.661589 \times(1.0290)^{89 / 180}=103.108770
$$

The accrued interest is EUR1.483333, and the flat price is EUR101.625437.

$$
\begin{aligned}
& A I=\frac{89}{180} \times 3=1.4833333 \\
& P V^{\text {Flat }}=103.108770-1.483333=101.625437
\end{aligned}
$$

(B) Stated annual yield-to-maturity of $6.00 \%$, or $3.00 \%$ per semiannual period:

The price at the beginning of the period is par value, as expected, because the coupon rate and the market discount rate are equal.

$$
P V=\frac{3}{(1.0300)^{1}}+\frac{3}{(1.0300)^{2}}+\cdots+\frac{103}{(1.0300)^{23}}=100.000000 .
$$

The full price on 18 June is EUR101.472251.

$$
P V^{F u l l}=100.000000 \times(1.0300)^{89 / 180}=101.472251
$$

The accrued interest is EUR1.483333, and the flat price is EUR99.988918.

$$
\begin{aligned}
& A I=\frac{89}{180} \times 3=1.4833333 \\
& P V^{\text {Flat }}=101.472251-1.483333=99.988918 .
\end{aligned}
$$

The flat price of the bond is a little below par value, even though the coupon rate and the yield-to-maturity are equal, because the accrued interest does not take into account the time value of money. The accrued interest is the interest earned by the owner of the bond for the time between the last coupon payment and the settlement date, 1.483333 per 100 of par value. However, that interest income is not received until the next coupon date. In theory, the accrued interest should be the present value of 1.483333 . In practice, however, accounting and financial reporting need to consider issues of practicality and materiality. For those reasons, the calculation of accrued interest in practice neglects the time value of money. Therefore, compared with theory, the reported accrued interest is a little "too high" and the flat price is a little "too low." The full price, however, is correct because it is the sum of the present values of the future cash flows, discounted using the market discount rate.

(C) Stated annual yield-to-maturity of $6.20 \%$, or $3.10 \%$ per semiannual period:

The price at the beginning of the period is 98.372607 per 100 of par value.

$P V=\frac{3}{(1.0310)^{1}}+\frac{3}{(1.0310)^{2}}+\cdots+\frac{103}{(1.0310)^{23}}=98.372607$

The full price on 18 June is EUR99.868805.

$P V^{\text {Full }}=98.372607 \times(1.0310)^{89 / 180}=99.868805$

The accrued interest is EUR1.483333, and the flat price is EUR98.385472.

$$
\begin{aligned}
& A I=\frac{89}{180} \times 3=1.4833333 \\
& P V^{\text {Flat }}=99.868805-1.483333=98.385472
\end{aligned}
$$

The accrued interest is the same in each case because it does not depend on the yield-to-maturity. The differences in the flat prices indicate the differences in the rate of return that is required by investors.

\section{Matrix Pricing}
Some fixed-rate bonds are not actively traded. Therefore, there is no market price available to calculate the rate of return required by investors. The same problem occurs for bonds that are not yet issued. In these situations, it is common to estimate the market discount rate and price based on the quoted or flat prices of more frequently traded comparable bonds. These comparable bonds have similar times-to-maturity, coupon rates, and credit quality. This estimation process is called matrix pricing.

For example, suppose that an analyst needs to value a three-year, $4 \%$ semiannual coupon payment corporate bond, Bond X. Assume that Bond $\mathrm{X}$ is not actively traded and that there are no recent transactions reported for this particular security. However, there are quoted prices for four corporate bonds that have very similar credit quality:

\begin{itemize}
  \item Bond A: Two-year, 3\% semiannual coupon payment bond trading at a price of 98.500

  \item Bond B: Two-year, $5 \%$ semiannual coupon payment bond trading at a price of 102.250

  \item Bond C: Five-year, $2 \%$ semiannual coupon payment bond trading at a price of 90.250

  \item Bond D: Five-year, 4\% semiannual coupon payment bond trading at a price of 99.125 The bonds are displayed in a matrix according to the coupon rate and the time-to-maturity. This matrix is shown in Exhibit 4.

\end{itemize}

\section{Exhibit 4: Matrix Pricing Example}
\begin{center}
\begin{tabular}{lcccc}
\hline
 & $\begin{array}{c}\mathbf{2 \%} \\ \text { Coupon }\end{array}$ & $\begin{array}{c}\mathbf{3 \%} \\ \text { Coupon }\end{array}$ & $\begin{array}{c}\mathbf{4 \%} \\ \text { Coupon }\end{array}$ & $\begin{array}{c}\mathbf{5 \%} \\ \text { Coupon }\end{array}$ \\
\hline
Two Years &  & 98.500 &  & 102.250 \\
 &  & $3.786 \%$ &  & $3.821 \%$ \\
Three Years &  & Bond X &  &  \\
Four Years &  &  &  &  \\
Five Years & 90.250 &  & 49.125 &  \\
 & $4.181 \%$ &  & $4.196 \%$ &  \\
\hline
\end{tabular}
\end{center}

In Exhibit 4, below each bond price is the yield-to-maturity. It is stated as the yield per semiannual period times two. For example, the yield-to-maturity on the two-year, $3 \%$ semiannual coupon payment corporate bond is $3.786 \%$.

$$
98.500=\frac{1.5}{(1+r)^{1}}+\frac{1.5}{(1+r)^{2}}+\frac{1.5}{(1+r)^{3}}+\frac{101.5}{(1+r)^{4}} ; r=0.01893 ; \times 2=0.03786 .
$$

Next, the analyst calculates the average yield for each year: $3.8035 \%$ for the two-year bonds and $4.1885 \%$ for the five-year bonds.

$$
\begin{aligned}
& \frac{0.03786+0.03821}{2}=0.038035 . \\
& \frac{0.04181+0.04196}{2}=0.041885 .
\end{aligned}
$$

The estimated three-year market discount rate can be obtained with linear interpolation. The interpolated yield is $3.9318 \%$.

$$
0.038035+\left(\frac{3-2}{5-2}\right) \times(0.041885-0.038035)=0.039318
$$

Using 3.9318\% as the estimated three-year annual market discount rate, the three-year, $4 \%$ semiannual coupon payment corporate bond has an estimated price of 100.191 per 100 of par value.

$$
\begin{aligned}
& \frac{2}{(1.019659)^{1}}+\frac{2}{(1.019659)^{2}}+\frac{2}{(1.019659)^{3}}+\frac{2}{(1.019659)^{4}}+\frac{2}{(1.019659)^{5}}+ \\
& \frac{102}{(1.019659)^{6}}=100.191 .
\end{aligned}
$$

Notice that $3.9318 \%$ is the stated annual rate. It is divided by two to get the yield per semiannual period: $(0.039318 / 2=0.019659)$.

Matrix pricing is also used in underwriting new bonds to get an estimate of the required yield spread over the benchmark rate. The benchmark rate typically is the yield-to-maturity on a government bond having the same or close to the same time-to-maturity. The spread is the difference between the yield-to-maturity on the new bond and the benchmark rate. The yield spread is the additional compensation required by investors for the difference in the credit risk, liquidity risk, tax status, and other risk premiums of the bond relative to the government bond. This spread is sometimes called the spread over the benchmark. Yield spreads are often stated in terms of basis points (bps), where one basis point equals one-hundredth of a percentage point. For example, if a yield-to-maturity is $2.25 \%$ and the benchmark rate is $1.50 \%$, the yield spread is $0.75 \%$, or 75 bps. Yield spreads are covered in more detail later. Suppose that a corporation is about to issue a five-year bond. The corporate issuer currently has a four-year, $3 \%$ annual coupon payment debt liability on its books. The price of that bond is 102.400 per 100 of par value. This is the full price, which is the same as the flat price because the accrued interest is zero. This implies that the coupon payment has just been made and there are four full years to maturity. The four-year rate of return required by investors for this bond is $2.36 \%$.

$$
102.400=\frac{3}{(1+r)^{1}}+\frac{3}{(1+r)^{2}}+\frac{3}{(1+r)^{3}}+\frac{103}{(1+r)^{4}} ; r=0.0236 .
$$

Suppose that there are no four-year government bonds to calculate the yield spread on this security. However, there are three-year and five-year government bonds that have yields-to-maturity of $0.75 \%$ and $1.45 \%$, respectively. The average of the two yields-to-maturity is $1.10 \%$, which is the estimated yield for the four-year government bond. Therefore, the estimated yield spread is $126 \mathrm{bps}$ over the implied benchmark rate $(0.0236-0.0110=0.0126)$.

There usually is a different yield spread for each maturity and for each credit rating. The term structure of "risk-free" rates, which is further discussed later, is the relationship between yields-to-maturity on "risk-free" bonds and times-to-maturity. The quotation marks around "risk-free" indicate that no bond is truly without risk. The primary component of the yield spread for many bonds is compensation for credit risk, not for time-to-maturity, and as a result, the yield spreads reflect the term structure of credit spreads. The term structure of credit spreads is the relationship between the spreads over the "risk-free" (or benchmark) rates and times-to-maturity. These term structures are covered in more detail later.

The issuer now has an estimate of the four-year yield spread: $126 \mathrm{bps}$. This spread is a reference point for estimating the five-year spread for the newly issued bond. Suppose that the term structure of credit spreads for bonds of the corporate issuer's quality indicates that five-year spreads are about $25 \mathrm{bps}$ higher than four-year spreads. Therefore, the estimated five-year required yield spread is 151 bps $(0.0126+0.0025=$ 0.0151 ). Given the yield-to-maturity of $1.45 \%$ on the five-year government bond, the expected market discount rate for the newly issued bond is $2.96 \%(0.0145+0.0151$ $=0.0296$ ). The corporation might set the coupon rate to be $3 \%$ and expect that the bond can be sold for a small premium above par value.

\section{EXAMPLE 6}
\section{Using Matrix Pricing to Estimate Bond Price}
\begin{enumerate}
  \item An analyst needs to assign a value to an illiquid four-year, $4.5 \%$ annual coupon payment corporate bond. The analyst identifies two corporate bonds that have similar credit quality: One is a three-year, $5.50 \%$ annual coupon payment bond priced at 107.500 per 100 of par value, and the other is a five-year, $4.50 \%$ annual coupon payment bond priced at 104.750 per 100 of par value. Using matrix pricing, the estimated price of the illiquid bond per 100 of par value is closest to:
A. 103.895 .
B. 104.991 .
C. 106.125 .
\end{enumerate}

\section{Solution:}
B is correct. The first step is to determine the yields-to-maturity on the observed bonds. The required yield on the three-year, $5.50 \%$ bond priced at 107.500 is $2.856 \%$. $107.500=\frac{5.50}{(1+r)^{1}}+\frac{5.50}{(1+r)^{2}}+\frac{105.50}{(1+r)^{3}} ; r=0.02856$

The required yield on the five-year, $4.50 \%$ bond priced at 104.750 is $3.449 \%$.

$$
104.750=\frac{4.50}{(1+r)^{1}}+\frac{4.50}{(1+r)^{2}}+\frac{4.50}{(1+r)^{3}}+\frac{4.50}{(1+r)^{4}}+\frac{104.50}{(1+r)^{5}} ; r=0.03449
$$

The estimated market discount rate for a four-year bond having the same credit quality is the average of two required yields:

$$
\frac{0.02856+0.03449}{2}=0.031525
$$

Given an estimated yield-to-maturity of $3.1525 \%$, the estimated price of the illiquid four-year, $4.50 \%$ annual coupon payment corporate bond is 104.991 per 100 of par value.

$$
\frac{4.50}{(1.031525)^{1}}+\frac{4.50}{(1.031525)^{2}}+\frac{4.50}{(1.031525)^{3}}+\frac{104.50}{(1.031525)^{4}}=104.991 \text {. }
$$

\section{Annual Yields for Varying Compounding Periods in the Year}
There are many ways to measure the rate of return on a fixed-rate bond investment. Consider a five-year, zero-coupon government bond. The purchase price today is 80 . The investor receives 100 at redemption in five years. One possible yield measure is $25 \%$-the gain of 20 divided by the amount invested, 80 . However, investors want a yield measure that is standardized to allow for comparison between bonds that have different times-to-maturity. Therefore, yield measures typically are annualized. A possible annual rate for this zero-coupon bond is $5 \%$ per year-25\% divided by five years. But for bonds maturing in more than one year, investors want an annualized and compounded yield-to-maturity. Money market rates on instruments maturing in one year or less typically are annualized but not compounded. They are stated on a simple interest basis. This concept is covered later.

In general, an annualized and compounded yield on a fixed-rate bond depends on the assumed number of periods in the year, which is called the periodicity of the annual rate. Typically, the periodicity matches the frequency of coupon payments. A bond that pays semiannual coupons has a stated annual yield-to-maturity for a periodicity of two-the rate per semiannual period times two. A bond that pays quarterly coupons has a stated annual yield for a periodicity of four-the rate per quarter times four. It is always important to know the periodicity of a stated annual rate.

The periodicity of the annual market discount rate for a zero-coupon bond is arbitrary because there are no coupon payments. For semiannual compounding, the annual yield-to-maturity on the five-year, zero-coupon bond priced at 80 per 100 of par value is stated to be $4.5130 \%$. This annual rate has a periodicity of two.

$$
80=\frac{100}{(1+r)^{10}} ; r=0.022565 ; \times 2=0.045130
$$

For quarterly compounding, the annual yield-to-maturity is stated to be $4.4880 \%$. This annual rate has a periodicity of four.

$$
80=\frac{100}{(1+r)^{20}} ; r=0.011220 ; \times 4=0.044880
$$

For monthly compounding, the annual yield-to-maturity is stated to be $4.4712 \%$. This annual rate has a periodicity of 12 .

$$
80=\frac{100}{(1+r)^{60}} ; r=0.003726 ; \times 12=0.044712 .
$$

For annual compounding, the yield-to-maturity is stated to be $4.5640 \%$. This annual rate has a periodicity of one.

$80=\frac{100}{(1+r)^{5}} ; r=0.045640 ; \times 1=0.045640$

This is known as an effective annual rate. An effective annual rate has a periodicity of one because there is just one compounding period in the year.

In this zero-coupon bond example, 2.2565\% compounded two times a year, $1.1220 \%$ compounded four times a year, and $0.3726 \%$ compounded 12 times a year are all equivalent to an effective annual rate of $4.5640 \%$. The compounded total return is the same for each expression for the annual rate. They differ in terms of the number of compounding periods per year-that is, in terms of the periodicity of the annual rate. For a given pair of cash flows, the stated annual rate and the periodicity are inversely related.

The most common periodicity for US dollar-denominated bond yields is two because most bonds in the US dollar market make semiannual coupon payments. An annual rate having a periodicity of two is known as a semiannual bond basis yield, or semiannual bond equivalent yield. Therefore, a semiannual bond basis yield is the yield per semiannual period times two. It is important to remember that "semiannual bond basis yield" and "yield per semiannual period" have different meanings. For example, if a bond yield is $2 \%$ per semiannual period, its annual yield is $4 \%$ when stated on a semiannual bond basis.

An important tool used in fixed-income analysis is to convert an annual yield from one periodicity to another. These are called periodicity, or compounding, conversions. A general formula to convert an annual percentage rate for $m$ periods per year, denoted $A P R_{m}$, to an annual percentage rate for $n$ periods per year, $A P R_{n}$, is Equation 7:

$$
\left(1+\frac{A P R_{m}}{m}\right)^{m}=\left(1+\frac{A P R_{n}}{n}\right)^{n} .
$$

For example, suppose that a three-year, 5\% semiannual coupon payment corporate bond is priced at 104 per 100 of par value. Its yield-to-maturity is $3.582 \%$, quoted on a semiannual bond basis for a periodicity of two: $0.01791 \times 2=0.03582$.

$$
104=\frac{2.5}{(1+r)^{1}}+\frac{2.5}{(1+r)^{2}}+\frac{2.5}{(1+r)^{3}}+\frac{2.5}{(1+r)^{4}}+\frac{2.5}{(1+r)^{5}}+\frac{102.5}{(1+r)^{6}} ; r=0.01791
$$

To compare this bond with others, an analyst converts this annualized yield-to-maturity to quarterly and monthly compounding. Doing so entails using Equation 7 to convert from a periodicity of $m=2$ to periodicities of $n=4$ and $n=12$.

$$
\begin{aligned}
& \left(1+\frac{0.03582}{2}\right)^{2}=\left(1+\frac{A P R_{4}}{4}\right)^{4} ; A P R_{4}=0.03566 \\
& \left(1+\frac{0.03582}{2}\right)^{2}=\left(1+\frac{A P R_{12}}{12}\right)^{12} ; A P R_{12}=0.03556 .
\end{aligned}
$$

An annual yield-to-maturity of $3.582 \%$ for semiannual compounding provides the same rate of return as annual yields of $3.566 \%$ and $3.556 \%$ for quarterly and monthly compounding, respectively. A general rule for these periodicity conversions is compounding more frequently at a lower annual rate corresponds to compounding less frequently at a higher annual rate. This rule can be used to check periodicity conversion calculations.

Equation 7 also applies to negative bond yields. Government bond yields have been negative in several countries, including Switzerland, Germany, Sweden, and Japan. For a simple example, consider a five-year, zero-coupon bond priced at 105 (\% of par value). Its yield-to-maturity is $-0.971 \%$ stated as an effective annual rate for a periodicity of 1 .

$$
105=\frac{100}{(1+r)^{5}} ; r=-0.00971
$$

Converting that to semiannual and monthly compounding for periodicities of 2 and 12 results in annual yields of $-0.973 \%$ and $-0.975 \%$, respectively.

$$
\begin{aligned}
& \left(1+\frac{-0.00971}{1}\right)^{1}=\left(1+\frac{A P R_{2}}{2}\right)^{2} ; A P R_{2}=-0.00973 . \\
& \left(1+\frac{-0.00971}{1}\right)^{1}=\left(1+\frac{A P R_{12}}{12}\right)^{12} ; A P R_{12}=-0.00975 .
\end{aligned}
$$

Compounding more frequently within the year results in a lower (more negative) yield-to-maturity.

\section{EXAMPLE 7}
\section{Yield Conversion Based on Periodicity}
\begin{enumerate}
  \item A five-year, $4.50 \%$ semiannual coupon payment government bond is priced at 98 per 100 of par value. Calculate the annual yield-to-maturity stated on a semiannual bond basis, rounded to the nearest basis point. Convert that annual yield to
\end{enumerate}

A. an annual rate that can be used for direct comparison with otherwise comparable bonds that make quarterly coupon payments and

B. an annual rate that can be used for direct comparison with otherwise comparable bonds that make annual coupon payments.

\section{Solution:}
The stated annual yield-to-maturity on a semiannual bond basis is $4.96 \%$ $(0.0248 \times 2=0.0496)$.

$$
\begin{gathered}
98=\frac{2.25}{(1+r)^{1}}+\frac{2.25}{(1+r)^{2}}+\frac{2.25}{(1+r)^{3}}+\frac{2.25}{(1+r)^{4}}+\frac{2.25}{(1+r)^{5}}+\frac{2.25}{(1+r)^{6}}+ \\
\frac{2.25}{(1+r)^{7}}+\frac{2.25}{(1+r)^{8}}+\frac{2.25}{(1+r)^{9}}+\frac{102.25}{(1+r)^{10}} ; r=0.0248 .
\end{gathered}
$$

A. Convert $4.96 \%$ from a periodicity of two to a periodicity of four:

$$
\left(1+\frac{0.0496}{2}\right)^{2}=\left(1+\frac{A P R_{4}}{4}\right)^{4} ; A P R_{4}=0.0493 \text {. }
$$

The annual percentage rate of $4.96 \%$ for compounding semiannually compares with $4.93 \%$ for compounding quarterly. That makes sense because increasing the frequency of compounding lowers the annual rate.

B. Convert $4.96 \%$ from a periodicity of two to a periodicity of one:

$\left(1+\frac{0.0496}{2}\right)^{2}=\left(1+\frac{A P R_{1}}{1}\right)^{1} ; A P R_{1}=0.0502$.

The annual rate of $4.96 \%$ for compounding semiannually compares with an effective annual rate of $5.02 \%$. Converting from more frequent to less frequent compounding entails raising the annual percentage rate.

\section{Yield Measures for Fixed-Rate Bonds}
An important concern for quoting and calculating bond yields-to-maturity is the actual timing of the cash flows. Consider a $6 \%$ semiannual payment corporate bond that matures on 15 March 2028. Suppose that for settlement on 23 January 2020, the bond is priced at 98.5 per 100 of par value to yield $6.236 \%$ quoted on a semiannual bond basis. Its coupon payments are scheduled for 15 March and 15 September of each year. The yield calculation implicitly assumes that the payments are made on those dates. It neglects the reality that 15 March 2020 is a Sunday and 15 March 2025 is a Saturday. In fact, the coupon payments will be made to investors on the following Monday.

Yield measures that neglect weekends and holidays are quoted on what is called street convention. The street convention yield-to-maturity is the internal rate of return on the cash flows assuming the payments are made on the scheduled dates. This assumption simplifies bond price and yield calculations and commonly is used in practice. Sometimes the true yield is also quoted. The true yield-to-maturity is the internal rate of return on the cash flows using the actual calendar of weekends and bank holidays. The true yield is never higher than the street convention yield because weekends and holidays delay the time to payment. The difference is typically small, no more than a basis point or two. Therefore, the true yield is not commonly used in practice. Sometimes, a government equivalent yield is quoted for a corporate bond. A government equivalent yield restates a yield-to-maturity based on a 30/360 day count to one based on actual/actual. The government equivalent yield on a corporate bond can be used to obtain the spread over the government yield. Doing so keeps the yields stated on the same day-count convention basis.

Another yield measure that is commonly quoted for fixed-income bonds is the current yield, also called the income or running yield. The current yield is the sum of the coupon payments received over the year divided by the flat price. For example, a 10-year, 2\% semiannual coupon payment bond is priced at 95 per 100 of par value. Its current yield is $2.105 \%$.

$$
\frac{2}{95}=0.02105
$$

The current yield is a crude measure of the rate of return to an investor because it neglects the frequency of coupon payments in the numerator and any accrued interest in the denominator. It focuses only on interest income. In addition to collecting and reinvesting coupon payments, the investor has a gain if the bond is purchased at a discount and is redeemed at par value. The investor has a loss if the bond is purchased at a premium and is redeemed at par value. Sometimes the simple yield on a bond is quoted. It is the sum of the coupon payments plus the straight-line amortized share of the gain or loss, divided by the flat price. Simple yields are used mostly to quote Japanese government bonds, known as "JGBs."

\section{EXAMPLE 8}
\section{Comparing Yields for Different Periodicities}
An analyst observes these reported statistics for two bonds.

\begin{center}
\begin{tabular}{lcc}
\hline
 & Bond A & Bond B \\
\hline
Annual Coupon Rate & $8.00 \%$ & $12.00 \%$ \\
Coupon Payment Frequency & Semiannually & Quarterly \\
Years to Maturity & 5 Years & 5 Years \\
Price (per 100 of par value) & 90 & 105 \\
\end{tabular}
\end{center}

\begin{center}
\begin{tabular}{lcl}
\hline
 & Bond A & Bond B \\
\hline
Current Yield & $8.889 \%$ & $11.429 \%$ \\
Yield-to-Maturity & $10.630 \%$ & $10.696 \%$ \\
\hline
\end{tabular}
\end{center}

\begin{enumerate}
  \item Confirm the calculation of the two yield measures for the two bonds.
\end{enumerate}

\section{Solution:}
Current yield for Bond A:

$\frac{8}{90}=0.08889$

Yield-to-maturity for Bond A:

$90=\frac{4}{(1+r)^{1}}+\frac{4}{(1+r)^{2}}+\cdots+\frac{104}{(1+r)^{10}} ; r=0.05315 ; \times 2=0.10630$.

Current yield for Bond B:

$\frac{12}{105}=0.11429$

Yield-to-maturity for Bond B:

$105=\frac{3}{(1+r)^{1}}+\frac{3}{(1+r)^{2}}+\cdots+\frac{103}{(1+r)^{20}} ; r=0.02674 ; \times 4=0.10696$.

\begin{enumerate}
  \setcounter{enumi}{1}
  \item The analyst believes that Bond $B$ has a little more risk than Bond A. How much additional compensation, in terms of a higher yield-to-maturity, does a buyer of Bond $\mathrm{B}$ receive for bearing this risk compared with Bond $\mathrm{A}$ ?
\end{enumerate}

\section{Solution:}
The yield-to-maturity on Bond $\mathrm{A}$ of $10.630 \%$ is an annual rate for compounding semiannually. The yield-to-maturity on Bond B of $10.696 \%$ is an annual rate for compounding quarterly. The difference in the yields is not 6.6 bps $(0.10696-0.10630=0.00066)$. It is essential to compare the yields for the same periodicity to make a statement about relative value.

A yield-to-maturity of $10.630 \%$ for a periodicity of two converts to $10.492 \%$ for a periodicity of four:

$$
\left(1+\frac{0.10630}{2}\right)^{2}=\left(1+\frac{A P R_{4}}{4}\right)^{4} ; A P R_{4}=0.10492
$$

A yield-to-maturity of $10.696 \%$ for a periodicity of four converts to $10.839 \%$ for a periodicity of two:

$$
\left(1+\frac{0.10696}{4}\right)^{4}=\left(1+\frac{A P R_{2}}{2}\right)^{2} ; A P R_{2}=0.10839
$$

The additional compensation for the greater risk in Bond B is 20.9 bps $(0.10839-0.10630=0.00209)$ when the yields are stated on a semiannual bond basis. The additional compensation is $20.4 \mathrm{bps}(0.10696-0.10492=$ 0.00204 ) when both are annualized for quarterly compounding.

If a fixed-rate bond contains an embedded option, other yield measures are used. An embedded option is part of the security and cannot be removed and sold separately. For example, a callable bond contains an embedded call option that gives the issuer the right to buy the bond back from the investor at specified prices on pre-determined dates. The preset dates usually coincide with coupon payment dates after a call protection period. A call protection period is the time during which the issuer of the bond is not allowed to exercise the call option.

Suppose that a seven-year, $8 \%$ annual coupon payment bond is first callable in four years. That gives the investor four years of protection against the bond being called. After the call protection period, the issuer might exercise the call option if interest rates decrease or the issuer's credit quality improves. Those circumstances allow the issuer to refinance the debt at a lower cost of funds. The preset prices that the issuer pays if the bond is called often are at a premium above par. For example, the "call schedule" for this bond might be that it is callable first at 102 (per 100 of par value) on the coupon payment date in four years, at 101 in five years, and at par value on coupon payment dates thereafter.

The yield-to-maturity on this seven-year, $8 \%$ callable bond is just one of several traditional yield measures for the investment. Others are yield-to-first-call, yield-to-second-call, and so on. If the current price for the bond is 105 per 100 of par value, the yield-to-first-call in four years is $6.975 \%$.

$105=\frac{8}{(1+r)^{1}}+\frac{8}{(1+r)^{2}}+\frac{8}{(1+r)^{3}}+\frac{8+102}{(1+r)^{4}}, \quad r=0.06975$

The yield-to-second-call in five years is $6.956 \%$.

$105=\frac{8}{(1+r)^{1}}+\frac{8}{(1+r)^{2}}+\frac{8}{(1+r)^{3}}+\frac{8}{(1+r)^{4}}+\frac{8+101}{(1+r)^{5}} ; r=0.06956$.

The yield-to-third-call is $6.953 \%$.

$105=\frac{8}{(1+r)^{1}}+\frac{8}{(1+r)^{2}}+\frac{8}{(1+r)^{3}}+\frac{8}{(1+r)^{4}}+\frac{8}{(1+r)^{5}}+\frac{8+100}{(1+r)^{6}} ; r=0.06953$

Finally, the yield-to-maturity is $7.070 \%$.

$$
\begin{aligned}
& 105=\frac{8}{(1+r)^{1}}+\frac{8}{(1+r)^{2}}+\frac{8}{(1+r)^{3}}+\frac{8}{(1+r)^{4}}+\frac{8}{(1+r)^{5}}+\frac{8}{(1+r)^{6}}+ \\
& \frac{8+100}{(1+r)^{7}} ; r=0.07070 .
\end{aligned}
$$

Each calculation is based on Equation 1, whereby the call price (or par value) is used for $F V$. The lowest of the sequence of yields-to-call and the yield-to-maturity is known as the yield-to-worst. In this case, it is the yield-to-third-call of $6.953 \%$. The intent of this yield measure is to provide to the investor the most conservative assumption for the rate of return.

The yield-to-worst is a commonly cited yield measure for fixed-rate callable bonds used by bond dealers and investors. However, a more precise approach is to use an option pricing model and an assumption about future interest rate volatility to value the embedded call option. The value of the embedded call option is added to the flat price of the bond to get the option-adjusted price. The investor bears the call risk (the bond issuer has the option to call), so the embedded call option reduces the value of the bond from the investor's perspective. The investor pays a lower price for the callable bond than if it were option-free. If the bond were non-callable, its price would be higher. The option-adjusted price is used to calculate the option-adjusted yield. The option-adjusted yield is the required market discount rate whereby the price is adjusted for the value of the embedded option. The value of the call option is the price of the option-free bond minus the price of the callable bond.

\section{Yield Measures for Floating-Rate Notes}
Floating-rate notes are very different from fixed-rate bonds. The interest payments on a floating-rate note, which often is called a floater or an FRN, are not fixed. Instead, they vary from period to period depending on the current level of a reference interest rate. The interest payments could go up or down; that is why they are called "floaters." The intent of an FRN is to offer the investor a security that has less market price risk than a fixed-rate bond when market interest rates fluctuate. In principle, a floater has a stable price even in a period of volatile interest rates. With a traditional fixed-income security, interest rate volatility affects the price because the future cash flows are constant. With a floating-rate note, interest rate volatility affects future interest payments.

The reference rate on a floating-rate note usually is a short-term money market rate. The three-month Libor (London Interbank Offered Rate) had long been the reference rate for many floating-rate notes. As discussed earlier, Libor is being phased out. We refer to the new rate as the Market Reference Rate (MRR). The principal on the floater typically is non-amortizing and is redeemed in full at maturity. As of June 2020, the reference rate is determined at the beginning of the period, and the interest payment is made at the end of the period. This payment structure is called "in arrears." The most common day-count conventions for calculating accrued interest on floaters are actual $/ 360$ and actual $/ 365$.

Although there are many varieties of FRNs, only the most common and traditional floaters are covered here. On these floaters, a specified yield spread is added to or subtracted from the reference rate. For example, the floater might reset its interest rate quarterly at MRR plus $0.50 \%$. This specified yield spread over the reference rate is called the quoted margin on the FRN. The role of the quoted margin is to compensate the investor for the difference in the credit risk of the issuer and that implied by the reference rate. For example, a company with a stronger credit rating than that of the banks or financial institutions from which MRR is computed may be able to obtain a "sub-MRR" cost of borrowed funds, which results in a negative quoted margin. An AAA rated company might be able to issue an FRN that pays MRR minus $0.25 \%$.

The required margin is the yield spread over or under the reference rate such that the FRN is priced at par value on a rate reset date. Suppose that a traditional floater is issued at par value and pays MRR plus $0.50 \%$. The quoted margin is $50 \mathrm{bps}$. If there is no change in the credit risk of the issuer, the required margin remains at $50 \mathrm{bps}$. On each quarterly reset date, the floater will be priced at par value. Between coupon dates, its flat price will be at a premium or discount to par value if MRR goes, respectively, down or up. However, if the required margin continues to be the same as the quoted margin, the flat price is "pulled to par" as the next reset date nears. At the reset date, any change in MRR is included in the interest payment for the next period.

Changes in the required margin usually come from changes in the issuer's credit risk. Changes in liquidity or tax status also could affect the required margin. Suppose that on a reset date, the required margin goes up to $75 \mathrm{bps}$ because of a downgrade in the issuer's credit rating. A floater having a quoted margin of $50 \mathrm{bps}$ now pays its investors a "deficient" interest payment. This FRN will be priced at a discount below par value. The amount of the discount is the present value of the deficient future cash flows. That annuity is 25 bps per period for the remaining life of the bond. It is the difference between the required and quoted margins. If the required margin goes down from $50 \mathrm{bps}$ to $40 \mathrm{bps}$, the FRN will be priced at a premium. The amount of the premium is the present value of the $10 \mathrm{bp}$ annuity for the "excess" interest payment each period.

Fixed-rate and floating-rate bonds are essentially the same with respect to changes in credit risk. With fixed-rate bonds, the premium or discount arises from a difference in the fixed coupon rate and the required yield-to-maturity. With floating-rate bonds, the premium or discount arises from a difference in the fixed quoted margin and the required margin. However, fixed-rate and floating-rate bonds are very different with respect to changes in benchmark interest rates.

The valuation of a floating-rate note needs a pricing model. Equation 8 is a simplified FRN pricing model. Following market practice, the required margin is called the discount margin.

$$
\begin{aligned}
& P V=\frac{\frac{(\text { Index }+Q M) \times F V}{m}}{\left(1+\frac{\text { Index }+D M}{m}\right)^{1}}+\frac{\frac{(\text { Index }+Q M) \times F V}{m}}{\left(1+\frac{\operatorname{Index}+D M}{m}\right)^{2}}+\cdots+ \\
& \frac{\frac{(\text { Index }+Q M) \times F V}{m}+F V}{\left(1+\frac{\operatorname{Index}+D M}{m}\right)^{N}}
\end{aligned}
$$

where

$P V=$ present value, or the price of the floating-rate note

Index $=$ reference rate, stated as an annual percentage rate

$Q M=$ quoted margin, stated as an annual percentage rate

$F V=$ future value paid at maturity, or the par value of the bond

$m=$ periodicity of the floating-rate note, the number of payment periods per year

$D M=$ discount margin, the required margin stated as an annual percentage rate

$N=$ number of evenly spaced periods to maturity

This equation is similar to Equation 1, which is the basic pricing formula for a fixed-rate bond given the market discount rate. In Equation $1, P M T$ is the coupon payment per period. Here, annual rates are used. The first interest payment is the annual rate for the period (Index $+Q M)$ times the par value $(F V)$ and divided by the number of periods in the year $(m)$. In Equation 1 , the market discount rate per period $(r)$ is used to discount the cash flows. Here, the discount rate per period is the reference rate plus the discount margin (Index $+D M$ ) divided by the periodicity $(m)$.

This is a simplified FRN pricing model, for several reasons. First, $P V$ is for a rate reset date when there are $N$ evenly spaced periods to maturity. There is no accrued interest so that the flat price is the full price. Second, the model assumes a 30/360 day-count convention so that the periodicity is an integer. Third, and most important, the same reference rate (Index) is used for all payment periods in both the numerators and denominators. More complex FRN pricing models use projected future rates for Index in the numerators and spot rates in the denominators. Therefore, the calculation for $D M$ depends on the simplifying assumptions in the pricing model.

Suppose that a two-year FRN pays MRR plus $0.50 \%$. Currently, MRR is $1.25 \%$. In Equation 8 , Index $=0.0125, Q M=0.0050$, and $m=2$. The numerators in Equation 8 , ignoring the repayment of principal, are 0.875 .

$\frac{(\operatorname{Index}+Q M) \times F V}{m}=\frac{(0.0125+0.0050) \times 100}{2}=0.875$

Suppose that the yield spread required by investors is 40 bps over the reference rate; $D M=0.0040$. The assumed discount rate per period is $0.825 \%$.

$\frac{\text { Index }+D M}{m}=\frac{0.0125+0.0040}{2}=0.00825$

Using Equation 8 for $N=4$, the FRN is priced at 100.196 per 100 of par value.

$\frac{0.875}{(1+0.00825)^{1}}+\frac{0.875}{(1+0.00825)^{2}}+\frac{0.875}{(1+0.00825)^{3}}+\frac{0.875+100}{(1+0.00825)^{4}}=100.196$

This floater is priced at a premium above par value because the quoted margin is greater than the discount margin.

A similar calculation is to estimate the discount margin given the market price of the floating-rate note. Suppose that a five-year FRN pays MRR plus $0.75 \%$ on a quarterly basis. Currently, MRR is $1.10 \%$. The price of the floater is 95.50 per 100 of par value, a discount below par value because of a downgrade in the issuer's credit rating. $\frac{(\text { Index }+Q M) \times F V}{m}=\frac{(0.0110+0.0075) \times 100}{4}=0.4625$

In Equation 8, use $P V=95.50$ and $N=20$.

$95.50=\frac{0.4625}{\left(1+\frac{0.0110+D M}{4}\right)^{1}}+\frac{0.4625}{\left(1+\frac{0.0110+D M}{4}\right)^{2}}+\cdots+\frac{0.4625+100}{\left(1+\frac{0.0110+D M}{4}\right)^{20}}$.

This has the same format as Equation 1, which can be used to solve for the market discount rate per period, $r=0.7045 \%$.

$95.50=\frac{0.4625}{(1+r)^{1}}+\frac{0.4625}{(1+r)^{2}}+\cdots+\frac{0.4625+100}{(1+r)^{20}} ; r=0.007045$.

This can be used to solve for $D M=1.718 \%$.

$0.007045=\frac{0.0110+D M}{4} ; D M=0.01718$

If this FRN was issued at par value, investors required at that time a spread of only 75 bps over MRR. Now, after the credit downgrade, investors require an estimated discount margin of $171.8 \mathrm{bps}$. The floater trades at a discount because the quoted margin remains fixed at $75 \mathrm{bps}$. The calculated discount margin is an estimate because it is based on a simplified FRN pricing model.The FRN pricing model in Equation 8 similarly applies to adjustable-rate loans made by banks and other non-securitized fixed-income instruments. Because a large portion of a bank's funding comes from short-duration investor deposits, banks prefer to make floating-rate unsecured loans to businesses and individuals, as opposed to fixed-rate loans, in order to help maintain a match between assets and liabilities on the balance sheet. This model is used to determine the appropriate interest rate to pay on demand or certificates of deposit (CDs), as well as to develop scorecards for risk-based floating-rate loan pricing.

\section{EXAMPLE 9}
\section{Calculating the Discount Margin for a Floating-Rate Note}
\begin{enumerate}
  \item A four-year French floating-rate note pays three-month Euribor (Euro Interbank Offered Rate, an index produced by the European Banking Federation) plus $1.25 \%$. The floater is priced at 98 per 100 of par value. Calculate the discount margin for the floater assuming that three-month Euribor is constant at $2 \%$. Assume the $30 / 360$ day-count convention and evenly spaced periods.
\end{enumerate}

\section{Solution:}
By assumption, the interest payment each period is 0.8125 per 100 of par value.

$$
\frac{(\text { Index }+Q M) \times F V}{m}=\frac{(0.0200+0.0125) \times 100}{4}=0.8125
$$

The discount margin can be estimated by solving for $D M$ in this equation.

$98=\frac{0.8125}{\left(1+\frac{0.0200+D M}{4}\right)^{1}}+\frac{0.8125}{\left(1+\frac{0.0200+D M}{4}\right)^{2}}+\cdots+$

$\frac{0.8125+100}{\left(1+\frac{0.0200+D M}{4}\right)^{16}}$

The solution for the discount rate per period is $0.9478 \%$.

$98=\frac{0.8125}{(1+r)^{1}}+\frac{0.8125}{(1+r)^{2}}+\cdots+\frac{0.8125+100}{(1+r)^{16}} ; r=0.009478$.

Therefore, $D M=1.791 \%$. $0.009478=\frac{0.0200+D M}{4} ; D M=0.01791$

The quoted margin is $125 \mathrm{bps}$ over the Euribor reference rate. Using the simplified FRN pricing model, it is estimated that investors require a 179.1 bp spread for the floater to be priced at par value.

\section{Yield Measures for Money Market Instruments}
Money market instruments are short-term debt securities. They range in time-to-maturity from overnight sale and repurchase agreements (repos) to one-year bank certificates of deposit. Money market instruments also include commercial paper, government issues of less than one year, bankers' acceptances, and time deposits based on such indexes as Libor and Euribor. Money market mutual funds are a major investor in such securities. These mutual funds can invest only in certain eligible money market securities.

There are several important differences in yield measures between the money market and the bond market:

\begin{enumerate}
  \item Bond yields-to-maturity are annualized and compounded. Yield measures in the money market are annualized but not compounded. Instead, the rate of return on a money market instrument is stated on a simple interest basis.

  \item Bond yields-to-maturity can be calculated using standard time-value-of-money analysis and with formulas programmed into a financial calculator. Money market instruments often are quoted using non-standard interest rates and require different pricing equations than those used for bonds.

  \item Bond yields-to-maturity usually are stated for a common periodicity for all times-to-maturity. Money market instruments having different times-to-maturity have different periodicities for the annual rate.

\end{enumerate}

In general, quoted money market rates are either discount rates or add-on rates. Although market conventions vary around the world, commercial paper, Treasury bills (US government securities issued with a maturity of one year or less), and bankers' acceptances often are quoted on a discount rate basis. Bank certificates of deposit, repos, and such indexes as Libor and Euribor are quoted on an add-on rate basis. It is important to understand that "discount rate" has a unique meaning in the money market. In general, discount rate means "interest rate used to calculate a present value"-for instance, "market discount rate" as used in our discussion. In the money market, however, discount rate is a specific type of quoted rate. Some examples will clarify this point.

Equation 9 is the pricing formula for money market instruments quoted on a discount rate basis.

$$
P V=F V \times\left(1-\frac{\text { Days }}{\text { Year }} \times D R\right) \text {, }
$$

where

$P V=$ present value, or the price of the money market instrument

$F V=$ future value paid at maturity, or the face value of the money market instrument

$$
\begin{aligned}
\text { Days } & =\text { number of days between settlement and maturity } \\
\text { Year } & =\text { number of days in the year } \\
D R & =\text { discount rate, stated as an annual percentage rate }
\end{aligned}
$$

Suppose that a 91-day US Treasury bill (T-bill) with a face value of USD10 million is quoted at a discount rate of $2.25 \%$ for an assumed 360-day year. Enter $F V=10,000,000$, Days $=91$, Year $=360$, and $D R=0.0225$. The price of the T-bill is USD9,943,125.

$$
P V=10,000,000 \times\left(1-\frac{91}{360} \times 0.0225\right)=9,943,125 \text {. }
$$

The unique characteristics of a money market discount rate can be examined with Equation 10, which transforms Equation 9 algebraically to isolate the $D R$ term.

$$
D R=\left(\frac{\text { Year }}{\text { Days }}\right) \times\left(\frac{F V-P V}{F V}\right) .
$$

The first term, Year/Days, is the periodicity of the annual rate. The second term reveals the odd character of a money market discount rate. The numerator, $F V-P V$, is the interest earned on the T-bill, USD56,875 $(=10,000,000-9,943,125)$, over the 91 days to maturity. However, the denominator is $F V$, not $P V$. In theory, an interest rate is the amount earned divided by the investment amount $(P V)$ - not divided by the total return at maturity, which includes the earnings $(F V)$. Therefore, by design, a money market discount rate understates the rate of return to the investor, and it understates the cost of borrowed funds to the issuer. That is because $P V$ is less than $F V$ (as long as $D R$ is greater than zero).

Equation 11 is the pricing formula for money market instruments quoted on an add-on rate basis.

$$
P V=\frac{F V}{\left(1+\frac{\text { Days }}{\text { Year }} \times A O R\right)}
$$

where

$P V=$ present value, principal amount, or the price of the money market instrument

$F V=$ future value, or the redemption amount paid at maturity including interest

Days $=$ number of days between settlement and maturity

Year $=$ number of days in the year

$A O R=$ add-on rate, stated as an annual percentage rate

Suppose that a Canadian pension fund buys a 180-day banker's acceptance (BA) with a quoted add-on rate of $4.38 \%$ for a 365 -day year. If the initial principal amount is CAD10 million, the redemption amount due at maturity is found by re-arranging Equation 11 and entering $P V=10,000,000$, Days $=180$, Year $=365$, and $A O R=0.0438$.

$$
F V=10,000,000+\left(10,000,000 \times \frac{180}{365} \times 0.0438\right)=10,216,000
$$

At maturity, the pension fund receives CAD10,216,000, the principal of CAD10 million plus interest of CAD216,000. The interest is calculated as the principal times the fraction of the year times the annual add-on rate. It is added to the principal to determine the redemption amount.

Suppose that after 45 days, the pension fund sells the BA to a dealer. At that time, the quoted add-on rate for a 135 -day BA is $4.17 \%$. The sale price for the BA can be calculated using Equation 11 for $F V=10,216,000$, Days $=135$, Year $=365$, and $A O R$ $=0.0417$. The sale price is CAD10,060,829.

$$
P V=\frac{10,216,000}{\left(1+\frac{135}{365} \times 0.0417\right)}=10,060,829 .
$$

The characteristics of an add-on rate can be examined with Equation 12, which transforms Equation 11 algebraically to isolate the $A O R$ term.

$$
A O R=\left(\frac{\text { Year }}{\text { Days }}\right) \times\left(\frac{F V-P V}{P V}\right) .
$$

This equation indicates that an add-on rate is a reasonable yield measure for a money market investment. The first term, Year/Days, is the periodicity of the annual rate. The second term is the interest earned, $F V-P V$, divided by $P V$, the amount invested.

The pension fund's rate of return on its 45-day investment in the banker's acceptance can be calculated with Equation 12. Enter Year $=365$, Days $=45, F V=10,060,829$, and $P V=10,000,000$. Notice that $F V$ here is the sale price, not the redemption amount.

$$
A O R=\left(\frac{365}{45}\right) \times\left(\frac{10,060,829-10,000,000}{10,000,000}\right)=0.04934 .
$$

The rate of return, stated on a 365 -day add-on rate basis, is $4.934 \%$. This result is an annual rate for a periodicity of 8.11 (= 365/45). Implicitly, this assumes that the investment can be replicated 8.11 times over the year.

Investment analysis is made difficult for money market securities because (1) some instruments are quoted on a discount rate basis and others on an add-on rate basis and (2) some are quoted for a 360-day year and others for a 365-day year. Another difference is that the "amount" of a money market instrument quoted on a discount rate basis typically is the face value paid at maturity. However, the "amount" when quoted on an add-on rate basis usually is the principal, the price at issuance. To make money market investment decisions, it is essential to compare instruments on a common basis. An example illustrates this point.

Suppose that an investor is comparing two money market instruments: (A) 90-day commercial paper quoted at a discount rate of $5.76 \%$ for a 360-day year and (B) a 90-day bank time deposit quoted at an add-on rate of $5.90 \%$ for a 365 -day year. Which offers the higher expected rate of return assuming that the credit risks are the same? The price of the commercial paper is 98.560 per 100 of face value, calculated using Equation 9 and entering $F V=100$, Days $=90$, Year $=360$, and $D R=0.0576$.

$$
P V=100 \times\left(1-\frac{90}{360} \times 0.0576\right)=98.560 \text {. }
$$

Next, use Equation 12 to solve for the $A O R$ for a 365-day year, whereby Year $=$ 365, Days $=90, F V=100$, and $P V=98.560$.

$$
A O R=\left(\frac{365}{90}\right) \times\left(\frac{100-98.560}{98.560}\right)=0.05925 .
$$

The 90-day commercial paper discount rate of $5.76 \%$ converts to an add-on rate for a 365-day year of $5.925 \%$. This converted rate is called a bond equivalent yield, or sometimes just an "investment yield." A bond equivalent yield is a money market rate stated on a 365-day add-on rate basis. If the risks are the same, the commercial paper offers 2.5 bps more in annual return than the bank time deposit.

\section{EXAMPLE 10}
\section{Comparing Money Market Instruments Based on Bond Equivalent Yields}
\begin{enumerate}
  \item Suppose that a money market investor observes quoted rates on the following four 180-day money market instruments:
\end{enumerate}

\begin{center}
\begin{tabular}{llcc}
\hline
$\begin{array}{l}\text { Money Market } \\ \text { Instrument }\end{array}$ & $\begin{array}{c}\text { Quotation } \\ \text { Basis }\end{array}$ & $\begin{array}{c}\text { Assumed Number of } \\ \text { Days in the Year }\end{array}$ & Quoted Rate \\
\hline
A & Discount Rate & 360 & $4.33 \%$ \\
B & Discount Rate & 365 & $4.36 \%$ \\
C & Add-On Rate & 360 & $4.35 \%$ \\
D & Add-On Rate & 365 & $4.45 \%$ \\
\hline
\end{tabular}
\end{center}

Calculate the bond equivalent yield for each instrument. Which instrument offers the investor the highest rate of return if the credit risks are the same?

\section{Solution:}
A. Use Equation 9 to get the price per 100 of par value, where $F V=100$, Days $=180$, Year $=360$, and $D R=0.0433$.

$P V=100 \times\left(1-\frac{180}{360} \times 0.0433\right)=97.835$.

Use Equation 12 to get the bond equivalent yield, where Year $=365$, Days $=180, F V=100$, and $P V=97.835$.

$A O R=\left(\frac{365}{180}\right) \times\left(\frac{100-97.835}{97.835}\right)=0.04487$.

The bond equivalent yield for Bond A is $4.487 \%$.

B. Use Equation 9 to get the price per 100 of face value, where $F V=100$, Days $=180$, Year $=365$, and $D R=0.0436$.

$P V=100 \times\left(1-\frac{180}{365} \times 0.0436\right)=97.850$.

Use Equation 12 to get the bond equivalent yield, where Year $=365$, Days $=180, F V=100$, and $P V=97.850$.

$A O R=\left(\frac{365}{180}\right) \times\left(\frac{100-97.850}{97.850}\right)=0.04456$.

The bond equivalent yield for Bond B is $4.456 \%$.

C. First, determine the redemption amount per 100 of principal $(P V=$ 100), where Days $=180$, Year $=360$, and $A O R=0.0435$.

$F V=100+\left(100 \times \frac{180}{360} \times 0.0435\right)=102.175$.

Use Equation 12 to get the bond equivalent yield, where Year $=365$, Days $=180, F V=102.175$, and $P V=100$.

$A O R=\left(\frac{365}{180}\right) \times\left(\frac{102.175-100}{100}\right)=0.04410$.

The bond equivalent yield for Bond $\mathrm{C}$ is $4.410 \%$. Another way to get the bond equivalent yield for Bond $C$ is to observe that the $A O R$ of $4.35 \%$ for a 360 -day year can be obtained using

Equation 12 for Year $=360$, Days $=180, F V=102.175$, and $P V=100$.

$A O R=\left(\frac{360}{180}\right) \times\left(\frac{102.175-100}{100}\right)=0.0435$

Therefore, an add-on rate for a 360-day year only needs to be multiplied by the factor of $365 / 360$ to get the 365-day year bond equivalent yield.

$\frac{365}{360} \times 0.0435=0.04410$

D. The quoted rate for Bond $\mathrm{D}$ of $4.45 \%$ is a bond equivalent yield, which is defined as an add-on rate for a 365-day year.

If the risks of these money market instruments are the same, Bond A offers the highest rate of return on a bond equivalent yield basis, $4.487 \%$.

The third difference between yield measures in the money market and the bond market is the periodicity of the annual rate. Because bond yields-to-maturity are computed using interest rate compounding, there is a well-defined periodicity. For instance, bond yields-to-maturity for semiannual compounding are annualized for a periodicity of two. Money market rates are computed using simple interest without compounding. In the money market, the periodicity is the number of days in the year divided by the number of days to maturity. Therefore, money market rates for different times-to-maturity have different periodicities.

Suppose that an analyst prefers to convert money market rates to a semiannual bond basis so that the rates are directly comparable to yields on bonds that make semiannual coupon payments. The quoted rate for a 90-day money market instrument is $10 \%$, quoted as a bond equivalent yield, which means its periodicity is $365 / 90$. Using Equation 7 , the conversion is from $m=365 / 90$ to $n=2$ for $A P R_{365 / 90}=0.10$.

$$
\left(1+\frac{0.10}{365 / 90}\right)^{365 / 90}=\left(1+\frac{A P R_{2}}{2}\right)^{2} ; A P R_{2}=0.10127
$$

Therefore, 10\% for a periodicity of $365 / 90$ corresponds to $10.127 \%$ for a periodicity of two. The difference is significant-12.7 bps. In general, the difference depends on the level of the annual percentage rate. When interest rates are lower, the difference between the annual rates for any two periodicities is reduced.

\section{THE MATURITY STRUCTURE OF INTEREST RATES}
define and compare the spot curve, yield curve on coupon bonds, par curve, and forward curve

define forward rates and calculate spot rates from forward rates, forward rates from spot rates, and the price of a bond using forward rates There are many reasons why the yields-to-maturity on any two bonds are different. Suppose that the yield-to-maturity is higher on Bond $\mathrm{X}$ than on Bond $\mathrm{Y}$. The following are some possible reasons for the difference between the yields:

\begin{itemize}
  \item Currency-Bond $\mathrm{X}$ could be denominated in a currency with a higher expected rate of inflation than the currency in which Bond $\mathrm{Y}$ is denominated.

  \item Credit risk-Bond $X$ could have a non-investment-grade rating of $B B$, and Bond $Y$ could have an investment-grade rating of AA.

  \item Liquidity-Bond X could be illiquid, and Bond $Y$ could be actively traded.

  \item Tax status-Interest income on Bond $\mathrm{X}$ could be taxable, whereas interest income on Bond $\mathrm{Y}$ could be exempt from taxation.

  \item Periodicity-Bond X could make a single annual coupon payment, and its yield-to-maturity could be quoted for a periodicity of one. Bond $Y$ could make monthly coupon payments, and its yield-to-maturity could be annualized for a periodicity of 12 .

\end{itemize}

Obviously, another reason is that Bond $\mathrm{X}$ and Bond $\mathrm{Y}$ could have different times-to-maturity. This factor explaining the differences in yields is called the maturity structure, or term structure, of interest rates. It involves the analysis of yield curves, which are relationships between yields-to-maturity and times-to-maturity. There are different types of yield curves, depending on the characteristics of the underlying bonds.

In theory, maturity structure should be analyzed for bonds that have the same properties other than time-to-maturity. The bonds should be denominated in the same currency and have the same credit risk, liquidity, and tax status. Their annual rates should be quoted for the same periodicity. Also, they should have the same coupon rate so that they each have the same degree of coupon reinvestment risk. In practice, maturity structure is analyzed for bonds for which these strong assumptions rarely hold.

The ideal dataset would be yields-to-maturity on a series of zero-coupon government bonds for a full range of maturities. This dataset is the government bond spot curve, sometimes called the zero or "strip" curve (because the coupon payments are "stripped" off the bonds). The spot, zero, or strip curve is a sequence of yields-to-maturity on zero-coupon bonds. Often, these government spot rates are interpreted as the "risk-free" yields; in this context, "risk-free" refers only to default risk. There still could be a significant amount of inflation risk to the investor, as well as liquidity risk.

A government bond spot curve is illustrated in Exhibit 5 for maturities ranging from 1 to 30 years. The annual yields are stated on a semiannual bond basis, which facilitates comparison to coupon-bearing bonds that make semiannual payments.

\section{Exhibit 5: A Government Bond Spot Curve}
\begin{center}
\includegraphics[max width=\textwidth]{2023_05_04_7b535d0a870224f62e3dg-585}
\end{center}

This spot curve is upward sloping and flattens for longer times-to-maturity. Longer-term government bonds usually have higher yields than shorter-term bonds. This pattern is typical under normal market conditions. Sometimes, a spot curve is downward sloping in that shorter-term yields are higher than longer-term yields. Such downward-sloping spot curves are called inverted yield curves. The theories that attempt to explain the shape of the yield curve and its implications for future financial market conditions are covered later.

This hypothetical spot curve is ideal for analyzing maturity structure because it best meets the "other things being equal" assumption. These government bonds presumably have the same currency, credit risk, liquidity, and tax status. Most importantly, they have no coupon reinvestment risk because there are no coupons to reinvest. However, most actively traded government and corporate bonds make coupon payments. Therefore, analysis of maturity structure usually is based on price data on government bonds that make coupon payments. These coupon bonds might not have the same liquidity and tax status. Older ("seasoned") bonds tend to be less liquid than newly issued debt because they are owned by "buy-and-hold" institutional and retail investors. Governments issue new debt for regular times-to-maturity-for instance, 5-year and 10-year bonds. The current 6-year bond could be a 10-year bond that was issued four years ago. Also, as interest rates fluctuate, older bonds are priced at a discount or premium to par value, which can lead to tax differences. In some countries, capital gains have different tax treatment than capital losses and interest income.

Analysts usually use only the most recently issued and actively traded government bonds to build a yield curve. These bonds have similar liquidity, and because they are priced closer to par value, they have fewer tax effects. A problem is that there are limited data for the full range of maturities. Therefore, it is necessary to interpolate between observed yields. Exhibit 6 illustrates a yield curve for a government that issues 2-year, 3-year, 5-year, 7-year, 10-year, and 30-year bonds that make semiannual coupon payments. Straight-line interpolation is used between those points on the yield curve for coupon bonds.

\section{Exhibit 6: A Government Bond Yield Curve}
\begin{center}
\includegraphics[max width=\textwidth]{2023_05_04_7b535d0a870224f62e3dg-586}
\end{center}

Exhibit 6 also includes yields for short-term government securities having 1 month, 3 months, 6 months, and 12 months to maturity. Although these money market instruments might have been issued and traded on a discount rate basis, they typically are reported as bond equivalent yields. It is important for the analyst to know whether they have been converted to the same periodicity as the longer-term government bonds. If not, the observed yield curve can be misleading because the number of periods in the year is not the same.

In addition to the yield curve on coupon bonds and the spot curve on zero-coupon bonds, maturity structure can be assessed using a par curve. A par curve is a sequence of yields-to-maturity such that each bond is priced at par value. The bonds, of course, are assumed to have the same currency, credit risk, liquidity, tax status, and annual yields stated for the same periodicity. Between coupon payment dates, the flat price (not the full price) is assumed to be equal to par value.

The par curve is obtained from a spot curve. On a coupon payment date, the following equation can be used to calculate a par rate given the sequence of spot rates.

$$
100=\frac{P M T}{\left(1+z_{1}\right)^{1}}+\frac{P M T}{\left(1+z_{2}\right)^{2}}+\cdots+\frac{P M T+100}{\left(1+z_{N}\right)^{N}} .
$$

This equation is very similar to Equation 2, whereby $P V=F V=100$. The problem is to solve for $P M T$ algebraically. Then, $P M T / 100$ is equal to the par rate per period.

An example illustrates the calculation of the par curve given a spot curve. Suppose the spot rates on government bonds are $5.263 \%$ for one year, $5.616 \%$ for two years, $6.359 \%$ for three years, and $7.008 \%$ for four years. These are effective annual rates. The one-year par rate is $5.263 \%$.

$$
100=\frac{P M T+100}{(1.05263)^{1}} ; P M T=5.263
$$

The two-year par rate is $5.606 \%$.

$$
100=\frac{P M T}{(1.05263)^{1}}+\frac{P M T+100}{(1.05616)^{2}} ; P M T=5.606
$$

The three-year and four-year par rates are $6.306 \%$ and $6.899 \%$, respectively.

$$
\begin{aligned}
& 100=\frac{P M T}{(1.05263)^{1}}+\frac{P M T}{(1.05616)^{2}}+\frac{P M T+100}{(1.06359)^{3}} ; P M T=6.306 \\
& 100=\frac{P M T}{(1.05263)^{1}}+\frac{P M T}{(1.05616)^{2}}+\frac{P M T}{(1.06359)^{3}}+\frac{P M T+100}{(1.07008)^{4}} ; P M T=6.899 .
\end{aligned}
$$

The fixed-income securities covered so far have been cash market securities. Money market securities often are settled on a "same day" or "cash settlement," basis. Other securities have a difference between the trade date and the settlement date. For instance, if a government bond trades on a $T+1$ basis, there is a one-day difference between the trade date and the settlement date. If a corporate bond trades on a $T+$ 3 basis, the seller delivers the bond and the buyer makes payment in three business days. Cash markets are also called spot markets, which can be confusing because "spot rate" can have two meanings. It can mean the rate on a bond traded in the spot, or cash, market. It can also mean yield on a zero-coupon bond, which is the meaning used in our discussion.

A forward market is for future delivery, beyond the usual settlement time period in the cash market. Agreement to the terms for the transaction is on the trade date, but delivery of the security and payment for it are deferred to a future date. A forward rate is the interest rate on a bond or money market instrument traded in a forward market. For example, suppose that in the cash market, a five-year zero-coupon bond is priced at 81 per 100 of par value. Its yield-to-maturity is $4.2592 \%$, stated on a semiannual bond basis.

$$
81=\frac{100}{(1+r)^{10}} ; r=0.021296 ; \times 2=0.042592
$$

Suppose that a dealer agrees to deliver a five-year bond two years into the future for a price of 75 per 100 of par value. The credit risk, liquidity, and tax status of this bond traded in the forward market are the same as those for this bond in the cash market. The forward rate is $5.8372 \%$.

$$
75=\frac{100}{(1+r)^{10}} ; r=0.029186 ; \times 2=0.058372
$$

The notation for forward rates is important to understand. Although finance textbook authors use varying notation, the most common market practice is to name this forward rate "the $2 y 5 y$." This term is pronounced "the two-year into five-year rate," or simply "the 2's, 5's." The idea is that the first number (two years) refers to the length of the forward period in years from today and the second number (five years) refers to the tenor of the underlying bond. The tenor is the remaining time-to-maturity for a bond (or a derivative contract). Therefore, $5.8372 \%$ is the "2y5y" forward rate for the zero-coupon bond-the five-year yield two years into the future. Note that the bond that will be a five-year zero in two years currently has seven years to maturity. In the money market, the forward rate usually refers to months. For instance, an analyst might inquire about the " $1 \mathrm{~m} 6 \mathrm{~m}$ " forward rate on Euribor, which is the rate on six-month Euribor one month into the future.

Implied forward rates (also known as forward yields) are calculated from spot rates. An implied forward rate is a breakeven reinvestment rate. It links the return on an investment in a shorter-term zero-coupon bond to the return on an investment in a longer-term zero-coupon bond. Suppose that the shorter-term bond matures in $A$ periods and the longer-term bond matures in $B$ periods. The yields-to-maturity per period on these bonds are denoted $z_{A}$ and $z_{B}$. The first is an $A$-period zero-coupon bond trading in the cash market. The second is a $B$-period zero-coupon cash market bond. The implied forward rate between period $A$ and period $B$ is denoted $I F R_{A, B-A}$. It is a forward rate on a security that starts in period $A$ and ends in period $B$. Its tenor is $B-A$ periods.

Equation 14 is a general formula for the relationship between the two spot rates and the implied forward rate.

$$
\left(1+z_{A}\right)^{A} \times\left(1+\operatorname{IFR}_{A, B-A}\right)^{B-A}=\left(1+z_{B}\right)^{B} .
$$

Suppose that the yields-to-maturity on three-year and four-year zero-coupon bonds are $3.65 \%$ and $4.18 \%$, respectively, stated on a semiannual bond basis. An analyst would like to know the "3y1y" implied forward rate, which is the implied one-year forward yield three years into the future. Therefore, $A=6$ (periods), $B=8$ (periods), $B-A=$ 2 (periods), $z_{6}=0.0365 / 2$ (per period), and $z_{8}=0.0418 / 2$ (per period).

$$
\begin{aligned}
& \left(1+\frac{0.0365}{2}\right)^{6} \times\left(1+I F R_{6,2}\right)^{2}=\left(1+\frac{0.0418}{2}\right)^{8} ; I F R_{6,2}=0.02889 ; \\
& \times 2=0.05778 .
\end{aligned}
$$

The "3y1y" implied forward yield is $5.778 \%$, annualized for a periodicity of two.

Equation 14 can be used to construct a forward curve. A forward curve is a series of forward rates, each having the same time frame. These forward rates might be observed on transactions in the derivatives market. Often, the forward rates are implied from transactions in the cash market. Exhibit 7 displays the forward curve that is calculated from the government bond spot curve shown in Exhibit 5. These are one-year forward rates stated on a semiannual bond basis.

\section{Exhibit 7: A Government Bond Spot Curve and Forward Curve}
\begin{center}
\includegraphics[max width=\textwidth]{2023_05_04_7b535d0a870224f62e3dg-589}
\end{center}

A forward rate can be interpreted as an incremental, or marginal, return for extending the time-to-maturity for an additional time period. Suppose an investor has a four-year investment horizon and is choosing between buying a three-year zero-coupon bond that is priced to yield $3.65 \%$ and a four-year zero that is priced to yield $4.18 \%$. The incremental, or marginal, return for the fourth year is $5.778 \%$, the "3y1y" implied forward rate. If the investor's view on future bond yields is that the one-year yield in three years is likely to be less than $5.778 \%$, the investor might prefer to buy the four-year bond. However, if the investor's view is that the one-year yield will be more than the implied forward rate, the investor might prefer the three-year bond and the opportunity to reinvest at the expected higher rate. That explains why an implied forward rate is the breakeven reinvestment rate. Implied forward rates are very useful to investors as well as bond issuers in making maturity decisions.

\section{EXAMPLE 11}
\section{Computing Forward Rates}
Suppose that an investor observes the following prices and yields-to-maturity on zero-coupon government bonds:

\begin{center}
\begin{tabular}{lcc}
\hline
Maturity & Price & Yield-to-Maturity \\
\hline
1 year & 97.50 & $2.548 \%$ \\
2 years & 94.25 & $2.983 \%$ \\
3 years & 91.75 & $2.891 \%$ \\
\hline
\end{tabular}
\end{center}

The prices are per 100 of par value. The yields-to-maturity are stated on a semiannual bond basis. 1. Compute the "1y1y" and "2y1y" implied forward rates, stated on a semiannual bond basis.

Solution:

The "1y1y" implied forward rate is $3.419 \%$. In Equation $14, A=2$ (periods), $B=4$ (periods), $B-A=2$ (periods), $z_{2}=0.02548 / 2$ (per period), and $z_{4}=$ $0.02983 / 2$ (per period).

$$
\begin{aligned}
& \left(1+\frac{0.02548}{2}\right)^{2} \times\left(1+I F R_{2,2}\right)^{2}=\left(1+\frac{0.02983}{2}\right)^{4} ; I F R_{2,2}=0.017095 ; \\
& \times 2=0.03419 .
\end{aligned}
$$

The "2y1y" implied forward rate is 2.707\%. In Equation 14, $A=4$ (periods), $B=6$ (periods), $B-A=2$ (periods), $z_{4}=0.02983 / 2$ (per period), and $z_{6}=$ $0.02891 / 2$ (per period).

$$
\begin{aligned}
& \left(1+\frac{0.02983}{2}\right)^{4} \times\left(1+I F R_{4,2}\right)^{2}=\left(1+\frac{0.02891}{2}\right)^{6} ; I F R_{4,2}=0.013536 ; \\
& \times 2=0.02707 .
\end{aligned}
$$

\begin{enumerate}
  \setcounter{enumi}{1}
  \item The investor has a three-year investment horizon and is choosing between (1) buying the two-year zero and reinvesting in another one-year zero in two years and (2) buying and holding to maturity the three-year zero. The investor decides to buy the two-year bond. Based on this decision, which of the following is the minimum yield-to-maturity the investor expects on one-year zeros two years from now?
A. $2.548 \%$
B. $2.707 \%$
C. $2.983 \%$
\end{enumerate}

Solution:

B is correct. The investor's view is that the one-year yield in two years will be greater than or equal to $2.707 \%$.

The "2y1y" implied forward rate of $2.707 \%$ is the breakeven reinvestment rate. If the investor expects the one-year rate in two years to be less than that, the investor would prefer to buy the three-year zero. If the investor expects the one-year rate in two years to be greater than $2.707 \%$, the investor might prefer to buy the two-year zero and reinvest the cash flow.

The forward curve has many applications in fixed-income analysis. Forward rates are used to make maturity decisions. They are used to identify arbitrage opportunities between transactions in the cash market for bonds and in derivatives markets. Forward rates are important in the valuation of derivatives, especially interest rate swaps and options. Those applications for the forward curve are covered elsewhere.

Forward rates can be used to value a fixed-income security in the same manner as spot rates because they are interconnected. The spot curve can be calculated from the forward curve, and the forward curve can be calculated from the spot curve. Either curve can be used to value a fixed-rate bond. An example will illustrate this process.

Suppose the current forward curve for one-year rates is the following:

\begin{center}
\begin{tabular}{llc}
\hline
Time Period &  & Forward Rate \\
\hline
 & $0 \mathrm{y} 1 \mathrm{y}$ & $1.88 \%$ \\
 & $1 \mathrm{y} 1 \mathrm{y}$ & $2.77 \%$ \\
$2 \mathrm{y} 1 \mathrm{y}$ & $3.54 \%$ &  \\
\end{tabular}
\end{center}

\begin{center}
\begin{tabular}{ccc}
\hline
Time Period & Forward Rate \\
\hline
$3 y 1 \mathrm{y}$ & $4.12 \%$ \\
\hline
\end{tabular}
\end{center}

These are annual rates stated for a periodicity of one. They are effective annual rates. The first rate, the " $0 y 1 y$," is the one-year spot rate. The others are one-year forward rates. Given these rates, the spot curve can be calculated as the geometric average of the forward rates.

The two-year implied spot rate is $2.3240 \%$.

$(1.0188 \times 1.0277)=\left(1+z_{2}\right)^{2} ; z_{2}=0.023240$.

The following are the equations for the three-year and four-year implied spot rates.

$$
\begin{aligned}
& (1.0188 \times 1.0277 \times 1.0354)=\left(1+z_{3}\right)^{3} ; z_{3}=0.027278 . \\
& (1.0188 \times 1.0277 \times 1.0354 \times 1.0412)=\left(1+z_{4}\right)^{4} ; z_{4}=0.030741 .
\end{aligned}
$$

The three-year implied spot rate is $2.7278 \%$, and the four-year spot rate is $3.0741 \%$.

Suppose that an analyst needs to value a four-year, $3.75 \%$ annual coupon payment bond that has the same risks as the bonds used to obtain the forward curve. Using the implied spot rates, the value of the bond is 102.637 per 100 of par value.

$\frac{3.75}{(1.0188)^{1}}+\frac{3.75}{(1.023240)^{2}}+\frac{3.75}{(1.027278)^{3}}+\frac{103.75}{(1.030741)^{4}}=102.637$.

The bond also can be valued using the forward curve.

$$
\begin{aligned}
& \frac{3.75}{(1.0188)}+\frac{3.75}{(1.0188 \times 1.0277)}+\frac{3.75}{(1.0188 \times 1.0277 \times 1.0354)} \\
& +\frac{103.75}{(1.0188 \times 1.0277 \times 1.0354 \times 1.0412)}=102.637 .
\end{aligned}
$$

\section{YIELD SPREADS}
compare, calculate, and interpret yield spread measures

A yield spread, in general, is the difference in yield between different fixed-income securities. This section describes a number of yield spread measures.

\section{Yield Spreads over Benchmark Rates}
In fixed-income security analysis, it is important to understand why bond prices and yields-to-maturity change. To do this, it is useful to separate a yield-to-maturity into two components: the benchmark and the spread. The benchmark yield for a fixed-income security with a given time-to-maturity is the base rate, often a government bond yield. The spread is the difference between the yield-to-maturity and the benchmark.

The reason for this separation is to distinguish between macroeconomic and microeconomic factors that affect the bond price and, therefore, its yield-to-maturity. The benchmark captures the macroeconomic factors: the expected rate of inflation in the currency in which the bond is denominated, general economic growth and the business cycle, foreign exchange rates, and the impact of monetary and fiscal policy. Changes in those factors impact all bonds in the market, and the effect is seen mostly in changes in the benchmark yield. The spread captures the microeconomic factors specific to the bond issuer and the bond itself: credit risk of the issuer and changes in the quality rating on the bond, liquidity and trading in comparable securities, and the tax status of the bond. It should be noted, however, that general yield spreads across issuers can widen and narrow with changes in macroeconomic factors.

Exhibit 8 illustrates the building blocks of the yield-to-maturity, starting with the benchmark and the spread. The benchmark is often called the risk-free rate of return. Also, the benchmark can be broken down into the expected real rate and the expected inflation rate in the economy. The yield spread is called the risk premium over the "risk-free" rate of return. The risk premium provides the investor with compensation for the credit and liquidity risks and possibly the tax impact of holding a specific bond.

\section{Exhibit 8: Yield-to-Maturity Building Blocks}
\begin{center}
\includegraphics[max width=\textwidth]{2023_05_04_7b535d0a870224f62e3dg-592}
\end{center}

The benchmark varies across financial markets. Fixed-rate bonds often use a government benchmark security with the same time-to-maturity as, or the closest time-to-maturity to, the specified bond. This benchmark is usually the most recently issued government bond and is called the on-the-run security. The on-the-run government bond is the most actively traded security and has a coupon rate closest to the current market discount rate for that maturity. That implies that it is priced close to par value. Seasoned government bonds are called off-the-run. On-the-run bonds typically trade at slightly lower yields-to-maturity than off-the-run bonds having the same or similar times-to-maturity because of differences in demand for the securities and, sometimes, differences in the cost of financing the government security in the repo market.

A frequently used benchmark for floating-rate notes has long been Libor, which is due to be phased out, with the Market Reference Rate to take its place. As a composite interbank rate, it is not a risk-free rate. The yield spread over a specific benchmark is referred to as the benchmark spread and is usually measured in basis points. If no benchmark exists for a specific bond's tenor or a bond has an unusual maturity, interpolation is used to derive an implied benchmark. Also, bonds with very long tenors are priced over the longest available benchmark bond. For example, 100-year bonds (often called "century bonds") in the United States are priced over the 30-year US Treasury benchmark rate.

In the United Kingdom, the United States, and Japan, the benchmark rate for fixed-rate bonds is a government bond yield. The yield spread in basis points over an actual or interpolated government bond is known as the G-spread. The spread over a government bond is the return for bearing greater credit, liquidity, and other risks relative to the sovereign bond. Euro-denominated corporate bonds are priced over a EUR interest rate swap benchmark. For example, a newly issued five-year EUR bond might be priced at a rate of "mid-swaps" plus 150 bps, where "mid-swaps" is the average of the bid and offered swap rates. The yield spread is over a five-year EUR swap rate rather than a government benchmark. Note that the government bond yield or swap rate used as the benchmark for a specific corporate bond will change over time as the remaining time-to-maturity changes.

The yield spread of a specific bond over the standard swap rate in that currency of the same tenor is known as the I-spread or interpolated spread to the swap curve. This yield spread over MRR allows comparison of bonds with differing credit and liquidity risks against an interbank lending benchmark. Issuers will use the spread above MRR to determine the relative cost of fixed-rate bonds versus floating-rate alternatives, such as an FRN or commercial paper. Investors use the spread over MRR as a measure of a bond's credit risk. Whereas a standard interest rate swap involves an exchange of fixed for floating cash flows based on a floating index, an asset swap converts the periodic fixed coupon of a specific bond to an MRR plus or minus a spread. If the bond is priced close to par, this conversion approximates the price of a bond's credit risk over the MRR index. Exhibit 9 illustrates these yield spreads using the Bloomberg Fixed Income Relative Value (FIRV) page.

This example is for the 3.75\% Apple bond that matures on 13 November 2047. The spreads are in the top-left corner of the page. The bond's flat asked price was 96.461 per 100 of par value on 12 July 2018 , and its yield-to-maturity was $3.955 \%$. On that date, the yield spread over a particular Treasury benchmark was 104 bps. Its G-spread over an interpolated government bond yield was also $104 \mathrm{bps}$. These two spreads sometimes differ by a few basis points, especially if the benchmark maturity differs from that of the underlying bond. The bond's I-spread was $109 \mathrm{bps}$. That Libor spread (to be replaced by MRR in the future) was a little larger than the G-spread because 30-year Treasury yields were slightly higher than 30-year Libor swap rates at that time. The use of these spreads in investor strategies will be covered in more detail later. In general, an analyst will track these spreads relative to their averages and historical highs and lows in an attempt to identify relative value.

\section{Yield Spreads over the Benchmark Yield Curve}
A yield curve shows the relationship between yields-to-maturity and times-to-maturity for securities with the same risk profile. For example, the government bond yield curve is the relationship between the yields of on-the-run government bonds and their times-to-maturity. The swap yield curve shows the relationship between fixed MRR swap rates and their times-to-maturity.

Each of these yield curves represents the term structure of benchmark interest rates, whether for "risk-free" government yields or "risky" fixed swap rates. Benchmark yield curves tend to be upward-sloping because investors typically demand a premium for holding longer-term securities. In general, investors face greater price risk for a given change in yield for longer-term bonds. This concept is covered further in the discussion of the topic of "Fixed-Income Risk and Return." The term structure of interest rates is dynamic, with short-term rates driven by central bank policy and longer-term rates affected by long-term growth and inflation expectations.

Isolating credit risk over varying times-to-maturity gives rise to a term structure of credit spreads that is distinct for each borrower. The G-spread and I-spread each use the same discount rate for each cash flow. Another approach is to calculate a constant yield spread over a government (or interest rate swap) spot curve instead. This spread is known as the zero-volatility spread (Z-spread) of a bond over the benchmark rate. In Exhibit 9, the Z-spread for the Apple bond was reported to be $107 \mathrm{bps}$.

The Z-spread over the benchmark spot curve can be calculated with Equation 15:

\begin{center}
\includegraphics[max width=\textwidth]{2023_05_04_7b535d0a870224f62e3dg-594}
\end{center}

$$
P V=\frac{P M T}{\left(1+z_{1}+Z\right)^{1}}+\frac{P M T}{\left(1+z_{2}+Z\right)^{2}}+\cdots+\frac{P M T+F V}{\left(1+z_{N}+Z\right)^{N}}
$$

The benchmark spot rates $-z_{1}, z_{2}, \ldots, z_{N}$-are derived from the government yield curve (or from fixed rates on interest rate swaps). $Z$ is the $Z$-spread per period and is the same for all time periods. In Equation 15, $N$ is an integer, so the calculation is on a coupon date when the accrued interest is zero. Sometimes, the Z-spread is called the "static spread" because it is constant (and has zero volatility). In practice, the Z-spread is usually calculated in a spreadsheet using a goal seek function or similar solver function.

The Z-spread is also used to calculate the option-adjusted spread (OAS) on a callable bond. The OAS, like the option-adjusted yield, is based on an option-pricing model and an assumption about future interest rate volatility. Then, the value of the embedded call option, which is stated in basis points per year, is subtracted from the yield spread. In particular, it is subtracted from the Z-spread:

$\mathrm{OAS}=$ Z-spread - Option value (in basis points per year).

This important topic is covered later.

\section{EXAMPLE 12}
\section{The G-Spread and the Z-Spread}
A $6 \%$ annual coupon corporate bond with two years remaining to maturity is trading at a price of 100.125 . The two-year, $4 \%$ annual payment government benchmark bond is trading at a price of 100.750 . The one-year and two-year government spot rates are $2.10 \%$ and $3.635 \%$, respectively, stated as effective annual rates.

\begin{enumerate}
  \item Calculate the G-spread, the spread between the yields-to-maturity on the corporate bond and the government bond having the same maturity.
\end{enumerate}

\section{Solution:}
The yield-to-maturity for the corporate bond is $5.932 \%$.

$$
100.125=\frac{6}{(1+r)^{1}}+\frac{106}{(1+r)^{2}} ; r=0.05932 .
$$

The yield-to-maturity for the government benchmark bond is $3.605 \%$.

$100.750=\frac{4}{(1+r)^{1}}+\frac{104}{(1+r)^{2}} ; r=0.03605$

The G-spread is 232.7 bps: $0.05932-0.03605=0.02327$.

\begin{enumerate}
  \setcounter{enumi}{1}
  \item Demonstrate that the Z-spread is $234.22 \mathrm{bps}$.
\end{enumerate}

\section{Solution:}
Solve for the value of the corporate bond using $z_{1}=0.0210, z_{2}=0.03635$, and $Z=0.023422$ :

$$
\begin{aligned}
& \frac{6}{(1+0.0210+0.023422)^{1}}+\frac{106}{(1+0.03635+0.023422)^{2}} \\
& =\frac{6}{(1.044422)^{1}}+\frac{106}{(1.059772)^{2}}=100.125
\end{aligned}
$$

\section{SUMMARY}
We have covered the principles and techniques that are used in the valuation of fixed-rate bonds, as well as floating-rate notes and money market instruments. These building blocks are used extensively in fixed-income analysis. The following are the main points made:

\begin{itemize}
  \item The market discount rate is the rate of return required by investors given the risk of the investment in the bond.

  \item A bond is priced at a premium above par value when the coupon rate is greater than the market discount rate.

  \item A bond is priced at a discount below par value when the coupon rate is less than the market discount rate.

  \item The amount of any premium or discount is the present value of the "excess" or "deficiency" in the coupon payments relative to the yield-to-maturity.

  \item The yield-to-maturity, the internal rate of return on the cash flows, is the implied market discount rate given the price of the bond.

  \item A bond price moves inversely with its market discount rate.

  \item The relationship between a bond price and its market discount rate is convex.

  \item The price of a lower-coupon bond is more volatile than the price of a higher-coupon bond, other things being equal.

  \item Generally, the price of a longer-term bond is more volatile than the price of a shorter-term bond, other things being equal. An exception to this phenomenon can occur on low-coupon (but not zero-coupon) bonds that are priced at a discount to par value.

  \item Assuming no default, premium and discount bond prices are "pulled to par" as maturity nears.

  \item A spot rate is the yield-to-maturity on a zero-coupon bond.

  \item A yield-to-maturity can be approximated as a weighted average of the underlying spot rates.

  \item Between coupon dates, the full (or invoice, or "dirty") price of a bond is split between the flat (or quoted, or "clean") price and the accrued interest.

  \item Flat prices are quoted to not misrepresent the daily increase in the full price as a result of interest accruals.

  \item Accrued interest is calculated as a proportional share of the next coupon payment using either the actual/actual or 30/360 methods to count days.

  \item Matrix pricing is used to value illiquid bonds by using prices and yields on comparable securities having the same or similar credit risk, coupon rate, and maturity.

  \item The periodicity of an annual interest rate is the number of periods in the year.

  \item A yield quoted on a semiannual bond basis is an annual rate for a periodicity of two. It is the yield per semiannual period times two.

  \item The general rule for periodicity conversions is that compounding more frequently at a lower annual rate corresponds to compounding less frequently at a higher annual rate.

  \item Street convention yields assume payments are made on scheduled dates, neglecting weekends and holidays. - The current yield is the annual coupon payment divided by the flat price, thereby neglecting as a measure of the investor's rate of return the time value of money, any accrued interest, and the gain from buying at a discount or the loss from buying at a premium.

  \item The simple yield is like the current yield but includes the straight-line amortization of the discount or premium.

  \item The yield-to-worst on a callable bond is the lowest of the yield-to-first-call, yield-to-second-call, and so on, calculated using the call price for the future value and the call date for the number of periods.

  \item The option-adjusted yield on a callable bond is the yield-to-maturity after adding the theoretical value of the call option to the price.

  \item A floating-rate note (floater, or FRN) maintains a more stable price than a fixed-rate note because interest payments adjust for changes in market interest rates.

  \item The quoted margin on a floater is typically the specified yield spread over or under the reference rate, which we refer to as the Market Reference Rate.

  \item The discount margin on a floater is the spread required by investors, and to which the quoted margin must be set, for the FRN to trade at par value on a rate reset date.

  \item Money market instruments, having one year or less time-to-maturity, are quoted on a discount rate or add-on rate basis.

  \item Money market discount rates understate the investor's rate of return (and the borrower's cost of funds) because the interest income is divided by the face value or the total amount redeemed at maturity, and not by the amount of the investment.

  \item Money market instruments need to be converted to a common basis for analysis.

  \item A money market bond equivalent yield is an add-on rate for a 365-day year.

  \item The periodicity of a money market instrument is the number of days in the year divided by the number of days to maturity. Therefore, money market instruments with different times-to-maturity have annual rates for different periodicities.

  \item In theory, the maturity structure, or term structure, of interest rates is the relationship between yields-to-maturity and times-to-maturity on bonds having the same currency, credit risk, liquidity, tax status, and periodicity.

  \item A spot curve is a series of yields-to-maturity on zero-coupon bonds.

  \item A frequently used yield curve is a series of yields-to-maturity on coupon bonds.

  \item A par curve is a series of yields-to-maturity assuming the bonds are priced at par value.

  \item In a cash market, the delivery of the security and cash payment is made on a settlement date within a customary time period after the trade date-for example, " $T+3 . "$

  \item In a forward market, the delivery of the security and cash payment are made on a predetermined future date.

  \item A forward rate is the interest rate on a bond or money market instrument traded in a forward market. - An implied forward rate (or forward yield) is the breakeven reinvestment rate linking the return on an investment in a shorter-term zero-coupon bond to the return on an investment in a longer-term zero-coupon bond.

  \item An implied forward curve can be calculated from the spot curve.

  \item Implied spot rates can be calculated as geometric averages of forward rates.

  \item A fixed-income bond can be valued using a market discount rate, a series of spot rates, or a series of forward rates.

  \item A bond yield-to-maturity can be separated into a benchmark and a spread.

  \item Changes in benchmark rates capture macroeconomic factors that affect all bonds in the market-inflation, economic growth, foreign exchange rates, and monetary and fiscal policy.

  \item Changes in spreads typically capture microeconomic factors that affect the particular bond-credit risk, liquidity, and tax effects.

  \item Benchmark rates are usually yields-to-maturity on government bonds or fixed rates on interest rate swaps.

  \item A G-spread is the spread over or under a government bond rate, and an I-spread is the spread over or under an interest rate swap rate.

  \item A G-spread or an I-spread can be based on a specific benchmark rate or on a rate interpolated from the benchmark yield curve.

  \item A Z-spread (zero-volatility spread) is based on the entire benchmark spot curve. It is the constant spread that is added to each spot rate such that the present value of the cash flows matches the price of the bond.

  \item An option-adjusted spread (OAS) on a callable bond is the Z-spread minus the theoretical value of the embedded call option.

\end{itemize}

\section{PRACTICE PROBLEMS}
\begin{enumerate}
  \item A portfolio manager is considering the purchase of a bond with a $5.5 \%$ coupon rate that pays interest annually and matures in three years. If the required rate of return on the bond is $5 \%$, the price of the bond per 100 of par value is closest to:
A. 98.65
B. 101.36 .
C. 106.43

  \item A bond with two years remaining until maturity offers a $3 \%$ coupon rate with interest paid annually. At a market discount rate of $4 \%$, the price of this bond per 100 of par value is closest to:
A. 95.34 .
B. 98.00
C. 98.11 .

  \item An investor who owns a bond with a $9 \%$ coupon rate that pays interest semiannually and matures in three years is considering its sale. If the required rate of return on the bond is $11 \%$, the price of the bond per 100 of par value is closest to:
A. 95.00 .
B. 95.11
C. 105.15

  \item A bond offers an annual coupon rate of $4 \%$, with interest paid semiannually. The bond matures in two years. At a market discount rate of $6 \%$, the price of this bond per 100 of par value is closest to:
A. 93.07 .
B. 96.28
C. 96.33

  \item A bond offers an annual coupon rate of $5 \%$, with interest paid semiannually. The bond matures in seven years. At a market discount rate of $3 \%$, the price of this bond per 100 of par value is closest to:
A. 106.60
B. 112.54
C. 143.90

  \item A zero-coupon bond matures in 15 years. At a market discount rate of $4.5 \%$ per year and assuming annual compounding, the price of the bond per 100 of par value is closest to:
A. 51.30 .
B. 51.67 C. 71.62 .

  \item Consider the following two bonds that pay interest annually:

\end{enumerate}

\begin{center}
\begin{tabular}{ccc}
\hline
Bond & Coupon Rate & Time-to-Maturity \\
\hline
A & $5 \%$ & 2 years \\
B & $3 \%$ & 2 years \\
\hline
\end{tabular}
\end{center}

At a market discount rate of $4 \%$, the price difference between Bond $\mathrm{A}$ and Bond $\mathrm{B}$ per 100 of par value is closest to:
A. 3.70 .
B. 3.77 .
C. 4.00 .

\section{The following information relates to questions}
 8-9\begin{center}
\begin{tabular}{cccc}
\hline
Bond & Price & Coupon Rate & Time-to-Maturity \\
\hline
A & 101.886 & $5 \%$ & 2 years \\
B & 100.000 & $6 \%$ & 2 years \\
C & 97.327 & $5 \%$ & 3 years \\
\hline
\end{tabular}
\end{center}

\begin{enumerate}
  \setcounter{enumi}{7}
  \item Which bond offers the lowest yield-to-maturity?
A. Bond A
B. Bond B
C. Bond C

  \item Which bond will most likely experience the smallest percent change in price if the market discount rates for all three bonds increase by $100 \mathrm{bps}$ ?
A. Bond A
B. Bond B
C. Bond C

  \item Suppose a bond's price is expected to increase by $5 \%$ if its market discount rate decreases by $100 \mathrm{bps}$. If the bond's market discount rate increases by $100 \mathrm{bps}$, the bond price is most likely to change by:
A. $5 \%$.
B. less than $5 \%$.
C. more than $5 \%$.

\end{enumerate}

\section{The following information relates to questions}
\section{$11-12$}
\begin{center}
\begin{tabular}{ccc}
\hline
Bond & Coupon Rate & Maturity (years) \\
\hline
A & $6 \%$ & 10 \\
B & $6 \%$ & 5 \\
C & $8 \%$ & 5 \\
\hline
\end{tabular}
\end{center}

All three bonds are currently trading at par value.

\begin{enumerate}
  \setcounter{enumi}{10}
  \item Relative to Bond C, for a $200 \mathrm{bp}$ decrease in the required rate of return, Bond $B$ will most likely exhibit a(n):
A. equal percentage price change.
B. greater percentage price change.
C. smaller percentage price change.

  \item Which bond will most likely experience the greatest percentage change in price if the market discount rates for all three bonds increase by 100 bps?
A. Bond A
B. Bond B
C. Bond C

  \item An investor considers the purchase of a two-year bond with a $5 \%$ coupon rate, with interest paid annually. Assuming the sequence of spot rates shown below, the price of the bond is closest to:

\end{enumerate}

\begin{center}
\begin{tabular}{cc}
\hline
Time-to-Maturity & Spot Rates \\
\hline
1 year & $3 \%$ \\
2 years & $4 \%$ \\
\hline
\end{tabular}
\end{center}

A. 101.93 .
B. 102.85 .
C. 105.81 .

\begin{enumerate}
  \setcounter{enumi}{13}
  \item A three-year bond offers a $10 \%$ coupon rate with interest paid annually. Assuming the following sequence of spot rates, the price of the bond is closest to:
\end{enumerate}

\begin{center}
\begin{tabular}{cc}
\hline
Time-to-Maturity & Spot Rates \\
\hline
1 year & $8.0 \%$ \\
2 years & $9.0 \%$ \\
3 years & $9.5 \%$ \\
\hline
\end{tabular}
\end{center}

A. 96.98 . B. 101.46 .

C. 102.95 .

\section{The following information relates to questions}
 15-17\begin{center}
\begin{tabular}{ccccc}
\hline
Bond & Coupon Rate & Time-to-Maturity & Time-to-Maturity & Spot Rates \\
\hline
$\mathrm{X}$ & $8 \%$ & 3 years & 1 year & $8 \%$ \\
$\mathrm{Y}$ & $7 \%$ & 3 years & 2 years & $9 \%$ \\
$\mathrm{Z}$ & $6 \%$ & 3 years & 3 years & $10 \%$ \\
\hline
\end{tabular}
\end{center}

All three bonds pay interest annually.

\begin{enumerate}
  \setcounter{enumi}{14}
  \item Based on the given sequence of spot rates, the price of Bond $\mathrm{X}$ is closest to:
A. 95.02 .
B. 95.28 .
C. 97.63 .

  \item Based on the given sequence of spot rates, the price of Bond $\mathrm{Y}$ is closest to:
A. 87.50 .
B. 92.54 .
C. 92.76 .

  \item Based on the given sequence of spot rates, the yield-to-maturity of Bond $\mathrm{Z}$ is closest to:
A. $9.00 \%$.
B. $9.92 \%$.
C. $11.93 \%$.

  \item Bond dealers most often quote the:
A. flat price.
B. full price.
C. full price plus accrued interest.

\end{enumerate}

The following information relates to questions 19-21

Bond G, described in the exhibit below, is sold for settlement on 16 June 2020.

$\begin{array}{ll}\text { Annual Coupon } & 5 \% \\ \text { Coupon Payment Frequency } & \text { Semiannual } \\ \text { Interest Payment Dates } & 10 \text { April and } 10 \text { October } \\ \text { Maturity Date } & 10 \text { October } 2022 \\ \text { Day-Count Convention } & 30 / 360 \\ \text { Annual Yield-to-Maturity } & 4 \%\end{array}$

\begin{enumerate}
  \setcounter{enumi}{18}
  \item The full price that Bond G settles at on 16 June 2020 is closest to:
A. 102.36.
B. 103.10 .
C. 103.65 .

  \item The accrued interest per 100 of par value for Bond $\mathrm{G}$ on the settlement date of 16 June 2020 is closest to:
A. 0.46 .
B. 0.73 .
C. 0.92 .

  \item The flat price for Bond G on the settlement date of 16 June 2020 is closest to:
A. 102.18 .
B. 103.10 .
C. 104.02 .

  \item Matrix pricing allows investors to estimate market discount rates and prices for bonds:
A. with different coupon rates.
B. that are not actively traded.
C. with different credit quality.

  \item When underwriting new corporate bonds, matrix pricing is used to get an estimate of the:
A. required yield spread over the benchmark rate.
B. market discount rate of other comparable corporate bonds.
C. yield-to-maturity on a government bond having a similar time-to-maturity.

  \item A bond with 20 years remaining until maturity is currently trading for 111 per 100 of par value. The bond offers a $5 \%$ coupon rate with interest paid semiannually. The bond's annual yield-to-maturity is closest to:
A. $2.09 \%$.
B. $4.18 \%$. C. $4.50 \%$.

  \item The annual yield-to-maturity, stated for with a periodicity of 12 , for a four-year, zero-coupon bond priced at 75 per 100 of par value is closest to:
A. $6.25 \%$.
B. $7.21 \%$.
C. $7.46 \%$.

  \item A five-year, 5\% semiannual coupon payment corporate bond is priced at 104.967 per 100 of par value. The bond's yield-to-maturity, quoted on a semiannual bond basis, is $3.897 \%$. An analyst has been asked to convert to a monthly periodicity. Under this conversion, the yield-to-maturity is closest to:
A. $3.87 \%$.
B. $4.95 \%$.
C. $7.67 \%$.

\end{enumerate}

\section{The following information relates to questions}
 27-30A bond with five years remaining until maturity is currently trading for 101 per 100 of par value. The bond offers a $6 \%$ coupon rate with interest paid semiannually. The bond is first callable in three years and is callable after that date on coupon dates according to the following schedule:

\begin{center}
\begin{tabular}{cc}
\hline
End of Year & Call Price \\
\hline
3 & 102 \\
4 & 101 \\
5 & 100 \\
\hline
\end{tabular}
\end{center}

\begin{enumerate}
  \setcounter{enumi}{26}
  \item The bond's annual yield-to-maturity is closest to:
A. $2.88 \%$.
B. $5.77 \%$.
C. $5.94 \%$.

  \item The bond's annual yield-to-first-call is closest to:
A. $3.12 \%$.
B. $6.11 \%$.
C. $6.25 \%$.

  \item The bond's annual yield-to-second-call is closest to:
A. $2.97 \%$.
B. $5.72 \%$.
C. $5.94 \%$.

  \item The bond's yield-to-worst is closest to:
A. $2.88 \%$.
B. $5.77 \%$.
C. $6.25 \%$.

  \item A two-year floating-rate note pays six-month Libor plus $80 \mathrm{bps}$. The floater is priced at 97 per 100 of par value. The current six-month MRR is $1.00 \%$. Assume a $30 / 360$ day-count convention and evenly spaced periods. The discount margin for the floater in basis points is closest to:
A. $180 \mathrm{bps}$.
B. $236 \mathrm{bps}$.
C. $420 \mathrm{bps}$.

  \item An analyst evaluates the following information relating to floating-rate notes (FRNs) issued at par value that have three-month MRR as a reference rate:

\end{enumerate}

\begin{center}
\begin{tabular}{ccc}
\hline
Floating-Rate Note & Quoted Margin & Discount Margin \\
\hline
X & $0.40 \%$ & $0.32 \%$ \\
Y & $0.45 \%$ & $0.45 \%$ \\
$\mathrm{Z}$ & $0.55 \%$ & $0.72 \%$ \\
\hline
\end{tabular}
\end{center}

Based only on the information provided, the FRN that will be priced at a premium on the next reset date is:
A. FRN X
B. FRN Y.
C. FRN Z

\begin{enumerate}
  \setcounter{enumi}{32}
  \item A 365-day year bank certificate of deposit has an initial principal amount of USD96.5 million and a redemption amount due at maturity of USD100 million. The number of days between settlement and maturity is 350 . The bond equivalent yield is closest to:
A. $3.48 \%$.
B. $3.65 \%$.
C. $3.78 \%$.

  \item The bond equivalent yield of a 180-day banker's acceptance quoted at a discount rate of $4.25 \%$ for a 360 -day year is closest to:
A. $4.31 \%$.
B. $4.34 \%$.
C. $4.40 \%$. 35. Which of the following statements describing a par curve is incorrect?

\end{enumerate}

A. A par curve is obtained from a spot curve.

B. All bonds on a par curve are assumed to have different credit risk.

C. A par curve is a sequence of yields-to-maturity such that each bond is priced at par value.

\begin{enumerate}
  \setcounter{enumi}{35}
  \item A yield curve constructed from a sequence of yields-to-maturity on zero-coupon bonds is the:
A. par curve.
B. spot curve.
C. forward curve.

  \item The rate interpreted to be the incremental return for extending the time-to-maturity of an investment for an additional time period is the:
A. add-on rate.
B. forward rate.
C. yield-to-maturity.

\end{enumerate}

The following information relates to questions 38-39

\begin{center}
\begin{tabular}{cc}
\hline
Time Period & Forward Rate \\
\hline
"0y1y" & $0.80 \%$ \\
"1y1y" & $1.12 \%$ \\
"2y1y" & $3.94 \%$ \\
“3y1y" & $3.28 \%$ \\
"4y1y" & $3.14 \%$ \\
\hline
\end{tabular}
\end{center}

All rates are annual rates stated for a periodicity of one (effective annual rates).

\begin{enumerate}
  \setcounter{enumi}{37}
  \item The three-year implied spot rate is closest to:
A. $1.18 \%$.
B. $1.94 \%$.
C. $2.28 \%$.

  \item The value per 100 of par value of a two-year, $3.5 \%$ coupon bond with interest payments paid annually is closest to:
A. 101.58 .
B. 105.01 . C. 105.82 .

  \item The spread component of a specific bond's yield-to-maturity is least likely impacted by changes in:
A. its tax status.
B. its quality rating.
C. inflation in its currency of denomination.

  \item The yield spread of a specific bond over the standard swap rate in that currency of the same tenor is best described as the:
A. I-spread.
B. Z-spread.
C. G-spread.

\end{enumerate}

\section{The following information relates to questions}
 42-42\begin{center}
\begin{tabular}{lccc}
\hline
Bond & Coupon Rate & Time-to-Maturity & Price \\
\hline
UK Government Benchmark Bond & $2 \%$ & 3 years & 100.25 \\
UK Corporate Bond & $5 \%$ & 3 years & 100.65 \\
\hline
\end{tabular}
\end{center}

Both bonds pay interest annually. The current three-year EUR interest rate swap benchmark is $2.12 \%$.

\begin{enumerate}
  \setcounter{enumi}{41}
  \item The G-spread in basis points on the UK corporate bond is closest to:
A. $264 \mathrm{bps}$.
B. 285 bps.
C. $300 \mathrm{bps}$.

  \item A corporate bond offers a $5 \%$ coupon rate and has exactly three years remaining to maturity. Interest is paid annually. The following rates are from the benchmark spot curve:

\end{enumerate}

\begin{center}
\begin{tabular}{cc}
\hline
Time-to-Maturity & Spot Rate \\
\hline
1 year & $4.86 \%$ \\
2 years & $4.95 \%$ \\
3 years & $5.65 \%$ \\
\hline
\end{tabular}
\end{center}

The bond is currently trading at a Z-spread of $234 \mathrm{bps}$. The value of the bond is closest to:
A. 92.38
B. 98.35 . C. 106.56 .

\begin{enumerate}
  \setcounter{enumi}{43}
  \item An option-adjusted spread (OAS) on a callable bond is the Z-spread:
\end{enumerate}

A. over the benchmark spot curve.

B. minus the standard swap rate in that currency of the same tenor.

C. minus the value of the embedded call option expressed in basis points per year.

\section{SOLUTIONS}
\begin{enumerate}
  \item B is correct. The bond price is closest to 101.36. The price is determined in the following manner:
\end{enumerate}

$P V=\frac{P M T}{(1+r)^{1}}+\frac{P M T}{(1+r)^{2}}+\frac{P M T+F V}{(1+r)^{3}}$

where

$P V=$ present value, or the price of the bond

$P M T=$ coupon payment per period

$F V=$ future value paid at maturity, or the par value of the bond

$r=$ market discount rate, or required rate of return per period

$P V=\frac{5.5}{(1+0.05)^{1}}+\frac{5.5}{(1+0.05)^{2}}+\frac{5.5+100}{(1+0.05)^{3}}$.

$P V=5.24+4.99+91.13=101.36$.

\begin{enumerate}
  \setcounter{enumi}{1}
  \item $\mathrm{C}$ is correct. The bond price is closest to 98.11 . The formula for calculating the price of this bond is
\end{enumerate}

$P V=\frac{P M T}{(1+r)^{1}}+\frac{P M T+F V}{(1+r)^{2}}$

where

$P V=$ present value, or the price of the bond

$P M T=$ coupon payment per period

$F V=$ future value paid at maturity, or the par value of the bond

$r=$ market discount rate, or required rate of return per period

$P V=\frac{3}{(1+0.04)^{1}}+\frac{3+100}{(1+0.04)^{2}}=2.88+95.23=98.11$

\begin{enumerate}
  \setcounter{enumi}{2}
  \item A is correct. The bond price is closest to 95.00 . The bond has six semiannual periods. Half of the annual coupon is paid in each period with the required rate of return also being halved. The price is determined in the following manner:
\end{enumerate}

$P V=\frac{P M T}{(1+r)^{1}}+\frac{P M T}{(1+r)^{2}}+\frac{P M T}{(1+r)^{3}}+\frac{P M T}{(1+r)^{4}}+\frac{P M T}{(1+r)^{5}}+\frac{P M T+F V}{(1+r)^{6}}$

where

$P V=$ present value, or the price of the bond

$P M T=$ coupon payment per period

$F V=$ future value paid at maturity, or the par value of the bond

$r=$ market discount rate, or required rate of return per period

$P V=\frac{4.5}{(1+0.055)^{1}}+\frac{4.5}{(1+0.055)^{2}}+\frac{4.5}{(1+0.055)^{3}}+\frac{4.5}{(1+0.055)^{4}}+\frac{4.5}{(1+0.055)^{5}}+\frac{4.5+100}{(1+0.055)^{6}}$

$$
P V=4.27+4.04+3.83+3.63+3.44+75.79=95.00
$$

\begin{enumerate}
  \setcounter{enumi}{3}
  \item B is correct. The bond price is closest to 96.28 . The formula for calculating this bond price is
\end{enumerate}

$P V=\frac{P M T}{(1+r)^{1}}+\frac{P M T}{(1+r)^{2}}+\frac{P M T}{(1+r)^{3}}+\frac{P M T+F V}{(1+r)^{4}}$

where

$P V=$ present value, or the price of the bond

$P M T=$ coupon payment per period

$F V=$ future value paid at maturity, or the par value of the bond

$r=$ market discount rate, or required rate of return per period

$P V=\frac{2}{(1+0.03)^{1}}+\frac{2}{(1+0.03)^{2}}+\frac{2}{(1+0.03)^{3}}+\frac{2+100}{(1+0.03)^{4}}$

$P V=1.94+1.89+1.83+90.62=96.28$

\begin{enumerate}
  \setcounter{enumi}{4}
  \item B is correct. The bond price is closest to 112.54 . The formula for calculating this bond price is
\end{enumerate}

$P V=\frac{P M T}{(1+r)^{1}}+\frac{P M T}{(1+r)^{2}}+\frac{P M T}{(1+r)^{3}}+\cdots+\frac{P M T+F V}{(1+r)^{14}}$

where

$P V=$ present value, or the price of the bond

$P M T=$ coupon payment per period

$F V=$ future value paid at maturity, or the par value of the bond

$r=$ market discount rate, or required rate of return per period

$P V=\frac{2.5}{(1+0.015)^{1}}+\frac{2.5}{(1+0.015)^{2}}+\frac{2.5}{(1+0.015)^{3}}+\cdots+\frac{2.5}{(1+0.015)^{13}}+\frac{2.5+100}{(1+0.015)^{14}}$

$P V=2.46+2.43+2.39+\ldots+2.06+83.21=112.54$

\begin{enumerate}
  \setcounter{enumi}{5}
  \item B is correct. The price of the zero-coupon bond is closest to 51.67. The price is determined in the following manner:
\end{enumerate}

$P V=\frac{100}{(1+r)^{N}}$

where

$P V=$ present value, or the price of the bond

$r=$ market discount rate, or required rate of return per period

$N=$ number of evenly spaced periods to maturity

$P V=\frac{100}{(1+0.045)^{15}}$

$P V=51.67$

\begin{enumerate}
  \setcounter{enumi}{6}
  \item B is correct. The price difference between Bonds A and B is closest to 3.77. One method for calculating the price difference between two bonds with identical terms to maturity is to use the following formula:
\end{enumerate}

$P V=\frac{P M T}{(1+r)^{1}}+\frac{P M T}{(1+r)^{2}}$

where

$P V=$ price difference

$P M T=$ coupon difference per period

$r=$ market discount rate, or required rate of return per period

In this case, the coupon difference is $(5 \%-3 \%)$, or $2 \%$.

$P V=\frac{2}{(1+0.04)^{1}}+\frac{2}{(1+0.04)^{2}}=1.92+1.85=3.77$

\begin{enumerate}
  \setcounter{enumi}{7}
  \item A is correct. Bond A offers the lowest yield-to-maturity. When a bond is priced at a premium above par value, the yield-to-maturity (YTM), or market discount rate, is less than the coupon rate. Bond $\mathrm{A}$ is priced at a premium, so its YTM is below its $5 \%$ coupon rate. Bond B is priced at par value, so its YTM is equal to its $6 \%$ coupon rate. Bond $C$ is priced at a discount below par value, so its YTM is above its $5 \%$ coupon rate.

  \item B is correct. Bond B will most likely experience the smallest percentage change in price if market discount rates increase by $100 \mathrm{bps}$. A higher-coupon bond has a smaller percentage price change than a lower-coupon bond when their market discount rates change by the same amount (the coupon effect). Also, a shorter-term bond generally has a smaller percentage price change than a longer-term bond when their market discount rates change by the same amount (the maturity effect). Bond $B$ will experience a smaller percentage change in price than Bond A because of the coupon effect. Bond B will also experience a smaller percentage change in price than Bond $C$ because of the coupon effect and the maturity effect.

  \item B is correct. The bond price is most likely to change by less than $5 \%$. The relationship between bond prices and market discount rate is not linear. The percentage price change is greater in absolute value when the market discount rate goes down than when it goes up by the same amount (the convexity effect). If a $100 \mathrm{bp}$ decrease in the market discount rate will cause the price of the bond to increase by $5 \%$, then a $100 \mathrm{bp}$ increase in the market discount rate will cause the price of the bond to decline by an amount less than $5 \%$.

  \item B is correct. Generally, for two bonds with the same time-to-maturity, a lower-coupon bond will experience a greater percentage price change than a higher-coupon bond when their market discount rates change by the same amount. Bond $B$ and Bond $C$ have the same time-to-maturity (five years); however, Bond B offers a lower coupon rate. Therefore, Bond B will likely experience a greater percentage change in price in comparison to Bond C.

  \item A is correct. Bond A will likely experience the greatest percentage change in price due to the coupon effect and the maturity effect. For two bonds with the same time-to-maturity, a lower-coupon bond has a greater percentage price change than a higher-coupon bond when their market discount rates change by the same amount. Generally, for the same coupon rate, a longer-term bond has a greater percentage price change than a shorter-term bond when their market discount rates change by the same amount. Relative to Bond $\mathrm{C}$, Bond $\mathrm{A}$ and Bond $\mathrm{B}$ offer a lower coupon rate of $6 \%$; however, Bond $A$ has a longer time-to-maturity than Bond B. Therefore, Bond A will likely experience the greater percentage change in price if the market discount rates for all three bonds increase by $100 \mathrm{bps}$.

  \item A is correct. The bond price is closest to 101.93 . The price is determined in the following manner:

\end{enumerate}

$P V=\frac{P M T}{\left(1+Z_{1}\right)^{1}}+\frac{P M T+F V}{\left(1+Z_{2}\right)^{2}}$

where

$P V=$ present value, or the price of the bond

$P M T=$ coupon payment per period

$F V=$ future value paid at maturity, or the par value of the bond

$Z_{1}=$ spot rate, or the zero-coupon yield, for Period 1

$Z_{2}=$ spot rate, or the zero-coupon yield, for Period 2

$P V=\frac{5}{(1+0.03)^{1}}+\frac{5+100}{(1+0.04)^{2}}$

$P V=4.85+97.08=101.93$.

\begin{enumerate}
  \setcounter{enumi}{13}
  \item B is correct. The bond price is closest to 101.46 . The price is determined in the following manner:
\end{enumerate}

$P V=\frac{P M T}{\left(1+Z_{1}\right)^{1}}+\frac{P M T}{\left(1+Z_{2}\right)^{2}}+\frac{P M T+F V}{\left(1+Z_{3}\right)^{3}}$,

where

$P V=$ present value, or the price of the bond

$P M T=$ coupon payment per period

$F V=$ future value paid at maturity, or the par value of the bond

$Z_{1}=$ spot rate, or the zero-coupon yield or zero rate, for Period 1

$Z_{2}=$ spot rate, or the zero-coupon yield or zero rate, for Period 2

$Z_{3}=$ spot rate, or the zero-coupon yield or zero rate, for Period 3

$P V=\frac{10}{(1+0.08)^{1}}+\frac{10}{(1+0.09)^{2}}+\frac{10+100}{(1+0.095)^{3}}$.

$P V=9.26+8.42+83.78=101.46$.

\begin{enumerate}
  \setcounter{enumi}{14}
  \item B is correct. The bond price is closest to 95.28 . The formula for calculating this bond price is $P V=\frac{P M T}{\left(1+Z_{1}\right)^{1}}+\frac{P M T}{\left(1+Z_{2}\right)^{2}}+\frac{P M T+F V}{\left(1+Z_{3}\right)^{3}}$
\end{enumerate}

where

$P V=$ present value, or the price of the bond

$P M T=$ coupon payment per period

$F V=$ future value paid at maturity, or the par value of the bond

$Z_{1}=$ spot rate, or the zero-coupon yield or zero rate, for Period 1

$Z_{2}=$ spot rate, or the zero-coupon yield or zero rate, for Period 2

$Z_{3}=$ spot rate, or the zero-coupon yield or zero rate, for Period 3

$P V=\frac{8}{(1+0.08)^{1}}+\frac{8}{(1+0.09)^{2}}+\frac{8+100}{(1+0.10)^{3}}$

$P V=7.41+6.73+81.14=95.28$

\begin{enumerate}
  \setcounter{enumi}{15}
  \item $C$ is correct. The bond price is closest to 92.76 . The formula for calculating this bond price is
\end{enumerate}

$P V=\frac{P M T}{\left(1+Z_{1}\right)^{1}}+\frac{P M T}{\left(1+Z_{2}\right)^{2}}+\frac{P M T+F V}{\left(1+Z_{3}\right)^{3}}$

where

$P V=$ present value, or the price of the bond

$P M T=$ coupon payment per period

$F V=$ future value paid at maturity, or the par value of the bond

$Z_{1}=$ spot rate, or the zero-coupon yield or zero rate, for Period 1

$Z_{2}=$ spot rate, or the zero-coupon yield or zero rate, for Period 2

$Z_{3}=$ spot rate, or the zero-coupon yield or zero rate, for Period 3

$P V=\frac{7}{(1+0.08)^{1}}+\frac{7}{(1+0.09)^{2}}+\frac{7+100}{(1+0.10)^{3}}$

$P V=6.48+5.89+80.39=92.76$

\begin{enumerate}
  \setcounter{enumi}{16}
  \item B is correct. The yield-to-maturity is closest to $9.92 \%$. The formula for calculating the price of Bond $\mathrm{Z}$ is $P V=\frac{P M T}{\left(1+Z_{1}\right)^{1}}+\frac{P M T}{\left(1+Z_{2}\right)^{2}}+\frac{P M T+F V}{\left(1+Z_{3}\right)^{3}}$
\end{enumerate}

where

$P V=$ present value, or the price of the bond

$P M T=$ coupon payment per period

$F V=$ future value paid at maturity, or the par value of the bond

$Z_{1}=$ spot rate, or the zero-coupon yield or zero rate, for Period 1

$Z_{2}=$ spot rate, or the zero-coupon yield or zero rate, for Period 2

$Z_{3}=$ spot rate, or the zero-coupon yield or zero rate, for Period 3

$P V=\frac{6}{(1+0.08)^{1}}+\frac{6}{(1+0.09)^{2}}+\frac{6+100}{(1+0.10)^{3}}$

$P V=5.56+5.05+79.64=90.25$

Using this price, the bond's yield-to-maturity can be calculated as

$P V=\frac{P M T}{(1+r)^{1}}+\frac{P M T}{(1+r)^{2}}+\frac{P M T+F V}{(1+r)^{3}}$

$90.25=\frac{6}{(1+r)^{1}}+\frac{6}{(1+r)^{2}}+\frac{6+100}{(1+r)^{3}}$

$r=9.92 \%$

\begin{enumerate}
  \setcounter{enumi}{17}
  \item A is correct. Bond dealers usually quote the flat price. When a trade takes place, the accrued interest is added to the flat price to obtain the full price paid by the buyer and received by the seller on the settlement date. The reason for using the flat price for quotation is to avoid misleading investors about the market price trend for the bond. If the full price were to be quoted by dealers, investors would see the price rise day after day even if the yield-to-maturity did not change. That is because the amount of accrued interest increases each day. After the coupon payment is made, the quoted price would drop dramatically. Using the flat price for quotation avoids that misrepresentation. The full price, flat price plus accrued interest, is not usually quoted by bond dealers. Accrued interest is included in the full price, and bond dealers do not generally quote the full price.

  \item B is correct. The bond's full price is 103.10 . The price is determined in the following manner:

\end{enumerate}

As of the beginning of the coupon period on 10 April 2020, there are 2.5 years (five semiannual periods) to maturity. These five semiannual periods occur on 10 October 2020, 10 April 2021, 10 October 2021, 10 April 2022, and 10 October 2022.

$P V=\frac{P M T}{(1+r)^{1}}+\frac{P M T}{(1+r)^{2}}+\frac{P M T}{(1+r)^{3}}+\frac{P M T}{(1+r)^{4}}+\frac{P M T+F V}{(1+r)^{5}}$

where

$P V=$ present value

$P M T=$ coupon payment per period

$F V=$ future value paid at maturity, or the par value of the bond

$r=$ market discount rate, or required rate of return per period $P V=\frac{2.5}{(1+0.02)^{1}}+\frac{2.5}{(1+0.02)^{2}}+\frac{2.5}{(1+0.02)^{3}}+\frac{2.5}{(1+0.02)^{4}}+\frac{2.5+100}{(1+0.02)^{5}}$

$P V=2.45+2.40+2.36+2.31+92.84=102.36$

The accrued interest period is identified as $66 / 180$. The number of days between 10 April 2020 and 16 June 2020 is 66 days, based on the 30/360 day-count convention (20 days remaining in April + 30 days in May +16 days in June $=66$ days total). The number of days between coupon periods is assumed to be 180 days using the 30/360 day convention.

$$
\begin{aligned}
& P V^{\text {Full }}=P V \times(1+r)^{66 / 180} . \\
& P V^{F u l l}=102.36 \times(1.02)^{66 / 180}=103.10 .
\end{aligned}
$$

\begin{enumerate}
  \setcounter{enumi}{19}
  \item $\mathrm{C}$ is correct. The accrued interest per 100 of par value is closest to 0.92 . The accrued interest is determined in the following manner: The accrued interest period is identified as 66/180. The number of days between 10 April 2020 and 16 June 2020 is 66 days, based on the $30 / 360$ day-count convention (20 days remaining in April + 30 days in May +16 days in June $=66$ days total). The number of days between coupon periods is assumed to be 180 days using the $30 / 360$ day convention.
\end{enumerate}

Accrued interest $=\frac{t}{T} \times P M T$

where

$t=$ number of days from the last coupon payment to the settlement date

$T=$ number of days in the coupon period

$t / T=$ fraction of the coupon period that has gone by since the last payment

$P M T=$ coupon payment per period

Accrued interest $=\frac{66}{180} \times \frac{5.00}{2}=0.92$

\begin{enumerate}
  \setcounter{enumi}{20}
  \item A is correct. The flat price of 102.18 is determined by subtracting the accrued interest (from Question 20) from the full price (from Question 19).
\end{enumerate}

$P V^{\text {Flat }}=P V^{\text {Full }}-$ Accrued interest.

$P V^{\text {Flat }}=103.10-0.92=102.18$

\begin{enumerate}
  \setcounter{enumi}{21}
  \item B is correct. For bonds not actively traded or not yet issued, matrix pricing is a price estimation process that uses market discount rates based on the quoted prices of similar bonds (similar times-to-maturity, coupon rates, and credit quality).

  \item A is correct. Matrix pricing is used in underwriting new bonds to get an estimate of the required yield spread over the benchmark rate. The benchmark rate is typically the yield-to-maturity on a government bond having the same or close to the same time-to-maturity. The spread is the difference between the yield-to-maturity on the new bond and the benchmark rate. The yield spread is the additional compensation required by investors for the difference in the credit risk, liquidity risk, and tax status of the bond relative to the government bond.

\end{enumerate}

In matrix pricing, the market discount rates of comparable bonds and the yield-to-maturity on a government bond having a similar time-to-maturity are not estimated. Rather, they are known and are used to estimate the required yield spread of a new bond.

\begin{enumerate}
  \setcounter{enumi}{23}
  \item B is correct. The formula for calculating this bond's yield-to-maturity is $P V=\frac{P M T}{(1+r)^{1}}+\frac{P M T}{(1+r)^{2}}+\frac{P M T}{(1+r)^{3}}+\cdots+\frac{P M T}{(1+r)^{39}}+\frac{P M T+F V}{(1+r)^{40}}$, where
\end{enumerate}

$P V=$ present value, or the price of the bond

$P M T=$ coupon payment per period

$F V=$ future value paid at maturity, or the par value of the bond

$r=$ market discount rate, or required rate of return per period

$111=\frac{2.5}{(1+r)^{1}}+\frac{2.5}{(1+r)^{2}}+\frac{2.5}{(1+r)^{3}}+\cdots+\frac{2.5}{(1+r)^{39}}+\frac{2.5+100}{(1+r)^{40}}$.

$r=0.0209$.

To arrive at the annualized yield-to-maturity, the semiannual rate of $2.09 \%$ must be multiplied by two. Therefore, the yield-to-maturity is equal to $2.09 \% \times 2=$ $4.18 \%$.

\begin{enumerate}
  \setcounter{enumi}{24}
  \item B is correct. The annual yield-to-maturity, stated for a periodicity of 12 , is $7.21 \%$. It is calculated as follows:
\end{enumerate}

$$
\begin{aligned}
& P V=\frac{F V}{(1+r)^{N}} . \\
& 75=\left(\frac{100}{(1+r)^{4 \times 12}}\right) . \\
& \frac{100}{75}=(1+r)^{48} . \\
& 1.33333=(1+r)^{48} . \\
& (1.33333)^{1 / 48}=\left[(1+r)^{48}\right]^{1 / 48} . \\
& 1.33333^{02083}=(1+r) . \\
& 1.00601=(1+r) . \\
& 1.00601-1=r . \\
& 0.00601=r . \\
& r \times 12=0.07212, \text { or approximately } 7.21 \% .
\end{aligned}
$$

\begin{enumerate}
  \setcounter{enumi}{25}
  \item A is correct. The yield-to-maturity, stated for a periodicity of 12 (monthly periodicity), is $3.87 \%$. The formula to convert an annual percentage rate (annual yield-to-maturity) from one periodicity to another is as follows:
\end{enumerate}

$$
\begin{aligned}
& \left(1+\frac{A P R_{m}}{m}\right)^{m}=\left(1+\frac{A P R_{n}}{n}\right)^{n} . \\
& \left(1+\frac{0.03897}{2}\right)^{2}=\left(1+\frac{A P R_{12}}{12}\right)^{12} . \\
& (1.01949)^{2}=\left(1+\frac{A P R_{12}}{12}\right)^{12} .
\end{aligned}
$$

$$
\begin{aligned}
& 1.03935=\left(1+\frac{A P R_{12}}{12}\right)^{12} \cdot \\
& (1.03935)^{1 / 12}=\left[\left(1+\frac{A P R_{12}}{12}\right)^{12}\right]^{1 / 12} \cdot \\
& 1.00322=\left(1+\frac{A P R_{12}}{12}\right) . \\
& 1.00322-1=\left(\frac{A P R_{12}}{12}\right) . \\
& A P R_{12}=0.00322 \times 12=0.03865, \text { or approximately } 3.87 \% .
\end{aligned}
$$

\begin{enumerate}
  \setcounter{enumi}{26}
  \item B is correct. The yield-to-maturity is $5.77 \%$. The formula for calculating this bond's yield-to-maturity is
\end{enumerate}

$P V=\frac{P M T}{(1+r)^{1}}+\frac{P M T}{(1+r)^{2}}+\frac{P M T}{(1+r)^{3}}+\cdots+\frac{P M T}{(1+r)^{9}}+\frac{P M T+F V}{(1+r)^{10}}$

where

$P V=$ present value, or the price of the bond

$P M T=$ coupon payment per period

$F V=$ future value paid at maturity, or the par value of the bond

$r=$ market discount rate, or required rate of return per period

$101=\frac{3}{(1+r)^{1}}+\frac{3}{(1+r)^{2}}+\frac{3}{(1+r)^{3}}+\cdots+\frac{3}{(1+r)^{9}}+\frac{3+100}{(1+r)^{10}}$

$r=0.02883$

To arrive at the annualized yield-to-maturity, the semiannual rate of $2.883 \%$ must be multiplied by two. Therefore, the yield-to-maturity is equal to $2.883 \% \times 2=$ $5.77 \%$ (rounded).

\begin{enumerate}
  \setcounter{enumi}{27}
  \item $\mathrm{C}$ is correct. The yield-to-first-call is $6.25 \%$. Given the first call date is exactly three years away, the formula for calculating this bond's yield-to-first-call is
\end{enumerate}

$P V=\frac{P M T}{(1+r)^{1}}+\frac{P M T}{(1+r)^{2}}+\frac{P M T}{(1+r)^{3}}+\cdots+\frac{P M T}{(1+r)^{5}}+\frac{P M T+F V}{(1+r)^{6}}$,

where

$P V=$ present value, or the price of the bond

$P M T=$ coupon payment per period

$F V=$ call price paid at call date

$r=$ market discount rate, or required rate of return per period

$101=\frac{3}{(1+r)^{1}}+\frac{3}{(1+r)^{2}}+\frac{3}{(1+r)^{3}}+\cdots+\frac{3}{(1+r)^{5}}+\frac{3+102}{(1+r)^{6}}$.

$r=0.03123$

To arrive at the annualized yield-to-first-call, the semiannual rate of $3.123 \%$ must be multiplied by two. Therefore, the yield-to-first-call is equal to $3.123 \% \times 2=$ $6.25 \%$ (rounded). 29. $\mathrm{C}$ is correct. The yield-to-second-call is $5.94 \%$. Given the second call date is exactly four years away, the formula for calculating this bond's yield-to-second-call is

$P V=\frac{P M T}{(1+r)^{1}}+\frac{P M T}{(1+r)^{2}}+\frac{P M T}{(1+r)^{3}}+\cdots+\frac{P M T}{(1+r)^{7}}+\frac{P M T+F V}{(1+r)^{8}}$,

where

$P V=$ present value, or the price of the bond

$P M T=$ coupon payment per period

$F V=$ call price paid at call date

$r=$ market discount rate, or required rate of return per period

$101=\frac{3}{(1+r)^{1}}+\frac{3}{(1+r)^{2}}+\frac{3}{(1+r)^{3}}+\cdots \frac{3}{(1+r)^{7}}+\frac{3+101}{(1+r)^{8}}$.

$r=0.0297$.

To arrive at the annualized yield-to-second-call, the semiannual rate of 2.97\% must be multiplied by two. Therefore, the yield-to-second-call is equal to $2.97 \% \times$ $2=5.94 \%$.

\begin{enumerate}
  \setcounter{enumi}{29}
  \item B is correct. The yield-to-worst is $5.77 \%$. The bond's yield-to-worst is the lowest of the sequence of yields-to-call and the yield-to-maturity. From Questions 27 -29 , we have the following yield measures for this bond:
\end{enumerate}

Yield-to-first-call: $6.25 \%$

Yield-to-second-call: 5.94\%

Yield-to-maturity: $5.77 \%$

Thus, the yield-to-worst is $5.77 \%$.

\begin{enumerate}
  \setcounter{enumi}{30}
  \item B is correct. The discount or required margin is $236 \mathrm{bps}$. Given the floater has a maturity of two years and is linked to six-month MRR, the formula for calculating the discount margin is
\end{enumerate}

$P V=\frac{\frac{(\text { Index }+Q M) \times F V}{m}}{\left(1+\frac{\text { Index }+D M}{m}\right)^{1}}+\frac{\frac{(\text { Index }+Q M) \times F V}{m}}{\left(1+\frac{\operatorname{Index}+D M}{m}\right)^{2}}+\cdots+\frac{\frac{(\text { Index }+Q M) \times F V}{m}+F V}{\left(1+\frac{\operatorname{Index}+D M}{m}\right)^{4}}$,

where

$P V=$ present value, or the price of the floating-rate note $=97$

Index $=$ reference rate, stated as an annual percentage rate $=0.01$

$Q M=$ quoted margin, stated as an annual percentage rate $=0.0080$

$F V=$ future value paid at maturity, or the par value of the bond $=100$

$m=$ periodicity of the floating-rate note, the number of payment periods per year $=2$

$D M=$ discount margin, the required margin stated as an annual percentage rate Substituting the given values:

$97=\frac{\frac{(0.01+0.0080) \times 100}{2}}{\left(1+\frac{0.01+D M}{2}\right)^{1}}+\frac{\frac{(0.01+0.0080) \times 100}{2}}{\left(1+\frac{0.01+D M}{2}\right)^{2}}+\cdots+\frac{\frac{(0.01+0.0080) \times 100}{2}+100}{\left(1+\frac{0.01+D M}{2}\right)^{4}}$. $97=\frac{0.90}{\left(1+\frac{0.01+D M}{2}\right)^{1}}+\frac{0.90}{\left(1+\frac{0.01+D M}{2}\right)^{2}}+\frac{0.90}{\left(1+\frac{0.01+D M}{2}\right)^{3}}+\frac{0.90+100}{\left(1+\frac{0.01+D M}{2}\right)^{4}}$

To calculate $D M$, begin by solving for the discount rate per period:

$97=\frac{0.90}{(1+r)^{1}}+\frac{0.90}{(1+r)^{2}}+\frac{0.90}{(1+r)^{3}}+\frac{0.90+100}{(1+r)^{4}}$.

$r=0.0168$

Now, solve for $D M$ :

$\frac{0.01+D M}{2}=0.0168$

$D M=0.0236$

The discount margin for the floater is equal to 236 bps.

\begin{enumerate}
  \setcounter{enumi}{31}
  \item A is correct. FRN $X$ will be priced at a premium on the next reset date because the quoted margin of $0.40 \%$ is greater than the discount, or required, margin of $0.32 \%$. The premium amount is the present value of the extra, or "excess," interest payments of $0.08 \%$ each quarter $(0.40 \%-0.32 \%)$. FRN Y will be priced at par value on the next reset date since there is no difference between the quoted and discount margins. FRN Z will be priced at a discount since the quoted margin is less than the required margin.

  \item $\mathrm{C}$ is correct. The bond equivalent yield is closest to $3.78 \%$. It is calculated as $A O R=\left(\frac{\text { Year }}{\text { Days }}\right) \times\left(\frac{F V-P V}{P V}\right)$

\end{enumerate}

where

$P V=$ present value, principal amount, or the price of the money market instrument

$F V=$ future value, or the redemption amount paid at maturity including interest

Days $=$ number of days between settlement and maturity

Year $=$ number of days in the year

$A O R=$ add-on rate, stated as an annual percentage rate (also called bond equivalent yield)

$A O R=\left(\frac{365}{350}\right) \times\left(\frac{100-96.5}{96.5}\right)$

$A O R=1.04286 \times 0.03627$

$A O R=0.03783$, or approximately $3.78 \%$

\begin{enumerate}
  \setcounter{enumi}{33}
  \item $C$ is correct. The bond equivalent yield is closest to $4.40 \%$. The present value of the banker's acceptance is calculated as $P V=F V \times\left(1-\frac{\text { Days }}{\text { Year }} \times D R\right)$
\end{enumerate}

where

$P V=$ present value, or price of the money market instrument

$F V=$ future value paid at maturity, or face value of the money market instrument

Days $=$ number of days between settlement and maturity

Year $=$ number of days in the year

$D R=$ discount rate, stated as an annual percentage rate

$P V=100 \times\left(1-\frac{\text { Days }}{\text { Year }} \times D R\right)$

$P V=100 \times\left(1-\frac{180}{360} \times 0.0425\right)$.

$P V=100 \times(1-0.02125)$

$P V=100 \times 0.97875$.

$P V=97.875$.

The bond equivalent yield (AOR) is calculated as

$A O R=\left(\frac{\text { Year }}{\text { Days }}\right) \times\left(\frac{F V-P V}{P V}\right)$,

where

$P V=$ present value, principal amount, or the price of the money market instrument $F V=$ future value, or the redemption amount paid at maturity including interest

Days $=$ number of days between settlement and maturity

Year $=$ number of days in the year

$A O R=$ add-on rate (bond equivalent yield), stated as an annual percentage rate

$A O R=\left(\frac{365}{180}\right) \times\left(\frac{100-P V}{P V}\right)$

$A O R=\left(\frac{365}{180}\right) \times\left(\frac{100-97.875}{97.875}\right)$.

$A O R=2.02778 \times 0.02171$

$A O R=0.04402$, or approximately $4.40 \%$.

Note that $P V$ is calculated using an assumed 360-day year and $A O R$ (bond equivalent yield) is calculated using a 365-day year.

\begin{enumerate}
  \setcounter{enumi}{34}
  \item B is correct. All bonds on a par curve are assumed to have similar, not different, credit risk. Par curves are obtained from spot curves, and all bonds used to derive the par curve are assumed to have the same credit risk, as well as the same periodicity, currency, liquidity, tax status, and annual yields. A par curve is a sequence of yields-to-maturity such that each bond is priced at par value.

  \item B is correct. The spot curve, also known as the strip, or zero, curve, is the yield curve constructed from a sequence of yields-to-maturity on zero-coupon bonds. The par curve is a sequence of yields-to-maturity such that each bond is priced at par value. The forward curve is constructed using a series of forward rates, each having the same time frame.

  \item B is correct. The forward rate can be interpreted to be the incremental or marginal return for extending the time-to-maturity of an investment for an additional time period. The add-on rate (bond equivalent yield) is a rate quoted for money market instruments, such as bank certificates of deposit, and indexes, such as MRR, Libor and Euribor. Yield-to-maturity is the internal rate of return on the bond's cash flows-the uniform interest rate such that when the bond's future cash flows are discounted at that rate, the sum of the present values equals the price of the bond. It is the implied market discount rate.

  \item B is correct. The three-year implied spot rate is closest to $1.94 \%$. It is calculated as the geometric average of the one-year forward rates:
$(1.0080 \times 1.0112 \times 1.0394)=\left(1+z_{3}\right)^{3}$.
$1.05945=\left(1+z_{3}\right)^{3}$.
$(1.05945)^{1 / 3}=\left[\left(1+z_{3}\right)^{3}\right]^{1 / 3}$.
$1.01944=1+z_{3}$
$1.01944-1=z_{3}$.
$0.01944=z_{3}$.
$z_{3}=1.944 \%$, or approximately $1.94 \%$.

  \item B is correct. The value per 100 of par value is closest to 105.01 . Using the forward curve, the bond price is calculated as follows:

\end{enumerate}

$\frac{3.5}{1.0080}+\frac{103.5}{(1.0080 \times 1.0112)}=3.47+101.54=105.01$.

\begin{enumerate}
  \setcounter{enumi}{39}
  \item C is correct. The spread component of a specific bond's yield-to-maturity is least likely impacted by changes in inflation in its currency of denomination. The effect of changes in macroeconomic factors, such as the expected rate of inflation in the currency of denomination, is seen mostly in changes in the benchmark yield. The spread or risk premium component is impacted by microeconomic factors specific to the bond and bond issuer, including tax status and quality rating.

  \item A is correct. The I-spread, or interpolated spread, is the yield spread of a specific bond over the standard swap rate in that currency of the same tenor. The yield spread in basis points over an actual or interpolated government bond is known as the G-spread. The Z-spread (zero-volatility spread) is the constant spread that is added to each spot rate such that the present value of the cash flows matches the price of the bond.

  \item B is correct. The G-spread is closest to 285 bps. The benchmark rate for UK fixed-rate bonds is the UK government benchmark bond. The euro interest rate spread benchmark is used to calculate the G-spread for euro-denominated corporate bonds, not UK bonds. The G-spread is calculated as follows:

\end{enumerate}

Yield-to-maturity on the UK corporate bond:

$100.65=\frac{5}{(1+r)^{1}}+\frac{5}{(1+r)^{2}}+\frac{105}{(1+r)^{3}} ; r=0.04762$, or $476 \mathrm{bps}$.

Yield-to-maturity on the UK government benchmark bond: $100.25=\frac{2}{(1+r)^{1}}+\frac{2}{(1+r)^{2}}+\frac{102}{(1+r)^{3}} ; r=0.01913$, or 191 bps.

The G-spread is $476-191=285 \mathrm{bps}$.

\begin{enumerate}
  \setcounter{enumi}{42}
  \item A is correct. The value of the bond is closest to 92.38 . The calculation is
\end{enumerate}

$$
\begin{aligned}
& P V=\frac{P M T}{\left(1+z_{1}+Z\right)^{1}}+\frac{P M T}{\left(1+z_{2}+Z\right)^{2}}+\frac{P M T+F V}{\left(1+z_{3}+Z\right)^{3}} \\
= & \frac{5}{(1+0.0486+0.0234)^{1}}+\frac{5}{(1+0.0495+0.0234)^{2}}+\frac{105}{(1+0.0565+0.0234)^{3}} \\
= & \frac{5}{1.0720}+\frac{5}{1.15111}+\frac{105}{1.25936}=4.66+4.34+83.38=92.38 .
\end{aligned}
$$

\begin{enumerate}
  \setcounter{enumi}{43}
  \item $C$ is correct. The option value in basis points per year is subtracted from the Z-spread to calculate the OAS. The Z-spread is the constant yield spread over the benchmark spot curve. The I-spread is the yield spread of a specific bond over the standard swap rate in that currency of the same tenor.
\end{enumerate}

\section*{LEARNING MODULE 
 4 }
\section{Introduction to Asset-Backed Securities}
Frank J. Fabozzi, PhD, CPA, CFA, is at EDHEC Business School (France).

\section{LEARNING OUTCOME}
\begin{center}
\begin{tabular}{c|l}
Mastery & The candidate should be able to: \\
\hline
$\square$ & $\begin{array}{l}\text { explain benefits of securitization for economies and financial } \\ \text { markets } \\ \text { describe securitization, including the parties involved in the process } \\ \text { and the roles they play } \\ \text { describe typical structures of securitizations, including credit } \\ \text { tranching and time tranching } \\ \text { describe types and characteristics of residential mortgage loans that } \\ \text { are typically securitized } \\ \text { describe types and characteristics of residential mortgage-backed } \\ \text { securities, including mortgage pass-through securities and } \\ \text { collateralized mortgage obligations, and explain the cash flows and } \\ \text { risks for each type } \\ \text { define prepayment risk and describe the prepayment risk of } \\ \text { mortgage-backed securities } \\ \text { describe characteristics and risks of commercial mortgage-backed } \\ \text { securities } \\ \text { describe types and characteristics of non-mortgage asset-backed } \\ \text { securities, including the cash flows and risks of each type } \\ \text { describe collateralized debt obligations, including their cash flows } \\ \text { and risks } \\ \text { describe characteristics and risks of covered bonds and how they } \\ \text { differ from other asset-backed securities }\end{array}$ \\
$\square$ &  \\
\end{tabular}
\end{center}$\square$

\section{INTRODUCTION: BENEFITS OF SECURITIZATION}
Previous readings examined the risk characteristics of various fixed-income instruments and the relationships between maturity, coupon, and interest rate changes. This reading introduces fixed-income instruments created through a process known as securitization. The securitization process transfers ownership of assets such as loans or receivables from the original owners into a special legal entity. The special legal entity then issues securities, using the asset cash flows to pay interest and repay the principal to investors. These securities are referred to generically as asset-backed securities (ABS), and the pool of assets from which their cash flows are generated is called collateral or securitized assets. These loans and receivables typically include residential mortgage loans (mortgages), commercial mortgages, automobile (auto) loans, student loans, bank loans, accounts receivable, or credit card receivables. While the ABS market in the United States remains the largest in the world, securitization has recently expanded in both Asia and Europe as well as to other income-yielding assets, such as airport landing slots, toll roads, and cell tower leases. The growing depth and breadth of the global ABS market underscores the importance of a solid understanding of securitization among issuers, investors, and financial analysts.

This reading describes the securitization process, discusses its benefits for both issuers and investors, and explains the investment characteristics of different types of ABS. Although investors can invest in some loan types via private credit funds or through secondary markets, securitization creates a more direct link between investors and borrowers for many types of loans and receivables. The terminology regarding ABS varies by jurisdiction. Mortgage-backed securities (MBS) are ABS backed by a pool of mortgages, and a distinction is sometimes made between MBS and ABS backed by non-mortgage assets. This distinction is common in the United States, for example, where typically the term "mortgage-backed securities" refers to securities backed by high-quality real estate mortgages and the term "asset-backed securities" refers to securities backed by other types of assets. Covered bonds date back to 18th century Europe and are similar to ABS, but they offer investors recourse to both the issuing financial institution and an underlying asset pool.

To underline the importance of securitization from a macroeconomic perspective, we discuss the benefits of securitization for economies and financial markets. Then, we describe the structure of a securitization, identifying the parties involved and their roles as well as typical structures. We further discuss securities backed by mortgages for real estate property, including residential MBS and commercial MBS, and review two common types of non-mortgage ABS-namely, those for auto loans and credit card receivables. The reading concludes with a description of collateralized debt obligations and covered bonds.

\section{Benefits of Securitization for Economies and Financial Markets}
The securitization of pools of loans and receivables into multiple securities provides economies and financial markets with several benefits.

Home or auto purchases have been traditionally financed by loans originated by financial institutions, such as commercial banks. For investors to gain exposure to these relatively illiquid loans, they must hold some combination of deposits, debt, or common equity issued by banks. This situation creates an additional intermediary (that is, the bank) between the borrowers and the investors. In addition, by being constrained to hold bank deposits and securities, investors cannot gain direct exposure to loans but are also affected by economic risks undertaken in other bank activities.

Securitization solves a number of these problems. It allows investors to achieve more direct legal claims on loan and receivables portfolios, enabling them to tailor interest rate and credit risk exposures to suit their specific needs. Disintermediation (that is, lessening the role of intermediaries) can effectively reduce borrower costs and enhance risk-adjusted investor returns. At the same time, banks can separate loan origination from financing, improving their profitability via origination fees and reducing capital requirements for loans that are sold. Securitization enables banks to expand lending origination beyond their balance sheets, ultimately benefiting individuals, governments, and companies that need to borrow.

Securitization also benefits investors by creating access to securities with profiles that match their risk, return, and maturity needs that are otherwise not directly available. For example, a pension fund with a long-term horizon can gain access to long-term real estate loans by investing in residential MBS without having to invest in bank bonds or stocks. Although few institutional or individual investors are willing to make or purchase real estate loans, auto loans, or credit card receivables directly, they may invest in a security backed by such loans or receivables. The ABS that are created by pooling these loans and receivables have characteristics similar to those of a standard bond and do not require the specialized resources and expertise needed to originate, monitor, and collect the payments from the underlying loans and receivables. As a result, investors can increase exposure to the risk-return characteristics of a wider range of underlying assets. Note that in many countries, the sale of $\mathrm{ABS}$ and similar instruments is restricted to investors who meet certain qualifications, such as those pertaining to net worth.

Securitization allows for the creation of tradable securities with better liquidity than that of the original loans on the bank's balance sheet. In making loans and receivables tradable, securitization makes financial markets more efficient. It also improves liquidity, which reduces liquidity risk in the financial system, as described later.

An important benefit of securitization for companies is that ABS provide an alternative means of funding operations that can be considered alongside bond, preferred equity, and common equity issuance. Companies that originate loans and receivables that can be securitized often compare and optimize the funding costs associated with each source of financing. As we will see, securitization is often less costly than a corporate bond issue secured by the same collateral.

For these reasons, securitization is beneficial to economies and financial markets and has been embraced by many sovereign governments. For example, the Italian government has used securitization since the late 1990s for privatizing public assets. In emerging markets, securitization is widely used. For example, in South America, companies and banks with high credit ratings have used securitization to sell receivables on exports, such as oil, to lower their funding costs.

Although securitization brings many benefits to economies, it is not without risks, and some of these risks are widely attributed to have precipitated the turmoil in financial markets during 2007-2009. Broadly, those risks fall into two categories: risks that relate primarily to the timing of the ABS's cash flows, such as contraction risk and extension risk, and risks related to the inherent credit risk of the loans and receivables backing the ABS. This reading describes these risks and discusses some of the structures used to mitigate them as well as redistribute them.

\section{HOW SECURITIZATION WORKS}
describe securitization, including the parties involved in the process and the roles they play

When assets are securitized, several legal and regulatory conditions must be satisfied. A number of parties participate in the process to facilitate the transaction and ensure these conditions are met. In this section, a typical securitization is described by way of a hypothetical example. The example describes the parties involved in a securitization and their roles. It also introduces the typical structures of securitizations, such as credit tranching and time tranching.

\section{An Example of a Securitization}
Mediquip, a hypothetical company, is a manufacturer of medical equipment that ranges in cost from US $\$ 50,000$ to US $\$ 300,000$. The majority of Mediquip's sales are made through loans granted by the company to its customers, and the medical equipment serves as collateral for the loans. These loans, which represent an asset to Mediquip, have maturities of five years and carry a fixed interest rate. They are fully amortizing with monthly payments; that is, the borrowers make equal payments each month consisting of interest payment and principal repayment. The total principal repaid from the 60 loan payments ( 12 months $\times 5$ years) is such that the amount borrowed is fully repaid at the end of the term.

Mediquip's credit department makes the decision about whether to extend credit to customers and services the loans that are made. Loan servicing refers to administering any aspect of a loan, including collecting payments from borrowers, notifying borrowers who may be delinquent, and recovering and disposing of the medical equipment if the borrower does not make the scheduled payments by a specified time. If one of its customers defaults, Mediquip can seize the medical equipment and sell it to try to recoup the remaining principal on the loan. Although the servicer of such loans need not be the originator of the loans, the assumption in this example is that Mediquip is the servicer.

The following is an illustration of how these loans can be securitized. Assume that Mediquip has US 200 million of loans. This amount is shown on Mediquip's balance sheet as an asset. Assume also that Mediquip wants to raise US\$200 million, which happens to be the amount of the loans. Because Mediquip's treasurer is aware of the potentially lower funding costs of securitization, he decides to raise the US $\$ 200$ million by securitizing the loans on the medical equipment rather than by issuing corporate bonds.

To do so, Mediquip sets up a separate legal entity called Medical Equipment Trust (MET), to which it sells the loans on the medical equipment. Such a legal entity is referred to as a special purpose entity (SPE) and sometimes also called a special purpose vehicle $(\mathrm{SPV})$ or a special purpose company. The legal form of the SPE varies by jurisdiction, but in almost all cases, the ultimate owner of the loans-MET in our example-is legally independent and is considered bankruptcy remote from the seller of the loans. Setting up a separate legal entity ensures that if Mediquip, the originator of the loans, files for bankruptcy, the loans backing the ABS that are issued by MET are secure within the SPE and creditors of Mediquip have no claim on them. Note that in some jurisdictions, the SPE may, in turn, transfer the loans to a trust or a limited company.

A securitization is diagramed in Exhibit 1. The SPE set up by Mediquip is called MET. The top of Exhibit 1 reflects Mediquip's business model as described abovethat is, the sale of medical equipment financed by loans (first oval). Mediquip sells to MET US\$200 million of loans (second oval) and receives from MET US\$200 million in cash (third oval); in this simplified example, the costs associated with the securitization are ignored. MET issues and sells securities that are backed by the pool of securitized loans (fourth oval) and receives cash (fifth oval). These securities are the ABS mentioned earlier, and the US 200 million of loans represent the collateral. The periodic cash payments that are received from the collateral-that is, the monthly payments made by Mediquip's customers that include both interest payment and principal repayment (sixth oval)-are used to make the periodic cash payments to the security holders-the investors who bought the ABS (seventh oval).

\section{Exhibit 1: Mediquip's Securitization}
\begin{center}
\includegraphics[max width=\textwidth]{2023_05_04_7b535d0a870224f62e3dg-627}
\end{center}

\section{Parties to a Securitization and Their Roles}
Securitization requires the publication of a prospectus, a document that contains information about the securitization. The three main parties to a securitization are

\begin{itemize}
  \item the seller of the collateral, sometimes called the depositor (Mediquip in our example);

  \item the SPE that purchases the loans or receivables and uses them as collateral to issue the ABS-MET in our example (the SPE is often referred to as the issuer in the prospectus because it is the entity that issues the securities; it may also be called the trust if the SPE is set up as a trust); and

  \item the servicer of the loans (Mediquip in our example).

\end{itemize}

Other parties are also involved in a securitization: independent accountants, lawyers/attorneys, trustees, underwriters, rating agencies, and financial guarantors. All these parties, including the servicer when it is different from the seller of the collateral, are referred to as third parties to the securitization.

A significant amount of legal documentation is involved in a securitization. Lawyers/ attorneys are responsible for preparing the legal documents. An important legal document is the purchase agreement between the seller of the collateral and the SPE, which sets forth the representations and warranties that the seller makes about the assets sold. These representations and warranties assure investors about the quality of the assets, an important consideration when assessing the risks associated with the ABS.

Another important legal document is the prospectus, which describes the structure of the securitization, including the priority and amount of payments to be made to the servicer, administrators, and the ABS holders. Securitizations often use several forms of credit enhancements, which are documented in the prospectus. Credit enhancements are provisions that are used to reduce the credit risk of a bond issue. They include (1) internal credit enhancements, such as subordination, overcollateralization, and reserve accounts, and (2) external credit enhancements, such as financial guarantees by banks or insurance companies, letters of credit, and cash collateral accounts. Securitizations often use subordination, which will be discussed later. Prior to the 2007-09 credit crisis, many securitizations included financial guarantees by a third party. The most common third-party financial guarantors are monoline insurance companies, or monoline insurers. A monoline insurer is a private insurance company whose business is restricted to providing guarantees for financial instruments such as ABS. Following the financial difficulties and downgrading of the major monoline insurers as a result of the financial crisis that began in the mortgage market in mid-2007, few structures have used financial guarantees from a monoline insurer.

A trustee or trustee agent is typically a financial institution with trust powers that safeguards the assets after they have been sold to the SPE, holds the funds due to the ABS holders until they are paid, and provides periodic information to the ABS holders. The information is provided in the form of remittance reports, which may be issued monthly, quarterly, or as agreed to in the terms of the prospectus.

Underwriters and rating agencies perform the same functions in a securitization as they do in a standard bond offering

\section{EXAMPLE 1}
\section{A Used Luxury Auto Securitization}
Used Luxury Auto (ULA) is a hypothetical company that has established a nationwide business in buying used luxury autos and then refurbishing them with the latest in electronic equipment (for instance, UBS ports and rear-view cameras). ULA Corp then sells these autos in the retail market, often financing the sales with promissory notes from the buyers via its ULA Credit Corp.

The following information is taken from a theoretical filing by ULA with the Securities and Exchange Commission for a securitization:

Issuer: ULA Trust 2020

Seller and Servicer: ULA Credit Corp

Notes:

$\$ 500,000,0004.00 \%$ ULA Trust contract-backed Class A notes, rated AAA

$\$ 250,000,0004.80 \%$ ULA Trust contract-backed Class B notes, rated A

Contracts: The assets underlying the notes are fixed-rate promissory notes relating to the purchase of used automobiles refurbished by ULA Corp.

\begin{enumerate}
  \item The collateral for this securitization is:
\end{enumerate}

A. ULA Trust contract-backed Class A and Class B notes.

B. Used automobiles refurbished by ULA Corp. C. Fixed-rate promissory notes relating to the purchase of used automobiles refurbished by ULA Corp.

\section{Solution:}
$\mathrm{C}$ is correct. The collateral is the pool of securitized assets from which the cash flows will be generated. It is the debt obligations that have been securitized. These contracts are the loans, called promissory notes, provided to purchasers of the used automobiles that were refurbished by ULA Corp.

\begin{enumerate}
  \setcounter{enumi}{1}
  \item The special purpose entity in this securitization is:
A. ULA Corp.
B. ULA Credit Corp.
C. ULA Trust 2020 .
\end{enumerate}

\section{Solution:}
C is correct. ULA Trust 2020 is the issuer of the ABS and, thus, the SPE. The SPE purchases the contracts that become the collateral from ULA Corp, the automobile refurbisher, but it is ULA Credit Corp that originates the loans and is, therefore, the seller of the collateral. ULA Credit Corp is also the servicer of the debt obligations.

\begin{enumerate}
  \setcounter{enumi}{2}
  \item ULA Credit Corp is responsible for:
\end{enumerate}

A. selling the collateral to the SPE and collecting payments from borrowers on the underlying promissory notes.

B. refurbishing the used motorcycles and collecting payments from borrowers on the underlying promissory notes.

C. selling the contract-backed Class A and Class B notes to investors and making the cash interest and principal payments to them.

\section{Solution:}
A is correct. ULA Credit Corp is the seller of the collateral, the promissory notes. As the servicer, it is responsible for collecting payments from borrowers, notifying borrowers who may be delinquent, and if necessary, recovering and disposing of the automobile if the borrower defaults.

\section{STRUCTURE OF A SECURITIZATION}
describe typical structures of securitizations, including credit tranching and time tranching

A simple securitization may involve the sale of only one class of bond or ABS. Let us call this class Bond Class A. Returning to the Mediquip and MET example, MET may raise US $\$ 200$ million by issuing 200,000 certificates for Bond Class A with a par value of US $\$ 1,000$ per certificate. Thus, each certificate holder is entitled to $1 / 200,000$ of the payments from the collateral after payment of servicing and other administrative fees. The structure of the securitization is often more complicated than a single class of ABS. As mentioned earlier, it is common for securitizations to include a form of internal credit enhancement called subordination, also referred to as credit tranching. In such a structure, there is more than one bond class or tranche, and the bond classes differ as to how they will share any losses resulting from defaults of the borrowers whose loans are in the collateral. The bond classes are classified as senior bond classes or subordinated bond classes-hence, the reason this structure is also referred to as a senior/subordinated structure. The subordinated bond classes are sometimes called "non-senior bond classes" or "junior bond classes." They function as credit protection for the more senior bond classes; that is, losses are realized by the subordinated bond classes before any losses are realized by the senior bond classes. This type of protection is also commonly referred to as a "waterfall" structure because of the cascading flow of payments between bond classes in the event of default.

For example, suppose MET issues two bond classes with a total par value of US $\$ 200$ million: Bond Class A, the senior bond class, with a par value of US $\$ 120$ million, and Bond Class B, the subordinated bond class, with a par value of US $\$ 80$ million. In this senior/subordinated structure, also referred to as credit tranching, Bond Class B will absorb losses up to US $\$ 80$ million. Thus, if defaults by Mediquip's customers do not exceed US $\$ 80$ million, Bond Class A will be fully repaid its US $\$ 120$ million. The purpose of this structure is to redistribute the credit risk associated with the collateral. The creation of a set of bond classes allows investors to choose the level of credit risk that they prefer to bear.

More than one subordinated bond class may be created. Suppose MET issues the following structure:

\begin{center}
\begin{tabular}{lc}
\hline
Bond Class & Par Value (US\$ millions) \\
\hline
A (senior) & 180 \\
B (subordinated) & 14 \\
C (subordinated) & 6 \\
\hline
Total & 200 \\
\hline
\end{tabular}
\end{center}

In this structure, Bond Class $A$ is the senior bond class whereas both Bond Class B and Bond Class $C$ are subordinated bond classes from the perspective of Bond Class A. The rules for the distribution of losses are as follows. All losses on the collateral are absorbed by Bond Class C before any losses are realized by Bond Class B and then Bond Class A. Consequently, if the losses on the collateral do not exceed US\$6 million, no losses will be realized by Bond Class A or Bond Class B. If the losses exceed US\$6 million, Bond Class B must absorb the losses up to an additional US 14 million. For example, if the total loss on the collateral is US 16 million, Bond Class C loses its entire par value of US\$6 million and Bond Class B realizes a loss of US\$10 million of its par value of US\$14 million. Bond Class A does not realize any loss in this scenario. Clearly, Bond Class A realizes a loss only if the total loss on the collateral exceeds US 20 million.

The structure of a securitization may also allow the redistribution of another type of risk, called "prepayment risk," among bond classes. Prepayment risk is the uncertainty that the cash flows will be different from the scheduled cash flows as set forth in the loan agreement because of the borrowers' ability to alter payments, usually to take advantage of interest rate movements. For example, when interest rates decline, borrowers tend to pay off part or all of their loans and refinance at lower interest rates. The creation of bond classes that possess different expected maturities is referred to as time tranching and is discussed later in this reading.

It is possible, and quite common, for a securitization to have structures with both credit tranching and time tranching.

\section{EXAMPLE 2}
\section{Bond Classes and Tranching}
\begin{enumerate}
  \item Return to the ULA securitization described in Example 1. Based on the information provided, the form of credit enhancement that the transaction most likely includes is:
A. time tranching.
B. credit tranching.
C. a financial guarantee.
\end{enumerate}

\section{Solution:}
$\mathrm{B}$ is correct. The ULA securitization includes two bond classes: Class A and Class B. Each bond class has a fixed but different interest rate; the interest rate increases from $4.00 \%$ for Class A notes to $4.80 \%$ for Class B notes. Thus, it is likely that the transaction has credit tranching and that the two bond classes display a senior/subordinated structure, with Class A notes being senior to Class B notes, the subordinated bond class. As the credit risk increases from the Class A notes to the Class B notes, so does the interest rate to reflect the additional compensation investors require for bearing the additional credit risk. The information provided does not give any indication of either time tranching or a financial guarantee.

\section{Key Role of the Special Purpose Entity}
The SPE plays a pivotal role in securitization. In fact, the setup of a legal entity that plays the same role as an SPE in terms of protecting the rights of ABS holders is a prerequisite in any country that wants to allow securitization. Indeed, without a provision in a country's legal system for the equivalent of an SPE, the benefits of using securitization by an entity seeking to raise funds would not exist. Let us explain why by returning to our example involving Mediquip and MET.

Assume that Mediquip has a credit rating from a credit-rating agency, such as Standard \& Poor's, Moody's Investors Service, or Fitch Ratings. A credit rating reflects the agency's opinion about the creditworthiness of an entity and/or the debt securities the entity issues. Suppose that the credit rating assigned to Mediquip is BB or $\mathrm{Ba} 2$. Such a credit rating means that Mediquip is below what is referred to as an investment-grade credit rating.

Assume again that Mediquip's treasurer wants to raise US 200 million and is contemplating doing so by issuing a five-year corporate bond rather than by securitizing the loans. The treasurer is, of course, concerned about the funding cost and would like the lowest possible interest rate available relative to some benchmark interest rate. The difference between the interest rate paid on the five-year corporate bond and the benchmark interest rate is the spread. The spread reflects the compensation investors require for buying the corporate bond, which is riskier than the bonds issued at the benchmark interest rate. The major factor affecting the spread is the issuer's credit rating-hence, the reason the spread is called a "credit spread."

Another factor that will influence the credit spread is whether the bond is backed by collateral. A corporate bond that has collateral is often referred to as a secured bond. The collateral usually reduces the credit spread, making the credit spread of a secured bond lower than that of an otherwise identical unsecured bond. In our example, Mediquip's treasurer can use the loans on the medical equipment as collateral for the secured corporate bond issue. Thus, if Mediquip issues a five-year corporate bond to raise US $\$ 200$ million, the credit spread will primarily reflect its credit rating, with a slight benefit for the use of collateral as explained below.

Now suppose that instead of using the loans as collateral for a secured corporate bond issue, Mediquip sells the loan contracts in an arm's length transaction to MET, the SPE. After the sale is completed, it is MET, not Mediquip, that legally owns them. As a result, if Mediquip is forced into bankruptcy while the loans are still outstanding, Mediquip's creditors cannot recover them because they are legally owned by another entity. Note that it is possible, however, that transfers made to bankruptcy-remote vehicles can be challenged as fraudulent conveyances and potentially unwound. The legal implication of setting up MET is that investors contemplating the purchase of any bond class backed by the cash flows from the pool of loans on the medical equipment will evaluate the credit risk associated with collecting the payments due on the receivables independently of Mediquip's credit rating.

Credit ratings are assigned to each bond class created in the securitization. They depend on the quality of the collateral-that is, how the rating agencies evaluate the credit risk of the pool of securitized loans or receivables. Depending on the structure of the securitization, each bond class receives a credit rating that reflects its credit risk, and some of the bond classes may have a better credit rating than the company that is seeking to raise funds. As a result, in aggregate, the funding cost of a securitization may be lower than that of a corporate bond issue. Access to lower funding cost is a key role of the SPE in a securitization.

You may ask why a securitization can be cheaper than a corporate bond issue secured by the same collateral. The reason is that the SPE would not be affected by the bankruptcy of the seller of the collateral. As mentioned above, the assets belong to the SPE, not to the entity that sold the assets to the SPE. In the United States and other countries, when a company is liquidated, creditors receive distributions based on the absolute priority rule to the extent that assets are available. The absolute priority rule is the principle that senior creditors are paid in full before subordinated creditors are paid anything. The absolute priority rule also guarantees the seniority of creditors relative to equityholders.

Whereas the absolute priority rule generally holds in liquidations, it has not always been upheld by the courts in reorganizations. Thus, although investors in the debt of a company may believe they have priority over the equityholders and priority over other classes of creditors, the actual outcome of a reorganization may be far different from the terms stated in the debt agreement; that is, there is no assurance that if the corporate bond has collateral, the rights of the bondholders will be respected. For this reason, the credit spread for a corporate bond backed by collateral does not decrease dramatically.

In the case of a securitization, the courts have in most jurisdictions no discretion to change seniority because the bankruptcy of a company does not affect the SPE. The rules set forth in the legal document, which describes how losses are to be absorbed by each bond class, are unaffected by the company's bankruptcy. This important decoupling of the credit risk of the entity needing funds from the bond classes issued by the SPE explains why the SPE's legal role is critical.

The SPE is bankruptcy remote from the seller of the collateral, which means that the bankruptcy of the seller of the collateral will not affect the holders of securities issued by the SPE and backed by the collateral. The security holders face credit risk only to the extent that the borrowers whose claims the SPE has purchased default on their loans. The SPE's ability to make cash payments to the security holders remains intact if the borrowers make the interest payments and/or the principal repayments on their loans. In many EU countries, the creditors are protected in the recognition of the securitization as a true sale. The SPE has full legal ownership of the securitized assets, which are de-recognized from the seller's balance sheet. However, it is important to note that not all countries have the same legal framework. Impediments with respect to $\mathrm{ABS}$ issuance have arisen in some countries because the concept of trust law is less well developed globally than it is in the United States and other developed countries. Thus, investors should be aware of the legal considerations that apply in the jurisdictions where they purchase ABS.

\section{EXAMPLE 3}
\section{Special Purpose Entity and Bankruptcy}
Agnelli Industries (Agnelli), a manufacturer of industrial machine tools based in Bergamo, Italy, has $€ 500$ million of corporate bonds outstanding. These bonds have a credit rating below investment grade. Agnelli has $€ 400$ million of receivables on its balance sheet that it would like to securitize. The receivables represent payments Agnelli expects to receive for machine tools it has sold to various customers in Europe. Agnelli sells the receivables to Agnelli Trust, a special purpose entity. Agnelli Trust then issues ABS, backed by the pool of receivables, with the following structure:

\begin{center}
\begin{tabular}{lc}
\hline
Bond Class & Par Value ( $\boldsymbol{\epsilon}$ millions) \\
\hline
A (senior) & 280 \\
B (subordinated) & 60 \\
C (subordinated) & 60 \\
\hline
Total & 400 \\
\hline
\end{tabular}
\end{center}

Bond Class A is given an investment-grade credit rating by the credit-rating agencies.

\begin{enumerate}
  \item Why does Bond Class A have a higher credit rating than the corporate bonds?
\end{enumerate}

\section{Solution:}
Bond Class A is issued by Agnelli Trust, an SPE that is bankruptcy remote from Agnelli. Thus, the investors who hold Agnelli's bonds and/or common shares have no legal claim on the cash flows from the securitized receivables that are the collateral for the ABS. As long as Agnelli's customers make the interest payments and/or principal repayments on their loans, Agnelli Trust will be able to make cash payments to the ABS investors. Because of the credit tranching, even if some of Agnelli's customers were to default on their loans, the losses would be realized by the subordinated Bond Classes B and $\mathrm{C}$ before any losses are realized by the senior Bond Class A. The credit risk associated with Bond Class A is, therefore, lower than that of Bond Classes $\mathrm{B}$ and $\mathrm{C}$ and the corporate bonds, justifying the investment-grade credit rating.

\begin{enumerate}
  \setcounter{enumi}{1}
  \item If Agnelli Industries files for bankruptcy after the issuance of the asset-backed security:
\end{enumerate}

A. Bond Classes A, B, and $\mathrm{C}$ will be unaffected.

B. Bond Classes A, B, and C will lose their entire par value. C. Losses will be realized by Bond Class C first, then by Bond Class B, and then by Bond Class A.

\section{Solution:}
A is correct. The ABS have been issued by Agnelli Trust, an SPE that is bankruptcy remote from Agnelli. If the securitization is viewed as resulting in a true sale, the fact that Agnelli files for bankruptcy does not affect the ABS holders. These ABS holders face credit risk only to the extent that $\mathrm{Ag}$ nelli's customers who bought the machine tools do not make the obligatory payments on their loans. As long as the customers continue to make payments, all three bond classes will receive their expected cash flows. These cash flows are completely and legally independent of anything that happens to Agnelli itself.

\begin{enumerate}
  \setcounter{enumi}{2}
  \item If one of Agnelli's customers defaults on its $€ 60$ million loan:
\end{enumerate}

A. Bond Classes A, B, and C will realize losses of $€ 20$ million each.

B. Bond Class $\mathrm{C}$ will realize losses of $€ 60$ million, but Bond Classes A and $B$ will be unaffected.

C. Bond Classes B and C will realize losses of $€ 30$ million each, but Bond Class A will be unaffected.

\section{Solution:}
$\mathrm{B}$ is correct. The rules for the distribution of losses are as follows. All losses on the collateral are absorbed by Bond Class $\mathrm{C}$ before any losses are realized by Bond Class B and then Bond Class A. Consequently, if the losses on the collateral are $€ 60$ million, which is the par value of Bond Class C, Bond Class $\mathrm{C}$ loses its entire par value, but Bond Classes A and B are unaffected.

\section{RESIDENTIAL MORTGAGE LOANS}
describe types and characteristics of residential mortgage loans that are typically securitized

Before describing the various types of residential mortgage-backed securities, this section briefly discusses the fundamental features of the underlying assets: residential mortgage loans. The mortgage designs described in this section are those that are typically securitized.

A mortgage loan, or simply mortgage, is a loan secured by the collateral of some specified real estate property that obliges the borrower (often someone wishing to buy a home) to make a predetermined series of payments to the lender (often initially a bank or mortgage company). The mortgage gives the lender the right to foreclose on the loan if the borrower defaults; that is, a foreclosure allows the lender to take possession of the mortgaged property and then sell it in order to recover funds toward satisfying the debt obligation.

Typically, the amount of the loan advanced to buy the property is less than the property's purchase price. The borrower makes a down payment, and the amount borrowed is the difference between the property's purchase price and the down payment. When the loan is first taken out, the borrower's equity in the property is equal to the down payment. Over time, as the market value of the property changes, the borrower's equity also changes. It also changes as the borrower makes mortgage payments that include principal repayment.

The ratio of the amount of the mortgage to the property's value is called the loan-to-value ratio (LTV). The lower the LTV, the higher the borrower's equity. From the lender's perspective, the higher the borrower's equity, the less likely the borrower is to default. Moreover, the lower the LTV, the more protection the lender has for recovering the amount loaned if the borrower does default and the lender repossesses and sells the property.

In the United States, market participants typically identify two types of mortgages based on the credit quality of the borrower: prime loans and subprime loans. Prime loans generally have borrowers of high credit quality with strong employment and credit histories, income sufficient to pay the loan obligation, and substantial equity in the underlying property. Subprime loans have borrowers with lower credit quality and/or are loans without a first lien on the property (that is, another party has a prior claim on the underlying property).

Throughout the world, there are a considerable number of mortgage designs. Mortgage design means the specification of (1) the maturity of the loan, (2) how the interest rate is determined, (3) how the principal is to be repaid (that is, the amortization schedule), (4) whether the borrower has the option to prepay and, in such cases, whether any prepayment penalties might be imposed, and (5) the rights of the lender in a foreclosure.

\section{Maturity}
In the United States, the typical term or number of years to maturity of a mortgage ranges from 15 to 30 years. For most European countries, a residential mortgage typically has a maturity between 20 and 40 years, whereas in France and Spain, it can be as long as 50 years. Japan is an extreme case, with mortgage maturities of 100 years.

\section{Interest Rate Determination}
The interest rate on a mortgage is called the mortgage rate, contract rate, or note rate. Determination of mortgage rates varies considerably among countries but may be categorized in four basic ways:

\begin{itemize}
  \item Fixed rate: The mortgage rate remains the same during the life of the mortgage. The United States and France have a high proportion of this type of interest rate determination. Although fixed-rate mortgages are not the dominant form in Germany, they do exist there.

  \item Adjustable or variable rate: The mortgage rate is reset periodicallydaily, weekly, monthly, or annually. The determination of the new mortgage rate for an adjustable-rate mortgage (ARM) at the reset date may be based on some reference rate or index (in which case, it is called an indexed-referenced ARM) or a rate determined at the lender's discretion (in which case, it is called a reviewable ARM). Residential mortgages in Australia, Ireland, South Korea, Spain, and the United Kingdom are dominated by adjustable-rate mortgages. In Australia, Ireland, and the United Kingdom, the reviewable ARM is standard. In South Korea and Spain, the indexed-referenced ARM is the norm. Canada and the United States have ARMs that are typically tied to an index or reference rate, although this type of mortgage rate is not the dominant form of interest rate determination. An important feature of an ARM is that it will usually have a maximum interest rate by which the mortgage rate can change at a reset date and a maximum interest rate that the mortgage rate can reach during the mortgage's life.

  \item Initial period fixed rate: The mortgage rate is fixed for some initial period and is then adjusted. The adjustment may call for a new fixed rate or for a variable rate. When the adjustment calls for a fixed rate, the mortgage is referred to as a rollover or renegotiable mortgage. This mortgage design is dominant in Canada, Denmark, Germany, the Netherlands, and Switzerland. When the mortgage starts out with a fixed rate and then switches to an adjustable rate after a specified initial term, the mortgage is referred to as a hybrid mortgage. Hybrid mortgages are popular in the United Kingdom.

  \item Convertible: The mortgage rate is initially either a fixed rate or an adjustable rate. At some point, the borrower has the option to convert the mortgage to a fixed rate or an adjustable rate for the remainder of the mortgage's life. Almost half of the mortgages in Japan are convertible.

\end{itemize}

\section{Amortization Schedule}
In most countries, residential mortgages are amortizing loans. The amortization of a loan means the gradual scheduled reduction of the principal over time. Assuming no prepayments are made by the borrower, the periodic mortgage payments made by the borrower consist of interest payments and scheduled principal repayments. As discussed in a previous reading, there are two types of amortizing loans: fully amortizing loans and partially amortizing loans. In a fully amortizing loan, the sum of all the scheduled principal repayments during the mortgage's life is such that when the last mortgage payment is made, the loan is fully repaid. Most residential mortgages in the United States are fully amortizing loans. In a partially amortizing loan, the sum of all the scheduled principal repayments is less than the amount borrowed. The final payment then includes the unpaid mortgage balance, sometimes referred to as a "balloon" payment.

If no scheduled principal repayment is specified for a certain number of years, the loan is said to be an interest-only mortgage. Interest-only mortgages have been available in Australia, Denmark, Finland, France, Germany, Greece, Ireland, the Netherlands, Portugal, South Korea, Spain, Switzerland, and the United Kingdom. Interest-only mortgages also have been available to a limited extent in the United States. A special type of interest-only mortgage is one in which there are no scheduled principal repayments over the entire life of the loan. In this case, the balloon payment is equal to the original loan amount. These mortgages, referred to as "interest-only lifetime mortgages" or "bullet mortgages," have been available in Denmark, the Netherlands, and the United Kingdom.

\section{Prepayment Options and Prepayment Penalties}
A prepayment is any payment of principal that exceeds the scheduled principal repayment. A mortgage may entitle the borrower to prepay all or part of the outstanding mortgage principal prior to the scheduled due date. This contractual provision is referred to as a prepayment option or an early repayment option. From the lender's or investor's viewpoint, the effect of a prepayment option is that the cash flow amounts and timing from a mortgage cannot be known with certainty. This risk was referred to earlier as prepayment risk. Prepayment risk affects all mortgages that allow prepayment, not just the level-payment, fixed-rate, fully amortizing mortgages. The mortgage may stipulate a monetary penalty when a borrower prepays within a certain time period after the mortgage is originated, which may extend for the full life of the loan. Such mortgage designs are referred to as prepayment penalty mortgages. The purpose of the prepayment penalty is to compensate the lender for the difference between the contract rate and the prevailing mortgage rate if the borrower prepays when interest rates decline. Hence, the prepayment penalty is effectively a mechanism that provides for yield maintenance for the lender. The method for calculating the penalty varies. Prepayment penalty mortgages are common in Europe. Although the proportion of prepayment penalty mortgages in the United States is small, they do exist.

\section{Rights of the Lender in a Foreclosure}
A mortgage can be a recourse loan or a non-recourse loan. When the borrower fails to make the contractual loan payments, the lender can repossess the property and sell it, but the proceeds received from the sale of the property may be insufficient to recoup the losses. In a recourse loan, the lender has a claim against the borrower for the shortfall between the amount of the outstanding mortgage balance and the proceeds received from the sale of the property. In a non-recourse loan, the lender does not have such a claim and thus can look only to the property to recover the outstanding mortgage balance. In the United States, recourse is typically determined by the state, and residential mortgages are non-recourse loans in many states. In contrast, residential mortgages in most European countries are recourse loans.

The recourse/non-recourse feature of a mortgage has implications for projecting the likelihood of defaults by borrowers, particularly in the case of what is sometimes called "underwater mortgages"-that is, mortgages for which the value of the property has declined below the amount owed by the borrower. For example, in the United States, where mortgages are typically non-recourse, the borrower may have an incentive to default on an underwater mortgage and allow the lender to foreclose on the property, even if resources are available to continue to make mortgage payments. This type of default by a borrower is referred to as a "strategic default." A strategic default, however, has negative consequences for the borrower, who will then have a lower credit score and a reduced ability to borrow in the future. Thus, not all borrowers faced with underwater mortgages will default. In countries where residential mortgages are recourse loans, a strategic default is less likely because the lender can seek to recover the shortfall from the borrower's other assets and/or income.

Now that the basics of residential mortgages have been set out, we can turn our attention to how these mortgages are securitized-that is, transformed into MBS. In the following sections, we focus on the US residential mortgage sector because it is the largest in the world and many non-US investors hold US MBS in their portfolios.

\section{EXAMPLE 4}
\section{Residential Mortgage Designs}
\begin{enumerate}
  \item In an interest-only mortgage, the borrower:
\end{enumerate}

A. does not have to repay the principal as long as she pays the interest.

B. does not have to make principal repayments for a certain number of years, after which she starts paying down the original loan amount.

C. does not have to make principal repayments over the entire life of the mortgage and pays down the original loan amount as a balloon payment.

\section{Solution:}
$\mathrm{B}$ is correct. In an interest-only mortgage, there is no scheduled principal repayment for a certain number of years, so the borrower starts paying down the original loan amount only after an initial period of interest-only payments. Some, but not all, interest-only mortgages do not have scheduled principal repayments over the entire life of the loan. These mortgages are called interest-only lifetime mortgages or bullet mortgages, and they require the borrower to pay back the original loan amount at maturity.

\begin{enumerate}
  \setcounter{enumi}{1}
  \item A bank advertises a mortgage with the following interest rate: $2.99 \%$ (12-month Euribor $+2.50 \%)$, resetting once a year. The mortgage is most likely:
\end{enumerate}

A. a hybrid mortgage.

B. an adjustable-rate mortgage.

C. an initial period fixed-rate mortgage.

Solution:

$\mathrm{B}$ is correct. An adjustable-rate mortgage is one for which the mortgage rate is typically based on some reference rate or index (indexed-referenced ARM) or a rate determined at the lender's discretion (reviewable ARM) and is reset periodically. A mortgage rate of 12 -month Euribor $+2.50 \%$, resetting once per year, suggests that the mortgage is an indexed-referenced ARM. The $2.99 \%$ rate is the current mortgage rate (that is, 12-month Euribor of $0.49 \%+2.50 \%$ ) and should not be taken as an indication that it is a fixed-rate, initial period fixed-rate, or hybrid mortgage.

\begin{enumerate}
  \setcounter{enumi}{2}
  \item If the borrower fails to make the contractual mortgage payments on a non-recourse mortgage, the lender:
\end{enumerate}

A. cannot foreclose the property.

B. can recover the outstanding mortgage balance only through the sale of the property.

C. can recover the outstanding mortgage balance through the sale of the property and the borrower's other assets and/or income.

Solution:

B is correct. In the case of a non-recourse mortgage, the lender can foreclose the property if the borrower fails to make the contractual mortgage payments. However, the lender can use only the proceeds from the property to recover the outstanding mortgage balance.

\section{MORTGAGE PASS-THROUGH SECURITIES}
describe types and characteristics of residential mortgage-backed
securities, including mortgage pass-through securities and
collateralized mortgage obligations, and explain the cash flows and
risks for each type
define prepayment risk and describe the prepayment risk of
mortgage-backed securities

The bonds created from the securitization of mortgages related to the purchase of residential properties are residential mortgage-backed securities (RMBS). In such countries as the United States, Canada, Japan, and South Korea, a distinction is often made between securities that are guaranteed by the government or a quasi-government entity and securities that are not. Quasi-government entities are usually created by governments to perform various functions for them. Examples of quasi-government entities include government-sponsored enterprises (GSEs) such as Fannie Mae (previously the Federal National Mortgage Association) and Freddie Mac (previously the Federal Home Loan Mortgage Corporation) in the United States and the Japan Housing Finance Agency (JHF).

In the United States, securities backed by residential mortgages are divided into three sectors: (1) those guaranteed by a federal agency, (2) those guaranteed by a GSE, and (3) those issued by private entities and that are not guaranteed by a federal agency or a GSE. The first two sectors are referred to as agency RMBS, and the third sector is referred to as non-agency RMBS. We devote significant attention to US agency and non-agency RMBS because these securities represent a large sector of the investment-grade bond market and are included in the portfolios of many global investors.

Agency RMBS include securities issued by federal agencies, such as the Government National Mortgage Association, popularly referred to as Ginnie Mae. This entity is a federally related institution because it is part of the US Department of Housing and Urban Development. As a result, the RMBS that it guarantees carry the full faith and credit of the US government with respect to timely payment of interest and repayment of principal.

Agency RMBS also include RMBS issued by GSEs, such as Fannie Mae and Freddie Mac. RMBS issued by GSEs do not carry the full faith and credit of the US government. Agency RMBS issued by GSEs differ from non-agency RMBS in two ways. First, the credit risk of the RMBS issued by Fannie Mae and Freddie Mac is reduced by the guarantee of the GSE itself, which charges a fee for insuring the issue. In contrast, non-agency RMBS use credit enhancements to reduce credit risk. The pool of securitized loans is another way in which RMBS issued by GSEs differ from non-agency RMBS. Loans included in agency RMBS must meet specific underwriting standards established by various government agencies. These standards set forth the maximum size of the loan, the loan documentation required, the maximum loan-to-value ratio, and whether insurance is required. Loans satisfying the underwriting standards for inclusion as collateral for an agency RMBS are called "conforming" mortgages; otherwise, loans are categorized as "non-conforming" mortgages.

This section starts with a discussion of agency RMBS, which include mortgage pass-through securities and collateralized mortgage obligations. Note that the popular Bloomberg Barclays US Aggregate Bond Index includes only agency RMBS that are mortgage pass-through securities in its mortgage sector definition. We then discuss non-agency RMBS.

\section{Mortgage Pass-Through Securities}
A mortgage pass-through security is a security created when one or more holders of mortgages form a pool of mortgages and sell shares or participation certificates in the pool. A pool can consist of several thousand or only a few mortgages. When a mortgage is used as collateral for a mortgage pass-through security, the mortgage is said to be securitized.

\section{Characteristics}
The cash flows of a mortgage pass-through security depend on the cash flows of the underlying pool of mortgages. The cash flows consist of monthly mortgage payments representing interest, the scheduled repayment of principal, and any prepayments. Cash payments are made to security holders each month. Neither the amount nor the timing of the cash flows from the pool of mortgages, however, is necessarily identical to that of the cash flow passed through to the security holders. In fact, the monthly cash flows of a mortgage pass-through security are less than the monthly cash flows of the underlying pool of mortgages by an amount equal to the servicing and other administrative fees.

The servicing fee is the charge related to administrative tasks, such as collecting monthly payments from borrowers, forwarding proceeds to owners of the loan, sending payment notices to borrowers, reminding borrowers when payments are overdue, maintaining records of the outstanding mortgage balance, initiating foreclosure proceedings if necessary, and providing tax information to borrowers when applicable. The servicing fee is typically a portion of the mortgage rate. The other administrative fees are those charged by the issuer or financial guarantor of the mortgage pass-through security for guaranteeing the issue.

A mortgage pass-through security's coupon rate is called the pass-through rate. The pass-through rate is lower than the mortgage rate on the underlying pool of mortgages by an amount equal to the servicing and other administrative fees. The pass-through rate that the investor receives is said to be "net interest" or "net coupon."

Not all mortgages included in a pool of securitized mortgages have the same mortgage rate and the same maturity. Consequently, for each mortgage pass-through security, a weighted average coupon rate (WAC) and a weighted average maturity (WAM) are determined. The WAC is calculated by weighting the mortgage rate of each mortgage in the pool by the percentage of the outstanding mortgage balance relative to the outstanding amount of all the mortgages in the pool. Similarly, the WAM is calculated by weighting the remaining number of months to maturity of each mortgage in the pool by the outstanding mortgage balance relative to the outstanding amount of all the mortgages in the pool. Example 5 illustrates the calculation of the WAC and WAM.

\section{EXAMPLE 5}
\section{Weighted Average Coupon Rate and Weighted Average}
 MaturityAssume that a pool includes three mortgages with the following characteristics:

\begin{center}
\begin{tabular}{cccc}
\hline
$\begin{array}{c}\text { Outstanding } \\ \text { Mortgage Balance } \\ \text { (US\$) }\end{array}$ & $\begin{array}{c}\text { Cumber of Months } \\ \text { Mortgage }\end{array}$ & $\begin{array}{c}\text { Coupon Rate (\%) }\end{array}$ & 3.1 \\
\hline
1 & 1,000 & 5.7 & 76 \\
2 & 3,000 &  &  \\
\end{tabular}
\end{center}

\begin{center}
\begin{tabular}{cccc}
\hline
$\begin{array}{c}\text { Outstanding } \\ \text { Mortgage Balance } \\ \text { (US\$) }\end{array}$ & Coupon Rate (\%) & $\begin{array}{c}\text { Number of Months } \\ \text { to Maturity }\end{array}$ &  \\
\hline
3 & 6,000 & 5.3 & 88 \\
\hline
\end{tabular}
\end{center}

The outstanding amount of three mortgages is US $\$ 10,000$. Thus, the weights of Mortgages 1,2, and 3 are $10 \%, 30 \%$, and $60 \%$, respectively.

The WAC is

$$
10 \% \times 5.1 \%+30 \% \times 5.7 \%+60 \% \times 5.3 \%=5.4 \% \text {. }
$$

The WAM is

$$
10 \% \times 34+30 \% \times 76+60 \% \times 88=79 \text { months } .
$$

\section{Prepayment Risk}
Mortgage pass-through security cash flows are uncertain because they depend on actual prepayments. As we noted earlier, this risk is called prepayment risk. Prepayment risk has two components: contraction risk and extension risk, both of which largely reflect changes in the general level of interest rates.

Contraction risk is the risk that when interest rates decline, actual prepayments will be higher than forecasted because homeowners will refinance at now-available lower interest rates. Thus, a security backed by mortgages will have a shorter maturity than was anticipated at the time of purchase. Holding a security whose maturity becomes shorter when interest rates decline has two adverse consequences for investors. First, investors must reinvest the proceeds at lower interest rates. Second, if the security is prepayable or callable, its price appreciation is not as great as that of an otherwise identical bond without a prepayment or call option.

In contrast, extension risk is the risk that when interest rates rise, prepayments will be lower than forecasted because homeowners are reluctant to give up the benefits of a contractual interest rate that now looks low. As a result, a security backed by mortgages will typically have a longer maturity than was anticipated at the time of purchase. From the investor's perspective, the value of the security has fallen because the higher interest rates translate into a lower price for the security, and the income investors receive (and can potentially reinvest) is typically limited to the interest payment and scheduled principal repayments.

\section{Prepayment Rate Measures}
In describing prepayments, market participants refer to the prepayment rate or prepayment speed. The two key prepayment rate measures are the single monthly mortality rate (SMM), a monthly measure, and its corresponding annualized rate, the conditional prepayment rate (CPR).

The SMM reflects the dollar amount of prepayment for the month as a fraction of the balance on the mortgage after accounting for the scheduled principal repayment for the month. It is calculated as follows:

$$
\mathrm{SMM}=\frac{\text { Prepayment for the month }}{\left(\begin{array}{l}
\text { Beginning outstanding mortgage balance for the month- } \\
\text { Scheduled principal repayment for the month }
\end{array}\right)}
$$

Note that the SMM is typically expressed as a percentage. When market participants describe the assumed prepayment for a pool of residential mortgages, they refer to the annualized SMM, which is the CPR. A CPR of $6 \%$, for example, means that approximately $6 \%$ of the outstanding mortgage balance at the beginning of the year is expected to be prepaid by the end of the year.

A key factor in the valuation of a mortgage pass-through security and other products derived from a pool of mortgages is forecasting the future prepayment rate. This task involves prepayment modeling. Prepayment modeling uses characteristics of the mortgage pool and other factors to develop a statistical model for forecasting future prepayments.

In the United States, market participants describe prepayment rates in terms of a prepayment pattern or benchmark over the life of a mortgage pool. This pattern is the Public Securities Association (PSA) prepayment benchmark, which is produced by the Securities Industry and Financial Markets Association (SIFMA). The PSA prepayment benchmark is expressed as a series of monthly prepayment rates. Based on historical patterns, it assumes that prepayment rates are low for newly originated mortgages and then speed up as the mortgages become seasoned. Slower or faster prepayment rates are then referred to as some percentage of the PSA prepayment benchmark. Rather than going into the details of the PSA prepayment benchmark, this discussion will rely on some PSA assumptions. What is important to remember is that the standard for the PSA model is 100 PSA; that is, at 100 PSA, investors can expect prepayments to follow the PSA prepayment benchmark-for example, an increase of prepayment rates of $0.20 \%$ for the first 30 months until they peak at $6 \%$ in Month 30 . A PSA assumption greater than 100 PSA means that prepayments are assumed to be faster than the standard model. In contrast, a PSA assumption lower than 100 PSA means that prepayments are assumed to be slower than the standard model.

\section{Cash Flow Construction}
Let us see how to construct the monthly cash flow for a hypothetical mortgage pass-through security. We assume the following:

\begin{itemize}
  \item The underlying pool of mortgages has a par value of US $\$ 800$ million.

  \item The mortgages are fixed-rate, level-payment, and fully amortizing loans.

  \item The WAC for the mortgages in the pool is $6 \%$.

  \item The WAM for the mortgages in the pool is 357 months.

  \item The pass-through rate is $5.5 \%$.

\end{itemize}

Exhibit 2 shows the cash flows to the mortgage pass-through security holders for selected months assuming a prepayment rate of 165 PSA. The SMM is given in in Column 3, and mortgage payments are given in Column 4 . The net interest payment in Column 5 is the amount available to pay security holders after servicing and other administrative fees. This amount is equal to the beginning outstanding mortgage balance in Column 2 multiplied by the pass-through rate of $5.5 \%$ and then divided by 12 . The scheduled principal repayment in Column 6 is the difference between the mortgage payment in Column 4 and the gross interest payment. The gross interest payment is equal to the beginning outstanding mortgage balance in Column 2 multiplied by the WAC of $6 \%$ and then divided by 12 . The prepayment in Column 7 is calculated by applying Equation 1, using the SMM provided in Column 3, the beginning outstanding mortgage balance in Column 2, and the scheduled principal repayment in Column 6. The total principal repayment in Column 8 is the sum of the scheduled principal repayment in Column 6 and the prepayments in Column 7. Subtracting this amount from the beginning outstanding mortgage balance for the month gives the beginning outstanding mortgage balance for the following month. Finally, the projected cash flow for this mortgage pass-through security in Column 9 is the sum of the net interest payment in Column 5 and the total principal repayment in Column 8. Exhibit 2: Monthly Cash Flow to Bondholders for a US $\$ \mathbf{8 0 0}$ Million Mortgage Pass-Through Security with a WAC of 6.0\%, a WAM of 357 Months, and a Pass-Through Rate of 5.5\%, Assuming a Prepayment Rate of 165 PSA

\begin{center}
\begin{tabular}{|c|c|c|c|c|c|c|c|c|}
\hline
(1) & (2) & (3) & (4) & (5) & (6) & (7) & (8) & (9) \\
\hline
Month & $\begin{array}{c}\text { Beginning } \\ \text { Outstanding } \\ \text { Mortgage } \\ \text { Balance (US\$) }\end{array}$ & $\begin{array}{l}\text { SMM } \\ \text { (\%) }\end{array}$ & $\begin{array}{c}\text { Mortgage } \\ \text { Payment } \\ \text { (US\$) }\end{array}$ & $\begin{array}{c}\text { Net } \\ \text { Interest } \\ \text { Payment } \\ \text { (US\$) }\end{array}$ & $\begin{array}{l}\text { Scheduled } \\ \text { Principal } \\ \text { Repayment } \\ \text { (US\$) }\end{array}$ & $\begin{array}{c}\text { Prepayment } \\ \text { (US\$) }\end{array}$ & $\begin{array}{c}\text { Total } \\ \text { Principal } \\ \text { Repayment } \\ \text { (US\$) }\end{array}$ & $\begin{array}{c}\text { Projected } \\ \text { Cash Flou } \\ \text { (US\$) }\end{array}$ \\
\hline
1 & $800,000,000$ & 0.111 & $4,810,844$ & $3,666,667$ & 810,844 & 884,472 & $1,695,316$ & $5,361,982$ \\
\hline
2 & $798,304,684$ & 0.139 & $4,805,520$ & $3,658,896$ & 813,996 & $1,104,931$ & $1,918,927$ & $5,577,823$ \\
\hline
3 & $796,385,757$ & 0.167 & $4,798,862$ & $3,650,101$ & 816,933 & $1,324,754$ & $2,141,687$ & $5,791,788$ \\
\hline
\multicolumn{9}{|l|}{$\square$} \\
\hline
29 & $674,744,235$ & 0.865 & $4,184,747$ & $3,092,578$ & 811,026 & $5,829,438$ & $6,640,464$ & $9,733,042$ \\
\hline
30 & $668,103,771$ & 0.865 & $4,148,550$ & $3,062,142$ & 808,031 & $5,772,024$ & $6,580,055$ & $9,642,198$ \\
\hline
\multicolumn{9}{|l|}{$\square$} \\
\hline
100 & $326,937,929$ & 0.865 & $2,258,348$ & $1,498,466$ & 623,659 & $2,822,577$ & $3,446,236$ & $4,944,702$ \\
\hline
101 & $323,491,693$ & 0.865 & $2,238,814$ & $1,482,670$ & 621,355 & $2,792,788$ & $3,414,143$ & $4,896,814$ \\
\hline
\multicolumn{9}{|l|}{$\square$} \\
\hline
200 & $103,307,518$ & 0.865 & 947,322 & 473,493 & 430,784 & 889,871 & $1,320,655$ & $1,794,148$ \\
\hline
201 & $101,986,863$ & 0.865 & 939,128 & 467,440 & 429,193 & 878,461 & $1,307,654$ & $1,775,094$ \\
\hline
\multicolumn{9}{|l|}{$\square$} \\
\hline
300 & $19,963,930$ & 0.865 & 397,378 & 91,501 & 297,559 & 170,112 & 467,670 & 559,172 \\
\hline
301 & $19,496,260$ & 0.865 & 393,941 & 89,358 & 296,460 & 166,076 & 462,536 & 551,893 \\
\hline
\multicolumn{9}{|c|}{$\square$} \\
\hline
356 & 484,954 & 0.865 & 244,298 & 2,223 & 241,873 & 2,103 & 243,976 & 246,199 \\
\hline
357 & 240,978 & 0.865 & 242,185 & 1,104 & 240,980 & 0 & 240,980 & 242,084 \\
\hline
\end{tabular}
\end{center}

Notes: The SMM in Column 3 is rounded, which results in some rounding error in the calculation of the prepayments in Column 7 and, thus, of the total principal repayments and the projected cash flows in Columns 8 and 9, respectively. Since the WAM is 357 months, the underlying mortgage pool is seasoned an average of three months, and therefore, based on a 165 PSA, the CPR is $0.132 \%$ in Month 1 (seasoned Month 4), and the pool seasons at 6\% in Month 27.

\section{Weighted Average Life}
A standard practice in the bond market is to refer to the maturity of a bond. This practice is not followed for MBS because principal repayments (scheduled principal repayments and prepayments) are made over the life of the security. Although an MBS has a "legal maturity," which is the date when the last scheduled principal repayment is due, the legal maturity does not reveal much about the actual principal repayments and the interest rate risk associated with the MBS. For example, a 30-year, option-free, corporate bond and an MBS with a 30-year legal maturity with the same coupon rate are not equivalent in terms of interest rate risk. Effective duration can be calculated for both the corporate bond and the MBS to assess the sensitivity of the securities to interest rate movements. But a measure widely used by market participants for MBS is the weighted average life, or simply the average life of the MBS. This measure gives investors an indication of how long they can expect to hold the MBS before it is paid off assuming interest rates stay at current levels and, thus, expected prepayments are realized. In other words, the average life of the MBS is the convention-based average time to receipt of all the projected principal repayments (scheduled principal repayments and projected prepayments).

A mortgage pass-through security's average life depends on the prepayment assumption, as illustrated in the following table. This table provides the average life of the mortgage pass-through security used in Exhibit 2 for various prepayment rates. Note that at the assumed prepayment rate of 165 PSA, the mortgage pass-through security has an average life of 8.6 years. The average life extends when the prepayment rate goes down and contracts rapidly as the prepayment rate goes up. So, at a prepayment rate of 600 PSA, the average life of the mortgage pass-through security is only 3.2 years.

\begin{center}
\begin{tabular}{lcccccc}
\hline
PSA assumption & 100 & 125 & 165 & 250 & 400 & 600 \\
\hline
Average life (years) & 11.2 & 10.1 & 8.6 & 6.4 & 4.5 & 3.2 \\
\hline
\end{tabular}
\end{center}

\section{ENVIRONMENTAL, SOCIAL, AND GOVERNANCE (ESG) RISKS FOR RMBS}
Beyond traditional factors that affect RMBS prepayment risk, such as loan-to-value ratios and FICO scores of residential borrowers, rating agencies have more recently incorporated ESG factors into their assessment of these securities. In the first example, the potential loss of mortgage principal in the event of a natural disaster is incorporated into the ratings process for a new securitization, whereas in the second example, several long outstanding RMBS backed by subprime mortgages were downgraded because of a potential decline in interest income as a result of the adverse social impact of the COVID-19 pandemic.

\section{Fitch Ratings Example}
As the physical risk of climate change increases, Fitch Ratings has incorporated natural disaster risk into its loan loss expectations by factoring in the impact of potential weather-related damage on the value of residential properties that collateralize the mortgages underlying an RMBS transaction. Fitch Ratings has created ESG Relevance Scores that are integrated into the existing research process (they range from 1 to 5 , with 5 representing ESG risk that is "highly relevant" to the credit decision and 1 representing ESG risk that is considered "irrelevant"). In the case of RMBS, Fitch assesses the geographic asset location and concentration within a given pool and compares it to regions that face greater climate risk. Fitch then assesses concentration penalties and adjusts stress assumptions for projected property losses from storm surges in cases of heightened environmental risk. For example, in 2019, Fitch published an ESG Relevance Score of 5 for an RMBS transaction with a large pool concentration in the US Gulf Coast region. Because this region is particularly susceptible to hurricane-related damage and over $75 \%$ of pool properties were concentrated in Louisiana and Texas, it faced a 1.16× probability of default adjustment with an increase in expected loan loss of over 100 bps due to the greater natural disaster and catastrophe risk in this pool versus other transactions.

\section{Moody's Investors Service Example}
In a 2020 report, Moody's Investor Service announced that ESG factors were a material credit consideration in one-third of its rating actions for private-sector issuers in 2019. The vast majority (88\%) cited governance issues, followed by social (20\%) and environmental (16\%) issues, and many actions referenced more than one ESG consideration. The COVID-19 pandemic has wide-ranging political, economic, and social consequences that can adversely affect the creditworthiness of residential borrowers. For example, in June 2020, Moody's downgraded the ratings of over 40 bonds from 25 different US RMBS transactions. These securities, originally issued between 2004 and 2006 and backed by subprime mortgages, were largely downgraded to $B 1$. In its rating action, Moody's noted the heightened risk of interest loss due to the 2020 slowdown in US economic activity and rising unemployment due to the coronavirus outbreak. Moody's cited unusually high downside risks to its forecast regarding the COVID-19 pandemic as a social risk under its ESG framework, with significant risks to public health and safety. Because many residential borrowers have been offered relief in the form of mortgage payment forbearance owing to the loss of income during the lockdown, these loans also face the possibility of modification in the future depending on the timing and pace of economic recovery.

\section{EXAMPLE 6}
\section{Mortgage Pass-Through Securities}
\begin{enumerate}
  \item A non-conforming mortgage:
\end{enumerate}

A. cannot be used as collateral in a mortgage-backed security.

B. does not satisfy the underwriting standards for inclusion as collateral for an agency residential mortgage-backed security.

C. does not give the lender a claim against the borrower for the shortfall between the amount of the outstanding mortgage balance and the proceeds from the sale of the property if the borrower defaults on the mortgage.

\section{Solution:}
B is correct. A non-conforming mortgage is one that does not satisfy the underwriting standards for inclusion as collateral for an agency RMBS. The standards specify the maximum size of the loan, the loan documentation required, the maximum loan-to-value ratio, and whether or not insurance is required for the loans in the pool.

\begin{enumerate}
  \setcounter{enumi}{1}
  \item The monthly cash flows of a mortgage pass-through security most likely:
A. are constant.
B. change when interest rates decline.
C. are equal to the cash flows of the underlying pool of mortgages.
\end{enumerate}

\section{Solution:}
B is correct. The monthly cash flows of a mortgage pass-through security are not equal to but rather depend on the cash flows of the underlying pool of mortgages. That said, their amount and timing cannot be known with certainty because of prepayments. When interest rates decline, borrowers are likely to prepay all or part of their outstanding mortgage balance, which will affect the monthly cash flows of the mortgage pass-through security. Remember that the fees related to servicing and guaranteeing the mortgages reduce the monthly cash flows of a mortgage pass-through security relative to those of the underlying pool of mortgages.

\begin{enumerate}
  \setcounter{enumi}{2}
  \item A prepayment rate of 80 PSA means that investors can expect:
\end{enumerate}

A. $80 \%$ of the par value of the mortgage pass-through security to be repaid prior to the security's maturity.

B. $80 \%$ of the borrowers whose mortgages are included in the collateral backing the mortgage pass-through security to prepay their mortgages.

C. the prepayment rate of the mortgages included in the collateral backing the mortgage pass-through security to be $80 \%$ of the monthly prepayment rates forecasted by the PSA model.

\section{Solution:}
$\mathrm{C}$ is correct. A prepayment rate of 80 PSA means that investors can expect the prepayment rate of the mortgages included in the collateral backing the mortgage pass-through security to be $80 \%$ of the monthly prepayment rates forecasted by the PSA model. For example, if the PSA model forecasts an increase in prepayment rates of $0.20 \%$ for the first 30 months until they peak at 6\% in Month 30, 80 PSA would assume an increase in prepayment rates of $0.16 \%(80 \% \times 0.20 \%)$ for the first 30 months until they peak at $4.80 \%(80 \%$ $\times 6 \%$ ) in Month 30. Thus, investors can expect slower prepayments than the PSA prepayment benchmark.

\begin{enumerate}
  \setcounter{enumi}{3}
  \item All else being equal, when interest rates decline:
\end{enumerate}

A. investors in mortgage pass-through securities face extension risk.

B. the weighted average maturity of a mortgage pass-through security lengthens.

C. the increase in the price of a mortgage pass-through security is less than the increase in the price of an otherwise identical bond with no prepayment option.

Solution:

$\mathrm{C}$ is correct. When interest rates decline, the prepayment rate on a mortgage pass-through security goes up because homeowners refinance at now-available lower interest rates. As a result, investors face contraction risk; that is, they receive payments faster than anticipated. Investors who decide to retain the security face the prospect of having to reinvest those payments at relatively low interest rates. Investors who decide to sell the security would have to do so at a price lower than that of an otherwise identical bond with no prepayment option and thus no prepayment risk.

\section{COLLATERALIZED MORTGAGE OBLIGATIONS AND NON-AGENCY RMBS}
describe types and characteristics of residential mortgage-backed securities, including mortgage pass-through securities and collateralized mortgage obligations, and explain the cash flows and risks for each type

As noted in the previous section, prepayment risk is an important consideration when investing in mortgage pass-through securities. Some institutional investors are concerned with extension risk, and others, with contraction risk. The structuring of a securitization can help redistribute the cash flows of mortgage-related products (mortgage pass-through securities or pools of loans) to different bond classes or tranches, which leads to the creation of securities that have different exposures to prepayment risk and thus different risk-return patterns relative to the mortgage-related product from which they were created. When the cash flows of mortgage-related products are redistributed to various tranches, the resulting securities are called collateralized mortgage obligations (CMOs). The mortgage-related products from which the cash flows are obtained are considered the collateral. Note that in contrast to a mortgage pass-through security, the collateral is not a pool of mortgages but a mortgage pass-through security. In fact, in practice, the collateral is usually a pool of mortgage pass-through securities-hence the reason market participants sometimes use the terms "collateral" and "mortgage pass-through securities" interchangeably.

The creation of a CMO cannot eliminate or change prepayment risk; it can only distribute the various forms of this risk among different bond classes. The CMO's major financial innovation is that securities can be created to closely satisfy the asset/liability needs of institutional investors, thereby broadening the appeal of mortgage-backed products.

A wide range of CMO structures exists. The major ones are reviewed in the following subsections.

\section{Sequential-Pay CMO Structures}
The first CMOs were structured so that each tranche would be retired sequentially. Such structures are called "sequential-pay CMOs." The rule for the monthly distribution of the principal repayments (scheduled principal repayment plus prepayments) to the tranches in this structure is as follows. First, distribute all principal payments to Tranche 1 until the principal balance for Tranche 1 is zero. After Tranche 1 is paid off, distribute all principal payments to Tranche 2 until the principal balance for Tranche 2 is zero, and so on.

To illustrate a sequential-pay CMO, let us use a hypothetical transaction called CMO-01. Assume that the collateral for CMO-01 is the mortgage pass-through security described in Exhibit 2. Recall that the total par value of the collateral is US $\$ 800$ million, the pass-through coupon rate is $5.5 \%$, the WAC is 6\%, and the WAM is 357 months. From this US $\$ 800$ million of collateral, four tranches are created, as shown in Exhibit 3. In this simple structure, the coupon rate is the same for each tranche and the same as the mortgage pass-through security's coupon rate. This feature is for simplicity; typically, the coupon rate varies by tranche based on the term structure of interest rates, among other factors.

\section{Exhibit 3: CMO-01: Sequential-Pay CMO Structure with Four Tranches}
\begin{center}
\begin{tabular}{lcc}
\hline
Tranche & Par Amount (US\$ millions) & Coupon Rate (\%) \\
\hline
A & 389 & 5.5 \\
B & 72 & 5.5 \\
C & 193 & 5.5 \\
D & 146 & 5.5 \\
Total & 800 &  \\
\hline
\end{tabular}
\end{center}

Payment rules: For payment of monthly coupon interest: Disburse monthly coupon interest to each tranche on the basis of the amount of principal outstanding for each tranche at the beginning of the month. For disbursement of principal payments: Disburse principal payments to Tranche A until it is completely paid off. After Tranche A is completely paid off, disburse principal payments to Tranche B until it is completely paid off. After Tranche B is completely paid off, disburse principal payments to Tranche $C$ until it is completely paid off. After Tranche $C$ is completely paid off, disburse principal payments to Tranche D until it is completely paid off. Remember that a CMO is created by redistributing the cash flows-interest payments and principal repayments-to the various tranches based on a set of payment rules. The payment rules at the bottom of Exhibit 3 describe how the cash flows from the mortgage pass-through security are to be distributed to the four tranches. CMO-01 has separate rules for the interest payment and the principal repayment, the latter being the sum of the scheduled principal repayment and the prepayments.

Although the payment rules for the distribution of the principal repayments are known, the precise principal repayment amount in each month is not. This amount will depend on the cash flow of the collateral, which depends on the actual prepayment rate of the collateral. The assumed prepayment rate (165 PSA in Exhibit 2) allows determining only the projected, not the actual, cash flow.

Consider what has been accomplished by creating the sequential-pay CMO-01 structure. Earlier, we saw that with a prepayment rate of 165 PSA, the mortgage pass-through security's average life was 8.6 years. Exhibit 4 reports the average life of the collateral and the four tranches assuming various actual prepayment rates. Note that the four tranches have average lives that are shorter or longer than the collateral, thereby attracting investors who have preferences for different average lives. For example, a pension fund that needs cash only after a few years because it expects a significant increase in the number of retirements after that time may opt for a tranche with a longer average life.

Exhibit 4: Average Life of the Collateral and the Four Tranches of CMO-01 for Various Actual Prepayment Rates

\begin{center}
\begin{tabular}{lccccc}
\hline
\multirow{2}{*}{$\begin{array}{l}\text { Prepayment } \\
\text { Rate }\end{array}$} & \multicolumn{4}{c}{Average Life (years)} &  \\
\cline { 2 - 6 }
 & Collateral & Tranche A & Tranche B & Tranche C & Tranche D \\
\hline
100 PSA & 11.2 & 4.7 & 10.4 & 15.1 & 24.0 \\
125 PSA & 10.1 & 4.1 & 8.9 & 13.2 & 22.4 \\
165 PSA & 8.6 & 3.4 & 7.3 & 10.9 & 19.8 \\
250 PSA & 6.4 & 2.7 & 5.3 & 7.9 & 15.2 \\
400 PSA & 4.5 & 2.0 & 3.8 & 5.3 & 10.3 \\
600 PSA & 3.2 & 1.6 & 2.8 & 3.8 & 7.0 \\
\hline
\end{tabular}
\end{center}

A major problem that remains is the considerable variability of the average lives of the tranches. How this problem can be handled is shown in the next section, but at this point, note that some protection against prepayment risk is provided for each tranche. The protection arises because prioritizing the distribution of principal (that is, establishing the payment rule for the principal repayment) effectively protects the shorter-term tranche (A in this structure) against extension risk. This protection actually comes from the longer-term tranches. Similarly, Tranches C and D provide protection against extension risk for Tranches A and B. At the same time, Tranches $\mathrm{C}$ and $\mathrm{D}$ benefit because they are provided protection against contraction risk; this protection comes from Tranches A and B. Thus, the sequential-pay CMO-01 structure allows investors concerned about extension risk to invest in Tranches A or B and those concerned about contraction risk to invest in Tranches C or D.

\section{CMO Structures Including Planned Amortization Class and Support Tranches}
A common structure in CMOs is to include planned amortization class (PAC) tranches, which offer greater predictability of the cash flows if the prepayment rate is within a specified band over the collateral's life. Remember that the creation of an MBS, whether it is a mortgage pass-through or a CMO, cannot make prepayment risk disappear. So where does the reduction of prepayment risk (both extension risk and contraction risk) that PAC tranches offer investors come from? The answer is that it comes from the existence of non-PAC tranches, called support tranches or companion tranches. The structure of the CMO makes the support tranches absorb prepayment risk first. Because PAC tranches have limited (but not complete) protection against both extension risk and contraction risk, they are said to provide two-sided prepayment protection.

The greater predictability of the cash flows for the PAC tranches occurs because a principal repayment schedule must be satisfied. When the prepayment rate is within the specified band, called the PAC band, all prepayment risk is absorbed by the support tranche. If the collateral prepayments are slower than forecasted, the support tranches do not receive any principal repayment until the PAC tranches receive their scheduled principal repayment. This rule reduces the extension risk of the PAC tranches. Similarly, if the collateral prepayments are faster than forecasted, the support tranches absorb any principal repayments in excess of the scheduled principal repayments. This rule reduces the contraction risk of the PAC tranches. Even if the prepayment rate is outside the PAC band, prepayment risk is first absorbed by the support tranche. Thus, the key to the prepayment protection that PAC tranches offer investors is the amount of support tranches outstanding. If the support tranches are paid off quickly because of faster-than-expected prepayments, they no longer provide any protection for the PAC tranches.

Support tranches expose investors to the highest level of prepayment risk. Therefore, investors must be particularly careful in assessing the cash flow characteristics of support tranches in order to reduce the likelihood of adverse portfolio consequences resulting from prepayments.

To illustrate how to create CMO structures including PAC and support tranches, we use again the US $\$ 800$ million mortgage pass-through security described in Exhibit 2, with a pass-through coupon rate of $5.5 \%$, a WAC of 6\%, and a WAM of 357 months as collateral. The creation of PAC tranches requires the specification of two PSA prepayment rates: a lower PSA prepayment assumption and an upper PSA prepayment assumption. The lower and upper PSA prepayment assumptions are called the "initial PAC collar" or the "initial PAC band." The PAC collar for a CMO is typically dictated by market conditions. In our example, we assume that the lower and upper PSA prepayment assumptions are 100 PSA and 250 PSA, respectively, so the initial PAC collar is 100-250 PSA.

Exhibit 5 shows a CMO structure called CMO-02 that contains only two tranches: a 5.5\% coupon PAC tranche created assuming an initial PAC collar of 100-250 PSA and a support tranche.

Exhibit 5: CMO-02: CMO Structure with One PAC Tranche and One Support Tranche

Tranche

Par Amount (US\$ millions)

Coupon Rate (\%)

\begin{center}
\begin{tabular}{lll}
\hline
P (PAC) & 487.6 & 5.5 \\
\hline
\end{tabular}
\end{center}

S (support) $312.4 \quad 5.5$

\begin{center}
\begin{tabular}{ccc}
\hline
Tranche & Par Amount (US\$ millions) & Coupon Rate (\%) \\
\hline
Total & 800.0 &  \\
\hline
\end{tabular}
\end{center}

Payment rules: For payment of monthly coupon interest: Disburse monthly coupon interest to each tranche based on the amount of principal outstanding for each tranche at the beginning of the month. For disbursement of principal payments: Disburse principal payments to Tranche $\mathrm{P}$ based on its schedule of principal repayments. Tranche $\mathrm{P}$ has priority with respect to current and future principal payments to satisfy the schedule. Any excess principal payments in a month over the amount necessary to satisfy the schedule for Tranche $\mathrm{P}$ are paid to Tranche $\mathrm{S}$. When Tranche $\mathrm{S}$ is completely paid off, all principal payments are to be made to Tranche $P$ regardless of the schedule.

Exhibit 6 reports the average life of the PAC and support tranches in CMO-02 assuming various actual prepayment rates. Note that between 100 PSA and 250 PSA, the average life of the PAC tranche is constant at 7.7 years. At slower or faster PSA rates, however, the schedule is broken and the average life changes-extending when the prepayment rate is less than 100 PSA and contracting when it is greater than 250 PSA. Even so, there is much less variability for the average life of the PAC tranche compared with that of the support tranche.

Exhibit 6: Average Life of the PAC Tranche and the Support Tranche of CMO-02 for Various Actual Prepayment Rates and an Initial PAC Collar of 100-250 PSA

\begin{center}
\begin{tabular}{lcc}
\hline
 & \multicolumn{2}{c}{Average Life (years)} \\
\cline { 2 - 3 }
Prepayment Rate & PAC Tranche (P) & Support Tranche (S) \\
\hline
50 PSA & 10.2 & 24.9 \\
75 PSA & 8.6 & 22.7 \\
100 PSA & 7.7 & 20.0 \\
165 PSA & 7.7 & 10.7 \\
250 PSA & 7.7 & 3.3 \\
400 PSA & 5.5 & 1.9 \\
600 PSA & 4.0 & 1.4 \\
\hline
\end{tabular}
\end{center}

Most CMO structures including PAC and support tranches have more than one PAC tranche. A sequence of six PAC tranches (that is, PAC tranches paid off in sequence as specified by a principal repayment schedule) is not uncommon. For example, consider CMO-03 in Exhibit 7, which contains four sequential PAC tranches (P-A, P-B, P-C, and P-D) and one support tranche. The total par amount of the PAC and support tranches is the same as for CMO-02 in Exhibit 5. The difference is that instead of one PAC tranche with a schedule, there are four PAC tranches with schedules. As described in the payment rules, the PAC tranches are paid off in sequence.

Exhibit 7: CMO-03: CMO Structure with Sequential PAC Tranches and One Support Tranche

\begin{center}
\begin{tabular}{lcc}
\hline
Tranche & Par Amount (US\$ million) & Coupon Rate (\%) \\
\hline
P-A (PAC) & 287.6 & 5.5 \\
P-B (PAC) & 90.0 & 5.5 \\
P-C (PAC) & 60.0 & 5.5 \\
\end{tabular}
\end{center}

\begin{center}
\begin{tabular}{lcc}
\hline
Tranche & Par Amount (US\$ million) & Coupon Rate (\%) \\
\hline
P-D (PAC) & 50.0 & 5.5 \\
S (support) & 312.4 & 5.5 \\
Total & 800.0 &  \\
\hline
\end{tabular}
\end{center}

Payment rules: For payment of monthly coupon interest: Disburse monthly coupon interest to each tranche based on the amount of principal outstanding for each tranche at the beginning of the month. For disbursement of principal payments: Disburse principal payments to Tranche $\mathrm{P}$-A based on its schedule of principal repayments. Tranche P-A has priority with respect to current and future principal payments to satisfy the schedule. Any excess principal payments in a month over the amount necessary to satisfy the schedule while P-A is outstanding is paid to Tranche $S$. Once $\mathrm{P}-\mathrm{A}$ is paid off, disburse principal payments to Tranche P-B based on its schedule of principal repayments. Tranche P-B has priority with respect to current and future principal payments to satisfy the schedule. Any excess principal payments in a month over the amount necessary to satisfy the schedule while P-B is outstanding are paid to Tranche S. The same rule applies for P-C and P-D. When Tranche $S$ is completely paid off, all principal payments are to be made to the outstanding PAC tranches regardless of the schedule.

\section{Other CMO Structures}
Often, there is a demand for tranches that have a floating rate. Although the collateral pays a fixed rate, it is possible to create a tranche with a floating rate. This is done by constructing a floater and an inverse floater combination from any of the fixed-rate tranches in the CMO structure. Because the floating-rate tranche pays a higher rate when interest rates go up and the inverse floater pays a lower rate when interest rates go up, they offset each other. Thus, a fixed-rate tranche can be used to satisfy the demand for a floating-rate tranche.

In a similar vein, other types of tranches to satisfy the various needs of investors are possible.

\section{EXAMPLE 7}
\section{Collateralized Mortgage Obligations}
\begin{enumerate}
  \item A collateralized mortgage obligation:
\end{enumerate}

A. eliminates prepayment risk.

B. is created from a pool of conforming loans.

C. redistributes various forms of prepayment risk among different bond classes.

\section{Solution:}
$\mathrm{C}$ is correct. CMOs are created by redistributing the cash flows of mortgage-related products, including mortgage pass-through securities, to different bond classes or tranches on the basis of a set of payment rules.

\begin{enumerate}
  \setcounter{enumi}{1}
  \item The variability in the average life of the PAC tranche of a CMO relative to the average life of the mortgage pass-through securities from which the CMO is created is:
A. lower.
B. the same.
C. higher.
\end{enumerate}

\section{Solution:}
A is correct. The purpose of creating different bond classes in a CMO is to provide a risk-return profile that is more suitable to investors than the riskreturn profile of the mortgage pass-through securities from which the CMO is created. The PAC tranche has considerably less variability in average life than the mortgage pass-through securities. In contrast, the support tranche has more variability in average life than the mortgage pass-through securities.

\begin{enumerate}
  \setcounter{enumi}{2}
  \item Referring to Exhibit 7, the tranche of CMO-03 that is most suitable for an investor concerned about contraction risk is:
\end{enumerate}

A. P-A (PAC).

B. P-D (PAC).

C. $\mathrm{S}$ (support).

Solution:

$\mathrm{B}$ is correct. Contraction risk is the risk that when interest rates decline, prepayments will be higher than expected and the security's maturity will become shorter than was anticipated at the time of purchase. PAC tranches offer investors protection against contraction risk (and extension risk). The PAC tranche that is most suitable for an investor concerned about contraction risk is P-D because it is the latest-payment PAC tranche; that is, any principal repayments in excess of the scheduled principal repayments are absorbed sequentially by the support tranche, then P-A, P-B, and, finally, P-D.

\begin{enumerate}
  \setcounter{enumi}{3}
  \item The tranche of a collateralized mortgage obligation that is most suitable for an investor who expects a fall in interest rates is:
\end{enumerate}

A. a fixed-rate tranche.

B. an inverse floating-rate tranche.

C. a PAC tranche.

Solution:

$\mathrm{B}$ is correct. The tranche of a CMO that is most suitable for an investor who expects a fall in interest rates is an inverse floating-rate tranche. The inverse floater pays a coupon rate that is inversely related to prevailing interest rates. Thus, if interest rates fall, the CMO's coupon rate will rise.

\begin{enumerate}
  \setcounter{enumi}{4}
  \item The investment that is most suitable for an investor who is willing and able to accept significant prepayment risk is:
\end{enumerate}

A. a mortgage pass-through security.

B. the support tranche of a collateralized mortgage obligation.

C. the inverse floating-rate tranche of a collateralized mortgage obligation.

\section{Solution:}
B is correct. The investment that is most suitable to an investor who is willing and able to accept significant prepayment risk is the support tranche of a collateralized mortgage obligation. Because the PAC tranche has a stable average life at prepayment rates within the PAC band, all prepayment risk is absorbed by the support tranche for prepayment rates within the band. Even at rates outside the PAC band, prepayment risk is first absorbed by the support tranche. Investors will be compensated for bearing prepayment risk in the sense that, if properly priced, the support tranche will have a higher expected rate of return than the PAC tranche.

\section{Non-Agency Residential Mortgage-Backed Securities}
Agency RMBS are those issued by Ginnie Mae, Fannie Mae, and Freddie Mac. RMBS issued by any other entity are non-agency RMBS. Entities that issue non-agency RMBS are typically thrift institutions, commercial banks, and private conduits. Private conduits may purchase non-conforming mortgages, pool them, and then sell mortgage pass-through securities whose collateral is the underlying pool of non-conforming mortgages. Because they are not guaranteed by the government or by a GSE, credit risk is an important consideration when investing in non-agency RMBS.

Non-agency RMBS share many features and structuring techniques with agency CMOs. However, because non-agency RMBS are not guaranteed by the US government or by a GSE that can provide protection against losses in the pool, some form of internal or external credit enhancement is necessary to make these securities attractive to investors. These credit enhancements allow investors to reduce credit risk or transfer credit risk between bond classes, thus enabling investors to choose the risk-return profile that best suits their needs. Credit enhancements also play an important role in obtaining favorable credit ratings, which make non-agency RMBS more marketable to investors. The level of credit enhancement is usually determined relative to a specific credit rating desired by the issuer for a security. Note that one of the consequences of the 2007-09 credit crisis has been an overall increase in the level of credit enhancement.

As mentioned earlier, subordination, or credit tranching, is a common form of credit enhancement. The subordination levels (that is, the amount of credit protection for a bond class) are set at the time of issuance. However, the subordination levels change over time, as voluntary prepayments and defaults occur. To protect investors in non-agency RMBS, a securitization is designed to keep the amount of credit enhancement from deteriorating over time. If the credit enhancement for senior tranches deteriorates because of poor performance of the collateral, a mechanism called the "shifting interest mechanism" locks out subordinated bond classes from receiving payments for a period. Many non-agency RMBS also include other credit enhancements, such as overcollateralization and reserve accounts.

When forecasting the future cash flows of non-agency RMBS, investors must consider two important components. The first is the assumed default rate for the collateral. The second is the recovery rate, because even though the collateral may default, not all of the outstanding mortgage balance may be lost. The repossession and subsequent sale of the recovered property may provide cash flows that will be available to pay bondholders. That amount is based on the assumed amount that will be recovered.

Now that we have discussed securities backed by a pool of residential mortgages, we turn to securities backed by a pool of commercial mortgages.

\section*{COMMERCIAL MORTGAGE-BACKED SECURITIES }
Commercial mortgage-backed securities (CMBS) are backed by a pool of commercial mortgages on income-producing property, such as multifamily properties (e.g., apartment buildings), office buildings, industrial properties (including warehouses), shopping centers, hotels, and health care facilities (e.g., senior housing care facilities). The collateral is a pool of commercial loans that were originated either to finance a commercial purchase or to refinance a prior mortgage obligation.

\section{Credit Risk}
In the United States and other countries where commercial mortgages are non-recourse loans, the lender can look only to the income-producing property backing the loan for interest payments and principal repayments. If a default occurs, the lender can foreclose the commercial property but it can use only the proceeds from the sale of that property to recover the principal outstanding, and it has no recourse to the borrower's other assets and/or income for any unpaid balance. The lender must view each property individually, and lenders evaluate each property using measures that have been found useful in assessing credit risk.

Two key indicators of potential credit performance are the loan-to-value ratio (LTV), which was discussed earlier, and the debt-service-coverage (DSC) ratio, sometimes referred to as DSCR. The DSC ratio is equal to the property's annual net operating income (NOI) divided by the debt service (that is, the annual amount of interest payments and principal repayments). The $\mathrm{NOI}$ is defined as the rental income reduced by cash operating expenses and a non-cash replacement reserve reflecting the depreciation of the property over time. A DSC ratio that exceeds 1.0 indicates that the cash flows from the property are sufficient to cover the debt service while maintaining the property in its initial state of repair. The higher the DSC ratio, the more likely it is that the borrower will be able to meet debt-servicing requirements from the property's cash flows.

\section{CMBS Structure}
A credit-rating agency determines the level of credit enhancement necessary to achieve a desired credit rating. For example, if specific loan-to-value and DSC ratios are needed and those ratios cannot be met at the loan level, subordination is used to achieve the desired credit rating.

Interest on the principal outstanding is paid to all tranches. Losses arising from loan defaults are charged against the outstanding principal balance of the CMBS tranche with the lowest priority. This tranche may not be rated by credit-rating agencies; in this case, this unrated tranche is called the "first-loss piece," "residual tranche," or "equity tranche." The total loss charged includes the amount previously advanced and the actual loss incurred in the sale of the loan's underlying property.

Two characteristics that are usually specific to CMBS structures are the presence of call protection and a balloon maturity provision.

\section{Call Protection}
A critical investment feature that distinguishes CMBS from RMBS is the protection against early prepayments available to investors known as call protection. An investor in an RMBS is exposed to considerable prepayment risk because the borrower has the right to prepay a loan, in whole or in part, before the scheduled principal repayment date. As explained previously, a borrower in the United States usually does not pay any penalty for prepayment. The discussion of CMOs highlighted how investors can purchase certain types of tranches (e.g., sequential-pay and PAC tranches) to modify or reduce prepayment risk.

CMBS investors have considerable call protection, which results in CMBS trading more like corporate bonds than RMBS. The call protection comes either at the structure level or at the loan level. Structural call protection is achieved when CMBS are structured to have sequential-pay tranches, by credit rating. A lower-rated tranche cannot be paid down until the higher-rated tranche is completely retired, so the AAA rated bonds must be paid off before the AA rated bonds are, and so on. Principal losses resulting from defaults, however, are affected from the bottom of the structure upward.

At the loan level, four mechanisms offer investors call protection:

\begin{itemize}
  \item A prepayment lockout, which is a contractual agreement that prohibits any prepayments during a specified period.

  \item Prepayment penalty points, which are predetermined penalties that a borrower who wants to refinance must pay to do so; a point is equal to $1 \%$ of the outstanding loan balance.

  \item A yield maintenance charge, also called a "make-whole charge" which is a penalty paid by the borrower that makes refinancing solely to get a lower mortgage rate uneconomical for the borrower. In its simplest terms, a yield maintenance charge is designed to make the lender indifferent as to the timing of prepayments.

  \item Defeasance, for which the borrower provides sufficient funds for the servicer to invest in a portfolio of government securities that replicates the cash flows that would exist in the absence of prepayments. The cash payments that must be met by the borrower are projected based on the terms of the loan. Then, a portfolio of government securities is constructed in such a way that the interest payments and the principal repayments from the portfolio will be sufficient to pay off each obligation when it comes due. When the last obligation is paid off, the value of the portfolio is zero (that is, there are no funds remaining). The cost of assembling such a portfolio is the cost of defeasing the loan that must be repaid by the issuer.

\end{itemize}

\section{Balloon Maturity Provision}
Many commercial loans backing CMBS are balloon loans that require a substantial principal repayment at maturity of the loan. If the borrower fails to make the balloon payment, the borrower is in default. The lender may extend the loan over a period known as the "workout period." In doing so, the lender may modify the original terms of the loan and charge a higher interest rate, called the "default interest rate," during the workout period.

The risk that a borrower will not be able to make the balloon payment because either the borrower cannot arrange for refinancing or cannot sell the property to generate sufficient funds to pay off the outstanding principal balance is called "balloon risk." Because the life of the loan is extended by the lender during the workout period, balloon risk is a type of extension risk.

\section{EXAMPLE 8}
\section{An Example of a Commercial Mortgage-Backed Security}
The following information is taken from a filing with the US Securities and Exchange Commission about a CMBS issued by a special purpose entity established by a major US commercial bank. The collateral for this CMBS was a pool of 72 fixed-rate mortgages secured by first liens (first claims) on various types of commercial, multifamily, and manufactured housing community properties.

\begin{center}
\begin{tabular}{|c|c|c|}
\hline
$\begin{array}{l}\text { Classes of Offered } \\ \text { Certificates }\end{array}$ & Initial Principal Amount (US\$) & $\begin{array}{c}\text { Initial Pass-Throug } \\ \text { Rate (\%) }\end{array}$ \\
\hline
A-1 & $75,176,000$ & 0.754 \\
\hline
$\mathrm{A}-2$ & $290,426,000$ & 1.987 \\
\hline
A-3 & $150,000,000$ & 2.815 \\
\hline
$A-4$ & $236,220,000$ & 3.093 \\
\hline
$A-A B$ & $92,911,000$ & 2.690 \\
\hline
$\mathrm{X}-\mathrm{A}$ & $948,816,000$ & 1.937 \\
\hline
A-S & $104,083,000$ & 3.422 \\
\hline
B & $75,423,000$ & 3.732 \\
\hline
C & $42,236,000$ &  \\
\hline
\end{tabular}
\end{center}

The filing included the following statements:

If you acquire Class $B$ certificates, then your rights to receive distributions of amounts collected or advanced on or in respect of the mortgage loans will be subordinated to those of the holders of the Class A-1, Class A-2, Class A-3, Class A-4, Class A-AB, Class X-A, and Class A-S certificates. If you acquire Class $C$ certificates, then your rights to receive distributions of amounts collected or advanced on or in respect of the mortgage loans will be subordinated to those of the holders of the Class B certificates and all other classes of offered certificates.

"Prepayment Penalty Description" or "Prepayment Provision" means the number of payments from the first due date through and including the maturity date for which a mortgage loan, as applicable, (i) is locked out from prepayment, (ii) provides for payment of a prepayment premium or yield maintenance charge in connection with a prepayment, (iii) or permits defeasance.

\begin{enumerate}
  \item Based on the information provided, this CMBS:
\end{enumerate}

A. did not include any credit enhancement.

B. included an internal credit enhancement.

C. included an external credit enhancement.

\section{Solution:}
$\mathrm{B}$ is correct. The CMBS included a senior/subordinated structure, which is a form of internal credit enhancement. Class B provided protection for all of the bond classes listed above it. Similarly, Class C provided protection for all other bond classes, including Class B; it was the first-loss piece, also called the residual tranche or equity tranche. Note that because it was the residual tranche, Class C had no specific pass-through rate. Investors in Class C will have priced it on the basis of some expected residual rate of return, but they could have done better or worse than expected depending on how interest rate movements and default rates affected the performance of the other tranches. 2. Based on the information provided, investors in this CMBS had prepayment protection at:
A. the loan level.
B. the structure level.
C. both the loan and structure levels.

\section{Solution:}
$\mathrm{C}$ is correct. This CMBS offered investors prepayment protection at both the structure and loan levels. The structural call protection was achieved thanks to the sequential-pay tranches. At the loan level, the CMBS included three of the four types of call protection-namely, a prepayment lockout, a yield maintenance charge, and defeasance.

\begin{enumerate}
  \setcounter{enumi}{2}
  \item Defeasance can be best described as:
\end{enumerate}

A. a predetermined penalty that a borrower who wants to refinance must pay to do so.

B. a contractual agreement that prohibits any prepayments during a specified period of time.

C. funds that the borrower must provide to replicate the cash flows that would exist in the absence of prepayments.

\section{Solution:}
$C$ is correct. Defeasance is a call protection at the loan level that requires the borrower to provide sufficient funds for the servicer to invest in a portfolio of government securities that replicates the cash flows that would exist in the absence of prepayments.

\begin{enumerate}
  \setcounter{enumi}{3}
  \item A risk that investors typically face when holding CMBS is:
A. call risk.
B. balloon risk.
C. contraction risk.
\end{enumerate}

\section{Solution:}
$B$ is correct. Because many commercial loans backing CMBS require a balloon payment, investors in CMBS typically face balloon risk-that is, the risk that if the borrower cannot arrange for refinancing or cannot sell the property to make the balloon payment, the CMBS may extend in maturity because the lender has to wait to obtain the outstanding principal until the borrower can make the balloon amount. Balloon risk is a type of extension risk.

\begin{enumerate}
  \setcounter{enumi}{4}
  \item The credit risk of a commercial mortgage-backed security is lower:
A. the lower the DSC ratio and the lower the LTV.
B. the lower the DSC ratio and the higher the LTV.
C. the higher the DSC ratio and the lower the LTV.
\end{enumerate}

\section{Solution:}
$C$ is correct. The DSC ratio and the LTV are key indicators of potential credit performance and thus allow investors to assess the credit risk of a CMBS. The DSC ratio is equal to the property's annual NOI divided by the annual amount of interest payments and principal repayments. So the higher the DSC ratio, the lower the CMBS's credit risk. The LTV is equal to the amount of the mortgage divided by the property's value. So the lower the LTV, the lower the CMBS's credit risk.

To this point, this reading has addressed the securitization of real estate property, both residential and commercial. We now discuss the securitization of debt obligations in which the underlying asset is not real estate.

\section{NON-MORTGAGE ASSET-BACKED SECURITIES}
describe types and characteristics of non-mortgage asset-backed securities, including the cash flows and risks of each type

Numerous types of non-mortgage assets have been used as collateral in securitization. The largest non-mortgage assets in most countries are auto loan and lease receivables, credit card receivables, personal loans, and commercial loans. What is important to keep in mind is that, regardless of the type of asset, ABS that are not guaranteed by a government or a quasi-government entity are subject to credit risk.

ABS can be categorized based on the way the collateral repays-that is, whether the collateral is amortizing or non-amortizing. Traditional residential mortgages and auto loans are examples of amortizing loans. The cash flows for an amortizing loan include interest payments, scheduled principal repayments, and any prepayments, if permissible. If the loan has no schedule for paying down the principal, it is a non-amortizing loan. Because a non-amortizing loan does not involve scheduled principal repayments, an ABS backed by non-amortizing loans is not affected by prepayment risk. Credit card receivable ABS are an example of ABS backed by non-amortizing loans.

Consider an ABS backed by a pool of 1,000 amortizing loans with a total par value of US\$100 million. Over time, some of the loans will be paid off; the amounts received from the scheduled principal repayment and any prepayments will be distributed to the bond classes based on the payment rule. Consequently, over time, the number of loans in the collateral will drop from 1,000 and the total par value will fall to less than US $\$ 100$ million.

Now, what happens if the collateral of the ABS is 1,000 non-amortizing loans? Some of these loans will be paid off in whole or in part before the maturity of the ABS. When those loans are paid off, what happens depends on whether the loans were paid off during or after the lockout period. The lockout or revolving period is the period during which the principal repaid is reinvested to acquire additional loans with a principal equal to the principal repaid. The reinvestment in new loans can result in the collateral including more or less than 1,000 loans, but the loans will still have a total par value of US $\$ 100$ million. When the lockout period is over, any principal that is repaid will not be used to reinvest in new loans but will instead be distributed to the bond classes.

This reading cannot cover all types of non-mortgage ABS. It focuses on the two popular non-mortgage ABS in most countries: auto loan ABS and credit card receivable ABS.

\section{Auto Loan ABS}
Auto loan ABS are backed by auto loans and lease receivables. The focus in this section is on the largest type of auto securitizations-that is, auto loan-backed securities. In some countries, auto loan-backed securities represent the largest or second largest sector of the securitization market.

The cash flows for auto loan-backed securities consist of scheduled monthly payments (that is, interest payments and scheduled principal repayments) and any prepayments. For securities backed by auto loans, prepayments result from sales and trade-ins requiring full payoff of the loan, repossession and subsequent resale of autos, insurance proceeds received upon loss or destruction of autos, and early payoffs of the loans.

All auto loan-backed securities have some form of credit enhancement, often a senior/subordinated structure. In addition, many auto loan-backed securities come with overcollateralization and a reserve account, often an excess spread account. Recall from a previous reading that the excess spread, sometimes called excess interest cash flow, is an amount that can be retained and deposited into a reserve account and that can serve as a first line of protection against losses.

To illustrate the typical structure of auto loan-backed securities, let us use the example of securities issued by Fideicomiso Financiero Autos VI. The collateral was a pool of 827 auto loans denominated in Argentine pesos (ARS). The loans were originated by BancoFinansur. The structure of the securitization included three bond classes:

\begin{center}
\begin{tabular}{lc}
Bond Class & $\begin{array}{c}\text { Outstanding Principal } \\ \text { Balance (ARS) }\end{array}$ \\
\hline
Class A Floating-Rate Debt Securities & $22,706,000$ \\
Class B Floating-Rate Debt Securities & $1,974,000$ \\
Certificates & $6,008,581$ \\
\hline
Total & $30,688,581$ \\
\hline
\end{tabular}
\end{center}

The certificates provided credit protection for Class B, and Class B provides credit protection for Class A. Further credit enhancement came from overcollateralization and the presence of an excess spread account. The reference rate for the floating-rate debt securities was BADLAR (Buenos Aires Deposits of Large Amount Rate), the benchmark rate for loans in Argentina. This reference rate is the average rate on 30-day deposits of at least ARS1 million. For Class A, the interest rate was BADLAR plus $450 \mathrm{bps}$, with a minimum rate of $18 \%$ and a maximum rate of $26 \%$; for Class $\mathrm{B}$, it was BADLAR plus $650 \mathrm{bps}$, with $20 \%$ and $28 \%$ as the minimum and maximum rates, respectively.

\section{EXAMPLE 9}
\section{An Example of an Auto Loan ABS}
The following information is from the prospectus supplement for US $\$ 877,670,000$ of auto loan ABS issued by XYZ Credit Automobile Receivables Trust 2019:

The collateral for this securitization is a pool of subprime automobile loan contracts secured for new and used automobiles and light-duty trucks and vans.

The issuing entity will issue seven sequential-pay classes of asset-backed notes pursuant to the indenture. The notes are designated as the "Class A-1 Notes," the "Class A-2 Notes," the "Class A-3 Notes," the "Class B Notes," the "Class C Notes," the "Class D Notes," and the "Class E Notes." The Class A-1 Notes, the Class A-2 Notes, and the Class A-3 Notes are the "Class A Notes." The Class A Notes, the Class B Notes, the Class C Notes, and the Class D Notes are being offered by this prospectus supplement and are sometimes referred to as the publicly offered notes. The Class E Notes are not being offered by this prospectus supplement and will initially be retained by the depositor or an affiliate of the depositor. The Class E Notes are sometimes referred to as the privately placed notes.

Each class of notes will have the initial note principal balance, interest rate, and final scheduled distribution date listed in the following tables:

Publicly Offered Notes

\begin{center}
\begin{tabular}{lccc}
\hline
 & $\begin{array}{c}\text { Initial Note } \\ \text { Principal Balance } \\ \text { (US\$) }\end{array}$ & $\begin{array}{c}\text { Interest Rate } \\ \mathbf{( \% )}\end{array}$ & $\begin{array}{c}\text { Final Scheduled } \\ \text { Distribution Date }\end{array}$ \\
\hline
A-1 (senior) & $168,000,000$ & 0.25 & 8 August 2020 \\
A-2 (senior) & $279,000,000$ & 0.74 & 8 November 2022 \\
A-3 (senior) & $192,260,000$ & 0.96 & 9 April 2024 \\
B (subordinated) & $68,870,000$ & 1.66 & 10 September 2024 \\
C (subordinated) & $85,480,000$ & 2.72 & 9 September 2025 \\
D (subordinated) & $84,060,000$ & 3.31 & 8 October 2025 \\
\hline
\end{tabular}
\end{center}

\begin{center}
\begin{tabular}{lccc}
\multicolumn{4}{c}{Privately Placed Notes} \\
\hline
 & $\begin{array}{c}\text { Initial Note } \\ \text { Principal Balance } \\ \text { (US\$) }\end{array}$ & $\begin{array}{c}\text { Interest Rate } \\ (\%)\end{array}$ & $\begin{array}{c}\text { Final Scheduled } \\ \text { Distribution Date }\end{array}$ \\
\hline
Class & $22,330,000$ & 4.01 & 8 January 2027 \\
\hline
\end{tabular}
\end{center}

Interest on each class of notes will accrue during each interest period at the applicable interest rate.

The overcollateralization amount represents the amount by which the aggregate principal balance of the automobile loan contracts exceeds the principal balance of the notes. On the closing date, the initial amount of overcollateralization is approximately US $\$ 49,868,074$, or $5.25 \%$ of the aggregate principal balance of the automobile loan contracts as of the cutoff date.

On the closing date, $2.0 \%$ of the expected initial aggregate principal balance of the automobile loan contracts will be deposited into the reserve account, which is approximately US\$18,997,361.

\begin{enumerate}
  \item The reference to subprime meant that:
\end{enumerate}

A. the asset-backed notes were rated below investment grade.

B. the automobile (auto) loan contracts were made to borrowers who did not have or could not document strong credit.

C. some of the auto loan contracts were secured by autos of low quality that may have been difficult to sell in case the borrower defaults.

Solution:

$\mathrm{B}$ is correct. A subprime loan is one granted to borrowers with lower credit quality, who have typically experienced prior credit difficulties or who cannot otherwise document strong credit. 2. Based on the information provided, if on the first distribution date there were losses on the loans of US 10 million:

A. none of the classes of notes will have incurred losses.

B. Class E notes will have incurred losses of US\$10 million.

C. Classes B, C, D, and $\mathrm{E}$ will have incurred losses pro rata of their initial note principal balances.

\section{Solution:}
A is correct. The amount of the loss (US $\$ 10$ million) was lower than the combined amount of overcollateralization and the reserve account (US $\$ 49,868,074+$ US $\$ 18,997,361=$ US\$68,865,435). Therefore, none of the classes of notes will have incurred losses.

\begin{enumerate}
  \setcounter{enumi}{2}
  \item Based on the information provided, if the first loss on the loans was US\$40 million over and above the protection provided by the internal credit enhancements and occurred in January 2020, which class(es) of notes realized losses?
\end{enumerate}

A. Class $\mathrm{E}$ and then Class D

B. Each class of subordinated notes in proportion to its principal balance

C. Class $E$ and then each class of subordinated notes in proportion to its principal balance

\section{Solution:}
A is correct. Once the amount of losses exceeds the amount of protection provided by the overcollateralization and the reserve account, losses are absorbed by the bond classes. Because it was a sequential-pay structure, Class E notes were the first ones to absorb losses, up to the principal amount of US\$22,330,000. It meant that there was still US $\$ 17,670,000$ to be absorbed by another bond class, which would have been the Class D notes.

\section{Credit Card Receivable ABS}
When a purchase is made on a credit card, the issuer of the credit card (the lender) extends credit to the cardholder (the borrower). Credit cards are issued by banks, credit card companies, retailers, and travel and entertainment companies. At the time of purchase, the cardholder agrees to repay the amount borrowed (that is, the cost of the item purchased) plus any applicable finance charges. The amount that the cardholder agrees to pay the issuer of the credit card is a receivable from the perspective of the issuer of the credit card. Credit card receivables are used as collateral for the issuance of credit card receivable ABS.

For a pool of credit card receivables, the cash flows consist of finance charges collected, fees, and principal repayments. Finance charges collected represent the periodic interest the credit card borrower is charged on the unpaid balance after the grace period, which may be fixed or floating. The floating rate may be capped; that is, it may have an upper limit because some countries have usury laws that impose a maximum interest rate. Fees include late payment fees and any annual membership fees.

Interest is paid to holders of credit card receivable ABS periodically (e.g., monthly, quarterly, or semiannually). As noted earlier, the collateral of credit card receivable ABS is a pool of non-amortizing loans. These loans have lockout periods during which the cash flows that are paid out to security holders are based only on finance charges collected and fees. When the lockout period is over, the principal that is repaid by the cardholders is no longer reinvested but instead is distributed to investors.

Some provisions in credit card receivable ABS require early principal amortization if specific events occur. Such provisions are referred to as "early amortization" or "rapid amortization" provisions and are included to safeguard the credit quality of the issue. The only way the principal cash flows can be altered is by the triggering of the early amortization provision.

To illustrate the typical structure of credit card receivable ABS, consider the Master Note Trust Series 20XX issued in March 20XX by a large, well-known financial institution. The originator of the credit card receivables was the issuer's retail bank, and the servicer was its finance company. The collateral was a pool of credit card receivables from several private-label and co-branded credit card issuers, including JCPenney, Lowe's Home Improvement, Sam's Club, Walmart, Gap, and Chevron. The structure of the US\$969,085,000 securitization was as follows: Class A notes for US\$800,000,000, Class B notes for US $\$ 100,946,373$, and Class C notes for US $\$ 68,138,802$. Thus, the issue had a senior/subordinate structure. The Class A notes were the senior notes and were rated Aaa by Moody's and AAA by Fitch. The Class B notes were rated A2 by Moody's and A+ by Fitch. The Class C notes were rated Baa2 by Moody's and BBB+ by Fitch.

\section{EXAMPLE 10}
\section{Credit Card Receivable ABS vs. Auto Loan ABS}
\begin{enumerate}
  \item Credit card receivable asset-backed securities differ from auto loan ABS in the following way:
\end{enumerate}

A. credit card loans are recourse loans, whereas auto loans are non-recourse loans.

B. the collateral for credit card receivable-backed securities is a pool of non-amortizing loans, whereas the collateral for auto loan ABS is a pool of amortizing loans.

C. credit card receivable-backed securities have regular principal repayments, whereas auto loan ABS include a lockout period during which the cash proceeds from principal repayments are reinvested in additional loan receivables.

\section{Solution:}
B is correct. A main difference between credit card receivable ABS and auto loan ABS is the type of loans that back the securities. For credit card receivable ABS, the collateral is a pool of non-amortizing loans. During the lockout period, the cash proceeds from principal repayments are reinvested in additional credit card receivables. When the lockout period is over, principal repayments are used to pay off the outstanding principal. For auto loan-backed securities, the collateral is a pool of amortizing loans. Security holders receive regular principal repayments. As a result, the outstanding principal balance declines over time.

\section{COLLATERALIZED DEBT OBLIGATIONS}
describe collateralized debt obligations, including their cash flows
and risks

Collateralized debt obligation (CDO) is a generic term used to describe a security backed by a diversified pool of one or more debt obligations: CDOs backed by corporate and emerging market bonds are collateralized bond obligations (CBOs); CDOs backed by leveraged bank loans are collateralized loan obligations (CLOs); CDOs backed by ABS, RMBS, CMBS, and other CDOs are structured finance CDOs; CDOs backed by a portfolio of credit default swaps for other structured securities are synthetic CDOs.

\section{CDO Structure}
A CDO involves the creation of an SPE. In a CDO, there is a need for a CDO manager, also called "collateral manager," to buy and sell debt obligations for and from the CDO's collateral (that is, the portfolio of assets) to generate sufficient cash flows to meet the obligations to the CDO bondholders.

The funds to purchase the collateral assets for a CDO are obtained from the issuance of debt obligations. These debt obligations are bond classes or tranches and include senior bond classes, mezzanine bond classes (that is, bond classes with credit ratings between senior and subordinated bond classes), and subordinated bond classes, often referred to as the residual or equity tranches. The motivation for investors to invest in senior or mezzanine bond classes is to earn a potentially higher yield than that on a comparably rated corporate bond by gaining exposure to debt products that they may not otherwise be able to purchase. Investors in equity tranches have the potential to earn an equity-type return, thereby offsetting the increased risk from investing in the subordinated class. The key to whether a CDO is viable depends upon whether a structure can be created that offers a competitive return for the subordinated tranche.

The basic economics of the CDO is that the funds are raised by the sale of the bond classes and the CDO manager invests those funds in assets. The CDO manager seeks to earn a rate of return higher than the aggregate cost of the bond classes. The return in excess of what is paid out to the bond classes accrues to the holders of the equity tranche and to the CDO manager. In other words, a CDO is a leveraged transaction in which those who invest in the equity tranche use borrowed funds (the bond classes issued) to generate a return above the funding cost.

As with ABS, each CDO bond class is structured to provide a specific level of risk for investors. CDOs impose restrictions on the CDO manager via various tests and limits that must be satisfied for the CDO to meet investors' varying risk appetites while still providing adequate protection for the senior bond class. If the CDO manager fails pre-specified tests, a provision is triggered that requires the payoff of the principal to the senior bond class until the tests are satisfied. This process effectively deleverages the CDO because the cheapest funding source for the CDO, the senior bond class, is reduced.

The ability of the CDO manager to make the interest payments and principal repayments depends on collateral performance. The proceeds to meet the obligations to the CDO bond classes can come from one or more of the following sources: interest payments from collateral assets, maturing of collateral assets, and sale of collateral assets. The cash flows and credit risks of CDOs are best illustrated by an example.

\section{An Example of a CDO Transaction}
Although various motivations may prompt a sponsor to create a CDO, the following example uses a CDO for which the purpose is to capture what market participants mistakenly label a CDO arbitrage transaction. The term "arbitrage" is not used here in the traditional sense-that is, a risk-free transaction that earns an expected positive net profit but requires no net investment of money. In this context, arbitrage is used in a loose sense to describe a transaction in which the motivation is to capture a spread between the potential collateral return and the funding cost.

To understand the structure of a CDO transaction and its risks, consider the following US\$100 million issue:

\begin{center}
\begin{tabular}{lll}
\hline
Tranche & Par Value (US\$ million) & Coupon Rate \\
\hline
Senior & 80 & MRR $^{\mathrm{a}}+70 \mathrm{bps}$ \\
Mezzanine & 10 & 10-year US Treasury rate $+200 \mathrm{bps}$ \\
Equity & 10 & - \\
\hline
\end{tabular}
\end{center}

${ }^{\mathrm{a}} M R R$ is the Market Reference Rate.

Suppose that the collateral consists of bonds that all mature in 10 years and that the coupon rate for every bond is the 10-year US Treasury rate plus 400 bps. Because the collateral pays a fixed rate (the 10-year US Treasury rate plus $400 \mathrm{bps}$ ) but the senior tranche requires a floating-rate payment (MRR plus $70 \mathrm{bps}$ ), the CDO manager enters into an interest rate swap agreement with another party. This interest rate swap is simply an agreement to periodically exchange fixed for floating interest payments based on a notional amount used to determine the interest payment for each party. The notional amount of the interest rate swap is the par value of the senior tranche-that is, US\$80 million in this example. Let us suppose that through the interest rate swap, the CDO manager agrees to do the following: (1) Pay a fixed rate each year equal to the 10-year US Treasury rate plus $100 \mathrm{bps}$, and (2) receive MRR.

Assume that the 10-year US Treasury rate at the time this CDO is issued is $7 \%$. Now, consider the annual cash flow for the first year. First, let us look at the collateral. Assuming no default, the collateral will pay an interest rate equal to the 10-year US Treasury rate of $7 \%$ plus $400 \mathrm{bps}-$ that is, $11 \%$. So, the interest payment is $11 \% \times$ US $\$ 100,000,000=$ US $\$ 11,000,000$.

Now, let us determine the interest that must be paid to the senior and mezzanine tranches. For the senior tranche, the interest payment is US $\$ 80,000,000 \times(M R R+$ $70 \mathrm{bps})$. For the mezzanine tranche, the coupon rate is the 10-year US Treasury rate plus $200 \mathrm{bps}$-that is, $9 \%$. So, the interest payment for the mezzanine tranche is $9 \%$ $\times$ US $\$ 10,000,000=$ US $\$ 900,000$.

Finally, consider the interest rate swap. In this agreement, the CDO manager agreed to pay the swap counterparty the 10-year US Treasury rate plus 100 bps-that is, $8 \%$-based on a notional amount of US $\$ 80$ million. So, the amount paid to the swap counterparty the first year is $8 \% \times$ US $\$ 80,000,000=$ US $\$ 6,400,000$. The amount received from the swap counterparty is MRR based on a notional amount of US $\$ 80$ million-that is, MRR $\times$ US $\$ 80,000,000$.

Combining this information, the CDO cash inflows are as follows:

Interest from collateral $\$ 11,000,000$

Interest from swap counterparty $\$ 80,000,000 \times M R R$

Total interest received $\$ 11,000,000+\$ 80,000,000 \times \mathrm{MRR}$

The CDO cash outflows are as follows:

Interest to senior tranche

$$
\$ 80,000,000 \times(\mathrm{MRR}+70 \mathrm{bps})
$$

Interest to mezzanine tranche

Interest to swap counterparty

Total interest paid

$\$ 900,000$

$\$ 6,400,000$

$\$ 7,300,000+\$ 80,000,000 \times(\mathrm{MRR}+70 \mathrm{bps})$

Netting the total interest received $(\$ 11,000,000+\$ 80,000,000 \times$ MRR) and the total interest paid- $\$ 7,300,000+\$ 80,000,000 \times(\mathrm{MRR}+70 \mathrm{bps})$-leaves a net interest of $\$ 3,700,000-\$ 80,000,000 \times 70$ bps $=$ US $\$ 3,140,000$. From this amount, any feesincluding the CDO manager's fees-must be paid. The balance is then the amount available to pay the equity tranche. Suppose the CDO manager's fees are US\$640,000. The cash flow available to the equity tranche for the first year is US $\$ 2.5$ million $(\$ 3,140,000-\$ 640,000)$. Because the equity tranche has a par value of US $\$ 10$ million and is assumed to be sold at par, the annual return is $25 \%$.

This example includes some simplifying assumptions. For instance, it is assumed that no defaults would occur. Furthermore, it is assumed that all securities purchased by the CDO manager are non-callable and, thus, that the coupon rate would not decline because of securities being called. Despite these simplifying assumptions, the example demonstrates the economics of an arbitrage CDO transaction, the need for an interest rate swap, and how the equity tranche realizes a return.

In practice, CDOs are subject to additional risks. For example, in the case of collateral defaults, the manager may fail to earn a return sufficient to pay off the investors in the senior and mezzanine tranches, resulting in a loss. Equity tranche investors may lose their entire investment. Even if payments are made to these investors, the realized return may be below the return expected at the time of purchase.

Moreover, after some period, the CDO manager must begin repaying principal to the senior and mezzanine tranches. The interest rate swap must be structured to take this requirement into account because the entire amount of the senior tranche is not outstanding for the life of the collateral.

\section{EXAMPLE 11}
\section{Collateralized Debt Obligations}
\begin{enumerate}
  \item An additional risk of an investment in an arbitrage collateralized debt obligation relative to an investment in an asset-backed security is:
\end{enumerate}

A. the default risk on the collateral assets.

B. the risk that the CDO manager will fail to earn a return sufficient to pay off the investors in the senior and the mezzanine tranches.

C. the risk due to the mismatch between the collateral making fixed-rate payments and the bond classes making floating-rate payments.

\section{Solution:}
B is correct. In addition to the risks associated with investments in ABS, such as the default risk on the collateral assets and the risk due to the potential mismatch between the collateral making fixed-rate payments and the bond classes making floating-rate payments, investors in CDOs face the risk that the CDO manager will fail to earn a return sufficient to pay off the investors in the senior and the mezzanine tranches. With an ABS, the cash flows from the collateral are used to pay off the holders of the bond classes without the active management of the collateral-that is, without a manager altering the composition of the debt obligations in the pool that is backing the securitization. In contrast, in an arbitrage CDO, a CDO manager buys and sells debt obligations with the dual purpose of not only paying off the holders of the bond classes but also generating an attractive/competitive return for the equity tranche and for the manager.

COVERED BONDS

describe characteristics and risks of covered bonds and how they differ from other asset-backed securities

As outlined in an earlier reading, covered bonds are senior debt obligations issued by a financial institution and backed by a segregated pool of assets that typically consist of commercial or residential mortgages or public sector assets. The covered bond (or Pfandbrief in German) originated over 250 years ago in Prussia and has been adopted by issuers in Europe, Asia, and Australia. Each country or jurisdiction specifies eligible collateral and structures permissible in the covered bond market, and the European Union is taking steps to harmonize these elements among its member states.

Covered bonds are similar to ABS but offer bondholders dual recourse-that is, to both the issuing financial institution and the underlying asset pool. In the case of ABS, the financial institution that originates loans transfers securitized assets to a bankruptcy-remote special legal entity, but the pool of underlying assets in a covered bond (or "cover pool") remains on the financial institution's balance sheet, against which covered bondholders retain a top-priority claim.

Whereas ABS often use credit tranching to create bond classes with different borrower default exposures, covered bonds usually consist of one bond class per cover pool. Another difference lies in the dynamic nature of the cover pool. In contrast to a static pool of mortgage loans that expose investors to prepayment risk as in the case of US mortgage-backed securities, cover pool sponsors must replace any prepaid or non-performing assets (i.e., assets that do not generate the promised cash flows) in the cover pool to ensure sufficient cash flows until the maturity of the covered bond.

Redemption regimes exist to align the covered bond's cash flows as closely as possible with the original maturity schedule in the event of default of a covered bond's financial sponsor. For example, in the case of hard-bullet covered bonds, if payments do not occur according to the original schedule, a bond default is triggered and bond payments are accelerated. Soft-bullet covered bonds delay the bond default and payment acceleration of bond cash flows until a new final maturity date, which is usually up to a year after the original maturity date. Conditional pass-through covered bonds, in contrast, convert to pass-through securities after the original maturity date if all bond payments have not yet been made.

Covered bonds have remained a relatively stable and reliable source of funding over time because of their dual recourse nature, strict eligibility criteria, dynamic cover pool, and redemption regimes in the event of sponsor default. As a result, covered bonds usually carry lower credit risks and offer lower yields than otherwise similar ABS.

\section{EXAMPLE 12}
\section{Covered Bonds}
\begin{enumerate}
  \item Which of the following statements about covered bonds and asset-backed securities is most accurate?
\end{enumerate}

A. Both covered bonds and ABS pass prepayment and extension risk of the underlying asset pool through to investors.

B. Both covered bonds and ABS offer investors recourse to both the bond's issuer and the underlying asset pool.

C. Covered bonds have a dynamic cover pool in which sponsors must replace prepaid or non-performing assets, whereas ABS, such as mortgage-backed securities, pass through default and prepayment risk to investors.

\section{Solution:}
$\mathrm{C}$ is correct. In contrast to a static pool of mortgage loans that expose investors to prepayment risk as in the case of US mortgage-backed securities, cover pool sponsors must replace any prepaid or non-performing assets in the cover pool to ensure sufficient cash flows until the covered bond matures. A is incorrect because the dynamic cover pool insulates investors from prepayment risk, whereas the static mortgage loan pool does not protect ABS investors from prepayment risk. B is incorrect since only covered bonds offer investors recourse to both the bond's issuer and the underlying asset pool.

\section{SUMMARY}
\begin{itemize}
  \item Securitization involves pooling debt obligations, such as loans or receivables, and creating securities backed by the pool of debt obligations called asset-backed securities (ABS). The cash flows of the debt obligations are used to make interest payments and principal repayments to the holders of the ABS.

  \item Securitization has several benefits. It allows investors direct access to liquid investments and payment streams that would be unattainable if all the financing were performed through banks. It enables banks to increase loan originations at economic scales greater than if they used only their own in-house loan portfolios. Thus, securitization contributes to lower costs of borrowing for entities raising funds, higher risk-adjusted returns to investors, and greater efficiency and profitability for the banking sector.

  \item The parties to a securitization include the seller of the collateral (pool of loans), the servicer of the loans, and the special purpose entity (SPE). The SPE is bankruptcy remote, which plays a pivotal role in the securitization.

  \item A common structure in a securitization is subordination, which leads to the creation of more than one bond class or tranche. Bond classes differ as to how they will share any losses resulting from defaults of the borrowers whose loans are in the collateral. The credit ratings assigned to the various bond classes depend on how the credit-rating agencies evaluate the credit risks of the collateral and any credit enhancements.

  \item The motivation for the creation of different types of structures is to redistribute prepayment risk and credit risk efficiently among different bond classes in the securitization. Prepayment risk is the uncertainty that the actual cash flows will be different from the scheduled cash flows as set forth in the loan agreements because borrowers may choose to repay the principal early to take advantage of interest rate movements.

  \item Because of the SPE, the securitization of a company's assets may include some bond classes that have better credit ratings than the company itself or its corporate bonds. Thus, the company's funding cost is often lower when raising funds through securitization than when issuing corporate bonds.

  \item A mortgage is a loan secured by the collateral of some specified real estate property that obliges the borrower to make a predetermined series of payments to the lender. The cash flow of a mortgage includes (1) interest, (2) scheduled principal payments, and (3) prepayments (any principal repaid in excess of the scheduled principal payment).

  \item The various mortgage designs throughout the world specify (1) the maturity of the loan; (2) how the interest rate is determined (i.e., fixed rate versus adjustable or variable rate); (3) how the principal is repaid (i.e., whether the loan is amortizing and if it is, whether it is fully amortizing or partially amortizing with a balloon payment); (4) whether the borrower has the option to prepay and if so, whether any prepayment penalties might be imposed; and (5) the rights of the lender in a foreclosure (i.e., whether the loan is a recourse or non-recourse loan).

  \item In the United States, there are three sectors for securities backed by residential mortgages: (1) those guaranteed by a federal agency (Ginnie Mae) whose securities are backed by the full faith and credit of the US government, (2) those guaranteed by a GSE (e.g., Fannie Mae and Freddie Mac) but not by the US government, and (3) those issued by private entities that are not guaranteed by a federal agency or a GSE. The first two sectors are referred to as agency residential mortgage-backed securities (RMBS), and the third sector, as non-agency RMBS.

  \item A mortgage pass-through security is created when one or more holders of mortgages form a pool of mortgages and sell shares or participation certificates in the pool. The cash flow of a mortgage pass-through security depends on the cash flow of the underlying pool of mortgages and consists of monthly mortgage payments representing interest, the scheduled repayment of principal, and any prepayments, net of servicing and other administrative fees.

  \item Market participants measure the prepayment rate using two measures: the single monthly mortality rate (SMM) and its corresponding annualized rate-namely, the conditional prepayment rate (CPR). For MBS, a measure widely used by market participants to assess effective duration is the weighted average life or simply the average life of the MBS.

  \item Market participants use the Public Securities Association (PSA) prepayment benchmark to describe prepayment rates. A PSA assumption greater than 100 PSA means that prepayments are assumed to occur faster than the benchmark, whereas a PSA assumption lower than 100 PSA means that prepayments are assumed to occur slower than the benchmark. - Prepayment risk includes two components: contraction risk and extension risk. The former is the risk that when interest rates decline, the security will have a shorter maturity than was anticipated at the time of purchase because homeowners will refinance at the new, lower interest rates. The latter is the risk that when interest rates rise, fewer prepayments will occur than what was anticipated at the time of purchase because homeowners are reluctant to give up the benefits of a contractual interest rate that now looks low.

  \item The creation of a collateralized mortgage obligation (CMO) can help manage prepayment risk by distributing the various forms of prepayment risk among different classes of bondholders. The CMO's major financial innovation is that the securities created more closely satisfy the asset/ liability needs of institutional investors, thereby broadening the appeal of mortgage-backed products.

  \item The most common types of CMO tranche are sequential-pay tranches, planned amortization class (PAC) tranches, support tranches, and floating-rate tranches.

  \item Non-agency RMBS share many features and structuring techniques with agency CMOs. However, they typically include two complementary mechanisms. First, the cash flows are distributed by rules that dictate the allocation of interest payments and principal repayments to tranches with various degrees of priority/seniority. Second, there are rules for the allocation of realized losses that specify that subordinated bond classes have lower payment priority than senior classes.

  \item In order to obtain favorable credit ratings, non-agency RMBS and non-mortgage ABS often require one or more credit enhancements. The most common forms of internal credit enhancement are senior/subordinated structures, reserve funds, and overcollateralization. In external credit enhancement, credit support in the case of defaults resulting in losses in the pool of loans is provided in the form of a financial guarantee by a third party to the transaction.

  \item Commercial mortgage-backed securities (CMBS) are securities backed by a pool of commercial mortgages on income-producing property.

  \item Two key indicators of the potential credit performance of CMBS are the debt-service-coverage (DSC) ratio and the loan-to-value ratio (LTV). The DSC ratio is the property's annual net operating income divided by the debt service.

  \item CMBS have considerable call protection, which allows CMBS to trade in the market more like corporate bonds than like RMBS. This call protection comes in two forms: at the structure level and at the loan level. The creation of sequential-pay tranches is an example of call protection at the structure level. At the loan level, four mechanisms offer investors call protection: prepayment lockouts, prepayment penalty points, yield maintenance charges, and defeasance.

  \item $\mathrm{ABS}$ are backed by a wide range of asset types. The most popular non-mortgage ABS are auto loan ABS and credit card receivable ABS. The collateral is amortizing for auto loan ABS and non-amortizing for credit card receivable ABS. As with non-agency RMBS, these ABS must offer credit enhancement to be appealing to investors. - A collateralized debt obligation (CDO) is a generic term used to describe a security backed by a diversified pool of one or more debt obligations (e.g., corporate and emerging market bonds, leveraged bank loans, ABS, RMBS, and CMBS).

  \item A CDO involves the creation of an SPE. The funds necessary to pay the bond classes come from a pool of loans that must be serviced. A CDO requires a collateral manager to buy and sell debt obligations for and from the CDO's portfolio of assets to generate sufficient cash flows to meet the obligations of the CDO bondholders and to generate a fair return for the equityholders.

  \item The structure of a CDO includes senior, mezzanine, and subordinated/ equity bond classes.

  \item Covered bonds are similar to ABS, but they differ because of their dual recourse nature, strict eligibility criteria, dynamic cover pool, and redemption regimes in the event of sponsor default.

\end{itemize}

\section{PRACTICE PROBLEMS}
\begin{enumerate}
  \item Securitization is beneficial for banks because it:
A. repackages bank loans into simpler structures.
B. increases the funds available for banks to lend.
C. allows banks to maintain ownership of their securitized assets.

  \item Securitization benefits financial markets by:
A. increasing the role of intermediaries.
B. establishing a barrier between investors and originating borrowers.
C. allowing investors to tailor credit risk and interest rate risk exposures to meet their individual needs.

  \item A benefit of securitization is the:
A. reduction in disintermediation.
B. simplification of debt obligations.
C. creation of tradable securities with greater liquidity than the original loans.

  \item Securitization benefits investors by:
A. providing more direct access to a wider range of assets.
B. reducing the inherent credit risk of pools of loans and receivables.
C. eliminating cash flow timing risks of an ABS, such as contraction and exten- sion risks.

  \item In a securitization, the special purpose entity (SPE) is responsible for the:
A. issuance of the asset-backed securities.
B. collection of payments from the borrowers.
C. recovery of underlying assets from delinquent borrowers.

  \item In a securitization, the collateral is initially sold by the:
A. issuer.
B. depositor.
C. underwriter.

  \item A special purpose entity issues asset-backed securities in the following structure.

\end{enumerate}

\begin{center}
\begin{tabular}{lc}
\hline
Bond Class & Par Value (€ millions) \\
\hline
A (senior) & 200 \\
B (subordinated) & 20 \\
\end{tabular}
\end{center}

\begin{center}
\begin{tabular}{lc}
\hline
Bond Class & Par Value ( $€$ millions) \\
\hline
$C$ (subordinated) & 5 \\
\hline
\end{tabular}
\end{center}

At which of the following amounts of default in par value would Bond Class A experience a loss?
A. $€ 20$ million
B. $€ 25$ million
C. $€ 26$ million

\begin{enumerate}
  \setcounter{enumi}{7}
  \item In a securitization, time tranching provides investors with the ability to choose between:
\end{enumerate}

A. extension and contraction risks.

B. senior and subordinated bond classes.

C. fully amortizing and partially amortizing loans.

\begin{enumerate}
  \setcounter{enumi}{8}
  \item The creation of bond classes with a waterfall structure for sharing losses is referred to as:
\end{enumerate}

A. time tranching.

B. credit tranching.

C. overcollateralization.

\begin{enumerate}
  \setcounter{enumi}{9}
  \item Which of the following statements related to securitization is correct?
\end{enumerate}

A. Time tranching addresses the uncertainty of a decline in interest rates.

B. Securitizations are rarely structured to include both credit tranching and time tranching.

C. Junior and senior bond classes differ in that junior classes can be paid off only at the bond's set maturity.

\begin{enumerate}
  \setcounter{enumi}{10}
  \item A goal of securitization is to:
\end{enumerate}

A. separate the seller's collateral from its credit ratings.

B. uphold the absolute priority rule in bankruptcy reorganizations.

C. account for collateral's primary influence on corporate bond credit spreads.

\begin{enumerate}
  \setcounter{enumi}{11}
  \item The last payment in a partially amortizing residential mortgage loan is best referred to as a:
\end{enumerate}

A. waterfall.

B. principal repayment.

C. balloon payment.

\begin{enumerate}
  \setcounter{enumi}{12}
  \item Which of the following characteristics of a residential mortgage loan would best protect the lender from a strategic default by the borrower?
A. Recourse
B. A prepayment option
C. Interest-only payments

  \item William Marolf obtains a EUR5 million mortgage loan from Bank Nederlandse. A year later, the principal on the loan is EUR4 million and Marolf defaults on the loan. Bank Nederlandse forecloses, sells the property for EUR2.5 million, and is entitled to collect the EUR1.5 million shortfall from Marolf. Marolf most likely had a:
A. bullet loan.
B. recourse loan.
C. non-recourse loan.

  \item Fran Martin obtains a non-recourse mortgage loan for $\$ 500,000$. One year later, when the outstanding balance of the mortgage is $\$ 490,000$, Martin cannot make his mortgage payments and defaults on the loan. The lender forecloses on the loan and sells the house for $\$ 315,000$. What amount is the lender entitled to claim from Martin?
A. $\$ 0$.
B. $\$ 175,000$.
C. $\$ 185,000$.

  \item A balloon payment equal to a mortgage's original loan amount is a characteristic of a:
A. bullet mortgage.
B. fully amortizing mortgage.
C. partially amortizing mortgage.

  \item Which of the following statements is correct concerning mortgage loan defaults?

\end{enumerate}

A. A non-recourse jurisdiction poses higher default risks for lenders.

B. In a non-recourse jurisdiction, strategic default will not affect the defaulting borrower's future access to credit.

C. When a recourse loan defaults, the mortgaged property is the lender's sole source for recovery of the outstanding mortgage balance.

\begin{enumerate}
  \setcounter{enumi}{17}
  \item If a mortgage borrower makes prepayments without penalty to take advantage of falling interest rates, the lender will most likely experience:
A. extension risk.
B. contraction risk.
C. yield maintenance.

  \item If interest rates increase, an investor who owns a mortgage pass-through security is most likely affected by:
A. credit risk.
B. extension risk.
C. contraction risk.

  \item In the context of mortgage-backed securities, a conditional prepayment rate (CPR) of $8 \%$ means that approximately $8 \%$ of the outstanding mortgage pool balance at the beginning of the year is expected to be prepaid:
A. in the current month.
B. by the end of the year.
C. over the life of the mortgages.

  \item For a mortgage pass-through security, which of the following risks most likely increases as interest rates decline?
A. Balloon
B. Extension
C. Contraction

  \item Compared with the weighted average coupon rate of its underlying pool of mortgages, the pass-through rate on a mortgage pass-through security is:
A. lower.
B. the same.
C. higher.

  \item The single monthly mortality rate (SMM) most likely:
A. increases as extension risk rises.
B. decreases as contraction risk falls.
C. stays fixed over time when the standard prepayment model remains at 100 PSA.

  \item Which of the following describes a typical feature of a non-agency residential mortgage-backed security (RMBS)?

\end{enumerate}

A. Senior/subordinated structure

B. A pool of conforming mortgages as collateral

C. A guarantee by a government-sponsored enterprise

\begin{enumerate}
  \setcounter{enumi}{24}
  \item Which of the following is most likely an advantage of collateralized mortgage obligations (CMOs)? CMOs can
A. eliminate prepayment risk.
B. be created directly from a pool of mortgage loans.
C. meet the asset/liability requirements of institutional investors. 26. The longest-term tranche of a sequential-pay CMO is most likely to have the lowest:
A. average life.
B. extension risk.
C. contraction risk.

  \item The tranches in a collateralized mortgage obligation that are most likely to provide protection for investors against both extension and contraction risk are:
A. planned amortization class (PAC) tranches.
B. support tranches.
C. sequential-pay tranches.

  \item Support tranches are most appropriate for investors who are:
A. concerned about their exposure to extension risk.
B. concerned about their exposure to concentration risk.
C. willing to accept prepayment risk in exchange for higher returns.

  \item Collateralized mortgage obligations are designed to:
A. eliminate contraction risk in support tranches.
B. distribute prepayment risk to various tranches.
C. eliminate extension risk in planned amortization tranches.

  \item Credit risk is an important consideration for commercial mortgage-backed securities (CMBS) if the CMBS are backed by mortgage loans that:
A. are non-recourse.
B. have call protection.
C. have prepayment penalty points.

  \item Which commercial mortgage-backed security characteristic causes a CMBS to trade more like a corporate bond than a residential mortgage-backed security?
A. Call protection
B. Internal credit enhancement
C. Debt-service-coverage ratio level

  \item A commercial mortgage-backed security does not meet the debt-to-service coverage at the loan level necessary to achieve a desired credit rating. Which of the following features would mostlikely improve the credit rating of the CMBS?
A. Subordination
B. Call protection
C. Balloon payments 33. If a default occurs in a non-recourse commercial mortgage-backed security, the lender will most likely:

\end{enumerate}

A. recover prepayment penalty points paid by the borrower to offset losses.

B. use only the proceeds received from the sale of the property to recover losses.

C. initiate a claim against the borrower for any shortfall resulting from the sale of the property.

\begin{enumerate}
  \setcounter{enumi}{33}
  \item Which of the following investments is least subject to prepayment risk?
A. Auto loan receivable-backed securities
B. Commercial mortgage-backed securities
C. Non-agency residential mortgage-backed securities

  \item An excess spread account incorporated into a securitization is designed to limit:
A. credit risk.
B. extension risk.
C. contraction risk.

  \item Which of the following best describes the cash flow that owners of credit card receivable asset-backed securities receive during the lockout period?
A. No cash flow
B. Only principal payments collected
C. Only finance charges collected and fees

  \item Which type of asset-backed security is not affected by prepayment risk?
A. Auto loan ABS
B. Residential MBS
C. Credit card receivable ABS

  \item In auto loan ABS, the form of credit enhancement that most likely serves as the first line of loss protection is the:
A. excess spread account.
B. sequential-pay structure.
C. proceeds from repossession sales.

  \item In credit card receivable ABS, principal cash flows can be altered only when the:
A. lockout period expires.
B. excess spread account is depleted.
C. early amortization provision is triggered.

  \item The CDO tranche with a credit-rating status between senior and subordinated bond classes is called the:
A. equity tranche.
B. residual tranche.
C. mezzanine tranche.

  \item The key to a CDO's viability is the creation of a structure with a competitive return for the:
A. senior tranche.
B. mezzanine tranche.
C. subordinated tranche.

  \item When the collateral manager fails pre-specified risk tests, a CDO is:
A. deleveraged by reducing the senior bond class.
B. restructured to reduce its most expensive funding source.
C. liquidated by paying off the bond classes in order of seniority.

  \item Which statement about covered bonds is least accurate?

\end{enumerate}

A. Covered bonds provide investors with dual recourse, to the cover pool and also to the issuer.

B. Covered bonds usually carry higher credit risks and offer higher yields than otherwise similar ABS.

C. Covered bonds have a dynamic cover pool, meaning sponsors must replace any prepaid or non-performing assets.

\section{SOLUTIONS}
\begin{enumerate}
  \item B is correct. Securitization increases the funds available for banks to lend because it allows banks to remove loans from their balance sheets and issue bonds that are backed by those loans. Securitization repackages relatively simple debt obligations, such as bank loans, into more complex, not simpler, structures. Securitization involves transferring ownership of assets from the original owner-in this case, the banks-into a special legal entity. As a result, banks do not maintain ownership of the securitized assets.

  \item C is correct. By removing the wall between ultimate investors and originating borrowers, investors can achieve better legal claims on the underlying mortgages and portfolios of receivables. This transparency allows investors to tailor interest rate risk and credit risk to their specific needs.

  \item $\mathrm{C}$ is correct. Securitization allows for the creation of tradable securities with greater liquidity than the original loans on a bank's balance sheet. Securitization results in lessening the roles of intermediaries, which increases disintermediation. Securitization is a process in which relatively simple debt obligations, such as loans, are repackaged into more complex structures.

  \item A is correct. Securitization allows investors to achieve more direct legal claims on loans and portfolios of receivables. As a result, investors can add to their portfolios exposure to the risk-return characteristics provided by a wider range of assets. $B$ is incorrect because securitization does not reduce credit risk but, rather, provides a structure to mitigate and redistribute the inherent credit risks of pools of loans and receivables.

\end{enumerate}

$\mathrm{C}$ is incorrect because securitization does not eliminate the timing risks associated with ABS cash flows but, rather, provides a structure to mitigate and redistribute those risks, such as contraction risk and extension risk.

\begin{enumerate}
  \setcounter{enumi}{4}
  \item A is correct. In a securitization, the special purpose entity is the special legal entity responsible for the issuance of the asset-backed securities. The servicer, not the SPE, is responsible for both the collection of payments from the borrowers and the recovery of underlying assets if the borrowers default on their loans.

  \item B is correct. In a securitization, the loans or receivables are initially sold by the depositor to the special purpose entity that uses them as collateral to issue the ABS.

\end{enumerate}

A is incorrect because the SPE, often referred to as the issuer, is the purchaser of the collateral rather than the seller of the collateral.

$\mathrm{C}$ is incorrect because the underwriter neither sells nor purchases the collateral in a securitization. The underwriter performs the same functions in a securitization as it does in a standard bond offering.

\begin{enumerate}
  \setcounter{enumi}{6}
  \item $C$ is correct. The first $€ 25(€ 5+€ 20)$ million in default are absorbed by the subordinated classes (C and B). The senior Class A bonds will experience a loss only when defaults exceed $€ 25$ million.

  \item A is correct. Time tranching is the process in which a set of bond classes or tranches is created that allow investors a choice in the type of prepayment riskextension or contraction-that they prefer to bear. Senior and subordinated bond classes are used in credit tranching. Credit tranching structures allow investors to choose the amount of credit risk that they prefer to bear. Fully and partially amortizing loans are two types of amortizing loans. 9. B is correct. Credit tranching is a form of credit enhancement called subordination in which bond classes or tranches differ as to how they will share losses resulting from defaults of the borrowers whose loans are part of the collateral. This type of protection is commonly referred to as a waterfall structure because of the cascading flow of payments between bond classes in the event of default.

\end{enumerate}

A is incorrect because time tranching involves the creation of bond classes that possess different expected maturities rather than bond classes that differ as to how credit losses will be shared. Time tranching involves the redistribution of prepayment risk, whereas credit tranching involves the redistribution of credit risk.

$\mathrm{C}$ is incorrect because although overcollateralization is a form of internal credit enhancement similar to subordination, it is the amount by which the principal amount of the pool of collateral exceeds the principal balance of the securities issued and backed by the collateral pool. Losses are absorbed first by the amount of overcollateralization and then according to the credit tranching structure.

\begin{enumerate}
  \setcounter{enumi}{9}
  \item A is correct. Time tranching is the creation of bond classes that possess different expected maturities so that prepayment risk can be redistributed among bond classes. When loan agreements provide borrowers the ability to alter payments, in the case of declining interest rates, this prepayment risk increases because borrowers tend to pay off part or all of their loans and refinance at lower interest rates.
\end{enumerate}

B is incorrect because it is possible and quite common for a securitization to have structures with both credit tranching and time tranching.

$\mathrm{C}$ is incorrect because the subordinated structures of junior and senior bond classes differ as to how they will share any losses relative to defaults of the borrowers whose loans are in the collateral pool. Junior classes offer protection for senior classes, with losses first realized by the former. The classes are distinguished not by scheduled repayment terms but, rather, by a loss sharing hierarchy in the event of borrower default.

\begin{enumerate}
  \setcounter{enumi}{10}
  \item A is correct. The legal implication of a special purpose entity, a prerequisite for securitization, is that investors contemplating the purchase of bond classes backed by the assets of the SPE will evaluate the credit risk of those assets independently from the credit rating of the entity that sold the assets to the SPE. This separation of the seller's collateral from its credit rating provides the opportunity for the SPE to access a lower aggregate funding cost than what the seller might otherwise obtain.
\end{enumerate}

$B$ is incorrect because the absolute priority rule, under which senior creditors are paid in full before subordinated creditors, has not always been upheld in bankruptcy reorganizations. There is no assurance that if a corporate bond has collateral, the rights of the bondholders will be respected. It is this uncertainty that creates the dominant influence of credit ratings over collateral in credit spreads. $\mathrm{C}$ is incorrect because corporate bond credit spreads will reflect the seller's credit rating primarily and the collateral only slightly. Securitization separates the seller's collateral from its credit rating, effectively altering the influence of collateral on the credit spread.

\begin{enumerate}
  \setcounter{enumi}{11}
  \item $\mathrm{C}$ is correct. In a partially amortizing loan, the sum of all the scheduled principal repayments is less than the amount borrowed. The last payment is for the remaining unpaid mortgage balance and is called the "balloon payment."

  \item A is correct. In a recourse loan, the lender has a claim against the borrower for the shortfall between the amount of the mortgage balance outstanding and the proceeds received from the sale of the property. A prepayment option is a benefit to the borrower and would thus not offer protection to the lender. An interest-only mortgage requires no principal repayment for a number of years and will not protect the lender from strategic default by the borrower.

  \item B is correct. Bank Nederlandse has a claim against Marolf for EUR1.5 million, the shortfall between the amount of the mortgage balance outstanding and the proceeds received from the sale of the property. This indicates that the mortgage loan is a recourse loan. The recourse/non-recourse feature indicates the rights of a lender in foreclosure. If Marolf had a non-recourse loan, the bank would have been entitled to only the proceeds from the sale of the underlying property, or EUR2.5 million. A bullet loan is a special type of interest-only mortgage for which there are no scheduled principal payments over the entire term of the loan. Since the unpaid balance is less than the original mortgage loan, it is unlikely that Marolf has an interest-only mortgage.

  \item A is correct. Because the loan has a non-recourse feature, the lender can look to only the underlying property to recover the outstanding mortgage balance and has no further claim against the borrower. The lender is simply entitled to foreclose on the home and sell it.

  \item A is correct. A bullet mortgage is a special type of interest-only mortgage in which there are no scheduled principal repayments over the entire life of the loan. At maturity, a balloon payment is required equal to the original loan amount.

\end{enumerate}

$\mathrm{B}$ is incorrect because with a fully amortizing mortgage, the sum of all the scheduled principal repayments during the mortgage's life is such that when the last mortgage payment is made, the loan is fully repaid, with no balloon payment required.

$\mathrm{C}$ is incorrect because with a partially amortizing mortgage, the sum of all the scheduled principal repayments is less than the amount borrowed, resulting in a balloon payment equal to the unpaid mortgage balance (rather than the original loan amount).

\begin{enumerate}
  \setcounter{enumi}{16}
  \item A is correct. In non-recourse loan jurisdictions, the borrower may have an incentive to default on an underwater mortgage and allow the lender to foreclose on the property because the lender has no claim against the borrower for the shortfall. For this reason, such defaults, known as strategic defaults, are more likely in non-recourse jurisdictions and less likely in recourse jurisdictions, where the lender does have a claim against the borrower for the shortfall.
\end{enumerate}

$B$ is incorrect because strategic defaults in non-recourse jurisdictions do have negative consequences for the defaulting borrowers in the form of a lower credit score and a reduced ability to borrow in the future. These negative consequences can be a deterrent in the incidence of underwater mortgage defaults.

$\mathrm{C}$ is incorrect because when a recourse loan defaults, the lender can look to both the property and the borrower to recover the outstanding mortgage balance. In a recourse loan, the lender has a claim against the borrower for the shortfall between the amount of the outstanding mortgage balance and the proceeds received from the sale of the property.

\begin{enumerate}
  \setcounter{enumi}{17}
  \item B is correct. Contraction risk is the risk that when interest rates decline, actual prepayments will be higher than forecasted. Extension risk is the risk that when interest rates rise, prepayments will be lower than forecasted. Yield maintenance results from prepayment penalties; the lender is protected from loss in yield by the imposition of prepayment penalties.

  \item B is correct. Extension risk is the risk that when interest rate rise, fewer prepayments will occur. Homeowners will be reluctant to give up the benefit of a contractual interest rate that is lower. As a result, the mortgage pass-through security becomes longer in maturity than anticipated at the time of purchase.

  \item $\mathrm{B}$ is correct. CPR is an annualized rate that indicates the percentage of the outstanding mortgage pool balance at the beginning of the year that is expected to be prepaid by the end of the year.

  \item C is correct. When interest rates decline, a mortgage pass-through security is subject to contraction risk. Contraction risk is the risk that when interest rates decline, actual prepayments will be higher than forecasted because borrowers will refinance at now-available lower interest rates. Thus, a security backed by mortgages will have a shorter maturity than was anticipated when the security was purchased.

  \item A is correct. The coupon rate of a mortgage pass-through security is called the pass-through rate, whereas the mortgage rate on the underlying pool of mortgages is calculated as a weighted average coupon rate (WAC). The pass-through rate is lower than the WAC by an amount equal to the servicing fee and other administrative fees.

  \item B is correct. The SMM is a monthly measure of the prepayment rate or prepayment speed. Contraction risk is the risk that when interest rates decline, actual prepayments will be higher than forecast. So if contraction risk falls, prepayments are likely to be lower than forecast, which would imply a decrease in the SMM.

\end{enumerate}

A is incorrect because the SMM is a monthly measure of the prepayment rate or prepayment speed. Extension risk is the risk that when interest rates rise, actual prepayments will be lower than forecast. So if extension risk rises, prepayments are likely to be lower than forecast, which would imply a decrease, not an increase, in the SMM.

$\mathrm{C}$ is incorrect because at 100 PSA, investors can expect prepayments to follow the PSA prepayment benchmark. Based on historical patterns, the PSA standard model assumes that prepayment rates are low for newly initiated mortgages and then speed up as mortgages season. Thus, 100 PSA does not imply that the SMM remains the same but, rather, implies that it will vary over the life of the mortgage.

\begin{enumerate}
  \setcounter{enumi}{23}
  \item A is correct. Non-agency RMBS are credit enhanced, either internally or externally, to make the securities more attractive to investors. The most common forms of internal credit enhancement are senior/subordinated structures, reserve accounts, and overcollateralization. Conforming mortgages are used as collateral for agency (not non-agency) mortgage pass-through securities. An agency RMBS, rather than a non-agency RMBS, issued by a GSE (government sponsored enterprise), is guaranteed by the respective GSE.

  \item C is correct. Using CMOs, securities can be created to closely satisfy the asset/ liability needs of institutional investors. The creation of a CMO cannot eliminate prepayment risk; it can only distribute the various forms of this risk among various classes of bondholders. The collateral of CMOs is mortgage-related products, not the mortgages themselves.

  \item $\mathrm{C}$ is correct. For a CMO with multiple sequential-pay tranches, the longest-term tranche will have the lowest contraction (prepayments greater than forecasted) risk because of the protection against this risk offered by the other tranches. The longest-term tranche is likely to have the highest average life and extension risk because it is the last tranche repaid in a sequential-pay tranche. 27. A is correct. PAC tranches have limited (but not complete) protection against both extension risk and contraction risk. This protection is provided by the support tranches. A sequential-pay tranche can protect against either extension risk or contraction risk but not both of these risks. The CMO structure with sequential-pay tranches allows investors concerned about extension risk to invest in shorter-term tranches and those concerned about contraction risk to invest in longer-term tranches.

  \item $\mathrm{C}$ is correct. The greater predictability of cash flows provided in the planned amortization class (PAC) tranches comes at the expense of support tranches. As a result, investors in support tranches are exposed to higher extension risk and contraction risk than investors in PAC tranches. Investors will be compensated for bearing this risk because support tranches have a higher expected return than PAC tranches.

  \item B is correct. CMOs are designed to redistribute cash flows of mortgage-related products to different bond classes or tranches through securitization. Although CMOs do not eliminate prepayment risk, they distribute prepayment risk among various classes of bondholders.

  \item A is correct. If commercial mortgage loans are non-recourse loans, the lender can look to only the income-producing property backing the loan for interest and principal repayment. If there is a default, the lender looks to the proceeds from the sale of the property for repayment and has no recourse against the borrower for any unpaid mortgage loan balance. Call protection and prepayment penalty points protect against prepayment risk.

  \item A is correct. With CMBS, investors have considerable call protection. An investor in an RMBS is exposed to considerable prepayment risk, but with CMBS, call protection is available to the investor at the structure and loan level. The call protection results in CMBS trading in the market more like a corporate bond than an RMBS. Both internal credit enhancement and the debt-service-coverage (DSC) ratio address credit risk, not prepayment risk.

  \item A is correct. If specific ratios of debt to service coverage are needed and those ratios cannot be met at the loan level, subordination is used to achieve the desired credit rating. Call protection protects investors against prepayment risk. Balloon payments increase the risk of the underlying loans.

  \item B is correct. In a non-recourse CMBS, the lender can look only to the income-producing property backing the loan for interest and principal repayment. If a default occurs, the lender can use only the proceeds from the sale of the property for repayment and has no recourse to the borrower for any unpaid balance.

  \item B is correct. A critical feature that differentiates CMBS from RMBS is the call protection provided to investors. An investor in an RMBS is exposed to considerable prepayment risk because the borrower has the right to prepay the loan before maturity. CMBS provide investors with considerable call protection that comes either at the structure level or at the loan level.

  \item A is correct. An excess spread account, sometimes called excess interest cash flow, is a form of internal credit enhancement that limits credit risk. It is an amount that can be retained and deposited into a reserve account and that can serve as a first line of protection against losses. An excess spread account does not limit prepayment risk-be it extension risk or contraction risk.

  \item C is correct. During the lockout period, the cash flow that is paid out to owners of credit card receivable asset-backed securities is based only on finance charges collected and fees.

  \item $C$ is correct. Because credit card receivable ABS are backed by non-amortizing loans that do not involve scheduled principal repayments, they are not affected by prepayment risk.

\end{enumerate}

A is incorrect because auto loan ABS are affected by prepayment risk since they are backed by amortizing loans involving scheduled principal repayments.

$B$ is incorrect because residential MBS are affected by prepayment risk since they are backed by amortizing loans involving scheduled principal repayments.

\begin{enumerate}
  \setcounter{enumi}{37}
  \item A is correct. In addition to a senior/subordinated (sequential-pay) structure, many auto loan ABS are structured with additional credit enhancement in the form of overcollateralization and a reserve account, often an excess spread account. The excess spread is an amount that can be retained and deposited into a reserve account that can serve as a first line of protection against losses.
\end{enumerate}

B is incorrect because in an auto loan ABS, losses are typically applied against the excess spread account and the amount of overcollateralization before the waterfall loss absorption of the sequential-pay structure.

$\mathrm{C}$ is incorrect because in auto loan ABS, proceeds from the repossession and resale of autos are prepayment cash flows rather than a form of credit enhancement for loss protection.

\begin{enumerate}
  \setcounter{enumi}{38}
  \item $C$ is correct. In credit card receivable ABS, the only way the principal cash flows can be altered is by triggering the early amortization provision. Such provisions are included in the ABS structure to safeguard the credit quality of the issue. A is incorrect because expiration of the lockout period does not result in the alteration of principal cash flows but instead defines when principal repayments are distributed to the ABS investors. During the lockout period, principal repayments by cardholders are reinvested. When the lockout period expires, principal repayments by cardholders are distributed to investors.
\end{enumerate}

$B$ is incorrect because the excess spread account is a credit enhancement for loss absorption. When the excess spread account is depleted, losses are applied against the overcollateralization amount followed by the senior/subordinated structure. The only way principal cash flows can be altered is by triggering the early amortization provision.

\begin{enumerate}
  \setcounter{enumi}{39}
  \item $C$ is correct. The mezzanine tranche consists of bond classes with credit ratings between senior and subordinated bond classes.
\end{enumerate}

A is incorrect because the equity tranche falls within and carries the credit rating applicable to the subordinated bond classes.

$B$ is incorrect because the residual tranche falls within and carries the credit ratings applicable to the subordinated bond classes.

\begin{enumerate}
  \setcounter{enumi}{40}
  \item $\mathrm{C}$ is correct. The key to whether a $\mathrm{CDO}$ is viable is whether a structure can be created that offers a competitive return for the subordinated tranche (often referred to as the residual or equity tranche). Investors in a subordinated tranche typically use borrowed funds (the bond classes issued) to generate a return above the funding cost.
\end{enumerate}

A is incorrect because the viability of a CDO depends on a structure that offers a competitive return for the subordinated tranche rather than the senior tranche. $\mathrm{B}$ is incorrect because the viability of a $\mathrm{CDO}$ depends on a structure that offers a competitive return for the subordinated tranche rather than the mezzanine tranche. 42. A is correct. When the collateral manager fails pre-specified tests, a provision is triggered that requires the payoff of the principal to the senior class until the tests are satisfied. This reduction of the senior class effectively deleverages the CDO because the CDO's cheapest funding source is reduced.

\begin{enumerate}
  \setcounter{enumi}{42}
  \item B is correct. Covered bonds usually carry lower credit risks and offer lower yields than otherwise similar ABS. The reason is, among other factors, covered bonds provide investors with dual recourse, to the cover pool and also to the issuer. Moreover, covered bonds have a dynamic cover pool, meaning sponsors must replace any prepaid or non-performing assets.
\end{enumerate}

\end{document}