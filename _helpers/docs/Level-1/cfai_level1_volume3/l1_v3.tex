\documentclass[10pt]{article}
\usepackage[utf8]{inputenc}
\usepackage[T1]{fontenc}
\usepackage{amsmath}
\usepackage{amsfonts}
\usepackage{amssymb}
\usepackage[version=4]{mhchem}
\usepackage{stmaryrd}
\usepackage{hyperref}
\hypersetup{colorlinks=true, linkcolor=blue, filecolor=magenta, urlcolor=cyan,}
\urlstyle{same}
\usepackage{graphicx}
\usepackage[export]{adjustbox}
\graphicspath{ {./images/} }
\usepackage{multirow}
\usepackage{CJKutf8}

\title{FINANCIAL STATEMENT ANALYSIS, CORPORATE ISSUERS }


\author{Business model type: Service}
\date{}


\DeclareUnicodeCharacter{00D7}{$\times$}

\begin{document}
\maketitle
CFA $^{\circledR}$ Program Curriculum 2023 • LEVEL 1 • VOLUME 3 @2022 by CFA Institute. All rights reserved. This copyright covers material written expressly for this volume by the editor/s as well as the compilation itself. It does not cover the individual selections herein that first appeared elsewhere. Permission to reprint these has been obtained by CFA Institute for this edition only. Further reproductions by any means, electronic or mechanical, including photocopying and recording, or by any information storage or retrieval systems, must be arranged with the individual copyright holders noted.

CFA $^{\circ}$, Chartered Financial Analyst ${ }^{\circ}$, AIMR-PPS $^{\circ}$, and GIPS ${ }^{\circ}$ are just a few of the trademarks owned by CFA Institute. To view a list of CFA Institute trademarks and the Guide for Use of CFA Institute Marks, please visit our website at \href{http://www.cfainstitute.org}{www.cfainstitute.org}.

This publication is designed to provide accurate and authoritative information in regard to the subject matter covered. It is sold with the understanding that the publisher is not engaged in rendering legal, accounting, or other professional service. If legal advice or other expert assistance is required, the services of a competent professional should be sought.

All trademarks, service marks, registered trademarks, and registered service marks are the property of their respective owners and are used herein for identification purposes only.

ISBN 978-1-950157-98-3 (paper)

ISBN 978-1-953337-25-2 (ebook)

2022

\section{CONTENTS}
How to Use the CFA Program Curriculum xiii

Errata $\quad$ xiii

Designing Your Personal Study Program $\quad$ xiii

CFA Institute Learning Ecosystem (LES) $\quad$ xiv

Feedback $\quad$ xiv

\section{Financial Statement Analysis}
Learning Module 1

Understanding Income Statements 3

Introduction $\quad 4$

Components and Format of the Income Statement $\quad 4$

$\begin{array}{lr}\text { Revenue Recognition } & 10\end{array}$

$\begin{array}{ll}\text { General Principles } & 11\end{array}$

Accounting Standards for Revenue Recognition $\quad 12$

Expense Recognition: General Principles $\quad 15$

General Principles $\quad 16$

Issues in Expense Recognition: Doubtful Accounts, Warranties $\quad 20$

$\begin{array}{lr}\text { Doubtful Accounts } & 20\end{array}$

$\begin{array}{ll}\text { Warranties } & 20\end{array}$

Issues in Expense Recognition: Depreciation and Amortization 21

Implications for Financial Analysts: Expense Recognition 25

Non-Recurring Items and Non-Operating Items: Discontinued Operations and Unusual or Infrequent items

Discontinued Operations $\quad 26$

$\begin{array}{ll}\text { Unusual or Infrequent Items } & 27\end{array}$

Non-Recurring Items: Changes in Accounting Policy 28

Non-Operating Items $\quad 31$

Earnings Per Share and Capital Structure and Basic EPS 32

$\begin{array}{ll}\text { Simple versus Complex Capital Structure } & 32\end{array}$

Basic EPS $\quad 33$

Diluted EPS: the If-Converted Method $\quad 35$

Diluted EPS When a Company Has Convertible Preferred Stock

Outstanding 35

Diluted EPS When a Company Has Convertible Debt Outstanding $\quad 37$

Diluted EPS: the Treasury Stock Method $\quad 38$

Other Issues with Diluted EPS and Changes in EPS $\quad 41$

$\begin{array}{ll}\text { Changes in EPS } & 42 \\ 4\end{array}$

Common-Size Analysis of the Income Statement 42

Common-Size Analysis of the Income Statement $\quad 43$

Income Statement Ratios $\quad 45$

Comprehensive Income $\quad 47$

$\begin{array}{lr}\text { Summary } & 50\end{array}$

Practice Problems $\quad 53$

$\begin{array}{ll}\text { Solutions } & 58\end{array}$ Current Assets: Cash and Cash Equivalents, Marketable Securities and

$\begin{array}{lr}\text { Trade Receivables } & 69\end{array}$

$\begin{array}{lr}\text { Current Assets } & 69\end{array}$

Current Assets: Inventories and Other Current Assets 73

$\begin{array}{ll}\text { Other Current Assets } & 74\end{array}$

$\begin{array}{ll}\text { Current liabilities } & 74\end{array}$

Non-Current Assets: Property, Plant and Equipment and Investment Property 78

$\begin{array}{lr}\text { Property, Plant, and Equipment } & 80\end{array}$

\begin{center}
\begin{tabular}{lr}
Investment Property & 81 \\
\hline
\end{tabular}
\end{center}

$\begin{array}{ll}\text { Non-Current Assets: Intangible Assets } & 81\end{array}$

\begin{center}
\begin{tabular}{lr}
Identifiable Intangibles & 82 \\
\hline
\end{tabular}
\end{center}

Non-Current Assets: Goodwill 84

$\begin{array}{ll}\text { Non-Current Assets: Financial Assets } & 87\end{array}$

Non-Current Assets: Deferred Tax Assets 90

$\begin{array}{ll}\text { Non-Current Liabilities } & 91\end{array}$

Long-term Financial Liabilities $\quad 92$

Deferred Tax Liabilities $\quad 92$

$\begin{array}{lr}\text { Components of Equity } & 93\end{array}$

Components of Equity $\quad 93$

$\begin{array}{ll}\text { Statement of Changes in Equity } & 96\end{array}$

Common Size Analysis of Balance Sheet 98

Common-Size Analysis of the Balance Sheet 98

$\begin{array}{lr}\text { Balance Sheet Ratios } & 105\end{array}$

$\begin{array}{lr}\text { Summary } & 108\end{array}$

$\begin{array}{lr}\text { Practice Problems } & 111\end{array}$

$\begin{array}{lr}\text { Solutions } & 117\end{array}$

Learning Module $3 \quad$ Understanding Cash Flow Statements 121

Introduction 122

Classification Of Cash Flows and Non-Cash Activities 123

Classification of Cash Flows and Non-Cash Activities 123

Cash Flow Statement: Differences Between IFRS and US GAAP 126

Cash Flow Statement: Direct and Indirect Methods for Reporting Cash

Flow from Operating Activities $\quad 127$

Cash Flow Statement: Indirect Method Under IFRS 128

Cash Flow Statement: Direct Method Under IFRS 131

Cash Flow Statement: Direct Method Under US GAAP 133

Cash Flow Statement: Indirect Method Under US GAAP 136

Linkages of Cash Flow Statement with the Income Statement and Balance

Sheet

138

Linkages of the Cash Flow Statement with the Income Statement and Balance Sheet Preparing the Cash Flow Statement: The Direct Method for Operating Activities

Operating Activities: Direct Method $\quad 141$

Preparing the Cash Flow Statement: Investing Activities 145

Preparing the Cash Flow Statement: Financing Activities 147

Long-Term Debt and Common Stock $\quad 147$

Dividends $\quad 147$

Preparing the Cash Flow Statement: Overall Statement of Cash Flows

Under the Direct Method 148

Preparing the Cash Flow Statement: Overall Statement of Cash Flows

Under the Indirect Method $\quad 149$

Conversion of Cash Flows from the Indirect to Direct Method 152

Cash Flow Statement Analysis: Evaluation of Sources and Uses of Cash 153

$\begin{array}{ll}\text { Evaluation of the Sources and Uses of Cash } & 154\end{array}$

Cash Flow Statement Analysis: Common Size Analysis 158

Cash Flow Statement Analysis: Free Cash Flow to Firm and Free Cash Flow to Equity 163

Cash Flow Statement Analysis: Cash Flow Ratios 165

Summary 166

Practice Problems $\quad 168$

$\begin{array}{ll}\text { Solutions } & 175\end{array}$

Learning Module 4

Financial Analysis Techniques 179

Introduction $\quad 179$

$\begin{array}{lr}\text { The Financial Analysis Process } & 180\end{array}$

$\begin{array}{lr}\text { Analytical Tools and Techniques } & 184\end{array}$

Financial Ratio Analysis $\quad 187$

The Universe of Ratios $\quad 188$

Value, Purposes, and Limitations of Ratio Analysis $\quad 190$

Sources of Ratios $\quad 191$

Common Size Balance Sheets and Income Statements 192

Common-Size Analysis of the Balance Sheet 192

Common-Size Analysis of the Income Statement 193

Cross-Sectional, Trend Analysis \& Relationships in Financial Statements 194

Trend Analysis $\quad 195$

Relationships among Financial Statements $\quad 197$

The Use of Graphs and Regression Analysis 198

\begin{center}
\includegraphics[max width=\textwidth]{2023_05_04_b5cfa4f1bc883752f121g-005}
\end{center}

$\begin{array}{ll}\text { Common Ratio Categories \& Interpretation and Context } & 200\end{array}$

$\begin{array}{ll}\text { Interpretation and Context } & 201\end{array}$

Activity Ratios $\quad 202$

$\begin{array}{lr}\text { Calculation of Activity Ratios } & 202\end{array}$

Interpretation of Activity Ratios $\quad 205$

$\begin{array}{ll}\text { Liquidity Ratios } & 208\end{array}$

$\begin{array}{lr}\text { Calculation of Liquidity Ratios } & 209\end{array}$

$\begin{array}{ll}\text { Interpretation of Liquidity Ratios } & 210\end{array}$

Solvency Ratios 213

$\begin{array}{lr}\text { Calculation of Solvency Ratios } & 214\end{array}$

Interpretation of Solvency Ratios $\quad 215$ Profitability Ratios $\quad 218$

$\begin{array}{lr}\text { Calculation of Profitability Ratios } & 218\end{array}$

$\begin{array}{lr}\text { Interpretation of Profitability Ratios } & 219\end{array}$

Integrated Financial Ratio Analysis $\quad 221$

The Overall Ratio Picture: Examples $\quad 222$

DuPont Analysis: The Decomposition of ROE 224

Equity Analysis and Valuation Ratios $\quad 229$

Valuation Ratios $\quad 230$

Industry-Specific Financial Ratios 233

Research on Financial Ratios in Credit and Equity Analysis 234

$\begin{array}{ll}\text { Credit Analysis } & 235\end{array}$

The Credit Rating Process $\quad 236$

Historical Research on Ratios in Credit Analysis 237

Business and Geographic Segments $\quad 238$

$\begin{array}{ll}\text { Segment Reporting Requirements } & 238\end{array}$

Segment Ratios $\quad 239$

Model Building and Forecasting 241

Summary $\quad 242$

References $\quad 244$

Practice Problems $\quad 245$

Solutions $\quad 252$

Learning Module 5

Inventories 255

Introduction 256

Cost of inventories $\quad 257$

Inventory valuation methods $\quad 258$

Specific Identification $\quad 259$

First-In, First-Out (FIFO) $\quad 259$

Weighted Average Cost $\quad 260$

Last-In, First-Out (LIFO) $\quad 260$

$\begin{array}{ll}\text { Calculations of cost of sales, gross profit, and ending inventory } & 260\end{array}$

$\begin{array}{ll}\text { Periodic versus perpetual inventory systems } & 263\end{array}$

$\begin{array}{ll}\text { Comparison of inventory valuation methods } & 265\end{array}$

The LIFO method and LIFO reserve $\quad 268$

LIFO Reserve $\quad 269$

$\begin{array}{ll}\text { LIFO liquidations } & 269\end{array}$

Inventory method changes $\quad 277$

Inventory adjustments $\quad 278$

Evaluation of inventory management: Disclosures \& ratios $\quad 285$

$\begin{array}{ll}\text { Presentation and Disclosure } & 285\end{array}$

Inventory Ratios $\quad 286$

Illustrations of inventory analysis: Adjusting LIFO to FIFO 287

Illustrations of inventory analysis: Impacts of writedowns 291

Summary $\quad 297$

$\begin{array}{lr}\text { Practice Problems } & 300\end{array}$

$\begin{array}{ll}\text { Solutions } & 317\end{array}$ Acquisition of Property, Plant and Equipment $\quad 327$

$\begin{array}{ll}\text { Acquisition of Long-Lived Assets } & 327\end{array}$

Property, Plant, and Equipment $\quad 327$

Acquisition of Intangible Assets 330

Intangible Assets Purchased in Situations Other Than Business

Combinations

Intangible Assets Developed Internally 331

Intangible Assets Acquired in a Business Combination 333

Capitalization versus Expensing: Impact on Financial Statements and Ratios 335

$\begin{array}{ll}\text { Capitalisation of Interest Costs } & 335 \\ & 340\end{array}$

Capitalisation of Interest and Internal Development Costs $\quad 342$

Depreciation of Long-Lived Assets: Methods and Calculation 347

Depreciation Methods and Calculation of Depreciation Expense $\quad 347$

Amortisation of Long-Lived Assets: Methods and Calculation 356

The Revaluation Model 357

$\begin{array}{lr}\text { Impairment of Assets } & 362\end{array}$

Impairment of Property, Plant, and Equipment 362

Impairment of Intangible Assets with a Finite Life $\quad 364$

Impairment of Intangibles with Indefinite Lives $\quad 364$

Impairment of Long-Lived Assets Held for Sale 365

$\begin{array}{ll}\text { Reversals of Impairments of Long-Lived Assets } & 365\end{array}$

Derecognition $\quad 365$

Sale of Long-Lived Assets $\quad 366$

Long-Lived Assets Disposed of Other Than by a Sale $\quad 366$

Presentation and Disclosure Requirements $\quad 368$

Using Disclosures in Analysis $\quad 375$

$\begin{array}{ll}\text { Investment Property } & 379\end{array}$

Summary $\quad 382$

Practice Problems $\quad 385$

$\begin{array}{ll}\text { Solutions } & 398\end{array}$

Learning Module $7 \quad$ Income Taxes 405

Introduction 406

Differences Between Accounting Profit and Taxable Income 406

Current and Deferred Tax Assets and Liabilities 408

Deferred Tax Assets and Liabilities 408

Determining the Tax Base of Assets and Liabilities 412

Determining the Tax Base of an Asset 412

$\begin{array}{lr}\text { Determining the Tax Base of a Liability } & 414\end{array}$

Changes in Income Tax Rates 416

Temporary and Permanent Differences Between Taxable and Accounting

$\begin{array}{lr}\text { Profit } & 417\end{array}$

$\begin{array}{lr}\text { Taxable Temporary Differences } & 418\end{array}$

Deductible Temporary Differences $\quad 418$

Examples of Taxable and Deductible Temporary Differences $\quad 419$

Exceptions to the Usual Rules for Temporary Differences $\quad 421$

Business Combinations and Deferred Taxes 422

Investments in Subsidiaries, Branches, Associates and Interests in Joint Ventures Unused Tax Losses and Tax Credits 422

Recognition and Measurement of Current and Deferred Tax 423

$\begin{array}{ll}\text { Recognition of a Valuation Allowance } & 424\end{array}$

Recognition of Current and Deferred Tax Charged Directly to Equity 424

Presentation and Disclosure $\quad 427$

Comparison of IFRS and US GAAP 433

Summary 435

$\begin{array}{lr}\text { Practice Problems } & 437\end{array}$

$\begin{array}{ll}\text { Solutions } & 442\end{array}$

\section{Learning Module 8}
\section{Learning Module 9}
Non-Current (Long-Term) Liabilities $\quad 445$

Introduction 445

Bonds Payable \& Accounting for Bond Issuance $\quad 446$

Accounting for Bond Issuance $\quad 446$

Accounting for Bond Amortisation, Interest Expense, and Interest Payments 450

$\begin{array}{lr}\text { Accounting for Bonds at Fair Value } & 455\end{array}$

Derecognition of Debt 458

Debt Covenants $\quad 461$

Presentation and Disclosure of Long-Term Debt 463

$\begin{array}{ll}\text { Leases } & 466\end{array}$

$\begin{array}{ll}\text { Examples of Leases } & 467\end{array}$

$\begin{array}{ll}\text { Advantages of Leasing } & 467\end{array}$

Lease Classification as Finance or Operating $\quad 467$

$\begin{array}{ll}\text { Financial Reporting of Leases } & 469\end{array}$

$\begin{array}{lr}\text { Lessee Accounting-IFRS } & 469\end{array}$

$\begin{array}{ll}\text { Lessee Accounting-US GAAP } & 471\end{array}$

$\begin{array}{ll}\text { Lessor Accounting } & 473\end{array}$

Introduction to Pensions and Other Post-Employment Benefits 475

$\begin{array}{lr}\text { Evaluating Solvency: Leverage and Coverage Ratios } & 479\end{array}$

Summary $\quad 482$

$\begin{array}{lr}\text { Practice Problems } & 485\end{array}$

$\begin{array}{ll}\text { Solutions } & 491\end{array}$

Financial Reporting Quality 497

Introduction \& Conceptual Overview 498

Conceptual Overview 499

$\begin{array}{lr}\text { GAAP, Decision Useful Financial Reporting } & 500\end{array}$

GAAP, Decision-Useful, but Sustainable? 501

$\begin{array}{ll}\text { Biased Accounting Choices } & 502\end{array}$

Within GAAP, but "Earnings Management" $\quad 510$

Departures from GAAP $\quad 511$

Differentiate between Conservative and Aggressive Accounting 512

Conservatism in Accounting Standards 513

Bias in the Application of Accounting Standards $\quad 515$

Context for Assessing Financial Reporting Quality 516

$\begin{array}{ll}\text { Motivations } & 516\end{array}$

Conditions Conducive to Issuing Low-Quality Financial Reports 517

Mechanisms That Discipline Financial Reporting Quality 517

$\begin{array}{lr}\text { Market Regulatory Authorities } & 518\end{array}$ $\begin{array}{lr}\text { Auditors } & 520\end{array}$

Private Contracting 523

Detection of Financial Reporting Quality Issues: Introduction \&

$\begin{array}{lr}\text { Presentation Choices } & 525\end{array}$

$\begin{array}{lr}\text { Presentation Choices } & 526\end{array}$

Accounting Choices and Estimates 532

How Accounting Choices and Estimates Affect Earnings and Balance

Sheets

533

How Choices that Affect the Cash Flow Statement 543

$\begin{array}{ll}\text { Choices that Affect Financial Reporting } & 547\end{array}$

$\begin{array}{ll}\text { Warning Signs } & 550\end{array}$

\begin{enumerate}
  \item Pay attention to revenue. $\quad 550$

  \item Pay attention to signals from inventories. 552

  \item Pay attention to capitalization policies and deferred costs. 552

  \item Pay attention to the relationship of cash flow and net income. 553

  \item Other potential warnings signs. 553

\end{enumerate}

Summary $\quad 556$

$\begin{array}{lr}\text { References } & 558\end{array}$

$\begin{array}{lr}\text { Practice Problems } & 559 \\ \text { Solutions }\end{array}$

$\begin{array}{ll}\text { Solutions } & 563\end{array}$

Learning Module 10

Applications of Financial Statement Analysis

Introduction \& Evaluating Past Financial Performance $\quad 567$

Application: Evaluating Past Financial Performance $\quad 568$

Application: Projecting Future Financial Performance as an Input to

Market Based Valuation $\quad 572$

Projecting Performance: An Input to Market-Based Valuation 573

$\begin{array}{ll}\text { Projecting Multiple-Period Performance } & 577\end{array}$

Application: Assessing Credit Risk $\quad 581$

Screening for Potential Equity Investments 583

Framework for Analyst Adjustments \& Adjustments to Investments \&

Adjustments to Inventory

$\begin{array}{ll}\text { A Framework for Analyst Adjustments } & 587\end{array}$

Analyst Adjustments Related to Investments $\quad 587$

Analyst Adjustments Related to Inventory 588

Adjustments Related to Property, Plant, and Equipment 591

Adjustments Related to Goodwill 593

$\begin{array}{ll}\text { Summary } & 596\end{array}$

$\begin{array}{lr}\text { References } & 597\end{array}$

Practice Problems 598

$\begin{array}{ll}\text { Solutions } & 601\end{array}$

\section{Corporate Issuers}
\section{Learning Module 1}
Corporate Structures and Ownership

$\begin{array}{ll}\text { Limited Partnership } & 608 \\ \text { Corporation (Limited Companies) } & 609 \\ \text { Public and Private Corporations } & 616 \\ \text { Exchange Listing and Share Ownership Transfer } & 616 \\ \text { Share Issuance } & 6219 \\ \text { Going Public from Private - IPO, Direct Listing, Acquisition } & 623 \\ \text { Life Cycle of Corporations } & 626 \\ \text { Lenders and Owners } & 633 \\ \text { Equity and Debt Risk-Return Profiles } & 634 \\ \text { Equity vs. Debt Conflicts of Interest } & 636 \\ \text { Summary } & 637 \\ \text { Practice Problems } & 636 \\ \text { Solutions } & \end{array}$

Learning Module 2

Introduction to Corporate Governance and Other ESG Considerations

$\begin{array}{ll}\text { Stakeholder Groups } & 640\end{array}$

$\begin{array}{lr}\text { Shareholder vs. Stakeholder Theory } & 641\end{array}$

$\begin{array}{lr}\text { Shareholders } & 641\end{array}$

$\begin{array}{lr}\text { Creditors/Debtholders } & 642\end{array}$

$\begin{array}{lr}\text { Board of Directors } & 642\end{array}$

$\begin{array}{lr}\text { Managers } & 643\end{array}$

Employees $\quad 643$

$\begin{array}{lr}\text { Customers } & 643\end{array}$

$\begin{array}{lr}\text { Suppliers } & 644\end{array}$

Governments $\quad 644$

Principal-Agent and Other Relationships $\quad 645$

Shareholder and Manager/Director Relationships $\quad 646$

Controlling and Minority Shareholder Relationships $\quad 648$

$\begin{array}{lr}\text { Manager and Board Relationships } & 651\end{array}$

$\begin{array}{lr}\text { Shareholder vs. Creditor (Debtholder) Interests } & 652\end{array}$

Corporate Governance and Mechanisms to Manage Stakeholder Risks 653

$\begin{array}{lr}\text { Shareholder Mechanisms } & 655\end{array}$

$\begin{array}{lr}\text { Creditor Mechanisms } & 658\end{array}$

Board of Director and Management Mechanisms $\quad 659$

$\begin{array}{lr}\text { Employee Mechanisms } & 662\end{array}$

$\begin{array}{lr}\text { Customer and Supplier Mechanisms } & 663\end{array}$

Government Mechanisms $\quad 663$

Corporate Governance and Stakeholder Management Risks and Benefits 665

$\begin{array}{ll}\text { Operational Risks and Benefits } & 665\end{array}$

Legal, Regulatory, or Reputational Risks and Benefits $\quad 667$

$\begin{array}{lr}\text { Financial Risks and Benefits } & 668\end{array}$

ESG Considerations in Investment Analysis $\quad 671$

Introduction to Environmental and Social Factors $\quad 672$

Evaluating ESG-Related Risks and Opportunities $\quad 673$

Environmental, Social, and Governance Investment Approaches $\quad 675$

$\begin{array}{lr}\text { ESG Investment Approaches } & 675\end{array}$

ESG Market Overview $\quad 678$

$\begin{array}{lr}\text { Summary } & 679\end{array}$ $\begin{array}{lr}\text { References } & 681\end{array}$

$\begin{array}{lr}\text { Practice Problems } & 682\end{array}$

$\begin{array}{ll}\text { Solutions } & 685\end{array}$

Learning Module 3

Business Models \& Risks $\quad 687$

Introductory Context/Motivation $\quad 687$

What Is a Business Model? $\quad 687$

$\begin{array}{lr}\text { Business Model Features } & 688\end{array}$

$\begin{array}{lr}\text { Customers, Market: Who } & 689\end{array}$

$\begin{array}{lr}\text { Firm Offering: What } & 690\end{array}$

$\begin{array}{lr}\text { Channels: Where } & 690\end{array}$

Pricing: How Much $\quad 693$

Value Proposition (Who + What + Where + How Much) $\quad 696$

$\begin{array}{ll}\text { Business Organization, Capabilities: How } & 698\end{array}$

$\begin{array}{lr}\text { Profitability and Unit Economics } & 699\end{array}$

$\begin{array}{ll}\text { Business Model Types } & 700\end{array}$

$\begin{array}{ll}\text { Business Model Innovation } & 701\end{array}$

Business Model Variations $\quad 702$

E-Commerce Business Models $\quad 702$

$\begin{array}{ll}\text { Network Effects and Platform Business Models } & 703\end{array}$

$\begin{array}{ll}\text { Crowdsourcing Business Models } & 703\end{array}$

$\begin{array}{ll}\text { Hybrid Business Models } & 703\end{array}$

$\begin{array}{ll}\text { Business Models: Financial Implications } & 706\end{array}$

$\begin{array}{ll}\text { External Factors } & 706\end{array}$

$\begin{array}{lr}\text { Firm-Specific Factors } & 707\end{array}$

Business Risks $\quad 711$

$\begin{array}{ll}\text { Summary } & 711\end{array}$

Macro Risk, Business Risk, and Financial Risk 711

$\begin{array}{lr}\text { Risk Impacts Are Cumulative } & 713\end{array}$

Business Risk: A Closer Look 713

$\begin{array}{lr}\text { Industry Risks } & 714\end{array}$

$\begin{array}{ll}\text { Industry Definition } & 715\end{array}$

$\begin{array}{lr}\text { Company-Specific Risks } & 715\end{array}$

$\begin{array}{ll}\text { Financial Risk } & 717\end{array}$

$\begin{array}{ll}\text { Measuring Operating and Financial Leverage } & 718\end{array}$

$\begin{array}{ll}\text { Summary } & 719\end{array}$

$\begin{array}{ll}\text { Practice Problems } & 721\end{array}$

$\begin{array}{ll}\text { Solutions } & 724\end{array}$

$\begin{array}{lrr}\text { Learning Module } 4 & \text { Capital Investments } & 727\end{array}$

Introduction $\quad 727$

Types of Capital Investments $\quad 728$

Business Maintenance $\quad 729$

Business Growth $\quad 730$

$\begin{array}{ll}\text { The Capital Allocation Process } & 732\end{array}$

Investment Decision Criteria $\quad 737$

Net Present Value $\quad 737$

Internal Rate of Return 739

Common Capital Allocation Pitfalls 742 Corporate Use of Capital Allocation $\quad 744$

$\begin{array}{lr}\text { Real Options } & 747\end{array}$

\begin{center}
\includegraphics[max width=\textwidth]{2023_05_04_b5cfa4f1bc883752f121g-012}
\end{center}

\begin{center}
\begin{tabular}{lr}
Sizing Options & 748 \\
\hline
Fexiblity Options & 748 \\
\hline
\end{tabular}
\end{center}

$\begin{array}{lr}\text { Flexibility Options } & 748\end{array}$

$\begin{array}{lr}\text { Fundamental Options } & 748\end{array}$

$\begin{array}{ll}\text { Summary } & 750\end{array}$

\begin{center}
\begin{tabular}{lr}
Practice Problems & 750 \\
\hline
$o l u t i o n s$ & 752 \\
\hline
\end{tabular}
\end{center}

$\begin{array}{ll}\text { Solutions } & 757\end{array}$

Learning Module $5 \quad$ Working Capital \& Liquidity 761

Introduction $\quad 761$

$\begin{array}{lr}\text { Financing Options } & 762\end{array}$

$\begin{array}{ll}\text { Internal Financing } & 763\end{array}$

External Financing: Financial Intermediaries 765

$\begin{array}{ll}\text { External Financing: Capital Markets } & 766\end{array}$

Working Capital, Liquidity, and Short-Term Funding Needs 768

$\begin{array}{ll}\text { Liquidity and Short-Term Funding } & 774\end{array}$

$\begin{array}{ll}\text { Primary Sources of Liquidity } & 775\end{array}$

Secondary Sources of Liquidity $\quad 776$

Drags and Pulls on Liquidity $\quad 777$

$\begin{array}{lr}\text { Measuring Liquidity } & 778\end{array}$

$\begin{array}{lr}\text { Evaluating Short-Term Financing Choices } & 782\end{array}$

Summary 784

$\begin{array}{lr}\text { Practice Problems } & 785\end{array}$

$\begin{array}{ll}\text { Solutions } & 788\end{array}$

\section{How to Use the CFA Program Curriculum}
The CFA Program exams measure your mastery of the core knowledge, skills, and abilities required to succeed as an investment professional. These core competencies are the basis for the Candidate Body of Knowledge $\left(\mathrm{CBOK}^{\mathrm{m}}\right)$ ). The CBOK consists of four components:

\begin{itemize}
  \item A broad outline that lists the major CFA Program topic areas (www. \href{http://cfainstitute.org/programs/cfa/curriculum/cbok}{cfainstitute.org/programs/cfa/curriculum/cbok})

  \item Topic area weights that indicate the relative exam weightings of the top-level topic areas (\href{http://www.cfainstitute.org/programs/cfa/curriculum}{www.cfainstitute.org/programs/cfa/curriculum})

  \item Learning outcome statements (LOS) that advise candidates about the specific knowledge, skills, and abilities they should acquire from curriculum content covering a topic area: LOS are provided in candidate study sessions and at the beginning of each block of related content and the specific lesson that covers them. We encourage you to review the information about the LOS on our website (\href{http://www.cfainstitute.org/programs/cfa/curriculum/}{www.cfainstitute.org/programs/cfa/curriculum/} study-sessions), including the descriptions of LOS "command words" on the candidate resources page at \href{http://www.cfainstitute.org}{www.cfainstitute.org}.

  \item The CFA Program curriculum that candidates receive upon exam registration

\end{itemize}

Therefore, the key to your success on the CFA exams is studying and understanding the CBOK. You can learn more about the CBOK on our website: www.cfainstitute. org/programs/cfa/curriculum/cbok.

The entire curriculum, including the practice questions, is the basis for all exam questions and is selected or developed specifically to teach the knowledge, skills, and abilities reflected in the $\mathrm{CBOK}$.

\section{ERRATA}
The curriculum development process is rigorous and includes multiple rounds of reviews by content experts. Despite our efforts to produce a curriculum that is free of errors, there are instances where we must make corrections. Curriculum errata are periodically updated and posted by exam level and test date online on the Curriculum Errata webpage (\href{http://www.cfainstitute.org/en/programs/submit-errata}{www.cfainstitute.org/en/programs/submit-errata}). If you believe you have found an error in the curriculum, you can submit your concerns through our curriculum errata reporting process found at the bottom of the Curriculum Errata webpage.

\section{DESIGNING YOUR PERSONAL STUDY PROGRAM}
An orderly, systematic approach to exam preparation is critical. You should dedicate a consistent block of time every week to reading and studying. Review the LOS both before and after you study curriculum content to ensure that you have mastered the applicable content and can demonstrate the knowledge, skills, and abilities described by the LOS and the assigned reading. Use the LOS self-check to track your progress and highlight areas of weakness for later review.

Successful candidates report an average of more than 300 hours preparing for each exam. Your preparation time will vary based on your prior education and experience, and you will likely spend more time on some study sessions than on others.

\section{CFA INSTITUTE LEARNING ECOSYSTEM (LES)}
Your exam registration fee includes access to the CFA Program Learning Ecosystem (LES). This digital learning platform provides access, even offline, to all of the curriculum content and practice questions and is organized as a series of short online lessons with associated practice questions. This tool is your one-stop location for all study materials, including practice questions and mock exams, and the primary method by which CFA Institute delivers your curriculum experience. The LES offers candidates additional practice questions to test their knowledge, and some questions in the LES provide a unique interactive experience.

\section{FEEDBACK}
Please send any comments or feedback to \href{mailto:info@cfainstitute.org}{info@cfainstitute.org}, and we will review your suggestions carefully.

\section{Financial Statement Analysis}
\section*{LEARNING MODULE 
 1 }
\section{Understanding Income Statements}
Elaine Henry, PhD, CFA, is at Stevens Institute of Technology (USA). Thomas R Robinson, PhD, CFA, CAIA, Robinson Global Investment Management LLC, (USA).

\section{LEARNING OUTCOME}
\begin{center}
\begin{tabular}{|c|c|}
\hline
Mastery & The candidate should be able to: \\
\hline
 & $\begin{array}{l}\text { describe the components of the income statement and alternative } \\ \text { presentation formats of that statement }\end{array}$ \\
\hline
 & $\begin{array}{l}\text { describe general principles of revenue recognition and accounting } \\ \text { standards for revenue recognition }\end{array}$ \\
\hline
 & $\begin{array}{l}\text { calculate revenue given information that might influence the choice } \\ \text { of revenue recognition method }\end{array}$ \\
\hline
 & $\begin{array}{l}\text { describe general principles of expense recognition, specific expense } \\ \text { recognition applications, and implications of expense recognition } \\ \text { choices for financial analysis }\end{array}$ \\
\hline
 & $\begin{array}{l}\text { describe the financial reporting treatment and analysis of } \\ \text { non-recurring items (including discontinued operations, unusual on } \\ \text { infrequent items) and changes in accounting policies }\end{array}$ \\
\hline
 & $\begin{array}{l}\text { contrast operating and non-operating components of the income } \\ \text { statement }\end{array}$ \\
\hline
 & $\begin{array}{l}\text { describe how earnings per share is calculated and calculate and } \\ \text { interpret a company's earnings per share (both basic and diluted } \\ \text { earnings per share) for both simple and complex capital structures }\end{array}$ \\
\hline
 & $\begin{array}{l}\text { contrast dilutive and antidilutive securities and describe the } \\ \text { implications of each for the earnings per share calculation }\end{array}$ \\
\hline
 & $\begin{array}{l}\text { formulate income statements into common-size income statements } \\ \text { evaluate a company's financial performance using common-size } \\ \text { income statements and financial ratios based on the income } \\ \text { statement }\end{array}$ \\
\hline
 & describe, calculate, and interpret comprehensive income \\
\hline
 & $\begin{array}{l}\text { describe other comprehensive income and identify major types of } \\ \text { items included in it }\end{array}$ \\
\hline
\end{tabular}
\end{center}

Note: Changes in accounting standards as well as new rulings and/or pronouncements issued after the publication of the readings on financial reporting and analysis may cause some of the information in these readings to become dated. Candidates are not responsible for anything that occurs after the readings were published. In addition, candidates are expected to be familiar with the analytical frameworks contained in the readings, as well as the implications of alternative accounting methods for financial analysis and valuation discussed in the readings. Candidates are also responsible for the content of accounting standards, but not for the actual reference numbers. Finally, candidates should be aware that certain ratios may be defined and calculated differently. When alternative ratio definitions exist and no specific definition is given, candidates should use the ratio definitions emphasized in the readings.

\section{INTRODUCTION}
The income statement presents information on the financial results of a company's business activities over a period of time. The income statement communicates how much revenue the company generated during a period and what costs it incurred in connection with generating that revenue. The basic equation underlying the income statement, ignoring gains and losses, is Revenue minus Expenses equals Net income. The income statement is also sometimes referred to as the "statement of operations," "statement of earnings," or "profit and loss (P\&L) statement." Under both International Financial Reporting Standards (IFRS) and US generally accepted accounting principles (US GAAP), the income statement may be presented as a separate statement followed by a statement of comprehensive income that begins with the profit or loss from the income statement or as a section of a single statement of comprehensive income. ${ }^{1}$ This reading focuses on the income statement, and the term income statement will be used to describe either the separate statement that reports profit or loss used for earnings per share calculations or that section of a statement of comprehensive income that reports the same profit or loss. The reading also includes a discussion of comprehensive income (profit or loss from the income statement plus other comprehensive income).

Investment analysts intensely scrutinize companies' income statements. Equity analysts are interested in them because equity markets often reward relatively highor low-earnings growth companies with above-average or below-average valuations, respectively, and because inputs into valuation models often include estimates of earnings. Fixed-income analysts examine the components of income statements, past and projected, for information on companies' abilities to make promised payments on their debt over the course of the business cycle. Corporate financial announcements frequently emphasize information reported in income statements, particularly earnings, more than information reported in the other financial statements.

This reading is organized as follows: Section 2 describes the components of the income statement and its format. Section 3 describes basic principles and selected applications related to the recognition of revenue, and Section 4 describes basic principles and selected applications related to the recognition of expenses. Section 5 covers non-recurring items and non-operating items. Section 6 explains the calculation of earnings per share. Section 7 introduces income statement analysis, and Section 8 explains comprehensive income and its reporting. A summary of the key points and practice problems in the CFA Institute multiple choice format complete the reading.

\section{COMPONENTS AND FORMAT OF THE INCOME STATEMENT}
describe the components of the income statement and alternative presentation formats of that statement

1 International Accounting Standard (IAS) 1, Presentation of Financial Statements, establishes the presentation and minimum content requirements of financial statements and guidelines for the structure of financial statements under IFRS. Under US GAAP, the Financial Accounting Standards Board Accounting Standards Codification ASC Section 220-10-45 [Comprehensive Income-Overall-Other Presentation Matters] discusses acceptable formats in which to present income, other comprehensive income, and comprehensive income. Exhibit 1, Exhibit 2, and Exhibit 3 show the income statements for Anheuser-Busch InBev SA/NV (AB InBev), a multinational beverage company based in Belgium, Molson Coors Brewing Company (Molson Coors), a US-based multinational brewing company, and Groupe Danone (Danone), a French food manufacturer. ${ }^{2}$ AB InBev and Danone report under IFRS, and Molson Coors reports under US GAAP. Note that both AB InBev and Molson Coors show three years' income statements and list the years in chronological order with the most recent year listed in the left-most column. In contrast, Danone shows two years of income statements and lists the years in chronological order from left to right with the most recent year in the right-most column. Different orderings of chronological information are common.

On the top line of the income statement, companies typically report revenue. Revenue generally refers to the amount charged for the delivery of goods or services in the ordinary activities of a business. Revenue may also be called sales or turnover. ${ }^{3}$ For the year ended 31 December 2017, AB InBev reports $\$ 56.44$ billion of revenue, Molson Coors reports $\$ 13.47$ billion of revenue (labeled "sales"), and Danone reports $€ 24.68$ billion of revenue (labeled "sales").

Revenue is reported after adjustments (e.g., for cash or volume discounts, or for other reductions), and the term net revenue is sometimes used to specifically indicate that the revenue has been adjusted (e.g., for estimated returns). For all three companies in Exhibit 1 through Exhibit 3, footnotes to their financial statements (not shown here) state that revenues are stated net of such items as returns, customer rebates, trade discounts, or volume-based incentive programs for customers.

In a comparative analysis, an analyst may need to reference information disclosed elsewhere in companies' annual reports-typically the notes to the financial statements and the Management Discussion and Analysis (MD\&A) - to identify the appropriately comparable revenue amounts. For example, excise taxes represent a significant expenditure for brewing companies. On its income statement, Molson Coors reports $\$ 13.47$ billion of revenue (labeled "sales") and $\$ 11.00$ billion of net revenue (labeled "net sales"), which equals sales minus $\$ 2.47$ billion of excise taxes. Unlike Molson Coors, AB InBev does not show the amount of excise taxes on its income statement. However, in its disclosures, $A B$ InBev notes that excise taxes (amounting to $\$ 15.4$ billion in 2017) have been deducted from the revenue amount shown on its income statement. Thus, the amount on AB InBev's income statement labeled "revenue" is more comparable to the amount on Molson Coors' income statement labeled "net sales."

Exhibit 1: Anheuser-Busch InBev SA/NV Consolidated Income Statement (in Millions of US Dollars) [Excerpt]

12 Months Ended December 31

\begin{center}
\begin{tabular}{|c|c|c|c|}
\hline
 & 2017 & 2016 & 2015 \\
\hline
Revenue & $\$ 56,444$ & $\$ 45,517$ & $\$ 43,604$ \\
\hline
Cost of sales & $(21,386)$ & $(17,803)$ & $(17,137)$ \\
\hline
Gross profit & 35,058 & 27,715 & 26,467 \\
\hline
Distribution expenses & $(5,876)$ & $(4,543)$ & $(4,259)$ \\
\hline
\end{tabular}
\end{center}

2 Following net income, the income statement also presents earnings per share, the amount of earnings per common share of the company. Earnings per share will be discussed in detail later in this reading, and the per-share display has been omitted from these exhibits to focus on the core income statement.

3 Sales is sometimes understood to refer to the sale of goods, whereas revenue can include the sale of goods or services; however, the terms are often used interchangeably. In some countries, the term "turnover" may be used in place of revenue.

\begin{center}
\begin{tabular}{|c|c|c|c|}
\hline
 & \multicolumn{3}{|c|}{12 Months Ended December 31} \\
\hline
 & 2017 & 2016 & 2015 \\
\hline
Sales and marketing expenses & $(8,382)$ & $(7,745)$ & $(6,913$ \\
\hline
Administrative expenses & $(3,841)$ & $(2,883)$ & $(2,560$ \\
\hline
Other operating income/(expenses) & 854 & 732 & 1,032 \\
\hline
Restructuring & $(468)$ & $(323)$ & $(171)$ \\
\hline
Business and asset disposal & $(39)$ & 377 & 524 \\
\hline
Acquisition costs business combinations & $(155)$ & $(448)$ & $(55)$ \\
\hline
Impairment of assets & - & - & $(82)$ \\
\hline
Judicial settlement & - & - & $(80)$ \\
\hline
Profit from operations & 17,152 & 12,882 & 13,904 \\
\hline
Finance cost & $(6,885)$ & $(9,216)$ & $(3,142$ \\
\hline
Finance income & 378 & 652 & 1,689 \\
\hline
Net finance income/(cost) & $(6,507)$ & $(8,564)$ & $(1,453$ \\
\hline
Share of result of associates and joint ventures & 430 & 16 & 10 \\
\hline
Profit before tax & 11,076 & 4,334 & 12,461 \\
\hline
Income tax expense & $(1,920)$ & $(1,613)$ & $(2,594$ \\
\hline
Profit from continuing operations & 9,155 & 2,721 & 9,867 \\
\hline
Profit from discontinued operations & 28 & 48 &  \\
\hline
Profit of the year & 9,183 & 2,769 & 9,867 \\
\hline
\multicolumn{4}{|l|}{Profit from continuing operations attributable to:} \\
\hline
Equity holders of $A B$ InBev & 7,968 & 1,193 & 8,273 \\
\hline
Non-controlling interest & 1,187 & 1,528 & 1,594 \\
\hline
\multicolumn{4}{|l|}{Profit of the year attributable to:} \\
\hline
Equity holders of AB InBev & 7,996 & 1,241 & 8,273 \\
\hline
Non-controlling interest & $\$ 1,187$ & $\$ 1,528$ & $\$ 1,594$ \\
\hline
\end{tabular}
\end{center}

Note: reported total amounts may have slight discrepancies due to rounding

\section{Exhibit 2: Molson Coors Brewing Company Consolidated Statement of Operations (in Millions of US}
 Dollars) [Excerpt]12 Months Ended

Dec. 31,2017

Dec. 31,2016

Dec. 31,2015

Sales

Excise taxes

Net sales

Cost of goods sold

Gross profit

Marketing, general and administrative expenses

Special items, net

Equity Income in MillerCoors

Operating income (loss)

Other income (expense), net

\begin{center}
\begin{tabular}{ccc}
$\$ 13,471.5$ & $\$ 6,597.4$ & $\$ 5,127.4$ \\
$(2,468.7)$ & $(1,712.4)$ & $(1,559.9)$ \\
\hline
$11,002.8$ & $4,885.0$ & $3,567.5$ \\
$(6,217.2)$ & $(2,987.5)$ & $(2,131.6)$ \\
\hline
$4,785.6$ & $1,897.5$ & $1,435.9$ \\
$(3,032.4)$ & $(1,589.8)$ & $(1,038.3)$ \\
$(28.1)$ & $2,522.4$ & $(346.7)$ \\
0 & 500.9 & 516.3 \\
\hline
$1,725.1$ & $3,331.0$ & 567.2 \\
\hline
\end{tabular}
\end{center}

\begin{center}
\begin{tabular}{|c|c|c|c|}
\hline
 & \multicolumn{3}{|c|}{12 Months Ended} \\
\hline
 & Dec. 31,2017 & Dec. 31,2016 & Dec. 31,2015 \\
\hline
Interest expense & $(349.3)$ & $(271.6)$ & $(120.3)$ \\
\hline
Interest income & 6.0 & 27.2 & 8.3 \\
\hline
Other income (expense), net & $(0.1)$ & $(29.7)$ & 0.9 \\
\hline
Total other income (expense), net & $(343.4)$ & $(274.1)$ & $(111.1)$ \\
\hline
Income (loss) from continuing operations before income taxes & $1,381.7$ & $3,056.9$ & 456.1 \\
\hline
Income tax benefit (expense) & 53.2 & $(1,055.2)$ & $(61.5)$ \\
\hline
Net income (loss) from continuing operations & $1,434.9$ & $2,001.7$ & 394.6 \\
\hline
Income (loss) from discontinued operations, net of tax & 1.5 & $(2.8)$ & 3.9 \\
\hline
Net income (loss) including noncontrolling interests & $1,436.4$ & $1,998.9$ & 398.5 \\
\hline
Net (income) loss attributable to noncontrolling interests & $(22.2)$ & $(5.9)$ & $(3.3)$ \\
\hline
Net income (loss) attributable to Molson Coors Brewing Company & $\$ 1,414.2$ & $\$ 1,993.0$ & $\$ 395.2$ \\
\hline
\end{tabular}
\end{center}

Exhibit 3: Groupe Danone Consolidated Income Statement (in Millions of Euros) [Excerpt]

\begin{center}
\begin{tabular}{|c|c|c|}
\hline
 & \multicolumn{2}{|c|}{Year Ended 31 December} \\
\hline
 & 2016 & 2017 \\
\hline
Sales & 21,944 & 24,677 \\
\hline
Cost of goods sold & $(10,744)$ & $(12,459)$ \\
\hline
Selling expense & $(5,562)$ & $(5,890)$ \\
\hline
General and administrative expense & $(2,004)$ & $(2,225)$ \\
\hline
Research and development expense & $(333)$ & $(342)$ \\
\hline
Other income (expense) & $(278)$ & $(219)$ \\
\hline
Recurring operating income & 3,022 & 3,543 \\
\hline
Other operating income (expense) & $(99)$ & 192 \\
\hline
Operating income & 2,923 & 3,734 \\
\hline
$\begin{array}{l}\text { Interest income on cash equivalents and } \\ \text { short-term investments }\end{array}$ & 130 & 151 \\
\hline
Interest expense & $(276)$ & $(414)$ \\
\hline
Cost of net debt & $(146)$ & $(263)$ \\
\hline
Other financial income & 67 & 137 \\
\hline
Other financial expense & $(214)$ & $(312)$ \\
\hline
Income before tax & 2,630 & 3,296 \\
\hline
Income tax expense & $(804)$ & $(842)$ \\
\hline
Net income from fully consolidated companies & 1,826 & 2,454 \\
\hline
Share of profit of associates & 1 & 109 \\
\hline
Net income & 1,827 & 2,563 \\
\hline
Net income - Group share & 1,720 & 2,453 \\
\hline
Net income - Non-controlling interests & 107 & 110 \\
\hline
\end{tabular}
\end{center}

Differences in presentations of items, such as expenses, are also common. Expenses reflect outflows, depletions of assets, and incurrences of liabilities in the course of the activities of a business. Expenses may be grouped and reported in different formats, subject to some specific requirements.

At the bottom of the income statement, companies report net income (companies may use other terms such as "net earnings" or "profit or loss"). For 2017, AB InBev reports $\$ 9,183$ million "Profit of the year", Molson Coors reports $\$ 1,436.4$ million of net income including noncontrolling interests, and Danone reports $€ 2,563$ million of net income. Net income is often referred to as the "bottom line." The basis for this expression is that net income is the final-or bottom-line item in an income statement. Because net income is often viewed as the single most relevant number to describe a company's performance over a period of time, the term "bottom line" sometimes is used in business to refer to any final or most relevant result.

Despite this customary terminology, note that each company presents additional items below net income: information about how much of that net income is attributable to the company itself and how much of that income is attributable to noncontrolling interests, also known as minority interests. The companies consolidate subsidiaries over which they have control. Consolidation means that they include all of the revenues and expenses of the subsidiaries even if they own less than 100 percent. Noncontrolling interest represents the portion of income that "belongs" to the minority shareholders of the consolidated subsidiaries, as opposed to the parent company itself. For AB InBev, \$7,996 million of the total profit is attributable to the shareholders of $\mathrm{AB}$ InBev, and \$1,187 million is attributable to noncontrolling interests. For Molson Coors, $\$ 1,414.2$ million is attributable to the shareholders of Molson Coors, and $\$ 22.2$ million is attributable to noncontrolling interests. For Danone, $€ 2,453$ million of the net income amount is attributable to shareholders of Groupe Danone and $€ 110$ million is attributable to noncontrolling interests.

Net income also includes gains and losses, which are increases and decreases in economic benefits, respectively, which may or may not arise in the ordinary activities of the business. For example, when a manufacturing company sells its products, these transactions are reported as revenue, and the costs incurred to generate these revenues are expenses and are presented separately. However, if a manufacturing company sells surplus land that is not needed, the transaction is reported as a gain or a loss. The amount of the gain or loss is the difference between the carrying value of the land and the price at which the land is sold. For example, in Exhibit 1, AB InBev reports a loss (proceeds, net of carrying value) of $\$ 39$ million on disposals of businesses and assets in fiscal 2017, and gains of \$377 million and \$524 million in 2016 and 2015, respectively. Details on these gains and losses can typically be found in the companies' disclosures. For example, AB InBev discloses that the $\$ 377$ million gain in 2016 was mainly from selling one of its breweries in Mexico.

The definition of income encompasses both revenue and gains and the definition of expenses encompasses both expenses that arise in the ordinary activities of the business and losses. ${ }^{4}$ Thus, net income (profit or loss) can be defined as: a) income minus expenses, or equivalently b) revenue plus other income plus gains minus expenses, or equivalently c) revenue plus other income plus gains minus expenses in the ordinary activities of the business minus other expenses, and minus losses. The last definition can be rearranged as follows: net income equals (i) revenue minus expenses in the ordinary activities of the business, plus (ii) other income minus other expenses, plus (iii) gains minus losses. In addition to presenting the net income, income statements also present items, including subtotals, which are significant to users of financial statements. Some of the items are specified by IFRS but other items are not specified. ${ }^{5}$ Certain items, such as revenue, finance costs, and tax expense, are required to be presented separately on the face of the income statement. IFRS additionally require that line items, headings, and subtotals relevant to understanding the entity's financial performance should be presented even if not specified. Expenses may be grouped together either by their nature or function. Grouping together expenses such as depreciation on manufacturing equipment and depreciation on administrative facilities into a single line item called "depreciation" is an example of a grouping by nature of the expense. An example of grouping by function would be grouping together expenses into a category such as cost of goods sold, which may include labour and material costs, depreciation, some salaries (e.g., salespeople's), and other direct sales related expenses. ${ }^{6}$ All three companies in Exhibit 1 through Exhibit 3 present their expenses by function, which is sometimes referred to "cost of sales" method.

One subtotal often shown in an income statement is gross profit or gross margin (that is revenue less cost of sales). When an income statement shows a gross profit subtotal, it is said to use a multi-step format rather than a single-step format. The AB InBev and Molson Coors income statements are examples of the multi-step format, whereas the Groupe Danone income statement is in a single-step format. For manufacturing and merchandising companies, gross profit is a relevant item and is calculated as revenue minus the cost of the goods that were sold. For service companies, gross profit is calculated as revenue minus the cost of services that were provided. In summary, gross profit is the amount of revenue available after subtracting the costs of delivering goods or services. Other expenses related to running the business are subtracted after gross profit.

Another important subtotal which may be shown on the income statement is operating profit (or, synonymously, operating income). Operating profit results from deducting operating expenses such as selling, general, administrative, and research and development expenses from gross profit. Operating profit reflects a company's profits on its business activities before deducting taxes, and for non-financial companies, before deducting interest expense. For financial companies, interest expense would be included in operating expenses and subtracted in arriving at operating profit because it relates to the operating activities for such companies. For some companies composed of a number of separate business segments, operating profit can be useful in evaluating the performance of the individual business segments, because interest and tax expenses may be more relevant at the level of the overall company rather than an individual segment level. The specific calculations of gross profit and operating profit may vary by company, and a reader of financial statements can consult the notes to the statements to identify significant variations across companies.

Operating profit is sometimes referred to as EBIT (earnings before interest and taxes). However, operating profit and EBIT are not necessarily the same. Note that in each of the Exhibit 1 through Exhibit 3, interest and taxes do not represent the only differences between earnings (net income, net earnings) and operating income. For example, AB InBev separately reports its share of associates' and joint ventures' income and Molson Coors separately reports some income from discontinued operations.

Exhibit 4 shows an excerpt from the income statement of CRA International, a company providing management consulting services. Accordingly, CRA deducts cost of services (rather than cost of goods) from revenues to derive gross profit. CRA's fiscal year ends on the Saturday nearest December 31st. Because of this fiscal year timeframe, CRA's fiscal year occasionally comprises 53 weeks rather than 52 weeks. Although the extra week is likely immaterial in computing year-to-year growth rates, it may have a material impact on a quarter containing the extra week. In general, an analyst should be alert to the effect of an extra week when making historical comparisons and forecasting future performance.

\section{Exhibit 4: CRA International Inc. Consolidated Statements of Operations (Excerpt) (in Thousands of Dollars)}
\begin{center}
\begin{tabular}{|c|c|c|c|}
\hline
 & \multicolumn{3}{|c|}{Fiscal Year Ended} \\
\hline
 & Dec. 30,2017 & Dec. 31,2016 & Jan. 02, 2016 \\
\hline
Revenues & $\$ 370,075$ & $\$ 324,779$ & $\$ 303,559$ \\
\hline
Costs of services (exclusive of depreciation and amortization) & 258,829 & 227,380 & 207,650 \\
\hline
Selling, general and administrative expenses & 86,537 & 70,584 & 72,439 \\
\hline
Depreciation and amortization & 8,945 & 7,896 & 6,552 \\
\hline
GNU goodwill impairment & - & - & 4,524 \\
\hline
Income from operations & 15,764 & 18,919 & 12,394 \\
\hline
\end{tabular}
\end{center}

Note: Remaining items omitted

Exhibit 1 through Exhibit 4 illustrate basic points about the income statement, including variations across the statements-some of which depend on the industry and/or country, and some of which reflect differences in accounting policies and practices of a particular company. In addition, some differences within an industry are primarily differences in terminology, whereas others are more fundamental accounting differences. Notes to the financial statements are helpful in identifying such differences.

Having introduced the components and format of an income statement, the next objective is to understand the actual reported numbers in it. To accurately interpret reported numbers, the analyst needs to be familiar with the principles of revenue and expense recognition-that is, how revenue and expenses are measured and attributed to a given accounting reporting period.

\section{REVENUE RECOGNITION}
describe general principles of revenue recognition and accounting standards for revenue recognition calculate revenue given information that might influence the choice of revenue recognition method

Revenue is the top line in an income statement, so we begin the discussion of line items in the income statement with revenue recognition. Accounting standards for revenue recognition (which we discuss later in this section) became effective at the beginning of 2018 and are nearly identical under IFRS and US GAAP. The revenue recognition standards for IFRS and US GAAP (IFRS 15 and ASC Topic 606, respectively) were issued in 2014 and resulted from an effort to achieve convergence, consistency, and transparency in revenue recognition globally. A first task is to explain some relevant accounting terminology. The terms revenue, sales, gains, losses, and net income (profit, net earnings) have been briefly defined. The IASB Conceptual Framework for Financial Reporting (2010), 7 referred to hereafter as the Conceptual Framework, further defines and discusses these income statement items. The ConceptualFramework explains that profit is a frequently used measure of performance and is composed of income and expenses. ${ }^{8}$ It defines income as follows:

Income is increases in economic benefits during the accounting period in the form of inflows or enhancements of assets or decreases of liabilities that result in increases in equity, other than those relating to contributions from equity participants. 9

In IFRS, the term "income" includes revenue and gains. Gains are similar to revenue, but they typically arise from secondary or peripheral activities rather than from a company's primary business activities. For example, for a restaurant, the sale of surplus restaurant equipment for more than its carrying value is referred to as a gain rather than as revenue. Similarly, a loss typically arises from secondary activities. Gains and losses may be considered part of operating activities (e.g., a loss due to a decline in the value of inventory) or may be considered part of non-operating activities (e.g., the sale of non-trading investments).

In the following simple hypothetical scenario, revenue recognition is straightforward: a company sells goods to a buyer for cash and does not allow returns, so the company recognizes revenue when the exchange of goods for cash takes place and measures revenue at the amount of cash received. In practice, however, determining when revenue should be recognized and at what amount is considerably more complex for reasons discussed in the following sections.

\section{General Principles}
An important aspect concerning revenue recognition is that it can occur independently of cash movements. For example, assume a company sells goods to a buyer on credit, so does not actually receive cash until some later time. A fundamental principle of accrual accounting is that revenue is recognized (reported on the income statement) when it is earned, so the company's financial records reflect revenue from the sale when the risk and reward of ownership is transferred; this is often when the company delivers the goods or services. If the delivery was on credit, a related asset, such as trade or accounts receivable, is created. Later, when cash changes hands, the company's financial records simply reflect that cash has been received to settle an account receivable. Similarly, there are situations when a company receives cash in advance and actually delivers the product or service later, perhaps over a period of time. In this case, the company would record a liability for unearned revenue when the cash is initially received, and revenue would be recognized as being earned over time as products and services are delivered. An example would be a subscription payment received for a publication that is to be delivered periodically over time.

7 The IASB is currently in the process of updating its Conceptual Framework for Financial Reporting. 8 Conceptual Framework, paragraph 4.24. The text on the elements of financial statements and their recognition and measurement is the same in the IASB Conceptual Frameworkfor Financial Reporting (2010) and the IASB Framework for the Preparation and Presentation of Financial Statements (1989).

9 Ibid., paragraph 4.25(a).

\section{Accounting Standards for Revenue Recognition}
The converged accounting standards issued by the IASB and FASB in May 2014 introduced some changes to the basic principles of revenue recognition and should enhance comparability. ${ }^{10}$ The content of the two standards is nearly identical, and this discussion pertains to both, unless specified otherwise. Issuance of this converged standard is significant because of the differences between IFRS and US GAAP on revenue recognition prior to the converged standard. The converged standard aims to provide a principles-based approach to revenue recognition that can be applied to many types of revenue-generating activities.

The core principle of the converged standard is that revenue should be recognized to "depict the transfer of promised goods or services to customers in an amount that reflects the consideration to which the entity expects to be entitled in an exchange for those goods or services." To achieve the core principle, the standard describes the application of five steps in recognizing revenue:

\begin{enumerate}
  \item Identify the contract(s) with a customer

  \item Identify the separate or distinct performance obligations in the contract

  \item Determine the transaction price

  \item Allocate the transaction price to the performance obligations in the contract

  \item Recognize revenue when (or as) the entity satisfies a performance obligation

\end{enumerate}

According to the standard, a contract is an agreement and commitment, with commercial substance, between the contacting parties. It establishes each party's obligations and rights, including payment terms. In addition, a contract exists only if collectability is probable. Each standard uses the same wording, but the threshold for probable collectability differs. Under IFRS, probable means more likely than not, and under US GAAP it means likely to occur. As a result, economically similar contracts may be treated differently under IFRS and US GAAP.

The performance obligations within a contract represent promises to transfer distinct good(s) or service(s). A good or service is distinct if the customer can benefit from it on its own or in combination with readily available resources and if the promise to transfer it can be separated from other promises in the contract. Each identified performance obligation is accounted for separately.

The transaction price is what the seller estimates will be received in exchange for transferring the good(s) or service(s) identified in the contract. The transaction price is then allocated to each identified performance obligation. Revenue is recognized when a performance obligation is fulfilled. Steps three and four address amount, and step five addresses timing of recognition. The amount recognized reflects expectations about collectability and (if applicable) an allocation to multiple obligations within the same contract. Revenue is recognized when the obligation-satisfying transfer is made.

Revenue should only be recognized when it is highly probable that it will not be subsequently reversed. This may result in the recording of a minimal amount of revenue upon sale when an estimate of total revenue is not reliable. The balance sheet will be required to reflect the entire refund obligation as a liability and will include an asset for the "right to returned goods" based on the carrying amount of inventory less costs of recovery.

When revenue is recognized, a contract asset is presented on the balance sheet. It is only at the point when all performance obligations have been met except for payment that a receivable appears on the seller's balance sheet. If consideration is received in advance of transferring good(s) or service(s), the seller presents a contract liability.

10 IFRS 15 Revenue from Contracts with Customers and FASB ASC Topic 606 (Revenue from Contracts with Customers). The entity will recognize revenue when it is able to satisfy the performance obligation by transferring control to the customer. Factors to consider when assessing whether the customer has obtained control of an asset at point in time:

\begin{itemize}
  \item Entity has a present right to payment,

  \item Customer has legal title,

  \item Customer has physical possession,

  \item Customer has the significant risks and rewards of ownership, and

  \item Customer has accepted the asset.

\end{itemize}

For a simple contract with only one deliverable at a single point in time, completing the five steps is straight-forward. For more complex contracts-such as when the performance obligations are satisfied over time, when the terms of the multi-period contracts change, when the performance obligation includes various components of goods and services, or when the compensation is "variable"-accounting choices can be less obvious. The steps in the standards are intended to provide guidance that can be generalized to most situations.

In addition, the standard provides many specific examples. These examples are intended to provide guidance as to how to approach more complex contracts. Some of these examples are summarized in Exhibit 5. Note that the end result for many examples may not differ substantially from that under revenue recognition standards that were in effect prior to the adoption of the converged standard; instead it is the conceptual approach and, in some cases, the terminology that will differ.

\section{Exhibit 5: Applying the Converged Revenue Recognition Standard}
The references in this exhibit are to Examples in IFRS 15 Revenue from Contracts with Customers (and ASU 2014-09 (FASB ASC Topic 606)), on which these summaries are based.

\section{Part 1 (ref. Example 10)}
Builder Co. enters into a contract with Customer Co. to construct a commercial building. Builder Co. identifies various goods and services to be provided, such as pre-construction engineering, construction of the building's individual components, plumbing, electrical wiring, and interior finishes. With respect to "Identifying the Performance Obligation," should Builder Co. treat each specific item as a separate performance obligation to which revenue should be allocated? The standard provides two criteria, which must be met, to determine if a good or service is distinct for purposes of identifying performance obligations. First, the customer can benefit from the good or service either on its own or together with other readily available resources. Second, the seller's "promise to transfer the good or service to the customer is separately identifiable from other promises in the contract." In this example, the second criterion is not met because it is the building for which the customer has contracted, not the separate goods and services. The seller will integrate all the goods and services into a combined output and each specific item should not be treated as a distinct good or service but accounted for together as a single performance obligation.

\section{Part 2 (ref. Example 8)}
Builder Co.'s contract with Customer Co. to construct the commercial building specifies consideration of $\$ 1$ million. Builder Co.s expected total costs are $\$ 700,000$. The Builder incurs $\$ 420,000$ in costs in the first year. Assuming that costs incurred provide an appropriate measure of progress toward completing the contract, how much revenue should Builder Co. recognize for the first year? The standard states that for performance obligations satisfied over time (e.g., where there is a long-term contract), revenue is recognized over time by measuring progress toward satisfying the obligation. In this case, the Builder has incurred $60 \%$ of the total expected costs $(\$ 420,000 / \$ 700,000)$ and will thus recognize $\$ 600,000(60 \% \times \$ 1$ million) in revenue for the first year.

This is the same amount of revenue that would be recognized using the "percentage-of-completion" method under previous accounting standards, but that term is not used in the converged standard. Instead, the standard refers to performance obligations satisfied over time and requires that progress toward complete satisfaction of the performance obligation be measured based on input method such as the one illustrated here (recognizing revenue based on the proportion of total costs that have been incurred in the period) or an output method (recognizing revenue based on units produced or milestones achieved).

\section{Part 3 (ref. Example 8)}
Assume that Builder Co.s contract with Customer Co. to construct the commercial building specifies consideration of $\$ 1$ million plus a bonus of $\$ 200,000$ if the building is completed within 2 years. Builder Co. has only limited experience with similar types of contracts and knows that many factors outside its control (e.g., weather, regulatory requirements) could cause delay. Builder Co.'s expected total costs are $\$ 700,000$. The Builder incurs $\$ 420,000$ in costs in the first year. Assuming that costs incurred provide an appropriate measure of progress toward completing the contract, how much revenue should Builder Co. recognize for the first year?

The standard addresses so-called "variable consideration" as part of determining the transaction price. A company is only allowed to recognize variable consideration if it can conclude that it will not have to reverse the cumulative revenue in the future. In this case, Builder Co. does not recognize any of the bonus in year one because it cannot reach the non-reversible conclusion given its limited experience with similar contracts and potential delays from factors outside its control.

\section{Part 4 (ref. Example 8)}
Assume all facts from Part 3. In the beginning of year two, Builder Co. and Customer Co. agree to change the building floor plan and modify the contract. As a result the consideration will increase by $\$ 150,000$, and the allowable time for achieving the bonus is extended by 6 months. Builder expects its costs will increase by $\$ 120,000$. Also, given the additional 6 months to earn the completion bonus, Builder concludes that it now meets the criteria for including the $\$ 200,000$ bonus in revenue. How should Builder account for this change in the contract? Note that previous standards did not provide a general framework for contract modifications. The converged standard provides guidance on whether a change in a contract is a new contract or a modification of an existing contract. To be considered a new contract, the change would need to involve goods and services that are distinct from the goods and services already transferred.

In this case, the change does not meet the criteria of a new contract and is therefore considered a modification of the existing contract, which requires the company to reflect the impact on a cumulative catch-up basis. Therefore, the company must update its transaction price and measure of progress. Builder's total revenue on the transaction (transaction price) is now $\$ 1.35$ million (\$1 million original plus the $\$ 150,000$ new consideration plus $\$ 200,000$ for the completion bonus). Builder Co.'s progress toward completion is now 51.2\% ( $\$ 420,000$ costs incurred divided by total expected costs of $\$ 820,000$ ). Based on the changes in the contract, the amount of additional revenue to be recognized is $\$ 91,200$, calculated as $(51.2 \% \times \$ 1.35$ million) minus the $\$ 600,000$ already recognized. The additional $\$ 91,200$ of revenue would be recognized as a "cumulative catch-up adjustment" on the date of the contract modification.

\section{Part 5 (ref. Example 45)}
Assume a Company operates a website that enables customers to purchase goods from various suppliers. The customers pay the Company in advance, and orders are nonrefundable. The suppliers deliver the goods directly to the customer, and the Company receives a $10 \%$ commission. Should the Company report Total Revenues equal to $100 \%$ of the sales amount (gross) or Total Revenues equal to $10 \%$ of the sales amount (net)? Revenues are reported gross if the Company is acting as a Principal and net if the Company is acting as an Agent.

In this example, the Company is an Agent because it isn't primarily responsible for fulfilling the contract, doesn't take any inventory risk or credit risk, doesn't have discretion in setting the price, and receives compensation in the form of a commission. Because the Company is acting as an Agent, it should report only the amount of commission as its revenue.

Some related costs require specific accounting treatment under the new standards. In particular, incremental costs of obtaining a contract and certain costs incurred to fulfill a contract must be capitalized under the new standards (i.e., reported as an asset on the balance sheet rather than as an expense on the income statement). If a company had previously expensed these incremental costs in the years prior to adopting the converged standard, all else equal, its profitability will initially appear higher under the converged standards.

The disclosure requirements are quite extensive. Companies are required at year end $^{11}$ to disclose information about contracts with customers disaggregated into different categories of contracts. The categories might be based on the type of product, the geographic region, the type of customer or sales channel, the type of contract pricing terms, the contract duration, or the timing of transfers. Companies are also required to disclose balances of any contract-related assets and liabilities and significant changes in those balances, remaining performance obligations and transaction price allocated to those obligations, and any significant judgments and changes in judgments related to revenue recognition. Significant judgments are those used in determining timing and amounts of revenue to be recognized.

The converged standard is expected to affect some industries more than others. For example, industries where bundled sales are common, such as the telecommunications and software industries, are expected to be significantly affected by the converged standard.

\section{EXPENSE RECOGNITION: GENERAL PRINCIPLES}
describe general principles of expense recognition, specific expense recognition applications, and implications of expense recognition choices for financial analysis

11 Interim period disclosures are required under IFRS and US GAAP but differ between them. Expenses are deducted against revenue to arrive at a company's net profit or loss. Under the IASB Conceptual Framework, expenses are "decreases in economic benefits during the accounting period in the form of outflows or depletions of assets or incurrences of liabilities that result in decreases in equity, other than those relating to distributions to equity participants." 12

The IASB Conceptual Framework also states:

The definition of expenses encompasses losses as well as those expenses that arise in the course of the ordinary activities of the enterprise. Expenses that arise in the course of the ordinary activities of the enterprise include, for example, cost of sales, wages and depreciation. They usually take the form of an outflow or depletion of assets such as cash and cash equivalents, inventory, property, plant and equipment.

Losses represent other items that meet the definition of expenses and may, or may not, arise in the course of the ordinary activities of the enterprise. Losses represent decreases in economic benefits and as such they are no different in nature from other expenses. Hence, they are not regarded as a separate element in this Conceptual Framework.

Losses include, for example, those resulting from disasters such as fire and flood, as well as those arising on the disposal of non-current assets. ${ }^{13}$

Similar to the issues with revenue recognition, in a simple hypothetical scenario, expense recognition would not be an issue. For instance, assume a company purchased inventory for cash and sold the entire inventory in the same period. When the company paid for the inventory, absent indications to the contrary, it is clear that the inventory cost has been incurred and when that inventory is sold, it should be recognized as an expense (cost of goods sold) in the financial records. Assume also that the company paid all operating and administrative expenses in cash within each accounting period. In such a simple hypothetical scenario, no issues of expense recognition would arise. In practice, however, as with revenue recognition, determining when expenses should be recognized can be somewhat more complex.

\section{General Principles}
In general, a company recognizes expenses in the period that it consumes (i.e., uses up) the economic benefits associated with the expenditure, or loses some previously recognized economic benefit. ${ }^{14}$

A general principle of expense recognition is the matching principle. Strictly speaking, IFRS do not refer to a "matching principle" but rather to a "matching concept" or to a process resulting in "matching of costs with revenues." 15 The distinction is relevant in certain standard setting deliberations. Under matching, a company recognizes some expenses (e.g., cost of goods sold) when associated revenues are recognized and thus, expenses and revenues are matched. Associated revenues and expenses are those that result directly and jointly from the same transactions or events. Unlike the simple scenario in which a company purchases inventory and sells all of the inventory within the same accounting period, in practice, it is more likely that some of the current period's sales are made from inventory purchased in a previous period or previous periods. It is also likely that some of the inventory purchased in the current period will remain unsold at the end of the current period and so will be sold in a following period. Matching requires that a company recognizes cost of goods sold in the same period as revenues from the sale of the goods.

12 IASB Conceptual Framework, paragraph 4.25(b).

13 Ibid., paragraphs 4.33-4.35.

14 Ibid., paragraph 4.49 .

15 Ibid., paragraph 4.50 . Period costs, expenditures that less directly match revenues, are reflected in the period when a company makes the expenditure or incurs the liability to pay. Administrative expenses are an example of period costs. Other expenditures that also less directly match revenues relate more directly to future expected benefits; in this case, the expenditures are allocated systematically with the passage of time. An example is depreciation expense.

Example 1 and Example 2 demonstrate matching applied to inventory and cost of goods sold.

\section{EXAMPLE 1}
The Matching of Inventory Costs with Revenues

\begin{enumerate}
  \item Kahn Distribution Limited (KDL), a hypothetical company, purchases inventory items for resale. At the beginning of 2018, Kahn had no inventory on hand. During 2018, KDL had the following transactions:
\end{enumerate}

\section{Inventory Purchases}
\begin{center}
\begin{tabular}{lll}
\hline
First quarter & 2,000 & units at $\$ 40$ per unit \\
Second quarter & 1,500 & units at $\$ 41$ per unit \\
Third quarter & 2,200 & units at $\$ 43$ per unit \\
Fourth quarter & 1,900 & units at $\$ 45$ per unit \\
\cline { 2 - 3 }
Total & 7,600 & units at a total cost of $\$ 321,600$ \\
\hline
\end{tabular}
\end{center}

KDL sold 5,600 units of in"ventory during the year at $\$ 50$ per unit, and received cash. KDL determines that there were 2,000 remaining units of inventory and specifically identifies that 1,900 were those purchased in the fourth quarter and 100 were purchased in the third quarter. What are the revenue and expense associated with these transactions during 2018 based on specific identification of inventory items as sold or remaining in inventory? (Assume that the company does not expect any products to be returned.)

\section{Solution:}
The revenue for 2018 would be $\$ 280,000$ (5,600 units $\times \$ 50$ per unit). Initially, the total cost of the goods purchased would be recorded as inventory (an asset) in the amount of $\$ 321,600$. During 2018, the cost of the 5,600 units sold would be expensed (matched against the revenue) while the cost of the 2,000 remaining unsold units would remain in inventory as follows:

\section{Cost of Goods Sold}
From the first quarter

2,000 units at $\$ 40$ per unit $=$ $\$ 80,000$

From the second quarter 1,500 units at $\$ 41$ per unit $=$ $\$ 61,500$

From the third quarter 2,100 units at $\$ 43$ per unit $=$

Total cost of goods sold

$\$ 90,300$

Cost of Goods Remaining in Inventory

From the third quarter $\quad 100$ units at $\$ 43$ per unit = $\$ 4,300$

From the fourth quarter 1,900 units at $\$ 45$ per unit $=$

$\$ 85,500$

Total remaining (or ending) inventory cost To confirm that total costs are accounted for: $\$ 231,800+\$ 89,800=$ $\$ 321,600$. The cost of the goods sold would be expensed against the revenue of $\$ 280,000$ as follows:

\begin{center}
\begin{tabular}{lc}
Revenue & $\$ 280,000$ \\
Cost of goods sold & 231,800 \\
Gross profit & $\$ 48,200$ \\
\hline
\end{tabular}
\end{center}

An alternative way to think about this is that the company created an asset (inventory) of $\$ 321,600$ as it made its purchases. At the end of the period, the value of the company's inv $v^{* *}$ entory on hand is $\$ 89,800$. Therefore, the amount of the Cost of goods sold expense recognized for the period should be the difference: $\$ 231,800$.

The remaining inventory amount of $\$ 89,800$ will be matched against revenue in a future year when the inventory items are sold.

\section{EXAMPLE 2}
\section{Alternative Inventory Costing Methods}
In Example 1, KDL was able to specifically identify which inventory items were sold and which remained in inventory to be carried over to later periods. This is called the specific identification method and inventory and cost of goods sold are based on their physical flow. It is generally not feasible to specifically identify which items were sold and which remain on hand, so accounting standards permit the assignment of inventory costs to costs of goods sold and to ending inventory using cost formulas (IFRS terminology) or cost flow assumptions (US GAAP). The cost formula or cost flow assumption determines which goods are assumed to be sold and which goods are assumed to remain in inventory. Both IFRS and US GAAP permit the use of the first in, first out (FIFO) method, and the weighted average cost method to assign costs.

Under the FIFO method, the oldest goods purchased (or manufactured) are assumed to be sold first and the newest goods purchased (or manufactured) are assumed to remain in inventory. Cost of goods in beginning inventory and costs of the first items purchased (or manufactured) flow into cost of goods sold first, as if the earliest items purchased sold first. Ending inventory would, therefore, include the most recent purchases. It turns out that those items specifically identified as sold in Example 1 were also the first items purchased, so in this example, under FIFO, the cost of goods sold would also be $\$ 231,800$, calculated as above.

The weighted average cost method assigns the average cost of goods available for sale to the units sold and remaining in inventory. The assignment is based on the average cost per unit (total cost of goods available for sale/total units available for sale) and the number of units sold and the number remaining in inventory.

For KDL, the weighted average cost per unit would be

$$
\$ 321,600 / 7,600 \text { units }=\$ 42.3158 \text { per unit }
$$

Cost of goods sold using the weighted average cost method would be

$$
5,600 \text { units at } \$ 42.3158=\$ 236,968
$$

Ending inventory using the weighted average cost method would be 2,000 units at $\$ 42.3158=\$ 84,632$

Another method is permitted under US GAAP but is not permitted under IFRS. This is the last in, first out (LIFO) method. Under the LIFO method, the newest goods purchased (or manufactured) are assumed to be sold first and the oldest goods purchased (or manufactured) are assumed to remain in inventory. Costs of the latest items purchased flow into cost of goods sold first, as if the most recent items purchased were sold first. Although this may seem contrary to common sense, it is logical in certain circumstances. For example, lumber in a lumberyard may be stacked up with the oldest lumber on the bottom. As lumber is sold, it is sold from the top of the stack, so the last lumber purchased and put in inventory is the first lumber out. Theoretically, a company should choose a method linked to the physical inventory flows. ${ }^{16}$ Under the LIFO method, in the KDL example, it would be assumed that the 2,000 units remaining in ending inventory would have come from the first quarter's purchases: ${ }^{17}$

Ending inventory 2,000 units at $\$ 40$ per unit $=\$ 80,000$

The remaining costs would be allocated to cost of goods sold under LIFO:

Total costs of $\$ 321,600$ less $\$ 80,000$ remaining in ending inventory $=\$ 241,600$

Alternatively, the cost of the last 5,600 units purchased is allocated to cost of goods sold under LIFO:

1,900 units at $\$ 45$ per unit $+2,200$ units at $\$ 43$ per unit $+1,500$ units at $\$ 41$ per unit

$=\$ 241,600$

An alternative way to think about expense recognition is that the company created an asset (inventory) of $\$ 321,600$ as it made its purchases. At the end of the period, the value of the company's inventory is $\$ 80,000$. Therefore, the amount of the Cost of goods sold expense recognized for the period should be the difference: $\$ 241,600$.

Exhibit 6 summarizes and compares inventory costing methods.

Exhibit 6: Summary Table on Inventory Costing Methods

\begin{center}
\begin{tabular}{|c|c|c|c|}
\hline
Method & Description & $\begin{array}{l}\text { Cost of Goods Sold When } \\ \text { Prices Are Rising, Relative to } \\ \text { Other Two Methods }\end{array}$ & $\begin{array}{l}\text { Ending Inventory When } \\ \text { Prices Are Rising, Relative to } \\ \text { Other Two Methods }\end{array}$ \\
\hline
FIFO (first in, first out) & $\begin{array}{l}\text { Costs of the earliest items pur- } \\ \text { chased flow to cost of goods sold } \\ \text { first }\end{array}$ & Lowest & Highest \\
\hline
LIFO (last in, first out) & $\begin{array}{l}\text { Costs of the most recent items } \\ \text { purchased flow to cost of goods } \\ \text { sold first }\end{array}$ & Highest" & Lowest" \\
\hline
\end{tabular}
\end{center}

16 Practically, the reason some companies choose to use LIFO in the United States is to reduce taxes. When prices and inventory quantities are rising, LIFO will normally result in higher cost of goods sold and lower income and hence lower taxes. US tax regulations require that if LIFO is used on a company's tax return, it must also be used on the company's GAAP financial statements.

17 If data on the precise timing of quarterly sales were available, the answer would differ because the cost of goods sold would be determined during the quarter rather than at the end of the quarter.

\begin{center}
\begin{tabular}{|c|c|c|c|}
\hline
Method & Description & $\begin{array}{l}\text { Cost of Goods Sold When } \\ \text { Prices Are Rising, Relative to } \\ \text { Other Two Methods }\end{array}$ & $\begin{array}{l}\text { Ending Inventory When } \\ \text { Prices Are Rising, Relative to } \\ \text { Other Two Methods }\end{array}$ \\
\hline
Weighted average cost & $\begin{array}{l}\text { Averages total costs over total } \\ \text { units available }\end{array}$ & Middle & Middle \\
\hline
\end{tabular}
\end{center}

"Assumes no LIFO layer liquidation. LIFO layer liquidation occurs when the volume of sales exceeds the volume of purchases in the period so that some sales are assumed to be made from existing, relatively low-priced inventory rather than from more recent purchases.

\section{ISSUES IN EXPENSE RECOGNITION: DOUBTFUL ACCOUNTS, WARRANTIES}
describe general principles of expense recognition, specific expense recognition applications, and implications of expense recognition choices for financial analysis

The following sections cover applications of the principles of expense recognition to certain common situations.

\section{Doubtful Accounts}
When a company sells its products or services on credit, it is likely that some customers will ultimately default on their obligations (i.e., fail to pay). At the time of the sale, it is not known which customer will default. (If it were known that a particular customer would ultimately default, presumably a company would not sell on credit to that customer.) One possible approach to recognizing credit losses on customer receivables would be for the company to wait until such time as a customer defaulted and only then recognize the loss (direct write-off method). Such an approach would usually not be consistent with generally accepted accounting principles.

Under the matching principle, at the time revenue is recognized on a sale, a company is required to record an estimate of how much of the revenue will ultimately be uncollectible. Companies make such estimates based on previous experience with uncollectible accounts. Such estimates may be expressed as a proportion of the overall amount of sales, the overall amount of receivables, or the amount of receivables overdue by a specific amount of time. The company records its estimate of uncollectible amounts as an expense on the income statement, not as a direct reduction of revenues.

\section{Warranties}
At times, companies offer warranties on the products they sell. If the product proves deficient in some respect that is covered under the terms of the warranty, the company will incur an expense to repair or replace the product. At the time of sale, the company does not know the amount of future expenses it will incur in connection with its warranties. One possible approach would be for a company to wait until actual expenses are incurred under the warranty and to reflect the expense at that time. However, this would not result in a matching of the expense with the associated revenue. Under the matching principle, a company is required to estimate the amount of future expenses resulting from its warranties, to recognize an estimated warranty expense in the period of the sale, and to update the expense as indicated by experience over the life of the warranty.

\section{ISSUES IN EXPENSE RECOGNITION: DEPRECIATION AND AMORTIZATION}
describe general principles of expense recognition, specific expense recognition applications, and implications of expense recognition choices for financial analysis

Companies commonly incur costs to obtain long-lived assets. Long-lived assets are assets expected to provide economic benefits over a future period of time greater than one year. Examples are land (property), plant, equipment, and intangible assets (assets lacking physical substance) such as trademarks. The costs of most long-lived assets are allocated over the period of time during which they provide economic benefits. The two main types of long-lived assets whose costs are not allocated over time are land and those intangible assets with indefinite useful lives.

Depreciation is the process of systematically allocating costs of long-lived assets over the period during which the assets are expected to provide economic benefits. "Depreciation" is the term commonly applied to this process for physical long-lived assets such as plant and equipment (land is not depreciated), and amortisation is the term commonly applied to this process for intangible long-lived assets with a finite useful life. ${ }^{18}$ Examples of intangible long-lived assets with a finite useful life include an acquired mailing list, an acquired patent with a set expiration date, and an acquired copyright with a set legal life. The term "amortisation" is also commonly applied to the systematic allocation of a premium or discount relative to the face value of a fixed-income security over the life of the security.

IFRS allow two alternative models for valuing property, plant, and equipment: the cost model and the revaluation model. ${ }^{19}$ Under the cost model, the depreciable amount of that asset (cost less residual value) is allocated on a systematic basis over the remaining useful life of the asset. Under the cost model, the asset is reported at its cost less any accumulated depreciation. Under the revaluation model, the asset is reported at its fair value. The revaluation model is not permitted under US GAAP. Although the revaluation model is permitted under IFRS, as noted earlier, it is not as widely used and thus we focus on the cost model here. There are two other differences between IFRS and US GAAP to note: IFRS require each component of an asset to be depreciated separately and US GAAP do not require component depreciation; and IFRS require an annual review of residual value and useful life, and US GAAP do not explicitly require such a review.

18 Intangible assets with indefinite life are not amortised. Instead, they are reviewed each period as to the reasonableness of continuing to assume an indefinite useful life and are tested at least annually for impairment (i.e., if the recoverable or fair value of an intangible asset is materially lower than its value in the company's books, the value of the asset is considered to be impaired and its value must be decreased) IAS 38, Intangible Assets and FASB ASC Topic 350 [Intangibles-Goodwill and Other].

19 IAS No. 16, Property, Plant, and Equipment. The method used to compute depreciation should reflect the pattern over which the economic benefits of the asset are expected to be consumed. IFRS do not prescribe a particular method for computing depreciation but note that several methods are commonly used, such as the straight-line method, diminishing balance method (accelerated depreciation), and the units of production method (depreciation varies depending upon production or usage).

The straight-line method allocates evenly the cost of long-lived assets less estimated residual value over the estimated useful life of an asset. (The term "straight line" derives from the fact that the annual depreciation expense, if represented as a line graph over time, would be a straight line. In addition, a plot of the cost of the asset minus the cumulative amount of annual depreciation expense, if represented as a line graph over time, would be a straight line with a negative downward slope.) Calculating depreciation and amortisation requires two significant estimates: the estimated useful life of an asset and the estimated residual value (also known as "salvage value") of an asset. Under IFRS, the residual value is the amount that the company expects to receive upon sale of the asset at the end of its useful life. Example 3 assumes that an item of equipment is depreciated using the straight-line method and illustrates how the annual depreciation expense varies under different estimates of the useful life and estimated residual value of an asset. As shown, annual depreciation expense is sensitive to both the estimated useful life and to the estimated residual value.

\section{EXAMPLE 3}
\section{Sensitivity of Annual Depreciation Expense to Varying Estimates of Useful Life and Residual Value}
Using the straight-line method of depreciation, annual depreciation expense is calculated as:

Cost - Residual value

Estimated useful life

Assume the cost of an asset is $\$ 10,000$. If, for example, the residual value of the asset is estimated to be $\$ 0$ and its useful life is estimated to be 5 years, the annual depreciation expense under the straight-line method would be $(\$ 10,000$ $-\$ 0$ )/5 years $=\$ 2,000$. In contrast, holding the estimated useful life of the asset constant at 5 years but increasing the estimated residual value of the asset to $\$ 4,000$ would result in annual depreciation expense of only $\$ 1,200$ [calculated as $(\$ 10,000-\$ 4,000) / 5$ years]. Alternatively, holding the estimated residual value at $\$ 0$ but increasing the estimated useful life of the asset to 10 years would result in annual depreciation expense of only $\$ 1,000$ [calculated as $(\$ 10,000-\$ 0) / 10$ years]. Exhibit 7 shows annual depreciation expense for various combinations of estimated useful life and residual value.

\section{Exhibit 7: Annual Depreciation Expense (in Dollars)}
Estimated

Useful Life

(Years) Estimated Residual Value

\begin{center}
\begin{tabular}{|c|c|c|c|c|c|c|}
\hline
 & 0 & 1,000 & 2,000 & 3,000 & 4,000 & 5,000 \\
\hline
2 & 5,000 & 4,500 & 4,000 & 3,500 & 3,000 & 2,500 \\
\hline
4 & 2,500 & 2,250 & 2,000 & 1,750 & 1,500 & 1,250 \\
\hline
\end{tabular}
\end{center}

\begin{center}
\includegraphics[max width=\textwidth]{2023_05_04_b5cfa4f1bc883752f121g-037}
\end{center}

Generally, alternatives to the straight-line method of depreciation are called accelerated methods of depreciation because they accelerate (i.e., speed up) the timing of depreciation. Accelerated depreciation methods allocate a greater proportion of the cost to the early years of an asset's useful life. These methods are appropriate if the plant or equipment is expected to be used up faster in the early years (e.g., an automobile). A commonly used accelerated method is the diminishing balance method, (also known as the declining balance method). The diminishing balance method is demonstrated in Example 4.

\section{EXAMPLE 4}
\section{An Illustration of Diminishing Balance Depreciation}
Assume the cost of computer equipment was $\$ 11,000$, the estimated residual value is $\$ 1,000$, and the estimated useful life is five years. Under the diminishing or declining balance method, the first step is to determine the straight-line rate, the rate at which the asset would be depreciated under the straight-line method. This rate is measured as 100 percent divided by the useful life or 20 percent for a five-year useful life. Under the straight-line method, $1 / 5$ or 20 percent of the depreciable cost of the asset (here, $\$ 11,000-\$ 1,000=\$ 10,000$ ) would be expensed each year for five years: The depreciation expense would be $\$ 2,000$ per year.

The next step is to determine an acceleration factor that approximates the pattern of the asset's wear. Common acceleration factors are 150 percent and 200 percent. The latter is known as double declining balance depreciation because it depreciates the asset at double the straight-line rate. Using the 200 percent acceleration factor, the diminishing balance rate would be 40 percent (20 percent $\times 2.0$ ). This rate is then applied to the remaining undepreciated balance of the asset each period (known as the net book value).

At the beginning of the first year, the net book value is $\$ 11,000$. Depreciation expense for the first full year of use of the asset would be 40 percent of $\$ 11,000$, or $\$ 4,400$. Under this method, the residual value, if any, is generally not used in the computation of the depreciation each period (the 40 percent is applied to $\$ 11,000$ rather than to $\$ 11,000$ minus residual value). However, the company will stop taking depreciation when the salvage value is reached.

At the beginning of Year 2, the net book value is measured as

\begin{center}
\begin{tabular}{lr}
Asset cost & $\$ 11,000$ \\
Less: Accumulated depreciation & $(4,400)$ \\
\hline
Net book value & $\$ 6,600$ \\
\hline
\end{tabular}
\end{center}

For the second full year, depreciation expense would be $\$ 6,600 \times 40$ percent, or $\$ 2,640$. At the end of the second year (i.e., beginning of the third year), a total of $\$ 7,040(\$ 4,400+\$ 2,640)$ of depreciation would have been recorded. So, the remaining net book value at the beginning of the third year would be

Asset cost

$\$ 11,000$ Less: Accumulated depreciation

Net book value

$\begin{array}{r}(7,040) \\ \hline \$ 3,960 \\ \hline\end{array}$

For the third full year, depreciation would be $\$ 3,960 \times 40$ percent, or $\$ 1,584$. At the end of the third year, a total of $\$ 8,624(\$ 4,400+\$ 2,640+\$ 1,584)$ of depreciation would have been recorded. So, the remaining net book value at the beginning of the fourth year would be

\begin{center}
\begin{tabular}{lr}
Asset cost & $\$ 11,000$ \\
Less: Accumulated depreciation & $(8,624)$ \\
\hline
Net book value & $\$ 2,376$ \\
\hline
\end{tabular}
\end{center}

For the fourth full year, depreciation would be $\$ 2,376 \times 40$ percent, or $\$ 950$. At the end of the fourth year, a total of $\$ 9,574(\$ 4,400+\$ 2,640+\$ 1,584+\$ 950)$ of depreciation would have been recorded. So, the remaining net book value at the beginning of the fifth year would be

\begin{center}
\begin{tabular}{lr}
Asset cost & $\$ 11,000$ \\
Less: Accumulated depreciation & $(9,574)$ \\
\hline
Net book value & $\$ 1,426$ \\
\end{tabular}
\end{center}

For the fifth year, if deprecation were determined as in previous years, it would amount to $\$ 570(\$ 1,426 \times 40$ percent). However, this would result in a remaining net book value of the asset below its estimated residual value of $\$ 1,000$. So, instead, only $\$ 426$ would be depreciated, leaving a $\$ 1,000$ net book value at the end of the fifth year.

\begin{center}
\begin{tabular}{lc}
Asset cost & $\begin{array}{r}\$ 11,000 \\ (10,000)\end{array}$ \\
Less: Accumulated depreciation & $\$ 1,000$ \\
\cline { 2 - 2 }
Net book value &  \\
\cline { 2 - 2 }
\end{tabular}
\end{center}

Companies often use a zero or small residual value, which creates problems for diminishing balance depreciation because the asset never fully depreciates. In order to fully depreciate the asset over the initially estimated useful life when a zero or small residual value is assumed, companies often adopt a depreciation policy that combines the diminishing balance and straight-line methods. An example would be a deprecation policy of using double-declining balance depreciation and switching to the straight-line method halfway through the useful life.

Under accelerated depreciation methods, there is a higher depreciation expense in early years relative to the straight-line method. This results in higher expenses and lower net income in the early depreciation years. In later years, there is a reversal with accelerated depreciation expense lower than straight-line depreciation. Accelerated depreciation is sometimes referred to as a conservative accounting choice because it results in lower net income in the early years of asset use.

For those intangible assets that must be amortised (those with an identifiable useful life), the process is the same as for depreciation; only the name of the expense is different. IFRS state that if a pattern cannot be determined over the useful life, then the straight-line method should be used. ${ }^{20}$ In most cases under IFRS and US GAAP, amortisable intangible assets are amortised using the straight-line method with no residual value. Goodwill ${ }^{21}$ and intangible assets with indefinite life are not amortised. Instead, they are tested at least annually for impairment (i.e., if the current value of an intangible asset or goodwill is materially lower than its value in the company's books, the value of the asset is considered to be impaired and its value in the company's books must be decreased).

In summary, to calculate depreciation and amortisation, a company must choose a method, estimate the asset's useful life, and estimate residual value. Clearly, different choices have a differing effect on depreciation or amortisation expense and, therefore, on reported net income.

\section{IMPLICATIONS FOR FINANCIAL ANALYSTS: EXPENSE RECOGNITION}
describe general principles of expense recognition, specific expense recognition applications, and implications of expense recognition choices for financial analysis

A company's estimates for doubtful accounts and/or for warranty expenses can affect its reported net income. Similarly, a company's choice of depreciation or amortisation method, estimates of assets' useful lives, and estimates of assets' residual values can affect reported net income. These are only a few of the choices and estimates that affect a company's reported net income.

As with revenue recognition policies, a company's choice of expense recognition can be characterized by its relative conservatism. A policy that results in recognition of expenses later rather than sooner is considered less conservative. In addition, many items of expense require the company to make estimates that can significantly affect net income. Analysis of a company's financial statements, and particularly comparison of one company's financial statements with those of another, requires an understanding of differences in these estimates and their potential impact.

If, for example, a company shows a significant year-to-year change in its estimates of uncollectible accounts as a percentage of sales, warranty expenses as a percentage of sales, or estimated useful lives of assets, the analyst should seek to understand the underlying reasons. Do the changes reflect a change in business operations (e.g., lower estimated warranty expenses reflecting recent experience of fewer warranty claims because of improved product quality)? Or are the changes seemingly unrelated to changes in business operations and thus possibly a signal that a company is manipulating estimates in order to achieve a particular effect on its reported net income?

As another example, if two companies in the same industry have dramatically different estimates for uncollectible accounts as a percentage of their sales, warranty expenses as a percentage of sales, or estimated useful lives as a percentage of assets, it is important to understand the underlying reasons. Are the differences consistent with differences in the two companies' business operations (e.g., lower uncollectible accounts for one company reflecting a different, more creditworthy customer base or possibly stricter credit policies)? Another difference consistent with differences in business operations would be a difference in estimated useful lives of assets if one

21 Goodwill is recorded in acquisitions and is the amount by which the price to purchase an entity exceeds the amount of net identifiable assets acquired (the total amount of identifiable assets acquired less liabilities assumed). of the companies employs newer equipment. Or, alternatively, are the differences seemingly inconsistent with differences in the two companies' business operations, possibly signaling that a company is manipulating estimates?

Information about a company's accounting policies and significant estimates are described in the notes to the financial statements and in the management discussion and analysis section of a company's annual report.

When possible, the monetary effect of differences in expense recognition policies and estimates can facilitate more meaningful comparisons with a single company's historical performance or across a number of companies. An analyst can use the monetary effect to adjust the reported expenses so that they are on a comparable basis.

Even when the monetary effects of differences in policies and estimates cannot be calculated, it is generally possible to characterize the relative conservatism of the policies and estimates and, therefore, to qualitatively assess how such differences might affect reported expenses and thus financial ratios.

\section{NON-RECURRING ITEMS AND NON-OPERATING ITEMS: DISCONTINUED OPERATIONS AND UNUSUAL OR INFREQUENT ITEMS}
describe the financial reporting treatment and analysis of non-recurring items (including discontinued operations, unusual or infrequent items) and changes in accounting policies

From a company's income statements, we can see its earnings from last year and in the previous year. Looking forward, the question is: What will the company earn next year and in the years after?

To assess a company's future earnings, it is helpful to separate those prior years' items of income and expense that are likely to continue in the future from those items that are less likely to continue. ${ }^{22}$ Some items from prior years are clearly not expected to continue in the future periods and are separately disclosed on a company's income statement. This is consistent with "An entity shall present additional line items, headings, and subtotals.. when such presentation is relevant to an understanding of the entity's financial performance."23 IFRS describe considerations that enter into the decision to present information other than that explicitly specified by a standard. Both IFRS and US GAAP specify that the results of discontinued operations should be reported separately from continuing operations. Other items that may be reported separately on a company's income statement, such as unusual items, items that occur infrequently, effects due to accounting changes, and non-operating income, require the analyst to make some judgments.

\section{Discontinued Operations}
When a company disposes of or establishes a plan to dispose of one of its component operations and will have no further involvement in the operation, the income statement reports separately the effect of this disposal as a "discontinued" operation under

22 In business writing, items expected to continue in the future are often described as "persistent" or "permanent," whereas those not expected to continue are described as "transitory."

23 IAS No. 1, Presentation of Financial Statements, paragraph 85. both IFRS and US GAAP. Financial standards provide various criteria for reporting the effect separately, which are generally that the discontinued component must be separable both physically and operationally. ${ }^{24}$

In Exhibit 1, AB InBev reported profit from discontinued operations of $\$ 28$ million in 2017 and $\$ 48$ million in 2016. In Exhibit 2, Molson Coors reported income from discontinued operations of $\$ 1.5$ million and $\$ 3.9$ million in 2017 and 2015, respectively, and a loss from discontinued operations of $\$ 2.8$ million in 2016.

Because the discontinued operation will no longer provide earnings (or cash flow) to the company, an analyst may eliminate discontinued operations in formulating expectations about a company's future financial performance.

\section{Unusual or Infrequent Items}
IFRS require that items of income or expense that are material and/or relevant to the understanding of the entity's financial performance should be disclosed separately. Unusual or infrequent items are likely to meet these criteria. Under US GAAP, material items that are unusual or infrequent, and that are both as of reporting periods beginning after December 15, 2015, are shown as part of a company's continuing operations but are presented separately. For example, restructuring charges, such as costs to close plants and employee termination costs, are considered part of a company's ordinary activities. As another example, gains and losses arising when a company sells an asset or part of a business, for more or less than its carrying value, are also disclosed separately on the income statement. These sales are considered ordinary business activities.

Highlighting the unusual or infrequent nature of these items assists an analyst in judging the likelihood that such items will reoccur. This meets the IFRS criteria of disclosing items that are relevant to the understanding of an entity's financial performance. In Exhibit 2, Molson Coors' income statement showed a separate line item for "Special Items, net." The company's footnotes provide details on the amount and explain that this line includes revenues or expenses that either they "do not believe to be indicative of [their] core operations, or they believe are significant to [their] current operating results warranting separate classification". In Exhibit 3, the income statement of Danone shows an amount for "Recurring operating income" followed by a separate line item for "other operating income (expense)", which is not included as a component of recurring income. Exhibit 8 presents an excerpt from Danone's additional disclosure about this non-recurring amount.

\section{Exhibit 8: Highlighting Infrequent Nature of Items-Excerpt from Groupe}
 Danone footnotes to its 2017 financial statementsNOTE 6. Events and Transactions Outside the Group's Ordinary Activities [Excerpt]

"Other operating income (expense) is defined under Recommendation 2013-03 of the French CNC relating to the format of consolidated financial statements prepared under international accounting standards, and comprises significant items that, because of their exceptional nature, cannot be viewed as inherent to Danone's current activities. These mainly include capital gains and losses on disposals of fully consolidated companies, impairment charges on goodwill, significant costs related to strategic restructuring and major external growth transactions, and incurred or estimated costs related to major crises and major litigation. Furthermore, in connection with Revised IFRS 3 and Revised IAS 27, Danone also classifies in Other operating income (expense) (i) acquisition costs related to business combinations, (ii) revaluation profit or loss accounted for following a loss of control, and (iii) changes in earn-outs related to business combinations and subsequent to the acquisition date.

"In 2017, the net Other operating income of $€ 192$ million consisted mainly of the following items:

\begin{center}
\begin{tabular}{lc}
\hline
(in $\epsilon$ millions) & $\begin{array}{c}\text { Related income } \\ \text { (expense) }\end{array}$ \\
\hline
Capital gain on disposal of Stonyfield & 628 \\
Compensation received following the decision of the Singapore arbi- & 105 \\
tration court in the Fonterra case & $(148)$ \\
Territorial risks, mainly in certain countries in the ALMA region & $(118)$ \\
Costs associated with the integration of WhiteWave & $(115)$ \\
Impairment of several intangible assets in Waters and Specialized &  \\
Nutrition Reporting entities & Remainder of table omitted \\
\end{tabular}
\end{center}

In Exhibit 8, Danone provides details on items considered to be "exceptional" items and not "inherent" to the company's current activities. The exceptional items include gains on asset disposals, receipts from a legal case, costs of integrating an acquisition, and impairment of intangible assets, among others. Generally, in forecasting future operations, an analyst would assess whether the items reported are likely to reoccur and also possible implications for future earnings. It is generally not advisable simply to ignore all unusual items.

\section{NON-RECURRING ITEMS: CHANGES IN ACCOUNTING POLICY}
describe the financial reporting treatment and analysis of non-recurring items (including discontinued operations, unusual or infrequent items) and changes in accounting policies

At times, standard setters issue new standards that require companies to change accounting policies. Depending on the standard, companies may be permitted to adopt the standards prospectively (in the future) or retrospectively (restate financial statements as though the standard existed in the past). In other cases, changes in accounting policies (e.g., from one acceptable inventory costing method to another) are made for other reasons, such as providing a better reflection of the company's performance. Changes in accounting policies are reported through retrospective application ${ }^{25}$ unless it is impractical to do so.

Retrospective application means that the financial statements for all fiscal years shown in a company's financial report are presented as if the newly adopted accounting principle had been used throughout the entire period. Notes to the financial statements

25 IAS No. 8, Accounting Policies, Changes in Accounting Estimates and Errors, and FASB ASC Topic 250 [Accounting Changes and Error Corrections]. describe the change and explain the justification for the change. Because changes in accounting principles are retrospectively applied, the financial statements that appear within a financial report are comparable.

Example 5 presents an excerpt from Microsoft Corporation's Form 10- $\mathrm{K}$ for the fiscal year ended 30 June 2018 describing a change in accounting principle resulting from the new revenue recognition standard. Microsoft elected to adopt the new standard 1 July 2017, earlier than the required adoption date. Microsoft also elected to use the "full retrospective method," which requires companies to restate prior periods' results. On its income statement, both 2016 and 2017 are presented as if the new standard had been used throughout both years. In the footnotes to its financial statements, Microsoft discloses the impact of the new standard.

\section{EXAMPLE 5}
\section{Microsoft Corporation Excerpt from Footnotes to the Financial Statements}
The most significant impact of the [new revenue recognition] standard relates to our accounting for software license revenue. Specifically, for Windows 10, we recognize revenue predominantly at the time of billing and delivery rather than ratably over the life of the related device. For certain multi-year commercial software subscriptions that include both distinct software licenses and SA, we recognize license revenue at the time of contract execution rather than over the subscription period. Due to the complexity of certain of our commercial license subscription contracts, the actual revenue recognition treatment required under the standard depends on contract-specific terms and in some instances may vary from recognition at the time of billing. Revenue recognition related to our hardware, cloud offerings (such as Office 365), LinkedIn, and professional services remains substantially unchanged. Refer to Impacts to Previously Reported Results below for the impact of adoption of the standard in our consolidated financial statements.

\begin{center}
\begin{tabular}{|c|c|c|c|}
\hline
$\begin{array}{l}\text { (In } \$ \text { millions, except per share } \\ \text { amounts) }\end{array}$ & $\begin{array}{c}\text { As } \\ \text { Previously } \\ \text { Reported }\end{array}$ & $\begin{array}{c}\text { New } \\ \text { Revenue } \\ \text { Standard } \\ \text { Adjustment }\end{array}$ & $\begin{array}{c}\text { As } \\ \text { Restated }\end{array}$ \\
\hline
\multicolumn{4}{|l|}{Income Statements} \\
\hline
\multicolumn{4}{|l|}{Year Ended June 30, 2017} \\
\hline
Revenue & 89,950 & 6,621 & 96,571 \\
\hline
Provision for income taxes & 1,945 & 2,467 & 4,412 \\
\hline
Net income & 21,204 & 4,285 & 25,489 \\
\hline
Diluted earnings per share & 2.71 & 0.54 & 3.25 \\
\hline
\multicolumn{4}{|l|}{Year Ended June 30, 2016} \\
\hline
Revenue & 85,320 & 5,834 & 91,154 \\
\hline
Provision for income taxes & 2,953 & 2,147 & 5,100 \\
\hline
Net income & 16,798 & 3,741 & 20,539 \\
\hline
Diluted earnings per share & 2.1 & 0.46 & 2.56 \\
\hline
\end{tabular}
\end{center}

\begin{enumerate}
  \item Question: Based on the above information, describe whether Microsoft's results appear better or worse under the new revenue recognition standard.
\end{enumerate}

\section{Solution:}
Microsoft's results appear better under the new revenue recognition standard. Revenues and income are higher under the new standard. The net profit margin is higher under the new standard. For 2017, the net profit margin is $26.4 \%$ (= 25,489/96,571) under the new standard versus $23.6 \%$ (= $21,204 / 89,950$ ) under the old standard. Reported revenue grew faster under the new standard. Revenue growth under the new standard was 5.9\% [= $(96,571 / 91,154)-1]$ compared to $5.4 \%$ [= $(89,950 / 85,320)-1)]$ under the old standard.

Microsoft's presentation of the effects of the new revenue recognition enables an analyst to identify the impact of the change in accounting standards.

Note that the new revenue recognition standard also offered companies the option of using a "modified retrospective" method of adoption. Under the modified retrospective approach, companies were not required to revise previously reported financial statements. Instead, they adjusted opening balances of retained earnings (and other applicable accounts) for the cumulative impact of the new standard.

In contrast to changes in accounting policies (such as whether to expense the cost of employee stock options), companies sometimes make changes in accounting estimates (such as the useful life of a depreciable asset). Changes in accounting estimates are handled prospectively, with the change affecting the financial statements for the period of change and future periods. No adjustments are made to prior statements, and the adjustment is not shown on the face of the income statement. Significant changes should be disclosed in the notes. Exhibit 9 provides an excerpt from the annual Form 10-K of Catalent Inc., a US-based biotechnology company, that illustrates a change in accounting estimate.

\section{Exhibit 9: Change in Accounting Estimate}
Catalent Inc. discloses a change in the method it uses to calculate both annual expenses related to its defined benefit pension plans. Rather than use a single, weighted-average discount rate in its calculations, the company will use the spot rates applicable to each projected cash flow.

\section{Post-Retirement and Pension Plans}
...The measurement of the related benefit obligations and the net periodic benefit costs recorded each year are based upon actuarial computations, which require management's judgment as to certain assumptions. These assumptions include the discount rates used in computing the present value of the benefit obligations and the net periodic benefit costs...

Effective June 30, 2016, the approach used to estimate the service and interest components of net periodic benefit cost for benefit plans was changed to provide a more precise measurement of service and interest costs. Historically, the Company estimated these service and interest components utilizing a single weighted-average discount rate derived from the yield curve used to measure the benefit obligation at the beginning of the period. Going forward, the Company has elected to utilize an approach that discounts the individual expected cash flows using the applicable spot rates derived from the yield curve over the projected cash flow period. The Company has accounted for this change as a change in accounting estimate that is inseparable from a change in accounting principle and accordingly has accounted for it prospectively.

Another possible adjustment is a correction of an error for a prior period (e.g., in financial statements issued for an earlier year). This cannot be handled by simply adjusting the current period income statement. Correction of an error for a prior period is handled by restating the financial statements (including the balance sheet, statement of owners' equity, and cash flow statement) for the prior periods presented in the current financial statements. ${ }^{26}$ Note disclosures are required regarding the error. These disclosures should be examined carefully because they may reveal weaknesses in the company's accounting systems and financial controls.

\section{NON-OPERATING ITEMS}
contrast operating and non-operating components of the income statement

Non-operating items are typically reported separately from operating income because they are material and/or relevant to the understanding of the entity's financial performance. Under IFRS, there is no definition of operating activities, and companies that choose to report operating income or the results of operating activities should ensure that these represent activities that are normally regarded as operating. Under US GAAP, operating activities generally involve producing and delivering goods and providing services and include all transactions and other events that are not defined as investing or financing activities. ${ }^{27}$ For example, if a non-financial service company invests in equity or debt securities issued by another company, any interest, dividends, or profits from sales of these securities will be shown as non-operating income. In general, for non-financial services companies, ${ }^{28}$ non-operating income that is disclosed separately on the income statement (or in the notes) includes amounts earned through investing activities.

Among non-operating items on the income statement (or accompanying notes), non-financial service companies also disclose the interest expense on their debt securities, including amortisation of any discount or premium. The amount of interest expense is related to the amount of a company's borrowings and is generally described in the notes to the financial statements. For financial service companies, interest income and expense are likely components of operating activities. (Note that the characterization of interest and dividends as non-operating items on the income statement is not necessarily consistent with the classification on the statement of cash flows. Specifically, under IFRS, interest and dividends received can be shown either as operating or as investing on the statement of cash flows, while under US GAAP interest and dividends received are shown as operating cash flows. Under IFRS, interest and dividends paid can be shown either as operating or as financing on the statement of cash flows, while under US GAAP, interest paid is shown as operating and dividends paid are shown as financing.)

26 Ibid.

27 FASB ASC Master Glossary.

28 Examples of financial services companies are insurance companies, banks, brokers, dealers, and investment companies. In practice, companies often disclose the interest expense and income separately, along with a net amount. For example, in Exhibit 1, ABN InBev's 2017 income statement shows finance cost of $\$ 6,885$ million, finance income of $\$ 378$ million, and net finance cost of $\$ 6,507$ million. Similarly, in Exhibit 3, Danone's 2017 income statement shows interest income of $€ 130$, interest expense of $€ 276$, and cost of net debt of $€ 146$.

For purposes of assessing a company's future performance, the amount of financing expense will depend on the company's financing policy (target capital structure) and borrowing costs. The amount of investing income will depend on the purpose and success of investing activities. For a non-financial company, a significant amount of financial income would typically warrant further exploration. What are the reasons underlying the company's investments in the securities of other companies? Is the company simply investing excess cash in short-term securities to generate income higher than cash deposits, or is the company purchasing securities issued by other companies for strategic reasons, such as access to raw material supply or research?

\section{EARNINGS PER SHARE AND CAPITAL STRUCTURE AND BASIC EPS}
describe how earnings per share is calculated and calculate and interpret a company's earnings per share (both basic and diluted earnings per share) for both simple and complex capital structures

One metric of particular importance to an equity investor is earnings per share (EPS). EPS is an input into ratios such as the price/earnings ratio. Additionally, each shareholder in a company owns a different number of shares. IFRS require the presentation of EPS on the face of the income statement for net profit or loss (net income) and profit or loss (income) from continuing operations. ${ }^{29}$ Similar presentation is required under US GAAP. ${ }^{30}$ This section outlines the calculations for EPS and explains how the calculation differs for a simple versus complex capital structure.

\section{Simple versus Complex Capital Structure}
A company's capital is composed of its equity and debt. Some types of equity have preference over others, and some debt (and other instruments) may be converted into equity. Under IFRS, the type of equity for which EPS is presented is referred to as ordinary. Ordinary shares are those equity shares that are subordinate to all other types of equity. The ordinary shareholders are basically the owners of the company-the equity holders who are paid last in a liquidation of the company and who benefit the most when the company does well. Under US GAAP, this ordinary equity is referred to as common stock or common shares, reflecting US language usage. The terms "ordinary shares," "common stock," and "common shares" are used interchangeably in the following discussion.

When a company has issued any financial instruments that are potentially convertible into common stock, it is said to have a complex capital structure. Examples of financial instruments that are potentially convertible into common stock include

29 IAS No. 33, Earnings Per Share.

30 FASB ASC Topic 260 [Earnings Per Share]. convertible bonds, convertible preferred stock, employee stock options, and warrants. ${ }^{31}$ If a company's capital structure does not include such potentially convertible financial instruments, it is said to have a simple capital structure.

The distinction between simple versus complex capital structure is relevant to the calculation of EPS because financial instruments that are potentially convertible into common stock could, as a result of conversion or exercise, potentially dilute (i.e., decrease) EPS. Information about such a potential dilution is valuable to a company's current and potential shareholders; therefore, accounting standards require companies to disclose what their EPS would be if all dilutive financial instruments were converted into common stock. The EPS that would result if all dilutive financial instruments were converted is called diluted EPS. In contrast, basic EPS is calculated using the reported earnings available to common shareholders of the parent company and the weighted average number of shares outstanding.

Companies are required to report both basic and diluted EPS as well as amounts for continuing operations. Exhibit 10 shows the per share amounts reported by $A B$ InBev at the bottom of its income statement that was presented in Exhibit 1 . The company's basic EPS ("before dilution") was $\$ 4.06$, and diluted EPS ("after dilution") was $\$ 3.98$ for 2017. In addition, in the same way that $A B$ InBev's income statement shows income from continuing operations separately from total income, EPS from continuing operations is also shown separately from total EPS. For 2017, the basic and diluted EPS from continuing operations were $\$ 4.04$ and $\$ 3.96$, respectively. Across all measures, AB InBev's EPS was much higher in 2017 than in 2016. An analyst would seek to understand the causes underlying the changes in EPS, a topic we will address following an explanation of the calculations of both basic and diluted EPS.

\section{Exhibit 10: AB InBev's Earnings Per Share}
12 Months Ended December 31

\begin{center}
\begin{tabular}{lccc}
\hline
 & $\mathbf{2 0 1 7}$ & $\mathbf{2 0 1 6}$ & $\mathbf{2 0 1 5}$ \\
\hline
Basic earnings per share & $\$ 4.06$ & $\$ 0.72$ & $\$ 5.05$ \\
Diluted earnings per share & 3.98 & 0.71 & 4.96 \\
$\begin{array}{l}\text { Basic earnings per share from continuing } \\ \text { operations }\end{array}$ & 4.04 & 0.69 & 5.05 \\
$\begin{array}{l}\text { Diluted earnings per share from continuing } \\ \text { operations }\end{array}$ & $\$ 3.96$ & $\$ 0.68$ & $\$ 4.96$ \\
\hline
\end{tabular}
\end{center}

\section{Basic EPS}
Basic EPS is the amount of income available to common shareholders divided by the weighted average number of common shares outstanding over a period. The amount of income available to common shareholders is the amount of net income remaining after preferred dividends (if any) have been paid. Thus, the formula to calculate basic EPS is:

$$
\text { Basic EPS }=\frac{\text { Net income }- \text { Preferred dividends }}{\text { Weighted average number of shares outstanding }}
$$

31 A warrant is a call option typically attached to securities issued by a company, such as bonds. A warrant gives the holder the right to acquire the company's stock from the company at a specified price within a specified time period. IFRS and US GAAP standards regarding earnings per share apply equally to call options, warrants, and equivalent instruments. The weighted average number of shares outstanding is a time weighting of common shares outstanding. For example, assume a company began the year with 2,000,000 common shares outstanding and repurchased 100,000 common shares on 1 July. The weighted average number of common shares outstanding would be the sum of $2,000,000$ shares $\times 1 / 2$ year $+1,900,000$ shares $\times 1 / 2$ year, or $1,950,000$ shares. So the company would use $1,950,000$ shares as the weighted average number of shares in calculating its basic EPS.

If the number of shares of common stock increases as a result of a stock dividend or a stock split, the EPS calculation reflects the change retroactively to the beginning of the period.

Example 6, Example 7, and Example 8 illustrate the computation of basic EPS.

\section{EXAMPLE 6}
\section{A Basic EPS Calculation (1)}
\begin{enumerate}
  \item For the year ended 31 December 2018, Shopalot Company had net income of $\$ 1,950,000$. The company had $1,500,000$ shares of common stock outstanding, no preferred stock, and no convertible financial instruments. What is Shopalot's basic EPS?
\end{enumerate}

\section{Solution:}
Shopalot's basic EPS is $\$ 1.30$ ( $\$ 1,950,000$ divided by $1,500,000$ shares).

\section{EXAMPLE 7}
\section{A Basic EPS Calculation (2)}
For the year ended 31 December 2018, Angler Products had net income of $\$ 2,500,000$. The company declared and paid $\$ 200,000$ of dividends on preferred stock. The company also had the following common stock share information:

Shares outstanding on 1 January 2018

$\begin{array}{r}1,000,000 \\ 200,000 \\ (100,000) \\ \hline\end{array}$

Shares repurchased (treasury shares) on 1 October 2018

Shares outstanding on 31 December 2018

$1,100,000$

\begin{enumerate}
  \item What is the company's weighted average number of shares outstanding?
\end{enumerate}

\section{Solution to 1:}
The weighted average number of shares outstanding is determined by the length of time each quantity of shares was outstanding:

\begin{center}
\begin{tabular}{lc}
$1,000,000 \times(3$ months $/ 12$ months $)=$ & 250,000 \\
$1,200,000 \times(6$ months $/ 12$ months $)=$ & 600,000 \\
$1,100,000 \times(3$ months $/ 12$ months $)=$ & 275,000 \\
\hline
Weighted average number of shares outstanding & $1,125,000$ \\
\hline
\end{tabular}
\end{center}

\begin{enumerate}
  \setcounter{enumi}{1}
  \item What is the company's basic EPS?
\end{enumerate}

\section{Solution to 2:}
Basic EPS $=($ Net income - Preferred dividends $) /$ Weighted average number of shares $=(\$ 2,500,000-\$ 200,000) / 1,125,000=\$ 2.04$

\section{EXAMPLE 8}
\section{A Basic EPS Calculation (3)}
\begin{enumerate}
  \item Assume the same facts as Example 7 except that on 1 December 2018, a previously declared 2-for-1 stock split took effect. Each shareholder of record receives two shares in exchange for each current share that he or she owns. What is the company's basic EPS?
\end{enumerate}

\section{Solution:}
For EPS calculation purposes, a stock split is treated as if it occurred at the beginning of the period. The weighted average number of shares would, therefore, be 2,250,000, and the basic EPS would be $\$ 1.02$ [= $(\$ 2,500,000-$ $\$ 200,000) / 2,250,000]$.

\section{DILUTED EPS: THE IF-CONVERTED METHOD}
describe how earnings per share is calculated and calculate and interpret a company's earnings per share (both basic and diluted earnings per share) for both simple and complex capital structures

If a company has a simple capital structure (in other words, one that includes no potentially dilutive financial instruments), then its basic EPS is equal to its diluted EPS. However, if a company has potentially dilutive financial instruments, its diluted EPS may differ from its basic EPS. Diluted EPS, by definition, is always equal to or less than basic EPS. The sections below describe the effects of three types of potentially dilutive financial instruments on diluted EPS: convertible preferred, convertible debt, and employee stock options. The final section explains why not all potentially dilutive financial instruments actually result in a difference between basic and diluted EPS.

\section{Diluted EPS When a Company Has Convertible Preferred Stock Outstanding}
When a company has convertible preferred stock outstanding, diluted EPS is calculated using the if-converted method. The if-converted method is based on what EPS would have been if the convertible preferred securities had been converted at the beginning of the period. In other words, the method calculates what the effect would have been if the convertible preferred shares converted at the beginning of the period. If the convertible shares had been converted, there would be two effects. First, the convertible preferred securities would no longer be outstanding; instead, additional common stock would be outstanding. Thus, under the if-converted method, the weighted average number of shares outstanding would be higher than in the basic EPS calculation. Second, if such a conversion had taken place, the company would not have paid preferred dividends. Thus, under the if-converted method, the net income available to common shareholders would be higher than in the basic EPS calculation.

Diluted EPS using the if-converted method for convertible preferred stock is equal to net income divided by the weighted average number of shares outstanding from the basic EPS calculation plus the additional shares of common stock that would be issued upon conversion of the preferred. Thus, the formula to calculate diluted EPS using the if-converted method for preferred stock is:

$$
\begin{aligned}
\text { Diluted EPS }= & \frac{\text { (Net income) }}{\begin{array}{l}
\text { (Weighted average number of shares } \\
\text { outstanding + New common shares that } \\
\text { would have been issued at conversion) }
\end{array}}
\end{aligned}
$$

A diluted EPS calculation using the if-converted method for preferred stock is provided in Example 9.

\section{EXAMPLE 9}
\section{A Diluted EPS Calculation Using the If-Converted Method for Preferred Stock}
\begin{enumerate}
  \item For the year ended 31 December 2018, Bright-Warm Utility Company (fictitious) had net income of $\$ 1,750,000$. The company had an average of 500,000 shares of common stock outstanding, 20,000 shares of convertible preferred, and no other potentially dilutive securities. Each share of preferred pays a dividend of $\$ 10$ per share, and each is convertible into five shares of the company's common stock. Calculate the company's basic and diluted EPS.
\end{enumerate}

\section{Solution:}
If the 20,000 shares of convertible preferred had each converted into 5 shares of the company's common stock, the company would have had an additional 100,000 shares of common stock (5 shares of common for each of the 20,000 shares of preferred). If the conversion had taken place, the company would not have paid preferred dividends of $\$ 200,000$ (\$10 per share for each of the 20,000 shares of preferred). As shown in Exhibit 11, the company's basic EPS was $\$ 3.10$ and its diluted EPS was $\$ 2.92$.

\begin{center}
\begin{tabular}{|c|c|c|}
\hline
 & Basic EPS & $\begin{array}{l}\text { Diluted EPS Using } \\ \text { If-Converted Method }\end{array}$ \\
\hline
Net income & $\$ 1,750,000$ & $\$ 1,750,000$ \\
\hline
Preferred dividend & $-200,000$ & 0 \\
\hline
Numerator & $\$ 1,550,000$ & $\$ 1,750,000$ \\
\hline
$\begin{array}{l}\text { Weighted average number of shares } \\ \text { outstanding }\end{array}$ & 500,000 & 500,000 \\
\hline
$\begin{array}{l}\text { Additional shares issued if preferred } \\ \text { converted }\end{array}$ & 0 & 100,000 \\
\hline
\end{tabular}
\end{center}

Exhibit 11: Calculation of Diluted EPS for Bright-Warm Utility Company Using the If-Converted Method: Case of Preferred Stock

\begin{center}
\begin{tabular}{lcc}
\hline
 & Basic EPS & $\begin{array}{c}\text { Diluted EPS Using } \\ \text { If-Converted Method }\end{array}$ \\
\hline
Denominator & 500,000 & 600,000 \\
\hline
EPS & $\$ 3.10$ & $\$ 2.92$ \\
\hline
\end{tabular}
\end{center}

\section{Diluted EPS When a Company Has Convertible Debt Outstanding}
When a company has convertible debt outstanding, the diluted EPS calculation also uses the if-converted method. Diluted EPS is calculated as if the convertible debt had been converted at the beginning of the period. If the convertible debt had been converted, the debt securities would no longer be outstanding; instead, additional shares of common stock would be outstanding. Also, if such a conversion had taken place, the company would not have paid interest on the convertible debt, so the net income available to common shareholders would increase by the after-tax amount of interest expense on the debt converted.

Thus, the formula to calculate diluted EPS using the if-converted method for convertible debt is:

$$
\text { Diluted EPS }=\frac{\begin{array}{l}
\text { (Net income }+ \text { After-tax interest on } \\
\text { convertible debt }- \text { Preferred dividends) }
\end{array}}{\begin{array}{l}
\text { (Weighted average number of shares } \\
\text { outstanding }+ \text { Additional common } \\
\text { shares that would have been } \\
\text { issued at conversion) }
\end{array}}
$$

A diluted EPS calculation using the if-converted method for convertible debt is provided in Example 10.

\section{EXAMPLE 10}
\section{A Diluted EPS Calculation Using the If-Converted Method for Convertible Debt}
\begin{enumerate}
  \item Oppnox Company (fictitious) reported net income of $\$ 750,000$ for the year ended 31 December 2018. The company had a weighted average of 690,000 shares of common stock outstanding. In addition, the company has only one potentially dilutive security: $\$ 50,000$ of 6 percent convertible bonds, convertible into a total of 10,000 shares. Assuming a tax rate of 30 percent, calculate Oppnox's basic and diluted EPS.
\end{enumerate}

\section{Solution:}
If the debt securities had been converted, the debt securities would no longer be outstanding and instead, an additional 10,000 shares of common stock would be outstanding. Also, if the debt securities had been converted, the company would not have paid interest of $\$ 3,000$ on the convertible debt, so net income available to common shareholders would have increased by $\$ 2,100$ [= \$3,000(1 - 0.30)] on an after-tax basis. Exhibit 12 illustrates the calculation of diluted EPS using the if-converted method for convertible debt.

\section{Exhibit 12: Calculation of Diluted EPS for Oppnox Company Using}
 the If-Converted Method: Case of a Convertible Bond\begin{center}
\begin{tabular}{|c|c|c|}
\hline
 & Basic EPS & $\begin{array}{l}\text { Diluted EPS Using } \\ \text { If-Converted Method }\end{array}$ \\
\hline
Net income & $\$ 750,000$ & $\$ 750,000$ \\
\hline
After-tax cost of interest &  & 2,100 \\
\hline
Numerator & $\$ 750,000$ & $\$ 752,100$ \\
\hline
$\begin{array}{l}\text { Weighted average number of shares } \\ \text { outstanding }\end{array}$ & 690,000 & 690,000 \\
\hline
If converted & 0 & 10,000 \\
\hline
Denominator & 690,000 & 700,000 \\
\hline
EPS & $\$ 1.09$ & $\mathbf{\$ 1 . 0 7}$ \\
\hline
\end{tabular}
\end{center}

\section{3}
\section{DILUTED EPS: THE TREASURY STOCK METHOD}
describe how earnings per share is calculated and calculate and interpret a company's earnings per share (both basic and diluted earnings per share) for both simple and complex capital structures

When a company has stock options, warrants, or their equivalents ${ }^{32}$ outstanding, diluted EPS is calculated as if the financial instruments had been exercised and the company had used the proceeds from exercise to repurchase as many shares of common stock as possible at the average market price of common stock during the period. The weighted average number of shares outstanding for diluted EPS is thus increased by the number of shares that would be issued upon exercise minus the number of shares that would have been purchased with the proceeds. This method is called the treasury stock method under US GAAP because companies typically hold repurchased shares as treasury stock. The same method is used under IFRS but is not named.

For the calculation of diluted EPS using this method, the assumed exercise of these financial instruments would have the following effects:

\begin{itemize}
  \item The company is assumed to receive cash upon exercise and, in exchange, to issue shares.

  \item The company is assumed to use the cash proceeds to repurchase shares at the weighted average market price during the period.

\end{itemize}

As a result of these two effects, the number of shares outstanding would increase by the incremental number of shares issued (the difference between the number of shares issued to the holders and the number of shares assumed to be repurchased by the company). For calculating diluted EPS, the incremental number of shares is weighted based upon the length of time the financial instrument was outstanding

32 Hereafter, options, warrants, and their equivalents will be referred to simply as "options" because the accounting treatment for EPS calculations is interchangeable for these instruments under IFRS and US GAAP. in the year. If the financial instrument was issued prior to the beginning of the year, the weighted average number of shares outstanding increases by the incremental number of shares. If the financial instruments were issued during the year, then the incremental shares are weighted by the amount of time the financial instruments were outstanding during the year.

The assumed exercise of these financial instruments would not affect net income. For calculating EPS, therefore, no change is made to the numerator. The formula to calculate diluted EPS using the treasury stock method (same method as used under IFRS but not named) for options is:

$$
\begin{aligned}
& \text { Diluted EPS }= \frac{\text { (Net income }- \text { Preferred dividends) }}{\text { [Weighted average number of shares }} \\
& \text { outstanding }+ \text { (New shares that would } \\
& \text { have been issued at option exercise- } \\
& \text { Shares that could have been purchased } \\
& \text { with cash received upon exercise) } \times \\
& \text { (Proportion of year during which the } \\
& \text { financial instruments were outstanding)] }
\end{aligned}
$$

A diluted EPS calculation using the treasury stock method for options is provided in Example 11.

\section{EXAMPLE 11}
\section{A Diluted EPS Calculation Using the Treasury Stock Method for Options}
\begin{enumerate}
  \item Hihotech Company (fictitious) reported net income of $\$ 2.3$ million for the year ended 30 June 2018 and had a weighted average of 800,000 common shares outstanding. At the beginning of the fiscal year, the company has outstanding 30,000 options with an exercise price of $\$ 35$. No other potentially dilutive financial instruments are outstanding. Over the fiscal year, the company's market price has averaged $\$ 55$ per share. Calculate the company's basic and diluted EPS.
\end{enumerate}

\section{Solution:}
Using the treasury stock method, we first calculate that the company would have received $\$ 1,050,000$ (\$35 for each of the 30,000 options exercised) if all the options had been exercised. The options would no longer be outstanding; instead, 30,000 shares of common stock would be outstanding. Under the treasury stock method, we assume that shares would be repurchased with the cash received upon exercise of the options. At an average market price of $\$ 55$ per share, the $\$ 1,050,000$ proceeds from option exercise, the company could have repurchased 19,091 shares. Therefore, the incremental number of shares issued is 10,909 (calculated as 30,000 minus 19,091 ). For the diluted EPS calculation, no change is made to the numerator. As shown in Exhibit 13, the company's basic EPS was $\$ 2.88$ and the diluted EPS was $\$ 2.84$. Exhibit 13: Calculation of Diluted EPS for Hihotech Company Using the Treasury Stock Method: Case of Stock Options

\begin{center}
\begin{tabular}{|c|c|c|}
\hline
 & Basic EPS & $\begin{array}{c}\text { Diluted EPS Using } \\ \text { Treasury Stock } \\ \text { Method }\end{array}$ \\
\hline
Net income & $\$ 2,300,000$ & $\$ 2,300,000$ \\
\hline
Numerator & $\$ 2,300,000$ & $\$ 2,300,000$ \\
\hline
$\begin{array}{l}\text { Weighted average number of shares } \\ \text { outstanding }\end{array}$ & 800,000 & 800,000 \\
\hline
If converted & 0 & 10,909 \\
\hline
Denominator & 800,000 & 810,909 \\
\hline
EPS & $\$ 2.88$ & $\$ 2.84$ \\
\hline
\end{tabular}
\end{center}

As noted, IFRS require a similar computation but does not refer to it as the "treasury stock method." The company is required to consider that any assumed proceeds are received from the issuance of new shares at the average market price for the period. These new "inferred" shares would be disregarded in the computation of diluted EPS, but the excess of the new shares that would be issued under options contracts minus the new inferred shares would be added to the weighted average number of shares outstanding. The results are the same as the treasury stock method, as shown in Example 12.

\section{EXAMPLE 12}
\section{Diluted EPS for Options under IFRS}
\begin{enumerate}
  \item Assuming the same facts as in Example 11, calculate the weighted average number of shares outstanding for diluted EPS under IFRS.
\end{enumerate}

\section{Solution:}
If the options had been exercised, the company would have received $\$ 1,050,000$. If this amount had been received from the issuance of new shares at the average market price of $\$ 55$ per share, the company would have issued 19,091 shares. IFRS refer to the 19,091 shares the company would have issued at market prices as the inferred shares. The number of shares issued under options $(30,000)$ minus the number of inferred shares $(19,091)$ equals 10,909 . This amount is added to the weighted average number of shares outstanding of 800,000 to get diluted shares of 810,909 . Note that this is the same result as that obtained under US GAAP; it is just derived in a different manner.

\textbackslash section\{OTHER ISSUES WITH DILUTED EPS AND CHANGES IN EPS

contrast dilutive and antidilutive securities and describe the implications of each for the earnings per share calculation

It is possible that some potentially convertible securities could be antidilutive (i.e., their inclusion in the computation would result in an EPS higher than the company's basic EPS). Under IFRS and US GAAP, antidilutive securities are not included in the calculation of diluted EPS. Diluted EPS should reflect the maximum potential dilution from conversion or exercise of potentially dilutive financial instruments. Diluted EPS will always be less than or equal to basic EPS. Example 13 provides an illustration of an antidilutive security.

\section{EXAMPLE 13}
\section{An Antidilutive Security}
\begin{enumerate}
  \item For the year ended 31 December 2018, Dim-Cool Utility Company (fictitious) had net income of $\$ 1,750,000$. The company had an average of 500,000 shares of common stock outstanding, 20,000 shares of convertible preferred, and no other potentially dilutive securities. Each share of preferred pays a dividend of $\$ 10$ per share, and each is convertible into three shares of the company's common stock. What was the company's basic and diluted EPS?
\end{enumerate}

\section{Solution:}
If the 20,000 shares of convertible preferred had each converted into 3 shares of the company's common stock, the company would have had an additional 60,000 shares of common stock (3 shares of common for each of the 20,000 shares of preferred). If the conversion had taken place, the company would not have paid preferred dividends of $\$ 200,000$ (\$10 per share for each of the 20,000 shares of preferred). The effect of using the if-converted method would be EPS of $\$ 3.13$, as shown in Exhibit 14. Because this is greater than the company's basic EPS of $\$ 3.10$, the securities are said to be antidilutive and the effect of their conversion would not be included in diluted EPS. Diluted EPS would be the same as basic EPS (i.e., \$3.10).

Exhibit 14: Calculation for an Antidilutive Security

Diluted EPS Using

Basic EPS

If-Converted Method

Net income

Preferred dividend

\section{Numerator}
Weighted average number of shares outstanding

If converted

Denominator

\begin{center}
\begin{tabular}{ccc}
$\$ 1,750,000$ &  &  \\
$-200,000$ & $\$ 1,750,000$ &  \\
\hline
$\$ 1,550,000$ & 0 &  \\
\hline
500,000 & $\$ 1,750,000$ &  \\
\cline { 3 - 3 }
0 & 500,000 &  \\
\cline { 3 - 4 }
 &  & 60,000 \\
\hline
500,000 & 560,000 &  \\
\hline
\end{tabular}
\end{center}

\begin{center}
\begin{tabular}{|c|c|c|c|}
\hline
 & Basic EPS & $\begin{array}{c}\text { Diluted EPS Using } \\ \text { If-Converted Method }\end{array}$ &  \\
\hline
EPS & $\$ 3.10$ & $\$ 3.13$ & $\begin{array}{l}\text { ←Exceeds basic EPS; security } \\ \text { is antidilutive and, therefore, } \\ \text { not included. Reported } \\ \text { diluted EPS= } \mathbf{\$ 3 . 1 0} \text {. }\end{array}$ \\
\hline
\end{tabular}
\end{center}

\section{Changes in EPS}
Having explained the calculations of both basic and diluted EPS, we return to an examination of changes in EPS. As noted above, AB InBev's fully diluted EPS from continuing operations increased from $\$ 0.68$ in 2016 to $\$ 3.96$ in 2017. In general, an increase in EPS results from an increase in net income, a decrease in the number of shares outstanding, or a combination of both. In the notes to its financial statements (not shown), AB InBev discloses that the weighted average number of shares for both the basic and fully-diluted calculations was greater in 2017 than in 2016. Thus, for AB InBev, the improvement in EPS from 2016 to 2017 was driven by an increase in net income. Changes in the numerator and denominator explain the changes in EPS arithmetically. To understand the business drivers of those changes requires further research. The next section presents analytical tools that an analyst can use to highlight areas for further examination.

\section{COMMON-SIZE ANALYSIS OF THE INCOME STATEMENT}
\begin{center}
\includegraphics[max width=\textwidth]{2023_05_04_b5cfa4f1bc883752f121g-056}
\end{center}

In this section, we apply two analytical tools to analyze the income statement: common-size analysis and income statement ratios. The objective of this analysis is to assess a company's performance over a period of time-compared with its own past performance or the performance of another company.

\section{Common-Size Analysis of the Income Statement}
Common-size analysis of the income statement can be performed by stating each line item on the income statement as a percentage of revenue. ${ }^{33}$ Common-size statements facilitate comparison across time periods (time series analysis) and across companies (cross-sectional analysis) because the standardization of each line item removes the effect of size.

To illustrate, Panel A of Exhibit 15 presents an income statement for three hypothetical companies in the same industry. Company A and Company B, each with $\$ 10$ million in sales, are larger (as measured by sales) than Company $C$, which has only $\$ 2$ million in sales. In addition, Companies A and B both have higher operating profit: $\$ 2$ million and $\$ 1.5$ million, respectively, compared with Company C's operating profit of only $\$ 400,000$.

How can an analyst meaningfully compare the performance of these companies? By preparing a common-size income statement, as illustrated in Panel B, an analyst can readily see that the percentages of Company C's expenses and profit relative to its sales are exactly the same as for Company A. Furthermore, although Company C's operating profit is lower than Company B's in absolute dollars, it is higher in percentage terms ( 20 percent for Company C compared with only 15 percent for Company B). For each $\$ 100$ of sales, Company C generates $\$ 5$ more operating profit than Company B. In other words, Company C is relatively more profitable than Company B based on this measure.

The common-size income statement also highlights differences in companies' strategies. Comparing the two larger companies, Company A reports significantly higher gross profit as a percentage of sales than does Company B (70 percent compared with 25 percent). Given that both companies operate in the same industry, why can Company A generate so much higher gross profit? One possible explanation is found by comparing the operating expenses of the two companies. Company A spends significantly more on research and development and on advertising than Company B. Expenditures on research and development likely result in products with superior technology. Expenditures on advertising likely result in greater brand awareness. So, based on these differences, it is likely that Company A is selling technologically superior products with a better brand image. Company B may be selling its products more cheaply (with a lower gross profit as a percentage of sales) but saving money by not investing in research and development or advertising. In practice, differences across companies are more subtle, but the concept is similar. An analyst, noting significant differences, would do more research and seek to understand the underlying reasons for the differences and their implications for the future performance of the companies.

\section{Exhibit 15}
Panel A: Income Statements for Companies A, B, and C (\$)

\begin{center}
\begin{tabular}{lccc}
\hline
 & $\mathbf{A}$ & $\mathbf{B}$ \\
\hline
Sales & $\$ 10,000,000$ & $\$ 10,000,000$ \\
Cost of sales & $3,000,000$ & $7,500,000,000$ \\
Gross profit & $7,000,000$ & $2,500,000$ \\
Selling, general, and administrative expenses & $1,000,000$ & $1,000,000$ \\
\cline { 4 - 5 }
\end{tabular}
\end{center}

33 This format can be distinguished as "vertical common-size analysis." As the reading on financial statement analysis discusses, there is another type of common-size analysis, known as "horizontal common-size analysis," that states items in relation to a selected base year value. Unless otherwise indicated, text references to "common-size analysis" refer to vertical analysis. Panel A: Income Statements for Companies A, B, and C (\$)

\begin{center}
\begin{tabular}{|c|c|c|c|}
\hline
 & $\mathbf{A}$ & B & C \\
\hline
Research and development & $2,000,000$ & - & 400,000 \\
\hline
Advertising & $2,000,000$ & - & 400,000 \\
\hline
Operating profit & $2,000,000$ & $1,500,000$ & 400,000 \\
\hline
\end{tabular}
\end{center}

Panel B: Common-Size Income Statements for Companies A, B, and C (\%)

\begin{center}
\begin{tabular}{lcc}
\hline
 & $\mathbf{A}$ & $\mathbf{B}$ \\
\hline
Sales & $100 \%$ & $100 \%$ \\
Cost of sales & 30 & $100 \%$ \\
Gross profit & 70 & 75 \\
Selling, general, and administrative expenses & 10 & 25 \\
Research and development & 20 & 10 \\
Advertising & 20 & 0 \\
Operating profit & 20 & 0 \\
\hline
\end{tabular}
\end{center}

Note: Each line item is expressed as a percentage of the company's sales.

For most expenses, comparison to the amount of sales is appropriate. However, in the case of taxes, it is more meaningful to compare the amount of taxes with the amount of pretax income. Using note disclosure, an analyst can then examine the causes for differences in effective tax rates. To project the companies' future net income, an analyst would project the companies' pretax income and apply an estimated effective tax rate determined in part by the historical tax rates.

Vertical common-size analysis of the income statement is particularly useful in cross-sectional analysis-comparing companies with each other for a particular time period or comparing a company with industry or sector data. The analyst could select individual peer companies for comparison, use industry data from published sources, or compile data from databases based on a selection of peer companies or broader industry data. For example, Exhibit 16 presents median common-size income statement data compiled for the components of the S\&P 500 classified into the $10 \mathrm{~S} \& \mathrm{P} /$ MSCI Global Industrial Classification System (GICS) sectors using 2017 data. Note that when compiling aggregate data such as this, some level of aggregation is necessary and less detail may be available than from peer company financial statements. The performance of an individual company can be compared with industry or peer company data to evaluate its relative performance.

Exhibit 16: Median Common-Size Income Statement Statistics for the S\&P 500 Classified by S\&P/MSCI GICS Sector Data for 2017

\begin{center}
\begin{tabular}{|c|c|c|c|c|c|c|}
\hline
 & Energy & Materials & Industrials & $\begin{array}{c}\text { Consumer } \\ \text { Discretionary }\end{array}$ & $\begin{array}{c}\text { Consumer } \\ \text { Staples }\end{array}$ & Health Care \\
\hline
Number of observations & 34 & 27 & 69 & 81 & 34 & 59 \\
\hline
Gross Margin & $37.7 \%$ & $33.0 \%$ & $36.8 \%$ & $37.6 \%$ & $43.4 \%$ & $59.0 \%$ \\
\hline
Operating Margin & $6.4 \%$ & $14.9 \%$ & $13.5 \%$ & $11.0 \%$ & $17.2 \%$ & $17.4 \%$ \\
\hline
Net Profit Margin & $4.9 \%$ & $9.9 \%$ & $8.8 \%$ & $6.0 \%$ & $10.9 \%$ & $7.2 \%$ \\
\hline
\end{tabular}
\end{center}

\begin{center}
\begin{tabular}{lccccc}
\hline
 & Financials & $\begin{array}{c}\text { Information } \\ \text { Technology }\end{array}$ & $\begin{array}{c}\text { Telecommunication } \\ \text { Services }\end{array}$ & 4 & 29 \\
Ueal Estate &  &  &  &  &  \\
\end{tabular}
\end{center}

Source: Based on data from Compustat. Operating margin based on EBIT (earnings before inter-

est and taxes.)

\section{INCOME STATEMENT RATIOS}
contrast dilutive and antidilutive securities and describe the implications of each for the earnings per share calculation

One aspect of financial performance is profitability. One indicator of profitability is net profit margin, also known as profit margin and return on sales, which is calculated as net income divided by revenue (or sales). ${ }^{34}$

Net profit margin $=\frac{\text { Net income }}{\text { Revenue }}$

Net profit margin measures the amount of income that a company was able to generate for each dollar of revenue. A higher level of net profit margin indicates higher profitability and is thus more desirable. Net profit margin can also be found directly on the common-size income statements.

For AB InBev, net profit margin based on continuing operations for 2017 was 16.2 percent (calculated as profit from continuing operations of $\$ 9,155$ million, divided by revenue of $\$ 56,444$ million). To judge this ratio, some comparison is needed. $A B$ InBev's profitability can be compared with that of another company or with its own previous performance. Compared with previous years, AB InBev's profitability is higher than in 2016 but lower than 2015. In 2016, net profit margin based on continuing operations was 6.0 percent, and in 2015 , it was 22.9 percent.

Another measure of profitability is the gross profit margin. Gross profit (gross margin) is calculated as revenue minus cost of goods sold, and the gross profit margin is calculated as the gross profit divided by revenue.

Gross profit margin $=\frac{\text { Gross profit }}{\text { Revenue }}$

The gross profit margin measures the amount of gross profit that a company generated for each dollar of revenue. A higher level of gross profit margin indicates higher profitability and thus is generally more desirable, although differences in gross profit margins across companies reflect differences in companies' strategies. For example, consider a company pursuing a strategy of selling a differentiated product (e.g., a product differentiated based on brand name, quality, superior technology, or patent protection). The company would likely be able to sell the differentiated product at a higher price than a similar, but undifferentiated, product and, therefore, would likely show a higher gross profit margin than a company selling an undifferentiated product.

34 In the definition of margin ratios of this type, "sales" is often used interchangeably with "revenue." "Return on sales" has also been used to refer to a class of profitability ratios having revenue in the denominator. Although a company selling a differentiated product would likely show a higher gross profit margin, this may take time. In the initial stage of the strategy, the company would likely incur costs to create a differentiated product, such as advertising or research and development, which would not be reflected in the gross margin calculation.

AB InBev's gross profit (shown in Exhibit 1) was $\$ 35,058$ million in 2017, $\$ 27,715$ million in 2016, and $\$ 26,467$ million in 2015. Expressing gross profit as a percentage of revenues, we see that the gross profit margin was 62.1 percent in $2017,60.9$ percent in 2016, and 60.7 percent in 2015. In absolute terms, AB InBev's gross profit was higher in 2016 than in 2015. However, AB InBev's gross profit margin was approximately constant between 2015 and 2016.

Exhibit 17 presents a common-size income statement for AB InBev, and highlights certain profitability ratios. The net profit margin and gross profit margin described above are just two of the many subtotals that can be generated from common-size income statements. Other "margins" used by analysts include the operating profit margin (profit from operations divided by revenue) and the pretax margin (profit before tax divided by revenue).

\section{Exhibit 17: AB InBev's Margins: Abbreviated Common-Size Income Statement}
12 Months Ended December 31

\begin{center}
\begin{tabular}{|c|c|c|c|c|c|c|}
\hline
 & \multicolumn{2}{|c|}{2017} & \multicolumn{2}{|c|}{2016} & \multicolumn{2}{|c|}{2015} \\
\hline
 & $\boldsymbol{\$}$ & $\%$ & $\mathbf{\$}$ & $\%$ & $\mathbf{\$}$ & $\%$ \\
\hline
Revenue & 56,444 & 100.0 & 45,517 & 100.0 & 43,604 & 100.0 \\
\hline
Cost of sales & $(21,386)$ & $(37.9)$ & $(17,803)$ & $(39.1)$ & $(17,137)$ & $(39.3)$ \\
\hline
Gross profit & 35,058 & 62.1 & 27,715 & 60.9 & 26,467 & 60.7 \\
\hline
Distribution expenses & $(5,876)$ & $(10.4)$ & $(4,543)$ & $(10.0)$ & $(4,259)$ & $(9.8)$ \\
\hline
Sales and marketing expenses & $(8,382)$ & $(14.9)$ & $(7,745)$ & $(17.0)$ & $(6,913)$ & $(15.9)$ \\
\hline
Administrative expenses & $(3,841)$ & $(6.8)$ & $(2,883)$ & $(6.3)$ & $(2,560)$ & $(5.9)$ \\
\hline
\multicolumn{7}{|c|}{Portions omitted} \\
\hline
Profit from operations & 17,152 & 30.4 & 12,882 & 28.3 & 13,904 & 31.9 \\
\hline
Finance cost & $(6,885)$ & $(12.2)$ & $(9,382)$ & $(20.6)$ & $(3,142)$ & $(7.2)$ \\
\hline
Finance income & 378 & 0.7 & 818 & 1.8 & 1,689 & 3.9 \\
\hline
Net finance income/(cost) & $(6,507)$ & $(11.5)$ & $(8,564)$ & $(18.8)$ & $(1,453)$ & $(3.3)$ \\
\hline
$\begin{array}{l}\text { Share of result of associates and joint } \\ \text { ventures }\end{array}$ & 430 & 0.8 & 16 & 0.0 & 10 & 0.0 \\
\hline
Profit before tax & 11,076 & 19.6 & 4,334 & 9.5 & 12,461 & 28.6 \\
\hline
Income tax expense & $(1,920)$ & $(3.4)$ & $(1,613)$ & $(3.5)$ & $(2,594)$ & $(5.9)$ \\
\hline
Profit from continuing operations & 9,155 & 16.2 & 2,721 & 6.0 & 9,867 & 22.6 \\
\hline
Profit from discontinued operations & 28 & 0.0 & 48 & 0.1 & - & - \\
\hline
Profit of the year & 9,183 & 16.3 & 2,769 & 6.1 & 9,867 & 22.6 \\
\hline
\end{tabular}
\end{center}

The profitability ratios and the common-size income statement yield quick insights about changes in a company's performance. For example, AB InBev's decrease in profitability in 2016 was not driven by a decrease in gross profit margin. Gross profit margin in 2016 was actually slightly higher than in 2015. The company's decrease in profitability in 2016 was driven in part by higher operating expenses and, in particular, by a significant increase in finance costs. The increased finance costs resulted from the 2016 merger with SABMiller. Valued at more than $\$ 100$ billion, the acquisition was one of the largest in history. The combination of AB InBev and SABMiller also explains the increase in revenue from around $\$ 45$ billion to over $\$ 56$ billion. The profitability ratios and the common-size income statement thus serve to highlight areas about which an analyst might wish to gain further understanding.

\section{COMPREHENSIVE INCOME}
describe, calculate, and interpret comprehensive income

describe other comprehensive income and identify major types of items included in it

The general expression for net income is revenue minus expenses. There are, however, certain items of revenue and expense that, by accounting convention, are excluded from the net income calculation. To understand how reported shareholders' equity of one period links with reported shareholders' equity of the next period, we must understand these excluded items, known as other comprehensive income. Under IFRS, other comprehensive income includes items of income and expense that are "not recognized in profit or loss as required or permitted by other IFRS." Total comprehensive income is "the change in equity during a period resulting from transaction and other events, other than those changes resulting from transactions with owners in their capacity as owners." 35

Under US GAAP, comprehensive income is defined as "the change in equity [net assets] of a business enterprise during a period from transactions and other events and circumstances from non-owner sources. It includes all changes in equity during a period except those resulting from investments by owners and distributions to owners." ${ }^{" 3}$ While the wording differs, comprehensive income is conceptually the same under IFRS and US GAAP.

Comprehensive income includes both net income and other revenue and expense items that are excluded from the net income calculation (collectively referred to as Other Comprehensive Income). Assume, for example, a company's beginning shareholders' equity is $€ 110$ million, its net income for the year is $€ 10$ million, its cash dividends for the year are $€ 2$ million, and there was no issuance or repurchase of common stock. If the company's actual ending shareholders' equity is $€ 123$ million, then $€ 5$ million $[€ 123-(€ 110+€ 10-€ 2)]$ has bypassed the net income calculation by being classified as other comprehensive income. If the company had no other comprehensive income, its ending shareholders' equity would have been $€ 118$ million $[€ 110+€ 10-€ 2]$.

Four types of items are treated as other comprehensive income under both IFRS and US GAAP. (The specific treatment of some of these items differs between the two sets of standards, but these types of items are common to both.)

\begin{itemize}
  \item Foreign currency translation adjustments. In consolidating the financial statements of foreign subsidiaries, the effects of translating the subsidiaries' balance sheet assets and liabilities at current exchange rates are included as other comprehensive income.
\end{itemize}

35 IAS 1, Presentation of Financial Statements.

36 FASB ASC Section 220-10-05 [Comprehensive Income-Overall-Overview and Background]. - Unrealized gains or losses on derivatives contracts accounted for as hedges. Changes in the fair value of derivatives are recorded each period, but certain changes in value are treated as other comprehensive income and thus bypass the income statement.

\begin{itemize}
  \item Unrealized holding gains and losses on a certain category of investment securities, namely, available-for-sale debt securities under US GAAP and securities designated as "fair value through other comprehensive income" under IFRS. (Note: IFRS, but not US GAAP, also includes a category of equity investments designated at fair value through other comprehensive income.)

  \item Certain costs of a company's defined benefit post-retirement plans that are not recognized in the current period.

\end{itemize}

In addition, under IFRS, other comprehensive income includes certain changes in the value of long-lived assets that are measured using the revaluation model rather than the cost model. Also, under IFRS, companies are not permitted to reclassify certain items of other comprehensive income to profit or loss, and companies must present separately the items of other comprehensive income that will and will not be reclassified subsequently to profit or loss.

The third type of item listed above is perhaps the simplest to illustrate. Holding gains on securities arise when a company owns securities over an accounting period, during which time the securities' value increases. Similarly, holding losses on securities arise when a company owns securities over a period during which time the securities' value decreases. If the company has not sold the securities (i.e., has not realized the gain or loss), its holding gain or loss is said to be unrealized. The question is: Should the company exclude unrealized gains and losses from income; reflect these unrealized holding gains and losses in its income statement (i.e., statement of profit and loss); or reflect these unrealized holding gains as other comprehensive income?

According to accounting standards, the answer depends on how the company has categorized the securities. Categorization depends on what the company intends to do with the securities (i.e., the business model for managing the asset) and on the cash flows of the security. Unrealized gains and losses are excluded from income for debt securities that the company intends to hold to maturity. These held-to-maturity debt securities are reported at their amortized cost, so no unrealized gains or losses are reported. For other securities reported at fair value, the unrealized gains or losses are reflected either in the income statement or as other comprehensive income.

Under US GAAP, unrealized gains and losses are reflected in the income statement for: (a) debt securities designated as trading securities; and (b) all investments in equity securities (other than investments giving rise to ownership positions that confer significant influence over the investee). The trading securities category pertains to a debt security that is acquired with the intent of selling it rather than holding it to collect the interest and principal payments. Also, under US GAAP, unrealized gains and losses are reflected as other comprehensive income for debt securities designated as available-for-sale securities. Available-for-sale debt securities are those not designated as either held-to-maturity or trading.

Under IFRS, unrealized gains and losses are reflected in the income statement for: (a) investments in equity investments, unless the company makes an irrevocable election otherwise; and (b) debt securities, if the securities do not fall into the other measurement categories or if the company makes an irrevocable election to show gains and losses on the income statement. These debt and equity investments are referred to as being measured at fair value through profit or loss. Also under IFRS, unrealized gains and losses are reflected as other comprehensive income for: (a) "debt securities held within a business model whose objective is achieved both by collecting contractual cash flows and selling financial assets"; and (b) equity investments for which the company makes an irrevocable election at initial recognition to show gains and losses as part of other comprehensive income. These debt and equity investments are referred to as being measured at fair value through other comprehensive income. Accounting for these securities is similar to accounting for US GAAP's available-for-sale debt securities.

Even where unrealized holding gains and losses are excluded from a company's net income (profit and loss), they are included in other comprehensive income and thus form a part of a company's comprehensive income.

\section{EXAMPLE 14}
Other Comprehensive Income

Assume a company's beginning shareholders' equity is $€ 200$ million, its net income for the year is $€ 20$ million, its cash dividends for the year are $€ 3$ million, and there was no issuance or repurchase of common stock. The company's actual ending shareholders' equity is $€ 227$ million.

\begin{enumerate}
  \item What amount has bypassed the net income calculation by being classified as other comprehensive income?
A. $€ 0$.
B. $€ 7$ million.
C. $€ 10$ million.
\end{enumerate}

\section{Solution to 1:}
$C$ is correct. If the company's actual ending shareholders' equity is $€ 227$ million, then $€ 10$ million [ $€ 227-(€ 200+€ 20-€ 3)$ ] has bypassed the net income calculation by being classified as other comprehensive income.

\begin{enumerate}
  \setcounter{enumi}{1}
  \item Which of the following statements best describes other comprehensive income?
\end{enumerate}

A. Income earned from diverse geographic and segment activities.

B. Income that increases stockholders' equity but is not reflected as part of net income.

C. Income earned from activities that are not part of the company's ordinary business activities.

\section{Solution to 2:}
$B$ is correct. Answers A and C are not correct because they do not specify whether such income is reported as part of net income and shown in the income statement.

\section{EXAMPLE 15}
Other Comprehensive Income in Analysis

\begin{enumerate}
  \item An analyst is looking at two comparable companies. Company A has a lower price/earnings (P/E) ratio than Company B, and the conclusion that has been suggested is that Company $A$ is undervalued. As part of examining this conclusion, the analyst decides to explore the question: What would the company's P/E look like if total comprehensive income per share-rather than net income per share-were used as the relevant metric?
\end{enumerate}

\begin{center}
\begin{tabular}{lcc}
\hline
 & Company A & Company B \\
\hline
Price & $\$ 35$ & $\$ 30$ \\
EPS & $\$ 1.60$ & $\$ 0.90$ \\
P/E ratio & $21.9 \times$ & $33.3 \times$ \\
Other comprehensive income (loss) \$ million & $(\$ 16.272)$ & $\$(1.757)$ \\
Shares (millions) & 22.6 & 25.1 \\
\hline
\end{tabular}
\end{center}

Solution:

As shown in the following table, part of the explanation for Company A's lower P/E ratio may be that its significant losses-accounted for as other comprehensive income $(\mathrm{OCI})$-are not included in the $\mathrm{P} / \mathrm{E}$ ratio.

\begin{center}
\begin{tabular}{lcc}
\hline
 & Company A & Company B \\
\hline
Price & $\$ 35$ & $\$ 30$ \\
EPS & $\$ 1.60$ & $\$ 0.90$ \\
OCI (loss) \$ million & $\$ 16.272)$ & $\$(1.757)$ \\
Shares (millions) & 22.6 & 25.1 \\
OCI (loss) per share & $\$(0.72)$ & $\$(0.07)$ \\
Comprehensive EPS = EPS + OCI per share & $\$ 0.88$ & $\$ 0.83$ \\
Price/Comprehensive EPS ratio & $39.8 \times$ & $36.1 \times$ \\
\hline
\end{tabular}
\end{center}

Both IFRS and US GAAP allow companies two alternative presentations. One alternative is to present two statements-a separate income statement and a second statement additionally including other comprehensive income. The other alternative is to present a single statement of other comprehensive income. Particularly in comparing financial statements of two companies, it is relevant to examine significant differences in comprehensive income.

\section{SUMMARY}
This reading has presented the elements of income statement analysis. The income statement presents information on the financial results of a company's business activities over a period of time; it communicates how much revenue the company generated during a period and what costs it incurred in connection with generating that revenue. A company's net income and its components (e.g., gross margin, operating earnings, and pretax earnings) are critical inputs into both the equity and credit analysis processes. Equity analysts are interested in earnings because equity markets often reward relatively high- or low-earnings growth companies with above-average or below-average valuations, respectively. Fixed-income analysts examine the components of income statements, past and projected, for information on companies' abilities to make promised payments on their debt over the course of the business cycle. Corporate financial announcements frequently emphasize income statements more than the other financial statements. Key points to this reading include the following:

\begin{itemize}
  \item The income statement presents revenue, expenses, and net income.

  \item The components of the income statement include: revenue; cost of sales; sales, general, and administrative expenses; other operating expenses; non-operating income and expenses; gains and losses; non-recurring items; net income; and EPS.

  \item An income statement that presents a subtotal for gross profit (revenue minus cost of goods sold) is said to be presented in a multi-step format. One that does not present this subtotal is said to be presented in a single-step format.

  \item Revenue is recognized in the period it is earned, which may or may not be in the same period as the related cash collection. Recognition of revenue when earned is a fundamental principle of accrual accounting.

  \item An analyst should identify differences in companies' revenue recognition methods and adjust reported revenue where possible to facilitate comparability. Where the available information does not permit adjustment, an analyst can characterize the revenue recognition as more or less conservative and thus qualitatively assess how differences in policies might affect financial ratios and judgments about profitability.

  \item As of the beginning of 2018, revenue recognition standards have converged. The core principle of the converged standards is that revenue should be recognized to "depict the transfer of promised goods or services to customers in an amount that reflects the consideration to which the entity expects to be entitled in an exchange for those goods or services."

  \item To achieve the core principle, the standard describes the application of five steps in recognizing revenue. The standard also specifies the treatment of some related contract costs and disclosure requirements.

  \item The general principles of expense recognition include a process to match expenses either to revenue (such as, cost of goods sold) or to the time period in which the expenditure occurs (period costs such as, administrative salaries) or to the time period of expected benefits of the expenditures (such as, depreciation).

  \item In expense recognition, choice of method (i.e., depreciation method and inventory cost method), as well as estimates (i.e., uncollectible accounts, warranty expenses, assets' useful life, and salvage value) affect a company's reported income. An analyst should identify differences in companies' expense recognition methods and adjust reported financial statements where possible to facilitate comparability. Where the available information does not permit adjustment, an analyst can characterize the policies and estimates as more or less conservative and thus qualitatively assess how differences in policies might affect financial ratios and judgments about companies' performance.

  \item To assess a company's future earnings, it is helpful to separate those prior years' items of income and expense that are likely to continue in the future from those items that are less likely to continue.

  \item Under IFRS, a company should present additional line items, headings, and subtotals beyond those specified when such presentation is relevant to an understanding of the entity's financial performance. Some items from prior years clearly are not expected to continue in future periods and are separately disclosed on a company's income statement. Under US GAAP, unusual and/or infrequently occurring items, which are material, are presented separately within income from continuing operations.

  \item Non-operating items are reported separately from operating items on the income statement. Under both IFRS and US GAAP, the income statement reports separately the effect of the disposal of a component operation as a "discontinued" operation.

  \item Basic EPS is the amount of income available to common shareholders divided by the weighted average number of common shares outstanding over a period. The amount of income available to common shareholders is the amount of net income remaining after preferred dividends (if any) have been paid.

  \item If a company has a simple capital structure (i.e., one with no potentially dilutive securities), then its basic EPS is equal to its diluted EPS. If, however, a company has dilutive securities, its diluted EPS is lower than its basic EPS.

  \item Diluted EPS is calculated using the if-converted method for convertible securities and the treasury stock method for options.

  \item Common-size analysis of the income statement involves stating each line item on the income statement as a percentage of sales. Common-size statements facilitate comparison across time periods and across companies of different sizes.

  \item Two income-statement-based indicators of profitability are net profit margin and gross profit margin.

  \item Comprehensive income includes both net income and other revenue and expense items that are excluded from the net income calculation.

\end{itemize}

\section{PRACTICE PROBLEMS}
\begin{enumerate}
  \item Expenses on the income statement may be grouped by:
A. nature, but not by function.
B. function, but not by nature.
C. either function or nature.

  \item An example of an expense classification by function is:

\end{enumerate}

A. tax expense.

B. interest expense.

C. cost of goods sold.

\begin{enumerate}
  \setcounter{enumi}{2}
  \item Denali Limited, a manufacturing company, had the following income statement information:
\end{enumerate}

Revenue

$\$ 4,000,000$

Cost of goods sold

$\$ 3,000,000$

Other operating expenses

$\$ 500,000$

Interest expense

$\$ 100,000$

Tax expense

$\$ 120,000$

Denali's gross profit is equal to:
A. $\$ 280,000$.
B. $\$ 500,000$.
C. $\$ 1,000,000$.

\begin{enumerate}
  \setcounter{enumi}{3}
  \item Under IFRS, income includes increases in economic benefits from:
\end{enumerate}

A. increases in liabilities not related to owners' contributions.

B. enhancements of assets not related to owners' contributions.

C. increases in owners' equity related to owners' contributions.

\begin{enumerate}
  \setcounter{enumi}{4}
  \item Fairplay had the following information related to the sale of its products during 2009, which was its first year of business:
Revenue
$\$ 1,000,000$
Returns of goods sold
$\$ 100,000$
Cash collected
$\$ 800,000$
Cost of goods sold
$\$ 700,000$
\end{enumerate}

Under the accrual basis of accounting, how much net revenue would be reported on Fairplay's 2009 income statement?
A. $\$ 200,000$.
B. $\$ 900,000$. C. $\$ 1,000,000$.

\begin{enumerate}
  \setcounter{enumi}{5}
  \item Apex Consignment sells items over the internet for individuals on a consignment basis. Apex receives the items from the owner, lists them for sale on the internet, and receives a 25 percent commission for any items sold. Apex collects the full amount from the buyer and pays the net amount after commission to the owner. Unsold items are returned to the owner after 90 days. During 2009, Apex had the following information:
\end{enumerate}

\begin{itemize}
  \item Total sales price of items sold during 2009 on consignment was $€ 2,000,000$.

  \item Total commissions retained by Apex during 2009 for these items was $€ 500,000$.

\end{itemize}

How much revenue should Apex report on its 2009 income statement?
A. $€ 500,000$.
B. $€ 2,000,000$.
C. $€ 1,500,000$.

\begin{enumerate}
  \setcounter{enumi}{6}
  \item A company previously expensed the incremental costs of obtaining a contract. All else being equal, adopting the May 2014 IASB and FASB converged accounting standards on revenue recognition makes the company's profitability initially appear:
A. lower.
B. unchanged.
C. higher.

  \item During 2009, Accent Toys Plc., which began business in October of that year, purchased 10,000 units of a toy at a cost of $\pounds 10$ per unit in October. The toy sold well in October. In anticipation of heavy December sales, Accent purchased 5,000 additional units in November at a cost of $€ 11$ per unit. During 2009, Accent sold 12,000 units at a price of $€ 15$ per unit. Under the first in, first out (FIFO) method, what is Accent's cost of goods sold for 2009?
A. $\pounds 120,000$.
B. $\pounds 122,000$.
C. $\pounds 124,000$.

  \item Using the same information as in the previous question, what would Accent's cost of goods sold be under the weighted average cost method?
A. $\pounds 120,000$.
B. $\pounds 122,000$.
C. $\pounds 124,000$.

  \item Which inventory method is least likely to be used under IFRS?

\end{enumerate}

A. First in, first out (FIFO).

B. Last in, first out (LIFO).

C. Weighted average. 11. At the beginning of 2009, Glass Manufacturing purchased a new machine for its assembly line at a cost of $\$ 600,000$. The machine has an estimated useful life of 10 years and estimated residual value of $\$ 50,000$. Under the straight-line method, how much depreciation would Glass take in 2010 for financial reporting purposes?
A. $\$ 55,000$
B. $\$ 60,000$.
C. $\$ 65,000$.

\begin{enumerate}
  \setcounter{enumi}{11}
  \item Using the same information as in Question 11, how much depreciation would Glass take in 2009 for financial reporting purposes under the double-declining balance method?
A. $\$ 60,000$.
B. $\$ 110,000$.
C. $\$ 120,000$.

  \item Which combination of depreciation methods and useful lives is most conservative in the year a depreciable asset is acquired?
A. Straight-line depreciation with a short useful life.
B. Declining balance depreciation with a long useful life.
C. Declining balance depreciation with a short useful life.

  \item Under IFRS, a loss from the destruction of property in a fire would most likely be classified as:
A. continuing operations.
B. discontinued operations.
C. other comprehensive income.

  \item A company chooses to change an accounting policy. This change requires that, if practical, the company restate its financial statements for:
A. all prior periods.
B. current and future periods.
C. prior periods shown in a report.

  \item For 2009, Flamingo Products had net income of $\$ 1,000,000$. At 1 January 2009 , there were 1,000,000 shares outstanding. On 1 July 2009 , the company issued 100,000 new shares for $\$ 20$ per share. The company paid $\$ 200,000$ in dividends to common shareholders. What is Flamingo's basic earnings per share for 2009?
A. $\$ 0.80$.
B. $\$ 0.91$
C. $\$ 0.95$.

  \item A company with no debt or convertible securities issued publicly traded common stock three times during the current fiscal year. Under both IFRS and US GAAP, the company's:

\end{enumerate}

A. basic EPS equals its diluted EPS.

B. capital structure is considered complex at year-end.

C. basic EPS is calculated by using a simple average number of shares outstanding.

\begin{enumerate}
  \setcounter{enumi}{17}
  \item For its fiscal year-end, Sublyme Corporation reported net income of $\$ 200$ million and a weighted average of 50,000,000 common shares outstanding. There are 2,000,000 convertible preferred shares outstanding that paid an annual dividend of $\$ 5$. Each preferred share is convertible into two shares of the common stock. The diluted EPS is closest to:
A. $\$ 3.52$.
B. $\$ 3.65$.
C. $\$ 3.70$.

  \item For its fiscal year-end, Calvan Water Corporation (CWC) reported net income of $\$ 12$ million and a weighted average of 2,000,000 common shares outstanding. The company paid $\$ 800,000$ in preferred dividends and had 100,000 options outstanding with an average exercise price of $\$ 20$. CWC's market price over the year averaged $\$ 25$ per share. CWC's diluted EPS is closest to:
A. $\$ 5.33$.
B. $\$ 5.54$.
C. $\$ 5.94$.

  \item Laurelli Builders (LB) reported the following financial data for year-end 31 December:

\end{enumerate}

$\begin{array}{lc}\text { Common shares outstanding, } 1 \text { January } & 2,020,000 \\ \text { Common shares issued as stock dividend, 1 June } & 380,000 \\ \text { Warrants outstanding, 1 January } & 500,000 \\ \text { Net income } & \$ 3,350,000 \\ \text { Preferred stock dividends paid } & \$ 430,000 \\ \text { Common stock dividends paid } & \$ 240,000\end{array}$

Which statement about the calculation of LB's EPS is most accurate?

A. LB's basic EPS is $\$ 1.12$.

B. LB's diluted EPS is equal to or less than its basic EPS.

C. The weighted average number of shares outstanding is 2,210,000.

\begin{enumerate}
  \setcounter{enumi}{20}
  \item Cell Services Inc. (CSI) had 1,000,000 average shares outstanding during all of 2009. During 2009, CSI also had 10,000 options outstanding with exercise prices of $\$ 10$ each. The average stock price of CSI during 2009 was $\$ 15$. For purposes of computing diluted earnings per share, how many shares would be used in the denominator?
\end{enumerate}

A. $1,003,333$.
B. $1,006,667$.
C. $1,010,000$.

\begin{enumerate}
  \setcounter{enumi}{21}
  \item When calculating diluted EPS, which of the following securities in the capital structure increases the weighted average number of common shares outstanding without affecting net income available to common shareholders?
A. Stock options
B. Convertible debt that is dilutive
C. Convertible preferred stock that is dilutive

  \item Which statement is most accurate? A common size income statement:

\end{enumerate}

A. restates each line item of the income statement as a percentage of net income.

B. allows an analyst to conduct cross-sectional analysis by removing the effect of company size.

C. standardizes each line item of the income statement but fails to help an analyst identify differences in companies' strategies.

\begin{enumerate}
  \setcounter{enumi}{23}
  \item Selected year-end financial statement data for Workhard are shown below.
\end{enumerate}

\begin{center}
\begin{tabular}{lc}
\hline
 & \$ millions \\
\hline
Beginning shareholders' equity & 475 \\
Ending shareholders' equity & 493 \\
Unrealized gain on available-for-sale securities & 5 \\
Unrealized loss on derivatives accounted for as hedges & -3 \\
Foreign currency translation gain on consolidation & 2 \\
Dividends paid & 1 \\
Net income & 15 \\
\hline
\end{tabular}
\end{center}

Workhard's comprehensive income for the year:
A. is $\$ 18$ million.
B. is increased by the derivatives accounted for as hedges.
C. includes $\$ 4$ million in other comprehensive income.

\begin{enumerate}
  \setcounter{enumi}{24}
  \item When preparing an income statement, which of the following items would most likely be classified as other comprehensive income?
A. A foreign currency translation adjustment
B. An unrealized gain on a security held for trading purposes
C. A realized gain on a derivative contract not accounted for as a hedge
\end{enumerate}

\section{SOLUTIONS}
\begin{enumerate}
  \item C is correct. IAS No. 1 states that expenses may be categorized by either nature or function.

  \item C is correct. Cost of goods sold is a classification by function. The other two expenses represent classifications by nature.

  \item C is correct. Gross margin is revenue minus cost of goods sold. Answer A represents net income and B represents operating income.

  \item B is correct. Under IFRS, income includes increases in economic benefits from increases in assets, enhancement of assets, and decreases in liabilities.

  \item B is correct. Net revenue is revenue for goods sold during the period less any returns and allowances, or $\$ 1,000,000$ minus $\$ 100,000=\$ 900,000$.

  \item A is correct. Apex is not the owner of the goods and should only report its net commission as revenue.

  \item $C$ is correct. Under the converged accounting standards, the incremental costs of obtaining a contract and certain costs incurred to fulfill a contract must be capitalized. If a company expensed these incremental costs in the years prior to adopting the converged standards, all else being equal, its profitability will appear higher under the converged standards.

  \item B is correct. Under the first in, first out (FIFO) method, the first 10,000 units sold came from the October purchases at $\pounds 10$, and the next 2,000 units sold came from the November purchases at $\pounds 11$.

  \item C is correct. Under the weighted average cost method:

\end{enumerate}

\begin{center}
\begin{tabular}{ccc}
October purchases & 10,000 units & $\$ 100,000$ \\
November purchases & 5,000 units & $\$ 55,000$ \\
\cline { 2 - 3 }
Total & 15,000 units & $\$ 155,000$ \\
\cline { 2 - 3 }
\end{tabular}
\end{center}

$\$ 155,000 / 15,000$ units $=\$ 10.3333$

$\$ 10.3333 \times 12,000$ units $=\$ 124,000$

\begin{enumerate}
  \setcounter{enumi}{9}
  \item B is correct. The last in, first out (LIFO) method is not permitted under IFRS. The other two methods are permitted.

  \item A is correct. Straight-line depreciation would be $(\$ 600,000-\$ 50,000) / 10$, or $\$ 55,000$.

  \item C is correct. Double-declining balance depreciation would be $\$ 600,000 \times 20$ percent (twice the straight-line rate). The residual value is not subtracted from the initial book value to calculate depreciation. However, the book value (carrying amount) of the asset will not be reduced below the estimated residual value.

  \item $C$ is correct. This would result in the highest amount of depreciation in the first year and hence the lowest amount of net income relative to the other choices.

  \item A is correct. A fire may be infrequent, but it would still be part of continuing operations and reported in the profit and loss statement. Discontinued operations relate to a decision to dispose of an operating division. 15. $\mathrm{C}$ is correct. If a company changes an accounting policy, the financial statements for all fiscal years shown in a company's financial report are presented, if practical, as if the newly adopted accounting policy had been used throughout the entire period; this retrospective application of the change makes the financial results of any prior years included in the report comparable. Notes to the financial statements describe the change and explain the justification for the change.

  \item $\mathrm{C}$ is correct. The weighted average number of shares outstanding for 2009 is $1,050,000$. Basic earnings per share would be $\$ 1,000,000$ divided by $1,050,000$, or $\$ 0.95$.

  \item A is correct. Basic and diluted EPS are equal for a company with a simple capital structure. A company that issues only common stock, with no financial instruments that are potentially convertible into common stock has a simple capital structure. Basic EPS is calculated using the weighted average number of shares outstanding.

  \item $\mathrm{C}$ is correct.

\end{enumerate}

Diluted EPS $=($ Net income $) /($ Weighted average number of shares outstanding + New common shares that would have been issued at conversion)

$$
\begin{aligned}
& =\$ 200,000,000 /[50,000,000+(2,000,000 \times 2)] \\
& =\$ 3.70
\end{aligned}
$$

The diluted EPS assumes that the preferred dividend is not paid and that the shares are converted at the beginning of the period.

\begin{enumerate}
  \setcounter{enumi}{18}
  \item B is correct. The formula to calculate diluted EPS is as follows:
\end{enumerate}

Diluted EPS $=($ Net income - Preferred dividends $) /[$ Weighted average number of shares outstanding + (New shares that would have been issued at option exercise - Shares that could have been purchased with cash received upon exercise) $\times$ (Proportion of year during which the financial instruments were outstanding)].

The underlying assumption is that outstanding options are exercised, and then the proceeds from the issuance of new shares are used to repurchase shares already outstanding:

Proceeds from option exercise $=100,000 \times \$ 20=\$ 2,000,000$

Shares repurchased $=\$ 2,000,000 / \$ 25=80,000$

The net increase in shares outstanding is thus $100,000-80,000=20,000$. Therefore, the diluted EPS for $C W C=(\$ 12,000,000-\$ 800,000) / 2,020,000=\$ 5.54$.

\begin{enumerate}
  \setcounter{enumi}{19}
  \item B is correct. LB has warrants in its capital structure; if the exercise price is less than the weighted average market price during the year, the effect of their conversion is to increase the weighted average number of common shares outstanding, causing diluted EPS to be lower than basic EPS. If the exercise price is equal to the weighted average market price, the number of shares issued equals the number of shares repurchased. Therefore, the weighted average number of common shares outstanding is not affected and diluted EPS equals basic EPS. If the exercise price is greater than the weighted average market price, the effect of their conversion is anti-dilutive. As such, they are not included in the calculation of basic EPS. LB's basic EPS is $\$ 1.22$ [= $(\$ 3,350,000-\$ 430,000) / 2,400,000]$. Stock dividends are treated as having been issued retroactively to the beginning of the period. 21. A is correct. With stock options, the treasury stock method must be used. Under that method, the company would receive $\$ 100,000(10,000 \times \$ 10)$ and would repurchase 6,667 shares $(\$ 100,000 / \$ 15)$. The shares for the denominator would be:
\end{enumerate}

$\begin{array}{lc}\text { Shares outstanding } & 1,000,000 \\ \text { Options exercises } & 10,000 \\ \text { Treasury shares purchased } & (6,667) \\ \text { Denominator } & 1,003,333\end{array}$

\begin{enumerate}
  \setcounter{enumi}{21}
  \item A is correct. When a company has stock options outstanding, diluted EPS is calculated as if the financial instruments had been exercised and the company had used the proceeds from the exercise to repurchase as many shares possible at the weighted average market price of common stock during the period. As a result, the conversion of stock options increases the number of common shares outstanding but has no effect on net income available to common shareholders. The conversion of convertible debt increases the net income available to common shareholders by the after-tax amount of interest expense saved. The conversion of convertible preferred shares increases the net income available to common shareholders by the amount of preferred dividends paid; the numerator becomes the net income.

  \item B is correct. Common size income statements facilitate comparison across time periods (time-series analysis) and across companies (cross-sectional analysis) by stating each line item of the income statement as a percentage of revenue. The relative performance of different companies can be more easily assessed because scaling the numbers removes the effect of size. A common size income statement states each line item on the income statement as a percentage of revenue. The standardization of each line item makes a common size income statement useful for identifying differences in companies' strategies.

  \item $\mathrm{C}$ is correct. Comprehensive income includes both net income and other comprehensive income.

\end{enumerate}

Other comprehensive income = Unrealized gain on available-for-sale securities Unrealized loss on derivatives accounted for as hedges + Foreign currency translation gain on consolidation

$=\$ 5$ million $-\$ 3$ million $+\$ 2$ million

$=\$ 4$ million

Alternatively,

Comprehensive income - Net income $=$ Other comprehensive income

Comprehensive income $=($ Ending shareholders equity - Beginning shareholders equity) + Dividends

$=(\$ 493$ million $-\$ 475$ million $)+\$ 1$ million

$=\$ 18$ million $+\$ 1$ million $=\$ 19$ million

Net income is $\$ 15$ million so other comprehensive income is $\$ 4$ million.

\begin{enumerate}
  \setcounter{enumi}{24}
  \item A is correct. Other comprehensive income includes items that affect shareholders' equity but are not reflected in the company's income statement. In consolidating the financial statements of foreign subsidiaries, the effects of translating the subsidiaries' balance sheet assets and liabilities at current exchange rates are included as other comprehensive income.
\end{enumerate}

\section{LEARNING MODULE 2}
\section{Understanding Balance Sheets}
Elaine Henry, PhD, CFA, is at Stevens Institute of Technology (USA). Thomas R Robinson, PhD, CFA, CAIA, Robinson Global Investment Management LLC, (USA).

\section{LEARNING OUTCOME}
\begin{center}
\begin{tabular}{c|l}
Mastery & The candidate should be able to: \\
\hline
$\square$ & $\begin{array}{l}\text { describe the elements of the balance sheet: assets, liabilities, and } \\ \text { equity } \\ \text { describe uses and limitations of the balance sheet in financial } \\ \text { analysis } \\ \text { describe alternative formats of balance sheet presentation }\end{array}$ \\
$\square$ & $\begin{array}{l}\text { contrast current and non-current assets and current and non-current } \\ \text { liabilities } \\ \text { describe different types of assets and liabilities and the measurement } \\ \text { bases of each } \\ \text { describe the components of shareholders' equity }\end{array}$ \\
$\square$ &  \\
$\square$ &  \\
\end{tabular}
\end{center}

Note: Changes in accounting standards as well as new rulings and/or pronouncements issued after the publication of the readings on financial reporting and analysis may cause some of the information in these readings to become dated. Candidates are not responsible for anything that occurs after the readings were published. In addition, candidates are expected to be familiar with the analytical frameworks contained in the readings, as well as the implications of alternative accounting methods for financial analysis and valuation discussed in the readings. Candidates are also responsible for the content of accounting standards, but not for the actual reference numbers. Finally, candidates should be aware that certain ratios may be defined and calculated differently. When alternative ratio definitions exist and no specific definition is given, candidates should use the ratio definitions emphasized in the readings.

\section{INTRODUCTION AND COMPONENTS OF THE BALANCE SHEET}
describe the elements of the balance sheet: assets, liabilities, and equity describe uses and limitations of the balance sheet in financial analysis The balance sheet provides information on a company's resources (assets) and its sources of capital (equity and liabilities/debt). This information helps an analyst assess a company's ability to pay for its near-term operating needs, meet future debt obligations, and make distributions to owners. The basic equation underlying the balance sheet is Assets = Liabilities + Equity.

Analysts should be aware that different types of assets and liabilities may be measured differently. For example, some items are measured at historical cost or a variation thereof and others at fair value. ${ }^{1}$ An understanding of the measurement issues will facilitate analysis. The balance sheet measurement issues are, of course, closely linked to the revenue and expense recognition issues affecting the income statement. Throughout this reading, we describe and illustrate some of the linkages between the measurement issues affecting the balance sheet and the revenue and expense recognition issues affecting the income statement.

This reading is organized as follows: In Section 2, we describe and give examples of the elements and formats of balance sheets. Section 3 discusses current assets and current liabilities. Section 4 focuses on assets, and Section 5 focuses on liabilities. Section 6 describes the components of equity and illustrates the statement of changes in shareholders' equity. Section 7 introduces balance sheet analysis. A summary of the key points and practice problems in the CFA Institute multiple-choice format conclude the reading.

\section{Components and Format of the Balance Sheet}
The balance sheet (also called the statement of financial position or statement of financial condition) discloses what an entity owns (or controls), what it owes, and what the owners' claims are at a specific point in time. ${ }^{2}$

The financial position of a company is described in terms of its basic elements (assets, liabilities, and equity):

\begin{itemize}
  \item Assets (A) are what the company owns (or controls). More formally, assets are resources controlled by the company as a result of past events and from which future economic benefits are expected to flow to the entity.

  \item Liabilities (L) are what the company owes. More formally, liabilities represent obligations of a company arising from past events, the settlement of which is expected to result in a future outflow of economic benefits from the entity.

  \item Equity (E) represents the owners' residual interest in the company's assets after deducting its liabilities. Commonly known as shareholders' equity or owners' equity, equity is determined by subtracting the liabilities from the assets of a company, giving rise to the accounting equation: $\mathbf{A}-\mathbf{L}=\mathbf{E}$ or $\mathbf{A}$ $=\mathbf{L}+\mathbf{E}$

\end{itemize}

The equation $\mathrm{A}=\mathrm{L}+\mathrm{E}$ is sometimes summarized as follows: The left side of the equation reflects the resources controlled by the company and the right side reflects how those resources were financed. For all financial statement items, an item should

1 IFRS and US GAAP define "fair value" as an exit price, i.e., the price that would be received to sell an asset or paid to transfer a liability in an orderly transaction between market participants at the measurement date (IFRS 13, FASB ASC Topic 820).

2 IFRS uses the term "statement of financial position" (IAS 1 Presentation of Financial Statements), and US GAAP uses the terms "balance sheet" and "statement of financial position" interchangeably (ASC 210-10-05 [Balance Sheet-Overall-Overview and Background]). only be recognized in the financial statements if it is probable that any future economic benefit associated with the item will flow to or from the entity and if the item has a cost or value that can be measured with reliability. ${ }^{3}$

The balance sheet provides important information about a company's financial condition, but the balance sheet amounts of equity (assets, net of liabilities) should not be viewed as a measure of either the market or intrinsic value of a company's equity for several reasons. First, the balance sheet under current accounting standards is a mixed model with respect to measurement. Some assets and liabilities are measured based on historical cost, sometimes with adjustments, whereas other assets and liabilities are measured based on a fair value, which represents its current value as of the balance sheet date. The measurement bases may have a significant effect on the amount reported. Second, even the items measured at current value reflect the value that was current at the end of the reporting period. The values of those items obviously can change after the balance sheet is prepared. Third, the value of a company is a function of many factors, including future cash flows expected to be generated by the company and current market conditions. Important aspects of a company's ability to generate future cash flows-for example, its reputation and management skills-are not included in its balance sheet.

\section{Balance Sheet Components}
To illustrate the components and formats of balance sheets, we show the major subtotals from two companies' balance sheets. Exhibit 1 and Exhibit 2 are based on the balance sheets of SAP Group and Apple Inc. SAP Group is a leading business software company based in Germany and prepares its financial statements in accordance with IFRS. Apple is a technology manufacturer based in the United States and prepares its financial statements in accordance with US GAAP. For purposes of discussion, Exhibit 1 and Exhibit 2 show only the main subtotals and totals of these companies' balance sheets. Additional exhibits throughout this reading will expand on these subtotals.

Exhibit 1: SAP Group Consolidated Statements of Financial Position

(Excerpt) (in millions of $\epsilon$ )

\begin{center}
\begin{tabular}{lcc}
\hline
 & \multicolumn{2}{c}{$\mathbf{3 1}$ December} \\
\hline
Assets & $\mathbf{2 0 1 7}$ & $\mathbf{2 0 1 6 *}$ \\
\hline
Total current assets & 11,930 & 11,564 \\
Total non-current assets & 30,567 & 32,713 \\
\cline { 2 - 3 }
Total assets & 42,497 & 44,277 \\
\hline
Equity and liabilities &  &  \\
\hline
Total current liabilities & 10,210 & 9,675 \\
Total non-current liabilities & 6,747 & 8,205 \\
\cline { 2 - 3 }
Total liabilities & 16,957 & 17,880 \\
\hline
Total equity & 25,540 & 26,397 \\
\hline
Equity and liabilities & 42,497 & 44,277 \\
\hline
\end{tabular}
\end{center}

Source: SAP Group 2017 annual report.

Notes: Numbers exactly from the annual report as prepared by the company, which reflects some rounding. * Numbers are the reclassified numbers from the SAP Group 2017 annual report.

Exhibit 2: Apple Inc. Consolidated Balance Sheets

(Excerpt)* (in millions of \$)

\begin{center}
\begin{tabular}{lcc}
\hline
Assets & $\mathbf{3 0}$ September $\mathbf{2 0 1 7}$ & $\mathbf{2 4}$ September 2016 \\
\hline
Total current assets & 128,645 & 106,869 \\
[All other assets]
 & 246,674 & 214,817 \\
\hline
Total assets & 375,319 & 321,686 \\
\hline
Liabilities and shareholders' equity & 100,814 &  \\
\hline
Total current liabilities & 140,458 & 79,006 \\
[Total non-current liabilities]
 & 241,272 & 114,431 \\
\hline
Total liabilities & 134,047 & 193,437 \\
\hline
Total shareholders' equity & 375,319 & 128,249 \\
\hline
Total liabilities and shareholders' &  & 321,686 \\
\hline
equity &  &  \\
\hline
\end{tabular}
\end{center}

"Note: The italicized subtotals presented in this excerpt are not explicitly shown on the face of the financial statement as prepared by the company.

Source: Apple Inc. 2017 annual report (Form 10K).

SAP Group uses the title Statement of Financial Position and Apple uses the title Balance Sheet. Despite their different titles, both statements report the three basic elements: assets, liabilities, and equity. Both companies are reporting on a consolidated basis, i.e., including all their controlled subsidiaries. The numbers in SAP Group's balance sheet are in millions of euro, and the numbers in Apple's balance sheet are in millions of dollars.

Balance sheet information is as of a specific point in time. These exhibits are from the companies' annual financial statements, so the balance sheet information is as of the last day of their respective fiscal years. SAP Group's fiscal year is the same as the calendar year and the balance sheet information is as of 31 December. Apple's fiscal year ends on the last Saturday of September, so the actual date changes from year to year. About every six years, Apple's fiscal year will include 53 weeks rather than 52 weeks. This feature of Apple's fiscal year should be noted, but in general, the extra week is more relevant to evaluating statements spanning a period of time (the income and cash flow statements) rather than the balance sheet which captures information as of a specific point in time.

A company's ability to pay for its short term operating needs relates to the concept of liquidity. With respect to a company overall, liquidity refers to the availability of cash to meet those short-term needs. With respect to a particular asset or liability, liquidity refers to its "nearness to cash." A liquid asset is one that can be easily converted into cash in a short period of time at a price close to fair market value. For example, a small holding of an actively traded stock is much more liquid than an investment in an asset such as a commercial real estate property, particularly in a weak property market.

The separate presentation of current and non-current assets and liabilities facilitates analysis of a company's liquidity position (at least as of the end of the fiscal period). Both IFRS and US GAAP require that the balance sheet distinguish between current and non-current assets and between current and non-current liabilities and present these as separate classifications. An exception to this requirement, under IFRS, is that the current and non-current classifications are not required if a liquidity-based presentation provides reliable and more relevant information. Presentations distinguishing between current and non-current elements are shown in Exhibit 1 and Exhibit 2. Exhibit 3 in Section 2.3 shows a liquidity-based presentation.

\section*{CURRENT AND NON-CURRENT CLASSIFICATION }
Assets that are held primarily for the purpose of trading or that are expected to be sold, used up, or otherwise realized in cash within one year or one operating cycle of the business, whichever is greater, after the reporting period are classified as current assets. A company's operating cycle is the average amount of time that elapses between acquiring inventory and collecting the cash from sales to customers. (When the entity's normal operating cycle is not clearly identifiable, its duration is assumed to be one year.) For a manufacturer, the operating cycle is the average amount of time between acquiring raw materials and converting these into cash from a sale. Examples of companies that might be expected to have operating cycles longer than one year include those operating in the tobacco, distillery, and lumber industries. Even though these types of companies often hold inventories longer than one year, the inventory is classified as a current asset because it is expected to be sold within an operating cycle. Assets not expected to be sold or used up within one year or one operating cycle of the business, whichever is greater, are classified as non-current assets (long-term, long-lived assets).

Current assets are generally maintained for operating purposes, and these assets include-in addition to cash-items expected to be converted into cash (e.g., trade receivables), used up (e.g., office supplies, prepaid expenses), or sold (e.g., inventories) in the current operating cycle. Current assets provide information about the operating activities and the operating capability of the entity. For example, the item "trade receivables" or "accounts receivable" would indicate that a company provides credit to its customers. Non-current assets represent the infrastructure from which the entity operates and are not consumed or sold in the current period. Investments in such assets are made from a strategic and longer term perspective.

Similarly, liabilities expected to be settled within one year or within one operating cycle of the business, whichever is greater, after the reporting period are classified as current liabilities. The specific criteria for classification of a liability as current include the following:

\begin{itemize}
  \item It is expected to be settled in the entity's normal operating cycle;

  \item It is held primarily for the purpose of being traded; 4

  \item It is due to be settled within one year after the balance sheet date; or

  \item The entity does not have an unconditional right to defer settlement of the liability for at least one year after the balance sheet date. 5

\end{itemize}

\begin{enumerate}
  \setcounter{enumi}{3}
  \item Examples of these are financial liabilities classified as held for trading in accordance with IAS 39, which is replaced by IFRS 9 effective for periods beginning on or after 1 January 2018.
\end{enumerate}

5 IAS 1, Presentation of Financial Statements, paragraph 69. IFRS specify that some current liabilities, such as trade payables and some accruals for employee and other operating costs, are part of the working capital used in the entity's normal operating cycle. Such operating items are classified as current liabilities even if they will be settled more than one year after the balance sheet date. All other liabilities are classified as non-current liabilities. Non-current liabilities include financial liabilities that provide financing on a long-term basis.

The excess of current assets over current liabilities is called working capital. The level of working capital provides analysts with information about the ability of an entity to meet liabilities as they fall due. Although adequate working capital is essential, excessive working capital should be so that funds that could be used more productively elsewhere are not inappropriately tied up.

A balance sheet with separately classified current and non-current assets and liabilities is referred to as a classified balance sheet. Classification also refers generally to the grouping of accounts into subcategories. Both companies' balance sheets that are summarized in Exhibits 1 and 2 are classified balance sheets. Although both companies' balance sheets present current assets before non-current assets and current liabilities before non-current liabilities, this is not required. IFRS does not specify the order or format in which a company presents items on a current/non-current classified balance sheet.

A liquidity-based presentation, rather than a current/non-current presentation, is used when such a presentation provides information that is reliable and more relevant. With a liquidity-based presentation, all assets and liabilities are presented broadly in order of liquidity.

Entities such as banks are candidates to use a liquidity-based presentation. Exhibit 3 presents the assets portion of the balance sheet of HSBC Holdings plc (HSBC), a global financial services company that reports using IFRS. HSBC's balance sheet is ordered using a liquidity-based presentation. As shown, the asset section begins with cash and balances at central banks. Less liquid items such as "Interest in associates and joint ventures" appear near the bottom of the asset listing.

Exhibit 3: HSBC Holdings plc Consolidated Statement of Financial Position (Excerpt: Assets Only) as of 31 December (in millions of US \$)

\begin{center}
\begin{tabular}{|c|c|c|}
\hline
Consolidated balance sheet - USD (\$) \$ in Millions & $\begin{array}{c}\text { Dec. 31, } \\ 2017\end{array}$ & $\begin{array}{c}\text { Dec. 31, } \\ 2016\end{array}$ \\
\hline
\multicolumn{3}{|l|}{Assets} \\
\hline
Cash and balances at central banks & $\$ 180,624$ & $\$ 128,009$ \\
\hline
Items in the course of collection from other banks & 6,628 & 5,003 \\
\hline
Hong Kong Government certificates of indebtedness & 34,186 & 31,228 \\
\hline
Trading assets & 287,995 & 235,125 \\
\hline
Financial assets designated at fair value & 29,464 & 24,756 \\
\hline
Derivatives & 219,818 & 290,872 \\
\hline
\end{tabular}
\end{center}

\begin{center}
\begin{tabular}{lcc}
\hline
 & Dec. 31, & Dec. 31, \\
Consolidated balance sheet - USD (\$) \$ in Millions & $\mathbf{2 0 1 7}$ & $\mathbf{2 0 1 6}$ \\
\hline
Loans and advances to banks & 90,393 & 88,126 \\
Loans and advances to customers & 962,964 & 861,504 \\
Reverse repurchase agreements - non-trading & 201,553 & 160,974 \\
Financial investments & 389,076 & 436,797 \\
Prepayments, accrued income and other assets & 67,191 & 63,909 \\
Current tax assets & 1,006 & 1,145 \\
Interests in associates and joint ventures & 22,744 & 20,029 \\
Goodwill and intangible assets & 23,453 & 21,346 \\
Deferred tax assets & 4,676 & 6,163 \\
Total assets & $2,521,771$ & $2,374,986$ \\
\hline
\end{tabular}
\end{center}

Source: HSBC Holdings plc 2017 Annual Report and Accounts.

\section{CURRENT ASSETS: CASH AND CASH EQUIVALENTS, MARKETABLE SECURITIES AND TRADE RECEIVABLES}
$$
\mid \begin{aligned}
& \text { describe different types of assets and liabilities and the measurement } \\
& \text { bases of each }
\end{aligned}
$$

This section examines current assets and current liabilities in greater detail.

\section{Current Assets}
Accounting standards require that certain specific line items, if they are material, must be shown on a balance sheet. Among the current assets' required line items are cash and cash equivalents, trade and other receivables, inventories, and financial assets (with short maturities). Companies present other line items as needed, consistent with the requirements to separately present each material class of similar items. As examples, Exhibit 4 and Exhibit 5 present balance sheet excerpts for SAP Group and Apple Inc. showing the line items for the companies' current assets.

\section{Exhibit 4: SAP Group Consolidated Statements of Financial Position}
 (Excerpt: Current Assets Detail) (in millions of $€$ )As of 31 December

\begin{center}
\begin{tabular}{lcc}
\hline
Assets & $\mathbf{2 0 1 7}$ & $\mathbf{2 0 1 6}$ \\
\hline
Cash and cash equivalents & $€ 4,011$ & $€ 3,702$ \\
Other financial assets & 990 & 1,124 \\
Trade and other receivables & 5,899 & 5,924 \\
Other non-financial assets & 725 & 581 \\
Tax assets & 306 & 233 \\
\hline
\end{tabular}
\end{center}

\begin{center}
\begin{tabular}{lcc}
\hline
 & \multicolumn{2}{c}{As of 31 December} \\
\hline
Assets & $\mathbf{2 0 1 7}$ & $\mathbf{2 0 1 6}$ \\
\hline
Total current assets & 11,930 & 11,564 \\
Total non-current assets & 30,567 & 32,713 \\
\cline { 2 - 3 }
Total assets & $\mathbf{4 2 , 4 9 7}$ & $\mathbf{4 4 , 2 7 7}$ \\
Total current liabilities & 10,210 & 9,674 \\
Total non-current liabilities & 6,747 & 8,205 \\
Total liabilities & 16,958 & 17,880 \\
Total equity & 25,540 & 26,397 \\
Total equity and liabilities & $€ 42,497$ & $\boldsymbol{6 4 4 , 2 7 7}$ \\
\hline
\end{tabular}
\end{center}

Source: SAP Group 2017 annual report.

Exhibit 5: Apple Inc. Consolidated Balance Sheet (Excerpt: Current Assets

Detail) * (in millions of $\$$ )

\begin{center}
\begin{tabular}{lcc}
\hline
 & $\mathbf{3 0}$ & $\mathbf{2 4}$ \\
 & September, & September, \\
Assets & $\mathbf{2 0 1 7}$ & $\mathbf{2 0 1 6}$ \\
\hline
Cash and cash equivalents & $\$ 20,289$ & $\$ 20,484$ \\
Short-term marketable securities & 53,892 & 46,671 \\
Accounts receivable, less allowances of $\$ 58$ and $\$ 53$, & 17,874 & 15,754 \\
respectively &  &  \\
Inventories & 4,855 & 2,132 \\
Vendor non-trade receivables & 17,799 & 13,545 \\
Other current assets & 13,936 & 8,283 \\
Total current assets & 128,645 & 106,869 \\
[All other assets]
 & 246,674 & 214,817 \\
Total assets & $\mathbf{3 7 5 , 3 1 9}$ & $\mathbf{3 2 1 , 6 8 6}$ \\
Total current liabilities & 100,814 & 79,006 \\
[Total non-current liabilities]
 & 140,458 & 114,431 \\
Total liabilities & 241,272 & 193,437 \\
Total shareholders' equity & 134,047 & 128,249 \\
Total liabilities and shareholders' equity & $\mathbf{\$ 3 7 5 , 3 1 9}$ & $\mathbf{\$ 3 2 1 , 6 8 6}$ \\
\hline
\end{tabular}
\end{center}

*Note: The italicized subtotals presented in this excerpt are not explicitly shown on the face of the financial statement as prepared by the company.

Source: Apple Inc. 2017 annual report (Form 10K).

\section{Cash and Cash Equivalents}
Cash equivalents are highly liquid, short-term investments that are so close to maturity, ${ }^{6}$ the risk is minimal that their value will change significantly with changes in interest rates. Cash and cash equivalents are financial assets. Financial assets, in general, are measured and reported at either amortised cost or fair value. Amortised cost is the

6 Generally, three months or less. historical cost (initially recognised cost) of the asset adjusted for amortisation and impairment. Under IFRS and US GAAP, fair value is based on an exit price, the price received to sell an asset or paid to transfer a liability in an orderly transaction between two market participants at the measurement date.

For cash and cash equivalents, amortised cost and fair value are likely to be immaterially different. Examples of cash equivalents are demand deposits with banks and highly liquid investments (such as US Treasury bills, commercial paper, and money market funds) with original maturities of three months or less. Cash and cash equivalents excludes amounts that are restricted in use for at least 12 months. For all companies, the Statement of Cash Flows presents information about the changes in cash over a period. For the fiscal year 2017, SAP Group's cash and cash equivalents increased from $€ 3,702$ million to $€ 4,011$ million, and Apple's cash and cash equivalents decreased from $\$ 20,484$ million to $\$ 20,289$ million.

\section{Marketable Securities}
Marketable securities are also financial assets and include investments in debt or equity securities that are traded in a public market, and whose value can be determined from price information in a public market. Examples of marketable securities include treasury bills, notes, bonds, and equity securities, such as common stocks and mutual fund shares. Companies disclose further detail in the notes to their financial statements about their holdings. For example, SAP Group discloses that its other financial assets consist of items such as time deposits, other receivables, and loans to employees and third parties. These do not fall into marketable securities and thus are more properly treated under trade receivables. Apple's short-term marketable securities, totaling $\$ 53.9$ billion and $\$ 46.7$ billion at the end of fiscal 2017 and 2016, respectively, include holdings of US treasuries, corporate securities, commercial paper, and time deposits. Financial assets such as investments in debt and equity securities involve a variety of measurement issues and will be addressed in Section 4.5.

\section{Trade Receivables}
Trade receivables, also referred to as accounts receivable, are another type of financial asset. These are amounts owed to a company by its customers for products and services already delivered. They are typically reported at net realizable value, an approximation of fair value, based on estimates of collectability. Several aspects of accounts receivable are usually relevant to an analyst. First, the overall level of accounts receivable relative to sales (a topic to be addressed further in ratio analysis) is important because a significant increase in accounts receivable relative to sales could signal that the company is having problems collecting cash from its customers.

A second relevant aspect of accounts receivable is the allowance for doubtful accounts. The allowance for doubtful accounts reflects the company's estimate of the amount of receivables that will ultimately be uncollectible. Additions to the allowance in a particular period are reflected as bad debt expenses, and the balance of the allowance for doubtful accounts reduces the gross receivables amount to a net amount that is an estimate of net realizable value. When specific receivables are deemed to be uncollectible, they are written off by reducing accounts receivable and the allowance for doubtful accounts. The allowance for doubtful accounts is called a contra account because it is netted against (i.e., reduces) the balance of accounts receivable, which is an asset account. SAP Group's balance sheet, for example, reports current net trade and other receivables of $€ 5,899$ million as of 31 December 2017. The amount of the allowance for doubtful accounts ( $€ 74$ million) is disclosed in the notes ${ }^{7}$ to the financial statements. Apple discloses the allowance for doubtful accounts on the face of the

7 Note 13 SAP Group 2017 Annual report balance sheet; as of 30 September 2017, the allowance was $\$ 58$ million. The $\$ 17,874$ million of accounts receivable on that date is net of the allowance. Apple's disclosures state that the allowance is based on "historical experience, the age of the accounts receivable balances, credit quality of the Company's customers, current economic conditions, and other factors that may affect customers' abilities to pay." The age of an accounts receivable balance refers to the length of time the receivable has been outstanding, including how many days past the due date.

Another relevant aspect of accounts receivable is the concentration of credit risk. For example, SAP Group's annual report discloses that concentration of credit risk is limited because they have a large customer base diversified across various industries, company sizes, and countries. Similarly, Apple's annual report notes that no single customer accounted for 10 percent or more of its revenues. However, Apple's disclosures for 2017 indicate that two customers individually represented $10 \%$ or more of its total trade receivables and its cellular network carriers accounted for $59 \%$ of trade receivables. Of its vendor non-trade receivables, three vendors represent $42 \%, 19 \%$, and $10 \%$ of the total. ${ }^{8}$

\section{EXAMPLE 1}
\section{Analysis of Accounts Receivable}
\begin{enumerate}
  \item Based on the balance sheet excerpt for Apple Inc. in Exhibit 5 , what percentage of its total accounts receivable in 2017 and 2016 does Apple estimate will be uncollectible?
\end{enumerate}

\section{Solution to 1:}
(\$ millions) The percentage of 2017 accounts receivable estimated to be uncollectible is 0.32 percent, calculated as $\$ 58 /(\$ 17,874+\$ 58)$. Note that the $\$ 17,874$ is net of the $\$ 58$ allowance, so the gross amount of accounts receivable is determined by adding the allowance to the net amount. The percentage of 2016 accounts receivable estimated to be uncollectible is 0.34 percent $[\$ 53 /(\$ 15,754+\$ 53)]$.

\begin{enumerate}
  \setcounter{enumi}{1}
  \item In general, how does the amount of allowance for doubtful accounts relate to bad debt expense?
\end{enumerate}

\section{Solution to 2:}
Bad debt expense is an expense of the period, based on a company's estimate of the percentage of credit sales in the period, for which cash will ultimately not be collected. The allowance for bad debts is a contra asset account, which is netted against the asset accounts receivable.

To record the estimated bad debts, a company recognizes a bad debt expense (which affects net income) and increases the balance in the allowance for doubtful accounts by the same amount. To record the write off of a particular account receivable, a company reduces the balance in the allowance for doubtful accounts and reduces the balance in accounts receivable by the same amount. 3. In general, what are some factors that could cause a company's allowance for doubtful accounts to decrease?

\section{Solution to 3:}
In general, a decrease in a company's allowance for doubtful accounts in absolute terms could be caused by a decrease in the amount of credit sales.

Some factors that could cause a company's allowance for doubtful accounts to decrease as a percentage of accounts receivable include the following:

\begin{itemize}
  \item Improvements in the credit quality of the company's existing customers (whether driven by a customer-specific improvement or by an improvement in the overall economy);

  \item Stricter credit policies (for example, refusing to allow less creditworthy customers to make credit purchases and instead requiring them to pay cash, to provide collateral, or to provide some additional form of financial backing); and/or

  \item Stricter risk management policies (for example, buying more insurance against potential defaults).

\end{itemize}

In addition to the business factors noted above, because the allowance is based on management's estimates of collectability, management can potentially bias these estimates to manipulate reported earnings. For example, a management team aiming to increase reported income could intentionally over-estimate collectability and under-estimate the bad debt expense for a period. Conversely, in a period of good earnings, management could under-estimate collectability and over-estimate the bad debt expense with the intent of reversing the bias in a period of poorer earnings.

\section{CURRENT ASSETS: INVENTORIES AND OTHER CURRENT ASSETS}
describe different types of assets and liabilities and the measurement bases of each

Inventories are physical products that will eventually be sold to the company's customers, either in their current form (finished goods) or as inputs into a process to manufacture a final product (raw materials and work-in-process). Like any manufacturer, Apple holds inventories. The 2017 balance sheet of Apple Inc. shows $\$ 4,855$ million of inventories. SAP Group's balance sheet does not include a line item for inventory, consistent with the fact that SAP Group is primarily a software and services provider.

Inventories are measured at the lower of cost and net realizable value (NRV) under IFRS. The cost of inventories comprises all costs of purchase, costs of conversion, and other costs incurred in bringing the inventories to their present location and condition. NRV is the estimated selling price less the estimated costs of completion and costs necessary to complete the sale. NRV is applicable for all inventories under IFRS. Under US GAAP, inventories are also measured at the lower of cost and NRV unless they are measured using the last-in, first-out (LIFO) or retail inventory methods. When using LIFO or the retail inventory methods, inventories are measured at the lower of cost or market value. US GAAP defines market value as current replacement cost but with upper and lower limits; the recorded value cannot exceed NRV and cannot be lower than NRV less a normal profit margin.

If the net realizable value or market value (under US GAAP, in certain cases) of a company's inventory falls below its carrying amount, the company must write down the value of the inventory. The loss in value is reflected in the income statement. For example, within its Management's Discussion and Analysis and notes, Apple indicates that the company reviews its inventory each quarter and records write-downs of inventory that has become obsolete, exceeds anticipated demand, or is carried at a value higher than its market value. Under IFRS, if inventory that was written down in a previous period subsequently increases in value, the amount of the original write-down is reversed. Subsequent reversal of an inventory write-down is not permitted under US GAAP.

When inventory is sold, the cost of that inventory is reported as an expense, "cost of goods sold." Accounting standards allow different valuation methods for determining the amounts that are included in cost of goods sold on the income statement and thus the amounts that are reported in inventory on the balance sheet. (Inventory valuation methods are referred to as cost formulas and cost flow assumptions under IFRS and US GAAP, respectively.) IFRS allows only the first-in, first-out (FIFO), weighted average cost, and specific identification methods. Some accounting standards (such as US GAAP) also allow last-in, first-out (LIFO) as an additional inventory valuation method. The LIFO method is not allowed under IFRS.

\section{Other Current Assets}
The amounts shown in "other current assets" reflect items that are individually not material enough to require a separate line item on the balance sheet and so are aggregated into a single amount. Companies usually disclose the components of other assets in a note to the financial statements. A typical item included in other current assets is prepaid expenses. Prepaid expenses are normal operating expenses that have been paid in advance. Because expenses are recognized in the period in which they are incurred-and not necessarily the period in which the payment is made-the advance payment of a future expense creates an asset. The asset (prepaid expenses) will be recognized as an expense in future periods as it is used up. For example, consider prepaid insurance. Assume a company pays its insurance premium for coverage over the next calendar year on 31 December of the current year. At the time of the payment, the company recognizes an asset (prepaid insurance expense). The expense is not incurred at that date; the expense is incurred as time passes (in this example, one-twelfth, $1 / 12$, in each following month). Therefore, the expense is recognized and the value of the asset is reduced in the financial statements over the course of the year.

SAP's notes to the financial statements disclose components of the amount shown as other non-financial assets on the balance sheet. The largest portion pertains to prepaid expenses, primarily prepayments for operating leases, support services, and software royalties. Apple's notes do not disclose components of other current assets.

\section{CURRENT LIABILITIES}
describe different types of assets and liabilities and the measurement bases of each Current liabilities are those liabilities that are expected to be settled in the entity's normal operating cycle, held primarily for trading, or due to be settled within 12 months after the balance sheet date. Exhibit 6 and Exhibit 7 present balance sheet excerpts for SAP Group and Apple Inc. showing the line items for the companies' current liabilities. Some of the common types of current liabilities, including trade payables, financial liabilities, accrued expenses, and deferred income, are discussed below.

Exhibit 6: SAP Group Consolidated Statements of Financial Position (Excerpt: Current Liabilities Detail) (in millions of $€$ )

As of 31 December

2017

2016

Assets

Total current assets

Total non-current assets

Total assets

\begin{center}
\begin{tabular}{cl}
11,930 & 11,564 \\
30,567 & 32,713 \\
\hline
42,497 & 44,277 \\
\hline
\end{tabular}
\end{center}

Equity and liabilities

Trade and other payables

1,151

Tax liabilities

\begin{center}
\begin{tabular}{cc}
 &  \\
1,151 & 1,281 \\
597 & 316 \\
1,561 & 1,813 \\
3,946 & 3,699 \\
184 & 183 \\
2,771 & 2,383 \\
\hline
10,210 & 9,674 \\
6,747 & 8,205 \\
\hline
16,958 & 17,880 \\
25,540 & 26,397 \\
\hline
$€ 42,497$ & $€ 44,277$ \\
\hline
\end{tabular}
\end{center}

Financial liabilities

Other non-financial liabilities

Provisions

Deferred income

Total current liabilities

Total non-current liabilities

Total liabilities

Total equity

Total equity and liabilities

Source: SAP Group 2017 annual report.

Exhibit 7: Apple Inc. Consolidated Balance Sheet (Excerpt: Current

Liabilities Detail)* (in millions of \$)

\begin{center}
\begin{tabular}{|c|c|c|}
\hline
Assets & $\begin{array}{c}30 \text { September } \\ 2017\end{array}$ & $\begin{array}{c}24 \text { September } \\ 2016\end{array}$ \\
\hline
Total current assets & 128,645 & 106,869 \\
\hline
[All other assets] & 246,674 & 214,817 \\
\hline
Total assets & 375,319 & 321,686 \\
\hline
\multicolumn{3}{|l|}{Liabilities and shareholders' equity} \\
\hline
Accounts payable & 49,049 & 37,294 \\
\hline
Accrued expenses & 25,744 & 22,027 \\
\hline
Deferred revenue & 7,548 & 8,080 \\
\hline
Commercial paper & 11,977 & 8,105 \\
\hline
Current portion of long-term debt & 6,496 & 3,500 \\
\hline
\end{tabular}
\end{center}

\begin{center}
\begin{tabular}{|c|c|c|}
\hline
Assets & $\begin{array}{c}30 \text { September } \\ 2017\end{array}$ & $\begin{array}{c}24 \text { September } \\ 2016\end{array}$ \\
\hline
Total current liabilities & 100,814 & 79,006 \\
\hline
[Total non-current liabilities] & 140,458 & 114,431 \\
\hline
Total liabilities & 241,272 & 193,437 \\
\hline
Total shareholders' equity & 134,047 & 128,249 \\
\hline
Total liabilities and shareholders' equity & 375,319 & 321,686 \\
\hline
\end{tabular}
\end{center}

"Note: The italicized subtotals presented in this excerpt are not explicitly shown on the face of the financial statement as prepared by the company.

Source: Apple Inc. 2017 annual report (Form 10K).

Trade payables, also called accounts payable, are amounts that a company owes its vendors for purchases of goods and services. In other words, these represent the unpaid amount as of the balance sheet date of the company's purchases on credit. An issue relevant to analysts is the trend in overall levels of trade payables relative to purchases (a topic to be addressed further in ratio analysis). Significant changes in accounts payable relative to purchases could signal potential changes in the company's credit relationships with its suppliers. The general term "trade credit" refers to credit provided to a company by its vendors. Trade credit is a source of financing that allows the company to make purchases and then pay for those purchases at a later date.

Financial liabilities that are due within one year or the operating cycle, whichever is longer, appear in the current liability section of the balance sheet. Financial liabilities include borrowings such as bank loans, notes payable (which refers to financial liabilities owed by a company to creditors, including trade creditors and banks, through a formal loan agreement), and commercial paper. In addition, any portions of long-term liabilities that are due within one year (i.e., the current portion of long-term liabilities) are also shown in the current liability section of the balance sheet. According to its footnote disclosures, most of SAP's $€ 1,561$ million of current financial liabilities is for bonds payable due in the next year. Apple shows $\$ 11,977$ million of commercial paper borrowing (short-term promissory notes issued by companies) and $\$ 6,496$ million of long-term debt due within the next year.

Accrued expenses (also called accrued expenses payable, accrued liabilities, and other non-financial liabilities) are expenses that have been recognized on a company's income statement but not yet been paid as of the balance sheet date. For example, SAP's 2017 balance sheet shows $€ 597$ million of tax liabilities. In addition to income taxes payable, other common examples of accrued expenses are accrued interest payable, accrued warranty costs, and accrued employee compensation (i.e., wages payable). SAP's notes disclose that the $€ 3,946$ million line item of other non-financial liabilities in 2017, for example, includes $€ 2,565$ million of employee-related liabilities.

Deferred income (also called deferred revenue or unearned revenue) arises when a company receives payment in advance of delivery of the goods and services associated with the payment. The company has an obligation either to provide the goods or services or to return the cash received. Examples include lease payments received at the beginning of a lease, fees for servicing office equipment received at the beginning of the service period, and payments for magazine subscriptions received at the beginning of the subscription period. SAP's balance sheet shows deferred income of $€ 2,771$ million at the end of 2017, up slightly from $€ 2,383$ million at the end of 2016 . Apple's balance sheet shows deferred revenue of $\$ 7,548$ million at the end of fiscal 2017, down slightly from $\$ 8,080$ million at the end of fiscal 2016. Example 3 presents each company's disclosures about deferred revenue and discusses some of the implications.

\section{EXAMPLE 2}
\section{Analysis of Deferred Revenue}
In the notes to its 2017 financial statements, SAP describes its deferred income as follows:

Deferred income consists mainly of prepayments made by our customers for cloud subscriptions and support; software support and services; fees from multiple-element arrangements allocated to undelivered elements; and amounts ... for obligations to perform under acquired customer contracts in connection with acquisitions.

Apple's deferred revenue also arises from sales involving multiple elements, some delivered at the time of sale and others to be delivered in the future. In addition, Apple recognizes deferred revenue in connection with sales of gifts cards as well as service contracts. In the notes to its 2017 financial statements, Apple describes its deferred revenue as follows:

The Company records deferred revenue when it receives payments in advance of the delivery of products or the performance of services. This includes amounts that have been deferred for unspecified and specified software upgrade rights and non-software services that are attached to hardware and software products. The Company sells gift cards redeemable at its retail and online stores ... The Company records deferred revenue upon the sale of the card, which is relieved upon redemption of the card by the customer. Revenue from AppleCare service and support contracts is deferred and recognized over the service coverage periods. AppleCare service and support contracts typically include extended phone support, repair services, web-based support resources and diagnostic tools offered under the Company's standard limited warranty.

\begin{enumerate}
  \item In general, in the period a transaction occurs, how would a company's balance sheet reflect $\$ 100$ of deferred revenue resulting from a sale? (Assume, for simplicity, that the company receives cash for all sales, the company's income tax payable is 30 percent based on cash receipts, and the company pays cash for all relevant income tax obligations as they arise. Ignore any associated deferred costs.)
\end{enumerate}

\section{Solution to 1:}
In the period that deferred revenue arises, the company would record a $\$ 100$ increase in the asset Cash and a $\$ 100$ increase in the liability Deferred Revenues. In addition, because the company's income tax payable is based on cash receipts and is paid in the current period, the company would record a $\$ 30$ decrease in the asset Cash and a $\$ 30$ increase in the asset Deferred Tax Assets. Deferred tax assets increase because the company has paid taxes on revenue it has not yet recognized for accounting purposes. In effect, the company has prepaid taxes from an accounting perspective.

\begin{enumerate}
  \setcounter{enumi}{1}
  \item In general, how does deferred revenue impact a company's financial statements in the periods following its initial recognition?
\end{enumerate}

\section{Solution to 2:}
In subsequent periods, the company will recognize the deferred revenue as it is earned. When the revenue is recognized, the liability Deferred Revenue will decrease. In addition, the tax expense is recognized on the income statement as the revenue is recognized and thus the associated amounts of Deferred Tax Assets will decrease. 3. Interpret the amounts shown by SAP as deferred income and by Apple as deferred revenue.

Solution to 3:

The deferred income on SAP's balance sheet and deferred revenue on Apple's balance sheet at the end of their respective 2017 fiscal years will be recognized as revenue, sales, or a similar item in income statements subsequent to the 2017 fiscal year, as the goods or services are provided or the obligation is reduced. The costs of delivering the goods or services will also be recognised.

\begin{enumerate}
  \setcounter{enumi}{3}
  \item Both accounts payable and deferred revenue are classified as current liabilities. Discuss the following statements:
\end{enumerate}

A. When assessing a company's liquidity, the implication of amounts in accounts payable differs from the implication of amounts in deferred revenue.

B. Some investors monitor amounts in deferred revenue as an indicator of future revenue growth.

\section{Solution to $4 A$ :}
The amount of accounts payable represents a future obligation to pay cash to suppliers. In contrast, the amount of deferred revenue represents payments that the company has already received from its customers, and the future obligation is to deliver the related services. With respect to liquidity, settling accounts payable will require cash outflows whereas settling deferred revenue obligations will not.

\section{Solution to $4 B$ :}
Some investors monitor amounts in deferred revenue as an indicator of future growth because the amounts in deferred revenue will be recognized as revenue in the future. Thus, growth in the amount of deferred revenue implies future growth of that component of a company's revenue.

\begin{center}
\includegraphics[max width=\textwidth]{2023_05_04_b5cfa4f1bc883752f121g-092}
\end{center}

\section{NON-CURRENT ASSETS: PROPERTY, PLANT AND EQUIPMENT AND INVESTMENT PROPERTY}
describe different types of assets and liabilities and the measurement bases of each

This section provides an overview of assets other than current assets, sometimes collectively referred to as non-current, long-term, or long-lived assets. The categories discussed are property, plant, and equipment; investment property; intangible assets; goodwill; financial assets; and deferred tax assets. Exhibit 8 and Exhibit 9 present balance sheet excerpts for SAP Group and Apple Inc. showing the line items for the companies' non-current assets. Exhibit 8: SAP Group Consolidated Statements of Financial Position (Excerpt: Non-Current Assets Detail) (in millions of $€$ )

\begin{center}
\begin{tabular}{lcc}
\hline
 & \multicolumn{2}{c}{As of 31 December} \\
\hline
Assets & $\mathbf{2 0 1 7}$ & $\mathbf{2 0 1 6}$ \\
\hline
Total current assets & 11,930 & 11,564 \\
\cline { 2 - 3 }
Goodwill & 21,274 & 23,311 \\
Intangible assets & 2,967 & 3,786 \\
Property, plant and equipment & 2,967 & 2,580 \\
Other financial assets & 1,155 & 1,358 \\
Trade and other receivables & 118 & 126 \\
Other non-financial assets & 621 & 532 \\
Tax assets & 443 & 450 \\
Deferred tax assets & 1,022 & 571 \\
Total non-current assets & 30,567 & 32,713 \\
Total assets & $\mathbf{4 2 , 4 9 7}$ & $\mathbf{4 4 , 2 7 7}$ \\
Total current liabilities & 10,210 & 9,674 \\
Total non-current liabilities & 6,747 & 8,205 \\
Total liabilities & $\mathbf{1 6 , 9 5 8}$ & $\mathbf{1 7 , 8 8 0}$ \\
Total equity & $\mathbf{2 5 , 5 4 0}$ & $\mathbf{2 6 , 3 9 7}$ \\
Total equity and liabilities & $\mathbf{4 2 7 7}$ &  \\
\hline
\end{tabular}
\end{center}

Source: SAP Group 2017 annual report.

Exhibit 9: Apple Inc. Consolidated Balance Sheet (Excerpt: Non-Current

Assets Detail)* (in millions of \$)

\begin{center}
\begin{tabular}{lcc}
\hline
Assets & $\mathbf{3 0}$ September & $\begin{array}{c}\mathbf{2 4} \text { September } \\ \text { 2016 }\end{array}$ \\
\hline
Total current assets & $\mathbf{2 0 1 7}$ & 106,869 \\
Long-term marketable securities & 128,645 & 170,430 \\
Property, plant and equipment, net & 194,714 & 27,010 \\
Goodwill & 33,783 & 5,414 \\
Acquired intangible assets, net & 5,717 & 3,206 \\
Other non-current assets & 2,298 & 8,757 \\
\hline
[All other assets] & 10,162 & 214,817 \\
\cline { 2 - 3 }
Total assets & 246,674 & 321,686 \\
\hline
Liabilities and shareholders' equity & 375,319 &  \\
\hline
Total current liabilities & 100,814 & 79,006 \\
[Total non-current liabilities]
 & 140,458 & 114,431 \\
\hline
Total liabilities & 241,272 & 193,437 \\
\hline
Total shareholders' equity & 134,047 & 128,249 \\
\hline
Total liabilities and shareholders' equity & 375,319 & 321,686 \\
\hline
\end{tabular}
\end{center}

"Note: The italicized subtotals presented in this excerpt are not explicitly shown on the face of the financial statement as prepared by the company. Source: Apple Inc. 2017 annual report (Form 10K).

\section{Property, Plant, and Equipment}
Property, plant, and equipment (PPE) are tangible assets that are used in company operations and expected to be used (provide economic benefits) over more than one fiscal period. Examples of tangible assets treated as property, plant, and equipment, include land, buildings, equipment, machinery, furniture, and natural resources such as mineral and petroleum resources. IFRS permits companies to report PPE using either a cost model or a revaluation model. ${ }^{9}$ While IFRS permits companies to use the cost model for some classes of assets and the revaluation model for others, the company must apply the same model to all assets within a particular class of assets. US GAAP permits only the cost model for reporting PPE.

Under the cost model, PPE is carried at amortised cost (historical cost less any accumulated depreciation or accumulated depletion, and less any impairment losses). Historical cost generally consists of an asset's purchase price, plus its delivery cost, and any other additional costs incurred to make the asset operable (such as costs to install a machine). Depreciation and depletion refer to the process of allocating (recognizing as an expense) the cost of a long-lived asset over its useful life. Land is not depreciated. Because PPE is presented on the balance sheet net of depreciation and depreciation expense is recognised in the income statement, the choice of depreciation method and the related estimates of useful life and salvage value impact both a company's balance sheet and income statement.

Whereas depreciation is the systematic allocation of cost over an asset's useful life, impairment losses reflect an unanticipated decline in value. Impairment occurs when the asset's recoverable amount is less than its carrying amount, with terms defined as follows under IFRS: $:^{10}$

\begin{itemize}
  \item Recoverable amount: The higher of an asset's fair value less cost to sell, and its value in use.

  \item Fair value less cost to sell: The amount obtainable in a sale of the asset in an arms-length transaction between knowledgeable willing parties, less the costs of the sale.

  \item Value in use: The present value of the future cash flows expected to be derived from the asset.

\end{itemize}

When an asset is considered impaired, the company recognizes the impairment loss in the income statement in the period the impairment is identified. Reversals of impairment losses are permitted under IFRS but not under US GAAP.

Under the revaluation model, the reported and carrying value for PPE is the fair value at the date of revaluation less any subsequent accumulated depreciation. Changes in the value of PPE under the revaluation model affect equity directly or profit and loss depending upon the circumstances.

In Exhibit 8 and Exhibit 9, SAP reports $€ 2,967$ million of PPE and Apple reports $\$ 33,783$ million of PPE at the end of fiscal year 2017. For SAP, PPE represents approximately 7 percent of total assets and for Apple, PPE represents approximately 9 percent of total assets. Both companies disclose in the notes that PPE are generally depreciated over their expected useful lives using the straight-line method.

9 IAS 16, Property, Plant and Equipment, paragraphs 29-31.

10 IAS 36, Impairment of Assets, paragraph 6. US GAAP uses a different approach to impairment.

\section{Investment Property}
Some property is not used in the production of goods or services or for administrative purposes. Instead, it is used to earn rental income or capital appreciation (or both). Under IFRS, such property is considered to be investment property. ${ }^{11}$ US GAAP does not include a specific definition for investment property. IFRS provides companies with the choice to report investment property using either a cost model or a fair value model. In general, a company must apply its chosen model (cost or fair value) to all of its investment property. The cost model for investment property is identical to the cost model for PPE: In other words, investment property is carried at cost less any accumulated depreciation and any accumulated impairment losses. Under the fair value model, investment property is carried at its fair value. When a company uses the fair value model to measure the value of its investment property, any gain or loss arising from a change in the fair value of the investment property is recognized in profit and loss, i.e., on the income statement, in the period in which it arises. ${ }^{12}$

Neither SAP Group nor Apple disclose ownership of investment property. The types of companies that typically hold investment property are real estate investment companies or property management companies. Entities such as life insurance companies and endowment funds may also hold investment properties as part of their investment portfolio.

\section{NON-CURRENT ASSETS: INTANGIBLE ASSETS}
$$
\mid \begin{aligned}
& \text { describe different types of assets and liabilities and the measurement } \\
& \text { bases of each }
\end{aligned}
$$

Intangible assets are identifiable non-monetary assets without physical substance. ${ }^{13}$ An identifiable asset can be acquired singly (can be separated from the entity) or is the result of specific contractual or legal rights or privileges. Examples include patents, licenses, and trademarks. The most common asset that is not a separately identifiable asset is accounting goodwill, which arises in business combinations and is discussed further in Section 4.4.

IFRS allows companies to report intangible assets using either a cost model or a revaluation model. The revaluation model can only be selected when there is an active market for an intangible asset. These measurement models are essentially the same as described for PPE. US GAAP permits only the cost model.

For each intangible asset, a company assesses whether the useful life of the asset is finite or indefinite. Amortisation and impairment principles apply as follows:

\begin{itemize}
  \item An intangible asset with a finite useful life is amortised on a systematic basis over the best estimate of its useful life, with the amortisation method and useful life estimate reviewed at least annually.

  \item Impairment principles for an intangible asset with a finite useful life are the same as for PPE.

  \item An intangible asset with an indefinite useful life is not amortised. Instead, at least annually, the reasonableness of assuming an indefinite useful life for the asset is reviewed and the asset is tested for impairment.

\end{itemize}

11 IAS 40, Investment Property.

12 IAS 40, Investment Property, paragraph 35.

13 IAS 38, Intangible Assets, paragraph 8. Financial analysts have traditionally viewed the values assigned to intangible assets, particularly goodwill, with caution. Consequently, in assessing financial statements, analysts often exclude the book value assigned to intangibles, reducing net equity by an equal amount and increasing pretax income by any amortisation expense or impairment associated with the intangibles. An arbitrary assignment of zero value to intangibles is not advisable; instead, an analyst should examine each listed intangible and assess whether an adjustment should be made. Note disclosures about intangible assets may provide useful information to the analyst. These disclosures include information about useful lives, amortisation rates and methods, and impairment losses recognised or reversed.

Further, a company may have developed intangible assets internally that can only be recognised in certain circumstances. Companies may also have assets that are never recorded on a balance sheet because they have no physical substance and are non-identifiable. These assets might include management skill, name recognition, a good reputation, and so forth. Such assets are valuable and are, in theory, reflected in the price at which the company's equity securities trade in the market (and the price at which the entirety of the company's equity would be sold in an acquisition transaction). Such assets may be recognised as goodwill if a company is acquired, but are not recognised until an acquisition occurs.

\section{Identifiable Intangibles}
Under IFRS, identifiable intangible assets are recognised on the balance sheet if it is probable that future economic benefits will flow to the company and the cost of the asset can be measured reliably. Examples of identifiable intangible assets include patents, trademarks, copyrights, franchises, licenses, and other rights. Identifiable intangible assets may have been created internally or purchased by a company. Determining the cost of internally created intangible assets can be difficult and subjective. For these reasons, under IFRS and US GAAP, the general requirement is that internally created identifiable intangibles are expensed rather than reported on the balance sheet.

IFRS provides that for internally created intangible assets, the company must separately identify the research phase and the development phase. ${ }^{14}$ The research phase includes activities that seek new knowledge or products. The development phase occurs after the research phase and includes design or testing of prototypes and models. IFRS require that costs to internally generate intangible assets during the research phase must be expensed on the income statement. Costs incurred in the development stage can be capitalized as intangible assets if certain criteria are met, including technological feasibility, the ability to use or sell the resulting asset, and the ability to complete the project.

US GAAP prohibits the capitalization as an asset of most costs of internally developed intangibles and research and development. All such costs usually must be expensed. Costs related to the following categories are typically expensed under IFRS and US GAAP. They include:

\begin{itemize}
  \item internally generated brands, mastheads, publishing titles, customer lists, etc.;

  \item start-up costs;

  \item training costs;

  \item administrative and other general overhead costs;

  \item advertising and promotion;

  \item relocation and reorganization expenses; and - redundancy and other termination costs.

\end{itemize}

Generally, acquired intangible assets are reported as separately identifiable intangibles (as opposed to goodwill) if they arise from contractual rights (such as a licensing agreement), other legal rights (such as patents), or have the ability to be separated and sold (such as a customer list).

\section{EXAMPLE 3}
\section{Measuring Intangible Assets}
\begin{enumerate}
  \item Alpha Inc., a motor vehicle manufacturer, has a research division that worked on the following projects during the year:
\end{enumerate}

Project $1 \quad$ Research aimed at finding a steering mechanism that does not operate like a conventional steering wheel but reacts to the impulses from a driver's fingers.

Project 2 The design of a prototype welding apparatus that is controlled electronically rather than mechanically. The apparatus has been determined to be technologically feasible, salable, and feasible to produce.

The following is a summary of the expenses of the research division (in thousands of $€$ ):

\begin{center}
\begin{tabular}{lccc}
\hline
 & General & Project 1 & Project 2 \\
\hline
Material and services & 128 & 935 & 620 \\
Labor &  &  &  \\
- Direct labor & - & 630 & - \\
- Administrative personnel & 720 & - & 470 \\
$\begin{array}{l}\text { Design, construction, and } \\ \text { testing }\end{array}$ & 270 & 450 &  \\
\hline
\end{tabular}
\end{center}

Five percent of administrative personnel costs can be attributed to each of Projects 1 and 2. Explain the accounting treatment of Alpha's costs for Projects 1 and 2 under IFRS and US GAAP.

\section{Solution:}
Under IFRS, the capitalization of development costs for Projects 1 and 2 would be as follows:

\begin{center}
\begin{tabular}{lll}
\hline
 & $\begin{array}{c}\text { Amount Capitalized as } \\ \left.\text { an Asset ( } \epsilon^{\prime} \mathbf{0 0 0}\right)\end{array}$ &  \\
\hline
Project 1: & $\begin{array}{l}\text { Classified as in the research stage, so } \\ \text { all costs are recognized as expenses }\end{array}$ & NIL \\
Project 2: & $\begin{array}{l}\text { Classified as in the development } \\ \text { stage, so costs may be capitalized. } \\ \text { Note that administrative costs are not } \\ \text { capitalized. }\end{array}$ &  \\
\hline
\end{tabular}
\end{center}

Under US GAAP, the costs of Projects 1 and 2 are expensed.

As presented in Exhibits 8 and 9, SAP's 2017 balance sheet shows $€ 2,967$ million of intangible assets, and Apple's 2017 balance sheet shows acquired intangible assets, net of $\$ 2,298$ million. SAP's notes disclose the types of intangible assets (software and database licenses, purchased software to be incorporated into its products, customer contracts, and acquired trademark licenses) and notes that all of its purchased intangible assets other than goodwill have finite useful lives and are amortised either based on expected consumption of economic benefits or on a straight-line basis over their estimated useful lives which range from two to 20 years. Apple's notes disclose that its acquired intangible assets consist primarily of patents and licenses, and almost the entire amount represents definite-lived and amortisable assets for which the remaining weighted-average amortisation period is 3.4 years as of 2017.

\begin{center}
\includegraphics[max width=\textwidth]{2023_05_04_b5cfa4f1bc883752f121g-098}
\end{center}

\section{NON-CURRENT ASSETS: GOODWILL}
describe different types of assets and liabilities and the measurement bases of each

When one company acquires another, the purchase price is allocated to all the identifiable assets (tangible and intangible) and liabilities acquired, based on fair value. If the purchase price is greater than the acquirer's interest in the fair value of the identifiable assets and liabilities acquired, the excess amount is recognized as an asset, described as goodwill. To understand why an acquirer would pay more to purchase a company than the fair value of the target company's identifiable assets net of liabilities, consider the following three observations. First, as noted, certain items not recognized in a company's own financial statements (e.g., its reputation, established distribution system, trained employees) have value. Second, a target company's expenditures in research and development may not have resulted in a separately identifiable asset that meets the criteria for recognition but nonetheless may have created some value. Third, part of the value of an acquisition may arise from strategic positioning versus a competitor or from perceived synergies. The purchase price might not pertain solely to the separately identifiable assets and liabilities acquired and thus may exceed the value of those net assets due to the acquisition's role in protecting the value of all of the acquirer's existing assets or to cost savings and benefits from combining the companies.

The subject of recognizing goodwill in financial statements has found both proponents and opponents among professionals. The proponents of goodwill recognition assert that goodwill is the present value of excess returns that a company is expected to earn. This group claims that determining the present value of these excess returns is analogous to determining the present value of future cash flows associated with other assets and projects. Opponents of goodwill recognition claim that the prices paid for acquisitions often turn out to be based on unrealistic expectations, thereby leading to future write-offs of goodwill.

Analysts should distinguish between accounting goodwill and economic goodwill. Economic goodwill is based on the economic performance of the entity, whereas accounting goodwill is based on accounting standards and is reported only in the case of acquisitions. Economic goodwill is important to analysts and investors, and it is not necessarily reflected on the balance sheet. Instead, economic goodwill is reflected in the stock price (at least in theory). Some financial statement users believe that goodwill should not be listed on the balance sheet, because it cannot be sold separately from the entity. These financial statement users believe that only assets that can be separately identified and sold should be reflected on the balance sheet. Other financial statement users analyze goodwill and any subsequent impairment charges to assess management's performance on prior acquisitions.

Under both IFRS and US GAAP, accounting goodwill arising from acquisitions is capitalized. Goodwill is not amortised but is tested for impairment annually. If goodwill is deemed to be impaired, an impairment loss is charged against income in the current period. An impairment loss reduces current earnings. An impairment loss also reduces total assets, so some performance measures, such as return on assets (net income divided by average total assets), may actually increase in future periods. An impairment loss is a non-cash item.

Accounting standards' requirements for recognizing goodwill can be summarized by the following steps:

A. The total cost to purchase the target company (the acquiree) is determined.

B. The acquiree's identifiable assets are measured at fair value. The acquiree's liabilities and contingent liabilities are measured at fair value. The difference between the fair value of identifiable assets and the fair value of the liabilities and contingent liabilities equals the net identifiable assets acquired.

C. Goodwill arising from the purchase is the excess of a) the cost to purchase the target company over b) the net identifiable assets acquired. Occasionally, a transaction will involve the purchase of net identifiable assets with a value greater than the cost to purchase. Such a transaction is called a "bargain purchase." Any gain from a bargain purchase is recognized in profit and loss in the period in which it arises. ${ }^{15}$

Companies are also required to disclose information that enables users to evaluate the nature and financial effect of business combinations. The required disclosures include, for example, the acquisition date fair value of the total cost to purchase the target company, the acquisition date amount recognized for each major class of assets and liabilities, and a qualitative description of the factors that make up the goodwill recognized.

Despite the guidance incorporated in accounting standards, analysts should be aware that the estimations of fair value involve considerable management judgment. Values for intangible assets, such as computer software, might not be easily validated when analyzing acquisitions. Management judgment about valuation in turn impacts current and future financial statements because identifiable intangible assets with definite lives are amortised over time. In contrast, neither goodwill nor identifiable intangible assets with indefinite lives are amortised; instead, as noted, both are tested annually for impairment.

The recognition and impairment of goodwill can significantly affect the comparability of financial statements between companies. Therefore, analysts often adjust the companies' financial statements by removing the impact of goodwill. Such adjustments include:

\begin{itemize}
  \item excluding goodwill from balance sheet data used to compute financial ratios, and

  \item excluding goodwill impairment losses from income data used to examine operating trends. In addition, analysts can develop expectations about a company's performance following an acquisition by taking into account the purchase price paid relative to the net assets and earnings prospects of the acquired company. Example 4 provides an historical example of goodwill impairment.

\end{itemize}

\section{EXAMPLE 4}
\section{Goodwill Impairment}
Safeway, Inc., is a North American food and drug retailer. On 25 February 2010, Safeway issued a press release that included the following information:

Safeway Inc. today reported a net loss of $\$ 1,609.1$ million ( $\$ 4.06$ per diluted share) for the 16-week fourth quarter of 2009. Excluding a non-cash goodwill impairment charge of $\$ 1,818.2$ million, net of tax (\$4.59 per diluted share), net income would have been $\$ 209.1$ million (\$0.53 per diluted share). Net income was $\$ 338.0$ million (\$0.79 per diluted share) for the 17-week fourth quarter of 2008.

In the fourth quarter of 2009, Safeway recorded a non-cash goodwill impairment charge of $\$ 1,974.2$ million ( $\$ 1,818.2$ million, net of tax). The impairment was due primarily to Safeway's reduced market capitalization and a weak economy....The goodwill originated from previous acquisitions.

Safeway's balance sheet as of 2 January 2010 showed goodwill of $\$ 426.6$ million and total assets of $\$ 14,963.6$ million. The company's balance sheet as of 3 January 2009 showed goodwill of $\$ 2,390.2$ million and total assets of $\$ 17,484.7$ million.

\begin{enumerate}
  \item How significant is this goodwill impairment charge?
\end{enumerate}

\section{Solution to 1:}
The goodwill impairment was more than 80 percent of the total value of goodwill and 11 percent of total assets, so it was clearly significant. (The charge of $\$ 1,974.2$ million equals 82.6 percent of the $\$ 2,390.2$ million of goodwill at the beginning of the year and 11.3 percent of the $\$ 17,484.7$ million total assets at the beginning of the year.)

\begin{enumerate}
  \setcounter{enumi}{1}
  \item With reference to acquisition prices, what might this goodwill impairment indicate?
\end{enumerate}

\section{Solution to 2:}
The goodwill had originated from previous acquisitions. The impairment charge implies that the acquired operations are now worth less than the price that was paid for their acquisition.

As presented in Exhibits 8 and 9, SAP's 2017 balance sheet shows $€ 21,274$ million of goodwill, and Apple's 2017 balance sheet shows goodwill of $\$ 5,717$ million. Goodwill represents 50.1 percent of SAP's total assets and only 1.5 percent of Apple's total assets. An analyst may be concerned that goodwill represents such a high proportion of SAP's total assets.

\section{NON-CURRENT ASSETS: FINANCIAL ASSETS}
describe different types of assets and liabilities and the measurement
bases of each

IFRS define a financial instrument as a contract that gives rise to a financial asset of one entity, and a financial liability or equity instrument of another entity. ${ }^{16}$ This section will focus on financial assets such as a company's investments in stocks issued by another company or its investments in the notes, bonds, or other fixed-income instruments issued by another company (or issued by a governmental entity). Financial liabilities such as notes payable and bonds payable issued by the company itself will be discussed in the liability portion of this reading. Some financial instruments may be classified as either an asset or a liability depending on the contractual terms and current market conditions. One example of such a financial instrument is a derivative. Derivatives are financial instruments for which the value is derived based on some underlying factor (interest rate, exchange rate, commodity price, security price, or credit rating) and for which little or no initial investment is required.

Financial instruments are generally recognized when the entity becomes a party to the contractual provisions of the instrument. In general, there are two basic alternative ways that financial instruments are measured subsequent to initial acquisition: fair value or amortised cost. Recall that fair value is the price that would be received to sell an asset or paid to transfer a liability in an orderly market transaction. ${ }^{17}$ The amortised cost of a financial asset (or liability) is the amount at which it was initially recognized, minus any principal repayments, plus or minus any amortisation of discount or premium, and minus any reduction for impairment.

Under IFRS, financial assets are subsequently measured at amortised cost if the asset's cash flows occur on specified dates and consist solely of principal and interest, and if the business model is to hold the asset to maturity. The concept is similar in US GAAP, where this category of asset is referred to as held-to-maturity. An example is an investment in a long-term bond issued by another company or by a government; the value of the bond will fluctuate, for example with interest rate movements, but if the bond is classified as a held-to-maturity investment, it will be measured at amortised cost on the balance sheet of the investing company. Other types of financial assets measured at amortised cost are loans to other companies.

Financial assets not measured at amortised cost subsequent to acquisition are measured at fair value as of the reporting date. For financial instruments measured at fair value, there are two basic alternatives in how net changes in fair value are recognized: as profit or loss on the income statement, or as other comprehensive income (loss) which bypasses the income statement. Note that these alternatives refer to unrealized changes in fair value, i.e., changes in the value of a financial asset that has not been sold and is still owned at the end of the period. Unrealized gains and losses are also referred to as holding period gains and losses. If a financial asset is sold within the period, a gain is realized if the selling price is greater than the carrying value and a loss is realized if the selling price is less than the carrying value. When a financial asset is sold, any realized gain or loss is reported on the income statement.

Under IFRS, financial assets are subsequently measured at fair value through other comprehensive income (i.e., any unrealized holding gains or losses are recognized in other comprehensive income) if the business model's objective involves both collecting contractual cash flows and selling the financial assets. This IFRS category applies

16 IAS 32, Financial Instruments: Presentation, paragraph 11.

17 IFRS 13 Fair Value Measurement and US GAAP ASC 820 Fair Value Measurement. specifically to debt investments, namely assets with cash flows occurring on specified dates and consisting solely of principal and interest. However, IFRS also permits equity investments to be measured at fair value through other comprehensive income if, at the time a company buys an equity investment, the company decides to make an irrevocable election to measure the asset in this manner. ${ }^{18}$ The concept is similar to the US GAAP investment category available-for-sale in which assets are measured at fair value, with any unrealized holding gains or losses recognized in other comprehensive income. However, unlike IFRS, the US GAAP category available-for-sale applies only to debt securities and is not permitted for investments in equity securities. ${ }^{19}$

Under IFRS, financial assets are subsequently measured at fair value through profit or loss (i.e., any unrealized holding gains or losses are recognized in the income statement) if they are not assigned to either of the other two measurement categories described above. In addition, IFRS allows a company to make an irrevocable election at acquisition to measure a financial asset in this category. Under US GAAP, all investments in equity securities (other than investments giving rise to ownership positions that confer significant influence over the investee) are measured at fair value with unrealized holding gains or losses recognized in the income statement. Under US GAAP, debt securities designated as trading securities are also measured at fair value with unrealized holding gains or losses recognized in the income statement. The trading securities category pertains to a debt security that is acquired with the intent of selling it rather than holding it to collect the interest and principal payments.

Exhibit 10 summarizes how various financial assets are classified and measured subsequent to acquisition.

\section{Exhibit 10: Measurement of Financial Assets}
\begin{center}
\begin{tabular}{lll}
\hline
 & Measured at Fair &  \\
$\begin{array}{l}\text { Measured at Cost or } \\ \text { Amortised Cost }\end{array}$ & $\begin{array}{l}\text { Value through Other } \\ \text { Comprehensive Income }\end{array}$ & $\begin{array}{l}\text { Measured at Fair Value } \\ \text { through Profit and Loss }\end{array}$ \\
\hline
\end{tabular}
\end{center}

\begin{itemize}
  \item Debt securities that are to be held to maturity.

  \item Loans and notes receivable

  \item Unquoted equity instruments (in limited circumstances where the fair value is not reliably measurable, cost may serve as a proxy (estimate) for fair value) - "Available-for-sale" debt securities (US GAAP); Debt securities where the business model involves both collecting interest and principal and selling the security (IFRS);

  \item Equity investments for which the company irrevocably elects this measurement at acquisition (IFRS only) - All equity securities unless the investment gives the investor significant influence (US GAAP only)

  \item "Trading" debt securities (US GAAP)

  \item Securities not assigned to either of the other two categories, or investments for which the company irrevocably elects this measurement ac acquisition (IFRS only)

\end{itemize}

To illustrate the different accounting treatments of the gains and losses on financial assets, consider an entity that invests $€ 100,000,000$ on 1 January $200 \mathrm{X}$ in a fixed-income security investment, with a 5 percent coupon paid semi-annually. After six months, the company receives the first coupon payment of $€ 2,500,000$. Additionally, market interest rates have declined such that the value of the fixed-income investment has increased by $€ 2,000,000$ as of 30 June 200X. Exhibit 11 illustrates how this situation will

18 IFRS 7 Financial Instruments: Disclosures, paragraph 8(h) and IFRS 9 Financial Instruments, paragraph 5.7.5.

19 US GAAP ASU 2016-01 and ASC 32X Investments. be portrayed in the balance sheet and income statement (ignoring taxes) of the entity concerned, under each of the following three measurement categories of financial assets: assets held for trading purposes, assets available for sale, and held-to-maturity assets.

\section{Exhibit 11: Accounting for Gains and Losses on Marketable Securities}
\begin{center}
\begin{tabular}{|c|c|c|c|}
\hline
IFRS Categories & $\begin{array}{c}\text { Measured at Cost or } \\ \text { Amortised Cost }\end{array}$ & $\begin{array}{c}\text { Measured at Fair } \\ \text { Value through Other } \\ \text { Comprehensive Income }\end{array}$ & $\begin{array}{c}\text { Measured at Fair } \\ \text { Value through Profit } \\ \text { and Loss }\end{array}$ \\
\hline
US GAAP Comparable Categories & Held to Maturity & $\begin{array}{c}\text { Available-for-Sale Debt } \\ \text { Securities }\end{array}$ & $\begin{array}{l}\text { Trading Debt } \\ \text { Securities }\end{array}$ \\
\hline
\end{tabular}
\end{center}

Income Statement For period 1 January-30 June

$200 x$

\begin{center}
\begin{tabular}{lccc}
\hline
Interest income & $2,500,000$ & $2,500,000$ & $2,500,000$ \\
Unrealized gains & - & - & $2,000,000$ \\
\cline { 2 - 3 }
Impact on profit and loss & $2,500,000$ & $2,500,000$ &  \\
\hline
\end{tabular}
\end{center}

Balance Sheet As of 30 June 200X

Assets

Cash and cash equivalents

\begin{center}
\begin{tabular}{ccc}
$2,500,000$ & $2,500,000$ & $2,500,000$ \\
$100,000,000$ & $100,000,000$ & $100,000,000$ \\
- & $2,000,000$ & $2,000,000$ \\
\hline
$102,500,000$ & $104,500,000$ & $104,500,000$ \\
\hline
\end{tabular}
\end{center}

Liabilities

\section{Equity}
Paid-in capital

Retained earnings

Accumulated other comprehensive income

\begin{center}
\begin{tabular}{ccc}
$100,000,000$ & $100,000,000$ & $100,000,000$ \\
$2,500,000$ & $2,500,000$ & $4,500,000$ \\
- & $2,000,000$ & - \\
\hline
$102,500,000$ & $104,500,000$ & $104,500,000$ \\
\hline
\end{tabular}
\end{center}

In the case of held-to-maturity securities, the income statement shows only the interest income (which is then reflected in retained earnings of the ending balance sheet). Because the securities are measured at cost rather than fair value, no unrealized gain is recognized. On the balance sheet, the investment asset is shown at its amortised cost of $€ 100,000,000$. In the case of securities classified as Measured at Fair Value through Other Comprehensive Income (IFRS) or equivalently as Available-for-sale debt securities (US GAAP), the income statement shows only the interest income (which is then reflected in retained earnings of the balance sheet). The unrealized gain does not appear on the income statement; instead, it would appear on a Statement of Comprehensive Income as Other Comprehensive Income. On the balance sheet, the investment asset is shown at its fair value of $€ 102,000,000$. (Exhibit 11 shows the unrealized gain on a separate line solely to highlight the impact of the change in value. In practice, the investments would be shown at their fair value on a single line.) In the case of securities classified as Measured at Fair Value through Profit and Loss (IFRS) or equivalently as trading debt securities (US GAAP), both the interest income and the unrealized gain are included on the income statement and thus reflected in retained earnings on the balance sheet.

In Exhibits 4 and 8, SAP's 2017 balance sheet shows other financial assets of $€ 990$ million (current) and $€ 1,155$ million (non-current). The company's notes disclose that the largest component of the current financial assets are loans and other financial receivables ( $€ 793$ million) and the largest component of the non-current financial assets is $€ 827$ million of available-for-sale equity investments.

In Exhibits 5 and 9, Apple's 2017 balance sheet shows $\$ 53,892$ million of short-term marketable securities and $\$ 194,714$ million of long-term marketable securities. In total, marketable securities represent more than 66 percent of Apple's $\$ 375.3$ billion in total assets. Marketable securities plus cash and cash equivalents represent around 72 percent of the company's total assets. Apple's notes disclose that most of the company's marketable securities are fixed-income securities issued by the US government or its agencies (\$60,237 million) and by other companies including commercial paper (\$153,451 million). In accordance with its investment policy, Apple invests in highly rated securities (which the company defines as investment grade) and limits its credit exposure to any one issuer. The company classifies its marketable securities as available for sale and reports them on the balance sheet at fair value. Unrealized gains and losses are reported in other comprehensive income.

\section{1}
\section{NON-CURRENT ASSETS: DEFERRED TAX ASSETS}
describe different types of assets and liabilities and the measurement bases of each

Portions of the amounts shown as deferred tax assets on SAP's balance sheet represent income taxes incurred prior to the time that the income tax expense will be recognized on the income statement. Deferred tax assets may result when the actual income tax payable based on income for tax purposes in a period exceeds the amount of income tax expense based on the reported financial statement income due to temporary timing differences. For example, a company may be required to report certain income for tax purposes in the current period but to defer recognition of that income for financial statement purposes to subsequent periods. In this case, the company will pay income tax as required by tax laws, and the difference between the taxes payable and the tax expense related to the income for which recognition was deferred on the financial statements will be reported as a deferred tax asset. When the income is subsequently recognized on the income statement, the related tax expense is also recognized which will reduce the deferred tax asset.

Also, a company may claim certain expenses for financial statement purposes that it is only allowed to claim in subsequent periods for tax purposes. In this case, as in the previous example, the financial statement income before taxes is less than taxable income. Thus, income taxes payable based on taxable income exceeds income tax expense based on accounting net income before taxes. The difference is expected to reverse in the future when the income reported on the financial statements exceeds the taxable income as a deduction for the expense becomes allowed for tax purposes. Deferred tax assets may also result from carrying forward unused tax losses and credits (these are not temporary timing differences). Deferred tax assets are only to be recognized if there is an expectation that there will be taxable income in the future, against which the temporary difference or carried forward tax losses or credits can be applied to reduce taxes payable.

\section{NON-CURRENT LIABILITIES}
describe different types of assets and liabilities and the measurement bases of each

All liabilities that are not classified as current are considered to be non-current or long-term. Exhibit 12 and Exhibit 13 present balance sheet excerpts for SAP Group and Apple Inc. showing the line items for the companies' non-current liabilities.

Both companies' balance sheets show non-current unearned revenue (deferred income for SAP Group and deferred revenue for Apple). These amounts represent the amounts of unearned revenue relating to goods and services expected to be delivered in periods beyond twelve months following the reporting period. The sections that follow focus on two common types of non-current (long-term) liabilities: long-term financial liabilities and deferred tax liabilities.

\section{Exhibit 12: SAP Group Consolidated Statements of Financial Position (Excerpt: Non-Current Liabilities Detail) (in millions of $€$ )}
as of 31 December

\begin{center}
\begin{tabular}{lcc}
\hline
 & $\mathbf{2 0 1 7}$ & $\mathbf{2 0 1 6}$ \\
\hline
Assets &  &  \\
Total current assets & 11,930 & 11,564 \\
Total non-current assets & 30,567 & 32,713 \\
\cline { 2 - 3 }
Total assets & 42,497 & 44,277 \\
Total current liabilities & 10,210 & 9,674 \\
\cline { 2 - 3 }
Trade and other payables & 119 & 127 \\
Tax liabilities & 470 & 365 \\
Financial liabilities & 5,034 & 6,481 \\
Other non-financial liabilities & 503 & 461 \\
Provisions & 303 & 217 \\
Deferred tax liabilities & 240 & 411 \\
Deferred income & 79 & 143 \\
Total non-current liabilities & 6,747 & 8,205 \\
Total liabilities & 16,958 & 17,880 \\
Total equity & 25,540 & 26,397 \\
Total equity and liabilities & $€ 42,497$ & $€ 44,277$ \\
\hline
\end{tabular}
\end{center}

Source: SAP Group 2017 annual report. Exhibit 13: Apple Inc. Consolidated Balance Sheet (Excerpt: Non-Current Liabilities Detail)* (in millions of \$)

\begin{center}
\begin{tabular}{lcc}
\hline
Assets & $\mathbf{3 0}$ September & $\begin{array}{c}\mathbf{2 4} \text { September } \\ \mathbf{2 0 1 6}\end{array}$ \\
\hline
Total current assets & 128,645 & 106,869 \\
[All other assets]
 & 246,674 & 214,817 \\
\cline { 2 - 3 }
Total assets & 375,319 & 321,686 \\
\hline
Liabilities and shareholders' equity &  &  \\
\hline
Total current liabilities & 100,814 & 79,006 \\
Deferred revenue, non-current & 2,836 & 2,930 \\
Long-term debt & 97,207 & 75,427 \\
Other non-current liabilities & 40,415 & 36,074 \\
\hline
[Total non-current liabilities] & 140,458 & 114,431 \\
\hline
Total liabilities & 241,272 & 193,437 \\
\hline
Total shareholders' equity & 134,047 & 128,249 \\
\hline
Total liabilities and shareholders' equity & 375,319 & 321,686 \\
\hline
\end{tabular}
\end{center}

"Note: The italicized subtotals presented in this excerpt are not explicitly shown on the face of the financial statement as prepared by the company.

Source: Apple Inc. 2017 annual report (Form 10K).

\section{Long-term Financial Liabilities}
Typical long-term financial liabilities include loans (i.e., borrowings from banks) and notes or bonds payable (i.e., fixed-income securities issued to investors). Liabilities such as loans payable and bonds payable are usually reported at amortised cost on the balance sheet. At maturity, the amortised cost of the bond (carrying amount) will be equal to the face value of the bond. For example, if a company issues $\$ 10,000,000$ of bonds at par, the bonds are reported as a long-term liability of $\$ 10$ million. The carrying amount (amortised cost) from the date of issue to the date of maturity remains at $\$ 10$ million. As another example, if a company issues $\$ 10,000,000$ of bonds at a price of 97.50 (a discount to par), the bonds are reported as a liability of $\$ 9,750,000$ at issue date. Over the bond's life, the discount of $\$ 250,000$ is amortised so that the bond will be reported as a liability of $\$ 10,000,000$ at maturity. Similarly, any bond premium would be amortised for bonds issued at a price in excess of face or par value.

In certain cases, liabilities such as bonds issued by a company are reported at fair value. Those cases include financial liabilities held for trading, derivatives that are a liability to the company, and some non-derivative instruments such as those which are hedged by derivatives.

SAP's balance sheet in Exhibit 12 shows $€ 5,034$ million of financial liabilities, and the notes disclose that these liabilities are mostly for bonds payable. Apple's balance sheet shows $\$ 97,207$ million of long-term debt, and the notes disclose that this debt includes floating- and fixed-rate notes with varying maturities.

\section{Deferred Tax Liabilities}
Deferred tax liabilities result from temporary timing differences between a company's income as reported for tax purposes (taxable income) and income as reported for financial statement purposes (reported income). Deferred tax liabilities result when taxable income and the actual income tax payable in a period based on it is less than the reported financial statement income before taxes and the income tax expense based on it. Deferred tax liabilities are defined as the amounts of income taxes payable in future periods in respect of taxable temporary differences. ${ }^{20}$ In contrast, in the previous discussion of unearned revenue, inclusion of revenue in taxable income in an earlier period created a deferred tax asset (essentially prepaid tax).

Deferred tax liabilities typically arise when items of expense are included in taxable income in earlier periods than for financial statement net income. This results in taxable income being less than income before taxes in the earlier periods. As a result, taxes payable based on taxable income are less than income tax expense based on accounting income before taxes. The difference between taxes payable and income tax expense results in a deferred tax liability-for example, when companies use accelerated depreciation methods for tax purposes and straight-line depreciation methods for financial statement purposes. Deferred tax liabilities also arise when items of income are included in taxable income in later periods-for example, when a company's subsidiary has profits that have not yet been distributed and thus have not yet been taxed.

SAP's balance sheet in Exhibit 12 shows $€ 240$ million of deferred tax liabilities. Apple's balance sheet in Exhibit 13 does not show a separate line item for deferred tax liabilities, however, note disclosures indicate that most of the $\$ 40,415$ million of other non-current liabilities reported on Apple's balance sheet represents deferred tax liabilities, which totaled $\$ 31,504$ million.

\section{COMPONENTS OF EQUITY}
describe the components of shareholders' equity

Equity is the owners' residual claim on a company's assets after subtracting its liabilities. ${ }^{21}$ It represents the claim of the owner against the company. Equity includes funds directly invested in the company by the owners, as well as earnings that have been reinvested over time. Equity can also include items of gain or loss that are not recognized on the company's income statement.

\section{Components of Equity}
Six main components typically comprise total owners' equity. The first five components listed below comprise equity attributable to owners of the parent company. The sixth component is the equity attributable to non-controlling interests.

\begin{enumerate}
  \item Capital contributed by owners (or common stock, or issued capital). The amount contributed to the company by owners. Ownership of a corporation is evidenced through the issuance of common shares. Common shares may have a par value (or stated value) or may be issued as no par shares (depending on regulations governing the incorporation). Where par or stated value requirements exist, it must be disclosed in the equity section of the balance sheet. In addition, the number of shares authorized, issued, and outstanding must be disclosed for each class of share issued by the company. The
\end{enumerate}

20 IAS 12, Income Taxes, paragraph 5.

21 IASB Conceptual Framework (2018), paragraph 4.4 (c) and FASB ASC 505-10-05-3 [Equity-Overview and Background] number of authorized shares is the number of shares that may be sold by the company under its articles of incorporation. The number of issued shares refers to those shares that have been sold to investors. The number of outstanding shares consists of the issued shares less treasury shares.

\begin{enumerate}
  \setcounter{enumi}{1}
  \item Preferred shares. Classified as equity or financial liabilities based upon their characteristics rather than legal form. For example, perpetual, non-redeemable preferred shares are classified as equity. In contrast, preferred shares with mandatory redemption at a fixed amount at a future date are classified as financial liabilities. Preferred shares have rights that take precedence over the rights of common shareholders-rights that generally pertain to receipt of dividends and receipt of assets if the company is liquidated.

  \item Treasury shares (or treasury stock or own shares repurchased). Shares in the company that have been repurchased by the company and are held as treasury shares, rather than being cancelled. The company is able to sell (reissue) these shares. A company may repurchase its shares when management considers the shares undervalued, needs shares to fulfill employees' stock options, or wants to limit the effects of dilution from various employee stock compensation plans. A repurchase of previously issued shares reduces shareholders' equity by the amount of the cost of repurchasing the shares and reduces the number of total shares outstanding. If treasury shares are subsequently reissued, a company does not recognize any gain or loss from the reissuance on the income statement. Treasury shares are non-voting and do not receive any dividends declared by the company.

  \item Retained earnings. The cumulative amount of earnings recognized in the company's income statements which have not been paid to the owners of the company as dividends.

  \item Accumulated other comprehensive income (or other reserves). The cumulative amount of other comprehensive income or loss. The term comprehensive income includes both a) net income, which is recognized on the income statement and is reflected in retained earnings, and b) other comprehensive income which is not recognized as part of net income and is reflected in accumulated other comprehensive income. ${ }^{22}$

  \item Noncontrolling interest (or minority interest). The equity interests of minority shareholders in the subsidiary companies that have been consolidated by the parent (controlling) company but that are not wholly owned by the parent company.

\end{enumerate}

Exhibit 14 and Exhibit 15 present excerpts of the balance sheets of SAP Group and Apple Inc., respectively, with detailed line items for each company's equity section. SAP's balance sheet indicates that the company has $€ 1,229$ million issued capital, and the notes to the financial statements disclose that the company has issued 1,229 million no-par common stock with a nominal value of $€ 1$ per share. SAP's balance sheet also indicates that the company has $€ 1,591$ million of treasury shares, and the notes to the financial statements disclose that the company holds 35 million of its shares as treasury shares. The line item share premium of $€ 570$ million includes amounts from treasury

22 IFRS defines Total comprehensive income as "the change in equity during a period resulting from transactions and other events, other than those changes resulting from transactions with owners in their capacity as owners. (IAS 1, Presentation of Financial Statements, paragraph 7. Similarly, US GAAP defines comprehensive income as "the change in equity [net assets] of a business entity during a period from transactions and other events and circumstances from nonowner sources. It includes all changes in equity during a period except those resulting from investments by owners and distributions to owners." (FASB ASC Master Glossary.) share transactions (and certain other transactions). The amount of retained earnings, $€ 24,794$ million, represents the cumulative amount of earnings that the company has recognized in its income statements, net of dividends. SAP's $€ 508$ million of "Other components of equity" includes the company's accumulated other comprehensive income. The notes disclose that this is composed of $€ 330$ million gains on exchange differences in translation, $€ 157$ million gains on remeasuring available-for-sale financial assets, and $€ 21$ million gains on cash flow hedges. The balance sheet next presents a subtotal for the amount of equity attributable to the parent company $€ 25,509$ million followed by the amount of equity attributable to non-controlling interests $€ 31$ million. Total equity includes both equity attributable to the parent company and equity attributable to non-controlling interests.

The equity section of Apple's balance sheet consists of only three line items: common stock, retained earnings, and accumulated other comprehensive income/(loss). Although Apple's balance sheet shows no treasury stock, the company does repurchase its own shares but cancels the repurchased shares rather than holding the shares in treasury. Apple's balance sheet shows that 5,126,201 thousand shares were issued and outstanding at the end of fiscal 2017 and 5,336,166 thousand shares were issued and outstanding at the end of fiscal 2016. Details on the change in shares outstanding is presented on the Statement of Shareholders' Equity in Exhibit 16, which shows that in 2017 Apple repurchased 246,496 thousand shares of its previously issued common stock and issued 36,531 thousand shares to employees.

\section{Exhibit 14: SAP Group Consolidated Statements of Financial Position}
 (Excerpt: Equity Detail) (in millions of $€$ )as of 31 December

\begin{center}
\begin{tabular}{lcc}
\hline
\multicolumn{1}{c}{Assets} & $\mathbf{2 0 1 7}$ & $\mathbf{2 0 1 6}$ \\
\hline
Total current assets & 11,930 & 11,564 \\
Total non-current assets & 30,567 & 32,713 \\
Total assets & 42,497 & 44,277 \\
Total current liabilities & 10,210 & 9,674 \\
Total non-current liabilities & 6,747 & 8,205 \\
Total liabilities & 16,958 & 17,880 \\
\cline { 2 - 3 }
Issued capital & 1,229 & 1,229 \\
Share premium & 570 & 599 \\
Retained earnings & 24,794 & 22,302 \\
Other components of equity & 508 & 3,346 \\
Treasury shares & $(1,591)$ & $(1,099)$ \\
Equity attributable to owners of parent & 25,509 & 26,376 \\
Non-controlling interests & 31 & 21 \\
Total equity & 25,540 & 26,397 \\
Total equity and liabilities & $€ 42,497$ & $€ 44,277$ \\
\hline
\end{tabular}
\end{center}

Source: SAP Group 2017 annual report. Exhibit 15: Apple Inc. Consolidated Balance Sheet (Excerpt: Equity Detail) (in millions of $\$$ ) (Number of shares are reflected in thousands)

\begin{center}
\begin{tabular}{|c|c|c|}
\hline
Assets & $\begin{array}{c}30 \text { September } \\ 2017\end{array}$ & $\begin{array}{c}24 \text { Septembe } \\ 2016\end{array}$ \\
\hline
Total current assets & 128,645 & 106,869 \\
\hline
[All other assets] & 246,674 & 214,817 \\
\hline
Total assets & 375,319 & 321,686 \\
\hline
\multicolumn{3}{|l|}{Liabilities and shareholders' equity} \\
\hline
Total current liabilities & 100,814 & 79,006 \\
\hline
[Total non-current liabilities] & 140,458 & 114,431 \\
\hline
Total liabilities & 241,272 & 193,437 \\
\hline
$\begin{array}{l}\text { Common stock and additional paid-in capital, } \\ \$ 0.00001 \text { par value: } 12,600,000 \text { shares authorized; } \\ 5,126,201 \text { and } 5,336,166 \text { shares issued and out- } \\ \text { standing, respectively }\end{array}$ & 35,867 & 31,251 \\
\hline
Retained earnings & 98,330 & 96,364 \\
\hline
Accumulated other comprehensive income/(loss) & $(150)$ & 634 \\
\hline
Total shareholders' equity & 134,047 & 128,249 \\
\hline
Total liabilities and shareholders' equity & 375,319 & 321.686 \\
\hline
\end{tabular}
\end{center}

Source: Apple Inc. 2017 annual report (10K).

\section{STATEMENT OF CHANGES IN EQUITY}
$$
\text { describe the components of shareholders' equity }
$$

The statement of changes in equity (or statement of shareholders' equity) presents information about the increases or decreases in a company's equity over a period. IFRS requires the following information in the statement of changes in equity:

\begin{itemize}
  \item total comprehensive income for the period;

  \item the effects of any accounting changes that have been retrospectively applied to previous periods;

  \item capital transactions with owners and distributions to owners; and

  \item reconciliation of the carrying amounts of each component of equity at the beginning and end of the year. ${ }^{23}$

\end{itemize}

Under US GAAP, the requirement as specified by the SEC is for companies to provide an analysis of changes in each component of stockholders' equity that is shown in the balance sheet. ${ }^{24}$

23 IAS 1, Presentation of Financial Statements, paragraph 106.

24 FASB ASC 505-10-S99 [Equity-Overall-SEC materials] indicates that a company can present the analysis of changes in stockholders' equity either in the notes or in a separate statement. Exhibit 16 presents an excerpt from Apple's Consolidated Statements of Changes in Shareholders' Equity. The excerpt shows only one of the years presented on the actual statement. It begins with the balance as of 24 September 2016 (i.e., the beginning of fiscal 2017) and presents the analysis of changes to 30 September 2017 in each component of equity that is shown on Apple's balance sheet. As noted above, the number of shares outstanding decreased from 5,336,166 thousand to 5,126,201 thousand as the company repurchased 246,496 thousand shares of its common stock and issued 36,531 thousand new shares which reduced the dollar balance of Paid-in Capital and Retained earnings by $\$ 913$ million and $\$ 581$ million, respectively. The dollar balance in common stock also increased by \$4,909 million in connection with share-based compensation. Retained earnings increased by $\$ 48,351$ million net income, minus $\$ 12,803$ million dividends, $\$ 33,001$ million for the share repurchase and $\$ 581$ million adjustment in connection with the stock issuance. For companies that pay dividends, the amount of dividends are shown separately as a deduction from retained earnings. The statement also provides details on the $\$ 784$ million change in Apple's Accumulated other comprehensive income. Note that the statement provides a subtotal for total comprehensive income that includes net income and each of the components of other comprehensive income.

Exhibit 16: Excerpt from Apple Inc.'s Consolidated Statements of Changes in Shareholders' Equity (in millions, except share amounts which are reflected in thousands)

\begin{center}
\begin{tabular}{|c|c|c|c|c|c|}
\hline
 & \multicolumn{2}{|c|}{$\begin{array}{c}\text { Common Stock and } \\
\text { Additional Paid-In Capital }\end{array}$} & \multirow{2}{*}{$\begin{array}{l}\text { Retained } \\
\text { Earnings }\end{array}$} & \multirow{2}{*}{$\begin{array}{c}\text { Accumulated Other } \\
\text { Comprehensive } \\
\text { Income/(Loss) }\end{array}$} & \multirow{2}{*}{$\begin{array}{c}\text { Total } \\
\text { Shareholders' } \\
\text { Equity }\end{array}$} \\
\hline
 & Shares & Amount &  &  \\
\hline
$\begin{array}{l}\text { Balances as of September } 24 \text {, } \\ 2016\end{array}$ & $5,336,166$ & 31,251 & 96,364 & 634 & 128,249 \\
\hline
Net income & - & - & 48,351 & - & 48,351 \\
\hline
$\begin{array}{l}\text { Other comprehensive income/ } \\ \text { (loss) }\end{array}$ & - & - & - & $(784)$ & $(784)$ \\
\hline
$\begin{array}{l}\text { Dividends and dividend equiva- } \\ \text { lents declared }\end{array}$ & - & - & $(12,803)$ & - & $(12,803)$ \\
\hline
Repurchase of common stock & $(246,496)$ & - & $(33,001)$ & - & $(33,001)$ \\
\hline
Share-based compensation & - & 4,909 & - & - & 4,909 \\
\hline
$\begin{array}{l}\text { Common stock issued, net of } \\ \text { shares withheld for employee } \\ \text { taxes }\end{array}$ & 36,531 & $(913)$ & $(581)$ & - & $(1,494)$ \\
\hline
$\begin{array}{l}\text { Tax benefit from equity awards, } \\ \text { including transfer pricing } \\ \text { adjustments }\end{array}$ & - & 620 & - & - & 620 \\
\hline
$\begin{array}{l}\text { Balances as of September 30, } \\ 2017\end{array}$ & $5,126,201$ & 35,867 & 98,330 & $(150)$ & 134,047 \\
\hline
\end{tabular}
\end{center}

\section{COMMON SIZE ANALYSIS OF BALANCE SHEET}
demonstrate the conversion of balance sheets to common-size balance sheets and interpret common-size balance sheets

This section describes two tools for analyzing the balance sheet: common-size analysis and balance sheet ratios. Analysis of a company's balance sheet can provide insight into the company's liquidity and solvency-as of the balance sheet date-as well as the economic resources the company controls. Liquidity refers to a company's ability to meet its short-term financial commitments. Assessments of liquidity focus a company's ability to convert assets to cash and to pay for operating needs. Solvency refers to a company's ability to meet its financial obligations over the longer term. Assessments of solvency focus on the company's financial structure and its ability to pay long-term financing obligations.

\section{Common-Size Analysis of the Balance Sheet}
The first technique, vertical common-size analysis, involves stating each balance sheet item as a percentage of total assets. ${ }^{25} \mathrm{Common}$-size statements are useful in comparing a company's balance sheet composition over time (time-series analysis) and across companies in the same industry. To illustrate, Panel A of Exhibit 17 presents a balance sheet for three hypothetical companies. Company $C$, with assets of $\$ 9.75$ million is much larger than Company A and Company B, each with only $\$ 3.25$ million in assets. The common-size balance sheet presented in Panel B facilitates a comparison of these different sized companies.

\section{Exhibit 17}
Panel A: Balance Sheets for Companies A, B, and C

(\$ Thousands)

A

B

C

ASSETS

Current assets

Cash and cash equivalents

\begin{center}
\begin{tabular}{|c|c|c|}
\hline
1,000 & 200 & 3,000 \\
\hline
900 & - & 300 \\
\hline
500 & 1,050 & 1,500 \\
\hline
100 & 950 & 300 \\
\hline
2,500 & 2,200 & 5,100 \\
\hline
750 & 750 & 4,650 \\
\hline
- & 200 & - \\
\hline
- & 100 & - \\
\hline
3,250 & 3,250 & 9,750 \\
\hline
\end{tabular}
\end{center}

Total assets

LIABILITIES AND SHAREHOLDERS' EQUITY

25 As discussed in the curriculum reading on financial statement analysis, another type of common-size analysis, known as "horizontal common-size analysis," states quantities in terms of a selected base-year value. Unless otherwise indicated, text references to "common-size analysis" refer to vertical analysis. Panel A: Balance Sheets for Companies A, B, and C

(\$ Thousands)

A

B

C

Current liabilities

Accounts payable

Total current liabilities

Long term bonds payable

Total liabilities

Total shareholders' equity

Total liabilities and shareholders' equity

\begin{center}
\begin{tabular}{|c|c|c|}
\hline
- & 2,500 & 600 \\
\hline
- & 2,500 & 600 \\
\hline
10 & 10 & 9,000 \\
\hline
10 & 2,510 & 9,600 \\
\hline
3,240 & 740 & 150 \\
\hline
3250 & 3250 & 9750 \\
\hline
\end{tabular}
\end{center}

Panel B: Common-Size Balance Sheets for Companies A, B, and C

(Percent)

A

B

C

ASSETS

Current assets

Cash and cash equivalents

\begin{center}
\begin{tabular}{|c|c|c|}
\hline
30.8 & 6.2 & 30.8 \\
\hline
27.7 & 0.0 & 3.1 \\
\hline
15.4 & 32.3 & 15.4 \\
\hline
3.1 & 29.2 & 3.1 \\
\hline
76.9 & 67.7 & 52.3 \\
\hline
23.1 & 23.1 & 47.7 \\
\hline
0.0 & 6.2 & 0.0 \\
\hline
0.0 & 3.1 & 0.0 \\
\hline
1000 & 1000 & 100.0 \\
\hline
\end{tabular}
\end{center}

Total assets

\section{LIABILITIES AND SHAREHOLDERS' EQUITY}
Current liabilities

Accounts payable

Total current liabilities

Long term bonds payable

Total liabilities

Total shareholders' equity

Total liabilities and shareholders' equity

\begin{center}
\begin{tabular}{|c|c|c|}
\hline
0.0 & 76.9 & 6.2 \\
\hline
0.0 & 76.9 & 6.2 \\
\hline
0.3 & 0.3 & 92.3 \\
\hline
0.3 & 77.2 & 98.5 \\
\hline
99.7 & 22.8 & 1.5 \\
\hline
100.0 & 100.0 & 100.0 \\
\hline
\end{tabular}
\end{center}

Most of the assets of Company $\mathrm{A}$ and $\mathrm{B}$ are current assets; however, Company $\mathrm{A}$ has nearly 60 percent of its total assets in cash and short-term marketable securities while Company B has only 6 percent of its assets in cash. Company A is more liquid than Company B. Company A shows no current liabilities (its current liabilities round to less than $\$ 10$ thousand), and it has cash on hand of $\$ 1.0$ million to meet any near-term financial obligations it might have. In contrast, Company B has $\$ 2.5$ million of current liabilities which exceed its available cash of only $\$ 200$ thousand. To pay those near-term obligations, Company B will need to collect some of its accounts receivables, sell more inventory, borrow from a bank, and/or raise more long-term capital (e.g., by issuing more bonds or more equity). Company $C$ also appears more liquid than Company B. It holds over 30 percent of its total assets in cash and short-term marketable securities, and its current liabilities are only 6.2 percent of the amount of total assets. Company C's $\$ 3.3$ million in cash and short-term marketable securities is substantially more than its current liabilities of $\$ 600$ thousand. Turning to the question of solvency, however, note that 98.5 percent of Company C's assets are financed with liabilities. If Company $\mathrm{C}$ experiences significant fluctuations in cash flows, it may be unable to pay the interest and principal on its long-term bonds. Company A is far more solvent than Company C, with less than one percent of its assets financed with liabilities.

Note that these examples are hypothetical only. Other than general comparisons, little more can be said without further detail. In practice, a wide range of factors affect a company's liquidity management and capital structure. The study of optimal capital structure is a fundamental issue addressed in corporate finance. Capital refers to a company's long-term debt and equity financing; capital structure refers to the proportion of debt versus equity financing.

Common-size balance sheets can also highlight differences in companies' strategies. Comparing the asset composition of the companies, Company $\mathrm{C}$ has made a greater proportional investment in property, plant, and equipment-possibly because it manufactures more of its products in-house. The presence of goodwill on Company B's balance sheet signifies that it has made one or more acquisitions in the past. In contrast, the lack of goodwill on the balance sheets of Company A and Company C suggests that these two companies may have pursued a strategy of internal growth rather than growth by acquisition. Company A may be in either a start-up or liquidation stage of operations as evidenced by the composition of its balance sheet. It has relatively little inventory and no accounts payable. It either has not yet established trade credit or it is in the process of paying off its obligations in the process of liquidating.

\section{EXAMPLE 5}
\section{Common-Size Analysis}
\begin{enumerate}
  \item Applying common-size analysis to the excerpts of SAP Group's balance sheets presented in Exhibits 4, 6, 8, and 12, answer the following: In 2017 relative to 2016, which of the following line items increased as a percentage of assets?
\end{enumerate}

A. Cash and cash equivalents.

B. Total current assets.

C. Total financial liabilities

D. Total deferred income.

Solution:

A, B, and D are correct. The following items increased as a percentage of total assets:

\begin{itemize}
  \item Cash and cash equivalents increased from 8.4 percent of total assets in $2016(€ 3,702 \div € 44,277)$ to 9.4 percent in $2017(€ 4,011 \div € 42,497)$.

  \item Total current assets increased from 26.1 percent of total assets in 2016 $(€ 11,564 \div € 44,277)$ to 28.1 percent in 2017 (€11,930 €42,497).

  \item Total deferred income increased from 5.7 percent of total assets in $2016((€ 2,383+€ 143) \div € 44,277)$ to 6.7 percent in $2017((€ 2,771$ $+€ 79) \div € 42,497)$

\end{itemize}

Total financial liabilities decreased both in absolute Euro amounts and as a percentage of total assets when compared with the previous year. Note that some amounts of the company's deferred income and financial liabilities are classified as current liabilities (shown in Exhibit 6) and some amounts are classified as non-current liabilities (shown in Exhibit 12). The total amounts-current and non-current-of deferred income and financial liabilities, therefore, are obtained by summing the amounts in Exhibits 6 and 12.

Overall, aspects of the company's liquidity position are somewhat stronger in 2017 compared to 2016. The company's cash balances as a percentage of total assets increased. While current liabilities increased as a percentage of total assets and total liabilities remained approximately the same percentage, the mix of liabilities shifted. Financial liabilities, which represent future cash outlays, decreased as a percentage of total assets, while deferred revenues, which represent cash received in advance of revenue recognition, increased.

Common-size analysis of the balance sheet is particularly useful in cross-sectional analysis-comparing companies to each other for a particular time period or comparing a company with industry or sector data. The analyst could select individual peer companies for comparison, use industry data from published sources, or compile data from databases. When analyzing a company, many analysts prefer to select the peer companies for comparison or to compile their own industry statistics.

Exhibit 18 presents common-size balance sheet data compiled for the 10 sectors of the S\&P 500 using 2017 data. The sector classification follows the S\&P/MSCI Global Industrial Classification System (GICS). The exhibit presents mean and median common-size balance sheet data for those companies in the S\&P 500 for which 2017 data was available in the Compustat database. ${ }^{26}$

Some interesting general observations can be made from these data:

\begin{itemize}
  \item Energy and utility companies have the largest amounts of property, plant, and equipment (PPE). Telecommunication services, followed by utilities, have the highest level of long-term debt. Utilities also use some preferred stock.

  \item Financial companies have the greatest percentage of total liabilities. Financial companies typically have relatively high financial leverage.

  \item Telecommunications services and utility companies have the lowest level of receivables.

  \item Inventory levels are highest for consumer discretionary. Materials and consumer staples have the next highest inventories.

  \item Information technology companies use the least amount of leverage as evidenced by the lowest percentages for long-term debt and total liabilities and highest percentages for common and total equity.

\end{itemize}

Example 6 discusses an analyst using cross-sectional common-size balance sheet data.

\section{EXAMPLE 6}
\section{Cross-Sectional Common-Size Analysis}
Jason Lu is comparing two companies in the computer industry to evaluate their relative financial position as reflected on their balance sheets. He has compiled the following vertical common-size data for Apple and Microsoft.

26 An entry of zero for an item (e.g., current assets) was excluded from the data, except in the case of preferred stock. Note that most financial institutions did not provide current asset or current liability data, so these are reported as not available in the database.

\begin{center}
\includegraphics[max width=\textwidth]{2023_05_04_b5cfa4f1bc883752f121g-116}
\end{center}

\begin{center}
\includegraphics[max width=\textwidth]{2023_05_04_b5cfa4f1bc883752f121g-117}
\end{center}

\section{Cross-Sectional Analysis: Consolidated Balance Sheets (as Percent of Total Assets)}
\begin{center}
\begin{tabular}{lcc}
\hline
 & Apple & Microsoft \\
\hline
ASSETS: & $\mathbf{3 0}$ September 2017 & $\mathbf{3 0}$ June 2017 \\
\hline
Current assets: &  & 3.2 \\
Cash and cash equivalents & 5.4 & 52.0 \\
Short-term marketable securities & 14.4 & 8.2 \\
Accounts receivable & 4.8 & 0.9 \\
Inventories & 1.3 & 0.0 \\
Vendor non-trade receivables & 4.7 & 2.0 \\
Other current assets & 3.7 & 66.3 \\
Total current assets & 34.3 & 2.5 \\
Long-term marketable securities & 51.9 & 9.8 \\
Property, plant and equipment, net & 9.0 & 14.6 \\
Goodwill & 1.5 & 4.2 \\
Acquired intangible assets, net & 0.6 & 2.6 \\
Other assets & 2.7 & 100.0 \\
\end{tabular}
\end{center}

LIABILITIES AND SHAREHOLDERS' EQUITY:

Current liabilities:

Accounts payable

Short-term debt

\begin{center}
\begin{tabular}{cc}
13.1 & 3.1 \\
3.2 & 3.8 \\
1.7 & 0.4 \\
6.9 & 2.7 \\
2.0 & 14.1 \\
0.0 & 2.6 \\
\hline
26.9 & 26.8 \\
25.9 & 31.6 \\
0.8 & 4.3 \\
10.8 & 7.3 \\
\hline
64.3 & 70.0 \\
\hline
\end{tabular}
\end{center}

Current portion of long-term debt

Accrued expenses

Deferred revenue

Other current liabilities

Total current liabilities

Long-term debt

Deferred revenue non-current

Other non-current liabilities

Total liabilities

Commitments and contingencies

Total shareholders' equity

Total liabilities and shareholders' equity

\begin{center}
\begin{tabular}{cc}
35.7 & 30.0 \\
\hline
100.0 & 100.0 \\
\hline
\end{tabular}
\end{center}

Source: Based on data from companies' annual reports.

From this data, Lu learns the following:

\begin{itemize}
  \item Apple and Microsoft have high levels of cash and short-term marketable securities, consistent with the information technology sector as reported in Exhibit 18. Apple also has a high balance in long-term marketable securities. This may reflect the success of the company's business model, which has generated large operating cash flows in recent years.

  \item Apple's level of accounts receivable is lower than Microsoft's and lower than the industry average. Further research is necessary to learn the extent to which this is related to Apple's cash sales through its own retail stores. An alternative explanation would be that the company has been selling/factoring receivables to a greater degree than the other companies; however, that explanation is unlikely given Apple's cash position. Additionally, Apple shows vendor non-trade receivables, reflecting arrangements with its contract manufacturers.

  \item Apple and Microsoft both have low levels of inventory, similar to industry medians as reported in Exhibit 18. Apple uses contract manufacturers and can rely on suppliers to hold inventory until needed. Additionally, in the Management Discussion and Analysis section of their annual report, Apple discloses $\$ 38$ billion of noncancelable manufacturing purchase obligations, $\$ 33$ billion of which is due within twelve months. These amounts are not currently recorded as inventory and reflect the use of contract manufacturers to assemble and test some finished products. The use of purchase commitments and contract manufacturers implies that inventory may be "understated." Microsoft's low level of inventory is consistent with its business mix which is more heavily weighted to software than to hardware.

  \item Apple and Microsoft have a level of property, plant, and equipment that is relatively close to the sector median as reported in Exhibit 18 .

  \item Apple has a very low amount of goodwill, reflecting its strategy to grow organically rather than through acquisition. Microsoft's level of goodwill, while higher than Apple's, is lower than the industry median and mean. Microsoft made a number of major acquisitions (for example, Nokia in 2014) but subsequently (in 2015) wrote off significant amounts of goodwill as an impairment charge.

  \item Apple's level of accounts payable is higher than the industry, but given the company's high level of cash and investments, it is unlikely that this is a problem.

  \item Apple's and Microsoft's levels of long-term debt are slightly higher than industry averages. Again, given the companies' high level of cash and investments, it is unlikely that this is a problem.

\end{itemize}

\section{BALANCE SHEET RATIOS}
calculate and interpret liquidity and solvency ratios

Ratios facilitate time-series and cross-sectional analysis of a company's financial position. Balance sheet ratios are those involving balance sheet items only. Each of the line items on a vertical common-size balance sheet is a ratio in that it expresses a balance sheet amount in relation to total assets. Other balance sheet ratios compare one balance sheet item to another. For example, the current ratio expresses current assets in relation to current liabilities as an indicator of a company's liquidity. Balance sheet ratios include liquidity ratios (measuring the company's ability to meet its short-term obligations) and solvency ratios (measuring the company's ability to meet long-term and other obligations). These ratios and others are discussed in a later reading. Exhibit 20 summarizes the calculation and interpretation of selected balance sheet ratios.

\section{Exhibit 20: Balance Sheet Ratios}
\begin{center}
\begin{tabular}{|c|c|c|}
\hline
Liquidity Ratios & Calculation & Indicates \\
\hline
Current & $\begin{array}{l}\text { Current assets } \div \text { Current } \\ \text { liabilities }\end{array}$ & Ability to meet current liabilities \\
\hline
Quick (acid test) & $\begin{array}{l}\text { (Cash }+ \text { Marketable securi- } \\ \text { ties }+ \text { Receivables }) \div \text { Current } \\ \text { liabilities }\end{array}$ & Ability to meet current liabilities \\
\hline
Cash & $\begin{array}{l}\text { (Cash + Marketable securities) } \\ \div \text { Current liabilities }\end{array}$ & Ability to meet current liabilities \\
\hline
\end{tabular}
\end{center}

\section{Solvency Ratios}
\begin{center}
\begin{tabular}{lll}
\hline
$\begin{array}{l}\text { Long-term } \\ \text { debt-to-equity }\end{array}$ & $\begin{array}{l}\text { Total long-term debt } \div \text { Total } \\ \text { equity }\end{array}$ & $\begin{array}{l}\text { Financial risk and financial } \\ \text { leverage }\end{array}$ \\
Debto-equity & Total debt $\div$ Total equity & $\begin{array}{l}\text { Financial risk and financial } \\ \text { leverage }\end{array}$ \\
Total debt & Total debt $\div$ Total assets & $\begin{array}{l}\text { Financial risk and financial } \\ \text { leverage }\end{array}$ \\
Financial leverage & Total assets $\div$ Total equity & $\begin{array}{l}\text { Financial risk and financial } \\ \text { leverage }\end{array}$ \\
\hline
\end{tabular}
\end{center}

\section{EXAMPLE 7}
\section{Ratio Analysis}
For the following ratio questions, refer to the balance sheet information for the SAP Group presented in Exhibits 1, 4, 6, 8, and 12.

\begin{enumerate}
  \item The current ratio for SAP Group at 31 December 2017 is closest to:
\end{enumerate}

A. 1.17 .

B. 1.20 .

C. 2.00 .

Solution to 1:

A is correct. SAP Group's current ratio (Current assets $\div$ Current liabilities) at 31 December 2017 is 1.17 ( $€ 11,930$ million $\div € 10,210$ million).

\begin{enumerate}
  \setcounter{enumi}{1}
  \item Which of the following liquidity ratios decreased in 2017 relative to 2016 ?
\end{enumerate}

A. Cash.

B. Quick.

C. Current.

Solution to 2:

A, B, and C are correct. The ratios are shown in the table below. The cash ratio, quick ratio, and current ratio are lower in 2017 than in 2016. The cash ratio is slightly higher.

\begin{center}
\begin{tabular}{|c|c|c|c|}
\hline
Liquidity Ratios & Calculation & $2017 €$ in millions & $2016 €$ in millions \\
\hline
Current & Current assets $\div$ Current liabilities & $€ 11,930 \div € 10,210=\mathbf{1 . 1 7}$ & $€ 11,564 \div € 9,674=\mathbf{1 . 2 0}$ \\
\hline
Quick (acid test) & $\begin{array}{l}(\text { Cash }+ \text { Marketable securities }+ \\ \text { Receivables }) \div \text { Current liabilities }\end{array}$ & $\begin{array}{l}(€ 4,011+€ 990+€ 5,899) \div \\ € 10,210=\mathbf{1 . 0 7}\end{array}$ & $\begin{array}{l}(€ 3,702+€ 1,124+€ 5,924) \div \\ € 9,674=\mathbf{1 . 1 1}\end{array}$ \\
\hline
Cash & $\begin{array}{l}(\text { Cash }+ \text { Marketable securities }) \div \\ \text { Current liabilities }\end{array}$ & $\begin{array}{l}(€ 4,011+€ 990 \div € 10,210 \\ =\mathbf{0 . 4 9}\end{array}$ & $\begin{array}{l}(€ 3,702+€ 1,124 \div € 9,674= \\ \mathbf{0 . 5 0}\end{array}$ \\
\hline
\end{tabular}
\end{center}

\begin{enumerate}
  \setcounter{enumi}{2}
  \item Which of the following leverage ratios decreased in 2017 relative to 2016 ?
\end{enumerate}

A. Debt-to-equity.

B. Financial leverage.

C. Long-term debt-to-equity.

\section{Solution to 3:}
A, B, and C are correct. The ratios are shown in the table below. All three leverage ratios decreased in 2017 relative to 2016.

\begin{center}
\begin{tabular}{llrr}
\hline
$\begin{array}{lrr}\text { Solvency } \\ \text { Ratios }\end{array}$ &  \\
\hline
$\begin{array}{llrl}\text { Long-term } & \text { Total long-term debt } \div & € 5,034 \div € 25,540 & € 6,481 \div € 26,397= \\ \text { debt-to-equity } & \text { Total equity } & =\mathbf{1 9 . 7 \%} & \mathbf{2 4 . 6 \%} \\ \text { Debt-to-equity } & \text { Total debt } \div \text { Total equity } & (€ 1,561+€ 5,034 & (€ 1,813+€ 6,481) \\ & & ) \div € 25,540= & \div € 26,397=\mathbf{3 1 . 4 \%} \\ \text { Financial } & \text { Total assets } \div \text { Total } & \mathbf{2 5 . 8 \%} & \\ \text { Leverage } & \text { equity } & € 42,497 \div & € 44,277 \div € 26,397 \\ & & € 25,540=\mathbf{1 . 6 6} & \mathbf{1 . 6 8}\end{array}$ &  \\
\hline
\end{tabular}
\end{center}

Cross-sectional financial ratio analysis can be limited by differences in accounting methods. In addition, lack of homogeneity of a company's operating activities can limit comparability. For diversified companies operating in different industries, using industry-specific ratios for different lines of business can provide better comparisons. Companies disclose information on operating segments. The financial position and performance of the operating segments can be compared to the relevant industry.

Ratio analysis requires a significant amount of judgment. One key area requiring judgment is understanding the limitations of any ratio. The current ratio, for example, is only a rough measure of liquidity at a specific point in time. The ratio captures only the amount of current assets, but the components of current assets differ significantly in their nearness to cash (e.g., marketable securities versus inventory). Another limitation of the current ratio is its sensitivity to end-of-period financing and operating decisions that can potentially impact current asset and current liability amounts. Another overall area requiring judgment is determining whether a ratio for a company is within a reasonable range for an industry. Yet another area requiring judgment is evaluating whether a ratio signifies a persistent condition or reflects only a temporary condition. Overall, evaluating specific ratios requires an examination of the entire operations of a company, its competitors, and the external economic and industry setting in which it is operating.

\section{SUMMARY}
The balance sheet (also referred to as the statement of financial position) discloses what an entity owns (assets) and what it owes (liabilities) at a specific point in time. Equity is the owners' residual interest in the assets of a company, net of its liabilities. The amount of equity is increased by income earned during the year, or by the issuance of new equity. The amount of equity is decreased by losses, by dividend payments, or by share repurchases.

An understanding of the balance sheet enables an analyst to evaluate the liquidity, solvency, and overall financial position of a company.

\begin{itemize}
  \item The balance sheet distinguishes between current and non-current assets and between current and non-current liabilities unless a presentation based on liquidity provides more relevant and reliable information.

  \item The concept of liquidity relates to a company's ability to pay for its near-term operating needs. With respect to a company overall, liquidity refers to the availability of cash to pay those near-term needs. With respect to a particular asset or liability, liquidity refers to its "nearness to cash."

  \item Some assets and liabilities are measured on the basis of fair value and some are measured at historical cost. Notes to financial statements provide information that is helpful in assessing the comparability of measurement bases across companies.

  \item Assets expected to be liquidated or used up within one year or one operating cycle of the business, whichever is greater, are classified as current assets. Assets not expected to be liquidated or used up within one year or one operating cycle of the business, whichever is greater, are classified as non-current assets.

  \item Liabilities expected to be settled or paid within one year or one operating cycle of the business, whichever is greater, are classified as current liabilities. Liabilities not expected to be settled or paid within one year or one operating cycle of the business, whichever is greater, are classified as non-current liabilities.

  \item Trade receivables, also referred to as accounts receivable, are amounts owed to a company by its customers for products and services already delivered. Receivables are reported net of the allowance for doubtful accounts.

  \item Inventories are physical products that will eventually be sold to the company's customers, either in their current form (finished goods) or as inputs into a process to manufacture a final product (raw materials and work-in-process). Inventories are reported at the lower of cost or net realizable value. If the net realizable value of a company's inventory falls below its carrying amount, the company must write down the value of the inventory and record an expense.

  \item Inventory cost is based on specific identification or estimated using the first-in, first-out or weighted average cost methods. Some accounting standards (including US GAAP but not IFRS) also allow last-in, first-out as an additional inventory valuation method.

  \item Accounts payable, also called trade payables, are amounts that a business owes its vendors for purchases of goods and services.

  \item Deferred revenue (also known as unearned revenue) arises when a company receives payment in advance of delivery of the goods and services associated with the payment received. - Property, plant, and equipment (PPE) are tangible assets that are used in company operations and expected to be used over more than one fiscal period. Examples of tangible assets include land, buildings, equipment, machinery, furniture, and natural resources such as mineral and petroleum resources.

  \item IFRS provide companies with the choice to report PPE using either a historical cost model or a revaluation model. US GAAP permit only the historical cost model for reporting PPE.

  \item Depreciation is the process of recognizing the cost of a long-lived asset over its useful life. (Land is not depreciated.)

  \item Under IFRS, property used to earn rental income or capital appreciation is considered to be an investment property. IFRS provide companies with the choice to report an investment property using either a historical cost model or a fair value model.

  \item Intangible assets refer to identifiable non-monetary assets without physical substance. Examples include patents, licenses, and trademarks. For each intangible asset, a company assesses whether the useful life is finite or indefinite.

  \item An intangible asset with a finite useful life is amortised on a systematic basis over the best estimate of its useful life, with the amortisation method and useful-life estimate reviewed at least annually. Impairment principles for an intangible asset with a finite useful life are the same as for PPE.

  \item An intangible asset with an indefinite useful life is not amortised. Instead, it is tested for impairment at least annually.

  \item For internally generated intangible assets, IFRS require that costs incurred during the research phase must be expensed. Costs incurred in the development stage can be capitalized as intangible assets if certain criteria are met, including technological feasibility, the ability to use or sell the resulting asset, and the ability to complete the project.

  \item The most common intangible asset that is not a separately identifiable asset is goodwill, which arises in business combinations. Goodwill is not amortised; instead it is tested for impairment at least annually.

  \item Financial instruments are contracts that give rise to both a financial asset of one entity and a financial liability or equity instrument of another entity. In general, there are two basic alternative ways that financial instruments are measured: fair value or amortised cost. For financial instruments measured at fair value, there are two basic alternatives in how net changes in fair value are recognized: as profit or loss on the income statement, or as other comprehensive income (loss) which bypasses the income statement.

  \item Typical long-term financial liabilities include loans (i.e., borrowings from banks) and notes or bonds payable (i.e., fixed-income securities issued to investors). Liabilities such as bonds issued by a company are usually reported at amortised cost on the balance sheet.

  \item Deferred tax liabilities arise from temporary timing differences between a company's income as reported for tax purposes and income as reported for financial statement purposes.

  \item Six potential components that comprise the owners' equity section of the balance sheet include: contributed capital, preferred shares, treasury shares, retained earnings, accumulated other comprehensive income, and non-controlling interest. - The statement of changes in equity reflects information about the increases or decreases in each component of a company's equity over a period.

  \item Vertical common-size analysis of the balance sheet involves stating each balance sheet item as a percentage of total assets.

  \item Balance sheet ratios include liquidity ratios (measuring the company's ability to meet its short-term obligations) and solvency ratios (measuring the company's ability to meet long-term and other obligations).

\end{itemize}

\section{PRACTICE PROBLEMS}
\begin{enumerate}
  \item Resources controlled by a company as a result of past events are:
A. equity.
B. assets.
C. liabilities.

  \item Equity equals:
A. Assets - Liabilities.
B. Liabilities - Assets.
C. Assets + Liabilities.

  \item Shareholders' equity reported on the balance sheet is most likely to differ from the market value of shareholders' equity because:
A. historical cost basis is used for all assets and liabilities.
B. some factors that affect the generation of future cash flows are excluded.
C. shareholders' equity reported on the balance sheet is updated continuously.

  \item The information provided by a balance sheet item is limited because of uncertainty regarding:
A. measurement of its cost or value with reliability.
B. the change in current value following the end of the reporting period.
C. the probability that any future economic benefit will flow to or from the entity.

  \item Distinguishing between current and non-current items on the balance sheet and presenting a subtotal for current assets and liabilities is referred to as:
A. a classified balance sheet.
B. an unclassified balance sheet.
C. a liquidity-based balance sheet.

  \item The most likely company to use a liquidity-based balance sheet presentation is a:
A. bank.
B. computer manufacturer holding inventories.
C. software company with trade receivables and payables.

  \item An example of a contra asset account is:

\end{enumerate}

A. depreciation expense.

B. sales returns and allowances. C. allowance for doubtful accounts.

\begin{enumerate}
  \setcounter{enumi}{7}
  \item The carrying value of inventories reflects:
A. their historical cost.
B. their current value.
C. the lower of historical cost or net realizable value.

  \item When a company pays its rent in advance, its balance sheet will reflect a reduction in:
A. assets and liabilities.
B. assets and shareholders' equity.
C. one category of assets and an increase in another.

  \item Which of the following is most likely classified as a current liability?

\end{enumerate}

A. Payment received for a product due to be delivered at least one year after the balance sheet date

B. Payments for merchandise due at least one year after the balance sheet date but still within a normal operating cycle

C. Payment on debt due in six months for which the company has the unconditional right to defer settlement for at least one year after the balance sheet date

\begin{enumerate}
  \setcounter{enumi}{10}
  \item Money received from customers for products to be delivered in the future is recorded as:
\end{enumerate}

A. revenue and an asset.

B. an asset and a liability.

C. revenue and a liability.

\begin{enumerate}
  \setcounter{enumi}{11}
  \item Accrued expenses (accrued liabilities) are:
\end{enumerate}

A. expenses that have been paid.

B. created when another liability is reduced.

C. expenses that have been reported on the income statement but not yet paid.

\begin{enumerate}
  \setcounter{enumi}{12}
  \item The most likely costs included in both the cost of inventory and property, plant, and equipment are:
A. selling costs.
B. storage costs.
C. delivery costs.

  \item All of the following are current assets except:

\end{enumerate}

A. cash.
B. goodwill.
C. inventories.

\begin{enumerate}
  \setcounter{enumi}{14}
  \item The initial measurement of goodwill is most likely affected by:
A. an acquisition's purchase price.
B. the acquired company's book value.
C. the fair value of the acquirer's assets and liabilities.

  \item For financial assets classified as trading securities, how are unrealized gains and losses reflected in shareholders' equity?
A. They are not recognized.
B. They flow through income into retained earnings.
C. They are a component of accumulated other comprehensive income.

  \item For financial assets classified as available for sale, how are unrealized gains and losses reflected in shareholders' equity?
A. They are not recognized.
B. They flow through retained earnings.
C. They are a component of accumulated other comprehensive income.

  \item For financial assets classified as held to maturity, how are unrealized gains and losses reflected in shareholders' equity?
A. They are not recognized.
B. They flow through retained earnings.
C. They are a component of accumulated other comprehensive income.

  \item Debt due within one year is considered:
A. current.
B. preferred.
C. convertible.

  \item The non-controlling (minority) interest in consolidated subsidiaries is presented on the balance sheet:
A. as a long-term liability.
B. separately, but as a part of shareholders' equity.
C. as a mezzanine item between liabilities and shareholders' equity.

  \item The item "retained earnings" is a component of:
A. assets.
B. liabilities. C. shareholders' equity.

  \item When a company buys shares of its own stock to be held in treasury, it records a reduction in:
A. both assets and liabilities.
B. both assets and shareholders' equity.
C. assets and an increase in shareholders' equity.

  \item A company has total liabilities of $€ 35$ million and total stockholders' equity of $€ 55$ million. Total liabilities are represented on a vertical common-size balance sheet by a percentage closest to:
A. $35 \%$.
B. $39 \%$.
C. $64 \%$.

  \item Which of the following would an analyst most likely be able to determine from a common-size analysis of a company's balance sheet over several periods?

\end{enumerate}

A. An increase or decrease in sales.

B. An increase or decrease in financial leverage.

C. A more efficient or less efficient use of assets.

\begin{enumerate}
  \setcounter{enumi}{24}
  \item Defining total asset turnover as revenue divided by average total assets, all else equal, impairment write-downs of long-lived assets owned by a company will most likely result in an increase for that company in:
\end{enumerate}

A. the debt-to-equity ratio but not the total asset turnover.

B. the total asset turnover but not the debt-to-equity ratio.

C. both the debt-to-equity ratio and the total asset turnover.

\begin{enumerate}
  \setcounter{enumi}{25}
  \item An investor concerned whether a company can meet its near-term obligations is most likely to calculate the:
A. current ratio.
B. return on total capital.
C. financial leverage ratio.

  \item The most stringent test of a company's liquidity is its:
A. cash ratio.
B. quick ratio.
C. current ratio.

  \item An investor worried about a company's long-term solvency would most likely examine its:

\end{enumerate}

A. current ratio. B. return on equity.

C. debt-to-equity ratio.

\begin{enumerate}
  \setcounter{enumi}{28}
  \item Using the information presented in Exhibit 4 of the reading, the quick ratio for SAP Group at 31 December 2017 is closest to:
A. 1.00 .
B. 1.07 .
C. 1.17

  \item Using the information presented in Exhibit 14 of the reading, the financial leverage ratio for SAP Group at 31 December 2017 is closest to:
A. 1.50 .
B. 1.66 .
C. 2.00 .

\end{enumerate}

\section{The following information relates to questions}
31-34

Exhibit 1: Common-Size Balance Sheets for Company A, Company B, and Sector Average

\begin{center}
\begin{tabular}{|c|c|c|c|}
\hline
 & $\begin{array}{c}\text { Company } \\ \text { A }\end{array}$ & $\begin{array}{c}\text { Company } \\ \text { B }\end{array}$ & $\begin{array}{l}\text { Sector } \\ \text { Averag }\end{array}$ \\
\hline
\multicolumn{4}{|l|}{ASSETS} \\
\hline
\multicolumn{4}{|l|}{Current assets} \\
\hline
Cash and cash equivalents & 5 & 5 & 7 \\
\hline
Marketable securities & 5 & 0 & 2 \\
\hline
Accounts receivable, net & 5 & 15 & 12 \\
\hline
Inventories & 15 & 20 & 16 \\
\hline
Prepaid expenses & 5 & 15 & 11 \\
\hline
Total current assets & 35 & 55 & 48 \\
\hline
Property, plant, and equipment, net & 40 & 35 & 37 \\
\hline
Goodwill & 25 & 0 & 8 \\
\hline
Other assets & 0 & 10 & 7 \\
\hline
Total assets & 100 & 100 & 100 \\
\hline
\end{tabular}
\end{center}

LIABILITIES AND SHAREHOLDERS' EQUITY

Current liabilities

Accounts payable

Short-term debt

Accrued expenses

\begin{center}
\begin{tabular}{ccc}
10 & 10 & 10 \\
25 & 10 & 15 \\
0 & 5 & 3 \\
\hline
\end{tabular}
\end{center}

LIABILITIES AND SHAREHOLDERS' EQUITY

\begin{center}
\begin{tabular}{lcccc}
\hline
Total current liabilities & 35 &  & 25 & 28 \\
Long-term debt & 45 &  & 20 & 28 \\
Other non-current liabilities & 0 & 10 & 7 &  \\
\cline { 2 - 4 }
Total liabilities & 80 & 55 & 63 &  \\
$\quad$ Total shareholders' equity & 20 & 100 &  &  \\
Total liabilities and shareholders' equity & 100 &  & 100 &  \\
\hline
\end{tabular}
\end{center}

\begin{enumerate}
  \setcounter{enumi}{30}
  \item Based on Exhibit 1, which statement is most likely correct?
\end{enumerate}

A. Company A has below-average liquidity risk.

B. Company B has above-average solvency risk.

C. Company A has made one or more acquisitions.

\begin{enumerate}
  \setcounter{enumi}{31}
  \item The quick ratio for Company A is closest to:
A. 0.43 .
B. 0.57 .
C. 1.00 .

  \item Based on Exhibit 1, the financial leverage ratio for Company B is closest to:
A. 0.55 .
B. 1.22 .
C. 2.22

  \item Based on Exhibit 1, which ratio indicates lower liquidity risk for Company A compared with Company B?
A. Cash ratio
B. Quick ratio
C. Current ratio

\end{enumerate}

\section{SOLUTIONS}
\begin{enumerate}
  \item B is correct. Assets are resources controlled by a company as a result of past events.

  \item A is correct. Assets $=$ Liabilities + Equity and, therefore, Assets - Liabilities $=$ Equity.

  \item B is correct. The balance sheet omits important aspects of a company's ability to generate future cash flows, such as its reputation and management skills. The balance sheet measures some assets and liabilities based on historical cost and measures others based on current value. Market value of shareholders' equity is updated continuously. Shareholders' equity reported on the balance sheet is updated for reporting purposes and represents the value that was current at the end of the reporting period.

  \item B is correct. Balance sheet information is as of a specific point in time, and items measured at current value reflect the value that was current at the end of the reporting period. For all financial statement items, an item should be recognized in the financial statements only if it is probable that any future economic benefit associated with the item will flow to or from the entity and if the item has a cost or value that can be measured with reliability.

  \item A is correct. A classified balance sheet is one that classifies assets and liabilities as current or non-current and provides a subtotal for current assets and current liabilities. A liquidity-based balance sheet broadly presents assets and liabilities in order of liquidity.

  \item A is correct. A liquidity-based presentation, rather than a current/non-current presentation, may be used by such entities as banks if broadly presenting assets and liabilities in order of liquidity is reliable and more relevant.

  \item $C$ is correct. A contra asset account is netted against (i.e., reduces) the balance of an asset account. The allowance for doubtful accounts reduces the balance of accounts receivable. Accumulated depreciation, not depreciation expense, is a contra asset account. Sales returns and allowances create a contra account that reduce sales, not an asset.

  \item C is correct. Under IFRS, inventories are carried at historical cost, unless net realizable value of the inventory is less. Under US GAAP, inventories are carried at the lower of cost or market.

  \item C is correct. Paying rent in advance will reduce cash and increase prepaid expenses, both of which are assets.

  \item B is correct. Payments due within one operating cycle of the business, even if they will be settled more than one year after the balance sheet date, are classified as current liabilities. Payment received in advance of the delivery of a good or service creates an obligation or liability. If the obligation is to be fulfilled at least one year after the balance sheet date, it is recorded as a non-current liability, such as deferred revenue or deferred income. Payments that the company has the unconditional right to defer for at least one year after the balance sheet may be classified as non-current liabilities.

  \item B is correct. The cash received from customers represents an asset. The obligation to provide a product in the future is a liability called "unearned income" or "unearned revenue." As the product is delivered, revenue will be recognized and the liability will be reduced.

  \item $C$ is correct. Accrued liabilities are expenses that have been reported on a company's income statement but have not yet been paid.

  \item $\mathrm{C}$ is correct. Both the cost of inventory and property, plant, and equipment include delivery costs, or costs incurred in bringing them to the location for use or resale.

  \item B is correct. Goodwill is a long-term asset, and the others are all current assets.

  \item A is correct. Initially, goodwill is measured as the difference between the purchase price paid for an acquisition and the fair value of the acquired, not acquiring, company's net assets (identifiable assets less liabilities).

  \item B is correct. For financial assets classified as trading securities, unrealized gains and losses are reported on the income statement and flow to shareholders' equity as part of retained earnings.

  \item $\mathrm{C}$ is correct. For financial assets classified as available for sale, unrealized gains and losses are not recorded on the income statement and instead are part of other comprehensive income. Accumulated other comprehensive income is a component of Shareholders' equity

  \item A is correct. Financial assets classified as held to maturity are measured at amortised cost. Gains and losses are recognized only when realized.

  \item A is correct. Current liabilities are those liabilities, including debt, due within one year. Preferred refers to a class of stock. Convertible refers to a feature of bonds (or preferred stock) allowing the holder to convert the instrument into common stock.

  \item B is correct. The non-controlling interest in consolidated subsidiaries is shown separately as part of shareholders' equity.

  \item C is correct. The item "retained earnings" is a component of shareholders' equity.

  \item B is correct. Share repurchases reduce the company's cash (an asset). Shareholders' equity is reduced because there are fewer shares outstanding and treasury stock is an offset to owners' equity.

  \item B is correct. Vertical common-size analysis involves stating each balance sheet item as a percentage of total assets. Total assets are the sum of total liabilities ( $\pounds 35$ million) and total stockholders' equity ( $\pounds 55$ million), or $\pounds 90$ million. Total liabilities are shown on a vertical common-size balance sheet as ( $\pounds 35$ million/ $\pounds 90$ million) $\approx 39 \%$.

  \item B is correct. Common-size analysis (as presented in the reading) provides information about composition of the balance sheet and changes over time. As a result, it can provide information about an increase or decrease in a company's financial leverage.

  \item $C$ is correct. Impairment write-downs reduce equity in the denominator of the debt-to-equity ratio but do not affect debt, so the debt-to-equity ratio is expected to increase. Impairment write-downs reduce total assets but do not affect revenue. Thus, total asset turnover is expected to increase.

  \item A is correct. The current ratio provides a comparison of assets that can be turned into cash relatively quickly and liabilities that must be paid within one year. The other ratios are more suited to longer-term concerns.

  \item A is correct. The cash ratio determines how much of a company's near-term obligations can be settled with existing amounts of cash and marketable securities.

  \item C is correct. The debt-to-equity ratio, a solvency ratio, is an indicator of financial risk.

  \item B is correct. The quick ratio ([Cash + Marketable securities + Receivables] : Current liabilities) is $1.07([=€ 4,011+€ 990+€ 5,899] \div € 10,210)$. As noted in the text, the largest component of the current financial assets are loans and other financial receivables. Thus, financial assets are included in the quick ratio but not the cash ratio.

  \item $B$ is correct. The financial leverage ratio (Total assets $\div$ Total equity) is 1.66 (= $€ 42,497 \div € 25,540)$

  \item C is correct. The presence of goodwill on Company A's balance sheet signifies that it has made one or more acquisitions in the past. The current, cash, and quick ratios are lower for Company A than for the sector average. These lower liquidity ratios imply above-average liquidity risk. The total debt, long-term debt-to-equity, debt-to-equity, and financial leverage ratios are lower for Company B than for the sector average. These lower solvency ratios imply below-average solvency risk.

\end{enumerate}

Current ratio is $(35 / 35)=1.00$ for Company A, versus $(48 / 28)=1.71$ for the sector average.

Cash ratio is $(5+5) / 35=0.29$ for Company A, versus $(7+2) / 28=0.32$ for the sector average.

Quick ratio is $(5+5+5) / 35=0.43$ for Company A, versus $(7+2+12) / 28=0.75$ for the sector average.

Total debt ratio is $(55 / 100)=0.55$ for Company B, versus $(63 / 100)=0.63$ for the sector average.

Long-term debt-to-equity ratio is $(20 / 45)=0.44$ for Company B, versus $(28 / 37)=$ 0.76 for the sector average.

Debt-to-equity ratio is $(55 / 45)=1.22$ for Company B, versus $(63 / 37)=1.70$ for the sector average.

Financial leverage ratio is $(100 / 45)=2.22$ for Company B, versus $(100 / 37)=2.70$ for the sector average.

\begin{enumerate}
  \setcounter{enumi}{31}
  \item A is correct. The quick ratio is defined as (Cash and cash equivalents + Marketable securities + receivables $\div$ Current liabilities. For Company A, this calculation is $(5+5+5) / 35=0.43$.

  \item $C$ is correct. The financial leverage ratio is defined as Total assets $\div$ Total equity. For Company B, total assets are 100 and total equity is 45 ; hence, the financial leverage ratio is $100 / 45=2.22$.

  \item A is correct. The cash ratio is defined as (Cash + Marketable securities)/Current liabilities. Company A's cash ratio, $(5+5) / 35=0.29$, is higher than $(5+0) / 25=$ 0.20 for Company B. LEARNING MODULE

\end{enumerate}

\begin{center}
\includegraphics[max width=\textwidth]{2023_05_04_b5cfa4f1bc883752f121g-135}
\end{center}

\section*{Understanding Cash Flow Statements }
 J. Hennie van Greuning, DCom, CFA, and Michael A. Broihahn, CPA, CIA, CFA.Elaine Henry, PhD, CFA, is at Stevens Institute of Technology (USA). Thomas R Robinson, PhD, CFA, CAIA, Robinson Global Investment Management LLC, (USA). J. Hennie van Greuning, DCom, CFA, is at BIBD (Brunei). Michael A. Broihahn, CPA, CIA, CFA, is at Barry University (USA).

\section{LEARNING OUTCOME}
\begin{center}
\begin{tabular}{c|l}
Mastery & The candidate should be able to: \\
\hline
$\square$ & $\begin{array}{l}\text { compare cash flows from operating, investing, and financing } \\ \text { activities and classify cash flow items as relating to one of those three } \\ \text { categories given a description of the items } \\ \text { describe how non-cash investing and financing activities are } \\ \text { reported } \\ \text { contrast cash flow statements prepared under International Financial } \\ \text { Reporting Standards (IFRS) and US generally accepted accounting } \\ \text { principles (US GAAP) } \\ \text { compare and contrast the direct and indirect methods of presenting } \\ \text { cash from operating activities and describe arguments in favor of } \\ \text { each method } \\ \text { describe how the cash flow statement is linked to the income } \\ \text { statement and the balance sheet } \\ \text { describe the steps in the preparation of direct and indirect cash flow } \\ \text { statements, including how cash flows can be computed using income } \\ \text { statement and balance sheet data } \\ \text { demonstrate the conversion of cash flows from the indirect to direct } \\ \text { method } \\ \text { analyze and interpret both reported and common-size cash flow } \\ \text { statements } \\ \text { calculate and interpret free cash flow to the firm, free cash flow to } \\ \text { equity, and performance and coverage cash flow ratios }\end{array}$ \\
$\square$ &  \\
\end{tabular}
\end{center}

Note: Changes in accounting standards as well as new rulings and/or pronouncements issued after the publication of the readings on financial reporting and analysis may cause some of the information in these readings to become dated. Candidates are not responsible for anything that occurs after the readings were published. In addition, candidates are expected to be familiar with the analytical frameworks contained in the readings, as well as the implications of alternative accounting methods for financial analysis and valuation discussed in the readings. Candidates are also responsible for the content of accounting standards, but not for the actual reference numbers. Finally, candidates should be aware that certain ratios may be defined and calculated differently. When alternative ratio definitions exist and no specific definition is given, candidates should use the ratio definitions emphasized in the readings.

\section{INTRODUCTION}
The cash flow statement provides information about a company's cash receipts and cash payments during an accounting period. The cash-based information provided by the cash flow statement contrasts with the accrual-based information from the income statement. For example, the income statement reflects revenues when earned rather than when cash is collected; in contrast, the cash flow statement reflects cash receipts when collected as opposed to when the revenue was earned. A reconciliation between reported income and cash flows from operating activities provides useful information about when, whether, and how a company is able to generate cash from its operating activities. Although income is an important measure of the results of a company's activities, cash flow is also essential. As an extreme illustration, a hypothetical company that makes all sales on account, without regard to whether it will ever collect its accounts receivable, would report healthy sales on its income statement and might well report significant income; however, with zero cash inflow, the company would not survive. The cash flow statement also provides a reconciliation of the beginning and ending cash on the balance sheet.

In addition to information about cash generated (or, alternatively, cash used) in operating activities, the cash flow statement provides information about cash provided (or used) in a company's investing and financing activities. This information allows the analyst to answer such questions as:

\begin{itemize}
  \item Does the company generate enough cash from its operations to pay for its new investments, or is the company relying on new debt issuance to finance them?

  \item Does the company pay its dividends to common stockholders using cash generated from operations, from selling assets, or from issuing debt?

\end{itemize}

Answers to these questions are important because, in theory, generating cash from operations can continue indefinitely, but generating cash from selling assets, for example, is possible only as long as there are assets to sell. Similarly, generating cash from debt financing is possible only as long as lenders are willing to lend, and the lending decision depends on expectations that the company will ultimately have adequate cash to repay its obligations. In summary, information about the sources and uses of cash helps creditors, investors, and other statement users evaluate the company's liquidity, solvency, and financial flexibility.

This reading explains how cash flow activities are reflected in a company's cash flow statement. The reading is organized as follows. Section 2 describes the components and format of the cash flow statement, including the classification of cash flows under International Financial Reporting Standards (IFRS) and US generally accepted accounting principles (GAAP) and the direct and indirect formats for presenting the cash flow statement. Section 3 discusses the linkages of the cash flow statement with the income statement and balance sheet and the steps in the preparation of the cash flow statement. Section 4 demonstrates the analysis of cash flow statements, including the conversion of an indirect cash flow statement to the direct method and how to use common-size cash flow analysis, free cash flow measures, and cash flow ratios used in security analysis. A summary of the key points and practice problems in the CFA Institute multiple-choice format conclude the reading.

\section{CLASSIFICATION OF CASH FLOWS AND NON-CASH ACTIVITIES}
compare cash flows from operating, investing, and financing activities and classify cash flow items as relating to one of those three categories given a description of the items describe how non-cash investing and financing activities are reported

The analyst needs to be able to extract and interpret information on cash flows from financial statements. The basic components and allowable formats of the cash flow statement are well established.

\begin{itemize}
  \item The cash flow statement has subsections relating specific items to the operating, investing, and financing activities of the company.

  \item Two presentation formats for the operating section are allowable: direct and indirect.

\end{itemize}

The following discussion presents these topics in greater detail.

\section{Classification of Cash Flows and Non-Cash Activities}
All companies engage in operating, investing, and financing activities. These activities are the classifications used in the cash flow statement under both IFRS and US GAAP and are described as follows: ${ }^{1}$

\begin{itemize}
  \item Operating activities include the company's day-to-day activities that create revenues, such as selling inventory and providing services, and other activities not classified as investing or financing. Cash inflows result from cash sales and from collection of accounts receivable. Examples include cash receipts from the provision of services and royalties, commissions, and other revenue. To generate revenue, companies undertake such activities as manufacturing inventory, purchasing inventory from suppliers, and paying employees. Cash outflows result from cash payments for inventory, salaries, taxes, and other operating-related expenses and from paying accounts payable. Additionally, operating activities include cash receipts and payments related to dealing securities or trading securities (as opposed to buying or selling securities as investments, as discussed below).

  \item Investing activities include purchasing and selling long-term assets and other investments. These long-term assets and other investments include property, plant, and equipment; intangible assets; other long-term assets; and both long-term and short-term investments in the equity and debt (bonds and loans) issued by other companies. For this purpose, investments in equity and debt securities exclude securities held for dealing or trading purposes, the purchase and sale of which are considered operating activities even for companies where this is not a primary business activity. Cash inflows in the investing category include cash receipts from the sale of non-trading securities; property, plant, and equipment; intangibles; and other long-term assets. Cash outflows include cash payments for the purchase of these assets.

  \item Financing activities include obtaining or repaying capital, such as equity and long-term debt. The two primary sources of capital are shareholders and creditors. Cash inflows in this category include cash receipts from issuing stock (common or preferred) or bonds and cash receipts from borrowing. Cash outflows include cash payments to repurchase stock (e.g., treasury stock) and to repay bonds and other borrowings. Note that indirect borrowing using accounts payable is not considered a financing activity-such borrowing is classified as an operating activity. The new IFRS standard relating to lease accounting (IFRS 16) affects how operating leases are represented in the cash flow statement. ${ }^{2}$ Under IFRS 16, operating leases are treated similarly to finance leases - that is, the interest component of lease payments will be reflected in either the operating or financing section, and the principal component of lease payments is included in the financing section.

\end{itemize}

\section{EXAMPLE 1}
\section{Net Cash Flow from Investing Activities}
\begin{enumerate}
  \item A company recorded the following in Year 1:
\end{enumerate}

$\begin{array}{lc}\text { Proceeds from issuance of long-term debt } & € 300,000 \\ \text { Purchase of equipment } & € 200,000 \\ \text { Loss on sale of equipment } & € 70,000 \\ \text { Proceeds from sale of equipment } & € 120,000 \\ \text { Equity in earnings of affiliate } & € 10,000\end{array}$

On the Year 1 statement of cash flows, the company would report net cash flow from investing activities closest to:

A. $(€ 150,000)$.

B. $(€ 80,000)$.

C. $€ 200,000$.

Solution:

B is correct. The only two items that would affect the investing section are the purchase of equipment and the proceeds from sale of equipment: $(€ 200,000)+€ 120,000=(€ 80,000)$. The loss on sale of equipment and the equity in earnings of affiliate affect net income but are not cash flows. The issuance of debt is a financing cash flow.

IFRS provide companies with choices in reporting some items of cash flow, particularly interest and dividends. IFRS explain that although for a financial institution interest paid and received would normally be classified as operating activities, for other entities, alternative classifications may be appropriate. For this reason, under IFRS, interest received may be classified either as an operating activity or as an investing activity. Under IFRS, interest paid may be classified as either an operating activity or

2 IFRS 16 is effective for fiscal years beginning 1 January 2019, with earlier voluntary adoption allowed. a financing activity. Furthermore, under IFRS, dividends received may be classified as either an operating activity or an investing activity and dividends paid may be classified as either an operating activity or a financing activity. Companies must use a consistent classification from year to year and disclose separately the amounts of interest and dividends received and paid and where the amounts are reported.

Under US GAAP, discretion is not permitted in classifying interest and dividends. Interest received and interest paid are reported as operating activities for all companies. ${ }^{3}$ Under US GAAP, dividends received are always reported as operating activities and dividends paid are always reported as financing activities.

\section{EXAMPLE 2}
\section{Operating versus Financing Cash Flows}
\begin{enumerate}
  \item On 31 December 2018, a company issued a $\pounds 30,000180$-day note at 8 percent and used the cash received to pay for inventory and issued $\pounds 110,000$ long-term debt at 11 percent annually and used the cash received to pay for new equipment. Which of the following most accurately reflects the combined effect of both transactions on the company's cash flows for the year ended 31 December 2018 under IFRS? Cash flows from:
\end{enumerate}

A. operations are unchanged.

B. financing increase $\pounds 110,000$.

C. operations decrease $\pounds 30,000$.

\section{Solution:}
$\mathrm{C}$ is correct. The payment for inventory would decrease cash flows from operations. The issuance of debt (both short-term and long-term debt) is part of financing activities and would increase cash flows from financing activities by $\pounds 140,000$. The purchase of equipment is an investing activity. Note that the treatment under US GAAP would be the same for these transactions.

Companies may also engage in non-cash investing and financing transactions. A non-cash transaction is any transaction that does not involve an inflow or outflow of cash. For example, if a company exchanges one non-monetary asset for another non-monetary asset, no cash is involved. Similarly, no cash is involved when a company issues common stock either for dividends or in connection with conversion of a convertible bond or convertible preferred stock. Because no cash is involved in non-cash transactions (by definition), these transactions are not incorporated in the cash flow statement. However, because such transactions may affect a company's capital or asset structures, any significant non-cash transaction is required to be disclosed, either in a separate note or a supplementary schedule to the cash flow statement.

\section{CASH FLOW STATEMENT: DIFFERENCES BETWEEN IFRS AND US GAAP}
contrast cash flow statements prepared under International Financial Reporting Standards (IFRS) and US generally accepted accounting principles (US GAAP)

As highlighted in the previous section, there are some differences in cash flow statements prepared under IFRS and US GAAP that the analyst should be aware of when comparing the cash flow statements of companies prepared in accordance with different sets of standards. The key differences are summarized in Exhibit 1. Most significantly, IFRS allow more flexibility in the reporting of such items as interest paid or received and dividends paid or received and in how income tax expense is classified.

US GAAP classify interest and dividends received from investments as operating activities, whereas IFRS allow companies to classify those items as either operating or investing cash flows. Likewise, US GAAP classify interest expense as an operating activity, even though the principal amount of the debt issued is classified as a financing activity. IFRS allow companies to classify interest expense as either an operating activity or a financing activity. US GAAP classify dividends paid to stockholders as a financing activity, whereas IFRS allow companies to classify dividends paid as either an operating activity or a financing activity.

US GAAP classify all income tax expenses as an operating activity. IFRS also classify income tax expense as an operating activity, unless the tax expense can be specifically identified with an investing or financing activity (e.g., the tax effect of the sale of a discontinued operation could be classified under investing activities).

\section{Exhibit 1: Cash Flow Statements: Differences between IFRS and US GAAP}
Topic

IFRS

US GAAP

Classification of cash flows:

\begin{itemize}
  \item Interest received

  \item Interest paid

  \item Dividends received

  \item Dividends paid

  \item Bank overdrafts

  \item Taxes paid Operating or investing

\end{itemize}

Operating or financing

Operating or investing

Operating or financing

Considered part of cash equivalents

Generally operating, but a portion can be allocated to investing or financing if it can be specifically identified with these categories

Format of statement

Direct or indirect; direct is encouraged Operating

Operating

Operating

Financing

Not considered part of cash and cash equivalents and classified as financing

Operating Direct or indirect; direct is encouraged. A reconciliation of net income to cash flow from operating activities must be provided regardless of method used

Sources: IAS 7; FASB ASC Topic 230; and "IFRS and US GAAP: Similarities and Differences,"

PricewaterhouseCoopers (November 2017), available at \href{http://www.pwc.com}{www.pwc.com}. Under either set of standards, companies currently have a choice of formats for presenting cash flow statements, as discussed in the next section.

\section{CASH FLOW STATEMENT: DIRECT AND INDIRECT METHODS FOR REPORTING CASH FLOW FROM OPERATING ACTIVITIES}
compare and contrast the direct and indirect methods of presenting cash from operating activities and describe arguments in favor of each method

There are two acceptable formats for reporting cash flow from operating activities (also known as cash flow from operations or operating cash flow), defined as the net amount of cash provided from operating activities: the direct and the indirect methods. The amount of operating cash flow is identical under both methods; only the presentation format of the operating cash flow section differs. The presentation format of the cash flows from investing and financing is exactly the same, regardless of which method is used to present operating cash flows.

The direct method shows the specific cash inflows and outflows that result in reported cash flow from operating activities. It shows each cash inflow and outflow related to a company's cash receipts and disbursements. In other words, the direct method eliminates any impact of accruals and shows only cash receipts and cash payments. The primary argument in favor of the direct method is that it provides information on the specific sources of operating cash receipts and payments. This is in contrast to the indirect method, which shows only the net result of these receipts and payments. Just as information on the specific sources of revenues and expenses is more useful than knowing only the net result-net income-the analyst gets additional information from a direct-format cash flow statement. The additional information is useful in understanding historical performance and in predicting future operating cash flows.

The indirect method shows how cash flow from operations can be obtained from reported net income as the result of a series of adjustments. The indirect format begins with net income. To reconcile net income with operating cash flow, adjustments are made for non-cash items, for non-operating items, and for the net changes in operating accruals. The main argument for the indirect approach is that it shows the reasons for differences between net income and operating cash flows. (However, the differences between net income and operating cash flows are equally visible on an indirect-format cash flow statement and in the supplementary reconciliation required under US GAAP if the company uses the direct method.) Another argument for the indirect method is that it mirrors a forecasting approach that begins by forecasting future income and then derives cash flows by adjusting for changes in balance sheet accounts that occur because of the timing differences between accrual and cash accounting.

IFRS and US GAAP both encourage the use of the direct method but permit either method. US GAAP encourage the use of the direct method but also require companies to present a reconciliation between net income and cash flow (which is equivalent to the indirect method). ${ }^{4}$ If the indirect method is chosen, no direct-format disclosures are required. The majority of companies, reporting under IFRS or US GAAP, present using the indirect method for operating cash flows.

Many users of financial statements prefer the direct format, particularly analysts and commercial lenders, because of the importance of information about operating receipts and payments in assessing a company's financing needs and capacity to repay existing obligations. Preparers argue that adjusting net income to operating cash flow, as in the indirect format, is easier and less costly than reporting gross operating cash receipts and payments, as in the direct format. With advances in accounting systems and technology, it is not clear that gathering the information required to use the direct method is difficult or costly. CFA Institute has advocated that standard setters require the use of the direct format for the main presentation of the cash flow statement, with indirect cash flows as supplementary disclosure. ${ }^{5}$

\section{CASH FLOW STATEMENT: INDIRECT METHOD UNDER IFRS}
\begin{center}
\includegraphics[max width=\textwidth]{2023_05_04_b5cfa4f1bc883752f121g-142}
\end{center}

Exhibit 2 presents the consolidated cash flow statement prepared under IFRS from Unilever Group's 2017 annual report. The statement, covering the fiscal years ended 31 December 2017, 2016, and 2015, shows the use of the indirect method. Unilever is an Anglo-Dutch consumer products company with headquarters in the United Kingdom and the Netherlands. ${ }^{6}$

Exhibit 2: Unilever Group Consolidated Cash Flow Statement ( $€$ millions)

For the year ended 31 December

\begin{center}
\begin{tabular}{rrr}
\hline
2016 & 2017 \\
\hline
\end{tabular}
\end{center}

\section{Cash flow from operating activities}
\begin{center}
\begin{tabular}{|c|c|c|c|}
\hline
Net profit & 6,486 & 5,547 & 5,259 \\
\hline
Taxation & 1,667 & 1,922 & 1,961 \\
\hline
$\begin{array}{l}\text { Share of net profit of joint ventures/associates and other income (loss) } \\ \text { from non-current investments and associates }\end{array}$ & $(173)$ & (231) & (198) \\
\hline
Net finance costs: & 877 & 563 & 493 \\
\hline
\end{tabular}
\end{center}

\begin{enumerate}
  \setcounter{enumi}{3}
  \item FASB ASC Section 230-10-45 [Statement of Cash Flows-Overall-Other Presentation Matters].
\end{enumerate}

5 A Comprehensive Business Reporting Model: Financial Reporting for Investors, CFA Institute Centre for Financial Market Integrity (July 2007), p. 13.

6 Unilever NV and Unilever PLC have independent legal structures, but a series of agreements enable the companies to operate as a single economic entity.

\begin{center}
\begin{tabular}{|c|c|c|c|}
\hline
 & \multicolumn{3}{|c|}{For the year ended 31 December} \\
\hline
 & 2017 & 2016 & 2015 \\
\hline
Operating profit & 8,857 & 7,801 & 7,515 \\
\hline
Depreciation, amortisation and impairment & 1,538 & 1,464 & 1,370 \\
\hline
Changes in working capital: & $(68)$ & 51 & 720 \\
\hline
Inventories & $(104)$ & 190 & $(129)$ \\
\hline
Trade and other current receivables & (506) & 142 & 2 \\
\hline
Trade payables and other liabilities & 542 & $(281)$ & 847 \\
\hline
Pensions and similar obligations less payments & $(904)$ & $(327)$ & $(385)$ \\
\hline
Provisions less payments & 200 & 65 & $(94)$ \\
\hline
Elimination of (profits)/losses on disposals & $(298)$ & 127 & 26 \\
\hline
Non-cash charge for share-based compensation & 284 & 198 & 150 \\
\hline
Other adjustments & $(153)$ & $(81)$ & 49 \\
\hline
Cash flow from operating activities & 9,456 & 9,298 & 9,351 \\
\hline
Income tax paid & $(2,164)$ & $(2,251)$ & $(2,021)$ \\
\hline
Net cash flow from operating activities & 7,292 & 7,047 & 7,330 \\
\hline
Interest received & 154 & 105 & 119 \\
\hline
Purchase of intangible assets & $(158)$ & $(232)$ & $(334)$ \\
\hline
Purchase of property, plant and equipment & $(1,509)$ & $(1,804)$ & $(1,867)$ \\
\hline
Disposal of property, plant and equipment & 46 & 158 & 127 \\
\hline
Acquisition of group companies, joint ventures and associates & $(4,896)$ & $(1,731)$ & $(1,897)$ \\
\hline
Disposal of group companies, joint ventures and associates & 561 & 30 & 199 \\
\hline
Acquisition of other non-current investments & $(317)$ & $(208)$ & $(78)$ \\
\hline
Disposal of other non-current investments & 251 & 173 & 127 \\
\hline
$\begin{array}{l}\text { Dividends from joint ventures, associates and other non-current } \\ \text { investments }\end{array}$ & 138 & 186 & 176 \\
\hline
(Purchase)/sale of financial assets & $(149)$ & 135 & $(111)$ \\
\hline
Net cash flow (used in)/from investing activities & $(5,879)$ & $(3,188)$ & $(3,539$ \\
\hline
Dividends paid on ordinary share capital & $(3,916)$ & $(3,609)$ & $(3,331)$ \\
\hline
Interest and preference dividends paid & $(470)$ & $(472)$ & $(579)$ \\
\hline
\multirow[t]{2}{*}{Net change in short-term borrowings} & 2,695 & 258 & 245 \\
\hline
 & 8,851 & 6,761 & 7,566 \\
\hline
\multicolumn{4}{|c|}{Additional financial liabilities} \\
\hline
Repayment of financial liabilities & $(2,604)$ & $(5,213)$ & $(6,270)$ \\
\hline
Capital element of finance lease rental payments & $(14)$ & $(35)$ & $(14)$ \\
\hline
Buy back of preference shares & $(448)$ & - & - \\
\hline
Repurchase of shares & $(5,014)$ & - & - \\
\hline
Other movements on treasury stock & $(204)$ & $(257)$ & $(276)$ \\
\hline
Other financing activities & $(309)$ & $(506)$ & $(373)$ \\
\hline
Net cash flow (used in)/from financing activities & $(1,433)$ & $(3,073)$ & $(3,032)$ \\
\hline
Net increase/(decrease) in cash and cash equivalents & $(20)$ & 786 & 759 \\
\hline
Cash and cash equivalents at the beginning of the year & 3,198 & 2,128 & 1,910 \\
\hline
\end{tabular}
\end{center}

\begin{center}
\begin{tabular}{|c|c|c|c|}
\hline
 & \multicolumn{3}{|c|}{For the year ended 31 December} \\
\hline
 & 2017 & 2016 & 2015 \\
\hline
Effect of foreign exchange rate changes & $(9)$ & 284 & $(541)$ \\
\hline
Cash and cash equivalents at the end of the year & 3,169 & 3,198 & 2,128 \\
\hline
\end{tabular}
\end{center}

Beginning first at the bottom of the statement, we note that cash increased from $€ 1,910$ million at the beginning of 2015 to $€ 3,169$ million at the end of 2017, with the largest increase occurring in 2016. To understand the changes, we next examine the sections of the statement. In each year, the primary cash inflow derived from operating activities, as would be expected for a mature company in a relatively stable industry. In each year, the operating cash flow was more than the reported net profit, again, as would be expected from a mature company, with the largest differences primarily arising from the add-back of depreciation. Also, in each year, the operating cash flow was more than enough to cover the company's capital expenditures. For example, in 2017, the company generated $€ 7,292$ million in net cash from operating activities and-as shown in the investing section-spent $€ 1,509$ million on property, plant, and equipment. The operating cash flow was also sufficient to cover acquisitions of other companies.

The financing section of the statement shows that each year the company returned more than $€ 3.3$ billion to its common shareholders through dividends and around $€ 500$ million to its debt holders and preferred shareholders via interest and dividends. In 2017, the company used cash to repurchase about $€ 5$ billion in common stock in and generated cash from increased borrowing. The increase in short-term borrowings ( $€ 2,695$ million) and additional financial liabilities (€8,851 million) exceeded the cash repayment of liabilities (€2,604 million).

Having examined each section of the statement, we return to the operating activities section of Unilever's cash flow statement, which presents a reconciliation of net profit to net cash flow from operating activities (i.e., uses the indirect method). The following discussion of certain adjustments to reconcile net profit to operating cash flows explains some of the main reconciliation adjustments and refers to the amounts in 2017. The first adjustment adds back the $€ 1,667$ million income tax expense (labeled "Taxation") that had been recognized as an expense in the computation of net profit. A $€ 2,164$ million deduction for the (cash) income taxes paid is then shown separately, as the last item in the operating activities section, consistent with the IFRS requirement that cash flows arising from income taxes be separately disclosed. The classification of taxes on income paid should be indicated. The classification is in operating activities unless the taxes can be specifically identified with financing or investing activities.

The next adjustment "removes" from the operating cash flow section the $€ 173$ million representing Unilever's share of joint ventures' income that had been included in the computation of net profit. A $€ 138$ million inflow of (cash) dividends received from those joint ventures is then shown in the investing activities section. Similarly, a $€ 877$ million adjustment removes the net finance costs from the operating activities section. Unilever then reports its $€ 154$ million (cash) interest received in the investing activities section and its $€ 470$ million (cash) interest paid (and preference dividends paid) in the financing activities section. The next adjustment in the operating section of this indirect-method statement adds back $€ 1,538$ million depreciation, amortisation, and impairment, all of which are expenses that had been deducted in the computation of net income but which did not involve any outflow of cash in the period. The $€ 68$ million adjustment for changes in working capital is necessary because these changes result from applying accrual accounting and thus do not necessarily correspond to the actual cash movement. These adjustments are described in greater detail in a later section. In summary, some observations from an analysis of Unilever's cash flow statement include:

\begin{itemize}
  \item Total cash increased from $€ 1,910$ million at the beginning of 2015 to $€ 3,169$ million at the end of 2017, with the largest increase occurring in 2016.

  \item In each year, the operating cash flow was more than the reported net profit, as would generally be expected from a mature company.

  \item In each year, the operating cash flow was more than enough to cover the company's capital expenditures.

  \item The company returned cash to its equity investors through dividends in each year and through share buybacks in 2017.

\end{itemize}

\textbackslash section\{CASH FLOW STATEMENT: DIRECT METHOD UNDER IFRS

\begin{center}
\includegraphics[max width=\textwidth]{2023_05_04_b5cfa4f1bc883752f121g-145}
\end{center}

In the direct format of the cash flow statement, the cash received from customers, as well as other operating items, is clearly shown.

Exhibit 3 presents a direct-method format cash flow statement prepared under IFRS for Telefónica Group, a diversified telecommunications company based in Madrid. ${ }^{7}$

\section{Exhibit 3: Telefónica Group Consolidated Statement of Cash Flows ( $€$ millions)}
\begin{center}
\begin{tabular}{|c|c|c|c|}
\hline
for the years ended 31 December & 2017 & 2016 & 2015 \\
\hline
\multicolumn{4}{|l|}{Cash flows from operating activities} \\
\hline
Cash received from operations & 63,456 & 63,514 & 67,582 \\
\hline
Cash paid from operations & $(46,929)$ & $(47,384)$ & $(50,833)$ \\
\hline
$\begin{array}{l}\text { Net interest and other financial expenses net of dividends } \\ \text { received }\end{array}$ & $(1,726)$ & $(2,143)$ & $(2,445)$ \\
\hline
Taxes paid & $(1,005)$ & $(649)$ & $(689)$ \\
\hline
Net cash flow provided by operating activities & 13,796 & 13,338 & 13,615 \\
\hline
\multicolumn{4}{|l|}{Cash flows from investing activities} \\
\hline
$\begin{array}{l}\text { (Payments on investments)/proceeds from the sale in property, } \\ \text { plant and equipment and intangible assets, net }\end{array}$ & $(8,992)$ & $(9,187)$ & $(10,256)$ \\
\hline
\end{tabular}
\end{center}

plant and equipment and intangible assets, net

7 This statement excludes the supplemental cash flow reconciliation provided at the bottom of the original cash flow statement by the company. Proceeds on disposals of companies, net of cash and cash equivalents disposed

Payments on investments in companies, net of cash and cash equivalents acquired

Proceeds on financial investments not included under cash equivalents

Payments made on financial investments not included under cash equivalents

(Payments)/proceeds on placements of cash surpluses not included under cash equivalents

\section{Government grants received}
Net cash used in investing activities

Cash flows from financing activities

Dividends paid

\begin{center}
\begin{tabular}{|c|c|c|}
\hline
2 & - & 7 \\
\hline
$(10,245)$ & $(8,208)$ & $(12,917$ \\
\hline
$(2,459)$ & $(2,906)$ & $(2,775)$ \\
\hline
2 & - & 4,255 \\
\hline
1,269 & $(660)$ & $(1,772)$ \\
\hline
646 & 656 & 83 \\
\hline
8,390 & 5,693 & 1,602 \\
\hline
4,844 & 10,332 & 8,784 \\
\hline
$(6,687)$ & $(6,873)$ & $(3,805)$ \\
\hline
$(6,711)$ & $(8,506)$ & $(9,858)$ \\
\hline
$(1,046)$ & $(1,956)$ & $(126)$ \\
\hline
$(1,752)$ & $(4,220)$ & $(3,612)$ \\
\hline
$(341)$ & 185 & $(1,000)$ \\
\hline
(2) & 26 & - \\
\hline
1,456 & 1,121 & $(3,914)$ \\
\hline
3,736 & 2,615 & 6,529 \\
\hline
5,192 & 3,736 & 2,615 \\
\hline
\end{tabular}
\end{center}

Proceeds/(payments) of treasury shares and other operations with shareholders and with minority interests

Operations with other equity holders

Proceeds on issue of debentures and bonds, and other debts

Proceeds on loans, borrowings and promissory notes

Repayments of debentures and bonds and other debts

Repayments of loans, borrowings and promissory notes

Financed operating payments and investments in property, plant and equipment and intangible assets payments

Net cash flow used in financing activities

Effect of changes in exchange rates

Effect of changes in consolidation methods and others

Net increase (decrease) in cash and cash equivalents during the period

Cash and cash equivalents at 1 January

Cash and cash equivalents at 31 December

As shown at the bottom of the statement, cash and cash equivalents decreased from $€ 6,529$ million at the beginning of 2015 to $€ 5,192$ million at the end of 2017 . The largest decrease in cash occurred in 2015. Cash from operations was the primary source of cash, consistent with the profile of a mature company in a relatively stable industry. Each year, the company generated significantly more cash from operations than it required for its capital expenditures. For example, in 2017, the company generated $€ 13.8$ billion cash from operations and spent-as shown in the investing section-only $€ 9$ billion on property, plant, and equipment, net of proceeds from sales. Another notable item from the investing section is the company's limited acquisition activity in 2017 and 2016 compared with 2015. In 2015 , the company made over $€ 3$ billion of acquisitions. As shown in the financing section, cash flows from financing was negative in all three years, although the components of the negative cash flows differed. In 2015, for example, the company generated cash with an equity issuance of $€ 4.2$ billion but made significant net repayments of debts resulting in negative cash from financing activities. In summary, some observations from an analysis of Telefónica's cash flow statement include

\begin{itemize}
  \item Total cash and cash equivalents decreased over the three-year period, with 2015 showing the biggest decrease.

  \item Cash from operating activities was large enough in each year to cover the company's capital expenditures.

  \item The amount paid for property, plant, and equipment and intangible assets was the largest investing expenditure each year.

  \item The company had a significant amount of acquisition activity in 2015.

  \item The company paid dividends each year although the amount in 2017 is somewhat lower than in prior years.

\end{itemize}

\section{CASH FLOW STATEMENT: DIRECT METHOD UNDER US GAAP}
contrast cash flow statements prepared under International Financial Reporting Standards (IFRS) and US generally accepted accounting principles (US GAAP) compare and contrast the direct and indirect methods of presenting cash from operating activities and describe arguments in favor of each method

Previously, we presented cash flow statements prepared under IFRS. In this section, we illustrate cash flow statements prepared under US GAAP. This section presents the cash flow statements of two companies, Tech Data Corporation and Walmart. Tech Data reports its operating activities using the direct method, whereas Walmart reports its operating activities using the more common indirect method.

Tech Data Corporation is a leading distributor of information technology products. Exhibit 4 presents comparative cash flow statements from the company's annual report for the fiscal years ended 31 January 2016 through 2018.

\section{Exhibit 4: Tech Data Corporation and Subsidiaries Consolidated Cash Flow Statements (in Thousands)}
Years Ended 31 January

2018

2017

2016

Cash flows from operating activities:

Cash received from customers

\begin{center}
\begin{tabular}{cccc}
$\$ 42,981,601$ & $\$ 29,427,357$ & $\$ 28,119,687$ &  \\
$(41,666,356)$ & $(28,664,222)$ & $(27,819,886)$ &  \\
$(86,544)$ & $(22,020)$ & $(20,264)$ &  \\
$(131,632)$ &  & $(84,272)$ & $(85,645)$ \\
\hline
$\mathbf{1 , 0 9 7 , 0 6 9}$ & $\mathbf{6 5 6 , 8 4 3}$ & $\mathbf{1 9 3 , 8 9 2}$ &  \\
\hline
\end{tabular}
\end{center}

Cash flows from investing activities:

Acquisition of business, net of cash acquired

$(2,249,849)$

$(2,916)$

Expenditures for property and equipment

$(192,235)$

$(24,971)$

$(20,917)$

\begin{center}
\begin{tabular}{|c|c|c|c|}
\hline
Years Ended 31 January & 2018 & 2017 & 2016 \\
\hline
Software and software development costs & $(39,702)$ & $(14,364)$ & $(13,055)$ \\
\hline
Proceeds from sale of subsidiaries & 0 & 0 & 20,020 \\
\hline
Net cash used in investing activities & $(2,481,786)$ & $(42,251)$ & $(41,800)$ \\
\hline
\multicolumn{4}{|l|}{Cash flows from financing activities:} \\
\hline
Borrowings on long-term debt & $1,008,148$ & 998,405 & - \\
\hline
Principal payments on long-term debt & $(861,394)$ & - & $(319)$ \\
\hline
Cash paid for debt issuance costs & $(6,348)$ & $(21,581)$ & - \\
\hline
Net borrowings on revolving credit loans & $(16,028)$ & 3,417 & 5,912 \\
\hline
Cash paid for purchase of treasury stock & - & - & $(147,003)$ \\
\hline
$\begin{array}{l}\text { Payments for employee withholdings on equity } \\ \text { awards }\end{array}$ & $(6,027)$ & $(4,479)$ & $(4,662)$ \\
\hline
Proceeds from the reissuance of treasury stock & 1,543 & 733 & 561 \\
\hline
Acquisition of earn-out payments & - & - & $(2,736)$ \\
\hline
$\begin{array}{l}\text { Net cash provided by (used in) financing } \\ \text { activities }\end{array}$ & 119,894 & 976,495 & $(148,247)$ \\
\hline
$\begin{array}{l}\text { Effect of exchange rate changes on cash and cash } \\ \text { equivalents }\end{array}$ & 94,860 & 3,335 & $(15,671)$ \\
\hline
$\begin{array}{l}\text { Net (decrease) increase in cash and cash } \\ \text { equivalents }\end{array}$ & $(1,169,963)$ & $1,594,422$ & $(11,826)$ \\
\hline
Cash and cash equivalents at beginning of year & $2,125,591$ & 531,169 & 542,995 \\
\hline
Cash and cash equivalents at end of year & $\$ 955,628$ & $\$ 2,125,591$ & $\$ 531,169$ \\
\hline
\end{tabular}
\end{center}

\section{Reconciliation of net income to net cash provided by operating activities:}
Net income

$\$ 116,641$

$\$ 195,095$

$\$ 265,736$

Adjustments to reconcile net income to net cash provided by operating activities:

Depreciation and amortization

Provision for losses on accounts receivable

150,046

54,437

57,253

Stock-based compensation expense

21,022

5,026

6,061

Loss on disposal of subsidiaries

29,381

13,947

14,890

Accretion of debt discount and debt issuance

$-$

\begin{itemize}
  \item 
\end{itemize}

699 costs

Deferred income taxes

$(11,002)$

2,387

Changes in operating assets and liabilities:

Accounts receivable

Inventories

$(554,627)$

$(91,961)$

$(297,637)$

Prepaid expenses and other assets

$(502,352)$

$(20,838)$

$(219,482)$

Accounts payable

32,963

66,027

$(44,384)$

Accrued expenses and other liabilities

Total adjustments

Net cash provided by operating activities

\begin{center}
\begin{tabular}{|c|c|c|}
\hline
$1,704,307$ & 459,146 & 426,412 \\
\hline
100,623 & $(13,869)$ & $(18,882)$ \\
\hline
980,428 & 461,748 & $(71,844)$ \\
\hline
$\$ 1,097,069$ & $\$ 656,843$ & $\$ 193,892$ \\
\hline
\end{tabular}
\end{center}

Tech Data Corporation prepares its cash flow statements under the direct method. The company's cash increased from $\$ 543$ million at the beginning of 2016 to $\$ 956$ million at the end of January 2018, with the biggest increase occurring in 2017. The 2017 increase was driven by changes in both operating cash flow and financing cash flow. In the cash flows from operating activities section of Tech Data's cash flow statements, the company identifies the amount of cash it received from customers, $\$ 43$ billion for 2018, and the amount of cash that it paid to suppliers and employees, $\$ 41.7$ billion for 2018. Cash receipts increased from $\$ 29.4$ billion in the prior year and cash paid also increased substantially. Net cash provided by operating activities was adequate to cover the company's investing activities in 2016 and 2017 but not in 2018, primarily because of increased amounts of cash used for acquisition of business. Related to this investing cash outflow for an acquisition, footnotes disclose that the major acquisition in 2018 accounted for the large increase in cash receipts and cash payments in the operating section. Also related to the 2018 acquisition, the financing section shows that the company borrowed more debt that it repaid in both 2017 and 2018. In 2017, borrowings on long-term debt were $\$ 998.4$ million, and net borrowings on revolving credit loans were $\$ 3.4$ million. In 2018 , the company generated cash by borrowing more long-term debt than it repaid but used cash to pay down its revolving credit loans. There are no dividend payments, although in 2016, the company paid \$147 million to repurchase its common stock.

Whenever the direct method is used, US GAAP require a disclosure note and a schedule that reconciles net income with the net cash flow from operating activities. Tech Data shows this reconciliation at the bottom of its consolidated statements of cash flows. The disclosure note and reconciliation schedule are exactly the information that would have been presented in the body of the cash flow statement if the company had elected to use the indirect method rather than the direct method. For 2018, the reconciliation highlights an increase in the company's accounts receivable, inventory, and payables.

In summary, some observations from an analysis of Tech Data's cash flow statement include:

\begin{itemize}
  \item The company's cash increased by over $\$ 412$ million over the three years ending in January 2018, with the biggest increase occurring in 2017.

  \item The company's operating cash was adequate to cover the company's investments in 2016 and 2017, but not in 2018 primarily because of a major acquisition.

  \item Related to the 2018 acquisition, the financing section shows an increase in long-term borrowings in 2017 and 2018, including a $\$ 998$ million increase in 2017.

  \item The company has not paid dividends in the past three years, but the financing section shows that in 2016 the company repurchased stock.

\end{itemize}

\section{CASH FLOW STATEMENT: INDIRECT METHOD UNDER US GAAP}
contrast cash flow statements prepared under International Financial Reporting Standards (IFRS) and US generally accepted accounting principles (US GAAP) compare and contrast the direct and indirect methods of presenting cash from operating activities and describe arguments in favor of each method

Walmart is a global retailer that conducts business under the names of Walmart and Sam's Club. Exhibit 5 presents the comparative cash flow statements from the company's annual report for the fiscal years ended 31 January 2018, 2017, and 2016.

Exhibit 5: Walmart Cash Flow Statements Fiscal Years Ended 31 January (\$ millions)

\begin{center}
\begin{tabular}{lcc}
\hline
Fiscal Year Ended 31 January & $\mathbf{2 0 1 8}$ & $\mathbf{2 0 1 7}$ \\
\hline
Cash flows from operating activities: &  &  \\
Consolidated net income & 10,523 &  \\
\cline { 2 - 4 }
 &  &  \\
\hline
\end{tabular}
\end{center}

Adjustments to reconcile income from continuing operations to net cash provided by operating activities:

Depreciation and amortization

Deferred income taxes

Loss on extinguishment of debt

Other operating activities

Changes in certain assets and liabilities, net of effects of acquisitions:

Receivables, net

Inventories

Accounts payable

Accrued liabilities

Accrued income taxes

Net cash provided by operating activities

Cash flows from investing activities:

Payments for property and equipment

Proceeds from disposal of property and equipment

Proceeds from the disposal of certain operations

Purchase of available for sale securities

Investment and business acquisitions, net of cash acquired

Other investing activities

Net cash used in investing activities

Cash flows from financing activities:

Net change in short-term borrowings

Proceeds from issuance of long-term debt

\begin{center}
\begin{tabular}{|c|c|c|}
\hline
10,529 & 10,080 & 9,454 \\
\hline
$(304)$ & 761 & $(672)$ \\
\hline
3,136 & - & - \\
\hline
1,210 & 206 & 1,410 \\
\hline
$(1.074)$ & $(402)$ & (19) \\
\hline
$(140)$ & 1,021 & $(703)$ \\
\hline
4,086 & 3,942 & 2,008 \\
\hline
928 & 1,280 & 1,466 \\
\hline
$(557)$ & 492 & $(472)$ \\
\hline
28,337 & 31,673 & 27,552 \\
\hline
$(10,051)$ & $(10,619)$ & $(11,477)$ \\
\hline
378 & 456 & 635 \\
\hline
1,046 & 662 & 246 \\
\hline
- & $(1,901)$ & - \\
\hline
$(375)$ & $(2,463)$ & - \\
\hline
$(58)$ & $(122)$ & $(79)$ \\
\hline
$(9,060)$ & $(13,987)$ & $(10,675)$ \\
\hline
4,148 & $(1,673)$ & 1,235 \\
\hline
7,476 & 137 & 39 \\
\hline
\end{tabular}
\end{center}

\begin{center}
\begin{tabular}{|c|c|c|c|}
\hline
Fiscal Year Ended 31 January & 2018 & 2017 & 2016 \\
\hline
Payments of long-term debt & $(13,061)$ & $(2,055)$ & $(4,432)$ \\
\hline
Payment for debt extinguishment or debt prepayment cost & $(3,059)$ & - & - \\
\hline
Dividends paid & $(6,124)$ & $(6,216)$ & $(6,294)$ \\
\hline
Purchase of Company stock & $(8,296)$ & $(8,298)$ & $(4,112)$ \\
\hline
Dividends paid to noncontrolling interest & $(690)$ & $(479)$ & $(719)$ \\
\hline
Purchase of noncontrolling interest & $(8)$ & $(90)$ & $(1,326)$ \\
\hline
Other financing activities & $(261)$ & $(398)$ & $(676)$ \\
\hline
Net cash used in financing activities & $(19,875)$ & $(19,072)$ & $(16,285)$ \\
\hline
Effect of exchange rates on cash and cash equivalents & 487 & $(452)$ & $(1,022)$ \\
\hline
Net increase (decrease) in cash and cash equivalents & $(111)$ & $(1,838)$ & $(430)$ \\
\hline
Cash and cash equivalents at beginning of yea & 6,867 & 8,705 & 9,135 \\
\hline
Cash and cash equivalents at end of yea & 6,756 & 6,867 & 8,705 \\
\hline
\multicolumn{4}{|l|}{Supplemental disclosure of cash flow information} \\
\hline
Income taxes paid & 6,179 & 4,507 & 8,111 \\
\hline
Interest paid & 2,450 & 2,351 & 2,540 \\
\hline
\end{tabular}
\end{center}

Walmart's cash flow statement indicates the following:

\begin{itemize}
  \item Cash and cash equivalents declined over the three years, from $\$ 9.1$ billion at the beginning of fiscal 2016 to $\$ 6.8$ billion at the end of fiscal 2018 .

  \item Operating cash flow was relatively steady at $\$ 27.6$ billion, $\$ 31.7$ billion, and $\$ 28.3$ billion in fiscal 2016, 2017, and 2018, respectively. Further, operating cash flow was significantly greater than the company's expenditures on property and equipment in every year.

  \item Over the three years, the company used significant amounts of cash to pay dividends and to repurchase its common stock. The company also repaid borrowing, particularly in fiscal 2018.

\end{itemize}

Walmart prepares its cash flow statements under the indirect method. In the cash flows from operating activities section of Walmart's cash flow statement, the company reconciles its net income for 2018 of $\$ 10.5$ billion to net cash provided by operating activities of $\$ 28.3$ billion. The largest adjustment is for depreciation and amortization of $\$ 10.5$ billion. Depreciation and amortization expense requires an adjustment because it was a non-cash expense on the income statement. As illustrated in previous examples, depreciation is the largest or one of the largest adjustments made by many companies in the reconciliation of net income to operating cash flow.

Whenever the indirect method is used, US GAAP mandate disclosure of how much cash was paid for interest and income taxes. Note that these are line items in cash flow statements using the direct method, so disclosure does not have to be mandated. Walmart discloses the amount of cash paid for income tax ( $\$ 6.2$ billion) and interest ( $\$ 2.5$ billion) at the bottom of its cash flow statements.

\section{LINKAGES OF CASH FLOW STATEMENT WITH THE INCOME STATEMENT AND BALANCE SHEET}
describe how the cash flow statement is linked to the income statement and the balance sheet

The indirect format of the cash flow statement demonstrates that changes in balance sheet accounts are an important factor in determining cash flows. The next section addresses the linkages between the cash flow statement and other financial statements.

\section{Linkages of the Cash Flow Statement with the Income Statement and Balance Sheet}
Recall the accounting equation that summarizes the balance sheet:

$$
\text { Assets }=\text { Liabilities }+ \text { Equity }
$$

Cash is an asset. The statement of cash flows ultimately shows the change in cash during an accounting period. The beginning and ending balances of cash are shown on the company's balance sheets for the previous and current years, and the bottom of the cash flow statement reconciles beginning cash with ending cash. The relationship, stated in general terms, is as shown below.

Beginning Balance Sheet at 31 December $20 \times 8$

Beginning cash

Ending Balance Sheet at 31 December $20 \times 9$

Plus: Cash receipts (from operating, investing, and Less: Cash payments (for operating, investing, and

Ending cash financing activities)

In the case of cash held in foreign currencies, there would also be an impact from changes in exchange rates. For example, Walmart's cash flow statement for 2018, presented in Exhibit 5, shows overall cash flows from operating, investing, and financing activities that total $\$(111)$ million during the year, including $\$ 487$ million net effect of exchange rates on cash and cash equivalents.

The body of Walmart's cash flow statement shows why the change in cash occurred; in other words, it shows the company's operating, investing, and financing activities (as well as the impact of foreign currency translation). The beginning and ending balance sheet values of cash and cash equivalents are linked through the cash flow statement.

The current assets and current liabilities sections of the balance sheet typically reflect a company's operating decisions and activities. Because a company's operating activities are reported on an accrual basis in the income statement, any differences between the accrual basis and the cash basis of accounting for an operating transaction result in an increase or decrease in some (usually) short-term asset or liability on the balance sheet. For example, if revenue reported using accrual accounting is higher than the cash actually collected, the result will typically be an increase in accounts receivable. If expenses reported using accrual accounting are lower than cash actually paid, the result will typically be a decrease in accounts payable or another accrued liability account ${ }^{8}$. As an example of how items on the balance sheet are related to the income statement and/or cash flow statement through the change in the beginning and ending balances, consider accounts receivable:

\begin{center}
\begin{tabular}{llll}
\hline
Beginning Balance Sheet & $\begin{array}{l}\text { Statement of Cash Flows for } \\ \text { at } 31 \text { December 20X8 }\end{array}$ & $\begin{array}{c}\text { Income Statement for Year } \\ \text { Ended 31 December 20X9 }\end{array}$ & $\begin{array}{l}\text { Year Ended 31 December } \\ \mathbf{2 0 X 9} 9\end{array}$ \\
\hline
Beginning accounts receivable & Plus: Revenues & $\begin{array}{l}\text { Minus: Cash collected from } \\ \text { customers }\end{array}$ & $\begin{array}{l}\text { Equals: Ending accounts } \\ \text { receivable }\end{array}$ \\
\hline
\end{tabular}
\end{center}

Knowing any three of these four items makes it easy to compute the fourth. For example, if you know beginning accounts receivable, revenues, and cash collected from customers, you can compute ending accounts receivable. Understanding the interrelationships among the balance sheet, income statement, and cash flow statement is useful not only in evaluating the company's financial health but also in detecting accounting irregularities. Recall the extreme illustration of a hypothetical company that makes sales on account without regard to future collections and thus reports healthy sales and significant income on its income statement yet lacks cash inflow. Such a pattern would occur if a company improperly recognized revenue.

A company's investing activities typically relate to the long-term asset section of the balance sheet, and its financing activities typically relate to the equity and long-term debt sections of the balance sheet. The next section demonstrates the preparation of cash flow information based on income statement and balance sheet information.

\section{PREPARING THE CASH FLOW STATEMENT: THE DIRECT METHOD FOR OPERATING ACTIVITIES}
describe the steps in the preparation of direct and indirect cash flow statements, including how cash flows can be computed using income statement and balance sheet data

The preparation of the cash flow statement uses data from both the income statement and the comparative balance sheets.

As noted earlier, companies often only disclose indirect operating cash flow information, whereas analysts prefer direct-format information. Understanding how cash flow information is put together will enable you to take an indirect statement apart and reconfigure it in a more useful manner. The result is an approximation of a direct cash flow statement, which-while not perfectly accurate-can be helpful to an analyst. The following demonstration of how an approximation of a direct cash flow statement is prepared uses the income statement and the comparative balance sheets for Acme Corporation (a fictitious retail company) shown in Exhibit 6 and Exhibit 7.

8 There are other less typical explanations of the differences. For example, if revenue reported using accrual accounting is higher than the cash actually collected, it is possible that it is the result of a decrease in an unearned revenue liability account. If expenses reported using accrual accounting are lower than cash actually paid, it is possible that it is the result of an increase in prepaid expenses, inventory, or another asset account.

\section{Exhibit 6: Acme Corporation Income Statement Year Ended 31 December}
 2018\begin{center}
\begin{tabular}{lcc}
Revenue (net) & $\$ 23,598$ &  \\
Cost of goods sold & 11,456 &  \\
Gross profit & $\$ 4,123$ & 12,142 \\
Salary and wage expense & 1,052 &  \\
Depreciation expense & 3,577 & 8,752 \\
Other operating expenses &  & 3,390 \\
Total operating expenses & 205 &  \\
Operating profit & $(246)$ & $(41)$ \\
Other revenues (expenses): &  & 3,349 \\
Gain on sale of equipment &  & 1,139 \\
\hline
Interest expense &  & $\$ 2,210$ \\
Income before tax &  &  \\
Income tax expense &  &  \\
Net income &  &  \\
\hline
\end{tabular}
\end{center}

Exhibit 7: Acme Corporation Comparative Balance Sheets 31 December 2018 and 2017

\begin{center}
\begin{tabular}{|c|c|c|c|}
\hline
 & 2018 & 2017 & Net Change \\
\hline
Cash & $\$ 1,011$ & $\$ 1,163$ & $\$(152)$ \\
\hline
Accounts receivable & 1,012 & 957 & 55 \\
\hline
Inventory & 3,984 & 3,277 & 707 \\
\hline
Prepaid expenses & 155 & 178 & $(23)$ \\
\hline
Total current assets & 6,162 & 5,575 & 587 \\
\hline
Land & 510 & 510 & - \\
\hline
Buildings & 3,680 & 3,680 & - \\
\hline
Equipment* & 8,798 & 8,555 & 243 \\
\hline
Less: accumulated depreciation & $(3,443)$ & $(2,891)$ & $(552)$ \\
\hline
Total long-term assets & 9,545 & 9,854 & $(309)$ \\
\hline
Total assets & $\$ 15,707$ & $\$ 15,429$ & $\$ 278$ \\
\hline
Accounts payable & $\$ 3,588$ & $\$ 3,325$ & $\$ 263$ \\
\hline
Salary and wage payable & 85 & 75 & 10 \\
\hline
Interest payable & 62 & 74 & $(12)$ \\
\hline
Income tax payable & 55 & 50 & 5 \\
\hline
Other accrued liabilities & 1,126 & 1,104 & 22 \\
\hline
Total current liabilities & 4,916 & 4,628 & 288 \\
\hline
Long-term debt & 3,075 & 3,575 & $(500)$ \\
\hline
Common stock & 3,750 & 4,350 & $(600)$ \\
\hline
Retained earnings & 3,966 & 2,876 & 1,090 \\
\hline
Total liabilities and equity & $\$ 15,707$ & $\$ 15,429$ & $\$ 278$ \\
\hline
\end{tabular}
\end{center}

*During 2018, Acme purchased new equipment for a total cost of $\$ 1,300$. No items impacted retained earnings other than net income and dividends.

The first step in preparing the cash flow statement is to determine the total cash flows from operating activities. The direct method of presenting cash from operating activities is illustrated in sections 3.2.1 through 3.2.4. Section 3.2.5 illustrates the indirect method of presenting cash flows from operating activities. Cash flows from investing activities and from financing activities are identical regardless of whether the direct or indirect method is used to present operating cash flows.

\section{Operating Activities: Direct Method}
We first determine how much cash Acme received from its customers, followed by how much cash was paid to suppliers and to employees as well as how much cash was paid for other operating expenses, interest, and income taxes.

\section{Cash Received from Customers}
The income statement for Acme reported revenue of $\$ 23,598$ for the year ended 31 December 2018. To determine the approximate cash receipts from its customers, it is necessary to adjust this revenue amount by the net change in accounts receivable for the year. If accounts receivable increase during the year, revenue on an accrual basis is higher than cash receipts from customers, and vice versa. For Acme Corporation, accounts receivable increased by $\$ 55$, so cash received from customers was $\$ 23,543$, as follows:

Revenue

$\$ 23,598$

Less: Increase in accounts receivable

Cash received from customers

$(55)$

$\$ 23,543$

Cash received from customers affects the accounts receivable account as follows:

Beginning accounts receivable

957

Plus revenue

23,598

Minus cash collected from customers

$(23,543)$

Ending accounts receivable

$\$ 1,012$

The accounts receivable account information can also be presented as follows:

Beginning accounts receivable

$\$ 957$

Plus revenue

23,598

Minus ending accounts receivable

Cash collected from customers

$(1,012)$

$\$ 23,543$

\section{EXAMPLE 3}
\section{Computing Cash Received from Customers}
\begin{enumerate}
  \item Blue Bayou, a fictitious advertising company, reported revenues of $\$ 50$ million, total expenses of $\$ 35$ million, and net income of $\$ 15$ million in the most recent year. If accounts receivable decreased by $\$ 12$ million, how much cash did the company receive from customers?
A. $\$ 38$ million.
B. $\$ 50$ million. C. $\$ 62$ million.
\end{enumerate}

\section{Solution:}
$\mathrm{C}$ is correct. Revenues of $\$ 50$ million plus the decrease in accounts receivable of $\$ 12$ million equals $\$ 62$ million cash received from customers. The decrease in accounts receivable means that the company received more in cash than the amount of revenue it reported.

"Cash received from customers" is sometimes referred to as "cash collections from customers" or "cash collections."

\section{Cash Paid to Suppliers}
For Acme, the cash paid to suppliers was $\$ 11,900$, determined as follows:

Cost of goods sold

Plus: Increase in inventory

Equals purchases from suppliers

Less: Increase in accounts payable

Cash paid to suppliers

\begin{center}
\begin{tabular}{c}
$\$ 11,456$ \\
707 \\
\hline
$\$ 12,163$ \\
$(263)$ \\
\hline
$\$ \mathbf{1 1 , 9 0 0}$ \\
\hline
\end{tabular}
\end{center}

There are two pieces to this calculation: the amount of inventory purchased and the amount paid for it. To determine purchases from suppliers, cost of goods sold is adjusted for the change in inventory. If inventory increased during the year, then purchases during the year exceeded cost of goods sold, and vice versa. Acme reported cost of goods sold of $\$ 11,456$ for the year ended 31 December 2018. For Acme Corporation, inventory increased by $\$ 707$, so purchases from suppliers was $\$ 12,163$. Purchases from suppliers affect the inventory account, as shown below:

$\begin{array}{lc}\text { Beginning inventory } & \$ 3,277 \\ \text { Plus purchases } & 12,163 \\ \text { Minus cost of goods sold } & (11,456) \\ \text { Ending inventory } & \$ 3,984\end{array}$

Acme purchased $\$ 12,163$ of inventory from suppliers in 2018, but is this the amount of cash that Acme paid to its suppliers during the year? Not necessarily. Acme may not have yet paid for all of these purchases and may yet owe for some of the purchases made this year. In other words, Acme may have paid less cash to its suppliers than the amount of this year's purchases, in which case Acme's liability (accounts payable) will have increased by the difference. Alternatively, Acme may have paid even more to its suppliers than the amount of this year's purchases, in which case Acme's accounts payable will have decreased.

Therefore, once purchases have been determined, cash paid to suppliers can be calculated by adjusting purchases for the change in accounts payable. If the company made all purchases with cash, then accounts payable would not change and cash outflows would equal purchases. If accounts payable increased during the year, then purchases on an accrual basis would be higher than they would be on a cash basis, and vice versa. In this example, Acme made more purchases than it paid in cash, so the balance in accounts payable increased. For Acme, the cash paid to suppliers was $\$ 11,900$, determined as follows:

\section{Purchases from suppliers}
Less: Increase in accounts payable

Cash paid to suppliers $\$ 12,163$

(263)

$\$ 11,900$ The amount of cash paid to suppliers is reflected in the accounts payable account, as shown below:

Beginning accounts payable $\quad \$ 3,325$

$\begin{array}{lr}\text { Plus purchases } & 12,163\end{array}$

Minus cash paid to suppliers

Ending accounts payable

$(11,900)$

\section{EXAMPLE 4}
\section{Computing Cash Paid to Suppliers}
\begin{enumerate}
  \item Orange Beverages Plc., a fictitious manufacturer of tropical drinks, reported cost of goods sold for the year of $\$ 100$ million. Total assets increased by $\$ 55$ million, but inventory declined by $\$ 6$ million. Total liabilities increased by $\$ 45$ million, but accounts payable decreased by $\$ 2$ million. How much cash did the company pay to its suppliers during the year?
A. $\$ 96$ million.
B. $\$ 104$ million.
C. $\$ 108$ million.
\end{enumerate}

\section{Solution:}
A is correct. Cost of goods sold of $\$ 100$ million less the decrease in inventory of $\$ 6$ million equals purchases from suppliers of $\$ 94$ million. The decrease in accounts payable of $\$ 2$ million means that the company paid $\$ 96$ million in cash (\$94 million plus $\$ 2$ million).

\section{Cash Paid to Employees}
To determine the cash paid to employees, it is necessary to adjust salary and wages expense by the net change in salary and wages payable for the year. If salary and wages payable increased during the year, then salary and wages expense on an accrual basis would be higher than the amount of cash paid for this expense, and vice versa. For Acme, salary and wages payable increased by $\$ 10$, so cash paid for salary and wages was $\$ 4,113$, as follows:

Salary and wages expense $\quad \$ 4,123$

Less: Increase in salary and wages payable

Cash paid to employees

$(10)$

$\$ 4,113$

The amount of cash paid to employees is reflected in the salary and wages payable account, as shown below:

\begin{center}
\includegraphics[max width=\textwidth]{2023_05_04_b5cfa4f1bc883752f121g-157}
\end{center}

Plus salary and wages expense $\quad 4,123$

Minus cash paid to employees

$(4,113)$

Ending salary and wages payable

$\$ 85$

\section{Cash Paid for Other Operating Expenses}
To determine the cash paid for other operating expenses, it is necessary to adjust the other operating expenses amount on the income statement by the net changes in prepaid expenses and accrued expense liabilities for the year. If prepaid expenses increased during the year, other operating expenses on a cash basis would be higher than on an accrual basis, and vice versa. Likewise, if accrued expense liabilities increased during the year, other operating expenses on a cash basis would be lower than on an accrual basis, and vice versa. For Acme Corporation, the amount of cash paid for operating expenses in 2018 was $\$ 3,532$, as follows:

$\begin{array}{lc}\text { Other operating expenses } & \$ 3,577 \\ \text { Less: Decrease in prepaid expenses } & (23) \\ \text { Less: Increase in other accrued liabilities } & (22) \\ \text { Cash paid for other operating expenses } & \mathbf{\$ 3 , 5 3 2}\end{array}$

\section{EXAMPLE 5}
\section{Computing Cash Paid for Other Operating Expenses}
\begin{enumerate}
  \item Black Ice, a fictitious sportswear manufacturer, reported other operating expenses of $\$ 30$ million. Prepaid insurance expense increased by $\$ 4$ million, and accrued utilities payable decreased by $\$ 7$ million. Insurance and utilities are the only two components of other operating expenses. How much cash did the company pay in other operating expenses?
A. $\$ 19$ million.
B. $\$ 33$ million.
C. $\$ 41$ million.
\end{enumerate}

\section{Solution:}
$C$ is correct. Other operating expenses of $\$ 30$ million plus the increase in prepaid insurance expense of $\$ 4$ million plus the decrease in accrued utilities payable of $\$ 7$ million equals $\$ 41$ million.

\section{Cash Paid for Interest}
The cash paid for interest is included in operating cash flows under US GAAP and may be included in operating or financing cash flows under IFRS. To determine the cash paid for interest, it is necessary to adjust interest expense by the net change in interest payable for the year. If interest payable increases during the year, then interest expense on an accrual basis will be higher than the amount of cash paid for interest, and vice versa. For Acme Corporation, interest payable decreased by $\$ 12$, and cash paid for interest was $\$ 258$, as follows:

\begin{center}
\begin{tabular}{lc}
Interest expense & $\$ 246$ \\
Plus: Decrease in interest payable & 12 \\
\cline { 2 - 2 }
Cash paid for interest & $\mathbf{\$ 2 5 8}$ \\
\cline { 2 - 2 }
\end{tabular}
\end{center}

Alternatively, cash paid for interest may also be determined by an analysis of the interest payable account, as shown below: Beginning interest payable $\quad \$ 74$

Plus interest expense 246

Minus cash paid for interest
Ending interest payable

\section{Cash Paid for Income Taxes}
To determine the cash paid for income taxes, it is necessary to adjust the income tax expense amount on the income statement by the net changes in taxes receivable, taxes payable, and deferred income taxes for the year. If taxes receivable or deferred tax assets increase during the year, income taxes on a cash basis will be higher than on an accrual basis, and vice versa. Likewise, if taxes payable or deferred tax liabilities increase during the year, income tax expense on a cash basis will be lower than on an accrual basis, and vice versa. For Acme Corporation, the amount of cash paid for income taxes in 2018 was $\$ 1,134$, as follows:

Income tax expense

Less: Increase in income tax payable

Cash paid for income taxes

\begin{center}
\begin{tabular}{c}
$\$ 1,139$ \\
$(5)$ \\
\hline
$\mathbf{\$ 1 , 1 3 4}$ \\
\hline
\end{tabular}
\end{center}

PREPARING THE CASH FLOW STATEMENT: INVESTING ACTIVITIES

describe the steps in the preparation of direct and indirect cash flow statements, including how cash flows can be computed using income statement and balance sheet data

The second and third steps in preparing the cash flow statement are to determine the total cash flows from investing activities and from financing activities. The presentation of this information is identical, regardless of whether the direct or indirect method is used for operating cash flows.

Purchases and sales of equipment were the only investing activities undertaken by Acme in 2018, as evidenced by the fact that the amounts reported for land and buildings were unchanged during the year. An informational note in Exhibit 7 tells us that Acme purchased new equipment in 2018 for a total cost of $\$ 1,300$. However, the amount of equipment shown on Acme's balance sheet increased by only $\$ 243$ (ending balance of $\$ 8,798$ minus beginning balance of $\$ 8,555$ ); therefore, Acme must have also sold or otherwise disposed of some equipment during the year. To determine the cash inflow from the sale of equipment, we analyze the equipment and accumulated depreciation accounts as well as the gain on the sale of equipment from Exhibits 6 and 7. Assuming that the entire accumulated depreciation is related to equipment, the cash received from sale of equipment is determined as follows.

The historical cost of the equipment sold was $\$ 1,057$. This amount is determined as follows: Beginning balance equipment (from balance sheet)

Plus equipment purchased (from informational note)

Minus ending balance equipment (from balance sheet)

$(8,798)$

Equals historical cost of equipment sold

$\$ 1,057$

The accumulated depreciation on the equipment sold was $\$ 500$, determined as follows:

Beginning balance accumulated depreciation (from balance sheet) $\quad \$ 2,891$

Plus depreciation expense (from income statement) $\quad 1,052$

$\begin{array}{lr}\text { Minus ending balance accumulated depreciation (from balance sheet) } & (3,443) \\ \text { Equals accumulated depreciation on equipment sold } & \$ 500\end{array}$

The historical cost information, accumulated depreciation information, and information from the income statement about the gain on the sale of equipment can be used to determine the cash received from the sale.

Historical cost of equipment sold (calculated above) $\quad \$ 1,057$

Less accumulated depreciation on equipment sold (calculated above)

$\begin{array}{ll}\text { Equals book value of equipment sold } & \$ 557\end{array}$

Plus gain on sale of equipment (from the income statement)

Equals cash received from sale of equipment

205

\section{EXAMPLE 6}
\section{Computing Cash Received from the Sale of Equipment}
\begin{enumerate}
  \item Copper, Inc., a fictitious brewery and restaurant chain, reported a gain on the sale of equipment of $\$ 12$ million. In addition, the company's income statement shows depreciation expense of $\$ 8$ million and the cash flow statement shows capital expenditure of $\$ 15$ million, all of which was for the purchase of new equipment.
\end{enumerate}

\begin{center}
\begin{tabular}{llll}
\hline
Balance sheet item & $\mathbf{1 2 / 3 1 / 2 0 1 7}$ & $\mathbf{1 2 / 3 1 / 2 0 1 8}$ & Change \\
\hline
Equipment & $\$ 100$ million & $\$ 109$ million & $\$ 9$ million \\
$\begin{array}{l}\text { Accumulated } \\ \text { depreciation-equipment }\end{array}$ & $\$ 30$ million & $\$ 36$ million & $\$ 6$ million \\
\hline
\end{tabular}
\end{center}

Using the above information from the comparative balance sheets, how much cash did the company receive from the equipment sale?

A. $\$ 12$ million.

B. $\$ 16$ million.

C. $\$ 18$ million.

Solution:

B is correct. Selling price (cash inflow) minus book value equals gain or loss on sale; therefore, gain or loss on sale plus book value equals selling price (cash inflow). The amount of gain is given, $\$ 12$ million. To calculate the book value of the equipment sold, find the historical cost of the equipment and the accumulated depreciation on the equipment.

\begin{itemize}
  \item Beginning balance of equipment of $\$ 100$ million plus equipment purchased of $\$ 15$ million minus ending balance of equipment of $\$ 109$ million equals historical cost of equipment sold, or $\$ 6$ million.

  \item Beginning accumulated depreciation on equipment of $\$ 30$ million plus depreciation expense for the year of $\$ 8$ million minus ending balance of accumulated depreciation of $\$ 36$ million equals accumulated depreciation on the equipment sold, or $\$ 2$ million.

  \item Therefore, the book value of the equipment sold was $\$ 6$ million minus $\$ 2$ million, or $\$ 4$ million.

  \item Because the gain on the sale of equipment was $\$ 12$ million, the amount of cash received must have been $\$ 16$ million.

\end{itemize}

PREPARING THE CASH FLOW STATEMENT: FINANCING ACTIVITIES

describe the steps in the preparation of direct and indirect cash flow statements, including how cash flows can be computed using income statement and balance sheet data

As with investing activities, the presentation of financing activities is identical, regardless of whether the direct or indirect method is used for operating cash flows.

\section{Long-Term Debt and Common Stock}
The change in long-term debt, based on the beginning 2018 (ending 2017) and ending 2018 balances in Exhibit 7, was a decrease of $\$ 500$. Absent other information, this indicates that Acme retired $\$ 500$ of long-term debt. Retiring long-term debt is a cash outflow relating to financing activities.

Similarly, the change in common stock during 2018 was a decrease of $\$ 600$. Absent other information, this indicates that Acme repurchased $\$ 600$ of its common stock. Repurchase of common stock is also a cash outflow related to financing activity.

\section{Dividends}
Recall the following relationship:

Beginning retained earnings + Net income - Dividends $=$ Ending retained earnings

Based on this relationship, the amount of cash dividends paid in 2018 can be determined from an analysis of retained earnings, as follows: Beginning balance of retained earnings (from the balance sheet) $\quad \$ 2,876$

Plus net income (from the income statement) $\quad 2,210$

Minus ending balance of retained earnings (from the balance sheet) $\quad \frac{(3,966)}{(3)}$

\begin{center}
\includegraphics[max width=\textwidth]{2023_05_04_b5cfa4f1bc883752f121g-162(1)}
\end{center}

Note that dividends paid are presented in the statement of changes in equity.

PREPARING THE CASH FLOW STATEMENT: OVERALL STATEMENT OF CASH FLOWS UNDER THE DIRECT METHOD

describe the steps in the preparation of direct and indirect cash flow statements, including how cash flows can be computed using income statement and balance sheet data

Exhibit 8 summarizes the information about Acme's operating, investing, and financing cash flows in the statement of cash flows. At the bottom of the statement, the total net change in cash is shown to be a decrease of $\$ 152$ (from $\$ 1,163$ to $\$ 1,011$ ). This decrease can also be seen on the comparative balance sheet in Exhibit 7. The cash provided by operating activities of $\$ 2,606$ was adequate to cover the net cash used in investing activities of $\$ 538$; however, the company's debt repayments, cash payments for dividends, and repurchase of common stock (i.e., its financing activities) of $\$ 2,220$ resulted in an overall decrease in cash of $\$ 152$.

Exhibit 8: Acme Corporation Cash Flow Statement (Direct Method) for Year Ended 31 December 2018

Cash flow from operating activities:

$\begin{array}{lr}\text { Cash received from customers } & \$ 23,543\end{array}$

$\begin{array}{ll}\text { Cash paid to suppliers } & (11,900)\end{array}$

Cash paid to employees $\quad(4,113)$

\begin{center}
\includegraphics[max width=\textwidth]{2023_05_04_b5cfa4f1bc883752f121g-162}
\end{center}

$\begin{array}{ll}\text { Cash paid for interest } & \text { (258) }\end{array}$

Cash paid for income tax

Net cash provided by operating activities

\begin{center}
\begin{tabular}{c}
$(1,134)$ \\
\hline
2,606 \\
\hline
\end{tabular}
\end{center}

Cash flow from investing activities:

Cash received from sale of equipment 762

Cash paid for purchase of equipment

Net cash used for investing activities

\begin{center}
\begin{tabular}{c}
$(1,300)$ \\
\hline
$(538)$ \\
\hline
\end{tabular}
\end{center}

Cash flow from financing activities:

Cash paid to retire long-term debt

Cash paid to retire common stock Cash paid for dividends

Net cash used for financing activities

Net increase (decrease) in cash

\begin{center}
\begin{tabular}{c}
$(1,120)$ \\
\hline
$(2,220)$ \\
\hline
$(152)$ \\
1,163 \\
\hline
$\$ 1,011$ \\
\hline
\end{tabular}
\end{center}

PREPARING THE CASH FLOW STATEMENT: OVERALL STATEMENT OF CASH FLOWS UNDER THE INDIRECT

METHOD

describe the steps in the preparation of direct and indirect cash flow statements, including how cash flows can be computed using income statement and balance sheet data

Using the alternative approach to reporting cash from operating activities, the indirect method, we will present the same amount of cash provided by operating activities. Under this approach, we reconcile Acme's net income of $\$ 2,210$ to its operating cash flow of $\$ 2,606$.

To perform this reconciliation, net income is adjusted for the following: a) any non-operating activities, b) any non-cash expenses, and c) changes in operating working capital items.

The only non-operating activity in Acme's income statement, the sale of equipment, resulted in a gain of $\$ 205$. This amount is removed from the operating cash flow section; the cash effects of the sale are shown in the investing section.

Acme's only non-cash expense was depreciation expense of $\$ 1,052$. Under the indirect method, depreciation expense must be added back to net income because it was a non-cash deduction in the calculation of net income.

Changes in working capital accounts include increases and decreases in the current operating asset and liability accounts. The changes in these accounts arise from applying accrual accounting; that is, recognizing revenues when they are earned and expenses when they are incurred instead of when the cash is received or paid. To make the working capital adjustments under the indirect method, any increase in a current operating asset account is subtracted from net income and a net decrease is added to net income. As described above, the increase in accounts receivable, for example, resulted from Acme recording income statement revenue higher than the amount of cash received from customers; therefore, to reconcile back to operating cash flow, that increase in accounts receivable must be deducted from net income. For current operating liabilities, a net increase is added to net income and a net decrease is subtracted from net income. As described above, the increase in wages payable, for example, resulted from Acme recording income statement expenses higher than the amount of cash paid to employees.

Exhibit 9 presents a tabulation of the most common types of adjustments that are made to net income when using the indirect method to determine net cash flow from operating activities.

\section{Exhibit 9: Adjustments to Net Income Using the Indirect Method}
Additions - Non-cash items

\begin{itemize}
  \item Depreciation expense of tangible assets

  \item Amortisation expense of intangible assets

  \item Depletion expense of natural resources

  \item Amortisation of bond discount

  \item Non-operating losses

  \item Loss on sale or write-down of assets

  \item Loss on retirement of debt

  \item Loss on investments accounted for under the equity method

  \item Increase in deferred income tax liability

  \item Changes in working capital resulting from accruing higher amounts for expenses than the amounts of cash payments or lower amounts for revenues than the amounts of cash receipts

  \item Decrease in current operating assets (e.g., accounts receivable, inventory, and prepaid expenses)

  \item Increase in current operating liabilities (e.g., accounts payable and accrued expense liabilities)

  \item Non-cash items (e.g., amortisation of bond premium)

  \item Non-operating items

  \item Gain on sale of assets

  \item Gain on retirement of debt

  \item Income on investments accounted for under the equity method

  \item Decrease in deferred income tax liability

  \item Changes in working capital resulting from accruing lower amounts for expenses than for cash payments or higher amounts for revenues than for cash receipts

  \item Increase in current operating assets (e.g., accounts receivable, inventory, and prepaid expenses)

  \item Decrease in current operating liabilities (e.g., accounts payable and accrued expense liabilities)

\end{itemize}

Accordingly, for Acme Corporation, the $\$ 55$ increase in accounts receivable and the $\$ 707$ increase in inventory are subtracted from net income and the $\$ 23$ decrease in prepaid expenses is added to net income. For Acme's current liabilities, the increases in accounts payable, salary and wage payable, income tax payable, and other accrued liabilities ( $\$ 263, \$ 10, \$ 5$, and $\$ 22$, respectively) are added to net income and the $\$ 12$ decrease in interest payable is subtracted from net income. Exhibit 10 presents the cash flow statement for Acme Corporation under the indirect method by using the information that we have determined from our analysis of the income statement and the comparative balance sheets. Note that the investing and financing sections are identical to the statement of cash flows prepared using the direct method. Exhibit 10: Acme Corporation Cash Flow Statement (Indirect Method) Year Ended 31 December 2018

Cash flow from operating activities:

Net income $\quad \$ 2,210$

$\begin{array}{lr}\text { Depreciation expense } & 1,052\end{array}$

$\begin{array}{ll}\text { Gain on sale of equipment } & \text { (205) }\end{array}$

Increase in accounts receivable

$\begin{array}{ll}\text { Increase in inventory } & \text { (707) }\end{array}$

$\begin{array}{ll}\text { Decrease in prepaid expenses } & 23\end{array}$

$\begin{array}{ll}\text { Increase in accounts payable } & 263\end{array}$

$\begin{array}{lr}\text { Increase in salary and wage payable } & 10\end{array}$

Decrease in interest payable $\quad$ (12)

$\begin{array}{ll}\text { Increase in income tax payable } & 5\end{array}$

\begin{center}
\includegraphics[max width=\textwidth]{2023_05_04_b5cfa4f1bc883752f121g-165}
\end{center}

Net cash provided by operating activities

2,606

Cash flow from investing activities:

$\begin{array}{ll}\text { Cash received from sale of equipment } & 762\end{array}$

Cash paid for purchase of equipment

Net cash used for investing activities

$(1,300)$

\begin{center}
\includegraphics[max width=\textwidth]{2023_05_04_b5cfa4f1bc883752f121g-165(1)}
\end{center}

Cash flow from financing activities:

Cash paid to retire long-term debt

Cash paid to retire common stock

Cash paid for dividends

Net cash used for financing activities

Net decrease in cash

$(1,120)$

Cash balance, 31 December 2017

Cash balance, 31 December 2018

$(152)$

\begin{center}
\includegraphics[max width=\textwidth]{2023_05_04_b5cfa4f1bc883752f121g-165(2)}
\end{center}

$\$ 1,011$

\section{EXAMPLE 7}
\section{Adjusting Net Income to Compute Operating Cash Flow}
\begin{enumerate}
  \item Based on the following information for Pinkerly Inc., a fictitious company, what are the total adjustments that the company would make to net income in order to derive operating cash flow?
\end{enumerate}

\begin{center}
\includegraphics[max width=\textwidth]{2023_05_04_b5cfa4f1bc883752f121g-165(3)}
\end{center}

A. Add $\$ 5$ million.

B. Add $\$ 21$ million.

C. Subtract $\$ 9$ million.

\section{Solution:}
A is correct. To derive operating cash flow, the company would make the following adjustments to net income: add depreciation (a non-cash expense) of $\$ 7$ million; add the decrease in inventory of $\$ 3$ million; add the increase in accounts payable of $\$ 10$ million; and subtract the increase in accounts receivable of $\$ 15$ million. Total additions of $\$ 20$ million and total subtractions of $\$ 15$ million result in net total additions of $\$ 5$ million.

CONVERSION OF CASH FLOWS FROM THE INDIRECT TO DIRECT METHOD

demonstrate the conversion of cash flows from the indirect to direct method

An analyst may desire to review direct-format operating cash flow to review trends in cash receipts and payments (such as cash received from customers or cash paid to suppliers). If a direct-format statement is not available, cash flows from operating activities reported under the indirect method can be converted to the direct method. Accuracy of conversion depends on adjustments using data available in published financial reports. The method described here is sufficiently accurate for most analytical purposes.

The three-step conversion process is demonstrated for Acme Corporation in Exhibit 11. Referring again to Exhibits 6 and 7 for Acme Corporation's income statement and balance sheet information, begin by disaggregating net income of $\$ 2,210$ into total revenues and total expenses (Step 1). Next, remove any non-operating and non-cash items (Step 2). For Acme, we therefore remove the non-operating gain on the sale of equipment of $\$ 205$ and the non-cash depreciation expense of $\$ 1,052$. Then, convert accrual amounts of revenues and expenses to cash flow amounts of receipts and payments by adjusting for changes in working capital accounts (Step 3). The results of these adjustments are the items of information for the direct format of operating cash flows. These line items are shown as the results of Step 3.

Exhibit 11: Conversion from the Indirect to the Direct Method

Step 1

Aggregate all revenue and all expenses Total revenues

Total expenses

Net income $\$ 23,803$

21,593

$\$ 2,210$ Remove all noncash items from aggre-

$(\$ 23,803-\$ 205)=$

$\$ 23,598$

gated revenues and expenses and break out remaining items into relevant cash flow items

Total expenses less noncash item expenses:

Revenue $\$ 23,598$

$(\$ 21,593-\$ 1,052)=$

$\$ 20,541$

Cost of goods sold

$\$ 11,456$

Salary and wage expenses

4,123

Other operating expenses

3,577

Interest expense 246

Income tax expense

Total

1,139

$\$ 20,541$

Step 3 Cash received from customers ${ }^{\mathrm{a}}$

$\$ 23,543$

Convert accrual amounts to cash flow

Cash paid to suppliers ${ }^{b}$

$(11,900)$

amounts by adjusting for working capital Cash paid to employees ${ }^{\mathrm{c}}$

changes

Cash paid for other operating expenses ${ }^{\mathrm{d}}$

Cash paid for interest ${ }^{\mathrm{e}}$

Cash paid for income $\operatorname{tax}^{\mathrm{f}}$

$(1,134)$

Net cash provided by operating activities

$\$ 2,606$

Calculations for Step 3:

a Revenue of $\$ 23,598$ less increase in accounts receivable of $\$ 55$.

b Cost of goods sold of $\$ 11,456$ plus increase in inventory of $\$ 707$ less increase in accounts payable of $\$ 263$.

c Salary and wage expense of \$4,123 less increase in salary and wage payable of \$10.

d Other operating expenses of \$3,577 less decrease in prepaid expenses of $\$ 23$ less increase in other accrued liabilities of $\$ 22$.

e Interest expense of $\$ 246$ plus decrease in interest payable of $\$ 12$.

${ }^{\mathrm{f}}$ Income tax expense of $\$ 1,139$ less increase in income tax payable of $\$ 5$

CASH FLOW STATEMENT ANALYSIS: EVALUATION OF SOURCES AND USES OF CASH

The analysis of a company's cash flows can provide useful information for understanding a company's business and earnings and for predicting its future cash flows. This section describes tools and techniques for analyzing the statement of cash flows, including the analysis of sources and uses of cash and cash flow, common-size analysis, and calculation of free cash flow measures and cash flow ratios.

\section{Evaluation of the Sources and Uses of Cash}
Evaluation of the cash flow statement should involve an overall assessment of the sources and uses of cash between the three main categories as well as an assessment of the main drivers of cash flow within each category, as follows:

\begin{enumerate}
  \item Evaluate where the major sources and uses of cash flow are between operating, investing, and financing activities.

  \item Evaluate the primary determinants of operating cash flow.

  \item Evaluate the primary determinants of investing cash flow.

  \item Evaluate the primary determinants of financing cash flow.

\end{enumerate}

\section{Step 1}
The major sources of cash for a company can vary with its stage of growth. For a mature company, it is expected and desirable that operating activities are the primary source of cash flows. Over the long term, a company must generate cash from its operating activities. If operating cash flow were consistently negative, a company would need to borrow money or issue stock (financing activities) to fund the shortfall. Eventually, these providers of capital need to be repaid from operations or they will no longer be willing to provide capital. Cash generated from operating activities can be used in either investing or financing activities. If the company has good opportunities to grow the business or other investment opportunities, it is desirable to use the cash in investing activities. If the company does not have profitable investment opportunities, the cash should be returned to capital providers, a financing activity. For a new or growth stage company, operating cash flow may be negative for some period of time as it invests in such assets as inventory and receivables (extending credit to new customers) in order to grow the business. This situation is not sustainable over the long term, so eventually the cash must start to come primarily from operating activities so that capital can be returned to the providers of capital. Lastly, it is desirable that operating cash flows are sufficient to cover capital expenditures (in other words, the company has free cash flow as discussed further in Section 4.3). In summary, major points to consider at this step are:

\begin{itemize}
  \item What are the major sources and uses of cash flow?

  \item Is operating cash flow positive and sufficient to cover capital expenditures?

\end{itemize}

\section{Step 2}
Turning to the operating section, the analysts should examine the most significant determinants of operating cash flow. Companies need cash for use in operations (for example, to hold receivables and inventory and to pay employees and suppliers) and receive cash from operating activities (for example, payments from customers). Under the indirect method, the increases and decreases in receivables, inventory, payables, and so on can be examined to determine whether the company is using or generating cash in operations and why. It is also useful to compare operating cash flow with net income. For a mature company, because net income includes non-cash expenses (depreciation and amortisation), it is expected and desirable that operating cash flow exceeds net income. The relationship between net income and operating cash flow is also an indicator of earnings quality. If a company has large net income but poor operating cash flow, it may be a sign of poor earnings quality. The company may be making aggressive accounting choices to increase net income but not be generating cash for its business. You should also examine the variability of both earnings and cash flow and consider the impact of this variability on the company's risk as well as the ability to forecast future cash flows for valuation purposes. In summary:

\begin{itemize}
  \item What are the major determinants of operating cash flow?

  \item Is operating cash flow higher or lower than net income? Why?

  \item How consistent are operating cash flows?

\end{itemize}

\section{Step 3}
Within the investing section, you should evaluate each line item. Each line item represents either a source or use of cash. This enables you to understand where the cash is being spent (or received). This section will tell you how much cash is being invested for the future in property, plant, and equipment; how much is used to acquire entire companies; and how much is put aside in liquid investments, such as stocks and bonds. It will also tell you how much cash is being raised by selling these types of assets. If the company is making major capital investments, you should consider where the cash is coming from to cover these investments (e.g., is the cash coming from excess operating cash flow or from the financing activities described in Step 4). If assets are being sold, it is important to determine why and to assess the effects on the company.

\section{Step 4}
Within the financing section, you should examine each line item to understand whether the company is raising capital or repaying capital and what the nature of its capital sources are. If the company is borrowing each year, you should consider when repayment may be required. This section will also present dividend payments and repurchases of stock that are alternative means of returning capital to owners. It is important to assess why capital is being raised or repaid.

We now provide an example of a cash flow statement evaluation.

\section{EXAMPLE 8}
\section{Analysis of the Cash Flow Statement}
\begin{enumerate}
  \item Derek Yee, CFA, is preparing to forecast cash flow for Groupe Danone as an input into his valuation model. He has asked you to evaluate the historical cash flow statement of Groupe Danone, which is presented in Exhibit 12. Groupe Danone prepares its financial statements in conformity with IFRS. Note that Groupe Danone presents the most recent period on the right. Exhibit 13 presents excerpts from Danone's 2017 Registration Document.
\end{enumerate}

Yee would like answers to the following questions:

\begin{itemize}
  \item What are the major sources of cash for Groupe Danone?

  \item What are the major uses of cash for Groupe Danone?

  \item Is cash flow from operating activities sufficient to cover capital expenditures?

  \item What is the relationship between net income and cash flow from operating activities?

  \item What types of financing cash flows does Groupe Danone have? Exhibit 12: Groupe Danone Consolidated Financial Statements Consolidated Statements of Cash Flows (in $€$ Millions)

\end{itemize}

Years Ended 31 December

Net income

Share of profits of associates net of dividends received

Depreciation, amortization and impairment of tangible and intangible assets

Increases in (reversals of) provisions

Change in deferred taxes

(Gains) losses on disposal of property, plant and equipment and financial investments

Expense related to Group performance shares

Cost of net financial debt

Net interest paid

Net change in interest income (expense)

Other components with no cash impact

Cash flows provided by operating activities, before changes in net working capital

(Increase) decrease in inventories

(Increase) decrease in trade receivables

Increase (decrease) in trade payables

Changes in other receivables and payables

Change in other working capital requirements

Cash flows provided by (used in) operating activities

Capital expenditure

Proceeds from the disposal of property, plant and equipment

Net cash outflows on purchases of subsidiaries and financial investments

Net cash inflows on disposal of subsidiaries and financial investments

(Increase) decrease in long-term loans and other long-term financial assets

Cash flows provided by (used in) investing activities

Increase in capital and additional paid-in capital

Purchases of treasury stock (net of disposals) and DANONE call options

Issue of perpetual subordinated debt securities

Interest on perpetual subordinated debt securities

Dividends paid to Danone shareholders

Buyout of non-controlling interests

Dividends paid

Contribution from non-controlling interests to capital increases

Transactions with non-controlling interests

Net cash flows on hedging derivatives

Bonds issued during the period

Bonds repaid during the period

Net cash flows from other current and non-current financial debt

Net cash flows from short-term investments

Cash flows provided by (used in) financing activities

Effect of exchange rate and other changes

\begin{center}
\begin{tabular}{|c|c|}
\hline
2016 & 2017 \\
\hline
1,827 & 2,563 \\
\hline
52 & $(54)$ \\
\hline
786 & 974 \\
\hline
51 & 153 \\
\hline
(65) & (353) \\
\hline
$(74)$ & $(284)$ \\
\hline
24 & 22 \\
\hline
149 & 265 \\
\hline
$(148)$ & (186) \\
\hline
- & 80 \\
\hline
13 & (15) \\
\hline
2,615 & 3,085 \\
\hline
(24) & $(122)$ \\
\hline
(110) & (190) \\
\hline
298 & 145 \\
\hline
(127) & 40 \\
\hline
37 & (127) \\
\hline
2,652 & 2,958 \\
\hline
$(925)$ & $(969)$ \\
\hline
27 & 45 \\
\hline
(66) & $(10,949)$ \\
\hline
110 & 441 \\
\hline
6 & (4) \\
\hline
$(848)$ & $(11,437)$ \\
\hline
46 & 47 \\
\hline
32 & 13 \\
\hline
- & 1,245 \\
\hline
- & - \\
\hline
$(985)$ & (279) \\
\hline
(295) & (107) \\
\hline
$(94)$ & $(86)$ \\
\hline
6 & 1 \\
\hline
$(383)$ & (193) \\
\hline
50 & (52) \\
\hline
11,237 & - \\
\hline
$(638)$ & $(1,487)$ \\
\hline
$(442)$ & $(564)$ \\
\hline
$(10,531)$ & 9,559 \\
\hline
$(1,616)$ & 8,289 \\
\hline
$(151)$ & 272 \\
\hline
\end{tabular}
\end{center}

\begin{center}
\begin{tabular}{lcc}
\hline
Years Ended 31 December & 2016 & 2017 \\
\hline
Increase (decrease) in cash and cash equivalents & 38 &  \\
Cash and cash equivalents at beginning of period & 519 &  \\
Cash and cash equivalents at end of period & 557 &  \\
\end{tabular}
\end{center}

Supplemental disclosures

Income tax payments during the year

$(1,116)$

Note: the numbers in the consolidated statement of cash flows were derived straight from company filings; some sub-totals may not sum exactly due to rounding by the company.

\section{Exhibit 13: Groupe Danone Excerpt from 2017 Registration Statement}
\section{Excerpt from Footnote 2 to the financial statements:}
... On July 7, 2016, Danone announced the signing of an agreement to acquire The WhiteWave Foods Company ("WhiteWave"), the global leader in plant-based foods and beverages and organic produce. The acquisition in cash, for USD 56.25 per share, represented, as of the date of the agreement, a total enterprise value of approximately USD 12.5 billion, including debt and certain other WhiteWave liabilities. ...

"Acquisition expenses recognized in Danone's consolidated financial statements totaled $€ 51$ million before tax, of which $€ 48$ million was recognized in 2016 in Other operating income (expense), with the balance recognized in 2017.

"WhiteWave's contribution to 2017 consolidated sales totaled $€ 2.7$ billion. Had the transaction been completed on January 1, 2017, the Group's 2017 consolidated sales would have been $€ 25.7$ billion, with recurring operating income of $€ 3.6$ billion.

"Meanwhile, integration expenses for the period totaled $€ 91$ million, recognized under Other operating income (expense)...

\section{Excerpt from Overview of Activities:}
“... As part of its transformation plan aimed at ensuring a safe journey to deliver strong, profitable and sustainable growth, Danone set objectives for 2020 that include like-for-like sales growth between $4 \%$ and $5 \% \ldots$. a recurring operating margin of over $16 \%$ in 2020 ... Finally, Danone will continue to focus on growing its free cash flow, which will contribute to financial deleverage with an objective of a ratio of Net debt/EBITDA below $3 x$ in 2020. Danone is committed to reaching a ROIC level around $12 \%$ in 2020.

\section{Solution:}
The major categories of cash flows can be summarized as follows (in $€$ millions):

Cash flows provided by operating activities

\begin{center}
\begin{tabular}{ccc}
$\mathbf{2 0 1 6}$ & $\mathbf{2 0 1 7}$ &  \\
\cline { 1 - 1 }
2,652 & 2,958 &  \\
$(848)$ & $(11,437)$ &  \\
$(1,616)$ & 8,289 &  \\
$(151)$ & 272 &  \\
\hline
38 &  & 81 \\
\hline
\end{tabular}
\end{center}

The primary source of cash for Groupe Danone in 2016 is operating activities. In both 2016 and 2017, there was sufficient operating cash flow to cover usual capital expenditures, and operating cash flow exceeded net income. Evaluating the five prior years [not shown in this Example], you confirm that Danone typically derives most of its cash from operating activities, reports operating cash flow greater than net income, and generates sufficient operating cash flow to cover capital expenditures.

The fact that the primary source of cash is from operations is positive and desirable for a mature company. Additionally, the fact that operating cash flow exceeds net income in both years is a positive sign. Finally, operating cash flows exceed normal capital expenditures, indicating that the company can fund capital expenditures from operations.

In 2017, however, the primary source of cash was financing activities, and the investing section shows significant use of cash for purchase of subsidiaries within investing activities. Footnotes disclose a major acquisition with an aggregate value of $€ 12.5$ billion, some of which was funded through proceeds from an earlier bond issuance, which appears as a financing cash flow in the financing section for 2016.

For purposes of Yee's cash flow forecast, the company's targets for free cash flow and debt reduction-as well as disclosures concerning the acquisition's impact on 2017 operating results-are potentially helpful.

\section{CASH FLOW STATEMENT ANALYSIS: COMMON SIZE ANALYSIS}
analyze and interpret both reported and common-size cash flow statements

In common-size analysis of a company's income statement, each income and expense line item is expressed as a percentage of net revenues (net sales). For the common-size balance sheet, each asset, liability, and equity line item is expressed as a percentage of total assets. For the common-size cash flow statement, there are two alternative approaches. The first approach is to express each line item of cash inflow (outflow) as a percentage of total inflows (outflows) of cash, and the second approach is to express each line item as a percentage of net revenue.

Exhibit 14 demonstrates the total cash inflows/total cash outflows method for Acme Corporation. Under this approach, each of the cash inflows is expressed as a percentage of the total cash inflows, whereas each of the cash outflows is expressed as a percentage of the total cash outflows. In Panel A, Acme's common-size statement is based on a cash flow statement using the direct method of presenting operating cash flows. Operating cash inflows and outflows are separately presented on the cash flow statement, and therefore, the common-size cash flow statement shows each of these operating inflows (outflows) as a percentage of total inflows (outflows). In Panel B, Acme's common-size statement is based on a cash flow statement using the indirect method of presenting operating cash flows. When a cash flow statement has been presented using the indirect method, operating cash inflows and outflows are not separately presented; therefore, the common-size cash flow statement shows only the net operating cash flow (net cash provided by or used in operating activities) as a percentage of total inflows or outflows, depending on whether the net amount was a cash inflow or outflow. Because Acme's net operating cash flow is positive, it is shown as a percentage of total inflows.

Exhibit 14: Acme Corporation Common-Size Cash Flow Statement Year

Ended 31 December 2018

Panel A. Direct Format for Cash Flow

\begin{center}
\begin{tabular}{|c|c|c|}
\hline
Inflows &  & $\begin{array}{l}\text { Percentage of } \\ \text { Total Inflows }\end{array}$ \\
\hline
Receipts from customers & $\$ 23,543$ & $96.86 \%$ \\
\hline
Sale of equipment & 762 & 3.14 \\
\hline
Total & $\$ 24,305$ & $100.00 \%$ \\
\hline
Outflows &  & $\begin{array}{l}\text { Percentage of } \\ \text { Total Outflows }\end{array}$ \\
\hline
Payments to suppliers & $\$ 11,900$ & $48.66 \%$ \\
\hline
Payments to employees & 4,113 & 16.82 \\
\hline
Payments for other operating expenses & 3,532 & 14.44 \\
\hline
Payments for interest & 258 & 1.05 \\
\hline
Payments for income tax & 1,134 & 4.64 \\
\hline
Purchase of equipment & 1,300 & 5.32 \\
\hline
Retirement of long-term debt & 500 & 2.04 \\
\hline
Retirement of common stock & 600 & 2.45 \\
\hline
Dividend payments & 1,120 & 4.58 \\
\hline
Total & $\$ 24,457$ & $100.00 \%$ \\
\hline
Net increase (decrease) in cash & $(\$ 152)$ &  \\
\hline
\end{tabular}
\end{center}

Panel B. Indirect Format for Cash Flow

\begin{center}
\begin{tabular}{|c|c|c|}
\hline
Inflows &  & $\begin{array}{l}\text { Percentage of } \\ \text { Total Inflows }\end{array}$ \\
\hline
Net cash provided by operating activities & $\$ 2,606$ & $77.38 \%$ \\
\hline
Sale of equipment & 762 & 22.62 \\
\hline
Total & $\$ 3,368$ & $100.00 \%$ \\
\hline
Outflows &  & $\begin{array}{l}\text { Percentage of } \\ \text { Total Outflows }\end{array}$ \\
\hline
Purchase of equipment & $\$ 1,300$ & $36.93 \%$ \\
\hline
Retirement of long-term debt & 500 & 14.20 \\
\hline
Retirement of common stock & 600 & 17.05 \\
\hline
Dividend payments & 1,120 & 31.82 \\
\hline
Total & $\$ 3,520$ & $100.00 \%$ \\
\hline
Net increase (decrease) in cash & $(\$ 152)$ &  \\
\hline
\end{tabular}
\end{center}

Exhibit 15 demonstrates the net revenue common-size cash flow statement for Acme Corporation. Under the net revenue approach, each line item in the cash flow statement is shown as a percentage of net revenue. The common-size statement in this exhibit has been developed based on Acme's cash flow statement using the indirect method for operating cash flows and using net revenue of $\$ 23,598$ as shown in Exhibit 6. Each line item of the reconciliation between net income and net operating cash flows is expressed as a percentage of net revenue. The common-size format makes it easier to see trends in cash flow rather than just looking at the total amount. This method is also useful to the analyst in forecasting future cash flows because individual items in the common-size statement (e.g., depreciation, fixed capital expenditures, debt borrowing, and repayment) are expressed as a percentage of net revenue. Thus, once the analyst has forecast revenue, the common-size statement provides a basis for forecasting cash flows for those items with an expected relation to net revenue.

Exhibit 15: Acme Corporation Common-Size Cash Flow Statement: Indirect Format Year Ended 31 December 2018

Percentage of Net Revenue

\section{Cash flow from operating activities:}
Net income

Depreciation expense

\begin{center}
\begin{tabular}{|c|c|}
\hline
$\$ 2,210$ & $9.37 \%$ \\
\hline
1,052 & 4.46 \\
\hline
$(205)$ & $(0.87)$ \\
\hline
(55) & $(0.23)$ \\
\hline
(707) & (3.00) \\
\hline
23 & 0.10 \\
\hline
263 & 1.11 \\
\hline
10 & 0.04 \\
\hline
(12) & $(0.05)$ \\
\hline
5 & 0.02 \\
\hline
22 & 0.09 \\
\hline
$\$ 2,606$ & 11.0 \\
\hline
\end{tabular}
\end{center}

Net cash provided by operating activities

\section{Cash flow from investing activities:}
Cash received from sale of equipment

Cash paid for purchase of equipment

Net cash used for investing activities

\begin{center}
\begin{tabular}{ccc}
$\$ 762$ & $3.23 \%$ &  \\
$(1,300)$ &  &  \\
\hline
$\$(538)$ &  & $(5.51)$ \\
\cline { 3 - 3 }
 &  & $(2.28) \%$ \\
\hline
\end{tabular}
\end{center}

\section{Cash flow from financing activities:}
Cash paid to retire long-term debt

Cash paid to retire common stock

$\$(500)$

$(2.12) \%$

Cash paid for dividends

Net cash used for financing activities

$(1,120)$

$(4.75)$

$\$(2,220)$

$(9.41) \%$

Net decrease in cash

$\$(152)$

$(0.64) \%$

\section{EXAMPLE 9}
\section{Analysis of a Common-Size Cash Flow Statement}
Andrew Potter is examining an abbreviated common-size cash flow statement for Apple Inc., a multinational technology company. The common-size cash flow statement was prepared by dividing each line item by total net sales for the same year.

Apple Inc. Common Size Statements OF Cash Flows as Percentage of Total Net Sales

12 Months Ended

\begin{center}
\begin{tabular}{ccc}
\hline
30 Sep. & 24 Sep. & 26 Sep. \\
2017 & 2016 & 2015 \\
\hline
\end{tabular}
\end{center}

\section{Statement of Cash Flows [Abstract]}
\section{Operating activities:}
Net income

$21.1 \% \quad 21.2 \% \quad 22.8 \%$

Adjustments to reconcile net income

to cash generated by operating

activities:

Depreciation and amortization

Share-based compensation expense

Deferred income tax expense

Other

Changes in operating assets and liabilities:

Accounts receivable, net

Inventories

Vendor non-trade receivables

Other current and non-current assets

Accounts payable

Deferred revenue

Other current and non-current

liabilities

Cash generated by operating

$\begin{array}{ccc}4.4 \% & 4.9 \% & 4.8 \% \\ 2.1 \% & 2.0 \% & 1.5 \% \\ 2.6 \% & 2.3 \% & 0.6 \% \\ -0.1 \% & 0.2 \% & 0.2 \%\end{array}$

activities

\section{Investing activities:}
Purchases of marketable securities

$\begin{array}{ccc}-69.6 \% & -66.0 \% & -71.2 \% \\ 13.9 \% & 9.9 \% & 6.2 \% \\ 41.3 \% & 42.0 \% & 46.0 \% \\ -0.1 \% & -0.1 \% & -0.1 \% \\ -5.4 \% & -5.9 \% & -4.8 \%\end{array}$

\begin{center}
\begin{tabular}{|c|c|c|c|}
\hline
 & \multicolumn{3}{|c|}{12 Months Ended} \\
\hline
 & $\begin{array}{l}30 \text { Sep. } \\ 2017\end{array}$ & $\begin{array}{c}24 \text { Sep. } \\ 2016\end{array}$ & $\begin{array}{c}26 \text { Sep. } \\ 2015\end{array}$ \\
\hline
$\begin{array}{l}\text { Payments for acquisition of intangible } \\ \text { assets }\end{array}$ & $-0.2 \%$ & $-0.4 \%$ & $-0.1 \%$ \\
\hline
Payments for strategic investments, net & $-0.2 \%$ & $-0.6 \%$ & $0.0 \%$ \\
\hline
Other & $0.1 \%$ & $-0.1 \%$ & $0.0 \%$ \\
\hline
Cash used in investing activities & $-20.3 \%$ & $-21.3 \%$ & $-24.1 \%$ \\
\hline
\multicolumn{4}{|l|}{Financing activities:} \\
\hline
$\begin{array}{l}\text { Proceeds from issuance of common } \\ \text { stock }\end{array}$ & $0.2 \%$ & $0.2 \%$ & $0.2 \%$ \\
\hline
Excess tax benefits from equity awards & $0.3 \%$ & $0.2 \%$ & $0.3 \%$ \\
\hline
$\begin{array}{l}\text { Payments for taxes related to net share } \\ \text { settlement of equity awards }\end{array}$ & $-0.8 \%$ & $-0.7 \%$ & $-0.6 \%$ \\
\hline
$\begin{array}{l}\text { Payments for dividends and dividend } \\ \text { equivalents }\end{array}$ & $-5.6 \%$ & $-5.6 \%$ & $-4.9 \%$ \\
\hline
Repurchases of common stock & $-14.4 \%$ & $-13.8 \%$ & $-15.1 \%$ \\
\hline
$\begin{array}{l}\text { Proceeds from issuance of term debt, } \\ \text { net }\end{array}$ & $12.5 \%$ & $11.6 \%$ & - \\
\hline
Repayments of term debt & $-1.5 \%$ & $-1.2 \%$ & $0.0 \%$ \\
\hline
Change in commercial paper, net & $1.7 \%$ & $-0.2 \%$ & $0.9 \%$ \\
\hline
Cash used in financing activities & $-7.6 \%$ & $-9.5 \%$ & $-7.6 \%$ \\
\hline
$\begin{array}{l}\text { Increase/(Decrease) in cash and cash } \\ \text { equivalents }\end{array}$ & $-0.1 \%$ & $-0.3 \%$ & $3.1 \%$ \\
\hline
\end{tabular}
\end{center}

Based on the information in the above exhibit:

\begin{enumerate}
  \item Discuss the significance of
\end{enumerate}

A. depreciation and amortization.

B. capital expenditures.

\section{Solution to 1:}
A. Apple's depreciation and amortization expense was consistently just less than 5\% of total net revenue in 2015 and 2016, declining to $4.4 \%$ in 2017.

B. Apple's level of capital expenditures is greater than depreciation and amortization in 2016 and 2017 whereas it was at about the same level as depreciation and amortization in 2015. In 2017 capital expenditures approached $6 \%$. This is an indication that Apple is doing more than replacing property, plant, and equipment, and is expanding those investments. With cash generated from operating activities exceeding $27 \%$ of sales in every year, however, Apple has more than enough cash flow from operations to fund these expenditures. 2. Compare Apple's operating cash flow as a percentage of revenue with Apple's net profit margin.

\section{Solution to 2:}
Apple's operating cash flow as a percentage of sales is much higher than net profit margin in every year. This gap appears to be declining however over the three year period. In 2015 net profit margin was $22.8 \%$ while operating cash flow as a percentage of sales was $34.8 \%$. By 2017 the net profit margin declined slightly to $21.1 \%$ while the operating cash flow as a percentage of sales declined more to $27.7 \%$. The primary difference appears to have been an increase in the level of receivables and inventory purchases, somewhat offset by an increase in accounts payable.

\begin{enumerate}
  \setcounter{enumi}{2}
  \item Discuss Apple's use of its positive operating cash flow.
\end{enumerate}

\section{Solution to 3:}
Apple has a very strong cash flow statement. Apple generates a large amount of operating cash flow in every year, exceeding net income. This cash flow is used for relatively modest purchases of property, plant and equipment, substantial purchases of marketable securities (investments), dividend payments and repurchases of its own stock.

\section{CASH FLOW STATEMENT ANALYSIS: FREE CASH FLOW TO FIRM AND FREE CASH FLOW TO EQUITY}
calculate and interpret free cash flow to the firm, free cash flow to equity, and performance and coverage cash flow ratios

It was mentioned earlier that it is desirable that operating cash flows are sufficient to cover capital expenditures. The excess of operating cash flow over capital expenditures is known generically as free cash flow. For purposes of valuing a company or its equity securities, an analyst may want to determine and use other cash flow measures, such as free cash flow to the firm (FCFF) or free cash flow to equity (FCFE).

FCFF is the cash flow available to the company's suppliers of debt and equity capital after all operating expenses (including income taxes) have been paid and necessary investments in working capital and fixed capital have been made. FCFF can be computed starting with net income as

$\mathrm{FCFF}=\mathrm{NI}+\mathrm{NCC}+\operatorname{Int}(1-$ Tax rate $)-$ FCInv $-\mathrm{WCInv}$

where

$\mathrm{NI}=$ Net income

$\mathrm{NCC}=$ Non-cash charges (such as depreciation and amortisation)

Int $=$ Interest expense

FCInv = Capital expenditures (fixed capital, such as equipment)

WCInv $=$ Working capital expenditures The reason for adding back interest is that FCFF is the cash flow available to the suppliers of debt capital as well as equity capital. Conveniently, FCFF can also be computed from cash flow from operating activities as

$\mathrm{FCFF}=\mathrm{CFO}+\operatorname{Int}(1-$ Tax rate $)-\mathrm{FCInv}$

CFO represents cash flow from operating activities under US GAAP or under IFRS where the company has included interest paid in operating activities. If interest paid was included in financing activities, then CFO does not have to be adjusted for Int(1 - Tax rate). Under IFRS, if the company has placed interest and dividends received in investing activities, these should be added back to CFO to determine FCFF. Additionally, if dividends paid were subtracted in the operating section, these should be added back in to compute FCFF.

The computation of FCFF for Acme Corporation (based on the data from Exhibits 6,7 , and 8 ) is as follows:

CFO $\quad \$ 2,606$

Plus: Interest paid times (1 - income tax rate)

$\left\{\$ 258\left[1-0.34^{\mathrm{a}}\right]\right\} \quad 170$

Less: Net investments in fixed capital

$(\$ 1,300-\$ 762)$

$\frac{(538)}{\$ 2,238}$

a Income tax rate of $0.34=($ Tax expense $\div$ Pretax income $)=(\$ 1,139 \div \$ 3,349)$.

FCFE is the cash flow available to the company's common stockholders after all operating expenses and borrowing costs (principal and interest) have been paid and necessary investments in working capital and fixed capital have been made. FCFE can be computed as

$\mathrm{FCFE}=\mathrm{CFO}-\mathrm{FCInv}+$ Net borrowing

When net borrowing is negative, debt repayments exceed receipts of borrowed funds. In this case, FCFE can be expressed as

$\mathrm{FCFE}=\mathrm{CFO}-\mathrm{FCInv}-$ Net debt repayment

The computation of FCFE for Acme Corporation (based on the data from Exhibits 6,7 , and 8 ) is as follows:

$\begin{array}{lr}\text { CFO } & \$ 2,606 \\ \text { Less: Net investments in fixed capital }(\$ 1,300-\$ 762) & (538) \\ \text { Less: Debt repayment } & (500) \\ { } } & \$ 1,568\end{array}$

Positive FCFE means that the company has an excess of operating cash flow over amounts needed for capital expenditures and repayment of debt. This cash would be available for distribution to owners.

\textbackslash section\{CASH FLOW STATEMENT ANALYSIS: CASH FLOW RATIOS

calculate and interpret free cash flow to the firm, free cash flow to equity, and performance and coverage cash flow ratios

The statement of cash flows provides information that can be analyzed over time to obtain a better understanding of the past performance of a company and its future prospects. This information can also be effectively used to compare the performance and prospects of different companies in an industry and of different industries. There are several ratios based on cash flow from operating activities that are useful in this analysis. These ratios generally fall into cash flow performance (profitability) ratios and cash flow coverage (solvency) ratios. Exhibit 15 summarizes the calculation and interpretation of some of these ratios.

\section{Exhibit 16: Cash Flow Ratios}
\begin{center}
\begin{tabular}{|c|c|c|}
\hline
Performance Ratios & Calculation & What It Measures \\
\hline
Cash flow to revenue & $\mathrm{CFO} \div$ Net revenue & $\begin{array}{l}\text { Operating cash generated per dollar of } \\ \text { revenue }\end{array}$ \\
\hline
Cash return on assets & $\mathrm{CFO} \div$ Average total assets & $\begin{array}{l}\text { Operating cash generated per dollar of asset } \\ \text { investment }\end{array}$ \\
\hline
Cash return on equity & $\mathrm{CFO} \div$ Average shareholders' equity & $\begin{array}{l}\text { Operating cash generated per dollar of owner } \\ \text { investment }\end{array}$ \\
\hline
Cash to income & $\mathrm{CFO} \div$ Operating income & Cash generating ability of operations \\
\hline
\includegraphics[max width=\textwidth]{2023_05_04_b5cfa4f1bc883752f121g-179}
 & $\begin{array}{l}(\mathrm{CFO}-\text { Preferred dividends }) \div \text { Number of } \\ \text { common shares outstanding }\end{array}$ & Operating cash flow on a per-share basis \\
\hline
\end{tabular}
\end{center}

\begin{center}
\begin{tabular}{|c|c|c|}
\hline
Coverage Ratios & Calculation & What It Measures \\
\hline
Debt coverage & $\mathrm{CFO} \div$ Total debt & Financial risk and financial leverage \\
\hline
Interest coverage $^{\mathrm{b}}$ & $\begin{array}{l}(\mathrm{CFO}+\text { Interest paid }+ \text { Taxes paid }) \div \text { Interest } \\ \text { paid }\end{array}$ & Ability to meet interest obligations \\
\hline
Reinvestment & $\mathrm{CFO} \div$ Cash paid for long-term assets & $\begin{array}{l}\text { Ability to acquire assets with operating cash } \\ \text { flows }\end{array}$ \\
\hline
Debt payment & $\begin{array}{l}\mathrm{CFO} \div \text { Cash paid for long-term debt } \\ \text { repayment }\end{array}$ & Ability to pay debts with operating cash flows \\
\hline
Dividend payment & $\mathrm{CFO} \div$ Dividends paid & $\begin{array}{l}\text { Ability to pay dividends with operating cash } \\ \text { flows }\end{array}$ \\
\hline
Investing and financing & $\begin{array}{l}\mathrm{CFO} \div \text { Cash outflows for investing and } \\ \text { financing activities }\end{array}$ & $\begin{array}{l}\text { Ability to acquire assets, pay debts, and make } \\ \text { distributions to owners }\end{array}$ \\
\hline
\end{tabular}
\end{center}

Notes:

a If the company reports under IFRS and includes total dividends paid as a use of cash in the operating section, total dividends should be added back to CFO as reported and then preferred dividends should be subtracted. Recall that CFO reported under US GAAP and IFRS may differ depending on the treatment of interest and dividends, received and paid.

$\mathrm{b}$ If the company reports under IFRS and included interest paid as a use of cash in the financing section, then interest paid should not be added back to the numerator.

\section{EXAMPLE 10}
\section{A Cash Flow Analysis of Comparables}
\begin{enumerate}
  \item Andrew Potter is comparing the cash-flow-generating ability of Microsoft with that of Apple Inc. He collects information from the companies' annual reports and prepares the following table.
\end{enumerate}

Cash Flow from Operating Activities as a Percentage of Total Net Revenue

\begin{center}
\begin{tabular}{lccc}
\hline
 & $\mathbf{2 0 1 7}(\%)$ & $\mathbf{2 0 1 6}(\%)$ & $\mathbf{2 0 1 5}(\%)$ \\
\hline
Microsoft & 43.9 & 39.1 & 31.7 \\
Apple Inc. & 27.7 & 30.5 & 34.8 \\
\hline
\end{tabular}
\end{center}

As a Percentage of Average Total Assets

\begin{center}
\begin{tabular}{lccc}
\hline
 & $\mathbf{2 0 1 7}(\%)$ & $\mathbf{2 0 1 6}(\%)$ & $\mathbf{2 0 1 5}$ (\%) \\
\hline
Microsoft & 18.2 & 18.1 & 17.1 \\
Apple Inc. & 18.2 & 21.5 & 31.1 \\
\hline
\end{tabular}
\end{center}

What is Potter likely to conclude about the relative cash-flow-generating ability of these two companies?

Solution:

On both measures-operating cash flow divided by revenue and operating cash flow divided by assets-both companies have overall strong results. However, Microsoft has higher cash flow from operating activities as a percentage of revenues in both 2016 and 2017. Further, Microsoft has an increasing trend. While Apple had a higher operating cash flow as a percent of revenue in 2015 compared to Microsoft, it has had a declining trend and was below Microsoft in the two more recent years. Microsoft's operating cash flow relative to assets is the same as Apple's in 2017 and relatively stable with a slight increase since 2015. Apple started the three years with a much stronger ratio but saw a declining trend such that its ratio is now at the same level as Microsoft. We should note that this ratio is heavily influenced by substantial investments in financial instruments that Apple has made over the years due to its strong historic cash flow.

\section{SUMMARY}
The cash flow statement provides important information about a company's cash receipts and cash payments during an accounting period as well as information about a company's operating, investing, and financing activities. Although the income statement provides a measure of a company's success, cash and cash flow are also vital to a company's long-term success. Information on the sources and uses of cash helps creditors, investors, and other statement users evaluate the company's liquidity, solvency, and financial flexibility. Key concepts are as follows:

\begin{itemize}
  \item Cash flow activities are classified into three categories: operating activities, investing activities, and financing activities. Significant non-cash transaction activities (if present) are reported by using a supplemental disclosure note to the cash flow statement.

  \item Cash flow statements under IFRS and US GAAP are similar; however, IFRS provide companies with more choices in classifying some cash flow items as operating, investing, or financing activities.

  \item Companies can use either the direct or the indirect method for reporting their operating cash flow:

  \item The direct method discloses operating cash inflows by source (e.g., cash received from customers, cash received from investment income) and operating cash outflows by use (e.g., cash paid to suppliers, cash paid for interest) in the operating activities section of the cash flow statement.

  \item The indirect method reconciles net income to operating cash flow by adjusting net income for all non-cash items and the net changes in the operating working capital accounts.

  \item The cash flow statement is linked to a company's income statement and comparative balance sheets and to data on those statements.

  \item Although the indirect method is most commonly used by companies, an analyst can generally convert it to an approximation of the direct format by following a simple three-step process.

  \item An evaluation of a cash flow statement should involve an assessment of the sources and uses of cash and the main drivers of cash flow within each category of activities.

  \item The analyst can use common-size statement analysis for the cash flow statement. Two approaches to developing the common-size statements are the total cash inflows/total cash outflows method and the percentage of net revenues method

  \item The cash flow statement can be used to determine free cash flow to the firm (FCFF) and free cash flow to equity (FCFE).

  \item The cash flow statement may also be used in financial ratios that measure a company's profitability, performance, and financial strength.

\end{itemize}

\section{PRACTICE PROBLEMS}
\begin{enumerate}
  \item The three major classifications of activities in a cash flow statement are:
\end{enumerate}

A. inflows, outflows, and net flows.

B. operating, investing, and financing.

C. revenues, expenses, and net income.

\begin{enumerate}
  \setcounter{enumi}{1}
  \item The sale of a building for cash would be classified as what type of activity on the cash flow statement?
\end{enumerate}

A. Operating.

B. Investing.

C. Financing.

\begin{enumerate}
  \setcounter{enumi}{2}
  \item Under which section of a manufacturing company's cash flow statement are the following activities reported?
\end{enumerate}

Item 1: Purchases of securities held for trading

Item 2: Purchases of securities held for investment

A. Both items are investing activities.

B. Only Item 1 is an operating activity.

C. Only Item 2 is an operating activity.

\begin{enumerate}
  \setcounter{enumi}{3}
  \item A conversion of a face value $\$ 1$ million convertible bond for $\$ 1$ million of common stock would most likely be:
\end{enumerate}

A. reported as a $\$ 1$ million investing cash inflow and outflow.

B. reported as a $\$ 1$ million financing cash outflow and inflow.

C. reported as supplementary information to the cash flow statement.

\begin{enumerate}
  \setcounter{enumi}{4}
  \item A company recently engaged in a non-cash transaction that significantly affected its property, plant, and equipment. The transaction is:
\end{enumerate}

A. reported under the investing section of the cash flow statement.

B. reported differently in cash flow from operations under the direct and indirect methods.

C. disclosed as a separate note or in a supplementary schedule to the cash flow statement.

\begin{enumerate}
  \setcounter{enumi}{5}
  \item Which of the following is an example of a financing activity on the cash flow statement under US GAAP?
\end{enumerate}

A. Payment of interest.

B. Receipt of dividends. C. Payment of dividends.

\begin{enumerate}
  \setcounter{enumi}{6}
  \item Interest paid is classified as an operating cash flow under:
\end{enumerate}

A. US GAAP but may be classified as either operating or investing cash flows under IFRS.

B. IFRS but may be classified as either operating or investing cash flows under US GAAP.

C. US GAAP but may be classified as either operating or financing cash flows under IFRS.

\begin{enumerate}
  \setcounter{enumi}{7}
  \item Cash flows from taxes on income must be separately disclosed under:
A. IFRS only.
B. US GAAP only.
C. both IFRS and US GAAP.

  \item A benefit of using the direct method rather than the indirect method when reporting operating cash flows is that the direct method:
A. mirrors a forecasting approach.
B. is easier and less costly.
C. provides specific information on the sources of operating cash flows.

  \item Which of the following is most likely to appear in the operating section of a cash flow statement under the indirect method?
A. Net income.
B. Cash paid to suppliers.
C. Cash received from customers.

  \item Which of the following components of the cash flow statement may be prepared under the indirect method under both IFRS and US GAAP?
A. Operating.
B. Investing.
C. Financing.

  \item Mabel Corporation (MC) reported accounts receivable of $\$ 66$ million at the end of its second fiscal quarter. MC had revenues of $\$ 72$ million for its third fiscal quarter and reported accounts receivable of $\$ 55$ million at the end of its third fiscal quarter. Based on this information, the amount of cash MC collected from customers during the third fiscal quarter is:
A. $\$ 61$ million.
B. $\$ 72$ million.
C. $\$ 83$ million.

  \item Red Road Company, a consulting company, reported total revenues of $\$ 100$ million, total expenses of $\$ 80$ million, and net income of $\$ 20$ million in the most recent year. If accounts receivable increased by $\$ 10$ million, how much cash did the company receive from customers?
A. $\$ 90$ million.
B. $\$ 100$ million.
C. $\$ 110$ million.

  \item In 2018, a company using US GAAP made cash payments of $\$ 6$ million for salaries, $\$ 2$ million for interest expense, and $\$ 4$ million for income taxes. Additional information for the company is provided in the table:

\end{enumerate}

\begin{center}
\begin{tabular}{lll}
\hline
(\$ millions) & $\mathbf{2 0 1 7}$ & $\mathbf{2 0 1 8}$ \\
\hline
Revenue & 42 & 37 \\
Cost of goods sold & 18 & 16 \\
Inventory & 36 & 40 \\
Accounts receivable & 22 & 19 \\
Accounts payable & 14 & 12 \\
\hline
\end{tabular}
\end{center}

Based only on the information given, the company's operating cash flow for 2018 is closest to:

A. $\$ 6$ million.

B. $\$ 10$ million.

C. $\$ 14$ million.

\begin{enumerate}
  \setcounter{enumi}{14}
  \item Green Glory Corp., a garden supply wholesaler, reported cost of goods sold for the year of $\$ 80$ million. Total assets increased by $\$ 55$ million, including an increase of $\$ 5$ million in inventory. Total liabilities increased by $\$ 45$ million, including an increase of $\$ 2$ million in accounts payable. The cash paid by the company to its suppliers is most likely closest to:
A. $\$ 73$ million.
B. $\$ 77$ million.
C. $\$ 83$ million.

  \item Purple Fleur S.A., a retailer of floral products, reported cost of goods sold for the year of $\$ 75$ million. Total assets increased by $\$ 55$ million, but inventory declined by $\$ 6$ million. Total liabilities increased by $\$ 45$ million, and accounts payable increased by $\$ 2$ million. The cash paid by the company to its suppliers is most likely closest to:
A. $\$ 67$ million.
B. $\$ 79$ million.
C. $\$ 83$ million.

  \item White Flag, a women's clothing manufacturer, reported salaries expense of $\$ 20$ million. The beginning balance of salaries payable was $\$ 3$ million, and the ending balance of salaries payable was $\$ 1$ million. How much cash did the company pay in salaries?
A. $\$ 18$ million.
B. $\$ 21$ million.
C. $\$ 22$ million.

  \item An analyst gathered the following information from a company's 2018 financial statements (in $\$$ millions):

\end{enumerate}

\begin{center}
\begin{tabular}{lcc}
\hline
Year ended 31 December & $\mathbf{2 0 1 7}$ & $\mathbf{2 0 1 8}$ \\
\hline
Net sales & 245.8 & 254.6 \\
Cost of goods sold & 168.3 & 175.9 \\
Accounts receivable & 73.2 & 68.3 \\
Inventory & 39.0 & 47.8 \\
Accounts payable & 20.3 & 22.9 \\
\hline
\end{tabular}
\end{center}

Based only on the information above, the company's 2018 statement of cash flows in the direct format would include amounts (in $\$$ millions) for cash received from customers and cash paid to suppliers, respectively, that are closest to:

\begin{center}
\begin{tabular}{lccc}
 & cash received from customers & cash paid to suppliers \\
\cline { 2 - 2 }
 & 249.7 & 169.7 \\
B & 259.5 & 174.5 \\
C & 259.5 & 182.1 \\
\end{tabular}
\end{center}

\begin{enumerate}
  \setcounter{enumi}{18}
  \item Golden Cumulus Corp., a commodities trading company, reported interest expense of $\$ 19$ million and taxes of $\$ 6$ million. Interest payable increased by $\$ 3$ million, and taxes payable decreased by $\$ 4$ million over the period. How much cash did the company pay for interest and taxes?
A. $\$ 22$ million for interest and $\$ 10$ million for taxes.
B. $\$ 16$ million for interest and $\$ 2$ million for taxes.
C. $\$ 16$ million for interest and $\$ 10$ million for taxes.

  \item The following information is extracted from Sweetfall Incorporated's financial statements.

\end{enumerate}

\begin{center}
\begin{tabular}{|c|c|c|c|}
\hline
\multicolumn{2}{|c|}{Income Statement} & \multicolumn{2}{|c|}{Balance Sheet Changes} \\
\hline
Revenue & $\$ 56,800$ & Decrease in accounts receivable & $\$ 1,324$ \\
\hline
Cost of goods sold & 27,264 & Decrease in inventory & 501 \\
\hline
Other operating expense & 562 & Increase in prepaid expense & 6 \\
\hline
Depreciation expense & 2,500 & Increase in accounts payable & 1,063 \\
\hline
\end{tabular}
\end{center}

The amount of cash Sweetfall Inc. paid to suppliers is:
A. $\$ 25,700$.
B. $\$ 26,702$.
C. $\$ 27,826$. 21. Silverago Incorporated, an international metals company, reported a loss on the sale of equipment of $\$ 2$ million in 2018. In addition, the company's income statement shows depreciation expense of $\$ 8$ million and the cash flow statement shows capital expenditure of $\$ 10$ million, all of which was for the purchase of new equipment. Using the following information from the comparative balance sheets, how much cash did the company receive from the equipment sale?

\begin{center}
\begin{tabular}{lccc}
\hline
Balance Sheet ltem & $\mathbf{1 2 / 3 1 / 2 0 1 7}$ & $\mathbf{1 2 / 3 1 / 2 0 1 8}$ & Change \\
\hline
Equipment & $\$ 100$ million & $\$ 105$ million & $\$ 5$ million \\
$\begin{array}{l}\text { Accumulated } \\ \text { depreciation-equipment }\end{array}$ & $\$ 40$ million & $\$ 46$ million & $\$ 6$ million \\
\hline
\end{tabular}
\end{center}

A. $\$ 1$ million.

B. $\$ 2$ million.

C. $\$ 3$ million.

\begin{enumerate}
  \setcounter{enumi}{21}
  \item Jaderong Plinkett Stores reported net income of $\$ 25$ million. The company has no outstanding debt. Using the following information from the comparative balance sheets (in millions), what should the company report in the financing section of the statement of cash flows in 2018?
\end{enumerate}

\begin{center}
\begin{tabular}{lccc}
\hline
Balance Sheet Item & $\mathbf{1 2 / 3 1 / 2 0 1 7}$ & $\mathbf{1 2 / 3 1 / 2 0 1 8}$ & Change \\
\hline
Common stock & $\$ 100$ & $\$ 102$ & $\$ 2$ \\
Additional paid-in capital common & $\$ 100$ & $\$ 140$ & $\$ 40$ \\
stock &  &  &  \\
Retained earnings & $\$ 100$ & $\$ 115$ & $\$ 15$ \\
Total stockholders' equity & $\$ 300$ & $\$ 357$ & $\$ 57$ \\
\hline
\end{tabular}
\end{center}

A. Issuance of common stock of $\$ 42$ million; dividends paid of $\$ 10$ million.

B. Issuance of common stock of $\$ 38$ million; dividends paid of $\$ 10$ million.

C. Issuance of common stock of $\$ 42$ million; dividends paid of $\$ 40$ million.

\begin{enumerate}
  \setcounter{enumi}{22}
  \item When computing net cash flow from operating activities using the indirect method, an addition to net income is most likely to occur when there is a:
A. gain on the sale of an asset.
B. loss on the retirement of debt.
C. decrease in a deferred tax liability.

  \item An analyst gathered the following information from a company's 2018 financial statements (in \$ millions):

\end{enumerate}

\begin{center}
\begin{tabular}{lcc}
\hline
Balances as of Year Ended 31 December & $\mathbf{2 0 1 7}$ & $\mathbf{2 0 1 8}$ \\
\hline
Retained earnings & 120 & 145 \\
Accounts receivable & 38 & 43 \\
Inventory & 45 & 48 \\
Accounts payable & 36 & 29 \\
\hline
\end{tabular}
\end{center}

In 2018, the company declared and paid cash dividends of $\$ 10$ million and recorded depreciation expense in the amount of $\$ 25$ million. The company considers dividends paid a financing activity. The company's 2018 cash flow from operations (in \$ millions) was closest to
A. 25 .
B. 45 .
C. 75 .

\begin{enumerate}
  \setcounter{enumi}{24}
  \item Based on the following information for Star Inc., what are the total net adjustments that the company would make to net income in order to derive operating cash flow?
\end{enumerate}

Year Ended

\begin{center}
\begin{tabular}{lccc}
\hline
Income Statement Item &  & $\mathbf{1 2 / 3 1 / 2 0 1 8}$ &  \\
\hline
Net income &  & $\$ 20$ million &  \\
Depreciation &  & $\$ 2$ million &  \\
\hline
 &  &  &  \\
\hline
Balance Sheet Item & $\mathbf{1 2 / 3 1 / 2 0 1 7}$ & $\mathbf{1 2 / 3 1 / 2 0 1 8}$ & Change \\
\hline
Accounts receivable & $\$ 25$ million & $\$ 22$ million & (\$3 million) \\
Inventory & $\$ 10$ million & $\$ 14$ million & $\$ 4$ million \\
Accounts payable & $\$ 8$ million & $\$ 13$ million & $\$ 5$ million \\
\hline
\end{tabular}
\end{center}

A. Add $\$ 2$ million.
B. Add $\$ 6$ million.
C. Subtract $\$ 6$ million.

\begin{enumerate}
  \setcounter{enumi}{25}
  \item The first step in cash flow statement analysis should be to:
A. evaluate consistency of cash flows.
B. determine operating cash flow drivers.
C. identify the major sources and uses of cash.

  \item Which of the following would be valid conclusions from an analysis of the cash flow statement for Telefónica Group presented in Exhibit 3?
A. The primary use of cash is financing activities.
B. The primary source of cash is operating activities.
C. Telefónica classifies dividends paid as an operating activity.

  \item Which is an appropriate method of preparing a common-size cash flow statement?

\end{enumerate}

A. Show each item of revenue and expense as a percentage of net revenue.

B. Show each line item on the cash flow statement as a percentage of net revenue. C. Show each line item on the cash flow statement as a percentage of total cash outflows.

\begin{enumerate}
  \setcounter{enumi}{28}
  \item Which of the following is an appropriate method of computing free cash flow to the firm?
\end{enumerate}

A. Add operating cash flows to capital expenditures and deduct after-tax interest payments.

B. Add operating cash flows to after-tax interest payments and deduct capital expenditures.

C. Deduct both after-tax interest payments and capital expenditures from operating cash flows.

\begin{enumerate}
  \setcounter{enumi}{29}
  \item An analyst has calculated a ratio using as the numerator the sum of operating cash flow, interest, and taxes and as the denominator the amount of interest. What is this ratio, what does it measure, and what does it indicate?
\end{enumerate}

A. This ratio is an interest coverage ratio, measuring a company's ability to meet its interest obligations and indicating a company's solvency.

B. This ratio is an effective tax ratio, measuring the amount of a company's operating cash flow used for taxes and indicating a company's efficiency in tax management.

C. This ratio is an operating profitability ratio, measuring the operating cash flow generated accounting for taxes and interest and indicating a company's liquidity.

\section{SOLUTIONS}
\begin{enumerate}
  \item B is correct. Operating, investing, and financing are the three major classifications of activities in a cash flow statement. Revenues, expenses, and net income are elements of the income statement. Inflows, outflows, and net flows are items of information in the statement of cash flows.

  \item B is correct. Purchases and sales of long-term assets are considered investing activities. Note that if the transaction had involved the exchange of a building for other than cash (for example, for another building, common stock of another company, or a long-term note receivable), it would have been considered a significant non-cash activity.

  \item B is correct. The purchase and sale of securities held for trading are considered operating activities even for companies in which this activity is not a primary business activity.

  \item C is correct. Non-cash transactions, if significant, are reported as supplementary information, not in the investing or financing sections of the cash flow statement.

  \item C is correct. Because no cash is involved in non-cash transactions, these transactions are not incorporated in the cash flow statement. However, non-cash transactions that significantly affect capital or asset structures are required to be disclosed either in a separate note or a supplementary schedule to the cash flow statement.

  \item $C$ is correct. Payment of dividends is a financing activity under US GAAP. Payment of interest and receipt of dividends are included in operating cash flows under US GAAP. Note that IFRS allow companies to include receipt of interest and dividends as either operating or investing cash flows and to include payment of interest and dividends as either operating or financing cash flows.

  \item $\mathrm{C}$ is correct. Interest expense is always classified as an operating cash flow under US GAAP but may be classified as either an operating or financing cash flow under IFRS.

  \item C is correct. Taxes on income are required to be separately disclosed under IFRS and US GAAP. The disclosure may be in the cash flow statement or elsewhere.

  \item $\mathrm{C}$ is correct. The primary argument in favor of the direct method is that it provides information on the specific sources of operating cash receipts and payments. Arguments for the indirect method include that it mirrors a forecasting approach and it is easier and less costly

  \item A is correct. Under the indirect method, the operating section would begin with net income and adjust it to arrive at operating cash flow. The other two items would appear in the operating section under the direct method.

  \item A is correct. The operating section may be prepared under the indirect method. The other sections are always prepared under the direct method.

  \item $\mathrm{C}$ is correct. The amount of cash collected from customers during the quarter is equal to beginning accounts receivable plus revenues minus ending accounts receivable: $\$ 66$ million $+\$ 72$ million $-\$ 55$ million $=\$ 83$ million. A reduction in accounts receivable indicates that cash collected during the quarter was greater than revenue on an accrual basis. 13. A is correct. Revenues of $\$ 100$ million minus the increase in accounts receivable of $\$ 10$ million equal $\$ 90$ million cash received from customers. The increase in accounts receivable means that the company received less in cash than it reported as revenue.

  \item A is correct.

\end{enumerate}

Operating cash flows $=$ Cash received from customers $-($ Cash paid to suppliers + Cash paid to employees + Cash paid for other operating expenses + Cash paid for interest + Cash paid for income taxes)

Cash received from customers $=$ Revenue + Decrease in accounts receivable $=\$ 37+\$ 3=\$ 40$ million

Cash paid to suppliers $=$ Cost of goods sold + Increase in inventory + Decrease in accounts payable

$=\$ 16+\$ 4+\$ 2=\$ 22$ million

Therefore, the company's operating cash flow $=\$ 40-\$ 22-$ Cash paid for salaries - Cash paid for interest - Cash paid for taxes $=\$ 40-\$ 22-\$ 6-\$ 2-\$ 4=\$ 6$ million.

\begin{enumerate}
  \setcounter{enumi}{14}
  \item $C$ is correct. Cost of goods sold of $\$ 80$ million plus the increase in inventory of $\$ 5$ million equals purchases from suppliers of $\$ 85$ million. The increase in accounts payable of $\$ 2$ million means that the company paid $\$ 83$ million in cash ( $\$ 85$ million minus $\$ 2$ million) to its suppliers.

  \item A is correct. Cost of goods sold of $\$ 75$ million less the decrease in inventory of $\$ 6$ million equals purchases from suppliers of $\$ 69$ million. The increase in accounts payable of $\$ 2$ million means that the company paid $\$ 67$ million in cash (\$69 million minus $\$ 2$ million).

  \item $C$ is correct. Beginning salaries payable of $\$ 3$ million plus salaries expense of $\$ 20$ million minus ending salaries payable of $\$ 1$ million equals $\$ 22$ million. Alternatively, the expense of $\$ 20$ million plus the $\$ 2$ million decrease in salaries payable equals $\$ 22$ million.

  \item $C$ is correct. Cash received from customers $=$ Sales + Decrease in accounts receivable $=254.6+4.9=259.5$. Cash paid to suppliers $=$ Cost of goods sold + Increase in inventory - Increase in accounts payable $=175.9+8.8-2.6=182.1$.

  \item $C$ is correct. Interest expense of $\$ 19$ million less the increase in interest payable of $\$ 3$ million equals interest paid of $\$ 16$ million. Tax expense of $\$ 6$ million plus the decrease in taxes payable of $\$ 4$ million equals taxes paid of $\$ 10$ million.

  \item A is correct. The amount of cash paid to suppliers is calculated as follows:

\end{enumerate}

$=$ Cost of goods sold - Decrease in inventory - Increase in accounts payable

$=\$ 27,264-\$ 501-\$ 1,063$

$=\$ 25,700$.

\begin{enumerate}
  \setcounter{enumi}{20}
  \item A is correct. Selling price (cash inflow) minus book value equals gain or loss on sale; therefore, gain or loss on sale plus book value equals selling price (cash inflow). The amount of loss is given $-\$ 2$ million. To calculate the book value of the equipment sold, find the historical cost of the equipment and the accumulated depreciation on the equipment.
\end{enumerate}

\begin{itemize}
  \item Beginning balance of equipment of $\$ 100$ million plus equipment purchased of $\$ 10$ million minus ending balance of equipment of $\$ 105$ million equals the historical cost of equipment sold, or $\$ 5$ million.

  \item Beginning accumulated depreciation of $\$ 40$ million plus depreciation expense for the year of $\$ 8$ million minus ending balance of accumulated depreciation of $\$ 46$ million equals accumulated depreciation on the equipment sold, or $\$ 2$ million.

  \item Therefore, the book value of the equipment sold was $\$ 5$ million minus $\$ 2$ million, or $\$ 3$ million.

  \item Because the loss on the sale of equipment was $\$ 2$ million, the amount of cash received must have been $\$ 1$ million.

\end{itemize}

\begin{enumerate}
  \setcounter{enumi}{21}
  \item A is correct. The increase of $\$ 42$ million in common stock and additional paid-in capital indicates that the company issued stock during the year. The increase in retained earnings of $\$ 15$ million indicates that the company paid $\$ 10$ million in cash dividends during the year, determined as beginning retained earnings of $\$ 100$ million plus net income of $\$ 25$ million minus ending retained earnings of $\$ 115$ million, which equals $\$ 10$ million in cash dividends.

  \item B is correct. An addition to net income is made when there is a loss on the retirement of debt, which is a non-operating loss. A gain on the sale of an asset and a decrease in deferred tax liability are both subtracted from net-income.

  \item B is correct. All dollar amounts are in millions. Net income (NI) for 2018 is $\$ 35$. This amount is the increase in retained earnings, $\$ 25$, plus the dividends paid, $\$ 10$. Depreciation of $\$ 25$ is added back to net income, and the increases in accounts receivable, $\$ 5$, and in inventory, $\$ 3$, are subtracted from net income because they are uses of cash. The decrease in accounts payable is also a use of cash and, therefore, a subtraction from net income. Thus, cash flow from operations is $\$ 25+\$ 10+\$ 25-\$ 5-\$ 3-\$ 7=\$ 45$.

  \item B is correct. To derive operating cash flow, the company would make the following adjustments to net income: Add depreciation (a non-cash expense) of $\$ 2$ million; add the decrease in accounts receivable of $\$ 3$ million; add the increase in accounts payable of $\$ 5$ million; and subtract the increase in inventory of $\$ 4$ million. Total additions would be $\$ 10$ million, and total subtractions would be $\$ 4$ million, which gives net additions of $\$ 6$ million.

  \item C is correct. An overall assessment of the major sources and uses of cash should be the first step in evaluating a cash flow statement.

  \item B is correct. The primary source of cash is operating activities. Cash flow provided by operating activity totaled $€ 13,796$ million in the most recent year. The primary use of cash is investing activities (total of $€ 10,245$ million). Dividends paid are classified as a financing activity.

  \item B is correct. An appropriate method to prepare a common-size cash flow statement is to show each line item on the cash flow statement as a percentage of net revenue. An alternative way to prepare a statement of cash flows is to show each item of cash inflow as a percentage of total inflows and each item of cash outflows as a percentage of total outflows.

  \item B is correct. Free cash flow to the firm can be computed as operating cash flows plus after-tax interest expense less capital expenditures. 30. A is correct. This ratio is an interest coverage ratio, measuring a company's ability to meet its interest obligations and indicating a company's solvency. This coverage ratio is based on cash flow information; another common coverage ratio uses a measure based on the income statement (earnings before interest, taxes, depreciation, and amortisation).

\end{enumerate}

\section{LEARNING MODULE
4}
\section{Financial Analysis Techniques}
by Elaine Henry, PhD, CFA, Thomas R. Robinson, PhD, CAIA, CFA, and J. Hennie van Greuning, DCom, CFA.

Elaine Henry, PhD, CFA, is at Stevens Institute of Technology (USA). Thomas R. Robinson, PhD, CAIA, CFA, Robinson Global Investment Management LLC, (USA). J. Hennie van Greuning, DCom, CFA, is at BIBD (Brunei).

\section{LEARNING OUTCOME}
\begin{center}
\begin{tabular}{|c|c|}
\hline
Mastery & The candidate should be able to: \\
\hline
 & $\begin{array}{l}\text { describe tools and techniques used in financial analysis, including } \\ \text { their uses and limitations }\end{array}$ \\
\hline
 & $\begin{array}{l}\text { identify, calculate, and interpret activity, liquidity, solvency, } \\ \text { profitability, and valuation ratios }\end{array}$ \\
\hline
 & $\begin{array}{l}\text { describe relationships among ratios and evaluate a company using } \\ \text { ratio analysis }\end{array}$ \\
\hline
 & $\begin{array}{l}\text { demonstrate the application of DuPont analysis of return on equity } \\ \text { and calculate and interpret effects of changes in its components }\end{array}$ \\
\hline
 & $\begin{array}{l}\text { calculate and interpret ratios used in equity analysis and credit } \\ \text { analysis }\end{array}$ \\
\hline
 & $\begin{array}{l}\text { explain the requirements for segment reporting and calculate and } \\ \text { interpret segment ratios }\end{array}$ \\
\hline
 & $\begin{array}{l}\text { describe how ratio analysis and other techniques can be used to } \\ \text { model and forecast earnings }\end{array}$ \\
\hline
\end{tabular}
\end{center}

Note: Changes in accounting standards as well as new rulings and/or pronouncements issued after the publication of the readings on financial reporting and analysis may cause some of the information in these readings to become dated. Candidates are not responsible for anything that occurs after the readings were published. In addition, candidates are expected to be familiar with the analytical frameworks contained in the readings, as well as the implications of alternative accounting methods for financial analysis and valuation discussed in the readings. Candidates are also responsible for the content of accounting standards, but not for the actual reference numbers. Finally, candidates should be aware that certain ratios may be defined and calculated differently. When alternative ratio definitions exist and no specific definition is given, candidates should use the ratio definitions emphasized in the readings.

\section{INTRODUCTION}
describe tools and techniques used in financial analysis, including their uses and limitations

Financial analysis tools can be useful in assessing a company's performance and trends in that performance. In essence, an analyst converts data into financial metrics that assist in decision making. Analysts seek to answer such questions as: How successfully has the company performed, relative to its own past performance and relative to its competitors? How is the company likely to perform in the future? Based on expectations about future performance, what is the value of this company or the securities it issues?

A primary source of data is a company's annual report, including the financial statements and notes, and management commentary (operating and financial review or management's discussion and analysis). This reading focuses on data presented in financial reports prepared under International Financial Reporting Standards (IFRS) and United States generally accepted accounting principles (US GAAP). However, financial reports do not contain all the information needed to perform effective financial analysis. Although financial statements do contain data about the past performance of a company (its income and cash flows) as well as its current financial condition (assets, liabilities, and owners' equity), such statements do not necessarily provide all the information useful for analysis nor do they forecast future results. The financial analyst must be capable of using financial statements in conjunction with other information to make projections and reach valid conclusions. Accordingly, an analyst typically needs to supplement the information found in a company's financial reports with other information, including information on the economy, industry, comparable companies, and the company itself.

This reading describes various techniques used to analyze a company's financial statements. Financial analysis of a company may be performed for a variety of reasons, such as valuing equity securities, assessing credit risk, conducting due diligence related to an acquisition, or assessing a subsidiary's performance. This reading will describe techniques common to any financial analysis and then discuss more specific aspects for the two most common categories: equity analysis and credit analysis.

Equity analysis incorporates an owner's perspective, either for valuation or performance evaluation. Credit analysis incorporates a creditor's (such as a banker or bondholder) perspective. In either case, there is a need to gather and analyze information to make a decision (ownership or credit); the focus of analysis varies because of the differing interest of owners and creditors. Both equity and credit analyses assess the entity's ability to generate and grow earnings, and cash flow, as well as any associated risks. Equity analysis usually places a greater emphasis on growth, whereas credit analysis usually places a greater emphasis on risks. The difference in emphasis reflects the different fundamentals of these types of investments: The value of a company's equity generally increases as the company's earnings and cash flow increase, whereas the value of a company's debt has an upper limit. ${ }^{1}$

The balance of this reading is organized as follows: Section 2 recaps the framework for financial statements and the place of financial analysis techniques within the framework. Section 3 provides a description of analytical tools and techniques. Section 4 explains how to compute, analyze, and interpret common financial ratios. Sections 5 through 8 explain the use of ratios and other analytical data in equity analysis, credit analysis, segment analysis, and forecasting, respectively. A summary of the key points and practice problems in the CFA Institute multiple-choice format conclude the reading.

\section{The Financial Analysis Process}
In financial analysis, it is essential to clearly identify and understand the final objective and the steps required to reach that objective. In addition, the analyst needs to know where to find relevant data, how to process and analyze the data (in other words, know the typical questions to address when interpreting data), and how to communicate the analysis and conclusions.

1 The upper limit is equal to the undiscounted sum of the principal and remaining interest payments (i.e., the present value of these contractual payments at a zero percent discount rate).

\section{The Objectives of the Financial Analysis Process}
Because of the variety of reasons for performing financial analysis, the numerous available techniques, and the often substantial amount of data, it is important that the analytical approach be tailored to the specific situation. Prior to beginning any financial analysis, the analyst should clarify the purpose and context, and clearly understand the following:

\begin{itemize}
  \item What is the purpose of the analysis? What questions will this analysis answer?

  \item What level of detail will be needed to accomplish this purpose?

  \item What data are available for the analysis?

  \item What are the factors or relationships that will influence the analysis?

  \item What are the analytical limitations, and will these limitations potentially impair the analysis?

\end{itemize}

Having clarified the purpose and context of the analysis, the analyst can select the set of techniques (e.g., ratios) that will best assist in making a decision. Although there is no single approach to structuring the analysis process, a general framework is set forth in Exhibit 1.2 The steps in this process were discussed in more detail in an earlier reading; the primary focus of this reading is on Phases 3 and 4, processing and analyzing data.

\section{Exhibit 1: A Financial Statement Analysis Framework}
Phase Sources of Information

\begin{enumerate}
  \item Articulate the purpose and context of the analysis.

  \item Collect input data.

  \item Process data.

  \item Analyze/interpret the processed data. - The nature of the analyst's function, such as evaluating an equity or debt investment or issuing a credit rating.

\end{enumerate}

\begin{itemize}
  \item Communication with client or supervisor on needs and concerns.

  \item Institutional guidelines related to developing specific work product.

  \item Financial statements, other financial data, questionnaires, and industry/economic data.

  \item Discussions with management, suppliers, customers, and competitors.

  \item Company site visits (e.g., to production facilities or retail stores).

  \item Data from the previous phase.

  \item Input data as well as processed data. Output

  \item Statement of the purpose or objective of analysis.

  \item A list (written or unwritten) of specific questions to be answered by the analysis.

  \item Nature and content of report to be provided.

  \item Timetable and budgeted resources for completion.

  \item Organized financial statements.

  \item Financial data tables.

  \item Completed questionnaires, if applicable.

  \item Adjusted financial statements.

  \item Common-size statements.

  \item Ratios and graphs.

  \item Forecasts.

  \item Analytical results.

\end{itemize}

2 Components of this framework have been adapted from van Greuning and Bratanovic (2003, p. 300)

and Benninga and Sarig (1997, pp. 134-156).

\begin{center}
\begin{tabular}{|c|c|c|}
\hline
Phase & Sources of Information & Output \\
\hline
$\begin{array}{l}\text { 5. Develop and communicate con- } \\ \text { clusions and recommendations } \\ \text { (e.g., with an analysis report). }\end{array}$ & $\begin{array}{l}\text { - Analytical results and previous reports. } \\ \text { - Institutional guidelines for published } \\ \text { reports. }\end{array}$ & $\begin{array}{l}\text { - Analytical report answering questions } \\ \text { posed in Phase } 1 . \\ \text { - Recommendation regarding the purpose } \\ \text { of the analysis, such as whether to make } \\ \text { an investment or grant credit. }\end{array}$ \\
\hline
6. Follow-up. & $\begin{array}{l}\text { Information gathered by periodically } \\ \text { repeating above steps as necessary to } \\ \text { determine whether changes to holdings } \\ \text { or recommendations are necessary. }\end{array}$ & - Updated reports and recommendations. \\
\hline
\end{tabular}
\end{center}

\section{Distinguishing between Computations and Analysis}
An effective analysis encompasses both computations and interpretations. A well-reasoned analysis differs from a mere compilation of various pieces of information, computations, tables, and graphs by integrating the data collected into a cohesive whole. Analysis of past performance, for example, should address not only what happened but also why it happened and whether it advanced the company's strategy. Some of the key questions to address include:

\begin{itemize}
  \item What aspects of performance are critical for this company to successfully compete in this industry?

  \item How well did the company's performance meet these critical aspects? (Established through computation and comparison with appropriate benchmarks, such as the company's own historical performance or competitors' performance.)

  \item What were the key causes of this performance, and how does this performance reflect the company's strategy? (Established through analysis.)

\end{itemize}

If the analysis is forward looking, additional questions include:

\begin{itemize}
  \item What is the likely impact of an event or trend? (Established through interpretation of analysis.)

  \item What is the likely response of management to this trend? (Established through evaluation of quality of management and corporate governance.)

  \item What is the likely impact of trends in the company, industry, and economy on future cash flows? (Established through assessment of corporate strategy and through forecasts.)

  \item What are the recommendations of the analyst? (Established through interpretation and forecasting of results of analysis.)

  \item What risks should be highlighted? (Established by an evaluation of major uncertainties in the forecast and in the environment within which the company operates.)

\end{itemize}

Example 1 demonstrates how a company's financial data can be analyzed in the context of its business strategy and changes in that strategy. An analyst must be able to understand the "why" behind the numbers and ratios, not just what the numbers and ratios are.

\section{EXAMPLE 1}
\section{Strategy Reflected in Financial Performance}
Apple Inc. engages in the design, manufacture, and sale of computer hardware, mobile devices, operating systems and related products, and services. It also operates retail and online stores. Microsoft develops, licenses, and supports software products, services, and technology devices through a variety of channels including retail stores in recent years. Selected financial data for 2015 through 2017 for these two companies are given below. Apple's fiscal year (FY) ends on the final Saturday in September (for example, FY2017 ended on 30 September 2017). Microsoft's fiscal year ends on 30 June (for example, FY2017 ended on 30 June 2017).

\section{Selected Financial Data for Apple (Dollars in Millions)}
\begin{center}
\begin{tabular}{lccc}
\hline
Fiscal year & $\mathbf{2 0 1 7}$ & $\mathbf{2 0 1 6}$ & $\mathbf{2 0 1 5}$ \\
\hline
Net sales (or Revenue) & 229,234 & 215,639 & 233,715 \\
Gross margin & 88,186 & 84,263 & 93,626 \\
Operating income & 61,344 & 60,024 & 71,230 \\
\hline
\end{tabular}
\end{center}

\section{Selected Financial Data for Microsoft (Dollars in Millions)*}
\begin{center}
\begin{tabular}{llll}
\hline
Fiscal year & $\mathbf{2 0 1 7}$ & $\mathbf{2 0 1 6}$ & $\mathbf{2 0 1 5}$ \\
\hline
Net sales (or Revenue) & 89,950 & 85,320 & 93,580 \\
Gross margin & 55,689 & 52,540 & 60,542 \\
Operating income & 22,326 & 20,182 & 18,161 \\
\hline
\end{tabular}
\end{center}

\begin{itemize}
  \item Microsoft revenue for 2017 and 2016 were subsequently revised in the company's 2018 10-K report due to changes in revenue recognition and lease accounting standards.
\end{itemize}

Source: 10-K reports for Apple and Microsoft.

Apple reported a 7.7 percent decrease in net sales from FY2015 to FY2016 and an increase of 6.3 percent from FY2016 to FY2017 for an overall slight decline over the three-year period. Gross margin decreased 10.0 percent from FY2015 to FY2016 and increased 4.7 percent from FY2016 to FY2017. This also represented an overall decline in gross margin over the three-year period. The company's operating income exhibited similar trends.

Microsoft reported an 8.8 percent decrease in net sales from FY2015 to FY2016 and an increase of 5.4 percent from FY2016 to FY2017 for an overall slight decline over the three-year period. Gross margin decreased 13.2 percent from FY2015 to FY2016 and increased 6.0 percent from FY2016 to FY2017. Similar to Apple, this represented an overall decline in gross margin over the three-year period. Microsoft's operating income on the other hand exhibited growth each year and for the three-year period. Overall growth in operating income was $23 \%$.

What caused Microsoft's growth in operating income while Apple and Microsoft had similar negative trends in sales and gross margin? Apple's decline in sales, gross margin, and operating income from FY2015 to FY2016 was caused by declines in iPhone sales and weakness in foreign currencies relative to the US dollar. FY2017 saw a rebound in sales of iPhones, Mac computers, and services offset somewhat by continued weaknesses in foreign currencies. Microsoft similarly had declines in revenue and gross margin from sales of its devices and Windows software in FY2016, as well as negative impacts from foreign currency weakness. Microsoft's increase in revenue and gross margin in FY2017 was driven by the acquisition of LinkedIn, higher sales of Microsoft Office software, and higher sales of cloud services. The driver in the continuous increase in operating income for Microsoft was a large decline over the three-year period in impairment, integration, and restructuring charges. Microsoft recorded a $\$ 10$ billion charge in FY2015 related to its phone business, and there were further charges of $\$ 1.1$ billion in FY2016 and \$306 million in FY2017. Absent these large write-offs, Microsoft would have had a trend similar to Apple's in operating income over the three-year period.

Analysts often need to communicate the findings of their analysis in a written report. Their reports should communicate how conclusions were reached and why recommendations were made. For example, a report might present the following:

\begin{itemize}
  \item the purpose of the report, unless it is readily apparent;

  \item relevant aspects of the business context:

  \item economic environment (country/region, macro economy, sector);

  \item financial and other infrastructure (accounting, auditing, rating agencies);

  \item legal and regulatory environment (and any other material limitations on the company being analyzed);

  \item evaluation of corporate governance and assessment of management strategy, including the company's competitive advantage(s);

  \item assessment of financial and operational data, including key assumptions in the analysis; and

  \item conclusions and recommendations, including limitations of the analysis and risks.

\end{itemize}

An effective narrative and well supported conclusions and recommendations are normally enhanced by using 3-10 years of data, as well as analytic techniques appropriate to the purpose of the report.

\section{ANALYTICAL TOOLS AND TECHNIQUES}
describe tools and techniques used in financial analysis, including their uses and limitations

The tools and techniques presented in this section facilitate evaluations of company data. Evaluations require comparisons. It is difficult to say that a company's financial performance was "good" without clarifying the basis for comparison. In assessing a company's ability to generate and grow earnings and cash flow, and the risks related to those earnings and cash flows, the analyst draws comparisons to other companies (cross-sectional analysis) and over time (trend or time-series analysis).

For example, an analyst may wish to compare the profitability of companies competing in a global industry. If the companies differ significantly in size and/or report their financial data in different currencies, comparing net income as reported is not useful. Ratios (which express one number in relation to another) and common-size financial statements can remove size as a factor and enable a more relevant comparison. To achieve comparability across companies reporting in different currencies, one approach is to translate all reported numbers into a common currency using exchange rates at the end of a period. Others may prefer to translate reported numbers using the average exchange rates during the period. Alternatively, if the focus is primarily on ratios, comparability can be achieved without translating the currencies.

The analyst may also want to examine comparable performance over time. Again, the nominal currency amounts of sales or net income may not highlight significant changes. To address this challenge, horizontal financial statements (whereby quantities are stated in terms of a selected base year value) can make such changes more apparent. Another obstacle to comparison is differences in fiscal year end. To achieve comparability, one approach is to develop trailing twelve months data, which will be described in a section below. Finally, it should be noted that differences in accounting standards can limit comparability.

\section{EXAMPLE 2}
\section{Ratio Analysis}
An analyst is examining the profitability of two international companies with large shares of the global personal computer market: Acer Inc. and Lenovo Group Limited. Acer has pursued a strategy of selling its products at affordable prices. In contrast, Lenovo aims to achieve higher selling prices by stressing the high engineering quality of its personal computers for business use. Acer reports in TWD, ${ }^{3}$ and Lenovo reports in USD. For Acer, fiscal year end is 31 December. For Lenovo, fiscal year end is 31 March; thus, FY2017 ended 31 March 2018.

The analyst collects the data shown in Exhibit 2 below. Use this information to answer the following questions:

\section{Exhibit 2}
Acer

\begin{center}
\begin{tabular}{lccccc}
\hline
TWD Millions & FY2013 & FY2014 & FY2015 & FY2016 & FY2017 \\
\hline
Revenue & 360,132 & 329,684 & 263,775 & 232,724 & 237,275 \\
Gross profit & 22,550 & 28,942 & 24,884 & 23,212 & 25,361 \\
Net income & $(20,519)$ & 1,791 & 604 & $(4,901)$ & 2,797 \\
\hline
\end{tabular}
\end{center}

\section{Lenovo}
\begin{center}
\begin{tabular}{|c|c|c|c|c|c|}
\hline
USD Millions & FY2013* & FY2014* & FY2015* & FY2016* & FY2017* \\
\hline
Revenue & 38,707 & 46,296 & 44,912 & 43,035 & 45,350 \\
\hline
Gross profit & 5,064 & 6,682 & 6,624 & 6,105 & 6,272 \\
\hline
Net income (Loss) & 817 & 837 & (145) & 530 & (127) - (- ) \\
\hline
\end{tabular}
\end{center}

\begin{itemize}
  \item Fiscal years for Lenovo end 31 March. Thus FY2017 represents the fiscal year ended 31 March 2018; the same applies respectively for prior years. 1. Which company is larger based on the amount of revenue, in US\$, reported in fiscal year 2017? For FY2017, assume the relevant, average exchange rate was $30.95 \mathrm{TWD} / \mathrm{USD}$.
\end{itemize}

\section{Solution to 1:}
Lenovo is much larger than Acer based on FY2017 revenues in USD terms. Lenovo's FY2017 revenues of \$USD45.35 billion are considerably higher than Acer's USD7.67 billion (= TWD237.275 million/30.95).

Acer: At the assumed average exchange rate of 30.95 TWD/USD, Acer's FY2017 revenues are equivalent to USD7.67 billion (= TWD237.275 million $\div 30.95 \mathrm{TWD} / \mathrm{USD})$

Lenovo: Lenovo's FY2017 revenues totaled USD45.35 billion.

Note: Comparing the size of companies reporting in different currencies requires translating reported numbers into a common currency using exchange rates at some point in time. This solution converts the revenues of Acer to billions of USD using the average exchange rate of the fiscal period. It would be equally informative (and would yield the same conclusion) to convert the revenues of Lenovo to TWD.

\begin{enumerate}
  \setcounter{enumi}{1}
  \item Which company had the higher revenue growth from FY2016 to FY2017? FY2013 to FY2017?
\end{enumerate}

\section{Solution to 2:}
The growth in Lenovo's revenue was much higher than Acer's in the most recent fiscal year and for the five-year period.

\begin{center}
\begin{tabular}{lcc}
\hline
 & $\begin{array}{c}\text { Change in Revenue FY2016 } \\ \text { versus FY2017 (\%) }\end{array}$ & $\begin{array}{c}\text { Change in Revenue FY2013 to } \\ \text { FY2017 (\%) }\end{array}$ \\
\hline
Acer & 1.96 & $(34.11)$ \\
Lenovo & 5.38 & 17.16 \\
\hline
\end{tabular}
\end{center}

The table shows two growth metrics. Calculations are illustrated using the revenue data for Acer:

The change in Acer's revenue for FY2016 versus FY2017 is 1.96\% percent calculated as $(237,275-232,724) \div 232,724$ or equivalently $(237,275 \div$ 232,724) - 1. The change in Acer's revenue from FY2013 to FY2017 is a decline of $34.11 \%$.

\begin{enumerate}
  \setcounter{enumi}{2}
  \item How do the companies compare, based on profitability?
\end{enumerate}

\section{Solution to 3:}
Profitability can be assessed by comparing the amount of gross profit to revenue and the amount of net income to revenue. The following table presents these two profitability ratios-gross profit margin (gross profit divided by revenue) and net profit margin (net income divided by revenue)-for each year.

\begin{center}
\begin{tabular}{lccccc}
\hline
Acer & FY2013 (\%) & FY2014 (\%) & FY2015 (\%) & FY2016 (\%) & FY2017 (\%) \\
\hline
Gross profit margin & 6.26 & 8.78 & 9.43 & 9.97 &  \\
Net profit margin & $(5.70)$ & 0.54 & 0.23 & $(2.11)$ & 1.18 \\
\hline
\end{tabular}
\end{center}

\begin{center}
\begin{tabular}{lccccc}
\hline
Lenovo & FY2013 (\%) & FY2014 (\%) & FY2015 (\%) & FY2016 (\%) & FY2017 (\%) \\
\hline
Gross profit margin & 13.08 & 14.43 & 14.75 & 14.19 &  \\
Net profit margin & 2.11 & 1.81 & $(0.32)$ & 1.23 & $(0.28)$ \\
\hline
\end{tabular}
\end{center}

The net profit margins indicate that both companies' profitability is relatively low. Acer's net profit margin is lower than Lenovo's in three out of the five years. Acer's gross profit margin increased each year but remains significantly below that of Lenovo. Lenovo's gross profit margin grew from FY2013 to FY2015 and then declined in FY2016 and FY2017. Overall, Lenovo is the more profitable company, likely attributable to its larger size and commensurate economies of scale. (Lenovo has the largest share of the personal computer market relative to other personal computer companies.)

Section 3.1 describes the tools and techniques of ratio analysis in more detail. Sections 3.2 to 3.4 describe other tools and techniques.

\section{FINANCIAL RATIO ANALYSIS}
describe tools and techniques used in financial analysis, including their uses and limitations

There are many relationships among financial accounts and various expected relationships from one point in time to another. Ratios are a useful way of expressing these relationships. Ratios express one quantity in relation to another (usually as a quotient).

Extensive academic research has examined the importance of ratios in predicting stock returns (Ou and Penman, 1989; Abarbanell and Bushee, 1998) or credit failure (Altman, 1968; Ohlson, 1980; Hopwood et al., 1994). This research has found that financial statement ratios are effective in selecting investments and in predicting financial distress. Practitioners routinely use ratios to derive and communicate the value of companies and securities.

Several aspects of ratio analysis are important to understand. First, the computed ratio is not "the answer." The ratio is an indicator of some aspect of a company's performance, telling what happened but not why it happened. For example, an analyst might want to answer the question: Which of two companies was more profitable? As demonstrated in the previous example, the net profit margin, which expresses profit relative to revenue, can provide insight into this question. Net profit margin is calculated by dividing net income by revenue: ${ }^{4}$

Net income

Revenue

Assume Company A has $€ 100,000$ of net income and Company B has $€ 200,000$ of net income. Company B generated twice as much income as Company A, but was it more profitable? Assume further that Company A has $€ 2,000,000$ of revenue, and thus a net profit margin of 5 percent, and Company B has $€ 6,000,000$ of revenue, and

4 The term "sales" is often used interchangeably with the term "revenues." Other times it is used to refer to revenues derived from sales of products versus services. The income statement usually reflects "revenues" or "sales" after returns and allowances (e.g., returns of products or discounts offered after a sale to induce the customer to not return a product). Additionally, in some countries, including the United Kingdom and South Africa, the term "turnover" is used in the sense of "revenue." thus a net profit margin of 3.33 percent. Expressing net income as a percentage of revenue clarifies the relationship: For each $€ 100$ of revenue, Company A earns $€ 5$ in net income, whereas Company B earns only $€ 3.33$ for each $€ 100$ of revenue. So, we can now answer the question of which company was more profitable in percentage terms: Company A was more profitable, as indicated by its higher net profit margin of 5 percent. Note that Company A was more profitable despite the fact that Company $B$ reported higher absolute amounts of net income and revenue. However, this ratio by itself does not tell us why Company A has a higher profit margin. Further analysis is required to determine the reason (perhaps higher relative sales prices or better cost control or lower effective tax rates).

Company size sometimes confers economies of scale, so the absolute amounts of net income and revenue are useful in financial analysis. However, ratios control for the effect of size, which enhances comparisons between companies and over time.

A second important aspect of ratio analysis is that differences in accounting policies (across companies and across time) can distort ratios, and a meaningful comparison may, therefore, involve adjustments to the financial data. Third, not all ratios are necessarily relevant to a particular analysis. The ability to select a relevant ratio or ratios to answer the research question is an analytical skill. Finally, as with financial analysis in general, ratio analysis does not stop with computation; interpretation of the result is essential. In practice, differences in ratios across time and across companies can be subtle, and interpretation is situation specific.

\section{The Universe of Ratios}
There are no authoritative bodies specifying exact formulas for computing ratios or providing a standard, comprehensive list of ratios. Formulas and even names of ratios often differ from analyst to analyst or from database to database. The number of different ratios that can be created is practically limitless. There are, however, widely accepted ratios that have been found to be useful. Section 4 of this reading will focus primarily on these broad classes and commonly accepted definitions of key ratios. However, the analyst should be aware that different ratios may be used in practice and that certain industries have unique ratios tailored to the characteristics of that industry. When faced with an unfamiliar ratio, the analyst can examine the underlying formula to gain insight into what the ratio is measuring. For example, consider the following ratio formula:

Operating income

Average total assets

Never having seen this ratio, an analyst might question whether a result of 12 percent is better than 8 percent. The answer can be found in the ratio itself. The numerator is operating income and the denominator is average total assets, so the ratio can be interpreted as the amount of operating income generated per unit of assets. For every $€ 100$ of average total assets, generating $€ 12$ of operating income is better than generating $€ 8$ of operating income. Furthermore, it is apparent that this particular ratio is an indicator of profitability (and, to a lesser extent, efficiency in use of assets in generating operating profits). When encountering a ratio for the first time, the analyst should evaluate the numerator and denominator to assess what the ratio is attempting to measure and how it should be interpreted. This is demonstrated in Example 3.

\section{EXAMPLE 3}
\section{Interpreting a Financial Ratio}
\begin{enumerate}
  \item A US insurance company reports that its "combined ratio" is determined by dividing losses and expenses incurred by net premiums earned. It reports the following combined ratios:
\end{enumerate}

\begin{center}
\begin{tabular}{lccccc}
\hline
Fiscal Year & $\mathbf{5}$ & $\mathbf{4}$ & $\mathbf{3}$ & $\mathbf{2}$ & $\mathbf{1}$ \\
\hline
Combined ratio & $90.1 \%$ & $104.0 \%$ & $98.5 \%$ & $104.1 \%$ & $101.1 \%$ \\
\hline
\end{tabular}
\end{center}

Explain what this ratio is measuring and compare the results reported for each of the years shown in the chart. What other information might an analyst want to review before making any conclusions on this information?

\section{Solution:}
The combined ratio is a profitability measure. The ratio is explaining how much costs (losses and expenses) were incurred for every dollar of revenue (net premiums earned). The underlying formula indicates that a lower value for this ratio is better. The Year 5 ratio of 90.1 percent means that for every dollar of net premiums earned, the costs were $\$ 0.901$, yielding a gross profit of $\$ 0.099$. Ratios greater than 100 percent indicate an overall loss. A review of the data indicates that there does not seem to be a consistent trend in this ratio. Profits were achieved in Years 5 and 3. The results for Years 4 and 2 show the most significant costs at approximately 104 percent.

The analyst would want to discuss this data further with management and understand the characteristics of the underlying business. He or she would want to understand why the results are so volatile. The analyst would also want to determine what should be used as a benchmark for this ratio.

The Operating income/Average total assets ratio shown above is one of many versions of the return on assets (ROA) ratio. Note that there are other ways of specifying this formula based on how assets are defined. Some financial ratio databases compute ROA using the ending value of assets rather than average assets. In limited cases, one may also see beginning assets in the denominator. Which one is right? It depends on what you are trying to measure and the underlying company trends. If the company has a stable level of assets, the answer will not differ greatly under the three measures of assets (beginning, average, and ending). However, if the assets are growing (or shrinking), the results will differ among the three measures. When assets are growing, operating income divided by ending assets may not make sense because some of the income would have been generated before some assets were purchased, and this would understate the company's performance. Similarly, if beginning assets are used, some of the operating income later in the year may have been generated only because of the addition of assets; therefore, the ratio would overstate the company's performance. Because operating income occurs throughout the period, it generally makes sense to use some average measure of assets. A good general rule is that when an income statement or cash flow statement number is in the numerator of a ratio and a balance sheet number is in the denominator, then an average should be used for the denominator. It is generally not necessary to use averages when only balance sheet numbers are used in both the numerator and denominator because both are determined as of the same date. However, in some instances, even ratios that only use balance sheet data may use averages. For example, return on equity (ROE), which is defined as net income divided by average shareholders' equity, can be decomposed into other ratios, some of which only use balance sheet data. In decomposing ROE into component ratios, if an average is used in one of the component ratios then it should be used in the other component ratios. The decomposition of ROE is discussed further in Section 4.6.2.

If an average is used, judgment is also required about what average should be used. For simplicity, most ratio databases use a simple average of the beginning and end-of-year balance sheet amounts. If the company's business is seasonal so that levels of assets vary by interim period (semiannual or quarterly), then it may be beneficial to take an average over all interim periods, if available. (If the analyst is working within a company and has access to monthly data, this can also be used.)

\section{Value, Purposes, and Limitations of Ratio Analysis}
The value of ratio analysis is that it enables a financial analyst to evaluate past performance, assess the current financial position of the company, and gain insights useful for projecting future results. As noted previously, the ratio itself is not "the answer" but is an indicator of some aspect of a company's performance. Financial ratios provide insights into:

\begin{itemize}
  \item economic relationships within a company that help analysts project earnings and free cash flow;

  \item a company's financial flexibility, or ability to obtain the cash required to grow and meet its obligations, even if unexpected circumstances develop;

  \item management's ability;

  \item changes in the company and/or industry over time; and

  \item comparability with peer companies or the relevant industry(ies).

\end{itemize}

There are also limitations to ratio analysis. Factors to consider include:

\begin{itemize}
  \item The heterogeneity or homogeneity of a company's operating activities.
\end{itemize}

Companies may have divisions operating in many different industries. This can make it difficult to find comparable industry ratios to use for comparison purposes.

\begin{itemize}
  \item The need to determine whether the results of the ratio analysis are consistent. One set of ratios may indicate a problem, whereas another set may indicate that the potential problem is only short term in nature.

  \item The need to use judgment. A key issue is whether a ratio for a company is within a reasonable range. Although financial ratios are used to help assess the growth potential and risk of a company, they cannot be used alone to directly value a company or its securities, or to determine its creditworthiness. The entire operation of the company must be examined, and the external economic and industry setting in which it is operating must be considered when interpreting financial ratios.

  \item The use of alternative accounting methods. Companies frequently have latitude when choosing certain accounting methods. Ratios taken from financial statements that employ different accounting choices may not be comparable unless adjustments are made. Some important accounting considerations include the following:

  \item FIFO (first in, first out), LIFO (last in, first out), or average cost inventory valuation methods (IFRS does not allow LIFO);

  \item Cost or equity methods of accounting for unconsolidated affiliates; - Straight line or accelerated methods of depreciation; and

  \item Operating or finance lease treatment for lessors (under US GAAP, the type of lease affects classifications of expenses; under IFRS, operating lease treatment for lessors is not applicable).

\end{itemize}

The expanding use of IFRS and past convergence efforts between IFRS and US GAAP make the financial statements of different companies more comparable and may overcome some of these difficulties. Nonetheless, there will remain accounting choices that the analyst must consider.

\section{Sources of Ratios}
Ratios may be computed using data obtained directly from companies' financial statements or from a database such as Bloomberg, Compustat, FactSet, or Thomson Reuters. The information provided by the database may include information as reported in companies' financial statements and ratios calculated based on the information. These databases are popular because they provide easy access to many years of historical data so that trends over time can be examined. They also allow for ratio calculations based on periods other than the company's fiscal year, such as for the trailing 12 months (TTM) or most recent quarter (MRQ).

\section{EXAMPLE 4}
\section{Trailing Twelve Months}
\begin{enumerate}
  \item On 15 July, an analyst is examining a company with a fiscal year ending on 31 December. Use the following data to calculate the company's trailing 12 month earnings (for the period ended 30 June 2018):
\end{enumerate}

\begin{itemize}
  \item Earnings for the year ended 31 December, 2017: $\$ 1,200$;

  \item Earnings for the six months ended 30 June 2017: $\$ 550$; and

  \item Earnings for the six months ended 30 June 2018: $\$ 750$.

\end{itemize}

\section{Solution:}
The company's trailing 12 months earnings is $\$ 1,400$, calculated as $\$ 1,200-$ $\$ 550+\$ 750$

Analysts should be aware that the underlying formulas for ratios may differ by vendor. The formula used should be obtained from the vendor, and the analyst should determine whether any adjustments are necessary. Furthermore, database providers often exercise judgment when classifying items. For example, operating income may not appear directly on a company's income statement, and the vendor may use judgment to classify income statement items as "operating" or "non-operating." Variation in such judgments would affect any computation involving operating income. It is therefore a good practice to use the same source for data when comparing different companies or when evaluating the historical record of a single company. Analysts should verify the consistency of formulas and data classifications by the data source. Analysts should also be mindful of the judgments made by a vendor in data classifications and refer back to the source financial statements until they are comfortable that the classifications are appropriate.

Collection of financial data from regulatory filings and calculation of ratios can be automated. The eXtensible Business Reporting Language (XBRL) is a mechanism that attaches "smart tags" to financial information (e.g., total assets), so that software can automatically collect the data and perform desired computations. The organization developing XBRL (\href{http://www.xbrl.org}{www.xbrl.org}) is an international nonprofit consortium of over 600 members from companies, associations, and agencies, including the International Accounting Standards Board. Many stock exchanges and regulatory agencies around the world now use XBRL for receiving and distributing public financial reports from listed companies.

Analysts can compare a subject company to similar (peer) companies in these databases or use aggregate industry data. For non-public companies, aggregate industry data can be obtained from such sources as Annual Statement Studies by the Risk Management Association or Dun \& Bradstreet. These publications typically provide industry data with companies sorted into quartiles. By definition, twenty-five percent of companies' ratios fall within the lowest quartile, 25 percent have ratios between the lower quartile and median value, and so on. Analysts can then determine a company's relative standing in the industry.

\section{COMMON SIZE BALANCE SHEETS AND INCOME STATEMENTS}
describe tools and techniques used in financial analysis, including their uses and limitations

Common-size analysis involves expressing financial data, including entire financial statements, in relation to a single financial statement item, or base. Items used most frequently as the bases are total assets or revenue. In essence, common-size analysis creates a ratio between every financial statement item and the base item.

Common-size analysis was demonstrated in readings for the income statement, balance sheet, and cash flow statement. In this section, we present common-size analysis of financial statements in greater detail and include further discussion of their interpretation.

\section{Common-Size Analysis of the Balance Sheet}
A vertical ${ }^{5}$ common-size balance sheet, prepared by dividing each item on the balance sheet by the same period's total assets and expressing the results as percentages, highlights the composition of the balance sheet. What is the mix of assets being used? How is the company financing itself? How does one company's balance sheet composition compare with that of peer companies, and what are the reasons for any differences?

A horizontal common-size balance sheet, prepared by computing the increase or decrease in percentage terms of each balance sheet item from the prior year or prepared by dividing the quantity of each item by a base year quantity of the item, highlights changes in items. These changes can be compared to expectations. The section on trend analysis below will illustrate a horizontal common-size balance sheet.

Exhibit 3 presents a vertical common-size (partial) balance sheet for a hypothetical company in two time periods. In this example, receivables have increased from 35 percent to 57 percent of total assets and the ratio has increased by 63 percent from Period 1 to Period 2. What are possible reasons for such an increase? The increase might

5 The term vertical analysis is used to denote a common-size analysis using only one reporting period or one base financial statement, whereas horizontal analysis refers to an analysis comparing a specific financial statement with prior or future time periods or to a cross-sectional analysis of one company with another. indicate that the company is making more of its sales on a credit basis rather than a cash basis, perhaps in response to some action taken by a competitor. Alternatively, the increase in receivables as a percentage of assets may have occurred because of a change in another current asset category, for example, a decrease in the level of inventory; the analyst would then need to investigate why that asset category has changed. Another possible reason for the increase in receivables as a percentage of assets is that the company has lowered its credit standards, relaxed its collection procedures, or adopted more aggressive revenue recognition policies. The analyst can turn to other comparisons and ratios (e.g., comparing the rate of growth in accounts receivable with the rate of growth in sales) to help determine which explanation is most likely.

\section{Exhibit 3: Vertical Common-Size (Partial) Balance Sheet for a Hypothetical}
 CompanyPeriod 1

Percent of Total Assets Period 2

Percent of Total Assets

\begin{center}
\begin{tabular}{lcc}
\hline
Cash & 25 & 15 \\
Receivables & 35 & 57 \\
Inventory & 35 & 20 \\
Fixed assets, net of &  & 8 \\
depreciation & 5 & 100 \\
\hline
Total assets & 100 & 8 \\
\hline
\end{tabular}
\end{center}

\section{Common-Size Analysis of the Income Statement}
A vertical common-size income statement divides each income statement item by revenue, or sometimes by total assets (especially in the case of financial institutions). If there are multiple revenue sources, a decomposition of revenue in percentage terms is useful. Exhibit 4 presents a hypothetical company's vertical common-size income statement in two time periods. Revenue is separated into the company's four services, each shown as a percentage of total revenue.

In this example, revenues from Service A have become a far greater percentage of the company's total revenue ( 30 percent in Period 1 and 45 percent in Period 2). What are possible reasons for and implications of this change in business mix? Did the company make a strategic decision to sell more of Service A, perhaps because it is more profitable? Apparently not, because the company's earnings before interest, taxes, depreciation, and amortisation (EBITDA) declined from 53 percent of sales to 45 percent, so other possible explanations should be examined. In addition, we note from the composition of operating expenses that the main reason for this decline in profitability is that salaries and employee benefits have increased from 15 percent to 25 percent of total revenue. Are more highly compensated employees required for Service A? Were higher training costs incurred in order to increase revenues from Service A? If the analyst wants to predict future performance, the causes of these changes must be understood.

In addition, Exhibit 4 shows that the company's income tax as a percentage of sales has declined dramatically (from 15 percent to 8 percent). Furthermore, taxes as a percentage of earnings before tax (EBT) (the effective tax rate, which is usually the more relevant comparison), have decreased from 36 percent $(=15 / 42)$ to 24 percent $(=8 / 34)$. Is Service A, which in Period 2 is a greater percentage of total revenue, provided in a jurisdiction with lower tax rates? If not, what is the explanation for the change in effective tax rate?

The observations based on Exhibit 4 summarize the issues that can be raised through analysis of the vertical common-size income statement.

Exhibit 4: Vertical Common-Size Income Statement for Hypothetical Company

\begin{center}
\begin{tabular}{|c|c|c|}
\hline
 & $\begin{array}{c}\text { Period } 1 \\ \text { Percent of Total } \\ \text { Revenue }\end{array}$ & $\begin{array}{c}\text { Period } 2 \\ \text { Percent of Tota } \\ \text { Revenue }\end{array}$ \\
\hline
Revenue source: Service A & 30 & 45 \\
\hline
Revenue source: Service B & 23 & 20 \\
\hline
Revenue source: Service C & 30 & 30 \\
\hline
Revenue source: Service D & 17 & 5 \\
\hline
Total revenue & 100 & 100 \\
\hline
\multicolumn{3}{|c|}{Operating expenses (excluding depreciation)} \\
\hline
Salaries and employee benefits & 15 & 25 \\
\hline
Administrative expenses & 22 & 20 \\
\hline
Rent expense & 10 & 10 \\
\hline
EBITDA & 53 & 45 \\
\hline
Depreciation and amortisation & 4 & 4 \\
\hline
EBIT & 49 & 41 \\
\hline
Interest paid & 7 & 7 \\
\hline
EBT & 42 & 34 \\
\hline
Income tax provision & 15 & 8 \\
\hline
Net income & 27 & 26 \\
\hline
\end{tabular}
\end{center}

$E B I T=$ earnings before interest and tax.

CROSS-SECTIONAL, TREND ANALYSIS \& RELATIONSHIPS IN FINANCIAL STATEMENTS describe tools and techniques used in financial analysis, including their uses and limitations

As noted previously, ratios and common-size statements derive part of their meaning through comparison to some benchmark. Cross-sectional analysis (sometimes called "relative analysis") compares a specific metric for one company with the same metric for another company or group of companies, allowing comparisons even though the companies might be of significantly different sizes and/or operate in different currencies. This is illustrated in Exhibit 5. Exhibit 5: Vertical Common-Size (Partial) Balance Sheet for Two Hypothetical Companies

\begin{center}
\begin{tabular}{lcc}
\hline
Assets & $\begin{array}{c}\text { Company 1 } \\ \text { Percent of Total Assets }\end{array}$ & $\begin{array}{c}\text { Company } 2 \\ \text { Percent of Total Assets }\end{array}$ \\
\hline
Cash & 38 & 12 \\
Receivables & 33 & 55 \\
Inventory & 27 & 24 \\
Fixed assets net of & 1 & 2 \\
depreciation & 1 & $\mathbf{1 0 0}$ \\
Investments & $\mathbf{1 0 0}$ &  \\
Total Assets &  &  \\
\hline
\end{tabular}
\end{center}

Exhibit 5 presents a vertical common-size (partial) balance sheet for two hypothetical companies at the same point in time. Company 1 is clearly more liquid (liquidity is a function of how quickly assets can be converted into cash) than Company 2, which has only 12 percent of assets available as cash, compared with the highly liquid Company 1 , which has 38 percent of assets available as cash. Given that cash is generally a relatively low-yielding asset and thus not a particularly efficient use of excess funds, why does Company 1 hold such a large percentage of total assets in cash? Perhaps the company is preparing for an acquisition, or maintains a large cash position as insulation from a particularly volatile operating environment. Another issue highlighted by the comparison in this example is the relatively high percentage of receivables in Company 2's assets, which may indicate a greater proportion of credit sales, overall changes in asset composition, lower credit or collection standards, or aggressive accounting policies.

\section{Trend Analysis ${ }^{6}$}
When looking at financial statements and ratios, trends in the data, whether they are improving or deteriorating, are as important as the current absolute or relative levels. Trend analysis provides important information regarding historical performance and growth and, given a sufficiently long history of accurate seasonal information, can be of great assistance as a planning and forecasting tool for management and analysts.

Exhibit 6 presents a partial balance sheet for a hypothetical company over five periods. The last two columns of the table show the changes for Period 5 compared with P+eriod 4, expressed both in absolute currency (in this case, dollars) and in percentages. A small percentage change could hide a significant currency change and vice versa, prompting the analyst to investigate the reasons despite one of the changes being relatively small. In this example, the largest percentage change was in investments, which decreased by 33.3 percent. ${ }^{7}$ However, an examination of the absolute currency amount of changes shows that investments changed by only $\$ 2$ million, and the more significant change was the $\$ 12$ million increase in receivables.

Another way to present data covering a period of time is to show each item in relation to the same item in a base year (i.e., a horizontal common-size balance sheet). Exhibit 7 and Exhibit 8 illustrate alternative presentations of horizontal common-size balance sheets. Exhibit 7 presents the information from the same partial balance

6 In financial statement analysis, the term "trend analysis" usually refers to comparisons across time periods of 3-10 years not involving statistical tools. This differs from the use of the term in the quantitative methods portion of the CFA curriculum, where "trend analysis" refers to statistical methods of measuring patterns in time-series data.

7 Percentage change is calculated as (Ending value - Beginning value)/Beginning value, or equivalently, (Ending value/Beginning value) -1 . sheet as in Exhibit 6, but indexes each item relative to the same item in Period 1. For example, in Period 2, the company had $\$ 29$ million cash, which is 74 percent or 0.74 of the amount of cash it had in Period 1. Expressed as an index relative to Period 1, where each item in Period 1 is given a value of 1.00 , the value in Period 2 would be $0.74(\$ 29 / \$ 39=0.74)$. In Period 3, the company had $\$ 27$ million cash, which is 69 percent of the amount of cash it had in Period $1(\$ 27 / \$ 39=0.69)$.

Exhibit 8 presents the percentage change in each item, relative to the previous year. For example, the change in cash from Period 1 to Period 2 was -25.6 percent $(\$ 29 / \$ 39-1=-0.256)$, and the change in cash from Period 2 to Period 3 was -6.9 percent $(\$ 27 / \$ 29-1=-0.069)$. An analyst will select the horizontal common-size balance that addresses the particular period of interest. Exhibit 7 clearly highlights that in Period 5 compared to Period 1 , the company has less than half the amount of cash, four times the amount of investments, and eight times the amount of property, plant, and equipment. Exhibit 8 highlights year-to-year changes: For example, cash has declined in each period. Presenting data this way highlights significant changes. Again, note that a mathematically big change is not necessarily an important change. For example, fixed assets increased 100 percent, i.e., doubled between Period 1 and 2; however, as a proportion of total assets, fixed assets increased from 1 percent of total assets to 2 percent of total assets. The company's working capital assets (receivables and inventory) are a far higher proportion of total assets and would likely warrant more attention from an analyst.

An analysis of horizontal common-size balance sheets highlights structural changes that have occurred in a business. Past trends are obviously not necessarily an accurate predictor of the future, especially when the economic or competitive environment changes. An examination of past trends is more valuable when the macroeconomic and competitive environments are relatively stable and when the analyst is reviewing a stable or mature business. However, even in less stable contexts, historical analysis can serve as a basis for developing expectations. Understanding of past trends is helpful in assessing whether these trends are likely to continue or if the trend is likely to change direction.

Exhibit 6: Partial Balance Sheet for a Hypothetical Company over Five Periods

\begin{center}
\begin{tabular}{|c|c|c|c|c|c|c|c|}
\hline
\multirow[b]{2}{*}{$\begin{array}{l}\text { Assets } \\
\text { (\$ Millions) }\end{array}$} & \multicolumn{5}{|c|}{Period} & \multirow{2}{*}{$\begin{array}{c}\text { Change } \\
4 \text { to } 5 \\
\text { (\$ Million) }\end{array}$} & \multirow{2}{*}{$\begin{array}{c}\text { Change } \\
4 \text { to } 5 \\
\text { (Percent) }\end{array}$} \\
\hline
 & 1 & 2 & 3 & 4 & 5 &  \\
\hline
Cash & 39 & 29 & 27 & 19 & 16 & -3 & -15.8 \\
\hline
Investments & 1 & 7 & 7 & 6 & 4 & -2 & -33.3 \\
\hline
Receivables & 44 & 41 & 37 & 67 & 79 & 12 & 17.9 \\
\hline
Inventory & 15 & 25 & 36 & 25 & 27 & 2 & 8.0 \\
\hline
$\begin{array}{l}\text { Fixed assets net of } \\ \text { depreciation }\end{array}$ & 1 & 2 & 6 & 9 & 8 & -1 & -11.1 \\
\hline
Total assets & 100 & 104 & 113 & 126 & 134 & 8 & 6.3 \\
\hline
\end{tabular}
\end{center}

\section{Exhibit 7: Horizontal Common-Size (Partial) Balance Sheet for a}
Hypothetical Company over Five Periods, with Each Item Expressed

Relative to the Same Item in Period One

\begin{center}
\begin{tabular}{|c|c|c|c|c|c|}
\hline
\multirow[b]{2}{*}{Assets} & \multicolumn{5}{|c|}{Period} \\
\hline
 & 1 & 2 & 3 & 4 & 5 \\
\hline
Cash & 1.00 & 0.74 & 0.69 & 0.49 & 0.41 \\
\hline
Investments & 1.00 & 7.00 & 7.00 & 6.00 & 4.00 \\
\hline
Receivables & 1.00 & 0.93 & 0.84 & 1.52 & 1.80 \\
\hline
Inventory & 1.00 & 1.67 & 2.40 & 1.67 & 1.80 \\
\hline
Fixed assets net of depreciation & 1.00 & 2.00 & 6.00 & 9.00 & 8.00 \\
\hline
Total assets & 1.00 & 1.04 & 1.13 & 1.26 & 1.34 \\
\hline
\end{tabular}
\end{center}

Exhibit 8: Horizontal Common-Size (Partial) Balance Sheet for a Hypothetical Company over Five Periods, with Percent Change in Each Item Relative to the Prior Period

\begin{center}
\begin{tabular}{|c|c|c|c|c|}
\hline
\multirow[b]{2}{*}{Assets} & \multicolumn{4}{|c|}{Period} \\
\hline
 & $2(\%)$ & $3(\%)$ & $4(\%)$ & $5(\%)$ \\
\hline
Cash & -25.6 & -6.9 & -29.6 & -15.8 \\
\hline
Investments & 600.0 & 0.0 & -14.3 & -33.3 \\
\hline
Receivables & -6.8 & -9.8 & 81.1 & 17.9 \\
\hline
Inventory & 66.7 & 44.0 & -30.6 & 8.0 \\
\hline
Fixed assets net of depreciation & 100.0 & 200.0 & 50.0 & -11.1 \\
\hline
Total assets & 4.0 & 8.7 & 11.5 & 6.3 \\
\hline
\end{tabular}
\end{center}

One measure of success is for a company to grow at a rate greater than the rate of the overall market in which it operates. Companies that grow slowly may find themselves unable to attract equity capital. Conversely, companies that grow too quickly may find that their administrative and management information systems cannot keep up with the rate of expansion.

\section{Relationships among Financial Statements}
Trend data generated by a horizontal common-size analysis can be compared across financial statements. For example, the growth rate of assets for the hypothetical company in Exhibit 6 can be compared with the company's growth in revenue over the same period of time. If revenue is growing more quickly than assets, the company may be increasing its efficiency (i.e., generating more revenue for every dollar invested in assets).

As another example, consider the following year-over-year percentage changes for a hypothetical company:

Revenue $+20 \%$

Net income $+25 \%$

Operating cash flow

$-10 \%$

Total assets Net income is growing faster than revenue, which indicates increasing profitability. However, the analyst would need to determine whether the faster growth in net income resulted from continuing operations or from non-operating, non-recurring items. In addition, the 10 percent decline in operating cash flow despite increasing revenue and net income clearly warrants further investigation because it could indicate a problem with earnings quality (perhaps aggressive reporting of revenue). Lastly, the fact that assets have grown faster than revenue indicates the company's efficiency may be declining. The analyst should examine the composition of the increase in assets and the reasons for the changes. Example 5 illustrates a historical example of a company where comparisons of trend data from different financial statements were actually indicative of aggressive accounting policies.

\section{EXAMPLE 5}
\section{Use of Comparative Growth Information ${ }^{8}$}
In July 1996, Sunbeam, a US company, brought in new management to turn the company around. In the following year, 1997, using 1996 as the base, the following was observed based on reported numbers:

$\begin{array}{ll}\text { Revenue } & +19 \% \\ \text { Inventory } & +58 \% \\ \text { Receivables } & +38 \%\end{array}$

It is generally more desirable to observe inventory and receivables growing at a slower (or similar) rate compared to revenue growth. Receivables growing faster than revenue can indicate operational issues, such as lower credit standards or aggressive accounting policies for revenue recognition. Similarly, inventory growing faster than revenue can indicate an operational problem with obsolescence or aggressive accounting policies, such as an improper overstatement of inventory to increase profits.

In this case, the explanation lay in aggressive accounting policies. Sunbeam was later charged by the US Securities and Exchange Commission with improperly accelerating the recognition of revenue and engaging in other practices, such as billing customers for inventory prior to shipment.

\section{THE USE OF GRAPHS AND REGRESSION ANALYSIS}
describe tools and techniques used in financial analysis, including their uses and limitations

Graphs facilitate comparison of performance and financial structure over time, highlighting changes in significant aspects of business operations. In addition, graphs provide the analyst (and management) with a visual overview of risk trends in a business. Graphs may also be used effectively to communicate the analyst's conclusions regarding financial condition and risk management aspects. Exhibit 9 presents the information from Exhibit $6 \mathrm{~A}$ in a stacked column format. The graph makes the significant decline in cash and growth in receivables (both in absolute terms and as a percentage of assets) readily apparent. In Exhibit 9, the vertical axis shows US\$ millions and the horizontal axis denotes the period.

Choosing the appropriate graph to communicate the most significant conclusions of a financial analysis is a skill. In general, pie graphs are most useful to communicate the composition of a total value (e.g., assets over a limited amount of time, say one or two periods). Line graphs are useful when the focus is on the change in amount for a limited number of items over a relatively longer time period. When the composition and amounts, as well as their change over time, are all important, a stacked column graph can be useful.

\section{Exhibit 9: Stacked Column Graph of Asset Composition of Hypothetical Company over Five Periods}
\begin{center}
\includegraphics[max width=\textwidth]{2023_05_04_b5cfa4f1bc883752f121g-213}
\end{center}

When comparing Period 5 with Period 4, the growth in receivables appears to be within normal bounds; but when comparing Period 5 with earlier periods, the dramatic growth becomes apparent. In the same manner, a simple line graph will also illustrate the growth trends in key financial variables. Exhibit 10 presents the information from Exhibit $6 \mathrm{~A}$ as a line graph, illustrating the growth of assets of a hypothetical company over five periods. The steady decline in cash, volatile movements of inventory, and dramatic growth of receivables is clearly illustrated. Again, the vertical axis is shown in US\$ millions and the horizontal axis denotes periods.

\section{Exhibit 10: Line Graph of Growth of Assets of Hypothetical Company over Five Periods}
\begin{center}
\includegraphics[max width=\textwidth]{2023_05_04_b5cfa4f1bc883752f121g-214}
\end{center}

\section{Regression Analysis}
When analyzing the trend in a specific line item or ratio, frequently it is possible simply to visually evaluate the changes. For more complex situations, regression analysis can help identify relationships (or correlation) between variables. For example, a regression analysis could relate a company's sales to GDP over time, providing insight into whether the company is cyclical. In addition, the statistical relationship between sales and GDP could be used as a basis for forecasting sales.

Other examples where regression analysis may be useful include the relationship between a company's sales and inventory over time, or the relationship between hotel occupancy and a company's hotel revenues. In addition to providing a basis for forecasting, regression analysis facilitates identification of items or ratios that are not behaving as expected, given historical statistical relationships.

\section{7}
\section{COMMON RATIO CATEGORIES \& INTERPRETATION AND CONTEXT}
identify, calculate, and interpret activity, liquidity, solvency, profitability, and valuation ratios

In the previous section, we focused on ratios resulting from common-size analysis. In this section, we expand the discussion to include other commonly used financial ratios and the broad classes into which they are categorized. There is some overlap with common-size financial statement ratios. For example, a common indicator of profitability is the net profit margin, which is calculated as net income divided by sales. This ratio appears on a vertical common-size income statement. Other ratios involve information from multiple financial statements or even data from outside the financial statements.

Because of the large number of ratios, it is helpful to think about ratios in terms of broad categories based on what aspects of performance a ratio is intended to detect. Financial analysts and data vendors use a variety of categories to classify ratios. The category names and the ratios included in each category can differ. Common ratio categories include activity, liquidity, solvency, profitability, and valuation. These categories are summarized in Exhibit 11. Each category measures a different aspect of the company's business, but all are useful in evaluating a company's overall ability to generate cash flows from operating its business and the associated risks.

\section{Exhibit 11: Categories of Financial Ratios}
Category Description

Activity Activity ratios measure how efficiently a company performs day-to-day tasks, such as the collection of receivables and management of inventory.

Liquidity Liquidity ratios measure the company's ability to meet its short-term obligations.

Solvency Solvency ratios measure a company's ability to meet long-term obligations. Subsets of these ratios are also known as "leverage" and "long-term debt" ratios.

Profitability Profitability ratios measure the company's ability to generate profits from its resources (assets).

Valuation Valuation ratios measure the quantity of an asset or flow (e.g., earnings) associated with ownership of a specified claim (e.g., a share or ownership of the enterprise).

These categories are not mutually exclusive; some ratios are useful in measuring multiple aspects of the business. For example, an activity ratio measuring how quickly a company collects accounts receivable is also useful in assessing the company's liquidity because collection of revenues increases cash. Some profitability ratios also reflect the operating efficiency of the business. In summary, analysts appropriately use certain ratios to evaluate multiple aspects of the business. Analysts also need to be aware of variations in industry practice in the calculation of financial ratios. In the text that follows, alternative views on ratio calculations are often provided.

\section{Interpretation and Context}
Financial ratios can only be interpreted in the context of other information, including benchmarks. In general, the financial ratios of a company are compared with those of its major competitors (cross-sectional and trend analysis) and to the company's prior periods (trend analysis). The goal is to understand the underlying causes of divergence between a company's ratios and those of the industry. Even ratios that remain consistent require understanding because consistency can sometimes indicate accounting policies selected to smooth earnings. An analyst should evaluate financial ratios based on the following:

\begin{enumerate}
  \item Company goals and strategy. Actual ratios can be compared with company objectives to determine whether objectives are being attained and whether the results are consistent with the company's strategy.

  \item Industry norms (cross-sectional analysis). A company can be compared with others in its industry by relating its financial ratios to industry norms or to a subset of the companies in an industry. When industry norms are used to make judgments, care must be taken because:

\end{enumerate}

\begin{itemize}
  \item Many ratios are industry specific, and not all ratios are important to all industries.

  \item Companies may have several different lines of business. This will cause aggregate financial ratios to be distorted. It is better to examine industry-specific ratios by lines of business.

  \item Differences in accounting methods used by companies can distort financial ratios.

  \item Differences in corporate strategies can affect certain financial ratios.

\end{itemize}

\begin{enumerate}
  \setcounter{enumi}{2}
  \item Economic conditions. For cyclical companies, financial ratios tend to improve when the economy is strong and weaken during recessions. Therefore, financial ratios should be examined in light of the current phase of the business cycle.
\end{enumerate}

The following sections discuss activity, liquidity, solvency, and profitability ratios in turn. Selected valuation ratios are presented later in the section on equity analysis.

\section{ACTIVITY RATIOS}
Activity ratios are also known as asset utilization ratios or operating efficiency ratios. This category is intended to measure how well a company manages various activities, particularly how efficiently it manages its various assets. Activity ratios are analyzed as indicators of ongoing operational performance-how effectively assets are used by a company. These ratios reflect the efficient management of both working capital and longer term assets. As noted, efficiency has a direct impact on liquidity (the ability of a company to meet its short-term obligations), so some activity ratios are also useful in assessing liquidity.

\section{Calculation of Activity Ratios}
Exhibit 12 presents the most commonly used activity ratios. The exhibit shows the numerator and denominator of each ratio.

\section{Exhibit 12: Definitions of Commonly Used Activity Ratios}
\begin{center}
\begin{tabular}{lll}
\hline
Activity Ratios & Numerator & Denominator \\
\hline
Inventory turnover & $\begin{array}{l}\text { Cost of sales or cost of } \\ \text { goods sold }\end{array}$ & Average inventory \\
Days of inventory on hand & Number of days in period & Inventory turnover \\
(DOH) & Revenue &  \\
Receivables turnover & Number of days in period & Receivables turnover \\
Days of sales outstanding (DSO) receivables &  &  \\
Payables turnover & Purchases & Average trade payables \\
Number of days of payables & Number of days in period & Payables turnover \\
Working capital turnover & Revenue & Average working capital \\
Fixed asset turnover & Revenue & Average net fixed assets \\
Total asset turnover & Revenue & Average total assets \\
\hline
\end{tabular}
\end{center}

Activity ratios measure how efficiently the company utilizes assets. They generally combine information from the income statement in the numerator with balance sheet items in the denominator. Because the income statement measures what happened during a period whereas the balance sheet shows the condition only at the end of the period, average balance sheet data are normally used for consistency. For example, to measure inventory management efficiency, cost of sales or cost of goods sold (from the income statement) is divided by average inventory (from the balance sheet). Most databases, such as Bloomberg and Baseline, use this averaging convention when income statement and balance sheet data are combined. These databases typically average only two points: the beginning of the year and the end of the year. The examples that follow based on annual financial statements illustrate that practice. However, some analysts prefer to average more observations if they are available, especially if the business is seasonal. If a semiannual report is prepared, an average can be taken over three data points (beginning, middle, and end of year). If quarterly data are available, a five-point average can be computed (beginning of year and end of each quarterly period) or a four-point average using the end of each quarterly period. Note that if the company's year ends at a low or high point for inventory for the year, there can still be bias in using three or five data points, because the beginning and end of year occur at the same time of the year and are effectively double counted.

Because cost of goods sold measures the cost of inventory that has been sold, this ratio measures how many times per year the entire inventory was theoretically turned over, or sold. (We say that the entire inventory was "theoretically" sold because in practice companies do not generally sell out their entire inventory.) If, for example, a company's cost of goods sold for a recent year was $€ 120,000$ and its average inventory was $€ 10,000$, the inventory turnover ratio would be 12 . The company theoretically turns over (i.e., sells) its entire inventory 12 times per year (i.e., once a month). (Again, we say "theoretically" because in practice the company likely carries some inventory from one month into another.) Turnover can then be converted to days of inventory on hand (DOH) by dividing inventory turnover into the number of days in the accounting period. In this example, the result is a DOH of 30.42 (365/12), meaning that, on average, the company's inventory was on hand for about 30 days, or, equivalently, the company kept on hand about 30 days' worth of inventory, on average, during the period.

Activity ratios can be computed for any annual or interim period, but care must be taken in the interpretation and comparison across periods. For example, if the same company had cost of goods sold for the first quarter (90 days) of the following year of $€ 35,000$ and average inventory of $€ 11,000$, the inventory turnover would be 3.18 times. However, this turnover rate is 3.18 times per quarter, which is not directly comparable to the 12 times per year in the preceding year. In this case, we can annualize the quarterly inventory turnover rate by multiplying the quarterly turnover by 4 (12 months/ 3 months; or by 4.06 , using 365 days/ 90 days) for comparison to the annual turnover rate. So, the quarterly inventory turnover is equivalent to a 12.72 annual inventory turnover (or 12.91 if we annualize the ratio using a 90-day quarter and a 365-day year). To compute the DOH using quarterly data, we can use the quarterly turnover rate and the number of days in the quarter for the numerator-or, we can use the annualized turnover rate and 365 days; either results in DOH of around 28.3, with slight differences due to rounding $(90 / 3.18=28.30$ and $365 / 12.91=28.27)$. Another time-related computational detail is that for companies using a 52/53-week annual period and for leap years, the actual days in the year should be used rather than 365 .

In some cases, an analyst may want to know how many days of inventory are on hand at the end of the year rather than the average for the year. In this case, it would be appropriate to use the year-end inventory balance in the computation rather than the average. If the company is growing rapidly or if costs are increasing rapidly, analysts should consider using cost of goods sold just for the fourth quarter in this computation because the cost of goods sold of earlier quarters may not be relevant. Example 6 further demonstrates computation of activity ratios using Hong Kong Stock Exchange(HKEX)-listed Lenovo Group Limited.

\section{EXAMPLE 6}
\section{Computation of Activity Ratios}
\begin{enumerate}
  \item An analyst would like to evaluate Lenovo Group's efficiency in collecting its trade accounts receivable during the fiscal year ended 31 March 2018 (FY2017). The analyst gathers the following information from Lenovo's annual and interim reports:
\end{enumerate}

\begin{center}
\begin{tabular}{lc}
\hline
 & US\$ in Thousands \\
\hline
Trade receivables as of 31 March 2017 & $4,468,392$ \\
Trade receivables as of 31 March 2018 & $4,972,722$ \\
Revenue for year ended 31 March 2018 & $45,349,943$ \\
\hline
\end{tabular}
\end{center}

Calculate Lenovo's receivables turnover and number of days of sales outstanding (DSO) for the fiscal year ended 31 March 2018.

\section{Solution:}
$$
\begin{aligned}
\text { Receivables turnover } & =\text { Revenue/Average receivables } \\
& =45,349,943 /[(4,468,392+4,972,722) / 2] \\
& =45,349,943 / 4,720,557 \\
& =9.6069 \text { times, or } 9.6 \text { rounded } \\
\mathrm{DSO} & =\begin{array}{l}
\text { Number of days in period } / \text { Receivables } \\
\text { turnover }
\end{array} \\
& =365 / 9.6 \\
& =38.0 \text { days }
\end{aligned}
$$

On average, it took Lenovo 38 days to collect receivables during the fiscal year ended 31 March 2018.

\section{Interpretation of Activity Ratios}
In the following section, we further discuss the activity ratios that were defined in Exhibit 12.

\section{Inventory Turnover and DOH}
Inventory turnover lies at the heart of operations for many entities. It indicates the resources tied up in inventory (i.e., the carrying costs) and can, therefore, be used to indicate inventory management effectiveness. A higher inventory turnover ratio implies a shorter period that inventory is held, and thus a lower DOH. In general, inventory turnover and DOH should be benchmarked against industry norms.

A high inventory turnover ratio relative to industry norms might indicate highly effective inventory management. Alternatively, a high inventory turnover ratio (and commensurately low DOH) could possibly indicate the company does not carry adequate inventory, so shortages could potentially hurt revenue. To assess which explanation is more likely, the analyst can compare the company's revenue growth with that of the industry. Slower growth combined with higher inventory turnover could indicate inadequate inventory levels. Revenue growth at or above the industry's growth supports the interpretation that the higher turnover reflects greater inventory management efficiency.

A low inventory turnover ratio (and commensurately high DOH) relative to the rest of the industry could be an indicator of slow-moving inventory, perhaps due to technological obsolescence or a change in fashion. Again, comparing the company's sales growth with the industry can offer insight.

\section{Receivables Turnover and DSO.}
The number of DSO represents the elapsed time between a sale and cash collection, reflecting how fast the company collects cash from customers to whom it offers credit. Although limiting the numerator to sales made on credit in the receivables turnover would be more appropriate, credit sales information is not always available to analysts; therefore, revenue as reported in the income statement is generally used as an approximation.

A relatively high receivables turnover ratio (and commensurately low DSO) might indicate highly efficient credit and collection. Alternatively, a high receivables turnover ratio could indicate that the company's credit or collection policies are too stringent, suggesting the possibility of sales being lost to competitors offering more lenient terms. A relatively low receivables turnover ratio would typically raise questions about the efficiency of the company's credit and collections procedures. As with inventory management, comparison of the company's sales growth relative to the industry can help the analyst assess whether sales are being lost due to stringent credit policies. In addition, comparing the company's estimates of uncollectible accounts receivable and actual credit losses with past experience and with peer companies can help assess whether low turnover reflects credit management issues. Companies often provide details of receivables aging (how much receivables have been outstanding by age). This can be used along with DSO to understand trends in collection, as demonstrated in Example 7.

\section{EXAMPLE 7}
\section{Evaluation of an Activity Ratio}
An analyst has computed the average DSO for Lenovo for fiscal years ended 31 March 2018 and 2017:

\begin{center}
\begin{tabular}{lcc}
\hline
 & FY2017 & FY2016 \\
\hline
Days of sales outstanding & 38.0 & 37.6 \\
\hline
\end{tabular}
\end{center}

Revenue increased from US\$43.035 billion for fiscal year ended 31 March 2017 (FY2016) to US\$45.350 billion for fiscal year ended 31 March 2018 (FY2017). The analyst would like to better understand the change in the company's DSO from FY2016 to FY2017 and whether the increase is indicative of any issues with the customers' credit quality. The analyst collects accounts receivable aging information from Lenovo's annual reports and computes the percentage of accounts receivable by days outstanding. This information is presented in Exhibit 13:

\section{Exhibit 13}
\begin{center}
\begin{tabular}{|c|c|c|c|c|c|c|}
\hline
 & \multicolumn{2}{|c|}{FY2017} & \multicolumn{2}{|c|}{FY2016} & \multicolumn{2}{|c|}{FY2015} \\
\hline
 & US\$000 & Percent & US\$000 & Percent & US\$000 & Percen \\
\hline
\multicolumn{7}{|l|}{Accounts receivable} \\
\hline
$0-30$ days & $3,046,240$ & 59.95 & $2,923,083$ & 63.92 & $3,246,600$ & 71.99 \\
\hline
$31-60$ days & $1,169,286$ & 23.01 & 985,251 & 21.55 & 617,199 & 13.69 \\
\hline
$61-90$ days & 320,183 & 6.30 & 283,050 & 6.19 & 240,470 & 5.33 \\
\hline
Over 90 days & 545,629 & 10.74 & 381,387 & 8.34 & 405,410 & 8.99 \\
\hline
Total & $5,081,338$ & 100.00 & $4,572,771$ & 100.00 & $4,509,679$ & 100.00 \\
\hline
$\begin{array}{l}\text { Less: Provision for } \\ \text { impairment }\end{array}$ & $-108,616$ & -2.14 & $-104,379$ & -2.28 & $-106,172$ & -2.35 \\
\hline
Trade receivables, net & $4,972,722$ & 97.86 & $4,468,392$ & 97.72 & $4,403,507$ & 97.65 \\
\hline
Total sales & $45,349,943$ &  & $43,034,731$ &  & $44,912,097$ &  \\
\hline
\end{tabular}
\end{center}

Note: Lenovo's footnotes disclose that general trade customers are provided with credit terms ranging from 0 to 120 days.

These data indicate that total accounts receivable increased by about $11.3 \%$ in FY2017 versus FY2016, while total sales increased by only 5.4\%. Further, the percentage of receivables in all categories older than 30 days has increased over the three-year period, indicating that customers are indeed taking longer to pay. On the other hand, the provision for impairment (estimate of uncollectible accounts) has declined as a percent of total receivables. Considering all this information, the company may be increasing customer financing purposely to drive its sales growth. They also may be underestimating the impairment. This should be investigated further by the analyst.

\section{Payables Turnover and the Number of Days of Payables}
The number of days of payables reflects the average number of days the company takes to pay its suppliers, and the payables turnover ratio measures how many times per year the company theoretically pays off all its creditors. For purposes of calculating these ratios, an implicit assumption is that the company makes all its purchases using credit. If the amount of purchases is not directly available, it can be computed as cost of goods sold plus ending inventory less beginning inventory. Alternatively, cost of goods sold is sometimes used as an approximation of purchases.

A payables turnover ratio that is high (low days payable) relative to the industry could indicate that the company is not making full use of available credit facilities; alternatively, it could result from a company taking advantage of early payment discounts. An excessively low turnover ratio (high days payable) could indicate trouble making payments on time, or alternatively, exploitation of lenient supplier terms. This is another example where it is useful to look simultaneously at other ratios. If liquidity ratios indicate that the company has sufficient cash and other short-term assets to pay obligations and yet the days payable ratio is relatively high, the analyst would favor the lenient supplier credit and collection policies as an explanation.

\section{Working Capital Turnover}
Working capital is defined as current assets minus current liabilities. Working capital turnover indicates how efficiently the company generates revenue with its working capital. For example, a working capital turnover ratio of 4.0 indicates that the company generates $€ 4$ of revenue for every $€ 1$ of working capital. A high working capital turnover ratio indicates greater efficiency (i.e., the company is generating a high level of revenues relative to working capital). For some companies, working capital can be near zero or negative, rendering this ratio incapable of being interpreted. The following two ratios are more useful in those circumstances.

\section{Fixed Asset Turnover}
This ratio measures how efficiently the company generates revenues from its investments in fixed assets. Generally, a higher fixed asset turnover ratio indicates more efficient use of fixed assets in generating revenue. A low ratio can indicate inefficiency, a capital-intensive business environment, or a new business not yet operating at full capacity - in which case the analyst will not be able to link the ratio directly to efficiency. In addition, asset turnover can be affected by factors other than a company's efficiency. The fixed asset turnover ratio would be lower for a company whose assets are newer (and, therefore, less depreciated and so reflected in the financial statements at a higher carrying value) than the ratio for a company with older assets (that are thus more depreciated and so reflected at a lower carrying value). The fixed asset ratio can be erratic because, although revenue may have a steady growth rate, increases in fixed assets may not follow a smooth pattern; so, every year-to-year change in the ratio does not necessarily indicate important changes in the company's efficiency.

\section{Total Asset Turnover}
The total asset turnover ratio measures the company's overall ability to generate revenues with a given level of assets. A ratio of 1.20 would indicate that the company is generating $€ 1.20$ of revenues for every $€ 1$ of average assets. A higher ratio indicates greater efficiency. Because this ratio includes both fixed and current assets, inefficient working capital management can distort overall interpretations. It is therefore helpful to analyze working capital and fixed asset turnover ratios separately.

A low asset turnover ratio can be an indicator of inefficiency or of relative capital intensity of the business. The ratio also reflects strategic decisions by management-for example, the decision whether to use a more labor-intensive (and less capital-intensive) approach to its business or a more capital-intensive (and less labor-intensive) approach.

When interpreting activity ratios, the analysts should examine not only the individual ratios but also the collection of relevant ratios to determine the overall efficiency of a company. Example 8 demonstrates the evaluation of activity ratios, both narrow (e.g., days of inventory on hand) and broad (e.g., total asset turnover) for a hypothetical manufacturer.

\section{EXAMPLE 8}
\section{Evaluation of Activity Ratios}
ZZZ Company is a hypothetical manufacturing company. As part of an analysis of management's operating efficiency, an analyst collects the following activity ratios from a data provider:

\begin{center}
\begin{tabular}{lcccc}
\hline
Ratio & $\mathbf{2 0 1 8}$ & $\mathbf{2 0 1 7}$ & $\mathbf{2 0 1 6}$ & $\mathbf{2 0 1 5}$ \\
\hline
DOH & 35.68 & 40.70 & 40.47 & 48.51 \\
DSO & 45.07 & 58.28 & 51.27 & 76.98 \\
Total asset turnover & 0.36 & 0.28 & 0.23 & 0.22 \\
\hline
\end{tabular}
\end{center}

These ratios indicate that the company has improved on all three measures of activity over the four-year period. The company appears to be managing its inventory more efficiently, is collecting receivables faster, and is generating a higher level of revenues relative to total assets. The overall trend appears good, but thus far, the analyst has only determined what happened. A more important question is why the ratios improved, because understanding good changes as well as bad ones facilitates judgments about the company's future performance. To answer this question, the analyst examines company financial reports as well as external information about the industry and economy. In examining the annual report, the analyst notes that in the fourth quarter of 2018 , the company experienced an "inventory correction" and that the company recorded an allowance for the decline in market value and obsolescence of inventory of about 15 percent of year-end inventory value (compared with about a 6 percent allowance in the prior year). This reduction in the value of inventory accounts for a large portion of the decline in DOH from 40.70 in 2017 to 35.68 in 2018. Management claims that this inventory obsolescence is a short-term issue; analysts can watch $\mathrm{DOH}$ in future interim periods to confirm this assertion. In any event, all else being equal, the analyst would likely expect $\mathrm{DOH}$ to return to a level closer to 40 days going forward.

More positive interpretations can be drawn from the total asset turnover. The analyst finds that the company's revenues increased more than 35 percent while total assets only increased by about 6 percent. Based on external information about the industry and economy, the analyst attributes the increased revenues both to overall growth in the industry and to the company's increased market share. Management was able to achieve growth in revenues with a comparatively modest increase in assets, leading to an improvement in total asset turnover. Note further that part of the reason for the increase in asset turnover is lower $\mathrm{DOH}$ and DSO.

\section{LIQUIDITY RATIOS}
identify, calculate, and interpret activity, liquidity, solvency, profitability, and valuation ratios Liquidity analysis, which focuses on cash flows, measures a company's ability to meet its short-term obligations. Liquidity measures how quickly assets are converted into cash. Liquidity ratios also measure the ability to pay off short-term obligations. In day-to-day operations, liquidity management is typically achieved through efficient use of assets. In the medium term, liquidity in the non-financial sector is also addressed by managing the structure of liabilities. (See the discussion on financial sector below.)

The level of liquidity needed differs from one industry to another. A particular company's liquidity position may vary according to the anticipated need for funds at any given time. Judging whether a company has adequate liquidity requires analysis of its historical funding requirements, current liquidity position, anticipated future funding needs, and options for reducing funding needs or attracting additional funds (including actual and potential sources of such funding).

Larger companies are usually better able to control the level and composition of their liabilities than smaller companies. Therefore, they may have more potential funding sources, including public capital and money markets. Greater discretionary access to capital markets also reduces the size of the liquidity buffer needed relative to companies without such access.

Contingent liabilities, such as letters of credit or financial guarantees, can also be relevant when assessing liquidity. The importance of contingent liabilities varies for the non-banking and banking sector. In the non-banking sector, contingent liabilities (usually disclosed in the footnotes to the company's financial statements) represent potential cash outflows, and when appropriate, should be included in an assessment of a company's liquidity. In the banking sector, contingent liabilities represent potentially significant cash outflows that are not dependent on the bank's financial condition. Although outflows in normal market circumstances typically may be low, a general macroeconomic or market crisis can trigger a substantial increase in cash outflows related to contingent liabilities because of the increase in defaults and business bankruptcies that often accompany such events. In addition, such crises are usually characterized by diminished levels of overall liquidity, which can further exacerbate funding shortfalls. Therefore, for the banking sector, the effect of contingent liabilities on liquidity warrants particular attention.

\section{Calculation of Liquidity Ratios}
Common liquidity ratios are presented in Exhibit 14. These liquidity ratios reflect a company's position at a point in time and, therefore, typically use data from the ending balance sheet rather than averages. The current, quick, and cash ratios reflect three measures of a company's ability to pay current liabilities. Each uses a progressively stricter definition of liquid assets.

The defensive interval ratio measures how long a company can pay its daily cash expenditures using only its existing liquid assets, without additional cash flow coming in. This ratio is similar to the "burn rate" often computed for start-up internet companies in the late 1990s or for biotechnology companies. The numerator of this ratio includes the same liquid assets used in the quick ratio, and the denominator is an estimate of daily cash expenditures. To obtain daily cash expenditures, the total of cash expenditures for the period is divided by the number of days in the period. Total cash expenditures for a period can be approximated by summing all expenses on the income statement-such as cost of goods sold; selling, general, and administrative expenses; and research and development expenses-and then subtracting any non-cash expenses, such as depreciation and amortisation. (Typically, taxes are not included.)

The cash conversion cycle, a financial metric not in ratio form, measures the length of time required for a company to go from cash paid (used in its operations) to cash received (as a result of its operations). The cash conversion cycle is sometimes expressed as the length of time funds are tied up in working capital. During this period of time, the company needs to finance its investment in operations through other sources (i.e., through debt or equity).

\section{Exhibit 14: Definitions of Commonly Used Liquidity Ratios}
\begin{center}
\begin{tabular}{|c|c|c|}
\hline
Liquidity Ratios & Numerator & Denominator \\
\hline
Current ratio & Current assets & Current liabilities \\
\hline
Quick ratio & $\begin{array}{l}\text { Cash }+ \text { Short-term marketable } \\ \text { investments }+ \text { Receivables }\end{array}$ & Current liabilities \\
\hline
Cash ratio & $\begin{array}{l}\text { Cash }+ \text { Short-term marketable } \\ \text { investments }\end{array}$ & Current liabilities \\
\hline
Defensive interval ratio & $\begin{array}{l}\text { Cash }+ \text { Short-term marketable } \\ \text { investments }+ \text { Receivables }\end{array}$ & Daily cash expenditures \\
\hline
\multicolumn{3}{|c|}{Additional Liquidity Measure} \\
\hline
$\begin{array}{l}\text { Cash conversion cycle } \\ \text { (net operating cycle) }\end{array}$ & \multicolumn{2}{|c|}{$\mathrm{DOH}+\mathrm{DSO}-$ Number of days of payables} \\
\hline
\end{tabular}
\end{center}

\section{Interpretation of Liquidity Ratios}
In the following, we discuss the interpretation of the five basic liquidity measures presented in Exhibit 14.

\section{Current Ratio}
This ratio expresses current assets in relation to current liabilities. A higher ratio indicates a higher level of liquidity (i.e., a greater ability to meet short-term obligations). A current ratio of 1.0 would indicate that the book value of its current assets exactly equals the book value of its current liabilities.

A lower ratio indicates less liquidity, implying a greater reliance on operating cash flow and outside financing to meet short-term obligations. Liquidity affects the company's capacity to take on debt. The current ratio implicitly assumes that inventories and accounts receivable are indeed liquid (which is presumably not the case when related turnover ratios are low).

\section{Quick Ratio}
The quick ratio is more conservative than the current ratio because it includes only the more liquid current assets (sometimes referred to as "quick assets") in relation to current liabilities. Like the current ratio, a higher quick ratio indicates greater liquidity.

The quick ratio reflects the fact that certain current assets-such as prepaid expenses, some taxes, and employee-related prepayments-represent costs of the current period that have been paid in advance and cannot usually be converted back into cash. This ratio also reflects the fact that inventory might not be easily and quickly converted into cash, and furthermore, that a company would probably not be able to sell all of its inventory for an amount equal to its carrying value, especially if it were required to sell the inventory quickly. In situations where inventories are illiquid (as indicated, for example, by low inventory turnover ratios), the quick ratio may be a better indicator of liquidity than is the current ratio.

\section{Cash Ratio}
The cash ratio normally represents a reliable measure of an entity's liquidity in a crisis situation. Only highly marketable short-term investments and cash are included. In a general market crisis, the fair value of marketable securities could decrease significantly as a result of market factors, in which case even this ratio might not provide reliable information.

\section{Defensive Interval Ratio}
This ratio measures how long the company can continue to pay its expenses from its existing liquid assets without receiving any additional cash inflow. A defensive interval ratio of 50 would indicate that the company can continue to pay its operating expenses for 50 days before running out of quick assets, assuming no additional cash inflows. A higher defensive interval ratio indicates greater liquidity. If a company's defensive interval ratio is very low relative to peer companies or to the company's own history, the analyst would want to ascertain whether there is sufficient cash inflow expected to mitigate the low defensive interval ratio.

\section{Cash Conversion Cycle (Net Operating Cycle)}
This metric indicates the amount of time that elapses from the point when a company invests in working capital until the point at which the company collects cash. In the typical course of events, a merchandising company acquires inventory on credit, incurring accounts payable. The company then sells that inventory on credit, increasing accounts receivable. Afterwards, it pays out cash to settle its accounts payable, and it collects cash in settlement of its accounts receivable. The time between the outlay of cash and the collection of cash is called the "cash conversion cycle." A shorter cash conversion cycle indicates greater liquidity. A short cash conversion cycle implies that the company only needs to finance its inventory and accounts receivable for a short period of time. A longer cash conversion cycle indicates lower liquidity; it implies that the company must finance its inventory and accounts receivable for a longer period of time, possibly indicating a need for a higher level of capital to fund current assets. Example 9 demonstrates the advantages of a short cash conversion cycle as well as how a company's business strategies are reflected in financial ratios.

\section{EXAMPLE 9}
\section{Evaluation of Liquidity Measures}
An analyst is evaluating the liquidity of Apple and calculates the number of days of receivables, inventory, and accounts payable, as well as the overall cash conversion cycle, as follows:

\begin{center}
\begin{tabular}{|c|c|c|c|}
\hline
 & FY2017 & FY2016 & FY2015 \\
\hline
DSO & 27 & 28 & 27 \\
\hline
$\mathrm{DOH}$ & 9 & 6 & 6 \\
\hline
$\begin{array}{l}\text { Less: Number of days } \\ \text { of payables }\end{array}$ & 112 & 101 & 86 \\
\hline
Equals: Cash conver- & (76) & (67) & (53) \\
\hline
\end{tabular}
\end{center}

The minimal DOH indicates that Apple maintains lean inventories, which is attributable to key aspects of the company's business model where manufacturing is outsourced. In isolation, the increase in number of days payable (from 86 days in FY2015 to 112 days in FY2017) might suggest an inability to pay suppliers; however, in Apple's case, the balance sheet indicates that the company has more than $\$ 70$ billion of cash and short-term investments, which would be more than enough to pay suppliers sooner if Apple chose to do so. Instead, Apple takes advantage of the favorable credit terms granted by its suppliers. The overall effect is a negative cash cycle, a somewhat unusual result. Instead of requiring additional capital to fund working capital as is the case for most companies, Apple has excess cash to invest for over 50 days during that three-year period (reflected on the balance sheet as short-term investments) on which it is earning, rather than paying, interest.

\section{EXAMPLE 10}
\section{Bounds and Context of Financial Measures}
The previous example focused on the cash conversion cycle, which many companies identify as a key performance metric. The less positive the number of days in the cash conversion cycle, typically, the better it is considered to be. However, is this always true?

This example considers the following question: If a larger negative number of days in a cash conversion cycle is considered to be a desirable performance metric, does identifying a company with a large negative cash conversion cycle necessarily imply good performance?

Using a historical example, National Datacomputer, a technology company, had large negative number of days in its cash conversion cycle during the 2005 to 2009 period. In 2008 its cash conversion cycle was 275.5 days.

\section{Exhibit 15: National Datacomputer Inc. (\$ millions)}
\begin{center}
\begin{tabular}{lcccccc}
\hline
Fiscal year & $\mathbf{2 0 0 4}$ & $\mathbf{2 0 0 5}$ & $\mathbf{2 0 0 6}$ & $\mathbf{2 0 0 7}$ & $\mathbf{2 0 0 8}$ & $\mathbf{2 0 0 9}$ \\
\hline
Sales & 3.248 & 2.672 & 2.045 & 1.761 & 1.820 & 1.723 \\
Cost of goods sold & 1.919 & 1.491 & 0.898 & 1.201 & 1.316 & 1.228 \\
Receivables, Total & 0.281 & 0.139 & 0.099 & 0.076 & 0.115 & 0.045 \\
Inventories, Total & 0.194 & 0.176 & 0.010 & 0.002 & 0.000 & 0.000 \\
Accounts payable & 0.223 & 0.317 & 0.366 & 1.423 & 0.704 & 0.674 \\
 &  &  &  &  &  &  \\
DSO &  & 28.69 & 21.24 & 18.14 & 19.15 & 16.95 \\
DOH &  & 45.29 & 37.80 & 1.82 & 0.28 & 0.00 \\
Less: Number of days of payables* &  & 66.10 & 138.81 & 271.85 & 294.97 & 204.79 \\
Equals: Cash conversion cycle & 7.88 & -79.77 & -251.89 & -275.54 & -187.84 &  \\
\hline
\end{tabular}
\end{center}

2008 and 2009 are reported as $\$ 0$ million; therefore, inventory turnover for 2009 cannot be measured. However, given inventory and average sales per day, DOH in 2009 is 0.00 .

Source: Raw data from Compustat. Ratios calculated.

The reason for the negative cash conversion cycle is that the company's accounts payable increased substantially over the period. An increase from approximately 66 days in 2005 to 295 days in 2008 to pay trade creditors is clearly a negative signal. In addition, the company's inventories disappeared, most likely because the company did not have enough cash to purchase new inventory and was unable to get additional credit from its suppliers. Of course, an analyst would have immediately noted the negative trends in these data, as well as additional data throughout the company's financial statements. In its MD\&A, the company clearly reports the risks as follows:

Because we have historically had losses and only a limited amount of cash has been generated from operations, we have funded our operating activities to date primarily from the sale of securities and from the sale of a product line in 2009. In order to continue to fund our operations, we may need to raise additional capital, through the sale of securities. We cannot be certain that any such financing will be available on acceptable terms, or at all. Moreover, additional equity financing, if available, would likely be dilutive to the holders of our common stock, and debt financing, if available, would likely involve restrictive covenants and a security interest in all or substantially all of our assets. If we fail to obtain acceptable financing when needed, we may not have sufficient resources to fund our normal operations which would have a material adverse effect on our business.

IF WE ARE UNABLE TO GENERATE ADEQUATE WORKING CAPITAL FROM OPERATIONS OR RAISE ADDITIONAL CAPITAL THERE IS SUBSTANTIAL DOUBT ABOUT THE COMPANY'S ABILITY TO CONTINUE AS A GOING CONCERN. (emphasis added by company)

Source: National Datacomputer Inc., 2009 Form 10-K, page 7.

Subsequently, the company's 2010 Form 10K reported:

“In January 2011, due to our inability to meet our financial obligations and the impending loss of a critical distribution agreement granting us the right to distribute certain products, our secured lenders ("Secured Parties") acting upon an event of default, sold certain of our assets (other than cash and accounts receivable) to Micronet, Ltd. ("Micronet"), an unaffiliated corporation pursuant to the terms of an asset purchase agreement between the Secured Parties and Micronet dated January 10, 2010 (the "Asset Purchase Agreement"). In order to induce Micronet to enter into the agreement, the Company also provided certain representations and warranties regarding certain business matters."

In summary, it is always necessary to consider ratios within bounds of reasonability and to understand the reasons underlying changes in ratios. Ratios must not only be calculated but must also be interpreted by an analyst.

\section{SOLVENCY RATIOS}
identify, calculate, and interpret activity, liquidity, solvency, profitability, and valuation ratios

Solvency refers to a company's ability to fulfill its long-term debt obligations. Assessment of a company's ability to pay its long-term obligations (i.e., to make interest and principal payments) generally includes an in-depth analysis of the components of its financial structure. Solvency ratios provide information regarding the relative amount of debt in the company's capital structure and the adequacy of earnings and cash flow to cover interest expenses and other fixed charges (such as lease or rental payments) as they come due.

Analysts seek to understand a company's use of debt for several main reasons. One reason is that the amount of debt in a company's capital structure is important for assessing the company's risk and return characteristics, specifically its financial leverage. Leverage is a magnifying effect that results from the use of fixed costs-costs that stay the same within some range of activity-and can take two forms: operating leverage and financial leverage.

Operating leverage results from the use of fixed costs in conducting the company's business. Operating leverage magnifies the effect of changes in sales on operating income. Profitable companies may use operating leverage because when revenues increase, with operating leverage, their operating income increases at a faster rate. The explanation is that, although variable costs will rise proportionally with revenue, fixed costs will not.

When financing a company (i.e., raising capital for it), the use of debt constitutes financial leverage because interest payments are essentially fixed financing costs. As a result of interest payments, a given percent change in EBIT results in a larger percent change in earnings before taxes (EBT). Thus, financial leverage tends to magnify the effect of changes in EBIT on returns flowing to equity holders. Assuming that a company can earn more on funds than it pays in interest, the inclusion of some level of debt in a company's capital structure may lower a company's overall cost of capital and increase returns to equity holders. However, a higher level of debt in a company's capital structure increases the risk of default and results in higher borrowing costs for the company to compensate lenders for assuming greater credit risk. Starting with Modigliani and Miller (1958) and Modigliani and Miller (1963), a substantial amount of research has focused on determining a company's optimal capital structure and the subject remains an important one in corporate finance.

In analyzing financial statements, an analyst aims to understand levels and trends in a company's use of financial leverage in relation to past practices and the practices of peer companies. Analysts also need to be aware of the relationship between operating leverage (results from the use of non-current assets with fixed costs) and financial leverage (results from the use of long-term debt with fixed costs). The greater a company's operating leverage, the greater the risk of the operating income stream available to cover debt payments; operating leverage can thus limit a company's capacity to use financial leverage.

A company's relative solvency is fundamental to valuation of its debt securities and its creditworthiness. Finally, understanding a company's use of debt can provide analysts with insight into the company's future business prospects because management's decisions about financing may signal their beliefs about a company's future. For example, the issuance of long-term debt to repurchase common shares may indicate that management believes the market is underestimating the company's prospects and that the shares are undervalued.

\section{Calculation of Solvency Ratios}
Solvency ratios are primarily of two types. Debt ratios, the first type, focus on the balance sheet and measure the amount of debt capital relative to equity capital. Coverage ratios, the second type, focus on the income statement and measure the ability of a company to cover its debt payments. These ratios are useful in assessing a company's solvency and, therefore, in evaluating the quality of a company's bonds and other debt obligations. Exhibit 16 describes commonly used solvency ratios. The first three of the debt ratios presented use total debt in the numerator. The definition of total debt used in these ratios varies among informed analysts and financial data vendors, with some using the total of interest-bearing short-term and long-term debt, excluding liabilities such as accrued expenses and accounts payable. (For calculations in this reading, we use this definition.) Other analysts use definitions that are more inclusive (e.g., all liabilities) or restrictive (e.g., long-term debt only, in which case the ratio is sometimes qualified as "long-term," as in "long-term debt-to-equity ratio"). If using different definitions of total debt materially changes conclusions about a company's solvency, the reasons for the discrepancies warrant further investigation.

\section{Exhibit 16: Definitions of Commonly Used Solvency Ratios}
Solvency Ratios Numerator Denominator

\section{Debt Ratios}
\begin{center}
\begin{tabular}{|c|c|c|}
\hline
Debt-to-assets ratio $^{a}$ & Total debt ${ }^{b}$ & Total assets \\
\hline
Debt-to-capital ratio & Total debt ${ }^{b}$ & Total debt $^{\mathrm{b}}+$ Total shareholders' equit \\
\hline
Debt-to-equity ratio & Total debt ${ }^{b}$ & Total shareholders' equity \\
\hline
Financial leverage ratio ${ }^{\mathrm{c}}$ & Average total assets & Average total equity \\
\hline
Debt-to-EBITDA & Total debt & EBITDA \\
\hline
\end{tabular}
\end{center}

Coverage Ratios

\begin{center}
\begin{tabular}{lll}
Interest coverage & EBIT & Interest payments \\
Fixed charge coverage & EBIT + Lease payments & Interest payments + Lease payments \\
\hline
\end{tabular}
\end{center}

a "Total debt ratio" is another name sometimes used for this ratio.

b In this reading, total debt is the sum of interest-bearing short-term and long-term debt.

${ }^{\mathrm{c}}$ Average total assets divided by average total equity is used for the purposes of this reading (in particular, Dupont analysis covered later). In practice, period-end total assets divided by period-end total equity is often used.

\section{Interpretation of Solvency Ratios}
In the following, we discuss the interpretation of the basic solvency ratios presented in Exhibit 16.

\section{Debt-to-Assets Ratio}
This ratio measures the percentage of total assets financed with debt. For example, a debt-to-assets ratio of 0.40 or 40 percent indicates that 40 percent of the company's assets are financed with debt. Generally, higher debt means higher financial risk and thus weaker solvency.

\section{Debt-to-Capital Ratio}
The debt-to-capital ratio measures the percentage of a company's capital (debt plus equity) represented by debt. As with the previous ratio, a higher ratio generally means higher financial risk and thus indicates weaker solvency.

\section{Debt-to-Equity Ratio}
The debt-to-equity ratio measures the amount of debt capital relative to equity capital. Interpretation is similar to the preceding two ratios (i.e., a higher ratio indicates weaker solvency). A ratio of 1.0 would indicate equal amounts of debt and equity, which is equivalent to a debt-to-capital ratio of 50 percent. Alternative definitions of this ratio use the market value of stockholders' equity rather than its book value (or use the market values of both stockholders' equity and debt).

\section{Financial Leverage Ratio}
This ratio (often called simply the "leverage ratio") measures the amount of total assets supported for each one money unit of equity. For example, a value of 3 for this ratio means that each $€ 1$ of equity supports $€ 3$ of total assets. The higher the financial leverage ratio, the more leveraged the company is in the sense of using debt and other liabilities to finance assets. This ratio is often defined in terms of average total assets and average total equity and plays an important role in the DuPont decomposition of return on equity that will be presented in Section 4.6.2.

\section{Debt-to-EBITDA Ratio}
This ratio estimates how many years it would take to repay total debt based on earnings before income taxes, depreciation and amortization (an approximation of operating cash flow).

\section{Interest Coverage}
This ratio measures the number of times a company's EBIT could cover its interest payments. Thus, it is sometimes referred to as "times interest earned." A higher interest coverage ratio indicates stronger solvency, offering greater assurance that the company can service its debt (i.e., bank debt, bonds, notes) from operating earnings.

\section{Fixed Charge Coverage}
This ratio relates fixed charges, or obligations, to the cash flow generated by the company. It measures the number of times a company's earnings (before interest, taxes, and lease payments) can cover the company's interest and lease payments. ${ }^{9}$ Similar to the interest coverage ratio, a higher fixed charge coverage ratio implies stronger solvency, offering greater assurance that the company can service its debt (i.e., bank debt, bonds, notes, and leases) from normal earnings. The ratio is sometimes used as an indication of the quality of the preferred dividend, with a higher ratio indicating a more secure preferred dividend.

Example 11 demonstrates the use of solvency ratios in evaluating the creditworthiness of a company.

\section{EXAMPLE 11}
\section{Evaluation of Solvency Ratios}
A credit analyst is evaluating the solvency of Eskom, a South African public utility based on financial statements for the year ended 31 March 2017. The following data are gathered from the company's 2017 annual report:

9 For computing this ratio, an assumption sometimes made is that one-third of the lease payment amount represents interest on the lease obligation and that the rest is a repayment of principal on the obligation. For this variant of the fixed charge coverage ratio, the numerator is EBIT plus one-third of lease payments and the denominator is interest payments plus one-third of lease payments.

\begin{center}
\begin{tabular}{lccc}
\hline
South African Rand, millions & $\mathbf{2 0 1 7}$ & $\mathbf{2 0 1 6}$ & $\mathbf{2 0 1 5}$ \\
\hline
Total Assets & 710,009 & 663,170 & 559,688 \\
Short Term Debt & 18,530 & 15,688 & 19,976 \\
Long Term Debt & 336,770 & 306,970 & 277,458 \\
Total Liabilities & 534,067 & 480,818 & 441,269 \\
Total Equity & 175,942 & 182,352 & 118,419 \\
\hline
\end{tabular}
\end{center}

\begin{enumerate}
  \item Calculate the company's financial leverage ratio for 2016 and 2017.
\end{enumerate}

\section{Solutions to 1:}
(Amounts are millions of Rand.)

For 2017 , average total assets were $(710,009+663,170) / 2=686,590$, and average total equity was $(175,942+182,352) / 2=179,147$. Thus, financial leverage was $686,590 / 179,942=3.83$. For 2016 , financial leverage was 4.07 .

\begin{center}
\begin{tabular}{lcc}
\hline
 & $\mathbf{2 0 1 7}$ & $\mathbf{2 0 1 6}$ \\
\hline
Average Assets & 686,590 & 611,429 \\
Average Equity & 179,147 & 150,386 \\
Financial Leverage & 3.83 & 4.07 \\
\hline
\end{tabular}
\end{center}

\begin{enumerate}
  \setcounter{enumi}{1}
  \item Interpret the financial leverage ratio calculated in Part A.
\end{enumerate}

\section{Solutions to 2:}
For 2017, every Rand in total equity supported R3.83 in total assets, on average. Financial leverage decreased from 2016 to 2017 on this measure.

\begin{enumerate}
  \setcounter{enumi}{2}
  \item What are the company's debt-to-assets, debt-to-capital, and debt-to-equity ratios for the three years?
\end{enumerate}

\section{Solutions to 3:}
(Amounts are millions of Rand other than ratios)

\begin{center}
\begin{tabular}{lccc}
\hline
 & $\mathbf{2 0 1 7}$ & $\mathbf{2 0 1 6}$ & $\mathbf{2 0 1 5}$ \\
\hline
Total Debt & 355,300 & 322,658 & 297,434 \\
Total Capital & 531,242 & 505,010 & 415,853 \\
Debt/Assets &  &  &  \\
Debt/Capital & $50.0 \%$ & $48.7 \%$ & $53.1 \%$ \\
Debt/Equity & $66.9 \%$ & $63.9 \%$ & $71.5 \%$ \\
\hline
\end{tabular}
\end{center}

\begin{enumerate}
  \setcounter{enumi}{3}
  \item Is there any discernable trend over the three years?
\end{enumerate}

\section{Solutions to 4:}
On all three metrics, the company's leverage decreased from 2015 to 2016 and increased from 2016 to 2017. For 2016 the decrease in leverage resulted from a conversion of subordinated debt into equity as well as additional issuance of equity. However, in 2017 debt levels increased again relative to assets, capital and equity indicating that the company's solvency has weakened. From a creditor's perspective, lower solvency (higher debt) indicates higher risk of default on obligations.

As with all ratio analysis, it is important to consider leverage ratios in a broader context. In general, companies with lower business risk and operations that generate steady cash flows are better positioned to take on more leverage without a commensurate increase in the risk of insolvency. In other words, a higher proportion of debt financing poses less risk of non-payment of interest and debt principal to a company with steady cash flows than to a company with volatile cash flows.

\section{1}
\section{PROFITABILITY RATIOS}
identify, calculate, and interpret activity, liquidity, solvency, profitability, and valuation ratios

The ability to generate profit on capital invested is a key determinant of a company's overall value and the value of the securities it issues. Consequently, many equity analysts would consider profitability to be a key focus of their analytical efforts.

Profitability reflects a company's competitive position in the market, and by extension, the quality of its management. The income statement reveals the sources of earnings and the components of revenue and expenses. Earnings can be distributed to shareholders or reinvested in the company. Reinvested earnings enhance solvency and provide a cushion against short-term problems.

\section{Calculation of Profitability Ratios}
Profitability ratios measure the return earned by the company during a period. Exhibit 17 provides the definitions of a selection of commonly used profitability ratios. Return-on-sales profitability ratios express various subtotals on the income statement (e.g., gross profit, operating profit, net profit) as a percentage of revenue. Essentially, these ratios constitute part of a common-size income statement discussed earlier. Return on investment profitability ratios measure income relative to assets, equity, or total capital employed by the company. For operating ROA, returns are measured as operating income, i.e., prior to deducting interest on debt capital. For ROA and ROE, returns are measured as net income, i.e., after deducting interest paid on debt capital. For return on common equity, returns are measured as net income minus preferred dividends (because preferred dividends are a return to preferred equity).

\section{Exhibit 17: Definitions of Commonly Used Profitability Ratios}
\begin{center}
\begin{tabular}{lll}
\hline
Profitability Ratios & Numerator & Denominator \\
\hline
Return on Sales $^{\mathrm{a}}$ &  &  \\
\hline
Gross profit margin & Gross profit & Revenue \\
Operating profit margin & Operating income ${ }^{\mathrm{b}}$ & Revenue \\
\end{tabular}
\end{center}

\begin{center}
\begin{tabular}{|c|c|c|}
\hline
Profitability Ratios & Numerator & Denominator \\
\hline
\multicolumn{3}{|l|}{Return on Sales $^{a}$} \\
\hline
Pretax margin & $\begin{array}{l}\text { EBT (earnings before tax but } \\ \text { after interest) }\end{array}$ & Revenue \\
\hline
Net profit margin & Net income & Revenue \\
\hline
\multicolumn{3}{|l|}{Return on Investment} \\
\hline
Operating ROA & Operating income & Average total assets \\
\hline
ROA & Net income & Average total assets \\
\hline
Return on total capital & EBIT & $\begin{array}{l}\text { Average short- and } \\ \text { long-term debt and } \\ \text { equity }\end{array}$ \\
\hline
$\mathrm{ROE}$ & Net income & Average total equity \\
\hline
Return on common equity & $\begin{array}{l}\text { Net income - Preferred } \\ \text { dividends }\end{array}$ & Average common equity \\
\hline
\end{tabular}
\end{center}

a "Sales" is being used as a synonym for "revenue."

b Some analysts use EBIT as a shortcut representation of operating income. Note that EBIT, strictly speaking, includes non-operating items such as dividends received and gains and losses on investment securities. Of utmost importance is that the analyst compute ratios consistently whether comparing different companies or analyzing one company over time.

\section{Interpretation of Profitability Ratios}
In the following, we discuss the interpretation of the profitability ratios presented in Exhibit 17. For each of the profitability ratios, a higher ratio indicates greater profitability.

\section{Gross Profit Margin}
Gross profit margin indicates the percentage of revenue available to cover operating and other expenses and to generate profit. Higher gross profit margin indicates some combination of higher product pricing and lower product costs. The ability to charge a higher price is constrained by competition, so gross profits are affected by (and usually inversely related to) competition. If a product has a competitive advantage (e.g., superior branding, better quality, or exclusive technology), the company is better able to charge more for it. On the cost side, higher gross profit margin can also indicate that a company has a competitive advantage in product costs.

\section{Operating Profit Margin}
Operating profit is calculated as gross profit minus operating costs. So, an operating profit margin increasing faster than the gross profit margin can indicate improvements in controlling operating costs, such as administrative overheads. In contrast, a declining operating profit margin could be an indicator of deteriorating control over operating costs.

\section{Pretax Margin}
Pretax income (also called "earnings before tax" or "EBT") is calculated as operating profit minus interest, and the pretax margin is the ratio of pretax income to revenue. The pretax margin reflects the effects on profitability of leverage and other (non-operating) income and expenses. If a company's pretax margin is increasing primarily as a result of increasing amounts of non-operating income, the analyst should evaluate whether this increase reflects a deliberate change in a company's business focus and, therefore, the likelihood that the increase will continue.

\section{Net Profit Margin}
Net profit, or net income, is calculated as revenue minus all expenses. Net income includes both recurring and non-recurring components. Generally, the net income used in calculating the net profit margin is adjusted for non-recurring items to offer a better view of a company's potential future profitability.

$R O A$

ROA measures the return earned by a company on its assets. The higher the ratio, the more income is generated by a given level of assets. Most databases compute this ratio as:

Net income

Average total assets

An issue with this computation is that net income is the return to equity holders, whereas assets are financed by both equity holders and creditors. Interest expense (the return to creditors) has already been subtracted in the numerator. Some analysts, therefore, prefer to add back interest expense in the numerator. In such cases, interest must be adjusted for income taxes because net income is determined after taxes. With this adjustment, the ratio would be computed as:

Net income + Interest expense ( 1 - Tax rate)

Average total assets

Alternatively, some analysts elect to compute ROA on a pre-interest and pre-tax basis (operating ROA in Exhibit 17) as:

$\frac{\text { Operating income or EBIT }}{\text { Average total assets }}$

In this ROA calculation, returns are measured prior to deducting interest on debt capital (i.e., as operating income or EBIT). This measure reflects the return on all assets invested in the company, whether financed with liabilities, debt, or equity. Whichever form of ROA is chosen, the analyst must use it consistently in comparisons to other companies or time periods.

\section{Return on Total Capital}
Return on total capital measures the profits a company earns on all of the capital that it employs (short-term debt, long-term debt, and equity). As with operating ROA, returns are measured prior to deducting interest on debt capital (i.e., as operating income or EBIT).

ROE

ROE measures the return earned by a company on its equity capital, including minority equity, preferred equity, and common equity. As noted, return is measured as net income (i.e., interest on debt capital is not included in the return on equity capital). A variation of ROE is return on common equity, which measures the return earned by a company only on its common equity.

Both ROA and ROE are important measures of profitability and will be explored in more detail in section 4.6.2. As with other ratios, profitability ratios should be evaluated individually and as a group to gain an understanding of what is driving profitability (operating versus non-operating activities). Example 12 demonstrates the evaluation of profitability ratios and the use of the management report (sometimes called management's discussion and analysis or management commentary) that accompanies financial statements to explain the trend in ratios.

\section{EXAMPLE 12}
\section{Evaluation of Profitability Ratios}
\begin{enumerate}
  \item Recall from Example 1 that an analysis found that Apple's gross margin declined over the three-year period FY2015 to FY2017. An analyst would like to further explore Apple's profitability using a five-year period. He gathers the following revenue data and calculates the following profitability ratios from information in Apple's annual reports:
\end{enumerate}

\begin{center}
\begin{tabular}{lccccc}
\hline
Dollars in millions & $\mathbf{2 0 1 7}$ & $\mathbf{2 0 1 6}$ & $\mathbf{2 0 1 5}$ & $\mathbf{2 0 1 4}$ & $\mathbf{2 0 1 3}$ \\
\hline
Sales & 229,234 & 215,639 & 233,715 & 182,795 & 170,910 \\
Gross Profit & 88,186 & 84,263 & 93,626 & 70,537 & 64,304 \\
Operating Income & 61,344 & 60,024 & 71,230 & 52,503 & 48,999 \\
Pre-tax Income & 64,089 & 61,372 & 72,515 & 53,483 & 50,155 \\
Net Income & 48,351 & 45,687 & 53,394 & 39,510 & 37,037 \\
 &  &  &  &  &  \\
Gross profit margin & $38.47 \%$ & $39.08 \%$ & $40.06 \%$ & $38.59 \%$ & $37.62 \%$ \\
Operating income margin & $26.76 \%$ & $27.84 \%$ & $30.48 \%$ & $28.72 \%$ & $28.67 \%$ \\
Pre-tax income & $27.96 \%$ & $28.46 \%$ & $31.03 \%$ & $29.26 \%$ & $29.35 \%$ \\
Net profit margin & $21.09 \%$ & $21.19 \%$ & $22.85 \%$ & $21.61 \%$ & $21.67 \%$ \\
\hline
\end{tabular}
\end{center}

Evaluate the overall trend in Apple's profitability ratios for the five-year period.

\section{Solution:}
Sales had increased steadily through 2015, dropped in 2016, and rebounded somewhat in 2017. As noted in Example 1, the sales decline in 2016 was related to a decline in iPhone sales and weakness in foreign currencies. Margins also rose from 2013 to 2015 and declined in 2016. However, in spite of the increase in sales in 2017, all margins declined slightly indicating costs were rising faster than sales. In spite of the fluctuations, Apple's bottom line net profit margin was relatively stable over the five-year period.

\section{INTEGRATED FINANCIAL RATIO ANALYSIS}
In prior sections, the text presented separately activity, liquidity, solvency, and profitability ratios. Prior to discussing valuation ratios, the following sections demonstrate the importance of examining a variety of financial ratios-not a single ratio or category of ratios in isolation-to ascertain the overall position and performance of a company. Experience shows that the information from one ratio category can be helpful in answering questions raised by another category and that the most accurate overall picture comes from integrating information from all sources. Section 4.6.1 provides some introductory examples of such analysis and Section 4.6.2 shows how return on equity can be analyzed into components related to profit margin, asset utilization (activity), and financial leverage.

\section{The Overall Ratio Picture: Examples}
This section presents two simple illustrations to introduce the use of a variety of ratios to address an analytical task. Example 13 shows how the analysis of a pair of activity ratios resolves an issue concerning a company's liquidity. Example 14 shows that examining the overall ratios of multiple companies can assist an analyst in drawing conclusions about their relative performances.

\section{EXAMPLE 13}
\section{A Variety of Ratios}
An analyst is evaluating the liquidity of a Canadian manufacturing company and obtains the following liquidity ratios:

\begin{center}
\begin{tabular}{llll}
\hline
Fiscal Year & $\mathbf{1 0}$ & $\mathbf{9}$ & $\mathbf{8}$ \\
\hline
Current ratio & 2.1 & 1.9 & 1.6 \\
Quick ratio & 0.8 & 0.9 & 1.0 \\
\hline
\end{tabular}
\end{center}

The ratios present a contradictory picture of the company's liquidity. Based on the increase in its current ratio from 1.6 to 2.1 , the company appears to have strong and improving liquidity; however, based on the decline of the quick ratio from 1.0 to 0.8 , its liquidity appears to be deteriorating. Because both ratios have exactly the same denominator, current liabilities, the difference must be the result of changes in some asset that is included in the current ratio but not in the quick ratio (e.g., inventories). The analyst collects the following activity ratios:

\begin{center}
\begin{tabular}{llll}
\hline
$\mathrm{DOH}$ & 55 & 45 & 30 \\
$\mathrm{DSO}$ & 24 & 28 & 30 \\
\hline
\end{tabular}
\end{center}

The company's DOH has deteriorated from 30 days to 55 days, meaning that the company is holding increasingly larger amounts of inventory relative to sales. The decrease in DSO implies that the company is collecting receivables faster. If the proceeds from these collections were held as cash, there would be no effect on either the current ratio or the quick ratio. However, if the proceeds from the collections were used to purchase inventory, there would be no effect on the current ratio and a decline in the quick ratio (i.e., the pattern shown in this example). Collectively, the ratios suggest that liquidity is declining and that the company may have an inventory problem that needs to be addressed.

\section{EXAMPLE 14}
\section{A Comparison of Two Companies (1)}
\begin{enumerate}
  \item An analyst collects the information ${ }^{10}$ shown in Exhibit 18 for two hypothetical companies:
\end{enumerate}

\section{Exhibit 18}
\begin{center}
\begin{tabular}{lcccc}
\hline
 & \multicolumn{4}{c}{Fiscal Year} \\
\hline
Anson Industries & $\mathbf{5}$ & $\mathbf{4}$ & $\mathbf{3}$ & $\mathbf{2}$ \\
\hline
Inventory turnover & 76.69 & 89.09 & 147.82 & 187.64 \\
DOH & 4.76 & 4.10 & 2.47 & 1.95 \\
Receivables turnover & 10.75 & 9.33 & 11.14 & 7.56 \\
DSO & 33.95 & 39.13 & 32.77 & 48.29 \\
Accounts payable turnover & 4.62 & 4.36 & 4.84 & 4.22 \\
Days payable & 78.97 & 83.77 & 75.49 & 86.56 \\
Cash from operations/Total liabilities & $31.41 \%$ & $11.15 \%$ & $4.04 \%$ & $8.81 \%$ \\
ROE & $5.92 \%$ & $1.66 \%$ & $1.62 \%$ & $-0.62 \%$ \\
ROA & $3.70 \%$ & $1.05 \%$ & $1.05 \%$ & $-0.39 \%$ \\
Net profit margin (Net income/Revenue) & $3.33 \%$ & $1.11 \%$ & $1.13 \%$ & $-0.47 \%$ \\
Total asset turnover (Revenue/Average & 1.11 & 0.95 & 0.93 & 0.84 \\
assets) &  &  &  &  \\
Leverage (Average assets/Average & 1.60 & 1.58 & 1.54 & 1.60 \\
equity) &  &  &  &  \\
\hline
\end{tabular}
\end{center}

\begin{center}
\begin{tabular}{lcccc}
\hline
 & \multicolumn{5}{c}{Fiscal Year} \\
\hline
Clarence Corporation & $\mathbf{5}$ & $\mathbf{4}$ & $\mathbf{3}$ & $\mathbf{2}$ &  \\
\hline
Inventory turnover & 9.19 & 9.08 & 7.52 & 14.84 &  \\
DOH & 39.73 & 40.20 & 48.51 & 24.59 &  \\
Receivables turnover & 8.35 & 7.01 & 6.09 & 5.16 &  \\
DSO & 43.73 & 52.03 & 59.92 & 70.79 &  \\
Accounts payable turnover & 6.47 & 6.61 & 7.66 & 6.52 &  \\
Days payable & 56.44 & 55.22 & 47.64 & 56.00 &  \\
Cash from operations/Total liabilities & $13.19 \%$ & $16.39 \%$ & $15.80 \%$ & $11.79 \%$ &  \\
ROE & $9.28 \%$ & $6.82 \%$ & $-3.63 \%$ & $-6.75 \%$ &  \\
ROA & $4.64 \%$ & $3.48 \%$ & $-1.76 \%$ & $-3.23 \%$ &  \\
Net profit margin (Net income/Revenue) & $4.38 \%$ & $3.48 \%$ & $-1.60 \%$ & $-2.34 \%$ &  \\
Total asset turnover (Revenue/Average & 1.06 & 1.00 & 1.10 & 1.38 &  \\
assets) &  &  &  &  &  \\
Leverage (Average assets/Average & 2.00 & 1.96 & 2.06 & 2.09 &  \\
equity) &  &  &  &  &  \\
\end{tabular}
\end{center}

10 Note that ratios are expressed in terms of two decimal places and are rounded. Therefore, expected relationships may not hold perfectly. Which of the following choices best describes reasonable conclusions an analyst might make about the companies' efficiency?

A. Over the past four years, Anson has shown greater improvement in efficiency than Clarence, as indicated by its total asset turnover ratio increasing from 0.84 to 1.11 .

B. In FY5, Anson's DOH of only 4.76 indicated that it was less efficient at inventory management than Clarence, which had DOH of 39.73 .

C. In FY5, Clarence's receivables turnover of 8.35 times indicated that it was more efficient at receivables management than Anson, which had receivables turnover of 10.75 .

\section{Solution:}
A is correct. Over the past four years, Anson has shown greater improvement in efficiency than Clarence, as indicated by its total asset turnover ratio increasing from 0.84 to 1.11 . Over the same period of time, Clarence's total asset turnover ratio has declined from 1.38 to 1.06 . Choices B and C are incorrect because $\mathrm{DOH}$ and receivables turnover are misinterpreted.

\section{DUPONT ANALYSIS: THE DECOMPOSITION OF ROE}
demonstrate the application of DuPont analysis of return on equity and calculate and interpret effects of changes in its components

As noted earlier, ROE measures the return a company generates on its equity capital. To understand what drives a company's ROE, a useful technique is to decompose ROE into its component parts. (Decomposition of ROE is sometimes referred to as DuPont analysis because it was developed originally at that company.) Decomposing ROE involves expressing the basic ratio (i.e., net income divided by average shareholders' equity) as the product of component ratios. Because each of these component ratios is an indicator of a distinct aspect of a company's performance that affects ROE, the decomposition allows us to evaluate how these different aspects of performance affected the company's profitability as measured by ROE. 11

Decomposing ROE is useful in determining the reasons for changes in ROE over time for a given company and for differences in ROE for different companies in a given time period. The information gained can also be used by management to determine which areas they should focus on to improve ROE. This decomposition will also show why a company's overall profitability, measured by ROE, is a function of its efficiency, operating profitability, taxes, and use of financial leverage. DuPont analysis shows the relationship between the various categories of ratios discussed in this reading and how they all influence the return to the investment of the owners.

Analysts have developed several different methods of decomposing ROE. The decomposition presented here is one of the most commonly used and the one found in popular research databases, such as Bloomberg. Return on equity is calculated as:

11 For purposes of analyzing ROE, this method usually uses average balance sheet factors; however, the math will work out if beginning or ending balances are used throughout. For certain purposes, these alternative methods may be appropriate. $\mathrm{ROE}=$ Net income/Average shareholders' equity

The decomposition of ROE makes use of simple algebra and illustrates the relationship between ROE and ROA. Expressing ROE as a product of only two of its components, we can write:

ROE $=\frac{\text { Net income }}{\text { Average shareholders' equity }}$

$=\frac{\text { Net income }}{\text { Average total assets }} \times \frac{\text { Average total assets }}{\text { Average shareholders' equity }}$

which can be interpreted as:

$\mathrm{ROE}=\mathrm{ROA} \times$ Leverage

In other words, ROE is a function of a company's ROA and its use of financial leverage ("leverage" for short, in this discussion). A company can improve its ROE by improving ROA or making more effective use of leverage. Consistent with the definition given earlier, leverage is measured as average total assets divided by average shareholders' equity. If a company had no leverage (no liabilities), its leverage ratio would equal 1.0 and ROE would exactly equal ROA. As a company takes on liabilities, its leverage increases. As long as a company is able to borrow at a rate lower than the marginal rate it can earn investing the borrowed money in its business, the company is making an effective use of leverage and ROE would increase as leverage increases. If a company's borrowing cost exceeds the marginal rate it can earn on investing in the business, ROE would decline as leverage increased because the effect of borrowing would be to depress ROA.

Using the data from Example 14 for Anson Industries, an analyst can examine the trend in ROE and determine whether the increase from an ROE of -0.625 percent in FY2 to 5.925 percent in FY5 is a function of ROA or the use of leverage:

\begin{center}
\begin{tabular}{|c|c|c|c|c|c|}
\hline
 & ROE & $=$ & ROA & $x$ & Leverage \\
\hline
FY5 & $5.92 \%$ &  & $3.70 \%$ &  & 1.60 \\
\hline
FY4 & $1.66 \%$ &  & $1.05 \%$ &  & 1.58 \\
\hline
FY3 & $1.62 \%$ &  & $1.05 \%$ &  & 1.54 \\
\hline
FY2 & $-0.62 \%$ &  & $-0.39 \%$ &  & 1.60 \\
\hline
\end{tabular}
\end{center}

Over the four-year period, the company's leverage factor was relatively stable. The primary reason for the increase in ROE is the increase in profitability measured by ROA.

Just as ROE can be decomposed, the individual components such as ROA can be decomposed. Further decomposing ROA, we can express ROE as a product of three component ratios:

$\frac{\text { Net income }}{\text { Average shareholders' equity }}=\frac{\text { Net income }}{\text { Revenue }} \times \frac{\text { Revenue }}{\text { Average total assets }}$

$\frac{\text { Net income }}{\text { Average shareholders' equity }}=\frac{\text { Net income }}{\text { Revenue }} \times \frac{\text { Revenue }}{\text { Average total assets }}$

Average total assets

$\times \overline{\text { Average shareholders' equity }}$

which can be interpreted as:

ROE $=$ Net profit margin $\times$ Total asset turnover $\times$ Leverage

The first term on the right-hand side of this equation is the net profit margin, an indicator of profitability: how much income a company derives per one monetary unit (e.g., euro or dollar) of sales. The second term on the right is the asset turnover ratio, an indicator of efficiency: how much revenue a company generates per one money unit of assets. Note that ROA is decomposed into these two components: net profit margin and total asset turnover. A company's ROA is a function of profitability (net profit margin) and efficiency (total asset turnover). The third term on the right-hand side of Equation 2 is a measure of financial leverage, an indicator of solvency: the total amount of a company's assets relative to its equity capital. This decomposition illustrates that a company's ROE is a function of its net profit margin, its efficiency, and its leverage. Again, using the data from Example 14 for Anson Industries, the analyst can evaluate in more detail the reasons behind the trend in ROE:12

\begin{center}
\begin{tabular}{|c|c|c|c|c|c|c|c|}
\hline
 & ROE & $=$ & Net profit margin & $x$ & Total asset turnover & $x$ & Leverage \\
\hline
FY5 & $5.92 \%$ &  & $3.33 \%$ &  & 1.11 &  & 1.60 \\
\hline
FY4 & $1.66 \%$ &  & $1.11 \%$ &  & 0.95 &  & 1.58 \\
\hline
FY3 & $1.62 \%$ &  & $1.13 \%$ &  & 0.93 &  & 1.54 \\
\hline
FY2 & $-0.62 \%$ &  & $-0.47 \%$ &  & 0.84 &  & 1.60 \\
\hline
\end{tabular}
\end{center}

This further decomposition confirms that increases in profitability (measured here as net profit margin) are indeed an important contributor to the increase in ROE over the four-year period. However, Anson's asset turnover has also increased steadily. The increase in ROE is, therefore, a function of improving profitability and improving efficiency. As noted above, ROE decomposition can also be used to compare the ROEs of peer companies, as demonstrated in Example 15.

\section{EXAMPLE 15}
\section{A Comparison of Two Companies (2)}
\begin{enumerate}
  \item Referring to the data for Anson Industries and Clarence Corporation in Example 14, which of the following choices best describes reasonable conclusions an analyst might make about the companies' ROE?
\end{enumerate}

A. Anson's inventory turnover of 76.69 indicates it is more profitable than Clarence.

B. The main driver of Clarence's superior ROE in FY5 is its more efficient use of assets.

C. The main drivers of Clarence's superior ROE in FY5 are its greater use of debt financing and higher net profit margin.

Solution:

$\mathrm{C}$ is correct. The main driver of Clarence's superior ROE (9.28 percent compared with only 5.92 percent for Anson) in FY5 is its greater use of debt financing (leverage of 2.00 compared with Anson's leverage of 1.60) and higher net profit margin ( 4.38 percent compared with only 3.33 percent for Anson). A is incorrect because inventory turnover is not a direct indicator of profitability. An increase in inventory turnover may indicate more efficient use of inventory which in turn could affect profitability; however, an increase in inventory turnover would also be observed if a company was selling more goods even if it was not selling those goods at a profit. B is incorrect because Clarence has less efficient use of assets than Anson, indicated by turnover of 1.06 for Clarence compared with Anson's turnover of 1.11.

To separate the effects of taxes and interest, we can further decompose the net profit margin and write:

12 Ratios are expressed in terms of two decimal places and are rounded. Therefore, ROE may not be the exact product of the three ratios.

$$
\begin{gathered}
\frac{\text { Net income }}{\text { Average shareholders' equity }}=\frac{\text { Net income }}{\text { EBT }} \times \frac{\mathrm{EBT}}{\text { EBIT }} \times \frac{\mathrm{EBIT}}{\text { Revenue }} \\
\times \frac{\text { Revenue }}{\text { Average total assets }} \times \frac{\text { Average total assets }}{\text { Average shareholders' equity }}
\end{gathered}
$$

which can be interpreted as:

$\mathrm{ROE}=$ Tax burden $\times$ Interest burden $\times$ EBIT margin $\times$ Total asset turnover $\times$

Leverage

This five-way decomposition is the one found in financial databases such as Bloomberg. The first term on the right-hand side of this equation measures the effect of taxes on ROE. Essentially, it reflects one minus the average tax rate, or how much of a company's pretax profits it gets to keep. This can be expressed in decimal or percentage form. So, a 30 percent tax rate would yield a factor of 0.70 or 70 percent. A higher value for the tax burden implies that the company can keep a higher percentage of its pretax profits, indicating a lower tax rate. A decrease in the tax burden ratio implies the opposite (i.e., a higher tax rate leaving the company with less of its pretax profits).

The second term on the right-hand side captures the effect of interest on ROE. Higher borrowing costs reduce ROE. Some analysts prefer to use operating income instead of EBIT for this term and the following term. Either operating income or EBIT is acceptable as long as it is applied consistently. In such a case, the second term would measure both the effect of interest expense and non-operating income on ROE.

The third term on the right-hand side captures the effect of operating margin (if operating income is used in the numerator) or EBIT margin (if EBIT is used) on ROE. In either case, this term primarily measures the effect of operating profitability on ROE.

The fourth term on the right-hand side is again the total asset turnover ratio, an indicator of the overall efficiency of the company (i.e., how much revenue it generates per unit of total assets). The fifth term on the right-hand side is the financial leverage ratio described above-the total amount of a company's assets relative to its equity capital.

This decomposition expresses a company's ROE as a function of its tax rate, interest burden, operating profitability, efficiency, and leverage. An analyst can use this framework to determine what factors are driving a company's ROE. The decomposition of ROE can also be useful in forecasting ROE based upon expected efficiency, profitability, financing activities, and tax rates. The relationship of the individual factors, such as ROA to the overall ROE, can also be expressed in the form of an ROE tree to study the contribution of each of the five factors, as shown in Exhibit 19 for Anson Industries. 13

Exhibit 19 shows that Anson's ROE of 5.92 percent in FY5 can be decomposed into ROA of 3.70 percent and leverage of 1.60 . ROA can further be decomposed into a net profit margin of 3.33 percent and total asset turnover of 1.11 . Net profit margin can be decomposed into a tax burden of 0.70 (an average tax rate of 30 percent), an interest burden of 0.90 , and an EBIT margin of 5.29 percent. Overall ROE is decomposed into five components.

13 Note that a breakdown of net profit margin was not provided in Example 14, but is added here.

\section{Exhibit 19: DuPont Analysis of Anson Industries' ROE: Fiscal Year 5}
\begin{center}
\includegraphics[max width=\textwidth]{2023_05_04_b5cfa4f1bc883752f121g-242}
\end{center}

Example 16 demonstrates how the five-component decomposition can be used to determine reasons behind the trend in a company's ROE.

\section{EXAMPLE 16}
\section{Five-Way Decomposition of ROE}
\begin{enumerate}
  \item An analyst examining Amsterdam PLC (a hypothetical company) wishes to understand the factors driving the trend in ROE over a four-year period. The analyst obtains and calculates the following data from Amsterdam's annual reports:
\end{enumerate}

\begin{center}
\begin{tabular}{lcccc}
\hline
 & $\mathbf{2 0 1 7}$ & $\mathbf{2 0 1 6}$ & $\mathbf{2 0 1 5}$ & $\mathbf{2 0 1 4}$ \\
\hline
ROE & $9.53 \%$ & $20.78 \%$ & $26.50 \%$ & $24.72 \%$ \\
Tax burden & $60.50 \%$ & $52.10 \%$ & $63.12 \%$ & $58.96 \%$ \\
Interest burden & $97.49 \%$ & $97.73 \%$ & $97.86 \%$ & $97.49 \%$ \\
EBIT margin & $7.56 \%$ & $11.04 \%$ & $13.98 \%$ & $13.98 \%$ \\
Asset turnover & 0.99 & 1.71 & 1.47 & 1.44 \\
Leverage & 2.15 & 2.17 & 2.10 & 2.14 \\
\hline
\end{tabular}
\end{center}

What might the analyst conclude?

\section{Solution:}
The tax burden measure has varied, with no obvious trend. In the most recent year, 2017, taxes declined as a percentage of pretax profit. (Because the tax burden reflects the relation of after-tax profits to pretax profits, the increase from 52.10 percent in 2016 to 60.50 percent in 2017 indicates that taxes declined as a percentage of pretax profits.) This decline in average tax rates could be a result of lower tax rates from new legislation or revenue in a lower tax jurisdiction. The interest burden has remained fairly constant over the four-year period indicating that the company maintains a fairly constant capital structure. Operating margin (EBIT margin) declined over the period, indicating the company's operations were less profitable. This decline is generally consistent with declines in oil prices in 2017 and declines in refining industry gross margins in 2016 and 2017. The company's efficiency (asset turnover) decreased in 2017. The company's leverage remained constant, consistent with the constant interest burden. Overall, the trend in ROE (declining substantially over the recent years) resulted from decreases in operating profits and a lower asset turnover. Additional research on the causes of these changes is required in order to develop expectations about the company's future performance.

The most detailed decomposition of ROE that we have presented is a five-way decomposition. Nevertheless, an analyst could further decompose individual components of a five-way analysis. For example, EBIT margin (EBIT/Revenue) could be further decomposed into a non-operating component (EBIT/Operating income) and an operating component (Operating income/Revenue). The analyst can also examine which other factors contributed to these five components. For example, an improvement in efficiency (total asset turnover) may have resulted from better management of inventory (DOH) or better collection of receivables (DSO).

\section{EQUITY ANALYSIS AND VALUATION RATIOS}
calculate and interpret ratios used in equity analysis and credit analysis

One application of financial analysis is to select securities as part of the equity portfolio management process. Analysts are interested in valuing a security to assess its merits for inclusion or retention in a portfolio. The valuation process has several steps, including:

\begin{enumerate}
  \item understanding the business and the existing financial profile

  \item forecasting company performance

  \item selecting the appropriate valuation model

  \item converting forecasts to a valuation

  \item making the investment decision

\end{enumerate}

Financial analysis assists in providing the core information to complete the first two steps of this valuation process: understanding the business and forecasting performance.

Fundamental equity analysis involves evaluating a company's performance and valuing its equity in order to assess its relative attractiveness as an investment. Analysts use a variety of methods to value a company's equity, including valuation ratios (e.g. the price-to-earnings or P/E ratio), discounted cash flow approaches, and residual income approaches (ROE compared with the cost of capital), among others. The following section addresses the first of these approaches-the use of valuation ratios.

\section{Valuation Ratios}
Valuation ratios have long been used in investment decision making. A well known example is the price to earnings ratio (P/E ratio)-probably the most widely cited indicator in discussing the value of equity securities-which relates share price to the earnings per share (EPS). Additionally, some analysts use other market multiples, such as price to book value (P/B) and price to cash flow (P/CF). The following sections explore valuation ratios and other quantities related to valuing equities.

\section{Calculation of Valuation Ratios and Related Quantities}
Exhibit 20 describes the calculation of some common valuation ratios and related quantities.

\section{Exhibit 20: Definitions of Selected Valuation Ratios and Related Quantities}
\begin{center}
\begin{tabular}{|c|c|c|}
\hline
Valuation Ratios & Numerator & Denominator \\
\hline
$\mathrm{P} / \mathrm{E}$ & Price per share & Earnings per share \\
\hline
$\mathrm{P} / \mathrm{CF}$ & Price per share & Cash flow per share \\
\hline
$\mathrm{P} / \mathrm{S}$ & Price per share & Sales per share \\
\hline
$\mathrm{P} / \mathrm{BV}$ & Price per share & Book value per share \\
\hline
Per-Share Quantities & Numerator & Denominator \\
\hline
Basic EPS & $\begin{array}{l}\text { Net income minus pre- } \\ \text { ferred dividends }\end{array}$ & $\begin{array}{l}\text { Weighted average number of } \\ \text { ordinary shares outstanding }\end{array}$ \\
\hline
Diluted EPS & $\begin{array}{l}\text { Adjusted income avail- } \\ \text { able for ordinary shares, } \\ \text { reflecting conversion of } \\ \text { dilutive securities }\end{array}$ & $\begin{array}{l}\text { Weighted average number of } \\ \text { ordinary and potential ordinary } \\ \text { shares outstanding }\end{array}$ \\
\hline
Cash flow per share & Cash flow from operations & $\begin{array}{l}\text { Weighted average number of } \\ \text { shares outstanding }\end{array}$ \\
\hline
EBITDA per share & EBITDA & $\begin{array}{l}\text { Weighted average number of } \\ \text { shares outstanding }\end{array}$ \\
\hline
Dividends per share & $\begin{array}{l}\text { Common dividends } \\ \text { declared }\end{array}$ & $\begin{array}{l}\text { Weighted average number of } \\ \text { ordinary shares outstanding }\end{array}$ \\
\hline
$\begin{array}{l}\text { Dividend-Related } \\ \text { Quantities }\end{array}$ & Numerator & Denominator \\
\hline
Dividend payout ratio & Common share dividends & $\begin{array}{l}\text { Net income attributable to com- } \\ \text { mon shares }\end{array}$ \\
\hline
Retention rate $(b)$ & $\begin{array}{l}\text { Net income attributable to } \\ \text { common shares - }\end{array}$ & $\begin{array}{l}\text { Net income attributable to com- } \\ \text { mon shares }\end{array}$ \\
\hline
 & Common share dividends &  \\
\hline
Sustainable growth rate & $b \times \mathrm{ROE}$ &  \\
\hline
\end{tabular}
\end{center}

The P/E ratio expresses the relationship between the price per share and the amount of earnings attributable to a single share. In other words, the P/E ratio tells us how much an investor in common stock pays per dollar of earnings.

Because P/E ratios are calculated using net income, the ratios can be sensitive to non-recurring earnings or one-time earnings events. In addition, because net income is generally considered to be more susceptible to manipulation than are cash flows, analysts may use price to cash flow as an alternative measure-particularly in situations where earnings quality may be an issue. EBITDA per share, because it is calculated using income before interest, taxes, and depreciation, can be used to eliminate the effect of different levels of fixed asset investment across companies. It facilitates comparison between companies in the same sector but at different stages of infrastructure maturity. Price to sales is calculated in a similar manner and is sometimes used as a comparative price metric when a company does not have positive net income.

Another price-based ratio that facilitates useful comparisons of companies' stock prices is price to book value, or $\mathrm{P} / \mathrm{B}$, which is the ratio of price to book value per share. This ratio is often interpreted as an indicator of market judgment about the relationship between a company's required rate of return and its actual rate of return. Assuming that book values reflect the fair values of the assets, a price to book ratio of one can be interpreted as an indicator that the company's future returns are expected to be exactly equal to the returns required by the market. A ratio greater than one would indicate that the future profitability of the company is expected to exceed the required rate of return, and values of this ratio less than one indicate that the company is not expected to earn excess returns. ${ }^{14}$

\section{Interpretation of Earnings per Share}
Exhibit 20 presented a number of per-share quantities that can be used in valuation ratios. In this section, we discuss the interpretation of one such critical quantity, earnings per share or EPS. 15

EPS is simply the amount of earnings attributable to each share of common stock. In isolation, EPS does not provide adequate information for comparison of one company with another. For example, assume that two companies have only common stock outstanding and no dilutive securities outstanding. In addition, assume the two companies have identical net income of $\$ 10$ million, identical book equity of $\$ 100$ million and, therefore, identical profitability (10 percent, using ending equity in this case for simplicity). Furthermore, assume that Company A has 100 million weighted average common shares outstanding, whereas Company B has 10 million weighted average common shares outstanding. So, Company A will report EPS of $\$ 0.10$ per share, and Company B will report EPS of $\$ 1$ per share. The difference in EPS does not reflect a difference in profitability - the companies have identical profits and profitability. The difference reflects only a different number of common shares outstanding. Analysts should understand in detail the types of EPS information that companies report:

Basic EPS provides information regarding the earnings attributable to each share of common stock. ${ }^{16}$ To calculate basic EPS, the weighted average number of shares outstanding during the period is first calculated. The weighted average number of shares consists of the number of ordinary shares outstanding at the beginning of the period, adjusted by those bought back or issued during the period, multiplied by a time-weighting factor.

14 For more detail on valuation ratios as used in equity analysis, see the curriculum reading "Equity Valuation: Concepts and Basic Tools."

15 For more detail on EPS calculation, see the reading "Understanding Income Statements."

16 IAS 33, Earnings per Share and FASB ASC Topic 260 [Earnings per Share]. Accounting standards generally require the disclosure of basic as well as diluted EPS (diluted EPS includes the effect of all the company's securities whose conversion or exercise would result in a reduction of basic EPS; dilutive securities include convertible debt, convertible preferred, warrants, and options). Basic EPS and diluted EPS must be shown with equal prominence on the face of the income statement for each class of ordinary share. Disclosure includes the amounts used as the numerators in calculating basic and diluted EPS, and a reconciliation of those amounts to the company's profit or loss for the period. Because both basic and diluted EPS are presented in a company's financial statements, an analyst does not need to calculate these measures for reported financial statements. Understanding the calculations is, however, helpful for situations requiring an analyst to calculate expected future EPS.

To calculate diluted EPS, earnings are adjusted for the after-tax effects assuming conversion, and the following adjustments are made to the weighted number of shares:

\begin{itemize}
  \item The weighted average number of shares for basic EPS, plus those that would be issued on conversion of all potentially dilutive ordinary shares. Potential ordinary shares are treated as dilutive when their conversion would decrease net profit per share from continuing ordinary operations.

  \item These shares are deemed to have been converted into ordinary shares at the beginning of the period or, if later, at the date of the issue of the shares.

  \item Options, warrants (and their equivalents), convertible instruments, contingently issuable shares, contracts that can be settled in ordinary shares or cash, purchased options, and written put options should be considered.

\end{itemize}

\section{Dividend-Related Quantities}
In this section, we discuss the interpretation of the dividend-related quantities presented in Exhibit 20. These quantities play a role in some present value models for valuing equities.

\section{Dividend Payout Ratio}
The dividend payout ratio measures the percentage of earnings that the company pays out as dividends to shareholders. The amount of dividends per share tends to be relatively fixed because any reduction in dividends has been shown to result in a disproportionately large reduction in share price. Because dividend amounts are relatively fixed, the dividend payout ratio tends to fluctuate with earnings. Therefore, conclusions about a company's dividend payout policies should be based on examination of payout over a number of periods. Optimal dividend policy, similar to optimal capital structure, has been examined in academic research and continues to be a topic of significant interest in corporate finance.

\section{Retention Rate}
The retention rate, or earnings retention rate, is the complement of the payout ratio or dividend payout ratio (i.e., 1 - payout ratio). Whereas the payout ratio measures the percentage of earnings that a company pays out as dividends, the retention rate is the percentage of earnings that a company retains. (Note that both the payout ratio and retention rate are both percentages of earnings. The difference in terminology-"ratio" versus "rate" versus "percentage"-reflects common usage rather than any substantive differences.)

Sustainable Growth Rate

A company's sustainable growth rate is viewed as a function of its profitability (measured as ROE) and its ability to finance itself from internally generated funds (measured as the retention rate). The sustainable growth rate is ROE times the retention rate. A higher $\mathrm{ROE}$ and a higher retention rate result in a higher sustainable growth rate. This calculation can be used to estimate a company's growth rate, a factor commonly used in equity valuation.

INDUSTRY-SPECIFIC FINANCIAL RATIOS

calculate and interpret ratios used in equity analysis and credit analysis

As stated earlier in this reading, a universally accepted definition and classification of ratios does not exist. The purpose of ratios is to serve as indicators of important aspects of a company's performance and value. Aspects of performance that are considered important in one industry may be irrelevant in another, and industry-specific ratios reflect these differences. For example, companies in the retail industry may report same-store sales changes because, in the retail industry, it is important to distinguish between growth that results from opening new stores and growth that results from generating more sales at existing stores. Industry-specific metrics can be especially important to the value of equity in early stage industries, where companies are not yet profitable.

In addition, regulated industries-especially in the financial sector-often are required to comply with specific regulatory ratios. For example, the banking sector's liquidity and cash reserve ratios provide an indication of banking liquidity and reflect monetary and regulatory requirements. Banking capital adequacy requirements attempt to relate banks' solvency requirements directly to their specific levels of risk exposure.

Exhibit 21 presents, for illustrative purposes only, some industry-specific and task-specific ratios. ${ }^{17}$

\section{Exhibit 21: Definitions of Some Common Industry- and Task-Specific Ratios}
\begin{center}
\begin{tabular}{lll}
\hline
Ratio & Numerator & Denominator \\
\hline
Business Risk &  &  \\
\hline
Coefficient of variation of & $\begin{array}{l}\text { Standard deviation of oper- } \\ \text { ating income }\end{array}$ & Average operating income \\
operating income & $\begin{array}{l}\text { Standard deviation of net } \\ \text { income }\end{array}$ & Average net income \\
$\begin{array}{ll}\text { Coefficient of variation of net }\end{array}$ & $\begin{array}{l}\text { Standard deviation of } \\ \text { revenue }\end{array}$ & Average revenue \\
Coefficient of variation of &  &  \\
\hline
\end{tabular}
\end{center}

\begin{center}
\begin{tabular}{lll}
\hline
Financial Sector Ratios & Numerator & Denominator \\
\hline
Capital adequacy-banks & Various components of & Various measures such as \\
 & capital & risk-weighted assets, market \\
 &  & risk exposure, or level of \\
 & operational risk assumed &  \\
\end{tabular}
\end{center}

17 There are many other industry- and task-specific ratios that are outside the scope of this reading. Resources such as Standard and Poor's Industry Surveys present useful ratios for each industry. Industry organizations may present useful ratios for the industry or a task specific to the industry.

\begin{center}
\begin{tabular}{|c|c|c|}
\hline
Financial Sector Ratios & Numerator & Denominator \\
\hline
$\begin{array}{l}\text { Monetary reserve requirement } \\ \text { (Cash reserve ratio) }\end{array}$ & Reserves held at central bank & Specified deposit liabilities \\
\hline
Liquid asset requirement & $\begin{array}{l}\text { Approved "readily market- } \\ \text { able" securities }\end{array}$ & Specified deposit liabilities \\
\hline
Net interest margin & Net interest income & Total interest-earning assets \\
\hline
Retail Ratios & Numerator & Denominator \\
\hline
$\begin{array}{l}\text { Same (or comparable) store } \\ \text { sales }\end{array}$ & $\begin{array}{l}\text { Average revenue growth year } \\ \text { over year for stores open in } \\ \text { both periods }\end{array}$ & Not applicable \\
\hline
$\begin{array}{l}\text { Sales per square meter (or } \\ \text { square foot) }\end{array}$ & Revenue & $\begin{array}{l}\text { Total retail space in square } \\ \text { meters (or square feet) }\end{array}$ \\
\hline
Service Companies & Numerator & Denominator \\
\hline
Revenue per employee & Revenue & Total number of employees \\
\hline
Net income per employee & Net income & Total number of employees \\
\hline
Hotel & Numerator & Denominator \\
\hline
Average daily rate & Room revenue & Number of rooms sold \\
\hline
Occupancy rate & Number of rooms sold & Number of rooms available \\
\hline
\end{tabular}
\end{center}

\section{RESEARCH ON FINANCIAL RATIOS IN CREDIT AND EQUITY ANALYSIS}
calculate and interpret ratios used in equity analysis and credit analysis

Some ratios may be particularly useful in equity analysis. The end product of equity analysis is often a valuation and investment recommendation. Theoretical valuation models are useful in selecting ratios that would be useful in this process. For example, a company's P/B is theoretically linked to ROE, growth, and the required return. ROE is also a primary determinant of residual income in a residual income valuation model. In both cases, higher ROE relative to the required return denotes a higher valuation. Similarly, profit margin is related to justified price-to-sales (P/S) ratios. Another common valuation method involves forecasts of future cash flows that are discounted back to the present. Trends in ratios can be useful in forecasting future earnings and cash flows (e.g., trends in operating profit margin and collection of customer receivables). Future growth expectations are a key component of all of these valuation models. Trends may be useful in assessing growth prospects (when used in conjunction with overall economic and industry trends). The variability in ratios and common-size data can be useful in assessing risk, an important component of the required rate of return in valuation models. A great deal of academic research has focused on the use of these fundamental ratios in evaluating equity investments. A classic study, Ou and Penman (1989a and 1989b), found that ratios and common-size metrics generated from accounting data were useful in forecasting earnings and stock returns. Ou and Penman examined 68 such metrics and found that these could be reduced to a more parsimonious list of relevant variables, including percentage changes in a variety of measures such as current ratio, inventory, and sales; gross and pretax margins; and returns on assets and equity. These variables were found to be useful in forecasting earnings and stock returns.

Subsequent studies have also demonstrated the usefulness of ratios in evaluation of equity investments and valuation. Lev and Thiagarajan (1993) examined fundamental financial variables used by analysts to assess whether they are useful in security valuation. They found that fundamental variables add about 70 percent to the explanatory power of earnings alone in predicting excess returns (stock returns in excess of those expected). The fundamental variables they found useful included percentage changes in inventory and receivables relative to sales, gross margin, sales per employee, and the change in bad debts relative to the change in accounts receivable, among others. Abarbanell and Bushee (1997) found some of the same variables useful in predicting future accounting earnings. Abarbanell and Bushee (1998) devised an investment strategy using these same variables and found that they can generate excess returns under this strategy.

Piotroski (2000) used financial ratios to supplement a value investing strategy and found that he can generate significant excess returns. Variables used by Piotroski include ROA, cash flow ROA, change in ROA, change in leverage, change in liquidity, change in gross margin, and change in inventory turnover.

This research shows that in addition to being useful in evaluating the past performance of a company, ratios can be useful in predicting future earnings and equity returns.

\section{CREDIT ANALYSIS}
calculate and interpret ratios used in equity analysis and credit analysis

Credit risk is the risk of loss caused by a counterparty's or debtor's failure to make a promised payment. For example, credit risk with respect to a bond is the risk that the obligor (the issuer of the bond) is not able to pay interest and principal according to the terms of the bond indenture (contract). Credit analysis is the evaluation of credit risk.

Approaches to credit analysis vary and, as with all financial analysis, depend on the purpose of the analysis and the context in which it is done. Credit analysis for specific types of debt (e.g., acquisition financing and other highly leveraged financing) often involves projections of period-by-period cash flows similar to projections made by equity analysts. Whereas the equity analyst may discount projected cash flows to determine the value of the company's equity, a credit analyst would use the projected cash flows to assess the likelihood of a company complying with its financial covenants in each period and paying interest and principal as due. ${ }^{18}$ The analysis would also include expectations about asset sales and refinancing options open to the company.

Credit analysis may relate to the borrower's credit risk in a particular transaction or to its overall creditworthiness. In assessing overall creditworthiness, one general approach is credit scoring, a statistical analysis of the determinants of credit default.

18 Financial covenants are clauses in bond indentures relating to the financial condition of the bond issuer. Another general approach to credit analysis is the credit rating process that is used, for example, by credit rating agencies to assess and communicate the probability of default by an issuer on its debt obligations (e.g., commercial paper, notes, and bonds). A credit rating can be either long term or short term and is an indication of the rating agency's opinion of the creditworthiness of a debt issuer with respect to a specific debt security or other obligation. Where a company has no debt outstanding, a rating agency can also provide an issuer credit rating that expresses an opinion of the issuer's overall capacity and willingness to meet its financial obligations. The following sections review research on the use of ratios in credit analysis and the ratios commonly used in credit analysis.

\section{The Credit Rating Process}
The credit rating process involves both the analysis of a company's financial reports as well as a broad assessment of a company's operations. In assigning credit ratings, rating agencies emphasize the importance of the relationship between a company's business risk profile and its financial risk.

For corporate entities, credit ratings typically reflect a combination of qualitative and quantitative factors. Qualitative factors generally include an industry's growth prospects, volatility, technological change, and competitive environment. At the individual company level, qualitative factors may include operational effectiveness, strategy, governance, financial policies, risk management practices, and risk tolerance. In contrast, quantitative factors generally include profitability, leverage, cash flow adequacy, and liquidity. ${ }^{19}$

When analyzing financial ratios, rating agencies normally investigate deviations of ratios from the median ratios of the universe of companies for which such ratios have been calculated and also use the median ratings as an indicator for the ratings grade given to a specific debt issuer. This so-called universe of rated companies frequently changes, and any calculations are obviously affected by economic factors as well as by mergers and acquisitions. International ratings include the influence of country and economic risk factors. Exhibit 22 presents a few key financial ratios used by Standard \& Poor's in evaluating industrial companies. Note that before calculating ratios, rating agencies make certain adjustments to reported financials such as adjusting debt to include off-balance sheet debt in a company's total debt.

Exhibit 22: Selected Credit Ratios

\begin{center}
\begin{tabular}{|c|c|c|}
\hline
Credit Ratio & Numerator ${ }^{a}$ & Denominator $^{a}$ \\
\hline
EBITDA interest coverage & EBITDA $^{b}$ & $\begin{array}{l}\text { Interest expense, including non-cash } \\ \text { interest on conventional debt } \\ \text { instruments }\end{array}$ \\
\hline
$\begin{array}{l}\text { FFO }^{\mathrm{C}} \text { (Funds from operations) } \\ \text { to debt }\end{array}$ & FFO & Total debt \\
\hline
Free operating cash flow to debt & $\begin{array}{l}\mathrm{CFO}^{\mathrm{d}} \text { (adjusted) } \\ \text { minus capital } \\ \text { expenditures }\end{array}$ & Total debt \\
\hline
EBIT margin & EBIT $^{\mathrm{e}}$ & Total revenues \\
\hline
EBITDA margin & EBITDA & Total revenues \\
\hline
\end{tabular}
\end{center}

19 Concepts in this paragraph are based on Standard \& Poor's General Criteria: Principles of Credit Ratings (2011). This represents the last updated version at the time of publication.

\begin{center}
\begin{tabular}{|c|c|c|}
\hline
Credit Ratio & Numerator $^{a}$ & Denominator $^{a}$ \\
\hline
Debt to EBITDA & Total debt & EBITDA \\
\hline
Return on capital & EBIT & $\begin{array}{l}\text { Average beginning-of-year and } \\ \text { end-of-year capital }\end{array}$ \\
\hline
\end{tabular}
\end{center}

a Note that both the numerator and the denominator definitions are adjusted from ratio to ratio and may not correspond to the definitions used elsewhere in this reading.

b EBITDA = earnings before interest, taxes, depreciation, and amortization.

c $F F O=$ funds from operations, defined as EBITDA minus net interest expense minus current tax expense (plus or minus all applicable adjustments).

d $C F O=$ cash flow from operations.

e $E B I T=$ earnings before interest and taxes.

${ }^{\mathrm{f}}$ Capital = debt plus noncurrent deferred taxes plus equity (plus or minus all applicable adjustments).

Source: Based on data from Standard \& Poor's Corporate Methodology: Ratios And Adjustments (2013).

This represents the last updated version at the time of publication.

\section{Historical Research on Ratios in Credit Analysis}
A great deal of academic and practitioner research has focused on determining which ratios are useful in assessing the credit risk of a company, including the risk of bankruptcy.

One of the earliest studies examined individual ratios to assess their ability to predict failure of a company up to five years in advance. Beaver (1967) found that six ratios could correctly predict company failure one year in advance 90 percent of the time and five years in advance at least 65 percent of the time. The ratios found effective by Beaver were cash flow to total debt, ROA, total debt to total assets, working capital to total assets, the current ratio, and the no-credit interval ratio (the length of time a company could go without borrowing). Altman (1968) and Altman, Haldeman, and Narayanan (1977) found that financial ratios could be combined in an effective model for predicting bankruptcy. Altman's initial work involved creation of a $Z$-score that was able to correctly predict financial distress. The $Z$-score was computed as

$$
\begin{aligned}
Z= & 1.2 \times(\text { Current assets }- \text { Current liabilities) } / \text { Total assets } \\
& +1.4 \times(\text { Retained earnings/Total assets }) \\
& +3.3 \times(\text { EBIT } / \text { Total assets }) \\
& +0.6 \times(\text { Market value of stock/Book value of liabilities }) \\
& +1.0 \times(\text { Sales/Total assets })
\end{aligned}
$$

In his initial study, a $Z$-score of lower than 1.81 predicted failure and the model was able to accurately classify 95 percent of companies studied into a failure group and a non-failure group. The original model was designed for manufacturing companies. Subsequent refinements to the models allow for other company types and time periods. Generally, the variables found to be useful in prediction include profitability ratios, coverage ratios, liquidity ratios, capitalization ratios, and earnings variability (Altman 2000).

Similar research has been performed on the ability of ratios to predict bond ratings and bond yields. For example, Ederington, Yawtiz, and Roberts (1987) found that a small number of variables (total assets, interest coverage, leverage, variability of coverage, and subordination status) were effective in explaining bond yields. Similarly, Ederington (1986) found that nine variables in combination could correctly classify more than 70 percent of bond ratings. These variables included ROA, long-term debt to assets, interest coverage, cash flow to debt, variability of coverage and cash flow, total assets, and subordination status. These studies have shown that ratios are effective in evaluating credit risk, bond yields, and bond ratings.

\section{BUSINESS AND GEOGRAPHIC SEGMENTS}
explain the requirements for segment reporting and calculate and interpret segment ratios

Analysts often need to evaluate the performance underlying business segments (subsidiary companies, operating units, or simply operations in different geographic areas) to understand in detail the company as a whole. Although companies are not required to provide full financial statements for segments, they are required to provide segment information under both IFRS and US GAAP. ${ }^{20}$

\section{Segment Reporting Requirements}
An operating segment is defined as a component of a company: a) that engages in activities that may generate revenue and create expenses, including a start-up segment that has yet to earn revenues, b) whose results are regularly reviewed by the company's senior management, and c) for which discrete financial information is available. ${ }^{21} \mathrm{~A}$ company must disclose separate information about any operating segment which meets certain quantitative criteria-namely, the segment constitutes 10 percent or more of the combined operating segments' revenue, assets, or profit. (For purposes of determining whether a segment constitutes 10 percent or more of combined profits or losses, the criteria is expressed in terms of the absolute value of the segment's profit or loss as a percentage of the greater of (i) the combined profits of all profitable segments and (ii) the absolute amount of the combined losses of all loss-making segments.) If, after applying these quantitative criteria, the combined revenue from external customers for all reportable segments combined is less than 75 percent of the total company revenue, the company must identify additional reportable segments until the 75 percent level is reached. Small segments might be combined as one if they share a substantial number of factors that define a business or geographical segment, or they might be combined with a similar significant reportable segment. Information about operating segments and businesses that are not reportable is combined in an "all other segments" category.

Companies may internally report business results in a variety of ways (e.g., product segments and geographical segments). Companies identify the segments for external reporting purposes considering the definition of an operating segment and using factors such as what information is reported to the board of directors and whether a manager is responsible for each segment. Companies must disclose the factors used to identify reportable segments and the types of products and services sold by each reportable segment

For each reportable segment, the following should also be disclosed:

\begin{itemize}
  \item a measure of profit or loss;

  \item a measure of total assets and liabilities ${ }^{22}$ (if these amounts are regularly reviewed by the company's chief decision-making officer);

  \item segment revenue, distinguishing between revenue to external customers and revenue from other segments;

\end{itemize}

20 IFRS 8, Operating Segments and FASB ASC Topic 280 [Segment Reporting].

21 IFRS 8, Operating Segments, paragraph 5.

22 IFRS 8 and FASB ASC Topic 280 are largely converged. One notable difference is that US GAAP does not require disclosure of segment liabilities, while IFRS requires disclosure of segment liabilities if that information is regularly provided to the company's "chief operating decision maker." - interest revenue and interest expense;

\begin{itemize}
  \item cost of property, plant, and equipment, and intangible assets acquired;

  \item depreciation and amortisation expense;

  \item other non-cash expenses;

  \item income tax expense or income; and

  \item share of the net profit or loss of an investment accounted for under the equity method.

\end{itemize}

Companies also must provide a reconciliation between the information of reportable segments and the consolidated financial statements in terms of segment revenue, profit or loss, assets, and liabilities.

Another disclosure required is the company's reliance on any single customer. If any single customer represents 10 percent or more of the company's total revenues, the company must disclose that fact. From an analysts' perspective, information about a concentrated customer base can be useful in assessing the risks faced by the company.

\section{Segment Ratios}
Based on the segment information that companies are required to present, a variety of useful ratios can be computed, as shown in Exhibit 23.

\section{Exhibit 23: Definitions of Segment Ratios}
\begin{center}
\begin{tabular}{lll}
\hline
Segment Ratios & Numerator & Denominator \\
\hline
Segment margin & Segment profit (loss) & Segment revenue \\
Segment turnover & Segment revenue & Segment assets \\
Segment ROA & Segment profit (loss) & Segment assets \\
Segment debt ratio & Segment liabilities & Segment assets \\
\hline
\end{tabular}
\end{center}

The segment margin measures the operating profitability of the segment relative to revenues, whereas the segment ROA measures the operating profitability relative to assets. Segment turnover measures the overall efficiency of the segment: how much revenue is generated per unit of assets. The segment debt ratio examines the level of liabilities (hence solvency) of the segment. Example 17 demonstrates the evaluation of segment ratios.

\section{EXAMPLE 17}
\section{The Evaluation of Segment Ratios}
\begin{enumerate}
  \item The information contained in Exhibit 24 relates to the business segments of Groupe Danone for 2016 and 2017 in millions of euro. According to the company's 2017 annual report the company operates in four business segments which are primarily evaluated on operating income and operating margin and in two geographic segments for which they also provide information on assets deployed.
\end{enumerate}

Evaluate the performance of the segments using the relative proportion of sales of each segment, the segment margins, segment ROA where available, and segment turnover where available.

\section{Exhibit 24: Group Danone Segment Disclosures (in $€$ millions)}
\begin{center}
\begin{tabular}{|c|c|c|c|c|}
\hline
\multirow[b]{2}{*}{Business Segments} & \multicolumn{2}{|c|}{2016} & \multicolumn{2}{|c|}{2017} \\
\hline
 & Sales & $\begin{array}{c}\text { Recurring } \\ \text { Operating Income }\end{array}$ & Sales & $\begin{array}{c}\text { Recurring Operating } \\ \text { Income }\end{array}$ \\
\hline
Fresh Dairy Products - International & 8,229 & 731 & 8,424 & 760 \\
\hline
Fresh Dairy Products - North America & 2,506 & 351 & 4,530 & 556 \\
\hline
Specialized Nutrition & 6,634 & 1,419 & 7,102 & 1,685 \\
\hline
Waters & 4,574 & 521 & 4,621 & 541 \\
\hline
Group Total & 21,944 & 3,022 & 24,677 & 3,542 \\
\hline
\end{tabular}
\end{center}

\begin{center}
\begin{tabular}{|c|c|c|c|c|c|c|}
\hline
 &  & 2016 &  &  & 2017 &  \\
\hline
Geographic Segments & Sales & $\begin{array}{c}\text { Recurring } \\ \text { Operating } \\ \text { Income }\end{array}$ & $\begin{array}{c}\text { Non-Current } \\ \text { Assets }\end{array}$ & Sales & $\begin{array}{c}\text { Recurring } \\ \text { Operating } \\ \text { Income }\end{array}$ & $\begin{array}{c}\text { Non-Current } \\ \text { Assets }\end{array}$ \\
\hline
$\begin{array}{l}\text { Europe and North } \\ \text { America }\end{array}$ & 10,933 & 1,842 & 11,532 & 13,193 & 2,048 & 22,517 \\
\hline
Rest of World & 11,011 & 1,180 & 9,307 & 11,484 & 1,495 & 8,433 \\
\hline
Group Total & 21,944 & 3,022 & 20,839 & 24,677 & 3,543 & 30,950 \\
\hline
\end{tabular}
\end{center}

Source: Company's 2017 Annual Report.

Solution:

\begin{center}
\begin{tabular}{|c|c|c|c|c|}
\hline
Business Segments & $\begin{array}{c}\text { Segment } \\ \text { Revenue } \\ \text { Percent }\end{array}$ & $\begin{array}{c}\text { Recurring } \\ \text { Operating Margin }\end{array}$ & $\begin{array}{c}\text { Segment } \\ \text { Revenue } \\ \text { Percent }\end{array}$ & $\begin{array}{c}\text { Recurring } \\ \text { Operating Margir }\end{array}$ \\
\hline
Fresh Dairy Products - International & $37.5 \%$ & $8.9 \%$ & $34.1 \%$ & $9.0 \%$ \\
\hline
Fresh Dairy Products - North America & $11.4 \%$ & $14.0 \%$ & $18.4 \%$ & $12.3 \%$ \\
\hline
Specialized Nutrition & $30.2 \%$ & $21.4 \%$ & $28.8 \%$ & $23.7 \%$ \\
\hline
Waters & $20.8 \%$ & $11.4 \%$ & $18.7 \%$ & $11.7 \%$ \\
\hline
Group Total & $100.0 \%$ & $13.8 \%$ & $100.0 \%$ & $14.4 \%$ \\
\hline
\end{tabular}
\end{center}

Business Segments

$2017 \%$ change in revenue

\begin{center}
\begin{tabular}{lc}
Fresh Dairy Products & $2.4 \%$ \\
- International &  \\
Fresh Dairy Products - North & $80.8 \%$ \\
America &  \\
Specialized Nutrition & $7.1 \%$ \\
Waters & $1.0 \%$ \\
Group Total & $12.5 \%$ \\
\hline
\end{tabular}
\end{center}

The business segment analysis shows that the largest proportion of the company's revenues occurs in the Fresh Dairy Products - International segment: $37.5 \%$ and $34.1 \%$ of the total in 2016 and 2017, respectively. The greatest increase in relative revenue, however, came from the Fresh Dairy Products - North America segment which grew by $80.8 \%$ and increased from $11.4 \%$ of total revenues in 2016 to $18.4 \%$ of total revenues in 2017. Examination of the company's full annual report reveals that Danone Group acquired a large health-oriented North American food company, Whitewave, in 2017. This caused the shift in the relative proportion of sales. The highest segment operating margin in both years comes from the Specialized Nutrition segment with operating margins of 21.4\% in 2016 increasing to $23.7 \%$ in 2017. Margins increased slightly in the Fresh Dairy Products - International and Waters segments, while margins declined in Fresh Dairy Products - North America. The latter is likely due to costs associated with the Whitewave acquisition.

2016

2017

\begin{center}
\begin{tabular}{|c|c|c|c|c|c|c|c|c|}
\hline
$\begin{array}{l}\text { Geographic } \\ \text { Segments }\end{array}$ & $\begin{array}{c}\text { Segment } \\ \text { Revenue } \\ \text { Percent }\end{array}$ & $\begin{array}{c}\text { Recurring } \\ \text { Operating } \\ \text { Margin }\end{array}$ & $\begin{array}{c}\text { Segment } \\ \text { ROA }\end{array}$ & $\begin{array}{c}\text { Segment } \\ \text { Asset } \\ \text { Turnover }\end{array}$ & $\begin{array}{c}\text { Segment } \\ \text { Revenue } \\ \text { Percent }\end{array}$ & $\begin{array}{c}\text { Recurring } \\ \text { Operating } \\ \text { Margin }\end{array}$ & $\begin{array}{c}\text { Segment } \\ \text { ROA }\end{array}$ & $\begin{array}{c}\text { Segment } \\ \text { Asset } \\ \text { Turnovel }\end{array}$ \\
\hline
$\begin{array}{l}\text { Europe } \\ \text { and North } \\ \text { America }\end{array}$ & $49.8 \%$ & $16.8 \%$ & $16.0 \%$ & 0.9 & $53.5 \%$ & $15.5 \%$ & $9.1 \%$ & 0.6 \\
\hline
Rest of World & $50.2 \%$ & $10.7 \%$ & $12.7 \%$ & 1.2 & $46.5 \%$ & $13.0 \%$ & $17.7 \%$ & 1.4 \\
\hline
Groun Total & $100.0 \%$ & $13.8 \%$ & $14.5 \%$ & 1.1 & $100.0 \%$ & $14.4 \%$ & $11.4 \%$ & 0.8 \\
\hline
\end{tabular}
\end{center}

As used in this table, ROA refers to operating income divided by ending assets, and Asset Turnover is defined as Revenue divided by non-current assets.

The geographic segment analysis shows that the company's sales are split roughly evenly between the two geographic segments. Operating margins were higher in the Europe and North America segment in both years but declined from $16.8 \%$ in 2016 to $15.5 \%$ in 2017, likely in connection with the North American acquisition of Whitewave. Operating margins in the rest of the world, however, increased in 2017. Segment return on assets and segment asset turnover declined significantly for the Europe and North America segment in 2017, again largely due to the acquisition of Whitewave. An examination of the annual report disclosures reveals that the large increase in segment assets came from intangible assets (mainly goodwill) recorded in the Whitewave acquisition. In contrast, segment return on assets and turnover improved significantly in the Rest of World segment.

\section{MODEL BUILDING AND FORECASTING}
describe how ratio analysis and other techniques can be used to model and forecast earnings

Analysts often need to forecast future financial performance. For example, analysts' EPS forecasts and related equity valuations are widely followed by Wall Street. Analysts use data about the economy, industry, and company in arriving at a company's forecast. The results of an analyst's financial analysis, including common-size and ratio analyses, are integral to this process, along with the judgment of the analysts. Based upon forecasts of growth and expected relationships among the financial statement data, the analyst can build a model (sometimes referred to as an "earnings model") to forecast future performance. In addition to budgets, pro forma financial statements are widely used in financial forecasting within companies, especially for use by senior executives and boards of directors. Last but not least, these budgets and forecasts are also used in presentations to credit analysts and others in obtaining external financing.

For example, based on a revenue forecast, an analyst may budget expenses based on expected common-size data. Forecasts of balance sheet and cash flow statements can be derived from expected ratio data, such as DSO. Forecasts are not limited to a single point estimate but should involve a range of possibilities. This can involve several techniques:

\begin{itemize}
  \item Sensitivity analysis: Also known as "what if" analysis, sensitivity analysis shows the range of possible outcomes as specific assumptions are changed; this could, in turn, influence financing needs or investment in fixed assets.

  \item Scenario analysis: This type of analysis shows the changes in key financial quantities that result from given (economic) events, such as the loss of customers, the loss of a supply source, or a catastrophic event. If the list of events is mutually exclusive and exhaustive and the events can be assigned probabilities, the analyst can evaluate not only the range of outcomes but also standard statistical measures such as the mean and median value for various quantities of interest.

  \item Simulation: This is computer-generated sensitivity or scenario analysis based on probability models for the factors that drive outcomes. Each event or possible outcome is assigned a probability. Multiple scenarios are then run using the probability factors assigned to the possible values of a variable.

\end{itemize}

\section{SUMMARY}
Financial analysis techniques, including common-size financial statements and ratio analysis, are useful in summarizing financial reporting data and evaluating the performance and financial position of a company. The results of financial analysis techniques provide important inputs into security valuation. Key facets of financial analysis include the following:

\begin{itemize}
  \item Common-size financial statements and financial ratios remove the effect of size, allowing comparisons of a company with peer companies (cross-sectional analysis) and comparison of a company's results over time (trend or time-series analysis).

  \item Activity ratios measure the efficiency of a company's operations, such as collection of receivables or management of inventory. Major activity ratios include inventory turnover, days of inventory on hand, receivables turnover, days of sales outstanding, payables turnover, number of days of payables, working capital turnover, fixed asset turnover, and total asset turnover.

  \item Liquidity ratios measure the ability of a company to meet short-term obligations. Major liquidity ratios include the current ratio, quick ratio, cash ratio, and defensive interval ratio. - Solvency ratios measure the ability of a company to meet long-term obligations. Major solvency ratios include debt ratios (including the debt-to-assets ratio, debt-to-capital ratio, debt-to-equity ratio, and financial leverage ratio) and coverage ratios (including interest coverage and fixed charge coverage).

  \item Profitability ratios measure the ability of a company to generate profits from revenue and assets. Major profitability ratios include return on sales ratios (including gross profit margin, operating profit margin, pretax margin, and net profit margin) and return on investment ratios (including operating ROA, ROA, return on total capital, ROE, and return on common equity).

  \item Ratios can also be combined and evaluated as a group to better understand how they fit together and how efficiency and leverage are tied to profitability.

  \item ROE can be analyzed as the product of the net profit margin, asset turnover, and financial leverage. This decomposition is sometimes referred to as DuPont analysis.

  \item Valuation ratios express the relation between the market value of a company or its equity (for example, price per share) and some fundamental financial metric (for example, earnings per share).

  \item Ratio analysis is useful in the selection and valuation of debt and equity securities and is a part of the credit rating process.

  \item Ratios can also be computed for business segments to evaluate how units within a business are performing.

  \item The results of financial analysis provide valuable inputs into forecasts of future earnings and cash flow.

\end{itemize}

\section{REFERENCES}
Abarbanell, J.S., B. J. Bushee. 1997. "Fundamental Analysis, Future Earnings, and Stock Prices." Journal of Accounting Research, vol. 35, no. 1:1-24. 10.2307/2491464

Abarbanell, J.S., B.J. Bushee. 1998. "Abnormal Returns to a Fundamental Analysis Strategy." Accounting Review, vol. 73, no. 1:19-46.

Altman, E. 1968. "Financial Ratios, Discriminant Analysis and the Prediction of Corporate Bankruptcy." Journal of Finance, vol. 23, no. 4:589-609. 10.2307/2978933

Altman, E., R. Haldeman, P. Narayanan. 1977. "Zeta Analysis: A New Model to Identify Bankruptcy Risk of Corporations." Journal of Banking \& Finance, vol. 1, no. 1. 10.1016/0378-4266(77)90017-6

Beaver, W. 1967. "Financial Ratios as Predictors of Failures." Empirical Research in Accounting, selected studies supplement to Journal of Accounting Research, 4 (1).

Benninga, Simon Z., Oded H. Sarig. 1997. Corporate Finance: A Valuation Approach. New York: McGraw-Hill Publishing.

Ederington, L.H. 1986. "Why Split Ratings Occur." Financial Management, vol. 15, no. 1:37-47. $10.2307 / 3665276$

Lev, B., S.R. Thiagarajan. 1993. "Fundamental Information Analysis." Journal of Accounting Research, vol. 31, no. 2:190-215. 10.2307/2491270

Modigliani, F., M. Miller. 1958. "The Cost of Capital, Corporation Finance and the Theory of Investment." American Economic Review, vol. 48:261-298.

Modigliani, F., M. Miller. 1963. "Corporate Income Taxes and the Cost of Capital: A Correction." American Economic Review, vol. 53:433-444.

Ou, J.A., S.H. Penman. 1989a. "Financial Statement Analysis and the Prediction of Stock Returns." Journal of Accounting and Economics, vol. 11, no. 4:295-329. 10.1016/0165-4101(89)90017-7

Ou, J.A., S.H. Penman. 1989b. "Accounting Measurement, Price-Earnings Ratio, and the Information Content of Security Prices." Journal of Accounting Research, vol. 27, no. Supplement:111-144. 10.2307/2491068

Piotroski, J.D. 2000. "Value Investing: The Use of Historical Financial Statement Information to Separate Winners from Losers." Journal of Accounting Research, vol. 38, no. Supplement:1-41. $10.2307 / 2672906$

Robinson, T., P. Munter. 2004. "Financial Reporting Quality: Red Flags and Accounting Warning Signs." Commercial Lending Review, vol. 19, no. 1:2-15.

van Greuning, H., S. Brajovic Bratanovic. 2003. Analyzing and Managing Banking Risk: A Framework for Assessing Corporate Governance and Financial Risk. Washington, DC: World Bank.

\section{PRACTICE PROBLEMS}
\begin{enumerate}
  \item Comparison of a company's financial results to other peer companies for the same time period is called:
A. technical analysis.
B. time-series analysis.
C. cross-sectional analysis.

  \item An analyst observes a decrease in a company's inventory turnover. Which of the following would most likely explain this trend?

\end{enumerate}

A. The company installed a new inventory management system, allowing more efficient inventory management.

B. Due to problems with obsolescent inventory last year, the company wrote off a large amount of its inventory at the beginning of the period.

C. The company installed a new inventory management system but experienced some operational difficulties resulting in duplicate orders being placed with suppliers.

\begin{enumerate}
  \setcounter{enumi}{2}
  \item Which of the following would best explain an increase in receivables turnover?
\end{enumerate}

A. The company adopted new credit policies last year and began offering credit to customers with weak credit histories.

B. Due to problems with an error in its old credit scoring system, the company had accumulated a substantial amount of uncollectible accounts and wrote off a large amount of its receivables.

C. To match the terms offered by its closest competitor, the company adopted new payment terms now requiring net payment within 30 days rather than 15 days, which had been its previous requirement.

\begin{enumerate}
  \setcounter{enumi}{3}
  \item Brown Corporation had average days of sales outstanding of 19 days in the most recent fiscal year. Brown wants to improve its credit policies and collection practices and decrease its collection period in the next fiscal year to match the industry average of 15 days. Credit sales in the most recent fiscal year were $\$ 300$ million, and Brown expects credit sales to increase to $\$ 390$ million in the next fiscal year. To achieve Brown's goal of decreasing the collection period, the change in the average accounts receivable balance that must occur is closest to:
A. $+\$ 0.41$ million.
B. $-\$ 0.41$ million.
C. $-\$ 1.22$ million.

  \item An analyst is interested in assessing both the efficiency and liquidity of Spherion PLC. The analyst has collected the following data for Spherion:

\end{enumerate}

\begin{center}
\begin{tabular}{lccc}
\hline
 & FY3 & FY2 & FY1 \\
\hline
Days of inventory on hand & 32 & 34 & 40 \\
\hline
\end{tabular}
\end{center}

\begin{center}
\begin{tabular}{lccc}
\hline
 & FY3 & FY2 & FY1 \\
\hline
Days sales outstanding & 28 & 25 & 23 \\
Number of days of payables & 40 & 35 & 35 \\
\hline
\end{tabular}
\end{center}

Based on this data, what is the analyst least likely to conclude?

A. Inventory management has contributed to improved liquidity.

B. Management of payables has contributed to improved liquidity.

C. Management of receivables has contributed to improved liquidity.

\begin{enumerate}
  \setcounter{enumi}{5}
  \item In order to assess a company's ability to fulfill its long-term obligations, an analyst would most likely examine:
A. activity ratios.
B. liquidity ratios.
C. solvency ratios.

  \item An analyst is evaluating the solvency and liquidity of Apex Manufacturing and has collected the following data (in millions of euro):

\end{enumerate}

\begin{center}
\begin{tabular}{|c|c|c|c|}
\hline
 & FY5 $(€)$ & FY4 (€) & FY3 (€) \\
\hline
Total debt & 2,000 & 1,900 & 1,750 \\
\hline
Total equity & 4,000 & 4,500 & 5,000 \\
\hline
\end{tabular}
\end{center}

Which of the following would be the analyst's most likely conclusion?

A. The company is becoming increasingly less solvent, as evidenced by the increase in its debt-to-equity ratio from 0.35 to 0.50 from FY3 to FY5.

B. The company is becoming less liquid, as evidenced by the increase in its debt-to-equity ratio from 0.35 to 0.50 from FY3 to FY5.

C. The company is becoming increasingly more liquid, as evidenced by the increase in its debt-to-equity ratio from 0.35 to 0.50 from FY3 to FY5.

\begin{enumerate}
  \setcounter{enumi}{7}
  \item With regard to the data in Problem 6, what would be the most reasonable explanation of the financial data?
\end{enumerate}

A. The decline in the company's equity results from a decline in the market value of this company's common shares.

B. The $€ 250$ increase in the company's debt from FY3 to FY5 indicates that lenders are viewing the company as increasingly creditworthy.

C. The decline in the company's equity indicates that the company may be incurring losses, paying dividends greater than income, and/or repurchasing shares.

\begin{enumerate}
  \setcounter{enumi}{8}
  \item An analyst observes the following data for two companies:
\end{enumerate}

\begin{center}
\begin{tabular}{lcc}
\hline
 & Company A (\$) & Company B (\$) \\
\hline
Revenue & 4,500 & 6,000 \\
\hline
\end{tabular}
\end{center}

\begin{center}
\begin{tabular}{lcc}
\hline
 & Company A (\$) & Company B (\$) \\
\hline
Net income & 50 & 1,000 \\
Current assets & 40,000 & 60,000 \\
Total assets & 100,000 & 700,000 \\
Current liabilities & 10,000 & 50,000 \\
Total debt & 60,000 & 150,000 \\
Shareholders' equity & 30,000 & 500,000 \\
\hline
\end{tabular}
\end{center}

Which of the following choices best describes reasonable conclusions that the analyst might make about the two companies' ability to pay their current and long-term obligations?

A. Company A's current ratio of 4.0 indicates it is more liquid than Company B, whose current ratio is only 1.2 , but Company B is more solvent, as indicated by its lower debt-to-equity ratio.

B. Company A's current ratio of 0.25 indicates it is less liquid than Company B, whose current ratio is 0.83 , and Company $\mathrm{A}$ is also less solvent, as indicated by a debt-to-equity ratio of 200 percent compared with Company B's debt-to-equity ratio of only 30 percent.

C. Company A's current ratio of 4.0 indicates it is more liquid than Company $\mathrm{B}$, whose current ratio is only 1.2 , and Company A is also more solvent, as indicated by a debt-to-equity ratio of 200 percent compared with Company B's debt-to-equity ratio of only 30 percent.

\begin{enumerate}
  \setcounter{enumi}{9}
  \item Which ratio would a company most likely use to measure its ability to meet short-term obligations?
A. Current ratio.
B. Payables turnover.
C. Gross profit margin.

  \item Which of the following ratios would be most useful in determining a company's ability to cover its lease and interest payments?
A. ROA.
B. Total asset turnover.
C. Fixed charge coverage.

  \item Assuming no changes in other variables, which of the following would decrease ROA?

\end{enumerate}

A. A decrease in the effective tax rate.

B. A decrease in interest expense.

C. An increase in average assets.

\section{The following information relates to questions}
13-16

The data in Exhibit 1 appear in the five-year summary of a major international company. A business combination with another major manufacturer took place in FY13.

\section{Exhibit 1}
\begin{center}
\begin{tabular}{lccccc}
\hline
 & FY10 & FY11 & FY12 & FY13 & FY14 \\
\hline
Financial statements & GBP $m$ & GBP m & GBP m & GBP m & GBP m \\
Income statements &  &  &  &  &  \\
Revenue & 4,390 & 3,624 & 3,717 & 8,167 & 11,366 \\
Profit before interest and taxation (EBIT) & 844 & 700 & 704 & 933 & 1,579 \\
Net interest payable & -80 & -54 & -98 & -163 & -188 \\
Taxation & -186 & -195 & -208 & -349 & -579 \\
Minorities & -94 & -99 & -105 & -125 & -167 \\
Profit for the year & 484 & 352 & 293 & 296 & 645 \\
Balance sheets &  &  &  &  &  \\
Fixed assets & 3,510 & 3,667 & 4,758 & 10,431 & 11,483 \\
Current asset investments, cash at bank & 316 & 218 & 290 & 561 & 682 \\
and in hand &  &  &  &  &  \\
Other current assets & 558 & 514 & 643 & 1,258 & 1,634 \\
Total assets & 4,384 & 4,399 & 5,691 & 12,250 & 13,799 \\
Interest bearing debt (long term) & -602 & $-1,053$ & $-1,535$ & $-3,523$ & $-3,707$ \\
Other creditors and provisions (current) & $-1,223$ & $-1,054$ & $-1,102$ & $-2,377$ & $-3,108$ \\
Total liabilities & $-1,825$ & $-2,107$ & $-2,637$ & $-5,900$ & $-6,815$ \\
Net assets & 2,559 & 2,292 & 3,054 & 6,350 & 6,984 \\
Shareholders' funds & 2,161 & 2,006 & 2,309 & 5,572 & 6,165 \\
Equity minority interests & 398 & 286 & 745 & 778 & 819 \\
Capital employed & 2,559 & 2,292 & 3,054 & 6,350 & 6,984 \\
Cash flow &  &  &  &  &  \\
Working capital movements & -53 & 5 & 71 & 85 & 107 \\
Net cash inflow from operating activities & 864 & 859 & 975 & 1,568 & 2,292 \\
\hline
\end{tabular}
\end{center}

\begin{enumerate}
  \setcounter{enumi}{12}
  \item The company's total assets at year-end FY9 were GBP 3,500 million. Which of the following choices best describes reasonable conclusions an analyst might make about the company's efficiency?
\end{enumerate}

A. Comparing FY14 with FY10, the company's efficiency improved, as indicated by a total asset turnover ratio of 0.86 compared with 0.64 .

B. Comparing FY14 with FY10, the company's efficiency deteriorated, as indicated by its current ratio.

C. Comparing FY14 with FY10, the company's efficiency deteriorated due to asset growth faster than turnover revenue growth. 14. Which of the following choices best describes reasonable conclusions an analyst might make about the company's solvency?

A. Comparing FY14 with FY10, the company's solvency improved, as indicated by an increase in its debt-to-assets ratio from 0.14 to 0.27 .

B. Comparing FY14 with FY10, the company's solvency deteriorated, as indicated by a decrease in interest coverage from 10.6 to 8.4.

C. Comparing FY14 with FY10, the company's solvency improved, as indicated by the growth in its profits to GBP 645 million.

\begin{enumerate}
  \setcounter{enumi}{14}
  \item Which of the following choices best describes reasonable conclusions an analyst might make about the company's liquidity?
\end{enumerate}

A. Comparing FY14 with FY10, the company's liquidity improved, as indicated by an increase in its debt-to-assets ratio from 0.14 to 0.27 .

B. Comparing FY14 with FY10, the company's liquidity deteriorated, as indicated by a decrease in interest coverage from 10.6 to 8.4.

C. Comparing FY14 with FY10, the company's liquidity improved, as indicated by an increase in its current ratio from 0.71 to 0.75 .

\begin{enumerate}
  \setcounter{enumi}{15}
  \item Which of the following choices best describes reasonable conclusions an analyst might make about the company's profitability?
\end{enumerate}

A. Comparing FY14 with FY10, the company's profitability improved, as indicated by an increase in its debt-to-assets ratio from 0.14 to 0.27 .

B. Comparing FY14 with FY10, the company's profitability deteriorated, as indicated by a decrease in its net profit margin from 11.0 percent to 5.7 percent.

C. Comparing FY14 with FY10, the company's profitability improved, as indicated by the growth in its shareholders' equity to GBP 6,165 million.

\begin{enumerate}
  \setcounter{enumi}{16}
  \item An analyst compiles the following data for a company:
\end{enumerate}

\begin{center}
\begin{tabular}{lccc}
\hline
 & FY13 & FY14 & FY15 \\
\hline
ROE & $19.8 \%$ & $20.0 \%$ & $22.0 \%$ \\
Return on total assets & $8.1 \%$ & $8.0 \%$ & $7.9 \%$ \\
Total asset turnover & 2.0 & 2.0 & 2.1 \\
\hline
\end{tabular}
\end{center}

Based only on the information above, the most appropriate conclusion is that, over the period FY13 to FY15, the company's:

A. net profit margin and financial leverage have decreased.

B. net profit margin and financial leverage have increased.

C. net profit margin has decreased but its financial leverage has increased.

\begin{enumerate}
  \setcounter{enumi}{17}
  \item A decomposition of ROE for Integra SA is as follows:
\end{enumerate}

\begin{center}
\begin{tabular}{lcc}
\hline
 & FY12 & FY11 \\
\hline
ROE & $18.90 \%$ & $18.90 \%$ \\
\hline
\end{tabular}
\end{center}

\begin{center}
\begin{tabular}{lcc}
\hline
 & FY12 & FY11 \\
\hline
Tax burden & 0.70 & 0.75 \\
Interest burden & 0.90 & 0.90 \\
EBIT margin & $10.00 \%$ & $10.00 \%$ \\
Asset turnover & 1.50 & 1.40 \\
Leverage & 2.00 & 2.00 \\
\hline
\end{tabular}
\end{center}

Which of the following choices best describes reasonable conclusions an analyst might make based on this ROE decomposition?

A. Profitability and the liquidity position both improved in FY12.

B. The higher average tax rate in FY12 offset the improvement in profitability, leaving ROE unchanged.

C. The higher average tax rate in FY12 offset the improvement in efficiency, leaving ROE unchanged.

\begin{enumerate}
  \setcounter{enumi}{18}
  \item A decomposition of ROE for Company A and Company B is as follows:
\end{enumerate}

\begin{center}
\begin{tabular}{lcccc}
\hline
 & \multicolumn{2}{c}{Company A} & \multicolumn{2}{c}{Company B} \\
\hline
 & FY15 & FY14 & FY15 & FY14 \\
\hline
ROE & $26.46 \%$ & $18.90 \%$ & $26.33 \%$ & $18.90 \%$ \\
Tax burden & 0.7 & 0.75 & 0.75 & 0.75 \\
Interest burden & 0.9 & 0.9 & 0.9 & 0.9 \\
EBIT margin & $7.00 \%$ & $10.00 \%$ & $13.00 \%$ & $10.00 \%$ \\
Asset turnover & 1.5 & 1.4 & 1.5 & 1.4 \\
Leverage & 4 & 2 & 2 & 2 \\
\hline
\end{tabular}
\end{center}

An analyst is most likely to conclude that:

A. Company A's ROE is higher than Company B's in FY15, and one explanation consistent with the data is that Company A may have purchased new, more efficient equipment.

B. Company A's ROE is higher than Company B's in FY15, and one explanation consistent with the data is that Company A has made a strategic shift to a product mix with higher profit margins.

C. The difference between the two companies' ROE in FY15 is very small and Company A's ROE remains similar to Company B's ROE mainly due to Company A increasing its financial leverage.

\begin{enumerate}
  \setcounter{enumi}{19}
  \item What does the $\mathrm{P} / \mathrm{E}$ ratio measure?
\end{enumerate}

A. The "multiple" that the stock market places on a company's EPS.

B. The relationship between dividends and market prices.

C. The earnings for one common share of stock.

\begin{enumerate}
  \setcounter{enumi}{20}
  \item A creditor most likely would consider a decrease in which of the following ratios to be positive news?
A. Interest coverage (times interest earned).
B. Debt-to-total assets.
C. Return on assets.

  \item When developing forecasts, analysts should most likely:

\end{enumerate}

A. develop possibilities relying exclusively on the results of financial analysis.

B. use the results of financial analysis, analysis of other information, and judgment.

C. aim to develop extremely precise forecasts using the results of financial analysis.

\section{SOLUTIONS}
\begin{enumerate}
  \item C is correct. Cross-sectional analysis involves the comparison of companies with each other for the same time period. Technical analysis uses price and volume data as the basis for investment decisions. Time-series or trend analysis is the comparison of financial data across different time periods.

  \item C is correct. The company's problems with its inventory management system causing duplicate orders would likely result in a higher amount of inventory and would, therefore, result in a decrease in inventory turnover. A more efficient inventory management system and a write off of inventory at the beginning of the period would both likely decrease the average inventory for the period (the denominator of the inventory turnover ratio), thus increasing the ratio rather than decreasing it.

  \item B is correct. A write off of receivables would decrease the average amount of accounts receivable (the denominator of the receivables turnover ratio), thus increasing this ratio. Customers with weaker credit are more likely to make payments more slowly or to pose collection difficulties, which would likely increase the average amount of accounts receivable and thus decrease receivables turnover. Longer payment terms would likely increase the average amount of accounts receivable and thus decrease receivables turnover.

  \item A is correct. The average accounts receivable balances (actual and desired) must be calculated to determine the desired change. The average accounts receivable balance can be calculated as an average day's credit sales times the DSO. For the most recent fiscal year, the average accounts receivable balance is $\$ 15.62$ million $[=(\$ 300,000,000 / 365) \times 19]$. The desired average accounts receivable balance for the next fiscal year is $\$ 16.03$ million $(=(\$ 390,000,000 / 365) \times 15)$. This is an increase of $\$ 0.41$ million (=16.03 million - 15.62 million). An alternative approach is to calculate the turnover and divide sales by turnover to determine the average accounts receivable balance. Turnover equals 365 divided by DSO. Turnover is $19.21(=365 / 19)$ for the most recent fiscal year and is targeted to be 24.33 (= 365/15) for the next fiscal year. The average accounts receivable balances are $\$ 15.62$ million (= $\$ 300,000,000 / 19.21)$, and $\$ 16.03$ million (= $\$ 390,000,000 / 24.33)$. The change is an increase in receivables of $\$ 0.41$ million

  \item $\mathrm{C}$ is correct. The analyst is unlikely to reach the conclusion given in Statement C because days of sales outstanding increased from 23 days in FY1 to 25 days in FY2 to 28 days in FY3, indicating that the time required to collect receivables has increased over the period. This is a negative factor for Spherion's liquidity. By contrast, days of inventory on hand dropped over the period FY1 to FY3, a positive for liquidity. The company's increase in days payable, from 35 days to 40 days, shortened its cash conversion cycle, thus also contributing to improved liquidity.

  \item C is correct. Solvency ratios are used to evaluate the ability of a company to meet its long-term obligations. An analyst is more likely to use activity ratios to evaluate how efficiently a company uses its assets. An analyst is more likely to use liquidity ratios to evaluate the ability of a company to meet its short-term obligations.

  \item A is correct. The company is becoming increasingly less solvent, as evidenced by its debt-to-equity ratio increasing from 0.35 to 0.50 from FY3 to FY5. The amount of a company's debt and equity do not provide direct information about the company's liquidity position.

\end{enumerate}

Debt to equity:

FY5: $2,000 / 4,000=0.5000$

FY4: $1,900 / 4,500=0.4222$

FY3: $1,750 / 5,000=0.3500$

\begin{enumerate}
  \setcounter{enumi}{7}
  \item C is correct. The decline in the company's equity indicates that the company may be incurring losses, paying dividends greater than income, or repurchasing shares. Recall that Beginning equity + New shares issuance - Shares repurchased + Comprehensive income - Dividends = Ending equity. The book value of a company's equity is not affected by changes in the market value of its common stock. An increased amount of lending does not necessarily indicate that lenders view a company as increasingly creditworthy. Creditworthiness is not evaluated based on how much a company has increased its debt but rather on its willingness and ability to pay its obligations. (Its financial strength is indicated by its solvency, liquidity, profitability, efficiency, and other aspects of credit analysis.)

  \item A is correct. Company A's current ratio of $4.0(=\$ 40,000 / \$ 10,000)$ indicates it is more liquid than Company B, whose current ratio is only $1.2(=\$ 60,000 / \$ 50,000)$. Company $\mathrm{B}$ is more solvent, as indicated by its lower debt-to-equity ratio of 30 percent $(=\$ 150,000 / \$ 500,000)$ compared with Company A's debt-to-equity ratio of 200 percent $(=\$ 60,000 / \$ 30,000)$.

  \item A is correct. The current ratio is a liquidity ratio. It compares the net amount of current assets expected to be converted into cash within the year with liabilities falling due in the same period. A current ratio of 1.0 would indicate that the company would have just enough current assets to pay current liabilities.

  \item $\mathrm{C}$ is correct. The fixed charge coverage ratio is a coverage ratio that relates known fixed charges or obligations to a measure of operating profit or cash flow generated by the company. Coverage ratios, a category of solvency ratios, measure the ability of a company to cover its payments related to debt and leases.

  \item $\mathrm{C}$ is correct. Assuming no changes in other variables, an increase in average assets (an increase in the denominator) would decrease ROA. A decrease in either the effective tax rate or interest expense, assuming no changes in other variables, would increase ROA.

  \item $C$ is correct. The company's efficiency deteriorated, as indicated by the decline in its total asset turnover ratio from $1.11\{=4,390 /[(4,384+3,500) / 2]\}$ for FY10 to $0.87\{=11,366 /[(12,250+13,799) / 2]\}$ for FY14. The decline in the total asset turnover ratio resulted from an increase in average total assets from GBP3,942 [= $(4,384+3,500) / 2]$ for FY10 to GBP13,024.5 for FY14, an increase of 230 percent, compared with an increase in revenue from GBP4,390 in FY10 to GBP11,366 in FY14, an increase of only 159 percent. The current ratio is not an indicator of efficiency.

  \item B is correct. Comparing FY14 with FY10, the company's solvency deteriorated, as indicated by a decrease in interest coverage from $10.6(=844 / 80)$ in FY10 to 8.4 $(=1,579 / 188)$ in FY14. The debt-to-asset ratio increased from $0.14(=602 / 4,384)$ in FY10 to $0.27(=3,707 / 13,799)$ in FY14. This is also indicative of deteriorating solvency. In isolation, the amount of profits does not provide enough information to assess solvency.

  \item C is correct. Comparing FY14 with FY10, the company's liquidity improved, as indicated by an increase in its current ratio from $0.71[=(316+558) / 1,223]$ in FY10 to $0.75[=(682+1,634) / 3,108]$ in FY14. Note, however, comparing only current investments with the level of current liabilities shows a decline in liquidity from $0.26(=316 / 1,223)$ in FY10 to $0.22(=682 / 3,108)$ in FY14. Debt-to-assets ratio and interest coverage are measures of solvency not liquidity.

  \item B is correct. Comparing FY14 with FY10, the company's profitability deteriorated, as indicated by a decrease in its net profit margin from 11.0 percent $(=$ $484 / 4,390)$ to 5.7 percent $(=645 / 11,366)$. Debt-to-assets ratio is a measure of solvency not an indicator of profitability. Growth in shareholders' equity, in isolation, does not provide enough information to assess profitability.

  \item $\mathrm{C}$ is correct. The company's net profit margin has decreased and its financial leverage has increased. ROA $=$ Net profit margin $\times$ Total asset turnover. ROA decreased over the period despite the increase in total asset turnover; therefore, the net profit margin must have decreased.

\end{enumerate}

ROE $=$ Return on assets $\times$ Financial leverage. $R O E$ increased over the period despite the drop in ROA; therefore, financial leverage must have increased.

\begin{enumerate}
  \setcounter{enumi}{17}
  \item $\mathrm{C}$ is correct. The increase in the average tax rate in FY12, as indicated by the decrease in the value of the tax burden (the tax burden equals one minus the average tax rate), offset the improvement in efficiency indicated by higher asset turnover) leaving ROE unchanged. The EBIT margin, measuring profitability, was unchanged in FY12 and no information is given on liquidity.

  \item C is correct. The difference between the two companies' ROE in 2010 is very small and is mainly the result of Company A's increase in its financial leverage, indicated by the increase in its Assets/Equity ratio from 2 to 4 . The impact of efficiency on ROE is identical for the two companies, as indicated by both companies' asset turnover ratios of 1.5. Furthermore, if Company A had purchased newer equipment to replace older, depreciated equipment, then the company's asset turnover ratio (computed as sales/assets) would have declined, assuming constant sales. Company A has experienced a significant decline in its operating margin, from 10 percent to 7 percent which, all else equal, would not suggest that it is selling more products with higher profit margins.

  \item A is correct. The P/E ratio measures the "multiple" that the stock market places on a company's EPS.

  \item B is correct. In general, a creditor would consider a decrease in debt to total assets as positive news. A higher level of debt in a company's capital structure increases the risk of default and will, in general, result in higher borrowing costs for the company to compensate lenders for assuming greater credit risk. A decrease in either interest coverage or return on assets is likely to be considered negative news.

  \item B is correct. The results of an analyst's financial analysis are integral to the process of developing forecasts, along with the analysis of other information and judgment of the analysts. Forecasts are not limited to a single point estimate but should involve a range of possibilities.

\end{enumerate}

\section*{LEARNINGMODULE }
\begin{center}
\includegraphics[max width=\textwidth]{2023_05_04_b5cfa4f1bc883752f121g-269}
\end{center}

\section{Inventories}
Michael A. Broihahn, CPA, CIA, CFA, is at Barry University (USA).

\section{LEARNING OUTCOME}
\begin{center}
\begin{tabular}{c|l}
Mastery & The candidate should be able to: \\
\hline
$\square$ & $\begin{array}{l}\text { contrast costs included in inventories and costs recognised as } \\ \text { expenses in the period in which they are incurred } \\ \text { describe different inventory valuation methods (cost formulas) } \\ \text { calculate and compare cost of sales, gross profit, and ending } \\ \text { inventory using different inventory valuation methods and using } \\ \text { perpetual and periodic inventory systems } \\ \text { calculate and explain how inflation and deflation of inventory costs } \\ \text { affect the financial statements and ratios of companies that use } \\ \text { different inventory valuation methods } \\ \text { explain LIFO reserve and LIFO liquidation and their effects on } \\ \text { financial statements and ratios } \\ \text { demonstrate the conversion of a company's reported financial } \\ \text { statements from LIFO to FIFO for purposes of comparison } \\ \text { describe the measurement of inventory at the lower of cost and net } \\ \text { realisable value } \\ \text { describe implications of valuing inventory at net realisable value for } \\ \text { financial statements and ratios } \\ \text { describe the financial statement presentation of and disclosures } \\ \text { relating to inventories } \\ \text { explain issues that analysts should consider when examining a } \\ \text { company's inventory disclosures and other sources of information } \\ \text { calculate and compare ratios of companies, including companies that } \\ \text { use different inventory methods } \\ \text { analyze and compare the financial statements of companies, } \\ \text { including companies that use different inventory methods }\end{array}$ \\
$\square$ &  \\
\end{tabular}
\end{center}$\square\left[\begin{array}{l}\square\end{array}\right.$

Note: Changes in accounting standards as well as new rulings and/or pronouncements issued after the publication of the readings on financial reporting and analysis may cause some of the information in these readings to become dated. Candidates are not responsible for anything that occurs after the readings were published. In addition, candidates are expected to be familiar with the analytical frameworks contained in the readings, as well as the implications of alternative accounting methods for financial analysis and valuation discussed in the readings. Candidates are also responsible for the content of accounting standards, but not for the actual reference numbers. Finally, candidates should be aware that certain ratios may be defined and calculated differently. When alternative ratio definitions exist and no specific definition is given, candidates should use the ratio definitions emphasized in the readings.

\section{INTRODUCTION}
Merchandising and manufacturing companies generate revenues and profits through the sale of inventory. Further, inventory may represent a significant asset on these companies' balance sheets. Merchandisers (wholesalers and retailers) purchase inventory, ready for sale, from manufacturers and thus account for only one type of inventory-finished goods inventory. Manufacturers, however, purchase raw materials from suppliers and then add value by transforming the raw materials into finished goods. They typically classify inventory into three different categories: ${ }^{1}$ raw materials, work in progress, ${ }^{2}$ and finished goods. Work-in-progress inventories have started the conversion process from raw materials but are not yet finished goods ready for sale. Manufacturers may report either the separate carrying amounts of their raw materials, work-in-progress, and finished goods inventories on the balance sheet or simply the total inventory amount. If the latter approach is used, the company must then disclose the carrying amounts of its raw materials, work-in-progress, and finished goods inventories in a footnote to the financial statements.

Inventories and cost of sales (cost of goods sold) ${ }^{3}$ are significant items in the financial statements of many companies. Comparing the performance of these companies is challenging because of the allowable choices for valuing inventories: Differences in the choice of inventory valuation method can result in significantly different amounts being assigned to inventory and cost of sales. Financial statement analysis would be much easier if all companies used the same inventory valuation method or if inventory price levels remained constant over time. If there was no inflation or deflation with respect to inventory costs and thus unit costs were unchanged, the choice of inventory valuation method would be irrelevant. However, inventory price levels typically do change over time.

International Financial Reporting Standards (IFRS) permit the assignment of inventory costs (costs of goods available for sale) to inventories and cost of sales by three cost formulas: specific identification, first-in, first-out (FIFO), and weighted average cost. ${ }^{4}$ US generally accepted accounting principles (US GAAP) allow the same three inventory valuation methods, referred to as cost flow assumptions in US GAAP, but also include a fourth method called last-in, first-out (LIFO). ${ }^{5}$ The choice of inventory valuation method affects the allocation of the cost of goods available for sale to ending inventory and cost of sales. Analysts must understand the various inventory valuation methods and the related impact on financial statements and financial ratios in order to evaluate a company's performance over time and relative to industry peers. The company's financial statements and related notes provide important information that the analyst can use in assessing the impact of the choice of inventory valuation method on financial statements and financial ratios.

This reading is organized as follows: Section 2 discusses the costs that are included in inventory and the costs that are recognised as expenses in the period in which they are incurred. Section 3 describes inventory valuation methods and compares the measurement of ending inventory, cost of sales and gross profit under each method, and when using periodic versus perpetual inventory systems. Section 4 describes the LIFO method, LIFO reserve, and effects of LIFO liquidations, and demonstrates the adjustments required to compare a company that uses LIFO with one that uses FIFO. Section 5 describes the financial statement effects of a change in inventory valuation

1 Other classifications are possible. Inventory classifications should be appropriate to the entity.

2 This category is commonly referred to as work in process under US GAAP.

3 Typically, cost of sales is IFRS terminology and cost of goods sold is US GAAP terminology.

4 International Accounting Standard (IAS) 2 [Inventories].

5 Financial Accounting Standards Board Accounting Standards Codification (FASB ASC) Topic 330 [Inventory]. method. Section 6 discusses the measurement and reporting of inventory when its value changes. Section 7 describes the presentation of inventories on the financial statements and related disclosures, discusses inventory ratios and their interpretation, and shows examples of financial analysis with respect to inventories. A summary and practice problems conclude the reading.

\section{COST OF INVENTORIES}
contrast costs included in inventories and costs recognised as expenses in the period in which they are incurred

Under IFRS, the costs to include in inventories are "all costs of purchase, costs of conversion, and other costs incurred in bringing the inventories to their present location and condition." ${ }^{6}$ The costs of purchase include the purchase price, import and tax-related duties, transport, insurance during transport, handling, and other costs directly attributable to the acquisition of finished goods, materials, and services. Trade discounts, rebates, and similar items reduce the price paid and the costs of purchase. The costs of conversion include costs directly related to the units produced, such as direct labour, and fixed and variable overhead costs. ${ }^{7}$ Including these product-related costs in inventory (i.e., as an asset) means that they will not be recognised as an expense (i.e., as cost of sales) on the income statement until the inventory is sold. US GAAP provide a similar description of the costs to be included in inventory. ${ }^{8}$

Both IFRS and US GAAP exclude the following costs from inventory: abnormal costs incurred as a result of waste of materials, labour or other production conversion inputs, any storage costs (unless required as part of the production process), and all administrative overhead and selling costs. These excluded costs are treated as expenses and recognised on the income statement in the period in which they are incurred. Including costs in inventory defers their recognition as an expense on the income statement until the inventory is sold. Therefore, including costs in inventory that should be expensed will overstate profitability on the income statement (because of the inappropriate deferral of cost recognition) and create an overstated inventory value on the balance sheet.

\section{EXAMPLE 1}
\section{Treatment of Inventory-Related Costs}
Acme Enterprises, a hypothetical company that prepares its financial statements in accordance with IFRS, manufactures tables. In 2018, the factory produced 900,000 finished tables and scrapped 1,000 tables. For the finished tables, raw material costs were $€ 9$ million, direct labour conversion costs were $€ 18$ million, and production overhead costs were $€ 1.8$ million. The 1,000 scrapped tables (attributable to abnormal waste) had a total production cost of $€ 30,000(€ 10,000$ raw material costs and $€ 20,000$ conversion costs; these costs are not included in

6 International Accounting Standard (IAS) 2 [Inventories].

7 Fixed production overhead costs (depreciation, factory maintenance, and factory management and administration) represent indirect costs of production that remain relatively constant regardless of the volume of production. Variable production overhead costs are indirect production costs (indirect labour and materials) that vary with the volume of production.

8 FASB Accounting Standards Codification ${ }^{\text {Te }}$ (ASC) Topic 330 [Inventory]. the $€ 9$ million raw material and $€ 19.8$ million total conversion costs of the finished tables). During the year, Acme spent $€ 1$ million for freight delivery charges on raw materials and $€ 500,000$ for storing finished goods inventory. Acme does not have any work-in-progress inventory at the end of the year.

\begin{enumerate}
  \item What costs should be included in inventory in 2018 ?
\end{enumerate}

\section{Solution to 1:}
Total inventory costs for 2018 are as follows:

\begin{center}
\begin{tabular}{lc}
\hline
Raw materials & $€ 9,000,000$ \\
Direct labour & $18,000,000$ \\
Production overhead & $1,800,000$ \\
Transportation for raw materials & $1,000,000$ \\
\cline { 2 - 2 }
Total inventory costs & $€ 29,800,000$ \\
\hline
\end{tabular}
\end{center}

\begin{enumerate}
  \setcounter{enumi}{1}
  \item What costs should be expensed in 2018 ?
\end{enumerate}

\section{Solution to 2:}
Total costs that should be expensed (not included in inventory) are as follows:

\begin{center}
\begin{tabular}{lc}
\hline
Abnormal waste & $€ 30,000$ \\
Storage of finished goods inventory & 500,000 \\
Total & $€ 530,000$ \\
\hline
\end{tabular}
\end{center}

\section{INVENTORY VALUATION METHODS}
$\square \quad$ describe different inventory valuation methods (cost formulas)

Generally, inventory purchase costs and manufacturing conversion costs change over time. As a result, the allocation of total inventory costs (i.e., cost of goods available for sale) between cost of sales on the income statement and inventory on the balance sheet will vary depending on the inventory valuation method used by the company. As mentioned in the introduction, inventory valuation methods are referred to as cost formulas and cost flow assumptions under IFRS and US GAAP, respectively. If the choice of method results in more cost being allocated to cost of sales and less cost being allocated to inventory than would be the case with other methods, the chosen method will cause, in the current year, reported gross profit, net income, and inventory carrying amount to be lower than if alternative methods had been used. Accounting for inventory, and consequently the allocation of costs, thus has a direct impact on financial statements and their comparability.

Both IFRS and US GAAP allow companies to use the following inventory valuation methods: specific identification; first-in, first-out (FIFO); and weighted average cost. US GAAP allow companies to use an additional method: last-in, first-out (LIFO). A company must use the same inventory valuation method for all items that have a similar nature and use. For items with a different nature or use, a different inventory valuation method can be used. ${ }^{9}$ When items are sold, the carrying amount of the inventory is recognised as an expense (cost of sales) according to the cost formula (cost flow assumption) in use.

Specific identification is used for inventory items that are not ordinarily interchangeable, whereas FIFO, weighted average cost, and LIFO are typically used when there are large numbers of interchangeable items in inventory. Specific identification matches the actual historical costs of the specific inventory items to their physical flow; the costs remain in inventory until the actual identifiable inventory is sold. FIFO, weighted average cost, and LIFO are based on cost flow assumptions. Under these methods, companies must make certain assumptions about which goods are sold and which goods remain in ending inventory. As a result, the allocation of costs to the units sold and to the units in ending inventory can be different from the physical movement of the items.

The choice of inventory valuation method would be largely irrelevant if inventory costs remained constant or relatively constant over time. Given relatively constant prices, the allocation of costs between cost of goods sold and ending inventory would be very similar under each of the four methods. Given changing price levels, however, the choice of inventory valuation method can have a significant impact on the amount of reported cost of sales and inventory. And the reported cost of sales and inventory balances affect other items, such as gross profit, net income, current assets, and total assets.

\section{Specific Identification}
The specific identification method is used for inventory items that are not ordinarily interchangeable and for goods that have been produced and segregated for specific projects. This method is also commonly used for expensive goods that are uniquely identifiable, such as precious gemstones. Under this method, the cost of sales and the cost of ending inventory reflect the actual costs incurred to purchase (or manufacture) the items specifically identified as sold and the items specifically identified as remaining in inventory. Therefore, this method matches the physical flow of the specific items sold and remaining in inventory to their actual cost.

\section{First-In, First-Out (FIFO)}
FIFO assumes that the oldest goods purchased (or manufactured) are sold first and the newest goods purchased (or manufactured) remain in ending inventory. In other words, the first units included in inventory are assumed to be the first units sold from inventory. Therefore, cost of sales reflects the cost of goods in beginning inventory plus the cost of items purchased (or manufactured) earliest in the accounting period, and the value of ending inventory reflects the costs of goods purchased (or manufactured) more recently. In periods of rising prices, the costs assigned to the units in ending inventory are higher than the costs assigned to the units sold. Conversely, in periods of declining prices, the costs assigned to the units in ending inventory are lower than the costs assigned to the units sold.

9 For example, if a clothing manufacturer produces both a retail line and one-of-a-kind designer garments, the retail line might be valued using FIFO and the designer garments using specific identification.

\section{Weighted Average Cost}
Weighted average cost assigns the average cost of the goods available for sale (beginning inventory plus purchase, conversion, and other costs) during the accounting period to the units that are sold as well as to the units in ending inventory. In an accounting period, the weighted average cost per unit is calculated as the total cost of the units available for sale divided by the total number of units available for sale in the period (Total cost of goods available for sale/Total units available for sale).

\section{Last-In, First-Out (LIFO)}
LIFO is permitted only under US GAAP. This method assumes that the newest goods purchased (or manufactured) are sold first and the oldest goods purchased (or manufactured), including beginning inventory, remain in ending inventory. In other words, the last units included in inventory are assumed to be the first units sold from inventory. Therefore, cost of sales reflects the cost of goods purchased (or manufactured) more recently, and the value of ending inventory reflects the cost of older goods. In periods of rising prices, the costs assigned to the units in ending inventory are lower than the costs assigned to the units sold. Conversely, in periods of declining prices, the costs assigned to the units in ending inventory are higher than the costs assigned to the units sold.

\section{4}
\section{CALCULATIONS OF COST OF SALES, GROSS PROFIT, AND ENDING INVENTORY}
calculate and compare cost of sales, gross profit, and ending inventory using different inventory valuation methods and using perpetual and periodic inventory systems

In periods of changing prices, the allocation of total inventory costs (i.e., cost of goods available for sale) between cost of sales on the income statement and inventory on the balance sheet will vary depending on the inventory valuation method used by the company. The following example illustrates how cost of sales, gross profit, and ending inventory differ based on the choice of inventory valuation method.

\section{EXAMPLE 2}
Inventory Cost Flow Illustration for the Specific Identification, Weighted Average Cost, FIFO, and LIFO Methods

Global Sales, Inc. (GSI) is a hypothetical Dubai-based distributor of consumer products, including bars of luxury soap. The soap is sold by the kilogram. GSI began operations in 2018, during which it purchased and received initially 100,000 $\mathrm{kg}$ of soap at 110 dirham (AED) $/ \mathrm{kg}$, then $200,000 \mathrm{~kg}$ of soap at $100 \mathrm{AED} / \mathrm{kg}$, and finally $300,000 \mathrm{~kg}$ of soap at $90 \mathrm{AED} / \mathrm{kg}$. GSI sold $520,000 \mathrm{~kg}$ of soap at 240 AED/kg. GSI stores its soap in its warehouse so that soap from each shipment received is readily identifiable. During 2018 , the entire $100,000 \mathrm{~kg}$ from the first shipment received, 180,000 $\mathrm{kg}$ of the second shipment received, and 240,000 $\mathrm{kg}$ of the final shipment received was sent to customers. Answers to the following questions should be rounded to the nearest 1,000 AED.

\begin{enumerate}
  \item What are the reported cost of sales, gross profit, and ending inventory balances for 2018 under the specific identification method?
\end{enumerate}

\section{Solution to 1:}
Under the specific identification method, the physical flow of the specific inventory items sold is matched to their actual cost.

Sales $=520,000 \times 240=124,800,000$ AED

Cost of sales $=(100,000 \times 110)+(180,000 \times 100)+(240,000 \times 90)$

$=50,600,000 \mathrm{AED}$

Gross profit $=124,800,000-50,600,000=74,200,000$ AED

Ending inventory $=(20,000 \times 100)+(60,000 \times 90)=7,400,000$ AED

Note that in spite of the segregation of inventory within the warehouse, it would be inappropriate to use specific identification for this inventory of interchangeable items. The use of specific identification could potentially result in earnings manipulation through the shipment decision.

\begin{enumerate}
  \setcounter{enumi}{1}
  \item What are the reported cost of sales, gross profit, and ending inventory balances for 2018 under the weighted average cost method?
\end{enumerate}

\section{Solution to 2:}
Under the weighted average cost method, costs are allocated to cost of sales and ending inventory by using a weighted average mix of the actual costs incurred for all inventory items. The weighted average cost per unit is determined by dividing the total cost of goods available for sale by the number of units available for sale.

Weighted average cost $=[(100,000 \times 110)+(200,000 \times 100)+(300,000 \times$ 90)]/600,000

$=96.667 \mathrm{AED} / \mathrm{kg}$

Sales $=520,000 \times 240=124,800,000$ AED

Cost of sales $=520,000 \times 96.667=50,267,000$ AED

Gross profit $=124,800,000-50,267,000=74,533,000$ AED

Ending inventory $=80,000 \times 96.667=7,733,360 \mathrm{AED}$

\begin{enumerate}
  \setcounter{enumi}{2}
  \item What are the reported cost of sales, gross profit, and ending inventory balances for 2018 under the FIFO method?
\end{enumerate}

\section{Solution to 3:}
Under the FIFO method, the oldest inventory units acquired are assumed to be the first units sold. Ending inventory, therefore, is assumed to consist of those inventory units most recently acquired.

$$
\text { Sales }=520,000 \times 240=124,800,000 \text { AED }
$$

Cost of sales $=(100,000 \times 110)+(200,000 \times 100)+(220,000 \times 90)$

$=50,800,000 \mathrm{AED}$

Gross profit $=124,800,000-50,800,000=74,000,000$ AED

Ending inventory $=80,000 \times 90=7,200,000$ AED

\begin{enumerate}
  \setcounter{enumi}{3}
  \item What are the reported cost of sales, gross profit, and ending inventory balances for 2018 under the LIFO method?
\end{enumerate}

\section{Solution to 4:}
Under the LIFO method, the newest inventory units acquired are assumed to be the first units sold. Ending inventory, therefore, is assumed to consist of the oldest inventory units.

Sales $=520,000 \times 240=124,800,000$ AED

Cost of sales $=(20,000 \times 110)+(200,000 \times 100)+(300,000 \times 90)$

$=49,200,000 \mathrm{AED}$

Gross profit $=124,800,000-49,200,000=75,600,000$ AED

Ending inventory $=80,000 \times 110=8,800,000 \mathrm{AED}$

The following table (in thousands of AED) summarizes the cost of sales, the ending inventory, and the cost of goods available for sale that were calculated for each of the four inventory valuation methods. Note that in the first year of operation, the total cost of goods available for sale is the same under all four methods. Subsequently, the cost of goods available for sale will typically differ because beginning inventories will differ. Also shown is the gross profit figure for each of the four methods. Because the cost of a $\mathrm{kg}$ of soap declined over the period, LIFO had the highest ending inventory amount, the lowest cost of sales, and the highest gross profit. FIFO had the lowest ending inventory amount, the highest cost of sales, and the lowest gross profit.

\begin{center}
\begin{tabular}{lccccc}
\hline
 & \multicolumn{2}{c}{$\begin{array}{c}\text { Weighted } \\
\text { Specific } \\
\text { ID }\end{array}$} & $\begin{array}{c}\text { Average } \\ \text { Cost }\end{array}$ & FIFO & LIFO \\
\hline
Inventory Valuation Method & $\begin{array}{c}\text { ID }\end{array}$ & 50,600 & 50,800 & 49,200 &  \\
Cost of sales & 7,400 & 7,733 & 7,200 & 8,800 &  \\
Ending inventory & 58,000 & 58,000 & 58,000 & 58,000 &  \\
Total cost of goods available &  &  &  &  &  \\
for sale & 74,200 & 74,533 & 74,000 & 75,600 &  \\
Gross profit &  &  &  &  &  \\
\end{tabular}
\end{center}

\section{PERIODIC VERSUS PERPETUAL INVENTORY SYSTEMS}
calculate and compare cost of sales, gross profit, and ending inventory using different inventory valuation methods and using perpetual and periodic inventory systems

Companies typically record changes to inventory using either a periodic inventory system or a perpetual inventory system. Under a periodic inventory system, inventory values and costs of sales are determined at the end of an accounting period. Purchases are recorded in a purchases account. The total of purchases and beginning inventory is the amount of goods available for sale during the period. The ending inventory amount is subtracted from the goods available for sale to arrive at the cost of sales. The quantity of goods in ending inventory is usually obtained or verified through a physical count of the units in inventory. Under a perpetual inventory system, inventory values and cost of sales are continuously updated to reflect purchases and sales.

Under either system, the allocation of goods available for sale to cost of sales and ending inventory is the same if the inventory valuation method used is either specific identification or FIFO. This is not generally true for the weighted average cost method. Under a periodic inventory system, the amount of cost of goods available for sale allocated to cost of sales and ending inventory may be quite different using the FIFO method compared to the weighted average cost method. Under a perpetual inventory system, inventory values and cost of sales are continuously updated to reflect purchases and sales. As a result, the amount of cost of goods available for sale allocated to cost of sales and ending inventory is similar under the FIFO and weighted average cost methods. Because of lack of disclosure and the dominance of perpetual inventory systems, analysts typically do not make adjustments when comparing a company using the weighted average cost method with a company using the FIFO method.

Using the LIFO method, the periodic and perpetual inventory systems will generally result in different allocations to cost of sales and ending inventory. Under either a perpetual or periodic inventory system, the use of the LIFO method will generally result in significantly different allocations to cost of sales and ending inventory compared to other inventory valuation methods. When inventory costs are increasing and inventory unit levels are stable or increasing, using the LIFO method will result in higher cost of sales and lower inventory carrying amounts than using the FIFO method. The higher cost of sales under LIFO will result in lower gross profit, operating income, income before taxes, and net income. Income tax expense will be lower under LIFO, causing the company's net operating cash flow to be higher. On the balance sheet, the lower inventory carrying amount will result in lower reported current assets, working capital, and total assets. Analysts must carefully assess the financial statement implications of the choice of inventory valuation method when comparing companies that use the LIFO method with companies that use the FIFO method.

Example 3 illustrates the impact of the choice of system under LIFO.

\section{EXAMPLE 3}
\section{Perpetual versus Periodic Inventory Systems}
\begin{enumerate}
  \item If GSI (the company in Example 2) had used a perpetual inventory system, the timing of purchases and sales would affect the amounts of cost of sales and inventory. Below is a record of the purchases, sales, and quantity of inventory on hand after the transaction in 2018.
\end{enumerate}

\begin{center}
\begin{tabular}{|c|c|c|c|}
\hline
Date & Purchased & Sold & Inventory on Hand \\
\hline
5 January & $100,000 \mathrm{~kg}$ at $110 \mathrm{AED} / \mathrm{kg}$ &  & $100,000 \mathrm{~kg}$ \\
\hline
1 February &  & $80,000 \mathrm{~kg}$ at $240 \mathrm{AED} / \mathrm{kg}$ & $20,000 \mathrm{~kg}$ \\
\hline
8 March & $200,000 \mathrm{~kg}$ at $100 \mathrm{AED} / \mathrm{kg}$ &  & $220,000 \mathrm{~kg}$ \\
\hline
6 April &  & $100,000 \mathrm{~kg}$ at $240 \mathrm{AED} / \mathrm{kg}$ & $120,000 \mathrm{~kg}$ \\
\hline
23 May &  & $60,000 \mathrm{~kg}$ at $240 \mathrm{AED} / \mathrm{kg}$ & $60,000 \mathrm{~kg}$ \\
\hline
7 July &  & $40,000 \mathrm{~kg}$ at $240 \mathrm{AED} / \mathrm{kg}$ & $20,000 \mathrm{~kg}$ \\
\hline
2 August & $300,000 \mathrm{~kg}$ at $90 \mathrm{AED} / \mathrm{kg}$ &  & $320,000 \mathrm{~kg}$ \\
\hline
5 September &  & $70,000 \mathrm{~kg}$ at $240 \mathrm{AED} / \mathrm{kg}$ & $250,000 \mathrm{~kg}$ \\
\hline
17 November &  & $90,000 \mathrm{~kg}$ at $240 \mathrm{AED} / \mathrm{kg}$ & $160,000 \mathrm{~kg}$ \\
\hline
\multirow[t]{2}{*}{8 December} &  & $80,000 \mathrm{~kg}$ at $240 \mathrm{AED} / \mathrm{kg}$ & $80,000 \mathrm{~kg}$ \\
\hline
 & $\begin{array}{l}\text { Total goods available for sale }= \\ 58,000,000 \text { AED }\end{array}$ & Total sales $=124,800,000 \mathrm{AED}$ &  \\
\hline
\end{tabular}
\end{center}

The amounts for total goods available for sale and sales are the same under either the perpetual or periodic system in this first year of operation. The carrying amount of the ending inventory, however, may differ because the perpetual system will apply LIFO continuously throughout the year. Under the periodic system, it was assumed that the ending inventory was composed of 80,000 units of the oldest inventory, which cost $110 \mathrm{AED} / \mathrm{kg}$.

What are the ending inventory, cost of sales, and gross profit amounts using the perpetual system and the LIFO method? How do these compare with the amounts using the periodic system and the LIFO method, as in Example 2?

\section{Solution:}
The carrying amounts of the inventory at the different time points using the perpetual inventory system are as follows:

\begin{center}
\begin{tabular}{lcc}
\hline
Date & Quantity on Hand & Quantities and Cost \\
\hline
5 January & $100,000 \mathrm{~kg}$ & $100,000 \mathrm{~kg}$ at $110 \mathrm{AED} / \mathrm{kg}$ \\
1 February & $20,000 \mathrm{~kg}$ & $20,000 \mathrm{~kg}$ at $110 \mathrm{AED} / \mathrm{kg}$ \\
8 March & $220,000 \mathrm{~kg}$ & $20,000 \mathrm{~kg}$ at $110 \mathrm{AED} / \mathrm{kg}+200,000,000 \mathrm{AED}$ \\
 & $120,000 \mathrm{~kg}$ & $\mathrm{~kg}$ at $100 \mathrm{AED} / \mathrm{kg}$ \\
6 April & $20,000 \mathrm{~kg}$ at $110 \mathrm{AED} / \mathrm{kg}+100,000$ &  \\
 & $\mathrm{~kg}$ at $100 \mathrm{AED} / \mathrm{kg}$ &  \\
23 May & $60,000 \mathrm{~kg}$ & $20,200,000 \mathrm{AED}$ \\
 & $20,000 \mathrm{~kg}$ & $20,000 \mathrm{~kg}$ at $110 \mathrm{AED} / \mathrm{kg}+40,000 \mathrm{~kg}$ \\
7 July & at $100 \mathrm{AED} / \mathrm{kg}$ &  \\
 & $20000 \mathrm{~kg}$ at $110 \mathrm{AED} / \mathrm{kg}$ &  \\
\end{tabular}
\end{center}

\begin{center}
\begin{tabular}{|c|c|c|c|}
\hline
Date & Quantity on Hand & Quantities and Cost & Carrying Amount \\
\hline
2 August & $320,000 \mathrm{~kg}$ & $\begin{array}{c}20,000 \mathrm{~kg} \text { at } 110 \mathrm{AED} / \mathrm{kg}+300,000 \\ \mathrm{~kg} \text { at } 90 \mathrm{AED} / \mathrm{kg}\end{array}$ & $29,200,000 \mathrm{AED}$ \\
\hline
5 September & $250,000 \mathrm{~kg}$ & $\begin{array}{c}20,000 \mathrm{~kg} \text { at } 110 \mathrm{AED} / \mathrm{kg}+230,000 \\ \mathrm{~kg} \text { at } 90 \mathrm{AED} / \mathrm{kg}\end{array}$ & 22,900,000 AED \\
\hline
17 November & $160,000 \mathrm{~kg}$ & $\begin{array}{c}20,000 \mathrm{~kg} \text { at } 110 \mathrm{AED} / \mathrm{kg}+140,000 \\ \mathrm{~kg} \text { at } 90 \mathrm{AED} / \mathrm{kg}\end{array}$ & $14,800,000 \mathrm{AED}$ \\
\hline
8 December & $80,000 \mathrm{~kg}$ & $\begin{array}{c}20,000 \mathrm{~kg} \text { at } 110 \mathrm{AED} / \mathrm{kg}+60,000 \mathrm{~kg} \\ \text { at } 90 \mathrm{AED} / \mathrm{kg}\end{array}$ & 7,600,000 AED \\
\hline
\end{tabular}
\end{center}

Perpetual system

Sales $=520,000 \times 240=124,800,000 \mathrm{AED}$

Cost of sales $=58,000,000-7,600,000=50,400,000 \mathrm{AED}$

Gross profit $=124,800,000-50,400,000=74,400,000$ AED

Ending inventory $=7,600,000 \mathrm{AED}$

Periodic system from Example 2

Sales $=520,000 \times 240=124,800,000 \mathrm{AED}$

Cost of sales $=(20,000 \times 110)+(200,000 \times 100)+(300,000 \times 90)$

$=49,200,000 \mathrm{AED}$

Gross profit $=124,800,000-49,200,000=75,600,000$ AED

Ending inventory $=80,000 \times 110=8,800,000 \mathrm{AED}$

In this example, the ending inventory amount is lower under the perpetual system because only $20,000 \mathrm{~kg}$ of the oldest inventory with the highest cost is assumed to remain in inventory. The cost of sales is higher and the gross profit is lower under the perpetual system compared to the periodic system.

\section{COMPARISON OF INVENTORY VALUATION METHODS}
calculate and explain how inflation and deflation of inventory costs affect the financial statements and ratios of companies that use different inventory valuation methods

As shown in Example 2, the allocation of the total cost of goods available for sale to cost of sales on the income statement and to ending inventory on the balance sheet varies under the different inventory valuation methods. In an environment of declining inventory unit costs and constant or increasing inventory quantities, FIFO (in comparison with weighted average cost or LIFO) will allocate a higher amount of the total cost of goods available for sale to cost of sales on the income statement and a lower amount to ending inventory on the balance sheet. Accordingly, because cost of sales will be higher under FIFO, a company's gross profit, operating profit, and income before taxes will be lower. Conversely, in an environment of rising inventory unit costs and constant or increasing inventory quantities, FIFO (in comparison with weighted average cost or LIFO) will allocate a lower amount of the total cost of goods available for sale to cost of sales on the income statement and a higher amount to ending inventory on the balance sheet. Accordingly, because cost of sales will be lower under FIFO, a company's gross profit, operating profit, and income before taxes will be higher.

The carrying amount of inventories under FIFO will more closely reflect current replacement values because inventories are assumed to consist of the most recently purchased items. The cost of sales under LIFO will more closely reflect current replacement value. LIFO ending inventory amounts are typically not reflective of current replacement value because the ending inventory is assumed to be the oldest inventory and costs are allocated accordingly. Example 4 illustrates the different results obtained by using either the FIFO or LIFO methods to account for inventory.

\section{EXAMPLE 4}
\section{Impact of Inflation Using LIFO Compared to FIFO}
Company L and Company F are identical in all respects except that Company L uses the LIFO method and Company F uses the FIFO method. Each company has been in business for five years and maintains a base inventory of 2,000 units each year. Each year, except the first year, the number of units purchased equaled the number of units sold. Over the five year period, unit sales increased 10 percent each year and the unit purchase and selling prices increased at the beginning of each year to reflect inflation of 4 percent per year. In the first year, 20,000 units were sold at a price of $\$ 15.00$ per unit and the unit purchase price was $\$ 8.00$.

\begin{enumerate}
  \item What was the end of year inventory, sales, cost of sales, and gross profit for each company for each of the five years?
\end{enumerate}

Solution to 1:

\begin{center}
\begin{tabular}{lccccc}
\hline
$\begin{array}{l}\text { Company L using } \\ \text { LIFO }\end{array}$ & Year 1 & Year 2 & Year 3 & Year 4 & Year 5 \\
\hline
Ending inventory $^{\mathrm{a}}$ & $\$ 16,000$ & $\$ 16,000$ & $\$ 16,000$ & $\$ 16,000$ & $\$ 16,000$ \\
Sales $^{\mathrm{b}}$ & $\$ 300,000$ & $\$ 343,200$ & $\$ 392,621$ & $\$ 449,158$ & $\$ 513,837$ \\
Cost of sales $^{\mathrm{c}}$ & 160,000 & 183,040 & 209,398 & 239,551 & 274,046 \\
Gross profit & $\$ 140,000$ & $\$ 160,160$ & $\$ 183,223$ & $\$ 209,607$ & $\$ 239,791$ \\
\hline
\end{tabular}
\end{center}

\includegraphics[max width=\textwidth, center]{2023_05_04_b5cfa4f1bc883752f121g-280}
the units acquired in the first year are assumed to remain in inventory.

${ }^{b}$ Sales Year $X=(20,000 \times \$ 15)(1.10)^{X-1}(1.04)^{X-1}$. The quantity sold increases by 10 percent each year and the selling price increases by 4 percent each year.

${ }^{\mathrm{c}}$ Cost of sales Year $X=(20,000 \times \$ 8)(1.10)^{X-1}(1.04)^{X-1}$. In Year 1, 20,000 units are sold with a cost of $\$ 8$. In subsequent years, the number of units purchased equals the number of units sold and the units sold are assumed to be those purchased in the year. The quantity purchased increases by 10 percent each year and the purchase price increases by 4 percent each year.

Note that if the company sold more units than it purchased in a year, inventory would decrease. This is referred to as LIFO liquidation. The cost of sales of the units sold in excess of those purchased would reflect the inventory carrying amount. In this example, each unit sold in excess of those purchased would have a cost of sales of $\$ 8$ and a higher gross profit.

\begin{center}
\begin{tabular}{lccccc}
\hline
$\begin{array}{l}\text { Company F using } \\ \text { FIFO }\end{array}$ & Year 1 & Year 2 & Year 3 & Year 4 & Year 5 \\
\hline
Ending inventory $^{\mathrm{a}}$ & $\$ 16,000$ & $\$ 16,640$ & $\$ 17,306$ & $\$ 17,998$ & $\$ 18,718$ \\
Sales $^{\mathrm{b}}$ & $\$ 300,000$ & $\$ 343,200$ & $\$ 392,621$ & $\$ 449,158$ & $\$ 513,837$ \\
Cost of sales $^{\mathrm{c}}$ & 160,000 & 182,400 & 208,732 & 238,859 & 273,326 \\
Gross profit & $\$ 140,000$ & $\$ 160,800$ & $\$ 183,889$ & $\$ 210,299$ & $\$ 240,511$ \\
\hline
\end{tabular}
\end{center}

${ }^{a}$ Ending Inventory Year $X=2,000$ units $\times$ Cost in Year $X=2,000$ units $\left[\$ 8 \times(1.04)^{\mathrm{X}-1}\right] .2,000$ units of the units acquired in Year $\mathrm{X}$ are assumed to remain in inventory.

\begin{center}
\includegraphics[max width=\textwidth]{2023_05_04_b5cfa4f1bc883752f121g-281}
\end{center}

${ }^{c}$ Cost of sales Year $1=\$ 160,000(=20,000$ units $\times \$ 8)$. There was no beginning inventory.

Cost of sales Year $X$ (where $X \neq 1$ ) = Beginning inventory plus purchases less ending inventory

$=($ Inventory at Year $X-1)+\left[(20,000 \times \$ 8)(1.10)^{X-1}(1.04)^{X-1}\right]-$ (Inventory at Year X)

$$
\begin{aligned}
& =2,000(\$ 8)(1.04)^{\mathrm{X}-2}+\left[(20,000 \times \$ 8)(1.10)^{\mathrm{X}-1}(1.04)^{\mathrm{X}-1}\right]-[2,000(\$ 8) \\
& \left.(1.04)^{\mathrm{X}-1}\right]
\end{aligned}
$$

For example, cost of sales Year $2=2,000(\$ 8)+[(20,000 \times \$ 8)(1.10)(1.04)]$ $-[2,000(\$ 8)(1.04)]=\$ 16,000+183,040-16,640=\$ 182,400$

\begin{enumerate}
  \setcounter{enumi}{1}
  \item Compare the inventory turnover ratios (based on ending inventory carrying amounts) and gross profit margins over the five year period and between companies.
\end{enumerate}

\section{Solution to 2:}
\begin{center}
\begin{tabular}{lccccccccccc}
\hline
 & \multicolumn{4}{c}{Company L} &  & \multicolumn{5}{c}{Company F} \\
\hline
Year & $\mathbf{1}$ & $\mathbf{2}$ & $\mathbf{3}$ & $\mathbf{4}$ & $\mathbf{5}$ & $\mathbf{1}$ & $\mathbf{2}$ & $\mathbf{3}$ & $\mathbf{4}$ & $\mathbf{5}$ \\
\hline
$\begin{array}{l}\text { Inventory } \\ \text { turnover }\end{array}$ & 10.0 & 11.4 & 13.1 & 15.0 & 17.1 & 10.0 & 11.0 & 12.1 & 13.3 & 14.6 \\
$\begin{array}{l}\text { Gross profit } \\ \text { margin (\%) }\end{array}$ & 46.7 & 46.7 & 46.7 & 46.7 & 46.7 & 46.7 & 46.9 & 46.8 & 46.8 & 46.8 \\
\hline
\end{tabular}
\end{center}

Inventory turnover ratio $=$ Cost of sales $\div$ Ending inventory. The inventory turnover ratio increased each year for both companies because the units sold increased, whereas the units in ending inventory remained unchanged. The increase in the inventory turnover ratio is higher for Company L because Company L's cost of sales is increasing for inflation but the inventory carrying amount is unaffected by inflation. It might appear that a company using the LIFO method manages its inventory more effectively, but this is deceptive. Both companies have identical quantities and prices of purchases and sales and only differ in the inventory valuation method used.

Gross profit margin $=$ Gross profit $\div$ Sales. The gross profit margin is stable under LIFO because both sales and cost of sales increase at the same rate of inflation. The gross profit margin is slightly higher under the FIFO method after the first year because a proportion of the cost of sales reflects an older purchase price.

\section{THE LIFO METHOD AND LIFO RESERVE}
explain LIFO reserve and LIFO liquidation and their effects on financial statements and ratios

demonstrate the conversion of a company's reported financial statements from LIFO to FIFO for purposes of comparison

The potential income tax savings are a benefit of using the LIFO method when inventory costs are increasing. The higher cash flows due to lower income taxes may make the company more valuable because the value of a company is based on the present value of its future cash flows. Under the LIFO method, ending inventory is assumed to consist of those units that have been held the longest. This generally results in ending inventories with carrying amounts lower than current replacement costs because inventory costs typically increase over time. Cost of sales will more closely reflect current replacement costs.

If the purchase prices (purchase costs) or production costs of inventory are increasing, the income statement consequences of using the LIFO method compared to other methods will include higher cost of sales, and lower gross profit, operating profit, income tax expense, and net income. The balance sheet consequences include lower ending inventory, working capital, total assets, retained earnings, and shareholders' equity. The lower income tax paid will result in higher net cash flow from operating activities. Some of the financial ratio effects are a lower current ratio, higher debt-to-equity ratios, and lower profitability ratios.

If the purchase prices or production costs of inventory are decreasing, it is unlikely that a company will use the LIFO method for tax purposes (and therefore for financial reporting purposes due to the LIFO conformity rule) because this will result in lower cost of sales, and higher taxable income and income taxes. However, if the company had elected to use the LIFO method and cannot justify changing the inventory valuation method for tax and financial reporting purposes when inventory costs begin to decrease, the income statement, balance sheet, and ratio effects will be opposite to the effects during a period of increasing costs.

\section{LIFO Reserve}
For companies using the LIFO method, US GAAP requires disclosure, in the notes to the financial statements or on the balance sheet, of the amount of the LIFO reserve. The LIFO reserve is the difference between the reported LIFO inventory carrying amount and the inventory amount that would have been reported if the FIFO method had been used (in other words, the FIFO inventory value less the LIFO inventory value). The disclosure provides the information that analysts need to adjust a company's cost of sales (cost of goods sold) and ending inventory balance based on the LIFO method, to the FIFO method.

To compare companies using LIFO with companies not using LIFO, inventory is adjusted by adding the disclosed LIFO reserve to the inventory balance that is reported on the balance sheet. The reported inventory balance, using LIFO, plus the LIFO reserve equals the inventory that would have been reported under FIFO. Cost of sales is adjusted by subtracting the increase in the LIFO reserve during the period from the cost of sales amount that is reported on the income statement. If the LIFO reserve has declined during the period, ${ }^{10}$ the decrease in the reserve is added to the cost of sales amount that is reported on the income statement. The LIFO reserve disclosure can be used to adjust the financial statements of a US company using the LIFO method to make them comparable with a similar company using the FIFO method.

\section{LIFO LIQUIDATIONS}
$\mid \begin{aligned} & \text { explain LIFO reserve and LIFO liquidation and their effects on } \\ & \text { financial statements and ratios }\end{aligned}$

In periods of rising inventory unit costs, the carrying amount of inventory under FIFO will always exceed the carrying amount of inventory under LIFO. The LIFO reserve may increase over time as the result of the increasing difference between the older costs used to value inventory under LIFO and the more recent costs used to value inventory under FIFO. Also, when the number of inventory units manufactured or purchased exceeds the number of units sold, the LIFO reserve may increase as the result of the addition of new LIFO layers (the quantity of inventory units is increasing and each increase in quantity creates a new LIFO layer).

When the number of units sold exceeds the number of units purchased or manufactured, the number of units in ending inventory is lower than the number of units in beginning inventory and a company using LIFO will experience a LIFO liquidation (some of the older units held in inventory are assumed to have been sold). If inventory unit costs have been rising from period to period and LIFO liquidation occurs, this will produce an inventory-related increase in gross profits. The increase in gross profits occurs because of the lower inventory carrying amounts of the liquidated units. The lower inventory carrying amounts are used for cost of sales and the sales are at the current prices. The gross profit on these units is higher than the gross profit that would be recognised using more current costs. These inventory profits caused by a LIFO liquidation, however, are one-time events and are not sustainable.

10 This typically results from a reduction in inventory units and is referred to as LIFO liquidation. LIFO liquidation is discussed in the next section. LIFO liquidations can occur for a variety of reasons. The reduction in inventory levels may be outside of management's control; for example, labour strikes at a supplier may force a company to reduce inventory levels to meet customer demands. In periods of economic recession or when customer demand is declining, a company may choose to reduce existing inventory levels rather than invest in new inventory. Analysts should be aware that management can potentially manipulate and inflate their company's reported gross profits and net income at critical times by intentionally reducing inventory quantities and liquidating older layers of LIFO inventory (selling some units of beginning inventory). During economic downturns, LIFO liquidation may result in higher gross profit than would otherwise be realised. If LIFO layers of inventory are temporarily depleted and not replaced by fiscal year-end, LIFO liquidation will occur resulting in unsustainable higher gross profits. Therefore, it is imperative to review the LIFO reserve footnote disclosures to determine if LIFO liquidation has occurred. A decline in the LIFO reserve from the prior period may be indicative of LIFO liquidation.

\section{EXAMPLE 5}
\section{Inventory Conversion from LIFO to FIFO}
Caterpillar Inc. (CAT), based in Peoria, Illinois, USA, is the largest maker of construction and mining equipment, diesel and natural gas engines, and industrial gas turbines in the world. Excerpts from CAT's consolidated financial statements are shown in Exhibit 1 and Exhibit 2; notes pertaining to CAT's inventories are presented in Exhibit 3. CAT's Management Discussion and Analysis (MD\&A) disclosure states that effective income tax rates were 28 percent for 2017 and 36 percent for 2016.

Exhibit 1: Caterpillar Inc. Consolidated Results of Operation (US\$ millions)

For the years ended 31 December

2017

2016

2015

Sales and revenues:

Sales of Machinery and Engines

Revenue of Financial Products

$\begin{array}{ccc}42,676 & 35,773 & 44,147 \\ 2,786 & 2,764 & 2,864 \\ 45,462 & 38,537 & 47,011 \\ & & \\ 31,049 & 28,309 & 33,546 \\ \vdots & \vdots & \vdots \\ 646 & 596 & 587 \\ \vdots & \vdots & \vdots \\ 41,056 & 38,039 & 43,226 \\ 4,406 & 498 & 3,785 \\ 531 & 505 & 507 \\ 207 & 146 & 161 \\ 4,082 & 139 & 4,439 \\ 3,339 & 192 & 916 \\ 743 & (53) & 2,523\end{array}$

\section{Operating costs:}
Cost of goods sold

$\vdots$

Interest expense of Financial Products

\begin{center}
\begin{tabular}{lccc}
\hline
For the years ended 31 December & $\mathbf{2 0 1 7}$ & $\mathbf{2 0 1 6}$ & $\mathbf{2 0 1 5}$ \\
\hline
Profit (loss) & 754 & $(67)$ & 2,512 \\
\hline
\end{tabular}
\end{center}

Exhibit 2: Caterpillar Inc. Consolidated Financial Position (US\$ millions)

31 December

2017

2016

Assets

Current assets:

Cash and short-term investments

$\vdots$

Inventories

Total current assets

$\vdots$

Total assets

$\begin{array}{ccc}8,261 & 7,168 & 6,460 \\ \vdots & \vdots & \vdots \\ 10,018 & 8,614 & 9,700 \\ 36,244 & 31,967 & 33,508 \\ \vdots & \vdots & \vdots \\ 76,962 & 74,704 & 78,342\end{array}$

\section{Liabilities}
Total current liabilities

$26,931 \quad 26,132 \quad 26,242$

:

Total liabilities

Stockholders' equity

Common stock of $\$ 1.00$ par value:

Authorized shares: 2,000,000,000

Issued shares (2017, 2016 and 2015 -

$5,593 \quad 5,277 \quad 5,238$

$814,894,624)$ at paid-in amount

Treasury stock (2017 - 217,268,852 shares;

$(17,005) \quad(17,478) \quad(17,640)$

$2016-228,408,600$ shares and 2015 -

$232,572,734$ shares) at cost

Profit employed in the business

$26,301 \quad 27,377 \quad 29,246$

Accumulated other comprehensive income

$(1,192) \quad(2,039) \quad(2,035)$

(loss)

Noncontrolling interests

Total stockholders' equity

\begin{center}
\begin{tabular}{ccc}
69 & 76 & 76 \\
13,766 & 13,213 & 14,885 \\
76,962 & 74,704 & 78,342 \\
\hline
\end{tabular}
\end{center}

Total liabilities and stockholders' equity

Exhibit 3: Caterpillar Inc. Selected Notes to Consolidated Financial Statements

Note 1. Operations and Summary of Significant Accounting Policies D. Inventories

Inventories are stated at the lower of cost or net realizable value. Cost is principally determined using the last-in, first-out (LIFO) method. The value of inventories on the LIFO basis represented about $65 \%$ of total inventories at December 31,2017 and about $60 \%$ of total inventories at December 31, 2016 and 2015. If the FIFO (first-in, first-out) method had been in use, inventories would have been $\$ 1,924$ million, $\$ 2,139$ million and $\$ 2,498$ million higher than reported at December 31, 2017, 2016 and 2015, respectively.

Note 7. Inventories

\begin{center}
\begin{tabular}{lccc}
\hline
$\mathbf{3 1}$ December (millions of dollars) & $\mathbf{2 0 1 7}$ & $\mathbf{2 0 1 6}$ & $\mathbf{2 0 1 5}$ \\
\hline
Raw Materials & 2,802 & 2,102 & 2,467 \\
Work-in-process & 2,254 & 1,719 & 1,857 \\
Finished goods & 4,761 & 4,576 & 5,122 \\
Supplies & 201 & 217 & 254 \\
Total inventories & 10,018 & 8,614 & 9,700 \\
\hline
\end{tabular}
\end{center}

We had long-term material purchase obligations of approximately $\$ 813$ million at December 31, 2017.

\begin{enumerate}
  \item What inventory values would CAT report for 2017, 2016, and 2015 if it had used the FIFO method instead of the LIFO method?
\end{enumerate}

Solution to 1:

\begin{center}
\begin{tabular}{lccc}
\hline
$\mathbf{3 1}$ December (millions of dollars) & $\mathbf{2 0 1 7}$ & $\mathbf{2 0 1 6}$ & $\mathbf{2 0 1 5}$ \\
\hline
Total inventories (LIFO method) & 10,018 & 8,614 & 9,700 \\
From Note 1.D (LIFO reserve) & 1,924 & 2,139 & 2,498 \\
Total inventories (FIFO method) & 11,942 & 10,753 & 12,198 \\
\hline
\end{tabular}
\end{center}

Note that the decrease in the LIFO reserve from 2015-2016 and again from 2016-2017 likely indicates a LIFO liquidation for both 2016 and 2017.

\begin{enumerate}
  \setcounter{enumi}{1}
  \item What amount would CAT's cost of goods sold for 2017 and 2016 be if it had used the FIFO method instead of the LIFO method?
\end{enumerate}

\section{Solution to 2:}
\begin{center}
\begin{tabular}{lcc}
\hline
31 December (millions of dollars) & $\mathbf{2 0 1 7}$ & $\mathbf{2 0 1 6}$ \\
\hline
Cost of goods sold (LIFO method) & 31,049 & 28,309 \\
Plus: Decrease in LIFO reserve** & 215 & 359 \\
Cost of goods sold (FIFO method) & 31,264 & 28,668 \\
\hline
\end{tabular}
\end{center}

\begin{itemize}
  \item From Note 1.D, the decrease in LIFO reserve for 2017 is $215(1,924$ - 2,139) and for 2016 is 359 $(2,139-2,498)$.
\end{itemize}

\begin{enumerate}
  \setcounter{enumi}{2}
  \item What net income (profit) would CAT report for 2017 and 2016 if it had used the FIFO method instead of the LIFO method?
\end{enumerate}

Solution to 3:

31 December (millions of dollars)

2017

2016

Net income (loss) (LIFO method)

Less: Increase in cost of goods sold (decrease

$-359$

in operating profit)

\begin{center}
\begin{tabular}{lcc}
\hline
31 December (millions of dollars) & $\mathbf{2 0 1 7}$ & $\mathbf{2 0 1 6}$ \\
\hline
Tax reduction on decreased operating profit* & 60 & 129 \\
Net income (loss) (FIFO method) & 599 & -297 \\
\hline
\end{tabular}
\end{center}

\begin{itemize}
  \item The reduction in taxes on the decreased operating profit are $60(215 \times 28 \%)$ for 2017 and 129 (359 $\times 36 \%)$ for 2016 .
\end{itemize}

\begin{enumerate}
  \setcounter{enumi}{3}
  \item By what amount would CAT's 2017 and 2016 net cash flow from operating activities decline if CAT used the FIFO method instead of the LIFO method?
\end{enumerate}

\section{Solution to 4:}
The effect on a company's net cash flow from operating activities is limited to the impact of the change on income taxes paid; changes in allocating inventory costs to ending inventory and cost of goods sold does not change any cash flows except income taxes. Consequently, the effect of using FIFO on CAT's net operating cash flow from operating activities would be an increase of $\$ 60$ million in 2017 and an increase of $\$ 129$ million in 2016 . These are the approximate incremental decreases in income taxes that CAT would have incurred if the FIFO method were used instead of the LIFO method (see solution to 3 above).

\begin{enumerate}
  \setcounter{enumi}{4}
  \item What is the cumulative amount of income tax savings that CAT has generated through 2017 by using the LIFO method instead of the FIFO method?
\end{enumerate}

\section{Solution to 5:}
Using the previously mentioned effective tax rates of 28 percent for 2017 and 36 percent for 2016 (as well as for earlier years), the cumulative amount of income tax savings that CAT has generated by using the LIFO method instead of FIFO is approximately $\$ 710$ million $(-215 \times 28 \%+2,139 \times 36 \%)$. Note 1.D indicates a LIFO reserve of $\$ 2,139$ million at the end of 2016 and a decrease in the LIFO reserve of $\$ 215$ million in 2017. Therefore, under the FIFO method, cumulative gross profits would have been $\$ 2,139$ million higher as of the end of 2016 and \$1,924 million higher as of the end of 2017. The estimated tax savings would be higher (lower) if income tax rates were assumed to be higher (lower).

\begin{enumerate}
  \setcounter{enumi}{5}
  \item What amount would be added to CAT's retained earnings (profit employed in the business) at 31 December 2017 if CAT had used the FIFO method instead of the LIFO method?
\end{enumerate}

\section{Solution to 6:}
The amount that would be added to CAT's retained earnings is $\$ 1,214$ million $(1,924-710)$ or $(-215 \times 72 \%+2,139 \times 64 \%)$. This represents the cumulative increase in operating profit due to the decrease in cost of goods sold (LIFO reserve of $\$ 1,924$ million) less the assumed taxes on that profit (\$710 million, see solution to 5 above). Some analysts advocate ignoring the tax consequences and suggest simply adjusting inventory and equity by the same amount. They argue that the reported equity of the firm is understated by the difference between the current value of its inventory (approximated by the value under FIFO) and its carrying value (value under LIFO). 7. What would be the change in Cat's cash balance if CAT had used the FIFO method instead of the LIFO method?

Solution to 7:

Under the FIFO method, an additional $\$ 710$ million is assumed to have been incurred for tax expenses. If CAT switched to FIFO, it would have an additional tax liability of $\$ 710$ million as a consequence of the restatement of financial statements to the FIFO method. This illustrates the significant immediate income tax liabilities that may arise in the year of transition from the LIFO method to the FIFO method. If CAT switched to FIFO for tax purposes, there would be a cash outflow of $\$ 710$ million for the additional taxes. However, because the company is not actually converting at this point for either tax or reporting purposes, it is appropriate to reflect a deferred tax liability rather than a reduction in cash. In this case for analysis purposes, under FIFO, inventory would increase by $\$ 1,924$ million, equity by $\$ 1,214$ million, and non-current liabilities by $\$ 710$ million.

\begin{enumerate}
  \setcounter{enumi}{7}
  \item Calculate and compare the following for 2017 under the LIFO method and the FIFO method: inventory turnover ratio, days of inventory on hand, gross profit margin, net profit margin, return on assets, current ratio, and total liabilities-to-equity ratio.
\end{enumerate}

\section{Solution to 8:}
CAT's ratios for 2017 under the LIFO and FIFO methods are as follows:

\begin{center}
\begin{tabular}{lcc}
\hline
 & LIFO & FIFO \\
\hline
Inventory turnover & 3.33 & 2.76 \\
Days of inventory on hand & 109.6 days & 132.2 days \\
Gross profit margin & $27.24 \%$ & $26.74 \%$ \\
Net profit margin & $1.66 \%$ & $1.32 \%$ \\
Return on assets & $0.99 \%$ & $0.77 \%$ \\
Current ratio & 1.35 & 1.42 \\
Total liabilities-to-equity ratio & 4.59 & 4.27 \\
\hline
\end{tabular}
\end{center}

Inventory turnover ratio $=$ Cost of goods sold $\div$ Average inventory

$$
\begin{aligned}
& \text { LIFO }=3.33=31,049 \div[(10,018+8,614) \div 2] \\
& \text { FIFO }=2.76=31,264 \div[(11,942+10,753) \div 2]
\end{aligned}
$$

The ratio is higher under LIFO because, given rising inventory costs, cost of goods sold will be higher and inventory carrying amounts will be lower under LIFO. If an analyst made no adjustment for the difference in inventory methods, it might appear that a company using the LIFO method manages its inventory more effectively.

Days of inventory on hand $=$ Number of days in period $\div$ Inventory turnover ratio

$$
\mathrm{LIFO}=109.6 \text { days }=(365 \text { days } \div 3.33)
$$

$\mathrm{FIFO}=132.2$ days $=(365$ days $\div 2.76)$

Without adjustment, a company using the LIFO method might appear to manage its inventory more effectively. This is primarily the result of the lower inventory carrying amounts under LIFO.

Gross profit margin $=$ Gross profit $\div$ Total revenue

$\mathrm{LIFO}=27.24$ percent $=[(42,676-31,049) \div 42,676]$

$\mathrm{FIFO}=26.74$ percent $=[(42,676-31,264) \div 42,676]$

Revenue of financial products is excluded from the calculation of gross profit. Gross profit is sales of machinery and engines less cost of goods sold. The gross profit margin is lower under FIFO because the cost of goods sold is higher from the LIFO reserve reduction.

Net profit margin $=$ Net income $\div$ Total revenue

$\mathrm{LIFO}=1.66$ percent $=(754 \div 45,462)$

$\mathrm{FIFO}=1.32$ percent $=(599 \div 45,462]$

The net profit margin is higher under LIFO because the cost of goods sold is lower due to the LIFO liquidation. The absolute percentage difference is less than that of the gross profit margin because of lower income taxes on the decreased income reported under FIFO and because net income is divided by total revenue including sales of machinery and engines and revenue of financial products. The company appears to be more profitable under LIFO.

Return on assets $=$ Net income $\div$ Average total assets

$\mathrm{LIFO}=0.99$ percent $=754 \div[(76,962+74,704) \div 2]$

$\mathrm{FIFO}=0.77$ percent $=599 \div[(76,962+1,924)+(74,704+2,139) \div 2]$

The total assets under FIFO are the LIFO total assets increased by the LIFO reserve. The return on assets is lower under FIFO because the of the lower net income due to the higher cost of goods sold as well as higher total assets due to the LIFO reserve adjustment. The company appears to be less profitable under FIFO.

Current ratio $=$ Current assets $\div$ Current liabilities

$\mathrm{LIFO}=1.35=(36,244 \div 26,931)$

$\mathrm{FIFO}=1.42=[(36,244+1,924) \div 26,931]$

The current ratio is lower under LIFO primarily because of lower inventory carrying amount. The company appears to be less liquid under LIFO.

Total liabilities-to-equity ratio $=$ Total liabilities $\div$ Total shareholders' equity

$\mathrm{LIFO}=4.59=(63,196 \div 13,766)$

$\mathrm{FIFO}=4.27=[(63,196+710) \div(13,766+1,214)]$

The ratio is higher under LIFO because the addition to retained earnings under FIFO reduces the ratio. The company appears to be more highly leveraged under LIFO. In summary, the company appears to be more profitable, less liquid, and more highly leveraged under LIFO. Yet, because a company's value is based on the present value of future cash flows, LIFO will increase the company's value because the cash flows are higher in earlier years due to lower taxes. LIFO is primarily used for the tax benefits it provides.

\section{EXAMPLE 6}
\section{LIFO Liquidation Illustration}
\begin{enumerate}
  \item Reliable Fans, Inc. (RF), a hypothetical company, sells high quality fans and has been in business since 2015. Exhibit 4 provides relevant data and financial statement information about RF's inventory purchases and sales of fan inventory for the years 2015 through 2018. RF uses the LIFO method and a periodic inventory system. What amount of RF's 2018 gross profit is due to LIFO liquidation?
\end{enumerate}

Exhibit 4: RF Financial Statement Information under LIFO

\begin{center}
\begin{tabular}{lcccc}
\hline
 & $\mathbf{2 0 1 5}$ & $\mathbf{2 0 1 6}$ & $\mathbf{2 0 1 7}$ & $\mathbf{2 0 1 8}$ \\
\hline
Fans units purchased & 12,000 & 12,000 & 12,000 & 12,000 \\
Purchase cost per fan & $\$ 100$ & $\$ 105$ & $\$ 110$ & $\$ 115$ \\
Fans units sold & 10,000 & 12,000 & 12,000 & 13,000 \\
Sales price per fan & $\$ 200$ & $\$ 205$ & $\$ 210$ & $\$ 215$ \\
LIFO Method &  &  &  &  \\
Beginning inventory & $\$ 0$ & $\$ 200,000$ & $\$ 200,000$ & $\$ 200,000$ \\
Purchases & $1,200,000$ & $1,260,000$ & $1,320,000$ & $1,380,000$ \\
Goods available for sale & $1,200,000$ & $1,460,000$ & $1,520,000$ & $1,580,000$ \\
Ending inventory" & $(200,000)$ & $(200,000)$ & $(200,000)$ & $(100,000)$ \\
Cost of goods sold & $\$ 1,000,000$ & $1,260,000$ & $\$ 1,320,000$ & $\$ 1,480,000$ \\
Income Statement &  &  &  &  \\
Sales & $\$ 2,000,000$ & $\$ 2,460,000$ & $\$ 2,520,000$ & $\$ 2,795,000$ \\
Cost of goods sold & $1,000,000$ & $1,260,000$ & $1,320,000$ & $1,480,000$ \\
Gross profit & $\$ 1,000,000$ & $\$ 1,200,000$ & $\$ 1,200,000$ & $\$ 1,315,000$ \\
Balance Sheet &  &  &  &  \\
Inventory & $\$ 200,000$ & $\$ 200,000$ & $\$ 200,000$ & $\$ 100,000$ \\
\hline
\end{tabular}
\end{center}

\begin{itemize}
  \item Ending inventory 2015, 2016, and $2017=(2,000 \times \$ 100)$; Ending inventory $2018=(1,000 \times \$ 100)$.
\end{itemize}

\section{Solution:}
RF's reported gross profit for 2018 is $\$ 1,315,000$. RF's 2018 gross profit due to LIFO liquidation is $\$ 15,000$. If RF had purchased 13,000 fans in 2018 rather than 12,000 fans, the cost of goods sold under the LIFO method would have been $\$ 1,495,000$ (13,000 fans sold at $\$ 115.00$ purchase cost per fan), and the reported gross profit would have been $\$ 1,300,000(\$ 2,795,000$ less $\$ 1,495,000)$. The gross profit due to LIFO liquidation is $\$ 15,000(\$ 1,315,000$ reported gross profit less the $\$ 1,300,000$ gross profit that would have been reported without the LIFO liquidation). The gross profit due to LIFO liquidation may also be determined by multiplying the number of units liquidated times the difference between the replacement cost of the units liquidated and their historical purchase cost. For RF, 1,000 units times $\$ 15$ (\$115 replacement cost per fan less the $\$ 100$ historical cost per fan) equals the $\$ 15,000$ gross profit due to LIFO liquidation.

\section{INVENTORY METHOD CHANGES}
$$
\begin{aligned}
& \text { describe different inventory valuation methods (cost formulas) } \\
& \text { demonstrate the conversion of a company's reported financial } \\
& \text { statements from LIFO to FIFO for purposes of comparison }
\end{aligned}
$$

Companies on rare occasion change inventory valuation methods. Under IFRS, a change in method is acceptable only if the change "results in the financial statements providing reliable and more relevant information about the effects of transactions, other events, or conditions on the business entity's financial position, financial performance, or cash flows." ${ }^{11}$ If the change is justifiable, then it is applied retrospectively.

This means that the change is applied to comparative information for prior periods as far back as is practicable. The cumulative amount of the adjustments relating to periods prior to those presented in the current financial statements is made to the opening balance of each affected component of equity (i.e., retained earnings or comprehensive income) of the earliest period presented. For example, if a company changes its inventory method in 2018 and it presents three years of comparative financial statements (2016, 2017, and 2018) in its annual report, it would retrospectively reflect this change as far back as possible. The change would be reflected in the three years of financial statements presented; the financial statements for 2016 and 2017 would be restated as if the new method had been used in these periods, and the cumulative effect of the change on periods prior to 2016 would be reflected in the 2016 opening balance of each affected component of equity. An exemption to the restatement applies when it is impracticable to determine either the period-specific effects or the cumulative effect of the change.

Under US GAAP, the conditions to make a change in accounting policy and the accounting for a change in inventory policy are similar to IFRS. ${ }^{12}$ US GAAP, however, requires companies to thoroughly explain why the newly adopted inventory accounting method is superior and preferable to the old method. If a company decides to change from LIFO to another inventory method, US GAAP requires a retrospective restatement as described above. However, if a company decides to change to the LIFO method, it must do so on a prospective basis and retrospective adjustments are not made to the financial statements. The carrying amount of inventory under the old method becomes the initial LIFO layer in the year of LIFO adoption.

Analysts should carefully evaluate changes in inventory valuation methods. Although the stated reason for the inventory change may be to better match inventory costs with sales revenue (or some other plausible business explanation), the real underlying (and unstated) purpose may be to reduce income tax expense (if changing to LIFO from FIFO or average cost), or to increase reported profits (if changing from

11 IAS 8 [Accounting Policies, Changes in Accounting Estimates and Errors].

12 FASB ASC Topic 250 [Accounting Changes and Error Corrections]. LIFO to FIFO or average cost). As always, the choice of inventory valuation method can have a significant impact on financial statements and the financial ratios that are derived from them. As a consequence, analysts must carefully consider the impact of the change in inventory valuation methods and the differences in inventory valuation methods when comparing a company's performance with that of its industry or its competitors.

\section{INVENTORY ADJUSTMENTS}
\begin{center}
\includegraphics[max width=\textwidth]{2023_05_04_b5cfa4f1bc883752f121g-292}
\end{center}

Significant financial risk can result from the holding of inventory. The cost of inventory may not be recoverable due to spoilage, obsolescence, or declines in selling prices. IFRS state that inventories shall be measured (and carried on the balance sheet) at the lower of cost and net realisable value. ${ }^{13}$ Net realisable value is the estimated selling price in the ordinary course of business less the estimated costs necessary to make the sale and estimated costs to get the inventory in condition for sale. The assessment of net realisable value is typically done item by item or by groups of similar or related items. In the event that the value of inventory declines below the carrying amount on the balance sheet, the inventory carrying amount must be written down to its net realisable value ${ }^{14}$ and the loss (reduction in value) recognised as an expense on the income statement. This expense may be included as part of cost of sales or reported separately.

In each subsequent period, a new assessment of net realisable value is made. Reversal (limited to the amount of the original write-down) is required for a subsequent increase in value of inventory previously written down. The reversal of any write-down of inventories is recognised as a reduction in cost of sales (reduction in the amount of inventories recognised as an expense).

US GAAP used to specify the lower of cost or market to value inventories. ${ }^{15}$ For fiscal years beginning after December 15, 2016, inventories measured using other than LIFO and retail inventory methods are measured at the lower of cost or net realisable value. This is broadly consistent with IFRS with one major difference: US GAAP prohibit the reversal of write-downs. For inventories measured using LIFO and retail inventory methods, market value is defined as current replacement cost subject to upper and lower limits. Market value cannot exceed net realisable value (selling price less reasonably estimated costs of completion and disposal). The lower limit of market value is net realisable value less a normal profit margin. Any write-down to market value or net realisable value reduces the value of the inventory, and the loss in value (expense) is generally reflected in the income statement in cost of goods sold.

13 IAS 2 paragraphs 28-33 [Inventories - Net realisable value].

14 Frequently, rather than writing inventory down directly, an inventory valuation allowance account is used. The allowance account is netted with the inventory accounts to arrive at the carrying amount that appears on the balance sheet.

15 FASB ASC Section 330-10-35 [Inventory - Overall - Subsequent Measurement]. An inventory write-down reduces both profit and the carrying amount of inventory on the balance sheet and thus has a negative effect on profitability, liquidity, and solvency ratios. However, activity ratios (for example, inventory turnover and total asset turnover) will be positively affected by a write-down because the asset base (denominator) is reduced. The negative impact on some key ratios, due to the decrease in profit, may result in the reluctance by some companies to record inventory write-downs unless there is strong evidence that the decline in the value of inventory is permanent. This is especially true under US GAAP where reversal of a write-down is prohibited.

IAS 2 [Inventories] does not apply to the inventories of producers of agricultural and forest products and minerals and mineral products, nor to commodity brokertraders. These inventories may be measured at net realisable value (fair value less costs to sell and complete) according to well-established industry practices. If an active market exists for these products, the quoted market price in that market is the appropriate basis for determining the fair value of that asset. If an active market does not exist, a company may use market determined prices or values (such as the most recent market transaction price) when available for determining fair value. Changes in the value of inventory (increase or decrease) are recognised in profit or loss in the period of the change. US GAAP is similar to IFRS in its treatment of inventories of agricultural and forest products and mineral ores. Mark-to-market inventory accounting is allowed for bullion.

\section{EXAMPLE 7}
\section{Accounting for Declines and Recoveries of Inventory Value}
Hatsumei Enterprises, a hypothetical company, manufactures computers and prepares its financial statements in accordance with IFRS. In 2017, the cost of ending inventory was $€ 5.2$ million but its net realisable value was $€ 4.9$ million. The current replacement cost of the inventory is $€ 4.7$ million. This figure exceeds the net realisable value less a normal profit margin. In 2018, the net realisable value of Hatsumei's inventory was $€ 0.5$ million greater than the carrying amount.

\begin{enumerate}
  \item What was the effect of the write-down on Hatsumei's 2017 financial statements? What was the effect of the recovery on Hatsumei's 2018 financial statements?
\end{enumerate}

\section{Solution to 1:}
For 2017, Hatsumei would write its inventory down to $€ 4.9$ million and record the change in value of $€ 0.3$ million as an expense on the income statement. For 2018, Hatsumei would increase the carrying amount of its inventory and reduce the cost of sales by $€ 0.3$ million (the recovery is limited to the amount of the original write-down).

\begin{enumerate}
  \setcounter{enumi}{1}
  \item Under US GAAP, if Hatsumei used the LIFO method, what would be the effects of the write-down on Hatsumei's 2017 financial statements and of the recovery on Hatsumei's 2018 financial statements?
\end{enumerate}

\section{Solution to 2:}
Under US GAAP, for 2017, Hatsumei would write its inventory down to $€ 4.7$ million and typically include the change in value of $€ 0.5$ million in cost of goods sold on the income statement. For 2018, Hatsumei would not reverse the write-down. 3. What would be the effect of the recovery on Hatsumei's 2018 financial statements if Hatsumei's inventory were agricultural products instead of computers?

\section{Solution to 3:}
If Hatsumei's inventory were agricultural products instead of computers, inventory would be measured at net realisable value and Hatsumei would, therefore, increase inventory by and record a gain of $€ 0.5$ million for 2018 .

Analysts should consider the possibility of an inventory write-down because the impact on a company's financial ratios may be substantial. The potential for inventory write-downs can be high for companies in industries where technological obsolescence of inventories is a significant risk. Analysts should carefully evaluate prospective inventory impairments (as well as other potential asset impairments) and their potential effects on the financial ratios when debt covenants include financial ratio requirements. The breaching of debt covenants can have a significant impact on a company.

Companies that use specific identification, weighted average cost, or FIFO methods are more likely to incur inventory write-downs than companies that use the LIFO method. Under the LIFO method, the oldest costs are reflected in the inventory carrying amount on the balance sheet. Given increasing inventory costs, the inventory carrying amounts under the LIFO method are already conservatively presented at the oldest and lowest costs. Thus, it is far less likely that inventory write-downs will occur under LIFO-and if a write-down does occur, it is likely to be of a lesser magnitude.

\section{EXAMPLE 8}
\section{Effect of Inventory Write-downs on Financial Ratios}
The Volvo Group, based in Göteborg, Sweden, is a leading supplier of commercial transport products such as construction equipment, trucks, busses, and drive systems for marine and industrial applications as well as aircraft engine components. ${ }^{16}$ Excerpts from Volvo's consolidated financial statements are shown in Exhibit 5 and Exhibit 6. Notes pertaining to Volvo's inventories are presented in Exhibit 7.

Exhibit 5: Volvo Group Consolidated Income Statements (Swedish Krona in millions, except per share data)

\begin{center}
\begin{tabular}{lccc}
\hline
For the years ended 31 December & $\mathbf{2 0 1 7}$ & $\mathbf{2 0 1 6}$ & $\mathbf{2 0 1 5}$ \\
\hline
Net sales & 334,748 & 301,914 & 312,515 \\
Cost of sales & $(254,581)$ & $(231,602)$ & $(240,653)$ \\
Gross income & 80,167 & 70,312 & 71,862 \\
$\vdots$ & $\vdots$ & $\vdots$ & $\vdots$ \\
Operating income & 30,327 & 20,826 & 23,318 \\
Interest income and similar credits & 164 & 240 & 257 \\
Income expenses and similar charges & $(1,852)$ & $(1,847)$ & $(2,366)$ \\
Other financial income and expenses & $(386)$ & 11 & $(792)$ \\
Income after financial items & 28,254 & 19,230 & 20,418 \\
Income taxes & $(6,971)$ & $(6,008)$ & $(5,320)$ \\
\end{tabular}
\end{center}

16 The Volvo line of automobiles has not been under the control and management of the Volvo Group since 1999.

\begin{center}
\begin{tabular}{lccc}
\hline
For the years ended 31 December & $\mathbf{2 0 1 7}$ & $\mathbf{2 0 1 6}$ & $\mathbf{2 0 1 5}$ \\
\hline
Income for the period & 21,283 & 13,223 & 15,099 \\
Attributable to: &  &  &  \\
Equity holders of the parent company & 20,981 & 13,147 & 15,058 \\
Minority interests & 302 & 76 & 41 \\
Profit & 21,283 & 13,223 & 15,099 \\
\hline
\end{tabular}
\end{center}

Exhibit 6: Volvo Group Consolidated Balance Sheets (Swedish Krona in millions)

\begin{center}
\begin{tabular}{lccc}
\hline
$\mathbf{3 1}$ December & $\mathbf{2 0 1 7}$ & $\mathbf{2 0 1 6}$ & $\mathbf{2 0 1 5}$ \\
\hline
Assets &  &  &  \\
Total non-current assets & 213,455 & 218,465 & 203,478 \\
Current assets: &  &  &  \\
Inventories & 52,701 & 48,287 & 44,390 \\
\begin{CJK}{UTF8}{mj}三\end{CJK} & $\vdots$ & $\vdots$ & $\vdots$ \\
Cash and cash equivalents & 36,092 & 23,949 & 21,048 \\
Total current assets & 199,039 & 180,301 & 170,687 \\
Total assets & 412,494 & 398,916 & 374,165 \\
 &  &  &  \\
Shareholders' equity and liabilities &  &  &  \\
Equity attributable to equity holders of & 107,069 & 96,061 & 83,810 \\
the parent company &  &  &  \\
Minority interests & 1,941 & 1,703 & 1,801 \\
Total shareholders' equity & 109,011 & 97,764 & 85,610 \\
Total non-current provisions & 29,147 & 29,744 & 26,704 \\
Total non-current liabilities & 96,213 & 104,873 & 91,814 \\
Total current provisions & 10,806 & 11,333 & 14,176 \\
Total current liabilities &  & 155,202 & 155,860 \\
Total shareholders' equity and & 398,916 & 374,165 &  \\
liabilities &  &  &  \\
\hline
\end{tabular}
\end{center}

\section{Exhibit 7: Volvo Group Selected Notes to Consolidated Financial}
 StatementsNote 17. Inventories

Accounting Policy

Inventories are reported at the lower of cost and net realisable value. The cost is established using the first-in, first-out principle (FIFO) and is based on the standard cost method, including costs for all direct manufacturing expenses and the attributable share of capacity and other related manufacturing-related costs. The standard costs are tested regularly and adjustments are made based on current conditions. Costs for research and development, selling, administration and financial expenses are not included. Net realisable value is calculated as the selling price less costs attributable to the sale.

Sources of Estimation Uncertainty

\section{Inventory obsolescence}
If the net realisable value is lower than cost, a valuation allowance is established for inventory obsolescence. The total inventory value, net of inventory obsolescence allowance, was: SEK (in millions) 52,701 as of December 2017 and 48,287 as of 31 December 2016.

Inventory

\begin{center}
\begin{tabular}{llll}
$\mathbf{3 1}$ December (millions of Krona) & $\mathbf{2 0 1 7}$ & $\mathbf{2 0 1 6}$ & $\mathbf{2 0 1 5}$ \\
\hline
Finished products & 32,304 & 31,012 & 27,496 \\
Production materials, etc. & 20,397 & 17,275 & 16,894 \\
Total & $\mathbf{5 2 , 7 0 1}$ & $\mathbf{4 8 , 2 8 7}$ & $\mathbf{4 4 , 3 9 0}$ \\
\hline
\end{tabular}
\end{center}

Increase (decrease) in allowance for inventory obsolescence

\begin{center}
\begin{tabular}{lccc}
\hline
$\mathbf{3 1}$ December (millions of Krona) & $\mathbf{2 0 1 7}$ & $\mathbf{2 0 1 6}$ & $\mathbf{2 0 1 5}$ \\
\hline
Opening balance & 3,683 & 3,624 & 3,394 \\
Change in allowance for inventory obso- & 304 & 480 & 675 \\
lescence charged to income &  &  &  \\
Scrapping & $(391)$ & $(576)$ & $(435)$ \\
Translation differences & $(116)$ & 177 & $(29)$ \\
Reclassifications, etc. & 8 & $(23)$ & 20 \\
Allowance for inventory obsolescence as & 3,489 & 3,683 & 3,624 \\
of 31 December &  &  &  \\
\hline
\end{tabular}
\end{center}

\begin{enumerate}
  \item What inventory values would Volvo have reported for 2017, 2016, and 2015 if it had no allowance for inventory obsolescence?
\end{enumerate}

Solution to 1:

\begin{center}
\begin{tabular}{lccc}
\hline
$\mathbf{3 1}$ December (Swedish krona in &  &  &  \\
millions) & $\mathbf{2 0 1 7}$ & $\mathbf{2 0 1 6}$ & $\mathbf{2 0 1 5}$ \\
\hline
Total inventories, net & 52,701 & 48,287 & 44,390 \\
$\begin{array}{l}\text { From Note 17. (Allowance for } \\ \text { obsolescence) }\end{array}$ & 3,489 & 3,683 & 3,624 \\
Total inventories (without allowance) & 56,190 & 51,970 & 48,014 \\
\hline
\end{tabular}
\end{center}

\begin{enumerate}
  \setcounter{enumi}{1}
  \item Assuming that any changes to the allowance for inventory obsolescence are reflected in the cost of sales, what amount would Volvo's cost of sales be for 2017 and 2016 if it had not recorded inventory write-downs in 2017 and 2016?
\end{enumerate}

\section{Solution to 2:}
31 December (Swedish krona in millions)

\begin{center}
\begin{tabular}{lcc}
\hline
31 December (Swedish krona in millions) & $\mathbf{2 0 1 7}$ & $\mathbf{2 0 1 6}$ \\
\hline
(Increase) decrease in allowance for & 194 & $(59)$ \\
obsolescence* &  &  \\
Cost of sales without allowance & 254,775 & 231,543 \\
\hline
\end{tabular}
\end{center}

\begin{itemize}
  \item From Note 17, the decrease in allowance for obsolescence for 2017 is $194(3,489-3,683)$ and the increase for 2016 is $59(3,683-3,624)$.
\end{itemize}

\begin{enumerate}
  \setcounter{enumi}{2}
  \item What amount would Volvo's profit (net income) be for 2017 and 2016 if it had not recorded inventory write-downs in 2017 and 2016? Volvo's effective income tax rate was reported as 25 percent for 2017 and 31 percent for 2016.
\end{enumerate}

\section{Solution to 3:}
\begin{center}
\begin{tabular}{lcc}
\hline
$\mathbf{3 1}$ December (Swedish krona in millions) & $\mathbf{2 0 1 7}$ & $\mathbf{2 0 1 6}$ \\
\hline
Profit (Net income) & 21,283 & 13,223 \\
Increase (reduction) in cost of sales & $(194)$ & 59 \\
Taxes (tax reduction) on operating profit* & 49 & $(18)$ \\
Profit (without allowance) & 21,138 & 13,264 \\
\hline
\end{tabular}
\end{center}

"Taxes (tax reductions) on the operating profit are assumed to be $49(194 \times 25 \%)$ for 2017 and -18 $(-59 \times 31 \%)$ for 2016 .

\begin{enumerate}
  \setcounter{enumi}{3}
  \item What would Volvo's 2017 profit (net income) have been if it had reversed all past inventory write-downs in 2017? This question is independent of 1,2 , and 3. The effective income tax rate was 25 percent for 2017.
\end{enumerate}

\section{Solution to 4:}
\begin{center}
\begin{tabular}{lc}
\hline
31 December (Swedish krona in millions) & $\mathbf{2 0 1 7}$ \\
\hline
Profit (Net income) & 21,283 \\
Reduction in cost of sales (increase in operating profit) & 3,489 \\
Taxes on increased operating profit* & -872 \\
Profit (after recovery of previous write-downs) & 23,900 \\
\hline
\end{tabular}
\end{center}

\begin{itemize}
  \item Taxes on the increased operating profit are assumed to be 872 (3,489 × 25\%) for 2017.
\end{itemize}

\begin{enumerate}
  \setcounter{enumi}{4}
  \item Compare the following for 2017 based on the numbers as reported and those assuming no allowance for inventory obsolescence as in questions 1 , 2, and 3: inventory turnover ratio, days of inventory on hand, gross profit margin, and net profit margin.
\end{enumerate}

\section{Solution to 5:}
The Volvo Group's financial ratios for 2017 with the allowance for inventory obsolescence and without the allowance for inventory obsolescence are as follows:

\begin{center}
\begin{tabular}{lcc}
\hline
 & $\begin{array}{c}\text { With Allowance } \\ \text { (As Reported) }\end{array}$ & $\begin{array}{c}\text { Without Allowance } \\ \text { (Adjusted) }\end{array}$ \\
\hline
Inventory turnover ratio & 5.04 & 4.71 \\
Days of inventory on hand & 72.4 & 77.5 \\
Gross profit margin & $23.95 \%$ & $23.89 \%$ \\
Net profit margin & $6.36 \%$ & $6.31 \%$ \\
\hline
\end{tabular}
\end{center}

Inventory turnover ratio $=$ Cost of sales $\div$ Average inventory

With allowance (as reported) $=5.04=254,581 \div[(52,701+48,287) \div 2]$

Without allowance (adjusted) $=4.71=254,775 \div[(56,190+51,970) \div 2]$

Inventory turnover is higher based on the numbers as reported because inventory carrying amounts will be lower with an allowance for inventory obsolescence. The company might appear to manage its inventory more efficiently when it has inventory write-downs.

Days of inventory on hand $=$ Number of days in period $\div$ Inventory turnover ratio

With allowance $($ as reported $)=72.4$ days $=(365$ days $\div 5.04)$

Without allowance (adjusted $)=77.5$ days $=(365$ days $\div 4.71)$

Days of inventory on hand are lower based on the numbers as reported because the inventory turnover is higher. A company with inventory write-downs might appear to manage its inventory more effectively. This is primarily the result of the lower inventory carrying amounts.

Gross profit margin $=$ Gross income $\div$ Net sales

With allowance $($ as reported $)=23.95$ percent $=(80,167 \div 334,748)$

Without allowance $($ adjusted $)=23.89$ percent $=[(80,167-194) \div 334,748]$

In this instance, the gross profit margin is slightly higher with inventory write-downs because the cost of sales is lower (due to the reduction in the allowance for inventory obsolescence). This assumes that inventory write-downs (and inventory write-down recoveries) are reported as part of cost of sales.

Net profit margin $=$ Profit $\div$ Net sales

With allowance (as reported) $=6.36$ percent $=(21,283 \div 334,748)$

Without allowance (adjusted) $=6.31$ percent $=(21,138 \div 334,748)$

In this instance, the net profit margin is higher with inventory write-downs because the cost of sales is lower (due to the reduction in the allowance for inventory obsolescence). The absolute percentage difference is less than that of the gross profit margin because of the income tax reduction on the decreased income without write-downs.

The profitability ratios (gross profit margin and net profit margin) for Volvo Group would have been slightly lower for 2017 if the company had not recorded inventory write-downs. The activity ratio (inventory turnover ratio) would appear less attractive without the write-downs. The inventory turnover ratio is slightly better (higher) with inventory write-downs because inventory write-downs decrease the average inventory (denominator), making inventory management appear more efficient with write-downs.

\begin{enumerate}
  \setcounter{enumi}{5}
  \item CAT (Example 5) has no disclosures indicative of either inventory write-downs or a cumulative allowance for inventory obsolescence in its 2017 financial statements. Provide a conceptual explanation as to why Volvo incurred inventory write-downs for 2017 but CAT did not.
\end{enumerate}

\section{Solution to 6:}
CAT uses the LIFO method whereas Volvo uses the FIFO method. Given increasing inventory costs, companies that use the FIFO inventory method are far more likely to incur inventory write-downs than those companies that use the LIFO method. This is because under the LIFO method, the inventory carrying amounts reflect the oldest costs and therefore the lowest costs given increasing inventory costs. Because inventory carrying amounts under the LIFO method are already conservatively presented, it is less likely that inventory write-downs will occur.

\section{EVALUATION OF INVENTORY MANAGEMENT: DISCLOSURES \& RATIOS}
$$
\begin{aligned}
& \text { describe the financial statement presentation of and disclosures } \\
& \text { relating to inventories } \\
& \text { explain issues that analysts should consider when examining a } \\
& \text { company's inventory disclosures and other sources of information }
\end{aligned}
$$

The choice of inventory valuation method impacts the financial statements. The financial statement items impacted include cost of sales, gross profit, net income, inventories, current assets, and total assets. Therefore, the choice of inventory valuation method also affects financial ratios that contain these items. Ratios such as current ratio, return on assets, gross profit margin, and inventory turnover are impacted. As a consequence, analysts must carefully consider inventory valuation method differences when evaluating a company's performance over time or when comparing its performance with the performance of the industry or industry competitors. Additionally, the financial statement items and ratios may be impacted by adjustments of inventory carrying amounts to net realisable value or current replacement cost.

\section{Presentation and Disclosure}
Disclosures are useful when analyzing a company. IFRS require the following financial statement disclosures concerning inventory:

a. the accounting policies adopted in measuring inventories, including the cost formula (inventory valuation method) used;

b. the total carrying amount of inventories and the carrying amount in classifications (for example, merchandise, raw materials, production supplies, work in progress, and finished goods) appropriate to the entity; c. the carrying amount of inventories carried at fair value less costs to sell;

d. the amount of inventories recognised as an expense during the period (cost of sales);

e. the amount of any write-down of inventories recognised as an expense in the period;

f. the amount of any reversal of any write-down that is recognised as a reduction in cost of sales in the period;

g. the circumstances or events that led to the reversal of a write-down of inventories; and

h. the carrying amount of inventories pledged as security for liabilities.

Inventory-related disclosures under US GAAP are very similar to the disclosures above, except that requirements (f) and (g) are not relevant because US GAAP do not permit the reversal of prior-year inventory write-downs. US GAAP also require disclosure of significant estimates applicable to inventories and of any material amount of income resulting from the liquidation of LIFO inventory.

\section{Inventory Ratios}
Three ratios often used to evaluate the efficiency and effectiveness of inventory management are inventory turnover, days of inventory on hand, and gross profit margin. ${ }^{17}$ These ratios are directly impacted by a company's choice of inventory valuation method. Analysts should be aware, however, that many other ratios are also affected by the choice of inventory valuation method, although less directly. These include the current ratio, because inventory is a component of current assets; the return-on-assets ratio, because cost of sales is a key component in deriving net income and inventory is a component of total assets; and even the debt-to-equity ratio, because the cumulative measured net income from the inception of a business is an aggregate component of retained earnings.

The inventory turnover ratio measures the number of times during the year a company sells (i.e., turns over) its inventory. The higher the turnover ratio, the more times that inventory is sold during the year and the lower the relative investment of resources in inventory. Days of inventory on hand can be calculated as days in the period divided by inventory turnover. Thus, inventory turnover and days of inventory on hand are inversely related. It may be that inventory turnover, however, is calculated using average inventory in the year whereas days of inventory on hand is based on the ending inventory amount. In general, inventory turnover and the number of days of inventory on hand should be benchmarked against industry norms and compared across years.

A high inventory turnover ratio and a low number of days of inventory on hand might indicate highly effective inventory management. Alternatively, a high inventory ratio and a low number of days of inventory on hand could indicate that the company does not carry an adequate amount of inventory or that the company has written down inventory values. Inventory shortages could potentially result in lost sales or production problems in the case of the raw materials inventory of a manufacturer. To assess which explanation is more likely, analysts can compare the company's inventory turnover and sales growth rate with those of the industry and review financial statement disclosures. Slower growth combined with higher inventory turnover could

17 Days of inventory on hand is also referred to as days in inventory and average inventory days outstanding. indicate inadequate inventory levels. Write-downs of inventory could reflect poor inventory management. Minimal write-downs and sales growth rates at or above the industry+'s growth rates would support the interpretation that the higher turnover reflects greater efficiency in managing inventory.

A low inventory turnover ratio and a high number of days of inventory on hand relative to industry norms could be an indicator of slow-moving or obsolete inventory. Again, comparing the company's sales growth across years and with the industry and reviewing financial statement disclosures can provide additional insight.

The gross profit margin, the ratio of gross profit to sales, indicates the percentage of sales being contributed to net income as opposed to covering the cost of sales. Firms in highly competitive industries generally have lower gross profit margins than firms in industries with fewer competitors. A company's gross profit margin may be a function of its type of product. A company selling luxury products will generally have higher gross profit margins than a company selling staple products. The inventory turnover of the company selling luxury products, however, is likely to be much lower than the inventory turnover of the company selling staple products.

\section{ILLUSTRATIONS OF INVENTORY ANALYSIS: ADJUSTING LIFO TO FIFO}
calculate and compare ratios of companies, including companies that use different inventory methods analyze and compare the financial statements of companies, including companies that use different inventory methods

IFRS and US GAAP require companies to disclose, either on the balance sheet or in the notes to the financial statements, the carrying amounts of inventories in classifications suitable to the company. For manufacturing companies, these classifications might include production supplies, raw materials, work in progress, and finished goods. For a retailer, these classifications might include significant categories of merchandise or the grouping of inventories with similar attributes. These disclosures may provide signals about a company's future sales and profits.

For example, a significant increase (attributable to increases in unit volume rather than increases in unit cost) in raw materials and/or work-in-progress inventories may signal that the company expects an increase in demand for its products. This suggests an anticipated increase in sales and profit. However, a substantial increase in finished goods inventories while raw materials and work-in-progress inventories are declining may signal a decrease in demand for the company's products and hence lower future sales and profit. This may also signal a potential future write down of finished goods inventory. Irrespective of the signal, an analyst should thoroughly investigate the underlying reasons for any significant changes in a company's raw materials, work-in-progress, and finished goods inventories.

Analysts also should compare the growth rate of a company's sales to the growth rate of its finished goods inventories, because this could also provide a signal about future sales and profits. For example, if the growth of inventories is greater than the growth of sales, this could indicate a decline in demand and a decrease in future earnings. The company may have to lower (mark down) the selling price of its products to reduce its inventory balances, or it may have to write down the value of its inventory because of obsolescence, both of which would negatively affect profits. Besides the potential for mark-downs or write-downs, having too much inventory on hand or the wrong type of inventory can have a negative financial effect on a company because it increases inventory related expenses such as insurance, storage costs, and taxes. In addition, it means that the company has less cash and working capital available to use for other purposes.

Inventory write-downs may have a substantial impact on a company's activity, profitability, liquidity, and solvency ratios. It is critical for the analyst to be aware of industry trends toward product obsolescence and to analyze the financial ratios for their sensitivity to potential inventory impairment. Companies can minimise the impact of inventory write-downs by better matching their inventory composition and growth with prospective customer demand. To obtain additional information about a company's inventory and its future sales, a variety of sources of information are available. Analysts should consider the Management Discussion and Analysis (MD\&A) or similar sections of the company's financial reports, industry related news and publications, and industry economic data.

When conducting comparisons, differences in the choice of inventory valuation method can significantly affect the comparability of financial ratios between companies. A restatement from the LIFO method to the FIFO method is critical to make a valid comparison with companies using a method other than the LIFO method such as those companies reporting under IFRS. Analysts should seek out as much information as feasible when analyzing the performance of companies.

\section{EXAMPLE 9}
\section{Comparative Illustration}
\begin{enumerate}
  \item Using CAT's LIFO numbers as reported and FIFO adjusted numbers (Example 5) and Volvo's numbers as reported (Example 8), compare the following for 2017: inventory turnover ratio, days of inventory on hand, gross profit margin, net profit margin, return on assets, current ratio, total liabilities-to-equity ratio, and return on equity. For the current ratio, include current provisions as part of current liabilities. For the total liabilities-to-equity ratio, include provisions in total liabilities.
\end{enumerate}

\section{Solution to 1:}
The comparisons between Caterpillar and Volvo for 2017 are as follows:

\begin{center}
\begin{tabular}{lccc}
\hline
 & CAT (LIFO) & CAT (FIFO) & Volvo \\
\hline
Inventory turnover ratio & 3.33 & 2.76 & 5.04 \\
Days of inventory on hand & 109.6 days & 132.2 days & 72.4 days \\
Gross profit margin & $27.24 \%$ & $26.74 \%$ & $23.95 \%$ \\
Net profit margin & $1.66 \%$ & $1.32 \%$ & $6.36 \%$ \\
Return on assets ${ }^{\mathrm{a}}$ & $0.99 \%$ & $0.77 \%$ & $5.25 \%$ \\
Current ratio ${ }^{b}$ & 1.35 & 1.42 & 1.12 \\
Total liabilities-to-equity ratio $^{\mathrm{c}}$ & 4.59 & 4.27 & 2.78 \\
Return on equity & $5.59 \%$ & $4.05 \%$ & $20.59 \%$ \\
\hline
\end{tabular}
\end{center}

Calculations for ratios previously calculated (see Examples 5 and 8) are not shown again.

a Return on assets $=$ Net income $\div$ Average total assets

Volvo $=5.25$ percent $=21,283 \div[(412,494+398,916) \div 2]$ ${ }^{\text {b }}$ Current ratio $=$ Current assets $\div$ Current liabilities

Volvo $=1.12=[199,039 \div(10,806+167,317)]$

The question indicates to include current provisions in current liabilities.

${ }^{\mathrm{c}}$ Total liabilities-to-equity ratio $=$ Total liabilities $\div$ Total shareholders' equity

Volvo $=2.78=[(29,147+96,213+10,806+167,317) \div 109,011]$

The question indicates to include provisions in total liabilities.

${ }^{\mathrm{d}}$ Return on equity $=$ Net income $\div$ Average shareholders' equity

CAT $($ LIFO $)=5.59$ percent $=754 \div[(13,766+13,213) \div 2]$

CAT $($ FIFO $)=4.05$ percent $=599 \div\{[(13,766+1,924-710)+(13,213+$

$2,139-770)] \div 2\}$

Volvo $=20.59$ percent $=21,283 \div[(109,011+97,764) \div 2]$

Comparing CAT (FIFO) and Volvo, it appears that Volvo manages its inventory more effectively. It has higher inventory turnover and fewer days of inventory on hand. Volvo appears to have superior profitability based on net profit margin. A primary reason for CAT's low profitability in 2017 was due to a substantial increase in the provision for income taxes. An analyst would likely further investigate CAT's increase in provision for income taxes, as well as other reported numbers, rather than reaching a conclusion based on ratios alone (in other words, try to identify the underlying causes of changes or differences in ratios).

\begin{enumerate}
  \setcounter{enumi}{1}
  \item How much do inventories represent as a component of total assets for CAT using LIFO numbers as reported and FIFO adjusted numbers, and for Volvo using reported numbers in 2017 and 2016? Discuss any changes that would concern an analyst.
\end{enumerate}

\section{Solution to 2:}
The 2017 and 2016 inventory to total assets ratios for CAT using LIFO and adjusted to FIFO and for Volvo as reported, are as follows:

\begin{center}
\begin{tabular}{lccc}
\hline
 & CAT (LIFO) & CAT (FIFO) & Volvo \\
\hline
2017 & $13.02 \%$ & $15.28 \%$ & $12.78 \%$ \\
2016 & $11.53 \%$ & $14.14 \%$ & $12.10 \%$ \\
\hline
\end{tabular}
\end{center}

Inventory to total assets

CAT (LIFO) $2017=13.02$ percent $=10,018 \div 76,962$

CAT (LIFO) $2016=11.53$ percent $=8,614 \div 74,704$

CAT (FIFO) $2017=15.28$ percent $=11,942 \div(76,962+1,924-710)$

CAT (FIFO) $2016=14.14$ percent $=10,753 \div(74,704+2,139-770)$

Volvo $2017=12.78$ percent $=52,701 \div 412,494$

Volvo $2016=12.10$ percent $=48,287 \div 398,916$

Based on the numbers as reported, CAT appears to have a similar percentage of assets tied up in inventory as Volvo. However, when CAT's inventory is adjusted to FIFO, it has a higher percentage of its assets tied up in inventory than Volvo. The increase in inventory as a percentage of total assets is cause for some concern. Higher inventory typically results in higher maintenance costs (for example, storage and financing costs). A build-up of slow moving or obsolete inventories may result in future inventory write-downs. In Volvo's Note 17 , the breakdown by inventory classification shows a significant increase in the inventory of production materials. Volvo may be planning on increasing production of more finished goods inventory (which has also increased). Looking at CAT's Note 7, all classifications of inventory seem to be increasing and because these are valued using the LIFO method, there is some cause for concern. The company must be increasing inventory quantities and adding new LIFO layers.

\begin{enumerate}
  \setcounter{enumi}{2}
  \item Using the reported numbers, compare the 2016 and 2017 growth rates of CAT and Volvo for sales, finished goods inventory, and inventories other than finished goods.
\end{enumerate}

\section{Solution to 3:}
CAT's and Volvo's 2017 and 2016 growth rates for sales ("Sales of machinery and engines" for CAT and "Net sales" for Volvo), finished goods, and inventories other than finished goods" are as follows:

\begin{center}
\begin{tabular}{lcc}
\hline
$\mathbf{2 0 1 7}$ & CAT & Volvo \\
\hline
Sales & $19.3 \%$ & $10.9 \%$ \\
Finished goods & $4.0 \%$ & $4.2 \%$ \\
Inventories other than finished goods & $30.2 \%$ & $18.1 \%$ \\
 &  &  \\
\hline
$\mathbf{2 0 1 6}$ &  & CAT \\
\hline
Sales & $-19.0 \%$ & $-3.4 \%$ \\
Finished goods & $-10.7 \%$ & $12.8 \%$ \\
Inventories other than finished goods & $-11.8 \%$ & $2.3 \%$ \\
\hline
\end{tabular}
\end{center}

Growth rate $=($ Value for year - Value for previous year $) /$ Value for previous year

2017 CAT

Sales $=19.3$ percent $=(42,676-35,773) \div 35,773$

Finished goods $=4.0$ percent $=(4,761-4,576) \div 4,576$

Inventories other than finished goods $=30.2$ percent $=[(2,802+2,254$ $+201)-(2,102+1,719+217)] \div(2,102+1,719+217)$

2017 Volvo

Sales $=10.9$ percent $=(334,748-301,914) \div 301,914$

Finished products $=4.2$ percent $=(32,304-31,012) \div 31,012$

Inventories other than finished products $=18.1$ percent $=(20,397-$ $17,275) \div 17,275$ Sales $=-19.0$ percent $=(35,773-44,147) \div 44,147$

Finished goods $=-10.7$ percent $=(4,576-5,122) \div 5,122$

Inventories other than finished goods $=-11.8$ percent $=[(2,102+$

$1,719+217)-(2,467+1,857+254)] \div(2,467+1,857+254)$

2016 Volvo

Sales $=-3.4$ percent $=(301,914-312,515) \div 312,515$

Finished products $=12.8$ percent $=(31,012-27,496) \div 27,496$

Inventories other than finished products $=2.3$ percent $=(17,275-$ $16,894) \div 16,894$

For both companies, the growth rates in finished goods inventory exceeds the growth rate in sales; this could be indicative of accumulating excess inventory. Volvo's growth rate in finished goods compared to its growth rate in sales is significantly higher but the lower growth rates in finished goods inventory for CAT is potentially a result of using the LIFO method versus the FIFO method. It appears Volvo is aware that an issue exists and is planning on cutting back production given the relatively small increase in inventories other than finished products. Regardless, an analyst should do further investigation before reaching any conclusion about a company's future prospects for sales and profit.

ILLUSTRATIONS OF INVENTORY ANALYSIS: IMPACTS OF WRITEDOWNS

calculate and compare ratios of companies, including companies that use different inventory methods

analyze and compare the financial statements of companies, including companies that use different inventory methods

\section{EXAMPLE 10}
\section{Single Company Illustration}
Selected excerpts from the consolidated financial statements and notes to consolidated financial statements for Jollof Inc., a hypothetical telecommunications company providing networking and communications solutions, are presented in Exhibit 8, Exhibit 9, and Exhibit 10. Exhibit 8 contains excerpts from the consolidated income statements, and Exhibit 9 contains excerpts from the consolidated balance sheets. Exhibit 10 contains excerpts from three of the notes to consolidated financial statements.

Note 1 (a) discloses that Jollof's finished goods inventories and work in progress are valued at the lower of cost or net realisable value. Note 2 (a) discloses that the impact of inventory and work in progress write-downs on Jollof's income before tax was a net reduction of $€ 239$ million in 2017, a net reduction of $€ 156$ million in 2016, and a net reduction of $€ 65$ million in $2015{ }^{18}$ The inventory impairment loss amounts steadily increased from 2015 to 2017 and are included as a component, (additions)/reversals, of Jollof's change in valuation allowance as disclosed in Note 3 (b) from Exhibit 10. Observe also that Jollof discloses its valuation allowance at 31 December 2017, 2016, and 2015 in Note 3 (b) and details on the allocation of the allowance are included in Note 3 (a). The $€ 549$ million valuation allowance is the total of a $€ 528$ million allowance for inventories and $\mathrm{a} € 21$ million allowance for work in progress on construction contracts. Finally, observe that the $€ 1,845$ million net value for inventories (excluding construction contracts) at 31 December 2017 in Note 3 (a) reconciles with the balance sheet amount for inventories and work in progress, net, on 31 December 2017, as presented in Exhibit 9.

The inventory valuation allowance represents the total amount of inventory write-downs taken for the inventory reported on the balance sheet (which is measured at the lower of cost or net realisable value). Therefore, an analyst can determine the historical cost of the company's inventory by adding the inventory valuation allowance to the reported inventory carrying amount on the balance sheet. The valuation allowance increased in magnitude and as a percentage of gross inventory values from 2015 to 2017.

\section{Exhibit 8: Alcatel-Lucent Consolidated Income Statements ( $€$ millions)}
\begin{center}
\begin{tabular}{|c|c|c|c|}
\hline
For years ended 31 December & 2017 & 2016 & 2015 \\
\hline
Revenues & 14,267 & 14,945 & 10,317 \\
\hline
Cost of sales & $(9,400)$ & $(10,150)$ & $(6,900)$ \\
\hline
Gross profit & 4,867 & 4,795 & 3,417 \\
\hline
Administrative and selling expenses & $(2,598)$ & $(2,908)$ & $(1,605)$ \\
\hline
Research and development costs & $(2,316$ & $(2,481)$ & $(1,235)$ \\
\hline
$\begin{array}{l}\text { Income from operating activities before restructuring costs, impairment of } \\ \text { assets, gain/(loss) on disposal of consolidated entities, and post-retirement } \\ \text { benefit plan amendments }\end{array}$ & $(47)$ & $(594)$ & 577 \\
\hline
Restructuring costs & $(472)$ & $(719)$ & $(594)$ \\
\hline
Impairment of assets & $(3,969)$ & $(2,473)$ & $(118)$ \\
\hline
Gain/(loss) on disposal of consolidated entities & (6) & - & 13 \\
\hline
Post-retirement benefit plan amendments & 39 & 217 & - \\
\hline
Income (loss) from operating activities & $(4,455)$ & $(3,569)$ & $(122)$ \\
\hline
$\vdots$ & $\vdots$ & $\vdots$ & $\vdots$ \\
\hline
Income (loss) from continuing operations & $(4,373)$ & $(3,433)$ & $(184)$ \\
\hline
Income (loss) from discontinued operations & 28 & 512 & 133 \\
\hline
Net income (loss) & $(4,345)$ & $(2,921)$ & 51 \\
\hline
\end{tabular}
\end{center}

Exhibit 9: Alcatel-Lucent Consolidated Balance Sheets (€ millions)

\begin{center}
\begin{tabular}{|c|c|c|c|}
\hline
31 December & 2017 & 2016 & 2015 \\
\hline
Inventories and work in progress, net & 1,845 & 1,877 & 1,898 \\
\hline
Amounts due from customers on construction contracts & 416 & 591 & 517 \\
\hline
Trade receivables and related accounts, net & 3,637 & 3,497 & 3,257 \\
\hline
Advances and progress payments & 83 & 92 & 73 \\
\hline
$\vdots$ & $\vdots$ & $\vdots$ & $\vdots$ \\
\hline
Total current assets & 12,238 & 11,504 & 13,629 \\
\hline
Total assets & 22,941 & 28,417 & 35,188 \\
\hline
$\vdots$ & $\vdots$ & $\vdots$ & $\vdots$ \\
\hline
Retained earnings, fair value, and other reserves & $(7,409)$ & $(3,210)$ & $(2,890)$ \\
\hline
$\vdots$ & $\vdots$ & $\vdots$ & $\vdots$ \\
\hline
Total shareholders' equity & 4,388 & 9,830 & 13,711 \\
\hline
Pensions, retirement indemnities, and other post-retirement benefits & 4,038 & 3,735 & 4,577 \\
\hline
Bonds and notes issued, long-term & 3,302 & 3,794 & 4,117 \\
\hline
Other long-term debt & 56 & 40 & 123 \\
\hline
Deferred tax liabilities & 968 & 1,593 & 2,170 \\
\hline
Other non-current liabilities & 372 & 307 & 232 \\
\hline
Total non-current liabilities & 8,736 & 9,471 & 11,219 \\
\hline
Provisions & 2,036 & 2,155 & 1,987 \\
\hline
Current portion of long-term debt & 921 & 406 & 975 \\
\hline
Customers' deposits and advances & 780 & 711 & 654 \\
\hline
Amounts due to customers on construction contracts & 158 & 342 & 229 \\
\hline
Trade payables and related accounts & 3,840 & 3,792 & 3,383 \\
\hline
Liabilities related to disposal groups held for sale & - & - & 1,349 \\
\hline
Current income tax liabilities & 155 & 59 & 55 \\
\hline
Other current liabilities & 1,926 & 1,651 & 1,625 \\
\hline
Total current liabilities & 9,817 & 9,117 & 10,257 \\
\hline
Total liabilities and shareholders' equity & 22,941 & 28,417 & 35,188 \\
\hline
\end{tabular}
\end{center}

Exhibit 10: Jollof Inc. Selected Notes to Consolidated Financial Statements

Note 1. Summary of Significant Accounting Policies

(a) Inventories and work in progress

Inventories and work in progress are valued at the lower of cost (including indirect production costs where applicable) or net realizable value. ${ }^{19}$ Net realizable value is the estimated sales revenue for a normal period of activity less expected completion and selling costs.

Note 2. Principal uncertainties regarding the use of estimates (a) Valuation allowance for inventories and work in progress

19 Cost approximates cost on a first-in, first-out basis. Inventories and work in progress are measured at the lower of cost or net realizable value. Valuation allowances for inventories and work in progress are calculated based on an analysis of foreseeable changes in demand, technology, or the market, in order to determine obsolete or excess inventories and work in progress.

The valuation allowances are accounted for in cost of sales or in restructuring costs, depending on the nature of the amounts concerned.

\begin{center}
\begin{tabular}{llll}
\hline
 & \multicolumn{2}{c}{31 December} &  \\
\hline
( $\boldsymbol{*}$ millions) & $\mathbf{2 0 1 7}$ & $\mathbf{2 0 1 6}$ & $\mathbf{2 0 1 5}$ \\
\hline
$\begin{array}{l}\text { Valuation allowance for inventories and work } \\ \text { in progress on construction contracts }\end{array}$ & $(549)$ & $(432)$ & 318 \\
$\begin{array}{l}\text { Impact of inventory and work in progress } \\ \text { write-downs on income (loss) before income } \\ \text { tax related reduction of goodwill and dis- } \\ \text { counted operations }\end{array}$ & $(239)$ & $(156)$ & $(65)$ \\
\hline
\end{tabular}
\end{center}

Note 3. Inventories and work in progress

(a) Analysis of net value

\begin{center}
\begin{tabular}{|c|c|c|c|}
\hline
(€ millions) & 2017 & 2016 & 2015 \\
\hline
Raw materials and goods & 545 & 474 & 455 \\
\hline
$\begin{array}{l}\text { Work in progress excluding construc- } \\ \text { tion contracts }\end{array}$ & 816 & 805 & 632 \\
\hline
Finished goods & 1,011 & 995 & 1,109 \\
\hline
$\begin{array}{l}\text { Gross value (excluding construction } \\ \text { contracts) }\end{array}$ & 2,373 & 2,274 & 2,196 \\
\hline
Valuation allowance & $(528)$ & $(396)$ & $(298)$ \\
\hline
$\begin{array}{l}\text { Net value (excluding construction } \\ \text { contracts) }\end{array}$ & 1,845 & 1,877 & 1,898 \\
\hline
$\begin{array}{l}\text { Work in progress on construction con- } \\ \text { tracts, gross* }\end{array}$ & 184 & 228 & 291 \\
\hline
Valuation allowance & $(21)$ & $(35)$ & $(19)$ \\
\hline
$\begin{array}{l}\text { Work in progress on construction } \\ \text { contracts, net }\end{array}$ & 163 & 193 & 272 \\
\hline
Total, net & 2,008 & 2,071 & 2,170 \\
\hline
\end{tabular}
\end{center}

\begin{itemize}
  \item Included in the amounts due from/to construction contracts.
\end{itemize}

(b) Change in valuation allowance

\begin{center}
\begin{tabular}{lccc}
\hline
( $\boldsymbol{\epsilon}$ millions) & $\mathbf{2 0 1 7}$ & $\mathbf{2 0 1 6}$ & $\mathbf{2 0 1 5}$ \\
\hline
At 1 January & $(432)$ & $(318)$ & $(355)$ \\
(Additions)/reversals & $(239)$ & $(156)$ & $(65)$ \\
Utilization & 58 & 32 & 45 \\
Changes in consolidation group & - & - & 45 \\
Net effect of exchange rate changes and & 63 & 10 & 12 \\
other changes & $(549)$ & $(432)$ & $(318)$ \\
At 31 December &  &  &  \\
\end{tabular}
\end{center}

Rounding differences may result in totals that are slightly different from the sum and from corresponding numbers in the note. 1. Calculate Jollof's inventory turnover, number of days of inventory on hand, gross profit margin, current ratio, debt-to-equity ratio, and return on total assets for 2017 and 2016 based on the numbers reported. Use an average for inventory and total asset amounts and year-end numbers for other ratio items. For debt, include only bonds and notes issued, long-term; other long-term debt; and current portion of long-term debt.

\section{Solution to 1:}
The financial ratios are as follows:

\begin{center}
\begin{tabular}{lcc}
\hline
 & $\mathbf{2 0 1 7}$ & $\mathbf{2 0 1 6}$ \\
\hline
Inventory turnover ratio & 5.05 & 5.38 \\
Number of days of inventory on hand & 72.3 days & 67.8 days \\
Gross profit margin & $34.1 \%$ & $32.1 \%$ \\
Current ratio & 1.25 & 1.26 \\
Debt-to-equity ratio & 0.98 & 0.43 \\
Return on total assets & $-16.9 \%$ & $-9.2 \%$ \\
\hline
\end{tabular}
\end{center}

Inventory turnover ratio $=$ Cost of sales $\div$ Average inventory

2017 inventory turnover ratio $=5.05=9,400 \div[(1,845+1,877) \div 2]$

2016 inventory turnover ratio $=5.38=10,150 \div[(1,877+1,898) \div 2]$

Number of days of inventory $=365$ days $\div$ Inventory turnover ratio

2017 number of days of inventory $=72.3$ days $=365$ days $\div 5.05$

2016 number of days of inventory $=67.8$ days $=365$ days $\div 5.38$

Gross profit margin $=$ Gross profit $\div$ Total revenue

2017 gross profit margin $=34.1 \%=4,867 \div 14,267$

2016 gross profit margin $=32.1 \%=4,795 \div 14,945$

Current ratio $=$ Current assets $\div$ Current liabilities

2017 current ratio $=1.25=12,238 \div 9,817$

2016 current ratio $=1.26=11,504 \div 9,117$

Debt-to-equity ratio $=$ Total debt $\div$ Total shareholders' equity

2017 debt-to-equity ratio $=0.98=(3,302+56+921) \div 4,388$

2016 debt-to-equity ratio $=0.43=(3,794+40+406) \div 9,830$

Return on assets $=$ Net income $\div$ Average total assets

2017 return on assets $=-16.9 \%=-4,345 \div[(22,941+28,417) \div 2]$

2016 return on assets $=-9.2 \%=-2,921 \div[(28,417+35,188) \div 2]$ 2. Based on the answer to Question 1, comment on the changes from 2016 to 2017.

\section{Solution to 2:}
From 2016 to 2017, the inventory turnover ratio declined and the number of days of inventory increased by 4.5 days. Jollof appears to be managing inventory less efficiently. The gross profit margin improved by 2.0 percent, from 32.1 percent in 2016 to 34.1 percent in 2017. The current ratio is relatively unchanged from 2016 to 2017 . The debt-to-equity ratio has risen significantly in 2017 compared to 2016. Although Jollofn's total debt has been relatively stable during this time period, the company's equity has been declining rapidly because of the cumulative effect of its net losses on retained earnings.

The return on assets is negative and deteriorated in 2017 compared to 2016. A larger net loss and lower total assets in 2017 resulted in a higher negative return on assets. The analyst should investigate the underlying reasons for the sharp decline in Jollof's return on assets. From Exhibit 8, it is apparent that Jollof's gross profit margins were insufficient to cover the administrative and selling expenses and research and development costs in 2016 and 2017. Large restructuring costs and asset impairment losses contributed to the loss from operating activities in both 2016 and 2017.

\begin{enumerate}
  \setcounter{enumi}{2}
  \item If Jollof had used the weighted average cost method instead of the FIFO method during 2017, 2016, and 2015, what would be the effect on Jollof's reported cost of sales and inventory carrying amounts? What would be the directional impact on the financial ratios that were calculated for Jollof in Question 1?
\end{enumerate}

\section{Solution to 3:}
If inventory replacement costs were increasing during 2015, 2016, and 2017 (and inventory quantity levels were stable or increasing), Jollof's cost of sales would have been higher and its gross profit margin would have been lower under the weighted average cost inventory method than what was reported under the FIFO method (assuming no inventory write-downs that would otherwise neutralize the differences between the inventory valuation methods). FIFO allocates the oldest inventory costs to cost of sales; the reported cost of sales would be lower under FIFO given increasing inventory costs. Inventory carrying amounts would be higher under the FIFO method than under the weighted average cost method because the more recently purchased inventory items would be included in inventory at their higher costs (again assuming no inventory write-downs that would otherwise neutralize the differences between the inventory valuation methods). Consequently, Jollof's reported gross profit, net income, and retained earnings would also be higher for those years under the FIFO method.

The effects on ratios are as follows:

\begin{itemize}
  \item The inventory turnover ratios would all be higher under the weighted average cost method because the numerator (cost of sales) would be higher and the denominator (inventory) would be lower than what was reported by Jollof under the FIFO method.

  \item The number of days of inventory would be lower under the weighted average cost method because the inventory turnover ratios would be higher. - The gross profit margin ratios would all be lower under the weighted average cost method because cost of sales would be higher under the weighted average cost method than under the FIFO method.

  \item The current ratios would all be lower under the weighted average cost method because inventory carrying values would be lower than under the FIFO method (current liabilities would be the same under both methods).

  \item The return-on-assets ratios would all be lower under the weighted average cost method because the incremental profit added to the numerator (net income) has a greater impact than the incremental increase to the denominator (total assets). By way of example, assume that a company has $€ 3$ million in net income and $€ 100$ million in total assets using the weighted average cost method. If the company reports another $€ 1$ million in net income by using FIFO instead of weighted average cost, it would then also report an additional $€ 1$ million in total assets (after tax). Based on this example, the return on assets is 3.00 percent $(€ 3 / € 100)$ under the weighted average cost method and 3.96 percent $(€ 4 / € 101)$ under the FIFO method.

  \item The debt-to-equity ratios would all be higher under the weighted average cost method because retained earnings would be lower than under the FIFO method (again assuming no inventory write-downs that would otherwise neutralize the differences between the inventory valuation methods).

\end{itemize}

Conversely, if inventory replacement costs were decreasing during 2015, 2016, and 2017 (and inventory quantity levels were stable or increasing), Jollof's cost of sales would have been lower and its gross profit and inventory would have been higher under the weighted average cost method than were reported under the FIFO method (assuming no inventory write-downs that would otherwise neutralize the differences between the inventory valuation methods). As a result, the ratio assessment that was performed above would result in directly opposite conclusions.

\section{SUMMARY}
The choice of inventory valuation method (cost formula or cost flow assumption) can have a potentially significant impact on inventory carrying amounts and cost of sales. These in turn impact other financial statement items, such as current assets, total assets, gross profit, and net income. The financial statements and accompanying notes provide important information about a company's inventory accounting policies that the analyst needs to correctly assess financial performance and compare it with that of other companies. Key concepts in this reading are as follows:

\begin{itemize}
  \item Inventories are a major factor in the analysis of merchandising and manufacturing companies. Such companies generate their sales and profits through inventory transactions on a regular basis. An important consideration in determining profits for these companies is measuring the cost of sales when inventories are sold. - The total cost of inventories comprises all costs of purchase, costs of conversion, and other costs incurred in bringing the inventories to their present location and condition. Storage costs of finished inventory and abnormal costs due to waste are typically treated as expenses in the period in which they occurred.

  \item The allowable inventory valuation methods implicitly involve different assumptions about cost flows. The choice of inventory valuation method determines how the cost of goods available for sale during the period is allocated between inventory and cost of sales.

  \item IFRS allow three inventory valuation methods (cost formulas): first-in, first-out (FIFO); weighted average cost; and specific identification. The specific identification method is used for inventories of items that are not ordinarily interchangeable and for goods or services produced and segregated for specific projects. US GAAP allow the three methods above plus the last-in, first-out (LIFO) method. The LIFO method is widely used in the United States for both tax and financial reporting purposes because of potential income tax savings.

  \item The choice of inventory method affects the financial statements and any financial ratios that are based on them. As a consequence, the analyst must carefully consider inventory valuation method differences when evaluating a company's performance over time or in comparison to industry data or industry competitors.

  \item A company must use the same cost formula for all inventories having a similar nature and use to the entity.

  \item The inventory accounting system (perpetual or periodic) may result in different values for cost of sales and ending inventory when the weighted average cost or LIFO inventory valuation method is used.

  \item Under US GAAP, companies that use the LIFO method must disclose in their financial notes the amount of the LIFO reserve or the amount that would have been reported in inventory if the FIFO method had been used. This information can be used to adjust reported LIFO inventory and cost of goods sold balances to the FIFO method for comparison purposes.

  \item LIFO liquidation occurs when the number of units in ending inventory declines from the number of units that were present at the beginning of the year. If inventory unit costs have generally risen from year to year, this will produce an inventory-related increase in gross profits.

  \item Consistency of inventory costing is required under both IFRS and US GAAP. If a company changes an accounting policy, the change must be justifiable and applied retrospectively to the financial statements. An exception to the retrospective restatement is when a company reporting under US GAAP changes to the LIFO method.

  \item Under IFRS, inventories are measured at the lower of cost and net realisable value. Net realisable value is the estimated selling price in the ordinary course of business less the estimated costs necessary to make the sale. Under US GAAP, inventories are measured at the lower of cost, market value, or net realisable value depending upon the inventory method used. Market value is defined as current replacement cost subject to an upper limit of net realizable value and a lower limit of net realizable value less a normal profit margin. Reversals of previous write-downs are permissible under IFRS but not under US GAAP.

  \item Reversals of inventory write-downs may occur under IFRS but are not allowed under US GAAP. - Changes in the carrying amounts within inventory classifications (such as raw materials, work-in-process, and finished goods) may provide signals about a company's future sales and profits. Relevant information with respect to inventory management and future sales may be found in the Management Discussion and Analysis or similar items within the annual or quarterly reports, industry news and publications, and industry economic data.

  \item The inventory turnover ratio, number of days of inventory ratio, and gross profit margin ratio are useful in evaluating the management of a company's inventory.

  \item Inventory management may have a substantial impact on a company's activity, profitability, liquidity, and solvency ratios. It is critical for the analyst to be aware of industry trends and management's intentions.

  \item Financial statement disclosures provide information regarding the accounting policies adopted in measuring inventories, the principal uncertainties regarding the use of estimates related to inventories, and details of the inventory carrying amounts and costs. This information can greatly assist analysts in their evaluation of a company's inventory management.

\end{itemize}

\section{PRACTICE PROBLEMS}
\begin{enumerate}
  \item Inventory cost is least likely to include:
\end{enumerate}

A. production-related storage costs.

B. costs incurred as a result of normal waste of materials.

C. transportation costs of shipping inventory to customers.

\begin{enumerate}
  \setcounter{enumi}{1}
  \item Mustard Seed PLC adheres to IFRS. It recently purchased inventory for $€ 100$ million and spent $€ 5$ million for storage prior to selling the goods. The amount it charged to inventory expense ( $€$ millions) was closest to:
A. $€ 95$.
B. $€ 100$.
C. $€ 105$.
\end{enumerate}

\section{The following information relates to questions}
\section{3-10}
Hans Annan, CFA, a food and beverage analyst, is reviewing Century Chocolate's inventory policies as part of his evaluation of the company. Century Chocolate, based in Switzerland, manufactures chocolate products and purchases and resells other confectionery products to complement its chocolate line. Annan visited Century Chocolate's manufacturing facility last year. He learned that cacao beans, imported from Brazil, represent the most significant raw material and that the work-in-progress inventory consists primarily of three items: roasted cacao beans, a thick paste produced from the beans (called chocolate liquor), and a sweetened mixture that needs to be "conched" to produce chocolate. On the tour, Annan learned that the conching process ranges from a few hours for lower-quality products to six days for the highest-quality chocolates. While there, Annan saw the facility's climate-controlled area where manufactured finished products (cocoa and chocolate) and purchased finished goods are stored prior to shipment to customers. After touring the facility, Annan had a discussion with Century Chocolate's CFO regarding the types of costs that were included in each inventory category.

Annan has asked his assistant, Joanna Kern, to gather some preliminary information regarding Century Chocolate's financial statements and inventories. He also asked Kern to calculate the inventory turnover ratios for Century Chocolate and another chocolate manufacturer for the most recent five years. Annan does not know Century Chocolate's most direct competitor, so he asks Kern to do some research and select the most appropriate company for the ratio comparison.

Kern reports back that Century Chocolate prepares its financial statements in accordance with IFRS. She tells Annan that the policy footnote states that raw materials and purchased finished goods are valued at purchase cost whereas work in progress and manufactured finished goods are valued at production cost. Raw material inventories and purchased finished goods are accounted for using the FIFO (first-in, first-out) method, and the weighted average cost method is used for other inventories. An allowance is established when the net realisable value of any inventory item is lower than the value calculated above. Kern provides Annan with the selected financial statements and inventory data for Century Chocolate shown in Exhibits 1 through 5. The ratio exhibit Kern prepared compares Century Chocolate's inventory turnover ratios to those of Gordon's Goodies, a US-based company. Annan returns the exhibit and tells Kern to select a different competitor that reports using IFRS rather than US GAAP. During this initial review, Annan asks Kern why she has not indicated whether Century Chocolate uses a perpetual or a periodic inventory system.

Kern replies that she learned that Century Chocolate uses a perpetual system but did not include this information in her report because inventory values would be the same under either a perpetual or periodic inventory system. Annan tells Kern she is wrong and directs her to research the matter.

While Kern is revising her analysis, Annan reviews the most recent month's Cocoa Market Review from the International Cocoa Organization. He is drawn to the statement that "the ICCO daily price, averaging prices in both futures markets, reached a 29-year high in US\$ terms and a 23-year high in SDRs terms (the SDR unit comprises a basket of major currencies used in international trade: US\$, euro, pound sterling and yen)." Annan makes a note that he will need to factor the potential continuation of this trend into his analysis.

\section{Exhibit 1: Century Chocolate Income Statements (CHF Millions)}
\begin{center}
\begin{tabular}{lcc}
\hline
For Years Ended 31 December & $\mathbf{2 0 1 8}$ & $\mathbf{2 0 1 7}$ \\
\hline
Sales & 95,290 & 93,248 \\
Cost of sales & $-41,043$ & $-39,047$ \\
Marketing, administration, and other expenses & $-35,318$ & $-42,481$ \\
Profit before taxes & $\mathbf{1 8 , 9 2 9}$ & $\mathbf{1 1 , 7 2 0}$ \\
Taxes & $-3,283$ & $-2,962$ \\
Profit for the period & $\mathbf{1 5 , 6 4 6}$ & $\mathbf{8 , 7 5 8}$ \\
\hline
\end{tabular}
\end{center}

\section{Exhibit 2: Century Chocolate Balance Sheets (CHF Millions)}
\begin{center}
\begin{tabular}{|c|c|c|}
\hline
31 December & 2018 & 2017 \\
\hline
Cash, cash equivalents, and short-term investments & 6,190 & 8,252 \\
\hline
Trade receivables and related accounts, net & 11,654 & 12,910 \\
\hline
Inventories, net & 8,100 & 7,039 \\
\hline
Other current assets & 2,709 & 2,812 \\
\hline
Total current assets & 28,653 & 31,013 \\
\hline
Property, plant, and equipment, net & 18,291 & 19,130 \\
\hline
Other non-current assets & 45,144 & 49,875 \\
\hline
Total assets & 92,088 & 100,018 \\
\hline
Trade and other payables & 10,931 & 12,299 \\
\hline
Other current liabilities & 17,873 & 25,265 \\
\hline
Total current liabilities & 28,804 & 37,564 \\
\hline
Non-current liabilities & 15,672 & 14,963 \\
\hline
Total liabilities & 44,476 & 52,527 \\
\hline
\end{tabular}
\end{center}

Equity

Share capital

\begin{center}
\begin{tabular}{ccc}
332 & 341 &  \\
47,280 &  & 47,150 \\
\hline
$\mathbf{4 7 , 6 1 2}$ &  & $\mathbf{4 7 , 4 9 1}$ \\
\hline
$\mathbf{9 2 , 0 8 8}$ & $\mathbf{1 0 0 , 0 1 8}$ &  \\
\hline
\end{tabular}
\end{center}

Exhibit 3: Century Chocolate Supplementary Footnote Disclosures:

Inventories (CHF Millions)

\begin{center}
\begin{tabular}{|c|c|c|}
\hline
31 December & 2018 & 2017 \\
\hline
Raw Materials & 2,154 & 1,585 \\
\hline
Work in Progress & 1,061 & 1,027 \\
\hline
Finished Goods & 5,116 & 4,665 \\
\hline
Total inventories before allowance & 8,331 & 7,277 \\
\hline
Allowance for write-downs to net realisable value & -231 & -238 \\
\hline
Total inventories net of allowance & 8,100 & 7,039 \\
\hline
\end{tabular}
\end{center}

\section{Exhibit 4: Century Chocolate Inventory Record for Purchased Lemon Drops}
\begin{center}
\begin{tabular}{llcc}
\hline
Date &  & Cartons & $\begin{array}{c}\text { Per Unit Amount } \\ \text { (CHF) }\end{array}$ \\
\hline
 & Beginning inventory & 100 & 22 \\
4 Feb 2018 & Purchase & 40 & 25 \\
3 Apr 2018 & Sale & 50 & 32 \\
23 Jul 2018 & Purchase & 70 & 30 \\
16 Aug 2018 & Sale & 100 & 32 \\
9 Sep 2018 & Sale & 35 & 32 \\
15 Nov 2018 & Purchase & 100 &  \\
\hline
\end{tabular}
\end{center}

Exhibit 5: Century Chocolate Net Realisable Value Information for Black Licorice Jelly Beans

\begin{center}
\begin{tabular}{lcc}
\hline
 & $\mathbf{2 0 1 8}$ & $\mathbf{2 0 1 7}$ \\
\hline
FIFO cost of inventory at 31 December (CHF) & 314,890 & 374,870 \\
Ending inventory at 31 December (Kilograms) & 77,750 & 92,560 \\
Cost per unit (CHF) & 4.05 & 4.05 \\
Net Realisable Value (CHF per Kilograms) & 4.20 & 3.95 \\
\hline
\end{tabular}
\end{center}

\begin{enumerate}
  \setcounter{enumi}{2}
  \item The costs least likely to be included by the $\mathrm{CFO}$ as inventory are:
\end{enumerate}

A. storage costs for the chocolate liquor. B. excise taxes paid to the government of Brazil for the cacao beans.

C. storage costs for chocolate and purchased finished goods awaiting shipment to customers.

\begin{enumerate}
  \setcounter{enumi}{3}
  \item What is the most likely justification for Century Chocolate's choice of inventory valuation method for its purchased finished goods?
A. It is the preferred method under IFRS.
B. It allocates the same per unit cost to both cost of sales and inventory.
C. Ending inventory reflects the cost of goods purchased most recently.

  \item In Kern's comparative ratio analysis, the 2018 inventory turnover ratio for Century Chocolate is closest to:
A. 5.07 .
B. 5.42
C. 5.55

  \item The most accurate statement regarding Annan's reasoning for requiring Kern to select a competitor that reports under IFRS for comparative purposes is that under US GAAP:
A. fair values are used to value inventory.
B. the LIFO method is permitted to value inventory.
C. the specific identification method is permitted to value inventory.

  \item Annan's statement regarding the perpetual and periodic inventory systems is most significant when which of the following costing systems is used?
A. LIFO.
B. FIFO.
C. Specific identification.

  \item Using the inventory record for purchased lemon drops shown in Exhibit 4, the cost of sales for 2018 will be closest to:
A. CHF 3,550 .
B. CHF 4,550 .
C. CHF 4,850 .

  \item Ignoring any tax effect, the 2018 net realisable value reassessment for the black licorice jelly beans will most likely result in:

\end{enumerate}

A. an increase in gross profit of CHF 7,775.

B. an increase in gross profit of CHF 11,670.

C. no impact on cost of sales because under IFRS, write-downs cannot be reversed. 10. If the trend noted in the ICCO report continues and Century Chocolate plans to maintain constant or increasing inventory quantities, the most likely impact on Century Chocolate's financial statements related to its raw materials inventory will be:

A. a cost of sales that more closely reflects current replacement values.

B. a higher allocation of the total cost of goods available for sale to cost of sales.

C. a higher allocation of the total cost of goods available for sale to ending inventory.

\section{The following information relates to questions}
\section{1-22}
\begin{enumerate}
  \setcounter{enumi}{10}
  \item Cinnamon Corp. started business in 2017 and uses the weighted average cost method. During 2017, it purchased 45,000 units of inventory at $€ 10$ each and sold 40,000 units for $€ 20$ each. In 2018, it purchased another 50,000 units at $€ 11$ each and sold 45,000 units for $€ 22$ each. Its 2018 cost of sales ( $€$ thousands) was closest to:
A. $€ 490$.
B. $€ 491$.
C. $€ 495$.

  \item Zimt AG started business in 2017 and uses the FIFO method. During 2017, it purchased 45,000 units of inventory at $€ 10$ each and sold 40,000 units for $€ 20$ each. In 2018, it purchased another 50,000 units at $€ 11$ each and sold 45,000 units for $€ 22$ each. Its 2018 ending inventory balance ( $€$ thousands) was closest to:
A. $€ 105$
B. $€ 109$
C. $€ 110$.

  \item Zimt AG uses the FIFO method, and Nutmeg Inc. uses the LIFO method. Compared to the cost of replacing the inventory, during periods of rising prices, the cost of sales reported by:

\end{enumerate}

A. Zimt is too low.

B. Nutmeg is too low.

C. Nutmeg is too high.

\begin{enumerate}
  \setcounter{enumi}{13}
  \item Zimt AG uses the FIFO method, and Nutmeg Inc. uses the LIFO method. Compared to the cost of replacing the inventory, during periods of rising prices the ending inventory balance reported by:
\end{enumerate}

A. Zimt is too high.

B. Nutmeg is too low. C. Nutmeg is too high.

\begin{enumerate}
  \setcounter{enumi}{14}
  \item Like many technology companies, TechnoTools operates in an environment of declining prices. Its reported profits will tend to be highest if it accounts for inventory using the:
A. FIFO method.
B. LIFO method.
C. weighted average cost method.

  \item Compared to using the weighted average cost method to account for inventory, during a period in which prices are generally rising, the current ratio of a company using the FIFO method would most likely be:
A. lower.
B. higher.
C. dependent upon the interaction with accounts payable.

  \item Zimt AG wrote down the value of its inventory in 2017 and reversed the write-down in 2018. Compared to the ratios that would have been calculated if the write-down had never occurred, Zimt's reported 2017:
A. current ratio was too high.
B. gross margin was too high.
C. inventory turnover was too high.

  \item Zimt AG wrote down the value of its inventory in 2017 and reversed the write-down in 2018. Compared to the results the company would have reported if the write-down had never occurred, Zimt's reported 2018:
A. profit was overstated.
B. cash flow from operations was overstated.
C. year-end inventory balance was overstated.

  \item Compared to a company that uses the FIFO method, during periods of rising prices a company that uses the LIFO method will most likely appear more:
A. liquid.
B. efficient.
C. profitable.

  \item Nutmeg, Inc. uses the LIFO method to account for inventory. During years in which inventory unit costs are generally rising and in which the company purchases more inventory than it sells to customers, its reported gross profit margin will most likely be:
A. lower than it would be if the company used the FIFO method.
B. higher than it would be if the company used the FIFO method.
C. about the same as it would be if the company used the FIFO method. 21. Compared to using the FIFO method to account for inventory, during periods of rising prices, a company using the LIFO method is most likely to report higher:

\end{enumerate}

A. net income.

B. cost of sales.

C. income taxes.

\begin{enumerate}
  \setcounter{enumi}{21}
  \item Carey Company adheres to US GAAP, whereas Jonathan Company adheres to IFRS. It is least likely that:
\end{enumerate}

A. Carey has reversed an inventory write-down.

B. Jonathan has reversed an inventory write-down.

C. Jonathan and Carey both use the FIFO inventory accounting method.

\section{The following information relates to questions}
 23-24A retail company is comparing different approaches to valuing inventory. The company has one product that it sells for $\$ 50$.

\section{Exhibit 1: Units Purchased and Sold (first quarter)}
\begin{center}
\begin{tabular}{|c|c|c|c|c|c|}
\hline
Date & Units Purchased & Purchase Price & Units Sold & Selling Price & Inventory Units on Hand \\
\hline
2 Jan & 1,000 & $\$ 20.00$ &  &  & 1,000 \\
\hline
17 Jan &  &  & 500 & $\$ 50.00$ & 500 \\
\hline
16 Feb & 1,000 & $\$ 18.00$ &  &  & 1,500 \\
\hline
3 Mar &  &  & 1,200 & $\$ 50.00$ & 300 \\
\hline
13 Mar & 1,000 & $\$ 17.00$ &  &  & 1,300 \\
\hline
23 Mar &  &  & 500 & $\$ 50.00$ & 800 \\
\hline
End of quarter totals: & 3,000 & $\$ 55,000$ & 2,200 & $\$ 110,000$ &  \\
\hline
\end{tabular}
\end{center}

Exhibit 2: Comparison of Inventory Methods and Models

\section{End of Quarter Valuations}
31 March

Perpetual LIFO Periodic LIFO

Perpetual FIFO

\begin{center}
\begin{tabular}{lccc}
\hline
Sales & $\$ 110,000$ & $\$ 110,000$ & $\$ 110,000$ \\
Ending inventory &  & $\$ 16,000$ & $\$ 13,600$ \\
Cost of goods sold &  & $\$ 39,000$ & $\$ 41,400$ \\
Gross profit &  & $\$ 71,000$ & $\$ 68,600$ \\
Inventory turnover ratio & $279 \%$ &  &  \\
\hline
\end{tabular}
\end{center}

Note: LIFO is last in, first out and FIFO is first in, first out. 23. What is the value of ending inventory for the first quarter if the company uses a perpetual LIFO inventory valuation method?
A. $\$ 14,500$
B. $\$ 15,000$
C. $\$ 16,000$

\begin{enumerate}
  \setcounter{enumi}{23}
  \item Which inventory accounting method results in the lowest inventory turnover ratio for the first quarter?
A. Periodic LIFO
B. Perpetual LIFO
C. Perpetual FIFO

  \item During periods of rising inventory unit costs, a company using the FIFO method rather than the LIFO method will report a lower:
A. current ratio.
B. inventory turnover.
C. gross profit margin.

  \item Compared with a company that uses the FIFO method, during a period of rising unit inventory costs, a company using the LIFO method will most likely appear more:
A. liquid.
B. efficient.
C. profitable.

  \item In a period of declining inventory unit costs and constant or increasing inventory quantities, which inventory method is most likely to result in a higher debt-to-equity ratio?
A. LIFO
B. FIFO
C. Weighted average cost

\end{enumerate}

\section{The following information relates to questions}
\section{8-33}
Robert Groff, an equity analyst, is preparing a report on Crux Corp. As part of his report, Groff makes a comparative financial analysis between Crux and its two main competitors, Rolby Corp. and Mikko Inc. Crux and Mikko report under US GAAP and Rolby reports under IFRS.

Groff gathers information on Crux, Rolby, and Mikko. The relevant financial information he compiles is in Exhibit 1. Some information on the industry is in Exhibit 2. Exhibit 1: Selected Financial Information (US\$ Millions)

\begin{center}
\begin{tabular}{lllll}
\hline
 & Crux & Rolby & Mikko \\
\hline
Inventory valuation method & LIFO & FIFO & LIFO \\
\hline
\end{tabular}
\end{center}

From the Balance Sheets

As of 31 December 2018

\begin{center}
\begin{tabular}{|c|c|c|c|}
\hline
Inventory, gross & 480 & 620 & 510 \\
\hline
Valuation allowance & 20 & 25 & 14 \\
\hline
Inventory, net & 460 & 595 & 496 \\
\hline
Total debt & 1,122 & 850 & 732 \\
\hline
Total shareholders' equity & 2,543 & 2,403 & 2,091 \\
\hline
\multicolumn{4}{|l|}{As of 31 December 2017} \\
\hline
Inventory, gross & 465 & 602 & 401 \\
\hline
Valuation allowance & 23 & 15 & 12 \\
\hline
Inventory, net & 442 & 587 & 389 \\
\hline
\end{tabular}
\end{center}

From the Income Statements

Year Ended 31 December 2018

\begin{center}
\begin{tabular}{|c|c|c|c|}
\hline
Revenues & 4,609 & 5,442 & 3,503 \\
\hline
Cost of goods sold $^{\text {a }}$ & 3,120 & 3,782 & 2,550 \\
\hline
Net income & 229 & 327 & 205 \\
\hline
$\begin{array}{l}{ }^{\mathrm{a}} \text { Charges included in cost of goods sold for } \\ \text { inventory write-downs* }\end{array}$ & 13 & 15 & 15 \\
\hline
\end{tabular}
\end{center}

\begin{itemize}
  \item This does not match the change in the inventory valuation allowance because the valuation allowance is reduced to reflect the valuation allowance attached to items sold and increased for additional necessary write-downs.
\end{itemize}

\section{LIFO Reserve}
$\begin{array}{lllll}\text { As of } 31 \text { December } 2018 & 55 & 0 & 77\end{array}$

$\begin{array}{lllll}\text { As of } 31 \text { December } 2017 & 72 & 0 & 50\end{array}$

\begin{center}
\begin{tabular}{|cccc}
As of 31 December 2016 & 96 & 0 & 43 \\
\hline
\end{tabular}
\end{center}

Tax Rate

Effective tax rate

$30 \%$

$30 \%$

$30 \%$

\section{Exhibit 2: Industry Information}
\begin{center}
\begin{tabular}{lccc}
\hline
 & $\mathbf{2 0 1 8}$ & $\mathbf{2 0 1 7}$ & $\mathbf{2 0 1 6}$ \\
\hline
Raw materials price index & 112 & 105 & 100 \\
Finished goods price index & 114 & 106 & 100 \\
\hline
\end{tabular}
\end{center}

To compare the financial performance of the three companies, Groff decides to convert LIFO figures into FIFO figures, and adjust figures to assume no valuation allowance is recognized by any company.

After reading Groff's draft report, his supervisor, Rachel Borghi, asks him the following questions: Question $1 \quad$ Which company's gross profit margin would best reflect current costs of the industry?

Question $2 \quad$ Would Rolby's valuation method show a higher gross profit margin than Crux's under an inflationary, a deflationary, or a stable price scenario?

Question 3 Which group of ratios usually appears more favorable with an inventory write-down?

\begin{enumerate}
  \setcounter{enumi}{27}
  \item Crux's inventory turnover ratio computed as of 31 December 2018, after the adjustments suggested by Groff, is closest to:
A. 5.67 .
B. 5.83.
C. 6.13 .

  \item Rolby's net profit margin for the year ended 31 December 2018, after the adjustments suggested by Groff, is closest to:
A. $6.01 \%$.
B. $6.20 \%$.
C. $6.28 \%$.

  \item Compared with its unadjusted debt-to-equity ratio, Mikko's debt-to-equity ratio as of 31 December 2018, after the adjustments suggested by Groff, is:
A. lower.
B. higher.
C. the same.

  \item The best answer to Borghi's Question 1 is:
A. Crux's.
B. Rolby's.
C. Mikko's.

  \item The best answer to Borghi's Question 2 is:
A. Stable.
B. Inflationary.
C. Deflationary.

  \item The best answer to Borghi's Question 3 is:

\end{enumerate}

A. Activity ratios.

B. Solvency ratios. C. Profitability ratios.

\section{The following information relates to questions}
 34-40ZP Corporation is a (hypothetical) multinational corporation headquartered in Japan that trades on numerous stock exchanges. ZP prepares its consolidated financial statements in accordance with US GAAP. Excerpts from ZP's 2018 annual report are shown in Exhibits 1-3.

\section{Exhibit 1: Consolidated Balance Sheets (¥ Millions)}
\begin{center}
\begin{tabular}{|c|c|c|}
\hline
31 December & 2017 & 2018 \\
\hline
\multicolumn{3}{|l|}{Current Assets} \\
\hline
Cash and cash equivalents & $¥ 542,849$ & $¥ 814,760$ \\
\hline
$\vdots$ & $\vdots$ & $\vdots$ \\
\hline
Inventories & 608,572 & 486,465 \\
\hline
$\vdots$ & $\vdots$ & $\vdots$ \\
\hline
Total current assets & $4,028,742$ & $3,766,309$ \\
\hline
$\vdots$ & $\vdots$ & $\vdots$ \\
\hline
Total assets & $¥ 10,819,440$ & $¥ 9,687,346$ \\
\hline
$\vdots$ & $\vdots$ & $\vdots$ \\
\hline
Total current liabilities & $¥ 3,980,247$ & $¥ 3,529,765$ \\
\hline
$\vdots$ & $\vdots$ & $\vdots$ \\
\hline
Total long-term liabilities & $2,663,795$ & $2,624,002$ \\
\hline
Minority interest in consolidated subsidiaries & 218,889 & 179,843 \\
\hline
Total shareholders' equity & $3,956,509$ & $3,353,736$ \\
\hline
Total liabilities and shareholders' equity & $¥ 10,819,440$ & $¥ 9,687,346$ \\
\hline
\end{tabular}
\end{center}

Exhibit 2: Consolidated Statements of Income (¥ Millions)

\begin{center}
\begin{tabular}{|c|c|c|c|}
\hline
For the years ended 31 December & 2016 & 2017 & 2018 \\
\hline
\multicolumn{4}{|l|}{Net revenues} \\
\hline
Sales of products & $¥ 7,556,699$ & $¥ 8,273,503$ & $¥ 6,391,240$ \\
\hline
\multirow{2}{*}{Financing operations} & 425,998 & 489,577 & 451,950 \\
\hline
 & $7,982,697$ & $8,763,080$ & $6,843,190$ \\
\hline
\multicolumn{4}{|l|}{Cost and expenses} \\
\hline
Cost of products sold & $6,118,742$ & $6,817,446$ & $5,822,805$ \\
\hline
Cost of financing operations & 290,713 & 356,005 & 329,128 \\
\hline
Selling, general and administrative & 827,005 & 832,837 & 844,927 \\
\hline
$\vdots$ & $\vdots$ & $\vdots$ & $\vdots$ \\
\hline
Operating income (loss) & 746,237 & 756,792 & $-153,670$ \\
\hline
$\vdots$ & $\vdots$ & $\vdots$ & $\vdots$ \\
\hline
\end{tabular}
\end{center}

\begin{center}
\begin{tabular}{lccc}
\hline
For the years ended 31 December & $\mathbf{2 0 1 6}$ & $\mathbf{2 0 1 7}$ & $\mathbf{2 0 1 8}$ \\
\hline
Net income & $¥ 548,011$ & $¥ 572,626$ & $-¥ 145,646$ \\
\hline
\end{tabular}
\end{center}

\section{Exhibit 3: Selected Disclosures in the 2018 Annual Report}
\section{Management's Discussion and Analysis of Financial Condition and Results of Operations}
Cost reduction efforts were offset by increased prices of raw materials, other production materials and parts ... Inventories decreased during fiscal 2018 by $¥ 122.1$ billion, or $20.1 \%$, to $¥ 486.5$ billion. This reflects the impacts of decreased sales volumes and fluctuations in foreign currency translation rates.

\section{Management \& Corporate Information}
Risk Factors

Industry and Business Risks

The worldwide market for our products is highly competitive. ZP faces intense competition from other manufacturers in the respective markets in which it operates. Competition has intensified due to the worldwide deterioration in economic conditions. In addition, competition is likely to further intensify because of continuing globalization, possibly resulting in industry reorganization. Factors affecting competition include product quality and features, the amount of time required for innovation and development, pricing, reliability, safety, economy in use, customer service and financing terms. Increased competition may lead to lower unit sales and excess production capacity and excess inventory. This may result in a further downward price pressure.

ZP's ability to adequately respond to the recent rapid changes in the industry and to maintain its competitiveness will be fundamental to its future success in maintaining and expanding its market share in existing and new markets.

\section{Notes to Consolidated Financial Statements}
\begin{enumerate}
  \setcounter{enumi}{1}
  \item Summary of significant accounting policies:
\end{enumerate}

Inventories. Inventories are valued at cost, not in excess of market. Cost is determined on the "average-cost" basis, except for the cost of finished products carried by certain subsidiary companies which is determined "last-in, first-out" ("LIFO") basis. Inventories valued on the LIFO basis totaled $¥ 94,578$ million and $¥ 50,037$ million at December 31,2017 and 2018, respectively. Had the "first-in, first-out" basis been used for those companies using the LIFO basis, inventories would have been $¥ 10,120$ million and $¥ 19,660$ million higher than reported at December 31, 2017 and 2018, respectively.

\section{Inventories:}
Inventories consist of the following:

\begin{center}
\begin{tabular}{lcc}
\hline
31 December (¥ Millions) & $\mathbf{2 0 1 7}$ & $\mathbf{2 0 1 8}$ \\
\hline
Finished goods & $¥ 403,856$ & $¥ 291,977$ \\
Raw materials & 99,869 & 85,966 \\
Work in process & 79,979 & 83,890 \\
Supplies and other & 24,868 & 24,632 \\
\cline { 2 - 3 }
\end{tabular}
\end{center}

\begin{center}
\begin{tabular}{lcc}
\hline
31 December (¥ Millions) & 2017 & 2018 \\
\hline
 & $¥ 608,572$ & $¥ 486,465$ \\
\hline
\end{tabular}
\end{center}

\begin{enumerate}
  \setcounter{enumi}{33}
  \item The MD\&A indicated that the prices of raw material, other production materials, and parts increased. Based on the inventory valuation methods described in Note 2, which inventory classification would least accurately reflect current prices?
\end{enumerate}

A. Raw materials.

B. Finished goods.

C. Work in process.

\begin{enumerate}
  \setcounter{enumi}{34}
  \item The 2017 inventory value as reported on the 2018 Annual Report if the company had used the FIFO inventory valuation method instead of the LIFO inventory valuation method for a portion of its inventory would be closest to:
A. $¥ 104,698$ million.
B. $¥ 506,125$ million.
C. $¥ 618,692$ million.

  \item If ZP had prepared its financial statement in accordance with IFRS, the inventory turnover ratio (using average inventory) for 2018 would be:
A. lower.
B. higher.
C. the same.

  \item Inventory levels decreased from 2017 to 2018 for all of the following reasons except:

\end{enumerate}

A. LIFO liquidation.

B. decreased sales volume.

C. fluctuations in foreign currency translation rates.

\begin{enumerate}
  \setcounter{enumi}{37}
  \item Which observation is most likely a result of looking only at the information reported in Note 9 ?
\end{enumerate}

A. Increased competition has led to lower unit sales.

B. There have been significant price increases in supplies.

C. Management expects a further downturn in sales during 2019.

\begin{enumerate}
  \setcounter{enumi}{38}
  \item Note 2 indicates that, "Inventories valued on the LIFO basis totaled $¥ 94,578$ million and $¥ 50,037$ million at December 31, 2017 and 2018, respectively." Based on this, the LIFO reserve should most likely:
\end{enumerate}

A. increase. B. decrease.

C. remain the same.

\begin{enumerate}
  \setcounter{enumi}{39}
  \item The Industry and Business Risk excerpt states that, "Increased competition may lead to lower unit sales and excess production capacity and excess inventory. This may result in a further downward price pressure." The downward price pressure could lead to inventory that is valued above current market prices or net realisable value. Any write-downs of inventory are least likely to have a significant effect on the inventory valued using:
\end{enumerate}

A. weighted average cost.

B. first-in, first-out (FIFO).

C. last-in, first-out (LIFO).

\begin{enumerate}
  \setcounter{enumi}{40}
  \item LIFO reserve is most likely to increase when inventory unit:
A. costs are increasing.
B. costs are decreasing.
C. levels are decreasing.

  \item A company using the LIFO method reports the following in $\pounds$ :

\end{enumerate}

\begin{center}
\begin{tabular}{lcc}
\hline
 & $\mathbf{2 0 1 8}$ & $\mathbf{2 0 1 7}$ \\
\hline
Cost of goods sold (COGS) & 50,800 & 48,500 \\
Ending inventories & 10,550 & 10,000 \\
LIFO reserve & 4,320 & 2,600 \\
\hline
\end{tabular}
\end{center}

Cost of goods sold for 2018 under the FIFO method is closest to:
A. $\pounds 48,530$.
B. $\pounds 49,080$.
C. $\pounds 52,520$.

\section{The following information relates to questions}
 43-48John Martinson, CFA, is an equity analyst with a large pension fund. His supervisor, Linda Packard, asks him to write a report on Karp Inc. Karp prepares its financial statements in accordance with US GAAP. Packard is particularly interested in the effects of the company's use of the LIFO method to account for its inventory. For this purpose, Martinson collects the financial data presented in Exhibits 1 and 2. Exhibit 1: Balance Sheet Information (US\$ Millions)

\begin{center}
\begin{tabular}{|c|c|c|}
\hline
As of 31 December & 2018 & 2017 \\
\hline
Cash and cash equivalents & 172 & 157 \\
\hline
Accounts receivable & 626 & 458 \\
\hline
Inventories & 620 & 539 \\
\hline
Other current assets & 125 & 65 \\
\hline
Total current assets & 1,543 & 1,219 \\
\hline
Property and equipment, net & 3,035 & 2,972 \\
\hline
Total assets & 4,578 & 4,191 \\
\hline
Total current liabilities & 1,495 & 1,395 \\
\hline
Long-term debt & 644 & 604 \\
\hline
Total liabilities & 2,139 & 1,999 \\
\hline
Common stock and paid in capital & 1,652 & 1,652 \\
\hline
Retained earnings & 787 & 540 \\
\hline
Total shareholders' equity & 2,439 & 2,192 \\
\hline
Total liabilities and shareholders' equity & 4,578 & 4,191 \\
\hline
\end{tabular}
\end{center}

\section{Exhibit 2: Income Statement Information (US\$ Millions)}
\begin{center}
\begin{tabular}{lcc}
\hline
For the Year Ended 31 December & $\mathbf{2 0 1 8}$ & $\mathbf{2 0 1 7}$ \\
\hline
Sales & 4,346 & 4,161 \\
Cost of goods sold & 2,211 & 2,147 \\
Depreciation and amortisation expense & 139 & 119 \\
Selling, general, and administrative expense & 1,656 & 1,637 \\
Interest expense & 31 & 18 \\
Income tax expense & 62 & 48 \\
Net income & 247 & 192 \\
\hline
\end{tabular}
\end{center}

Martinson finds the following information in the notes to the financial statements:

\begin{itemize}
  \item The LIFO reserves as of 31 December 2018 and 2017 are $\$ 155$ million and $\$ 117$ million respectively, and

  \item The effective income tax rate applicable to Karp for 2018 and earlier periods is 20 percent.

\end{itemize}

\begin{enumerate}
  \setcounter{enumi}{42}
  \item If Karp had used FIFO instead of LIFO, the amount of inventory reported as of 31 December 2018 would have been closest to:
\end{enumerate}

A. $\$ 465$ million.

B. $\$ 658$ million.

C. $\$ 775$ million. 44. If Karp had used FIFO instead of LIFO, the amount of cost of goods sold reported by Karp for the year ended 31 December 2018 would have been closest to:
A. $\$ 2,056$ million.
B. $\$ 2,173$ million.
C. $\$ 2,249$ million.

\begin{enumerate}
  \setcounter{enumi}{44}
  \item If Karp had used FIFO instead of LIFO, its reported net income for the year ended 31 December 2018 would have been higher by an amount closest to:
A. $\$ 30$ million.
B. $\$ 38$ million.
C. $\$ 155$ million.

  \item If Karp had used FIFO instead of LIFO, Karp's retained earnings as of 31 December 2018 would have been higher by an amount closest to:
A. $\$ 117$ million
B. $\$ 124$ million.
C. $\$ 155$ million.

  \item If Karp had used FIFO instead of LIFO, which of the following ratios computed as of 31 December 2018 would most likely have been lower?
A. Cash ratio.
B. Current ratio.
C. Gross profit margin.

  \item If Karp had used FIFO instead of LIFO, its debt to equity ratio computed as of 31 December 2018 would have:
A. increased.
B. decreased.
C. remained unchanged.

  \item If inventory unit costs are increasing from period-to-period, a LIFO liquidation is most likely to result in an increase in:
A. gross profit.
B. LIFO reserve.
C. inventory carrying amounts.

  \item Carrying inventory at a value above its historical cost would most likely be permitted if:
A. the inventory was held by a producer of agricultural products.
B. financial statements were prepared using US GAAP.
C. the change resulted from a reversal of a previous write-down. 51. Eric's Used Book Store prepares its financial statements in accordance with IFRS. Inventory was purchased for $\pounds 1$ million and later marked down to $\pounds 550,000$. One of the books, however, was later discovered to be a rare collectible item, and the inventory is now worth an estimated $\pounds 3$ million. The inventory is most likely reported on the balance sheet at:
A. $\pounds 550,000$.
B. $\pounds 1,000,000$.
C. $\pounds 3,000,000$.

  \item Fernando's Pasta purchased inventory and later wrote it down. The current net realisable value is higher than the value when written down. Fernando's inventory balance will most likely be:

\end{enumerate}

A. higher if it complies with IFRS.

B. higher if it complies with US GAAP.

C. the same under US GAAP and IFRS.

\begin{enumerate}
  \setcounter{enumi}{52}
  \item A write down of the value of inventory to its net realizable value will have a positive effect on the:
A. balance sheet.
B. income statement.
C. inventory turnover ratio.

  \item Company A adheres to US GAAP and Company B adheres to IFRS. Which of the following is most likely to be disclosed on the financial statements of both companies?

\end{enumerate}

A. Any material income resulting from the liquidation of LIFO inventory

B. The amount of inventories recognized as an expense during the period

C. The circumstances that led to the reversal of a write down of inventories

\begin{enumerate}
  \setcounter{enumi}{54}
  \item Which of the following most likely signals that a manufacturing company expects demand for its product to increase?
\end{enumerate}

A. Finished goods inventory growth rate higher than the sales growth rate

B. Higher unit volumes of work in progress and raw material inventories

C. Substantially higher finished goods, with lower raw materials and work-in-process

\section{SOLUTIONS}
\begin{enumerate}
  \item C is correct. Transportation costs incurred to ship inventory to customers are an expense and may not be capitalized in inventory. (Transportation costs incurred to bring inventory to the business location can be capitalized in inventory.)
\end{enumerate}

Storage costs required as part of production, as well as costs incurred as a result of normal waste of materials, can be capitalized in inventory. (Costs incurred as a result of abnormal waste must be expensed.)

\begin{enumerate}
  \setcounter{enumi}{1}
  \item B is correct. Inventory expense includes costs of purchase, costs of conversion, and other costs incurred in bringing the inventories to their present location and condition. It does not include storage costs not required as part of production.

  \item $\mathrm{C}$ is correct. The storage costs for inventory awaiting shipment to customers are not costs of purchase, costs of conversion, or other costs incurred in bringing the inventories to their present location and condition and are not included in inventory. The storage costs for the chocolate liquor occur during the production process and are thus part of the conversion costs. Excise taxes are part of the purchase cost.

  \item $\mathrm{C}$ is correct. The carrying amount of inventories under FIFO will more closely reflect current replacement values because inventories are assumed to consist of the most recently purchased items. FIFO is an acceptable, but not preferred, method under IFRS. Weighted average cost, not FIFO, is the cost formula that allocates the same per unit cost to both cost of sales and inventory.

  \item $\quad B$ is correct. Inventory turnover $=$ Cost of sales/Average inventor $y=$ $41,043 / 7,569.5=5.42$. Average inventory is $(8,100+7,039) / 2=7,569.5$.

  \item B is correct. For comparative purposes, the choice of a competitor that reports under IFRS is requested because LIFO is permitted under US GAAP.

  \item A is correct. The carrying amount of the ending inventory may differ because the perpetual system will apply LIFO continuously throughout the year, liquidating layers as sales are made. Under the periodic system, the sales will start from the last layer in the year. Under FIFO, the sales will occur from the same layers regardless of whether a perpetual or periodic system is used. Specific identification identifies the actual products sold and remaining in inventory, and there will be no difference under a perpetual or periodic system.

  \item B is correct. The cost of sales is closest to CHF 4,550. Under FIFO, the inventory acquired first is sold first. Using Exhibit 4, a total of 310 cartons were available for sale $(100+40+70+100)$ and 185 cartons were sold $(50+100+35)$, leaving 125 in ending inventory. The FIFO cost would be as follows:

\end{enumerate}

100 (beginning inventory) $\times 22=2,200$

$40(4$ February 2009) $\times 25=1,000$

$45(23$ July 2009$) \times 30=1,350$

Cost of sales $=2,200+1,000+1,350=\operatorname{CHF} 4,550$

\begin{enumerate}
  \setcounter{enumi}{8}
  \item A is correct. Gross profit will most likely increase by CHF 7,775. The net realisable value has increased and now exceeds the cost. The write-down from 2017 can be reversed. The write-down in 2017 was 9,256 [92,560 × $(4.05-3.95)]$. IFRS require the reversal of any write-downs for a subsequent increase in value of inventory previously written down. The reversal is limited to the lower of the subsequent increase or the original write-down. Only 77,750 kilograms remain in inventory; the reversal is $77,750 \times(4.05-3.95)=7,775$. The amount of any reversal of a write-down is recognised as a reduction in cost of sales. This reduction results in an increase in gross profit.

  \item $C$ is correct. Using the FIFO method to value inventories when prices are rising will allocate more of the cost of goods available for sale to ending inventories (the most recent purchases, which are at higher costs, are assumed to remain in inventory) and less to cost of sales (the oldest purchases, which are at lower costs, are assumed to be sold first).

  \item B is correct. Cinnamon uses the weighted average cost method, so in 2018, 5,000 units of inventory were 2017 units at $€ 10$ each and 50,000 were 2008 purchases at $€ 11$. The weighted average cost of inventory during 2008 was thus $(5,000 \times 10)$ $+(50,000 \times 11)=50,000+550,000=€ 600,000$, and the weighted average cost was approximately $€ 10.91=€ 600,000 / 55,000$. Cost of sales was $€ 10.91 \times 45,000$, which is approximately $€ 490,950$.

  \item $C$ is correct. Zimt uses the FIFO method, and thus the first 5,000 units sold in 2018 depleted the 2017 inventory. Of the inventory purchased in 2018, 40,000 units were sold and 10,000 remain, valued at $€ 11$ each, for a total of $€ 110,000$.

  \item A is correct. Zimt uses the FIFO method, so its cost of sales represents units purchased at a (no longer available) lower price. Nutmeg uses the LIFO method, so its cost of sales is approximately equal to the current replacement cost of inventory.

  \item B is correct. Nutmeg uses the LIFO method, and thus some of the inventory on the balance sheet was purchased at a (no longer available) lower price. Zimt uses the FIFO method, so the carrying value on the balance sheet represents the most recently purchased units and thus approximates the current replacement cost.

  \item B is correct. In a declining price environment, the newest inventory is the lowest-cost inventory. In such circumstances, using the LIFO method (selling the newer, cheaper inventory first) will result in lower cost of sales and higher profit.

  \item B is correct. In a rising price environment, inventory balances will be higher for the company using the FIFO method. Accounts payable are based on amounts due to suppliers, not the amounts accrued based on inventory accounting.

  \item $\mathrm{C}$ is correct. The write-down reduced the value of inventory and increased cost of sales in 2017. The higher numerator and lower denominator mean that the inventory turnover ratio as reported was too high. Gross margin and the current ratio were both too low.

  \item A is correct. The reversal of the write-down shifted cost of sales from 2018 to 2017. The 2017 cost of sales was higher because of the write-down, and the 2018 cost of sales was lower because of the reversal of the write-down. As a result, the reported 2018 profits were overstated. Inventory balance in 2018 is the same because the write-down and reversal cancel each other out. Cash flow from operations is not affected by the non-cash write-down, but the higher profits in 2018 likely resulted in higher taxes and thus lower cash flow from operations.

  \item B is correct. LIFO will result in lower inventory and higher cost of sales. Gross margin (a profitability ratio) will be lower, the current ratio (a liquidity ratio) will be lower, and inventory turnover (an efficiency ratio) will be higher. 20. A is correct. LIFO will result in lower inventory and higher cost of sales in periods of rising costs compared to FIFO. Consequently, LIFO results in a lower gross profit margin than FIFO.

  \item B is correct. The LIFO method increases cost of sales, thus reducing profits and the taxes thereon.

  \item A is correct. US GAAP do not permit inventory write-downs to be reversed.

  \item A is correct. A perpetual inventory system updates inventory values and quantities and cost of goods sold continuously to reflect purchases and sales. The ending inventory of 800 units consists of 300 units at $\$ 20$ and 500 units at $\$ 17$.

\end{enumerate}

$(300 \times \$ 20)+(500 \times \$ 17)=\$ 14,500$

\begin{enumerate}
  \setcounter{enumi}{23}
  \item A is correct. In an environment with falling inventory costs and declining inventory levels, periodic LIFO will result in a higher ending inventory value and lower cost of goods sold versus perpetual LIFO and perpetual FIFO methods. This results in a lower inventory turnover ratio, which is calculated as follows:
\end{enumerate}

Inventory turnover ratio $=$ Cost of goods sold/Ending inventory

The inventory turnover ratio using periodic LIFO is $\$ 39,000 / \$ 16,000=244 \%$ or 2.44 times.

The inventory turnover ratio using perpetual LIFO is $279 \%$ or 2.79 times, which is provided in Exhibit 1 (= 40,500/14,500 from previous question).

The inventory turnover for perpetual FIFO is $\$ 41,400 / \$ 13,600=304 \%$ or 3.04 times.

\begin{enumerate}
  \setcounter{enumi}{24}
  \item B is correct. During a period of rising inventory costs, a company using the FIFO method will allocate a lower amount to cost of goods sold and a higher amount to ending inventory as compared with the LIFO method. The inventory turnover ratio is the ratio of cost of sales to ending inventory. A company using the FIFO method will produce a lower inventory turnover ratio as compared with the LIFO method. The current ratio (current assets/current liabilities) and the gross profit margin [gross profit/sales $=$ (sales less cost of goods sold)/sales] will be higher under the FIFO method than under the LIFO method in periods of rising inventory unit costs.

  \item B is correct. During a period of rising inventory prices, a company using the LIFO method will have higher cost of cost of goods sold and lower inventory compared with a company using the FIFO method. The inventory turnover ratio will be higher for the company using the LIFO method, thus making it appear more efficient. Current assets and gross profit margin will be lower for the company using the LIFO method, thus making it appear less liquid and less profitable.

  \item B is correct. In an environment of declining inventory unit costs and constant or increasing inventory quantities, FIFO (in comparison with weighted average cost or LIFO) will have higher cost of goods sold and lower net income and inventory. Because both inventory and net income are lower, total equity is lower, resulting in a higher debt-to-equity ratio.

  \item B is correct. Crux's adjusted inventory turnover ratio must be computed using cost of goods sold (COGS) under FIFO and excluding charges for increases in valuation allowances.

\end{enumerate}

COGS $($ adjusted $)=$ COGS $($ LIFO method $)-$ Charges included in cost of goods sold for inventory write-downs - Change in LIFO reserve

$=\$ 3,120$ million -13 million $-(55$ million -72 million $)$

$=\$ 3,124$ million

Note: Minus the change in LIFO reserve is equivalent to plus the decrease in LIFO reserve. The adjusted inventory turnover ratio is computed using average inventory under FIFO.

Ending inventory $($ FIFO $)=$ Ending inventory $($ LIFO $)+$ LIFO reserve

Ending inventory $2018($ FIFO) $=\$ 480+55=\$ 535$

Ending inventory $2017($ FIFO) $=\$ 465+72=\$ 537$

Average inventory $=(\$ 535+537) / 2=\$ 536$

Therefore, adjusted inventory turnover ratio equals:

Inventory turnover ratio $=\mathrm{COGS} /$ Average inventory $=\$ 3,124 / \$ 536=5.83$

\begin{enumerate}
  \setcounter{enumi}{28}
  \item B is correct. Rolby's adjusted net profit margin must be computed using net income (NI) under FIFO and excluding charges for increases in valuation allowances.
\end{enumerate}

NI (adjusted) = NI (FIFO method) + Charges, included in cost of goods sold for inventory write-downs, after tax

$$
\begin{aligned}
& =\$ 327 \text { million }+15 \text { million } \times(1-30 \%) \\
& =\$ 337.5 \text { million }
\end{aligned}
$$

Therefore, adjusted net profit margin equals:

Net profit margin $=\mathrm{NI} /$ Revenues $=\$ 337.5 / \$ 5,442=6.20 \%$

\begin{enumerate}
  \setcounter{enumi}{29}
  \item A is correct. Mikko's adjusted debt-to-equity ratio is lower because the debt (numerator) is unchanged and the adjusted shareholders' equity (denominator) is higher. The adjusted shareholders' equity corresponds to shareholders' equity under FIFO, excluding charges for increases in valuation allowances. Therefore, adjusted shareholders' equity is higher than reported (unadjusted) shareholders' equity.

  \item C is correct. Mikko's and Crux's gross margin ratios would better reflect the current gross margin of the industry than Rolby because both use LIFO. LIFO recognizes as cost of goods sold the cost of the most recently purchased units, therefore, it better reflects replacement cost. However, Mikko's gross margin ratio best reflects the current gross margin of the industry because Crux's LIFO reserve is decreasing. This could reflect a LIFO liquidation by Crux which would distort gross profit margin.

  \item B is correct. The FIFO method shows a higher gross profit margin than the LIFO method in an inflationary scenario, because FIFO allocates to cost of goods sold the cost of the oldest units available for sale. In an inflationary environment, these units are the ones with the lowest cost.

  \item A is correct. An inventory write-down increases cost of sales and reduces profit and reduces the carrying value of inventory and assets. This has a negative effect on profitability and solvency ratios. However, activity ratios appear positively affected by a write-down because the asset base, whether total assets or inventory (denominator), is reduced. The numerator, sales, in total asset turnover is unchanged, and the numerator, cost of sales, in inventory turnover is increased. Thus, turnover ratios are higher and appear more favorable as the result of the write-down.

  \item B is correct. Finished goods least accurately reflect current prices because some of the finished goods are valued under the "last-in, first-out" ("LIFO") basis. The costs of the newest units available for sale are allocated to cost of goods sold, leaving the oldest units (at lower costs) in inventory. ZP values raw materials and work in process using the weighted average cost method. While not fully reflecting current prices, some inflationary effect will be included in the inventory values.

  \item $\mathrm{C}$ is correct. FIFO inventory $=$ Reported inventory + LIFO reserve $=¥ 608,572+$ $10,120=¥ 618,692$. The LIFO reserve is disclosed in Note 2 of the notes to consolidated financial statements.

  \item A is correct. The inventory turnover ratio would be lower. The average inventory would be higher under FIFO and cost of products sold would be lower by the increase in LIFO reserve. LIFO is not permitted under IFRS.

\end{enumerate}

Inventory turnover ratio $=$ Cost of products sold $\div$ Average inventory

2018 inventory turnover ratio as reported $=10.63$

$=¥ 5,822,805 /[(608,572+486,465) / 2]$.

2018 inventory turnover ratio adjusted to FIFO as necessary $=10.34$

$=[¥ 5,822,805-(19,660-10,120)] /[(608,572+10,120+486,465+19,660) / 2]$.

\begin{enumerate}
  \setcounter{enumi}{36}
  \item A is correct. No LIFO liquidation occurred during 2018; the LIFO reserve increased from $¥ 10,120$ million in 2008 to $¥ 19,660$ million in 2018. Management stated in the MD\&A that the decrease in inventories reflected the impacts of decreased sales volumes and fluctuations in foreign currency translation rates.

  \item $\mathrm{C}$ is correct. Finished goods and raw materials inventories are lower in 2018 when compared to 2017. Reduced levels of inventory typically indicate an anticipated business contraction.

  \item B is correct. The decrease in LIFO inventory in 2018 would typically indicate that more inventory units were sold than produced or purchased. Accordingly, one would expect a liquidation of some of the older LIFO layers and the LIFO reserve to decrease. In actuality, the LIFO reserve increased from $¥ 10,120$ million in 2017 to $¥ 19,660$ million in 2018. This is not to be expected and is likely caused by the increase in prices of raw materials, other production materials, and parts of foreign currencies as noted in the MD\&A. An analyst should seek to confirm this explanation.

  \item B is correct. If prices have been decreasing, write-downs under FIFO are least likely to have a significant effect because the inventory is valued at closer to the new, lower prices. Typically, inventories valued using LIFO are less likely to incur inventory write-downs than inventories valued using weighted average cost or FIFO. Under LIFO, the oldest costs are reflected in the inventory carrying value on the balance sheet. Given increasing inventory costs, the inventory carrying values under the LIFO method are already conservatively presented at the oldest and lowest costs. Thus, it is far less likely that inventory write-downs will occur under LIFO; and if a write-down does occur, it is likely to be of a lesser magnitude.

  \item A is correct. LIFO reserve is the FIFO inventory value less the LIFO inventory value. In periods of rising inventory unit costs, the carrying amount of inventory under FIFO will always exceed the carrying amount of inventory under LIFO. The LIFO reserve may increase over time as a result of the increasing difference between the older costs used to value inventory under LIFO and the more recent costs used to value inventory under FIFO. When inventory unit levels are decreasing, the company will experience a LIFO liquidation, reducing the LIFO reserve.

  \item B is correct. The adjusted COGS under the FIFO method is equal to COGS under the LIFO method less the increase in LIFO reserve:

\end{enumerate}

COGS $($ FIFO) $=$ COGS (LIFO) - Increase in LIFO reserve

$\operatorname{COGS}(F I F O)=\pounds 50,800-(\pounds 4,320-\pounds 2,600)$

COGS $($ FIFO $)=\pounds 49,080$

\begin{enumerate}
  \setcounter{enumi}{42}
  \item $\mathrm{C}$ is correct. Karp's inventory under FIFO equals Karp's inventory under LIFO plus the LIFO reserve. Therefore, as of 31 December 2018, Karp's inventory under FIFO equals:
\end{enumerate}

Inventory (FIFO method $)=$ Inventory $($ LIFO method $)+$ LIFO reserve

$=\$ 620$ million +155 million

$=\$ 775$ million

\begin{enumerate}
  \setcounter{enumi}{43}
  \item B is correct. Karp's cost of goods sold (COGS) under FIFO equals Karp's cost of goods sold under LIFO minus the increase in the LIFO reserve. Therefore, for the year ended 31 December 2018, Karp's cost of goods sold under FIFO equals:
\end{enumerate}

COGS (FIFO method) = COGS (LIFO method) - Increase in LIFO reserve

$=\$ 2,211$ million $-(155$ million -117 million $)$

$=\$ 2,173$ million

\begin{enumerate}
  \setcounter{enumi}{44}
  \item A is correct. Karp's net income (NI) under FIFO equals Karp's net income under LIFO plus the after-tax increase in the LIFO reserve. For the year ended 31 December 2018, Karp's net income under FIFO equals:
\end{enumerate}

NI $($ FIFO method $)=$ NI $($ LIFO method $)+$ Increase in LIFO reserve $\times(1-$ Tax rate)

$=\$ 247$ million +38 million $\times(1-20 \%)$

$=\$ 277.4$ million

Therefore, the increase in net income is:

Increase in NI = NI (FIFO method) - NI (LIFO method)

$=\$ 277$ million -247 million

$=\$ 30.4$ million 46. B is correct. Karp's retained earnings (RE) under FIFO equals Karp's retained earnings under LIFO plus the after-tax LIFO reserve. Therefore, for the year ended 31 December 2018, Karp's retained earnings under FIFO equals:

RE $($ FIFO method $)=$ RE $($ LIFO method $)+$ LIFO reserve $\times(1-$ Tax rate $)$

$=\$ 787$ million +155 million $\times(1-20 \%)$

$=\$ 911$ million

Therefore, the increase in retained earnings is:

Increase in RE $=$ RE (FIFO method) - RE (LIFO method)

$=\$ 911$ million -787 million

$=\$ 124$ million

\begin{enumerate}
  \setcounter{enumi}{46}
  \item A is correct. The cash ratio (cash and cash equivalents $\div$ current liabilities) would be lower because cash would have been less under FIFO. Karp's income before taxes would have been higher under FIFO, and consequently taxes paid by Karp would have also been higher and cash would have been lower. There is no impact on current liabilities. Both Karp's current ratio and gross profit margin would have been higher if FIFO had been used. The current ratio would have been higher because inventory under FIFO increases by a larger amount than the cash decreases for taxes paid. Because the cost of goods sold under FIFO is lower than under LIFO, the gross profit margin would have been higher.

  \item B is correct. If Karp had used FIFO instead of LIFO, the debt-to-equity ratio would have decreased. No change in debt would have occurred, but shareholders' equity would have increased as a result of higher retained earnings.

  \item A is correct. When the number of units sold exceeds the number of units purchased, a company using LIFO will experience a LIFO liquidation. If inventory unit costs have been rising from period-to-period and a LIFO liquidation occurs, it will produce an increase in gross profit as a result of the lower inventory carrying amounts of the liquidated units (lower cost per unit of the liquidated units).

  \item A is correct. IFRS allow the inventories of producers and dealers of agricultural and forest products, agricultural produce after harvest, and minerals and mineral products to be carried at net realisable value even if above historical cost. (US GAAP treatment is similar.)

  \item B is correct. Under IFRS, the reversal of write-downs is required if net realisable value increases. The inventory will be reported on the balance sheet at $\pounds 1,000,000$. The inventory is reported at the lower of cost or net realisable value. Under US GAAP, inventory is carried at the lower of cost or market value. After a write-down, a new cost basis is determined and additional revisions may only reduce the value further. The reversal of write-downs is not permitted.

  \item A is correct. IFRS require the reversal of inventory write-downs if net realisable values increase; US GAAP do not permit the reversal of write-downs.

  \item $C$ is correct. Activity ratios (for example, inventory turnover and total asset turnover) will be positively affected by a write down to net realizable value because the asset base (denominator) is reduced. On the balance sheet, the inventory carrying amount is written down to its net realizable value and the loss in value (expense) is generally reflected on the income statement in cost of goods sold, thus reducing gross profit, operating profit, and net income. 54. B is correct. Both US GAAP and IFRS require disclosure of the amount of inventories recognized as an expense during the period. Only US GAAP allows the LIFO method and requires disclosure of any material amount of income resulting from the liquidation of LIFO inventory. US GAAP does not permit the reversal of prior-year inventory write downs.

  \item B is correct. A significant increase (attributable to increases in unit volume rather than increases in unit cost) in raw materials and/or work-in-progress inventories may signal that the company expects an increase in demand for its products. If the growth of finished goods inventories is greater than the growth of sales, it could indicate a decrease in demand and a decrease in future earnings. A substantial increase in finished goods inventories while raw materials and work-in-progress inventories are declining may signal a decrease in demand for the company's products.

\end{enumerate}

\section*{LEARNING MODULE 
 6 }
\section{Long-Lived Assets}
Elaine Henry, PhD, CFA, is at Stevens Institute of Technology (USA). Elizabeth A. Gordon, PhD, MBA, CPA, is at Temple University (USA).

\section{LEARNING OUTCOME}
\begin{center}
\begin{tabular}{|c|c|}
\hline
Mastery & The candidate should be able to: \\
\hline
 & $\begin{array}{l}\text { identify and contrast costs that are capitalised and costs that are } \\ \text { expensed in the period in which they are incurred }\end{array}$ \\
\hline
 & $\begin{array}{l}\text { compare the financial reporting of the following types of intangible } \\ \text { assets: purchased, internally developed, acquired in a business } \\ \text { combination }\end{array}$ \\
\hline
 & $\begin{array}{l}\text { explain and evaluate how capitalising versus expensing costs in the } \\ \text { period in which they are incurred affects financial statements and } \\ \text { ratios }\end{array}$ \\
\hline
 & $\begin{array}{l}\text { describe the different depreciation methods for property, plant, and } \\ \text { equipment and calculate depreciation expense }\end{array}$ \\
\hline
 & $\begin{array}{l}\text { describe how the choice of depreciation method and assumptions } \\ \text { concerning useful life and residual value affect depreciation expense, } \\ \text { financial statements, and ratios }\end{array}$ \\
\hline
 & $\begin{array}{l}\text { explain and evaluate how impairment, revaluation, and } \\ \text { derecognition of property, plant, and equipment and intangible } \\ \text { assets affect financial statements and ratios }\end{array}$ \\
\hline
 & $\begin{array}{l}\text { describe the different amortisation methods for intangible assets } \\ \text { with finite lives and calculate amortisation expense }\end{array}$ \\
\hline
 & $\begin{array}{l}\text { describe how the choice of amortisation method and assumptions } \\ \text { concerning useful life and residual value affect amortisation expense, } \\ \text { financial statements, and ratios }\end{array}$ \\
\hline
 & describe the revaluation model \\
\hline
 & $\begin{array}{l}\text { explain the impairment of property, plant, and equipment and } \\ \text { intangible assets }\end{array}$ \\
\hline
 & $\begin{array}{l}\text { explain the derecognition of property, plant, and equipment and } \\ \text { intangible assets }\end{array}$ \\
\hline
\end{tabular}
\end{center}

Note: Changes in accounting standards as well as new rulings and/or pronouncements issued after the publication of the readings on financial reporting and analysis may cause some of the information in these readings to become dated. Candidates are not responsible for anything that occurs after the readings were published. In addition, candidates are expected to be familiar with the analytical frameworks contained in the readings, as well as the implications of alternative accounting methods for financial analysis and valuation discussed in the readings. Candidates are also responsible for the content of accounting standards, but not for the actual reference numbers. Finally, candidates should be aware that certain ratios may be defined and calculated differently. When alternative ratio definitions exist and no specific definition is given, candidates should use the ratio definitions emphasized in the readings.

\section{LEARNING OUTCOME}
\begin{center}
\begin{tabular}{c|l}
Mastery & The candidate should be able to: \\
$\square$ & $\begin{array}{l}\text { describe the financial statement presentation of and disclosures } \\ \text { relating to property, plant, and equipment and intangible assets } \\ \text { analyze and interpret financial statement disclosures regarding } \\ \text { property, plant, and equipment and intangible assets } \\ \text { compare the financial reporting of investment property with that of } \\ \text { property, plant, and equipment }\end{array}$ \\
$\square$ &  \\
\end{tabular}
\end{center}

\begin{center}
\includegraphics[max width=\textwidth]{2023_05_04_b5cfa4f1bc883752f121g-340}
\end{center}

\section{INTRODUCTION}
Long-lived assets, also referred to as non-current assets or long-term assets, are assets that are expected to provide economic benefits over a future period of time, typically greater than one year. ${ }^{1}$ Long-lived assets may be tangible, intangible, or financial assets. Examples of long-lived tangible assets, typically referred to as property, plant, and equipment and sometimes as fixed assets, include land, buildings, furniture and fixtures, machinery and equipment, and vehicles; examples of long-lived intangible assets (assets lacking physical substance) include patents and trademarks; and examples of long-lived financial assets include investments in equity or debt securities issued by other entities. The scope of this reading is limited to long-lived tangible and intangible assets (hereafter, referred to for simplicity as long-lived assets).

The first issue in accounting for a long-lived asset is determining its cost at acquisition. The second issue is how to allocate the cost to expense over time. The costs of most long-lived assets are capitalised and then allocated as expenses in the profit or loss (income) statement over the period of time during which they are expected to provide economic benefits. The two main types of long-lived assets with costs that are typically not allocated over time are land, which is not depreciated, and those intangible assets with indefinite useful lives. Additional issues that arise are the treatment of subsequent costs incurred related to the asset, the use of the cost model versus the revaluation model, unexpected declines in the value of the asset, classification of the asset with respect to intent (for example, held for use or held for sale), and the derecognition of the asset.

This reading is organised as follows. Section 2 describes and illustrates accounting for the acquisition of long-lived assets, with particular attention to the impact of capitalizing versus expensing expenditures. Section 3 describes the allocation of the costs of long-lived assets over their useful lives. Section 4 discusses the revaluation model that is based on changes in the fair value of an asset. Section 5 covers the concepts of impairment (unexpected decline in the value of an asset). Section 6 describes accounting for the derecognition of long-lived assets. Section 7 describes financial statement presentation, disclosures, and analysis of long-lived assets. Section 8 discusses differences in financial reporting of investment property compared with property, plant, and equipment. A summary is followed by practice problems.

1 In some industries, inventory is held longer than one year but is nonetheless reported as a current asset.

\section{ACQUISITION OF PROPERTY, PLANT AND EQUIPMENT}
\begin{center}
\includegraphics[max width=\textwidth]{2023_05_04_b5cfa4f1bc883752f121g-341}
\end{center}

identify and contrast costs that are capitalised and costs that are expensed in the period in which they are incurred

\section{Acquisition of Long-Lived Assets}
Upon acquisition, property, plant, and equipment (tangible assets with an economic life of longer than one year and intended to be held for the company's own use) are recorded on the balance sheet at cost, which is typically the same as their fair value. ${ }^{2}$ Accounting for an intangible asset depends on how the asset is acquired. If several assets are acquired as part of a group, the purchase price is allocated to each asset on the basis of its fair value. An asset's cost potentially includes expenditures additional to the purchase price.

A key concept in accounting for expenditures related to long-lived assets is whether and when such expenditures are capitalised (i.e., included in the asset shown on the balance sheet) versus expensed (i.e., treated as an expense of the period on the income statement). After examining the specific treatment of certain expenditures, we will consider the general financial statement impact of capitalising versus expensing and two analytical issues related to the decision-namely, the effects on an individual company's trend analysis and on comparability across companies.

\section{Property, Plant, and Equipment}
This section primarily discusses the accounting treatment for the acquisition of long-lived tangible assets (property, plant, and equipment) through purchase. Assets can be acquired by methods other than purchase. ${ }^{3}$ When an asset is exchanged for another asset, the asset acquired is recorded at fair value if reliable measures of fair value exist. Fair value is the fair value of the asset given up unless the fair value of the asset acquired is more clearly evident. If there is no reliable measure of fair value, the acquired asset is measured at the carrying amount of the asset given up. In this case, the carrying amount of the assets is unchanged, and no gain or loss is reported.

Typically, accounting for the exchange involves removing the carrying amount of the asset given up, adding a fair value for the asset acquired, and reporting any difference between the carrying amount and the fair value as a gain or loss. A gain would be reported when the fair value used for the newly acquired asset exceeds the carrying amount of the asset given up. A loss would be reported when the fair value used for the newly acquired asset is less than the carrying amount of the asset given up.

When property, plant, or equipment is purchased, the buyer records the asset at cost. In addition to the purchase price, the buyer also includes, as part of the cost of an asset, all the expenditures necessary to get the asset ready for its intended use.

2 Fair value is defined in International Financial Reporting Standards (IFRS) and under US generally accepted accounting principles (US GAAP) in the Financial Accounting Standards Board (FASB) Accounting Standards Codification (ASC) as "the price that would be received to sell an asset or paid to transfer a liability in an orderly transaction between market participants at the measurement date." [IFRS 13 and FASB ASC Topic 820]

3 IAS 16 Property, Plant and Equipment, paragraphs 24-26 [Measurement of Cost]; IAS 38 Intangible Assets, paragraphs 45-47 [Exchange of Assets]; and FASB ASC Section 845-10-30 [Nonmonetary Transactions - Overall - Initial Measurement]. For example, freight costs borne by the purchaser to get the asset to the purchaser's place of business and special installation and testing costs required to make the asset usable are included in the total cost of the asset.

Subsequent expenditures related to long-lived assets are included as part of the recorded value of the assets on the balance sheet (i.e., capitalised) if they are expected to provide benefits beyond one year in the future and are expensed if they are not expected to provide benefits in future periods. Expenditures that extend the original life of the asset are typically capitalised. Example 1 illustrates the difference between costs that are capitalised and costs that are expensed in a period.

\section{EXAMPLE 1}
\section{Acquisition of PPE}
Assume a (hypothetical) company, Trofferini S.A., incurred the following expenditures to purchase a towel and tissue roll machine: $€ 10,900$ purchase price including taxes, $€ 200$ for delivery of the machine, $€ 300$ for installation and testing of the machine, and $€ 100$ to train staff on maintaining the machine. In addition, the company paid a construction team $€ 350$ to reinforce the factory floor and ceiling joists to accommodate the machine's weight. The company also paid $€ 1,500$ to repair the factory roof (a repair expected to extend the useful life of the factory by five years) and $€ 1,000$ to have the exterior of the factory and adjoining offices repainted for maintenance reasons. The repainting neither extends the life of factory and offices nor improves their usability.

\begin{enumerate}
  \item Which of these expenditures will be capitalised and which will be expensed?
\end{enumerate}

\section{Solution to 1:}
The company will capitalise as part of the cost of the machine all costs that are necessary to get the new machine ready for its intended use: $€ 10,900$ purchase price, $€ 200$ for delivery, $€ 300$ for installation and testing, and $€ 350$ to reinforce the factory floor and ceiling joists to accommodate the machine's weight (which was necessary to use the machine and does not increase the value of the factory). The $€ 100$ to train staff is not necessary to get the asset ready for its intended use and will be expensed.

The company will capitalise the expenditure of $€ 1,500$ to repair the factory roof because the repair is expected to extend the useful life of the factory. The company will expense the $€ 1,000$ to have the exterior of the factory and adjoining offices repainted because the painting does not extend the life or alter the productive capacity of the buildings.

\begin{enumerate}
  \setcounter{enumi}{1}
  \item How will the treatment of these expenditures affect the company's financial statements?
\end{enumerate}

\section{Solution to 2:}
The costs related to the machine that are capitalised- $€ 10,900$ purchase price, $€ 200$ for delivery, $€ 300$ for installation and testing, and $€ 350$ to prepare the factory-will increase the carrying amount of the machine asset as shown on the balance sheet and will be included as investing cash outflows. The item related to the factory that is capitalised-the $€ 1,500$ roof repairwill increase the carrying amount of the factory asset as shown on the balance sheet and is an investing cash outflow. The expenditures of $€ 100$ to train staff and $€ 1,000$ to paint are expensed in the period and will reduce the amount of income reported on the company's income statement (and thus reduce retained earnings on the balance sheet) and the operating cash flow.

Example 1 describes capitalising versus expensing in the context of purchasing property, plant, and equipment. When a company constructs an asset (or acquires an asset that requires a long period of time to get ready for its intended use), borrowing costs incurred directly related to the construction are generally capitalised. Constructing a building, whether for sale (in which case, the building is classified as inventory) or for the company's own use (in which case, the building is classified as a long-lived asset), typically requires a substantial amount of time. To finance construction, any borrowing costs incurred prior to the asset being ready for its intended use are capitalised as part of the cost of the asset. The company determines the interest rate to use on the basis of its existing borrowings or, if applicable, on a borrowing specifically incurred for constructing the asset. If a company takes out a loan specifically to construct a building, the interest cost on that loan during the time of construction would be capitalised as part of the building's cost. Under IFRS, but not under US GAAP, income earned on temporarily investing the borrowed monies decreases the amount of borrowing costs eligible for capitalisation.

Thus, a company's interest costs for a period are included either on the balance sheet (to the extent they are capitalised as part of an asset) or on the income statement (to the extent they are expensed). If the interest expenditure is incurred in connection with constructing an asset for the company's own use, the capitalised interest appears on the balance sheet as a part of the relevant long-lived asset (i.e., property, plant, and equipment). The capitalised interest is expensed over time as the property is depreciated and is thus part of subsequent years' depreciation expense rather than interest expense of the current period. If the interest expenditure is incurred in connection with constructing an asset to sell (for example, by a home builder), the capitalised interest appears on the company's balance sheet as part of inventory. The capitalised interest is expensed as part of the cost of goods sold when the asset is sold. Interest payments made prior to completion of construction that are capitalised are classified as an investing cash outflow. Expensed interest may be classified as an operating or financing cash outflow under IFRS and is classified as an operating cash outflow under US GAAP.

\section{EXAMPLE 2}
\section{Capitalised Borrowing Costs}
BILDA S.A., a hypothetical company, borrows $€ 1,000,000$ at an interest rate of 10 percent per year on 1 January 2010 to finance the construction of a factory that will have a useful life of 40 years. Construction is completed after two years, during which time the company earns $€ 20,000$ by temporarily investing the loan proceeds.

\begin{enumerate}
  \item What is the amount of interest that will be capitalised under IFRS, and how would that amount differ from the amount that would be capitalised under US GAAP?
\end{enumerate}

\section{Solution to 1:}
The total amount of interest paid on the loan during construction is $€ 200,000(=€ 1,000,000 \times 10 \% \times 2$ years). Under IFRS, the amount of borrowing cost eligible for capitalisation is reduced by the $€ 20,000$ interest income from temporarily investing the loan proceeds, so the amount to be capitalised is $€ 180,000$. Under US GAAP, the amount to be capitalised is $€ 200,000$.

\begin{enumerate}
  \setcounter{enumi}{1}
  \item Where will the capitalised borrowing cost appear on the company's financial statements?
\end{enumerate}

\section{Solution to 2:}
The capitalised borrowing costs will appear on the company's balance sheet as a component of property, plant, and equipment. In the years prior to completion of construction, the interest paid will appear on the statement of cash flows as an investment activity. Over time, as the property is depreciated, the capitalised interest component is part of subsequent years' depreciation expense on the company's income statement.

compare the financial reporting of the following types of intangible assets: purchased, internally developed, acquired in a business combination

Intangible assets are assets lacking physical substance. Intangible assets include items that involve exclusive rights, such as patents, copyrights, trademarks, and franchises. Under IFRS, identifiable intangible assets must meet three definitional criteria. They must be (1) identifiable (either capable of being separated from the entity or arising from contractual or legal rights), (2) under the control of the company, and (3) expected to generate future economic benefits. In addition, two recognition criteria must be met: (1) It is probable that the expected future economic benefits of the asset will flow to the company, and (2) the cost of the asset can be reliably measured. Goodwill, which is not considered an identifiable intangible asset, ${ }^{4}$ arises when one company purchases another and the acquisition price exceeds the fair value of the net identifiable assets (both the tangible assets and the identifiable intangible assets, minus liabilities) acquired.

Accounting for an intangible asset depends on how it is acquired. The following sections describe accounting for intangible assets obtained in three ways: purchased in situations other than business combinations, developed internally, and acquired in business combinations.

4 The IFRS definition of an intangible asset as an "identifiable non-monetary asset without physical substance" applies to intangible assets not specifically dealt with in standards other than IAS 38. The definition of intangible assets under US GAAP\_ "assets (other than financial assets) that lack physical substance"includes goodwill in the definition of an intangible asset.

\section{Intangible Assets Purchased in Situations Other Than Business}
\section{Combinations}
Intangible assets purchased in situations other than business combinations, such as buying a patent, are treated at acquisition the same as long-lived tangible assets; they are recorded at their fair value when acquired, which is assumed to be equivalent to the purchase price. If several intangible assets are acquired as part of a group, the purchase price is allocated to each asset on the basis of its fair value.

In deciding how to treat individual intangible assets for analytical purposes, analysts are particularly aware that companies must use a substantial amount of judgment and numerous assumptions to determine the fair value of individual intangible assets. For analysis, therefore, understanding the types of intangible assets acquired can often be more useful than focusing on the values assigned to the individual assets. In other words, an analyst would typically be more interested in understanding what assets a company acquired (for example, franchise rights) than in the precise portion of the purchase price a company allocated to each asset. Understanding the types of assets a company acquires can offer insights into the company's strategic direction and future operating potential.

\section{Intangible Assets Developed Internally}
In contrast with the treatment of construction costs of tangible assets, the costs to internally develop intangible assets are generally expensed when incurred. There are some situations, however, in which the costs incurred to internally develop an intangible asset are capitalised. The general analytical issues related to the capitalising-versus-expensing decision apply here-namely, comparability across companies and the effect on an individual company's trend analysis.

The general requirement that costs to internally develop intangible assets be expensed should be compared with capitalising the cost of acquiring intangible assets in situations other than business combinations. Because costs associated with internally developing intangible assets are usually expensed, a company that has internally developed such intangible assets as patents, copyrights, or brands through expenditures on R\&D or advertising will recognise a lower amount of assets than a company that has obtained intangible assets through external purchase. In addition, on the statement of cash flows, costs of internally developing intangible assets are classified as operating cash outflows whereas costs of acquiring intangible assets are classified as investing cash outflows. Differences in strategy (developing versus acquiring intangible assets) can thus impact financial ratios.

IFRS require that expenditures on research (or during the research phase of an internal project) be expensed rather than capitalised as an intangible asset. ${ }^{5}$ Research is defined as "original and planned investigation undertaken with the prospect of gaining new scientific or technical knowledge and understanding." ${ }^{6}$ The "research phase of an internal project" refers to the period during which a company cannot demonstrate that an intangible asset is being created-for example, the search for alternative materials or systems to use in a production process. In contrast with the treatment of research-phase expenditures, IFRS allow companies to recognise an intangible asset arising from development expenditures (or the development phase of an internal project) if certain criteria are met, including a demonstration of the technical feasibility of completing the intangible asset and the intent to use or sell the asset. Development is defined as "the application of research findings or other knowledge to a plan or design for the production of new or substantially improved materials, devices, products, processes, systems or services before the start of commercial production or use." ${ }^{.7}$

Generally, US GAAP require that both research and development costs be expensed as incurred but require capitalisation of certain costs related to software development. ${ }^{8}$ Costs incurred to develop a software product for sale are expensed until the product's technological feasibility is established and are capitalised thereafter. Similarly, companies expense costs related to the development of software for internal use until it is probable that the project will be completed and that the software will be used as intended. Thereafter, development costs are capitalised. The probability that the project will be completed is easier to demonstrate than is technological feasibility. The capitalised costs, related directly to developing software for sale or internal use, include the costs of employees who help build and test the software. The treatment of software development costs under US GAAP is similar to the treatment of all costs of internally developed intangible assets under IFRS.

\section{EXAMPLE 3}
\section{Software Development Costs}
Assume REH AG, a hypothetical company, incurs expenditures of $€ 1,000$ per month during the fiscal year ended 31 December 2019 to develop software for internal use. Under IFRS, the company must treat the expenditures as an expense until the software meets the criteria for recognition as an intangible asset, after which time the expenditures can be capitalised as an intangible asset.

\begin{enumerate}
  \item What is the accounting impact of the company being able to demonstrate that the software met the criteria for recognition as an intangible asset on 1 February versus 1 December?
\end{enumerate}

\section{Solution to 1:}
If the company is able to demonstrate that the software met the criteria for recognition as an intangible asset on 1 February, the company would recognise the $€ 1,000$ expended in January as an expense on the income statement for the fiscal year ended 31 December 2019. The other $€ 11,000$ of expenditures would be recognised as an intangible asset (on the balance sheet). Alternatively, if the company is not able to demonstrate that the software met the criteria for recognition as an intangible asset until 1 December, the company would recognise the $€ 11,000$ expended in January through November as an expense on the income statement for the fiscal year ended 31 December 2019, with the other $€ 1,000$ of expenditures recognised as an intangible asset.

7 IAS 38 Intangible Assets, paragraph 8 [Definitions].

8 FASB ASC Section 350-40-25 [Intangibles-Goodwill and Other - Internal-Use Software - Recognition] and FASB ASC Section 985-20-25 [Software - Costs of Software to be Sold, Leased, or Marketed Recognition] specify US GAAP accounting for software development costs for software for internal use and for software to be sold, respectively. 2. How would the treatment of expenditures differ if the company reported under US GAAP and it had established in 2018 that the project was likely to be completed and the software used to perform the function intended?

\section{Solution to 2:}
Under US GAAP, the company would capitalise the entire $€ 12,000$ spent to develop software for internal use.

\section{Intangible Assets Acquired in a Business Combination}
When one company acquires another company, the transaction is accounted for using the acquisition method of accounting. ${ }^{9}$ Under the acquisition method, the company identified as the acquirer allocates the purchase price to each asset acquired (and each liability assumed) on the basis of its fair value. If the purchase price exceeds the sum of the amounts that can be allocated to individual identifiable assets and liabilities, the excess is recorded as goodwill. Goodwill cannot be identified separately from the business as a whole.

Under IFRS, the acquired individual assets include identifiable intangible assets that meet the definitional and recognition criteria. ${ }^{10}$ Otherwise, if the item is acquired in a business combination and cannot be recognised as a tangible or identifiable intangible asset, it is recognised as goodwill. Under US GAAP, there are two criteria to judge whether an intangible asset acquired in a business combination should be recognised separately from goodwill: The asset must be either an item arising from contractual or legal rights or an item that can be separated from the acquired company. Examples of intangible assets treated separately from goodwill include the intangible assets previously mentioned that involve exclusive rights (patents, copyrights, franchises, licenses), as well as such items as internet domain names and video and audiovisual materials.

Exhibit 1 describes how AB InBev allocated the $\$ 103$ billion purchase consideration in its 2016 acquisition of SABMiller Group. The combined company was renamed Anheuser-Busch InBev SA/NV. The majority of the intangible asset valuation relates to brands with indefinite life ( $\$ 19.9$ billion of the $\$ 20.0$ billion total). Of $\$ 63.0$ billion total assets acquired, assets to be divested were valued at $\$ 24.8$ billion and assets to be held for were valued at $\$ 38.2$ billion. In total, intangible assets represent 52 percent of the total assets to be held for use. In addition, $\$ 74.1$ billion of goodwill was recognized in the transaction.

\section{Exhibit 1: Acquisition of Intangible Assets through a Business Combination}
Excerpt from the 2016 annual report of AB InBev:

“On 10 October 2016, AB InBev announced the ... successful completion of the business combination with the former SABMiller Group ("SAB").

"The transaction resulted in 74.1 billion US dollar of goodwill provisionally allocated primarily to the businesses in Colombia, Ecuador, Peru, Australia, South Africa and other African, Asia Pacific and Latin American countries. The factors that contributed to the recognition of goodwill include the acquisition of an assembled workforce and the premiums paid for cost synergies expected to be achieved in SABMiller. Management's

9 Both IFRS and US GAAP require the use of the acquisition method in accounting for business combinations (IFRS 3 and FASB ASC Section 805).

10 As previously described, the definitional criteria are identifiability, control by the company, and expected future benefits. The recognition criteria are probable flows of the expected economic benefits to the company and measurability. assessment of the future economic benefits supporting recognition of this goodwill is in part based on expected savings through the implementation of $\mathrm{AB}$ InBev best practices such as, among others, a zero based budgeting program and initiatives that are expected to bring greater efficiency and standardization, generate cost savings and maximize purchasing power. Goodwill also arises due to the recognition of deferred tax liabilities in relation to the preliminary fair value adjustments on acquired intangible assets for which the amortization does not qualify as a tax deductible expense. None of the goodwill recognized is deductible for tax purposes.

"The majority of the intangible asset valuation relates to brands with indefinite life, valued for a total amount of 19.9 billion US dollar. The valuation of the brands with indefinite life is based on a series of factors, including the brand history, the operating plan and the countries in which the brands are sold. The fair value of brands was estimated by applying a combination of known valuation methodologies, such as the royalty relief and excess earnings valuation approaches.

"The intangibles with an indefinite life mainly include the Castle and Carling brand families in Africa, the Aguila and Poker brand families in Colombia, the Cristal and Pilsner brand families in Ecuador, and the Carlton brand family in Australia.

"Assets held for sale were recognized in relation to the divestiture of SABMiller's interests in the MillerCoors LLC joint venture and certain of SABMiller's portfolio of Miller brands outside of the U.S. to Molson Coors Brewing company; the divestiture of SABMiller's European premium brands to Asahi Group Holdings, Ltd and the divestiture of SABMiller's interest in China Resources Snow Breweries Ltd. to China Resources Beer (Holdings) Co. Ltd. ...." [Excerpt]

The following is a summary of the provisional allocation of AB InBev's purchase price of SABMiller:

\begin{center}
\begin{tabular}{lc}
\hline
Assets & $\mathbf{\$}$ million \\
\hline
Property, plant and equipment & 9,060 \\
Intangible assets & 20,040 \\
Investment in associates & 4,386 \\
Inventories & 977 \\
Trade and other receivables & 1,257 \\
Cash and cash equivalents & 1,410 \\
Assets held for sale & 24,805 \\
All other assets & 1,087 \\
Total assets & 63,022 \\
Total liabilities & $-27,769$ \\
Net identified assets and liabilities & $\mathbf{3 5 , 2 5 3}$ \\
Non-controlling interests & $-6,200$ \\
Goodwill on acquisition & $\mathbf{7 4 , 0 8 3}$ \\
Purchase consideration & $\mathbf{1 0 3 , 1 3 6}$ \\
\hline
\end{tabular}
\end{center}

Table is excerpted from the company's 2016 Annual Report. Portions of detail are omitted, and subtotals are shown in italics. Source: AB InBev 2016 Annual Report, pp. 82-85.

\textbackslash section\{CAPITALIZATION VERSUS EXPENSING: IMPACT ON FINANCIAL STATEMENTS AND RATIOS

explain and evaluate how capitalising versus expensing costs in the period in which they are incurred affects financial statements and ratios

This section discusses the implications for financial statements and ratios of capitalising versus expensing costs in the period in which they are incurred. We first summarize the general financial statement impact of capitalising versus expensing and two analytical issues related to the decision-namely the effect on an individual company's trend analysis and on comparability across companies.

In the period of the expenditure, an expenditure that is capitalised increases the amount of assets on the balance sheet and appears as an investing cash outflow on the statement of cash flows. After initial recognition, a company allocates the capitalised amount over the asset's useful life as depreciation or amortisation expense (except assets that are not depreciated, i.e., land, or amortised, e.g., intangible assets with indefinite lives). This expense reduces net income on the income statement and reduces the value of the asset on the balance sheet. Depreciation and amortisation are non-cash expenses and therefore, apart from their effect on taxable income and taxes payable, have no impact on the cash flow statement. In the section of the statement of cash flows that reconciles net income to operating cash flow, depreciation and amortisation expenses are added back to net income.

Alternatively, an expenditure that is expensed reduces net income by the after-tax amount of the expenditure in the period it is made. No asset is recorded on the balance sheet and thus no depreciation or amortisation occurs in subsequent periods. The lower amount of net income is reflected in lower retained earnings on the balance sheet. An expenditure that is expensed appears as an operating cash outflow in the period it is made. There is no effect on the financial statements of subsequent periods.

Example 4 illustrates the impact on the financial statements of capitalising versus expensing an expenditure.

\section{EXAMPLE 4}
\section{General Financial Statement Impact of Capitalising Versus Expensing}
Assume two identical (hypothetical) companies, CAP Inc. (CAP) and NOW Inc. (NOW), start with $€ 1,000$ cash and $€ 1,000$ common stock. Each year the companies recognise total revenues of $€ 1,500 \mathrm{cash}$ and make cash expenditures, excluding an equipment purchase, of $€ 500$. At the beginning of operations, each company pays $€ 900$ to purchase equipment. CAP estimates the equipment will have a useful life of three years and an estimated salvage value of $€ 0$ at the end of the three years. NOW estimates a much shorter useful life and expenses the equipment immediately. The companies have no other assets and make no other asset purchases during the three-year period. Assume the companies pay no dividends, earn zero interest on cash balances, have a tax rate of 30 percent, and use the same accounting method for financial and tax purposes.

The left side of Exhibit 2 shows CAP's financial statements; i.e., with the expenditure capitalised and depreciated at $€ 300$ per year based on the straight-line method of depreciation ( $€ 900$ cost minus $€ 0$ salvage value equals $€ 900$, divided by a three-year life equals $€ 300$ per year). The right side of the exhibit shows NOW's financial statements, with the entire $€ 900$ expenditure treated as an expense in the first year. All amounts are in euro.

\section{Exhibit 2: Capitalising versus Expensing}
\begin{center}
\includegraphics[max width=\textwidth]{2023_05_04_b5cfa4f1bc883752f121g-350}
\end{center}

\begin{enumerate}
  \item Which company reports higher net income over the three years? Total cash flow? Cash from operations?
\end{enumerate}

\section{Solution to 1:}
Neither company reports higher total net income or cash flow over the three years. The sum of net income over the three years is identical ( $€ 1,470$ total) whether the $€ 900$ is capitalised or expensed. Also, the sum of the change in cash ( 1,470 total) is identical under either scenario. CAP reports higher cash from operations by an amount of $€ 900$ because, under the capitalisation scenario, the $€ 900$ purchase is treated as an investing cash flow.

Note: Because the companies use the same accounting method for both financial and taxable income, absent the assumption of zero interest on cash balances, expensing the $€ 900$ would have resulted in higher income and cash flow for NOW because the lower taxes paid in the first year ( $€ 30$ versus $€ 210)$ would have allowed NOW to earn interest income on the tax savings.

\begin{enumerate}
  \setcounter{enumi}{1}
  \item Based on ROE and net profit margin, how does the profitability of the two companies compare?
\end{enumerate}

\section{Solution to 2:}
In general, Ending shareholders' equity $=$ Beginning shareholders' equity + Net income + Other comprehensive income - Dividends + Net capital contributions from shareholders. Because the companies in this example do not have other comprehensive income, did not pay dividends, and reported no capital contributions from shareholders, Ending retained earnings $=$ Beginning retained earnings + Net income, and Ending shareholders' equity $=$ Beginning shareholders' equity + Net income .

ROE is calculated as Net income divided by Average shareholders' equity, and Net profit margin is calculated as Net income divided by Total revenue. For example, CAP had Year 1 ROE of 39 percent $(€ 490 /[(€ 1,000+$ $€ 1,490) / 2])$, and Year 1 net profit margin of 33 percent $(€ 490 / € 1,500)$.

CAP Inc.

Now Inc.

\section{Capitalise $€ 900$ as asset and}
 depreciateExpense $€ 900$ immediately

\begin{center}
\begin{tabular}{lccclcccc}
\hline
For year & $\mathbf{1}$ & $\mathbf{2}$ & $\mathbf{3}$ & For year & $\mathbf{1}$ & $\mathbf{2}$ & $\mathbf{3}$ \\
\hline
ROE & $39 \%$ & $28 \%$ & $\mathbf{3}$ & ROE & $7 \%$ & $49 \%$ & $33 \%$ \\
Net profit & $33 \%$ & $33 \%$ & $33 \%$ & Net profit & $5 \%$ & $47 \%$ & $47 \%$ \\
margin &  &  &  & margin &  &  &  \\
\hline
\end{tabular}
\end{center}

As shown, compared to expensing, capitalising results in higher profitability ratios ( $\mathrm{ROE}$ and net profit margin) in the first year, and lower profitability ratios in subsequent years. For example, CAP's Year $1 \mathrm{ROE}$ of 39 percent was higher than NOW's Year 1 ROE of 7 percent, but in Years 2 and 3, NOW reports superior profitability.

Note also that NOW's superior growth in net income between Year 1 and Year 2 is not attributable to superior performance compared to CAP but rather to the accounting decision to recognise the expense sooner than CAP. In general, all else equal, accounting decisions that result in recognising expenses sooner will give the appearance of greater subsequent growth. Comparison of the growth of the two companies' net incomes without an awareness of the difference in accounting methods would be misleading. As a corollary, NOW's income and profitability exhibit greater volatility across the three years, not because of more volatile performance but rather because of the different accounting decision. 3. Why does NOW report change in cash of $€ 70$ in Year 1, while CAP reports total change in cash of $(€ 110)$ ?

Solution to 3:

NOW reports an increase in cash of $€ 70$ in Year 1, while CAP reports a decrease in cash of $€ 110$ because NOW's taxes were $€ 180$ lower than CAP's taxes ( $€ 30$ versus $€ 210)$.

Note that this problem assumes the accounting method used by each company for its tax purposes is identical to the accounting method used by the company for its financial reporting. In many countries, companies are allowed to use different depreciation methods for financial reporting and taxes, which may give rise to deferred taxes.

As shown, discretion regarding whether to expense or capitalise expenditures can impede comparability across companies. Example 4 assumes the companies purchase a single asset in one year. Because the sum of net income over the three-year period is identical whether the asset is capitalised or expensed, it illustrates that although capitalising results in higher profitability compared to expensing in the first year, it results in lower profitability in the subsequent years. Conversely, expensing results in lower profitability in the first year but higher profitability in later years, indicating a favorable trend.

Similarly, shareholders' equity for a company that capitalises the expenditure will be higher in the early years because the initially higher profits result in initially higher retained earnings. Example 4 assumes the companies purchase a single asset in one year and report identical amounts of total net income over the three-year period, so shareholders' equity (and retained earnings) for the firm that expenses will be identical to shareholders' equity (and retained earnings) for the capitalising firm at the end of the three-year period.

Although Example 4 shows companies purchasing an asset only in the first year, if a company continues to purchase similar or increasing amounts of assets each year, the profitability-enhancing effect of capitalising continues if the amount of the expenditures in a period continues to be more than the depreciation expense. Example 5 illustrates this point.

\section{EXAMPLE 5}
\section{Impact of Capitalising Versus Expensing for Ongoing Purchases}
A company buys a $\pounds 300$ computer in Year 1 and capitalises the expenditure. The computer has a useful life of three years and an expected salvage value of $\pounds 0$, so the annual depreciation expense using the straight-line method is $\pounds 100$ per year. Compared to expensing the entire $\pounds 300$ immediately, the company's pre-tax profit in Year 1 is $\pounds 200$ greater.

\begin{enumerate}
  \item Assume that the company continues to buy an identical computer each year at the same price. If the company uses the same accounting treatment for each of the computers, when does the profit-enhancing effect of capitalising versus expensing end?
\end{enumerate}

\section{Solution to 1:}
The profit-enhancing effect of capitalising versus expensing would end in Year 3. In Year 3, the depreciation expense on each of the three computers bought in Years 1, 2, and 3 would total $\pounds 300(\pounds 100+\pounds 100+\pounds 100)$. There- fore, the total depreciation expense for Year 3 will be exactly equal to the capital expenditure in Year 3. The expense in Year 3 would be $\pounds 300$, regardless of whether the company capitalised or expensed the annual computer purchases.

\begin{enumerate}
  \setcounter{enumi}{1}
  \item If the company buys another identical computer in Year 4, using the same accounting treatment as the prior years, what is the effect on Year 4 profits of capitalising versus expensing these expenditures?
\end{enumerate}

\section{Solution to 2:}
There is no impact on Year 4 profits. As in the previous year, the depreciation expense on each of the three computers bought in Years 2, 3, and 4 would total $\pounds 300(\pounds 100+\pounds 100+\pounds 100)$. Therefore, the total depreciation expense for Year 4 will be exactly equal to the capital expenditure in Year 4. Pre-tax profits would be reduced by $\pounds 300$, regardless of whether the company capitalised or expensed the annual computer purchases.

Compared to expensing an expenditure, capitalising the expenditure typically results in greater amounts reported as cash from operations. Capitalised expenditures are typically treated as an investment cash outflow whereas expenses reduce operating cash flows. Because cash flow from operating activities is an important consideration in some valuation models, companies may try to maximise reported cash flow from operations by capitalising expenditures that should be expensed. Valuation models that use free cash flow will consider not only operating cash flows but also investing cash flows. Analysts should be alert to evidence of companies manipulating reported cash flow from operations by capitalising expenditures that should be expensed.

In summary, holding all else constant, capitalising an expenditure enhances current profitability and increases reported cash flow from operations. The profitability-enhancing effect of capitalising continues so long as capital expenditures exceed the depreciation expense. Profitability-enhancing motivations for decisions to capitalise should be considered when analyzing performance. For example, a company may choose to capitalise more expenditures (within the allowable bounds of accounting standards) to achieve earnings targets for a given period. Expensing a cost in the period reduces current period profits but enhances future profitability and thus enhances the profit trend. Profit trend-enhancing motivations should also be considered when analyzing performance. If the company is in a reporting environment which requires identical accounting methods for financial reporting and taxes (unlike the United States, which permits companies to use depreciation methods for reporting purposes that differ from the depreciation method required by tax purposes), then expensing will have a more favorable cash flow impact because paying lower taxes in an earlier period creates an opportunity to earn interest income on the cash saved.

In contrast with the relatively simple examples above, it is generally neither possible nor desirable to identify individual instances involving discretion about whether to capitalise or expense expenditures. An analyst can, however, typically identify significant items of expenditure treated differently across companies. The items of expenditure giving rise to the most relevant differences across companies will vary by industry. This cross-industry variation is apparent in the following discussion of the capitalisation of expenditures.

\section{CAPITALISATION OF INTEREST COSTS}
explain and evaluate how capitalising versus expensing costs in the period in which they are incurred affects financial statements and ratios

As noted above, companies generally must capitalise interest costs associated with acquiring or constructing an asset that requires a long period of time to get ready for its intended use. 11

As a consequence of this accounting treatment, a company's interest costs for a period can appear either on the balance sheet (to the extent they are capitalised) or on the income statement (to the extent they are expensed).

If the interest expenditure is incurred in connection with constructing an asset for the company's own use, the capitalised interest appears on the balance sheet as a part of the relevant long-lived asset. The capitalised interest is expensed over time as the property is depreciated-and is thus part of depreciation expense rather than interest expense. If the interest expenditure is incurred in connection with constructing an asset to sell, for example by a real estate construction company, the capitalised interest appears on the company's balance sheet as part of inventory. The capitalised interest is then expensed as part of the cost of sales when the asset is sold.

The treatment of capitalised interest poses certain issues that analysts should consider. First, capitalised interest appears as part of investing cash outflows, whereas expensed interest typically reduces operating cash flow. US GAAP reporting companies are required to categorize interest in operating cash flow, and IFRS reporting companies can categorize interest in operating, investing, or financing cash flows. Although the treatment is consistent with accounting standards, an analyst may want to examine the impact on reported cash flows. Second, interest coverage ratios are solvency indicators measuring the extent to which a company's earnings (or cash flow) in a period covered its interest costs. To provide a true picture of a company's interest coverage, the entire amount of interest expenditure, both the capitalised portion and the expensed portion, should be used in calculating interest coverage ratios. Additionally, if a company is depreciating interest that it capitalised in a previous period, income should be adjusted to eliminate the effect of that depreciation. Example 6 illustrates the calculations.

\section{EXAMPLE 6}
\section{Effect of Capitalised Interest Costs on Coverage Ratios and Cash Flow}
Melco Resorts \& Entertainment Limited (NASDAQ: MLCO), a Hong Kong SAR based casino company which is listed on the NASDAQ stock exchange and prepares financial reports under US GAAP, disclosed the following information in one of the footnotes to its 2017 financial statements: "Interest and amortization of deferred financing costs associated with major development and construction projects is capitalized and included in the cost of the project. .... Total interest expenses incurred amounted to $\$ 267,065, \$ 252,600$, and $\$ 253,168$, of which $\$ 37,483, \$ 29,033$, and $\$ 134,838$ were capitalized during the years ended December

11 IAS 23 [Borrowing Costs] and FASB ASC Subtopic 835-20 [Interest - Capitalization of Interest] specify respectively IFRS and US GAAP for capitalisation of interest costs. Although the standards are not completely converged, the standards are in general agreement. 31, 2017, 2016, and 2015, respectively. Amortization of deferred financing costs of $\$ 26,182, \$ 48,345$, and $\$ 38,511$, net of amortization capitalized of nil, nil, and $\$ 5,458$, were recorded during the years ended December 31, 2017, 2016, and 2015, respectively." (Form 20-F filed 12 April 2018). Cash payments for deferred financing costs were reported in cash flows from financing activities.

Exhibit 3: Melco Resorts \& Entertainment Limited Selected Data, as

Reported (Dollars in thousands)

\begin{center}
\begin{tabular}{lccc}
\hline
 & $\mathbf{2 0 1 7}$ & $\mathbf{2 0 1 6}$ & $\mathbf{2 0 1 5}$ \\
\hline
EBIT (from income statement) & 544,865 & 298,663 & 58,553 \\
Interest expense (from income statement) & 229,582 & 223,567 & 118,330 \\
Capitalized interest (from footnote) & 37,483 & 29,033 & 134,838 \\
Amortization of deferred financing costs & 26,182 & 48,345 & 38,511 \\
(from footnote) &  &  &  \\
 &  &  &  \\
Net cash provided by operating activities & $1,162,500$ & $1,158,128$ & 522,026 \\
Net cash from (used) in investing activities & $(410,226)$ & 280,604 & $(469,656)$ \\
Net cash from (used) in financing & $(1,046,041)$ & $(1,339,717)$ & $(29,688)$ \\
activities &  &  &  \\
\hline
\end{tabular}
\end{center}

Notes: EBIT represents "Income (Loss) Before Income Tax" plus "Interest expenses, net of capitalized interest" from the income statement.

\begin{enumerate}
  \item Calculate and interpret Melco's interest coverage ratio with and without capitalised interest.
\end{enumerate}

\section{Solution to 1:}
Interest coverage ratios with and without capitalised interest were as follows:

For 2017

$2.37(\$ 544,865 \div \$ 229,582)$ without adjusting for capitalised interest; and

$2.14[(\$ 544,865+\$ 26,182) \div(\$ 229,582+\$ 37,483)]$ including an adjustment to EBIT for depreciation of previously capitalised interest and an adjustment to interest expense for the amount of interest capitalised in 2017.

For 2016

$1.34(\$ 298,663 \div \$ 223,567)$ without adjusting for capitalised interest; and

$1.37[(\$ 298,663+\$ 48,345) \div(\$ 223,567+\$ 29,033)]$ including an adjustment to EBIT for depreciation of previously capitalised interest and an adjustment to interest expense for the amount of interest capitalised in 2016.

For 2015

$0.49(\$ 58,533 \div \$ 118,330)$ without adjusting for capitalised interest; and

$0.38[(\$ 58,533+\$ 38,511) \div(\$ 118,330+\$ 134,838)]$ including an adjustment to EBIT for depreciation of previously capitalised interest and an adjustment to interest expense for the amount of interest capitalised in 2015.

The above calculations indicate that Melco's interest coverage improved in 2017 compared to the previous two years. In both 2017 and 2015, the coverage ratio is lower when adjusted for capitalised interest. 2. Calculate Melco's percentage change in operating cash flow from 2016 to 2017. Assuming the financial reporting does not affect reporting for income taxes, what were the effects of capitalised interest on operating and investing cash flows?

\section{Solution to 2:}
If the interest had been expensed rather than capitalised, operating cash flows would have been lower in all three years. On an adjusted basis, but not an unadjusted basis, the company's operating cash flow declined in 2017 compared to 2016. On an unadjusted basis, for 2017 compared with 2016, Melco's operating cash flow increased by 0.4 percent in $2017[(\$ 1,162,500 \div$ $\$ 1,158,128)-1$. Including adjustments to expense all interest costs, Melco's operating cash flow also decreased by 0.4 percent in $2017\{[\$ 1,162,500-$ $\$ 37,483) \div(\$ 1,158,128-\$ 29,033)]-1\}$.

If the interest had been expensed rather than capitalised, financing cash flows would have been higher in all three years.

The treatment of capitalised interest raises issues for consideration by an analyst. First, capitalised interest appears as part of investing cash outflows, whereas expensed interest reduces operating or financing cash flow under IFRS and operating cash flow under US GAAP. An analyst may want to examine the impact on reported cash flows of interest expenditures when comparing companies. Second, interest coverage ratios are solvency indicators measuring the extent to which a company's earnings (or cash flow) in a period covered its interest costs. To provide a true picture of a company's interest coverage, the entire amount of interest, both the capitalised portion and the expensed portion, should be used in calculating interest coverage ratios.

Generally, including capitalised interest in the calculation of interest coverage ratios provides a better assessment of a company's solvency. In assigning credit ratings, rating agencies include capitalised interest in coverage ratios. For example, Standard \& Poor's calculates the EBIT interest coverage ratio as EBIT divided by gross interest (defined as interest prior to deductions for capitalised interest or interest income).

Maintaining a minimum interest coverage ratio is a financial covenant often included in lending agreements, e.g., bank loans and bond indentures. The definition of the coverage ratio can be found in the company's credit agreement. The definition is relevant because treatment of capitalised interest in calculating coverage ratios would affect an assessment of how close a company's actual ratios are to the levels specified by its financial covenants and thus the probability of breaching those covenants.

\section{CAPITALISATION OF INTEREST AND INTERNAL DEVELOPMENT COSTS}
explain and evaluate how capitalising versus expensing costs in the period in which they are incurred affects financial statements and ratios

As noted above, accounting standards require companies to capitalise software development costs after a product's feasibility is established. Despite this requirement, judgment in determining feasibility means that companies' capitalisation practices may differ. For example, as illustrated in Exhibit 4, Microsoft judges product feasibility to be established very shortly before manufacturing begins and, therefore, effectively expenses-rather than capitalises-research and development costs.

\section{Exhibit 4: Disclosure on Software Development Costs}
Excerpt from Management's Discussion and Analysis (MD\&A) of Microsoft Corporation, Application of Critical Accounting Policies, Research and Development Costs:

"Costs incurred internally in researching and developing a computer software product are charged to expense until technological feasibility has been established for the product. Once technological feasibility is established, all software costs are capitalized until the product is available for general release to customers. Judgment is required in determining when technological feasibility of a product is established. We have determined that technological feasibility for our software products is reached after all high-risk development issues have been resolved through coding and testing. Generally, this occurs shortly before the products are released to production. The amortization of these costs is included in cost of revenue over the estimated life of the products."

Source: Microsoft Corporation Annual Report on Form 10-K 2017, p. 45.

Expensing rather than capitalising development costs results in lower net income in the current period. Expensing rather than capitalising will continue to result in lower net income so long as the amount of the current-period development expenses is higher than the amortisation expense that would have resulted from amortising prior periods' capitalised development costs-the typical situation when a company's development costs are increasing. On the statement of cash flows, expensing rather than capitalising development costs results in lower net operating cash flows and higher net investing cash flows. This is because the development costs are reflected as operating cash outflows rather than investing cash outflows.

In comparing the financial performance of a company that expenses most or all software development costs, such as Microsoft, with another company that capitalises software development costs, adjustments can be made to make the two comparable. For the company that capitalises software development costs, an analyst can adjust (a) the income statement to include software development costs as an expense and to exclude amortisation of prior years' software development costs; (b) the balance sheet to exclude capitalised software (decrease assets and equity); and (c) the statement of cash flows to decrease operating cash flows and decrease cash used in investing by the amount of the current period development costs. Any ratios that include income, long-lived assets, or cash flow from operations-such as return on equity-will also be affected.

\section{EXAMPLE 7}
\section{Software Development Costs}
You are working on a project involving the analysis of JHH Software, a (hypothetical) software development company that established technical feasibility for its first product in 2017. Part of your analysis involves computing certain market-based ratios, which you will use to compare JHH to another company that expenses all of its software development expenditures. Relevant data and excerpts from the company's annual report are included in Exhibit 5. Exhibit 5: JHH SOFTWARE (Dollars in Thousands, Except Per-Share Amounts)

\begin{center}
\includegraphics[max width=\textwidth]{2023_05_04_b5cfa4f1bc883752f121g-358}
\end{center}

Footnote disclosure of accounting policy for software development:

Expenses that are related to the conceptual formulation and design of software products are expensed to research and development as incurred. The company capitalises expenses that are incurred to produce the finished product after technological feasibility has been established.

\begin{enumerate}
  \item Compute the following ratios for JHH based on the reported financial statements for fiscal year ended 31 December 2018, with no adjustments. Next, determine the approximate impact on these ratios if the company had expensed rather than capitalised its investments in software. (Assume the financial reporting does not affect reporting for income taxes. There would be no change in the effective tax rate.)
\end{enumerate}

A. P/E: Price/Earnings per share B. P/CFO: Price/Operating cash flow per share

C. EV/EBITDA: Enterprise value/EBITDA, where enterprise value is defined as the total market value of all sources of a company's financing, including equity and debt, and EBITDA is earnings before interest, taxes, depreciation, and amortisation.

\section{Solution to 1:}
(Dollars are in thousands, except per-share amounts.) JHH's 2019 ratios are presented in the following table:

\begin{center}
\begin{tabular}{|c|c|c|}
\hline
Ratios & As reported & As adjusted \\
\hline
P/E ratio & 30.0 & 42.9 \\
\hline
P/CFO & 19.0 & 31.6 \\
\hline
EV/EBITDA & 163 & 24.7 \\
\hline
\end{tabular}
\end{center}

A. Based on the information as reported, the $\mathrm{P} / \mathrm{E}$ ratio was $30.0(\$ 42 \div$ \$1.40). Based on EPS adjusted to expense software development costs, the P/E ratio was $42.9(\$ 42 \div \$ 0.98)$.

\begin{itemize}
  \item Price: Assuming that the market value of the company's equity is based on its fundamentals, the price per share is $\$ 42$, regardless of a difference in accounting.

  \item EPS: As reported, EPS was $\$ 1.40$. Adjusted EPS was $\$ 0.98$.

\end{itemize}

Expensing software development costs would have reduced JHH's 2018 operating income by $\$ 6,000$, but the company would have reported no amortisation of prior years' software costs, which would have increased operating income by $\$ 2,000$. The net change of $\$ 4,000$ would have reduced operating income from the reported $\$ 13,317$ to $\$ 9,317$. The effective tax rate for $2018(\$ 3,825$ $\div \$ 13,317$ ) is $28.72 \%$, and using this effective tax rate would give an adjusted net income of $\$ 6,641$ [\$9,317 $\times(1-0.2872)]$, compared to $\$ 9,492$ before the adjustment. The EPS would therefore be reduced from the reported $\$ 1.40$ to $\$ 0.98$ (adjusted net income of $\$ 6,641$ divided by 6,780 shares).

B. Based on information as reported, the P/CFO was $19.0(\$ 42 \div \$ 2.21)$.

Based on CFO adjusted to expense software development costs, the P/ CFO was $31.6(\$ 42 \div \$ 1.33)$.

\begin{itemize}
  \item Price: Assuming that the market value of the company's equity is based on its fundamentals, the price per share is $\$ 42$, regardless of a difference in accounting.

  \item CFO per share, as reported, was $\$ 2.21$ (total operating cash flows $\$ 15,007 \div 6,780$ shares).

  \item CFO per share, as adjusted, was $\$ 1.33$. The company's $\$ 6,000$ expenditure on software development costs was reported as a cash outflow from investing activities, so expensing those costs would reduce cash from operating activities by $\$ 6,000$, from the reported $\$ 15,007$ to $\$ 9,007$. Dividing adjusted total operating cash flow of $\$ 9,007$ by 6,780 shares results in cash flow per share of $\$ 1.33$.

\end{itemize}

C. Based on information as reported, the EV/EBITDA was $16.3(\$ 284,760$ $\div \$ 17,517)$. Based on EBITDA adjusted to expense software development costs, the EV/EBITDA was $24.7(\$ 284,760 \div \$ 11,517)$ - Enterprise Value: Enterprise value is the sum of the market value of the company's equity and debt. JHH has no debt, and therefore the enterprise value is equal to the market value of its equity. The market value of its equity is $\$ 284,760$ ( $\$ 42$ per share $\times 6,780$ shares).

\begin{itemize}
  \item EBITDA, as reported, was $\$ 17,517$ (earnings before interest and taxes of $\$ 13,317$ plus $\$ 2,200$ depreciation plus $\$ 2,000$ amortisation).

  \item EBITDA, adjusted for expensing software development costs by the inclusion of $\$ 6,000$ development expense and the exclusion of $\$ 2,000$ amortisation of prior expense, would be $\$ 11,517$ (earnings before interest and taxes of $\$ 9,317$ plus $\$ 2,200$ depreciation plus $\$ 0$ amortisation).

\end{itemize}

\begin{enumerate}
  \setcounter{enumi}{1}
  \item Interpret the changes in the ratios.
\end{enumerate}

\section{Solution to 2:}
Expensing software development costs would decrease historical profits, operating cash flow, and EBITDA, and would thus increase all market multiples. So JHH's stock would appear more expensive if it expensed rather than capitalised the software development costs.

If the unadjusted market-based ratios were used in the comparison of JHH to its competitor that expenses all software development expenditures, then JHH might appear to be under-priced when the difference is solely related to accounting factors. JHH's adjusted market-based ratios provide a better basis for comparison.

For the company in Example 7, current period software development expenditures exceed the amortisation of prior periods' capitalised software development expenditures. As a result, expensing rather than capitalising software development costs would have the effect of lowering income. If, however, software development expenditures slowed such that current expenditures were lower than the amortisation of prior periods' capitalised software development expenditures, then expensing software development costs would have the effect of increasing income relative to capitalising it.

This section illustrated how decisions about capitalising versus expensing impact financial statements and ratios. Earlier expensing lowers current profits but enhances trends, whereas capitalising now and expensing later enhances current profits. Having described the accounting for acquisition of long-lived assets, we now turn to the topic of measuring long-lived assets in subsequent periods.

\section{DEPRECIATION OF LONG-LIVED ASSETS: METHODS AND CALCULATION}
describe the different depreciation methods for property, plant, and equipment and calculate depreciation expense

describe how the choice of depreciation method and assumptions concerning useful life and residual value affect depreciation expense, financial statements, and ratios

explain and evaluate how impairment, revaluation, and derecognition of property, plant, and equipment and intangible assets affect financial statements and ratios

Under the cost model of reporting long-lived assets, which is permitted under IFRS and required under US GAAP, the capitalised costs of long-lived tangible assets (other than land, which is not depreciated) and intangible assets with finite useful lives are allocated to subsequent periods as depreciation and amortisation expenses. Depreciation and amortisation are effectively the same concept, with the term depreciation referring to the process of allocating tangible assets' costs and the term amortisation referring to the process of allocating intangible assets' costs. ${ }^{12}$ The alternative model of reporting long-lived assets is the revaluation model, which is permitted under IFRS but not under US GAAP. Under the revaluation model, a company reports the long-lived asset at fair value rather than at acquisition cost (historical cost) less accumulated depreciation or amortisation, as in the cost model.

An asset's carrying amount is the amount at which the asset is reported on the balance sheet. Under the cost model, at any point in time, the carrying amount (also called carrying value or net book value) of a long-lived asset is equal to its historical cost minus the amount of depreciation or amortisation that has been accumulated since the asset's purchase (assuming that the asset has not been impaired, a topic which will be addressed in Section 5). Companies may present on the balance sheet the total net amount of property, plant, and equipment and the total net amount of intangible assets. However, more detail is disclosed in the notes to financial statements. The details disclosed typically include the acquisition costs, the depreciation and amortisation expenses, the accumulated depreciation and amortisation amounts, the depreciation and amortisation methods used, and information on the assumptions used to depreciate and amortise long-lived assets.

\section{Depreciation Methods and Calculation of Depreciation Expense}
Depreciation methods include the straight-line method, in which the cost of an asset is allocated to expense evenly over its useful life; accelerated methods, in which the allocation of cost is greater in earlier years; and the units-of-production method, in which the allocation of cost corresponds to the actual use of an asset in a particular period. The choice of depreciation method affects the amounts reported on the financial statements, including the amounts for reported assets and operating

12 Depletion is the term applied to a similar concept for natural resources; costs associated with those resources are allocated to a period on the basis of the usage or extraction of those resources. and net income. This, in turn, affects a variety of financial ratios, including fixed asset turnover, total asset turnover, operating profit margin, operating return on assets, and return on assets.

Using the straight-line method, depreciation expense is calculated as depreciable cost divided by estimated useful life and is the same for each period. Depreciable cost is the historical cost of the tangible asset minus the estimated residual (salvage) value. ${ }^{13} \mathrm{~A}$ commonly used accelerated method is the declining balance method, in which the amount of depreciation expense for a period is calculated as some percentage of the carrying amount (i.e., cost net of accumulated depreciation at the beginning of the period). When an accelerated method is used, depreciable cost is not used to calculate the depreciation expense but the carrying amount should not be reduced below the estimated residual value. In the units-of-production method, the amount of depreciation expense for a period is based on the proportion of the asset's production during the period compared with the total estimated productive capacity of the asset over its useful life. The depreciation expense is calculated as depreciable cost times production in the period divided by estimated productive capacity over the life of the asset. Equivalently, the company may estimate a depreciation cost per unit (depreciable cost divided by estimated productive capacity) and calculate depreciation expense as depreciation cost per unit times production in the period. Regardless of the depreciation method used, the carrying amount of the asset is not reduced below the estimated residual value. Example 8 provides an example of these depreciation methods.

\section{EXAMPLE 8}
\section{Alternative Depreciation Methods}
You are analyzing three hypothetical companies: EVEN-LI Co., SOONER Inc., and AZUSED Co. At the beginning of Year 1, each company buys an identical piece of box manufacturing equipment for $\$ 2,300$ and has the same assumptions about useful life, estimated residual value, and productive capacity. The annual production of each company is the same, but each company uses a different method of depreciation. As disclosed in each company's notes to the financial statements, each company's depreciation method, assumptions, and production are as follows:

\section{Depreciation method}
\begin{itemize}
  \item EVEN-LI Co.: straight-line method

  \item SOONER Inc.: double-declining balance method (the rate applied to the carrying amount is double the depreciation rate for the straight-line method)

  \item AZUSED Co.: units-of-production method

\end{itemize}

\section{Assumptions and production}
\begin{itemize}
  \item Estimated residual value: $\$ 100$

  \item Estimated useful life: 4 years

  \item Total estimated productive capacity: 800 boxes

  \item Production in each of the four years: 200 boxes in the first year, 300 in the second year, 200 in the third year, and 100 in the fourth year

\end{itemize}

13 The residual value is the estimated amount that an entity will obtain from disposal of the asset at the end of its useful life. 1. Using the following template for each company, record its beginning and ending net book value (carrying amount), end-of-year accumulated depreciation, and annual depreciation expense for the box manufacturing equipment.

Template:

$\begin{array}{cccc}\begin{array}{c}\text { Beginning Net } \\ \text { Book Value }\end{array} & \begin{array}{c}\text { Depreciation } \\ \text { Expense }\end{array} & \begin{array}{c}\text { Accumulated } \\ \text { Depreciation }\end{array} & \begin{array}{c}\text { Ending Net } \\ \text { Book Value }\end{array}\end{array}$

Year 1

Year 2

Year 3

Year 4

\section{Solution to 1:}
For each company, the following information applies: Beginning net book value in Year 1 equals the purchase price of $\$ 2,300$; accumulated year-end depreciation equals the balance from the previous year plus the current year's depreciation expense; ending net book value (carrying amount) equals original cost minus accumulated year-end depreciation (which is the same as beginning net book value minus depreciation expense); and beginning net book value in Years 2, 3, and 4 equals the ending net book value of the prior year. The following text and filled-in templates describe how depreciation expense is calculated for each company.

EVEN-LI Co. uses the straight-line method, so depreciation expense in each year equals $\$ 550$, which is calculated as ( $\$ 2,300$ original cost $-\$ 100$ residual value)/4 years. The net book value at the end of Year 4 is the estimated residual value of $\$ 100$.

\begin{center}
\begin{tabular}{lcccc}
\hline
 & $\begin{array}{c}\text { Beginning } \\ \text { Net Book } \\ \text { Value }\end{array}$ & $\begin{array}{c}\text { Depreciation } \\ \text { Expense }\end{array}$ & $\begin{array}{c}\text { Accumulated } \\ \text { Year-End } \\ \text { Depreciation }\end{array}$ & $\begin{array}{c}\text { Ending Net } \\ \text { Book Value }\end{array}$ \\
\hline
Year 1 & $\$ 2,300$ & $\$ 550$ & $\$ 550$ & $\$ 1,750$ \\
Year 2 & 1,750 & 550 & 1,100 & 1,200 \\
Year 3 & 1,200 & 550 & 1,650 & 650 \\
Year 4 & 650 & 550 & 2,200 & 100 \\
\hline
\end{tabular}
\end{center}

SOONER Inc. uses the double-declining balance method. The depreciation rate for the double-declining balance method is double the depreciation rate for the straight-line method. The depreciation rate under the straight-line method is 25 percent (100 percent divided by 4 years). Thus, the depreciation rate for the double-declining balance method is 50 percent (2 times 25 percent). The depreciation expense for the first year is $\$ 1,150$ (50 percent of $\$ 2,300$ ). Note that under this method, the depreciation rate of 50 percent is applied to the carrying amount (net book value) of the asset, without adjustment for expected residual value. Because the carrying amount of the asset is not depreciated below its estimated residual value, however, the depreciation expense in the final year of depreciation decreases the ending net book value (carrying amount) to the estimated residual value.

\begin{center}
\begin{tabular}{lcccc}
\hline
 & $\begin{array}{c}\text { Beginning } \\ \text { Net Book } \\ \text { Value }\end{array}$ & $\begin{array}{c}\text { Depreciation } \\ \text { Expense }\end{array}$ & $\begin{array}{c}\text { Accumulated } \\ \text { Year-End } \\ \text { Depreciation }\end{array}$ & $\begin{array}{c}\text { Ending Net } \\ \text { Book Value }\end{array}$ \\
\hline
SOONER Inc. & $\$ 2,300$ & $\$ 1,150$ & $\$ 1,150$ & $\$ 1,150$ \\
Year 1 & 1,150 & 575 & 1,725 & 575 \\
Year 3 & 575 & 288 & 2,013 & 287 \\
Year 4 & 287 & 187 & 2,200 & 100 \\
\hline
\end{tabular}
\end{center}

Another common approach (not required in this question) is to use an accelerated method, such as the double-declining method, for some period (a year or more) and then to change to the straight-line method for the remaining life of the asset. If SOONER had used the double-declining method for the first year and then switched to the straight-line method for Years 2, 3 , and 4 , the depreciation expense would be $\$ 350[(\$ 1,150-\$ 100$ estimated residual value)/3 years] a year for Years 2,3 , and 4 . The results for SOONER under this alternative approach are shown below.

\begin{center}
\begin{tabular}{lcccc}
\hline
 & $\begin{array}{c}\text { Beginning } \\ \text { Net Book } \\ \text { Value }\end{array}$ & $\begin{array}{c}\text { Depreciation } \\ \text { Expense }\end{array}$ & $\begin{array}{c}\text { Accumulated } \\ \text { Year-End } \\ \text { Depreciation }\end{array}$ & $\begin{array}{c}\text { Ending Net } \\ \text { Book Value }\end{array}$ \\
\hline
Year 1 & $\$ 2,300$ & $\$ 1,150$ & $\$ 1,150$ & $\$ 1,150$ \\
Year 2 & 1,150 & 350 & 1,500 & 800 \\
Year 3 & 800 & 350 & 1,850 & 450 \\
Year 4 & 450 & 350 & 2,200 & 100 \\
\hline
\end{tabular}
\end{center}

AZUSED Co. uses the units-of-production method. Dividing the equipment's total depreciable cost by its total productive capacity gives a cost per unit of $\$ 2.75$, calculated as ( $\$ 2,300$ original cost $-\$ 100$ residual value)/800. The depreciation expense recognised each year is the number of units produced times $\$ 2.75$. For Year 1 , the amount of depreciation expense is $\$ 550$ (200 units times $\$ 2.75$ ). For Year 2, the amount is $\$ 825$ (300 units times $\$ 2.75)$. For Year 3, the amount is $\$ 550$. For Year 4 , the amount is $\$ 275$.

\begin{center}
\begin{tabular}{lcccc}
\hline
 & $\begin{array}{c}\text { Beginning } \\ \text { Net Book } \\ \text { Value }\end{array}$ & $\begin{array}{c}\text { Depreciation } \\ \text { Expense }\end{array}$ & $\begin{array}{c}\text { Accumulated } \\ \text { Year-End } \\ \text { Depreciation }\end{array}$ & $\begin{array}{c}\text { Ending Net } \\ \text { Book Value }\end{array}$ \\
\hline
Year 1 & $\$ 2,300$ & $\$ 550$ & $\$ 550$ & $\$ 1,750$ \\
Year 2 & 1,750 & 825 & 1,375 & 925 \\
Year 3 & 925 & 550 & 1,925 & 375 \\
Year 4 & 375 & 275 & 2,200 & 100 \\
\hline
\end{tabular}
\end{center}

\begin{enumerate}
  \setcounter{enumi}{1}
  \item Explain the significant differences in the timing of the recognition of the depreciation expense.
\end{enumerate}

\section{Solution to 2:}
All three methods result in the same total amount of accumulated depreciation over the life of the equipment. The significant differences are simply in the timing of the recognition of the depreciation expense. The straight-line method recognises the expense evenly, the accelerated method recognises most of the expense in the first year, and the units-of-production method recognises the expense on the basis of production (or use of the asset). Under all three methods, the ending net book value is $\$ 100$.

\begin{enumerate}
  \setcounter{enumi}{2}
  \item For each company, assume that sales, earnings before interest, taxes, depreciation, and amortization, and assets other than the box manufacturing equipment are as shown in the following table. Calculate the total asset turnover ratio, the operating profit margin, and the operating return on assets for each company for each of the four years. Discuss the ratios, comparing results within and across companies.
\end{enumerate}

\begin{center}
\begin{tabular}{lccc}
\hline
 & Sales & $\begin{array}{c}\text { Earnings before } \\ \text { Interest, Taxes, } \\ \text { Depreciation, } \\ \text { and Amortization }\end{array}$ & $\begin{array}{c}\text { Carrying Amount of Total } \\ \text { Assets, Excluding the Box } \\ \text { Manufacturing Equipment, } \\ \text { at Year End* }\end{array}$ \\
\hline
Year 1 & $\$ 300,000$ & $\$ 36,000$ & $\$ 30,000$ \\
Year 2 & 320,000 & 38,400 & 32,000 \\
Year 3 & 340,000 & 40,800 & 34,000 \\
Year 4 & 360,000 & 43,200 & 36,000 \\
\hline
\end{tabular}
\end{center}

"Assume that total assets at the beginning of Year 1, including the box manufacturing equipment, had a value of $\$ 30,300$. Assume that depreciation expense on assets other than the box manufacturing equipment totaled $\$ 1,000$ per year.

\section{Solution to 3:}
Total asset turnover ratio $=$ Total revenue $\div$ Average total assets

Operating profit margin $=$ Earnings before interest and taxes $\div$ Total revenue

Operating return on assets

$=$ Earnings before interest and taxes $\div$ Average total assets

Ratios are shown in the table below, and details of the calculations for Years 1 and 2 are described after discussion of the ratios.

\begin{center}
\begin{tabular}{|c|c|c|c|c|c|c|c|c|c|}
\hline
\multirow[b]{2}{*}{Ratio*} & \multicolumn{3}{|c|}{EVEN-LI Co.} & \multicolumn{3}{|c|}{SOONER Inc.} & \multicolumn{3}{|c|}{AZUSED Co.} \\
\hline
 & AT & PM (\%) & ROA (\%) & AT & PM (\%) & ROA (\%) & AT & PM (\%) & ROA (\%) \\
\hline
Year 1 & 9.67 & 11.48 & 111.04 & 9.76 & 11.28 & 110.17 & 9.67 & 11.48 & 111.04 \\
\hline
Year 2 & 9.85 & 11.52 & 113.47 & 10.04 & 11.51 & 115.57 & 9.90 & 11.43 & 113.10 \\
\hline
Year 3 & 10.02 & 11.54 & 115.70 & 10.17 & 11.62 & 118.21 & 10.10 & 11.54 & 116.64 \\
\hline
Year 4 & 10.18 & 11.57 & 117.74 & 10.23 & 11.67 & 119.42 & 10.22 & 11.65 & 118.98 \\
\hline
\end{tabular}
\end{center}

" $A T=$ Total asset turnover ratio. $P M=$ Operating profit margin. $R O A=$ Operating return on assets.

For all companies, the asset turnover ratio increased over time because sales grew at a faster rate than that of the assets. SOONER had consistently higher asset turnover ratios than the other two companies, however, because higher depreciation expense in the earlier periods decreased its average total assets. In addition, the higher depreciation in earlier periods resulted in SOONER having lower operating profit margin and operating ROA in the first year and higher operating profit margin and operating ROA in the later periods. SOONER appears to be more efficiently run, on the basis of its higher asset turnover and greater increases in profit margin and ROA over time; however, these comparisons reflect differences in the companies' choice of depreciation method. In addition, an analyst might question the sustainability of the extremely high ROAs for all three companies because such high profitability levels would probably attract new competitors, which would likely put downward pressure on the ratios.

\section{EVEN-LI Co.}
Year 1:

Total asset turnover ratio $=300,000 /[(30,300+30,000+1,750) / 2]$

$=300,000 / 31,025=9.67$

Operating profit margin $=(36,000-1,000-550) / 300,000$

$=34,450 / 300,000=11.48 \%$

Operating ROA $=34,450 / 31,025=111.04 \%$

Year 2:

Total asset turnover ratio $=320,000 /[(30,000+1,750+32,000+1,200) / 2]$

$=320,000 / 32,475=9.85$

Operating profit margin $=(38,400-1,000-550) / 320,000$

$=36,850 / 320,000=11.52 \%$

Operating ROA $=36,850 / 32,475=113.47 \%$

SOONER Inc.

Year 1:

Total asset turnover ratio $=300,000 /[(30,300+30,000+1,150) / 2]$

$=300,000 / 30,725=9.76$

Operating profit margin $=(36,000-1,000-1,150) / 300,000$

$=33,850 / 300,000=11.28 \%$

Operating ROA $=33,850 / 30,725=110.17 \%$

Year 2:

Total asset turnover ratio $=320,000 /[(30,000+1,150+32,000+575) / 2]$

$=320,000 / 31,862.50=10.04$

Operating profit margin $=(38,400-1,000-575) / 320,000$

$=36,825 / 320,000=11.51 \%$

Operating $\mathrm{ROA}=36,825 / 31,862.50=115.57 \%$

\section{AZUSED Co.}
Year 1:

Total asset turnover ratio $=300,000 /[(30,300+30,000+1,750) / 2]$

$=300,000 / 31,025=9.67$

Operating profit margin $=(36,000-1,000-550) / 300,000$

$=34,450 / 300,000=11.48 \%$

Operating ROA $=34,450 / 31,025=111.04 \%$

Year 2:

Total asset turnover ratio $=320,000 /[(30,000+1,750+32,000+925) / 2]$

$=320,000 / 32,337.50=9.90$

Operating profit margin $=(38,400-1,000-825) / 320,000$

$=36,575 / 320,000=11.43 \%$

Operating ROA $=36,575 / 32,337.50=113.10 \%$

In many countries, a company must use the same depreciation methods for both financial and tax reporting. In other countries, including the United States, a company need not use the same depreciation method for financial reporting and taxes. As a result of using different depreciation methods for financial and tax reporting, pre-tax income on the income statement and taxable income on the tax return may differ. Thus, the amount of tax expense computed on the basis of pre-tax income and the amount of taxes actually owed on the basis of taxable income may differ. Although these differences eventually reverse because the total depreciation is the same regardless of the timing of its recognition in financial statements versus on tax returns, during the period of the difference, the balance sheet will show what is known as deferred taxes. For instance, if a company uses straight-line depreciation for financial reporting and an accelerated depreciation method for tax purposes, the company's financial statements will report lower depreciation expense and higher pre-tax income in the first year, compared with the amount of depreciation expense and taxable income in its tax reporting. (Compare the depreciation expense in Year 1 for EVEN-LI Co. and SOONER Inc. in the previous example.) Tax expense calculated on the basis of the financial statements' pre-tax income will be higher than taxes payable on the basis of taxable income; the difference between the two amounts represents a deferred tax liability. The deferred tax liability will be reduced as the difference reverses (i.e., when depreciation for financial reporting is higher than the depreciation for tax purposes) and the income tax is paid.

Significant estimates required for calculating depreciation include the useful life of the asset (or its total lifetime productive capacity) and its expected residual value at the end of that useful life. A longer useful life and higher expected residual value decrease the amount of annual depreciation expense relative to a shorter useful life and lower expected residual value. Companies should review their estimates periodically to ensure they remain reasonable. IFRS require companies to review estimates annually.

Although no significant differences exist between IFRS and US GAAP with respect to the definition of depreciation and the acceptable depreciation methods, IFRS require companies to use a component method of depreciation. ${ }^{14}$ Companies are required to separately depreciate the significant components of an asset (parts of an item with a cost that is significant in relation to the total cost and/or with different useful lives) and thus require additional estimates for the various components. For instance, it may be appropriate to depreciate separately the engine, frame, and interior furnishings of an aircraft. Under US GAAP, the component method of depreciation is allowed but is seldom used in practice. ${ }^{15}$ The following example illustrates depreciating components of an asset.

\section{EXAMPLE 9}
\section{Illustration of Depreciating Components of an Asset}
CUTITUP Co., a hypothetical company, purchases a milling machine, a type of machine used for shaping metal, at a total cost of $\$ 10,000$. $\$ 2,000$ was estimated to represent the cost of the rotating cutter, a significant component of the machine. The company expects the machine to have a useful life of eight years and a residual value of $\$ 3,000$ and that the rotating cutter will need to be replaced every two years. Assume the entire residual value is attributable to the milling machine itself, and assume the company uses straight-line depreciation for all assets.

\begin{enumerate}
  \item How much depreciation expense would the company report in Year 1 if it uses the component method of depreciation, and how much depreciation expense would the company report in Year 1 if it does not use the component method?
\end{enumerate}

\section{Solution to 1:}
Depreciation expense in Year 1 under the component method would be $\$ 1,625$. For the portion of the machine excluding the cutter, the depreciable base is total cost minus the cost attributable to the cutter minus the estimated residual value $=\$ 10,000-\$ 2,000-\$ 3,000=\$ 5,000$. Depreciation expense for the machine excluding the cutter in the first year equals $\$ 625$ (depreciable cost divided by the useful life of the machine $=\$ 5,000 / 8$ years). For the cutter, the depreciation expense equals $\$ 1,000$ (depreciable cost divided by the useful life of the cutter $=\$ 2,000 / 2$ years). Thus, the total depreciation expense for Year 1 under the component method is $\$ 1,625$ (the sum of the depreciation expenses of the two components $=\$ 625+\$ 1,000)$. Depreciation expense in Year 2 would also be $\$ 1,625$.

If the company does not use the component method, depreciation expense in Year 1 is $\$ 875$ (the depreciable cost of the total milling machine divided by its useful life $=[\$ 10,000-\$ 3,000] / 8$ years). Depreciation expense in Year 2 would also be $\$ 875$.

\begin{enumerate}
  \setcounter{enumi}{1}
  \item Assuming a new cutter with an estimated two-year useful life is purchased at the end of Year 2 for $\$ 2,000$, what depreciation expenses would the com- pany report in Year 3 if it uses the component method and if it does not use the component method?
\end{enumerate}

\section{Solution to 2:}
Assuming that at the end of Year 2, the company purchases a new cutter for $\$ 2,000$ with an estimated two-year life, under the component method, the depreciation expense in Year 3 will remain at $\$ 1,625$. If the company does not use the component method and purchases a new cutter with an estimated two-year life for $\$ 2,000$ at the end of Year 2, the depreciation expense in Year 3 will be $\$ 1,875[\$ 875+(\$ 2,000 / 2)=\$ 875+\$ 1,000]$.

\begin{enumerate}
  \setcounter{enumi}{2}
  \item Assuming replacement of the cutter every two years at a price of $\$ 2,000$, what is the total depreciation expense over the eight years if the company uses the component method compared with the total depreciation expense if the company does not use the component method?
\end{enumerate}

\section{Solution to 3:}
Over the eight years, assuming replacement of the cutters every two years at a price of $\$ 2,000$, the total depreciation expense will be $\$ 13,000[\$ 1,625 \times 8$ years] when the component method is used. When the component method is not used, the total depreciation expense will also be $\$ 13,000$ [\$875 $\times$ 2 years $+\$ 1,875 \times 6$ years]. This amount equals the total expenditures of $\$ 16,000[\$ 10,000+3$ cutters $\times \$ 2,000]$ less the residual value of $\$ 3,000$.

\begin{enumerate}
  \setcounter{enumi}{3}
  \item How many different items must the company estimate in the first year to compute depreciation expense for the milling machine if it uses the component method, and how does this compare with what would be required if it does not use the component method?
\end{enumerate}

\section{Solution to 4:}
The following table summarizes the estimates required in the first year to compute depreciation expense if the company does or does not use the component method:

\begin{center}
\begin{tabular}{lcc}
\hline
Estimate & $\begin{array}{c}\text { Required using } \\ \text { component } \\ \text { method? }\end{array}$ & $\begin{array}{c}\text { Required if not } \\ \text { using component } \\ \text { method? }\end{array}$ \\
\hline
Useful life of milling machine & Yes & Yes \\
Residual value of milling machine & Yes & Yes \\
$\begin{array}{l}\text { Portion of machine cost attributable to } \\ \text { cutter }\end{array}$ & Yes & No \\
$\begin{array}{lcl}\text { Portion of residual value attributable to } \\ \text { cutter }\end{array}$ & Yes & No \\
Useful life of cutter & Yes & No \\
\hline
\end{tabular}
\end{center}

Total depreciation expense may be allocated between the cost of sales and other expenses. Within the income statement, depreciation expense of assets used in production is usually allocated to the cost of sales, and the depreciation expense of assets not used in production may be allocated to some other expense category. For instance, depreciation expense may be allocated to selling, general, and administrative expenses if depreciable assets are used in those functional areas. Notes to the financial statements sometimes disclose information regarding which income statement line items include depreciation expense, although the exact amount of detail disclosed by individual companies varies.

\section{AMORTISATION OF LONG-LIVED ASSETS: METHODS AND CALCULATION}
\begin{center}
\includegraphics[max width=\textwidth]{2023_05_04_b5cfa4f1bc883752f121g-370}
\end{center}

Amortisation is similar in concept to depreciation. The term amortisation applies to intangible assets, and the term depreciation applies to tangible assets. Both terms refer to the process of allocating the cost of an asset over the asset's useful life. Only those intangible assets assumed to have finite useful lives are amortised over their useful lives, following the pattern in which the benefits are used up. Acceptable amortisation methods are the same as the methods acceptable for depreciation. Assets assumed to have an indefinite useful life (in other words, without a finite useful life) are not amortised. An intangible asset is considered to have an indefinite useful life when there is "no foreseeable limit to the period over which the asset is expected to generate net cash inflows" for the company. ${ }^{16}$

Intangible assets with finite useful lives include an acquired customer list expected to provide benefits to a direct-mail marketing company for two to three years, an acquired patent or copyright with a specific expiration date, an acquired license with a specific expiration date and no right to renew the license, and an acquired trademark for a product that a company plans to phase out over a specific number of years. Examples of intangible assets with indefinite useful lives include an acquired license that, although it has a specific expiration date, can be renewed at little or no cost and an acquired trademark that, although it has a specific expiration, can be renewed at a minimal cost and relates to a product that a company plans to continue selling for the foreseeable future.

As with depreciation for a tangible asset, the calculation of amortisation for an intangible asset requires the original amount at which the intangible asset is recognised and estimates of the length of its useful life and its residual value at the end of its useful life. Useful lives are estimated on the basis of the expected use of the asset, considering any factors that may limit the life of the asset, such as legal, regulatory, contractual, competitive, or economic factors.

16 IAS 38 Intangible Assets, paragraph 88.

\section{EXAMPLE 10}
\section{Amortisation Expense}
\begin{enumerate}
  \item IAS 38 Intangible Assets provides illustrative examples regarding the accounting for intangible assets, including the following:
\end{enumerate}

A direct-mail marketing company acquires a customer list and expects that it will be able to derive benefit from the information on the list for at least one year, but no more than three years. The customer list would be amortised over management's best estimate of its useful life, say 18 months. Although the direct-mail marketing company may intend to add customer names and other information to the list in the future, the expected benefits of the acquired customer list relate only to the customers on that list at the date it was acquired.

In this example, in what ways would management's decisions and estimates affect the company's financial statements?

\section{Solution:}
Because the acquired customer list is expected to generate future economic benefits for a period greater than one year, the cost of the list should be capitalised and not expensed. The acquired customer list is determined to not have an indefinite life and must be amortised. Management must estimate the useful life of the customer list and must select an amortisation method. In this example, the list appears to have no residual value. Both the amortisation method and the estimated useful life affect the amount of the amortisation expense in each period. A shorter estimated useful life, compared with a longer estimated useful life, results in a higher amortisation expense each year over a shorter period, but the total accumulated amortisation expense over the life of the intangible asset is unaffected by the estimate of the useful life. Similarly, the total accumulated amortisation expense over the life of the intangible asset is unaffected by the choice of amortisation method. The amortisation expense per period depends on the amortisation method. If the straight-line method is used, the amortisation expense is the same for each year of useful life. If an accelerated method is used, the amortisation expense will be higher in earlier years.

\section{THE REVALUATION MODEL}
describe the revaluation model

explain and evaluate how impairment, revaluation, and derecognition of property, plant, and equipment and intangible assets affect financial statements and ratios

The revaluation model is an alternative to the cost model for the periodic valuation and reporting of long-lived assets. IFRS permit the use of either the revaluation model or the cost model, but the revaluation model is not allowed under US GAAP. Revaluation changes the carrying amounts of classes of long-lived assets to fair value (the fair value must be measured reliably). Under the cost model, carrying amounts are historical costs less accumulated depreciation or amortisation. Under the revaluation model, carrying amounts are the fair values at the date of revaluation less any subsequent accumulated depreciation or amortisation.

IFRS allow companies to value long-lived assets either under a cost model at historical cost minus accumulated depreciation or amortisation or under a revaluation model at fair value. In contrast, US accounting standards require that the cost model be used. A key difference between the two models is that the cost model allows only decreases in the values of long-lived assets compared with historical costs but the revaluation model may result in increases in the values of long-lived assets to amounts greater than historical costs.

IFRS allow a company to use the cost model for some classes of assets and the revaluation model for others, but the company must apply the same model to all assets within a particular class of assets and must revalue all items within a class to avoid selective revaluation. Examples of different classes of assets include land, land and buildings, machinery, motor vehicles, furniture and fixtures, and office equipment. The revaluation model may be used for classes of intangible assets but only if an active market for the assets exists, because the revaluation model may only be used if the fair values of the assets can be measured reliably. For practical purposes, the revaluation model is rarely used for either tangible or intangible assets, but its use is especially rare for intangible assets.

Under the revaluation model, whether an asset revaluation affects earnings depends on whether the revaluation initially increases or decreases an asset class' carrying amount. If a revaluation initially decreases the carrying amount of the asset class, the decrease is recognised in profit or loss. Later, if the carrying amount of the asset class increases, the increase is recognised in profit or loss to the extent that it reverses a revaluation decrease of the same asset class previously recognised in profit or loss. Any increase in excess of the reversal amount will not be recognised in the income statement but will be recorded directly to equity in a revaluation surplus account. An upward revaluation is treated the same as the amount in excess of the reversal amount. In other words, if a revaluation initially increases the carrying amount of the asset class, the increase in the carrying amount of the asset class bypasses the income statement and goes directly to equity under the heading of revaluation surplus. Any subsequent decrease in the asset's value first decreases the revaluation surplus and then goes to income. When an asset is retired or disposed of, any related amount of revaluation surplus included in equity is transferred directly to retained earnings.

Asset revaluations offer several considerations for financial statement analyses. First, an increase in the carrying amount of depreciable long-lived assets increases total assets and shareholders' equity, so asset revaluations that increase the carrying amount of an asset can be used to reduce reported leverage. Defining leverage as average total assets divided by average shareholders' equity, increasing both the numerator (assets) and denominator (equity) by the same amount leads to a decline in the ratio. (Mathematically, when a ratio is greater than one, as in this case, an increase in both the numerator and the denominator by the same amount leads to a decline in the ratio.) Therefore, the leverage motivation for the revaluation should be considered in analysis. For example, a company may revalue assets up if it is seeking new capital or approaching leverage limitations set by financial covenants.

Second, assets revaluations that decrease the carrying amount of the assets reduce net income. In the year of the revaluation, profitability measures such as return on assets and return on equity decline. However, because total assets and shareholders' equity are also lower, the company may appear more profitable in future years. Additionally, reversals of downward revaluations also go through income, thus increasing earnings. Managers can then opportunistically time the reversals to manage earnings and increase income. Third, asset revaluations that increase the carrying amount of an asset initially increase depreciation expense, total assets, and shareholders' equity. Therefore, profitability measures, such as return on assets and return on equity, would decline. Although upward asset revaluations also generally decrease income (through higher depreciation expense), the increase in the value of the long-lived asset is presumably based on increases in the operating capacity of the asset, which will likely be evidenced in increased future revenues.

Finally, an analyst should consider who did the appraisal-i.e. an independent external appraiser or management-and how often revaluations are made. Appraisals of the fair value of long-lived assets involve considerable judgment and discretion. Presumably, appraisals of assets from independent external sources are more reliable. How often assets are revalued can provide an indicator of whether their reported value continues to be representative of their fair values.

The next two examples illustrate revaluation of long-lived assets under IFRS.

\section{EXAMPLE 11}
Revaluation Resulting in an Increase in Carrying Amount Followed by Subsequent Revaluation Resulting in a Decrease in Carrying Amount

UPFIRST, a hypothetical manufacturing company, has elected to use the revaluation model for its machinery. Assume for simplicity that the company owns a single machine, which it purchased for $€ 10,000$ on the first day of its fiscal period, and that the measurement date occurs simultaneously with the company's fiscal period end.

\begin{enumerate}
  \item At the end of the first fiscal period after acquisition, assume the fair value of the machine is determined to be $€ 11,000$. How will the company's financial statements reflect the asset?
\end{enumerate}

\section{Solution to 1:}
At the end of the first fiscal period, the company's balance sheet will show the asset at a value of $€ 11,000$. The $€ 1,000$ increase in the value of the asset will appear in other comprehensive income and be accumulated in equity under the heading of revaluation surplus.

\begin{enumerate}
  \setcounter{enumi}{1}
  \item At the end of the second fiscal period after acquisition, assume the fair value of the machine is determined to be $€ 7,500$. How will the company's financial statements reflect the asset?
\end{enumerate}

\section{Solution to 2:}
At the end of the second fiscal period, the company's balance sheet will show the asset at a value of $€ 7,500$. The total decrease in the carrying amount of the asset is $€ 3,500$ ( $€ 11,000-€ 7,500)$. Of the $€ 3,500$ decrease, the first $€ 1,000$ will reduce the amount previously accumulated in equity under the heading of revaluation surplus. The other $€ 2,500$ will be shown as a loss on the income statement.

\section{EXAMPLE 12}
\section{Revaluation Resulting in a Decrease in Asset's Carrying Amount Followed by Subsequent Revaluation Resulting in an Increase in Asset's Carrying Amount}
DOWNFIRST, a hypothetical manufacturing company, has elected to use the revaluation model for its machinery. Assume for simplicity that the company owns a single machine, which it purchased for $€ 10,000$ on the first day of its fiscal period, and that the measurement date occurs simultaneously with the company's fiscal period end.

\begin{enumerate}
  \item At the end of the first fiscal period after acquisition, assume the fair value of the machine is determined to be $€ 7,500$. How will the company's financial statements reflect the asset?
\end{enumerate}

\section{Solution to 1:}
At the end of the first fiscal period, the company's balance sheet will show the asset at a value of $€ 7,500$. The $€ 2,500$ decrease in the value of the asset will appear as a loss on the company's income statement.

\begin{enumerate}
  \setcounter{enumi}{1}
  \item At the end of the second fiscal period after acquisition, assume the fair value of the machine is determined to be $€ 11,000$. How will the company's financial statements reflect the asset?
\end{enumerate}

\section{Solution to 2:}
At the end of the second fiscal period, the company's balance sheet will show the asset at a value of $€ 11,000$. The total increase in the carrying amount of the asset is an increase of $€ 3,500(€ 11,000-€ 7,500)$. Of the $€ 3,500$ increase, the first $€ 2,500$ reverses a previously reported loss and will be reported as a gain on the income statement. The other $€ 1,000$ will bypass profit or loss and be reported as other comprehensive income and be accumulated in equity under the heading of revaluation surplus.

Exhibit 6 provides two examples of disclosures concerning the revaluation model. The first disclosure is an excerpt from the 2006 annual report of KPN, a Dutch telecommunications and multimedia company. The report was produced at a time during which any IFRS-reporting company with a US stock exchange listing was required to explain differences between its reporting under IFRS and its reporting if it had used US GAAP. ${ }^{17}$ One of these differences, as previously noted, is that US GAAP do not allow revaluation of fixed assets held for use. KPN's disclosure states that the company elected to report a class of fixed assets (cables) at fair value and explained that under US GAAP, using the cost model, the value of the asset class would have been $€ 350$ million lower. The second disclosure is an excerpt from the 2017 annual report of Avianca Holdings S.A., a Latin American airline that reports under IFRS and uses the revaluation model for one component of its fixed assets.

17 On 15 November 2007, the SEC approved rule amendments under which financial statements from foreign private issuers in the United States will be accepted without reconciliation to US GAAP if the financial statements are prepared in accordance with IFRS as issued by the International Accounting Standards Board. The rule took effect for the 2007 fiscal year. As a result, companies such as KPN no longer need to provide reconciliations to US GAAP.

\section{Exhibit 6: Impact of Revaluation}
\begin{enumerate}
  \item Excerpt from the annual report of Koninklijke KPN N.V. explaining certain differences between IFRS and US GAAP regarding "Deemed cost fixed assets":
\end{enumerate}

KPN elected the exemption to revalue certain of its fixed assets upon the transition to IFRS to fair value and to use this fair value as their deemed cost. KPN applied the depreciated replacement cost method to determine this fair value. The revalued assets pertain to certain cables, which form part of property, plant \& equipment. Under US GAAP, this revaluation is not allowed and therefore results in a reconciling item. As a result, the value of these assets as of December 31, 2006 under US GAAP is EUR 350 million lower (2005: EUR 415 million; 2004: EUR 487 million) than under IFRS.

Source: KPN's Form 20-F, p. 168, filed 1 March 2007.

\begin{enumerate}
  \setcounter{enumi}{1}
  \item The 2017 annual report of Avianca Holdings S.A. and Subsidiaries shows $\$ 58.4$ million of "Revaluation and Other Reserves" as a component of Equity on its balance sheet and $\$ 31.0$ million in Other Comprehensive Income for the current year's "Revaluation of Administrative Property". The relevant footnote disclosure explains:
\end{enumerate}

"Administrative property in Bogota, Medellín, El Salvador, and San Jose is recorded at fair value less accumulated depreciation on buildings and impairment losses recognized at the date of revaluation. Valuations are performed with sufficient frequency to ensure that the fair value of a revalued asset does not differ materially from its carrying amount. A revaluation reserve is recorded in other comprehensive income and credited to the asset revaluation reserve in equity. However, to the extent that it reverses a revaluation deficit of the same asset previously recognized in profit or loss, the increase is recognized in profit and loss. A revaluation deficit is recognized in the income statement, except to the extent that it offsets an existing surplus on the same asset recognized in the asset revaluation reserve. Upon disposal, any revaluation reserve relating to the particular asset being sold is transferred to retained earnings.

Source: AVIANCA HOLDINGS S.A. Form 20-F filed 01 May 2018.

Clearly, the use of the revaluation model as opposed to the cost model can have a significant impact on the financial statements of companies. This has potential consequences for comparing financial performance using financial ratios of companies that use different models.

\section{IMPAIRMENT OF ASSETS}
explain the impairment of property, plant, and equipment and intangible assets

explain and evaluate how impairment, revaluation, and derecognition of property, plant, and equipment and intangible assets affect financial statements and ratios

In contrast with depreciation and amortisation charges, which serve to allocate the depreciable cost of a long-lived asset over its useful life, impairment charges reflect an unanticipated decline in the value of an asset. Both IFRS and US GAAP require companies to write down the carrying amount of impaired assets. Impairment reversals for identifiable, long-lived assets are permitted under IFRS but typically not under US GAAP.

An asset is considered to be impaired when its carrying amount exceeds its recoverable amount. Although IFRS and US GAAP define recoverability differently (as described below), in general, impairment losses are recognised when the asset's carrying amount is not recoverable. The following paragraphs describe accounting for impairment for different categories of assets.

\section{Impairment of Property, Plant, and Equipment}
Accounting standards do not require that property, plant, and equipment be tested annually for impairment. Rather, at the end of each reporting period (generally, a fiscal year), a company assesses whether there are indications of asset impairment. If there is no indication of impairment, the asset is not tested for impairment. If there is an indication of impairment, such as evidence of obsolescence, decline in demand for products, or technological advancements, the recoverable amount of the asset should be measured in order to test for impairment. For property, plant, and equipment, impairment losses are recognised when the asset's carrying amount is not recoverable; the carrying amount is more than the recoverable amount. The amount of the impairment loss will reduce the carrying amount of the asset on the balance sheet and will reduce net income on the income statement. The impairment loss is a non-cash item and will not affect cash from operations.

IFRS and US GAAP differ somewhat both in the guidelines for determining that impairment has occurred and in the measurement of an impairment loss. Under IAS 36, an impairment loss is measured as the excess of carrying amount over the recoverable amount of the asset. The recoverable amount of an asset is defined as "the higher of its fair value less costs to sell and its value in use." Value in use is based on the present value of expected future cash flows. Under US GAAP, assessing recoverability is separate from measuring the impairment loss. The carrying amount of an asset "group" is considered not recoverable when it exceeds the undiscounted expected future cash flows of the group. If the asset's carrying amount is considered not recoverable, the impairment loss is measured as the difference between the asset's fair value and carrying amount.

\section{EXAMPLE 13}
\section{Impairment of Property, Plant, and Equipment}
Sussex, a hypothetical manufacturing company in the United Kingdom, has a machine it uses to produce a single product. The demand for the product has declined substantially since the introduction of a competing product. The company has assembled the following information with respect to the machine:

Carrying amount $\pounds 18,000$

Undiscounted expected future cash flows $\quad \pounds 19,000$

Present value of expected future cash flows $\quad \pounds 16,000$

$\begin{array}{lr}\text { Fair value if sold } & \pounds 17,000\end{array}$

$\begin{array}{lr}\text { Costs to sell } & \pounds 2,000\end{array}$

\begin{enumerate}
  \item Under IFRS, what would the company report for the machine?
\end{enumerate}

\section{Solution to 1:}
Under IFRS, the company would compare the carrying amount $(\pounds 18,000)$ with the higher of its fair value less costs to sell $(\pounds 15,000)$ and its value in use $(\pounds 16,000)$. The carrying amount exceeds the value in use, the higher of the two amounts, by $\pounds 2,000$. The machine would be written down to the recoverable amount of $\pounds 16,000$, and a loss of $\pounds 2,000$ would be reported in the income statement. The carrying amount of the machine is now $\pounds 16,000$. A new depreciation schedule based on the carrying amount of $\pounds 16,000$ would be developed.

\begin{enumerate}
  \setcounter{enumi}{1}
  \item Under US GAAP, what would the company report for the machine?
\end{enumerate}

\section{Solution to 2:}
Under US GAAP, the carrying amount $(\pounds 18,000)$ is compared with the undiscounted expected future cash flows $(\pounds 19,000)$. The carrying amount is less than the undiscounted expected future cash flows, so the carrying amount is considered recoverable. The machine would continue to be carried at $\pounds 18,000$, and no loss would be reported.

In Example 13, a write down in the value of a piece of property, plant, and equipment occurred under IFRS but not under US GAAP. In Example 14, a write down occurs under both IFRS and US GAAP.

\section{EXAMPLE 14}
\section{Impairment of Property, Plant, and Equipment}
Essex, a hypothetical manufacturing company, has a machine it uses to produce a single product. The demand for the product has declined substantially since the introduction of a competing product. The company has assembled the following information with respect to the machine:

Carrying amount

$\pounds 18,000$

Undiscounted expected future cash flows

$\pounds 16,000$

Present value of expected future cash flows

$\pounds 14,000$

Fair value if sold

$\pounds 10,000$ Costs to sell

\begin{enumerate}
  \item Under IFRS, what would the company report for the machine?
\end{enumerate}

\section{Solution to 1:}
Under IFRS, the company would compare the carrying amount $(\pounds 18,000)$ with the higher of its fair value less costs to sell $(\pounds 8,000)$ and its value in use $(\pounds 14,000)$. The carrying amount exceeds the value in use, the higher of the two amounts, by $\pounds 4,000$. The machine would be written down to the recoverable amount of $\pounds 14,000$, and a loss of $\pounds 4,000$ would be reported in the income statement. The carrying amount of the machine is now $\pounds 14,000$. A new depreciation schedule based on the carrying amount of $€ 14,000$ would be developed.

\begin{enumerate}
  \setcounter{enumi}{1}
  \item Under US GAAP, what would the company report for the machine?
\end{enumerate}

\section{Solution to 2:}
Under US GAAP, the carrying amount $(\pounds 18,000)$ is compared with the undiscounted expected future cash flows $(\pounds 16,000)$. The carrying amount exceeds the undiscounted expected future cash flows, so the carrying amount is considered not recoverable. The machine would be written down to fair value of $\pounds 10,000$, and a loss of $\pounds 8,000$ would be reported in the income statement. The carrying amount of the machine is now $\pounds 10,000$. A new depreciation schedule based on the carrying amount of $\pounds 10,000$ would be developed.

Example 14 shows that the write down to value in use under IFRS can be less than the write down to fair value under US GAAP. The difference in recognition of impairment losses is ultimately reflected in difference in book value of equity.

\section{Impairment of Intangible Assets with a Finite Life}
Intangible assets with a finite life are amortised (carrying amount decreases over time) and may become impaired. As is the case with property, plant, and equipment, the assets are not tested annually for impairment. Instead, they are tested only when significant events suggest the need to test. The company assesses at the end of each reporting period whether a significant event suggesting the need to test for impairment has occurred. Examples of such events include a significant decrease in the market price or a significant adverse change in legal or economic factors. Impairment accounting for intangible assets with a finite life is essentially the same as for tangible assets; the amount of the impairment loss will reduce the carrying amount of the asset on the balance sheet and will reduce net income on the income statement.

\section{Impairment of Intangibles with Indefinite Lives}
Intangible assets with indefinite lives are not amortised. Instead, they are carried on the balance sheet at historical cost but are tested at least annually for impairment. Impairment exists when the carrying amount exceeds its fair value.

\section{Impairment of Long-Lived Assets Held for Sale}
A long-lived (non-current) asset is reclassified as held for sale rather than held for use when management's intent is to sell it and its sale is highly probable. (Additionally, accounting standards require that the asset must be available for immediate sale in its present condition. ${ }^{18}$ For instance, assume a building is no longer needed by a company and management's intent is to sell it, if the transaction meets the accounting criteria, the building is reclassified from property, plant, and equipment to non-current assets held for sale. At the time of reclassification, assets previously held for use are tested for impairment. If the carrying amount at the time of reclassification exceeds the fair value less costs to sell, an impairment loss is recognised and the asset is written down to fair value less costs to sell. Long-lived assets held for sale cease to be depreciated or amortised.

\section{Reversals of Impairments of Long-Lived Assets}
After an asset has been deemed impaired and an impairment loss has been reported, the asset's recoverable amount could potentially increase. For instance, a lawsuit appeal may successfully challenge a patent infringement by another company, with the result that a patent previously written down has a higher recoverable amount. IFRS permit impairment losses to be reversed if the recoverable amount of an asset increases regardless of whether the asset is classified as held for use or held for sale. Note that IFRS permit the reversal of impairment losses only. IFRS do not permit the revaluation to the recoverable amount if the recoverable amount exceeds the previous carrying amount. Under US GAAP, the accounting for reversals of impairments depends on whether the asset is classified as held for use or held for sale. ${ }^{19}$ Under US GAAP, once an impairment loss has been recognised for assets held for use, it cannot be reversed. In other words, once the value of an asset held for use has been decreased by an impairment charge, it cannot be increased. For assets held for sale, if the fair value increases after an impairment loss, the loss can be reversed.

\section{DERECOGNITION}
\begin{center}
\includegraphics[max width=\textwidth]{2023_05_04_b5cfa4f1bc883752f121g-379}
\end{center}

A company derecognises an asset (i.e., removes it from the financial statements) when the asset is disposed of or is expected to provide no future benefits from either use or disposal. A company may dispose of a long-lived operating asset by selling it, exchanging it, abandoning it, or distributing it to existing shareholders. As previously described, non-current assets that management intends to sell or to distribute to existing shareholders and which meet the accounting criteria (immediately available for sale in current condition and the sale is highly probable) are reclassified as non-current assets held for sale.

18 IFRS 5 Non-current Assets Held for Sale and Discontinued Operations.

19 FASB ASC Section 360-10-35 [Property, Plant, and Equipment - Overall - Subsequent Measurement].

\section{Sale of Long-Lived Assets}
The gain or loss on the sale of long-lived assets is computed as the sales proceeds minus the carrying amount of the asset at the time of sale. An asset's carrying amount is typically the net book value (i.e., the historical cost minus accumulated depreciation), unless the asset's carrying amount has been changed to reflect impairment and/or revaluation, as previously discussed.

\section{EXAMPLE 15}
\section{Calculation of Gain or Loss on the Sale of Long-Lived}
Assets

\begin{enumerate}
  \item Moussilauke Diners Inc., a hypothetical company, as a result of revamping its menus to focus on healthier food items, sells 450 used pizza ovens and reports a gain on the sale of $\$ 1.2$ million. The ovens had a carrying amount of $\$ 1.9$ million (original cost of $\$ 5.1$ million less $\$ 3.2$ million of accumulated depreciation). At what price did Moussilauke sell the ovens?
A. $\$ 0.7$ million
B. $\$ 3.1$ million
C. $\$ 6.3$ million
\end{enumerate}

\section{Solution:}
B is correct. The ovens had a carrying amount of $\$ 1.9$ million, and Moussilauke recognised a gain of $\$ 1.2$ million. Therefore, Moussilauke sold the ovens at a price of $\$ 3.1$ million. The gain on the sale of $\$ 1.2$ million is the selling price of $\$ 3.1$ million minus the carrying amount of $\$ 1.9$ million. Ignoring taxes, the cash flow from the sale is $\$ 3.1$ million, which would appear as a cash inflow from investing.

A gain or loss on the sale of an asset is disclosed on the income statement, either as a component of other gains and losses or in a separate line item when the amount is material. A company typically discloses further detail about the sale in the management discussion and analysis and/or financial statement footnotes. In addition, a statement of cash flows prepared using the indirect method adjusts net income to remove any gain or loss on the sale from operating cash flow and to include the amount of proceeds from the sale in cash from investing activities. Recall that the indirect method of the statement of cash flows begins with net income and makes all adjustments to arrive at cash from operations, including removal of gains or losses from non-operating activities.

\section{Long-Lived Assets Disposed of Other Than by a Sale}
Long-lived assets to be disposed of other than by a sale (e.g., abandoned, exchanged for another asset, or distributed to owners in a spin-off) are classified as held for use until disposal or until they meet the criteria to be classified as held for sale or held for distribution. ${ }^{20}$ Thus, the long-lived assets continue to be depreciated and tested for impairment, unless their carrying amount is zero, as required for other long-lived assets owned by the company.

20 In a spin-off, shareholders of the parent company receive a proportional number of shares in a new, separate entity. When an asset is retired or abandoned, the accounting is similar to a sale, except that the company does not record cash proceeds. Assets are reduced by the carrying amount of the asset at the time of retirement or abandonment, and a loss equal to the asset's carrying amount is recorded.

When an asset is exchanged, accounting for the exchange typically involves removing the carrying amount of the asset given up, adding a fair value for the asset acquired, and reporting any difference between the carrying amount and the fair value as a gain or loss. The fair value used is the fair value of the asset given up unless the fair value of the asset acquired is more clearly evident. If no reliable measure of fair value exists, the acquired asset is measured at the carrying amount of the asset given up. A gain is reported when the fair value used for the newly acquired asset exceeds the carrying amount of the asset given up. A loss is reported when the fair value used for the newly acquired asset is less than the carrying amount of the asset given up. If the acquired asset is valued at the carrying amount of the asset given up because no reliable measure of fair value exists, no gain or loss is reported.

When a spin-off occurs, typically, an entire cash generating unit of a company with all its assets is spun off. As an illustration of a spin-off, Fiat Chrysler Automobiles (FCA) spun off its ownership of Ferrari in 2016. Prior to the spinoff, FCA had sold 10 percent of its ownership of Ferrari in an IPO and recognized an increase in Shareholders' equity of $€ 873$ million (the difference between the consideration it received in the IPO of $€ 866$ million and the carrying amount of the equity interest sold of $€ 7$ million.) In contrast, the spin-off, in which FCA distributed its ownership in Ferrari to the existing FCA shareholders, did not result in any gain or loss.

FCA's spinoff was completed on 3 January 2016, with each FCA shareholder receiving one common share of Ferrari N.V. for every ten common shares of FCA. In its financial statements for the prior fiscal year, FCA shows the assets and liabilities of Ferrari as held for distribution. Specifically, its balance sheet includes $€ 3,650$ million Assets Held for Distribution as a component of current assets and $€ 3,584$ million Liabilities Held for Distribution. Exhibit 7 includes excerpts from the company's 31 December 2015 annual report.

\section{Exhibit 7: Fiat Chrysler Automobiles (FCA) Excerpts from Notes to the Consolidated Financial Statements - 2015 Annual Report}
\section{Ferrari Spin-off and Discontinued Operations}
"As the spin-off of Ferrari N.V. became highly probable with the aforementioned shareholders' approval and since it was available for immediate distribution at that date, the Ferrari segment met the criteria to be classified as a disposal group held for distribution to owners and a discontinued operation pursuant to IFRS 5 - Non-current Assets Held for Sale and Discontinued Operations."

The following assets and liabilities of the Ferrari segment were classified as held for distribution at December 31, 2015:

At December 31, 2015

Assets classified as held for distribution

$(€$ million)

Goodwill

786

Other intangible assets

$\begin{array}{lr}\text { Property, plant and equipment } & 627\end{array}$

$\begin{array}{lr}\text { Other non-current assets } & 134\end{array}$

$\begin{array}{lr}\text { Receivables from financing activities } & 1,176\end{array}$

$\begin{array}{lr}\text { Cash and cash equivalents } & 182\end{array}$

\begin{center}
\begin{tabular}{|c|c|}
\hline
 & $\begin{array}{c}\text { At December } 31 \\ 2015\end{array}$ \\
\hline
Other current assets & 448 \\
\hline
Total Assets held for distribution & 3,650 \\
\hline
\multicolumn{2}{|c|}{Liabilities classified as held for distribution} \\
\hline
Provisions & 224 \\
\hline
Debt & 2,256 \\
\hline
Other current liabilities & 624 \\
\hline
Trade payables & 480 \\
\hline
Total Liabilities held for distribution & 3,584 \\
\hline
\end{tabular}
\end{center}

Source: Fiat Chrysler Automobiles (FCA)'s Form 20-F for the year ending 31 December 2015.

\section{PRESENTATION AND DISCLOSURE REQUIREMENTS}
describe the financial statement presentation of and disclosures relating to property, plant, and equipment and intangible assets

Under IFRS, for each class of property, plant, and equipment, a company must disclose the measurement bases, the depreciation method, the useful lives (or, equivalently, the depreciation rate) used, the gross carrying amount and the accumulated depreciation at the beginning and end of the period, and a reconciliation of the carrying amount at the beginning and end of the period. ${ }^{21}$ In addition, disclosures of restrictions on title and pledges as security of property, plant, and equipment and contractual agreements to acquire property, plant, and equipment are required. If the revaluation model is used, the date of revaluation, details of how the fair value was obtained, the carrying amount under the cost model, and the revaluation surplus must be disclosed.

The disclosure requirements under US GAAP are less exhaustive. ${ }^{22}$ A company must disclose the depreciation expense for the period, the balances of major classes of depreciable assets, accumulated depreciation by major classes or in total, and a general description of the depreciation method(s) used in computing depreciation expense with respect to the major classes of depreciable assets.

Under IFRS, for each class of intangible assets, a company must disclose whether the useful lives are indefinite or finite. If finite, for each class of intangible asset, a company must disclose the useful lives (or, equivalently, the amortisation rate) used, the amortisation methods used, the gross carrying amount and the accumulated amortisation at the beginning and end of the period, where amortisation is included on the income statement, and a reconciliation of the carrying amount at the beginning and end of the period. ${ }^{23}$ If an asset has an indefinite life, the company must disclose the carrying amount of the asset and why it is considered to have an indefinite life. Similar to property, plant, and equipment, disclosures of restrictions on title and pledges as security of intangible assets and contractual agreements to acquire intangible assets

21 IAS 16 Property, Plant and Equipment, paragraphs 73-79 [Disclosure].

22 FASB ASC Section 360-10-50 [Property, Plant, and Equipment - Overall - Disclosure].

23 IAS 38 Intangible Assets, paragraphs 118-128 [Disclosure]. are required. If the revaluation model is used, the date of revaluation, details of how the fair value was obtained, the carrying amount under the cost model, and the revaluation surplus must be disclosed.

Under US GAAP, companies are required to disclose the gross carrying amounts and accumulated amortisation in total and by major class of intangible assets, the aggregate amortisation expense for the period, and the estimated amortisation expense for the next five fiscal years. ${ }^{24}$

The disclosures related to impairment losses also differ under IFRS and US GAAP. Under IFRS, a company must disclose for each class of assets the amounts of impairment losses and reversals of impairment losses recognised in the period and where those are recognised on the financial statements. ${ }^{25}$ The company must also disclose in aggregate the main classes of assets affected by impairment losses and reversals of impairment losses and the main events and circumstances leading to recognition of these impairment losses and reversals of impairment losses. Under US GAAP, there is no reversal of impairment losses for assets held for use. The company must disclose a description of the impaired asset, what led to the impairment, the method of determining fair value, the amount of the impairment loss, and where the loss is recognised on the financial statements. ${ }^{26}$

Disclosures about long-lived assets appear throughout the financial statements: in the balance sheet, the income statement, the statement of cash flows, and the notes. The balance sheet reports the carrying value of the asset. For the income statement, depreciation expense may or may not appear as a separate line item. Under IFRS, whether the income statement discloses depreciation expense separately depends on whether the company is using a 'nature of expense' method or a 'function of expense' method. Under the nature of expense method, a company aggregates expenses "according to their nature (for example, depreciation, purchases of materials, transport costs, employee benefits and advertising costs), and does not reallocate them among functions within the entity." ${ }^{17}$ Under the function of expense method, a company classifies expenses according to the function, for example as part of cost of sales or of SG\&A (selling, general, and administrative expenses). At a minimum, a company using the function of expense method must disclose cost of sales, but the other line items vary.

The statement of cash flows reflects acquisitions and disposals of fixed assets in the investing section. In addition, when prepared using the indirect method, the statement of cash flows typically shows depreciation expense (or depreciation plus amortisation) as a line item in the adjustments of net income to cash flow from operations. The notes to the financial statements describe the company's accounting method(s), the range of estimated useful lives, historical cost by main category of fixed asset, accumulated depreciation, and annual depreciation expense.

To illustrate financial statement presentation and disclosures, the following example provides excerpts relating to intangible assets and property, plant, and equipment from the annual report of Orange SA for the year ended 31 December 2017.

24 FASB ASC Section 350-30-50 [Intangibles - General - Disclosure].

25 IAS 36 Impairment of Assets, paragraphs 126-137 [Disclosure].

26 FASB ASC Section 360-10-50 [Property, Plant, and Equipment - Overall - Disclosure] and FASB ASC Section 350-30-50 [Intangibles - General - Disclosure].

27 IAS 1 paragraph 102.

\section{EXAMPLE 16}
\section{Financial Statement Presentation and Disclosures for Long-Lived Assets}
The following exhibits include excerpts from the annual report for the year ended 31 December 2017 of Orange SA, an international telecommunications company based in France.

Exhibit 8: OrangeSA

Excerpts from the 2017 Consolidated Financial Statements

(Note that only selected line items/data are shown for illustrative purposes)

\begin{center}
\begin{tabular}{|c|c|c|c|}
\hline
\multicolumn{4}{|c|}{$\begin{array}{l}\text { Excerpt from Consolidated income statement } \\
\text { EUR }(€) € \text { in Millions }\end{array}$} \\
\hline
 & \multicolumn{3}{|c|}{12 Months Ended} \\
\hline
 & $\begin{array}{l}\text { Dec. 31, } \\ 2017\end{array}$ & $\begin{array}{l}\text { Dec. 31, } \\ 2016\end{array}$ & $\begin{array}{l}\text { Dec. 31, } \\ 2015\end{array}$ \\
\hline
Revenues & $€ 41,096$ & $€ 40,918$ & $€ 40,236$ \\
\hline
$\ldots$ & $\ldots$ & $\ldots$ & $\ldots$ \\
\hline
Depreciation and amortization & $(6,846)$ & $(6,728)$ & $(6,465)$ \\
\hline
... & $\ldots$ & $\ldots$ & ... \\
\hline
Impairment of goodwill & (20) & \includegraphics[max width=\textwidth]{2023_05_04_b5cfa4f1bc883752f121g-384(3)}
 &  \\
\hline
Impairment of fixed assets & \includegraphics[max width=\textwidth]{2023_05_04_b5cfa4f1bc883752f121g-384(2)}
 & \includegraphics[max width=\textwidth]{2023_05_04_b5cfa4f1bc883752f121g-384(1)}
 & \includegraphics[max width=\textwidth]{2023_05_04_b5cfa4f1bc883752f121g-384}
 \\
\hline
$\cdots$ & $\ldots$ & $\ldots$ & $\cdots$ \\
\hline
Operating income & 4,917 & 4,077 & 4,742 \\
\hline
$\cdots$ & $\cdots$ & $\cdots$ & $\cdots$ \\
\hline
$\begin{array}{l}\text { Consolidated net income of continuing } \\ \text { operations }\end{array}$ & 2,114 & 1,010 & 2,510 \\
\hline
$\begin{array}{l}\text { Consolidated net income of discontinued } \\ \text { operations (EE) }\end{array}$ & 29 & 2,253 & 448 \\
\hline
Consolidated net income & 2,143 & 3,263 & 2,958 \\
\hline
$\begin{array}{l}\text { Net income attributable to owners of the } \\ \text { parent company }\end{array}$ & 1,906 & 2,935 & 2,652 \\
\hline
Non-controlling interests & $€ 237$ & $€ 328$ & $€ 306$ \\
\hline
\end{tabular}
\end{center}

Excerpt from the Consolidated statement of financial position EUR $(\epsilon) €$ in Millions

\begin{center}
\begin{tabular}{lccc}
\hline
Assets & $\begin{array}{c}\text { Dec. 31, } \\ \mathbf{2 0 1 7}\end{array}$ & $\begin{array}{c}\text { Dec. 31, } \\ \mathbf{2 0 1 6}\end{array}$ & Dec. 31, 2015 \\
\hline
Goodwill & $€ 27,095$ & $€ 27,156$ & $€ 27,071$ \\
Other intangible assets & 14,339 & 14,602 & 14,327 \\
Property, plant and equipment & 26,665 & 25,912 & 25,123 \\
$\ldots$ & $\ldots$ & $\ldots$ & $\ldots$ \\
\cline { 2 - 4 }
Total non-current assets & 74,035 & 74,819 & 71,330 \\
$\ldots$ & $\ldots$ & $\ldots$ & $\ldots$ \\
Total current assets & 20,679 & 19,849 & 14,312 \\
\end{tabular}
\end{center}

Excerpt from the Consolidated statement of financial position EUR $(€) €$ in Millions

\begin{center}
\begin{tabular}{|c|c|c|c|}
\hline
Assets & $\begin{array}{c}\text { Dec. 31, } \\ 2017\end{array}$ & $\begin{array}{c}\text { Dec. } 31 \text {, } \\ 2016\end{array}$ & Dec. 31,2015 \\
\hline
Assets held for sale &  &  & 5,788 \\
\hline
Total assets & 94,714 & 94,668 & 91,430 \\
\hline
\multicolumn{4}{|l|}{Equity and liabilities} \\
\hline
$\cdots$ & $\cdots$ & $\cdots$ & $\cdots$ \\
\hline
Total equity & 32,942 & 33,174 & 33,267 \\
\hline
$\cdots$ & $\cdots$ & $\cdots$ & $\cdots$ \\
\hline
Total non-current liabilities & 32,736 & 35,590 & 36,537 \\
\hline
$\cdots$ & $\cdots$ & $\cdots$ & $\cdots$ \\
\hline
Total current liabilities & 29,036 & 25,904 & 21,626 \\
\hline
Total equity and liabilities & 94,714 & 94,668 & 91,430 \\
\hline
\end{tabular}
\end{center}

\section{Exhibit 9: Orange}
Excerpts from the 2017 Notes to the Consolidated Financial Statements

\section{Excerpt from Note 7.2 Goodwill}
[Excerpt] Reconciliation of Changes in Goodwill ( $\epsilon$ in Millions)

12 Months Ended

\begin{center}
\begin{tabular}{|c|c|c|c|}
\hline
 & $\begin{array}{c}\text { Dec. 31, } \\ 2017\end{array}$ & $\begin{array}{c}\text { Dec. 31, } \\ 2016\end{array}$ & $\begin{array}{l}\text { Dec. 31, } \\ 2015\end{array}$ \\
\hline
Gross Value in the opening balance & $€ 32,689$ & $€ 32,606$ & $€ 30,271$ \\
\hline
Acquisitions & 38 & 904 & 2,333 \\
\hline
Disposals & 0 & (6) & $(69)$ \\
\hline
Translation adjustment & $(40)$ & $(815)$ & 73 \\
\hline
Reclassifications and other items & 0 & 0 & $(2)$ \\
\hline
Reclassification to assets held for sale & 0 & 0 & 0 \\
\hline
Gross Value in Closing Balance & 32,687 & 32,689 & 32,606 \\
\hline
$\begin{array}{l}\text { Accumulated Impairment losses in the } \\ \text { opening balance }\end{array}$ & $(5,533)$ & $(5,535)$ & $(5,487)$ \\
\hline
Impairment & $(20)$ & $(772)$ & 0 \\
\hline
Disposals & 0 & 0 & 0 \\
\hline
Translation adjustment & (39) & 774 & $(48)$ \\
\hline
Reclassifications and other items & 0 & 0 & 0 \\
\hline
Reclassification to assets held for sale & 0 & 0 & 0 \\
\hline
$\begin{array}{l}\text { Accumulated Impairment losses in the } \\ \text { closing balance }\end{array}$ & $€(5,592)$ & $€(5,533)$ & $€(5,535)$ \\
\hline
Net book value of goodwill & 27,095 & 27,156 & 27,071 \\
\hline
\end{tabular}
\end{center}

Excerpt" from Note 7.3 Key assumptions used to determine recoverable amounts as of 31 December 2017

The parameters used for the determination of recoverable amount of the main consolidated operations are set forth below:

\begin{center}
\begin{tabular}{lccccc}
\hline
 & France & Spain & Poland & Belgium & Romania \\
\hline
Perpetuity growth rate & $0.8 \%$ & $1.5 \%$ & $1.0 \%$ & $0.5 \%$ & $2.3 \%$ \\
Post-tax discount rate & $5.5 \%$ & $8.6 \%$ & $8.3 \%$ & $6.8 \%$ & $8.8 \%$ \\
\hline
\end{tabular}
\end{center}

Excerpt" from Note 7.4 Sensitivity of recoverable amounts as of 31 December 2017

The level of sensitivity presented allows readers of the financial statements to estimate the impact in their own assessment.

\begin{center}
\begin{tabular}{lccccc}
(in billions of euros) & France & Spain & Poland & Belgium & Romania \\
\hline
$\begin{array}{l}\text { Decrease by 1\% in perpetuity } \\ \text { growth rate }\end{array}$ & 10.4 & 1.6 & 0.6 & 0.3 & 0.3 \\
$\begin{array}{l}\text { An increase by 1\% in post-tax } \\ \text { discount rate }\end{array}$ & 11.4 & 2.0 & 0.6 & 0.3 & 0.3 \\
\hline
\end{tabular}
\end{center}

\begin{itemize}
  \item Table extracted presents only selected assumptions and selected countries.
\end{itemize}

The company's annual report provides more detail.

Goodwill is not amortized. It is tested for impairment at least annually and more frequently when there is an indication that it may be impaired .... These tests are performed at the level of each Cash Generating Unit (CGU) (or group of CGUs)... To determine whether an impairment loss should be recognized, the carrying value of the assets and liabilities of the CGUs or groups of CGUs is compared to recoverable amount, for which Orange uses mostly the value in use.... Value in use is the present value of the future expected cash flows. Cash flow projections are based on economic and regulatory assumptions, license renewal assumptions and forecast trading and investment activity drawn up by the Group's management...

\section{Excerpt from Note 8.3 Other intangible assets - Net book value}
\begin{center}
\begin{tabular}{lccc}
\hline
 & \multicolumn{3}{c}{December 31} \\
\hline
(in millions of euros) & $\mathbf{2 0 1 7}$ & $\mathbf{2 0 1 6}$ & $\mathbf{2 0 1 5}$ \\
\hline
Telecommunications licenses & 6,233 & 6,440 & 5,842 \\
Orange brand & 3,133 & 3,133 & 3,133 \\
Other brands & 88 & 102 & 137 \\
Customer bases & 555 & 703 & 729 \\
Software & 3,946 & 3,781 & 3,815 \\
Other intangible assets & 384 & 443 & 671 \\
Total & $€ 14,339$ & $€ 14,602$ & $€ 14,327$ \\
\hline
\end{tabular}
\end{center}

Excerpt from Note 8.4 Property, plant and equipment - Net book value

December 31

\begin{center}
\begin{tabular}{lccc}
\hline
(in millions of euros) & $\mathbf{2 0 1 7}$ & $\mathbf{2 0 1 6}$ & $\mathbf{2 0 1 5}$ \\
\hline
Land and buildings & 2,535 & 2,661 & 2,733 \\
Network and terminals & 22,880 & 21,984 & 21,194 \\
IT equipment & 802 & 784 & 787 \\
Other property, plant and equipment & 448 & 483 & 409 \\
\cline { 2 - 4 }
Total & $€ 26,665$ & $€ 25,912$ & $€ 25,123$ \\
\hline
\end{tabular}
\end{center}

\section{Exhibit 10: Orange}
Excerpt from the 2017 Analysis of the Group's financial position and earnings

"Orange group operating income stood at 4,077 million euros in 2016, compared with 4,742 million euros in 2015 on a historical basis, a drop of $14.0 \%$ or 665 million euros. This drop on a historical basis was largely attributable to:

\begin{itemize}
  \item the recognition, in 2016, of 772 million euros in impairment loss of goodwill ... and 207 million euros in impairment loss of fixed assets ... primarily relating to:

  \item Poland for 507 million euros. This impairment loss mainly reflects a decline in competitiveness in the ADSL market, a deterioration in revenue assumptions in the mobile market and an increase in the post-tax discount rate due to the downgrading of the country's sovereign rating by the rating agencies,

  \item Egypt for 232 million euros. This impairment loss reflects the financial terms of the $4 \mathrm{G}$ license awarded in 2016, the sharp depreciation of the Egyptian pound and increased political and economic uncertainty,

  \item in the Congo (DRC), for 109 million euros. This impairment loss reflects political and economic uncertainty, a decline in purchasing power with a knock-on effect on the consumption of telecommunications products and services and an increased regulatory burden (particularly connected with the implementation of customer identification),

  \item Cameroon for 90 million euros. This impairment loss reflects a decline in voice revenues following the surge in messaging services and in VoIP of Over-The-Top (OTT) providers and heightened competition in the mobile market,

  \item and Niger for 26 million euros;

  \item and the 263 million euro increase in depreciation and amortization 1. What proportion of Orange's total assets as of December 31, 2017, is represented by goodwill and other intangible assets?

\end{itemize}

Solution to 1:

As of 31 December 2017, goodwill represents $28 \%(=27,095 \div 97,714)$ of Orange's total assets. Other intangible assets represent $15 \%(=14,339 \div$ 97,714). Data are from the company's balance sheet in Exhibit 8.

\begin{enumerate}
  \setcounter{enumi}{1}
  \item What is the largest component of the company's impairment losses during the year ending December 2016?
\end{enumerate}

\section{Solution to 2:}
The largest component of the $€ 772$ impairment loss on goodwill and the $€$ 207 million impairment loss of fixed assets related to a $€ 507$ million loss in Poland. The company attributed the loss to a decline in the competitiveness of the market for its ADSL technology, a reduction in revenue assumptions, and an increase in the discount rate resulting from the downgrading of the country's debt rating. From Exhibit 10.

[The company's financial statements define ADSL (Asymmetrical Digital Subscriber Line) as a "broadband data transmission technology on the traditional telephone network. It enables broadband data transmission (first and foremost Internet access) via twisted paired copper cable (the most common type of telephone line found in buildings)."]

\begin{enumerate}
  \setcounter{enumi}{2}
  \item The company discloses that it determines whether an impairment loss should be recognized by comparing the carrying value of a unit's assets and liabilities to the "recoverable amount" for which the company uses mostly the value in use. How does the company determine value in use?
\end{enumerate}

\section{Solution to 3:}
The company determines value in use - which it uses as a unit's assets and liabilities "recoverable amount" in impairment testing - as the present value of the future expected cash flows. The cash flow projections are based on management's assumptions. From Note 7.4 in Exhibit 9.

\begin{enumerate}
  \setcounter{enumi}{3}
  \item By what amount would the estimated recoverable value of the company's operations in France, Spain, Poland, Belgium and Romania change if the company decreased its estimate of the perpetuity growth rate by $1 \%$ ? By what amount would the estimated recoverable value of these operations change if the company increased its estimate of the post-tax discount rate by $1 \% ?$
\end{enumerate}

\section{Solution to 4:}
If the company decreased its estimate of the perpetuity growth rate by $1 \%$, the estimated recoverable value of the company's operations in France, Spain, Poland, Belgium and Romania would change by $€ 13.2$ billion $(=10.4+$ $1.6+0.6+0.3+0.3)$. A decrease in estimated growth decreases the present value of the cash flows. If the company increased its estimate of the post-tax discount rate by $1 \%$, the estimated recoverable value of these operations would change by $€ 14.6$ billion $(=11.4+2.0+0.6+0.3+0.3)$. An increase in the discount rate decreases the present value of cash flows. Data are from Note 7.4 in Exhibit 9. 5. What are the largest components of other intangible assets as of December 31, 2017? What is the largest component of property, plant and equipment as of December 31, 2017?

\section{Solution to 5:}
The largest components of other intangible assets as of December 31, 2017, are telecommunications licenses, software, and the Orange brand, reported at $€ 6,233$ million, $€ 3,946$ million, and $€ 3,133$ million, respectively. The largest component of property, plant and equipment as of December 31, 2017, is network and terminals ( $€ 22,880$ million.) Data are from Note 8.3 and 8.4 in Exhibit 9.

Note that the exhibits in the previous example contain relatively brief excerpts from the company's disclosures. The complete text of the disclosures concerning the company's non-current assets spans numerous different footnotes, some of which are several pages long. Overall, an analyst can use the disclosures to understand a company's investments in tangible and intangible assets, how those investments changed during a reporting period, how those changes affected current performance, and what those changes might indicate about future performance.

\section{USING DISCLOSURES IN ANALYSIS}
analyze and interpret financial statement disclosures regarding property, plant, and equipment and intangible assets

Ratios used in analyzing fixed assets include the fixed asset turnover ratio and several asset age ratios. The fixed asset turnover ratio (total revenue divided by average net fixed assets) reflects the relationship between total revenues and investment in PPE. The higher this ratio, the higher the amount of sales a company is able to generate with a given amount of investment in fixed assets. A higher asset turnover ratio is often interpreted as an indicator of greater efficiency.

Asset age ratios generally rely on the relationship between historical cost and depreciation. Under the revaluation model (permitted under IFRS but not US GAAP), the relationship between carrying amount, accumulated depreciation, and depreciation expense will differ when the carrying amount differs significantly from the depreciated historical cost. Therefore, the following discussion of asset age ratios applies primarily to PPE reported under the cost model.

Asset age and remaining useful life, two asset age ratios, are important indicators of a company's need to reinvest in productive capacity. The older the assets and the shorter the remaining life, the more a company may need to reinvest to maintain productive capacity. The average age of a company's asset base can be estimated as accumulated depreciation divided by depreciation expense. The average remaining life of a company's asset base can be estimated as net PPE divided by depreciation expense. These estimates simply reflect the following relationships for assets accounted for on a historical cost basis: total historical cost minus accumulated depreciation equals net PPE; and, under straight-line depreciation, total historical cost less salvage value divided by estimated useful life equals annual depreciation expense. Equivalently, total historical cost less salvage value divided by annual depreciation expense equals estimated useful life. Assuming straight-line depreciation and no salvage value (for simplicity), we have the following:

\begin{center}
\begin{tabular}{lll}
\hline
Estimated total useful life $=\quad \begin{array}{l}\text { Time elapsed since pur }-\quad+\quad \text { Estimated remaining life } \\ \text { chase (Age) }\end{array}$ &  &  \\
Historical cost $\div$ annual $=\quad$ Estimated total useful &  &  \\
depreciation expense &  &  \\
Historical cost & $=$ & Accumulated \\
 & Acpreciation &  \\
 &  & det PPE \\
\end{tabular}
\end{center}

Equivalently,

\begin{center}
\begin{tabular}{|c|c|c|c|c|}
\hline
Estimated total useful life & $=$ & $\begin{array}{l}\text { Estimated age of } \\ \text { equipment }\end{array}$ & + & Estimated remaining life \\
\hline
$\begin{array}{l}\text { Historical cost } \div \\ \text { annual depreciation } \\ \text { expense }\end{array}$ & $=$ & $\begin{array}{l}\text { Accumulated deprecia- } \\ \text { tion } \div \text { annual deprecia- } \\ \text { tion expense }\end{array}$ & + & $\begin{array}{l}\text { Net PPE } \div \\ \text { annual depreciation } \\ \text { expense }\end{array}$ \\
\hline
\end{tabular}
\end{center}

The application of these estimates can be illustrated by a hypothetical example of a company with a single depreciable asset. Assume the asset initially cost $\$ 100$, had an estimated useful life of 10 years, and an estimated salvage value of $\$ 0$. Each year, the company records a depreciation expense of $\$ 10$, so accumulated depreciation will equal $\$ 10$ times the number of years since the asset was acquired (when the asset is 7 years old, accumulated depreciation will be $\$ 70$ ). Equivalently, the age of the asset will equal accumulated depreciation divided by the annual depreciation expense.

In practice, such estimates are difficult to make with great precision. Companies use depreciation methods other than the straight-line method and have numerous assets with varying useful lives and salvage values, including some assets that are fully depreciated, so this approach produces an estimate only. Moreover, fixed asset disclosures are often quite general. Consequently, these estimates may be primarily useful to identify areas for further investigation.

One further measure compares a company's current reinvestment in productive capacity. Comparing annual capital expenditures to annual depreciation expense provides an indication of whether productive capacity is being maintained. It is a very general indicator of the rate at which a company is replacing its PPE relative to the rate at which PPE is being depreciated.

\section{EXAMPLE 17}
Using Fixed Asset Disclosure to Compare Companies' Fixed Asset Turnover and Average Age of Depreciable Assets

You are analyzing the property, plant, and equipment of three international telecommunications companies:

\begin{itemize}
  \item Orange SA, which we discussed previously, has been listed on Euronext Paris (symbol ORA) and on the New York Stock Exchange (symbol ORAN) since 1997. At December 31, 2017, the French government retained $22.95 \%$ of the share capital.

  \item BCE Inc., Canada's largest communications company, provides wireless, wireline, Internet, TV and business communications across Canada. BCE's shares are publicly traded on the Toronto Stock Exchange and on the New York Stock Exchange (TSX, NYSE: BCE). - Verizon Communications Inc. is a US-based global provider of communications, information and entertainment products and services to consumers, businesses and governmental agencies. Verizon's shares are listed on the New York Stock Exchange and the NASDAQ Global Select Market (symbol VZ).

\end{itemize}

Exhibit 11 presents selected information from the companies' financial statements.

\section{Exhibit 11}
\begin{center}
\begin{tabular}{lccc}
\hline
 & Orange & BCE Inc & Verizon \\
\hline
Currency, Millions of: & Euro $\epsilon$ & Canadian \$ & US \$ \\
\hline
Historical cost total PPE, end of year & $€ 97,092$ & $\$ 69,230$ & $\$ 246,498$ \\
Accumulated depreciation, end of year & 70,427 & 45,197 & 157,930 \\
Net PPE, end of year & 26,665 & 24,033 & 88,568 \\
Net PPE, beginning of year & 25,912 & 22,346 & 84,751 \\
Revenues & 41,096 & 22,719 & 126,034 \\
Annual depreciation expense & 4,708 & 3,037 & 14,741 \\
Capital expenditure & 5,677 & 4,149 & 17,247 \\
Land included in PPE & Not separated & Not separated & 806 \\
Accounting standards & IFRS & IFRS & US GAAP \\
PPE measurement & Historical cost & Historical cost & Historical cost \\
Depreciation method & Straight-line & Straight-line & Straight-line \\
\hline
\end{tabular}
\end{center}

Sources: Companies' 2017 Annual Financial Reports.

\begin{enumerate}
  \item Based on the above data for each company, estimate the total useful life, age, and remaining useful life of PPE.
\end{enumerate}

\section{Solution to 1:}
The following table presents the estimated total useful life, estimated age, and estimated remaining useful life of PPE for each of the companies.

\begin{center}
\begin{tabular}{lccc}
\hline
Estimates & Orange & BCE Inc & Verizon \\
\hline
Estimated total useful life (years) & 20.6 & 22.8 & 16.7 \\
Estimated age (years) & 15.0 & 14.9 & 10.7 \\
Estimated remaining life (years) & 5.7 & 7.9 & 6.0 \\
\hline
\end{tabular}
\end{center}

The computations are demonstrated using Verizon's data (\$ millions). The estimated total useful life of PPE is total historical cost of PPE of $\$ 246,498$ divided by annual depreciation expense of $\$ 14,741$, giving 16.7 years. Estimated age and estimated remaining life are obtained by dividing accumulated depreciation of $\$ 157,930$ and net PPE of $\$ 88,568$ by the annual depreciation expense of $\$ 14,741$, giving 10.7 years and 6.0 years, respectively. Ideally, the estimates of asset lives illustrated in this example should exclude land, which is not depreciable, when the information is available; however, both Orange and BCE report Land and Buildings as a combined amount. We will use Verizon, for which land appeared to be disclosed separately in the above table, to illustrate the estimates with adjusting for land. As an illustration of the calculations to exclude land, excluding Verizon's land would give an estimated total useful life for the non-land PPE of 16.7 years [(total cost $€ 246,498$ minus land cost of $\$ 806$ ) divided by annual depreciation expense of $€ 14,741$ million]. The estimate is essentially unchanged from the estimate including land because land represents such a small component of Verizon's PPE.

\begin{enumerate}
  \setcounter{enumi}{1}
  \item Interpret the estimates. What items might affect comparisons across these companies?
\end{enumerate}

\section{Solution to 2:}
The estimated total useful life suggests that Orange and BCE depreciate PPE over a much longer period than Verizon: 20.6 and 22.8 years for Orange and BCE, respectively, versus 16.7 years for Verizon.

The estimated age of the equipment suggests that Verizon has the newest PPE with an estimated age of 10.7 years. Additionally, the estimates suggest that around 73 percent of Orange's assets' useful lives have passed (15.0 years $\div 20.6$ years, or equivalently, $€ 70,427$ million $\div € 97,092$ million). In comparison, around 65 and 64 percent of the useful lives of the PPE of BCE and Verizon, respectively, have passed.

Items that can affect comparisons across the companies include business differences, such as differences in composition of the companies' operations and differences in acquisition and divestiture activity. This result can be compared, to an extent, to the useful lives and asset mix disclosed in the companies' footnotes; however, differences in disclosures, e.g. in the categories of assets disclosed, can affect comparisons.

\begin{enumerate}
  \setcounter{enumi}{2}
  \item How does each company's 2017 depreciation expense compare to its capital expenditures for the year?
\end{enumerate}

\section{Solution to 3:}
All three companies' capital expenditure exceeds its depreciation expense. Rounding to the nearest $10 \%$, capital expenditure as a percentage of depreciation is 120 percent for Orange, 140 percent for BCE, and 120 percent for Verizon. All three companies are replacing PPE at a faster rate than the PPE is being depreciated, consistent with the companies' somewhat older asset base.

\begin{enumerate}
  \setcounter{enumi}{3}
  \item Calculate and compare fixed asset turnover for each company.
\end{enumerate}

\section{Solution to 4:}
Fixed asset turnover is calculated as total revenues divided by average net PPE. Orange's fixed asset turnover is $1.6(=41,096 /((26,665+25,912) / 2)$. BCE's fixed asset turnover is 1.0, and Verizon's fixed asset turnover is 1.5. Orange's and Verizon's higher levels of fixed asset turnover indicate these companies, compared to BCE, are able to generate more sales per unit of investment in fixed assets.

\section{INVESTMENT PROPERTY}
compare the financial reporting of investment property with that of property, plant, and equipment

Investment property is defined under IFRS as property that is owned (or, in some cases, leased under a finance lease) for the purpose of earning rentals or capital appreciation or both. ${ }^{28}$ An example of investment property is a building owned by a company and leased out to tenants. In contrast, other long-lived tangible assets (i.e., property considered to be property, plant, and equipment) are owner-occupied properties used for producing the company's goods and services or for housing the company's administrative activities. Investment properties do not include long-lived tangible assets held for sale in the ordinary course of business. For example, the houses and property owned by a housing construction company are considered to be its inventory.

Under IFRS, companies are allowed to value investment properties using either a cost model or a fair value model. The cost model is identical to the cost model used for property, plant, and equipment. If the cost model is used, the fair value of investment property must be disclosed. ${ }^{29}$ The fair value model, however, differs from the revaluation model used for property, plant, and equipment. Under the revaluation model, whether an asset revaluation affects net income depends on whether the revaluation initially increases or decreases the carrying amount of the asset. In contrast, under the fair value model, all changes in the fair value of the asset affect net income. To use the fair value model, a company must be able to reliably determine the property's fair value on a continuing basis.

Example 18 presents an excerpt from the annual report of a property company reporting under IFRS.

\section{EXAMPLE 18}
\section{Financial Statement Presentation and Disclosures for Long-Lived Assets}
The following exhibit presents information and excerpts from the annual report for the year ended 31 December 2017 of intu properties plc, a property company headquartered in London that owns, develops and manages shopping centres in the United Kingdom and Spain. Its common stock is listed in London and Johannesburg.

Exhibit 12: Information and excerpts from the Annual Report of intu properties plc (Currency in $\pounds$ millions)

\section{Financial Information}
\begin{center}
\begin{tabular}{lccc}
\hline
$\begin{array}{l}\text { Financial } \\ \text { Statement }\end{array}$ & Item Label & $\begin{array}{c}\text { Amount } \\ \mathbf{2 0 1 7}\end{array}$ & $\begin{array}{c}\text { Amount } \\ \mathbf{2 0 1 6}\end{array}$ \\
\hline
Balance Sheet & $\begin{array}{c}\text { Investment and development } \\ \text { property }\end{array}$ & $9,179.4$ & $9,212.1$ \\
\hline
\end{tabular}
\end{center}

28 IAS 40 Investment Property prescribes the accounting treatment for investment property.

29 Ibid., paragraph 32.

\begin{center}
\begin{tabular}{lccc}
\hline
Financial &  & Amount & Amount \\
Statement & Item Label & $\mathbf{2 0 1 7}$ & $\mathbf{2 0 1 6}$ \\
\hline
Balance Sheet & Plant and equipment & 12.2 & 7.6 \\
Balance Sheet & Total assets & $10,794.5$ & $10,369.2$ \\
Income Statement & Net rental income & 423.4 & 406.1 \\
Income Statement & Revaluation of investment and & 30.8 & $(78.0)$ \\
 & development property &  &  \\
\hline
\end{tabular}
\end{center}

\section{Excerpt from Note 2 Accounting policies}
Investment and development property

Investment and development property is owned or leased by the Group and held for long-term rental income and capital appreciation.

The Group has elected to use the fair value model. Properties are initially recognised at cost and subsequently revalued at the balance sheet date to fair value as determined by professionally qualified external valuers on the basis of market value with the exception of certain development land where an assessment of fair value may be made internally. External valuations are received for significant development land once required planning permissions are obtained. The cost of investment and development property includes capitalised interest and other directly attributable outgoings incurred during development. Interest is capitalised on the basis of the average interest rate on the relevant debt outstanding. Interest ceases to be capitalised on the date of practical completion.

Gains or losses arising from changes in the fair value of investment and development property are recognised in the income statement. Depreciation is not provided in respect of investment and development property. Gains or losses arising on the sale of investment and development property are recognised when the significant risks and rewards of ownership have been transferred to the buyer. The gain or loss recognised is the proceeds received less the carrying value of the property and costs directly associated with the sale.

\section{Plant and equipment}
Plant and equipment consists of vehicles, fixtures, fittings and other equipment. Plant and equipment is stated at cost less accumulated depreciation and any accumulated impairment losses. Depreciation is charged to the income statement on a straight- line basis over an asset's estimated useful life up to a maximum of five years.

\section{Excerpt from Note 14 Investment and development property}
The market value of investment and development property at 31 December 2017 includes $\pounds 8,831.9$ million (31 December 2016: $\pounds 9,088.6$ million) in respect of investment property and $\pounds 376.5$ million (31 December 2016: $\pounds 153.2$ million) in respect of development property. ...All the Group's significant investment and development property relates to prime shopping centres which are of a similar nature and share characteristics and risks....

\section{Valuation methodology}
The fair value of the Group's investment and development property at 31 December 2017 was determined by independent external valuers ... Fair values for investment properties are calculated using the present value income approach. ...The key driver of the property valuations is the terms of the leases in place at the valuation date. These determine the majority of the cash flow profile of the property for a number of years and therefore form the base of the valuation... 1. How do the assets included in the balance sheet line item "Investment and development property" differ from the assets included in the balance sheet line item "Plant and equipment"?

\section{Solution to 1:}
The assets included in the balance sheet line item "Investment and development property" are shopping centres which the company holds for long-term rental income and capital appreciation. In 2017, the company reported net rental income of $\pounds 423.4$ million. The balance sheet line item "Plant and equipment" includes vehicles, fixtures, fittings, and other equipment used by the company in its operations.

\begin{enumerate}
  \setcounter{enumi}{1}
  \item How does the valuation model used by the company for its investment and development property differ from the valuation model used for its plant and equipment?
\end{enumerate}

\section{Solution to 2:}
The valuation model used by the company for its investment and development property is the fair value model, in which properties are initially recognised at cost and subsequently revalued and shown on the balance sheet at fair value. All changes in the fair value of the asset affect net income. The company employs external valuation experts to determine the fair value, which is based on expected future cash flow from rental income.

The valuation model used for its plant and equipment is the historical cost model in which properties are shown on the balance sheet at cost minus accumulated depreciation and any impairment losses.

\begin{enumerate}
  \setcounter{enumi}{2}
  \item How does accounting for depreciation differ for investment and development property versus plant and equipment?
\end{enumerate}

\section{Solution to 3:}
Depreciation in accounting refers to the allocation of the cost of a long-lived asset over its useful life. No depreciation is recorded for investment and development property. Depreciation expense for plant and equipment is calculated on a straight-line basis over the asset's estimated useful life.

\begin{enumerate}
  \setcounter{enumi}{3}
  \item Do the revaluation gains and losses on investment and development properties indicate that the properties have been sold?
\end{enumerate}

\section{Solution to 4:}
No. The revaluation gains and losses on investment properties arise from changes in the fair value of properties that are owned by the company. The company reported a revaluation gain of $€ 30.8$ million in 2017 and a revaluation loss of $€ 78.0$ million in 2016.

Sales of property would have resulted in a gain or loss on disposal, calculated as the proceeds minus the carrying value of the property and related selling costs.

In general, a company must apply its chosen model (cost or fair value) to all of its investment property. If a company chooses the fair value model for its investment property, it must continue to use the fair value model until it disposes of the property or changes its use such that it is no longer considered investment property (e.g., it becomes owner-occupied property or part of inventory). The company must continue to use the fair value model for that property even if transactions on comparable properties, used to estimate fair value, become less frequent.

Certain valuation issues arise when a company changes the use of property such that it moves from being an investment property to owner-occupied property or part of inventory. If a company's chosen model for investment property is the cost model, such transfers do not change the carrying amount of the property transferred. If a company's chosen model is the fair value model, transfers from investment property to owner-occupied property or to inventory are made at fair value. In other words, the property's fair value at the time of transfer is considered to be its cost for ongoing accounting for the property. If a company's chosen model for investment property is the fair value model and it transfers a property from owner-occupied to investment property, the change in measurement of the property from depreciated cost to fair value is treated like a revaluation. If a company's chosen model is the fair value model and it transfers a property from inventory to investment property, any difference between the inventory carrying amount and the property's fair value at the time of transfer is recognised as profit or loss.

Investment property appears as a separate line item on the balance sheet. Companies are required to disclose whether they use the fair value model or the cost model for their investment property. If the company uses the fair value model, it must make additional disclosures about how it determines fair value and must provide reconciliation between the beginning and ending carrying amounts of investment property. If the company uses the cost model, it must make additional disclosures similar to those for property, plant, and equipment-for example, the depreciation method and useful lives must be disclosed. In addition, if the company uses the cost model, it must also disclose the fair value of investment property.

Under US GAAP, there is no specific definition of investment property. Most operating companies and real estate companies in the United States that hold investment-type property use the historical cost model.

\section{SUMMARY}
Understanding the reporting of long-lived assets at inception requires distinguishing between expenditures that are capitalised (i.e., reported as long-lived assets) and those that are expensed. Once a long-lived asset is recognised, it is reported under the cost model at its historical cost less accumulated depreciation (amortisation) and less any impairment or under the revaluation model at its fair value. IFRS permit the use of either the cost model or the revaluation model, whereas US GAAP require the use of the cost model. Most companies reporting under IFRS use the cost model. The choice of different methods to depreciate (amortise) long-lived assets can create challenges for analysts comparing companies.

Key points include the following:

\begin{itemize}
  \item Expenditures related to long-lived assets are capitalised as part of the cost of assets if they are expected to provide future benefits, typically beyond one year. Otherwise, expenditures related to long-lived assets are expensed as incurred.

  \item Although capitalising expenditures, rather than expensing them, results in higher reported profitability in the initial year, it results in lower profitability in subsequent years; however, if a company continues to purchase similar or increasing amounts of assets each year, the profitability-enhancing effect of capitalisation continues. - Capitalising an expenditure rather than expensing it results in a greater amount reported as cash from operations because capitalised expenditures are classified as an investing cash outflow rather than an operating cash outflow.

  \item Companies must capitalise interest costs associated with acquiring or constructing an asset that requires a long period of time to prepare for its intended use.

  \item Including capitalised interest in the calculation of interest coverage ratios provides a better assessment of a company's solvency.

  \item IFRS require research costs be expensed but allow all development costs (not only software development costs) to be capitalised under certain conditions. Generally, US accounting standards require that research and development costs be expensed; however, certain costs related to software development are required to be capitalised.

  \item When one company acquires another company, the transaction is accounted for using the acquisition method of accounting in which the company identified as the acquirer allocates the purchase price to each asset acquired (and each liability assumed) on the basis of its fair value. Under acquisition accounting, if the purchase price of an acquisition exceeds the sum of the amounts that can be allocated to individual identifiable assets and liabilities, the excess is recorded as goodwill.

  \item The capitalised costs of long-lived tangible assets and of intangible assets with finite useful lives are allocated to expense in subsequent periods over their useful lives. For tangible assets, this process is referred to as depreciation, and for intangible assets, it is referred to as amortisation.

  \item Long-lived tangible assets and intangible assets with finite useful lives are reviewed for impairment whenever changes in events or circumstances indicate that the carrying amount of an asset may not be recoverable.

  \item Intangible assets with an indefinite useful life are not amortised but are reviewed for impairment annually.

  \item Impairment disclosures can provide useful information about a company's expected cash flows.

  \item Methods of calculating depreciation or amortisation expense include the straight-line method, in which the cost of an asset is allocated to expense in equal amounts each year over its useful life; accelerated methods, in which the allocation of cost is greater in earlier years; and the units-of-production method, in which the allocation of cost corresponds to the actual use of an asset in a particular period.

  \item Estimates required for depreciation and amortisation calculations include the useful life of the equipment (or its total lifetime productive capacity) and its expected residual value at the end of that useful life. A longer useful life and higher expected residual value result in a smaller amount of annual depreciation relative to a shorter useful life and lower expected residual value.

  \item IFRS permit the use of either the cost model or the revaluation model for the valuation and reporting of long-lived assets, but the revaluation model is not allowed under US GAAP.

  \item Under the revaluation model, carrying amounts are the fair values at the date of revaluation less any subsequent accumulated depreciation or amortisation. - In contrast with depreciation and amortisation charges, which serve to allocate the cost of a long-lived asset over its useful life, impairment charges reflect an unexpected decline in the fair value of an asset to an amount lower than its carrying amount.

  \item IFRS permit impairment losses to be reversed, with the reversal reported in profit. US GAAP do not permit the reversal of impairment losses.

  \item The gain or loss on the sale of long-lived assets is computed as the sales proceeds minus the carrying amount of the asset at the time of sale.

  \item Estimates of average age and remaining useful life of a company's assets reflect the relationship between assets accounted for on a historical cost basis and depreciation amounts.

  \item The average remaining useful life of a company's assets can be estimated as net PPE divided by depreciation expense, although the accounting useful life may not necessarily correspond to the economic useful life.

  \item Long-lived assets reclassified as held for sale cease to be depreciated or amortised. Long-lived assets to be disposed of other than by a sale (e.g., by abandonment, exchange for another asset, or distribution to owners in a spin-off) are classified as held for use until disposal. Thus, they continue to be depreciated and tested for impairment.

  \item Investment property is defined as property that is owned (or, in some cases, leased under a finance lease) for the purpose of earning rentals, capital appreciation, or both.

  \item Under IFRS, companies are allowed to value investment properties using either a cost model or a fair value model. The cost model is identical to the cost model used for property, plant, and equipment, but the fair value model differs from the revaluation model used for property, plant, and equipment. Unlike the revaluation model, under the fair value model, all changes in the fair value of investment property affect net income.

  \item Under US GAAP, investment properties are generally measured using the cost model.

\end{itemize}

\section{PRACTICE PROBLEMS}
\begin{enumerate}
  \item JOOVI Inc. has recently purchased and installed a new machine for its manufacturing plant. The company incurred the following costs:
\end{enumerate}

\begin{center}
\begin{tabular}{lr}
\hline
Purchase price & $\$ 12,980$ \\
Freight and insurance & $\$ 1,200$ \\
Installation & $\$ 700$ \\
Testing & $\$ 100$ \\
Maintenance staff training costs & $\$ 500$ \\
\hline
\end{tabular}
\end{center}

The total cost of the machine to be shown on JOOVI's balance sheet is closest to:
A. $\$ 14,180$.
B. $\$ 14,980$.
C. $\$ 15,480$.

\begin{enumerate}
  \setcounter{enumi}{1}
  \item Which costs incurred with the purchase of property and equipment are expensed?
A. Delivery charges
B. Installation and testing
C. Training required to use the property and equipment

  \item When constructing an asset for sale, directly related borrowing costs are most likely:
A. expensed as incurred.
B. capitalized as part of inventory.
C. capitalized as part of property, plant, and equipment.

  \item BAURU, S.A., a Brazilian corporation, borrows capital from a local bank to finance the construction of its manufacturing plant. The loan has the following conditions:

\end{enumerate}

Borrowing date

Amount borrowed

Annual interest rate

Term of the loan

Payment method 1 January 2009

500 million Brazilian real (BRL)

14 percent

3 years

Annual payment of interest only. Principal amortization is due at the end of the loan term.

The construction of the plant takes two years, during which time BAURU earned BRL 10 million by temporarily investing the loan proceeds. Which of the following is the amount of interest related to the plant construction (in BRL million) that can be capitalized in BAURU's balance sheet?
A. 130
B. 140 . C. 210 .

\begin{enumerate}
  \setcounter{enumi}{4}
  \item After reading the financial statements and footnotes of a company that follows IFRS, an analyst identified the following intangible assets:
\end{enumerate}

\begin{itemize}
  \item product patent expiring in 40 years;

  \item copyright with no expiration date; and

  \item goodwill acquired 2 years ago in a business combination.

\end{itemize}

Which of these assets is an intangible asset with a finite useful life?

\begin{center}
\begin{tabular}{ccccc}
 & Product Patent & Copyright & Goodwill &  \\
\cline { 2 - 2 }
 & Yes & Yes & Yes & No \\
B & No & No & No &  \\
C & Yes & Yes &  &  \\
\end{tabular}
\end{center}

\begin{enumerate}
  \setcounter{enumi}{5}
  \item Intangible assets with finite useful lives mostly differ from intangible assets with infinite useful lives with respect to accounting treatment of:
A. revaluation.
B. impairment.
C. amortization.

  \item Costs incurred for intangible assets are generally expensed when they are:
A. internally developed.
B. individually acquired.
C. acquired in a business combination.

  \item Under US GAAP, when assets are acquired in a business combination, goodwill most likely arises from:

\end{enumerate}

A. contractual or legal rights.

B. assets that can be separated from the acquired company.

C. assets that are neither tangible nor identifiable intangible assets.

\begin{enumerate}
  \setcounter{enumi}{8}
  \item All else equal, in the fiscal year when long-lived equipment is purchased:
\end{enumerate}

A. depreciation expense increases.

B. cash from operations decreases.

C. net income is reduced by the amount of the purchase.

\begin{enumerate}
  \setcounter{enumi}{9}
  \item Companies $X$ and $Z$ have the same beginning-of-the-year book value of equity and the same tax rate. The companies have identical transactions throughout the year and report all transactions similarly except for one. Both companies acquire a $\pounds 300,000$ printer with a three-year useful life and a salvage value of $\pounds 0$ on 1 January of the new year. Company $\mathrm{X}$ capitalizes the printer and depreciates it on a straight-line basis, and Company $\mathrm{Z}$ expenses the printer. The following year-end information is gathered for Company X.
\end{enumerate}

\begin{center}
\begin{tabular}{lc}
\hline
 & $\begin{array}{c}\text { Company } \mathbf{X} \\ \text { As of } 31 \text { December }\end{array}$ \\
\hline
Ending shareholders' equity & $\pounds 10,000,000$ \\
Tax rate &  \\
Dividends & $25 \%$ \\
Net income & $\pounds 0.00$ \\
\hline
\end{tabular}
\end{center}

Based on the information given, Company Z's return on equity using year-end equity will be closest to:
A. $5.4 \%$.
B. $6.1 \%$.
C. $7.5 \%$.

\section{The following information relates to questions}
\section{$11-14$}
Melanie Hart, CFA, is a transportation analyst. Hart has been asked to write a research report on Altai Mountain Rail Company (AMRC). Like other companies in the railroad industry, AMRC's operations are capital intensive, with significant investments in such long-lived tangible assets as property, plant, and equipment. In November of 2008, AMRC's board of directors hired a new team to manage the company. In reviewing the company's 2009 annual report, Hart is concerned about some of the accounting choices that the new management has made. These choices differ from those of the previous management and from common industry practice. Hart has highlighted the following statements from the company's annual report:

Statement 1 “In 2009, AMRC spent significant amounts on track replacement and similar improvements. AMRC expensed rather than capitalised a significant proportion of these expenditures."

Statement 2 "AMRC uses the straight-line method of depreciation for both financial and tax reporting purposes to account for plant and equipment."

Statement 3 “In 2009, AMRC recognized an impairment loss of $€ 50$ million on a fleet of locomotives. The impairment loss was reported as 'other income' in the income statement and reduced the carrying amount of the assets on the balance sheet."

Exhibit 1 and Exhibit 2 contain AMRC's 2009 consolidated income statement and balance sheet. AMRC prepares its financial statements in accordance with International Financial Reporting Standards. Exhibit 1: Consolidated Statement of Income

\begin{center}
\begin{tabular}{|c|c|c|c|c|}
\hline
\multirow[b]{2}{*}{For the Years Ended 31 December} & \multicolumn{2}{|c|}{2009} & \multicolumn{2}{|c|}{2008} \\
\hline
 & $€$ Millions & $\%$ Revenues & $€$ Millions & \% Revenues \\
\hline
Operating revenues & 2,600 & 100.0 & 2,300 & 100.0 \\
\hline
\multicolumn{5}{|l|}{Operating expenses} \\
\hline
Depreciation & $(200)$ & $(7.7)$ & $(190)$ & $(8.3)$ \\
\hline
Other operating expense & $(1,590)$ & $(61.1)$ & $(1,515)$ & $(65.9)$ \\
\hline
Total operating expenses & $(1,790)$ & $(68.8)$ & $(1,705)$ & $(74.2)$ \\
\hline
Operating income & 810 & 31.2 & 595 & 25.8 \\
\hline
Other income & $(50)$ & $(1.9)$ & - & 0.0 \\
\hline
Interest expense & $(73)$ & $(2.8)$ & $(69)$ & $(3.0)$ \\
\hline
Income before taxes & 687 & 26.5 & 526 & 22.8 \\
\hline
Income taxes & $(272)$ & $(10.5)$ & $(198)$ & $(8.6)$ \\
\hline
Net income & 415 & 16 & 328 & 14.2 \\
\hline
\end{tabular}
\end{center}

Exhibit 2: Consolidated Balance Sheet

As of 31 December

2009

2008

\begin{center}
\begin{tabular}{|c|c|c|c|c|}
\hline
Assets & $€$ Millions & $\%$ Assets & $€$ Millions & $\%$ Assets \\
\hline
Current assets & 500 & 9.4 & 450 & 8.5 \\
\hline
\multicolumn{5}{|l|}{Property \& equipment:} \\
\hline
Land & 700 & 13.1 & 700 & 13.2 \\
\hline
Plant \& equipment & 6,000 & 112.1 & 5,800 & 109.4 \\
\hline
Total property \& equipment & 6,700 & 125.2 & 6,500 & 122.6 \\
\hline
Accumulated depreciation & $(1,850)$ & $(34.6)$ & $(1,650)$ & $(31.1)$ \\
\hline
Net property \& equipment & 4,850 & 90.6 & 4,850 & 91.5 \\
\hline
Total assets & 5,350 & 100.0 & 5,300 & 100.0 \\
\hline
\multicolumn{5}{|l|}{Liabilities and Shareholders' Equity} \\
\hline
Current liabilities & 480 & 9.0 & 430 & 8.1 \\
\hline
Long-term debt & 1,030 & 19.3 & 1,080 & 20.4 \\
\hline
Other long-term provisions and liabilities & 1,240 & 23.1 & 1,440 & 27.2 \\
\hline
Total liabilities & 2,750 & 51.4 & 2,950 & 55.7 \\
\hline
\multicolumn{5}{|l|}{Shareholders' equity} \\
\hline
Common stock and paid-in-surplus & 760 & 14.2 & 760 & 14.3 \\
\hline
Retained earnings & 1,888 & 35.5 & 1,600 & 30.2 \\
\hline
Other comprehensive losses & $(48)$ & $(0.9)$ & $(10)$ & $(0.2)$ \\
\hline
Total shareholders' equity & 2,600 & 48.6 & 2,350 & 44.3 \\
\hline
Total liabilities \& shareholders' equity & 5,350 & 100.0 & 5,300 & 100.0 \\
\hline
\end{tabular}
\end{center}

\begin{enumerate}
  \setcounter{enumi}{10}
  \item With respect to Statement 1 , which of the following is the most likely effect of management's decision to expense rather than capitalise these expenditures?
\end{enumerate}

A. 2009 net profit margin is higher than if the expenditures had been capitalised.

B. 2009 total asset turnover is lower than if the expenditures had been capitalised.

C. Future profit growth will be higher than if the expenditures had been capitalised.

\begin{enumerate}
  \setcounter{enumi}{11}
  \item With respect to Statement 2, what would be the most likely effect in 2010 if AMRC were to switch to an accelerated depreciation method for both financial and tax reporting?
A. Net profit margin would increase.
B. Total asset turnover would decrease.
C. Cash flow from operating activities would increase.

  \item With respect to Statement 3, what is the most likely effect of the impairment loss?

\end{enumerate}

A. Net income in years prior to 2009 was likely understated.

B. Net profit margins in years after 2009 will likely exceed the 2009 net profit margin.

C. Cash flow from operating activities in 2009 was likely lower due to the impairment loss.

\begin{enumerate}
  \setcounter{enumi}{13}
  \item Based on Exhibits 1 and 2, the best estimate of the average remaining useful life of the company's plant and equipment at the end of 2009 is:
A. 20.75 years
B. 24.25 years.
C. 30.00 years.
\end{enumerate}

\section{The following information relates to questions}
\section{5-20}
Brian Jordan is interviewing for a junior equity analyst position at Orion Investment Advisors. As part of the interview process, Mary Benn, Orion's Director of Research, provides Jordan with information about two hypothetical companies, Alpha and Beta, and asks him to comment on the information on their financial statements and ratios. Both companies prepare their financial statements in accordance with International Financial Reporting Standards (IFRS) and are identical in all respects except for their accounting choices.

Jordan is told that at the beginning of the current fiscal year, both companies purchased a major new computer system and began building new manufacturing plants for their own use. Alpha capitalised and Beta expensed the cost of the computer system; Alpha capitalised and Beta expensed the interest costs associated with the construction of the manufacturing plants.

Benn asks Jordan, "What was the impact of these decisions on each company's current fiscal year financial statements and ratios?"

Jordan responds, "Alpha's decision to capitalise the cost of its new computer system instead of expensing it results in lower net income, lower total assets, and higher cash flow from operating activities in the current fiscal year. Alpha's decision to capitalise its interest costs instead of expensing them results in a lower fixed asset turnover ratio and a higher interest coverage ratio."

Jordan is told that Alpha uses the straight-line depreciation method and Beta uses an accelerated depreciation method; both companies estimate the same useful lives for long-lived assets. Many companies in their industry use the units-of-production method.

Benn asks Jordan, "What are the financial statement implications of each depreciation method, and how do you determine a company's need to reinvest in its productive capacity?"

Jordan replies, "All other things being equal, the straight-line depreciation method results in the least variability of net profit margin over time, while an accelerated depreciation method results in a declining trend in net profit margin over time. The units-of-production can result in a net profit margin trend that is quite variable. I use a three-step approach to estimate a company's need to reinvest in its productive capacity. First, I estimate the average age of the assets by dividing net property, plant, and equipment by annual depreciation expense. Second, I estimate the average remaining useful life of the assets by dividing accumulated depreciation by depreciation expense. Third, I add the estimates of the average remaining useful life and the average age of the assets in order to determine the total useful life."

Jordan is told that at the end of the current fiscal year, Alpha revalued a manufacturing plant; this increased its reported carrying amount by 15 percent. There was no previous downward revaluation of the plant. Beta recorded an impairment loss on a manufacturing plant; this reduced its carrying by 10 percent.

Benn asks Jordan "What was the impact of these decisions on each company's current fiscal year financial ratios?"

Jordan responds, "Beta's impairment loss increases its debt to total assets and fixed asset turnover ratios, and lowers its cash flow from operating activities. Alpha's revaluation increases its debt to capital and return on assets ratios, and reduces its return on equity."

At the end of the interview, Benn thanks Jordan for his time and states that a hiring decision will be made shortly.

\begin{enumerate}
  \setcounter{enumi}{14}
  \item Jordan's response about the financial statement impact of Alpha's decision to capitalise the cost of its new computer system is most likely correct with respect to:
\end{enumerate}

A. lower net income.

B. lower total assets.

C. higher cash flow from operating activities.

\begin{enumerate}
  \setcounter{enumi}{15}
  \item Jordan's response about the ratio impact of Alpha's decision to capitalise interest costs is most likely correct with respect to the:
\end{enumerate}

A. interest coverage ratio.

B. fixed asset turnover ratio.

C. interest coverage and fixed asset turnover ratios. 17. Jordan's response about the impact of the different depreciation methods on net profit margin is most likely incorrect with respect to:
A. accelerated depreciation.
B. straight-line depreciation.
C. units-of-production depreciation.

\begin{enumerate}
  \setcounter{enumi}{17}
  \item Jordan's response about his approach to estimating a company's need to reinvest in its productive capacity is most likely correct regarding:
A. estimating the average age of the asset base.
B. estimating the total useful life of the asset base.
C. estimating the average remaining useful life of the asset base.

  \item Jordan's response about the effect of Beta's impairment loss is most likely incorrect with respect to the impact on its:
A. debt to total assets.
B. fixed asset turnover.
C. cash flow from operating activities.

  \item Jordan's response about the effect of Alpha's revaluation is most likely correct with respect to the impact on its:
A. return on equity.
B. return on assets.
C. debt to capital ratio.

  \item A financial analyst is studying the income statement effect of two alternative depreciation methods for a recently acquired piece of equipment. She gathers the following information about the equipment's expected production life and use:

\end{enumerate}

\begin{center}
\begin{tabular}{lcccccc}
\hline
 & Year 1 & Year 2 & Year 3 & Year 4 & Year 5 & Total \\
\hline
Units of production & 2,000 & 2,000 & 2,000 & 2,000 & 2,500 & 10,500 \\
\hline
\end{tabular}
\end{center}

Compared with the units-of-production method of depreciation, if the company uses the straight-line method to depreciate the equipment, its net income in Year 1 will most likely be:
A. lower.
B. higher.
C. the same.

\begin{enumerate}
  \setcounter{enumi}{21}
  \item A company purchases a piece of equipment for $€ 1,500$. The equipment is expected to have a useful life of five years and no residual value. In the first year of use, the units of production are expected to be $15 \%$ of the equipment's lifetime production capacity and the equipment is expected to generate $€ 1,500$ of revenue and incur $€ 500$ of cash expenses.
\end{enumerate}

The depreciation method yielding the lowest operating profit on the equipment in the first year of use is:
A. straight line.
B. units of production.
C. double-declining balance.

\begin{enumerate}
  \setcounter{enumi}{22}
  \item Juan Martinez, CFO of VIRMIN, S.A., is selecting the depreciation method to use for a new machine. The machine has an expected useful life of six years. Production is expected to be relatively low initially but to increase over time. The method chosen for tax reporting must be the same as the method used for financial reporting. If Martinez wants to minimize tax payments in the first year of the machine's life, which of the following depreciation methods is Martinez most likely to use?
\end{enumerate}

A. Straight-line method.

B. Units-of-production method.

C. Double-declining balance method.

\section{The following information relates to questions}
\section{4-25}
Miguel Rodriguez of MARIO S.A., an Uruguayan corporation, is computing the depreciation expense of a piece of manufacturing equipment for the fiscal year ended 31 December 2009. The equipment was acquired on 1 January 2009. Rodriguez gathers the following information (currency in Uruguayan pesos, UYP):

Cost of the equipment

Estimated residual value

Expected useful life

Total productive capacity

Production in FY 2009

Expected production for the next 7 years UYP 1,200,000

UYP 200,000

8 years

800,000 units

135,000 units

95,000 units each year

\begin{enumerate}
  \setcounter{enumi}{23}
  \item If MARIO uses the straight-line method, the amount of depreciation expense on MARIO's income statement related to the manufacturing equipment is closest to:
A. 125,000 .
B. 150,000 .
C. 168,750 .

  \item If MARIO uses the units-of-production method, the amount of depreciation expense (in UYP) on MARIO's income statement related to the manufacturing equipment is closest to:
A. 118,750 .
B. 168,750 .
C. 202,500 . 26. A company purchases equipment for $\$ 200,000$ with a five-year useful life and salvage value of zero. It uses the double-declining balance method of depreciation for two years, then shifts to straight-line depreciation at the beginning of Year 3. Compared with annual depreciation expense under the double-declining balance method, the resulting annual depreciation expense in Year 4 is:
A. smaller.
B. the same.
C. greater.

  \item Which of the following amortization methods is most likely to evenly distribute the cost of an intangible asset over its useful life?
A. Straight-line method.
B. Units-of-production method.
C. Double-declining balance method.

  \item Which of the following will cause a company to show a lower amount of amortization of intangible assets in the first year after acquisition?

\end{enumerate}

A. A higher residual value.

B. A higher amortization rate.

C. A shorter useful life.

\begin{enumerate}
  \setcounter{enumi}{28}
  \item An analyst in the finance department of BOOLDO S.A., a French corporation, is computing the amortization of a customer list, an intangible asset, for the fiscal year ended 31 December 2009. She gathers the following information about the asset:
\end{enumerate}

Acquisition cost

Acquisition date

Expected residual value at time of acquisition $€ 2,300,000$

1 January 2008

$€ 500,000$

The customer list is expected to result in extra sales for three years after acquisition. The present value of these expected extra sales exceeds the cost of the list.

If the analyst uses the straight-line method, the amount of accumulated amortization related to the customer list as of 31 December 2009 is closest to:
A. $€ 600,000$.
B. $€ 1,200,000$.
C. $€ 1,533,333$.

\begin{enumerate}
  \setcounter{enumi}{29}
  \item A financial analyst is analyzing the amortization of a product patent acquired by MAKETTI S.p.A., an Italian corporation. He gathers the following information about the patent:
\end{enumerate}

Acquisition cost

Acquisition date

Patent expiration date

Total plant capacity of patented product $€ 5,800,000$

1 January 2009

31 December 2015

40,000 units per year Production of patented product in fiscal year ended $31 \quad 20,000$ units

December 2009

Expected production of patented product during life of the $\quad 175,000$ units patent

If the analyst uses the units-of-production method, the amortization expense on the patent for fiscal year 2009 is closest to:
A. $€ 414,286$.
B. $€ 662,857$.
C. $€ 828,571$.

\begin{enumerate}
  \setcounter{enumi}{30}
  \item A company acquires a patent with an expiration date in six years for $¥ 100$ million. The company assumes that the patent will generate economic benefits that will decline over time and decides to amortize the patent using the double-declining balance method. The annual amortization expense in Year 4 is closest to:
A. $¥ 6.6$ million.
B. $¥ 9.9$ million.
C. $¥ 19.8$ million.

  \item A company is comparing straight-line and double-declining balance amortization methods for a non-renewable six-year license, acquired for $€ 600,000$. The difference between the Year 4 ending net book values using the two methods is closest to:
A. $€ 81,400$.
B. $€ 118,600$.
C. $€ 200,000$.

  \item MARU S.A. de C.V., a Mexican corporation that follows IFRS, has elected to use the revaluation model for its property, plant, and equipment. One of MARU's machines was purchased for 2,500,000 Mexican pesos (MXN) at the beginning of the fiscal year ended 31 March 2010. As of 31 March 2010, the machine has a fair value of MXN 3,000,000. Should MARU show a profit for the revaluation of the machine?

\end{enumerate}

A. Yes.

B. No, because this revaluation is recorded directly in equity.

C. No, because value increases resulting from revaluation can never be recognized as a profit.

\begin{enumerate}
  \setcounter{enumi}{33}
  \item An analyst is studying the impairment of the manufacturing equipment of WLP Corp., a UK-based corporation that follows IFRS. He gathers the following information about the equipment:
\end{enumerate}

Fair value

Costs to sell

Value in use

Net carrying amount $\pounds 16,800,000$

$\pounds 800,000$

$\pounds 14,500,000$

$\pounds 19,100,000$ The amount of the impairment loss on WLP Corp.'s income statement related to its manufacturing equipment is closest to:
A. $\pounds 2,300,000$.
B. $\pounds 3,100,000$.
C. $\pounds 4,600,000$.

\begin{enumerate}
  \setcounter{enumi}{34}
  \item Under IFRS, an impairment loss on a property, plant, and equipment asset is measured as the excess of the carrying amount over the asset's:
A. fair value.
B. recoverable amount.
C. undiscounted expected future cash flows.

  \item The impairment of intangible assets with finite lives affects:
A. the balance sheet but not the income statement.
B. the income statement but not the balance sheet.
C. both the balance sheet and the income statement.

  \item A financial analyst at BETTO S.A. is analyzing the result of the sale of a vehicle for 85,000 Argentine pesos (ARP) on 31 December 2009. The analyst compiles the following information about the vehicle:

\end{enumerate}

\begin{center}
\begin{tabular}{ll}
\hline
Acquisition cost of the vehicle & ARP 100,000 \\
Acquisition date & 1 January 2007 \\
Estimated residual value at acquisition date & ARP 10,000 \\
Expected useful life & 9 years \\
Depreciation method & Straight-line \\
\hline
\end{tabular}
\end{center}

The result of the sale of the vehicle is most likely:
A. a loss of ARP 15,000 .
B. a gain of ARP 15,000.
C. a gain of ARP 18,333 .

\begin{enumerate}
  \setcounter{enumi}{37}
  \item CROCO S.p.A sells an intangible asset with a historical acquisition cost of $€ 12$ million and an accumulated amortization of $€ 2$ million and reports a loss on the sale of $€ 3.2$ million. Which of the following amounts is most likely the sale price of the asset?
A. $€ 6.8$ million
B. $€ 8.8$ million
C. $€ 13.2$ million

  \item The gain or loss on a sale of a long-lived asset to which the revaluation model has been applied is most likely calculated using sales proceeds less:

\end{enumerate}

A. carrying amount. B. carrying amount adjusted for impairment.

C. historical cost net of accumulated depreciation.

\begin{enumerate}
  \setcounter{enumi}{39}
  \item According to IFRS, all of the following pieces of information about property, plant, and equipment must be disclosed in a company's financial statements and footnotes except for:
\end{enumerate}

A. useful lives.

B. acquisition dates.

C. amount of disposals.

\begin{enumerate}
  \setcounter{enumi}{40}
  \item According to IFRS, all of the following pieces of information about intangible assets must be disclosed in a company's financial statements and footnotes except for:
\end{enumerate}

A. fair value.

B. impairment loss.

C. amortization rate.

\begin{enumerate}
  \setcounter{enumi}{41}
  \item Which of the following is a required financial statement disclosure for long-lived intangible assets under US GAAP?
\end{enumerate}

A. The useful lives of assets

B. The reversal of impairment losses

C. Estimated amortization expense for the next five fiscal years

\begin{enumerate}
  \setcounter{enumi}{42}
  \item Which of the following characteristics is most likely to differentiate investment property from property, plant, and equipment?
A. It is tangible.
B. It earns rent.
C. It is long-lived.

  \item If a company uses the fair value model to value investment property, changes in the fair value of the asset are least likely to affect:
A. net income.
B. net operating income.
C. other comprehensive income.

  \item Investment property is most likely to:
A. earn rent.
B. be held for resale.
C. be used in the production of goods and services. 46. A company is most likely to:

\end{enumerate}

A. use a fair value model for some investment property and a cost model for other investment property.

B. change from the fair value model when transactions on comparable properties become less frequent.

C. change from the fair value model when the company transfers investment property to property, plant, and equipment.

\begin{enumerate}
  \setcounter{enumi}{46}
  \item Under the revaluation model for property, plant, and equipment and the fair model for investment property:
\end{enumerate}

A. fair value of the asset must be able to be measured reliably.

B. net income is affected by all changes in the fair value of the asset.

C. net income is never affected if the asset increases in value from its carrying amount.

\begin{enumerate}
  \setcounter{enumi}{47}
  \item Under IFRS, what must be disclosed under the cost model of valuation for investment properties?
A. Useful lives
B. The method for determining fair value
C. Reconciliation between beginning and ending carrying amounts of invest- ment property
\end{enumerate}

\section{SOLUTIONS}
\begin{enumerate}
  \item B is correct. Only costs necessary for the machine to be ready to use can be capitalized. Therefore, Total capitalized costs $=12,980+1,200+700+100=\$ 14,980$.

  \item $\mathrm{C}$ is correct. When property and equipment are purchased, the assets are recorded on the balance sheet at cost. Costs for the assets include all expenditures required to prepare the assets for their intended use. Any other costs are expensed. Costs to train staff for using the machine are not required to prepare the property and equipment for their intended use, and these costs are expensed.

  \item B is correct. When a company constructs an asset, borrowing costs incurred directly related to the construction are generally capitalized. If the asset is constructed for sale, the borrowing costs are classified as inventory.

  \item A is correct. Borrowing costs can be capitalized under IFRS until the tangible asset is ready for use. Also, under IFRS, income earned on temporarily investing the borrowed monies decreases the amount of borrowing costs eligible for capitalization. Therefore, Total capitalized interest $=(500$ million $\times 14 \% \times 2$ years $)-$ 10 million $=130$ million.

  \item B is correct. A product patent with a defined expiration date is an intangible asset with a finite useful life. A copyright with no expiration date is an intangible asset with an indefinite useful life. Goodwill is no longer considered an intangible asset under IFRS and is considered to have an indefinite useful life.

  \item $\mathrm{C}$ is correct. An intangible asset with a finite useful life is amortized, whereas an intangible asset with an indefinite useful life is not.

  \item A is correct. The costs to internally develop intangible assets are generally expensed when incurred.

  \item $\mathrm{C}$ is correct. Under both International Financial Reporting Standards (IFRS) and US GAAP, if an item is acquired in a business combination and cannot be recognized as a tangible asset or identifiable intangible asset, it is recognized as goodwill. Under US GAAP, assets arising from contractual or legal rights and assets that can be separated from the acquired company are recognized separately from goodwill.

  \item A is correct. In the fiscal year when long-lived equipment is purchased, the assets on the balance sheet increase and depreciation expense on the income statement increases because of the new long-lived asset.

  \item B is correct. Company Z's return on equity based on year-end equity value will be $6.1 \%$. Company $\mathrm{Z}$ will have an additional $\pounds 200,000$ of expenses compared with Company X. Company Z expensed the printer for $\pounds 300,000$ rather than capitalizing the printer and having a depreciation expense of $\pounds 100,000$ like Company X. Company Z's net income and shareholders' equity will be $\pounds 150,000$ lower (= $\pounds 200,000 \times 0.75)$ than that of Company X.

\end{enumerate}

$$
\begin{aligned}
& \text { ROE }=\left(\frac{\text { Net income }}{\text { Shareholders' Equity }}\right) \\
& =\pounds 600,000 / \pounds 9,850,000 \\
& =0.61=6.1 \%
\end{aligned}
$$

\begin{enumerate}
  \setcounter{enumi}{10}
  \item $\mathrm{C}$ is correct. Expensing rather than capitalising an investment in long-term assets will result in higher expenses and lower net income and net profit margin in the current year. Future years' incomes will not include depreciation expense related to these expenditures. Consequently, year-to-year growth in profitability will be higher. If the expenses had been capitalised, the carrying amount of the assets would have been higher and the 2009 total asset turnover would have been lower.

  \item C is correct. In 2010, switching to an accelerated depreciation method would increase depreciation expense and decrease income before taxes, taxes payable, and net income. Cash flow from operating activities would increase because of the resulting tax savings.

  \item B is correct. 2009 net income and net profit margin are lower because of the impairment loss. Consequently, net profit margins in subsequent years are likely to be higher. An impairment loss suggests that insufficient depreciation expense was recognized in prior years, and net income was overstated in prior years. The impairment loss is a non-cash item and will not affect operating cash flows.

  \item A is correct. The estimated average remaining useful life is 20.75 years.

\end{enumerate}

Estimate of remaining useful life $=$ Net plant and equipment $\div$ Annual depreciation expense

Net plant and equipment $=$ Gross P \& E - Accumulated depreciation

$=€ 6000-€ 1850=€ 4150$

Estimate of remaining useful life $=$ Net $\mathrm{P} \& \mathrm{E} \div$ Depreciation expense

$$
=€ 4150 \div € 200=20.75
$$

\begin{enumerate}
  \setcounter{enumi}{14}
  \item C is correct. The decision to capitalise the costs of the new computer system results in higher cash flow from operating activities; the expenditure is reported as an outflow of investing activities. The company allocates the capitalised amount over the asset's useful life as depreciation or amortisation expense rather than expensing it in the year of expenditure. Net income and total assets are higher in the current fiscal year.

  \item B is correct. Alpha's fixed asset turnover will be lower because the capitalised interest will appear on the balance sheet as part of the asset being constructed. Therefore, fixed assets will be higher and the fixed asset turnover ratio (total revenue/average net fixed assets) will be lower than if it had expensed these costs. Capitalised interest appears on the balance sheet as part of the asset being constructed instead of being reported as interest expense in the period incurred. However, the interest coverage ratio should be based on interest payments, not interest expense (earnings before interest and taxes/interest payments), and should be unchanged. To provide a true picture of a company's interest coverage, the entire amount of interest expenditure, both the capitalised portion and the expensed portion, should be used in calculating interest coverage ratios.

  \item A is correct. Accelerated depreciation will result in an improving, not declining, net profit margin over time, because the amount of depreciation expense declines each year. Under straight-line depreciation, the amount of depreciation expense will remain the same each year. Under the units-of-production method, the amount of depreciation expense reported each year varies with the number of units produced.

  \item B is correct. The estimated average total useful life of a company's assets is calculated by adding the estimates of the average remaining useful life and the average age of the assets. The average age of the assets is estimated by dividing accumulated depreciation by depreciation expense. The average remaining useful life of the asset base is estimated by dividing net property, plant, and equipment by annual depreciation expense.

  \item $\mathrm{C}$ is correct. The impairment loss is a non-cash charge and will not affect cash flow from operating activities. The debt to total assets and fixed asset turnover ratios will increase, because the impairment loss will reduce the carrying amount of fixed assets and therefore total assets.

  \item A is correct. In an asset revaluation, the carrying amount of the assets increases. The increase in the asset's carrying amount bypasses the income statement and is reported as other comprehensive income and appears in equity under the heading of revaluation surplus. Therefore, shareholders' equity will increase but net income will not be affected, so return on equity will decline. Return on assets and debt to capital ratios will also decrease.

  \item A is correct. If the company uses the straight-line method, the depreciation expense will be one-fifth ( 20 percent) of the depreciable cost in Year 1. If it uses the units-of-production method, the depreciation expense will be 19 percent $(2,000 / 10,500)$ of the depreciable cost in Year 1 . Therefore, if the company uses the straight-line method, its depreciation expense will be higher and its net income will be lower.

  \item $\mathrm{C}$ is correct. The operating income or earnings before interest and taxes will be lowest for the method that results in the highest depreciation expense. The double-declining balance method results in the highest depreciation expense in the first year of use.

\end{enumerate}

Depreciation expense:

Straight line $=€ 1,500 / 5=€ 300$.

Double-declining balance $=€ 1,500 \times 0.40=€ 600$.

Units of production $=€ 1,500 \times 0.15=€ 225$.

\begin{enumerate}
  \setcounter{enumi}{22}
  \item $\mathrm{C}$ is correct. If Martinez wants to minimize tax payments in the first year of the machine's life, he should use an accelerated method, such as the double-declining balance method.

  \item A is correct. Using the straight-line method, depreciation expense amounts to Depreciation expense $=(1,200,000-200,000) / 8$ years $=125,000$.

  \item B is correct. Using the units-of-production method, depreciation expense amounts to

\end{enumerate}

Depreciation expense $=(1,200,000-200,000) \times(135,000 / 800,000)=168,750$.

\begin{enumerate}
  \setcounter{enumi}{25}
  \item $C$ is correct. Shifting at the end of Year 2 from double-declining balance to straight-line depreciation methodology results in depreciation expense being the same in each of Years 3, 4, and 5. Shifting to the straight-line methodology at the beginning of Year 3 results in a greater depreciation expense in Year 4 than would have been calculated using the double-declining balance method.
\end{enumerate}

Depreciation expense Year 4 (Using double-declining balance method all five years)

$=2 \times$ Annual depreciation \% using straight-line method $\times$ Carrying amount at end of Year 3

$=40 \% \times \$ 43,200$

Depreciation expense Year 4 with switch to straight-line method in Year 3

$=1 / 3 \times$ Remaining depreciable cost at start of Year 3

$=1 / 3 \times \$ 72,000$

$=\$ 24,000$

\begin{enumerate}
  \setcounter{enumi}{26}
  \item A is correct. The straight-line method is the method that evenly distributes the cost of an asset over its useful life because amortization is the same amount every year.

  \item A is correct. A higher residual value results in a lower total depreciable cost and, therefore, a lower amount of amortization in the first year after acquisition (and every year after that).

  \item B is correct. Using the straight-line method, accumulated amortization amounts to

\end{enumerate}

Accumulated amortization $=[(2,300,000-500,000) / 3$ years $] \times 2$ years $=1,200,000$

\begin{enumerate}
  \setcounter{enumi}{29}
  \item B is correct. Using the units-of-production method, depreciation expense amounts to
\end{enumerate}

Depreciation expense $=5,800,000 \times(20,000 / 175,000)=662,857$

\begin{enumerate}
  \setcounter{enumi}{30}
  \item B is correct. As shown in the following calculations, under the double-declining balance method, the annual amortization expense in Year 4 is closest to $¥ 9.9$ million.
\end{enumerate}

Annual amortization expense $=2 \times$ Straight-line amortization rate $\times$ Net book value.

Amortization expense Year $4=33.3 \% \times ¥ 29.6$ million = $¥ 9.9$ million.

\begin{enumerate}
  \setcounter{enumi}{31}
  \item A is correct. As shown in the following calculations, at the end of Year 4, the difference between the net book values calculated using straight-line versus double-declining balance is closest to $€ 81,400$.
\end{enumerate}

Net book value end of Year 4 using straight-line method $=€ 600,000-[4 \times$ $(€ 600,000 / 6)]=€ 200,000$.

Net book value end of Year 4 using double-declining balance method $=€ 600,000$ $(1-33.33 \%)^{4} \approx € 118,600$.

\begin{enumerate}
  \setcounter{enumi}{32}
  \item B is correct. In this case, the value increase brought about by the revaluation should be recorded directly in equity. The reason is that under IFRS, an increase in value brought about by a revaluation can only be recognized as a profit to the extent that it reverses a revaluation decrease of the same asset previously recognized in the income statement. 34. B is correct. The impairment loss equals $\pounds 3,100,000$.
\end{enumerate}

Impairment $=\max ($ Fair value less costs to sell; Value in use $)-$ Net carrying amount

$=\max (16,800,000-800,000 ; 14,500,000)-19,100,000$

$=-3,100,000$.

\begin{enumerate}
  \setcounter{enumi}{34}
  \item B is correct. Under IFRS, an impairment loss is measured as the excess of the carrying amount over the asset's recoverable amount. The recoverable amount is the higher of the asset's fair value less costs to sell and its value in use. Value in use is a discounted measure of expected future cash flows. Under US GAAP, assessing recoverability is separate from measuring the impairment loss. If the asset's carrying amount exceeds its undiscounted expected future cash flows, the asset's carrying amount is considered unrecoverable and the impairment loss is measured as the excess of the carrying amount over the asset's fair value.

  \item $\mathrm{C}$ is correct. The carrying amount of the asset on the balance sheet is reduced by the amount of the impairment loss, and the impairment loss is reported on the income statement.

  \item B is correct. The result on the sale of the vehicle equals

\end{enumerate}

Gain or loss on the sale $=$ Sale proceeds - Carrying amount

$=$ Sale proceeds - (Acquisition cost - Accumulated depreciation)

$=85,000-\{100,000-[((100,000-10,000) / 9$ years $) \times 3$ years $]\}$

$=15,000$.

\begin{enumerate}
  \setcounter{enumi}{37}
  \item A is correct. Gain or loss on the sale $=$ Sale proceeds - Carrying amount. Rearranging this equation, Sale proceeds = Carrying amount + Gain or loss on sale. Thus, Sale price $=(12$ million -2 million $)+(-3.2$ million $)=6.8$ million.

  \item A is correct. The gain or loss on the sale of long-lived assets is computed as the sales proceeds minus the carrying amount of the asset at the time of sale. This is true under the cost and revaluation models of reporting long-lived assets. In the absence of impairment losses, under the cost model, the carrying amount will equal historical cost net of accumulated depreciation.

  \item B is correct. IFRS do not require acquisition dates to be disclosed.

  \item A is correct. IFRS do not require fair value of intangible assets to be disclosed.

  \item C is correct. Under US GAAP, companies are required to disclose the estimated amortization expense for the next five fiscal years. Under US GAAP, there is no reversal of impairment losses. Disclosure of the useful lives-finite or indefinite and additional related details-is required under IFRS.

  \item B is correct. Investment property earns rent. Investment property and property, plant, and equipment are tangible and long-lived.

  \item $\mathrm{C}$ is correct. When a company uses the fair value model to value investment property, changes in the fair value of the property are reported in the income statement-not in other comprehensive income.

  \item A is correct. Investment property earns rent. Inventory is held for resale, and property, plant, and equipment are used in the production of goods and services.

  \item $C$ is correct. A company will change from the fair value model to either the cost model or revaluation model when the company transfers investment property to property, plant, and equipment.

  \item A is correct. Under both the revaluation model for property, plant, and equipment and the fair model for investment property, the asset's fair value must be able to be measured reliably. Under the fair value model, net income is affected by all changes in the asset's fair value. Under the revaluation model, any increase in an asset's value to the extent that it reverses a previous revaluation decrease will be recognized on the income statement and increase net income.

  \item A is correct. Under IFRS, when using the cost model for its investment properties, a company must disclose useful lives. The method for determining fair value, as well as reconciliation between beginning and ending carrying amounts of investment property, is a required disclosure when the fair value model is used.

\end{enumerate}

\section*{LEARNING MODULE }
\section{7}
\section{Income Taxes}
Elbie Louw, PhD, CFA, CIPM (South Africa). Michael A. Broihahn, CPA, CIA, CFA, is at Barry University (USA).

\section{LEARNING OUTCOME}
\begin{center}
\includegraphics[max width=\textwidth]{2023_05_04_b5cfa4f1bc883752f121g-419}
\end{center}

Note: Changes in accounting standards as well as new rulings and/or pronouncements issued after the publication of the readings on financial reporting and analysis may cause some of the information in these readings to become dated. Candidates are not responsible for anything that occurs after the readings were published. In addition, candidates are expected to be familiar with the analytical frameworks contained in the readings, as well as the implications of alternative accounting methods for financial analysis and valuation discussed in the readings. Candidates are also responsible for the content of accounting standards, but not for the actual reference numbers. Finally, candidates should be aware that certain ratios may be defined and calculated differently. When alternative ratio definitions exist and no specific definition is given, candidates should use the ratio definitions emphasized in the readings.

\section{INTRODUCTION}
For those companies reporting under International Financial Reporting Standards (IFRS), IAS 12 [Income Taxes] covers accounting for a company's income taxes and the reporting of deferred taxes. For those companies reporting under United States generally accepted accounting principles (US GAAP), FASB ASC Topic 740 [Income Taxes] is the primary source for information on accounting for income taxes. Although IFRS and US GAAP follow similar conventions on many income tax issues, there are some key differences that will be discussed in the reading.

Differences between how and when transactions are recognized for financial reporting purposes relative to tax reporting can give rise to differences in tax expense and related tax assets and liabilities. To reconcile these differences, companies that report under either IFRS or US GAAP create a provision on the balance sheet called deferred tax assets or deferred tax liabilities, depending on the nature of the situation.

Deferred tax assets or liabilities usually arise when accounting standards and tax authorities recognize the timing of revenues and expenses at different times. Because timing differences such as these will eventually reverse over time, they are called "temporary differences." Deferred tax assets represent taxes that have been recognized for tax reporting purposes (or often the carrying forward of losses from previous periods) but have not yet been recognized on the income statement prepared for financial reporting purposes. Deferred tax liabilities represent tax expense that has appeared on the income statement for financial reporting purposes, but has not yet become payable under tax regulations.

This reading provides a primer on the basics of income tax accounting and reporting. The reading is organized as follows. Section 2 describes the differences between taxable income and accounting profit. Section 3 explains the determination of tax base, which relates to the valuation of assets and liabilities for tax purposes. Section 4 discusses several types of timing differences between the recognition of taxable and accounting profit. Section 5 examines unused tax losses and tax credits. Section 6 describes the recognition and measurement of current and deferred tax. Section 7 discusses the disclosure and presentation of income tax information on companies' financial statements and illustrates its practical implications for financial analysis. Section 8 provides an overview of the similarities and differences for income-tax reporting between IFRS and US GAAP. A summary of the key points and practice problems in the CFA Institute multiple-choice format conclude the reading.

\section{2}
\section{DIFFERENCES BETWEEN ACCOUNTING PROFIT AND TAXABLE INCOME}
describe the differences between accounting profit and taxable income and define key terms, including deferred tax assets, deferred tax liabilities, valuation allowance, taxes payable, and income tax expense

A company's accounting profit is reported on its income statement in accordance with prevailing accounting standards. Accounting profit (also referred to as income before taxes or pretax income) does not include a provision for income tax expense. ${ }^{1}$

1 As defined under IAS 12 , paragraph 5. A company's taxable income is the portion of its income that is subject to income taxes under the tax laws of its jurisdiction. Because of different guidelines for how income is reported on a company's financial statements and how it is measured for income tax purposes, accounting profit and taxable income may differ.

A company's taxable income is the basis for its income tax payable (a liability) or recoverable (an asset), which is calculated on the basis of the company's tax rate and appears on its balance sheet. A company's tax expense, or tax benefit in the case of a recovery, appears on its income statement and is an aggregate of its income tax payable (or recoverable in the case of a tax benefit) and any changes in deferred tax assets and liabilities.

When a company's taxable income is greater than its accounting profit, then its income taxes payable will be higher than what would have otherwise been the case had the income taxes been determined based on accounting profit. Deferred tax assets, which appear on the balance sheet, arise when an excess amount is paid for income taxes (taxable income higher than accounting profit) and the company expects to recover the difference during the course of future operations. Actual income taxes payable will thus exceed the financial accounting income tax expense (which is reported on the income statement and is determined based on accounting profit). Related to deferred tax assets is a valuation allowance, which is a reserve created against deferred tax assets. The valuation allowance is based on the likelihood of realizing the deferred tax assets in future accounting periods. Deferred tax liabilities, which also appear on the balance sheet, arise when a deficit amount is paid for income taxes and the company expects to eliminate the deficit over the course of future operations. In this case, financial accounting income tax expense exceeds income taxes payable.

Income tax paid in a period is the actual amount paid for income taxes (not a provision, but the actual cash outflow). The income tax paid may be less than the income tax expense because of payments in prior periods or refunds received in the current period. Income tax paid reduces the income tax payable, which is carried on the balance sheet as a liability.

The tax base of an asset or liability is the amount at which the asset or liability is valued for tax purposes, whereas the carrying amount is the amount at which the asset or liability is valued according to accounting principles. ${ }^{2}$ Differences between the tax base and the carrying amount also result in differences between accounting profit and taxable income. These differences can carry through to future periods. For example, a tax loss carry forward occurs when a company experiences a loss in the current period that may be used to reduce future taxable income. The company's tax expense on its income statement must not only reflect the taxes payable based on taxable income, but also the effect of these differences.

2 The terms "tax base" and "tax basis" are interchangeable. "Tax basis" is more commonly used in the United States. Similarly, "carrying amount" and "book value" refer to the same concept.

\section*{CURRENT AND DEFERRED TAX ASSETS AND LIABILITIES }
\begin{abstract}
explain how deferred tax liabilities and assets are created and the factors that determine how a company's deferred tax liabilities and assets should be treated for the purposes of financial analysis calculate income tax expense, income taxes payable, deferred tax assets, and deferred tax liabilities, and calculate and interpret the adjustment to the financial statements related to a change in the income tax rate
\end{abstract}

A company's current tax liability is the amount payable in taxes and is based on current taxable income. If the company expects to receive a refund for some portion previously paid in taxes, the amount recoverable is referred to as a current tax asset. The current tax liability or asset may, however, differ from what the liability would have been if it was based on accounting profit rather than taxable income for the period. Differences in accounting profit and taxable income are the result of the application of different rules. Such differences between accounting profit and taxable income can occur in several ways, including:

\begin{itemize}
  \item Revenues and expenses may be recognized in one period for accounting purposes and a different period for tax purposes;

  \item Specific revenues and expenses may be either recognized for accounting purposes and not for tax purposes; or not recognized for accounting purposes but recognized for tax purposes;

  \item The carrying amount and tax base of assets and/or liabilities may differ;

  \item The deductibility of gains and losses of assets and liabilities may vary for accounting and income tax purposes;

  \item Subject to tax rules, tax losses of prior years might be used to reduce taxable income in later years, resulting in differences in accounting and taxable income (tax loss carryforward); and

  \item Adjustments of reported financial data from prior years might not be recognized equally for accounting and tax purposes or might be recognized in different periods.

\end{itemize}

\section{Deferred Tax Assets and Liabilities}
Deferred tax assets represent taxes that have been paid (or often the carrying forward of losses from previous periods) but have not yet been recognized on the income statement. Deferred tax liabilities occur when financial accounting income tax expense is greater than regulatory income tax expense. Deferred tax assets and liabilities usually arise when accounting standards and tax authorities recognize the timing of taxes due at different times; for example, when a company uses accelerated depreciation when reporting to the tax authority (to increase expense and lower tax payments in the early years) but uses the straight-line method on the financial statements. Although not similar in treatment on a year-to-year basis (e.g., depreciation of 5 percent on a straight-line basis may be permitted for accounting purposes whereas 10 percent is allowed for tax purposes) over the life of the asset, both approaches allow for the total cost of the asset to be depreciated (or amortized). Because these timing differences will eventually reverse or self-correct over the course of the asset's depreciable life, they are called "temporary differences."

Any deferred tax asset or liability is based on temporary differences that result in an excess or a deficit amount paid for taxes, which the company expects to recover from future operations. Because taxes will be recoverable or payable at a future date, it is only a temporary difference and a deferred tax asset or liability is created. Changes in the deferred tax asset or liability on the balance sheet reflect the difference between the amounts recognized in the previous period and the current period. The changes in deferred tax assets and liabilities are added to income tax payable to determine the company's income tax expense (or credit) as it is reported on the income statement.

At the end of each fiscal year, deferred tax assets and liabilities are recalculated by comparing the tax bases and carrying amounts of the balance sheet items. Identified temporary differences should be assessed on whether the difference will result in future economic benefits. For example, Pinto Construction (a hypothetical company) depreciates equipment on a straight-line basis of 10 percent per year. The tax authorities allow depreciation of 15 percent per year. At the end of the fiscal year, the carrying amount of the equipment for accounting purposes would be greater than the tax base of the equipment thus resulting in a temporary difference. A deferred tax item may only be created if it is not doubtful that the company will realize economic benefits in the future. In our example, the equipment is used in the core business of Pinto Construction. If the company is a going concern and stable, there should be no doubt that future economic benefits will result from the equipment and it would be appropriate to create the deferred tax item.

Should it be doubtful that future economic benefits will be realized from a temporary difference (such as Pinto Construction being under liquidation), the temporary difference will not lead to the creation of a deferred tax asset or liability. If a deferred tax asset or liability resulted in the past, but the criteria of economic benefits is not met on the current balance sheet date, then, under IFRS, an existing deferred tax asset or liability related to the item will be reversed. Under US GAAP, a valuation allowance is established. In assessing future economic benefits, much is left to the discretion of the auditor in assessing the temporary differences and the issue of future economic benefits.

\section{EXAMPLE 1}
The following information pertains to a hypothetical company, Reston Partners:

\section{Reston Partners Consolidated Income Statement}
\begin{center}
\begin{tabular}{|c|c|c|c|}
\hline
Period Ending 31 March (£ Millions) & Year 3 & Year 2 & Year 1 \\
\hline
Revenue & $\pounds 40,000$ & $\pounds 30,000$ & $\pounds 25,000$ \\
\hline
Other net gains & 2,000 & 0 & 0 \\
\hline
$\begin{array}{l}\text { Changes in inventories of finished goods } \\ \text { and work in progress }\end{array}$ & 400 & 180 & 200 \\
\hline
Raw materials and consumables used & $(5,700)$ & $(4,000)$ & $(8,000)$ \\
\hline
Depreciation expense & $(2,000)$ & $(2,000)$ & $(2,000)$ \\
\hline
Other expenses & $(6,000)$ & $(5,900)$ & $(4,500)$ \\
\hline
Interest expense & $(2,000)$ & $(3,000)$ & $(6,000)$ \\
\hline
Profit before tax & $\pounds 26,700$ & $\pounds 15,280$ & $\pounds 4,700$ \\
\hline
\end{tabular}
\end{center}

The financial performance and accounting profit of Reston Partners on this income statement is based on accounting principles appropriate for the jurisdiction in which Reston Partners operates. The principles used to calculate accounting profit (profit before tax in the example above) may differ from the principles applied for tax purposes (the calculation of taxable income). For illustrative purposes, however, assume that all income and expenses on the income statement are treated identically for tax and accounting purposes except depreciation.

The depreciation is related to equipment owned by Reston Partners. For simplicity, assume that the equipment was purchased at the beginning of the Year 1. Depreciation should thus be calculated and expensed for the full year. Assume that accounting standards permit equipment to be depreciated on a straight-line basis over a 10-year period, whereas the tax standards in the jurisdiction specify that equipment should be depreciated on a straight-line basis over a 7-year period. For simplicity, assume a salvage value of $\pounds 0$ at the end of the equipment's useful life. Both methods will result in the full depreciation of the asset over the respective tax or accounting life.

The equipment was originally purchased for $\pounds 20,000$. In accordance with accounting standards, over the next 10 years the company will recognize annual depreciation of $\pounds 2,000(\pounds 20,000 \div 10)$ as an expense on its income statement and for the determination of accounting profit. For tax purposes, however, the company will recognize $\pounds 2,857$ ( $\pounds 20,000 \div 7$ ) in depreciation each year. Each fiscal year the depreciation expense related to the use of the equipment will, therefore, differ for tax and accounting purposes (tax base vs. carrying amount), resulting in a difference between accounting profit and taxable income.

The previous income statement reflects accounting profit (depreciation at $\pounds 2,000$ per year). The following table shows the taxable income for each fiscal year.

\begin{center}
\begin{tabular}{lccc}
\hline
Taxable Income ( $\pounds$ Millions) & Year 3 & Year 2 & Year 1 \\
\hline
Revenue & $\pounds 40,000$ & $\pounds 30,000$ & $\pounds 25,000$ \\
Other net gains & 2,000 & 0 & 0 \\
Changes in inventories of finished goods & 400 & 180 & 200 \\
and work in progress &  &  &  \\
Raw materials and consumables used & $(5,700)$ & $(4,000)$ & $(8,000)$ \\
Depreciation expense & $(2,857)$ & $(2,857)$ & $(2,857)$ \\
Other expenses & $(6,000)$ & $(5,900)$ & $(4,500)$ \\
Interest expense & $(2,000)$ & $(3,000)$ & $(6,000)$ \\
Taxable income & $\pounds 25,843$ & $\pounds 14,423$ & $\pounds 3,843$ \\
\hline
\end{tabular}
\end{center}

The carrying amount and tax base for the equipment is as follows:

\begin{center}
\begin{tabular}{|c|c|c|c|}
\hline
(£ Millions) & Year 3 & Year 2 & Year 1 \\
\hline
$\begin{array}{l}\text { Equipment value for accounting purposes } \\ \text { (carrying amount) (depreciation of } € 2,000 / \\ \text { year) }\end{array}$ & $\pounds 14,000$ & $\pounds 16,000$ & $\pounds 18,000$ \\
\hline
$\begin{array}{l}\text { Equipment value for tax purposes (tax } \\ \text { base) (depreciation of } \pounds 2,857 / \text { year) }\end{array}$ & $\pounds 11,429$ & $\pounds 14,286$ & $\pounds 17,143$ \\
\hline
Difference & $\pounds 2,571$ & $\pounds 1,714$ & $\pounds 857$ \\
\hline
\end{tabular}
\end{center}

At each balance sheet date, the tax base and carrying amount of all assets and liabilities must be determined. The income tax payable by Reston Partners will be based on the taxable income of each fiscal year. If a tax rate of 30 percent is assumed, then the income taxes payable for Year 1, Year 2, and Year 3 are $\pounds 1,153$ $(30 \% \times 3,843), \pounds 4,327(30 \% \times 14,423)$, and $\pounds 7,753(30 \% \times 25,843)$. Remember, though, that if the tax obligation is calculated based on accounting profits, it will differ because of the differences between the tax base and the carrying amount of equipment. The difference in each fiscal year is reflected in the table above. In each fiscal year the carrying amount of the equipment exceeds its tax base. For tax purposes, therefore, the asset tax base is less than its carrying value under financial accounting principles. The difference results in a deferred tax liability.

\begin{center}
\begin{tabular}{lccc}
\hline
( $₫$ Millions) & Year $\mathbf{3}$ & Year 2 & Year 1 \\
\hline
Deferred tax liability & $\pounds 771$ & $\pounds 514$ & $\pounds 257$ \\
$\quad$ (Difference between tax base and carrying amount $) \times$ tax rate &  &  &  \\
Year 1: $\pounds(18,000-17,143) \times 30 \%=257$ &  &  &  \\
Year 2: $\pounds(16,000-14,286) \times 30 \%=514$ &  &  &  \\
Year 3: $\pounds(14,000-11,429) \times 30 \%=771$ &  &  &  \\
\end{tabular}
\end{center}

The comparison of the tax base and carrying amount of equipment shows what the deferred tax liability should be on a particular balance sheet date. In each fiscal year, only the change in the deferred tax liability should be included in the calculation of the income tax expense reported on the income statement prepared for accounting purposes.

On the income statement, the company's income tax expense will be the sum of change in the deferred tax liability and the income tax payable.

\begin{center}
\begin{tabular}{lcccc}
\hline
( Millions) & Year 3 & Year 2 & Year 1 \\
\hline
$\begin{array}{l}\text { Income tax payable (based on tax } \\ \text { accounting) }\end{array}$ & $\pounds 7,753$ & $\pounds 4,327$ & $\pounds 1,153$ \\
Change in deferred tax liability &  &  &  \\
Income tax (based on financial &  &  &  \\
accounting) &  &  &  \\
\end{tabular}
\end{center}

Note that because the different treatment of depreciation is a temporary difference, the income tax on the income statement is 30 percent of the accounting profit, although only a part is income tax payable and the rest is a deferred tax liability.

The consolidated income statement of Reston Partners including income tax is presented as follows:

\section{Reston Partners Consolidated Income Statement}
\begin{center}
\begin{tabular}{|c|c|c|c|}
\hline
Period Ending 31 March ( \textbackslash pm Millions) & Year 3 & Year 2 & Year 1 \\
\hline
Revenue & $\pounds 40,000$ & $\pounds 30,000$ & $\pounds 25,000$ \\
\hline
Other net gains & 2,000 & 0 & 0 \\
\hline
$\begin{array}{l}\text { Changes in inventories of finished goods } \\ \text { and work in progress }\end{array}$ & 400 & 180 & 200 \\
\hline
Raw materials and consumables used & $(5,700)$ & $(4,000)$ & $(8,000)$ \\
\hline
Depreciation expense & $(2,000)$ & $(2,000)$ & $(2,000)$ \\
\hline
Other expenses & $(6,000)$ & $(5,900)$ & $(4,500)$ \\
\hline
Interest expense & $(2,000)$ & $(3,000)$ & $(6,000)$ \\
\hline
Profit before tax & $\pounds 26,700$ & $\pounds 15,280$ & $\pounds 4,700$ \\
\hline
Income tax & $(8,010)$ & $(4,584)$ & $(1,410)$ \\
\hline
Profit after tax & $\pounds 18,690$ & $\pounds 10,696$ & $\pounds 3,290$ \\
\hline
\end{tabular}
\end{center}

Any amount paid to the tax authorities will reduce the liability for income tax payable and be reflected on the statement of cash flows of the company.

\section{DETERMINING THE TAX BASE OF ASSETS AND LIABILITIES}
calculate the tax base of a company's assets and liabilities

As mentioned in Section 2, temporary differences arise from a difference in the tax base and carrying amount of assets and liabilities. The tax base of an asset or liability is the amount attributed to the asset or liability for tax purposes, whereas the carrying amount is based on accounting principles. Such a difference is considered temporary if it is expected that the taxes will be recovered or payable at a future date.

\section{Determining the Tax Base of an Asset}
The tax base of an asset is the amount that will be deductible for tax purposes in future periods as the economic benefits become realized and the company recovers the carrying amount of the asset.

For example, our previously mentioned Reston Partners (from Example 1) depreciates equipment on a straight-line basis at a rate of 10 percent per year. The tax authorities allow depreciation of approximately 15 percent per year. At the end of the fiscal year, the carrying amount of equipment for accounting purposes is greater than the asset tax base thus resulting in a temporary difference.

\section{EXAMPLE 2}
\section{Determining the Tax Base of an Asset}
\begin{enumerate}
  \item The following information pertains to Entiguan Sports, a hypothetical developer of products used to treat sports-related injuries. (The treatment of items for accounting and tax purposes is based on hypothetical accounting and tax standards and is not specific to a particular jurisdiction.) Calculate the tax base and carrying amount for each item.

  \item Dividends receivable: On its balance sheet, Entiguan Sports reports dividends of $€ 1$ million receivable from a subsidiary. Assume that dividends are not taxable.

  \item Development costs: Entiguan Sports capitalized development costs of $€ 3$ million during the year. Entiguan amortized $€ 500,000$ of this amount during the year. For tax purposes amortization of 25 percent per year is allowed.

  \item Research costs: Entiguan incurred $€ 500,000$ in research costs, which were all expensed in the current fiscal year for financial reporting purposes. Assume that applicable tax legislation requires research costs to be expensed over a four-year period rather than all in one year. 4. Accounts receivable: Included on the income statement of Entiguan Sports is a provision for doubtful debt of $€ 125,000$. The accounts receivable amount reflected on the balance sheet, after taking the provision into account, amounts to $€ 1,500,000$. The tax authorities allow a deduction of 25 percent of the gross amount for doubtful debt.

\end{enumerate}

Solutions:

\begin{center}
\begin{tabular}{|c|c|c|c|}
\hline
 & $\begin{array}{l}\text { Carrying Amount } \\ (€)\end{array}$ & Tax Base $(€)$ & $\begin{array}{l}\text { Temporary } \\ \text { Difference }(€)\end{array}$ \\
\hline
1. Dividends receivable & $1,000,000$ & $1,000,000$ & 0 \\
\hline
2. Development costs & $2,500,000$ & $2,250,000$ & 250,000 \\
\hline
3. Research costs & 0 & 375,000 & $(375,000)$ \\
\hline
4. Accounts receivable & $1,500,000$ & $1,218,750$ & 281,250 \\
\hline
\end{tabular}
\end{center}

\section{Comments:}
\begin{enumerate}
  \item Dividends receivable: Although the dividends received are economic benefits from the subsidiary, we are assuming that dividends are not taxable. Therefore, the carrying amount equals the tax base for dividends receivable.

  \item Development costs: First, we assume that development costs will generate economic benefits for Entiguan Sports. Therefore, it may be included as an asset on the balance sheet for the purposes of this example. Second, the amortization allowed by the tax authorities exceeds the amortization accounted for based on accounting rules. Therefore, the carrying amount of the asset exceeds its tax base. The carrying amount is $(€ 3,000,000-€ 500,000)=€ 2,500,000$ whereas the tax base is $[€ 3,000,000-(25 \% \times € 3,000,000)]=€ 2,250,000$.

  \item Research costs: We assume that research costs will result in future economic benefits for the company. If this were not the case, creation of a deferred tax asset or liability would not be allowed. The tax base of research costs exceeds their carrying amount. The carrying amount is $€ 0$ because the full amount has been expensed for financial reporting purposes in the year in which it was incurred. Therefore, there would not have been a balance sheet item "Research costs" for tax purposes, and only a proportion may be deducted in the current fiscal year. The tax base of the asset is ( $5500,000-€ 500,000 / 4)=€ 375,000$.

  \item Accounts receivable: The economic benefits that should have been received from accounts receivable have already been included in revenues included in the calculation of the taxable income when the sales occurred. Because the receipt of a portion of the accounts receivable is doubtful, the provision is allowed. The provision, based on tax legislation, results in a greater amount allowed in the current fiscal year than would be the case under accounting principles. This results in the tax base of accounts receivable being lower than its carrying amount. Note that the example specifically states that the balance sheet amount for accounts receivable after the provision for accounting purposes amounts to $€ 1,500,000$. Therefore, accounts receivable before any provision was $€ 1,500,000+€ 125,000=€ 1,625,000$. The tax base is calculated as $(€ 1,500,000+€ 125,000)-[25 \% \times(€ 1,500,000+€ 125,000)]=$ $€ 1,218,750$.

\end{enumerate}

\section{Determining the Tax Base of a Liability}
The tax base of a liability is the carrying amount of the liability less any amounts that will be deductible for tax purposes in the future. With respect to payments from customers received in advance of providing the goods and services, the tax base of such a liability is the carrying amount less any amount of the revenue that will not be taxable in future. Keep in mind the following fundamental principle: In general, a company will recognize a deferred tax asset or liability when recovery/settlement of the carrying amount will affect future tax payments by either increasing or reducing the taxable profit. Remember, an analyst is not only evaluating the difference between the carrying amount and the tax base, but the relevance of that difference on future profits and losses and thus by implication future taxes.

IFRS offers specific guidelines with regard to revenue received in advance: IAS 12 states that the tax base is the carrying amount less any amount of the revenue that will not be taxed at a future date. Under US GAAP, an analysis of the tax base would result in a similar outcome. The tax legislation within the jurisdiction will determine the amount recognized on the income statement and whether the liability (revenue received in advance) will have a tax base greater than zero. This will depend on how tax legislation recognizes revenue received in advance.

\section{EXAMPLE 3}
\section{Determining the Tax Base of a Liability}
\begin{enumerate}
  \item The following information pertains to the hypothetical company Entiguan Sports for the fiscal year -end. The treatment of items for accounting and tax purposes is based on fictitious accounting and tax standards and is not specific to a particular jurisdiction. Calculate the tax base and carrying amount for each item.

  \item Donations: Entiguan Sports made donations of $€ 100,000$ in the current fiscal year. The donations were expensed for financial reporting purposes, but are not tax deductible based on applicable tax legislation.

  \item Interest received in advance: Entiguan Sports received in advance interest of $€ 300,000$. The interest is taxed because tax authorities recognize the interest to accrue to the company (part of taxable income) on the date of receipt.

  \item Rent received in advance: Entiguan recognized $€ 10$ million for rent received in advance from a lessee for an unused warehouse building. Rent received in advance is deferred for accounting purposes but taxed on a cash basis.

  \item Loan: Entiguan Sports secured a long-term loan for $€ 550,000$ in the current fiscal year. Interest is charged at 13.5 percent per annum and is payable at the end of each fiscal year.

\end{enumerate}

Solutions:

\begin{center}
\begin{tabular}{|c|c|c|c|}
\hline
 & $\begin{array}{l}\text { Carrying } \\ \text { Amount }(€)\end{array}$ & $\begin{array}{c}\text { Tax Base } \\ (\epsilon)\end{array}$ & $\begin{array}{c}\text { Temporary } \\ \text { Difference ( } \epsilon\end{array}$ \\
\hline
1. Donations & 0 & 0 & 0 \\
\hline
2. Interest received in advance & 300,000 & 0 & $(300,000)$ \\
\hline
3. Rent received in advance & $10,000,000$ & 0 & $(10,000,000)$ \\
\hline
\end{tabular}
\end{center}

\begin{center}
\begin{tabular}{|c|c|c|c|}
\hline
 & $\begin{array}{c}\text { Carrying } \\ \text { Amount (€) }\end{array}$ & $\begin{array}{c}\text { Tax Base } \\ (\epsilon)\end{array}$ & $\begin{array}{c}\text { Temporary } \\ \text { Difference }(€)\end{array}$ \\
\hline
4. Loan (capital) & 0 & 0 & 0 \\
\hline
Interest paid & 0 & 0 & 0 \\
\hline
\end{tabular}
\end{center}

\section{Comments:}
\begin{enumerate}
  \item Donations: The amount of $€ 100,000$ was immediately expensed on Entiguan's income statement; therefore, the carrying amount is $€ 0$. Tax legislation does not allow donations to be deducted for tax purposes, so the tax base of the donations equals the carrying amount. Note that while the carrying amount and tax base are the same, the difference in the treatment of donations for accounting and tax purposes (expensed for accounting purposes, but not deductible for tax purposes) represents a permanent difference (a difference that will not be reversed in future). Permanent and temporary differences are elaborated on in Section 4 and it will refer to this particular case with an expanded explanation.

  \item Interest received in advance: Based on the information provided, for tax purposes, interest is deemed to accrue to the company on the date of receipt. For tax purposes, it is thus irrelevant whether it is for the current or a future accounting period; it must be included in taxable income in the financial year received. Interest received in advance is, for accounting purposes though, included in the financial period in which it is deemed to have been earned. For this reason, the interest income received in advance is a balance sheet liability. It was not included on the income statement because the income relates to a future financial year. Because the full $€ 300,000$ is included in taxable income in the current fiscal year, the tax base is $€ 300,000-300,000=$ $€ 0$. Note that although interest received in advance and rent received in advance are both taxed, the timing depends on how the particular item is treated in tax legislation.

  \item Rent received in advance: The result is similar to interest received in advance. The carrying amount of rent received in advance would be $€ 10,000,000$ while the tax base is $€ 0$.

  \item Loan: Repayment of the loan has no tax implications. The repayment of the capital amount does not constitute an income or expense. The interest paid is included as an expense in the calculation of taxable income as well as accounting income. Therefore, the tax base and carrying amount is $€ 0$. For clarity, the interest paid that would be included on the income statement for the year amounts to $13.5 \% \times$ $€ 550,000=€ 74,250$ if the loan was acquired at the beginning of the current fiscal year.

\end{enumerate}

\section{CHANGES IN INCOME TAX RATES}
calculate income tax expense, income taxes payable, deferred tax assets, and deferred tax liabilities, and calculate and interpret the adjustment to the financial statements related to a change in the income tax rate

evaluate the effect of tax rate changes on a company's financial statements and ratios

The measurement of deferred tax assets and liabilities is based on current tax law. But if there are subsequent changes in tax laws or new income tax rates, existing deferred tax assets and liabilities must be adjusted for the effects of these changes. The resulting effects of the changes are also included in determining accounting profit in the period of change.

When income tax rates change, the deferred tax assets and liabilities are adjusted to the new tax rate. If income tax rates increase, deferred taxes (that is, the deferred tax assets and liabilities) will also increase. Likewise, if income tax rates decrease, deferred taxes will decrease. A decrease in tax rates decreases deferred tax liabilities, which reduces future tax payments to the taxing authorities. A decrease in tax rates will also decrease deferred tax assets, which reduces their value toward the offset of future tax payments to the taxing authorities.

To illustrate the effect of a change in tax rate, consider Example 1 again. In that illustration, the timing difference that led to the recognition of a deferred tax liability for Reston Partners was attributable to differences in the method of depreciation and the related effects on the accounting carrying value and the asset tax base. The relevant information is restated below.

The carrying amount and tax base for the equipment is:

\begin{center}
\begin{tabular}{|c|c|c|c|}
\hline
(£ Millions) & Year 3 & Year 2 & Year 1 \\
\hline
$\begin{array}{l}\text { Equipment value for accounting purposes (carrying } \\ \text { amount) (depreciation of } \pounds 2,000 / \text { year) }\end{array}$ & $\pounds 14,000$ & $\pounds 16,000$ & $\pounds 18,000$ \\
\hline
$\begin{array}{l}\text { Equipment value for tax purposes (tax base) } \\ \text { (depreciation of } € 2,857 / \text { year) }\end{array}$ & $\pounds 11,429$ & $\pounds 14,286$ & $\pounds 17,143$ \\
\hline
Difference & $\pounds 2,571$ & $\pounds 1,714$ & $\pounds 857$ \\
\hline
\end{tabular}
\end{center}

At a 30 percent income tax rate, the deferred tax liability was then determined as follows:

\begin{center}
\begin{tabular}{lccc}
\hline
(£ Millions) & Year 3 & Year 2 & Year 1 \\
\hline
Deferred tax liability & $\pounds 771$ & $\pounds 514$ & $\pounds 257$ \\
(Difference between tax base and carrying amount) &  &  &  \\
Year 1: $\pounds(18,000-17,143) \times 30 \%=\pounds 257$ &  &  &  \\
Year 2: $\pounds(16,000-14,286) \times 30 \%=\pounds 514$ &  &  &  \\
Year 3: $\pounds(14,000-11,429) \times 30 \%=\pounds 771$ &  &  &  \\
\end{tabular}
\end{center}

For this illustration, assume that the taxing authority has changed the income tax rate to 25 percent for Year 3. Although the difference between the carrying amount and the tax base of the depreciable asset are the same, the deferred tax liability for 2017 will be $\pounds 643$ (instead of $\pounds 771$ or a reduction of $\pounds 128$ in the liability). 2017: $\pounds(14,000$ $-11,429) \times 25 \%=\pounds 643$. Reston Partners' provision for income tax expense is also affected by the change in tax rates. Taxable income for Year 3 will now be taxed at a rate of 25 percent. The benefit of the Year 3 accelerated depreciation tax shield is now only $\pounds 214$ ( $\pounds 857 \times$ $25 \%$ ) instead of the previous $\pounds 257$ (a reduction of $\pounds 43$ ). In addition, the reduction in the beginning carrying value of the deferred tax liability for Year 3 (the year of change) further reduces the income tax expense for Year 3 . The reduction in income tax expense attributable to the change in tax rate is $\pounds 86$. Year 3: $(30 \%-25 \%) \times \pounds 1,714$ $=\pounds 86$. Note that these two components together account for the reduction in the deferred tax liability $(\pounds 43+\pounds 86=\pounds 129)$.

As may be seen from this discussion, changes in the income tax rate have an effect on a company's deferred tax asset and liability carrying values as well as an effect on the measurement of income tax expense in the year of change. The analyst must thus note that proposed changes in tax law can have a quantifiable effect on these accounts (and any related financial ratios that are derived from them) if the proposed changes are subsequently enacted into law.

\section{TEMPORARY AND PERMANENT DIFFERENCES BETWEEN TAXABLE AND ACCOUNTING PROFIT}
 identify and contrast temporary versus permanent differences inpre-tax accounting income and taxable income

Temporary differences arise from a difference between the tax base and the carrying amount of assets and liabilities. The creation of a deferred tax asset or liability from a temporary difference is only possible if the difference reverses itself at some future date and to such an extent that the balance sheet item is expected to create future economic benefits for the company. IFRS and US GAAP both prescribe the balance sheet liability method for recognition of deferred tax. This balance sheet method focuses on the recognition of a deferred tax asset or liability should there be a temporary difference between the carrying amount and tax base of balance sheet items. ${ }^{3}$

Permanent differences are differences between tax and financial reporting of revenue (expenses) that will not be reversed at some future date. Because they will not be reversed at a future date, these differences do not give rise to deferred tax. These items typically include

\begin{itemize}
  \item Income or expense items not allowed by tax legislation, and

  \item Tax credits for some expenditures that directly reduce taxes.

\end{itemize}

3 Previously, IAS 12 required recognition of deferred tax based on the deferred method (also known as the income statement method), which focused on timing differences. Timing differences are differences in the recognition of income and expenses for accounting and tax purposes that originate in one period and will reverse in a future period. Given the definition of timing differences, all timing differences are temporary differences, such as the different treatment of depreciation for tax and accounting purposes (although the timing is different with regard to the allowed depreciation for tax and accounting purposes, the asset will eventually be fully depreciated). Because no deferred tax item is created for permanent differences, all permanent differences result in a difference between the company's effective tax rate and statutory tax rate. The effective tax rate is also influenced by different statutory taxes should an entity conduct business in more than one tax jurisdiction. The formula for the reported effective tax rate is thus equal to:

Reported effective tax rate $=$ Income tax expense $\div$

Pretax income (accounting profit)

The net change in deferred tax during a reporting period is the difference between the balance of the deferred tax asset or liability for the current period and the balance of the previous period.

\section{Taxable Temporary Differences}
Temporary differences are further divided into two categories, namely taxable temporary differences and deductible temporary differences. Taxable temporary differences are temporary differences that result in a taxable amount in a future period when determining the taxable profit as the balance sheet item is recovered or settled. Taxable temporary differences result in a deferred tax liability when the carrying amount of an asset exceeds its tax base and, in the case of a liability, when the tax base of the liability exceeds its carrying amount.

Under US GAAP, a deferred tax asset or liability is not recognized for unamortizable goodwill. Under IFRS, a deferred tax account is not recognized for goodwill arising in a business combination. Since goodwill is a residual, the recognition of a deferred tax liability would increase the carrying amount of goodwill. Discounting deferred tax assets or liabilities is generally not allowed for temporary differences related to business combinations as it is for other temporary differences.

IFRS provides an exemption (that is, deferred tax is not provided on the temporary difference) for the initial recognition of an asset or liability in a transaction that: a) is not a business combination (e.g., joint ventures, branches and unconsolidated investments); and b) affects neither accounting profit nor taxable profit at the time of the transaction. US GAAP does not provide an exemption for these circumstances.

As a simple example of a temporary difference with no recognition of deferred tax liability, assume that a holding company of various leisure related businesses and holiday resorts buys an interest in a hotel in the current financial year. The goodwill related to the transaction will be recognized on the financial statements, but the related tax liability will not, as it relates to the initial recognition of goodwill.

\section{Deductible Temporary Differences}
Deductible temporary differences are temporary differences that result in a reduction or deduction of taxable income in a future period when the balance sheet item is recovered or settled. Deductible temporary differences result in a deferred tax asset when the tax base of an asset exceeds its carrying amount and, in the case of a liability, when the carrying amount of the liability exceeds its tax base. The recognition of a deferred tax asset is only allowed to the extent there is a reasonable expectation of future profits against which the asset or liability (that gave rise to the deferred tax asset) can be recovered or settled.

To determine the probability of sufficient future profits for utilization, one must consider the following: 1) Sufficient taxable temporary differences must exist that are related to the same tax authority and the same taxable entity; and 2) The taxable temporary differences that are expected to reverse in the same periods as expected for the reversal of the deductible temporary differences. As with deferred tax liabilities, IFRS states that deferred tax assets should not be recognized in cases that would arise from the initial recognition of an asset or liability in transactions that are not a business combination and when, at the time of the transaction, there is no impact on either accounting or taxable profit. Subsequent to initial recognition under IFRS and US GAAP, any deferred tax assets that arise from investments in subsidiaries, branches, associates, and interests in joint ventures are recognized as a deferred tax asset.

IFRS and US GAAP allow the creation of a deferred tax asset in the case of tax losses and tax credits. These two unique situations will be further elaborated on in Section 6. IAS 12 does not allow the creation of a deferred tax asset arising from negative goodwill. Negative goodwill arises when the amount that an entity pays for an interest in a business is less than the net fair market value of the portion of assets and liabilities of the acquired company, based on the interest of the entity.

\section{Examples of Taxable and Deductible Temporary Differences}
Exhibit 1 summarizes how differences between the tax bases and carrying amounts of assets and liabilities give rise to deferred tax assets or deferred tax liabilities.

\section{Exhibit 1: Treatment of Temporary Differences}
\begin{center}
\begin{tabular}{lll}
\hline
Balance Sheet Item & Carrying Amount vs. Tax Base & Results in Deferred Tax Asset/Liability \\
\hline
Asset & Carrying amount $>$ tax base & Deferred tax liability \\
Asset & Carrying amount $<$ tax base & Deferred tax asset \\
Liability & Carrying amount $>$ tax base & Deferred tax asset \\
Liability & Carrying amount $<$ tax base & Deferred tax liability \\
\hline
\end{tabular}
\end{center}

\section{EXAMPLE 4}
\section{Taxable and Deductible Temporary Differences}
\begin{enumerate}
  \item Examples 2 and 3 illustrated how to calculate the tax base of assets and liabilities, respectively. Based on the information provided in Examples 2 and 3, indicate whether the difference in the tax base and carrying amount of the assets and liabilities are temporary or permanent differences and whether a deferred tax asset or liability will be recognized based on the difference identified.
\end{enumerate}

\section{Solution to Example 2:}
\begin{center}
\begin{tabular}{|c|c|c|c|c|}
\hline
 & $\begin{array}{l}\text { Carrying Amount } \\ (\epsilon)\end{array}$ & Tax Base $(\epsilon)$ & $\begin{array}{c}\text { Temporary } \\ \text { Difference }(€)\end{array}$ & $\begin{array}{l}\text { Will Result in Deferred Tax Asset/ } \\ \text { Liability }\end{array}$ \\
\hline
1. Dividends receivable & $1,000,000$ & $1,000,000$ & 0 & $N / A$ \\
\hline
2. Development costs & $2,500,000$ & $2,250,000$ & 250,000 & Deferred tax liability \\
\hline
3. Research costs & 0 & 375,000 & $(375,000)$ & Deferred tax asset \\
\hline
4. Accounts receivable & $1,500,000$ & $1,218,750$ & 281,250 & Deferred tax liability \\
\hline
\end{tabular}
\end{center}

Example 2 included comments on the calculation of the carrying amount and tax base of the assets.

\begin{enumerate}
  \item Dividends receivable: As a result of non-taxability, the carrying amount equals the tax base of dividends receivable. This constitutes a permanent difference and will not result in the recognition of any deferred tax asset or liability. A temporary difference constitutes a difference that will, at some future date, be reversed. Although the timing of recognition is different for tax and accounting purposes, in the end the full carrying amount will be expensed/recognized as income. A permanent difference will never be reversed. Based on tax legislation, dividends from a subsidiary are not recognized as income. Therefore, no amount will be reflected as dividend income when calculating the taxable income, and the tax base of dividends receivable must be the total amount received, namely $€ 1,000,000$. The taxable income and accounting profit will permanently differ with the amount of dividends receivable, even on future financial statements as an effect on the retained earnings reflected on the balance sheet.

  \item Development costs: The difference between the carrying amount and tax base is a temporary difference that, in the future, will reverse. In this fiscal year, it will result in a deferred tax liability.

  \item Research costs: The difference between the carrying amount and tax base is a temporary difference that results in a deferred tax asset. Remember the explanation in Section 2 for deferred tax assets-a deferred tax asset arises because of an excess amount paid for taxes (when taxable income is greater than accounting profit), which is expected to be recovered from future operations. Based on accounting principles, the full amount was deducted resulting in a lower accounting profit, while the taxable income by implication, should be greater because of the lower amount expensed.

  \item Accounts receivable: The difference between the carrying amount and tax base of the asset is a temporary difference that will result in a deferred tax liability.

\end{enumerate}

Solution to Example 3:

\begin{center}
\begin{tabular}{|c|c|c|c|c|}
\hline
 & $\begin{array}{c}\text { Carrying } \\ \text { Amount (€) }\end{array}$ & Tax Base $(€)$ & $\begin{array}{c}\text { Temporary } \\ \text { Difference }(€)\end{array}$ & $\begin{array}{l}\text { Will Result in Deferred Tax } \\ \text { Asset/Liability }\end{array}$ \\
\hline
1. Donations & 0 & 0 & 0 & $N / A$ \\
\hline
2. Interest received in advance & 300,000 & 0 & $(300,000)$ & Deferred tax asset \\
\hline
3. Rent received in advance & $10,000,000$ & 0 & $(10,000,000)$ & Deferred tax asset \\
\hline
4. Loan (capital) & 550,000 & 550,000 & 0 & $N / A$ \\
\hline
Interest paid & 0 & 0 & 0 & $N / A$ \\
\hline
\end{tabular}
\end{center}

Example 3 included extensive comments on the calculation of the carrying amount and tax base of the liabilities.

\begin{enumerate}
  \item Donations: It was assumed that tax legislation does not allow donations to be deducted for tax purposes. No temporary difference results from donations, and thus a deferred tax asset or liability will not be recognized. This constitutes a permanent difference. 2. Interest received in advance: Interest received in advance results in a temporary difference that gives rise to a deferred tax asset. A deferred tax asset arises because of an excess amount paid for taxes (when taxable income is greater than accounting profit), which is expected to be recovered from future operations.

  \item Rent received in advance: The difference between the carrying amount and tax base is a temporary difference that leads to the recognition of a deferred tax asset.

  \item Loan: There are no temporary differences as a result of the loan or interest paid, and thus no deferred tax item is recognized.

\end{enumerate}

\section{EXCEPTIONS TO THE USUAL RULES FOR TEMPORARY DIFFERENCES}
$\square \quad \begin{aligned} & \text { identify and contrast temporary versus permanent differences in } \\ & \text { pre-tax accounting income and taxable income }\end{aligned}$

In some situations the carrying amount and tax base of a balance sheet item may vary at initial recognition. For example, a company may deduct a government grant from the initial carrying amount of an asset or liability that appears on the balance sheet. For tax purposes, such grants may not be deducted when determining the tax base of the balance sheet item. In such circumstances, the carrying amount of the asset or liability will be lower than its tax base. Differences in the tax base of an asset or liability as a result of the circumstances described above may not be recognized as deferred tax assets or liabilities.

For example, a government may offer grants to Small, Medium, and Micro Enterprises (SMME) in an attempt to assist these entrepreneurs in their endeavors that contribute to the country's GDP and job creation. Assume that a particular grant is offered for infrastructure needs (office furniture, property, plant, and equipment, etc.). In these circumstances, although the carrying amount will be lower than the tax base of the asset, the related deferred tax may not be recognized. As mentioned earlier, deferred tax assets and liabilities should not be recognized in cases that would arise from the initial recognition of an asset or liability in transactions that are not a business combination and when, at the time of the transaction, there is no impact on either accounting or taxable profit.

A deferred tax liability will also not be recognized at the initial recognition of goodwill. Although goodwill may be treated differently across tax jurisdictions, which may lead to differences in the carrying amount and tax base of goodwill, IAS 12 does not allow the recognition of such a deferred tax liability. Any impairment that an entity should, for accounting purposes, impose on goodwill will again result in a temporary difference between its carrying amount and tax base. Any impairment that an entity should, for accounting purposes, impose on goodwill and if part of the goodwill is related to the initial recognition, that part of the difference in tax base and carrying amount should not result in any deferred taxation because the initial deferred tax liability was not recognized. Any future differences between the carrying amount and tax base as a result of amortization and the deductibility of a portion of goodwill constitute a temporary difference for which provision should be made.

\section{Business Combinations and Deferred Taxes}
The fair value of assets and liabilities acquired in a business combination is determined on the acquisition date and may differ from the previous carrying amount. It is highly probable that the values of acquired intangible assets, including goodwill, would differ from their carrying amounts. This temporary difference will affect deferred taxes as well as the amount of goodwill recognized as a result of the acquisition.

\section{Investments in Subsidiaries, Branches, Associates and Interests in Joint Ventures}
Investments in subsidiaries, branches, associates and interests in joint ventures may lead to temporary differences on the consolidated versus the parent's financial statements. The related deferred tax liabilities as a result of temporary differences will be recognized unless both of the following criterion are satisfied:

\begin{itemize}
  \item The parent is in a position to control the timing of the future reversal of the temporary difference, and

  \item It is probable that the temporary difference will not reverse in the future.

\end{itemize}

With respect to deferred tax assets related to subsidiaries, branches, and associates and interests, deferred tax assets will only be recognized if the following criteria are satisfied:

\begin{itemize}
  \item The temporary difference will reverse in the future, and

  \item Sufficient taxable profits exist against which the temporary difference can be used.

\end{itemize}

\section{UNUSED TAX LOSSES AND TAX CREDITS}
explain recognition and measurement of current and deferred tax items

IAS 12 allows the recognition of unused tax losses and tax credits only to the extent that it is probable that in the future there will be taxable income against which the unused tax losses and credits can be applied. Under US GAAP, a deferred tax asset is recognized in full but is then reduced by a valuation allowance if it is more likely than not that some or all of the deferred tax asset will not be realized. The same requirements for creation of a deferred tax asset as a result of deductible temporary differences also apply to unused tax losses and tax credits. The existence of tax losses may indicate that the entity cannot reasonably be expected to generate sufficient future taxable income. All other things held constant, the greater the history of tax losses, the greater the concern regarding the company's ability to generate future taxable profits.

Should there be concerns about the company's future profitability, then the deferred tax asset may not be recognized until it is realized. When assessing the probability that sufficient taxable profit will be generated in the future, the following criteria can serve as a guide:

\begin{itemize}
  \item If there is uncertainty as to the probability of future taxable profits, a deferred tax asset as a result of unused tax losses or tax credits is only recognized to the extent of the available taxable temporary differences; - Assess the probability that the entity will in fact generate future taxable profits before the unused tax losses and/or credits expire pursuant to tax rules regarding the carry forward of the unused tax losses;

  \item Verify that the above is with the same tax authority and based on the same taxable entity;

  \item Determine whether the past tax losses were a result of specific circumstances that are unlikely to be repeated; and

  \item Discover if tax planning opportunities are available to the entity that will result in future profits. These may include changes in tax legislation that is phased in over more than one financial period to the benefit of the entity.

\end{itemize}

It is imperative that the timing of taxable and deductible temporary differences also be considered before creating a deferred tax asset based on unused tax credits.

\section{RECOGNITION AND MEASUREMENT OF CURRENT AND DEFERRED TAX}
\begin{center}
\includegraphics[max width=\textwidth]{2023_05_04_b5cfa4f1bc883752f121g-437}
\end{center}

Current taxes payable or recoverable from tax authorities are based on the applicable tax rates at the balance sheet date. Deferred taxes should be measured at the tax rate that is expected to apply when the asset is realized or the liability settled. With respect to the income tax for a current or prior period not yet paid, it is recognized as a tax liability until paid. Any amount paid in excess of any tax obligation is recognized as an asset. The income tax paid in excess or owed to tax authorities is separate from deferred taxes on the company's balance sheet.

When measuring deferred taxes in a jurisdiction, there are different forms of taxation such as income tax, capital gains tax (any capital gains made), or secondary tax on companies (tax payable on the dividends that a company declares) and possibly different tax bases for a balance sheet item (as in the case of government grants influencing the tax base of an asset such as property). In assessing which tax laws should apply, it is dependent on how the related asset or liability will be settled. It would be prudent to use the tax rate and tax base that is consistent with how it is expected the tax base will be recovered or settled.

Although deferred tax assets and liabilities are related to temporary differences expected to be recovered or settled at some future date, neither are discounted to present value in determining the amounts to be booked. Both must be adjusted for changes in tax rates.

Deferred taxes as well as income taxes should always be recognized on the income statement of an entity unless it pertains to:

\begin{itemize}
  \item Taxes or deferred taxes charged directly to equity, or

  \item A possible provision for deferred taxes relates to a business combination. The carrying amount of the deferred tax assets and liabilities should also be assessed. The carrying amounts may change even though there may have been no change in temporary differences during the period evaluated. This can result from:

  \item Changes in tax rates;

  \item Reassessments of the recoverability of deferred tax assets; or

  \item Changes in the expectations for how an asset will be recovered and what influences the deferred tax asset or liability.

\end{itemize}

All unrecognized deferred tax assets and liabilities must be reassessed at the balance sheet date and measured against the criteria of probable future economic benefits. If such a deferred asset is likely to be recovered, it may be appropriate to recognize the related deferred tax asset.

Different jurisdictions have different requirements for determining tax obligations that can range from different forms of taxation to different tax rates based on taxable income. When comparing financial statements of entities that conduct business in different jurisdictions subject to different tax legislation, the analyst should be cautious in reaching conclusions because of the potentially complex tax rules that may apply.

\section{Recognition of a Valuation Allowance}
Deferred tax assets must be assessed at each balance sheet date. If there is any doubt whether the deferral will be recovered, then the carrying amount should be reduced to the expected recoverable amount. Should circumstances subsequently change and suggest the future will lead to recovery of the deferral, the reduction may be reversed.

Under US GAAP, deferred tax assets are reduced by creating a valuation allowance. Establishing a valuation allowance reduces the deferred tax asset and income in the period in which the allowance is established. Should circumstances change to such an extent that a deferred tax asset valuation allowance may be reduced, the reversal will increase the deferred tax asset and operating income. Because of the subjective judgment involved, an analyst should carefully scrutinize any such changes.

\section{Recognition of Current and Deferred Tax Charged Directly to}
\section{Equity}
In general, IFRS and US GAAP require that the recognition of deferred tax liabilities and current income tax should be treated similarly to the asset or liability that gave rise to the deferred tax liability or income tax based on accounting treatment. Should an item that gives rise to a deferred tax liability be taken directly to equity, the same should hold true for the resulting deferred tax.

The following are examples of such items:

\begin{itemize}
  \item Revaluation of property, plant, and equipment (revaluations are not permissible under US GAAP);

  \item Long-term investments at fair value;

  \item Changes in accounting policies;

  \item Errors corrected against the opening balance of retained earnings;

  \item Initial recognition of an equity component related to complex financial instruments; and

  \item Exchange rate differences arising from the currency translation procedures for foreign operations. Whenever it is determined that a deferred tax liability will not be reversed, an adjustment should be made to the liability. The deferred tax liability will be reduced and the amount by which it is reduced should be taken directly to equity. Any deferred taxes related to a business combination must also be recognized in equity.

\end{itemize}

Depending on the items that gave rise to the deferred tax liabilities, an analyst should exercise judgment regarding whether the taxes should be included with deferred tax liabilities or whether it should be taken directly to equity. It may be more appropriate simply to ignore deferred taxes.

\section{EXAMPLE 5}
\section{Taxes Charged Directly to Equity}
The following information pertains to Khaleej Company (a hypothetical company). A building owned by Khaleej Company was originally purchased for $€ 1,000,000$ on 1 January 20x1. For accounting purposes, buildings are depreciated at 5 percent a year on a straight-line basis, and depreciation for tax purposes is 10 percent a year on a straight-line basis. On the first day of $20 \times 3$, the building is revalued at $€ 1,200,000$. It is estimated that the remaining useful life of the building from the date of revaluation is 20 years. Important: For tax purposes the revaluation of the building is not recognized.

Based on the information provided, the following illustrates the difference in treatment of the building for accounting and tax purposes.

\begin{center}
\begin{tabular}{|c|c|c|}
\hline
 & $\begin{array}{c}\text { Carrying Amount } \\ \text { of Building }\end{array}$ & $\begin{array}{l}\text { Tax Base of } \\ \text { Building }\end{array}$ \\
\hline
Balance on 1 January $20 \times 1$ & $€ 1,000,000$ & $€ 1,000,000$ \\
\hline
Depreciation $20 \times 1$ & $(50,000)$ & $(100,000)$ \\
\hline
Balance on 31 December $20 \times 1$ & $€ 950,000$ & $€ 900,000$ \\
\hline
Depreciation $20 \times 2$ & $(50,000)$ & $(100,000)$ \\
\hline
Balance on 31 December 20x2 & $€ 900,000$ & $€ 800,000$ \\
\hline
Revaluation on 1 January $20 \times 3$ & 300,000 & $\mathrm{n} / \mathrm{a}$ \\
\hline
Balance on 1 January $20 \times 3$ & $€ 1,200,000$ & $€ 800,000$ \\
\hline
Depreciation $20 \times 3$ & $(60,000)$ & $(100,000)$ \\
\hline
Balance on 31 December 20x3 & $€ 1,140,000$ & $€ 700,000$ \\
\hline
\multicolumn{3}{|l|}{Accumulated depreciation} \\
\hline
Balance on 1 January $20 \times 1$ & $€ 0$ & $€ 0$ \\
\hline
Depreciation $20 \times 1$ & 50,000 & 100,000 \\
\hline
Balance on 31 December $20 \times 1$ & $€ 50,000$ & $€ 100,000$ \\
\hline
Depreciation $20 \times 2$ & 50,000 & 100,000 \\
\hline
Balance on 31 December 20x2 & $€ 100,000$ & $€ 200,000$ \\
\hline
Revaluation at 1 January $20 \times 3$ & $(100,000)$ & $\mathrm{n} / \mathrm{a}$ \\
\hline
Balance on 1 January $20 \times 3$ & $€ 0$ & $€ 200,000$ \\
\hline
Depreciation $20 \times 3$ & 60,000 & 100,000 \\
\hline
Balance on 30 November $20 \times 3$ & $€ 60,000$ & $€ 300,000$ \\
\hline
\end{tabular}
\end{center}

\begin{center}
\begin{tabular}{lcc}
\hline
 & Carrying Amount & Tax Base \\
\hline
On 31 December 20x1 & $€ 950,000$ & $€ 900,000$ \\
On 31 December 20x2 & $€ 900,000$ & $€ 800,000$ \\
\end{tabular}
\end{center}

\begin{center}
\begin{tabular}{lcl}
\hline
 & Carrying Amount & Tax Base \\
\hline
On 31 December 20x3 & $€ 1,140,000$ & $€ 700,000$ \\
\hline
\end{tabular}
\end{center}

31 December 20x1: On 31 December 20x1, different treatments for depreciation expense result in a temporary difference that gives rise to a deferred tax liability. The difference in the tax base and carrying amount of the building was a result of different depreciation amounts for tax and accounting purposes. Depreciation appears on the income statement. For this reason the deferred tax liability will also be reflected on the income statement. If we assume that the applicable tax rate in $20 \times 1$ was 40 percent, then the resulting deferred tax liability will be $40 \%$ $\times(€ 950,000-€ 900,000)=€ 20,000$.

31 December 20x2: As of 31 December 20x2, the carrying amount of the building remains greater than the tax base. The temporary difference again gives rise to a deferred tax liability. Again, assuming the applicable tax rate to be 40 percent, the deferred tax liability from the building is $40 \% \times(€ 900,000-€ 800,000)=€ 40,000$.

31 December 20x3: On 31 December 20x3, the carrying amount of the building again exceeds the tax base. This is not the result of disposals or additions, but is a result of the revaluation at the beginning of the $20 \times 3$ fiscal year and the different rates of depreciation. The deferred tax liability would seem to be $40 \%$ $\times(€ 1,140,000-€ 700,000)=€ 176,000$, but the treatment is different than it was for the 20x1 and 20x2. In 20x3, revaluation of the building gave rise to a balance sheet equity account, namely "Revaluation Surplus" in the amount of $€ 300,000$, which is not recognized for tax purposes.

The deferred tax liability would usually have been calculated as follows:

\begin{center}
\begin{tabular}{|c|c|c|c|}
\hline
 & $20 \times 3$ & $20 \times 2$ & $20 \times 1$ \\
\hline
$\begin{array}{l}\text { Deferred tax liability (closing balance at } \\ \text { end of fiscal year) }\end{array}$ & $€ 176,000$ & $€ 40,000$ & $€ 20,000$ \\
\hline
\multicolumn{4}{|l|}{$\begin{array}{l}\text { (Difference between tax base and carrying } \\
\text { amount) }\end{array}$} \\
\hline
\multicolumn{4}{|l|}{$20 \times 1: €(950,000-900,000) \times 40 \%=20,000$} \\
\hline
\multicolumn{4}{|l|}{20x2: $€(900,000-800,000) \times 40 \%=40,000$} \\
\hline
$20 \times 3 \cdot €(1.140 .000-700.000) \times 40 \%=176.000$ &  &  &  \\
\hline
\end{tabular}
\end{center}

The change in the deferred tax liability in 20x1 is $€ 20,000$, in 20x2: $€ 20,000$ $(€ 40,000-€ 20,000)$ and, it would seem, in 20x3: $€ 136,000$ ( $€ 176,000-€ 40,000)$. In 20x3, although it would seem that the balance for deferred tax liability should be $€ 176,000$, the revaluation is not recognized for tax purposes. Only the portion of the difference between the tax base and carrying amount that is not a result of the revaluation is recognized as giving rise to a deferred tax liability.

The effect of the revaluation surplus and the associated tax effects are accounted for in a direct adjustment to equity. The revaluation surplus is reduced by the tax provision associated with the excess of the fair value over the carry value and it affects retained earnings ( $€ 300,000 \times 40 \%=€ 120,000)$.

The deferred tax liability that should be reflected on the balance sheet is thus not $€ 176,000$ but only $€ 56,000$ ( $€ 176,000-€ 120,000)$. Given the balance of deferred tax liability at the beginning of the $20 \times 3$ fiscal year in the amount of $€ 40,000$, the change in the deferred tax liability is only $€ 56,000-€ 40,000=€ 16,000$. In the future, at the end of each year, an amount equal to the depreciation as a result of the revaluation minus the deferred tax effect will be transferred from the revaluation reserve to retained earnings. In $20 \times 3$ this will amount to a portion of depreciation resulting from the revaluation, $€ 15,000(€ 300,000 \div 20)$, minus the deferred tax effect of $€ 6,000(€ 15,000 \times 40 \%)$, thus $€ 9,000$.

\section{PRESENTATION AND DISCLOSURE}
analyze disclosures relating to deferred tax items and the effective tax rate reconciliation and explain how information included in these disclosures affects a company's financial statements and financial ratios

We will discuss the presentation and disclosure of income tax related information by way of example. The Consolidated Statements of Operations (Income Statements) and Consolidated Balance Sheets for Micron Technology (MU), a global technology company based in the US, are provided in Exhibit 2 and Exhibit 3, respectively. Exhibit 4 provides the income tax note disclosures for MU for the 2015, 2016, and 2017 fiscal years.

MU's income tax provision (i.e., income tax expense) for fiscal year 2017 is $\$ 114$ million (see Exhibit 2). The income tax note disclosure in Exhibit 4 reconciles how the income tax provision was determined beginning with MU's reported income before taxes (shown in Exhibit 4 as $\$ 5,196$ million for fiscal year 2017). The note disclosure then denotes the income tax provision for 2017 that is current (\$153 million), which is then offset by the deferred tax benefit for foreign taxes ( $\$ 39$ million), for a net income tax provision of $\$ 114$ million. Exhibit 4 further shows a reconciliation of how the income tax provision was derived from the US federal statutory rate. Many public companies comply with this required disclosure by displaying the information in percentage terms, but MU has elected to provide the disclosure in absolute dollar amounts. From this knowledge, for 2017 we can see that the dollar amount shown for US federal income tax provision at the statutory rate $(\$ 1,819$ million) was determined by multiplying MU's income before taxes by the 35 percent US federal statutory rate $(\$ 5,196 \times 0.35=\$ 1,819)$.

In addition, the note disclosure in Exhibit 4 provides detailed information about the derivation of the deferred tax assets (\$766 million for 2017) and deferred tax liabilities (\$17 million for 2017). These deferred tax assets are shown separately on MU's consolidated balance sheet for fiscal year 2017 with noncurrent assets (see Exhibit 3), while the deferred tax liabilities are included in other noncurrent liabilities (also see Exhibit 3).

Exhibit 2: Micron Technology, Inc. Consolidated Statements of Operations (Amounts in US\$ Millions except Per Share)

\begin{center}
\begin{tabular}{|c|c|c|c|}
\hline
For the Year Ended & $\begin{array}{c}31 \text { Aug. } \\ 2017\end{array}$ & $\begin{array}{c}1 \text { Sept. } \\ 2016\end{array}$ & $\begin{array}{c}3 \text { Sept. } \\ 2015\end{array}$ \\
\hline
Net sales & 20,322 & $\$ 12,399$ & $\$ 16,192$ \\
\hline
Cost of goods sold & 11,886 & 9,894 & 10,977 \\
\hline
\end{tabular}
\end{center}

Income Taxes

\begin{center}
\begin{tabular}{|c|c|c|c|}
\hline
For the Year Ended & $\begin{array}{l}31 \text { Aug. } \\ 2017\end{array}$ & $\begin{array}{c}1 \text { Sept. } \\ 2016\end{array}$ & $\begin{array}{l}3 \text { Sept. } \\ 2015\end{array}$ \\
\hline
Gross margin & 8,436 & 2,505 & 5,215 \\
\hline
Selling, general and administrative & 743 & 659 & 719 \\
\hline
Research and development & 1,824 & 1,617 & 1,540 \\
\hline
Restructure and asset impairments & 18 & 67 & 3 \\
\hline
Other operating (income) expense, net & $(17)$ & (6) & $(45)$ \\
\hline
Operating income & 5,868 & 168 & 2,998 \\
\hline
Interest income (expense), net & $(560)$ & $(395)$ & $(336)$ \\
\hline
Other non-operating income (expense), net & $(112)$ & $(54)$ & $(53)$ \\
\hline
Income tax (provision) benefit & $(114)$ & (19) & $(157)$ \\
\hline
$\begin{array}{l}\text { Equity in net income (loss) of equity } \\ \text { method investees }\end{array}$ & 8 & 25 & 447 \\
\hline
$\begin{array}{l}\text { Net income (loss) attributable to noncon- } \\ \text { trolling interests }\end{array}$ & $(1)$ & (1) & - \\
\hline
Net income (loss) attributable to Micron & $\$ 5,089$ & $\$(276)$ & $\$ 2,899$ \\
\hline
\multicolumn{4}{|l|}{Earnings (loss) per share:} \\
\hline
Basic & $\$ 4.67$ & $\$(0.27)$ & $\$ 2.71$ \\
\hline
Diluted & $\$ 4.41$ & $\$(0.27)$ & $\$ 2.47$ \\
\hline
\multicolumn{4}{|l|}{$\begin{array}{l}\text { Number of shares used in per share } \\
\text { calculations: }\end{array}$} \\
\hline
Basic & 1,089 & 1,036 & 1,070 \\
\hline
Diluted & 1,154 & 1,036 & 1,170 \\
\hline
\end{tabular}
\end{center}

Exhibit 3: Micron Technology, Inc. Consolidated Balance Sheets (Amounts in US\$ Millions)

As of

31 Aug. 2017

1 Sept. 2016

Assets

Cash and equivalents

\begin{center}
\begin{tabular}{ccc}
$\$ 5,109$ & $\$ 4,140$ &  \\
319 & 258 &  \\
3,759 & 2,068 &  \\
3,123 & 2,889 &  \\
147 &  & 140 \\
\cline { 2 - 3 }
12,457 & 9,495 &  \\
617 & 414 &  \\
19,431 & 14,686 &  \\
16 &  & 1,364 \\
387 & 464 &  \\
766 &  & 657 \\
1,662 &  & 460 \\
\hline
$\$ 35,336$ &  &  \\
\cline { 1 - 2 }
 &  &  \\
\hline
\end{tabular}
\end{center}

$\begin{array}{lll}\text { Short-term investments } & 319 & 258\end{array}$

$\begin{array}{lrr}\text { Receivables } & 3,759 & 2,068\end{array}$

Inventories

Other current assets

Total current assets

Long-term marketable investments

Property, plant and equipment, net

Equity method investments

Intangible assets, net

Deferred tax assets

Other noncurrent assets

Total assets

$\$ 3,664$

$\$ 3,879$

Accounts payable and accrued expenses

\begin{center}
\begin{tabular}{|c|c|c|}
\hline
As of & 31 Aug. 2017 & 1 Sept. 2016 \\
\hline
Current debt & 1,262 & 756 \\
\hline
Total current liabilities & 5,334 & 4,835 \\
\hline
Long-term debt & 9,872 & 9,154 \\
\hline
Other noncurrent liabilities & 639 & 623 \\
\hline
Total liabilities & 15,845 & 14,612 \\
\hline
Redeemable convertible notes & 21 & - \\
\hline
Micron shareholder's equity &  &  \\
\hline
$\begin{array}{l}\text { Common stock of } \$ 0.10 \text { par value, } 3,000 \\ \text { shares authorized, } 1,116 \text { shares issued and } \\ 1,112 \text { shares outstanding (1,094 issued and } \\ 1,040 \text { outstanding as of September } 1,2016)\end{array}$ & 112 & 109 \\
\hline
Additional capital & 8,287 & 7,736 \\
\hline
Retained earnings & 10,260 & 5,299 \\
\hline
$\begin{array}{l}\text { Treasury stock, } 4 \text { shares held ( } 54 \text { as of } \\ \text { September } 1,2016)\end{array}$ & $(67)$ & $(1,029)$ \\
\hline
$\begin{array}{l}\text { Accumulated other comprehensive income } \\ \text { (loss) }\end{array}$ & 29 & $(35)$ \\
\hline
Total Micron shareholders' equity & 18,621 & 12,080 \\
\hline
Noncontrolling interests in subsidiaries & 849 & 848 \\
\hline
Total equity & 19,470 & 12,928 \\
\hline
Total liabilities and shareholders' equity & 35,336 & $\$ 27,540$ \\
\hline
\end{tabular}
\end{center}

\section{Exhibit 4: Micron Technology, Inc. Income Taxes Note to the Consolidated}
 Financial StatementsIncome (loss) before taxes and the income tax (provision) benefit consisted of the following:

(in US\$ Millions)

2017

2016

2015

Income (loss) before income taxes, net income (loss) attributable to noncontrolling interests, and equity in net income (loss) of equity method investees

Foreign

US

\begin{center}
\begin{tabular}{cccc}
$\$ 5,252$ &  & $\$(353)$ &  \\
$(56)$ &  & 72 & $\$ 2,431$ \\
\cline { 4 - 4 }
 &  &  & 178 \\
\hline
$(281)$ &  & $\$ 2,609$ &  \\
\hline
\end{tabular}
\end{center}

Income tax (provision) benefit:

Current:

Foreign

$\$(152)$

State

(1)

US federal

Deferred:

US federal

State

Foreign

\begin{center}
\begin{tabular}{|c|c|c|}
\hline
- & - & 6 \\
\hline
(153) & (28) & (88) \\
\hline
- & 39 & 15 \\
\hline
- & 2 & 1 \\
\hline
39 & (32) & $(85)$ \\
\hline
\end{tabular}
\end{center}

Income Taxes

\begin{center}
\begin{tabular}{lccccc}
\hline
(in US\$ Millions) & $\mathbf{2 0 1 7}$ &  & $\mathbf{2 0 1 6}$ & $\mathbf{2 0 1 5}$ \\
\hline
Income tax (provision) & 39 &  & 9 & $(69)$ \\
\cline { 2 - 3 }
 & $\$(114)$ &  & $\$(19)$ & $\$(157)$ \\
\hline
\end{tabular}
\end{center}

The company's income tax (provision) computed using the US federal statutory rate and the company's income tax (provision) benefit is reconciled as follows:

\begin{center}
\begin{tabular}{|c|c|c|c|}
\hline
(US\$ Millions) & 2017 & 2016 & 2015 \\
\hline
$\begin{array}{l}\text { US federal income tax (provision) benefit at } \\ \text { statutory rate }\end{array}$ & $\$(1,819)$ & $\$ 98$ & $\$(913)$ \\
\hline
Foreign tax rate differential & 1,571 & $(300)$ & 515 \\
\hline
Change in valuation allowance & 64 & 63 & 260 \\
\hline
Change in unrecognized tax benefits & 12 & 52 & $(118)$ \\
\hline
Tax credits & 66 & 48 & 53 \\
\hline
Noncontrolling investment transactions & - & - & 57 \\
\hline
Other & $(8)$ & 20 & $(11)$ \\
\hline
Income tax (provision) benefit & $(114)$ & $\$(19)$ & $\$(157)$ \\
\hline
\end{tabular}
\end{center}

State taxes reflect investment tax credits of $\$ 233$ million as at 31 August 2017. Deferred income taxes reflect the net tax effects of temporary differences between the bases of assets and liabilities for financial reporting and income tax purposes. The company's deferred tax assets and liabilities consist of the following as of the end of the periods shown below:

\begin{center}
\begin{tabular}{|c|c|c|}
\hline
(US\$ Millions) & 2017 & 2016 \\
\hline
\multicolumn{3}{|l|}{Deferred tax assets:} \\
\hline
Net operating loss and tax credit carryforwards & $\$ 3,426$ & $\$ 3,014$ \\
\hline
Accrued salaries, wages, and benefits & 211 & 142 \\
\hline
Other accrued liabilities & 59 & 76 \\
\hline
Other & 86 & 65 \\
\hline
Gross deferred assets & 3,782 & 3,297 \\
\hline
Less valuation allowance & $(2,321)$ & $(2,107)$ \\
\hline
Deferred tax assets, net of valuation allowance & 1,461 & 1,190 \\
\hline
\multicolumn{3}{|l|}{Deferred tax liabilities:} \\
\hline
Debt discount & $(145)$ & $(170)$ \\
\hline
Property, plant, and equipment & $(300)$ & $(135)$ \\
\hline
Unremitted earnings on certain subsidiaries & $(123)$ & $(121)$ \\
\hline
Product and process technology & $(85)$ & $(81)$ \\
\hline
Other & $(59)$ & $(28)$ \\
\hline
Deferred tax liabilities & $(712)$ & $(535)$ \\
\hline
Net deferred tax assets & $\$ 749$ & $\$ 655$ \\
\hline
\end{tabular}
\end{center}

Reported as:

Current deferred tax assets (included in other current $\quad \$-\quad$ \$assets)

Deferred tax assets

\begin{center}
\begin{tabular}{lcc}
\hline
(US\$ Millions) & $\mathbf{2 0 1 7}$ & $\mathbf{2 0 1 6}$ \\
\hline
$\begin{array}{l}\text { Deferred tax liabilities (included in other noncurrent } \\ \text { liabilities) }\end{array}$ & $(17)$ & $(2)$ \\
\cline { 2 - 3 }
Net deferred tax assets & $\$ 749$ &  \\
\cline { 3 - 4 }
\end{tabular}
\end{center}

The company has a valuation allowance against substantially all of its US net deferred tax assets. As of 31 August 2017, the company had aggregate US tax net operating loss carryforwards of $\$ 3.88$ billion and unused US tax credit carryforwards of $\$ 416$ million. The company also has unused state tax net operating loss carryforwards of $\$ 1.95$ billion and unused state tax credits of $\$ 233$ million. The net operating loss carryforwards and the tax credit carryforwards expire between 2018 to 2037.

The changes in valuation allowance of $\$ 64$ million and $\$ 63$ million in 2017 and 2016, respectively, are primarily a result of uncertainties of realizing certain US and foreign net operating losses and certain tax credit carryforwards.

Provision has been made for deferred taxes on undistributed earnings of non-US subsidiaries to the extent that dividend payments from such companies are expected to result in additional tax liability. Remaining undistributed earnings of $\$ 12.91$ billion as of 31 August 2017 have been indefinitely reinvested. Determination of the amount of unrecognized deferred tax liability on these unremitted earnings is not practicable.

\section{EXAMPLE 6}
\section{Financial Analysis Example}
Use the financial statement information and disclosures provided by MU in Exhibit 2, Exhibit 3, and Exhibit 4 to answer the following questions:

\begin{enumerate}
  \item MU discloses a valuation allowance of $\$ 2,321$ million (see Exhibit 4) against gross deferred assets of $\$ 3,782$ million in 2017. Does the existence of this valuation allowance have any implications concerning MU's future earnings prospects?
\end{enumerate}

\section{Solution to 1:}
According to Exhibit 4, MU's deferred tax assets expire gradually until 2037 (2018 to 2037 for the net operating loss carryforwards and the tax credit carryforwards).

Because the company is still relatively young, it is likely that most of these expirations occur toward the end of that period. Because cumulative US tax net operating loss carryforwards total $\$ 3.88$ billion, the valuation allowance could imply that $\mathrm{MU}$ is not reasonably expected to earn $\$ 3.88$ billion over the next 20 years. However, as we can see in Exhibit 2, MU earned a profit for 2017 and 2015, thereby showing that the allowance could be adjusted downward if the company continues to generate profits in the future, and making it more likely than not that the deferred tax asset would be recognized. 2. How would MU's deferred tax assets and deferred tax liabilities be affected if the federal statutory tax rate was changed to 21 percent?

Solution to 2:

MU's total deferred tax assets exceed total deferred tax liabilities by $\$ 749$ million. A change in the federal statutory tax rate to 21 percent from the current rate of 35 percent would make these net deferred assets less valuable. Also, because it is possible that the deferred tax asset valuation allowance could be adjusted downward in the future (see discussion to solution 1), the impact could be far greater in magnitude.

\begin{enumerate}
  \setcounter{enumi}{2}
  \item How would reported earnings have been affected if MU were not using a valuation allowance?
\end{enumerate}

\section{Solution to 3:}
The disclosure in Exhibit 4 shows that the increase in the valuation allowance increased the income tax provision as reported on the income statement by $\$ 64$ million in 2017. Additional potential reductions in the valuation allowance could similarly reduce reported income taxes (actual income taxes would not be affected by a valuation allowance established for financial reporting) in future years (see discussion to solution 1).

\begin{enumerate}
  \setcounter{enumi}{3}
  \item How would MU's $\$ 3.88$ billion in net operating loss carryforwards in 2017 (see Exhibit 4) affect the valuation that an acquiring company would be willing to offer?
\end{enumerate}

\section{Solution to 4:}
If an acquiring company is profitable, it may be able to use MU's tax loss carryforwards to offset its own tax liabilities. The value to an acquirer would be the present value of the carryforwards, based on the acquirer's tax rate and expected timing of realization. The higher the acquiring company's tax rate, and the more profitable the acquirer, the sooner it would be able to benefit. Therefore, an acquirer with a high current tax rate would theoretically be willing to pay more than an acquirer with a lower tax rate.

\begin{enumerate}
  \setcounter{enumi}{4}
  \item Under what circumstances should the analyst consider MU's deferred tax liability as debt or as equity? Under what circumstances should the analyst exclude MU's deferred tax liability from both debt and equity when calculating the debt-to-equity ratio?
\end{enumerate}

\section{Solution to 5:}
The analyst should classify the deferred tax liability as debt if the liability is expected to reverse with subsequent tax payment. If the liability is not expected to reverse, there is no expectation of a cash outflow and the liability should be treated as equity. By way of example, future company losses may preclude the payment of any income taxes, or changes in tax laws could result in taxes that are never paid. The deferred tax liability should be excluded from both debt and equity when both the amounts and timing of tax payments resulting from the reversals of temporary differences are uncertain.

\section{COMPARISON OF IFRS AND US GAAP}
identify the key provisions of and differences between income tax accounting under International Financial Reporting Standards (IFRS) and US generally accepted accounting principles (GAAP)

As mentioned earlier, though IFRS and US GAAP follow similar conventions on many tax issues, there are some notable differences. Exhibit 5 summarizes many of the key similarities and differences between IFRS and US GAAP. Though both frameworks require a provision for deferred taxes, there are differences in the methodologies.

\section{Exhibit 5: Deferred Income Tax Issues IFRS and US GAAP Methodology Similarities and Differences}
\section{Introduction}
The objective in accounting for income taxes is to recognize the amount of taxes + currently payable or refundable and deferred taxes. Income tax expense is the current tax expense (or recovery) plus the period change in deferred taxes (net of tax arising from a business combination or recorded outside profit or loss).

Unpaid taxes for current and prior periods are recognized as a liability, and an asset is recognised if the amount already paid exceeds the amount due. A prior tax loss benefit used to recover previous period current tax is also an asset.

In general, deferred taxes are recognized using an asset and liability approach which focuses on temporary differences arising between the tax base of an asset or liability and its carrying amount in the statement of financial position. Deferred taxes are recognized for the future tax consequences of events that have been recognized in an entity's financial statements or tax returns.

Deferred taxes are not recognized for:

\begin{itemize}
  \item The initial recognition of goodwill

  \item The initial recognition of an asset or liability in a non-business combination transaction and where accounting profit or taxable profit (tax loss) is not affected

  \item Taxable temporary differences from investments in subsidiaries/branches/ associates, and interests in joint ventures in which the parent etc. is able to control the timing of the reversal of the temporary difference, and it is probable that the temporary difference will not reverse in the foreseeable future Similar to IFRS.

\end{itemize}

The approach to calculating current taxes is similar to IFRS with some exceptions, such as the treatment of taxes on the elimination of intercompany profits.

US GAAP also follows an asset and liability approach to calculating deferred taxes, although there are some differences in the application relative to IFRS.

Deferred taxes are not recognized for:

\begin{itemize}
  \item Goodwill for which amortization is not deductible for tax purposes

  \item Unlike IFRS, US GAAP does not have a similar exception

  \item An excess of the amount for the financial reporting over the tax basis of an investment in a foreign subsidiary or a foreign corporate joint venture that is essentially permanent in duration, unless it becomes apparent that those temporary differences will reverse in the foreseeable future. Unlike IFRS, this exception does not apply to domestic subsidiaries and corporate joint venture and investments in equity investees.

\end{itemize}

Unlike IFRS, recognition of deferred taxes is prohibited for differences that are remeasured from the local currency into the functional currency using historical exchange rates and that result from changes in exchange rates or indexing for tax purposes. Deferred taxes should be recognized for the difference between the carrying amount determined by using the historical exchange rate and the relevant tax base, which may have been affected by exchange rate changes or tax indexing.

\section{Recognition and measurement}
Current tax liabilities and assets for the current and prior periods are measured at amounts expected to be paid to (recovered from) the taxation authorities based on tax rates (and tax laws) that have been enacted or substantially enacted by the end of the reporting period.

Deferred tax assets are measured at the tax rates that are expected to apply when the asset is realized or the liability is settled, based on tax rates (and tax laws) that have been enacted or substantively enacted by the end of the reporting period.

Deferred tax assets are recognized to the extent that it is probable (more likely than not) that taxable profit will be available to utilize the deductible temporary difference or carryforward of unused tax losses or tax credits. End of reporting period reviews may reduce the carrying amount if sufficient taxable profit is no longer probable such as to allow the utilization benefit of all or part of that deferred tax asset, and any such reduction is reversed if it subsequently becomes probable again.

Current and deferred taxes are recognized outside profit or loss if the tax relates to items that are recognized, in the same or a different period, in other comprehensive income (OCI), or directly to equity.

Deferred tax assets and liabilities are not discounted.

\section{Presentation and disclosure}
Deferred tax assets and liabilities are offset if the entity has a legally enforceable right to offset current tax assets against current tax liabilities and the deferred tax assets and deferred tax liabilities relate to income taxes levied by the same taxing authority on either the same taxable entity, or different taxable entities that intend either to settle current tax assets and liabilities on a net basis or to simultaneously realize/settle the asset/ liability.

Deferred tax assets and liabilities are presented as separate line items in the statement of financial position. If a classified statement of financial position is used, deferred taxes are classified as noncurrent. Similar to IFRS.

Measurement of current and deferred tax assets and liabilities is based on enacted tax law. Deferred tax assets and liabilities are measured using enacted tax rate(s) expected to apply to taxable income in periods in which deferred tax is expected to be settled or realized. Unlike IFRS, use of substantially enacted tax rates is not permitted.

Unlike IFRS, deferred tax assets are recognized in full and reduced by a valuation allowance if it is more likely than not that some portion or all of the deferred tax assets will not be realized.

Similar to IFRS, the tax effects of certain items occurring during the year are charged or credited directly to OCI or to related components of shareholders' equity.

Similar to IFRS.

All deferred taxes are offset and presented as a single amount.

Deferred tax assets and deferred tax liabilities are presented as noncurrent in a classified statement of financial position, which aligns with IFRS. IFRS

All entities must disclose an explanation of the relationship between tax expense and accounting profit using either or both of the following formats:

\begin{itemize}
  \item A numerical reconciliation between tax expense (income) and the product of accounting profit multiplied by the applicable tax rate(s) including disclosure of the basis on which the applicable rate is computed.

  \item A numerical reconciliation between the average effective tax rate and the applicable tax rate, including disclosure of the basis on which the applicable tax rate is computed.

\end{itemize}

\section{US GAAP}
Public companies must disclose a reconciliation using percentages or dollar amounts of the reported amount of income tax expense attributable to continuing operations for the year to that amount of income tax expense that would result from applying domestic federal statutory tax rates to pretax income from continuing operations.

Nonpublic enterprises must disclose the nature of significant reconciling items but may omit a numerical reconciliation.

Sources: IFRS: IAS 12 and 32. US GAAP: ASC 740. "Comparison between US GAAP and IFRS

Standards," Section 5.3 Taxation, Grant Thornton, April 2017. "IFRS and US GAAP: similarities and differences", PricewaterhouseCoopers LLC, 2018.

\section{SUMMARY}
Income taxes are a significant category of expense for profitable companies. Analyzing income tax expenses is often difficult for the analyst because there are many permanent and temporary timing differences between the accounting that is used for income tax reporting and the accounting that is used for financial reporting on company financial statements. The financial statements and notes to the financial statements of a company provide important information that the analyst needs to assess financial performance and to compare a company's financial performance with other companies. Key concepts in this reading are as follows:

\begin{itemize}
  \item Differences between the recognition of revenue and expenses for tax and accounting purposes may result in taxable income differing from accounting profit. The discrepancy is a result of different treatments of certain income and expenditure items.

  \item The tax base of an asset is the amount that will be deductible for tax purposes as an expense in the calculation of taxable income as the company expenses the tax basis of the asset. If the economic benefit will not be taxable, the tax base of the asset will be equal to the carrying amount of the asset.

  \item The tax base of a liability is the carrying amount of the liability less any amounts that will be deductible for tax purposes in the future. With respect to revenue received in advance, the tax base of such a liability is the carrying amount less any amount of the revenue that will not be taxable in the future.

  \item Temporary differences arise from recognition of differences in the tax base and carrying amount of assets and liabilities. The creation of a deferred tax asset or liability as a result of a temporary difference will only be allowed if the difference reverses itself at some future date and to the extent that it is expected that the balance sheet item will create future economic benefits for the company.

  \item Permanent differences result in a difference in tax and financial reporting of revenue (expenses) that will not be reversed at some future date. Because it will not be reversed at a future date, these differences do not constitute temporary differences and do not give rise to a deferred tax asset or liability. - Current taxes payable or recoverable are based on the applicable tax rates on the balance sheet date of an entity; in contrast, deferred taxes should be measured at the tax rate that is expected to apply when the asset is realized or the liability settled.

  \item All unrecognized deferred tax assets and liabilities must be reassessed on the appropriate balance sheet date and measured against their probable future economic benefit.

  \item Deferred tax assets must be assessed for their prospective recoverability. If it is probable that they will not be recovered at all or partly, the carrying amount should be reduced. Under US GAAP, this is done through the use of a valuation allowance.

\end{itemize}

\section{PRACTICE PROBLEMS}
\begin{enumerate}
  \item In early 2018 Sanborn Company must pay the tax authority $€ 37,000$ on the income it earned in 2017. This amount was recorded on the company's 31 December 2017 financial statements as:
A. taxes payable.
B. income tax expense.
C. a deferred tax liability.

  \item Using the straight-line method of depreciation for reporting purposes and accelerated depreciation for tax purposes would most likely result in a:
A. valuation allowance.
B. deferred tax asset.
C. temporary difference.

  \item Income tax expense reported on a company's income statement equals taxes payable, plus the net increase in:
A. deferred tax assets and deferred tax liabilities.
B. deferred tax assets, less the net increase in deferred tax liabilities.
C. deferred tax liabilities, less the net increase in deferred tax assets.

  \item Analysts should treat deferred tax liabilities that are expected to reverse as:
A. equity.
B. liabilities.
C. neither liabilities nor equity.

  \item When accounting standards require an asset to be expensed immediately but tax rules require the item to be capitalized and amortized, the company will most likely record:
A. a deferred tax asset.
B. a deferred tax liability.
C. no deferred tax asset or liability.

  \item A company incurs a capital expenditure that may be amortized over five years for accounting purposes, but over four years for tax purposes. The company will most likely record:
A. a deferred tax asset.
B. a deferred tax liability.
C. no deferred tax asset or liability.

  \item A company receives advance payments from customers that are immediately tax- able but will not be recognized for accounting purposes until the company fulfills its obligation. The company will most likely record:
A. a deferred tax asset.
B. a deferred tax liability.
C. no deferred tax asset or liability.

\end{enumerate}

\section{The following information relates to questions}
\section{8-10}
The tax effects of temporary differences that give rise to deferred tax assets and liabilities are as follows (\$ thousands):

\begin{center}
\begin{tabular}{|c|c|c|}
\hline
 & Year 3 & Year 2 \\
\hline
\multicolumn{3}{|l|}{Deferred tax assets:} \\
\hline
Accrued expenses & $\$ 8,613$ & $\$ 7,927$ \\
\hline
Tax credit and net operating loss carryforwards & 2,288 & 2,554 \\
\hline
LIFO and inventory reserves & 5,286 & 4,327 \\
\hline
Other & 2,664 & 2,109 \\
\hline
Deferred tax assets & 18,851 & 16,917 \\
\hline
Valuation allowance & $(1,245)$ & $(1,360)$ \\
\hline
Net deferred tax assets & $\$ 17,606$ & $\$ 15,557$ \\
\hline
\multicolumn{3}{|l|}{Deferred tax liabilities:} \\
\hline
Depreciation and amortization & $\$(27,338)$ & $\$(29,313)$ \\
\hline
Compensation and retirement plans & $(3,831)$ & $(8,963)$ \\
\hline
Other & $(1,470)$ & $(764)$ \\
\hline
Deferred tax liabilities & $(32,639)$ & $(39,040)$ \\
\hline
Net deferred tax liability & $(\$ 15,033)$ & $(\$ 23,483)$ \\
\hline
\end{tabular}
\end{center}

\begin{enumerate}
  \setcounter{enumi}{7}
  \item A reduction in the statutory tax rate would most likely benefit the company's:
\end{enumerate}

A. income statement and balance sheet.

B. income statement but not the balance sheet.

C. balance sheet but not the income statement.

\begin{enumerate}
  \setcounter{enumi}{8}
  \item If the valuation allowance had been the same in Year 3 as it was in Year 2, the company would have reported $\$ 115$ higher:
\end{enumerate}

A. net income.

B. deferred tax assets.

C. income tax expense.

\begin{enumerate}
  \setcounter{enumi}{9}
  \item Compared to the provision for income taxes in Year 3, the company's cash tax payments were:
\end{enumerate}

A. lower. B. higher.

C. the same.

\begin{enumerate}
  \setcounter{enumi}{10}
  \item When accounting standards require recognition of an expense that is not permitted under tax laws, the result is a:
A. deferred tax liability.
B. temporary difference.
C. permanent difference.

  \item When certain expenditures result in tax credits that directly reduce taxes, the company will most likely record:
A. a deferred tax asset.
B. a deferred tax liability.
C. no deferred tax asset or liability.

  \item Zimt AG presents its financial statements in accordance with US GAAP. In Year 3, Zimt discloses a valuation allowance of $\$ 1,101$ against total deferred tax assets of $\$ 19,201$. In Year 2, Zimt disclosed a valuation allowance of $\$ 1,325$ against total deferred tax assets of $\$ 17,325$. The change in the valuation allowance most likely indicates that Zimt's:
A. deferred tax liabilities were reduced in Year 3.
B. expectations of future earning power has increased.
C. expectations of future earning power has decreased.

  \item Cinnamon, Inc. recorded a total deferred tax asset in Year 3 of $\$ 12,301$, offset by a $\$ 12,301$ valuation allowance. Cinnamon most likely:

\end{enumerate}

A. fully utilized the deferred tax asset in Year 3.

B. has an equal amount of deferred tax assets and deferred tax liabilities.

C. expects not to earn any taxable income before the deferred tax asset expires.

\begin{enumerate}
  \setcounter{enumi}{14}
  \item Deferred tax liabilities should be treated as equity when:
A. they are not expected to reverse.
B. the timing of tax payments is uncertain.
C. the amount of tax payments is uncertain.

  \item When both the timing and amount of tax payments are uncertain, analysts should treat deferred tax liabilities as:
A. equity.
B. liabilities.
C. neither liabilities nor equity.

\end{enumerate}

\section{The following information relates to questions}
 17-19Note I: Income Taxes

The components of earnings before income taxes are as follows (\$ thousands):

\begin{center}
\begin{tabular}{|c|c|c|c|}
\hline
 & Year 3 & Year 2 & Year 1 \\
\hline
\multicolumn{4}{|c|}{Earnings before income taxes:} \\
\hline
United States & $\$ 88,157$ & $\$ 75,658$ & $\$ 59,973$ \\
\hline
Foreign & 116,704 & 113,509 & 94,760 \\
\hline
Total & $\$ 204,861$ & $\$ 189,167$ & $\$ 154,733$ \\
\hline
\end{tabular}
\end{center}

The components of the provision for income taxes are as follows (\$ thousands):

\begin{center}
\begin{tabular}{|c|c|c|c|}
\hline
 & Year 3 & Year 2 & Year 1 \\
\hline
\multicolumn{4}{|c|}{Income taxes} \\
\hline
\multicolumn{4}{|c|}{Current:} \\
\hline
Federal & $\$ 30,632$ & $\$ 22,031$ & $\$ 18,959$ \\
\hline
\multirow[t]{2}{*}{Foreign} & 28,140 & 27,961 & 22,263 \\
\hline
 & $\$ 58,772$ & $\$ 49,992$ & $\$ 41,222$ \\
\hline
\multicolumn{4}{|c|}{Deferred:} \\
\hline
Federal & $(\$ 4,752)$ & $\$ 5,138$ & $\$ 2,336$ \\
\hline
\multirow[t]{2}{*}{Foreign} & 124 & 1,730 & 621 \\
\hline
 & $(4,628)$ & 6,868 & 2,957 \\
\hline
Total & $\$ 54,144$ & $\$ 56,860$ & $\$ 44,179$ \\
\hline
\end{tabular}
\end{center}

\begin{enumerate}
  \setcounter{enumi}{16}
  \item In Year 3, the company's US GAAP income statement recorded a provision for income taxes closest to:
A. $\$ 30,632$.
B. $\$ 54,144$.
C. $\$ 58,772$.

  \item The company's effective tax rate was highest in:
A. Year 1.
B. Year 2 .
C. Year 3 .

  \item Compared to the company's effective tax rate on US income, its effective tax rate on foreign income was:
A. lower in each year presented.
B. higher in each year presented.
C. higher in some periods and lower in others.

\end{enumerate}

\section{The following information relates to questions
20-22}
A company's provision for income taxes resulted in effective tax rates attributable to loss from continuing operations before cumulative effect of change in accounting principles that varied from the statutory federal income tax rate of 34 percent, as summarized in the table below.

\begin{center}
\begin{tabular}{lccc}
\hline
Year Ended $\mathbf{3 0}$ June & Year $\mathbf{3}$ & Year 2 & Year $\mathbf{1}$ \\
\hline
Expected federal income tax expense (bene- & $(\$ 112,000)$ & $\$ 768,000$ & $\$ 685,000$ \\
fit) from continuing operations at 34 percent &  &  &  \\
Expenses not deductible for income tax & 357,000 & 32,000 & 51,000 \\
purposes &  &  &  \\
State income taxes, net of federal benefit & 132,000 & 22,000 & 100,000 \\
Change in valuation allowance for deferred & $(150,000)$ & $(766,000)$ & $(754,000)$ \\
tax assets &  &  & $\$ 82,000$ \\
Income tax expense & $\$ 227,000$ & $\$ 56,000$ &  \\
\end{tabular}
\end{center}

\begin{enumerate}
  \setcounter{enumi}{19}
  \item In Year 3, the company's net income (loss) was closest to:
A. $(\$ 217,000)$.
B. $(\$ 329,000)$.
C. $(\$ 556,000)$.

  \item The $\$ 357,000$ adjustment in Year 3 most likely resulted in:
A. an increase in deferred tax assets.
B. an increase in deferred tax liabilities.
C. no change to deferred tax assets and liabilities.

  \item Over the three years presented, changes in the valuation allowance for deferred tax assets were most likely indicative of:
A. decreased prospect for future profitability.
B. increased prospects for future profitability.
C. assets being carried at a higher value than their tax base.

\end{enumerate}

\section{SOLUTIONS}
\begin{enumerate}
  \item A is correct. The taxes a company must pay in the immediate future are taxes payable.

  \item $\mathrm{C}$ is correct. Because the differences between tax and financial accounting will correct over time, the resulting deferred tax liability, for which the expense was charged to the income statement but the tax authority has not yet been paid, will be a temporary difference. A valuation allowance would only arise if there was doubt over the company's ability to earn sufficient income in the future to require paying the tax.

  \item C is correct. Higher reported tax expense relative to taxes paid will increase the deferred tax liability, whereas lower reported tax expense relative to taxes paid increases the deferred tax asset.

  \item B is correct. If the liability is expected to reverse (and thus require a cash tax payment) the deferred tax represents a future liability.

  \item A is correct. The capitalization will result in an asset with a positive tax base and zero carrying value. The amortization means the difference is temporary. Because there is a temporary difference on an asset resulting in a higher tax base than carrying value, a deferred tax asset is created.

  \item B is correct. The difference is temporary, and the tax base will be lower (because of more rapid amortization) than the carrying value of the asset. The result will be a deferred tax liability.

  \item A is correct. The advances represent a liability for the company. The carrying value of the liability exceeds the tax base (which is now zero). A deferred tax asset arises when the carrying value of a liability exceeds its tax base.

  \item A is correct. A lower tax rate would increase net income on the income statement, and because the company has a net deferred tax liability, the net liability position on the balance sheet would also improve (be smaller).

  \item $\mathrm{C}$ is correct. The reduction in the valuation allowance resulted in a corresponding reduction in the income tax provision.

  \item B is correct. The net deferred tax liability was smaller in Year 3 than it was in Year 2, indicating that in addition to meeting the tax payments provided for in Year 3 the company also paid taxes that had been deferred in prior periods.

  \item $C$ is correct. Accounting items that are not deductible for tax purposes will not be reversed and thus result in permanent differences.

  \item C is correct. Tax credits that directly reduce taxes are a permanent difference, and permanent differences do not give rise to deferred tax.

  \item B is correct. The valuation allowance is taken against deferred tax assets to represent uncertainty that future taxable income will be sufficient to fully utilize the assets. By decreasing the allowance, Zimt is signaling greater likelihood that future earnings will be offset by the deferred tax asset.

  \item $\mathrm{C}$ is correct. The valuation allowance is taken when the company will "more likely than not" fail to earn sufficient income to offset the deferred tax asset. Because the valuation allowance equals the asset, by extension the company expects no taxable income prior to the expiration of the deferred tax assets.

  \item A is correct. If the liability will not reverse, there will be no required tax payment in the future and the "liability" should be treated as equity.

  \item $C$ is correct. The deferred tax liability should be excluded from both debt and equity when both the amounts and timing of tax payments resulting from the reversals of temporary differences are uncertain.

  \item B is correct. The income tax provision in Year 3 was $\$ 54,144$, consisting of $\$ 58,772$ in current income taxes, of which $\$ 4,628$ were deferred.

  \item B is correct. The effective tax rate of 30.1 percent $(\$ 56,860 / \$ 189,167)$ was higher than the effective rates in Year 2 and Year 3.

  \item A is correct. In Year 3 the effective tax rate on foreign operations was 24.2 percent $[(\$ 28,140+\$ 124) / \$ 116,704]$ and the effective US tax rate was $[(\$ 30,632$ - $\$ 4,752) / \$ 88,157]=29.4$ percent. In Year 2 the effective tax rate on foreign operations was 26.2 percent and the US rate was 35.9 percent. In Year 1 the foreign rate was 24.1 percent and the US rate was 35.5 percent.

  \item $\mathrm{C}$ is correct. The income tax provision at the statutory rate of 34 percent is a benefit of $\$ 112,000$, suggesting that the pre-tax income was a loss of $\$ 112,000 / 0.34$ $=(\$ 329,412)$. The income tax provision was $\$ 227,000 .(\$ 329,412)-\$ 227,000=$ $(\$ 556,412)$.

  \item C is correct. Accounting expenses that are not deductible for tax purposes result in a permanent difference, and thus do not give rise to deferred taxes.

  \item B is correct. Over the three-year period, changes in the valuation allowance reduced cumulative income taxes by $\$ 1,670,000$. The reductions to the valuation allowance were a result of the company being "more likely than not" to earn sufficient taxable income to offset the deferred tax assets.

\end{enumerate}

\section{LEARNING MODULE 8}
\section{Non-Current (Long-Term) Liabilities}
Elizabeth A. Gordon, PhD, MBA, CPA, is at Temple University (USA). Elaine Henry, PhD, CFA, is at Stevens Institute of Technology (USA).

\section{LEARNING OUTCOME}
\begin{center}
\begin{tabular}{l|l}
Mastery & The candidate should be able to: \\
\hline
$\square$ & $\begin{array}{l}\text { determine the initial recognition, initial measurement and } \\ \text { subsequent measurement of bonds } \\ \text { describe the effective interest method and calculate interest expense, } \\ \text { amortisation of bond discounts/premiums, and interest payments } \\ \text { explain the derecognition of debt }\end{array}$ \\
$\square$ & $\begin{array}{l}\text { describe the role of debt covenants in protecting creditors } \\ \text { describe the financial statement presentation of and disclosures } \\ \text { relating to debt } \\ \text { explain motivations for leasing assets instead of purchasing them } \\ \square\end{array}$ \\
$\begin{array}{l}\text { explain the financial reporting of leases from a lessee's perspective } \\ \text { explain the financial reporting of leases from a lessor's perspective } \\ \square\end{array}$ &  \\
$\begin{array}{l}\text { compare the presentation and disclosure of defined contribution and } \\ \text { defined benefit pension plans } \\ \text { calculate and interpret leverage and coverage ratios }\end{array}$ &  \\
\end{tabular}
\end{center}

Note: Changes in accounting standards as well as new rulings and/or pronouncements issued after the publication of the readings on financial reporting and analysis may cause some of the information in these readings to become dated. Candidates are not responsible for anything that occurs after the readings were published. In addition, candidates are expected to be familiar with the analytical frameworks contained in the readings, as well as the implications of alternative accounting methods for financial analysis and valuation discussed in the readings. Candidates are also responsible for the content of accounting standards, but not for the actual reference numbers. Finally, candidates should be aware that certain ratios may be defined and calculated differently. When alternative ratio definitions exist and no specific definition is given, candidates should use the ratio definitions emphasized in the readings.

\section{INTRODUCTION}
A non-current liability (long-term liability) broadly represents a probable sacrifice of economic benefits in periods generally greater than one year in the future. Common types of non-current liabilities reported in a company's financial statements include long-term debt (e.g., bonds payable, long-term notes payable), leases, pension liabilities, and deferred tax liabilities. This reading focuses on bonds payable, leases, and pension liabilities.

This reading is organised as follows. Section 2 introduces bonds and the accounting for their issuance. Section 3 discusses the recording of interest expense and interest payments as well as the amortisation of discount or premium. Section 4 describes fair value accounting for bonds, an alternative to the amortised cost approach. Section 5 discusses the repayment of principal when bonds are redeemed or reach maturity, which requires derecognition from the financial statements. Section 6 covers debt covenants. Section 7 describes the financial statement presentation and disclosures about debt financings. Section 8 discusses leases, including the benefits of leasing and accounting for leases by both lessees and lessors. Section 9 introduces pension accounting and the resulting non-current liabilities. Section 10 discusses the use of leverage and coverage ratios in evaluating solvency. Section 11 concludes and summarises the reading.

\section{BONDS PAYABLE \& ACCOUNTING FOR BOND ISSUANCE}
determine the initial recognition, initial measurement and subsequent measurement of bonds

This section discusses accounting for bonds payable-a common form of long-term debt. In some contexts (e.g., some government debt obligations), the word "bond" is used only for a debt security with a maturity of 10 years or longer; "note" refers to a debt security with a maturity between 2 and 10 years; and "bill" refers to a debt security with a maturity of less than 2 years. In this reading, we use the terms bond and note interchangeably because the accounting treatments of bonds payable and long-term notes payable are similar. In the following sections, we discuss bond issuance (initial recognition and measurement); bond amortisation, interest expense, and interest payments; market rates and fair value (subsequent measurement); repayment of bonds, including retirements and redemptions (derecognition); and other issues concerning disclosures related to debt. We also discuss debt covenants.

\section{Accounting for Bond Issuance}
Bonds are contractual promises made by a company (or other borrowing entity) to pay cash in the future to its lenders (i.e., bondholders) in exchange for receiving cash in the present. The terms of a bond contract are contained in a document called an indenture. The cash or sales proceeds received by a company when it issues bonds is based on the value (price) of the bonds at the time of issue; the price at the time of issue is determined as the present value of the future cash payments promised by the company in the bond agreement.

Ordinarily, bonds contain promises of two types of future cash payments: 1) the face value of the bonds, and 2) periodic interest payments. The face value of the bonds is the amount of cash payable by the company to the bondholders when the bonds mature. The face value is also referred to as the principal, par value, stated value, or maturity value. The date of maturity of the bonds (the date on which the face value is paid to bondholders) is stated in the bond contract and typically is a number of years in the future. Periodic interest payments are made based on the interest rate promised in the bond contract applied to the bonds' face value. The interest rate promised in the contract, which is the rate used to calculate the periodic interest payments, is referred to as the coupon rate, nominal rate, or stated rate. Similarly, the periodic interest payment is referred to as the coupon payment or simply the coupon. For fixed rate bonds (the primary focus of our discussion here), the coupon rate remains unchanged throughout the life of the bonds. The frequency with which interest payments are made is also stated in the bond contract. For example, bonds paying interest semi-annually will make two interest payments per year. ${ }^{1}$

The future cash payments are discounted to the present to arrive at the market value of the bonds. The market rate of interest is the rate demanded by purchasers of the bonds given the risks associated with future cash payment obligations of the particular bond issue. The market rate of interest at the time of issue often differs from the coupon rate because of interest rate fluctuations that occur between the time the issuer establishes the coupon rate and the day the bonds are actually available to investors. If the market rate of interest when the bonds are issued equals the coupon rate, the market value (price) of the bonds will equal the face value of the bonds. Thus, ignoring

issuance costs, the issuing company will receive sales proceeds (cash) equal to the face value of the bonds. When a bond is issued at a price equal to its face value, the bond is said to have been issued at par.

If the coupon rate when the bonds are issued is higher than the market rate, the market value of the bonds-and thus the amount of cash the company receives-will be higher than the face value of the bonds. In other words, the bonds will sell at a premium to face value because they are offering an attractive coupon rate compared to current market rates. If the coupon rate is lower than the market rate, the market value and thus the sale proceeds from the bonds will be less than the face value of the bonds; the bond will sell at a discount to face value. The market rate at the time of issuance is the effective interest rate or borrowing rate that the company incurs on the debt. The effective interest rate is the discount rate that equates the present value of the two types of promised future cash payments to their selling price. For the issuing company, interest expense reported for the bonds in the financial statements is based on the effective interest rate.

On the issuing company's statement of cash flows, the cash received (sales proceeds) from issuing bonds is reported as a financing cash inflow. On the issuing company's balance sheet at the time of issue, bonds payable normally are measured and reported at the sales proceeds. In other words, the bonds payable are initially reported at the face value of the bonds minus any discount, or plus any premium.

Using a three-step approach, the following two examples illustrate accounting for bonds issued at face value and then accounting for bonds issued at a discount to face value. Accounting for bonds issued at a premium involves steps similar to the steps followed in the examples below. For simplicity, these examples assume a flat interest rate yield curve (i.e., that the market rate of interest is the same for each period). More-precise bond valuations use the interest rate applicable to each time period in which a payment of interest or principal occurs.

\section{EXAMPLE 1}
\section{Bonds Issued at Face Value}
\begin{enumerate}
  \item Debond Corp. (a hypothetical company) issues $\pounds 1,000,000$ worth of five-year bonds, dated 1 January 2018, when the market interest rate on bonds of comparable risk and terms is 5 percent per annum. The bonds pay 5 percent
\end{enumerate}

$\overline{1 \text { Interest rates are stated }}$ on an annual basis regardless of the frequency of payment. interest annually on 31 December. What are the sales proceeds of the bonds when issued, and how is the issuance reflected in the financial statements?

Solution:

Calculating the value of the bonds at issuance and thus the sales proceeds involves three steps: 1 ) identifying key features of the bonds and the market interest rate, 2) determining future cash outflows, and 3) discounting the future cash flows to the present.

First, identify key features of the bonds and the market interest rate necessary to determine sales proceeds:

Face value (principal):

Time to maturity: $\quad 5$ years

Coupon rate: $\quad 5 \%$

Market rate at issuance: $\quad 5 \%$

Frequency of interest payments: annual

Interest payment: $\quad \pounds 50,000$ Each annual interest payment is the face value times the coupon rate $(\pounds 1,000,000 \times 5 \%)$. If interest is paid other than annually, adjust the interest rate to match the interest payment period (e.g., divide the annual coupon rate by two for semi-annual interest payments).

Second, determine future cash outflows. Debond will pay bondholders $\pounds 1,000,000$ when the bonds mature in five years. On 31 December of each year until the bonds mature, Debond will make an interest payment of $\pounds 50,000$.

Third, sum the present value ${ }^{2}$ of the future payments of interest and principal to obtain the value of the bonds and thus the sales proceeds from issuing the bonds. In this example, the sum is $\pounds 1,000,000=(\pounds 216,474+\pounds 783,526)$.

\begin{center}
\begin{tabular}{|c|c|c|c|c|c|}
\hline
Date & $\begin{array}{c}\text { Interest } \\ \text { Payment }\end{array}$ & $\begin{array}{c}\text { Present Value at } \\ \text { Market Rate } \\ (5 \%)\end{array}$ & $\begin{array}{c}\text { Face Value } \\ \text { Payment }\end{array}$ & $\begin{array}{c}\text { Present Value at } \\ \text { Market Rate } \\ (5 \%)\end{array}$ & $\begin{array}{c}\text { Total } \\ \text { Present Value }\end{array}$ \\
\hline
31 December 2018 & $\pounds 50,000$ & $\pounds 47,619$ &  &  &  \\
\hline
31 December 2019 & 50,000 & 45,352 &  &  &  \\
\hline
31 December 2020 & 50,000 & 43,192 &  &  &  \\
\hline
31 December 2021 & 50,000 & 41,135 &  &  &  \\
\hline
31 December 2022 & 50,000 & 39,176 & $\pounds 1,000,000$ & $\pounds 783,526$ &  \\
\hline
\multirow[t]{2}{*}{Total} &  & $\pounds 216,474$ &  & $\pounds 783,526$ & $\pounds 1,000,000$ \\
\hline
 &  &  &  &  & Sales Proceeds \\
\hline
\end{tabular}
\end{center}

The sales proceeds of the bonds when issued are $\pounds 1,000,000$. There is no discount or premium because these bonds are issued at face value. The issuance is reflected on the balance sheet as an increase of cash and an increase in a long-term liability, bonds payable, of $\pounds 1,000,000$. The issuance is reflected in the statement of cash flows as a financing cash inflow of $\pounds 1,000,000$.

2 Alternative ways to calculate the present value include 1) to treat the five annual interest payments as an annuity and use the formula for finding the present value of an annuity and then add the present value of the principal payment, or 2) to use a financial calculator to calculate the total present value. The price of bonds is often expressed as a percentage of face value. For example, the price of bonds issued at par, as in Example 1, is 100 (i.e., 100 percent of face value). In Example 2, in which bonds are issued at a discount, the price is 95.79 (i.e., 95.79 percent of face value).

\section{EXAMPLE 2}
\section{Bonds Issued at a Discount}
\begin{enumerate}
  \item Debond Corp. issues $\pounds 1,000,000$ worth of five-year bonds, dated 1 January 2018, when the market interest rate on bonds of comparable risk and terms is 6 percent. The bonds pay 5 percent interest annually on 31 December. What are the sales proceeds of the bonds when issued, and how is the issuance reflected in the financial statements?
\end{enumerate}

\section{Solution:}
The key features of the bonds and the market interest rate are:

Face value (principal):

Time to maturity:

Coupon rate:

Market rate at issuance:

Frequency of interest payments:

Interest payment: $\pounds 1,000,000$

5 years

$5 \%$

$6 \%$

annual

$\pounds 50,000$ Each annual interest payment is the face value times the coupon rate $(\pounds 1,000,000 \times 5 \%)$.

The future cash outflows (interest payments and face value payment), the present value of the future cash outflows, and the total present value are:

\begin{center}
\begin{tabular}{|c|c|c|c|c|c|}
\hline
Date & $\begin{array}{l}\text { Interest } \\ \text { Payment }\end{array}$ & $\begin{array}{c}\text { Present Value at } \\ \text { Market Rate } \\ (6 \%)\end{array}$ & $\begin{array}{c}\text { Face Value } \\ \text { Payment }\end{array}$ & $\begin{array}{c}\text { Present Value at } \\ \text { Market Rate } \\ (6 \%)\end{array}$ & $\begin{array}{c}\text { Total } \\ \text { Present Value }\end{array}$ \\
\hline
31 December 2018 & $\pounds 50,000$ & $\pounds 47,170$ &  &  &  \\
\hline
31 December 2019 & 50,000 & 44,500 &  &  &  \\
\hline
31 December 2020 & 50,000 & 41,981 &  &  &  \\
\hline
31 December 2021 & 50,000 & 39,605 &  &  &  \\
\hline
31 December 2022 & 50,000 & 37,363 & $\pounds 1,000,000$ & $\pounds 747,258$ &  \\
\hline
\multirow[t]{2}{*}{Total} &  & $\pounds 210,618$ &  & $\pounds 747,258$ & $\pounds 957,876$ \\
\hline
 &  &  &  &  & Sales Proceeds \\
\hline
\end{tabular}
\end{center}

The sales proceeds of the bonds when issued are $\pounds 957,876$. The bonds sell at a discount of $\pounds 42,124=(\pounds 1,000,000-\pounds 957,876)$ because the market rate when the bonds are issued (6 percent) is greater than the bonds' coupon rate (5 percent). The issuance is reflected on the balance sheet as an increase of cash and an increase in a long-term liability, bonds payable, of $\pounds 957,876$. The bonds payable is composed of the face value of $\pounds 1,000,000$ minus a discount of $\pounds 42,124$. The issuance is reflected in the statement of cash flows as a financing cash inflow of $\pounds 957,876$. In Example 2, the bonds were issued at a discount to face value because the bonds' coupon rate of 5 percent was less than the market rate. Bonds are issued at a premium to face value when the bonds' coupon rate exceeds the market rate.

Bonds issued with a coupon rate of zero (zero-coupon bonds) are issued at a discount to face value if the market rate is greater than zero. If the market rate is zero or negative, zero-coupon bonds will be issued at par or at premium, respectively. The value of zero-coupon bonds is based on the present value of the principal payment only because there are no periodic interest payments.

Such issuance costs as printing, legal fees, commissions, and other types of charges are costs incurred when bonds are issued. Under International Financial Reporting Standards (IFRS), all debt issuance costs are included in the measurement of the liability, bonds payable. Under US generally accepted accounting principles (US GAAP), companies generally used to show these debt issuance costs as an asset (a deferred charge), which was amortised on a straight-line basis to the relevant expense (e.g., legal fees) over the life of the bonds. Under US GAAP, debt issuance costs are deducted from the related debt liability. Companies reporting under US GAAP may still report debt issuance costs for lines of credit as an asset because the SEC indicated that it would not object to this treatment. Under IFRS and US GAAP, cash outflows related to bond issuance costs are included in the financing section of the statement of cash flows, usually netted against bond proceeds.

\section{ACCOUNTING FOR BOND AMORTISATION, INTEREST EXPENSE, AND INTEREST PAYMENTS}
determine the initial recognition, initial measurement and subsequent measurement of bonds

describe the effective interest method and calculate interest expense, amortisation of bond discounts/premiums, and interest payments

In this section, we discuss accounting and reporting for bonds after they are issued. Most companies maintain the historical cost (sales proceeds) of the bonds after issuance, and they amortise any discount or premium over the life of the bond. The amount reported on the balance sheet for bonds is thus the historical cost plus or minus the cumulative amortisation, which is referred to as amortised cost. Companies also have the option to report the bonds at their current fair values.

The rationale for reporting the bonds at amortised historical cost is the company's intention to retain the debt until it matures. Therefore, changes in the underlying economic value of the debt are not relevant from the issuing company's perspective. From an investor's perspective, however, analysis of a company's underlying economic liabilities and solvency is more difficult when debt is reported at amortised historical cost. The rest of this section illustrates accounting and reporting of bonds at amortised historical cost. Section 2.3 discusses the alternative of reporting bonds at fair value.

Companies initially report bonds as a liability on their balance sheet at the amount of the sales proceeds net of issuance costs under both IFRS and US GAAP. The amount at which bonds are reported on the company's balance sheet is referred to as the carrying amount, carrying value, book value, or net book value. If the bonds are issued at par, the initial carrying amount will be identical to the face value, and usually the carrying amount will not change over the life of the bonds. ${ }^{3}$ For bonds issued at face value, the amount of periodic interest expense will be the same as the amount of periodic interest payment to bondholders.

If, however, the market rate differs from the bonds' coupon rate at issuance such that the bonds are issued at a premium or discount, the premium or discount is amortised systematically over the life of the bonds as a component of interest expense. For bonds issued at a premium to face value, the carrying amount of the bonds is initially greater than the face value. As the premium is amortised, the carrying amount (amortised cost) of the bonds will decrease to the face value. The reported interest expense will be less than the coupon payment. For bonds issued at a discount to face value, the carrying amount of the bonds is initially less than the face value. As the discount is amortised, the carrying amount (amortised cost) of the bonds will increase to the face value. The reported interest expense will be higher than the coupon payment.

The accounting treatment for bonds issued at a discount reflects the fact that the company essentially paid some of its borrowing costs at issuance by selling its bonds at a discount. Rather than there being an actual cash transfer in the future, this "payment" was made in the form of accepting less than the face value for the bonds at the date of issuance. The remaining borrowing cost occurs as a cash interest payment to investors each period. The total interest expense reflects both components of the borrowing cost: the periodic interest payments plus the amortisation of the discount. The accounting treatment for bonds issued at a premium reflects the fact that the company essentially received a reduction on its borrowing costs at issuance by selling its bonds at a premium. Rather than there being an actual reduced cash transfer in the future, this "reduction" was made in the form of receiving more than face value for the bonds at the date of issuance. The total interest expense reflects both components of the borrowing cost: the periodic interest payments less the amortisation of the premium. When the bonds mature, the carrying amount will be equal to the face value regardless of whether the bonds were issued at face value, a discount, or a premium.

Two methods for amortising the premium or discount of bonds that were issued at a price other than par are the effective interest rate method and the straight-line method. The effective interest rate method is required under IFRS and preferred under US GAAP because it better reflects the economic substance of the transaction. The effective interest rate method applies the market rate in effect when the bonds were issued (historical market rate or effective interest rate) to the current amortised cost (carrying amount) of the bonds to obtain interest expense for the period. The difference between the interest expense (based on the effective interest rate and amortised cost) and the interest payment (based on the coupon rate and face value) is the amortisation of the discount or premium. The straight-line method of amortisation evenly amortises the premium or discount over the life of the bond, similar to straight-line depreciation on long-lived assets. Under either method, as the bond approaches maturity, the amortised cost approaches face value.

Example 3 illustrates both methods of amortisation for bonds issued at a discount. Example 4 shows amortisation for bonds issued at a premium.

\section{EXAMPLE 3}
\section{Amortising a Bond Discount}
Debond Corp. issues $\pounds 1,000,000$ face value of five-year bonds, dated 1 January 2017, when the market interest rate is 6 percent. The sales proceeds are $\pounds 957,876$. The bonds pay 5 percent interest annually on 31 December.

3 If a company reports debt at fair value, rather than amortised cost, the carrying value may change. 1. What is the interest payment on the bonds each year?

\section{Solution to 1:}
The interest payment equals $\pounds 50,000$ annually ( $\pounds 1,000,000 \times 5 \%)$.

\begin{enumerate}
  \setcounter{enumi}{1}
  \item What amount of interest expense on the bonds would be reported in 2017 and 2018 using the effective interest rate method?
\end{enumerate}

\section{Solution to 2:}
The sales proceeds of $\pounds 957,876$ are less than the face value of $\pounds 1,000,000$; the bonds were issued at a discount of $\pounds 42,124$. The bonds are initially reported as a long-term liability, bonds payable, of $\pounds 957,876$, which comprises the face value of $\pounds 1,000,000$ minus a discount of $\pounds 42,124$. The discount is amortised over time, ultimately, increasing the carrying amount (amortised cost) to face value.

Under the effective interest rate method, interest expense on the bonds is calculated as the bonds' carrying amount times the market rate in effect when the bonds are issued (effective interest rate). For 2017, interest expense is $\pounds 57,473=(\pounds 957,876 \times 6 \%)$. The amount of the discount amortised in 2017 is the difference between the interest expense of $\pounds 57,473$ and the interest payment of $\pounds 50,000$ (i.e., $\pounds 7,473$ ). The bonds' carrying amount increases by the discount amortisation; at 31 December 2017, the bonds' carrying amount is $\pounds 965,349$ (beginning balance of $\pounds 957,876$ plus $\pounds 7,473$ discount amortisation). At this point, the carrying amount reflects a remaining unamortised discount of $\pounds 34,651$ ( $\pounds 42,124$ discount at issuance minus $\pounds 7,473$ amortised).

For 2018, interest expense is $\pounds 57,921=(\pounds 965,349 \times 6 \%)$, the carrying amount of the bonds on 1 January 2018 times the effective interest rate. The amount of the discount amortised in 2018 is the difference between the interest expense of $\pounds 57,921$ and the interest payment of $\pounds 50,000$ (i.e., $\pounds 7,921$ ). At 31 December 2018, the bonds' carrying amount is $\pounds 973,270$ (beginning balance of $\pounds 965,349$ plus $\pounds 7,921$ discount amortisation).

The following table illustrates interest expense, discount amortisation, and carrying amount (amortised cost) over the life of the bonds.

\begin{center}
\begin{tabular}{|c|c|c|c|c|c|}
\hline
Year & $\begin{array}{l}\text { Carrying Amount } \\ \text { (beginning of year) }\end{array}$ & $\begin{array}{l}\text { Interest Expense (at } \\ \text { effective interest rate } \\ \text { of } 6 \% \text { ) }\end{array}$ & $\begin{array}{l}\text { Interest Payment } \\ \text { (at coupon rate of } \\ \mathbf{5 \% )}\end{array}$ & $\begin{array}{c}\text { Amortisation } \\ \text { of Discount }\end{array}$ & $\begin{array}{c}\text { Carrying } \\ \text { Amount (end o } \\ \text { year) }\end{array}$ \\
\hline
 & (a) & (b) & (c) & (d) & (e) \\
\hline
2017 & $\pounds 957,876$ & $\pounds 57,473$ & $\pounds 50,000$ & $\pounds 7,473$ & $\pounds 965,349$ \\
\hline
2018 & 965,349 & 57,921 & 50,000 & 7,921 & 973,270 \\
\hline
2019 & 973,270 & 58,396 & 50,000 & 8,396 & 981,666 \\
\hline
2020 & 981,666 & 58,900 & 50,000 & 8,900 & 990,566 \\
\hline
2021 & 990,566 & 59,434 & 50,000 & 9,434 & $1,000,000$ \\
\hline
Total &  & $\pounds 292,124$ & $\pounds 250,000$ & $\pounds 42,124$ &  \\
\hline
\end{tabular}
\end{center}

\begin{enumerate}
  \setcounter{enumi}{2}
  \item Determine the reported value of the bonds (i.e., the carrying amount) at 31 December 2017 and 2018, assuming the effective interest rate method is used to amortise the discount.
\end{enumerate}

\section{Solution to 3:}
The carrying amounts of the bonds at 31 December 2017 and 2018 are $\pounds 965,349$ and $\pounds 973,270$, respectively. Observe that the carrying amount of the bonds issued at a discount increases over the life of the bonds. At maturity, 31 December 2021, the carrying amount of the bonds equals the face value of the bonds. The carrying amount of the bonds will be reduced to zero when the principal payment is made.

\begin{enumerate}
  \setcounter{enumi}{3}
  \item What amount of interest expense on the bonds would be reported under the straight-line method of amortising the discount?
\end{enumerate}

\section{Solution to 4:}
Under the straight-line method, the discount (or premium) is evenly amortised over the life of the bonds. In this example, the $\pounds 42,124$ discount would be amortised by $\pounds 8,424.80$ ( $\pounds 42,124$ divided by 5 years) each year under the straight-line method. So, the annual interest expense under the straight-line method would be $\pounds 58,424.80(\pounds 50,000$ plus $\pounds 8,424.80)$.

The accounting and reporting for zero-coupon bonds is similar to the example above except that no interest payments are made; thus, the amount of interest expense each year is the same as the amount of the discount amortisation for the year.

\section{EXAMPLE 4}
\section{Amortising a Bond Premium}
Prembond Corp. issues $\pounds 1,000,000$ face value of five-year bonds, dated 1 January 2017, when the market interest rate is 4 percent. The sales proceeds are $\pounds 1,044,518$. The bonds pay 5 percent interest annually on 31 December.

\begin{enumerate}
  \item What is the interest payment on the bonds each year?
\end{enumerate}

\section{Solution to 1:}
The interest payment equals $\pounds 50,000$ annually $(\pounds 1,000,000 \times 5 \%)$.

\begin{enumerate}
  \setcounter{enumi}{1}
  \item What amount of interest expense on the bonds would be reported in 2017 and 2018 using the effective interest rate method?
\end{enumerate}

\section{Solution to 2:}
The sales proceeds of $\pounds 1,044,518$ are more than the face value of $\pounds 1,000,000$; the bonds were issued at a premium of $\pounds 44,518$. The bonds are initially reported as a long-term liability, bonds payable, of $\pounds 1,044,518$, which comprises the face value of $\pounds 1,000,000$ plus a premium of $\pounds 44,518$. The premium is amortised over time, ultimately decreasing the carrying amount (amortised cost) to face value.

Under the effective interest rate method, interest expense on the bonds is calculated as the bonds' carrying amount times the market rate in effect when the bonds are issued (effective interest rate). For 2017, interest ex- pense is $\pounds 41,781=(\pounds 1,044,518 \times 4 \%)$. The amount of the premium amortised in 2017 is the difference between the interest expense of $\pounds 41,781$ and the interest payment of $\pounds 50,000$ (i.e., $\pounds 8,219$ ). The bonds' carrying amount decreases by the premium amortisation; at 31 December 2017, the bonds' carrying amount is $\pounds 1,036,299$ (beginning balance of $\pounds 1,044,518$ less $\pounds 8,219$ premium amortisation). At this point, the carrying amount reflects a remaining unamortised premium of $\pounds 36,299$ ( $\pounds 44,518$ premium at issuance minus $\pounds 8,219$ amortised).

For 2018, interest expense is $\pounds 41,452=(\pounds 1,036,299 \times 4 \%)$. The amount of the premium amortised in 2011 is the difference between the interest expense of $\pounds 41,452$ and the interest payment of $\pounds 50,000$ (i.e., $\pounds 8,548)$. At 31 December 2018, the bonds' carrying amount is $\pounds 1,027,751$ (beginning balance of $\pounds 1,036,299$ less $\pounds 8,548$ premium amortisation).

The following table illustrates interest expense, premium amortisation, and carrying amount (amortised cost) over the life of the bonds.

\begin{center}
\begin{tabular}{|c|c|c|c|c|c|}
\hline
Year & $\begin{array}{l}\text { Carrying Amount } \\ \text { (beginning of year) }\end{array}$ & $\begin{array}{c}\text { Interest Expense (at } \\ \text { effective interest rate } \\ \text { of } 4 \% \text { ) }\end{array}$ & $\begin{array}{l}\text { Interest Payment (at } \\ \text { coupon rate of } 5 \% \text { ) }\end{array}$ & $\begin{array}{c}\text { Amortisation of } \\ \text { Premium }\end{array}$ & $\begin{array}{c}\text { Carrying Amount } \\ \text { (end of year) }\end{array}$ \\
\hline
 & (a) & (b) & (c) & (d) & (e) \\
\hline
2017 & $\pounds 1,044,518$ & $\pounds 41,781$ & $\pounds 50,000$ & $\pounds 8,219$ & $\pounds 1,036,299$ \\
\hline
2018 & $1,036,299$ & 41,452 & 50,000 & 8,548 & $1,027,751$ \\
\hline
2019 & $1,027,751$ & 41,110 & 50,000 & 8,890 & $1,018,861$ \\
\hline
2020 & $1,018,861$ & 40,754 & 50,000 & 9,246 & $1,009,615$ \\
\hline
2021 & $1,009,615$ & 40,385 & 50,000 & 9,615 & $1,000,000$ \\
\hline
Total &  &  &  & $\pounds 44,518$ &  \\
\hline
\end{tabular}
\end{center}

\begin{enumerate}
  \setcounter{enumi}{2}
  \item Determine the reported value of the bonds (i.e., the carrying amount) at 31 December 2017 and 2018, assuming the effective interest rate method is used to amortise the premium.
\end{enumerate}

\section{Solution to 3:}
The carrying amounts of the bonds at 31 December 2017 and 2018 are $\pounds 1,036,299$ and $\pounds 1,027,751$, respectively. Observe that the carrying amount of the bonds issued at a premium decreases over the life of the bonds. At maturity, 31 December 2021, the carrying amount of the bonds equals the face value of the bonds. The carrying amount of the bonds will be reduced to zero when the principal payment is made.

\begin{enumerate}
  \setcounter{enumi}{3}
  \item What amount of interest expense on the bonds would be reported under the straight-line method of amortising the premium?
\end{enumerate}

\section{Solution to 4:}
Under the straight-line method, the premium is evenly amortised over the life of the bonds. In this example, the $\pounds 44,518$ premium would be amortised by $\pounds 8,903.64$ ( $\pounds 44,518$ divided by 5 years) each year under the straight-line method. So, the annual interest expense under the straight-line method would be $\pounds 41,096.36$ ( $\pounds 50,000$ less $\pounds 8,903.64$ ). The reporting of interest payments on the statement of cash flows can differ under IFRS and US GAAP. Under IFRS, interest payments on bonds can be included as an outflow in either the operating section or the financing section of the statement of cash flows. US GAAP requires interest payments on bonds to be included as an operating cash outflow. (Some financial statement users consider the placement of interest payments in the operating section to be inconsistent with the placement of bond issue proceeds in the financing section of the statement of cash flows.) Typically, cash interest paid is not shown directly on the statement of cash flows, but companies are required to disclose interest paid separately.

Amortisation of a discount (premium) is a non-cash item and thus, apart from its effect on pretax income, has no effect on cash flow. In the section of the statement of cash flows that reconciles net income to operating cash flow, amortisation of a discount (premium) is added back to (subtracted from) net income.

\section{ACCOUNTING FOR BONDS AT FAIR VALUE}
determine the initial recognition, initial measurement and subsequent measurement of bonds

Reporting bonds at amortised historical costs (historical cost plus or minus the cumulative amortisation) reflects the market rate at the time the bonds were issued (i.e., historical market rate or effective interest rate). As market interest rates change, the bonds' carrying amount diverges from the bonds' fair market value. When market interest rates decline, the fair value of a bond with a fixed coupon rate increases. As a result, a company's economic liabilities may be higher than its reported debt based on amortised historical cost. Conversely, when market interest rates increase, the fair value of a bond with a fixed coupon rate decreases and the company's economic liability may be lower than its reported debt. Using financial statement amounts based on amortised cost may underestimate (or overestimate) a company's debt-to-total-capital ratio and similar leverage ratios.

Companies have the option to report financial liabilities at fair value. Financial liabilities reported at fair value are designated as financial liabilities at fair value through profit or loss under IFRS, or, equivalently under US GAAP, as liabilities under the fair value option. Even if a company does not opt to report financial liabilities at fair value, the availability of fair value information in the financial statements has increased. IFRS and US GAAP require fair value disclosures in the financial statements unless the carrying amount approximates fair value or the fair value cannot be reliably measured. ${ }^{4}$

A company electing to measure a liability at fair value will report decreases in the liability's fair value as income and increases in the liability's fair value as losses. Because changes in a liability's fair value can result from changes in market rates and/ or changes in the credit quality of the issuing company, accounting standards require companies to present separately the portion of the change resulting from changes in their own credit risk. Specifically, the company will report the portion of the change in value attributable to changes in their credit risk in other comprehensive income. Only the portion of the change in value not attributable to changes in their credit risk will be recognised in profit or loss. ${ }^{5}$ As of the end of 2018, few companies have selected the option to report financial liabilities at fair value. Those that have are primarily companies in the financial sector. Reporting standards for financial investments and derivatives already required these companies to report a significant portion of their assets at fair values. Measuring financial liabilities at other than fair value, when financial assets are measured at fair value, results in earnings volatility. This volatility is the result of using different bases of measurement for financial assets and financial liabilities. Furthermore, when a liability is related to a specific asset, using different measurement bases creates an accounting mismatch. Goldman Sachs elected to account for some financial liabilities at fair value under the fair value option. In its fiscal year 2017 10-K filing (page 136), Goldman explains this choice:

"The primary reasons for electing the fair value option are to:

\begin{itemize}
  \item Reflect economic events in earnings on a timely basis;

  \item Mitigate volatility in earnings from using different measurement attributes (e.g., transfers of financial instruments owned accounted for as financings are recorded at fair value, whereas the related secured financing would be recorded on an accrual basis absent electing the fair value option); and

  \item Address simplification and cost-benefit considerations (e.g., accounting for hybrid financial instruments at fair value in their entirety versus bifurcation of embedded derivatives and hedge accounting for debt hosts)."

\end{itemize}

Most companies, as required under IFRS and US GAAP, disclose the fair values of financial liabilities. The primary exception to the disclosure occurs when fair value cannot be reliably measured. Example 5 illustrates ING Group's fair value disclosures, including the fair values of long-term debt.

\section{EXAMPLE 5}
Fair Value Disclosures of Debt and Financial Instruments

ING Group 2017 Form 20-F

ING Group [Condensed] Balance Sheet as of 31 December 2017 and 2016 [Liabilities Only]

\begin{center}
\begin{tabular}{lcc}
\hline
Amounts in billions of euros & $\mathbf{2 0 1 7}$ & $\mathbf{2 0 1 6}$ \\
\hline
Deposits from banks & 36.8 & 32.0 \\
Customer deposits & 539.8 & 522.9 \\
Financial liabilities at fair value through & 87.2 & 99.0 \\
profit or loss & 18.9 & 20.1 \\
Other liabilities & 112.1 & 120.4 \\
Debt securities in issue/subordinated &  &  \\
loans & $\mathbf{7 9 4 . 8}$ & $\mathbf{7 9 4 . 4}$ \\
Total liabilities &  &  \\
\end{tabular}
\end{center}

The following are excerpts from the footnotes to ING Group's financial statements.

\section{Excerpt from Note 1 Accounting Policies}
\section{Financial assets and liabilities at fair value through profit or loss}
... Financial liabilities at fair value through profit or loss comprise the following sub-categories: trading liabilities, non-trading derivatives, and other financial liabilities designated at fair value through profit or loss by management. Trading liabilities include equity securities, debt securities, funds on deposit, and derivatives.

A financial asset or financial liability is classified at fair value through profit or loss if acquired principally for the purpose of selling in the short term or if designated by management as such. Management will designate a financial asset or a financial liability as such only if this eliminates a measurement inconsistency or if the related assets and liabilities are managed on a fair value basis....

\section{Financial liabilities at amortised cost}
Financial liabilities at amortised cost include the following sub-categories: preference shares classified as debt, debt securities in issue, subordinated loans, and deposits from banks and customer deposits.

Financial liabilities at amortised cost are recognised initially at their issue proceeds (fair value of consideration received) net of transaction costs incurred. Liabilities in this category are subsequently stated at amortised cost; any difference between proceeds, net of transaction costs, and the redemption value is recognised in the statement of profit or loss over the period of the liability using the effective interest method....

\section{Excerpt from Note 16 Debt securities in issue}
Debt securities in issue relate to debentures and other issued debt securities with either fixed interest rates or interest rates based on floating interest rate levels, such as certificates of deposit and accepted bills issued by ING Group, except for subordinated items. Debt securities in issue do not include debt securities presented as Financial liabilities at fair value through profit or loss.

\section{Excerpt from Note 37 Fair value of assets and liabilities}
Fair Value of Financial Liabilities as of 31 December 2017 and 2016

\begin{center}
\begin{tabular}{|c|c|c|c|c|}
\hline
\multirow[b]{2}{*}{Amounts in millions of euros} & \multicolumn{2}{|c|}{Estimated Fair Value} & \multicolumn{2}{|c|}{$\begin{array}{c}\text { Statement of } \\
\text { Financial Position } \\
\text { Value }\end{array}$} \\
\hline
 & 2017 & 2016 & 2017 & 2016 \\
\hline
\multicolumn{5}{|l|}{Financial liabilities} \\
\hline
Deposits from banks & 36,868 & 32,352 & 36,821 & 31,964 \\
\hline
Customer deposits & 540,547 & 523,850 & 539,828 & 522,908 \\
\hline
\multicolumn{5}{|l|}{$\begin{array}{l}\text { Financial liabilities at fair value } \\
\text { through profit or loss }\end{array}$} \\
\hline
- trading liabilities & 73,596 & 83,167 & 73,596 & 83,167 \\
\hline
- non-trading derivatives & 2,331 & 3,541 & 2,331 & 3,541 \\
\hline
$\begin{array}{l}\text { - designated as at fair value } \\ \text { through profit or loss }\end{array}$ & 11,215 & 12,266 & 11,215 & 12,266 \\
\hline
Other liabilities & 14,488 & 15,247 & 14,488 & 15,247 \\
\hline
Debt securities in issue & 96,736 & 103,559 & 96,086 & 103,234 \\
\hline
\multirow[t]{2}{*}{Subordinated loans} & 16,457 & 17,253 & 15,968 & 17,223 \\
\hline
 & 792,238 & 791,235 & 790,333 & 789,550 \\
\hline
\end{tabular}
\end{center}

Use the condensed balance sheet and excerpts from the notes to ING Group's financial statements shown above to address the following questions:

\begin{enumerate}
  \item As of 31 December 2017, what proportion of the amount of liabilities on ING Group's balance sheet is reported at fair value through profit or loss?
\end{enumerate}

Solution to 1:

Of ING Group's total $€ 794.8$ billion liabilities, 11 percent (=€87.2 billion/€794.8 billion) are reported at fair value through profit or loss.

\begin{enumerate}
  \setcounter{enumi}{1}
  \item As of 31 December 2017 and 2016, what is the percent difference between the carrying amount and fair value of the debt securities that are shown on ING Group's balance sheet at amortised cost?
\end{enumerate}

\section{Solution to 2:}
ING's debt securities that are shown on the balance sheet at amortised cost appear in the line labeled "Debt securities in issue". Note 1 states that "Debt securities in issue" are reported at amortised cost. Note 16 indicates that this line item relates to debentures and other issued debt securities, and thus we exclude subordinated loans and deposits from banks and customer deposits in this case (there are no preference shares classified as debt listed in the Note 37 excerpt).

According to the above excerpt from Note 37, in each year the fair value of ING's debt securities is slightly higher than its carrying amount. The difference is $0.7 \%[=(96,736 / 96,086)-1]$ on 31 December 2017 and $0.3 \%$ [= $(103,559 / 103,234)-1]$ on 31 December 2016.

\section{DERECOGNITION OF DEBT}
explain the derecognition of debt

Once bonds are issued, a company may leave the bonds outstanding until maturity or redeem the bonds before maturity either by calling the bonds (if the bond issue includes a call provision) or by purchasing the bonds in the open market. If the bonds remain outstanding until the maturity date, the company pays bondholders the face value of the bonds at maturity. The discount or premium on the bonds would be fully amortised at maturity; the carrying amount would equal face value. Upon repayment, the bonds payable account is reduced by the carrying amount at maturity (face value) of the bonds, and cash is reduced by an equal amount. Repayment of the bonds appears in the statement of cash flows as a financing cash outflow.

If a company decides to redeem bonds before maturity and thus extinguish the liability early, the bonds payable account is reduced by the carrying amount of the redeemed bonds. The difference between the cash required to redeem the bonds and the carrying amount of the bonds is a gain or loss on the extinguishment of debt. Under IFRS, debt issuance costs are included in the measurement of the liability and are thus part of its carrying amount. Under US GAAP, debt issuance costs are accounted for separately from bonds payable and are amortised over the life of the bonds. Any unamortised debt issuance costs must be written off at the time of redemption and included in the gain or loss on debt extinguishment.

For example, a company reporting under IFRS has a $\pounds 10$ million bond issuance with a carrying amount equal to its face value and five years remaining until maturity. The company redeems the bonds at a call price of 103 . The redemption cost is $\pounds 10.3$ million $(=\pounds 10$ million $\times 103 \%)$. The company's loss on redemption would be $\pounds 300$ thousand ( $\pounds 10$ million carrying amount minus $\pounds 10.3$ million cash paid to redeem the callable bonds).

A gain or loss on the extinguishment of debt is reported on the income statement, in a separate line item, when the amount is material. A company typically discloses further detail about the extinguishment in the management discussion and analysis (MD\&A) and/or notes to the financial statements. ${ }^{6}$ In addition, in a statement of cash flows prepared using the indirect method, net income is adjusted to remove any gain or loss on the extinguishment of debt from operating cash flows and the cash paid to redeem the bonds is classified as cash used for financing activities. (Recall that the indirect method of the statement of cash flows begins with net income and makes necessary adjustments to arrive at cash from operations, including removal of gains or losses from non-operating activities.)

To illustrate the financial statement impact of the extinguishment of debt, consider the notes payable repurchase in Example 6 below.

\section{EXAMPLE 6}
\section{Debt Extinguishment Disclosure}
The following excerpts are from the 2018 annual report of Monte Rock Inc. (a hypothetical company). In its statement of cash flows, the company uses the indirect method to reconcile net income with net cash (used in) provided by operations.

\section{Excerpt from Consolidated Statements of Income}
For the years ended 31 December 2018, 2017, and 2016

\section{8}
Revenues:

$\vdots$

Total revenues

:

Total operating expenses

Income from operations

Other income (expense):

Gain on debt extinguishment

$2,345,000$

\begin{center}
\begin{tabular}{c}
$100,279,900$ \\
\hline
$4,629,000$ \\
$\vdots$ \\
$2,345,000$ \\
\end{tabular}
\end{center}

$96,879,000$

;

6 We use the term MD\&A generally to refer to any management commentary provided on a company's financial condition, changes in financial condition, and results of operations. In the United States, the Securities and Exchange Commission (SEC) requires a management discussion and analysis for companies listed on US public markets. Reporting requirements for such a commentary as the SEC-required MD\&A vary across exchanges, but some are similar to the SEC requirements. The IASB issued an IFRS Practice Statement, "Management Commentary," in December 2010 to guide all companies reporting under IFRS.

\begin{center}
\begin{tabular}{|c|c|c|c|}
\hline
 & 2018 & 2017 & 2016 \\
\hline
Total other income (expense), net & $11,236,100$ & $(14,257,000)$ & $(7,085,800)$ \\
\hline
Net income & $\$ 15,865,100$ & $\$ 2,019,200$ & $\$ 18,774,300$ \\
\hline
\end{tabular}
\end{center}

Excerpt from Consolidated Statements of Cash Flows

For the years ended 31 December 2018, 2017, and 2016

\begin{center}
\begin{tabular}{|c|c|c|c|}
\hline
 & 2018 & 2017 & 2016 \\
\hline
\multicolumn{4}{|l|}{CASH FLOWS FROM OPERATING ACTIVITIES:} \\
\hline
Net Income & $\$ 15,865,100$ & $\$ 2,019,200$ & $\$ 18,774,300$ \\
\hline
\multicolumn{4}{|l|}{$\begin{array}{l}\text { Adjustments to reconcile net income to net cash (used } \\
\text { in) provided by operating activities: }\end{array}$} \\
\hline
$\vdots$ & $\vdots$ & $\vdots$ & $\vdots$ \\
\hline
Gain on debt extinguishment & $(2,345,000)$ & - & - \\
\hline
$\vdots$ & $\vdots$ & $\vdots$ & $\vdots$ \\
\hline
Total adjustments & $(16,636,000)$ & $38,842,400$ & $19,815,800$ \\
\hline
Net cash (used in) provided by operating activities & $(770,900)$ & $40,861,600$ & $38,590,100$ \\
\hline
$\vdots$ & $\vdots$ & $\vdots$ & $\vdots$ \\
\hline
\multicolumn{4}{|l|}{CASH FLOWS FROM FINANCING ACTIVITIES:} \\
\hline
Payments for debt financing costs & $(294,000)$ & $(1,526,500)$ & $(1,481,500)$ \\
\hline
$\vdots$ & $\vdots$ & $\vdots$ & $\vdots$ \\
\hline
Purchase of debt securities & $(2,155,000)$ & - & $(5,000,000)$ \\
\hline
$\vdots$ & $\vdots$ & $\vdots$ & $\vdots$ \\
\hline
Payments of unsecured debt & - & $(31,402,960)$ & $(1,356,000)$ \\
\hline
\end{tabular}
\end{center}

\section{Excerpt from NOTE 8: BONDS PAYABLE}
On December 12, 2014, the Company issued $\$ 25$ million of unsecured bonds... Interest on the bonds is equal to Libor plus $4 \%$, payable quarterly in arrears. ... During the 4 th quarter of 2018 , the Company repurchased the unsecured bonds with a face value of $\$ 4.5$ million and realized a $\$ 2.3$ million gain.

\begin{enumerate}
  \item The balance in bonds payable was reduced at redemption by:
\end{enumerate}

A. $\$ 2,155,000$.

B. $\$ 2,345,000$.

C. $\$ 4,500,000$.

Solution to 1:

$C$ is correct. The bonds payable is reduced at redemption by the carrying amount of the bonds redeemed. The cash paid to extinguish the bonds plus the gain on redemption equals the carrying amount of the bonds. The carrying amount of the bonds was $\$ 4,500,000$. In this case, the carrying amount equals the face value. The company recognised a gain of $\$ 2,345,000$ when it extinguished the debt of $\$ 4,500,000$ by paying only $\$ 2,155,000$.

\begin{enumerate}
  \setcounter{enumi}{1}
  \item How much cash did the Company pay to redeem the bonds?
\end{enumerate}

A. $\$ 2,155,000$

B. $\$ 2,345,000$ C. $\$ 4,500,000$

\section{Solution to 2:}
A is correct. As shown in the Statement of Cash flow, the company paid $\$ 2,155,000$ to redeem the bonds. The company recognised a gain of $\$ 2,345,000$ when it extinguished the debt of $\$ 4,500,000$ by paying only $\$ 2,155,000$.

\section{DEBT COVENANTS}
$\square \quad$ describe the role of debt covenants in protecting creditors

Borrowing agreements (the bond indenture) often include restrictions called covenants that protect creditors by restricting activities of the borrower. Debt covenants benefit borrowers to the extent that they lower the risk to the creditors and thus reduce the cost of borrowing. Affirmative covenants restrict the borrower's activities by requiring certain actions. For instance, covenants may require that the borrower maintain certain ratios above a specified amount or perform regular maintenance on real assets used as collateral. Negative covenants require that the borrower not take certain actions. These covenants may restrict the borrower's ability to invest, pay dividends, or make other operating and strategic decisions that might adversely affect the company's ability to pay interest and principal.

Common covenants include limitations on how borrowed monies can be used, maintenance of collateral pledged as security (if any), restrictions on future borrowings, and requirements that limit dividends. Covenants may also specify minimum acceptable levels of financial ratios, such as debt-to-equity ratio, current ratio, or interest coverage.

When a company violates a debt covenant, it is a breach of contract. Depending on the severity of the breach and the terms of the contract, lenders may choose to waive the covenant, be entitled to a penalty payment or higher interest rate, renegotiate, or call for payment of the debt. Bond contracts typically require that the decision to call for immediate repayment be made, on behalf of all the bondholders, by holders of some minimum percentage of the principal amount of the bond issue.

Example 7 illustrates common disclosures related to debt covenants included in financial statement disclosures (notes to the financial statements).

\section{EXAMPLE 7}
\section{Illustration of Debt Covenant Disclosures}
The following excerpts are from the 2017 Form 20-F filing of TORM plc, a tanker company which describes itself as one of the world's largest carriers of refined oil products. TORM plc was established in 2016 following the restructuring of TORM A/S.

The following excerpt is from the Risk Factors section of TORM's fiscal year 2017 Form 20-F.

Our current debt facilities impose restrictions on our financial and operational flexibility. Our debt facilities impose, and any future debt facility may impose, covenants and other operating and financial restrictions on our ability to, among other things, pay dividends, charter-in vessels, incur additional debt, sell vessels or refrain from procuring the timely release of arrested vessels. Our debt facilities require us to maintain various financial ratios, including a specified minimum liquidity requirement, a minimum equity requirement and a collateral maintenance requirement. Our ability to comply with these restrictions and covenants is dependent on our future performance and our ability to operate our fleet and may be affected by events beyond our control, including fluctuating vessel values. We may therefore need to seek permission from our lenders in order to engage in certain corporate actions.

Failure to comply with the covenants and financial and operational restrictions under our debt facilities may lead to an event of default under those agreements. An event of default may lead to an acceleration of the repayment of debt. In addition, any default or acceleration under our existing debt facilities or agreements governing our other existing or future indebtedness is likely to lead to an acceleration of the repayment of debt under any other debt instruments that contain cross-acceleration or cross-default provisions. If all or a part of our indebtedness is accelerated, we may not be able to repay that indebtedness or borrow sufficient funds to refinance that debt, which could have a material adverse effect on our future performance, results of operations, cash flows and financial position and could lead to bankruptcy or other insolvency proceedings.

... As of December 31, 2017, we were in compliance with the financial covenants contained in our debt facilities.

The following excerpt is from the Liquidity and Capital Resources section in TORM's fiscal year 2017 Form 20-F.

The DSF [Danish Ship Finance] Facility contains, among others, the following financial and other covenants:

\begin{itemize}
  \item Loan-to-value. If at any time the aggregate market value of the vessels and the value of any additional security is less than $133 \%$ of the loan amount less amounts on credit in the deposit accounts and reserve account and the value of any additional security, the borrower and guarantors shall, within 30 days of a written request, post additional security or prepay the loan to reduce the excess to zero.

  \item Free Liquidity. Minimum unencumbered cash and cash equivalents ... of the higher of $\$ 75$ million and $5 \%$ of our total debt, of which $\$ 40$ million is required to be unencumbered cash and cash equivalents.

  \item Equity Ratio. The ratio of market value adjusted shareholders' equity to total market value adjusted assets shall be at least $25 \%$.

  \item Dividends. We are restricted from making any distributions, including payment of dividends and repayments of shareholders loans, except ....

\end{itemize}

\begin{enumerate}
  \item Which of the covenants described in the above excerpts are affirmative covenants?
\end{enumerate}

\section{Solution to 1:}
Examples of affirmative covenants in the above excerpts are from TORM's disclosure about the DSF Facility and include: the requirement for TORM to maintain a loan-to-value relationship such that the assets securing the loan (the vessels) are 133\% of the loan amount; the requirement for TORM to maintain "free liquidity" (i.e., a minimum level of cash and cash equivalents); and the requirement that the equity ratio be at least $25 \%$. These covenants require the issuer to do something. The dividend covenant requiring that TORM not take certain actions (i.e., not pay dividends unless certain conditions are met) is a negative covenant. In addition, the excerpt from Risk Factors describes other negative covenants in TORM's debt facilities including restrictions on chartering-in vessels, incurring additional debt, or selling vessels.

\begin{enumerate}
  \setcounter{enumi}{1}
  \item Based on the excerpt above, what are the potential consequences of breaching the loan covenants?
\end{enumerate}

\section{Solution to 2:}
A breach of a loan covenant by TORM-an event of default-may result in the entire amount of its debt becoming due.

\section{PRESENTATION AND DISCLOSURE OF LONG-TERM DEBT}
describe the financial statement presentation of and disclosures relating to debt

The non-current (long-term) liabilities section of the balance sheet usually includes a single line item of the total amount of a company's long-term debt due after one year, with the portion of long-term debt due in the next twelve months shown as a current liability. Notes to the financial statements provide more information on the types and nature of a company's debt. These note disclosures can be used to determine the amount and timing of future cash outflows. The notes generally include stated and effective interest rates, maturity dates, restrictions imposed by creditors (covenants), and collateral pledged (if any). The amount of scheduled debt repayments for the next five years also is shown in the notes.

Example 8 contains an excerpt from the 2017 Form 10-K of Johnson \& Johnson (J\&J), a US manufacturer of health care products.

\section{EXAMPLE 8}
\section{Illustration of Long-Term Debt Disclosures}
Exhibit 1 is an excerpt from Note 7 of Johnson \& Johnson's 2017 annual report that illustrates financial statement disclosures for long-term debt, including type and nature of long-term debt, carrying amounts, effective interest rates, and required payments over the next five years. Johnson \& Johnson reports its debt at amortised cost.

\section{Exhibit 1: Johnson $\&$ Johnson Borrowings}
The components of long-term debt are as follows:

\begin{center}
\begin{tabular}{lcccc}
\hline
(Dollars in Millions) & $\mathbf{2 0 1 7}$ & $\begin{array}{c}\text { Effective } \\ \text { Rate \% }\end{array}$ & $\mathbf{2 0 1 6}$ & $\begin{array}{c}\text { Effective } \\ \text { Rate \% }\end{array}$ \\
\hline
$5.55 \%$ Debentures due 2017 & $\$-$ & - & $\$ 1,000$ & 5.55 \\
$1.125 \%$ Notes due 2017 & - & - & 699 & 1.15 \\
$5.15 \%$ Debentures due 2018 & 900 & 5.18 & 899 & 5.18 \\
\end{tabular}
\end{center}

The components of long-term debt are as follows:

\begin{center}
\begin{tabular}{|c|c|c|c|c|}
\hline
(Dollars in Millions) & 2017 & $\begin{array}{c}\text { Effective } \\ \text { Rate \% }\end{array}$ & 2016 & $\begin{array}{c}\text { Effective } \\ \text { Rate \% }\end{array}$ \\
\hline
1.65\% Notes due 2018 & 597 & 1.70 & 600 & 1.70 \\
\hline
$\begin{array}{l}4.75 \% \text { Notes due } 2019(1 \mathrm{~B} \\ \text { Euro } 1.1947)^{2} /(1 \mathrm{~B} \text { Euro } \\ 1.0449)^{3}\end{array}$ & 1,192 & 5.83 & 1,041 & 5.83 \\
\hline
$1.875 \%$ Notes due 2019 & 496 & 1.93 & 499 & 1.93 \\
\hline
$0.89 \%$ Notes due 2019 & 300 & 1.75 & 299 & 1.20 \\
\hline
$1.125 \%$ Notes due 2019 & 699 & 1.13 & 699 & 1.13 \\
\hline
$\begin{array}{l}\text { 3\% Zero Coupon Convertible } \\ \text { Subordinated Debentures due } \\ 2020\end{array}$ & 60 & 3.00 & 84 & 3.00 \\
\hline
$\begin{array}{c}\text { 2.95\% Debentures due } 2020 \\ \text { [PORTIONS OMITTED] }\end{array}$ & 547 & 3.15 & 546 & 3.15 \\
\hline
Subtotal & 32,174 & $3.19^{1}$ & 24,146 & $3.33^{1}$ \\
\hline
Less current portion & 1,499 &  & 1,704 &  \\
\hline
Total long-term debt & $\$ 30,675$ &  & $\$ 22,442$ &  \\
\hline
\end{tabular}
\end{center}

(1) Weighted average effective rate.

(2) Translation rate at December 31, 2017.

(3) Translation rate at January 1, 2017.

"Fair value of the long-term debt was estimated using market prices, which were corroborated by quoted broker prices and significant other observable inputs.

"The Company has access to substantial sources of funds at numerous banks worldwide. In September 2017, the Company secured a new 364-day Credit Facility. Total credit available to the Company approximates $\$ 10$ billion, which expires on September 13, 2018. Interest charged on borrowings under the credit line agreements is based on either bids provided by banks, the prime rate or London Interbank Offered Rates (Libor), plus applicable margins. Commitment fees under the agreements are not material... Throughout 2017, the Company continued to have access to liquidity through the commercial paper market. Short-term borrowings and the current portion of long-term debt amounted to approximately $\$ 3.9$ billion at the end of 2017 , of which $\$ 2.3$ billion was borrowed under the Commercial Paper Program, $\$ 1.5$ billion is the current portion of the long term debt, and the remainder principally represents local borrowing by international subsidiaries. ..."

Aggregate maturities of long-term obligations commencing in 2018 are (dollars in millions):

\begin{center}
\begin{tabular}{cccccc}
\hline
$\mathbf{2 0 1 8}$ & $\mathbf{2 0 1 9}$ & $\mathbf{2 0 2 0}$ & $\mathbf{2 0 2 1}$ & $\mathbf{2 0 2 2}$ & After 2022 \\
\hline
$\$ 1,499$ & 2,752 & 1,105 & 1,797 & 2,189 & 22,832 \\
\hline
\end{tabular}
\end{center}

Use the information in Exhibit 1 to answer the following questions:

\begin{enumerate}
  \item Why are the effective interest rates unchanged from 2016 to 2017 for most of the borrowings listed?
\end{enumerate}

Solution to 1:

The effective rate typically refers to the market rate at which the bonds are issued and typically does not change from year to year. 2. Why does the carrying amount of the "1.125\% Notes due 2019" remain the same in 2016 and $2017 ?$

\section{Solution to 2:}
The carrying amount of the "1.125\% Notes due 2019" remains the same because the effective interest rate at which the debentures were issued (1.13\%) is approximately the same as the coupon rate indicating that the notes were issued approximately at par. Thus, there would be no amortization of a premium or discount to affect the carrying amount of the notes, and assuming no repurchases, the carrying amount would not change.

\begin{enumerate}
  \setcounter{enumi}{2}
  \item Why is the carrying amount of the "4.75\% Notes due 2019 " higher in 2017 than in $2016 ?$
\end{enumerate}

\section{Solution to 3:}
The notes are denominated in Euros, with a face value of $€ 1$ billion. The dollar/euro translation exchange rate at the end of 2017 was higher than the exchange rate at the end of 2016 (1.1947 versus 1.0449). That increase explains part of the increase in carrying value. In addition, the effective interest rate of $5.83 \%$ is higher than the $4.75 \%$ coupon rate - implying that the notes were issued at a discount. Thus, the increase in the carrying amount of the notes also reflects the amortisation of the issuance discount.

In this reading, we focus on accounting for simple debt contracts. Debt contracts can take on additional features, which lead to more complexity. For instance, convertible debt and debt with warrants are more complex instruments that have both debt and equity features. Convertible debt gives the debt holder the option to exchange the debt for equity. Bonds issued with warrants give holders the right to purchase shares of the issuer's common stock at a specific price, similar to stock options. Issuance of bonds with warrants is more common by non-US companies. Example 9 provides an example of a financial statement disclosure of bonds with warrants.

\section{EXAMPLE 9}
\section{Financial Statement Disclosure of Bonds with Warrants}
\begin{enumerate}
  \item The following describes a company's issuance of convertible bonds with warrants.
\end{enumerate}

On 1 February 2018, the Company issued convertible bonds with stock warrants due 2024 with an aggregate principal amount of RMB 30 billion (the "Bonds with Warrants"). The Bonds with Warrants with fixed interest rate of $0.8 \%$ per annum and interest payable annually, were issued at par value of RMB 100. Each lot of the Bonds with Warrants, comprising ten Bonds with Warrants are entitled to warrants (the "Warrants") to subscribe 50.5 shares of the Company during the 5 trading days prior to 3 March 2020 at an initial exercise price of RMB 19.68 per share, subject to adjustment for, amongst other things, cash dividends, subdivision or consolidation of shares, bonus issues, rights issues, capital distribution, change of control and other events which have a dilutive effect on the issued share capital of the Company.

If all warrants were exercised, how many shares would be subscribed for?

\section{Solution:}
$1,515,000,000$ shares would be subscribed for [aggregate principal amount divided by par value of a lot times shares subscribed per lot $=($ RMB $30,000,000,000 /$ RMB $100 \times 10) \times 50.5$ shares].

In addition to disclosures in the notes to the financial statements, an MD\&A commonly provides other information about a company's capital resources, including debt financing and off-balance-sheet financing. In the MD\&A, management often provides a qualitative discussion on any material trends, favorable or unfavorable, in capital resources and indicates any expected material changes in their mix and relative cost. Additional quantitative information is typically provided, including schedules summarising a company's contractual obligations (e.g., bond payables) and other commitments (e.g., lines of credit and guarantees) in total and over the next five years.

\section{LEASES}
explain motivations for leasing assets instead of purchasing them explain the financial reporting of leases from a lessee's perspective explain the financial reporting of leases from a lessor's perspective

A lease is a contract that conveys the right to use an asset for a period of time in exchange for consideration. The party who uses the asset and pays the consideration is the lessee, and the party who owns the asset, grants the right to use the asset, and receives consideration is the lessor. For a contract to be a lease, it must

a. identify a specific underlying asset,

b. give the customer the right to obtain largely all of the economic benefits from the asset over the contract term, and

c. give the customer, not the supplier, the ability to direct how and for what objective the underlying asset is used.

For example, a contract between a customer and a trucking company is a lease if the contract identifies a specific truck, allows the customer exclusive use of it during the contract term, and lets the customer direct its use. However, if the customer contracts with a trucking company to ship goods for a fee, the contract would not be a lease, because a specific truck is not identified nor does the customer obtain largely all of the economic benefits from the truck over the contract term.

Leasing is a way to obtain the benefits of the asset without purchasing it outright. From the perspective of a lessee, it is a form of financing that resembles acquiring an asset with a note payable. From the perspective of a lessor, a lease is a form of investment and can also be an effective selling strategy, because customers generally prefer to pay in installments.

\section{Examples of Leases}
Leasing is among the most prevalent forms of financing. In 2014, the International Accounting Standards Board found that more than 14,000 publicly listed companies were lessees and they owed over $\$ 3.3$ trillion in future lease payments in aggregate. ${ }^{7}$ The following exhibit illustrates several examples of these arrangements.

Lessee

Alibaba

Copa Airlines

Lease Disclosure Excerpt

“The Company entered into operating lease agreements primarily for shops and malls, offices, warehouses, and land."

"The Company leases some aircraft under long-term lease agreements with an average duration of 10 years. Other leased assets include real estate, airport and terminal facilities, sales offices, maintenance facilities, and general offices."

Facebook "We have entered into various non-cancelable operating lease agreements for certain of our offices, data center, land, colocations, and equipment."

Standard Bank "The group leases various offices, branch space, and ATM space."

Sources: Companies' 2020 and 2019 annual reports.

Lessors are often banks, although there are some large independent leasing companies, such as AerCap Holdings N.V., which describes itself as "the global leader in aircraft leasing, .. . leasing aircraft to customers in every major geographical region. As of December 31, 2019, [it] owned 939 aircraft."

\section{Advantages of Leasing}
There are several advantages to leasing an asset compared with purchasing it:

\begin{itemize}
  \item Less cash is needed up front. Leases typically require little, if any, down payment.

  \item Cost effectiveness: Leases are a form of secured borrowing; in the event of non-payment, the lessor simply repossesses the leased asset. As a result, the effective interest rate for a lease is typically lower than what the lessee would pay on an unsecured loan or bond.

  \item Convenience and lower risks associated with asset ownership, such as obsolescence.

\end{itemize}

From the perspective of a lessor, leasing has advantages over selling outright, which include earning interest income over the lease term and increasing the market for its product by offering customers the ability to use or control an asset while paying smaller amounts over time.

\section{Lease Classification as Finance or Operating}
Leases can resemble either the purchase of an asset or a rental contract. For example, a 10-year lease of an automobile with a 10-year useful life for monthly payments that, in aggregate, are equal to the fair value of the automobile is effectively a debt-financed

7 IFRS, "IASB Shines Light on Leases by Bringing Them onto the Balance Sheet" (13 January 2016). www. \href{http://ifrs.org/news-and-events/2016/01/iasb-shines-light-on-leases-by-bringing-them-onto-the-balance-sheet}{ifrs.org/news-and-events/2016/01/iasb-shines-light-on-leases-by-bringing-them-onto-the-balance-sheet}. 8 AerCap Holdings N.V. annual report for the fiscal year ended 31 December 312019 on Form 20 -F. purchase of that automobile. In contrast, a 1-year lease of a machine with a 20-year useful life resembles a rental contract. A lease that resembles a purchase is classified as a finance lease. All other leases are operating leases.

More specifically, a lease is a finance lease if $a n y$ of the following five criteria are met. These criteria are the same for the International Financial Reporting Standards (IFRS) and US generally accepted accounting principles (GAAP). If none of the criteria are met, the lease is an operating lease. The same criteria are used for both lessees and lessors in classifying the lease.

\begin{enumerate}
  \item The lease transfers ownership of the underlying asset to the lessee.

  \item The lessee has an option to purchase the underlying asset and is reasonably certain it will do so.

  \item The lease term is for a major part of the asset's useful life.

  \item The present value of the sum of the lease payments equals or exceeds substantially all of the fair value of the asset.

  \item The underlying asset has no alternative use to the lessor.

\end{enumerate}

\section{EXAMPLE 10}
\section{Lease Identification and Classification}
\begin{enumerate}
  \item Company C enters a contract with Company $\mathrm{D}$ that requires Company $\mathrm{C}$ to pay 1100 million at the end of each of the next two years to Company D for exclusive use of a specific machine over that time period. The present value of the payments is $¥ 186$ million. At the end of the contract, Company $\mathrm{C}$ will return the machine to Company $D$. The contract does not contain a purchase option. The machine can be used in many applications by many types of customers. The remaining useful life of the machine is four years, and its fair value is $¥ 190$ million. This contract is:
\end{enumerate}

A. not a lease.

B. an operating lease.

C. a finance lease.

\section{Solution to 1:}
$\mathrm{C}$ is correct. This contract is a lease because a specific asset is identified, Company $\mathrm{C}$ will exclusively use it, and Company $\mathrm{C}$ will have the ability to direct its use. The contract is a finance lease because one of the five criteria is met: The present value of the lease payments equals substantially all of the fair value $(186 / 190=98 \%)$.

\begin{enumerate}
  \setcounter{enumi}{1}
  \item If the fair value of the machine in Question 1 was $¥ 300$ million, would the classification of the contract change?
\end{enumerate}

A. No

B. Yes, from an operating lease to a finance lease

C. Yes, from a finance lease to an operating lease

\section{Solution to 2:}
$\mathrm{C}$ is correct. This change would result in the lease not meeting any of the five criteria for a finance lease. If a lease does not meet any of the five criteria, it is an operating lease.

\section{Financial Reporting of Leases}
The financial reporting of a lease depends on whether the party is the lessee or lessor, whether the party reports with IFRS or US GAAP, and the classification of the lease as finance or operating. Additionally, for lessees, there are lease accounting exemptions for certain lease contracts: If its term is 12 months or less (IFRS and US GAAP) or it is for a "low-value asset," up to $\$ 5,000$ in sales price (IFRS only), then the lessee can elect to simply expense the lease payments on a straight-line basis. These exemptions are not available to lessors. The following diagram illustrates the different permutations for lease accounting.

\section{Lease Classifications for Lessee and Lessor:}
\begin{center}
\includegraphics[max width=\textwidth]{2023_05_04_b5cfa4f1bc883752f121g-483}
\end{center}

Fortunately, lessor accounting under both IFRS and US GAAP is substantially identical, and the differences in treatment for lessees are modest.

\section{Lessee Accounting -IFRS}
Under IFRS, there is a single accounting model for both finance and operating leases for lessees. At lease inception, the lessee records a lease payable liability and a "right-of-use" (ROU) asset on its balance sheet, both equal to the present value of future lease payments. The discount rate used in the present value calculation is either the rate implicit in the lease or an estimated secured borrowing rate.

The lease liability is subsequently reduced by each lease payment using the effective interest method. Each lease payment is composed of interest expense, which is the product of the lease liability and the discount rate, and principal repayment, which is the difference between the interest expense and lease payment.

The ROU asset is subsequently amortized, often on a straight-line basis, over the lease term. So, although the lease liability and ROU asset begin with the same carrying value on the balance sheet, they typically diverge in subsequent periods because the principal repayment that reduces the lease liability and the amortization expense that reduces the ROU asset are calculated differently.

The following list shows how the transaction affects the financial statements:

\begin{enumerate}
  \item The lease liability net of principal repayments and the ROU asset net of accumulated amortization are reported on the balance sheet.

  \item Interest expense on the lease liability and the amortization expense related to the ROU asset are reported separately on the income statement. 3. The principal repayment component of the lease payment is reported as a cash outflow under financing activities on the statement of cash flows, and depending on the lessee's reporting policies, interest expense is reported under either operating or financing activities on the statement of cash flows.

\end{enumerate}

\section{EXAMPLE 11}
\section{Lessee Accounting-IFRS}
Proton Enterprises, a hypothetical manufacturer based in Germany, is offered the following terms to lease a machine: five-year lease with an implied interest rate of $10 \%$ and an annual lease payment of EUR100,000 per year payable at the end of each year. The present value of the machinery is therefore EUR379,079 (in Microsoft Excel, the formula is $=\mathrm{PV}(10 \%, 5,-100,000)$. The asset will be amortized over the five-year lease term on a straight-line basis. Proton reports under IFRS.

What would be the impact of this lease on Proton's:

\begin{enumerate}
  \item balance sheet at the beginning of the year?
\end{enumerate}

\section{Solution to 1:}
Proton would report a EUR379,079 lease liability and ROU asset.

\begin{enumerate}
  \setcounter{enumi}{1}
  \item income statement during the following year?
\end{enumerate}

\section{Solution to 2:}
Interest expense and amortization expense are reported on the income statement. In Year 2, interest expense is EUR31,699 and amortization expense is EUR 75,816, as illustrated in the following tables:

\begin{center}
\begin{tabular}{|c|c|c|c|c|}
\hline
 & $\begin{array}{c}\text { Lease } \\ \text { Payment }\end{array}$ & $\begin{array}{c}\text { Interest Expense } \\ \text { (10\% × Lease } \\ \text { Liability) }\end{array}$ & $\begin{array}{l}\text { Principal Repayment } \\ \text { (Payment - Interest) }\end{array}$ & $\begin{array}{l}\text { Lease } \\ \text { Liability }\end{array}$ \\
\hline
 & FO.1 & FO. 2 & FO.3 & FO.4 \\
\hline
Year 0 &  &  &  & 379,079 \\
\hline
Year 1 & 100,000 & 37,908 & 62,092 & 316,987 \\
\hline
Year 2 & 100,000 & 31,699 & 68,301 & 248,685 \\
\hline
Year 3 & 100,000 & 24,869 & 75,131 & 173,554 \\
\hline
Year 4 & 100,000 & 17,355 & 82,645 & 90,909 \\
\hline
Year 5 & 100,000 & 9,091 & 90,909 & 0 \\
\hline
Total & 500,000 & 120,921 & 379,079 &  \\
\hline
\end{tabular}
\end{center}

\begin{center}
\begin{tabular}{lcc}
\hline
 & Amortization Expense & ROU Asset \\
\hline
Straight-Line F.1 & F.2 &  \\
\hline
Year 0 &  & 379,079 \\
Year 1 & 75,816 & 303,263 \\
Year 2 & 75,816 & 227,447 \\
Year 3 & 75,816 & 151,631 \\
Year 4 & 75,816 & 75,816 \\
\end{tabular}
\end{center}

\begin{center}
\begin{tabular}{lcc}
\hline
 & Amortization Expense & ROU Asset \\
\hline
Year 5 & 75,816 & 0 \\
\hline
Total & 379,079 &  \\
\hline
\end{tabular}
\end{center}

Note: Totals may not sum due to rounding.

\begin{enumerate}
  \setcounter{enumi}{2}
  \item statement of cash flows during the following year?
\end{enumerate}

\section{Solution to 3:}
Principal repayments are reported as a cash outflow under financing activities on the statement of cash flows, and depending on Proton's reporting policies, interest expense is reported under operating or financing activities on the statement of cash flows. From the previous tables, Year 2 principal repayment is EUR68,301 and interest expense is EUR31,699, for a total of EUR100,000.

\section{Lessee Accounting-US GAAP}
Under US GAAP, there are two accounting models for lessees: one for finance leases and another for operating leases. The finance lease accounting model is identical to the lessee accounting model for IFRS. The operating lease accounting model is different.

At operating lease inception, the lessee records a lease payable liability and a corresponding right-of-use asset on its balance sheet that are subsequently reduced by the principal repayment component of the lease payment and amortization, respectively, in the same manner that an IFRS lessee would.

The key difference between an operating lease and a finance lease is how the amortization of the ROU asset is calculated. For an operating lease, the lessee's ROU asset amortization expense is the lease payment minus the interest expense. The implication is that the total expense reported on the income statement (interest plus amortization) will equal the lease payment and that the lease liability and the $\mathrm{ROU}$ asset will always equal each other because the principal repayment and amortization are calculated in an identical manner.

The following list shows how the transaction appears on the financial statements:

a. The lease liability net of principal repayments and the ROU asset net of accumulated amortization are reported on the balance sheet.

b. Interest expense on the lease liability and the amortization expense related to the ROU asset are reported as a single line titled "lease expense" as an operating expense on the income statement. The interest and amortization components are not reported separately, nor are they grouped with other types of interest and amortization expense (e.g., interest on a bond, amortization of an intangible asset).

c. The entire lease payment is reported as a cash outflow under operating activities on the statement of cash flows. The interest and principal repayment components are not reported separately.

\section{EXAMPLE 12}
\section{Lessee Accounting - Operating Lease under US GAAP}
If Proton Enterprises classified the lease of the machinery from Example 11 as an operating lease:

\begin{enumerate}
  \item how would its financial statements differ, if at all?
\end{enumerate}

\section{Solution to 1:}
The first step is to construct the lease liability and ROU asset amortization tables under an operating lease scenario. The lease liability amortization is the same as the finance lease columns FO.1-FO.4 in Example 11.

\begin{center}
\begin{tabular}{lccc}
\hline
 & Amortization Expense & ROU Asset & Lease Expense \\
\hline
(Lease Payment & (Amortization + &  &  \\
 & - Interest) & Interest) &  \\
 & 0.1 & 0.2 & 0.3 \\
\hline
Year 0 & 379,079 & 100,000 &  \\
Year 1 & 62,092 & 316,987 & 100,000 \\
Year 2 & 68,301 & 248,685 & 100,000 \\
Year 3 & 75,131 & 173,554 & 100,000 \\
Year 4 & 82,645 & 90,909 & 100,000 \\
Year 5 & 90,909 & 0 & 500,000 \\
\hline
Total & 379,078 &  &  \\
\hline
\end{tabular}
\end{center}

Now we can compare the financial statement impacts under both finance and operating lease scenarios.

Balance Sheet $\quad$ Year $1 \quad$ Year $2 \quad$ Year $3 \quad$ Year $4 \quad$ Year 5

Finance lease:

ROU asset, net: F.2 $2 \quad 303,263 \quad 227,447 \quad 151,631 \quad 75,816 \quad 0$

$\begin{array}{llllll}\text { Lease liability, net: } \quad 316,987 \quad 248,685 & 173,554 & 90,909 & 0\end{array}$

FO.4

Operating lease:

ROU asset, net: $\mathrm{O} .2 \quad 316,987 \quad 248,685 \quad 173,554 \quad 90,909 \quad 0$

$\begin{array}{llllll}\text { Lease liability, net: } \quad 316,987 \quad 248,685 \quad 173,554 & 90,909 & 0\end{array}$

FO.4

The ROU asset is lower in each period under a finance lease because the amortization expense is higher.

\begin{center}
\begin{tabular}{lccccc}
\hline
Income Statement & Year 1 & Year 2 & Year 3 & Year 4 & Year 5 \\
\hline
Finance lease: &  &  &  &  &  \\
Amortization: F.1 & 75,816 & 75,816 & 75,816 & 75,816 & 75,816 \\
Interest: FO.2 & 37,908 & 31,699 & 24,869 & 17,355 & 9,091 \\
\cline { 2 - 6 }
$\quad$ Total & 113,724 & 107,515 & 100,685 & 93,171 & 84,907 \\
Operating lease: &  &  &  &  &  \\
Lease expense: O.3 & 100,000 & 100,000 & 100,000 & 100,000 & 100,000 \\
\hline
\end{tabular}
\end{center}

Total expense is higher for a finance lease in Years 1-3 but lower in Years 4 and 5. The largest difference is classification; amortization and interest are presented separately for a finance lease, whereas operating lease expense is an operating expense.

\begin{center}
\begin{tabular}{|c|c|c|c|c|c|}
\hline
$\begin{array}{l}\text { Statement of Cash } \\ \text { Flows }\end{array}$ & Year 1 & Year 2 & Year 3 & Year 4 & Year 5 \\
\hline
\multicolumn{6}{|l|}{Finance lease:} \\
\hline
$\begin{array}{l}\text { Cash flow from } \\ \text { operating activities }\end{array}$ & $(37,908)$ & $(31,699)$ & $(24,869)$ & $(17,355)$ & $(9,091)$ \\
\hline
$\begin{array}{l}\text { Cash flow from } \\ \text { financing activities }\end{array}$ & $(62,902)$ & $(68,301)$ & $(75,131)$ & $(82,645)$ & $(90,909)$ \\
\hline
Total & $(100,000)$ & $(100,000)$ & $(100,000)$ & $(100,000)$ & $(100,000)$ \\
\hline
\multicolumn{6}{|l|}{Operating lease:} \\
\hline
$\begin{array}{l}\text { Cash flows from } \\ \text { operating activities }\end{array}$ & $(100,000)$ & $(100,000)$ & $(100,000)$ & $(100,000)$ & $(100,000)$ \\
\hline
\end{tabular}
\end{center}

The difference on the statement of cash flows is only in classification, because in both cases the total cash outflow is equal to the lease payment.

\begin{enumerate}
  \setcounter{enumi}{1}
  \item how would the classification, all else equal, affect EBITDA margin, total asset turnover, and cash flow per share?
\end{enumerate}

\section{Solution to 2:}
The following table shows how the classification affects the indicated financial ratios.

\begin{center}
\begin{tabular}{|c|c|c|}
\hline
Ratio & Formula & $\begin{array}{c}\text { Impact of Using an Operating Lease } \\ \text { Instead of a Finance Lease }\end{array}$ \\
\hline
EBITDA margin & $\frac{\text { EBITDA }}{\text { Total revenues }}$ & $\begin{array}{l}\text { Lower: Lease expense is classified } \\ \text { as an operating expense rather than } \\ \text { interest and amortization. }\end{array}$ \\
\hline
Asset turnover & $\frac{\text { Total revenues }}{\text { Total assets }}$ & $\begin{array}{l}\text { Lower: Total assets are higher under } \\ \text { an operating lease because the ROU } \\ \text { asset is amortized at a slower pace in } \\ \text { Years } 1-3 \text {. }\end{array}$ \\
\hline
$\begin{array}{l}\text { Cash flow per } \\ \text { share }\end{array}$ & $\frac{\text { Cash flow from operations }}{\text { Shares outstanding }}$ & $\begin{array}{l}\text { Lower: Cash flow from operations } \\ \text { is lower because the entire lease } \\ \text { payment is included in operating } \\ \text { activities versus solely interest expense } \\ \text { for a finance lease. }\end{array}$ \\
\hline
\end{tabular}
\end{center}

\section{Lessor Accounting}
The accounting for lessors is substantially identical under IFRS and US GAAP. Under both accounting standards, lessors classify leases as finance or operating leases, which determines the financial reporting. Although lessors under US GAAP recognize finance leases as either "sales-type" or "direct financing," the distinction is immaterial from an analyst's perspective. At finance lease inception, the lessor recognizes a lease receivable asset equal to the present value of future lease payments and de-recognizes the leased asset, simultaneously recognizing any difference as a gain or loss. The discount rate used in the present value calculation is the rate implicit in the lease.

The lease receivable is subsequently reduced by each lease payment using the effective interest method. Each lease payment is composed of interest income, which is the product of the lease receivable and the discount rate, and principal proceeds, which equals the difference between the interest income and cash receipt.

The following list shows how the transaction affects the financial statements:

\begin{enumerate}
  \item Lease receivable net of principal proceeds is reported on the balance sheet.

  \item Interest income is reported on the income statement. If leasing is a primary business activity for the entity, as it commonly is for financial institutions and independent leasing companies, it is reported as revenue.

  \item The entire cash receipt is reported under operating activities on the statement of cash flows.

\end{enumerate}

The accounting treatment for an operating lease is different in that, because the contract is essentially a rental agreement, the lessor keeps the leased asset on its books and recognizes lease revenue on a straight-line basis. Interest revenue is not recognized because the transaction is not considered a financing.

The following list shows how the transaction affects the financial statements:

\begin{enumerate}
  \item The balance sheet is not affected. The lessor continues to recognize the leased asset at cost net of accumulated depreciation.

  \item Lease revenue is recognized on a straight-line basis on the income statement. Depreciation expense continues to be recognized.

  \item The entire cash receipt is reported under operating activities on the statement of cash flows. This is the same as a finance lease.

\end{enumerate}

\section{EXAMPLE 13}
\section{Lessor Accounting}
\begin{enumerate}
  \item Let's examine Proton's machine lease from Example 11 and Example 12 from the perspective of the lessor. Assume that the carrying value of the asset immediately prior to the lease is EUR350,000, accumulated depreciation is zero, and the lessor elects to depreciate it on a straight-line basis over five years.
\end{enumerate}

How are the lessor's financial statements affected by the classification of the lease as a finance or operating lease?

\section{Solution}
The difference on the balance sheet is material, because a finance lease requires the lessor to de-recognize the asset and recognize a lease receivable, whereas an operating lease lessor continues to recognize the asset and depreciate it over its useful life. In this case, where the present value of the lease payments is well above the carrying value of the asset, the finance lease classification results in a significant increase in assets.

\begin{center}
\begin{tabular}{lccccc}
\hline
Balance Sheet & Year 1 & Year 2 & Year 3 & Year 4 & Year 5 \\
\hline
Finance lease: &  &  &  &  &  \\
Lease receivable, net & 316,987 & 248,685 & 173,554 & 90,909 & 0 \\
Operating lease: &  &  &  &  &  \\
\end{tabular}
\end{center}

The difference on the income statement is also material, because a finance lease lessor recognizes interest revenue under the effective interest method whereas the operating lease lessor recognizes straight-line lease revenue.

\begin{center}
\begin{tabular}{lccccc}
\hline
Income Statement & Year 1 & Year 2 & Year 3 & Year 4 & Year 5 \\
\hline
Finance lease: &  &  &  &  &  \\
Interest revenue & 37,908 & 31,699 & 24,869 & 17,355 & 9,091 \\
Operating lease: &  &  &  &  &  \\
Lease revenue & 100,000 & 100,000 & 100,000 & 100,000 & 100,000 \\
\hline
\end{tabular}
\end{center}

The statement of cash flows, however, is no different for the lessor under a finance or operating lease: The entire cash inflow from the lease payment is recognized under operating activities.

Statement of Cash Flows Year $1 \quad$ Year $2 \quad$ Year $3 \quad$ Year $4 \quad$ Year 5

Finance lease:

Cash flows from operating

$\begin{array}{llllll}\text { activities } & 100,000 & 100,000 & 100,000 & 100,000 & 100,000\end{array}$

Operating lease:

Cash flows from operating

$\begin{array}{llllll}\text { activities } & 100,000 & 100,000 & 100,000 & 100,000 & 100,000\end{array}$

\section{INTRODUCTION TO PENSIONS AND OTHER POST-EMPLOYMENT BENEFITS}
compare the presentation and disclosure of defined contribution and defined benefit pension plans

Pensions and other post-employment benefits give rise to non-current liabilities reported by many companies. Companies may offer various types of benefits to their employees following retirement, such as pension plans, health care plans, medical insurance, and life insurance. Pension plans often are the most significant post-employment benefits provided to retired employees.

The accounting and reporting for pension plans depends on the type of pension plan offered. Two common types of pension plans are defined contribution pension plans and defined benefit pension plans. Under a defined-contribution plan, a company contributes an agreed-upon (defined) amount into the plan. The agreed-upon amount is the pension expense. The amount the company contributes to the plan is treated as an operating cash outflow. The only impact on assets and liabilities is a decrease in cash, although if some portion of the agreed-upon amount has not been paid by fiscal year-end, a liability would be recognised on the balance sheet. Because the amount of the contribution is defined and the company has no further obligation once the contribution has been made, accounting for a defined-contribution plan is fairly straightforward.

Accounting for a defined-benefit plan is more complicated. Under a defined-benefit plan, a company makes promises of future benefits to be paid to the employee during retirement. For example, a company could promise an employee annual pension payments equal to 70 percent of his final salary at retirement until death. Estimating the eventual amount of the obligation arising from that promise requires the company to make many assumptions, such as the employee's expected salary at retirement and the number of years the employee is expected to live beyond retirement. The company estimates the future amounts to be paid and discounts the future estimated amounts to a present value (using a rate reflective of a high-quality corporate bond yield) to determine the pension obligation. The discount rate used to determine the pension obligation significantly affects the amount of the pension obligation. The pension obligation is allocated over the employee's employment as part of pension expense.

Most defined-benefit pension plans are funded through a separate legal entity, typically a pension trust fund. A company makes payments into the pension fund, and retirees are paid from the fund. The payments that a company makes into the fund are invested until they are needed to pay the retirees. If the fair value of the fund's assets is higher than the present value of the estimated pension obligation, the plan has a surplus and the company's balance sheet will reflect a net pension asset. ${ }^{9}$ Conversely, if the present value of the estimated pension obligation exceeds the fund's assets, the plan has a deficit and the company's balance sheet will reflect a net pension liability. ${ }^{10}$ Thus, a company reports either a net pension asset or a net pension liability. Each period, the change in the net pension asset or liability is recognised either in profit or loss or in other comprehensive income.

Under IFRS, the change in the net pension asset or liability each period is viewed as having three general components. Two of the components of this change are recognised as pension expense in profit and loss: (1) employees' service costs, and (2) the net interest expense or income accrued on the beginning net pension asset or liability. The service cost during the period for an employee is the present value of the increase in the pension benefit earned by the employee as a result of providing one more year of service. The service cost also includes past service costs, which are changes in the present value of the estimated pension obligation related to employees' service in prior periods, such as might arise from changes in the plan. The net interest expense or income represents the change in value of the net defined benefit pension asset or liability and is calculated as the net pension asset or liability multiplied by the discount rate used in estimating the present value of the pension obligation. The third component of the change in the net pension asset or liability during a period - "remeasurements" - is recognised in other comprehensive income. Remeasurements are not amortised into profit or loss over time.

9 The amount of any reported net pension asset is capped the amount of any expected future economic benefits to the company from the plan; this cap is referred to as the asset ceiling.

10 The description of accounting for pensions presented in this reading corresponds to the June 2011 version of IAS 19 Employee Benefits, which took effect on 1 January 2013. Both IFRS and US GAAP require companies to present the amount of net pension liability or asset on the balance sheet. Remeasurements include (a) actuarial gains and losses and (b) the actual return on plan assets less any return included in the net interest expense or income. Actuarial gains and losses can occur when changes are made to the assumptions on which a company bases its estimated pension obligation (e.g., employee turnover, mortality rates, retirement ages, compensation increases). The actual return on plan assets includes interest, dividends and other income derived from the plan assets, including realized and unrealized gains or losses. The actual return typically differs from the amount included in the net interest expense or income, which is calculated using a rate reflective of a high-quality corporate bond yield; plan assets are typically allocated across various asset classes, including equity as well as bonds.

Under US GAAP, the change in net pension asset or liability each period is viewed as having five components, some of which are recognised in profit and loss in the period incurred and some of which are recognised in other comprehensive income and amortised into profit and loss over time. The three components recognised in profit and loss in the period incurred are (1) employees' service costs for the period, (2) interest expense accrued on the beginning pension obligation, and (3) expected return on plan assets, which is a reduction in the amount of expense recognised. The other two components are past service costs and actuarial gains and losses. Past service costs are recognised in other comprehensive income in the period in which they arise and then subsequently amortised into pension expense over the future service period of the employees covered by the plan. Actuarial gains and losses are typically also recognised in other comprehensive income in the period in which they occur and then amortised into pension expense over time. In effect, this treatment allows companies to "smooth" the effects on pension expense over time for these latter two components. US GAAP does permit companies to immediately recognize actuarial gains and losses in profit and loss.

Similar to other forms of employee compensation for a manufacturing company, the pension expense related to production employees is added to inventory and expensed through cost of sales (cost of goods sold). For employees not involved directly in the production process, the pension expense is included with salaries and other administrative expenses. Therefore, pension expense is not directly reported on the income statement. Rather, extensive disclosures are included in the notes to the financial statements.

Example 14 presents excerpts from the balance sheet and pension-related disclosures in BT Group plc's Annual Report for the year ended 31 March 2018.

\section{EXAMPLE 14}
\section{BT Group plc: Excerpt from Balance Sheets}
Below is an excerpt of BT Group plc's balance sheet from the annual report for the year ended 31 March 2018. BT reports under IFRS.

\begin{center}
\begin{tabular}{lccc}
\hline
Non-current liabilities, GBP million & $\begin{array}{c}\text { Mar. 31, } \\ \mathbf{2 0 1 8}\end{array}$ & $\begin{array}{c}\text { Mar. 31, } \\ \mathbf{2 0 1 7}\end{array}$ & $\begin{array}{c}\text { Mar. 31, } \\ \mathbf{2 0 1 6}\end{array}$ \\
\hline
Loans and other borrowings & 11,994 & 10,081 & 11,025 \\
Derivative financial instruments & 787 & 869 & 863 \\
Retirement benefit obligations & 6,371 & 9,088 & 6,382 \\
Other payables & 1,326 & 1,298 & 1,106 \\
Deferred tax liabilities & 1,340 & 1,240 & 1,262 \\
Provisions & 452 & 536 & 565 \\
Non-current liabilities & 22,270 & 23,112 & 21,203 \\
\hline
\end{tabular}
\end{center}

\section{Pension-Related Disclosures}
The following are excerpts of pension-related disclosures from BT Group plc's 2018 Annual Report.

\section{Extract from Note 3 "Summary of Significant Accounting Policies"}
\section{Retirement benefits}
The group's net obligation in respect of defined benefit pension plans is the present value of the defined benefit obligation less the fair value of the plan assets.

The calculation of the obligation is performed by a qualified actuary using the projected unit credit method and key actuarial assumptions at the balance sheet date.

The income statement expense is allocated between an operating charge and net finance income or expense. The operating charge reflects the increase in the defined benefit obligation resulting from the pension benefit earned by active employees in the current period, the costs of administering the plans and any past service costs/credits such as those arising from curtailments or settlements. The net finance income or expense reflects the interest on the net retirement benefit obligations recognised in the group balance sheet, based on the discount rate at the start of the year. Actuarial gains and losses are recognised in full in the period in which they occur and are presented in the group statement of comprehensive income.

The group also operates defined contribution pension plans and the income statement expense represents the contributions payable for the year.

\section{Extract from Note 20 "Retirement Benefit Plans"}
 Information on Defined Benefit Pension Plans\begin{center}
\begin{tabular}{llll}
\hline
$\mathbf{\pm m}$ & $\mathbf{2 0 1 8}$ & $\mathbf{2 0 1 7}$ & $\mathbf{2 0 1 6}$ \\
\hline
Present value of liabilities & $\mathbf{5 7 , 3 2 7}$ & 60,200 & 50,350 \\
Fair value of plan assets & $\mathbf{5 0 , 9 5 6}$ & 51,112 & 43,968 \\
\hline
\end{tabular}
\end{center}

Use information in the excerpts to answer the following questions:

\begin{enumerate}
  \item What type(s) of pension plans does BT have?
\end{enumerate}

\section{Solution to 1:}
Note 3 "Summary of Significant Accounting Policies" indicates that the company has both defined contribution and defined benefit pension plans.

\begin{enumerate}
  \setcounter{enumi}{1}
  \item What proportion of BT's total non-current liabilities are related to its retirement benefit obligations?
\end{enumerate}

\section{Solution to 2:}
Retirement benefit obligations represent $29 \%, 39 \%$, and 30\% of BT's total non-current liabilities for the years 2018, 2017, and 2016. Using 2018 to illustrate, $\pounds 6,371 / \pounds 22,270=29 \%$. $(\pounds$ million $)$ 3. Describe how BT's retirement benefit obligation is calculated.

\section{Solution to 3:}
Note 3 "Summary of Significant Accounting Policies" indicates that BT's Retirement benefit obligation is calculated as the present value of the defined benefit obligation minus the fair value of the plan assets.

Using data from Note 20 "Retirement Benefit Plans" the retirement benefit obligation for each year can be calculated. Using 2018 to illustrate, £57,327 $\pounds 50,956=\pounds 6,371$ ( $\pounds$ million).

\section{EVALUATING SOLVENCY: LEVERAGE AND COVERAGE RATIOS}
calculate and interpret leverage and coverage ratios

Solvency refers to a company's ability to meet its long-term debt obligations, including both principal and interest payments. In evaluating a company's solvency, ratio analyses can provide information about the relative amount of debt in the company's capital structure and the adequacy of earnings and cash flow to cover interest expense and other fixed charges (such as lease or rental payments) as they come due. Ratios are useful to evaluate a company's performance over time compared to the performance of other companies and industry norms. Ratio analysis has the advantage of allowing the comparison of companies regardless of their size and reporting currency.

The two primary types of solvency ratios are leverage ratios and coverage ratios. Leverage ratios focus on the balance sheet and measure the extent to which a company uses liabilities rather than equity to finance its assets. Coverage ratios focus on the income statement and cash flows and measure the ability of a company to cover its debt-related payments.

Exhibit 2 describes the two types of commonly used solvency ratios. The first three leverage ratios use total debt in the numerator. ${ }^{11}$ The debt-to-assets ratio expresses the percentage of total assets financed with debt. Generally, the higher the ratio, the higher the financial risk and thus the weaker the solvency. The debt-to-capitalratio measures the percentage of a company's total capital (debt plus equity) financed through debt. The debt-to-equity ratio measures the amount of debt financing relative to equity financing. A debt-to-equity ratio of 1.0 indicates equal amounts of debt and equity, which is the same as a debt-to-capital ratio of 50 percent. Interpretations of these ratios are similar. Higher debt-to-capital or debt-to-equity ratios imply weaker solvency. A caveat must be made when comparing debt ratios of companies in different countries. Within certain countries, companies historically have obtained more capital from debt than equity financing, so debt ratios tend to be higher for companies in these countries.

11 For calculations in this reading, total debt is the sum of interest-bearing short-term and long-term debt, excluding non-interest-bearing liabilities, such as accrued expenses, accounts payable, and deferred income taxes. This definition of total debt differs from other definitions that are more inclusive (e.g., all liabilities) or more restrictive (e.g., long-term debt only). If the use of different definitions of total debt materially changes conclusions about a company's solvency, the reasons for the discrepancies should be further investigated.

\section{Exhibit 2: Definitions of Commonly Used Solvency Ratios}
\begin{center}
\begin{tabular}{|c|c|c|}
\hline
Solvency Ratios & Numerator & Denominator \\
\hline
\multicolumn{3}{|l|}{Leverage ratios} \\
\hline
Debt-to-assets ratio & Total debt ${ }^{\mathrm{a}}$ & Total assets \\
\hline
Debt-to-capital ratio & Total debt ${ }^{\mathrm{a}}$ & $\begin{array}{l}\text { Total debt }{ }^{\mathrm{a}}+\text { Total shareholders } \\ \text { equity }\end{array}$ \\
\hline
Debt-to-equity ratio & Total debt ${ }^{\mathrm{a}}$ & Total shareholders' equity \\
\hline
Financial leverage ratio & Average total assets & Average shareholders' equity \\
\hline
\multicolumn{3}{|l|}{Coverage ratios} \\
\hline
Interest coverage ratio & EBIT $^{b}$ & Interest payments \\
\hline
Fixed charge coverage ratio & $\begin{array}{l}\mathrm{EBIT}^{\mathrm{b}}+\text { lease } \\ \text { payments }\end{array}$ & $\begin{array}{l}\text { Interest payments + lease } \\ \text { payments }\end{array}$ \\
\hline
\end{tabular}
\end{center}

${ }^{\text {a }}$ In this reading, debt is defined as the sum of interest-bearing short-term and long-term debt.

${ }^{\mathrm{b}}$ EBIT is earnings before interest and taxes.

The financial leverage ratio (also called the 'leverage ratio' or 'equity multiplier') measures the amount of total assets supported by one money unit of equity. For example, a value of 4 for this ratio means that each $€ 1$ of equity supports $€ 4$ of total assets. The higher the financial leverage ratio, the more leveraged the company in the sense of using debt and other liabilities to finance assets. This ratio often is defined in terms of average total assets and average total equity and plays an important role in the DuPont decomposition of return on equity. ${ }^{12}$

The interest coverage ratio measures the number of times a company's EBIT could cover its interest payments. A higher interest coverage ratio indicates stronger solvency, offering greater assurance that the company can service its debt from operating earnings. The fixed charge coverage ratio relates fixed financing charges, or obligations, to the cash flow generated by the company. It measures the number of times a company's earnings (before interest, taxes, and lease payments) can cover the company's interest and lease payments.

Example 15 demonstrates the use of solvency ratios in evaluating the creditworthiness of a company.

\section{EXAMPLE 15}
\section{Evaluating Solvency Ratios}
A credit analyst is evaluating and comparing the solvency of two companies-BT Group plc (BT) and Telefonica S A (Telefonica). The following data are gathered from the companies' fiscal 2017 annual reports (line item titles may vary between the two companies):

\begin{center}
\begin{tabular}{|c|c|c|c|c|}
\hline
 & \multicolumn{2}{|c|}{$\begin{array}{l}\text { BT Group plc } \\
\text { ( } \pounds \text { millions) }\end{array}$} & \multicolumn{2}{|c|}{$\begin{array}{c}\text { Telefonica S A } \\
\text { (€ millions) }\end{array}$} \\
\hline
 & 31-Mar-18 & 31-Mar-17 & 31-Dec-17 & 31-Dec-16 \\
\hline
Short-term borrowings & 2,281 & 2,632 & 9,414 & 14,749 \\
\hline
\end{tabular}
\end{center}

12 The basic DuPont decomposition is: Return on Equity = Net income/Average shareholders' equity = (Sales/Average total assets) $\times($ Net income/Sales $) \times$ (Average total assets/Average shareholders' equity).

\begin{center}
\begin{tabular}{lcccc}
\hline
 & \multicolumn{2}{c}{$\begin{array}{c}\text { BT Group plc } \\
\text { (\pounds millions) }\end{array}$} & \multicolumn{2}{c}{$\begin{array}{c}\text { Telefonica S A } \\
\text { ( } \boldsymbol{\epsilon} \text { millions) }\end{array}$} \\
\hline
 & 31-Mar-18 & 31-Mar-17 & 31-Dec-17 & 31-Dec-16 \\
\hline
Long-term debt & 11,994 & 10,081 & 46,332 & 45,612 \\
Total shareholders' equity & 10,304 & 8,335 & 26,618 & 28,385 \\
Total assets & 42,759 & 42,372 & 115,066 & 123,641 \\
EBIT* & 3,381 & 3,167 & 6,791 & 5,469 \\
Interest expense & 776 & 817 & 3,363 & 4,476 \\
\hline
\end{tabular}
\end{center}

\begin{itemize}
  \item Operating profit (or operating income) is used as a proxy for EBIT for both companies.
\end{itemize}

Use the above information to answer the following questions:

\begin{enumerate}
  \item With regard to leverage ratios of $\mathrm{BT}$ and Telefonica:
\end{enumerate}

A. What are each company's debt-to-assets, debt-to-capital, and debt-to-equity ratios for 2017 and 2016?

B. Comment on any changes in the calculated leverage ratios from year-to-year for each company.

C. Comment on the calculated leverage ratios of BT Group plc compared to Telefonica SA.

\section{Solution to 1:}
A. The debt-to-assets, debt-to-capital, and debt-to-equity ratios are as follows, with supporting calculations from each company's most recent year demonstrated below.

\begin{center}
\begin{tabular}{lcccc}
\hline
 & \multicolumn{2}{c}{BT Group plc} & \multicolumn{2}{c}{Telefonica S A} \\
\hline
 & 31-Mar-18 & 31-Mar-17 & 31-Dec-17 & 31-Dec-16 \\
\hline
Debt-to-assets & $33.4 \%$ & $30.0 \%$ & $48.4 \%$ & $48.8 \%$ \\
Debt-to-capital & $58.1 \%$ & $60.4 \%$ & $67.7 \%$ & $68.0 \%$ \\
Debt-to-equity & 1.39 & 1.53 & 2.09 & 2.13 \\
\hline
\end{tabular}
\end{center}

\begin{center}
\begin{tabular}{|c|c|c|}
\hline
 & BT Group plc & Telefonica S A \\
\hline
 & 31-Mar-18 & 31-Dec-17 \\
\hline
Debt-to-assets & $\begin{array}{c}33.4 \%= \\ (2,281+11,994) / 42,759\end{array}$ & $\begin{array}{c}48.4 \%= \\ (9,414+46,332) / 115,066\end{array}$ \\
\hline
Debt-to-capital & $\begin{aligned} & 58.1 \%= \\ &(2,281+11,994) /(2,281+11,994+10,304)\end{aligned}$ & $\begin{array}{c}67.7 \%= \\ (9,414+46,332) /(9,414+ \\ 46,332+26,618)\end{array}$ \\
\hline
Debt-to-equity & $\begin{array}{c}1.39= \\ (2,281+11,994) / 10,304\end{array}$ & $\begin{array}{c}2.09= \\ (9,414+46,332) / 26,618\end{array}$ \\
\hline
\end{tabular}
\end{center}

B. BT's debt-to-assets ratio increased, while its debt-to-capital and debt-to-equity ratios both decreased. The decrease in BT's debt-to-capital and debt-to-equity ratios resulted primarily from the company's increase in total equity and indicate stronger solvency. In addition, we observe that BT decreased its short-term borrowings and increased its long-term debt. Telefonica's leverage ratios appear fairly similar, albeit slightly lower, for 2017 compared to 2016. Similar to BT, it appears that Telefonica shifted away from short borrowings to long-term debt in 2017.

C. In both years, all three of BT's leverage ratios were lower than Telefonica's. Based on these ratios, this may imply higher solvency of BT relative to Telefonica.

\begin{enumerate}
  \setcounter{enumi}{1}
  \item With regard to coverage ratios of BT and Telefonica:
\end{enumerate}

A. What is each company's interest coverage ratio for 2017 and 2016 ?

B. Comment on any changes in the interest coverage ratio from year to year for each company.

C. Comment on the interest coverage ratio of BT Group plc compared to Telefonica SA.

Solution to 2:

A. The interest coverage ratios are as follows, with supporting calculations from each company's most recent year demonstrated below.

\begin{center}
\begin{tabular}{lcccc}
\hline
 & \multicolumn{2}{c}{BT Group plc} & \multicolumn{2}{c}{Telefonica S A} \\
\hline
 & 31-Mar-18 & 31-Mar-17 & 31-Dec-17 & 31-Dec-16 \\
\hline
Interest coverage ratio & 4.36 & 3.88 & 2.02 & 1.22 \\
\hline
 &  &  &  &  \\
\hline
BT Group plc & Telefonica S A &  &  &  \\
\hline
Interest coverage ratio & 31-Mar-18 & 31-Dec-17 &  &  \\
\hline
$4.36=3,381 / 776$ & $2.02=6,791 / 3,363$ &  &  &  \\
\hline
\end{tabular}
\end{center}

B. Both companies' interest coverage ratios increased from 2017 to 2018, indicating an improvement in solvency, consistent with the conclusions drawn from the companies' ratios in question 1. Both companies have sufficient operating earnings to cover interest payments.

C. BT's ability to cover interest payments is greater than Telefonica's, although both companies have sufficient operating earnings to service its interest payments. This comparison indicates that BT has greater financial strength than Telefonica, which is also consistent with the conclusions drawn from a comparison of the companies' ratios in question 1.

\section{SUMMARY}
Non-current liabilities arise from different sources of financing and different types of creditors. Bonds are a common source of financing from debt markets. Key points in accounting and reporting of non-current liabilities include the following:

\begin{itemize}
  \item The sales proceeds of a bond issue are determined by discounting future cash payments using the market rate of interest at the time of issuance (effective interest rate). The reported interest expense on bonds is based on the effective interest rate. - Future cash payments on bonds usually include periodic interest payments (made at the stated interest rate or coupon rate) and the principal amount at maturity.

  \item When the market rate of interest equals the coupon rate for the bonds, the bonds will sell at par (i.e., at a price equal to the face value). When the market rate of interest is higher than the bonds' coupon rate, the bonds will sell at a discount. When the market rate of interest is lower than the bonds' coupon rate, the bonds will sell at a premium

  \item An issuer amortises any issuance discount or premium on bonds over the life of the bonds.

  \item If a company redeems bonds before maturity, it reports a gain or loss on debt extinguishment computed as the net carrying amount of the bonds (including bond issuance costs under IFRS) less the amount required to redeem the bonds.

  \item Debt covenants impose restrictions on borrowers, such as limitations on future borrowing or requirements to maintain a minimum debt-to-equity ratio.

  \item The carrying amount of bonds is typically the amortised historical cost, which can differ from their fair value.

  \item Companies are required to disclose the fair value of financial liabilities, including debt. Although permitted to do so, few companies opt to report debt at fair values on the balance sheet.

  \item A lease is a contract in which a lessor grants the lessee the exclusive right to use a specific underlying asset for a period of time in exchange for payments.

  \item Leasing is a common arrangement because it has several advantages over purchasing an asset outright: less upfront cash commitment, generally low interest rates, and lower risks associated with ownership such as obsolescence.

  \item Leases are classified as operating or finance leases. Finance leases resemble an asset purchase or sale while operating leases resemble a rental agreement.

  \item US GAAP and IFRS share the same accounting treatment for lessors but differ for lessees. IFRS has a single accounting model for both operating leases and finance lease lessees, while US GAAP has an accounting model for each.

  \item Lessees reporting under IFRS and finance lease lessees reporting under US GAAP

  \item Recognize a lease liability and corresponding right-of-use asset on the balance sheet, equal to the present value of lease payments. The liability is subsequently reduced using the effective interest method and the right-of-use asset is amortized. Interest expense and amortization expense are shown separately on the income statement. The statement of cash flows shows the entire lease payment.

  \item Operating lease lessees reporting under US GAAP

  \item Recognize a lease liability and corresponding right-of-use asset on the balance sheet, equal to the present value of lease payments. The liability is subsequently reduced using the effective interest method, but the amortization of the right-of-use asset is the lease payment less the interest expense. Interest expense and amortization expense are shown together as a single operating expense on the income statement. - Finance lease lessors (IFRS and US GAAP)

  \item Recognize a lease receivable asset equal to the present value of future lease payments and de-recognize the leased asset, simultaneously recognizing any difference as a gain or loss. The lease receivable is subsequently reduced by each lease payment using the effective interest method. Interest income is reported on the income statement, typically as revenue, and the entire cash receipt is reported under operating activities on the statement of cash flows.

  \item Operating lease lessors (IFRS and US GAAP)

  \item The balance sheet is not affected: the lessor continues to recognize the underlying asset and depreciate it. Lease revenue is recognized on a straight-line basis on the income statement and the entire cash receipt is reported under operating activities on the statement of cash flows.

  \item Two types of pension plans are defined contribution plans and defined benefits plans. In a defined contribution plan, the amount of contribution into the plan is specified (i.e., defined) and the amount of pension that is ultimately paid by the plan (received by the retiree) depends on the performance of the plan's assets. In a defined benefit plan, the amount of pension that is ultimately paid by the plan (received by the retiree) is defined, usually according to a benefit formula.

  \item Under a defined contribution pension plan, the cash payment made into the plan is recognised as pension expense.

  \item Under both IFRS and US GAAP, companies must report the difference between the defined benefit pension obligation and the pension assets as an asset or liability on the balance sheet. An underfunded defined benefit pension plan is shown as a non-current liability.

  \item Under IFRS, the change in the defined benefit plan net asset or liability is recognised as a cost of the period, with two components of the change (service cost and net interest expense or income) recognised in profit and loss and one component (remeasurements) of the change recognised in other comprehensive income.

  \item Under US GAAP, the change in the defined benefit plan net asset or liability is also recognised as a cost of the period with three components of the change (current service costs, interest expense on the beginning pension obligation, and expected return on plan assets) recognised in profit and loss and two components (past service costs and actuarial gains and losses) typically recognised in other comprehensive income.

  \item Solvency refers to a company's ability to meet its long-term debt obligations.

  \item In evaluating solvency, leverage ratios focus on the balance sheet and measure the amount of debt financing relative to equity financing.

  \item In evaluating solvency, coverage ratios focus on the income statement and cash flows and measure the ability of a company to cover its interest payments.

\end{itemize}

\section{PRACTICE PROBLEMS}
\begin{enumerate}
  \item A company issues $€ 1$ million of bonds at face value. When the bonds are issued, the company will record a:
A. cash inflow from investing activities.
B. cash inflow from financing activities.
C. cash inflow from operating activities.

  \item At the time of issue of $4.50 \%$ coupon bonds, the effective interest rate was $5.00 \%$. The bonds were most likely issued at:
A. par.
B. a discount.
C. a premium.

  \item Oil Exploration LLC paid $\$ 45,000$ in printing, legal fees, commissions, and other costs associated with its recent bond issue. It is most likely to record these costs on its financial statements as:

\end{enumerate}

A. an asset under US GAAP and reduction of the carrying value of the debt under IFRS.

B. a liability under US GAAP and reduction of the carrying value of the debt under IFRS.

C. a cash outflow from investing activities under both US GAAP and IFRS.

\begin{enumerate}
  \setcounter{enumi}{3}
  \item A company issues $\$ 1,000,000$ face value of 10 -year bonds on 1 January 2015 when the market interest rate on bonds of comparable risk and terms is $5 \%$. The bonds pay $6 \%$ interest annually on 31 December. At the time of issue, the bonds payable reflected on the balance sheet is closest to:
A. $\$ 926,399$.
B. $\$ 1,000,000$,
C. $\$ 1,077,217$.

  \item Midland Brands issues three-year bonds dated 1 January 2015 with a face value of $\$ 5,000,000$. The market interest rate on bonds of comparable risk and term is $3 \%$. If the bonds pay $2.5 \%$ annually on 31 December, bonds payable when issued are most likely reported as closest to:
A. $\$ 4,929,285$.
B. $\$ 5,000,000$.
C. $\$ 5,071,401$.

  \item A firm issues a bond with a coupon rate of $5.00 \%$ when the market interest rate is $5.50 \%$ on bonds of comparable risk and terms. One year later, the market interest rate increases to $6.00 \%$. Based on this information, the effective interest rate is:

\end{enumerate}

A. $5.00 \%$. B. $5.50 \%$.

C. $6.00 \%$.

\begin{enumerate}
  \setcounter{enumi}{6}
  \item On 1 January 2010, Elegant Fragrances Company issues $\pounds 1,000,000$ face value, five-year bonds with annual interest payments of $\pounds 55,000$ to be paid each 31 December. The market interest rate is 6.0 percent. Using the effective interest rate method of amortisation, Elegant Fragrances is most likely to record:
\end{enumerate}

A. an interest expense of $\pounds 55,000$ on its 2010 income statement.

B. a liability of $\pounds 982,674$ on the 31 December 2010 balance sheet.

C. a $\pounds 58,736$ cash outflow from operating activity on the 2010 statement of cash flows.

\begin{enumerate}
  \setcounter{enumi}{7}
  \item Consolidated Enterprises issues $€ 10$ million face value, five-year bonds with a coupon rate of 6.5 percent. At the time of issuance, the market interest rate is 6.0 percent. Using the effective interest rate method of amortisation, the carrying value after one year will be closest to:
A. $€ 10.17$ million.
B. $€ 10.21$ million.
C. $€ 10.28$ million.

  \item A company issues $€ 10,000,000$ face value of 10 -year bonds dated 1 January 2015 when the market interest rate on bonds of comparable risk and terms is $6 \%$. The bonds pay $7 \%$ interest annually on 31 December. Based on the effective interest rate method, the interest expense on 31 December 2015 is closest to:
A. $€ 644,161$
B. $€ 700,000$.
C. $€ 751,521$.

  \item A company issues $\$ 30,000,000$ face value of five-year bonds dated 1 January 2015 when the market interest rate on bonds of comparable risk and terms is $5 \%$. The bonds pay $4 \%$ interest annually on 31 December. Based on the effective interest rate method, the carrying amount of the bonds on 31 December 2015 is closest to:
A. $\$ 28,466,099$.
B. $\$ 28,800,000$.
C. $\$ 28,936,215$.

  \item Lesp Industries issues five-year bonds dated 1 January 2015 with a face value of $\$ 2,000,000$ and $3 \%$ coupon rate paid annually on 31 December. The market interest rate on bonds of comparable risk and term is $4 \%$. The sales proceeds of the bonds are $\$ 1,910,964$. Under the effective interest rate method, the interest expense in 2017 is closest to:

\end{enumerate}

A. $\$ 77,096$.

B. $\$ 77,780$. C. $\$ 77,807$.

\begin{enumerate}
  \setcounter{enumi}{11}
  \item For a bond issued at a premium, using the effective interest rate method, the:
\end{enumerate}

A. carrying amount increases each year.

B. amortization of the premium increases each year.

C. premium is evenly amortized over the life of the bond.

\begin{enumerate}
  \setcounter{enumi}{12}
  \item Comte Industries issues $\$ 3,000,000$ worth of three-year bonds dated 1 January 2015. The bonds pay interest of $5.5 \%$ annually on 31 December. The market interest rate on bonds of comparable risk and term is $5 \%$. The sales proceeds of the bonds are $\$ 3,040,849$. Under the straight-line method, the interest expense in the first year is closest to:
A. $\$ 150,000$.
B. $\$ 151,384$.
C. $\$ 152,042$.

  \item Innovative Inventions, Inc. needs to raise $€ 10$ million. If the company chooses to issue zero-coupon bonds, its debt-to-equity ratio will most likely:
A. rise as the maturity date approaches.
B. decline as the maturity date approaches.
C. remain constant throughout the life of the bond.

  \item Fairmont Golf issued fixed rate debt when interest rates were 6 percent. Rates have since risen to 7 percent. Using only the carrying amount (based on historical cost) reported on the balance sheet to analyze the company's financial position would most likely cause an analyst to:

\end{enumerate}

A. overestimate Fairmont's economic liabilities.

B. underestimate Fairmont's economic liabilities.

C. underestimate Fairmont's interest coverage ratio.

\begin{enumerate}
  \setcounter{enumi}{15}
  \item The management of Bank EZ repurchases its own bonds in the open market. They pay $€ 6.5$ million for bonds with a face value of $€ 10.0$ million and a carrying value of $€ 9.8$ million. The bank will most likely report:
A. other comprehensive income of $€ 3.3$ million.
B. other comprehensive income of $€ 3.5$ million.
C. a gain of $€ 3.3$ million on the income statement.

  \item A company redeems $\$ 1,000,000$ face value bonds with a carrying value of $\$ 990,000$. If the call price is 104 the company will:
A. reduce bonds payable by $\$ 1,000,000$.
B. recognize a loss on the extinguishment of debt of $\$ 50,000$.
C. recognize a gain on the extinguishment of debt of $\$ 10,000$. 18. Which of the following is an example of an affirmative debt covenant? The borrower is:
A. prohibited from entering into mergers.
B. prevented from issuing excessive additional debt.
C. required to perform regular maintenance on equipment pledged as collateral.

  \item Debt covenants are least likely to place restrictions on the issuer's ability to:
A. pay dividends.
B. issue additional debt.
C. issue additional equity.

  \item Regarding a company's debt obligations, which of the following is most likely presented on the balance sheet?

\end{enumerate}

A. Effective interest rate

B. Maturity dates for debt obligations

C. The portion of long-term debt due in the next 12 months

\begin{enumerate}
  \setcounter{enumi}{20}
  \item Beginning with fiscal year 2019, for leases with a term longer than one year, lessees report a right-to-use asset and a lease liability on the balance sheet:
\end{enumerate}

A. only for finance leases.

B. only for operating leases.

C. for both finance and operating leases.

\begin{enumerate}
  \setcounter{enumi}{21}
  \item For a lessor, the leased asset appears on the balance sheet and continues to be depreciated when the lease is classified as:
A. a finance lease.
B. a sales-type lease.
C. an operating lease.

  \item Under US GAAP, a lessor's reported revenues at lease inception will be highest if the lease is classified as:
A. a sales-type lease.
B. an operating lease.
C. a direct financing lease.

  \item Under both IFRS and US GAAP, a lessor in an operating lease recognizes:

\end{enumerate}

A. selling profit at lease inception.

B. a lease asset comprising the lease receivable and relevant residual value at lease inception. C. lease receipts as income and related costs, including depreciation, as expenses over the lease term.

\begin{enumerate}
  \setcounter{enumi}{24}
  \item Compared with a finance lease, an operating lease:
A. is similar to renting an asset.
B. is equivalent to the purchase of an asset.
C. term is for the majority of the economic life of the leased asset.

  \item Under US GAAP, a lessee's accounting for a long-term finance lease after inception will include:
A. recognizing a single lease expense.
B. recording depreciation expense on the right-of-use asset.
C. increasing the balance of the lease liability by a portion of the lease payment.

  \item A company enters into a finance lease agreement to acquire the use of an asset for three years with lease payments of $€ 19,000,000$ starting next year. The leased asset has a fair market value of $€ 49,000,000$ and the present value of the lease payments is $€ 47,250,188$. Based on this information, the value of the lease liability reported on the company's balance sheet at lease inception is closest to:
A. $€ 47,250,188$.
B. $€ 49,000,000$.
C. $€ 57,000,000$.

  \item Penben Corporation has a defined benefit pension plan. At 31 December, its pension obligation is $€ 10$ million and pension assets are $€ 9$ million. Under either IFRS or US GAAP, the reporting on the balance sheet would be closest to which of the following?

\end{enumerate}

A. $€ 10$ million is shown as a liability, and $€ 9$ million appears as an asset.

B. $€ 1$ million is shown as a net pension obligation.

C. Pension assets and obligations are not required to be shown on the balance sheet but only disclosed in footnotes.

\begin{enumerate}
  \setcounter{enumi}{28}
  \item The following information is associated with a company that offers its employees a defined benefit plan:
Fair value of fund's assets
$\$ 1,500,000,000$
Estimated pension obligations
$\$ 2,600,000,000$
Present value of estimated pension obligations
$\$ 1,200,000,000$
\end{enumerate}

Based on this information, the company's balance sheet will present a net pension:

A. asset of $\$ 300,000,000$.

B. asset of $\$ 1,400,000,000$.

C. liability of $\$ 1,100,000,000$. 30. The following presents selected financial information for a company:

\begin{center}
\begin{tabular}{lc}
\hline
 & \$ Millions \\
\hline
Short-term borrowing & 4,231 \\
Current portion of long-term interest-bearing debt & 29 \\
Long-term interest-bearing debt & 925 \\
Average shareholders' equity & 18,752 \\
Average total assets & 45,981 \\
\hline
\end{tabular}
\end{center}

The financial leverage ratio is closest to:
A. 0.113
B. 0.277
C. 2.452

\begin{enumerate}
  \setcounter{enumi}{30}
  \item An analyst evaluating three industrial companies calculates the following ratios:
\end{enumerate}

\begin{center}
\begin{tabular}{lccc}
\hline
 & Company A & Company B & Company C \\
\hline
Debt-to-Equity & $23.5 \%$ & $22.5 \%$ & $52.5 \%$ \\
Interest Coverage & 15.6 & 49.5 & 45.5 \\
\hline
\end{tabular}
\end{center}

The company with both the lowest financial leverage and the greatest ability to meet interest payments is:

A. Company A.

B. Company B.

C. Company C.

\begin{enumerate}
  \setcounter{enumi}{31}
  \item An analyst evaluating a company's solvency gathers the following information:
\end{enumerate}

\begin{center}
\begin{tabular}{lc}
\hline
 & \$ Millions \\
\hline
Short-term interest-bearing debt & 1,258 \\
Long-term interest-bearing debt & 321 \\
Total shareholder's equity & 4,285 \\
Total assets & 8,750 \\
EBIT & 2,504 \\
Interest payments & 52 \\
\hline
\end{tabular}
\end{center}

The company's debt-to-assets ratio is closest to:
A. 0.18 .
B. 0.27
C. 0.37 .

\section{SOLUTIONS}
\begin{enumerate}
  \item B is correct. The company receives $€ 1$ million in cash from investors at the time the bonds are issued, which is recorded as a financing activity.

  \item B is correct. The effective interest rate is greater than the coupon rate and the bonds will be issued at a discount.

  \item A is correct. Under US GAAP, expenses incurred when issuing bonds are generally recorded as an asset and amortised to the related expense (legal, etc.) over the life of the bonds. Under IFRS, they are included in the measurement of the liability. The related cash flows are financing activities.

  \item $\mathrm{C}$ is correct. The bonds will be issued at a premium because the coupon rate is higher than the market interest rate. The future cash outflows, the present value of the cash outflows, and the total present value are as follows:

\end{enumerate}

\begin{center}
\begin{tabular}{|c|c|c|c|c|c|}
\hline
Date & $\begin{array}{c}\text { Interest } \\ \text { Payment (\$) }\end{array}$ & $\begin{array}{c}\text { Present Value at } \\ \text { Market Rate } 5 \% \\ \text { (\$) }\end{array}$ &  & $\begin{array}{c}\text { Present Value at } \\ \text { Market Rate } 5 \% \\ \text { (\$) }\end{array}$ & $\begin{array}{c}\text { Total Present } \\ \text { Value (\$) }\end{array}$ \\
\hline
31 December 2015 & $60,000.00$ & $57,142.86$ &  &  &  \\
\hline
31 December 2016 & $60,000.00$ & $54,421.77$ &  &  &  \\
\hline
31 December 2017 & $60,000.00$ & $51,830.26$ &  &  &  \\
\hline
31 December 2018 & $60,000.00$ & $49,362.15$ &  &  &  \\
\hline
31 December 2019 & $60,000.00$ & $47,011.57$ &  &  &  \\
\hline
31 December 2020 & $60,000.00$ & $44,772.92$ &  &  &  \\
\hline
31 December 2021 & $60,000.00$ & $42,640.88$ &  &  &  \\
\hline
31 December 2022 & $60,000.00$ & $40,610.36$ &  &  &  \\
\hline
31 December 2023 & $60,000.00$ & $38,676.53$ &  &  &  \\
\hline
\multirow[t]{3}{*}{31 December 2024} & $60,000.00$ & $36,834.80$ & $1,000,000.00$ & $613,913.25$ &  \\
\hline
 &  & $463,304.10$ &  & $613,913.25$ & $1,077,217.35$ \\
\hline
 &  &  &  &  & Sales Proceed \\
\hline
\end{tabular}
\end{center}

The following illustrates the keystrokes for many financial calculators to calculate sales proceeds of $\$ 1,077,217.35$ :

\begin{center}
\begin{tabular}{ll}
\hline
Calculator Notation & Numerical Value for This Problem \\
\hline
$\mathrm{N}$ & 10 \\
$\% i$ or I/Y & 5 \\
FV & $\$ 1,000,000.00$ \\
PMT & $\$ 60,000.00$ \\
PV compute & $\mathrm{X}$ \\
\hline
\end{tabular}
\end{center}

Thus, the sales proceeds are reported on the balance sheet as an increase in long-term liability, bonds payable of $\$ 1,077,217$.

\begin{enumerate}
  \setcounter{enumi}{4}
  \item A is correct. The bonds payable reported at issue is equal to the sales proceeds. The interest payments and future value of the bond must be discounted at the market interest rate of $3 \%$ to determine the sales proceeds.
\end{enumerate}

\begin{center}
\begin{tabular}{|c|c|c|c|c|c|}
\hline
Date & $\begin{array}{c}\text { Interest } \\ \text { Payment }\end{array}$ & $\begin{array}{c}\text { Present } \\ \text { Value at } \\ \text { Market Rate } \\ (3 \%)\end{array}$ & $\begin{array}{c}\text { Face Value } \\ \text { Payment }\end{array}$ & $\begin{array}{c}\text { Present Value } \\ \text { at Market Rate } \\ (3 \%)\end{array}$ & Total Present Value \\
\hline
31 December 2015 & $\$ 125,000.00$ & $\$ 121,359.22$ &  &  &  \\
\hline
31 December 2016 & $\$ 125,000.00$ & $\$ 117,824.49$ &  &  &  \\
\hline
31 December 2017 & $\$ 125,000.00$ & $\$ 114,392.71$ & $\$ 5,000,000.00$ & $\$ 4,575,708.30$ &  \\
\hline
Total &  & $\$ 353,576.42$ &  & $\$ 4,575,708.30$ & $\$ 4,929,284.72$ \\
\hline
\end{tabular}
\end{center}

The following illustrates the keystrokes for many financial calculators to calculate sales proceeds of $\$ 4,929,284.72$ :

\begin{center}
\begin{tabular}{ll}
\hline
Calculator Notation & Numerical Value for This Problem \\
\hline
$\mathrm{N}$ & 3 \\
$\% i$ or I/Y & 3.0 \\
$\mathrm{FV}$ & $\$ 5,000,000.00$ \\
PMT & $\$ 125,000.00$ \\
PV compute & $\mathrm{X}$ \\
\hline
\end{tabular}
\end{center}

\begin{enumerate}
  \setcounter{enumi}{5}
  \item B is correct. The market interest rate at the time of issuance is the effective interest rate that the company incurs on the debt. The effective interest rate is the discount rate that equates the present value of the coupon payments and face value to their selling price. Consequently, the effective interest rate is $5.50 \%$.

  \item B is correct. The bonds will be issued at a discount because the market interest rate is higher than the stated rate. Discounting the future payments to their present value indicates that at the time of issue, the company will record $\pounds 978,938$ as both a liability and a cash inflow from financing activities. Interest expense in 2010 is $\pounds 58,736$ ( $\pounds 978,938$ times 6.0 percent). During the year, the company will pay cash of $\pounds 55,000$ related to the interest payment, but interest expense on the income statement will also reflect $\pounds 3,736$ related to amortisation of the initial discount ( $\pounds 58,736$ interest expense less the $\pounds 55,000$ interest payment). Thus, the value of the liability at 31 December 2010 will reflect the initial value $(\pounds 978,938)$ plus the amortised discount $(\pounds 3,736)$, for a total of $\pounds 982,674$. The cash outflow of $\pounds 55,000$ may be presented as either an operating or financing activity under IFRS.

  \item A is correct. The coupon rate on the bonds is higher than the market rate, which indicates that the bonds will be issued at a premium. Taking the present value of each payment indicates an issue date value of $€ 10,210,618$. The interest expense is determined by multiplying the carrying amount at the beginning of the period $(€ 10,210,618)$ by the market interest rate at the time of issue ( 6.0 percent) for an interest expense of $€ 612,637$. The value after one year will equal the beginning value less the amount of the premium amortised to date, which is the difference between the amount paid $(€ 650,000)$ and the expense accrued $(€ 612,637)$ or $€ 37,363 . € 10,210,618-€ 37,363=€ 10,173,255$ or $€ 10.17$ million.

  \item A is correct. The future cash outflows, the present value of the cash outflows, and the total present value are as follows:

\end{enumerate}

\begin{center}
\begin{tabular}{|c|c|c|c|c|c|}
\hline
Date & $\begin{array}{c}\text { Interest } \\ \text { Payment }(€)\end{array}$ & $\begin{array}{c}\text { Present Value at } \\ \text { Market Rate } 6 \% \text { (€) }\end{array}$ &  & $\begin{array}{c}\text { Present Value at } \\ \text { Market Rate } 6 \% \\ (€)\end{array}$ & $\begin{array}{l}\text { Total Present Value } \\ (€)\end{array}$ \\
\hline
31 December 2015 & $700,000.00$ & $660,377.36$ &  &  &  \\
\hline
31 December 2016 & $700,000.00$ & $622,997.51$ &  &  &  \\
\hline
31 December 2017 & $700,000.00$ & $587,733.50$ &  &  &  \\
\hline
31 December 2018 & $700,000.00$ & $554,465.56$ &  &  &  \\
\hline
31 December 2019 & $700,000.00$ & $523,080.72$ &  &  &  \\
\hline
31 December 2020 & $700,000.00$ & $493,472.38$ &  &  &  \\
\hline
31 December 2021 & $700,000.00$ & $465,539.98$ &  &  &  \\
\hline
31 December 2022 & $700,000.00$ & $439,188.66$ &  &  &  \\
\hline
31 December 2023 & $700,000.00$ & $414,328.92$ &  &  &  \\
\hline
\multirow[t]{3}{*}{31 December 2024} & $700,000.00$ & $390,876.34$ & $10,000,000.00$ & $5,583,947.77$ &  \\
\hline
 &  & $5,152,060.94$ &  & $5,583,947.77$ & $10,736,008.71$ \\
\hline
 &  &  &  &  & Sales Proceeds \\
\hline
\end{tabular}
\end{center}

The following illustrates the keystrokes for many financial calculators to calculate sales proceeds of $€ 10,736,008.71$ :

\begin{center}
\begin{tabular}{ll}
\hline
Calculator Notation & Numerical Value for This Problem \\
\hline
$\mathrm{N}$ & 10 \\
$\% i$ or I/Y & 6 \\
FV & $\$ 10,000,000.00$ \\
PMT & $\$ 700,000.00$ \\
PV compute & $\mathrm{X}$ \\
\hline
\end{tabular}
\end{center}

The interest expense is calculated by multiplying the carrying amount at the beginning of the year by the effective interest rate at issuance. As a result, the interest expense at 31 December 2015 is $€ 644,161(€ 10,736,008.71 \times 6 \%)$.

\begin{enumerate}
  \setcounter{enumi}{9}
  \item C is correct. The future cash outflows, the present value of the cash outflows, and the total present value are as follows:
\end{enumerate}

\begin{center}
\begin{tabular}{|c|c|c|c|c|c|}
\hline
Date & $\begin{array}{c}\text { Interest } \\ \text { Payment (\$) }\end{array}$ & $\begin{array}{c}\text { Present Value at } \\ \text { Market Rate } 5 \% \\ \text { (\$) }\end{array}$ &  & $\begin{array}{c}\text { Present Value at } \\ \text { Market Rate 5\% (\$) }\end{array}$ & $\begin{array}{l}\text { Total Present Value } \\ \text { (\$) }\end{array}$ \\
\hline
31 December 2015 & $1,200,000$ & $1,142,857.14$ &  &  &  \\
\hline
31 December 2016 & $1,200,000$ & $1,088,435.37$ &  &  &  \\
\hline
31 December 2017 & $1,200,000$ & $1,036,605.12$ &  &  &  \\
\hline
31 December 2018 & $1,200,000$ & $987,242.97$ &  &  &  \\
\hline
\multirow[t]{2}{*}{31 December 2019} & $1,200,000$ & $940,231.40$ & $30,000,000$ & $23,505,785.00$ &  \\
\hline
 &  & $5,195,372.00$ &  & $23,505,785.00$ & $28,701,157.00$ \\
\hline
\end{tabular}
\end{center}

Sales Proceeds

The following illustrates the keystrokes for many financial calculators to calculate sales proceeds of $\$ 28,701,157.00$ :

Calculator Notation Numerical Value for This Problem

\begin{center}
\begin{tabular}{ll}
\hline
Calculator Notation & Numerical Value for This Problem \\
\hline
$\% i$ or I/Y & 5 \\
FV & $\$ 30,000,000.00$ \\
PMT & $\$ 1,200,000.00$ \\
PV compute & $\mathrm{X}$ \\
\hline
\end{tabular}
\end{center}

The following table illustrates interest expense, premium amortization, and carrying amount (amortized cost) for 2015.

\begin{center}
\includegraphics[max width=\textwidth]{2023_05_04_b5cfa4f1bc883752f121g-508}
\end{center}

\begin{enumerate}
  \setcounter{enumi}{10}
  \item $B$ is correct. The interest expense for a given year is equal to the carrying amount at the beginning of the year times the effective interest of $4 \%$. Under the effective interest rate method, the difference between the interest expense and the interest payment (based on the coupon rate and face value) is the discount amortized in the period, which increases the carrying amount annually. For 2017, the interest expense is the beginning carrying amount $(\$ 1,944,499)$ times the effective interest of $4 \%$.
\end{enumerate}

\begin{center}
\begin{tabular}{|c|c|c|c|c|c|}
\hline
Year & $\begin{array}{l}\text { Carrying Amount } \\ \text { (beginning) }\end{array}$ & $\begin{array}{l}\text { Interest Expense (at } \\ \text { effective interest of } \\ 4 \% \text { ) }\end{array}$ & $\begin{array}{l}\text { Interest Payment (at } \\ \text { coupon rate of } 3 \% \text { ) }\end{array}$ & $\begin{array}{c}\text { Amortization of } \\ \text { Discount }\end{array}$ & $\begin{array}{c}\text { Carrying Amoun } \\ \text { (end of year) }\end{array}$ \\
\hline
2015 & $\$ 1,910,964$ & $\$ 76,439$ & $\$ 60,000.00$ & $\$ 16,439$ & $\$ 1,927,403$ \\
\hline
2016 & $\$ 1,927,403$ & $\$ 77,096$ & $\$ 60,000.00$ & $\$ 17,096$ & $\$ 1,944,499$ \\
\hline
2017 & $\$ 1,944,499$ & $\$ 77,780$ & $\$ 60,000.00$ & $\$ 17,780$ & $\$ 1,962,279$ \\
\hline
\end{tabular}
\end{center}

\begin{enumerate}
  \setcounter{enumi}{11}
  \item $\mathrm{B}$ is correct. The amortization of the premium equals the interest payment minus the interest expense. The interest payment is constant and the interest expense decreases as the carrying amount decreases. As a result, the amortization of the premium increases each year.

  \item B is correct. Under the straight-line method, the bond premium is amortized equally over the life of the bond. The annual interest payment is $\$ 165,000$ $(\$ 3,000,000 \times 5.5 \%)$ and annual amortization of the premium under the straight-line method is $\$ 13,616[(\$ 3,040,849-\$ 3,000,000) / 3)]$. The interest expense is the interest payment less the amortization of the premium $(\$ 165,000-$ $\$ 13,616=\$ 151,384)$.

  \item A is correct. The value of the liability for zero-coupon bonds increases as the discount is amortised over time. Furthermore, the amortised interest will reduce earnings at an increasing rate over time as the value of the liability increases. Higher relative debt and lower relative equity (through retained earnings) will cause the debt-to-equity ratio to increase as the zero-coupon bonds approach maturity.

  \item A is correct. When interest rates rise, bonds decline in value. Thus, the carrying amount of the bonds being carried on the balance sheet is higher than the market value. The company could repurchase the bonds for less than the carrying amount, so the economic liabilities are overestimated. Because the bonds are issued at a fixed rate, there is no effect on interest coverage.

  \item C is correct. A gain of $€ 3.3$ million (carrying amount less amount paid) will be reported on the income statement.

  \item B is correct. If a company decides to redeem a bond before maturity, bonds payable is reduced by the carrying amount of the debt. The difference between the cash required to redeem the bonds and the carrying amount of the bonds is a gain or loss on the extinguishment of debt. Because the call price is 104 and the face value is $\$ 1,000,000$, the redemption cost is $104 \%$ of $\$ 1,000,000$ or $\$ 1,040,000$. The company's loss on redemption would be $\$ 50,000$ ( $\$ 990,000$ carrying amount of debt minus $\$ 1,040,000$ cash paid to redeem the callable bonds).

  \item C is correct. Affirmative covenants require certain actions of the borrower. Requiring the company to perform regular maintenance on equipment pledged as collateral is an example of an affirmative covenant because it requires the company to do something. Negative covenants require that the borrower not take certain actions. Prohibiting the borrower from entering into mergers and preventing the borrower from issuing excessive additional debt are examples of negative covenants.

  \item C is correct. Covenants protect debtholders from excessive risk taking, typically by limiting the issuer's ability to use cash or by limiting the overall levels of debt relative to income and equity. Issuing additional equity would increase the company's ability to meet its obligations, so debtholders would not restrict that ability.

  \item $C$ is correct. The non-current liabilities section of the balance sheet usually includes a single line item of the total amount of a company's long-term debt due after 1 year, and the current liabilities section shows the portion of a company's long-term debt due in the next 12 months. Notes to the financial statements generally present the stated and effective interest rates and maturity dates for a company's debt obligations

  \item C is correct. Beginning with fiscal year 2019 , lessees report a right-of-use asset and a lease liability for all leases longer than one year. An exception under IFRS exists for leases when the underlying asset is of low value.

  \item C is correct. When a lease is classified as an operating lease, the underlying asset remains on the lessor's balance sheet. The lessor will record a depreciation expense that reduces the asset's value over time.

  \item A is correct. A sales-type lease treats the lease as a sale of the asset, and revenue is recorded at the time of sale equal to the value of the leased asset. Under a direct financing lease, only interest income is reported as earned. Under an operating lease, revenue from lease receipts is reported when collected.

  \item $\mathrm{C}$ is correct. Lessor accounting for an operating lease under US GAAP is similar to that under IFRS: Over the lease term, the lessor recognizes lease receipts as income and recognizes related costs, including depreciation of the leased asset, as expenses. Under IFRS, at inception of a finance lease-not an operating lease-the lessor derecognizes the underlying leased asset and recognizes a lease asset comprising the lease receivable and relevant residual value. Further, an IFRS-reporting lessor will recognize selling profit at the beginning of all leases that are not classified as operating leases. In contrast, a US GAAP-reporting lessor will recognize selling profit only on sales-type leases at the beginning of the lease term.

  \item A is correct. An operating lease is an agreement that allows the lessee to use an asset for a period of time. Thus, an operating lease is similar to renting an asset, whereas a finance lease is equivalent to the purchase of an asset by the lessee that is directly financed by the lessor.

  \item B is correct. A lessee's accounting for a long-term finance lease under US GAAP and after lease inception includes recording depreciation expense on the right-of-use asset, recognizing interest expense on the lease liability, and reducing the balance of the lease liability for the portion of the lease payments that represents repayment of the lease liability. A lessee's accounting for an operating lease under US GAAP and after lease inception will recognize a single lease expense, which is a straight-line allocation of the cost of the lease over its term.

  \item A is correct. Under the revised reporting standards under IFRS and US GAAP, a lessee must recognize an asset and a lease liability at inception of each of its leases (with an exception for short-term leases). The lessee reports a "right-of-use" (ROU) asset and a lease liability, calculated essentially as the present value of fixed lease payments, on its balance sheet. Thus, at lease inception, the company will record a lease liability on the balance sheet of $€ 47,250,188$.

  \item B is correct. The company will report a net pension obligation of $€ 1$ million equal to the pension obligation ( $€ 10$ million) less the plan assets ( $€ 9$ million).

  \item A is correct. A company that offers a defined benefit plan makes payments into a pension fund and the retirees are paid from the fund. The payments that a company makes into the fund are invested until they are needed to pay retirees. If the fair value of the fund's assets is higher than the present value of the estimated pension obligation, the plan has a surplus and the company's balance sheet will reflect a net pension asset. Because the fair value of the fund's assets is $\$ 1,500,000,000$ and the present value of estimated pension obligations is $\$ 1,200,000,000$, the company will present a net pension asset of $\$ 300,000,000$ on its balance sheet.

  \item C is correct. The financial leverage ratio is calculated as follows:

\end{enumerate}

\begin{center}
\includegraphics[max width=\textwidth]{2023_05_04_b5cfa4f1bc883752f121g-510}
\end{center}

\begin{enumerate}
  \setcounter{enumi}{30}
  \item B is correct. Company B has the lowest debt-to-equity ratio, indicating the lowest financial leverage, and the highest interest coverage ratio, indicating the greatest number of times that EBIT covers interest payments.

  \item A is correct because the debt-to-assets (total debt)/(total assets) ratio is $(1,258+$ $321) /(8,750)=1,579 / 8,750=0.18$

\end{enumerate}

\section{LEARNING MODULE}
\begin{center}
\includegraphics[max width=\textwidth]{2023_05_04_b5cfa4f1bc883752f121g-511}
\end{center}

\section{Financial Reporting Quality}
by Jack T. Ciesielski, CPA, CFA, Elaine Henry, PhD, CFA, and Thomas I. Selling, PhD, CPA.

Jack T. Ciesielski, CPA, CFA, is at R.G. Associates, Inc., former publisher of The Analyst's Accounting Observer (USA). Elaine Henry, PhD, CFA, is at Stevens Institute of Technology (USA). Thomas I. Selling, PhD, CPA, is at the Cox School of Business, Southern Methodist University (USA).

\section{LEARNING OUTCOME}
\begin{center}
\begin{tabular}{c|l}
Mastery & The candidate should be able to: \\
\hline
$\square$ & $\begin{array}{l}\text { compare and contrast financial reporting quality with the quality } \\ \text { of reported results (including quality of earnings, cash flow, and } \\ \text { balance sheet items) } \\ \text { describe a spectrum for assessing financial reporting quality }\end{array}$ \\
$\square$ & $\begin{array}{l}\text { explain the difference between conservative and aggressive } \\ \text { accounting } \\ \text { describe motivations that might cause management to issue financial } \\ \text { reports that are not high quality } \\ \text { describe conditions that are conducive to issuing low-quality, or even } \\ \text { fraudulent, financial reports } \\ \text { describe mechanisms that discipline financial reporting quality and } \\ \text { the potential limitations of those mechanisms }\end{array}$ \\
$\square$ & $\begin{array}{l}\text { describe presentation choices, including non-GAAP measures, that } \\ \text { could be used to influence an analyst's opinion } \\ \text { describe accounting methods (choices and estimates) that could be } \\ \text { used to manage earnings, cash flow, and balance sheet items } \\ \text { describe accounting warning signs and methods for detecting } \\ \text { manipulation of information in financial reports }\end{array}$ \\
$\square$ &  \\
\end{tabular}
\end{center}

Note: Changes in accounting standards as well as new rulings and/or pronouncements issued after the publication of the readings on financial reporting and analysis may cause some of the information in these readings to become dated. Candidates are not responsible for anything that occurs after the readings were published. In addition, candidates are expected to be familiar with the analytical frameworks contained in the readings, as well as the implications of alternative accounting methods for financial analysis and valuation discussed in the readings. Candidates are also responsible for the content of accounting standards, but not for the actual reference numbers. Finally, candidates should be aware that certain ratios may be defined and calculated differently. When alternative ratio definitions exist and no specific definition is given, candidates should use the ratio definitions emphasized in the readings.

\section{INTRODUCTION \& CONCEPTUAL OVERVIEW}
compare and contrast financial reporting quality with the quality of reported results (including quality of earnings, cash flow, and balance sheet items)

describe a spectrum for assessing financial reporting quality

Ideally, analysts would always have access to financial reports that are based on sound financial reporting standards, such as those from the International Accounting Standards Board (IASB) and the Financial Accounting Standards Board (FASB), and are free from manipulation. But, in practice, the quality of financial reports can vary greatly. High-quality financial reporting provides information that is useful to analysts in assessing a company's performance and prospects. Low-quality financial reporting contains inaccurate, misleading, or incomplete information.

Extreme lapses in financial reporting quality have given rise to high-profile scandals that resulted not only in investor losses but also in reduced confidence in the financial system. Financial statement users who were able to accurately assess financial reporting quality were better positioned to avoid losses. These lapses illustrate the challenges analysts face as well as the potential costs of failing to recognize practices that result in misleading or inaccurate financial reports. ${ }^{1}$ Examples of misreporting can provide an analyst with insight into various signals that may indicate poor-quality financial reports.

This reading addresses financialreporting quality, which pertains to the quality of information in financial reports, including disclosures in notes. High-quality reporting provides decision-useful information, which is relevant and faithfully represents the economic reality of the company's activities during the reporting period as well as the company's financial condition at the end of the period. A separate but interrelated attribute of quality is quality of reported results or earnings quality, which pertains to the earnings and cash generated by the company's actual economic activities and the resulting financial condition. The term "earnings quality" is commonly used in practice and will be used broadly to encompass the quality of earnings, cash flow, and/or balance sheet items. High-quality earnings result from activities that a company will likely be able to sustain in the future and provide a sufficient return on the company's investment. The concepts of earnings quality and financial reporting quality are interrelated because a correct assessment of earnings quality is possible only when there is some basic level of financial reporting quality. Beyond this basic level, as the quality of reporting increases, the ability of financial statement users to correctly assess earnings quality and to develop expectations for future performance arguably also increases.

Section 2 provides a conceptual overview of reporting quality. Section 3 discusses motivations that might cause, and conditions that might enable, management to issue financial reports that are not high quality and mechanisms that aim to provide discipline to financial reporting quality. Section 4 describes choices made by management that can affect financial reporting quality-presentation choices, accounting methods, and estimates-as well as warning signs of poor-quality financial reporting.

1 In this reading, the examples of misleading or inaccurate financial reports occurred in prior years-not because there are no current examples of questionable financial reporting, but rather because it has been conclusively resolved that misreporting occurred in the historical examples.

\section{Conceptual Overview}
As indicated in the introduction, financial reporting quality and results or earnings quality are interrelated attributes of quality. Exhibit 1 illustrates this interrelationship and its implications.

Exhibit 1: Relationships between Financial Reporting Quality and Earnings Quality

Financial Reporting Quality

\begin{center}
\begin{tabular}{|c|c|c|c|}
\hline
 &  & Low & High \\
\hline
\multirow{2}{*}{$\begin{array}{l}\text { Earnings } \\
\text { (Results) } \\
\text { Quality }\end{array}$} & \multirow{2}{*}{$\begin{array}{l}\text { High } \\
\text { Low }\end{array}$} & \multirow{2}{*}{$\begin{array}{l}\text { LOW financial reporting } \\
\text { quality impedes assessment of } \\
\text { earnings quality and impedes } \\
\text { valuation. }\end{array}$} & $\begin{array}{l}\text { HIGH financial reporting qual- } \\ \text { ity enables assessment. HIGH } \\ \text { earnings quality increases } \\ \text { company value. }\end{array}$ \\
\hline
 &  &  & $\begin{array}{l}\text { HIGH financial reporting qual- } \\ \text { ity enables assessment. LOW } \\ \text { earnings quality decreases } \\ \text { company value. }\end{array}$ \\
\hline
\end{tabular}
\end{center}

As can be seen in Exhibit 1, if financial reporting quality is low, the information provided is of little use in assessing the company's performance, and thus in making investment and other decisions.

Financial reporting quality varies across companies. High-quality reports contain information that is relevant, complete, neutral, and free from error. The lowest-quality reports contain information that is pure fabrication. Earnings (results) quality can range from high and sustainable to low and unsustainable. Providers of resources prefer high and sustainable earnings. Combining the two measures of quality-financial reporting and earnings-the overall quality of financial reports from a user perspective can be thought of as spanning a continuum from the highest to the lowest. Exhibit 2 presents a quality spectrum that provides a basis for evaluating better versus poorer quality reports. This spectrum ranges from reports that are of high financial reporting quality and reflect high and sustainable earnings quality to reports that are not useful because of poor financial reporting quality. Exhibit 2: Quality Spectrum of Financial Reports

\begin{center}
\includegraphics[max width=\textwidth]{2023_05_04_b5cfa4f1bc883752f121g-514}
\end{center}

\section{GAAP, DECISION USEFUL FINANCIAL REPORTING}
describe a spectrum for assessing financial reporting quality

At the top of the spectrum, labeled in Exhibit 2 as "GAAP, decision-useful, sustainable, and adequate returns," are high-quality reports that provide useful information about high-quality earnings.

\begin{itemize}
  \item High-quality financial reports conform to the generally accepted accounting principles (GAAP) of the jurisdiction, such as International Financial Reporting Standards (IFRS), US GAAP, or other home-country GAAP. The exhibit uses the term GAAP to refer generically to the accounting standards accepted in a company's jurisdiction.

  \item In addition to conforming to GAAP, high-quality financial reports also embody the characteristics of decision-useful information such as those defined in the Conceptual Framework. ${ }^{2}$ Recall that the fundamental characteristics of useful information are relevance and faithful representation. Relevant information is defined as information that can affect a decision and encompasses the notion of materiality. (Information is considered material if

\end{itemize}

2 The characteristics of decision-useful information are identical under IFRS and US GAAP. In September 2010, the IASB adopted the Conceptual Framework for Financial Reporting in place of the Framework for the Preparation and Presentation of Financial Statements (1989). The Conceptual Framework represents the partial completion of a joint convergence project between the IASB and FASB on an updated framework. The Conceptual Framework (2010) contains two updated chapters: "The Objective of Financial Reporting" and "Qualitative Characteristics of Useful Financial Information." The remainder of the material in the Conceptual Framework is from the Framework (1989) and will be updated as the project is completed. Also in September 2010, the FASB issued Concepts Statement 8, "Conceptual Framework for Financial Reporting," to replace Concepts Statements 1 and 2. "omitting it or misstating it could influence decisions that users make on the basis of the financial information of a specific reporting entity."3) Faithful representation of economic events is complete, neutral, and free from error. The Conceptual Framework also enumerates enhancing characteristics of useful information: comparability, verifiability, timeliness, and understandability. Of course, the desirable characteristics for financial information require trade-offs. For example, financial reports must balance the aim of providing information that is produced quickly enough to be timely and thus relevant, and yet not so quickly that errors occur. Financial reports must balance the aim of providing information that is complete but not so exhaustive that immaterial information is included. High-quality information results when these and other tradeoffs are made in an unbiased, skillful manner.

\begin{itemize}
  \item High-quality earnings indicate an adequate level of return on investment and derive from activities that a company will likely be able to sustain in the future. An adequate level of return on investment exceeds the cost of the investment and also equals or exceeds the expected return. Sustainable activities and sustainable earnings are those expected to recur in the future. Sustainable earnings that provide a high return on investment contribute to higher valuation of a company and its securities.
\end{itemize}

\section{GAAP, Decision-Useful, but Sustainable?}
The next level down in Exhibit 2, "GAAP, decision-useful, but sustainable?" refers to circumstances in which high-quality reporting provides useful information, but that information reflects results or earnings that are not sustainable (lower earnings quality). The earnings may not be sustainable because the company cannot expect earnings that generate the same level of return on investment in the future or because the earnings, although replicable, will not generate sufficient return on investment to sustain the company. Earnings quality is low in both cases. Reporting can be high quality even when the economic reality being depicted is not of high quality. For example, consider a company that generates a loss, or earnings that do not provide an adequate return on investment, or earnings that resulted from non-recurring activities. The relatively undesirable economic reality could nonetheless be depicted in financial reporting that provides high-quality, decision-useful information.

Exhibit 3 presents an excerpt from the fiscal year 2014 first-quarter results of Toyota Motor Corporation, a Japanese automobile company. As highlighted by a Wall Street Journal article, ${ }^{4}$ the company sold fewer cars but reported an $88 \%$ increase in operating profits compared with the prior year, primarily because of the change in exchange rates. The weaker yen benefited Toyota both because the company manufactures more cars in Japan (compared with its competitors) and because the company sells a significant number of cars outside of Japan. Exchange rate weakening is a less sustainable source of profits than manufacturing and selling cars. In summary, this example is a case of high-quality financial reporting coupled with lower earnings quality.

3 Text from conceptual frameworks referenced in Note 4.

4 Back (2013). Exhibit 3: Excerpt from Toyota Motor Corporation's Consolidated Financial Results for FY2014 First Quarter Ending 30 June 2013

Consolidated vehicle unit sales in Japan and overseas decreased by 37 thousand units, or 1.6\%, to 2,232 thousand units in FY2014 first quarter (the three months ended June 30, 2013) compared with FY2013 first quarter (the three months ended June 30, 2012). Vehicle unit sales in Japan decreased by 51 thousand units, or 8.8\%, to 526 thousand units in FY2014 first quarter compared with FY2013 first quarter. Meanwhile, overseas vehicle unit sales increased by 14 thousand units, or $0.8 \%$, to 1,706 thousand units in FY2014 first quarter compared with FY2013 first quarter.

As for the results of operations, net revenues increased by 753.7 billion yen, or 13.7\%, to 6,255.3 billion yen in FY2014 first quarter compared with FY2013 first quarter, and operating income increased by 310.2 billion yen, or 87.9\%, to 663.3 billion yen in FY2014 first quarter compared with FY2013 first quarter. The factors contributing to an increase in operating income were the effects of changes in exchange rates of 260.0 billion yen, cost reduction efforts of 70.0 billion yen, marketing efforts of 30.0 billion yen and other factors of 10.2 billion yen. On the other hand, the factors contributing to a decrease in operating income were the increase in expenses and others of 60.0 billion yen.

\section*{BIASED ACCOUNTING CHOICES }
The next level down in the spectrum in Exhibit 2 is "Within GAAP, but biased choices." Biased choices result in financial reports that do not faithfully represent the economic substance of what is being reported. The problem with bias in financial reporting, as with other deficiencies in reporting quality, is that it impedes an investor's ability to correctly assess a company's past performance, to accurately forecast future performance, and thus to appropriately value the company.

Choices are deemed to be "aggressive" if they increase a company's reported performance and financial position in the period under review. The choice can increase the amount of revenues, earnings, and/or operating cash flow reported for the period, or decrease expenses, and/or reduce the level of debt reported on the balance sheet. Aggressive choices may lead to a reduction in the company's reported performance and in its financial position in later periods. In contrast, choices are deemed "conservative" if they decrease a company's performance and financial position in the reporting period. This can include lowering the reported revenues, earnings, and/or operating cash flow reported or increasing expenses, or recording a higher level of debt on the balance sheet. Conservative choices may lead to a rise in the company's reported performance and financial position in later periods.

Another type of bias is understatement of earnings volatility, so-called earnings "smoothing". Earnings smoothing can result from conservative choices to understate earnings in periods when a company's operations are performing well, building up (often hidden) reserves that allow aggressive choices in periods when its operations are struggling. Biased choices can be made not only in the context of reported amounts but also in the context of how information is presented. For example, companies can disclose information transparently, which facilitates analysis, or they can disclose it in a manner that aims to obscure unfavorable and/or emphasize favorable information.

\section{EXAMPLE 1}
\section{Quality of Financial Reports}
PACCAR Inc. designs, manufactures, and distributes trucks and related aftermarket parts that are sold worldwide under the Kenworth, Peterbilt, and DAF nameplates. In 2013, the US SEC charged PACCAR for various accounting deficiencies that "clouded their financial reporting to investors in the midst of the financial crisis." The SEC complaint cites the company's 2009 segment reporting. Exhibit 4 presents an excerpt from the notes to PACCAR's financial statements, and Exhibit 5 presents an excerpt from the management's discussion and analysis (MD\&A) of PACCAR's annual report.

\section{Exhibit 4: Excerpt from Notes to PACCAR's 2009 Financial Statements}
\section{S. SEGMENT AND RELATED INFORMATION}
PACCAR operates in two principal segments, Truck and Financial Services.

The Truck segment includes the manufacture of trucks and the distribution of related aftermarket parts, both of which are sold through a network of independent dealers... The Financial Services segment is composed of finance and leasing products and services provided to truck customers and dealers ... Included in All Other is PACCAR's industrial winch manufacturing business. Also within this category are other sales, income and expenses not attributable to a reportable segment, including a portion of corporate expense.

Business Segment Data (\$ millions)

\begin{center}
\begin{tabular}{lccc}
\hline
 & $\mathbf{2 0 0 9}$ & $\mathbf{2 0 0 8}$ & $\mathbf{2 0 0 7}$ \\
\hline
Income before Income Taxes &  &  &  \\
Truck & $\$ 25.9$ & $\$ 1,156.5$ & $\$ 1,352.8$ \\
All other & 42.2 & 6.0 & 32.0 \\
\cline { 2 - 4 }
 & 68.1 & $1,162.5$ & $1,384.8$ \\
Financial services & 84.6 & 216.9 & 284.1 \\
Investment income & 22.3 & 84.6 & 95.4 \\
\cline { 2 - 4 }
 & $\$ 175.0$ & $\$ 1,464.0$ & $\$ 1,764.3$ \\
\hline
\end{tabular}
\end{center}

\section{Exhibit 5: Excerpt from MD\&A of PACCAR's 2009 Annual Report}
Net sales and revenues and gross margins for truck units and aftermarket parts are provided below. The aftermarket parts gross margin includes direct revenues and costs, but excludes certain truck segment costs.

\begin{center}
\begin{tabular}{lccc}
\hline
 & $\mathbf{2 0 0 9}$ & $\mathbf{2 0 0 8}$ & \% Change \\
\hline
Net Sales and Revenues &  &  &  \\
Trucks & $\$ 5,103.30$ & $\$ 11,281.30$ & -55 \\
Aftermarket parts & $1,890.70$ & $2,266.10$ & -17 \\
 & $\$ 6,994.00$ & $\$ 13,547.40$ & -48 \\
Gross Margin &  &  &  \\
Trucks & $-\$ 46.6$ & $\$ 1,141.70$ & -104 \\
Aftermarket parts & 625.7 & 795.20 & -21 \\
 & $\$ 579.1$ & $\$ 1,936.90$ & -70 \\
\hline
\end{tabular}
\end{center}

\begin{enumerate}
  \item Based on the segment data excerpted from the notes to the financial statements, was PACCAR's truck segment profitable in 2009?
\end{enumerate}

\section{Solution to 1:}
Yes, the segment data presented in the note to the financial statements indicates that the Truck segment earned \$25.9 million in 2009.

\begin{enumerate}
  \setcounter{enumi}{1}
  \item Based on the data about the truck's gross margin presented in the MD\&A, was PACCAR's truck segment profitable in 2009?
\end{enumerate}

\section{Solution to 2:}
No, the segment data presented in the MD\&A indicates that the Truck segment had a negative gross margin.

\begin{enumerate}
  \setcounter{enumi}{2}
  \item What is the main difference between the note presentation and the MD\&A presentation?
\end{enumerate}

\section{Solution to 3:}
The main difference between the note presentation and the MD\&A presentation is that the aftermarket parts business is combined with the trucks business in the notes but separated in the MD\&A. Although the data are not exactly comparable in the two disclosures (because the note shows income before taxes and the MD\&A shows gross profit), the two disclosures present a different picture of PACCAR's profits from truck sales.

\begin{enumerate}
  \setcounter{enumi}{3}
  \item The SEC complaint stated that "PACCAR failed to report the operating results of its aftermarket parts business separately from its truck sales business as required under segment reporting requirements, which are in place to ensure that investors gain the same insight into a company as its executives." Is the PACCAR situation an example of issues with financial reporting quality, earnings quality, or both?
\end{enumerate}

\section{Solution to 4:}
The PACCAR situation appears to be an example of issues with both financial reporting quality and earnings quality. The substantial decrease in truck sales and the negative gross margin reflect poor earnings quality. The failure to disclose clear segment information is an instance of poor financial reporting quality. While choices exist within GAAP for the presentation of a desired economic picture, non-GAAP reporting adds yet another dimension of management discretion. Non-GAAP reporting of financial metrics not in compliance with generally accepted accounting principles such as US GAAP and IFRS includes both financial metrics and operating metrics. ${ }^{5}$ Non-GAAP financial metrics relate directly to the financial statements. A common non-GAAP financial metric is "non-GAAP earnings," which are created by companies "that adjust standards-compliant earnings to exclude items required by accounting standards or to include items not permitted by accounting standards" (Ciesielski and Henry, 2017). In contrast, non-GAAP operating metrics do not relate directly to the financial statements and include metrics that are typically industry-driven, such as subscriber numbers, active users, and occupancy rates.

Non-GAAP financial reporting has become increasingly common, presenting challenges to analysts. An important challenge is that non-GAAP financial reporting diminishes comparability across financial statements. The adjustments that companies make to create non-GAAP earnings, for example, are generally ad hoc and thus differ significantly. When evaluating non-GAAP metrics, investors must decide the extent to which specific adjustments should be incorporated into their analyses and forecasts. ${ }^{6}$

Another challenge arises from differences in terminology. Non-GAAP earnings are sometimes referred to as underlying earnings, adjusted earnings, recurring earnings, core earnings, or similar. Exhibit 6 provides an example from Jaguar Land Rover Automotive plc (JLR), a subsidiary of Tata Motors Ltd. The company prepares its financial reports under IFRS. The exhibit is an excerpt from JLR's 2016/17 annual report and uses the term "alternative performance measures". Exhibit 7 is from Tata Motors Ltd's Form 6-K filed with the US SEC, containing supplemental information regarding JLR and using the term "non-IFRS Financial Measures". The information in the two exhibits is essentially identical, but the terminology and formatting differ.

\section{Exhibit 6}
JLR's 2016/17 Annual Report: Footnote 3 [Excerpt]

\section{3) ALTERNATIVE PERFORMANCE MEASURES}
Many companies use alternative performance measures (APMs) to provide helpful additional information for users of their financial statements, telling a clearer story of how the business has performed over the period... These measures exclude certain items that are included in comparable statutory measures....

Reconciliations between these alternative performance measures and statutory reported measures are shown below.

\section{EBIT AND EBITDA ( $\mathbf{E m})$}
\begin{center}
\begin{tabular}{lc}
\hline
Year ended 31 March & 2017 \\
\hline
EBITDA & 2,955 \\
Depreciation and amortisation & $-1,656$ \\
Share of profit/(loss) of equity accounted investments & 159 \\
EBIT & $\mathbf{1 , 4 5 8}$ \\
\end{tabular}
\end{center}

5 The term "non-GAAP" refers generally to all metrics that are non-compliant with generally accepted accounting principles and thus includes "non-IFRS" metrics.

6 A survey of non-GAAP earnings in the S\&P 500 is presented in Ciesielski and Henry (2017). Some observers even recommend that investors shift their focus from a company's earnings to a company's "strategic assets" and the contribution of these assets to its competitive edge (Gu and Lev, 2017).

\begin{center}
\begin{tabular}{lc}
Year ended 31 March & $\mathbf{2 0 1 7}$ \\
\hline
Foreign exchange (loss)/gain on derivatives & -11 \\
Unrealised gain/(loss) on commodities & 148 \\
Foreign exchange loss on loans & -101 \\
Finance income & 33 \\
Finance expense (net) & -68 \\
Exceptional item & 151 \\
Profit before tax & $\mathbf{1 , 6 1 0}$ \\
\hline
\end{tabular}
\end{center}

\section{Exhibit 7}
\section{Tata Motors Ltd. SEC Form 6-K [Excerpt]}
\section{Non-IFRS Financial Measures}
This Report includes references to certain non-IFRS measures, including EBITDA, EBIT ... [These measures] and related ratios should not be considered in isolation and are not measures of JLR's financial performance or liquidity under IFRS and should not be considered as an alternative to profit or loss for the period or any other performance measures derived in accordance with IFRS or as an alternative to cash flow from operating, investing or financing activities or any other measure of JLR's liquidity derived in accordance with IFRS. ... In addition, EBITDA, EBIT.... as defined, may not be comparable to other similarly titled measures used by other companies.

\section{Exhibit 8: to Form 6-K Supplemental Information Regarding the Jaguar and}
 Land Rover Business of Tata Motors Limited [Excerpt from Exhibit 1]The reconciliation of JLR's EBIT and EBITDA to profit for the period line item is:

\begin{center}
\begin{tabular}{lc}
\hline
Fiscal year ended March $\mathbf{3 1}, \mathbf{2 0 1 7}$ & $\mathbf{E m}$ \\
\hline
Profit for the period & 1,272 \\
Add back taxation & 338 \\
Add/(less) back exceptional charge/(credit) & -151 \\
Add back/(less) foreign exchange (gains)/loss - financing & 101 \\
Add back/(less) foreign exchange (gains)/loss - derivatives & 11 \\
Add back/(less) unrealized commodity losses/(gains) - unrealized derivatives & -148 \\
Less finance income & -33 \\
Add back finance expense (net) & 68 \\
EBIT & $\mathbf{1 , 4 5 8}$ \\
Add back depreciation and amortization & 1,656 \\
Add/(less) back share of loss/(profit) from equity accounted investees & -159 \\
EBITDA & 2,955 \\
\hline
\end{tabular}
\end{center}

Management emphasis on non-GAAP financial measures to deflect attention from less-than-desirable GAAP financial results is an example of an aggressive presentation choice. Since 2003, if a company uses a non-GAAP financial measure ${ }^{7}$ in an SEC filing, it is required to display the most directly comparable GAAP measure with equal prominence and to provide a reconciliation between the non-GAAP measure and the equivalent GAAP measure. In other words, a company is not allowed to give more prominence to a non-GAAP financial measure in an SEC filing.

Similarly, the IFRS Practice Statement "Management Commentary," issued December 2010, requires disclosures when non-IFRS measures are included in financial reports:

If information from the financial statements has been adjusted for inclusion in management commentary, that fact should be disclosed. If financial performance measures that are not required or defined by IFRSs are included within management commentary, those measures should be defined and explained, including an explanation of the relevance of the measure to users. When financial performance measures are derived or drawn from the financial statements, those measures should be reconciled to measures presented in the financial statements that have been prepared in accordance with IFRSs. (Page 17)

The reconciliation between as-reported measures (GAAP financial measures presented in the financial statements) and as-adjusted measures (non-GAAP financial measures presented in places other than the financial statements) can provide important information.

The European Securities and Markets Authority (ESMA) published guidelines in October 2015 (ESMA Guidelines on Alternative Performance Measures) covering such points as the definition of APMs, reconciliation to GAAP, explanation of the metrics' relevance, and consistency over time. We discuss ESMA in more detail later in this reading.

\section{EXAMPLE 2}
\section{Presentation of Non-GAAP Financial Measures}
Convatec Group PLC (Convatec), a global medical products manufacturer, raised $\$ 1.8$ billion via an initial public offering (IPO) on the London Stock Exchange in 2016. The company had been purchased by private equity firms from Bristol-Myers Squibb in 2008 for $\$ 4.1$ billion. Exhibit 7 presents excerpts from the company's regulatory filing at the London Stock Exchange announcing its full year 2016 results.

\section{Exhibit 9: Excerpt from ConvaTec's Press Release for Full Year 2016}
Results

Headline: "Strong results, delivering on strategy"

CEO Review [Excerpt]

At constant currency, revenue grew $4 \%$ to $\$ 1,688$ million and adjusted EBITDA was $\$ 508$ million, up 6.5\% at constant currency...

[Footnote] Constant currency growth 'CER' is calculated by restating 2016 results using 2015 foreign exchange rates for the relevant period.

7 Non-domestic private issuers can file financial statements prepared in accordance with IFRS without reconciliation to US GAAP. The SEC recognizes US GAAP and IFRS as GAAP. Consolidated Statement of Profit or Loss for the year ended 31 December 2016 ( $\$ \mathrm{~m})$

\begin{center}
\begin{tabular}{lcc}
\hline
 & $\mathbf{2 0 1 6}$ & $\mathbf{2 0 1 5}$ \\
\hline
Revenue & $1,688.3$ & $1,650.4$ \\
Cost of goods sold & -821.0 & -799.9 \\
\cline { 2 - 3 }
Gross profit & 867.3 & 850.5 \\
Selling and distribution expenses & -357.0 & -346.7 \\
General and administrative expenses & -318.2 & -233.1 \\
Research and development expenses & -38.1 & -40.3 \\
\cline { 2 - 3 }
profit & 154.0 & 230.4 \\
Finance costs & -271.4 & -303.6 \\
Other expense, net & -8.4 & -37.1 \\
Loss before income taxes & -125.8 & -110.3 \\
Income tax (expense) benefit & -77.0 & 16.9 \\
\cline { 2 - 3 }
Net loss & -202.8 & -93.4 \\
\hline
\end{tabular}
\end{center}

\section{Non-IFRS Financial Information [Excerpt]}
This release contains certain financial measures that are not defined or recognised under IFRS. These measures are referred to as "Adjusted" measures... These measures are not measurements of financial performance or liquidity under IFRS and should not replace measures of liquidity or operating profit that are derived in accordance with IFRS.

\section{Reconciliation to adjusted earnings [Excerpt]}
\begin{center}
\begin{tabular}{|c|c|c|c|c|c|c|c|c|c|}
\hline
2016 & Reported & (a) & (b) & (c) & (d) & (e) & (f) & (g) & Adjusted \\
\hline
Revenue & $1,688.3$ & - & - & - & - & - & - & - & $1,688.3$ \\
\hline
\multicolumn{10}{|l|}{$\ldots$} \\
\hline
Operating profit & 154.0 & 155.1 & 30.9 & 11.7 & 0.8 & - & 90.2 & 29.5 & 472.2 \\
\hline
\multicolumn{10}{|l|}{$\cdots$} \\
\hline
$\begin{array}{l}\text { (Loss) profit before income } \\ \text { taxes }\end{array}$ & -125.8 & 155.1 & 30.9 & 11.7 & 0.8 & 37.6 & 90.2 & 29.5 & 230.0 \\
\hline
Income tax $^{\text {expense }^{(\mathrm{h})}}$ & -77.0 &  &  &  &  &  &  &  & -51.2 \\
\hline
Net (loss) profit & -202.8 &  &  &  &  &  &  &  & 178.8 \\
\hline
\end{tabular}
\end{center}

(a) Represents an adjustment to exclude (i) acquisition-related amortisation expense ... (ii) accelerated depreciation ...related to the closure of certain manufacturing facilities, and (iii) impairment charges and assets write offs related to property, plant and equipment and intangible assets ....

(b) Represents restructuring costs and other related costs ...

(c) Represents remediation costs which include regulatory compliance costs related to FDA activities, IT enhancement costs, and professional service fees associated with activities that were undertaken in respect of the Group's compliance function and to strengthen its control environment within finance.

(d) Represents costs primarily related to (i) corporate development activities and (ii) a settlement of ordinary course multi-year patent-related litigations in 2015 ....

(e) Represents adjustments to exclude (i) loss on extinguishment of debt and write off of deferred financing fees ... and (ii) foreign exchange related transactions.

(f) Represents an adjustment to exclude (i) share-based compensation expense ... arising from pre-IPO employee equity grants and (ii) pre-IPO ownership structure related costs, including management fees to Nordic Capital and Avista (refer to Note 6 Related Party Transactions for further information).

(g) Represents IPO related costs, primary advisory fees.

(h) Adjusted income tax expense/benefit is income tax (expense) benefit net of tax adjustments.

\section{Adjusted EBITDA [Excerpt]}
Adjusted EBITDA is defined as Adjusted EBIT...further adjusted to exclude (i) software and R\&D amortisation, (ii) depreciation and (iii) post-IPO share-based compensation.

The following table reconciles the Group's Adjusted EBIT to Adjusted EBITDA.

\begin{center}
\begin{tabular}{lc}
\hline
 & $\mathbf{2 0 1 6} \mathbf{( \$ \mathbf { m } )}$ \\
\hline
Adjusted EBIT & 472.2 \\
Software and R\&D amortization & 6.7 \\
Depreciation & 27.9 \\
Post-IPO share-based compensation & 0.8 \\
\hline
Adjusted EBITDA & 507.6 \\
\hline
\end{tabular}
\end{center}

\begin{enumerate}
  \item Based on the information provided, explain the differences between the following two disclosures contained in Convatec's press release:
\end{enumerate}

A. The CEO Review of 2016 results, at the beginning of the release, states that "revenue grew $4 \%$ to $\$ 1,688$ million."

B. Convatec's Consolidated Statement of Profit or Loss shows 2016 revenues of $\$ 1,688.3$ million and 2015 revenues of $\$ 1,650.4$ million.

\section{Solution to 1:}
The amount of revenue reported on the company's income statement conforms to International Financial Reporting Standards (IFRS). Using the amounts from the income statement, the company's total revenue increased by $2.3 \%(=\$ 1,688.3 / \$ 1,650.4-1)$. The revenue growth rate of $4 \%$ in the CEO review is a non-IFRS measure, calculated on a "constant currency" basis, which the footnote describes as a comparison using 2016 revenues restated at 2015 foreign exchange rates.

\begin{enumerate}
  \setcounter{enumi}{1}
  \item Based on the information provided, explain the differences between the following two disclosures contained in Convatecs's earnings release:
\end{enumerate}

A. The CEO Review of 2016 results states that "adjusted EBITDA was \$508 million, up $6.5 \%$ at constant currency."

B. Convatec's Consolidated Statement of Profit or Loss shows 2016 net loss of $\$ 202.8$ million and 2015 net loss of $\$ 93.4$ million.

\section{Solution to 2:}
The amounts reported on the company's income statement conform to IFRS. Using amounts from the income statement, the company reported a loss in 2016 of $\$ 202.8$ million, which was more than twice as large a loss as the $\$ 93.4$ million loss reported in 2015. Also referring to the income statement, the company reported 2016 operating profit (referred to elsewhere as EBIT) of $\$ 154.0$ million, a decline of $33.2 \%$ from the $\$ 230.4$ million operating profit reported in 2016.

In contrast, the "Adjusted EBITDA" amount highlighted in the CEO Review is neither defined nor recognized under IFRS. It is a non-IFRS measure. To create the Adjusted EBITDA, the company first begins with EBIT (called Operating profit in excerpts II and III) of $\$ 154.0$ and creates Adjusted EBIT (\$472.2 million) by adding back 8 different expenses that IFRS requires the company to recognize. These adjustments are listed beneath the first tabular reconciliation in Items a through g. After developing Adjusted EBIT, the company creates Adjusted EBITDA (\$507.6 million) by adding back a further 3 different expenses that IFRS requires the company to recognize.

Overall, there are three key differences between Disclosures A and B: (1) Most importantly, disclosure A refers to a non-IFRS metric rather than an IFRS-compliant metric; (2) Disclosure A refers to operating profit, which was positive, rather than to net income, which was negative; and (3) Disclosure A highlights a positive economic outcome-i.e., an increase, on a currency adjusted basis. An analyst should be aware of the alternative means by which earnings announcements can paint a positive picture of companies' results.

Often, poor reporting quality occurs simultaneously with poor earnings quality; for example, aggressive accounting choices are made to obscure poor performance. It is also possible, of course, for poor reporting quality to occur with high-quality earnings. Although a company with good performance would not require aggressive accounting choices to obscure poor performance, it might nonetheless produce poor-quality reports for other reasons. A company with good performance might be unable to produce high-quality reports because of inadequate internal systems.

Another scenario in which poor reporting quality might occur simultaneously with high quality earnings is that a company with good performance might deliberately produce reports based on "conservative" rather than aggressive accounting choicesthat is, choices that make current performance look worse. One motivation might be to avoid unwanted political attention. Another motivation could arise in a period in which management had already exceeded targets before the end of the period and thus made conservative accounting choices that would delay reporting profits until the following period (so-called "hidden reserves"). Similar motivations might also contribute to accounting choices that create the appearance that the trajectory of future results would appear more attractive. For example, a company might make choices to accelerate losses in the first year of an acquisition or the first year of a new CEO's tenure so that the trajectory of future results would appear more attractive.

Overall, unbiased financial reporting is the ideal. Some investors may prefer conservative choices rather than aggressive ones, however, because a positive surprise is easier to tolerate than a negative surprise. Biased reporting, whether conservative or aggressive, adversely affects a user's ability to assess a company.

The quality spectrum considers the more intuitive situation in which less-than-desired underlying economics are the central motivation for poor reporting quality. In addition, it is necessary to have some degree of reporting quality in order to evaluate earnings quality. Proceeding down the spectrum, therefore, the concepts of reporting quality and earnings quality become progressively less distinguishable.

\section{Within GAAP, but "Earnings Management"}
The next level down on the spectrum in Exhibit 2 is labeled "Within GAAP, but 'earnings management." The term "earnings management" is defined here as making intentional choices that create biased financial reports. ${ }^{8}$ The distinction between earnings management and biased choices is subtle and, primarily, a matter of intent. Earnings management represents "deliberate actions to influence reported earnings

8 Various definitions have appeared in academic research. Closest to the discussion here is Schipper (1989), which uses the term "earnings management" to mean "'disclosure management' in the sense of a purposeful intervention in the external financial reporting process, with the intent of obtaining some private gain (as opposed to, say, merely facilitating the neutral operation of the process)." and their interpretation" (Ronen and Yaari, 2008). Earnings can be "managed" upward (increased) by taking real actions, such as deferring research and development (R\&D) expenses into the next reporting period. Alternatively, earnings can be increased by accounting choices, such as changing accounting estimates. For example, the amount of estimated product returns, bad debt expense, or asset impairment could be decreased to create higher earnings. Because it is difficult to determine intent, we include earnings management under the biased choices discussion.

\section{DEPARTURES FROM GAAP}
describe a spectrum for assessing financial reporting quality

The next levels down on the spectrum in Exhibit 2 mark departures from GAAP. Financial reporting that departs from GAAP can generally be considered low quality. In such situations, earnings quality is likely difficult or impossible to assess because comparisons with earlier periods and/or other entities cannot be made. An example of improper accounting was Enron (accounting issues revealed in 2001), whose inappropriate use of off-balance-sheet structures and other complex transactions resulted in vastly understated indebtedness as well as overstated profits and operating cash flow. Another notorious example of improper accounting was WorldCom (accounting issues discovered in 2002), a company that by improperly capitalizing certain expenditures dramatically understated its expenses and thus overstated its profits. More recently, New Century Financial (accounting issues revealed in 2007) issued billions of dollars of subprime mortgages and improperly reserved only minimal amounts for loan repurchase losses. Each of these companies subsequently filed for bankruptcy.

In the 1980s, Polly Peck International (PPI) reported currency losses, incurred in the normal course of operations, directly through equity rather than in its profit and loss statements. In the 1990 s, Sunbeam improperly reported revenues from "bill-and-hold" sales and also manipulated the timing of expenses in an effort to falsely portray outstanding performance of its then-new chief executive.

At the bottom of the quality spectrum, fabricated reports portray fictitious events, either to fraudulently obtain investments by misrepresenting the company's performance and/or to obscure fraudulent misappropriation of the company's assets. Examples of fraudulent reporting are unfortunately easy to find, although they were not necessarily easy to identify at the time. In the 1970s, Equity Funding Corp. created fictitious revenues and even fictitious policyholders. In the 1980s, Crazy Eddie's reported fictitious inventory as well as fictitious revenues supported by fake invoices. In 2004, Parmalat reported fictitious bank balances.

\section{EXAMPLE 3}
\section{Spectrum for Assessing Quality of Financial Reports}
Jake Lake, a financial analyst, has identified several items in the financial reports of several (hypothetical) companies. Describe each of these items in the context of the financial reporting quality spectrum.

\begin{enumerate}
  \item ABC Co.'s 2018 earnings totaled $\$ 233$ million, including a $\$ 100$ million gain from selling one of its less profitable divisions. ABC's earnings for the prior three years totaled $\$ 120$ million, $\$ 107$ million, and $\$ 111$ million. The com- pany's financial reports are extremely clear and detailed, and the company's earnings announcement highlights the one-time nature of the $\$ 100$ million gain.
\end{enumerate}

\section{Solution to 1:}
ABC's 2018 total earnings quality can be viewed as low because nearly half of the earnings are derived from a non-sustainable activity, namely the sale of a division. ABC's 2018 quality of earnings from continuing operations may be high because the amounts are fairly consistent from year to year, although an analyst would undertake further analysis to confirm earnings quality. In general, a user of financial reports should look beyond the bottom-line net income. The description provided suggests that the company's reporting quality is high; the reports are clear and detailed, and the one-time nature of the $\$ 100$ million gain is highlighted.

\begin{enumerate}
  \setcounter{enumi}{1}
  \item DEF Co. discloses that in 2018, it changed the depreciable life of its equipment from 3 years to 15 years. Equipment represents a substantial component of the company's assets. The company's disclosures indicate that the change is permissible under the accounting standards of its jurisdiction but provide only limited explanation of the change.
\end{enumerate}

\section{Solution to 2:}
DEF's accounting choice appears to be within permissible accounting standards, but its effect is to substantially lower depreciation expense and thus to increase earnings for the year. The quality of reported earnings is questionable. Although the new level of earnings may be sustainable, similar increases in earnings for future periods might not be achievable, because increasing earnings solely by changing accounting estimates is likely not sustainable. In addition, the description provided suggests that the company's reporting quality is low because it offers only a limited explanation for the change.

\begin{enumerate}
  \setcounter{enumi}{2}
  \item GHI Co's R\&D expenditures for the past five years have been approximately $3 \%$ of sales. In 2018, the company significantly reduced its R\&D expenditures. Without the reduction in R\&D expenditures, the company would have reported a loss. No explanation is disclosed.
\end{enumerate}

\section{Solution to 3:}
GHI's operational choice to reduce its R\&D may reflect real earnings management because the change enabled the company to avoid reporting a loss. In addition, the description provided suggests that the company's reporting quality is low because it does not offer an explanation for the change.

\section{DIFFERENTIATE BETWEEN CONSERVATIVE AND AGGRESSIVE ACCOUNTING}
explain the difference between conservative and aggressive accounting This section returns to the implications of conservative and aggressive accounting choices. As mentioned earlier, unbiased financial reporting is the ideal. But some investors may prefer or be perceived to prefer conservative rather than aggressive accounting choices because a positive surprise is acceptable. In contrast, management may make, or be perceived to make, aggressive accounting choices because they increase the company's reported performance and financial position.

Aggressive accounting choices in the period under review may decrease the company's reported performance and financial position in later periods, which creates a sustainability issue. Conservative choices do not typically create a sustainability issue because they decrease the company's reported performance and financial position, and may increase them in later periods. In terms of establishing expectations for the future, however, financial reporting that is relevant and faithfully representative is the most useful.

A common presumption is that financial reports are typically biased upward, but that is not always the case. Although accounting standards ideally promote unbiased financial reporting, some accounting standards may specifically require a conservative treatment of a transaction or an event. Also, managers may choose to take a conservative approach when applying standards. It is important that an analyst consider the possibility of conservative choices and their effects.

At its most extreme, conservatism follows accounting practices that "anticipate no profit, but anticipate all losses" (Bliss, 1924). But in general, conservatism means that revenues may be recognized once a verifiable and legally enforceable receivable has been generated and that losses need not be recognized until it becomes "probable" that an actual loss will be incurred. Conservatism is not an absolute but is characterized by degrees, such as "the accountant's tendency to require a higher degree of verification to recognize good news as gains than to recognize bad news as losses" (Basu, 1997). From this perspective, "verification" (e.g., physical existence of inventories, evidence of costs incurred or to be incurred, or establishment of rights and obligations on legal grounds) drives the degree of conservatism. For recognition of revenues, a higher degree of verification would be required than for expenses.

\section{Conservatism in Accounting Standards}
The Conceptual Framework supports neutrality of information: "A neutral depiction is without bias in the selection or presentation of financial information."9 Neutralitylack of upward or downward bias-is considered a desirable characteristic of financial reporting. Conservatism directly conflicts with the characteristic of neutrality because the asymmetric nature of conservatism leads to bias in measuring assets and liabilities-and ultimately, earnings.

Despite efforts to support neutrality in financial reporting, many conservatively biased standards remain. Standards across jurisdictions may differ on the extent of conservatism embedded within them. An analyst should be aware of the implications of accounting standards for the financial reports. An example is the different treatment by IFRS and US GAAP of the impairment of long-lived assets. ${ }^{10}$ Both IFRS and US GAAP specify an impairment analysis protocol that begins with an assessment of whether recent events indicate that the economic benefit from an individual or group of long-lived assets may be less than its carrying amount(s). From that point on, however, the two regimes diverge:

\begin{itemize}
  \item Under IFRS, if the "recoverable amount" (the higher of fair value less costs to sell and value in use) is less than the carrying amount, then an impairment charge will be recorded.

  \item Under US GAAP, an impairment charge will be recorded only when the sum of the undiscounted future cash flows expected to be derived from the asset(s) is less than the carrying amount(s). If the undiscounted future cash flows are less than the carrying amount, the asset is written down to fair value.

\end{itemize}

To illustrate the difference in application, assume that a factory is the unit of account eligible for impairment testing. Its carrying amount is $\$ 10,000,000$; "fair value" and "recoverable amount" are both $\$ 6,000,000$; and the undiscounted future net cash flows associated with the factory total $\$ 10,000,000$. Under IFRS, an impairment charge of $\$ 4,000,000$ would be recorded; but under US GAAP, no impairment charge would be recognized.

Thus, on its face, IFRS would be regarded as more conservative than US GAAP because impairment losses would normally be recognized earlier under IFRS than under US GAAP. But, taking the analysis one step further, such a broad generalization may not hold up. For example, if an asset is impaired under both IFRS and US GAAP and the asset's value in use exceeds its fair value, the impairment loss under US GAAP will be greater. Also, IFRS permits the recognition of recoveries of the recoverable amount in subsequent periods if evidence indicates that the recoverable amount has subsequently increased. In contrast, US GAAP prohibits the subsequent write-up of an asset after an impairment charge has been taken; it would recognize the asset's increased value only when the asset is ultimately sold.

Common examples of conservatism in accounting standards include the following:

\begin{itemize}
  \item Research costs. Because the future benefit of research costs is uncertain at the time the costs are incurred, both US GAAP and IFRS require immediate expensing instead of capitalization.

  \item Litigation losses. When it becomes "probable" that a cost will be incurred, both US GAAP and IFRS require expense recognition, even though a legal liability may not be incurred until a future date.

  \item Insurance recoverables. Generally, a company that receives payment on an insurance claim may not recognize a receivable until the insurance company acknowledges the validity of the claimed amount.

\end{itemize}

Watts (2003) reviews empirical studies of conservatism, and identifies four potential benefits of conservatism:

\begin{itemize}
  \item Given asymmetrical information, conservatism may protect the contracting parties with less information and greater risk. This protection is necessary because the contracting party may be at a disadvantage. For example, corporations that access debt markets have limited liability, and lenders thus have limited recourse to recover their losses from shareholders. As another example, executives who receive earnings-based bonuses might not be subject to having those bonuses "clawed back" if earnings are subsequently discovered to be overstated.
\end{itemize}

\begin{center}
\includegraphics[max width=\textwidth]{2023_05_04_b5cfa4f1bc883752f121g-528}
\end{center}

\begin{itemize}
  \item Conservatism reduces the possibility of litigation and, by extension, litigation costs. Rarely, if ever, is a company sued because it understated good news or overstated bad news.

  \item Conservative rules may protect the interests of regulators and politicians by reducing the possibility that fault will be found with them if companies overstate earnings or assets.

  \item In many tax jurisdictions, financial and tax reporting rules are linked. For example, in Germany and Japan, only deductions taken against reported income can be deducted against taxable income. Hence, companies can reduce the present value of their tax payments by electing conservative accounting policies for certain types of events.

\end{itemize}

Analysts should consider possible conservative and aggressive biases and their consequences when examining financial reports. Current-period financial reports may be unbiased, upward biased through aggressive accounting choices, downward biased through conservative accounting choices, or biased through a combination of conservative and aggressive accounting choices.

\section{Bias in the Application of Accounting Standards}
Any application of accounting standards, whether the standard itself is neutral or not, often requires significant amounts of judgment. Characterizing the application of an accounting standard as conservative or aggressive is more a matter of intent rather than definition.

Careful analysis of disclosures, facts, and circumstances contributes to making an accurate inference of intent. Management seeking to manipulate earnings may take a longer view by sacrificing short-term profitability in order to ensure higher profits in later periods. One example of biased accounting in the guise of conservatism is the so-called "big bath" restructuring charges. Both US GAAP and IFRS provide for accrual of future costs associated with restructurings, and these costs are often associated with and presented along with asset impairments. But in some instances, companies use the accounting provisions to estimate "big" losses in the period under review so that performance in future periods will appear better. Having observed numerous instances of manipulative practices in the late 1990s, in which US companies set up opportunities to report higher profits in future periods that were not connected with performance in those periods, the SEC staff issued rules that narrowed the circumstances under which costs can be categorized as part of a "non-recurring" restructuring event and enhanced the transparency surrounding restructuring charges and asset impairments. ${ }^{11}$

A similar manifestation of "big bath" accounting is often referred to as "cookie jar reserve accounting." Both US GAAP and IFRS require accruals of estimates of future non-payments of loans. In his 1998 speech "The 'Numbers Game," SEC chair Arthur Levitt expressed the general concern that corporations were overstating loans and other forms of loss allowances for the purpose of smoothing income over time. ${ }^{12}$ In 2003, the SEC issued interpretive guidance that essentially requires a company to provide a separate section in management's discussion and analysis (MD\&A) titled "Critical Accounting Estimates." 13 If the effects of subjective estimates and judgments of highly uncertain matters are material to stakeholders (investors, customers, suppliers, and

11 SEC, "Restructuring and Impairment Charges," Staff Accounting Bulletin (SAB) No. 100 (1999): www. \href{http://sec.gov/interps/account/sab100.htm}{sec.gov/interps/account/sab100.htm}.

12 Arthur Levitt, "The 'Numbers Game," Remarks given at NYU Center for Law and Business (28 September 1998): \href{http://www.sec.gov/news/speech/speecharchive/1998/spch220.txt}{www.sec.gov/news/speech/speecharchive/1998/spch220.txt}.

13 SEC, "Commission Guidance Regarding Management's Discussion and Analysis of Financial Condition and Results of Operations," Financial Reporting Release (FRR) No. 72 (2003): \href{http://www.sec.gov/rules/interp/33-8350}{www.sec.gov/rules/interp/33-8350}. htm. other users of the financial statements), disclosures of their nature and exposure to uncertainty should be made in the MD\&A. This requirement is in addition to required disclosures in the notes to the financial statements.

\begin{center}
\includegraphics[max width=\textwidth]{2023_05_04_b5cfa4f1bc883752f121g-530}
\end{center}

\section{CONTEXT FOR ASSESSING FINANCIAL REPORTING QUALITY}
describe motivations that might cause management to issue financial reports that are not high quality

describe conditions that are conducive to issuing low-quality, or even fraudulent, financial reports

In assessing financial reporting quality, it is useful to consider whether a company's managers may be motivated to issue financial reports that are not high quality. If motivation exists, an analyst should consider whether the reporting environment is conducive to managers' misreporting. It is important to consider mechanisms within the reporting environment that discipline financial reporting quality, such as the regulatory regime.

\section{Motivations}
Managers may be motivated to issue financial reports that are not high quality to mask poor performance, such as loss of market share or lower profitability than competitors. Lewis (2012) stated, "A firm experiencing performance problems, particularly those it considers transient, may induce a response that inflates current earnings numbers in exchange for lower future earnings."

\begin{itemize}
  \item Even when there is no need to mask poor performance, managers frequently have incentives to meet or beat market expectations as reflected in analysts' forecasts and/or management's own forecasts. Exceeding forecasts typically increases the stock price, if only temporarily. Additionally, exceeding forecasts can increase management compensation that is linked to increases in stock price or to reported earnings. Graham, Harvey, and Rajgopal (2005) found that the CFOs they surveyed view earnings as the most important financial metric to financial markets. Achieving (or exceeding) particular benchmarks, including prior-year earnings and analysts' forecasts, is very important. The authors examined a variety of motivations for why managers might "exercise accounting discretion to achieve some desirable earnings goal." Motivations to meet earnings benchmarks include equity market effects (for example, building credibility with market participants and positively affecting stock price) and trade effects (for example, enhancing reputation with customers and suppliers). Equity market effects are the most powerful incentives, but trade effects are important, particularly for smaller companies.

  \item Career concerns and incentive compensation may motivate accounting choices. For example, managers might be concerned that working for a company that performs poorly will limit their future career opportunities or that they will not receive a bonus based on exceeding a particular earnings target. In both cases, management might be motivated to make accounting choices to increase earnings. In a period of marginally poor performance, a manager might accelerate or inflate revenues and/or delay or under report expenses. Conversely, in a period of strong performance, a manager might delay revenue recognition or accelerate expense recognition to increase the probability of exceeding the next period's targets (i.e., to "bank" some earnings for the next period.) The surveyed managers indicated a greater concern with career implications of reported results than with incentive compensation implications.

\end{itemize}

Avoiding debt covenant violations can motivate managers to inflate earnings. Graham, Harvey, and Rajgopal's survey indicates that avoidance of bond covenant violation is important to highly leveraged and unprofitable companies but relatively unimportant overall.

\section{Conditions Conducive to Issuing Low-Quality Financial Reports}
As discussed, deviations from a neutral presentation of financial results could be driven by management choices or by a jurisdiction's financial reporting standards. Ultimately, a decision to issue low-quality, or even fraudulent, financial reports is made by an individual or individuals. Why individuals make such choices is not always immediately apparent. For example, why would the newly appointed CEO of Sunbeam, who already had a net worth of more than $\$ 100$ million, commit accounting fraud by improperly reporting revenues from "bill-and-hold" sales and manipulating the timing of expenses, rather than admit to lower-than-expected financial results?

Typically, three conditions exist when low-quality financial reports are issued: opportunity, motivation, and rationalization. Opportunity can be the result of internal conditions, such as poor internal controls or an ineffective board of directors, or external conditions, such as accounting standards that provide scope for divergent choices or minimal consequences for an inappropriate choice. Motivation can result from pressure to meet some criteria for personal reasons, such as a bonus, or corporate reasons, such as concern about financing in the future. Rationalization is important because if an individual is concerned about a choice, he or she needs to be able to justify it to him-or herself.

Former Enron CFO Andrew Fastow, speaking at the 2013 Association of Certified Fraud Examiners Annual Fraud Conference, indicated that he knew at the time he was doing something wrong but followed procedure to justify his decision (Pavlo, 2013). He made sure to get management and board approval, as well as legal and accounting opinions, and to include appropriate disclosures. The incentive and corporate culture was to create earnings rather than focus on long-term value. Clearly, as reflected in his prison sentence, he did something that was not only wrong but illegal.

\section{MECHANISMS THAT DISCIPLINE FINANCIAL REPORTING QUALITY}
$\square \quad \begin{aligned} & \text { describe mechanisms that discipline financial reporting quality and } \\ & \text { the potential limitations of those mechanisms }\end{aligned}$

Markets potentially discipline financial reporting quality. Companies and nations compete for capital, and the cost of capital is a function of perceived risk-including the risk that a company's financial statements will skew investors' expectations. Thus, in the absence of other conflicting economic incentives, a company seeking to minimize its long-term cost of capital should aim to provide high-quality financial reports. In addition to markets, other mechanisms that discipline financial reporting quality include market regulatory authorities, auditors, and private contracts.

\section{Market Regulatory Authorities}
Companies seeking to minimize the cost of capital should maximize reporting quality, but as discussed earlier, conflicting incentives often exist. For this reason, national regulations, and the regulators that establish and enforce rules, can play a significant role in financial reporting quality. Many of the world's securities regulators are members of the International Organization of Securities Commissions (IOSCO). IOSCO is recognized as the "global standard setter for the securities sector" although it does not actually set standards but rather establishes objectives and principles to guide securities and capital market regulation. IOSCO's membership includes more than 120 securities regulators and 80 other securities market participants, such as stock exchanges. ${ }^{14}$

One member of IOSCO is ESMA , an independent EU authority with a mission to "enhance the protection of investors and reinforce stable and well-functioning financial markets in the European Union."15ESMA organizes financial reporting enforcement activities through a forum consisting of European enforcers from European Economic Area countries. Direct supervision and enforcement activities are performed at the national level. For example, the Financial Conduct Authority (FCA) is the IOSCO member with primary responsibility for securities regulation in the United Kingdom. ESMA reported that European enforcers examined the interim and/or annual financial statements of 1,141 issuers in 2017, which in turn led to enforcement actions for 328 issuers with the following outcomes: 12 required reissuances of financial statements, 71 public corrective notes, and 245 required corrections in future financial statements. ${ }^{16}$

Another member of IOSCO is the US regulatory authority, the Securities and Exchange Commission. The SEC is responsible for overseeing approximately 9,100 US public companies (along with investment advisers, broker/dealers, securities exchanges, and other entities) and reviews the disclosures of these companies at least once every three years with the aim of improving information available to investors and potentially uncovering possible violations of securities laws. ${ }^{17}$ In 2017, the SEC reported that it had filed 754 total and 446 standalone enforcement actions, about $20 \%$ of which concerned issuer reporting/accounting and auditing. ${ }^{18}$

Examples of regulatory bodies in Asia include the Financial Services Agency in Japan, the China Securities Regulatory Commission, and the Securities and Exchange Board of India. Examples of regulatory bodies in South America include the Comisión Nacional de Valores in Argentina, Comissão de Valores Mobiliários in Brazil, and Comisión para el Mercado Financiero in Chile. A full list of IOSCO members can be found on the organization's website.

14 Visit \href{http://www.iosco.org}{www.iosco.org} for more information.

15 Text from ESMA's mission statement on their website: \href{http://www.esma.europa.eu}{www.esma.europa.eu}.

16 ESMA, "Enforcement and Regulatory Activities of Accounting Enforcers in 2017," ESMA32-63-424, European Securities and Markets Authority (03 April 2018): \href{http://www.esma.europa.eu}{www.esma.europa.eu}.

17 SEC, "FY2013 Congressional Justification," Securities and Exchange Commission (February 2012): www. \href{http://sec.gov/about/secfy13congbudgjust.pdf}{sec.gov/about/secfy13congbudgjust.pdf}.

18 SEC, Securities and Exchange Commission Division of Enforcement Annual Report, "A Look Back at Fiscal Year 2017" \href{http://www.sec.gov/report}{www.sec.gov/report}. Typical features of a regulatory regime that most directly affect financial reporting quality include the following:

\begin{itemize}
  \item Registration requirements. Market regulators typically require publicly traded companies to register securities before offering the securities for sale to the public. A registration document typically contains current financial statements, other relevant information about the risks and prospects of the company issuing the securities, and information about the securities being offered.

  \item Disclosure requirements. Market regulators typically require publicly traded companies to make public periodic reports, including financial reports and management comments. Standard-setting bodies, such as the IASB and FASB, are typically private sector, self-regulated organizations with board members who are experienced accountants, auditors, users of financial statements, and academics. Regulatory authorities, such as the Accounting and Corporate Regulatory Authority in Singapore, the Securities and Exchange Commission in the United States, the Securities and Exchange Commission in Brazil, and the Financial Reporting Council in the United Kingdom, have the legal authority to enforce financial reporting requirements and exert other controls over entities that participate in the capital markets within their jurisdiction. In other words, generally, standard-setting bodies set the standards, and regulatory authorities recognize and enforce those standards. Without the recognition of standards by regulatory authorities, the private-sector standard-setting bodies would have no authority. Regulators often retain the legal authority to establish financial reporting standards in their jurisdiction and can overrule the private-sector standard-setting bodies.

  \item Auditing requirements. Market regulators typically require companies' financial statements to be accompanied by an audit opinion attesting that the financial statements conform to the relevant set of accounting standards. Some regulators, such as the SEC in the United States, require an additional audit opinion attesting to the effectiveness of the company's internal controls over financial reporting.

  \item Management commentaries. Regulations typically require publicly traded companies' financial reports to include statements by management. For example, the FCA in the United Kingdom requires a management report containing "(1) a fair review of the issuer's business; and (2) a description of the principal risks and uncertainties facing the issuer" (Disclosure Guidance and Transparency Rules sourcebook.)

  \item Responsibility statements. Regulations typically require a statement from the person or persons responsible for the company's filings. Such statements require the responsible individuals to explicitly acknowledge responsibility and to attest to the correctness of the financial reports. Some regulators, such as the SEC in the United States, require formal certifications that carry specific legal penalties for false certifications.

  \item Regulatory review of filings. Regulators typically undertake a review process to ensure that the rules have been followed. The review process typically covers all initial registrations and a sample of subsequent periodic financial reports. - Enforcement mechanisms. Regulators are granted various powers to enforce the securities market rules. Such powers can include assessing fines, suspending or permanently barring market participants, and bringing criminal prosecutions. Public announcements of disciplinary actions are also a type of enforcement mechanism.

\end{itemize}

In summary, market regulatory authorities play a central role in encouraging high-quality financial reporting.

\section{Auditors}
As noted, regulatory authorities typically require that publicly traded companies' financial statements be audited by an independent auditor. Private companies also obtain audit opinions for their financial statements, either voluntarily or because audit reports are required by an outside party, such as providers of debt or equity capital.

Audit opinions provide financial statement users with some assurance that the information complies with the relevant set of accounting standards and presents the company's information fairly. Exhibit 10, Exhibit 11, Exhibit 12, and Exhibit 13 provide excerpts from the independent auditors' reports for GlaxoSmithKline plc, Alibaba Group Holding Limited, Apple Inc., and Tata Motors Limited, respectively. For each company, the auditor issued an unqualified opinion on the financial statements, indicating that the financial statements present fairly the company's performance in accordance with relevant standards. (Note: The term "unqualified opinion" means that the opinion did not include any qualifications or exceptions; the term is synonymous with the less formal term "clean opinion." Unqualified opinions are the most common.) Other items in the audit reports reflect the specific requirements of the company's regulatory regime. For example, the audit report for GlaxoSmithKline spans nine pages and includes opinions on the company's financial statements as well as the Strategic Report and the Directors' Report. This audit report also includes disclosures about "Key audit matters," in accordance with International Standards on Auditing (ISAs) issued by the International Auditing and Assurance Standards Board (IAASB) in 2015 and effective for periods ending on or after December 15, 2016.

The excerpts for Alibaba, Apple, and Tata Motors show the auditors' opinions on the companies' financial statements and additionally the SEC-required opinions on the effectiveness of the companies' internal controls because these companies are listed in the United States. For Alibaba, a single report includes both unqualified opinions: (i) the financial statements present fairly the financial position, results of operations, and cash flows... in conformity with US GAAP; and (ii) the company maintained effective control over financial reporting. For Apple, the first report includes the unqualified opinion on the financial statements, and the second report includes the unqualified opinion on the company's effective internal controls. For Tata Motors, the first report includes the unqualified opinion that the financial statements present the company's position and results fairly in accordance with IFRS. (The SEC permits non-US companies to report using US GAAP, IFRS as issued by the IASB, or home-country GAAP.) However, the second report includes an adverse opinion on the effectiveness of the company's internal controls: "In our opinion, because of the effect of the material weakness ... the company has not maintained effective internal control." The report explains that the material weakness involved a third party's inappropriate access to the company's systems. The report further states that although the material weakness resulted in ineffective internal controls, it did not affect the audit opinion on the financial statements. Elsewhere in Tata Motors' annual report (not shown in the excerpt), the company discloses that the weakness did not result in a financial misstatement and that it has undertaken remedial measures. Exhibit 10: Excerpts from Audit Opinion of PricewaterhouseCoopers LLP from the 2017 Annual Report (pages 149-157) of GlaxoSmithKline plc

In our opinion, GlaxoSmithKline plc's Group financial statements (the "financial statements"):

\begin{itemize}
  \item give a true and fair view of the state of the Group's affairs as at 31 December 2017 and of its profit and cash flows for the year then ended;

  \item have been properly prepared in accordance with International Financial Reporting Standards ("IFRSs") as adopted by the European Union; and

  \item have been prepared in accordance with the requirements of the Companies Act 2006 and Article 4 of the IAS Regulation.

\end{itemize}

In our opinion, the Group financial statements have been properly prepared in accordance with IFRSs as issued by the IASB.

...

Key audit matters

Key audit matters are those matters that, in the auditors' professional judgement, were of most significance in the audit of the financial statements of the current period and include the most significant assessed risks of material misstatement (whether or not due to fraud) identified by the auditors, including those which had the greatest effect on: the overall audit strategy; the allocation of resources in the audit; and directing the efforts of the engagement team. These matters, and any comments we make on the results of our procedures thereon, were addressed in the context of our audit of the financial statements as a whole, and in forming our opinion thereon, and we do not provide a separate opinion on these matters. This is not a complete list of all risks identified by our audit.

...

In our opinion, based on the work undertaken in the course of the audit, the information given in the Strategic Report and Directors' Report for the year ended 31 December 2017 is consistent with the financial statements and has been prepared in accordance with applicable legal requirements.

\section{Exhibit 11: Excerpts from Audit Opinion of PricewaterhouseCoopers Hong}
 Kong, SAR from the Annual Report (SEC Form 20-F, Pages F-2 and F-3) of Alibaba Group Holding Limited for the year ended March 31, 2018In our opinion, the consolidated financial statements referred to above present fairly, in all material respects, the financial position of the Company as of March 31, 2017 and 2018, and the results of their operations and their cash flows for each of the three years in the period ended March 31, 2018 in conformity with accounting principles generally accepted in the United States of America. Also in our opinion, the Company maintained, in all material respects, effective internal control over financial reporting as of March 31, 2018, based on criteria established in Internal Control - Integrated Framework (2013) issued by the COSO.

Exhibit 12: Excerpt from Audit Opinion of Ernst \& Young from the Annual Report (SEC Form 10-K, pages 70 and 71) of Apple Inc. for the year ended September 30, 2017

[From the Financial Statement Opinion] We have audited the accompanying consolidated balance sheets of Apple Inc. as of September 30, 2017 and September 24, 2016, and the related consolidated statements of operations, comprehensive income, shareholders' equity and cash flows for each of the three years in the period ended September 30, 2017.

In our opinion, the financial statements referred to above present fairly, in all material respects, the consolidated financial position of Apple Inc. at September 30, 2017 and September 24, 2016, and the consolidated results of its operations and its cash flows for each of the three years in the period ended September 30, 2017, in conformity with U.S. generally accepted accounting principles.

We also have audited, in accordance with the standards of the Public Company Accounting Oversight Board (United States), Apple Inc.'s internal control over financial reporting as of September 30, 2017, based on criteria established in Internal Control - Integrated Framework issued by the Committee of Sponsoring Organizations of the Treadway Commission (2013 framework) and our report dated November 3, 2017 expressed an unqualified opinion thereon.

[From the Internal Controls Opinion]

We have audited Apple Inc.'s internal control over financial reporting as of September 30, 2017, based on criteria established in Internal Control - Integrated Framework issued by the Committee of Sponsoring Organizations of the Treadway Commission (2013 framework) ("the COSO criteria").

In our opinion, Apple Inc. maintained, in all material respects, effective internal control over financial reporting as of September 30, 2017, based on the COSO criteria.

We also have audited, in accordance with the standards of the Public Company Accounting Oversight Board (United States), the 2017 consolidated financial statements of Apple Inc. and our report dated November 3, 2017 expressed an unqualified opinion thereon.

Exhibit 13: Excerpt from Audit Opinion of KPMG Mumbai, India from the Annual Report (SEC Form 20-F, pages F2 to F4) of Tata Motors Limited for the year ended March 31, 2018

\section{Opinion on the Consolidated Financial Statements}
We have audited the accompanying consolidated balance sheet of Tata Motors Limited and its subsidiaries (the "Company") as of March 31, 2018, the related consolidated income statement, statement of comprehensive income, statement of cash flows, and statement of changes in equity for the year ended March 31, 2018, and the related notes and financial statement schedule 1 (collectively, the consolidated financial statements).

In our opinion, the consolidated financial statements present fairly, in all material respects, the financial position of the Company as of March 31, 2018, and the results of its operations and its cash flows for the year ended March 31, 2018, in conformity with the International Financial Reporting Standards as issued by the International Accounting Standards Board ("IFRS").

We also have audited, in accordance with the standards of the Public Company Accounting Oversight Board (United States) (PCAOB), the Company's internal control over financial reporting as of March 31, 2018, based on criteria established in Internal Control - Integrated Framework (2013) issued by the Committee of Sponsoring Organizations of the Treadway Commission, and our report dated July, 31, 2018 expressed an adverse opinion on the effectiveness of the Company's internal control over financial reporting.

...

Opinion on Internal Control Over Financial Reporting

We have audited Tata Motors Limited's and subsidiaries' (the Company) internal control over financial reporting as of March 31, 2018, based on criteria established in Internal Control - Integrated Framework (2013) issued by the Committee of Sponsoring Organizations of the Treadway Commission. In our opinion, because of the effect of the material weakness described below, on the achievement of the objectives of the control criteria, the Company has not maintained effective internal control over financial reporting as of March 31, 2018, based on criteria established in Internal Control - Integrated Framework (2013) issued by the Committee of Sponsoring Organizations of the Treadway Commission.

...

A material weakness is a deficiency, or a combination of deficiencies, in internal control over financial reporting, such that there is a reasonable possibility that a material misstatement of the company's annual or interim financial statements will not be prevented or detected on a timely basis. A material weakness related to inappropriate system access restrictions at a third party logistics provider has been identified and included in management's assessment. The material weakness was considered in determining the nature, timing, and extent of audit tests applied in our audit of the 2018 consolidated financial statements, and this report does not affect our report on those consolidated financial statements.

Although audit opinions provide discipline for financial reporting quality, inherent limitations exist. First, an audit opinion is based on a review of information prepared by the company. If a company deliberately intends to deceive its auditor, a review of information might not uncover misstatements. Second, an audit is based on sampling, and the sample might not reveal misstatements. Third, an "expectations gap" may exist between the auditor's role and the public's expectation of auditors. An audit is not typically intended to detect fraud; it is intended to provide assurance that the financial reports are fairly presented. Finally, the company being audited pays the audit fees, often established through a competitive process. This situation could provide an auditor with an incentive to show leniency to the company being audited, particularly if the auditor's firm provides additional services to the company.

\section{Private Contracting}
Aspects of private contracts, such as loan agreements or investment contracts, can serve as mechanisms to discipline financial reporting quality. Many parties that have a contractual arrangement with a company have an incentive to monitor that company's performance and to ensure that the company's financial reports are high quality. For example, loan agreements often contain loan covenants, which create specifically tailored financial reporting requirements that are legally binding for the issuer. As noted earlier, avoidance of debt covenant violation is a potential motivation for managers to inflate earnings. As another example, an investment contract could contain provisions giving investors the option to recover all or part of their investment if certain financial triggers occur. Such provisions could motivate the investee's managers to manipulate reported results to avoid the financial triggers.

Because the financial reports prepared by the investees or borrowers directly affect the contractual outcomes-potentially creating a motivation for misreportinginvestors and lenders are motivated to monitor financial reports and to ensure that they are high quality.

\section{EXAMPLE 4}
\section{Financial Reporting Manipulation: Motivations and Disciplining Mechanisms}
For each of the following two scenarios, identify (1) factors that might motivate the company's managers to manipulate reported financial amounts and (2) applicable mechanisms that could discipline financial reporting quality.

\begin{enumerate}
  \item ABC Co. is a private company. Bank NTBig has made a loan to ABC Co. $A B C$ is required to maintain a minimum 2.0 interest coverage ratio. In its most recent financial reports, ABC reported earnings before interest and taxes of $\$ 1,200$ and interest expense of $\$ 600$. In the report's notes, the company discloses that it changed the estimated useful life of its property, plant, and equipment during the year. Depreciation was approximately $\$ 150$ lower as a result of this change in estimate.
\end{enumerate}

\section{Solution to 1:}
The need to maintain a minimum interest coverage ratio of 2.0 might motivate ABC's managers to manipulate reported financial amounts. The company's coverage ratio based on the reported amounts is exactly equal to 2.0. If ABC's managers had not changed the estimated useful life of the property, plant, and equipment, the coverage ratio would have fallen below the required level.

EBIT, as reported

$\$ 1,200$

Impact on depreciation expense of changed assumptions about useful life

$\frac{150}{\$ 1,050}$

EBIT, as adjusted

$\begin{array}{ll}\text { Interest expense } & \$ 600\end{array}$

$\begin{array}{ll}\text { Coverage ratio, as reported } & 2.00\end{array}$

$\begin{array}{ll}\text { Coverage ratio, as adjusted } & 1.75\end{array}$

The potential disciplining mechanisms include the auditors, who will assess the reasonableness of the depreciable lives estimates. In addition, the lenders will carefully scrutinize the change in estimate because the company only barely achieved the minimum coverage ratio and would not have achieved the minimum without the change in accounting estimate.

\begin{enumerate}
  \setcounter{enumi}{1}
  \item DEF Co. is a publicly traded company. For the most recent quarter, the average of analysts' forecasts for earnings per share was $\$ 2.50$. In its quarterly earnings announcement, DEF reported net income of $\$ 3,458,780$. The number of common shares outstanding was 1,378,000. DEF's main product is a hardware device that includes a free two-year service contract in the selling price. Based on management estimates, the company allocates a portion of revenues to the hardware device, which it recognizes immediately, and a portion to the service contract, which it defers and recognizes over the two years of the contract. Based on the disclosures, a higher percentage of reve- nue was allocated to hardware than in the past, with an estimated after-tax impact on net income of $\$ 27,000$.
\end{enumerate}

\section{Solution to 2:}
The desire to meet or exceed the average of analysts' forecasts for earnings per share might motivate DEF Co.'s managers to manipulate reported financial amounts. As illustrated in the following calculations, the impact of allocating a greater portion of revenue to hardware enabled the company to exceed analysts' earnings per share forecasts by $\$ 0.01$.

Net income, as reported

$\$ 3,458,780$

Impact on gross profit of changed revenue recognition, net of tax

27,000

Net income, as adjusted

$\$ 3,431,780$

Weighted average number of shares

$1,378,000$

Earnings per share, as reported

$\$ 2.51$

Earnings per share, as adjusted

$\$ 2.49$

The potential disciplining mechanisms include the auditors, market regulators, financial analysts, and financial journalists.

DETECTION OF FINANCIAL REPORTING QUALITY ISSUES: INTRODUCTION \& PRESENTATION CHOICES

describe presentation choices, including non-GAAP measures, that could be used to influence an analyst's opinion describe accounting methods (choices and estimates) that could be used to manage earnings, cash flow, and balance sheet items

Choices in the application of accounting standards abound, which is perhaps one reason why accounting literature and texts are so voluminous. Compounding the complexity, measurement often depends on estimates of economic phenomena. Two estimates might be justifiable, but they may have significantly different effects on the company's financial statements. As discussed earlier, the choice of a particular estimate may depend on the motivations of the reporting company's managers. With many choices available, and the inherent flexibility of estimates in the accounting process, managers have many tools for managing and meeting analysts' expectations through financial reporting.

An understanding of the choices that companies make in financial reporting is fundamental to evaluating the overall quality-both financial reporting and earnings quality-of the reports produced. Choices exist both in how information is presented (financial reporting quality) and in how financial results are calculated (earnings quality). Choices in presentation (financial reporting quality) may be fairly transparent to investors. Choices in the calculation of financial results (earnings quality), however, are more difficult to discern because they can be deeply embedded in the construction of reported financial results.

The availability of accounting choices enables managers to affect the reporting of financial results. Some choices increase performance and financial position in the reporting period (aggressive choices), and others increase them in later periods (conservative choices). A manager that wants to increase performance and financial position in the reporting period could:

\begin{itemize}
  \item Recognize revenue prematurely;

  \item Use non-recurring transactions to increase profits;

  \item Defer expenses to later periods;

  \item Measure and report assets at higher values; and/or

  \item Measure and report liabilities at lower values.

\end{itemize}

A manager that wants to increase performance and financial position in a later period could:

\begin{itemize}
  \item Defer current income to a later period (save income for a "rainy day"); and/ or

  \item Recognize future expenses in a current period, setting the table for improving future performance.

\end{itemize}

The following sections describe some of the potential choices for how information is presented and how accounting elements [assets, liabilities, owners' equity, revenue and gains (income), and expenses and losses] are recognized, measured, and reported. In addition to choices within GAAP, companies may prepare fraudulent reports. For example, these reports may include non-existent revenue or assets. Section 4 concludes with some of the warning signs that can indicate poor-quality financial reports.

\section{Presentation Choices}
The technology boom of the 1990s and the internet bubble of the early 2000s featured companies, popular with investors, that often shared the same characteristic: They could not generate enough current earnings to justify their stock prices using the traditional price-to-earnings ratio (P/E) approaches to valuation. Many investors chose to explain these apparent anomalies by rationalizing that the old focus on profits and traditional valuation approaches no longer applied to such companies. Strange new metrics for determining operating performance emerged. Website operators spoke of the "eyeballs" they had captured in a quarter, or the "stickiness" of their websites for web surfers' visits. Various versions of "pro forma earnings" - that is, "non-GAAP earnings measures"-became a financial reporting staple of the era.

Many technology companies were accomplished practitioners of pro forma reporting, but they were not the first to use it. In the early 1990s, downsizing of large companies was a commonplace event, and massive restructuring charges obscured the operating performance at many established companies. For example, as it learned to cope in a world that embraced the personal computer rather than mainframe computing, International Business Machines (IBM) reported massive restructuring charges in 1991, 1992, and 1993: $\$ 3.7$ billion, $\$ 11.6$ billion, and $\$ 8.9$ billion, respectively. IBM was not alone. Sears incurred $\$ 2.7$ billion of restructuring charges in 1993, and AT\&T reported restructuring charges of $\$ 7.7$ billion in 1995. These events were not isolated; restructuring charges were a standard quarterly reporting event. To counter perceptions that their operations were floundering, and supposedly to assist investors in evaluating operating performance, companies often sanitized earnings releases by excluding restructuring charges in pro forma measures of financial performance. Accounting principles for reporting business combinations also played a role in boosting the popularity of pro forma earnings. Before 2001, acquisitions of one company by another often resulted in goodwill amortization charges that made subsequent earnings reports look weak. Complicating matters, there were two accounting methods for recording acquisitions: pooling-of-interests and purchase methods. The now-extinct pooling-of-interests treatment was difficult for companies to achieve because of the many restrictive criteria for its use, but it was greatly desired because it did not result in goodwill amortization charges. In the technology boom period, acquisitions were common and many were reported as purchases, with consequential goodwill amortization dragging down earnings for as long as 40 years under the then-existing rules. Acquisitive companies reporting under purchase accounting standards perceived themselves to be at a reporting disadvantage compared with companies able to apply pooling-of-interests. They responded by presenting earnings adjusted for the exclusion of amortization of intangible assets and goodwill.

Because investors try to make intercompany comparisons on a consistent basis, earnings before interest, taxes, depreciation, and amortization has become an extremely popular performance measure. EBITDA is widely viewed as eliminating noisy reporting signals. That noise may be introduced by different accounting methods among companies for depreciation, amortization of intangible assets, and restructuring charges. Companies may construct and report their own version of EBITDA, sometimes referring to it as "adjusted EBITDA," by adding to the list of items to exclude from net income. Items that analysts might encounter include the following:

\begin{itemize}
  \item Rental payments for operating leases, resulting in EBITDAR (earnings before interest, taxes, depreciation, amortization, and rentals);

  \item Equity-based compensation, usually justified on the grounds that it is a non-cash expense;

  \item Acquisition-related charges;

  \item Impairment charges for goodwill or other intangible assets;

  \item Impairment charges for long-lived assets;

  \item Litigation costs; and

  \item Loss/gain on debt extinguishments.

\end{itemize}

Among other incentives for the spread of non-GAAP earnings measures are loan covenants. Lenders may make demands on a borrowing company that require achieving and maintaining performance criteria that use GAAP net income as a starting point but arrive at a measure suitable to the lender. The company may use this measure as its preferred non-GAAP metric in earnings releases, and also when describing its liquidity or solvency situation in the management commentary (called management discussion and analysis in the United States).

As mentioned earlier, if a company uses a non-GAAP financial measure in an SEC filing, it must display the most directly comparable GAAP measure with equal prominence and provide a reconciliation between the two. Management must explain why it believes that the non-GAAP financial measure provides useful information regarding the company's financial condition and operations. Management must also disclose additional purposes, if material, for which it uses the non-GAAP financial measures.

Similarly, IFRS requires a definition and explanation of any non-IFRS measures included in financial reports, including why the measure is potentially relevant to users. Management must provide reconciliations of non-IFRS measures with IFRS measures. There is a concern that management may use non-GAAP measures to distract attention from GAAP measures. The SEC intended that the definition of non-GAAP financial measures would capture all measures with the effect of depicting either:

\begin{itemize}
  \item a measure of performance that differs from that presented in the financial statements, such as income or loss before taxes or net income or loss, as calculated in accordance with GAAP; or

  \item a measure of liquidity that differs from cash flow or cash flow from operations computed in accordance with GAAP. ${ }^{19}$

\end{itemize}

The SEC prohibits the exclusion of charges or liabilities requiring cash settlement from any non-GAAP liquidity measures, other than EBIT and EBITDA. Also prohibited is the calculation of a non-GAAP performance measure intended to eliminate or smooth items tagged as non-recurring, infrequent, or unusual when such items are very likely to occur again. The SEC views the period within two years of either before or after the reporting date as the relevant time frame for considering whether a charge or gain is a recurring item. Example 5 describes a case of misuse and misreporting of non-GAAP measures.

\section{EXAMPLE 5}
\section{Misuse and Misreporting of Non-GAAP Measures}
Groupon is an online discount merchant. In the company's initial S-1 registration statement in 2011, then-CEO Andrew Mason gave prospective investors an up-front warning in a section entitled "We don't measure ourselves in conventional ways", which described Groupon's adjusted consolidated segment operating income (adjusted CSOI) measure. Exhibit 14 provides excerpts from a section entitled "Non-GAAP Financial Measures," which offered a more detailed explanation. Exhibit 15, also from the initial registration statement, shows a reconciliation of CSOI to the most comparable US GAAP measure. In its review, the SEC took the position that online marketing expenses were a recurring cost of business. Groupon responded that the marketing costs were similar to acquisition costs, not recurring costs, and that "we'll ramp down marketing just as fast as we ramped it up, reducing the customer acquisition part of our marketing expenses" as time passes. ${ }^{20}$

Eventually, and after much negative publicity, Groupon changed its non-GAAP measure. Exhibit 16 shows an excerpt from the final prospectus filed in November, after the SEC's review. Use the three exhibits to answer the questions that follow.

\section{Exhibit 14: Groupon's "Non-GAAP Financial Measures"}
\section{Disclosures from June S-1 Filing}
Adjusted CSOI is operating income of our two segments, North America and International, adjusted for online marketing expense, acquisition-related costs and stock-based compensation expense. Online marketing expense primarily represents the cost to acquire new subscribers and is dictated by the amount of growth we wish to pursue. Acquisition-related costs are non-recurring non-cash items related to certain of our acquisitions. Stock-based compensation expense is a non-cash item. We consider Adjusted CSOI to be an important measure of the performance of our business as it excludes expenses that are non-cash or otherwise not indicative of future operating expenses. We believe it is important to view Adjusted CSOI as a complement to our entire consolidated statements of operations.

Our use of Adjusted CSOI has limitations as an analytical tool, and you should not consider this measure in isolation or as a substitute for analysis of our results as reported under GAAP. Some of these limitations are:

\begin{itemize}
  \item Adjusted CSOI does not reflect the significant cash investments that we currently are making to acquire new subscribers;

  \item Adjusted CSOI does not reflect the potentially dilutive impact of issuing equity-based compensation to our management team and employees or in connection with acquisitions;

  \item Adjusted CSOI does not reflect any interest expense or the cash requirements necessary to service interest or principal payments on any indebtedness that we may incur;

  \item Adjusted CSOI does not reflect any foreign exchange gains and losses;

  \item Adjusted CSOI does not reflect any tax payments that we might make, which would represent a reduction in cash available to us;

  \item Adjusted CSOI does not reflect changes in, or cash requirements for, our working capital needs; and

  \item Other companies, including companies in our industry, may calculate Adjusted CSOI differently or may use other financial measures to evaluate their profitability, which reduces the usefulness of it as a comparative measure.

\end{itemize}

Because of these limitations, Adjusted CSOI should not be considered as a measure of discretionary cash available to us to invest in the growth of our business. When evaluating our performance, you should consider Adjusted CSOI alongside other financial performance measures, including various cash flow metrics, net loss and our other GAAP results.

\section{Exhibit 15: Groupon's "Adjusted CSOI"}
\section{Excerpt from June S-1 Filing}
The following is a reconciliation of CSOI to the most comparable US GAAP measure, "loss from operations," for the years ended December 31, 2008, 2009, and 2010 and the three months ended March 31, 2010 and 2011:

\begin{center}
\begin{tabular}{lcccccc}
\hline
 &  & \multicolumn{2}{c}{$\begin{array}{c}\text { Year Ended } \\
\text { December 31, }\end{array}$} & \multicolumn{2}{c}{$\begin{array}{c}\text { Three Months Ended } \\
\text { March 31, }\end{array}$} \\
\hline
(in \$ thousands) & $\mathbf{2 0 0 8}$ & $\mathbf{2 0 0 9}$ & $\mathbf{2 0 1 0}$ & $\mathbf{2 0 1 0}$ & $\mathbf{2 0 1 1}$ \\
\hline
$\begin{array}{l}\text { (Loss) Income from }\end{array}$ &  &  &  &  &  \\
operations &  &  &  &  &  \\
$\begin{array}{l}\text { Adjustments }\end{array}$ & $(1,632)$ & $(1,077)$ & $(420,344)$ & 8,571 & $(117,148)$ \\
$\begin{array}{l}\text { Online marketing } \\ \text { Stock-based } \\ \text { compensation }\end{array}$ & 162 & 4,446 & 241,546 & 3,904 & 179,903 \\
 & 24 & 115 & 36,168 & 116 & 18,864 \\
\end{tabular}
\end{center}

\begin{center}
\begin{tabular}{lccccc}
\hline
 &  & \multicolumn{2}{c}{$\begin{array}{c}\text { Year Ended } \\
\text { December 31, }\end{array}$} & \multicolumn{2}{c}{$\begin{array}{c}\text { Three Months Ended } \\
\text { March 31, }\end{array}$} \\
\hline
(in \$ thousands) & $\mathbf{2 0 0 8}$ & $\mathbf{2 0 0 9}$ & $\mathbf{2 0 1 0}$ & $\mathbf{2 0 1 0}$ & $\mathbf{2 0 1 1}$ \\
\hline
Acquisition-related & - & - & 203,183 & - & - \\
\cline { 2 - 6 }
Total adjustments & 186 & 4,561 & 480,897 & 4,020 & 198,767 \\
\cline { 2 - 6 }
Adjusted CSOI & $(1,446)$ & 3,484 & 60,553 & 12,591 & 81,619 \\
\hline
\end{tabular}
\end{center}

\section{Exhibit 16: Groupon's "CSOI"}
\section{Excerpt from Revised S-1 Filing}
The following is a reconciliation of CSOI to the most comparable US GAAP measure, "loss from operations," for the years ended December 31, 2008, 2009, and 2010 and the nine months ended September 30, 2010 and 2011:

\begin{center}
\begin{tabular}{lcccccc}
\hline
 &  & \multicolumn{2}{c}{Year Ended} &  &  \\
December 31, &  & \multicolumn{2}{c}{$\begin{array}{c}\text { Nine Months Ended } \\
\text { September 30, }\end{array}$} &  &  \\
\hline
(in \$ thousands) & $\mathbf{2 0 0 8}$ & $\mathbf{2 0 0 9}$ & $\mathbf{2 0 1 0}$ & $\mathbf{2 0 1 0}$ & $\mathbf{2 0 1 1}$ \\
\hline
Loss from operations & $(1,632)$ & $(1,077)$ & $(420,344)$ & $(84,215)$ & $(218,414)$ \\
Adjustments: &  &  &  &  &  \\
\end{tabular}
\end{center}

\begin{enumerate}
  \item What cautions did Groupon include along with its description of the "Adjusted CSOI" metric?
\end{enumerate}

\section{Solution to 1:}
Groupon cautioned that the "Adjusted CSOI" metric should not be considered in isolation, should not be considered as a substitute for analysis using GAAP results, and "should not be considered a measure of discretionary cash flow." The company lists numerous limitations, primarily citing items that adjusted CSOI did not reflect.

\begin{enumerate}
  \setcounter{enumi}{1}
  \item Groupon excludes "online marketing" from "Adjusted CSOI." How does the exclusion of this expense compare with the SEC's limits on non-GAAP performance measures?
\end{enumerate}

\section{Solution to 2:}
The SEC specifies that non-GAAP measures should not eliminate items tagged as non-recurring, infrequent, or unusual when such items may be very likely to occur again. Because the online marketing expense occurred in every period reported and is likely to occur again, exclusion of this item appears contrary to SEC requirements. 3. In the first quarter of 2011 , what was the effect of excluding online marketing expenses on the calculation of "Adjusted CSOI"?

\section{Solution to 3:}
As shown in Exhibit 15, in the first quarter of 2011, the exclusion of the online marketing expense was enough to swing the company from a net loss under US GAAP reporting to a profit-at least, a profit as defined by adjusted CSOI. Using adjusted CSOI as a performance measure, the company showed results that were $35 \%$ higher for the first quarter of 2011 compared with the entire previous year.

\begin{enumerate}
  \setcounter{enumi}{3}
  \item For 2010, how did results under the revised non-GAAP metric compare with the originally reported metric?
\end{enumerate}

\section{Solution to 4:}
As shown in Exhibit 16, the revised metric is now called "CSOI" and no longer refers to "Adjusted CSOI." For 2010, results under the revised non-GAAP metric, which includes online marketing costs, shows a loss of $\$ 180,993,000$ instead of a profit of $\$ 60,553,000$.

In the case described in Example 5, Groupon changed its reporting and corrected the non-GAAP metric that the SEC had identified as misleading. In other cases, the SEC has pursued enforcement actions against companies for reporting misleading non-GAAP information. One such action was brought in 2009 against SafeNet Inc., where the SEC charged the company with improperly classifying ordinary operating expenses as non-recurring. This related to the integration of an acquired company and exclusion of the expenses from non-GAAP earnings in order to exceed earnings targets. A second action was brought by the SEC in 2017 against MDC Partners Inc. ("MDCA") for improper reconciliation of a non-GAAP measure and for improperly displaying the non-GAAP measure with greater prominence in its earnings releases. The case was brought after the company agreed to follow the rules but then failed to do so, as evidenced by the remark in the SEC's action: "Despite agreeing to comply with non-GAAP financial measure disclosure rules in December 2012 correspondence with the [SEC's] Division of Corporation Finance, MDCA continued to violate those rules for six quarters ..." Exhibit 17 presents the headline and sub-headings for one of MDC Partners' earnings announcements that was the subject of the enforcement action.

\section{Exhibit 17: MDC Partners Inc. Press Release [Excerpt]}
\section{SEC Form 8-K filed 24 April 2014}
This excerpt shows the headline, sub-heads, and lead sentence of the company's press release announcing periodic earnings.

MDC PARTNERS INC. REPORTS RECORD RESULTS FOR THE THREE MONTHS ENDED MARCH 31, 2014

ORGANIC REVENUE GROWTH OF 8.3\%, EBITDA GROWTH OF 18.1\% AND 90 BASIS POINTS OF MARGIN IMPROVEMENT

FREE CASH FLOW GROWTH OF 34.0\%

INCREASED 2014 GUIDANCE IMPLIES YEAR-OVER-YEAR EBITDA GROWTH OF +13.5\% TO + 16.1\%, MARGIN IMPROVEMENT OF 60 TO 70 BASIS POINTS, AND FREE CASH FLOW GROWTH OF + 15.8\% $T O+20.2 \%$ FIRST QUARTER HIGHLIGHTS:

\begin{itemize}
  \item Revenue increased to $\$ 292.6$ million from $\$ 265.6$ million, an increase of $10.1 \%$

  \item Organic revenue increased 8.3\%

  \item EBITDA increased to $\$ 36.4$ million from $\$ 30.8$ million, an increase of $18.1 \%$

  \item EBITDA margin increased 90 basis points to $12.5 \%$ from $11.6 \%$

  \item Free Cash Flow increased to $\$ 20.6$ million from $\$ 15.4$ million, an increase of $34.0 \%$

  \item Net New Business wins totaled \$24.4 million

\end{itemize}

NEW YORK, NY (April 24, 2014) - MDC Partners Inc. (NASDAQ: MDCA; TSX: MDZ.A) today announced financial results for the three months ended March 31, 2014.

In general, management may choose to construct non-GAAP financial measures not only to help investors better understand the company's performance but also to paint a more flattering picture of its performance. In some cases, management may attempt to present non-GAAP measures in a way that diverts attention from the standards-compliant financial information that it is required to present.

\section{ACCOUNTING CHOICES AND ESTIMATES}
describe accounting methods (choices and estimates) that could be used to manage earnings, cash flow, and balance sheet items

Choices do not necessarily involve complex accounting standards. Something as simple as the shipping terms for goods delivered to customers can have a profound effect on the timing of revenue. On the last day of the first quarter, suppose a company ships $\$ 10,000$ of goods to a customer on the terms "free on board (FOB) shipping point," arriving the next day. This shipping term means that the customer takes title to the goods, and bears the risk of loss, at the time the goods leave the seller's loading dock. Barring any issues with collectability of the receivable, or a likelihood of a return, the seller would be able to recognize revenue on the sale along with the associated profit. That revenue and profit would be recognized in the first quarter of the year. Change the point at which the goods' title transfers to the customer to "FOB destination" and the revenue pattern will be completely different. Under these terms, the title-and risk of loss-transfers to the customer when the goods arrive at their destination, which is the customer's address. The seller cannot recognize the sale and profit until the shipment arrives the following day, which is the start of a new accounting period.

A simple change in shipping terms can make the difference between revenue and profits in the reporting period or postponing them until the next period. Shipping terms can also influence management behavior. To "make the numbers," managers might push product out the door prematurely under FOB shipping point arrangements in order to reflect as much revenue as possible in the reporting period. Alternatively, in the case of an over-abundance of orders, the company could run the risk of exceeding analysts' consensus estimates by a large margin. Management might be uncomfortable with this situation because investors might extrapolate too much from one reporting period in which expectations were exceeded. Management might want to prevent investors from becoming too optimistic and, if possible, delay revenue recognition until the next quarter. This result could be accomplished by fulfilling customer orders by initiating delivery on the last day of the quarter, with shipping terms set as FOB destination. By doing so, title would transfer in the next accounting period. Another possibility in this scenario is that if the customers insisted on FOB shipping point terms, the selling company could simply delay shipment until after the close of the quarter.

This illustration also highlights a difficult distinction for investors to make. A company may use accounting as a tool to aggressively promote earnings growth-as in the example with the premature shipment of goods with FOB shipping point terms-but it may be aggressively managing the business flow by slacking off on shipping goods when business is "too good," as in the second example. In either case, a desired management outcome is obtained by a simple change in shipping terms. Yet, many investors might be inclined to say that the second example is a conservative kind of earnings management and accept it, even though it artificially masks the actual economic activity that occurred at the time.

\section{How Accounting Choices and Estimates Affect Earnings and Balance Sheets}
Assumptions about inventory cost flows provide another example of how accounting choices can affect financial reporting. Companies may assume that their purchases of inventory items are sold to customers on a first-in-first-out (FIFO) basis, with the result that the remaining inventory reflects the most recent costs. Alternatively, they may assume that their purchases of inventory items are sold to customers on a weighted-average cost basis. Example 6 makes the point that merely choosing a cost flow assumption can affect profitability.

\section{EXAMPLE 6}
\section{Effect of Cost Flow Assumption}
A company starts operations with no inventory at the beginning of a fiscal year and makes purchases of a good for resale five times during the period at increasing prices. Each purchase is for the same number of units of the good. The purchases, and the cost of goods available for sale, appear in the following table. Notice that the price per unit has increased by $140 \%$ by the end of the period.

\begin{center}
\begin{tabular}{lccc}
\hline
 & Units & Price & Cost \\
\hline
Purchase 1 & 5 & $\$ 100$ & $\$ 500$ \\
Purchase 2 & 5 & 150 & 750 \\
Purchase 3 & 5 & 180 & 900 \\
Purchase 4 & 5 & 200 & 1,000 \\
Purchase 5 & 5 & 240 & 1,200 \\
\cline { 3 - 4 }
Cost of goods available for sale &  & $\$ 4,350$ &  \\
\hline
\end{tabular}
\end{center}

During the period, the company sells, at $\$ 250$ each, all of the goods purchased except for five of them. Although the ending inventory consists of five units, the cost attached to those units can vary greatly. 1. What are the ending inventory and cost of goods sold if the company uses the FIFO method of inventory costing?

\section{Solution to 1:}
The ending inventory and cost of goods sold if the company uses the FIFO method of inventory costing are $\$ 1,200$ and $\$ 3,150$.

\begin{enumerate}
  \setcounter{enumi}{1}
  \item What are the ending inventory and cost of goods sold if the company uses the weighted-average method of inventory costing?
\end{enumerate}

\section{Solution to 2:}
The ending inventory and cost of goods sold if the company uses the weighted-average method of inventory costing are $\$ 870$ and $\$ 3,480$.

\begin{enumerate}
  \setcounter{enumi}{2}
  \item Compare cost of goods sold and gross profit calculated under the two methods.
\end{enumerate}

\section{Solution to 3:}
The following table shows how the choice of inventory costing methodsFIFO versus weighted average-affects the cost of goods sold and gross profit.

\begin{center}
\begin{tabular}{lcc}
\hline
Cost Flow Assumption & FIFO & Weighted Average \\
\hline
Cost of goods available for sale & $\$ 4,350$ & $\$ 4,350$ \\
Ending inventory (5 units) & $(1,200)$ & $(870)$ \\
Cost of goods sold & $\$ 3,150$ & $\$ 3,480$ \\
Sales & $\$ 5,000$ & $\$ 5,000$ \\
Cost of goods sold & 3,150 & 3,480 \\
Gross profit & $\$ 1,850$ & $\$ 1,520$ \\
Gross profit margin & $37.0 \%$ & $30.4 \%$ \\
\hline
\end{tabular}
\end{center}

Note: Average inventory cost is calculated as Cost of goods available for sale/Units purchased = $\$ 4,350 / 25=\$ 174$. There are five units in ending inventory, yielding an inventory value of $\$ 870$.

Depending on which cost flow assumption the company uses, the end-of-period inventory is either $\$ 870$ (under the weighted-average method) or $\$ 1,200$ (under FIFO). The choice of method results in a difference of $\$ 330$ in gross profit and 6.6\% in gross profit margin.

The previous example is simplified and extreme for purposes of illustration clarity, but the point is important: Management's choice among acceptable inventory assumptions and methods affects profit. The selection of an inventory costing method is a policy decision, and companies cannot arbitrarily switch from one method to another. The selection does matter to profitability, however, and it also matters to the balance sheet.

In periods of changing prices, the FIFO cost assumption will provide a more current picture of ending inventory value, because the most recent purchases will remain in inventory. The balance sheet will be more relevant to investors. Under the weighted-average cost assumption, however, the balance sheet will display a blend of old and new costs. During inflationary periods, the value of the inventory will be understated: The company will not be able to replenish its inventory at the value shown. At the same time, the weighted-average inventory cost method ensures that the more current costs are shown in cost of sales, making the income statement more relevant than under the FIFO assumption. Trade-offs exist, and investors should be aware of how accounting choices affect financial reports. High-quality financial reporting provides users with sufficient information to assess the effects of accounting choices.

Estimates abound in financial reporting because of the use of accrual accounting, which attempts to show the effects of all economic events on a company during a particular period. Accrual accounting stands in contrast to cash basis accounting, which shows only the cash transactions conducted by a company. Although a high degree of certainty exists with reporting only cash transactions, much information is hidden. For instance, a company with growing revenues that makes the majority of its sales on credit would be understating its revenues for each period if it reported only cash transactions. On an accrual basis, revenues reflect all transactions that occurred, whether they transacted on a cash basis or credit-extended basis. Estimates enter the process because some facts related to events occurring in a particular period might not yet be known. Estimates can be well grounded in reality and applied to present a complete picture of the events affecting a company, or they can be management tools for achieving a desired financial picture.

To illustrate how estimates can affect financial reporting, consider sales made on credit. A company sells $\$ 1,000,000$ of merchandise on credit and records the sale just before year end. Under accrual accounting, that amount is included in revenues and accounts receivable. The company's managers know from experience that they will never collect every dollar of the accounts receivable. Past experience is that, on average, only $97 \%$ of accounts receivable is collected. The company would estimate an amount of the uncollectible accounts at the time the sales occur and record an uncollectible accounts expense of $\$ 30,000$, lowering earnings. The other side of the entry would be to establish an allowance for uncollectible accounts of $\$ 30,000$. This allowance would be a contra asset account, presented as an offset to accounts receivable. The accounts receivable, net of the allowance for uncollectible accounts, would be stated at $\$ 970,000$, which is the amount of cash the company ultimately expects to receive. If cash-basis accounting had been used, no revenues or accounts receivable would have been reported even though sales of merchandise had occurred. Accrual accounting, which contains estimates about future events, provides a much fuller picture of what transpired in the period than pure cash-basis accounting.

Yet, accrual accounting poses temptations to managers to manage the numbers, rather than to manage the business. Suppose a company's managers realize that the company will not meet analysts' consensus estimates in a particular quarter, and further, their bonus pay is dependent on reaching specified earnings targets. By offering special payment terms, or discounts, the managers may induce customers to take delivery of products that they would normally not order, so they could ship the products on FOB shipping point terms and recognize the revenues in the current quarter. They could even be so bold as to ship the goods under those terms even if the customer did not order them, in the hope that the customer would keep them or, at worst, return them in the next accounting period. Their aim would be to move the product off the company's property with FOB shipping point terms.

To further improve earnings in order to meet the consensus estimates, the company's managers might revise their estimate of the uncollectible accounts. The company's collection history shows a typical non-collection rate of $3 \%$ of sales, but the managers might rationalize the use of a $2 \%$ non-collection rate. This change will reduce the allowance for uncollectible accounts and the expense reported for the period. The managers might be able to justify the reduction on the grounds that the sales occurred in a part of the country that was experiencing an improved economic outlook, or that the company's collection history had been biased by the inclusion of a prolonged period of economic downturn. Whatever the justification, it would be hard to prove that the new estimate was completely right or wrong until time had passed. Because proof of the reliability of estimates is rarely available at the time the estimate is recorded, managers have a readily available means for manipulating earnings at their discretion.

ConAgra Foods Inc. provides an example of how the allowance for uncollectible accounts may be manipulated to manage earnings. ${ }^{21}$ A subsidiary, called United Agri-Products (UAP), engaged in several improper accounting practices, one of them being the understatement of uncollectible accounts expense for several years. Exhibit 18 presents an excerpt from the SEC's Accounting and Auditing Enforcement Release.

\section{Exhibit 18: SEC's Accounting and Auditing Enforcement Release Regarding United Agri-Products}
... Generally, UAP's policy required that accounts which were past due between 90 days and one year should be reserved at $50 \%$, and accounts over one year past due were to be reserved at $100 \%$.

... In FY 1999 and continuing through FY 2000, UAP had substantial bad debt problems. In FY 2000, certain former UAP senior executives were informed that UAP needed to record an additional $\$ 50$ million of bad debt expense. Certain former UAP senior executives were aware that in FY 1999 the size of the bad debt at certain IOCs had been substantial enough that it could have negatively impacted those IOC's ability to achieve PBT (profits before taxes) targets. In addition, just prior to the end of UAP's FY 2000, the former UAP COO (chief operating officer), in the presence of other UAP employees, ordered that UAP's bad debt reserve be reduced by $\$ 7$ million in order to assist the Company in meeting its PBT target for the fiscal year.

... At the end of FY 2000, former UAP senior executives reported financial results to ConAgra which they knew, or were reckless in not knowing, overstated UAP's income before income taxes because UAP had failed to record sufficient bad debt expense. The misconduct with respect to bad debt expense caused ConAgra to overstate its reported income before income taxes by $\$ 7$ million, or $1.13 \%$, in FY 2000. At the Agricultural Products' segment level, the misconduct caused that segment's reported operating profit to be overstated by $5.05 \%$.

Deferred-tax assets provide a similar example of choices in estimates affecting the earnings outcome. Deferred-tax assets may arise when a company reports a net operating loss under tax accounting rules. A company may record a deferred-tax asset based on the expectation that losses in the reporting period will offset expected future profits and reduce the company's future income tax liability. Accounting standards require that the deferred tax asset be reduced by a "valuation allowance" to account for the possibility that the company will be unable to generate enough profit to use all of the available tax benefits. ${ }^{22}$

Assume a company loses $€ 1$ billion in 2012, generating a net operating loss of the same amount for tax purposes. The company's income tax rate is $25 \%$, and it will be able to apply the net operating loss to its taxable income for the next 10 years. The net operating loss results in a deferred tax asset with a nominal value of $€ 250$ million $(25 \% \times € 1,000,000,000)$. Initial recognition would result in a deferred tax asset of $€ 250$ million and a credit to deferred tax expense of $€ 250$ million. The company must address the question of whether or not the $€ 250$ million will ever be completely applied to future income. It may be experiencing increased competition and other circumstances

21 Accounting and Auditing Enforcement Release No. 2542, "SEC v. James Charles Blue, Randy Cook, and Victor Campbell,) United States District Court for the District of Colorado, Civ. Action No. 07-CV-00095 REB-MEH (17 January 2007).

22 See Accounting Standards Codification 740-10-30-16 to 25, "Establishment of a Valuation Allowance for Deferred Tax Assets." that resulted in the $€ 1$ billion loss, and it may be unreasonable to assume it will have taxable income against which to apply the loss. In fact, the company's managers might believe it is reasonable to assume only that it will survive for five years, and with marginal profitability. The $€ 250$ million deferred tax asset is thus overstated if no valuation allowance is recorded to offset it.

The managers believe that only $€ 100$ million of the net operating losses will actually be applied to the company's taxable income. That belief implies that only $€ 25$ million of the tax benefits will ever be realized. The deferred tax assets reported on the balance sheet should not exceed this amount. The company should record a valuation allowance of $€ 225$ million, which would offset the deferred tax asset balance of $€ 250$ million, resulting in a net deferred tax asset balance of $€ 25$ million. There would also be $\mathrm{a} € 225$ million credit to the deferred tax provision. It is important to understand that the valuation allowance should be revised whenever facts and circumstances change.

The ultimate value of the deferred tax asset is driven by management's outlook for the future-and that outlook may be influenced by other factors. If the company needs to stay in compliance with debt covenants and needs every euro of value that can be justified by the outlook, its managers may take a more optimistic view of the future and keep the valuation allowance artificially low (in other words, the net deferred tax asset high).

PowerLinx Inc. provides an example of how over-optimism about the realizability of a deferred tax asset can lead to misstated financial reports. PowerLinx was a maker of security video cameras, underwater cameras, and accessories. Aside from fraudulently reporting $90 \%$ of its fiscal year 2000 revenue, PowerLinx had problems with valuation of its deferred tax assets. Exhibit 19 provides an excerpt from the SEC's Accounting and Auditing Enforcement Release with emphasis added. ${ }^{23}$

\section{Exhibit 19: SEC's Accounting and Auditing Enforcement Release Regarding PowerLinx}
PowerLinx improperly recorded on its fiscal year 2000 balance sheet a deferred tax asset of $\$ 1,439,322$ without any valuation allowance. The tax asset was material, representing almost forty percent of PowerLinx's total assets of $\$ 3,841,944$. PowerLinx also recorded deferred tax assets of $\$ 180,613$, $\$ 72,907$, and $\$ 44,921$, respectively, in its financial statements for the first three quarters of 2000.

PowerLinx did not have a proper basis for recording the deferred tax assets. The company had accumulated significant losses in 2000 and had no historical operating basis from which to conclude that it would be profitable in future years. Underwater camera sales had declined significantly and the company had devoted most of its resources to developing its SecureView product. The sole basis for PowerLinx's "expectation" of future profitability was the purported $\$ 9$ million backlog of SecureView orders, which management assumed would generate taxable income; however, this purported backlog, which predated Bauer's hiring, did not reflect actual demand for SecureView cameras and, consequently, was not a reasonable or reliable indicator of future profitability.

Another example of misstated financial results caused by improper reflection of the realizability of a deferred tax asset occurred with Hampton Roads Bankshares Inc. ("HRBS"), a commercial bank with deteriorating loan portfolio quality and commensurate losses in the years following the financial crisis. The company reported a deferred tax asset related to its loan losses; however, it did not establish a valuation allowance against its deferred tax asset. This decision was based on dubious projections

23 Accounting and Auditing Enforcement Release No. 2448, "In the Matter of Douglas R. Bauer, Respondent," SEC (27 June 2006): \href{http://www.sec.gov/litigation/admin/2006/34-54049.pdf}{www.sec.gov/litigation/admin/2006/34-54049.pdf}. indicating that the company would earn the necessary future taxable income "to fully utilize the [deferred tax asset] DTA over the applicable carry-forward period."24 Over time, it became clear that the earnings projections were not realistic, and ultimately the company restated its financial results to include a valuation allowance against almost the entire deferred tax asset. Exhibit 18 presents an excerpt from the company's amended Form 10-Q/A containing the restatement.

\section{Exhibit 20: Hampton Roads Bankshares, Inc. Form 10-Q/A filed August 13,}
 2010 [Excerpt from footnotes]\section{NOTE B - RESTATEMENT OF CONSOLIDATED FINANCIAL STATEMENTS}
Subsequent to filing the Company's annual report on Form 10-K for the year ended December 31, 2009 and its Form 10-Q for the three months ended March 31, 2010 the Company determined that a valuation allowance on its deferred tax assets should be recognized as of December 31, 2009. The Company decided to establish a valuation allowance against the deferred tax asset because it is uncertain when it will realize this asset.

Accordingly, the December 31, 2009 consolidated balance sheet and the March 31, 2010 consolidated financial statements have been restated to account for this determination. The effect of this change in the consolidated financial statements was as follows (in thousands, except per share amounts).

Consolidated Balance Sheet at March 31, 2010

\begin{center}
\begin{tabular}{lccc}
\hline
 & As Reported & Adjustment & As Restated \\
\hline
Deferred tax assets, net & $\$ 70,323$ & $\$(70,323)$ & - \\
Total assets & $3,016,470$ & $(70,323)$ & $\$ 2,946,147$ \\
Retained earnings deficit & $(158,621)$ & $(70,323)$ & $(228,944)$ \\
Total shareholder's equity & 156,509 & $(70,323)$ & 86,186 \\
Total Liabilities and sharehold- & $3,016,470$ & $(70,323)$ & $2,946,147$ \\
ers' equity &  &  &  \\
\hline
\end{tabular}
\end{center}

Consolidated Balance Sheet at December 31, 2009

\begin{center}
\begin{tabular}{lccc}
\hline
 & As Reported & Adjustment & As Restated \\
\hline
Deferred tax assets, net & $\$ 56,380$ & $\$(55,983)$ & $\$ 397$ \\
Total assets & $2,975,559$ & $(55,983)$ & $2,919,576$ \\
Retained earnings deficit & $(132,465)$ & $(55,983)$ & $(188,488)$ \\
Total shareholder's equity & 180,996 & $(55,983)$ & 125,013 \\
Total Liabilities and sharehold- & $2,975,559$ & $(55,983)$ & $2,919,576$ \\
ers' equity &  &  &  \\
\hline
\end{tabular}
\end{center}

24 Accounting and Auditing Enforcement Release No. 3600, "In the Matter of Hampton Roads Bankshares Inc., Respondent," SEC (5 December 2014) \href{https://www.sec.gov/litigation/admin/2014/34-73750.pdf}{https://www.sec.gov/litigation/admin/2014/34-73750.pdf}. Another example of how choices and estimates can affect reported results lies in the selection of a depreciation method for allocating the cost of long-lived assets to accounting periods subsequent to their acquisition. A company's managers may choose to depreciate long-lived assets (1) on a straight-line basis, with each year bearing the same amount of depreciation expense; (2) using an accelerated method, with greater depreciation expense recognition in the earlier part of an asset's life; or (3) using an activity-based depreciation method, which allocates depreciation expense based on units of use or production. Depreciation expense is affected by another set of choices and estimates regarding the salvage value of the assets being depreciated. A salvage value of zero will always increase depreciation expense under any method compared with the choice of a non-zero salvage value.

Assume a company invests $\$ 1,000,000$ in manufacturing equipment and expects it to have a useful economic life of 10 years. During its expected life, the equipment will produce 400,000 units of product, or $\$ 2.50$ depreciation expense per unit produced. When it is disposed of at the end of its expected life, the company's managers expect to realize no value for the equipment. The following table shows the differences in the three alternative methods of depreciation: straight-line, accelerated on a double-declining balance basis, and units-of-production method, with no salvage value assumed at the end of the equipment's life.

\begin{center}
\includegraphics[max width=\textwidth]{2023_05_04_b5cfa4f1bc883752f121g-553}
\end{center}

${ }^{1}$ Declining balance rate of $20 \%$ calculated as 10-year life being equivalent to 10\% annual depreciation rate, multiplied by $2=20 \%$.

The straight-line method allocates the cost of the equipment evenly to all 10 years of the equipment's life. The double-declining balance method will have a higher allocation of cost to the earlier years of the equipment's life. As its name implies, the depreciation expense will decline in each succeeding year because it is based on a fixed rate applied to a declining balance. The rate used was double the straight-line rate, but it could have been any other rate that the company's managers believed was representative of the way the actual equipment depreciation occurred. Notice that the double-declining balance method also results in an incomplete depreciation of the machine at the end of 10 years; a balance of $\$ 107,374 .(=\$ 1,000,000-\$ 892,626)$ remains at the end of the expected life, which will result in a loss upon the retirement of the equipment if the company's expectation of zero salvage value turns out to be correct. Some companies may choose to depreciate the equipment to its expected salvage value, zero in this case, in its final year of use. Some companies may use a policy of switching to straight-line depreciation after the mid-life of its depreciable assets in order to fully depreciate them. That particular pattern is coincidentally displayed in the units-of-production example, in which the equipment is used most heavily in the earliest part of its useful life, and then levels off to much less utilization in the second half of the expected life.

Exhibit 21 shows the different expense allocation patterns of the methods over the same life. Each will affect earnings differently.

\section{Exhibit 21: Expense Allocation Patterns of Different Depreciation Methods}
\begin{center}
\includegraphics[max width=\textwidth]{2023_05_04_b5cfa4f1bc883752f121g-554}
\end{center}

The company's managers could justify any of these methods. Each might fairly represent the way the equipment will be consumed over its expected economic life, which is a subjective estimate itself. The choices of methods and lives can profoundly affect reported income. These choices are not proven right or wrong until far into the future-but managers must estimate their effects in the present.

Exhibit 22 shows the effects of the three different methods on operating profit and operating profit margins, assuming that the production output of the equipment generates revenues of $\$ 500,000$ each year and $\$ 200,000$ of cash operating expenses are incurred, leaving $\$ 300,000$ of operating profit before depreciation expense.

Exhibit 22: Effects of Depreciation Methods on Operating Profit

\begin{center}
\begin{tabular}{|c|c|c|c|}
\hline
\multirow[b]{2}{*}{Year} & \multicolumn{3}{|c|}{Straight Line} \\
\hline
 & Depreciation & Operating Profit & Operating Profit Margin \\
\hline
1 & $\$ 100,000$ & $\$ 200,000$ & $40.0 \%$ \\
\hline
2 & 100,000 & 200,000 & $40.0 \%$ \\
\hline
3 & 100,000 & 200,000 & $40.0 \%$ \\
\hline
4 & 100,000 & 200,000 & $40.0 \%$ \\
\hline
5 & 100,000 & 200,000 & $40.0 \%$ \\
\hline
6 & 100,000 & 200,000 & $40.0 \%$ \\
\hline
7 & 100,000 & 200,000 & $40.0 \%$ \\
\hline
8 & 100,000 & 200,000 & $40.0 \%$ \\
\hline
\end{tabular}
\end{center}

\begin{center}
\begin{tabular}{lccc}
\hline
 & \multicolumn{3}{c}{Straight Line} \\
\hline
Year & Depreciation & Operating Profit & Operating Profit Margin \\
\hline
9 & 200,000 & $40.0 \%$ &  \\
10 & 100,000 & 200,000 & $40.0 \%$ \\
\hline
\end{tabular}
\end{center}

Double Declining Balance

\begin{center}
\begin{tabular}{lccc}
\hline
Year & Depreciation & Operating Profit & Operating Profit Margin \\
\hline
1 & $\$ 200,000$ & $\$ 100,000$ & $20.0 \%$ \\
2 & 160,000 & 140,000 & $28.0 \%$ \\
3 & 128,000 & 172,000 & $34.4 \%$ \\
4 & 102,400 & 197,600 & $39.5 \%$ \\
5 & 81,920 & 218,080 & $43.6 \%$ \\
6 & 65,536 & 234,464 & $46.9 \%$ \\
7 & 52,429 & 247,571 & $49.5 \%$ \\
8 & 41,943 & 258,057 & $51.6 \%$ \\
9 & 33,554 & 266,446 & $53.3 \%$ \\
10 & $134,218^{*}$ & 165,782 & $33.2 \%$ \\
\end{tabular}
\end{center}

\begin{center}
\begin{tabular}{lccc}
\hline
\multicolumn{3}{c}{Units of Production} &  \\
\hline
Year & Depreciation & Operating Profit & Operating Profit Margin \\
\hline
1 & $\$ 225,000$ & $\$ 75,000$ & $15.0 \%$ \\
2 & 200,000 & 100,000 & $20.0 \%$ \\
3 & 175,000 & 125,000 & $25.0 \%$ \\
4 & 150,000 & 150,000 & $30.0 \%$ \\
5 & 125,000 & 175,000 & $35.0 \%$ \\
6 & 25,000 & 275,000 & $55.0 \%$ \\
7 & 275,000 & $55.0 \%$ &  \\
8 & 25,000 & 275,000 & $55.0 \%$ \\
9 & 275,000 & $55.0 \%$ &  \\
10 & 25,000 & 55,000 &  \\
\hline
 & 25,000 &  &  \\
\hline
\end{tabular}
\end{center}

\begin{itemize}
  \item Includes $\$ 107,374$ of undepreciated basis, treated as depreciation expense in final year of service.
\end{itemize}

The straight-line method shows consistent operating profit margins, and the other two methods show varying degrees of increasing operating profit margins as the depreciation expense decreases over time.

The example above shows the differences among alternative methods, but even more depreciation expense variation is possible by changing estimated lives and assumptions about salvage value. For instance, change the expected life assumption to 5 years from 10 and add an expectation that the equipment will have a 10\% salvage value at the end of its expected life. Exhibit 23 shows the revised depreciation calculations. Notice that under the double-declining balance method, the depreciation rate is applied to the gross cost, unlike the other two methods. The straight-line method and the units-of-production method subtract the salvage value from the cost before depreciation expense is calculated. Also note that the assumption about the usage of the equipment is revised so that it is depreciated only to its salvage value of $\$ 100,000$ by the end of its estimated life. The total depreciation under each method is $\$ 900,000$.

\section{Exhibit 23: Depreciation Calculations for Each Method in Changed Scenario}
\begin{center}
\begin{tabular}{|c|c|c|c|c|c|c|c|}
\hline
 & $\begin{array}{c}\text { Straight-Line } \\ \text { Method }\end{array}$ & Double & clining Ba & ce Method & Uni & 5-of-Productior & Method \\
\hline
Year & $\begin{array}{c}\text { Depreciation } \\ \text { Expense }\end{array}$ & Balance & $\begin{array}{c}\text { Declining } \\ \text { Balance } \\ \text { Rate }^{1}\end{array}$ & $\begin{array}{c}\text { Depreciation } \\ \text { Expense }\end{array}$ & $\begin{array}{c}\text { Units } \\ \text { Produced }\end{array}$ & $\begin{array}{c}\text { Depreciation } \\ \text { Rate/Unit }\end{array}$ & $\begin{array}{c}\text { Depreciation } \\ \text { Expense }\end{array}$ \\
\hline
1 & $\$ 180,000$ & $\$ 1,000,000$ & $40 \%$ & $\$ 400,000$ & 100,000 & $\$ 2.25$ & $\$ 225,000$ \\
\hline
2 & 180,000 & 600,000 & $40 \%$ & 240,000 & 90,000 & $\$ 2.25$ & 202,500 \\
\hline
3 & 180,000 & 360,000 & $40 \%$ & 144,000 & 80,000 & $\$ 2.25$ & 180,000 \\
\hline
4 & 180,000 & 216,000 & $40 \%$ & 86,400 & 70,000 & $\$ 2.25$ & 157,500 \\
\hline
5 & 180,000 & 129,600 & $40 \%$ & $29,600^{2}$ & 60,000 & $\$ 2.25$ & 135,000 \\
\hline
Total & $\$ 900,000$ &  &  & $\$ 900,000$ & 400,000 &  & $\$ 900,000$ \\
\hline
\end{tabular}
\end{center}

${ }^{1}$ Declining balance rate of $40 \%$ calculated as 5-year life being equivalent to 20\% annual depreciation rate, multiplied by $2=40 \%$.

${ }^{2}$ Depreciation calculated as $\$ 29,600$ instead of $40 \% \times \$ 129,600$. Rote application of the declining-balance rate would have resulted in $\$ 51,840$ of expense, which would have depreciated the asset below salvage value.

Exhibit 24 shows the different expense allocation patterns of the methods over the five-year expected life, and assuming a $10 \%$ salvage value. Although each method is distinctly different in the timing of the cost allocation over time, the variation is less pronounced than over the longer life used in the previous example.

Exhibit 24: Expense Allocation Patterns of Depreciation Methods in Changed Scenario

\begin{center}
\includegraphics[max width=\textwidth]{2023_05_04_b5cfa4f1bc883752f121g-556}
\end{center}

One of the clearest examples of how choices affect both the balance sheet and income statement can be found in capitalization practices. In classifying a payment made, management must determine whether the payment will benefit only the current period-making it an expense-or whether it will benefit future periods, leading to classification as a cost to be capitalized as an asset. This management judgment embodies an implicit forecast of how the item acquired by the payment will be used, or not used, in the future.

That judgment can be biased by the powerful effect a capitalization policy can have on earnings. Every amount capitalized on the balance sheet as a building, an item of inventory, a deferred cost, or any "other asset" is an amount that does not get recognized as an expense in the reporting period.

A real-life example can be found in the case of WorldCom Inc., a telecom concern that grew rapidly in the late 1990s. Much of WorldCom's financial reporting was eventually found to be fraudulent. An important part of the misreporting centered on its treatment of what is known in the telecom industry as "line costs". These are the costs of carrying a voice call or data transmission from its starting point to its ending point, and they represented WorldCom's largest expense. WorldCom's chief financial officer decided to capitalize such costs instead of treating them as an operating expense. As a consequence, from the second quarter of 1999 through the first quarter of 2002, WorldCom increased its operating income by $\$ 7$ billion. In three of the five quarters in which the improper line cost capitalization took place, WorldCom would have recognized pre-tax losses instead of profits. 25

Similarly, acquisitions are an area in which managers must exercise judgment. An allocation of the purchase price must be made to all of the different assets acquired based on their fair values, and those fair values are not always objectively verifiable. Management may have to make its own estimate of fair values for assets acquired, and it may be biased towards a low estimate for the values of depreciable assets in order to depress future depreciation expense. Another benefit to keeping depreciable asset values low is that the amount of the purchase price that cannot be allocated to specific assets is classified as goodwill, which is neither depreciated nor amortized in future reporting periods.

Goodwill reporting has choices of its own. Although goodwill has no effect on future earnings when unimpaired, annual testing of its fair value may reveal that the excess of price paid over the fair value of assets may not be recoverable, which should lead to a write-down of goodwill. The estimation process for the fair value of goodwill may depend heavily on projections of future performance. Those projections may be biased upward in order to avoid a goodwill write-down.

\textbackslash section\{HOW CHOICES THAT AFFECT THE CASH FLOW STATEMENT

describe accounting methods (choices and estimates) that could be used to manage earnings, cash flow, and balance sheet items

The cash flow statement consists of three sections: the operating section, which shows the cash generated or used by operations; the investing section, which shows cash used for investments or provided by their disposal; and the financing section, which shows transactions attributable to financing activities.

25 See Report of Investigation by the Special Investigative Committee of the Board of Directors of WorldCom, Inc., by Dennis R. Beresford, Nicholas deB. Katzenbach, \& C.B. Rogers, Jr.PP 9-11: www.sec. gov/Archives/edgar/data/723527/000093176303001862/dex991.htm. The operating section is closely scrutinized by investors. Many of them consider it a reality check on reported earnings, on the grounds that earnings attributable to accrual accounting only and unsupported by actual cash flows may indicate earnings manipulation. Such investors believe that amounts shown for cash generated by operations is more insulated from managerial manipulation than the income statement. Cash generated by operations can be managed to an extent, however.

The operating section of the cash flow statement can be shown either under the direct method or the indirect method. Under the direct method, "entities are encouraged to report major classes of gross cash receipts and gross cash payments and their arithmetic sum - the net cash flow from operating activities." ${ }^{26}$ In practice, companies rarely use the direct method. Instead, they use the indirect method, which shows a reconciliation of net income to cash provided by operations. The reconciliation shows the non-cash items affecting net income along with changes in working capital accounts affecting cash from operations. Exhibit 25 provides an example of the indirect presentation method.

\section{Exhibit 25: Indirect Presentation Method}
$\begin{array}{ll}\text { Cash Flows from Operating Activities (\$ millions) } & 2018\end{array}$

Net income $\quad \$ 3,000$

Adjustments to reconcile net income to net cash provided by operating activities:

$\begin{array}{lr}\text { Provision for doubtful receivables } & 10\end{array}$

$\begin{array}{lr}\text { Provision for depreciation and amortization } & 1,000\end{array}$

Goodwill impairment charges $\quad 35$

$\begin{array}{lr}\text { Share-based compensation expense } & 100\end{array}$

$\begin{array}{lr}\text { Provision for deferred income taxes } & 200\end{array}$

Changes in assets and liabilities:

Trade, notes and financing receivables related to sales $\quad(2,000)$

$\begin{array}{lr}\text { Inventories } & (1,500)\end{array}$

\begin{center}
\begin{tabular}{lc}
Accounts payable & 1,200 \\
Accrued income taxes payable/receivable & $(80)$ \\
Retirement benefits & 90 \\
Other & \begin{tabular}{c}
$(250)$ \\
\hline
Net cash provided by operating activities \\
\hline
\end{tabular} \\
\hline
\end{tabular}
\end{center}

Whether the indirect method or direct method is used, simple choices exist for managers to improve the appearance of cash flow provided by operations without actually improving it. One such choice is in the area of accounts payable management, shaded in Exhibit 25. Assume that the accounts payable balance is $\$ 5,200$ million at the end of the period, an increase of $\$ 1,200$ million from its previous year-end balance of $\$ 4,000$ million. The $\$ 1,200$ million increase in accounts payable matched increased expenses and/or assets but did not require cash. If the company's managers had further delayed paying creditors $\$ 500$ million until the day after the balance sheet date, they could have increased the cash provided by operating activities by $\$ 500$ million.

26 Accounting Standards Codification Section 230-10-45-25, "Reporting Operating, Investing, and Financing Activities." The direct method and indirect method are similar in IFRS, as addressed in IAS 7, Paragraph 18. If the managers believe that cash generated from operations is a metric of focus for investors, they can impress them with artificially strong cash flow by simply stretching the accounts payable credit period.

What might alert investors to such machinations? They need to examine the composition of the operations section of the cash flow statement-if they do not, then nothing will ever alert them. Studying changes in the working capital can reveal unusual patterns that may indicate manipulation of the cash provided by operations.

Another practice that might lead an investor to question the quality of cash provided by operations is to compare a company's cash generation with an industry-wide level or with the cash operating performance of one or more similar competitors. Cash generation performance can be measured several ways. One way is to compare the relationship between cash generated by operations and net income. Cash generated by operations in excess of net income signifies better quality of earnings, whereas a chronic excess of net income over cash generated by operations should be a cause for concern; it may signal the use of accounting methods to simply raise net income instead of depicting financial reality. Another way to measure cash generation performance is to compare cash generated by operations with debt service, capital expenditures, and dividends (if any). When there is a wide variance between the company's cash generation performance and that of its benchmarks, investors should seek an explanation and carefully examine the changes in working capital accounts.

Because investors may focus on cash from operations as an important metric, managers may resort to managing the working capital accounts as described in order to present the most favorable picture. But there are other ways to do this. A company may misclassify operating uses of cash into either the investing or financing sections of the cash flow statement, which enhances the appearance of cash generated by operating activities.

Dynegy Inc. provides an example of manipulation of cash from operations through clever construction of contracts and assistance from an unconsolidated special purpose entity named ABG Gas Supply LLC (ABG). In April 2001, Dynegy entered into a contract for the purchase of natural gas from ABG. According to the contract, Dynegy would purchase gas at below-market rates from ABG for nine months and sell it at the current market rate. The nine-month term coincided with Dynegy's 2001 year-end and would result in gains backed by cash flows. Dynegy also agreed to buy gas at above-market rates from $\mathrm{ABG}$ for the following 51 months and sell it at the current market rate. The contract was reported at its fair value at the end of fiscal year 2001. It had no effect on net income for the year. The earlier portion of the contract resulted in a gain, supported by $\$ 300$ million of cash flow, but the latter portion of the contract resulted in non-cash losses that offset the profit. The mark-to-market rules required the recognition of both gains and losses from all parts of the contract, and hence the net effect on earnings was zero.

In April 2002, a Wall Street Journal article exposed the chicanery, thanks to leaked documents. The SEC required Dynegy to restate the cash flow statement by reclassifying $\$ 300$ million from the operating section of the cash flow statement to the financing section, on the grounds that Dynegy had used ABG as a conduit to effectively borrow $\$ 300$ million from Citigroup. The bank had extended credit to ABG, which it used to finance its losses on the contract (Lee, 2012).

Another area of flexibility in cash flow reporting is found in the area of interest capitalization, which creates differences between total interest payments and total interest costs. ${ }^{27}$ Assume a company incurs total interest cost of $\$ 30,000$, composed of $\$ 3,000$ of discount amortization and $\$ 27,000$ of interest payments. Of the $\$ 30,000$, two-thirds of it $(\$ 20,000)$ is expensed; the remaining third $(\$ 10,000)$ is capitalized as plant assets. If the company uses the same interest expense/capitalization proportions

27 See Nurnberg and Largay (1998) and Nurnberg (2006). to allocate the interest payments between operating and investing activities, then it will report $\$ 18,000(2 / 3 \times \$ 27,000)$ as an operating outflow and $\$ 9,000(1 / 3 \times \$ 27,000)$ as an investing outflow. The company might also choose to offset the entire $\$ 3,000$ of non-cash discount amortization against the $\$ 20,000$ treated as expense, resulting in an operating outflow as low as $\$ 17,000$, or as much as $\$ 20,000$ if it allocated all of the non-cash discount amortization to interest capitalized as investing activities. Similarly, the investing outflow could be as much as $\$ 10,000$ or as little as $\$ 7,000$, depending on the treatment of the non-cash discount amortization. There are choices within the choices, all in areas where investors believe choices do not even exist. Nurnberg and Largay (1998) note that companies apparently favor the method that reports the lowest operating outflow, presumably to maximize reported cash from operations.

Investors and analysts need to be aware that presentation choices permitted in IAS 7, "Statement of Cash Flows," offer flexibility in classification of certain items in the cash flow statement. This flexibility can drastically change the results in the operating section of the cash flow statement. An excerpt from IAS 7, Paragraphs 33 and 34, provides the background:

\begin{enumerate}
  \setcounter{enumi}{32}
  \item Interest paid and interest and dividends received are usually classified as operating cash flows for a financial institution. However, there is no consensus on the classification of these cash flows for other entities. Interest paid and interest and dividends received may be classified as operating cash flows because they enter into the determination of profit or loss. Alternatively, interest paid and interest and dividends received may be classified as financing cash flows and investing cash flows respectively, because they are costs of obtaining financial resources or returns on investments.

  \item Dividends paid may be classified as a financing cash flow because they are a cost of obtaining financial resources. Alternatively, dividends paid may be classified as a component of cash flows from operating activities in order to assist users to determine the ability of an entity to pay dividends out of operating cash flows. [Emphasis added.]

\end{enumerate}

By allowing a choice of operating or financing for the placement of interest and dividends received or paid, IAS 7 gives a company's managers the opportunities to select the presentation that gives the best-looking picture of operating performance. An example is Norse Energy Corp. ASA, a Norwegian gas explorer and producer, which changed its classifications of interest paid and interest received in 2007 (Gordon, Henry, Jorgensen, and Linthicum, 2017). Interest paid was switched to financing instead of decreasing cash generated from operations. Norse Energy also switched its classification of interest received to investing from operating cash flow. The net effect of these changes was to report positive, rather than negative, operating cash flows in both 2007 and 2008. With these simple changes, the company could also change the perception of its operations. The cash flow statement formerly presented the appearance of a company with operations that used more cash than it generated, and it possibly raised questions about the sustainability of operations. After the revision, the operating section of the cash flow statement depicted a much more viable operation.

Exhibit 26 shows the net effect of the reclassifications on Norse Energy's cash flows.

\section{Exhibit 26: Reclassification of Cash Flows (amounts in \$ millions)}
\begin{center}
\begin{tabular}{|c|c|c|c|c|c|c|}
\hline
 & \multicolumn{2}{|c|}{$\begin{array}{l}\text { As Reported } \\
\text { (following } 2007 \\
\text { reclassification) }\end{array}$} & \multicolumn{2}{|c|}{$\begin{array}{l}\text { Adjustments, If No } \\
\text { Reclassification* }\end{array}$} & \multicolumn{2}{|c|}{$\begin{array}{c}\text { Pro-forma } \\
\text { (if no reclassification) }\end{array}$} \\
\hline
 & 2008 & 2007 & 2008 & 2007 & 2008 & 2007 \\
\hline
Operating & $\$ 5.30$ & $\$ 2.80$ & $(\$ 13.70)$ & $(\$ 14.40)$ & $(\$ 8.40)$ & $(\$ 11.60)$ \\
\hline
Investing & $\$ 0.90$ & $(\$ 56.80)$ & $(\$ 9.00)$ & $(\$ 3.50)$ & $(\$ 8.10)$ & $(\$ 60.30)$ \\
\hline
Financing & $(\$ 16.60)$ & $\$ 34.50$ & $\$ 22.70$ & $\$ 17.90$ & $\$ 6.10$ & $\$ 52.40$ \\
\hline
Total & $(\$ 10.40)$ & $(\$ 19.50)$ & $\$ 0$ & $\$ 0$ & $(\$ 10.40)$ & $(\$ 19.50)$ \\
\hline
\end{tabular}
\end{center}

\begin{itemize}
  \item The adjustments reverse the addition of interest received to investing and instead add it to operating. The adjustments also reverse the deduction of interest paid from financing and instead subtract it from operating.
\end{itemize}

\section{CHOICES THAT AFFECT FINANCIAL REPORTING}
describe accounting methods (choices and estimates) that could be used to manage earnings, cash flow, and balance sheet items

Exhibit 27 summarizes some of the areas where choices can be made that affect financial reports.

\section{Exhibit 27: Areas Where Choices and Estimates Affect Financial Reporting}
Area of Choice/

Estimate

Revenue recognition Analyst Concerns

\begin{itemize}
  \item How is revenue recognized: upon shipment or upon delivery of goods?

  \item Is the company engaging in "channel stuffing"-the practice of overloading a distribution channel with more product than it is normally capable of selling? This can be accomplished by inducing customers to buy more through unusual discounts, the threat of near-term price increases, or both-or simply by shipping goods that were not ordered. These transactions may be corrected in a subsequent period and may result in restated results. Are accounts receivable relative to revenues abnormally high for relative to the company's history or to its peers? If so, channel stuffing may have occurred.

  \item Is there unusual activity in the allowance for sales returns relative to past history?

  \item Does the company's days sales outstanding show any collection issues that might indicate shipment of unneeded or unwanted goods to customers? Area of Choice/

\end{itemize}

Estimate

Analyst Concerns

\begin{itemize}
  \item Does the company engage in "bill-and-hold" transactions? This is when a customer purchases goods but requests that they remain with the seller until a later date. This kind of transaction makes it possible for a seller to manufacture fictitious sales by declaring end-of-period inventory as "sold but held," with a minimum of effort and phony documentation.

  \item Does the company use rebates as part of its marketing approach? If so, how significantly do the estimates of rebate fulfillment affect net revenues, and have any unusual breaks with history occurred?

  \item Does the company separate its revenue arrangements into multiple deliverables of goods or services? This area is one of great revenue recognition flexibility, and also one that provides little visibility to investors. They simply cannot examine a company's arrangements and decide for themselves whether revenue has been properly allocated to different components of a contract. If a company uses multiple deliverable arrangements with its customers as a routine matter, investors might be more sensitive to revenue reporting risks. In seeking a comfort level, they might ask the following questions: Does the company explain adequately how it determines the different allocations of deliverables and how revenue is recognized on each one? Do deferred revenues result? If not, does it seem reasonable that there are no deferred revenues for this kind of arrangement? Are there unusual trends in revenues and receivables, particularly with regard to cash conversion? If an investor is not satisfied with the answers to these questions, he or she might be more comfortable with other investment choices.

\end{itemize}

Long-lived assets: Depreciation policies - Do the estimated life spans of the associated assets make sense, or are they unusually low compared with others in the same industry?

\begin{itemize}
  \item Have there been changes in depreciable lives that have a positive effect on current earnings?

  \item Do recent asset write-downs indicate that company policy on asset lives might need to be reconsidered?

\end{itemize}

Intangibles:

Capitalization policies

\begin{itemize}
  \item Does the company capitalize expenditures related to intangibles, such as software? Does its balance sheet show any R\&D capitalized as a result of acquisitions? Or, if the company is an IFRS filer, has it capitalized any internally generated development costs?

  \item How do the company's capitalization policies compare with the competition?

  \item Are amortization policies reasonable?

\end{itemize}

Allowance for doubt-

\begin{itemize}
  \item Are additions to such allowances lower or higher than in the ful accounts/loan loss past?
\end{itemize}

reserves - Does the collection experience justify any difference from historical provisioning?

\begin{itemize}
  \item Is there a possibility that any lowering of the allowance may be the result of industry difficulties along with the difficulty of meeting earnings expectations? Area of Choice/ Estimate
\end{itemize}

Inventory cost methods Analyst Concerns

\begin{itemize}
  \item Does the company use a costing method that produces fair reporting results in view of its environment? How do its inventory methods compare with others in its industry? Are there differences that will make comparisons uneven if there are unusual changes in inflation?

  \item Does the company use reserves for obsolescence in its inventory valuation? If so, are they subject to unusual fluctuations that might indicate adjusting them to arrive at a specified earnings result?

  \item If a company reports under US GAAP and uses last-in-first-out (LIFO) inventory accounting, does LIFO liquidation (the assumed sale of old, lower-cost layers of inventory) occur through inventory reduction programs? This inventory reduction may generate earnings without supporting cash flow, and management may intentionally reduce the layers to produce specific earnings benefits.

\end{itemize}

Tax asset valuation accounts

\begin{itemize}
  \item Tax assets, if present, must be stated at the value at which management expects to realize them, and an allowance must be set up to restate tax assets to the level expected to eventually be converted into cash. Determining the allowance involves an estimate of future operations and tax payments. Does the amount of the valuation allowance seem reasonable, overly optimistic, or overly pessimistic?

  \item Are there contradictions between the management commentary and the allowance level, or the tax note and the allowance level? There cannot be an optimistic management commentary and a fully reserved tax asset, or vice versa. One of them has to be wrong.

  \item Look for changes in the tax asset valuation account. It may be $100 \%$ reserved at first, and then "optimism" increases whenever an earnings boost is needed. Lowering the reserve decreases tax expense and increases net income.

\end{itemize}

Goodwill

\begin{itemize}
  \item Companies must annually assess goodwill balances for impairment on a qualitative basis. If further testing appears necessary, it is based on estimates of the fair value of the reporting units (US GAAP issuers), or cash-generating units (IFRS issuers), associated with goodwill balances. The tests are based on subjective estimates, including future cash flows and the employment of discount rates.

  \item Do the disclosures on goodwill testing suggest that the exercise was skewed to avoid impairment charges?

\end{itemize}

Warranty reserves

\begin{itemize}
  \item Have additions to the reserves been reduced, perhaps to make earnings targets? Examine the trend in the charges of actual costs against the reserves: Do they support or contradict the warranty provisioning activity? Do the actual costs charged against the reserve give the analyst any indication about the quality of the products sold?
\end{itemize}

Related-party transactions Analyst Concerns

\begin{itemize}
  \item Is the company engaged in transactions that disproportionately benefit members of management? Does one company have control over another's destiny through supply contracts or other dealings?

  \item Do extensive dealings take place with non-public companies that are under management control? If so, those companies could absorb losses (through supply arrangements that are unfavorable to them, for example) in order to make the public company's performance look good. This scenario may provide opportunities for an owner to cash out.

\end{itemize}

The most important lesson is that choices exist among accounting methods and estimates, and an analyst needs a working knowledge of them in order to understand whether management may have made choices to achieve a desired result.

\section{WARNING SIGNS}
describe accounting warning signs and methods for detecting manipulation of information in financial reports

The choices management makes to achieve desired results leave a trail, like tracks in sand or snow. The evidence, or warning signs, of information manipulation in financial reports is directly linked to the basic means of manipulation: biased revenue recognition and biased expense recognition. The bias may relate to timing and/ or location of recognition. An example of the timing issue is that a company may choose to defer expenses by capitalizing them. Regarding location, it may recognize a loss in other comprehensive income or directly through equity, rather than through the profit and loss statement. The alert investor or analyst should do the following to find warning signs.

\section{1) Pay attention to revenue.}
The single largest number on the income statement is revenue, and revenue recognition is a recurring source of accounting manipulation and even outright fraud. Answering the question, "Is revenue higher or lower than the previous comparable period?" is not sufficient. Many analytical procedures can be routinely performed to identify warning signals of malfeasance:

\begin{itemize}
  \item Examine the accounting policies note for a company's revenue recognition policies.

  \item Consider whether the policies make it easier to prematurely recognize revenue, such as recognizing revenue immediately upon shipment of goods, or if the company uses bill-and-hold arrangements whereby a sale is recognized before goods are actually shipped to the customer.

  \item Barter transactions may exist, which can be difficult to value properly. - Rebate programs involve many estimates, including forecasts of the amount of rebates that will ultimately be incurred, which can have significant effects on revenue recognition.

  \item Multiple-deliverable arrangements of goods and services are common, but clarity about the timing of revenue recognition for each item or service delivered is necessary for the investor to be comfortable with the reporting of revenues.

\end{itemize}

Although none of these decisions necessarily violates accounting standards, each involves significant judgement and warrants close attention if other warning signs are present.

\begin{itemize}
  \item Look at revenue relationships. Compare a company's revenue growth with its primary competitors or its industry peer group.

  \item If a company's revenue growth is out of line with its competitors, its industry, or the economy, the investor or analyst needs to understand the reasons for the outperformance. It may be a result of superior management or products and services, but not all management is superior, nor are the products and services of their companies. Revenue quality might be suspect, and the investor should take additional analytical steps.

  \item Compare accounts receivable with revenues over several years.

  \item Examine the trend to determine whether receivables are increasing as a percentage of total revenues. If so, a company might be engaging in channel-stuffing activities, or worse, recording fictitious sales transactions.

  \item Calculate receivables turnover for several years:

  \item Examine the trend for unusual changes and seek an explanation if they exist.

  \item Compare a company's days sales outstanding (DSO) or receivables turnover with that of relevant competitors or an industry peer group and determine whether the company is an outlier.

\end{itemize}

An increase in DSO or decrease in receivables turnover could suggest that some revenues are recorded prematurely or are even fictitious, or that the allowance for doubtful accounts is insufficient.

\begin{itemize}
  \item Examine asset turnover. If a company's managers make poor asset allocation choices, revenues may not be sufficient to justify the investment. Be particularly alert when asset allocation choices involve acquisitions of entire companies. If post-acquisition revenue generation is weak, managers might reach for revenue growth anywhere it can be found. That reach for growth might result in accounting abuses.
\end{itemize}

Revenues, divided by total assets, indicate the productivity of assets in generating revenues. If the company's asset turnover is continually declining, or lagging the asset turnover of competitors or the industry, it may portend future asset write-downs, particularly in the goodwill balances of acquisitive companies.

\section{2) Pay attention to signals from inventories.}
Although inventory is not a component of every company's asset base, its presence creates an opportunity for accounting manipulation.

\begin{itemize}
  \item Look at inventory relationships. Because revenues involve items sold from inventory, the kind of examination an investor should perform on inventory is similar to that for revenues.

  \item Compare growth in inventories with competitors and industry benchmarks. If a company's inventory growth is out of line with its peers, without any concurrent sales growth, then it may be simply the result of poor inventory management-an operational inefficiency that might affect an investor's view of a company. It may also signal obsolescence problems in the company's inventory that have not yet been recognized through markdowns to the inventory's net realizable value. Reported gross and net profits could be overstated because of overstated inventory.

  \item Calculate the inventory turnover ratio. This ratio is the cost of sales divided by the average ending inventory. Declining inventory turnover could also suggest obsolescence problems that should be recognized.

  \item Companies reporting under US GAAP may use LIFO inventory cost flow assumptions. When this assumption is part of the accounting policies, and a company operates in an inflationary environment, investors should note whether old, low-cost inventory costs have been passed through current earnings and artificially improved gross, operating, and net profits.

\end{itemize}

\section{3) Pay attention to capitalization policies and deferred costs.}
In a study of enforcement actions over a five-year period, the SEC found that improper revenue recognition was the most prevalent accounting issue. ${ }^{28}$ Suppression of expenses was the next most prevalent problem noted. As the earlier discussion of WorldCom showed, improper capitalization practices can result in a significant misstatement of financial results.

\begin{itemize}
  \item Examine the company's accounting policy note for its capitalization policy for long-term assets, including interest costs, and for its handling of other deferred costs. Compare the company's policy with the industry practice. If the company is the only one capitalizing certain costs while other industry participants treat them as expenses, a red flag is raised. If an outlier company of this type is encountered, it would be useful to cross-check such a company's asset turnover and profitability margins with others in its industry. An investor might expect such a company to be more profitable than its competitors, but the investor might also have lower confidence in the quality of the reported numbers.
\end{itemize}

28 SEC, "Report Pursuant to Section 704 of the Sarbanes-Oxley Act of 2002" (\href{http://www.sec.gov/news/studies/}{www.sec.gov/news/studies/} sox704report.pdf): 5-6. 4) Pay attention to the relationship of cash flow and net income.

Net income propels stock prices, but cash flow pays bills. Management can manipulate either one, but sooner or later, net income must be realized in cash if a company is to remain viable. When net income is higher than cash provided by operations, one possibility is that aggressive accrual accounting policies have shifted current expenses to later periods. Increasing earnings in the presence of declining cash generated by operations might signal accounting irregularities.

\begin{itemize}
  \item Construct a time series of cash generated by operations divided by net income. If the ratio is consistently below 1.0 or has declined repeatedly, there may be problems in the company's accrual accounts.
\end{itemize}

\section{5) Other potential warnings signs.}
Other areas that might suggest the need for further analysis include the following:

\begin{itemize}
  \item Depreciation methods and useful lives. As discussed earlier, selection of depreciation methods and useful lives can greatly influence profitability. An investor should compare a company's policies with those of its peers to determine whether it is particularly lenient in its effects on earnings. Investors should likewise compare the length of depreciable lives used by a company with those used by its peers.

  \item Fourth-quarter surprises. An investor should be suspicious of possible earnings management if a company routinely disappoints investors with poor earnings or overachieves in the fourth quarter of the year when no seasonality exists in the business. The company may be over- or under-reporting profits in the first three quarters of the year.

  \item Presence of related-party transactions. Related-party transactions often arise when a company's founders are still very active in managing the company, with much of their wealth tied to the company's fortunes. They may be more biased in their view of a company's performance because it relates directly to their own wealth and reputations, and they may be able to transact business with the company in ways that may not be detected. For instance, they may purchase unsellable inventory from the company for disposal in another company of their own in order to avoid markdowns.

  \item Non-operating income or one-time sales included in revenue. To disguise weakening revenue growth, or just to enhance revenue growth, a company might classify non-operating income items into revenues or fail to clarify the nature of revenues. In the first quarter of 1997, Sunbeam Corporation included one-time disposal of product lines in sales without indicating that such non-recurring sales were included in revenues. This inclusion gave investors a false impression of the company's sustainable revenue-generating capability.

  \item Classification of expenses as "non-recurring." To make operating performance look more attractive, managers might carve out "special items" in the income statement. Particularly when these items appear period after period, equity investors might find their interests best served by not accepting the carve-out of serial "special items" and instead focusing on the net income line in evaluating performance over long periods. - Gross/operating margins out of line with competitors or industry. This disparity is an ambivalent warning sign. It might signal superior management ability. But it might also signal the presence of accounting manipulations to add a veneer of superior management ability to the company's reputation. Only the compilation and examination of other warning signals will enable an investor or analyst to decide which signal is being given.

\end{itemize}

Warning signals are just that: signals, not indisputable declarations of accounting manipulation guilt. Investors and analysts need to evaluate them cohesively, not on an isolated basis. When an investor finds a number of these signals, the investee company should be viewed with caution or even discarded in favor of alternatives.

Furthermore, as discussed earlier, context is important in judging the value of warning signals. A few examples of facts and circumstances to be aware of are as follows.

\begin{itemize}
  \item Younger companies with an unblemished record of meeting growth projections. It is plausible, especially for a younger company with new and popular product offerings, to generate above-average returns for a period of time. But, as demand dissipates, products mature, and competitors challenge for market share, management may seek to extend its recent record of rapid growth in sales and profitability by unconventional means. At this point, the "earnings games" begin: aggressive estimates, drawing down "cookie jar" reserves, selling assets for accounting gains, taking on excess leverage, or entering into financial transactions with no apparent business purpose other than financial statement "window dressing."

  \item Management has adopted a minimalist approach to disclosure. Confidence in accounting quality depends on disclosure. If management does not seem to take seriously its obligation to provide information, one needs to be concerned. For example, for a large company, management might claim that it has only one reportable segment, or its commentary might be similar from period to period. A plausible explanation for minimalist disclosure policies could be that management is protecting investors' interests by withholding valuable information from competitors. But, this is not necessarily the case. For example, after Sony Corporation acquired CBS Records and Columbia Pictures, it incurred substantial losses for a number of years. Yet, Sony chose to hide its negative trends and doubtful future prospects by aggregating the results within a much larger "Entertainment Division." In 1998, after Sony ultimately wrote off much of the goodwill associated with these ill-fated acquisitions, the SEC sanctioned Sony and its CFO for failing to separately discuss them in MD\&A in a balanced manner. ${ }^{29}$

  \item Management fixation on earnings reports. Beware of companies whose management appears to be fixated on reported earnings, sometimes to the detriment of attending to real drivers of value. Indicators of excessive earnings fixation include the aggressive use of non-GAAP measures of performance, special items, or non-recurring charges. Another indicator of earnings fixation is highly decentralized operations in which division managers' compensation packages are heavily weighted toward the attainment of reported earnings or non-GAAP measures of performance.

\end{itemize}

29 Accounting and Auditing Enforcement Release No. 1061, "In the Matter of Sony Corporation and Sumio Sano, Respondents," SEC (5 August 1998).

\section{Company Culture}
A company's culture is an intangible that investors should bear in mind when they are evaluating financial statements for the possibility of accounting manipulation. A management's highly competitive mentality may serve investors well when the company conducts business (assuming that actions taken are not unethical, illegal, or harmfully myopic), but that kind of thinking should not extend to communications with the owners of the company: the shareholders. That mentality can lead to the kind of accounting gamesmanship seen in the early part of the century. In examining financial statements for warning signs of manipulation, the investor should consider whether that mindset exists in the preparation of the financial statements.

One notable example of the mindset comes from one of the most recognized corporate names in the world, General Electric. In the mid-1980s, GE acquired Kidder Peabody, and it was ultimately determined that much of the earnings that Kidder had reported were bogus. As a consequence, GE would announce within two days that it would take a non-cash write-off of $\$ 350$ million. Here is how former CEO/Chair Jack Welsh described the ensuing meeting with senior management in his memoir, Straight from the Gut:

"The response of our business leaders to the crisis was typical of the GE culture [emphasis added]. Even though the books had closed on the quarter, many immediately offered to pitch in to cover the Kidder gap. Some said they could find an extra $\$ 10$ million, $\$ 20$ million, and even $\$ 30$ million from their businesses to offset the surprise. Though it was too late, their willingness to help was a dramatic contrast to the excuses I had been hearing from the Kidder people." (p. 225)

It appears that the corporate governance apparatus fostered a GE culture that extended the concept of teamwork to the point of "sharing" profits to win one for the team as a whole, which is incompatible with the concept of neutral financial reporting. Although research is not conclusive on this question, it may also be worth considering that predisposition to earnings manipulation is more likely to be present when the CEO and board chair are one and the same, or when the audit committee of the board essentially serves at the pleasure of the CEO and lacks financial reporting sophistication. Finally, one could discuss whether the financial reporting environment today would reward or penalize a CEO who openly endorsed a view that he could legitimately exercise financial reporting discretion-albeit within limits-for the purpose of artificially smoothing earnings.

\section{Restructuring and/or impairment charges.}
At times, a company's stock price has been observed to rise after it recognized a "big bath" charge to reported earnings. The conventional wisdom explaining the stock price rise is that accounting recognition signals something positive: that management is now ready to part with the lagging portion of a company, so as to redirect its attention and talents to more-profitable activities. Consequently, the earnings charge should be disregarded for being solely related to past events.

The analyst should also consider, however, that the events leading ultimately to the big bath on the financial statements did not happen overnight, even though the accounting for those events occurs at a subsequent point. Management may want to communicate that the accounting adjustments reflect the company's new path, but the restructuring charge also indicates that the old path of reported earnings was not real. In particular, expenses reported in prior years were very likely understatedeven assuming that no improper financial statement manipulation had occurred. To extrapolate historical earnings trends, an analyst should consider making pro forma analytical adjustments to prior years' earnings to reflect a reasonable division of the latest period's restructuring and impairment charges.

\section{Management has a merger and acquisition orientation.}
Tyco International Ltd. acquired more than 700 companies from 1996 to 2002. Even assuming the best of intentions regarding financial reporting, a growth-at-any-cost corporate culture poses a severe challenge to operational and financial reporting controls. In Tyco's case, the SEC found that it consistently and fraudulently understated assets acquired (lowering future depreciation and amortization charges) and overstated liabilities assumed (avoiding expense recognition and potentially increasing earnings in future periods). ${ }^{30}$

\section{SUMMARY}
Financial reporting quality varies across companies. The ability to assess the quality of a company's financial reporting is an important skill for analysts. Indications of low-quality financial reporting can prompt an analyst to maintain heightened skepticism when reading a company's reports, to review disclosures critically when undertaking financial statement analysis, and to incorporate appropriate adjustments in assessments of past performance and forecasts of future performance.

\begin{itemize}
  \item Financial reporting quality can be thought of as spanning a continuum from the highest (containing information that is relevant, correct, complete, and unbiased) to the lowest (containing information that is not just biased or incomplete but possibly pure fabrication).

  \item Reporting quality, the focus of this reading, pertains to the information disclosed. High-quality reporting represents the economic reality of the company's activities during the reporting period and the company's financial condition at the end of the period.

  \item Results quality (commonly referred to as earnings quality) pertains to the earnings and cash generated by the company's actual economic activities and the resulting financial condition, relative to expectations of current and future financial performance. Quality earnings are regarded as being sustainable, providing a sound platform for forecasts.

  \item An aspect of financial reporting quality is the degree to which accounting choices are conservative or aggressive. "Aggressive" typically refers to choices that aim to enhance the company's reported performance and financial position by inflating the amount of revenues, earnings, and/or operating cash flow reported in the period; or by decreasing expenses for the period and/or the amount of debt reported on the balance sheet.

  \item Conservatism in financial reports can result from either (1) accounting standards that specifically require a conservative treatment of a transaction or an event or (2) judgments made by managers when applying accounting standards that result in conservative results.

  \item Managers may be motivated to issue less-than-high-quality financial reports in order to mask poor performance, to boost the stock price, to increase personal compensation, and/or to avoid violation of debt covenants.

\end{itemize}

30 Accounting and Auditing Enforcement Release No. 2414, "SEC Brings Settled Charges Against Tyco International Ltd. Alleging Billion Dollar Accounting Fraud," SEC (17 April 2006): \href{http://www.sec.gov/litigation/}{www.sec.gov/litigation/} litreleases/2006/lr19657.htm. - Conditions that are conducive to the issuance of low-quality financial reports include a cultural environment that result in fewer or less transparent financial disclosures, book/tax conformity that shifts emphasis toward legal compliance and away from fair presentation, and limited capital markets regulation.

\begin{itemize}
  \item Mechanisms that discipline financial reporting quality include the free market and incentives for companies to minimize cost of capital, auditors, contract provisions specifically tailored to penalize misreporting, and enforcement by regulatory entities.

  \item Pro forma earnings (also commonly referred to as non-GAAP or non-IFRS earnings) adjust earnings as reported on the income statement. Pro forma earnings that exclude negative items are a hallmark of aggressive presentation choices.

  \item Companies are required to make additional disclosures when presenting any non-GAAP or non-IFRS metric.

  \item Managers' considerable flexibility in choosing their companies' accounting policies and in formulating estimates provides opportunities for aggressive accounting.

  \item Examples of accounting choices that affect earnings and balance sheets include inventory cost flow assumptions, estimates of uncollectible accounts receivable, estimated realizability of deferred tax assets, depreciation method, estimated salvage value of depreciable assets, and estimated useful life of depreciable assets.

  \item Cash from operations is a metric of interest to investors that can be enhanced by operating choices, such as stretching accounts payable, and potentially by classification choices.

\end{itemize}

\section{REFERENCES}
Back, Aaron. 2013. "Toyota, What a Difference the Yen Makes." Wall Street Journal (4 August 2013).

Basu, Sudipta. 1997. "The Conservatism Principle and the Asymmetric Timeliness of Earnings." Journal of Accounting and Economics, vol. 24, no. 1 (December):3-37. 10.1016/ S0165-4101(97)00014-1

Bliss, James Harris. 1924. Management through Accounts. New York: Ronald Press Company.

Ciesielski, Jack T, Elaine Henry. 2017. "Accounting's Tower of Babel: Key Considerations in Assessing Non-GAAP Earnings." Financial Analysts Journal, vol. 73, no. 2:34-50.

Gordon, Elizabeth, Elaine Henry, Bjorn Jorgensen, Cheryl Linthicum. 2017. "Flexibility in Cash-Flow Classification under IFRS: Determinants and Consequences." Review of Accounting Studies, vol. 22, no. 2:839-872.

Graham, John, Campbell Harvey, Shiva Rajgopal. 2005. “The Economic Implications of Corporate Financial Reporting." Journal of Accounting and Economics, vol. 40, no. 1 (December):3-73. $10.1016 /$ j.jacceco.2005.01.002

Lewis, Craig M. 2012. "Risk Modeling at the SEC: The Accounting Quality Model" Speech, the Financial Executives International Committee on Finance and Information Technology (13 December): \href{http://www.sec.gov/news/speech/2012/spch121312cml.htm}{www.sec.gov/news/speech/2012/spch121312cml.htm}.

Nurnberg, H. 2006. "Perspectives on the Cash Flow Statement under FASB Statement No. 95." Center for Excellence in Accounting and Security Analysis Occasional Paper Series. Columbia Business School.

Nurnberg, H., J. Largay. 1998. "Interest Payments in the Cash Flow Statement." Accounting Horizons, vol. 12, no. 4 (December):407-418.

Pavlo, Walter. 2013. "Fmr Enron CFO Andrew Fastow Speaks At ACFE Annual Conference," Forbes (26 June): \href{http://www.forbes.com/sites/walterpavlo/2013/06/26/fmr-enron-cfo-andre}{www.forbes.com/sites/walterpavlo/2013/06/26/fmr-enron-cfo-andre} w-fastow-speaks-at-acfe-annual-conference/.

Ronen, Joshua, Varda Yaari. 2008. Earnings Management: Emerging Insights in Theory, Practice, and Research. New York: Springer.

Schipper, Katherine. 1989. "Commentary on Earnings Management." Accounting Horizons, vol. 3, no. 4 (December):91-102.

Watts, Ross. 2003. "Conservatism in Accounting Part I: Explanations and Implications." Accounting Horizons, vol. 17, no. 3 (September):207-221. 10.2308/acch.2003.17.3.207

\section{PRACTICE PROBLEMS}
\begin{enumerate}
  \item In contrast to earnings quality, financial reporting quality most likely pertains to:
A. sustainable earnings.
B. relevant information.
C. adequate return on investment.

  \item The information provided by a low-quality financial report will most likely:
A. decrease company value.
B. indicate earnings are not sustainable.
C. impede the assessment of earnings quality.

  \item To properly assess a company's past performance, an analyst requires:
A. high earnings quality.
B. high financial reporting quality.
C. both high earnings quality and high financial reporting quality.

  \item Low quality earnings most likely reflect:
A. low-quality financial reporting.
B. company activities which are unsustainable.
C. information that does not faithfully represent company activities.

  \item Earnings that result from non-recurring activities most likely indicate:
A. lower-quality earnings.
B. biased accounting choices.
C. lower-quality financial reporting.

  \item Which attribute of financial reports would most likely be evaluated as optimal in the financial reporting spectrum?
A. Conservative accounting choices
B. Sustainable and adequate returns
C. Emphasized pro forma earnings measures

  \item Financial reports of the lowest level of quality reflect:
A. fictitious events.
B. biased accounting choices.
C. accounting that is non-compliant with GAAP.

  \item When earnings are increased by deferring research and development (R\&D) investments until the next reporting period, this choice is considered:
A. non-compliant accounting.
B. earnings management as a result of a real action.
C. earnings management as a result of an accounting choice.

  \item A high-quality financial report may reflect:
A. earnings smoothing.
B. low earnings quality.
C. understatement of asset impairment.

  \item If a particular accounting choice is considered aggressive in nature, then the financial performance for the reporting period would most likely:
A. be neutral.
B. exhibit an upward bias.
C. exhibit a downward bias.

  \item Which of the following is most likely to reflect conservative accounting choices?

\end{enumerate}

A. Decreased reported earnings in later periods

B. Increased reported earnings in the period under review

C. Increased debt reported on the balance sheet at the end of the current period

\begin{enumerate}
  \setcounter{enumi}{11}
  \item Which of the following is most likely to be considered a potential benefit of accounting conservatism?
\end{enumerate}

A. A reduction in litigation costs

B. Less biased financial reporting

C. An increase in current period reported performance

\begin{enumerate}
  \setcounter{enumi}{12}
  \item Which of the following statements most likely describes a situation that would motivate a manager to issue low-quality financial reports?
\end{enumerate}

A. The manager's compensation is tied to stock price performance.

B. The manager has increased the market share of products significantly.

C. The manager has brought the company's profitability to a level higher than competitors.

\begin{enumerate}
  \setcounter{enumi}{13}
  \item Which of the following concerns would most likely motivate a manager to make conservative accounting choices?
\end{enumerate}

A. Attention to future career opportunities

B. Expected weakening in the business environment

C. Debt covenant violation risk in the current period 15. Which of the following conditions best explains why a company's manager would obtain legal, accounting, and board level approval prior to issuing low-quality financial reports?
A. Motivation
B. Opportunity
C. Rationalization

\begin{enumerate}
  \setcounter{enumi}{15}
  \item A company is experiencing a period of strong financial performance. In order to increase the likelihood of exceeding analysts' earnings forecasts in the next reporting period, the company would most likely undertake accounting choices for the period under review that:
A. inflate reported revenue.
B. delay expense recognition.
C. accelerate expense recognition.

  \item Which of the following situations represents a motivation, rather than an opportunity, to issue low-quality financial reports?
A. Poor internal controls
B. Search for a personal bonus
C. Inattentive board of directors

  \item Which of the following situations will most likely motivate managers to inflate reported earnings?
A. Possibility of bond covenant violation
B. Earnings in excess of analysts' forecasts
C. Earnings that are greater than the previous year

  \item Which of the following best describes an opportunity for management to issue low-quality financial reports?
A. Ineffective board of directors
B. Pressure to achieve some performance level
C. Corporate concerns about financing in the future

  \item An audit opinion of a company's financial reports is most likely intended to:
A. detect fraud.
B. reveal misstatements.
C. assure that financial information is presented fairly.

  \item If a company uses a non-GAAP financial measure in an SEC filing, then the company must:

\end{enumerate}

A. give more prominence to the non-GAAP measure if it is used in earnings releases. B. provide a reconciliation of the non-GAAP measure and equivalent GAAP measure.

C. exclude charges requiring cash settlement from any non-GAAP liquidity measures.

\begin{enumerate}
  \setcounter{enumi}{21}
  \item A company wishing to increase earnings in the reporting period may choose to:
\end{enumerate}

A. decrease the useful life of depreciable assets.

B. lower estimates of uncollectible accounts receivables.

C. classify a purchase as an expense rather than a capital expenditure.

\begin{enumerate}
  \setcounter{enumi}{22}
  \item Which technique most likely increases the cash flow provided by operations?
\end{enumerate}

A. Stretching the accounts payable credit period

B. Applying all non-cash discount amortization against interest capitalized

C. Shifting classification of interest paid from financing to operating cash flows

\begin{enumerate}
  \setcounter{enumi}{23}
  \item Bias in revenue recognition would least likely be suspected if:
\end{enumerate}

A. the firm engages in barter transactions.

B. reported revenue is higher than the previous quarter.

C. revenue is recognized before goods are shipped to customers.

\begin{enumerate}
  \setcounter{enumi}{24}
  \item Which of the following is an indication that a company may be recognizing revenue prematurely? Relative to its competitors, the company's:
A. asset turnover is decreasing.
B. receivables turnover is increasing.
C. days sales outstanding is increasing.

  \item Which of the following would most likely signal that a company may be using aggressive accrual accounting policies to shift current expenses to later periods? Over the last five-year period, the ratio of cash flow to net income has:

\end{enumerate}

A. increased each year.

B. decreased each year.

C. fluctuated from year to year.

\begin{enumerate}
  \setcounter{enumi}{26}
  \item An analyst reviewing a firm with a large reported restructuring charge to earnings should:
\end{enumerate}

A. view expenses reported in prior years as overstated.

B. disregard it because it is solely related to past events.

C. consider making pro forma adjustments to prior years' earnings.

\section{SOLUTIONS}
\begin{enumerate}
  \item B is correct. Financial reporting quality pertains to the quality of information in financial reports. High-quality financial reporting provides decision-useful information, which is relevant and faithfully represents the economic reality of the company's activities. Earnings of high quality are sustainable and provide an adequate level of return. Highest-quality financial reports reflect both high financial reporting quality and high earnings quality.

  \item C is correct. Financial reporting quality pertains to the quality of the information contained in financial reports. High-quality financial reports provide decision-useful information that faithfully represents the economic reality of the company. Low-quality financial reports impede assessment of earnings quality. Financial reporting quality is distinguishable from earnings quality, which pertains to the earnings and cash generated by the company's actual economic activities and the resulting financial condition. Low-quality earnings are not sustainable and decrease company value.

  \item B is correct. Financial reporting quality pertains to the quality of the information contained in financial reports. If financial reporting quality is low, the information provided is of little use in assessing the company's performance. Financial reporting quality is distinguishable from earnings quality, which pertains to the earnings and cash generated by the company's actual economic activities and the resulting financial condition.

  \item B is correct. Earnings quality pertains to the earnings and cash generated by the company's actual economic activities and the resulting financial condition. Low-quality earnings are likely not sustainable over time because the company does not expect to generate the same level of earnings in the future or because earnings will not generate sufficient return on investment to sustain the company. Earnings that are not sustainable decrease company value. Earnings quality is distinguishable from financial reporting quality, which pertains to the quality of the information contained in financial reports.

  \item A is correct. Earnings that result from non-recurring activities are unsustainable. Unsustainable earnings are an example of lower-quality earnings. Recognizing earnings that result from non-recurring activities is neither a biased accounting choice nor indicative of lower quality financial reporting because it faithfully represents economic events.

  \item B is correct. At the top of the quality spectrum of financial reports are reports that conform to GAAP, are decision useful, and have earnings that are sustainable and offer adequate returns. In other words, these reports have both high financial reporting quality and high earnings quality.

  \item A is correct. Financial reports span a quality continuum from high to low based on decision-usefulness and earnings quality (see Exhibit 2 of the reading). The lowest-quality reports portray fictitious events, which may misrepresent the company's performance and/or obscure fraudulent misappropriation of the company's assets.

  \item B is correct. Deferring research and development (R\&D) investments into the next reporting period is an example of earnings management by taking a real action

  \item B is correct. High-quality financial reports offer useful information, meaning in- formation that is relevant and faithfully represents actual performance. Although low earnings quality may not be desirable, if the reported earnings are representative of actual performance, they are consistent with high-quality financial reporting. Highest-quality financial reports reflect both high financial reporting quality and high earnings quality.

  \item B is correct. Aggressive accounting choices aim to enhance the company's reported performance by inflating the amount of revenues, earnings, and/or operating cash flow reported in the period. Consequently, the financial performance for that period would most likely exhibit an upward bias.

  \item $C$ is correct. Accounting choices are considered conservative if they decrease the company's reported performance and financial position in the period under review. Conservative choices may increase the amount of debt reported on the balance sheet. They may decrease the revenues, earnings, and/or operating cash flow reported for the period and increase those amounts in later periods.

  \item A is correct. Conservatism reduces the possibility of litigation and, by extension, litigation costs. Rarely, if ever, is a company sued because it understated good news or overstated bad news. Accounting conservatism is a type of bias in financial reporting that decreases a company's reported performance. Conservatism directly conflicts with the characteristic of neutrality.

  \item A is correct. Managers often have incentives to meet or beat market expectations, particularly if management compensation is linked to increases in stock prices or to reported earnings.

  \item B is correct. Managers may be motivated to understate earnings in the reporting period and increase the probability of meeting or exceeding the next period's earnings target.

  \item C is correct. Typically, conditions of opportunity, motivation, and rationalization exist when individuals issue low-quality financial reports. Rationalization occurs when an individual is concerned about a choice and needs to be able to justify it to herself or himself. If the manager is concerned about a choice in a financial report, she or he may ask for other opinions to convince herself or himself that it is okay.

  \item $\mathrm{C}$ is correct. In a period of strong financial performance, managers may pursue accounting choices that increase the probability of exceeding earnings forecasts for the next period. By accelerating expense recognition or delaying revenue recognition, managers may inflate earnings in the next period and increase the likelihood of exceeding targets.

  \item B is correct. Motivation can result from pressure to meet some criteria for personal reasons, such as a bonus, or corporate reasons, such as concern about future financing. Poor internal controls and an inattentive board of directors offer opportunities to issue low-quality financial reports.

  \item A is correct. The possibility of bond covenant violations may motivate managers to inflate earnings in the reporting period. In so doing, the company may be able to avoid the consequences associated with violating bond covenants.

  \item A is correct. Opportunities to issue low-quality financial reports include internal conditions, such as an ineffective board of directors, and external conditions, such as accounting standards that provide scope for divergent choices. Pressure to achieve a certain level of performance and corporate concerns about future financing are examples of motivations to issue low-quality financial reports. Typically, three conditions exist when low-quality financial reports are issued: opportunity, motivation, and rationalization.

  \item C is correct. An audit is intended to provide assurance that the company's financial reports are presented fairly, thus providing discipline regarding financial reporting quality. Regulatory agencies usually require that the financial statements of publicly traded companies be audited by an independent auditor to provide assurance that the financial statements conform to accounting standards. Privately held companies may also choose to obtain audit opinions either voluntarily or because an outside party requires it. An audit is not typically intended to detect fraud. An audit is based on sampling and it is possible that the sample might not reveal misstatements.

  \item B is correct. If a company uses a non-GAAP financial measure in an SEC filing, it is required to provide the most directly comparable GAAP measure with equivalent prominence in the filing. In addition, the company is required to provide a reconciliation between the non-GAAP measure and the equivalent GAAP measure. Similarly, IFRS require that any non-IFRS measures included in financial reports must be defined and their potential relevance explained. The non-IFRS measures must be reconciled with IFRS measures.

  \item B is correct. If a company wants to increase reported earnings, the company's managers may reduce the allowance for uncollected accounts and the related expense reported for the period. Decreasing the useful life of depreciable assets would increase depreciation expense and decrease earnings in the reporting period. Classifying a purchase as an expense, rather than capital expenditure, would decrease earnings in the reporting period. The use of accrual accounting may result in estimates in financial reports, because all facts associated with events may not be known at the time of recognition. These estimates can be grounded in reality or managed by the company to present a desired financial picture.

  \item A is correct. Managers can temporarily show a higher cash flow from operations by stretching the accounts payable credit period. In other words, the managers delay payments until the next accounting period. Applying all non-cash discount amortization against interest capitalized causes reported interest expenses and operating cash outflow to be higher, resulting in a lower cash flow provided by operations. Shifting the classification of interest paid from financing to operating cash flows lowers the cash flow provided by operations.

  \item B is correct. Bias in revenue recognition can lead to manipulation of information presented in financial reports. Addressing the question as to whether revenue is higher or lower than the previous period is not sufficient to determine if there is bias in revenue recognition. Additional analytical procedures must be performed to identify warning signals of accounting malfeasance. Barter transactions are difficult to value properly and may result in bias in revenue recognition. Policies that make it easier to prematurely recognize revenue, such as before goods are shipped to customers, may be a warning sign of accounting malfeasance.

  \item $C$ is correct. If a company's days sales outstanding (DSO) is increasing relative to competitors, this may be a signal that revenues are being recorded prematurely or are even fictitious. There are numerous analytical procedures that can be performed to provide evidence of manipulation of information in financial reporting. These warning signs are often linked to bias associated with revenue recognition and expense recognition policies.

  \item B is correct. If the ratio of cash flow to net income for a company is consistently below 1 or has declined repeatedly over time, this may be a signal of manipula- tion of information in financial reports through aggressive accrual accounting policies. When net income is consistently higher than cash provided by operations, one possible explanation is that the company may be using aggressive accrual accounting policies to shift current expenses to later periods.

  \item C is correct. To extrapolate historical earnings trends, an analyst should consider making pro forma analytical adjustments of prior years' earnings to reflect in those prior years a reasonable share of the current period's restructuring and impairment charges.

\end{enumerate}

\textbackslash section\{LEARNING MODULE

\begin{center}
\includegraphics[max width=\textwidth]{2023_05_04_b5cfa4f1bc883752f121g-581}
\end{center}

\section{Applications of Financial Statement Analysis}
by Thomas R. Robinson, PhD, CAIA, CFA, J. Hennie van Greuning, DCom, CFA, Elaine Henry, PhD, CFA, and Michael A. Broihahn, CPA, CIA, CFA.

Thomas R Robinson, PhD, CAIA, CFA Robinson Global Investment Management LLC, (USA). J. Hennie van Greuning, DCom, CFA, is at BIBD (Brunei). Elaine Henry, PhD, CFA, is at Stevens Institute of Technology (USA). Michael A. Broihahn, CPA, CIA, CFA, is at Barry University (USA).

\section{LEARNING OUTCOME}
\begin{center}
\begin{tabular}{c|l}
Mastery & The candidate should be able to: \\
$\square$ & $\begin{array}{l}\text { evaluate a company's past financial performance and explain how a } \\ \text { company's strategy is reflected in past financial performance } \\ \text { demonstrate how to forecast a company's future net income and } \\ \text { cash flow } \\ \text { describe the role of financial statement analysis in assessing the } \\ \text { credit quality of a potential debt investment } \\ \text { describe the use of financial statement analysis in screening for } \\ \text { potential equity investments }\end{array}$ \\
$\square \quad \begin{array}{l}\text { explain appropriate analyst adjustments to a company's financial } \\ \text { statements to facilitate comparison with another company }\end{array}$ &  \\
\end{tabular}
\end{center}

Note: Changes in accounting standards as well as new rulings and/or pronouncements issued after the publication of the readings on financial reporting and analysis may cause some of the information in these readings to become dated. Candidates are not responsible for anything that occurs after the readings were published. In addition, candidates are expected to be familiar with the analytical frameworks contained in the readings, as well as the implications of alternative accounting methods for financial analysis and valuation discussed in the readings. Candidates are also responsible for the content of accounting standards, but not for the actual reference numbers. Finally, candidates should be aware that certain ratios may be defined and calculated differently. When alternative ratio definitions exist and no specific definition is given, candidates should use the ratio definitions emphasized in the readings.

\section{INTRODUCTION \& EVALUATING PAST FINANCIAL PERFORMANCE}
evaluate a company's past financial performance and explain how a company's strategy is reflected in past financial performance

This reading presents several important applications of financial statement analysis. Among the issues we will address are the following:

\begin{itemize}
  \item What are the key questions to address in evaluating a company's past financial performance? - How can an analyst approach forecasting a company's future net income and cash flow?

  \item How can financial statement analysis be used to evaluate the credit quality of a potential fixed-income investment?

  \item How can financial statement analysis be used to screen for potential equity investments?

  \item How can differences in accounting methods affect financial ratio comparisons between companies, and what are some adjustments analysts make to reported financials to facilitate comparability among companies.

\end{itemize}

The reading "Financial Statement Analysis: An Introduction" described a framework for conducting financial statement analysis. Consistent with that framework, prior to undertaking any analysis, an analyst should explore the purpose and context of the analysis. The purpose and context guide further decisions about the approach, the tools, the data sources, and the format in which to report results of the analysis, and also suggest which aspects of the analysis are most important. Having identified the purpose and context, the analyst should then be able to formulate the key questions that the analysis must address. The questions will suggest the data the analyst needs to collect to objectively address the questions. The analyst then processes and analyzes the data to answer these questions. Conclusions and decisions based on the analysis are communicated in a format appropriate to the context, and follow-up is undertaken as required. Although this reading will not formally present applications as a series of steps, the process just described is generally applicable.

Section 2 of this reading describes the use of financial statement analysis to evaluate a company's past financial performance, and Section 3 describes basic approaches to projecting a company's future financial performance. Section 4 presents the use of financial statement analysis in assessing the credit quality of a potential debt investment. Section 5 concludes the survey of applications by describing the use of financial statement analysis in screening for potential equity investments. Analysts often encounter situations in which they must make adjustments to a company's reported financial results to increase their accuracy or comparability with the financials of other companies. Section 6 illustrates several common types of analyst adjustments. Section 7 presents a summary, and practice problems in the CFA Institute multiple-choice format conclude the reading.

\section{Application: Evaluating Past Financial Performance}
Analysts examine a company's past financial performance for a number of reasons. Cross-sectional analysis of financial performance facilitates understanding of the comparability of companies for a market-based valuation. ${ }^{1}$ Analysis of a company's historical performance over time can provide a basis for a forward-looking analysis of the company. Both cross-sectional and trend analysis can provide information for evaluating the quality and performance of a company's management.

An evaluation of a company's past performance addresses not only what happened (i.e., how the company performed) but also why it happened-the causes behind the performance and how the performance reflects the company's strategy. Evaluative

1 Pinto et al. (2010) describe market-based valuation as using price multiples-ratios of a stock's market price to some measure of value per share (e.g., price-to-earnings ratios). Although the valuation method may be used independently of an analysis of a company's past financial performance, such an analysis may provide reasons for differences in companies' price multiples. judgments assess whether the performance is better or worse than a relevant benchmark, such as the company's own historical performance, a competitor's performance, or market expectations. Some key analytical questions include the following:

\begin{itemize}
  \item How and why have corporate measures of profitability, efficiency, liquidity, and solvency changed over the periods being analyzed?

  \item How do the level and trend in a company's profitability, efficiency, liquidity, and solvency compare with the corresponding results of other companies in the same industry? What factors explain any differences?

  \item What aspects of performance are critical for a company to successfully compete in its industry, and how did the company perform relative to those critical performance aspects?

  \item What are the company's business model and strategy, and how did they influence the company's performance as reflected in, for example, its sales growth, efficiency, and profitability?

\end{itemize}

Data available to answer these questions include the company's (and its competitors') financial statements, materials from the company's investor relations department, corporate press releases, and non-financial-statement regulatory filings, such as proxies. Useful data also include industry information (e.g., from industry surveys, trade publications, and government sources), consumer information (e.g., from consumer satisfaction surveys), and information that is gathered by the analyst firsthand (e.g., through on-site visits). Processing the data typically involves creating common-size financial statements, calculating financial ratios, and reviewing or calculating industry-specific metrics. Example 1 illustrates the effects of strategy on performance and the use of basic economic reasoning in interpreting results.

\section{EXAMPLE 1}
\section{A Change in Products Reflected in Financial Performance}
Apple Inc. is a company that has evolved and adapted over time. In its 1994 Prospectus filed with the US SEC, Apple identified itself as "one of the world's leading personal computer technology companies." At that time, most of its revenue was generated by computer sales. In the prospectus, however, Apple stated, "The Company's strategy is to expand its market share in the personal computing industry while developing and expanding into new related business such as Personal Interactive Electronics and Apple Business Systems." Over time, products other than computers became significant generators of revenue and profit.

In 2005, an article in Barron's said, "In the last year, the iPod has become Apple's best-selling product, bringing in a third of revenues for the Cupertino, Calif. firm . . Little noticed by these iPod zealots, however is a looming threat ... Wireless phone companies are teaming up with the music industry to make most mobile phones into music players" (Barron's 27 June 2005, p. 19). The threat noted by Barron's was not unnoticed or ignored by Apple.

In June 2007, Apple itself entered the mobile phone market with the launch of the original iPhone, followed in June 2008 by the second-generation iPhone 3G (a handheld device combining the features of a mobile phone, an iPod, and an internet connection device). Soon after, the company launched the iTunes App Store, which allows users to download third-party applications onto their iPhones. As noted in a 2009 Business Week article, Apple "is the world's largest music distributor, having passed Wal-Mart Stores in early 2008. Apple sells around $90 \%$ of song downloads and $75 \%$ of digital music players in the United States" (Business Week, 28 September 2009, p. 34). Product innovations continue as evidenced by the introduction of the iPad in January 2010.

In analyzing the historical performance of Apple in 2018, an analyst might refer to the information presented in Exhibit 1, which shows sales, profitability, sales by product line and product mix.

\section{Exhibit 1: (dollars in millions)}
\begin{center}
\begin{tabular}{|c|c|c|c|c|c|c|c|c|}
\hline
$\begin{array}{l}\text { Sales and } \\ \text { Profitability }\end{array}$ & 2017 & 2016 & 2015 & 2014 & 2013 & 2012 & 2011 & 2010 \\
\hline
Sales & 229,234 & 215,639 & 233,715 & 182,795 & 170,910 & 156,608 & 108,249 & 65,225 \\
\hline
Cost of goods sold & 141,048 & 131,376 & 140,089 & 112,258 & 106,606 & 87,846 & 64,431 & 39,541 \\
\hline
Gross profit & 88,186 & 84,263 & 93,626 & 70,537 & 64,304 & 68,762 & 43,818 & 25,684 \\
\hline
Gross margin & $38.5 \%$ & $39.1 \%$ & $40.1 \%$ & $38.6 \%$ & $37.6 \%$ & $43.9 \%$ & $40.5 \%$ & $39.4 \%$ \\
\hline
\multicolumn{9}{|l|}{$\begin{array}{l}\text { Net sales by } \\
\text { product }\end{array}$} \\
\hline
$\mathrm{Mac}$ & 25,850 & 22,831 & 25,471 & 24,079 & 21,483 & 23,221 & 21,783 & 17,480 \\
\hline
iPhone and related & 141,319 & 136,700 & 155,041 & 101,991 & 91,279 & 78,692 & 45,998 & 25,177 \\
\hline
iPad and related & 19,222 & 20,628 & 23,227 & 30,283 & 31,980 & 30,945 & 19,168 & 4,957 \\
\hline
Services & 29,980 & 24,348 & 19,909 & 18,063 & 16,051 & 12,890 & 9,373 & 10,110 \\
\hline
$\begin{array}{l}\text { Other (includes } \\ \text { iPod) }\end{array}$ & 12,863 & 11,132 & 10,067 & 8,379 & 10,117 & 10,760 & 11,927 & 7,501 \\
\hline
Total & 229,234 & 215,639 & 233,715 & 182,795 & 170,910 & 156,508 & 108,249 & 65,225 \\
\hline
\multicolumn{9}{|l|}{$\begin{array}{l}\text { Net sales \% by } \\
\text { product }\end{array}$} \\
\hline
Mac & $11.3 \%$ & $10.6 \%$ & $10.9 \%$ & $13.2 \%$ & $12.6 \%$ & $14.8 \%$ & $20.1 \%$ & $26.8 \%$ \\
\hline
iPhone and related & $61.6 \%$ & $63.4 \%$ & $66.3 \%$ & $55.8 \%$ & $53.4 \%$ & $50.3 \%$ & $42.5 \%$ & $38.6 \%$ \\
\hline
iPad and related & $8.4 \%$ & $9.6 \%$ & $9.9 \%$ & $16.6 \%$ & $18.7 \%$ & $19.8 \%$ & $17.7 \%$ & $7.6 \%$ \\
\hline
Services & $13.1 \%$ & $11.3 \%$ & $8.5 \%$ & $9.9 \%$ & $9.4 \%$ & $8.2 \%$ & $8.7 \%$ & $15.5 \%$ \\
\hline
$\begin{array}{l}\text { Other (includes } \\ \text { iPod) }\end{array}$ & $5.6 \%$ & $5.2 \%$ & $4.3 \%$ & $4.6 \%$ & $5.9 \%$ & $6.9 \%$ & $11.0 \%$ & $11.5 \%$ \\
\hline
Total & $100.0 \%$ & $100.0 \%$ & $100.0 \%$ & $100.0 \%$ & $100.0 \%$ & $100.0 \%$ & $100.0 \%$ & $100.0 \%$ \\
\hline
\end{tabular}
\end{center}

Source: Apple 10-K filings.

Using the information provided, address the following:

\begin{enumerate}
  \item How have sales and gross margin changed over time?
\end{enumerate}

\section{Solution to 1:}
Since 2010 total sales have increased from $\$ 65$ billion to $\$ 229$ billion. This represents an annualized growth rate of almost $20 \%$. There was only one year that did not have sales growth in dollars (2016). Gross margin has ranged from $37.6 \%$ to $43.9 \%$. Gross margin increased from 2010 , when the iPad was introduced, through 2012, when it reached its peak. Gross margin then declined in 2013 and trended upward through 2015. There were modest declines in gross margin after 2015. 2. How has the company's product mix changed since the introduction of the iPad in 2010, and what might this change suggest for an analyst in evaluating Apple's profitability over time and its ability to maintain that profitability?

\section{Solution to 2:}
When the iPad was introduced in 2010 it received a significant share of the product mix, rising to $17.7 \%$ in 2011 , the first full year after introduction. The iPad's product mix share approached 20\% share in 2012 and then declined slightly for two years before a larger decline down to a relatively stable product mix share of around 9\%. This could be explained by reaching fairly widespread adoption. The iPhone also gained significant product mix share, rising steadily from $38.6 \%$ in 2010 to $66.3 \%$ in 2015 . Share declined slightly since 2015 but still remains the largest of Apple's product segments at more than $60 \%$. Sales of their original product, the Mac, have declined from more than $25 \%$ of sales to around $10 \%$. Services have changed significantly but have shown a steady increase in recent years, most likely due to Apple's music and other media subscription plans. Initially a blockbuster product, the iPod is now included in "other," and this is the largest driver of the decline in that category over time.

Apple had a history of introducing new products every few years, but in recent years the company has not created new product categories. Instead the company has periodically introduced new models of iPads and iPhones. The recent decline in margins is attributable in part to the lack of new products and services and highlights the importance of product innovation to Apple in maintaining historically healthy margins.

In calculating and interpreting financial statement ratios, an analyst needs to be aware of the potential impact on the financial statements and related ratios of companies reporting under different accounting standards, such as international financial reporting standards (IFRS), US generally accepted accounting principles (US GAAP), or other home-country GAAP. Furthermore, even within a given set of accounting standards, companies still have discretion to choose among acceptable methods. A company also may make different assumptions and estimates even when applying the same method as another company. Therefore, making selected adjustments to a company's financial statement data may be useful to facilitate comparisons with other companies or with the industry overall. Examples of such analyst adjustments will be discussed in Section 6.

Non-US companies that use any acceptable body of accounting standards (other than IFRS or US GAAP) and file with the US SEC (because their shares or depositary receipts based on their shares trade in the United States) are required to reconcile their net income and shareholders' equity accounts to US GAAP. Note that in 2007, the SEC eliminated the reconciliation requirement for non-US companies using IFRS and filing with the SEC, however companies may still voluntarily provide this information for comparison purposes.

In general, because the reconciliation data are no longer required by the SEC, we cannot always determine whether differences in net income, equity, and thus ROE also exist between IFRS and the companies' home-country GAAP (including US GAAP).

Comparison of the levels and trends in a company's performance provide information about how the company performed. The company's management presents its view about causes underlying its performance in the management commentary or management discussion and analysis (MD\&A) section of its annual report and during periodic conference calls with analysts and investors. To gain additional understanding of the causes underlying a company's performance, an analyst can review industry information or seek information from additional sources, such as consumer surveys. The results of an analysis of past performance provide a basis for reaching conclusions and making recommendations. For example, an analysis undertaken as the basis for a forward-looking study might conclude that a company's future performance is or is not likely to reflect continuation of recent historical trends. As another example, an analysis to support a market-based valuation of a company might focus on whether the company's profitability and growth outlook, which is better (worse) than the peer group median, justifies its relatively high (low) valuation. This analysis would consider market multiples, such as price-to-earnings ratio (P/E), price-to-book ratio, and total invested capital to EBITDA (earnings before interest, taxes, depreciation, and amortization). ${ }^{2}$ As another example, an analysis undertaken as part of an evaluation of the management of two companies might result in conclusions about whether one company has grown as fast as another company, or as fast as the industry overall, and whether each company has maintained profitability while growing.

\section{APPLICATION: PROJECTING FUTURE FINANCIAL PERFORMANCE AS AN INPUT TO MARKET BASED VALUATION}
demonstrate how to forecast a company's future net income and
cash flow

Projections of future financial performance are used in determining the value of a company or its equity component. Projections of future financial performance are also used in credit analysis-particularly in project finance or acquisition finance-to determine whether a company's cash flows will be adequate to pay the interest and principal on its debt and to evaluate whether a company will likely remain in compliance with its financial covenants.

Sources of data for analysts' projections include some or all of the following: the company's projections, the company's previous financial statements, industry structure and outlook, and macroeconomic forecasts.

Evaluating a company's past performance may provide a basis for forward-looking analyses. An evaluation of a company's business and economic environment and its history may persuade the analyst that historical information constitutes a valid basis for such analyses and that the analyst's projections may be based on the continuance of past trends, perhaps with some adjustments. Alternatively, in the case of a major acquisition or divestiture, for a start-up company, or for a company operating in a volatile industry, past performance may be less relevant to future performance.

Projections of a company's near-term performance may be used as an input to market-based valuation or relative valuation (i.e., valuation based on price multiples). Such projections may involve projecting next year's sales and using the common-size income statement to project major expense items or particular margins on sales (e.g., gross profit margin or operating profit margin). These calculations will then lead to the development of an income measure for a valuation calculation, such as net income, earnings per share (EPS) or EBITDA. More complex projections of a company's future performance involve developing a more detailed analysis of the components of performance for multiple periods-for example, projections of sales and gross margin

2 Total invested capital is the sum of market value of common equity, book value of preferred equity, and face value of debt. by product line, projection of operating expenses based on historical patterns, and projection of interest expense based on requisite debt funding, interest rates, and applicable taxes. Furthermore, a projection should include sensitivity analyses applied to the major assumptions.

\section{Projecting Performance: An Input to Market-Based Valuation}
One application of financial statement analysis involves projecting a company's near-term performance as an input to market-based valuation. For example, an analyst might project a company's sales and profit margin to estimate EPS and then apply a projected P/E to establish a target price for the company's stock.

Analysts often take a top-down approach to projecting a company's sales. ${ }^{3}$ First, industry sales are projected on the basis of their historical relationship with some macroeconomic indicator, such as growth in real gross domestic product (GDP). In researching the automobile industry, for example, the analyst may find that the industry's annual domestic unit car sales (number of cars sold in domestic markets) bears a relationship to annual changes in real GDP. Regression analysis is often used to establish the parameters of such relationships. Other factors in projecting sales may include consumer income or tastes, technological developments, and the availability of substitute products or services. After industry sales are projected, a company's market share is projected. Company-level market share projections may be based on historical market share and a forward-looking assessment of the company's competitive position. The company's sales are then estimated as its projected market share multiplied by projected total industry sales.

After developing a sales forecast for a company, an analyst can choose among various methods for forecasting income and cash flow. An analyst must decide on the level of detail to consider in developing forecasts. For example, separate forecasts may be made for individual expense items or for more aggregated expense items, such as total operating expenses. Rather than stating a forecast in terms of expenses, the forecast might be stated in terms of a forecasted profit margin (gross, operating, or net). The net profit margin, in contrast to the gross or operating profit margins, is affected by financial leverage and tax rates, which are subject to managerial and legal/regulatory revisions; therefore, historical data may sometimes be more relevant for projecting gross or operating profit margins than for projecting net profit margins. Whatever the margin used, the forecasted amount of profit for a given period is the product of the forecasted amount of sales and the forecast of the selected profit margin.

As Example 2 illustrates, for relatively mature companies operating in non-volatile product markets, historical information on operating profit margins can provide a useful starting point for forecasting future operating profits (at least over short forecasting horizons). Historical operating profit margins are typically less reliable for projecting future margins for a new or relatively volatile business or one with significant fixed costs (which can magnify the volatility of operating margins).

3 The discussion in this paragraph is indebted to Benninga and Sarig (1997).

\section{EXAMPLE 2}
\section{Using Historical Operating Profit Margins to Forecast Operating Profit}
One approach to projecting operating profit is to determine a company's average operating profit margin over the previous several years and apply that margin to a forecast of the company's sales. Use the following information on three companies to answer Questions 1 and 2 below:

\begin{itemize}
  \item Johnson \& Johnson (JNJ). This US health care conglomerate, founded in 1887, had 2017 sales of around $\$ 76.5$ billion from its three main businesses: pharmaceuticals, medical devices and diagnostics, and consumer products.

  \item BHP Billiton (BHP). This company, with group headquarters in Australia and secondary headquarters in London, is the world's largest natural resources company, reporting revenue of approximately US $\$ 38.3$ billion for the fiscal year ended June 2017. The company mines, processes, and markets coal, copper, nickel, iron, bauxite, and silver and also has substantial petroleum operations.

  \item Baidu. This Chinese company, which was established in 2000 and went public on NASDAQ in 2005, is the leading Chinese language search engine. The company's revenues for 2017 were 84.8 billion renminbi (RMB), an increase of 20 percent from 2016 and almost 4 times revenue in 2012.

\end{itemize}

Using the additional information given, state and justify whether actual results support the usefulness of the stable operating margin assumption.

\begin{enumerate}
  \item For each of the three companies, state and justify whether the suggested forecasting method (applying the average operating profit over the previous several years to a forecast of sales) would be a reasonable starting point for projecting future operating profit.
\end{enumerate}

\section{Solution to 1:}
JNJ. Because JNJ is an established company with diversified operations in relatively stable businesses, the suggested approach to projecting the company's operating profit would be a reasonable starting point.

BHP. Because commodity prices tend to be volatile and the mining industry is relatively capital intensive, the suggested approach to projecting BHP's operating profit would probably not be a useful starting point.

Baidu. Compared to the other two companies, Baidu has a more limited operating history and remains in a period of rapid growth. These aspects about the company suggests that the broad approach to projecting operating profit would not be a useful starting point for Baidu.

\begin{enumerate}
  \setcounter{enumi}{1}
  \item Assume that the 2017 forecast of sales was perfect and, therefore, equal to the realized sales by the company in 2017. Compare the forecast of 2017 operating profit, using an average of the previous five years' operating profit margins, with the actual 2017 operating profit reported by the company given the following additional information:
\end{enumerate}

\begin{itemize}
  \item JNJ: For the five years prior to 2017, JNJ's average operating profit margin was approximately 25.6 percent. The company's actual operating profit for 2017 was $\$ 18.2$ billion.

  \item BHP: For the four years prior to the year ending June 2017, BHP's average operating profit margin was approximately 24.0 percent. The company's actual operating profit for the year ended June 2017 was US $\$ 11.8$ billion.

  \item Baidu: Over the four years prior to 2017, Baidu's average operating profit margin was approximately 28.5 percent. The company's actual operating profit for 2017 was RMB15.7 billion.

\end{itemize}

\section{Solution to 2:}
JNJ. JNJ's actual operating profit margin for 2017 was 23.8 percent ( $\$ 18.2$ billion divided by sales of $\$ 76.5$ billion), which is a little less than company's five-year average operating profit margin of approximately 25.6 percent.

BHP. BHP's actual operating profit margin for the year ended June 2017 was 30.8 percent ( $\$ 11.8$ billion divided by sales of $\$ 38.3$ billion). If the company's average profit margin of 24.0 percent had been applied to perfectly forecasted sales, the forecasted operating profit would have been approximately US $\$ 9.2$ billion, around 22 percent lower than actual operating profit.

Baidu. Baidu's actual operating profit margin for 2017 was 18.5 percent (RMB15.7 billion divided by sales of RMB84.8 billion). If the average profit margin of 28.5 percent had been applied to perfectly forecasted sales, the forecasted operating profit would have been approximately RMB24.2 billion, or around 54 percent higher than Baidu's actual operating profit.

Although prior years' profit margins can provide a useful starting point in projections for companies with relatively stable business, the underlying data should, nonetheless, be examined to identify items that are not likely to occur again in the following year(s). Such non-recurring (i.e., transitory) items should be removed from computations of any profit amount or profit margin that will be used in projections. Example 3 illustrates this principle.

\section{EXAMPLE 3}
\section{Issues in Forecasting}
Following are excerpts from the 2017 annual report of Textron, a global aircraft, defense and industrial company.

Textron Consolidated Statements of Operations for each of the years in the three-year period ended December 31

(In millions, except per share data)

\begin{center}
\begin{tabular}{|c|c|c|c|}
\hline
(In millions, except per share data) & 2017 & 2016 & 2015 \\
\hline
Manufacturing revenues & $\$ 14,129$ & $\$ 13,710$ & $\$ 13,340$ \\
\hline
Finance revenues & 69 & 78 & 83 \\
\hline
Total revenues & 14,198 & 13,788 & 13,423 \\
\hline
\multicolumn{4}{|l|}{Costs, expenses and other} \\
\hline
Cost of sales & 11,795 & 11,311 & 10,979 \\
\hline
Selling and administrative expense & 1,337 & 1,304 & 1,304 \\
\hline
Interest expense & 174 & 174 & 169 \\
\hline
Special charges & 130 & 123 & - \\
\hline
Total costs, expenses and other & 13,436 & 12,912 & 12,45 \\
\hline
\multicolumn{4}{|l|}{Income from continuing operations before} \\
\hline
income taxes & 762 & 876 & 971 \\
\hline
Income tax expense & 456 & 33 & 273 \\
\hline
Income from continuing operations & 306 & 843 & 698 \\
\hline
$\begin{array}{l}\text { Income (loss) from discontinued operations, net } \\ \text { of income taxes* }\end{array}$ & 1 & 119 & (1) \\
\hline
Net income & 307 & 962 & 697 \\
\hline
\end{tabular}
\end{center}

Footnotes:

2017 Note 12 Special Charges

In 2016, we initiated a plan to restructure and realign our businesses by implementing headcount reductions, facility consolidations and other actions in order to improve overall operating efficiency across Textron. Under this plan, Textron Systems discontinued production of its sensor-fuzed weapon product within its Weapons and Sensors operating unit, we combined our Jacobsen business with the Textron Specialized Vehicles business by consolidating facilities and general and administrative functions, and we reduced headcount at Textron Aviation, as well as other businesses and corporate functions. In December 2017, we decided to take additional restructuring actions to further consolidate operating facilities and streamline product lines, primarily within the Bell, Textron Systems and Industrial segments, which resulted in additional special charges of $\$ 45$ million in the fourth quarter of 2017. We recorded total special charges of $\$ 213$ million since the inception of the 2016 plan, which included $\$ 97$ million of severance costs, $\$ 84$ million of asset impairments and $\$ 32$ million in contract terminations and other costs. Of these amounts, $\$ 83$ million was incurred at Textron Systems, $\$ 63$ million at Textron Aviation, $\$ 38$ million at Industrial, $\$ 28$ million at Bell and $\$ 1$ million at Corporate. The total headcount reduction under this plan is expected to be approximately 2,100 positions, representing $5 \%$ of our workforce.

In connection with the acquisition of Arctic Cat, as discussed in Note 2, we initiated a restructuring plan in the first quarter of 2017 to integrate this business into our Textron Specialized Vehicles business within the Industrial segment and reduce operating redundancies and maximize efficiencies. Under the Arctic Cat plan, we recorded restructuring charges of $\$ 28$ million in 2017, which included $\$ 19$ million of severance costs, largely related to change-of-control provisions, and $\$ 9$ million of contract termination and other costs. In addition, we recorded $\$ 12$ million of acquisition-related integration and transaction costs in 2017.

2016 Financial Statement General Footnote "Income from discontinued operations, net of income taxes for the year ended December 31, 2016 primarily includes the settlement of a U.S. federal income tax audit. See Note 13 to the Consolidated Financial Statements for additional information.

2016 Note 13 Income Taxes

The provision for income taxes for 2016 included a benefit of $\$ 319$ million to reflect the settlement with the U.S. Internal Revenue Service Office of Appeals for our 1998 to 2008 tax years, which resulted in a $\$ 206$ million benefit attributable to continuing operations and $\$ 113$ million attributable to discontinued operations.

Source: Textron annual reports.

\section{Discussion:}
Results of discontinued operations and restructuring charges should generally not be included when assessing past performance or when forecasting future net income. For purposes of evaluating the company's ongoing operating and net profit margins the special charges related to restructuring and the special tax benefit related to discontinued operations should be removed. For example, the company's operating margin for 2017 including special charges would be 5.4\% (\$762 million/\$14,198 million). Excluding special charges, the operating margin would be 6.3\% (\$762 million + $\$ 130$ million)/\$14,198 million. Similarly, the net profit margin would be determined by eliminating the income from discontinued operations, particularly for 2016.

In general, when earnings projections are used as a foundation for market-based valuations, an analyst will make appropriate allowance for transitory components of past earnings. Occasionally, an analyst will observe that a company takes special charges virtually every year. In such cases, they are not transitory and should not be removed in evaluating past and future margins.

\section{PROJECTING MULTIPLE-PERIOD PERFORMANCE}
$$
\mid \begin{aligned}
& \text { demonstrate how to forecast a company's future net income and } \\
& \text { cash flow }
\end{aligned}
$$

Projections of future financial performance over multiple periods are needed in valuation models that estimate the value of a company or its equity by discounting future cash flows. The value of a company or its equity developed in this way can then be compared with its current market price as a basis for investment decisions.

Projections of future performance are also used for credit analysis. These projections are important in assessing a borrower's ability to repay interest and principal of debt obligations. Investment recommendations depend on the needs and objectives of the client and on an evaluation of the risk of the investment relative to its expected return-both of which are a function of the terms of the debt obligation itself as well as financial market conditions. Terms of the debt obligation include amount, interest rate, maturity, financial covenants, and collateral. Example 4 presents an elementary illustration of net income and cash flow forecasting to illustrate a format for analysis and some basic principles. In Example 4, assumptions are shown first; then, the period-by-period abbreviated financial statement resulting from the assumptions is shown.

Depending on the use of the forecast, an analyst may choose to compute further, more specific cash flow metrics. For example, free cash flow to equity, which is used in discounted cash flow approaches to equity valuation, can be estimated as net income adjusted for noncash items, minus investment in net working capital and in net fixed assets, plus net borrowing.

\section{EXAMPLE 4}
\section{Basic Example of Financial Forecasting}
\begin{enumerate}
  \item Assume a company is formed with $\$ 100$ of equity capital, all of which is immediately invested in working capital. Assumptions are as follows:
\end{enumerate}

\begin{center}
\begin{tabular}{lr}
\hline
Variable & Assumption \\
\hline
First-year sales & $\$ 100$ \\
Sales growth & $10 \%$ per year \\
Cost of goods sold/Sales & $20 \%$ \\
Operating expense/Sales & $70 \%$ \\
Interest income rate & $5 \%$ \\
Tax rate & $30 \%$ \\
Working capital as percent of sales & $90 \%$ \\
Dividends & Non-dividend paying \\
\end{tabular}
\end{center}

Based on this information, forecast the company's net income and cash flow for five years.

\section{Solution:}
Exhibit 2 shows the net income forecasts in Line 7 and cash flow forecasts (“Change in cash") in Line 18.

\section{Exhibit 2: Basic Financial Forecasting}
\begin{center}
\begin{tabular}{|c|c|c|c|c|c|c|}
\hline
 & \multicolumn{6}{|c|}{Time Period} \\
\hline
 & 0 & 1 & 2 & 3 & 4 & 5 \\
\hline
(1) Sales &  & 100.0 & 110.0 & 121.0 & 133.1 & 146.4 \\
\hline
(2) Cost of goods sold &  & $(20.0)$ & $(22.0)$ & $(24.2)$ & $(26.6)$ & $(29.3)$ \\
\hline
(3) Operating expenses &  & $(70.0)$ & $(77.0)$ & $(84.7)$ & $(93.2)$ & $(102.5)$ \\
\hline
(4) Interest income &  & 0.0 & 0.9 & 0.8 & 0.8 & 0.7 \\
\hline
(5) Income before tax &  & 10.0 & 11.9 & 12.9 & 14.1 & 15.3 \\
\hline
(6) Taxes &  & (3.0) & (3.6) & (3.9) & $(4.2)$ & (4.6) \\
\hline
(7) Net income &  & 7.0 & 8.3 & 9.0 & 9.9 & 10.7 \\
\hline
(8) Cash/Borrowing & 0.0 & 17.0 & 16.3 & 15.4 & 14.4 & 13.1 \\
\hline
(9) Working capital (non-cash) & 100.0 & 90.0 & 990 & 108.9 & 119.8 & 131.8 \\
\hline
\end{tabular}
\end{center}

\begin{center}
\begin{tabular}{|c|c|c|c|c|c|c|}
\hline
 & \multicolumn{6}{|c|}{Time Period} \\
\hline
 & 0 & 1 & 2 & 3 & 4 & 5 \\
\hline
(10) Total assets & 100.0 & 107.0 & 115.3 & 124.3 & 134.2 & 144.9 \\
\hline
(11) Liabilities & 0.0 & 0.0 & 0.0 & 0.0 & 0.0 & 0.0 \\
\hline
(12) Equity & 100.0 & 107.0 & 115.3 & 124.3 & 134.2 & 144.9 \\
\hline
(13) Total liabilities + Equity & 100.0 & 107.0 & 115.3 & 124.3 & 134.2 & 144.9 \\
\hline
(14) Net income &  & 7.0 & 8.3 & 9.0 & 9.9 & 10.7 \\
\hline
(15) Plus: Non-cash items &  & 0.0 & 0.0 & 0.0 & 0.0 & 0.0 \\
\hline
(16) Less: Investment in working capital &  & -10.0 & 9.0 & 9.9 & 10.9 & 12.0 \\
\hline
(17) Less: Investment in fixed capital &  & 0.0 & 0.0 & 0.0 & 0.0 & 0.0 \\
\hline
(18) Change in cash &  & 17.0 & -0.7 & -0.9 & -1.0 & -1.3 \\
\hline
(19) Beginning cash &  & 0.0 & 17.0 & 16.3 & 15.4 & 14.4 \\
\hline
(20) Ending cash &  & 17.0 & 16.3 & 15.4 & 14.4 & 13.1 \\
\hline
\end{tabular}
\end{center}

Exhibit 2 indicates that at time 0 , the company is formed with $\$ 100$ of equity capital (Line 12). All of the company's capital is assumed to be immediately invested in working capital (Line 9). In future periods, because it is assumed that no dividends are paid, book equity increases each year by the amount of net income (Line 14). Future periods' required working capital (Line 9) is assumed to be 90 percent of annual sales (Line 1). Sales are assumed to be $\$ 100$ in the first period and to grow at a constant rate of 10 percent per year (Line 1). The cost of goods sold is assumed to be constant at 20 percent of sales (Line 2), so the gross profit margin is 80 percent. Operating expenses are assumed to be 70 percent of sales each year (Line 3). Interest income (Line 4) is calculated as 5 percent of the beginning balance of cash/borrowing or the ending balance of the previous period (Line 8) and is an income item when there is a cash balance, as in this example. (If available cash is inadequate to cover required cash outflows, the shortfall is presumed to be covered by borrowing. This borrowing would be shown as a negative balance on Line 8 and an associated interest expense on Line 4. Alternatively, a forecast can be presented with separate lines for cash and borrowing.) Taxes of 30 percent are deducted to obtain net income (Line 7).

To calculate each period's cash flow, begin with net income (Line 7 = Line 14), add back any noncash items, such as depreciation (Line 15), deduct investment in working capital in the period or change in working capital over the period (Line 16), and deduct investment in fixed capital in the period (Line 17). ${ }^{4}$ In this simple example, we are assuming that the company does not invest in any fixed capital (long-term assets) but, rather, rents furnished office space. Therefore, there is no depreciation and noncash items are zero. Each period's change in cash (Line 18) is added to the beginning cash balance (Line 19) to obtain the ending cash balance (Line 20 = Line 8 ).

Example 4 is simplified to demonstrate some principles of forecasting. In practice, each aspect of a forecast presents a range of challenges. Sales forecasts may be very detailed, with separate forecasts for each year of each product line, each geographical, and/or each business segment. Sales forecasts may be based on past results (for

\begin{enumerate}
  \setcounter{enumi}{3}
  \item Working capital represents funds that must be invested in the daily operations of a business to, for example, carry inventory and accounts receivable. The term "investment" in this context means "addition to" or "increase in." The "investment in fixed capital" is also referred to as "capital expenditure" ("capex"). relatively stable businesses), management forecasts, industry studies, and/or macroeconomic forecasts. Similarly, gross profit margins may be based on past results or forecasted relationships and may be detailed. Expenses other than cost of goods sold may be broken down into more detailed line items, each of which may be forecasted on the basis of its relationship with sales (if variable) or on the basis of its historical levels. Working capital requirements may be estimated as a proportion of the amount of sales (as in Example 4) or the change in sales or as a compilation of specific forecasts for inventory, receivables, and payables. Most forecasts will involve some investment in fixed assets, in which case, depreciation amounts affect taxable income and net income but not cash flow. Example 4 makes the simplifying assumption that interest is paid on the beginning-of-year cash balance.
\end{enumerate}

Example 4 develops a series of point estimates for future net income and cash flow. In practice, forecasting generally includes an analysis of the risk in forecasts-in this case, an assessment of the impact on income and cash flow if the realized values of variables differ significantly from the assumptions used in the base case or if actual sales are much different from forecasts. Quantifying the risk in forecasts requires an analysis of the economics of the company's businesses and expense structure and the potential impact of events affecting the company, the industry, and the economy in general. When that investigation is completed, the analyst can use scenario analysis or Monte Carlo simulation to assess risk. Scenario analysis involves specifying assumptions that differ from those used as the base-case assumptions. In Example 4, the projections of net income and cash flow could be recast in a more pessimistic scenario, with assumptions changed to reflect slower sales growth and higher costs. A Monte Carlo simulation involves specifying probability distributions of values for variables and random sampling from those distributions. In the analysis in Example 4, the projections would be repeatedly recast with the selected values for the drivers of net income and cash flow, thus permitting the analyst to evaluate a range of possible results and the probability of simulating the possible actual outcomes.

An understanding of financial statements and ratios can enable an analyst to make more detailed projections of income statement, balance sheet, and cash flow statement items. For example, an analyst may collect information on normal inventory and receivables turnover and use this information to forecast accounts receivable, inventory, and cash flows based on sales projections rather than use a composite working capital investment assumption, as in Example 4.

As the analyst makes detailed forecasts, he or she must ensure that the forecasts are consistent with each other. For instance, in Example 5, the analyst's forecast concerning days of sales outstanding (which is an estimate of the average time to collect payment from sales made on credit) should flow from a model of the company that yields a forecast of the change in the average accounts receivable balance. Otherwise, predicted days of sales outstanding and accounts receivable will not be mutually consistent.

\section{EXAMPLE 5}
\section{Consistency of Forecasts}
\begin{enumerate}
  \item Brown Corporation, a hypothetical company, had an average days-of-sales-outstanding (DSO) period of 19 days in 2017. An analyst thinks that Brown's DSO will decline in 2018 (because of expected improvements in the company's collections department) to match the industry average of 15 days. Total sales (all on credit) in 2017 were $\$ 300$ million, and Brown expects total sales (all on credit) to increase to $\$ 320$ million in 2018. To achieve the lower DSO, the change in the average accounts receivable balance from 2017 to 2018 that must occur is closest to:
A. $-\$ 3.51$ million.
B. $-\$ 2.46$ million.
C. $\$ 2.46$ million
D. $\$ 3.51$ million.
\end{enumerate}

\section{Solution:}
B is correct. The first step is to calculate accounts receivable turnover from the DSO collection period. Receivable turnover equals 365/19 (DSO) = 19.2 for 2017 and $365 / 15=24.3$ in 2018 . Next, the analyst uses the fact that the average accounts receivable balance equals sales/receivable turnover to conclude that for 2017 , average accounts receivable was $\$ 300,000,000 / 19.2$ $=\$ 15,625,000$ and for 2018 , it must equal $\$ 320,000,000 / 24.3=\$ 13,168,724$. The difference is a reduction in receivables of $\$ 2,456,276$.

The next section illustrates the application of financial statement analysis to credit risk analysis.

\section{APPLICATION: ASSESSING CREDIT RISK}
describe the role of financial statement analysis in assessing the credit quality of a potential debt investment

Credit risk is the risk of loss caused by a counterparty's or debtor's failure to make a promised payment. For example, credit risk with respect to a bond is the risk that the obligor (the issuer of the bond) will not be able to pay interest and/or principal according to the terms of the bond indenture (contract). Credit analysis is the evaluation of credit risk. Credit analysis may relate to the credit risk of an obligor in a particular transaction or to an obligor's overall creditworthiness.

In assessing an obligor's overall creditworthiness, one general approach is credit scoring, a statistical analysis of the determinants of credit default. Credit analysis for specific types of debt (e.g., acquisition financing and other highly leveraged financing) typically involves projections of period-by-period cash flows.

Whatever the techniques adopted, the analytical focus of credit analysis is on debt-paying ability. Unlike payments to equity investors, payments to debt investors are limited by the agreed contractual interest. If a company experiences financial success, its debt becomes less risky but its success does not increase the amount of payments to its debtholders. In contrast, if a company experiences financial distress, it may be unable to pay interest and principal on its debt obligations. Thus, credit analysis has a special concern with the sensitivity of debt-paying ability to adverse events and economic conditions-cases in which the creditor's promised returns may be most at risk. Because those returns are generally paid in cash, credit analysis usually focuses on cash flow rather than accrual income. Typically, credit analysts use return measures related to operating cash flow because it represents cash generated internally, which is available to pay creditors. These themes are reflected in Example 6, which illustrates Moody's application of four quantitative factors in the credit analysis of the aerospace and defense industry. ${ }^{5}$ These factors include

\begin{enumerate}
  \item scale,

  \item business profile,

  \item leverage and coverage, and

  \item financial policy.

\end{enumerate}

"Scale" relates to a company's sensitivity to adverse events, adverse economic conditions, and other factors-such as market leadership, purchasing power with suppliers, and access to capital markets-that may affect debt-paying ability. "Business profile" represents a company's competitive position, stability of revenues, product and geographic diversity, growth prospects, and the stability and volatility of cash flows. "Leverage and coverage" reflects a company's "financial flexibility" and viability. Finally, "financial policy" relates to a company's financial risk tolerance and its capital structure.

\section{EXAMPLE 6}
Moody's Evaluation of Quantifiable Rating Factors for the Aerospace and Defense Industry

\begin{enumerate}
  \item Moody's considers four broad rating factors for the aerospace and defense industry: scale; business profile, leverage and coverage; and financial policy. A company's ratings for each of these factors are weighted and aggregated in determining the overall credit rating assigned. The broad factors, the sub-factors, and weightings are as follows:
\end{enumerate}

\begin{center}
\begin{tabular}{llcc}
\hline
Broad Factor & Sub-factors & $\begin{array}{c}\text { Sub-factor } \\ \text { Weighting } \\ \text { (\%) }\end{array}$ & $\begin{array}{c}\text { Broad Factor } \\ \text { Weighting } \\ (\%)\end{array}$ \\
\hline
Scale & Total revenue &  &  \\
Operating profit & 10 & 25 &  \\
Business profile & $\begin{array}{l}\text { Competitive position } \\ \text { Expected revenue stability }\end{array}$ & 15 &  \\
Leverage and & Debt/EBITDA & 10 & 20 \\
coverage & Retained cash flow/Net debt & 15 &  \\
Financial policy & Financial policy & 10 &  \\
\hline
Total &  & 20 & 20 \\
\hline
\end{tabular}
\end{center}

a Retained cash flow is defined by Moody's as cash flow before working capital and after dividends.

Why might the leverage and coverage factor be weighted higher compared to the other rating factors?

5 The information in this paragraph and in Example 7 are based upon the "Rating Methodology: Aerospace and Defense Industry" (Moody's, 2018).

\section{Solution:}
The level of debt relative to earnings and cash flow is a critical factor in assessing creditworthiness. Higher levels of debt for a company typically result in a higher risk in meeting interest and principal payments on its debt obligations.

A point to note regarding Example 6 is that the rating factors and the metrics used to represent each can vary by industry group.

Analyses of a company's historical and projected financial statements are an integral part of the credit evaluation process. Moody's and other rating agencies compute a variety of ratios in assessing creditworthiness. A comparison of a company's ratios with the ratios of its peers is informative in evaluating relative creditworthiness, as demonstrated in Example 7.

\section{EXAMPLE 7}
\section{Peer Comparison of Ratios}
\begin{enumerate}
  \item A credit analyst is assessing the efficiency and leverage of two aerospace and defense companies based on certain sub-factors identified by Moody's in Example 7. The analyst collects the information from the companies' annual reports and calculates the following ratios:
\end{enumerate}

\begin{center}
\begin{tabular}{lcc}
\hline
 & Company $\mathbf{1}$ & Company 2 \\
\hline
Debt/EBITDA & 9.3 & 4.1 \\
Retained cash flow/Net debt & $2.6 \%$ & $9.6 \%$ \\
EBIT/Interest & 5.7 & 8.2 \\
\hline
\end{tabular}
\end{center}

Based solely on the data given, which company is more likely to be assigned a higher credit rating, and why?

\section{Solution:}
The ratio comparisons are all in favor of Company 2, which has a lower level of debt relative to EBITDA, higher retained cash flow to net debt, and higher interest coverage. Based only on the data given, Company 2 is likely to be assigned a higher credit rating.

In calculating credit ratios, such as those presented in Example 8, analysts typically make certain adjustments to reported financial statements. We describe some common adjustments later in the reading.

Financial statement analysis, especially financial ratio analysis, can also be an important tool in selecting equity investments, as discussed in the next section.

\section{SCREENING FOR POTENTIAL EQUITY INVESTMENTS}
describe the use of financial statement analysis in screening for potential equity investments Ratios constructed from financial statement data and market data are often used to screen for potential equity investments. Screening is the application of a set of criteria to reduce a set of potential investments to a smaller set having certain desired characteristics. Criteria involving financial ratios generally involve comparing one or more ratios with some pre-specified target or cutoff values.

A security selection approach incorporating financial ratios may be applied whether the investor uses top-down analysis or bottom-up analysis. Top-down analysis involves identifying attractive geographical segments and/or industry segments, from which the investor chooses the most attractive investments. Bottom-up analysis involves selection of specific investments from all companies within a specified investment universe. Regardless of the direction, screening for potential equity investments aims to identify companies that meet specific criteria. An analysis of this type may be used as the basis for directly forming a portfolio, or it may be undertaken as a preliminary part of a more thorough analysis of potential investment targets.

Fundamental to this type of analysis are decisions about which metrics to use as screens, how many metrics to include, what values of those metrics to use as cutoff points, and what weighting to give each metric. Metrics may include not only financial ratios but also characteristics such as market capitalization or membership as a component security in a specified index. Exhibit 3 presents a hypothetical example of a simple stock screen based on the following criteria: a valuation ratio (P/E) less than a specified value, a solvency ratio measuring financial leverage (calculated as total liabilities/total assets) not exceeding a specified value, positive operating margin, and dividend yield (dividends per share divided by price per share) greater than a specified value. Exhibit 3 shows the results of applying the screen in August 2018 to a set of 6,406 US companies with market capitalization greater than $\$ 100$ million, which compose a hypothetical equity manager's investment universe.

\section{Exhibit 3: Example of a Stock Screen}
\begin{center}
\begin{tabular}{lcc}
\hline
 & Stocks Meeting Criterion &  \\
\hline
Criterion & Number & Percent of Total \\
\hline
Market Capitalization $>\$ 100$ million & 4,357 & $68.01 \%$ \\
P/E $<15$ & 1,104 & $17.23 \%$ \\
Total liabilities/Total Assets $\leq 0.9$ & 61 & $0.95 \%$ \\
Operating Income/Sales $>0$ & 3,509 & $54.78 \%$ \\
Dividend yield $>0.5 \%$ & 2,391 & $37.32 \%$ \\
Meeting all five criteria simultaneously & 17 & $0.27 \%$ \\
\hline
\end{tabular}
\end{center}

Source for data: \href{http://google.com/finance/}{http://google.com/finance/}.

Several points about the screen in Exhibit 3 are consistent with many screens used in practice:

\begin{itemize}
  \item Some criteria serve as checks on the results from applying other criteria. In this hypothetical example, the second criterion selects stocks that appear relatively cheaply valued. The stocks might be cheap for a good reason, however, such as poor profitability or excessive financial leverage. So, the requirement for net income to be positive serves as a check on profitability, and the limitation on financial leverage serves as a check on financial risk. Of course, financial ratios or other statistics cannot generally control for exposure to certain other types of risk (e.g., risk related to regulatory developments or technological innovation). - If all the criteria were completely independent of each other, the set of stocks meeting all five criteria would be 2, equal to 6,406 times 0.023 percent-the product of the fraction of stocks satisfying the five criteria individually (i.e., $0.6801 \times 0.1723 \times 0.0095 \times 0.5478 \times 0.3732=0.000228$, or 0.023 percent). As the screen illustrates, criteria are often not independent, and the result is that more securities pass the screening than if criteria were independent. In this example, 17 of the securities pass all five screens simultaneously. For an example of the lack of independence, we note that dividend-paying status is probably positively correlated with the ability to generate positive operating margin. If stocks that pass one test tend to also pass another, few are eliminated after the application of the second test.

  \item The results of screens can sometimes be relatively concentrated in a subset of the sectors represented in the benchmark. The financial leverage criterion in Exhibit 3 would exclude banking stocks, for example. What constitutes a high or low value of a measure of a financial characteristic can be sensitive to the industry in which a company operates.

\end{itemize}

Screens can be used by both growth investors (focused on investing in high-earnings-growth companies), value investors (focused on paying a relatively low share price in relation to earnings or assets per share), and market-oriented investors (an intermediate grouping of investors whose investment disciplines cannot be clearly categorized as value or growth). Growth screens would typically feature criteria related to earnings growth and/or momentum. Value screens, as a rule, feature criteria setting upper limits for the value of one or more valuation ratios. Market-oriented screens would not strongly emphasize valuation or growth criteria. The use of screens involving financial ratios may be most common among value investors.

An analyst may want to evaluate how a portfolio based on a particular screen would have performed historically. For this purpose, the analyst uses a process known as "back-testing." Back-testing applies the portfolio selection rules to historical data and calculates what returns would have been earned if a particular strategy had been used. The relevance of back-testing to investment success in practice, however, may be limited. Haugen and Baker (1996) described some of these limitations:

\begin{itemize}
  \item Survivorship bias: If the database used in back-testing eliminates companies that cease to exist because of a bankruptcy or merger, then the remaining companies collectively will appear to have performed better.

  \item Look-ahead bias: If a database includes financial data updated for restatements (where companies have restated previously issued financial statements to correct errors or reflect changes in accounting principles), then there is a mismatch between what investors would have actually known at the time of the investment decision and the information used in the back-testing.

  \item Data-snooping bias: If researchers build a model on the basis of previous researchers' findings, then use the same database to test that model, they are not actually testing the model's predictive ability. When each step is backward looking, the same rules may or may not produce similar results in the future. The predictive ability of the model's rules can validly be tested only by using future data. One academic study has argued that the apparent ability of value strategies to generate excess returns is largely explainable as the result of collective data snooping (Conrad, Cooper, and Kaul, 2003).

\end{itemize}

\section{EXAMPLE 8}
\section{Ratio-Based Screening for Potential Equity Investments}
Below are two alternative strategies under consideration by an investment firm:

Strategy $A$ : Invest in stocks that are components of a global equity index, have a ROE above the median ROE of all stocks in the index, and have a $\mathrm{P} / \mathrm{E}$ less than the median $\mathrm{P} / \mathrm{E}$.

Strategy $B$ : Invest in stocks that are components of a broad-based US equity index, have a ratio of price to operating cash flow in the lowest quartile of companies in the index, and have shown increases in sales for at least the past three years.

Both strategies were developed with the use of back-testing.

\begin{enumerate}
  \item How would you characterize the two strategies?
\end{enumerate}

\section{Solution to 1:}
Strategy A appears to aim for global diversification and combines a requirement for high relative profitability with a traditional measure of value (low P/E). Strategy B focuses on both large and small companies in a single market and apparently aims to identify companies that are growing and have a lower price multiple based on cash flow from operations.

\begin{enumerate}
  \setcounter{enumi}{1}
  \item What concerns might you have about using such strategies?
\end{enumerate}

\section{Solution to 2:}
The use of any approach to investment decisions depends on the objectives and risk profile of the investor. With that crucial consideration in mind, we note that ratio-based benchmarks may be an efficient way to screen for potential equity investments. In screening, however, many questions arise.

First, unintentional selections can be made if criteria are not specified carefully. For example, Strategy A might unintentionally select a loss-making company with negative shareholders' equity because negative net income divided by negative shareholders' equity arithmetically results in a positive ROE. Strategy B might unintentionally select a company with negative operating cash flow because price to operating cash flow will be negative and thus very low in the ranking. In both cases, the analyst can add additional screening criteria to avoid unintentional selections; these additional criteria could include requiring positive shareholders' equity in Strategy A and requiring positive operating cash flow in Strategy B.

Second, the inputs to ratio analysis are derived from financial statements, and companies may differ in the financial standards they apply (e.g., IFRS versus US GAAP), the specific accounting method(s) they choose within those allowed by the reporting standards, and/or the estimates made in applying an accounting method.

Third, back-testing may not provide a reliable indication of future performance because of survivorship bias, look-ahead bias, or data-snooping bias. Also, as suggested by finance theory and by common sense, the past is not necessarily indicative of the future.

Fourth, implementation decisions can dramatically affect returns. For example, decisions about frequency and timing of portfolio re-evaluation and changes affect transaction costs and taxes paid out of the portfolio.

\section{FRAMEWORK FOR ANALYST ADJUSTMENTS \& ADJUSTMENTS TO INVESTMENTS \& ADJUSTMENTS TO INVENTORY}
explain appropriate analyst adjustments to a company's financial
statements to facilitate comparison with another company

When comparing companies that use different accounting methods or estimate key accounting inputs in different ways, analysts frequently adjust a company's financials. In this section, we first provide a framework for considering potential analyst adjustments to facilitate such comparisons and then provide examples of such adjustments. In practice, required adjustments vary widely. The examples presented here are not intended to be comprehensive but, rather, to illustrate the use of adjustments to facilitate a meaningful comparison.

\section{A Framework for Analyst Adjustments}
In this discussion of potential analyst adjustments to a company's financial statements, we use a framework focused on the balance sheet. Because the financial statements are interrelated, however, adjustments to items reported on one statement may also be reflected in adjustments to items on another financial statement. For example, an analyst adjustment to inventory on the balance sheet affects cost of goods sold on the income statement (and thus also affects net income and, subsequently, the retained earnings account on the balance sheet).

Regardless of the particular order in which an analyst considers the items that may require adjustment for comparability, the following aspects are appropriate:

\begin{itemize}
  \item Importance (materiality). Is an adjustment to this item likely to affect the conclusions? In other words, does it matter? For example, in an industry where companies require minimal inventory, does it matter that two companies use different inventory accounting methods?

  \item Body of standards. Is there a difference in the body of standards being used (US GAAP versus IFRS)? If so, in which areas is the difference likely to affect a comparison?

  \item Methods. Is there a difference in accounting methods used by the companies being compared?

  \item Estimates. Is there a difference in important estimates used by the companies being compared?

\end{itemize}

The following sections illustrate analyst adjustments-first, those relating to the asset side of the balance sheet and then those relating to the liability side.

\section{Analyst Adjustments Related to Investments}
Accounting for investments in the debt and equity securities of other companies (other than investments accounted for under the equity method and investments in consolidated subsidiaries) depends on management's intention (i.e., whether to actively trade the securities, make them available for sale, or in the case of debt securities, hold them to maturity). When securities are classified as "financial assets measured at fair value through profit or loss" (similar to "trading" securities in US GAAP), unrealized gains and losses are reported in the income statement. When securities are classified as "financial assets measured at fair value through other comprehensive income" (similar to "available-for-sale" securities in US GAAP), unrealized gains and losses are not reported in the income statement and, instead, are recognized in equity. If two otherwise comparable companies have significant differences in the classification of investments, analyst adjustments may be useful to facilitate comparison.

\section{Analyst Adjustments Related to Inventory}
With inventory, adjustments may be required for different accounting methods. As described in previous readings, a company's decision about inventory method will affect the value of inventory shown on the balance sheet as well as the value of inventory that is sold (cost of goods sold). If a company not reporting under IFRS ${ }^{6}$ uses LIFO (last-in, first-out) and another uses FIFO (first-in, first-out), comparability of the financial results of the two companies will suffer. Companies that use the LIFO method, must also, however, disclose the value of their inventory under the FIFO method. To recast inventory values for a company using LIFO reporting on a FIFO basis, the analyst adds the ending balance of the LIFO reserve to the ending value of inventory under LIFO accounting. To adjust cost of goods sold to a FIFO basis, the analyst subtracts the change in the LIFO reserve from the reported cost of goods sold under LIFO accounting. Example 9 illustrates the use of a disclosure of the value of inventory under the FIFO method to make a more consistent comparison of the current ratios of two companies reporting in different methods.

\section{EXAMPLE 9}
\section{Adjustment for a Company Using LIFO Accounting for Inventories}
An analyst is comparing the financial performance of LP Technology Corporation (LP Tech), a hypothetical company, with the financial performance of a similar company that uses IFRS for reporting. The company reporting under IFRS uses the FIFO method of inventory accounting. Therefore, the analyst converts LP Tech's results to a comparable basis. Exhibit 4 provides balance sheet information on LP Tech.

Exhibit 4: Data for LP Technology Corporation

\begin{center}
\begin{tabular}{lcc}
\hline
 & $\mathbf{3 0}$ June &  \\
\hline
Total current assets & $\mathbf{2 0 1 8}$ & $\mathbf{2 0 1 7}$ \\
\hline
Total current liabilities & 820.2 & 749.7 \\
 & 218.1 & 198.5 \\
NOTE 6. INVENTORIES &  &  \\
Inventories consist of the following (\$ millions): &  &  \\
\hline
\end{tabular}
\end{center}

30 June

2018

2017

Raw materials

$\$ 30.7$

$\$ 29.5$

6 IAS No. 2 does not permit the use of LIFO.

\begin{center}
\begin{tabular}{|c|c|c|}
\hline
 & \multicolumn{2}{|c|}{30 June} \\
\hline
 & 2018 & 2017 \\
\hline
Work in process & 109.1 & 90.8 \\
\hline
\multirow[t]{2}{*}{Finished goods} & 63.8 & 65.1 \\
\hline
 & $\$ 203.6$ & $\$ 185.4$ \\
\hline
\end{tabular}
\end{center}

If the first-in, first-out method of inventory had been used instead of the LIFO method, inventories would have been $\$ 331.8$ and $\$ 305.8$ million higher as of June 30, 2018 and 2017, respectively.

"We value most of our inventory using the LIFO method, which could be repealed resulting in adverse effects on our cash flows and financial condition.

The cost of our inventories is primarily determined using the Last-In First-Out ("LIFO") method. Under the LIFO inventory valuation method, changes in the cost of raw materials and production activities are recognized in cost of sales in the current period even though these materials and other costs may have been incurred at significantly different values due to the length of time of our production cycle. Generally in a period of rising prices, LIFO recognizes higher costs of goods sold, which both reduces current income and assigns a lower value to the year-end inventory. Recent proposals have been initiated aimed at repealing the election to use the LIFO method for income tax purposes. According to these proposals, generally taxpayers that currently use the LIFO method would be required to revalue their LIFO inventory to its first-in, first-out ("FIFO") value. As of June 30, 2018, if the FIFO method of inventory had been used instead of the LIFO method, our inventories would have been about $\$ 332$ million higher. This increase in inventory would result in a one time increase in taxable income which would be taken into account ratably over the first taxable year and the following several taxable years. The repeal of LIFO could result in a substantial tax liability which could adversely impact our cash flows and financial condition."

\begin{enumerate}
  \item Based on the information in Exhibit 4, calculate LP Tech's current ratio under FIFO and LIFO for 2017 and 2018.
\end{enumerate}

\section{Solution to 1:}
The calculations of LP Tech's current ratio (current assets divided by current liabilities) are as follows:

\begin{center}
\begin{tabular}{|c|c|c|}
\hline
 & 2018 & 2017 \\
\hline
\multicolumn{3}{|l|}{I. Current ratio (unadjusted)} \\
\hline
Total current assets & $\$ 820.2$ & $\$ 749.7$ \\
\hline
Total current liabilities & $\$ 218.1$ & $\$ 198.5$ \\
\hline
Current ratio (unadjusted) & 3.8 & 3.8 \\
\hline
\multicolumn{3}{|l|}{II. Current ratio (adjusted)} \\
\hline
Total current assets & $\$ 820.2$ & $\$ 749.7$ \\
\hline
Adjust inventory to FIFO, add: & 331.8 & 305.8 \\
\hline
Total current assets (adjusted) & $\$ 1,152$ & $\$ 1,056$ \\
\hline
Total current liabilities & 218.1 & 198.5 \\
\hline
Current ratio (adjusted) & 5.3 & 5.3 \\
\hline
\end{tabular}
\end{center}

To adjust the LIFO inventory to FIFO, add the excess amounts of FIFO cost over LIFO cost to LIFO inventory and increase current assets by an equal amount. The effect of adjusting inventory on the current ratio is to increase the current ratio from 3.8 to 5.3 in both 2017 and 2018. LP Tech has greater liquidity according to the adjusted current ratio.

\begin{enumerate}
  \setcounter{enumi}{1}
  \item LP Tech makes the following disclosure in the risk section of its MD\&A. Assuming an effective tax rate of 35 percent, estimate the impact on LPTC's tax liability.
\end{enumerate}

\section{Solution to 2:}
Assuming an effective tax rate of 35 percent, we find the total increase in LP Tech's tax liability to be $\$ 116.1$ million $(0.35 \times \$ 331.8$ million).

\begin{enumerate}
  \setcounter{enumi}{2}
  \item LP Tech reported cash flow from operations of $\$ 115.2$ million for the year ended 30 June 2018. In comparison with the company's operating cash flow, how significant is the additional potential tax liability?
\end{enumerate}

\section{Solution to 3:}
The additional tax liability would be greater than the entire amount of the company's cash flow from operations of $\$ 115.2$ million; the additional tax liability would be apportioned, however, over several years.

In summary, the information disclosed by companies that use LIFO allows an analyst to calculate the value of the company's inventory as if the company were using the FIFO method. If the LIFO method is used for a substantial part of a company's inventory and the LIFO reserve is large relative to reported inventory, however, the adjustment to a FIFO basis can be important for comparison of the LIFO-reporting company with a company that uses the FIFO method of inventory valuation. Example 10 illustrates a case in which such an adjustment would have a major impact on an analyst's conclusions.

\section{EXAMPLE 10}
\section{Analyst Adjustment to Inventory Value for Comparability in a Current Ratio Comparison}
\begin{enumerate}
  \item Company A reports under IFRS and uses the FIFO method of inventory accounting. Company B reports under US GAAP and uses the LIFO method. Exhibit 5 gives data pertaining to current assets, LIFO reserves, and current liabilities of these companies.
\end{enumerate}

Exhibit 5: Data for Companies Accounting for Inventory on Different Bases

\begin{center}
\begin{tabular}{lcc}
\hline
 & $\begin{array}{c}\text { Company A } \\ \text { (FIFO) }\end{array}$ & $\begin{array}{c}\text { Company B } \\ \text { (LIFO) }\end{array}$ \\
\hline
Current assets (includes inventory) & $\$ 300,000$ & $\$ 80,000$ \\
LIFO reserve & $\mathrm{NA}$ & $\$ 20,000$ \\
Current liabilities & $\$ 150,000$ & $\$ 45,000$ \\
\hline
\end{tabular}
\end{center}

$N A=$ not applicable

Based on the data given in Exhibit 5, compare the liquidity of the two companies as measured by the current ratio.

\section{Solution:}
Company A's current ratio is 2.0. Based on unadjusted balance sheet data, Company B's current ratio is 1.78 . Company A's higher current ratio indicates that Company A appears to be more liquid than Company B; however, the use of unadjusted data for Company B is not appropriate for making comparisons with Company A.

After adjusting Company B's inventory to a comparable basis (i.e., to a FIFO basis), the conclusion changes. The following table summarizes the results when Company B's inventory is left on a LIFO basis and when it is placed on a FIFO basis for comparability with Company A.

\begin{center}
\begin{tabular}{|c|c|c|c|}
\hline
 & \multirow[b]{2}{*}{$\begin{array}{c}\text { Company A } \\
\text { (FIFO) }\end{array}$} & \multicolumn{2}{|c|}{Company B} \\
\hline
 &  & $\begin{array}{l}\text { Unadjusted } \\ \text { (LIFO basis) }\end{array}$ & $\begin{array}{c}\text { Adjusted } \\ \text { (FIFO basis) }\end{array}$ \\
\hline
$\begin{array}{l}\text { Current assets (includes } \\ \text { inventory) }\end{array}$ & $\$ 300,000$ & $\$ 80,000$ & $\$ 100,000$ \\
\hline
Current liabilities & $\$ 150,000$ & $\$ 45,000$ & $\$ 45,000$ \\
\hline
Current ratio & 2.00 & 1.78 & 2.22 \\
\hline
\end{tabular}
\end{center}

When both companies' inventories are stated on a FIFO basis, Company B appears to be the more liquid, as indicated by its current ratio of 2.22 versus Company A's ratio of 2.00 .

The adjustment to place Company B's inventory on a FIFO basis was significant because Company B was assumed to use LIFO for its entire inventory and its inventory reserve was $\$ 20,000 / \$ 80,000=0.25$, or 25 percent of its reported inventory.

As mentioned earlier, an analyst can also adjust the cost of goods sold for a company using LIFO to a FIFO basis by subtracting the change in the amount of the LIFO reserve from cost of goods sold. Such an adjustment would be appropriate for making profitability comparisons with a company reporting on a FIFO basis and is important to make when the impact of the adjustment would be material.

\section{ADJUSTMENTS RELATED TO PROPERTY, PLANT, AND EQUIPMENT}
explain appropriate analyst adjustments to a company's financial statements to facilitate comparison with another company

Management generally has considerable discretion in determination of depreciation expense. Depreciation expense affects the values of reported net income and reported net fixed assets. Analysts often consider management's choices related to depreciation as a qualitative factor in evaluating the quality of a company's financial reporting, and in some cases, analysts may adjust reported depreciation expense for a specific analytical purpose.

The amount of depreciation expense depends on both the accounting method and the estimates used in the calculations. Companies can use the straight-line method, an accelerated method, or a usage method to depreciate fixed assets (other than land). The straight-line method reports an equal amount of depreciation expense each period, and the expense is computed as the depreciable cost divided by the estimated useful life of the asset (when acquired, an asset's depreciable cost is calculated as its total cost minus its estimated salvage value). Accelerated methods depreciate the asset more quickly; they apportion a greater amount of the depreciable cost to depreciation expense in the earlier periods. Usage-based methods depreciate an asset in proportion to its usage. In addition to selecting a depreciation method, companies must estimate an asset's salvage value and useful life to compute depreciation.

Disclosures required for depreciation often do not facilitate specific adjustments, so comparisons of companies concerning their decisions in depreciating assets are often qualitative and general. The accounts that are associated with depreciation include the balance sheet accounts for gross property, plant, and equipment (PPE) and accumulated depreciation; the income statement amount for depreciation expense; and the statement of cash flows disclosure of capital expenditure (capex) and asset disposals. The relationships among these items can reveal various pieces of information. Note, however, that PPE typically includes a mix of assets with different depreciable lives and salvage values, so the items in the following list reflect general relationships in the total pool of assets.

\begin{itemize}
  \item Accumulated depreciation divided by gross PPE, from the balance sheet, suggests how much of the useful life of the company's overall asset base has passed.

  \item Accumulated depreciation divided by depreciation expense suggests how many years' worth of depreciation expense have already been recognized (i.e., the average age of the asset base).

  \item Net PPE (net of accumulated depreciation) divided by depreciation expense is an approximate indicator of how many years of useful life remain for the company's overall asset base.

  \item Gross PPE divided by depreciation expense suggests the average life of the assets at installation.

  \item Capex divided by the sum of gross PPE plus capex can suggest what percentage of the asset base is being renewed through new capital investment.

  \item Capex in relation to asset disposal provides information on growth of the asset base.

\end{itemize}

As Example 11 shows, these relationships can be evaluated for companies in an industry to suggest differences in their strategies for asset utilization or areas for further investigation.

\section{EXAMPLE 11}
\section{Differences in Depreciation}
An analyst is evaluating the financial statements of two companies in the same industry. The companies have similar strategies with respect to the use of equipment in manufacturing their products. The following information is provided (amounts in millions):

\begin{center}
\begin{tabular}{lcc}
\hline
 & Company A & Company B \\
\hline
Net PPE & $\$ 1,200$ & $\$ 750$ \\
Depreciation expense & $\$ 120$ & $\$ 50$ \\
\hline
\end{tabular}
\end{center}

\begin{enumerate}
  \item Based on the information given, estimate the average remaining useful lives of the asset bases of Company A and Company B.
\end{enumerate}

\section{Solution to 1:}
The estimated average remaining useful life of Company A's asset base is 10 years (calculated as net PPE divided by depreciation expense, or $\$ 1,200 / \$ 120$ $=10$ years). For Company B, the average remaining useful life of the asset base appears to be far longer, 15 years $(\$ 750 / \$ 50)$.

\begin{enumerate}
  \setcounter{enumi}{1}
  \item Suppose that, based on a physical inspection of the companies' plants and other industry information, the analyst believes that the actual remaining useful lives of Company A's and Company B's assets are roughly equal at 10 years. Based only on the facts given, what might the analyst conclude about Company B's reported net income?
\end{enumerate}

\section{Solution to 2:}
If 10 years were used to calculate Company B's depreciation expense, the expense would be $\$ 75$ million (i.e., $\$ 25$ million higher than reported) and higher depreciation expense would decrease net income. The analyst might conclude that Company B's reported net income reflects relatively more aggressive accounting estimates than estimates reflected in Company A's reported net income.

\section{ADJUSTMENTS RELATED TO GOODWILL}
$$
\square \quad \begin{aligned}
& \text { explain appropriate analyst adjustments to a company's financial } \\
& \text { statements to facilitate comparison with another company }
\end{aligned}
$$

Goodwill arises when one company purchases another for a price that exceeds the fair value of the net identifiable assets acquired. Net identifiable assets include current assets, fixed assets, and certain intangible assets that have value and meet recognition criteria under accounting standards. A broad range of intangible assets might require valuation in the context of a business combination-for example, brands, technology, and customer lists. Goodwill is recorded as an asset and essentially represents the difference between the purchase price and the net identifiable assets. For example, assume ParentCo purchases TargetCo for a purchase price of $\$ 400$ million and the fair value of TargetCo's identifiable assets is $\$ 300$ million (which includes the fair values of current assets, fixed assets, and a recognized brand). ParentCo will record total assets of $\$ 400$ million consisting of $\$ 300$ million in identifiable assets (including the fair value of the brand) and $\$ 100$ million of goodwill. The goodwill is tested annually for impairment and if the value of the goodwill is determined to be impaired, ParentCo will then reduce the amount of the asset and report a write-off resulting from impairment. One of the conceptual difficulties with goodwill arises in comparative financial statement analysis. Consider, for example, two hypothetical US companies, one of which has grown by making an acquisition and the other of which has grown internally. Assume that the economic value of the two companies is identical: Each has an identically valuable branded product, well-trained workforce, and proprietary technology. The company that has grown by acquisition will have recorded the transaction to acquire the target company and its underlying net assets on the basis of the total consideration paid for the acquisition. The company that has grown internally will have done so by incurring expenditures for advertising, staff training, and research, all of which are expensed as incurred under US GAAP. Given the immediate expensing, the value of the internally generated assets is not capitalized onto the balance sheet and is thus not directly reflected on the company's balance sheet (revenues, income, and cash flows should reflect the benefits derived from the investment in the intangible assets). Ratios based on asset values and/or income, including profitability ratios (such as ROA) and market value to book value (MV/BV), ${ }^{7}$ will generally differ for the two companies because of differences in the accounting values of assets and income related to acquired intangibles and goodwill, although, by assumption, the economic value of the companies is identical.

\section{EXAMPLE 12}
\section{Ratio Comparisons for Goodwill}
Miano Marseglia is an analyst who is evaluating the relative valuation of two securities brokerage companies: TD Ameritrade Holding Corporation (AMTD) and the Charles Schwab Corporation (SCHW). As one part of an overall analysis, Marseglia would like to see how the two companies compare with each other and with the industry based on market value to book value. Because both companies are large players in the industry, Marseglia expects them to sell at a higher MV/BV than the financial services sector median of 2.2. He collects the following data on the two companies.

\begin{center}
\begin{tabular}{|c|c|c|}
\hline
 & SCHW & AMTD \\
\hline
$\begin{array}{l}\text { Market capitalization on } 30 \text { August } 2018 \text { (mar- } \\ \text { ket price per share times the number of shares } \\ \text { outstanding) }\end{array}$ & $\$ 68,620$ & $\$ 33,247$ \\
\hline
$\begin{array}{l}\text { Total shareholders' equity (as of } 30 \text { June } 2018 \text { for } \\ \text { both companies) }\end{array}$ & $\$ 20,097$ & $\$ 7,936$ \\
\hline
Goodwill & $\$ 1,227$ & $\$ 4,198$ \\
\hline
Other intangible assets & $\$ 93$ & $\$ 1,363$ \\
\hline
\end{tabular}
\end{center}

Marseglia computes the MV/BV for the companies as follows:

SCHW $\$ 68,620 / \$ 20,097=3.4$

AMTD $\$ 33,247 / \$ 7,936=4.2$

As expected, each company appears to be selling at a premium to the sector median MV/BV of 2.2. The companies have similar MV/BVs (i.e., they are somewhat equally valued relative to the book value of shareholders' equity). Marseglia is concerned, however, because he notes that AMTD has significant

$7 \mathrm{MV} / \mathrm{BV}$ equals the total market value of the stock (the market capitalization) divided by total stockholders' equity. It is also referred to as the price-to-book ratio because it can also be calculated as price per share divided by stockholders' equity per share. amounts of goodwill and acquired intangible assets. He wonders what the relative value would be if the MV/BV were computed after adjusting book value, first, to remove goodwill and, second, to remove all intangible assets. Book value reduced by all intangible assets (including goodwill) is known as "tangible book value."

\begin{enumerate}
  \item Compute the MV/BV adjusted for goodwill and the price/tangible book value for each company.
\end{enumerate}

\section{Solution to 1:}
\begin{center}
\begin{tabular}{|c|c|c|}
\hline
 & \multicolumn{2}{|c|}{(\$ millions)} \\
\hline
 & SCHW & AMTD \\
\hline
Total stockholders' equity & $\$ 20,097$ & $\$ 7,93$ \\
\hline
Less: Goodwill & $\$ 1,227$ & $\$ 4,19$ \\
\hline
Book value, adjusted & $\$ 18,870$ & $\$ 3,73$ \\
\hline
Adiusted MV/BV & 3.6 & 8.9 \\
\hline
\end{tabular}
\end{center}

\begin{center}
\begin{tabular}{|c|c|c|}
\hline
 & \multicolumn{2}{|c|}{(\$ millions)} \\
\hline
 & SCHW & AMTD \\
\hline
Total stockholders' equity & $\$ 20,097$ & $\$ 7,936$ \\
\hline
Less: Goodwill & $\$ 1,227$ & $\$ 4,198$ \\
\hline
Less: Other intangible assets & $\$ 93$ & $\$ 1,363$ \\
\hline
Tangible book value & $\$ 18,777$ & $\$ 2,375$ \\
\hline
MV/tangible book value & 3.7 & 14.0 \\
\hline
\end{tabular}
\end{center}

\begin{enumerate}
  \setcounter{enumi}{1}
  \item Which company appears to be a better value based solely on this data? (Note that the MV/BV is only one part of a broader analysis. Much more evidence related to the valuations and the comparability of the companies would be required to reach a conclusion about whether one company is a better value.)
\end{enumerate}

\section{Solution to 2:}
After adjusting for goodwill, SCHW appears to be selling for a much lower price relative to book value than does AMTD (3.6 versus 8.9) after adjusting for goodwill. The difference is more extreme after adjusting for other intangibles.

\section{SUMMARY}
This reading described selected applications of financial statement analysis, including the evaluation of past financial performance, the projection of future financial performance, the assessment of credit risk, and the screening of potential equity investments. In addition, the reading introduced analyst adjustments to reported financials. In all cases, the analyst needs to have a good understanding of the financial reporting standards under which the financial statements were prepared. Because standards evolve over time, analysts must stay current in order to make good investment decisions.

The main points in the reading are as follows:

\begin{itemize}
  \item Evaluating a company's historical performance addresses not only what happened but also the causes behind the company's performance and how the performance reflects the company's strategy.

  \item The projection of a company's future net income and cash flow often begins with a top-down sales forecast in which the analyst forecasts industry sales and the company's market share. By projecting profit margins or expenses and the level of investment in working and fixed capital needed to support projected sales, the analyst can forecast net income and cash flow.

  \item Projections of future performance are needed for discounted cash flow valuation of equity and are often needed in credit analysis to assess a borrower's ability to repay interest and principal of a debt obligation.

  \item Credit analysis uses financial statement analysis to evaluate credit-relevant factors, including tolerance for leverage, operational stability, and margin stability.

  \item When ratios constructed from financial statement data and market data are used to screen for potential equity investments, fundamental decisions include which metrics to use as screens, how many metrics to include, what values of those metrics to use as cutoff points, and what weighting to give each metric.

  \item Analyst adjustments to a company's reported financial statements are sometimes necessary (e.g., when comparing companies that use different accounting methods or assumptions). Adjustments can include those related to investments; inventory; property, plant, and equipment; and goodwill.

\end{itemize}

\section{REFERENCES}
Benninga, Simon Z., Oded H. Sarig. 1997. Corporate Finance: A Valuation Approach. New York: McGraw-Hill Publishing.

Conrad, J., M. Cooper, G. Kaul. 2003. "Value versus Glamour." Journal of Finance, vol. 58, no. 5:1969-1996. 10.1111/1540-6261.00594

Haugen, R.A., N.L. Baker. 1996. "Commonality in the Determinants of Expected Stock Returns." Journal of Financial Economics, vol. 41, no. 3:401-439. 10.1016/0304-405X(95)00868-F

\section{PRACTICE PROBLEMS}
\begin{enumerate}
  \item Projecting profit margins into the future on the basis of past results would be most reliable when the company:
\end{enumerate}

A. is in the commodities business.

B. operates in a single business segment.

C. is a large, diversified company operating in mature industries.

\begin{enumerate}
  \setcounter{enumi}{1}
  \item Galambos Corporation had an average receivables collection period of 19 days in 2003. Galambos has stated that it wants to decrease its collection period in 2004 to match the industry average of 15 days. Credit sales in 2003 were $\$ 300$ million, and analysts expect credit sales to increase to $\$ 400$ million in 2004 . To achieve the company's goal of decreasing the collection period, the change in the average accounts receivable balance from 2003 to 2004 that must occur is closest to:
A. $-\$ 420,000$
B. $\$ 420,000$.
C. $\$ 836,000$.

  \item Credit analysts are likely to consider which of the following in making a rating recommendation?

\end{enumerate}

A. Business risk but not financial risk

B. Financial risk but not business risk

C. Both business risk and financial risk

\begin{enumerate}
  \setcounter{enumi}{3}
  \item When screening for potential equity investments based on return on equity, to control risk, an analyst would be most likely to include a criterion that requires:
\end{enumerate}

A. positive net income.

B. negative net income.

C. negative shareholders' equity.

\begin{enumerate}
  \setcounter{enumi}{4}
  \item One concern when screening for stocks with low price-to-earnings ratios is that companies with low P/Es may be financially weak. What criterion might an analyst include to avoid inadvertently selecting weak companies?
\end{enumerate}

A. Net income less than zero

B. Debt-to-total assets ratio below a certain cutoff point

C. Current-year sales growth lower than prior-year sales growth

\begin{enumerate}
  \setcounter{enumi}{5}
  \item When a database eliminates companies that cease to exist because of a merger or bankruptcy, this can result in:
\end{enumerate}

A. look-ahead bias.

B. back-testing bias. C. survivorship bias.

\begin{enumerate}
  \setcounter{enumi}{6}
  \item In a comprehensive financial analysis, financial statements should be:
\end{enumerate}

A. used as reported without adjustment.

B. adjusted after completing ratio analysis.

C. adjusted for differences in accounting standards, such as international financial reporting standards and US generally accepted accounting principles.

\begin{enumerate}
  \setcounter{enumi}{7}
  \item When comparing a US company that uses the last in, first out (LIFO) method of inventory with companies that prepare their financial statements under international financial reporting standards (IFRS), analysts should be aware that according to IFRS, the LIFO method of inventory:
\end{enumerate}

A. is never acceptable.

B. is always acceptable.

C. is acceptable when applied to finished goods inventory only.

\begin{enumerate}
  \setcounter{enumi}{8}
  \item An analyst is evaluating the balance sheet of a US company that uses last in, first out (LIFO) accounting for inventory. The analyst collects the following data:
\end{enumerate}

\begin{center}
\begin{tabular}{lcc}
\hline
 & 31 Dec 05 & 31 Dec 06 \\
\hline
Inventory reported on balance sheet & $\$ 500,000$ & $\$ 600,000$ \\
LIFO reserve & $\$ 50,000$ & $\$ 70,000$ \\
Average tax rate & $30 \%$ & $30 \%$ \\
\hline
\end{tabular}
\end{center}

After adjusting the amounts to convert to the first in, first out (FIFO) method, inventory at 31 December 2006 would be closest to:
A. $\$ 600,000$.
B. $\$ 620,000$.
C. $\$ 670,000$.

\begin{enumerate}
  \setcounter{enumi}{9}
  \item An analyst gathered the following data for a company (\$ millions):
\end{enumerate}

\begin{center}
\begin{tabular}{lcc}
\hline
 & $\mathbf{3 1}$ Dec 2000 & $\mathbf{3 1}$ Dec 2001 \\
\hline
Gross investment in fixed assets & $\$ 2.8$ & $\$ 2.8$ \\
Accumulated depreciation & $\$ 1.2$ & $\$ 1.6$ \\
\hline
\end{tabular}
\end{center}

The average age and average depreciable life of the company's fixed assets at the end of 2001 are closest to:

\begin{center}
\begin{tabular}{lcc}
\hline
 & Average Age & Average Depreciable Life \\
\hline
A & 1.75 years & 7 years \\
B & 1.75 years & 14 years \\
C & 4.00 years & 7 years \\
\end{tabular}
\end{center}

\begin{enumerate}
  \setcounter{enumi}{10}
  \item To compute tangible book value, an analyst would:
A. add goodwill to stockholders' equity.
B. add all intangible assets to stockholders' equity.
C. subtract all intangible assets from stockholders' equity.
\end{enumerate}

\section{SOLUTIONS}
\begin{enumerate}
  \item C is correct. For a large, diversified company, margin changes in different business segments may offset each other. Furthermore, margins are most likely to be stable in mature industries.

  \item C is correct. Accounts receivable turnover is equal to $365 / 19$ (collection period in days) $=19.2$ for 2003 and needs to equal 365/15 $=24.3$ in 2004 for Galambos to meet its goal. Sales/turnover equals the accounts receivable balance. For 2003, $\$ 300,000,000 / 19.2=\$ 15,625,000$, and for 2004, $\$ 400,000,000 / 24.3=\$ 16,460,905$. The difference of $\$ 835,905$ is the increase in receivables needed for Galambos to achieve its goal.

  \item $\mathrm{C}$ is correct. Credit analysts consider both business risk and financial risk.

  \item A is correct. Requiring that net income be positive would eliminate companies that report a positive return on equity only because both net income and shareholders' equity are negative.

  \item B is correct. A lower value of debt/total assets indicates greater financial strength. Requiring that a company's debt/total assets be below a certain cutoff point would allow the analyst to screen out highly leveraged and, therefore, potentially financially weak companies.

  \item C is correct. Survivorship bias exists when companies that merge or go bankrupt are dropped from the database and only surviving companies remain. Look-ahead bias involves using updated financial information in back-testing that would not have been available at the time the decision was made. Back-testing involves testing models in prior periods and is not, itself, a bias.

  \item $\mathrm{C}$ is correct. Financial statements should be adjusted for differences in accounting standards (as well as accounting and operating choices). These adjustments should be made prior to common-size and ratio analysis.

  \item A is correct. LIFO is not permitted under IFRS.

  \item $\mathrm{C}$ is correct. To convert LIFO inventory to FIFO inventory, the entire LIFO reserve must be added back: $\$ 600,000+\$ 70,000=\$ 670,000$.

  \item $\mathrm{C}$ is correct. The company made no additions to or deletions from the fixed asset account during the year, so depreciation expense is equal to the difference in accumulated depreciation at the beginning of the year and the end of the year, or $\$ 0.4$ million. Average age is equal to accumulated depreciation/depreciation expense, or $\$ 1.6 / \$ 0.4=4$ years. Average depreciable life is equal to ending gross investment $/$ depreciation expense $=\$ 2.8 / \$ 0.4=7$ years.

  \item C is correct. Tangible book value removes all intangible assets, including goodwill, from the balance sheet.

\end{enumerate}

\section{Corporate Issuers}
\section*{LEARNING MODULE 
 1 }
\section{Corporate Structures and Ownership}
Vahan Janjigian, PhD, CFA, is at Greenwich Wealth Management, LLC (Greenwich, CT, USA).

\section{LEARNING OUTCOME}
\begin{center}
\begin{tabular}{|c|c|}
\hline
Mastery & The candidate should be able to: \\
\hline
$\square$ & $\begin{array}{l}\text { compare business structures and describe key features of corporate } \\ \text { issuers }\end{array}$ \\
\hline
ז & compare public and private companies \\
\hline
Г & compare the financial claims and motivations of lenders and owners \\
\hline
\end{tabular}
\end{center}

INTRODUCTION

compare business structures and describe key features of corporate issuers

In 1997, Martin Eberhard and Marc Tarpenning, an engineer and a computer scientist, started a company called NuvoMedia to make an electronic book reader they called the Rocket eBook, a precursor to the Kindle eBook popularized by Amazon. Three years after it was founded, NuvoMedia was sold for USD187 million.

Soon after, the two entrepreneurs decided to form a new company, this one focused on making electric cars. They named this company in honor of the inventor Nikola Tesla. Because this was a high-risk, capital-intensive endeavor, they used only some of their newfound wealth and sought other investors with expertise in electric vehicles and fundraising capabilities. Elon Musk, an entrepreneur with a shared vision in the commercialization of electric sports cars, joined the team.

In addition to making an initial investment of USD6.3 million in Tesla, Musk also helped raise more money from other venture capitalists. Due to conflicts that were not disclosed, Eberhard and Tarpenning resigned just before Tesla came out with its first vehicle, the Roadster, in 2008. Musk took over as CEO and led Tesla's initial public offering in 2010, which raised USD226 million. In many ways, Tesla's story is typical of how businesses begin and succeed. They are often started by founders with significant knowledge or technical expertise but who may lack the skills required to manage a business as it grows larger. Capital is needed to fund growth and is initially raised through private channels. Private investors often get involved in the management of the company, especially if they have a large investment at stake. Eventually, even larger amounts of capital are required, and the company is acquired or taken public.

Here we examine different forms of business structures, focusing on corporations and the securities they issue to capital providers.

\section{BUSINESS STRUCTURES}
compare business structures and describe key features of corporate issuers

While the focus here is on corporations, it is important to recognize that other business structures exist and to understand how they compare with one another. Our focus here is on four areas:

\begin{enumerate}
  \item Legal Relationship - the legal relationship between the owner(s) and the business.

  \item Owner-Operator Relationship - the relationship between the owner(s) of the business and those who operate the business.

  \item Business Liability - the extent to which individuals have liability for actions undertaken by the business or its business debts. Liability can be unlimited or limited in nature.

  \item Taxation - the treatment of profits or losses generated by the business for tax purposes.

\end{enumerate}

While there are numerous forms of business structures, some with variations, we discuss only the more common forms shown in Exhibit 1.

\section{Exhibit 1: Common Business Structures}
\begin{itemize}
  \item Sole Proprietorship

  \item General Partnership

  \item Limited Partnership

  \item Corporation

\end{itemize}

\section{Sole Proprietorship (Sole Trader)}
The simplest business structure is the sole proprietorship, also called the sole trader, shown in Exhibit 2. In a sole proprietorship, the owner personally funds the capital needed to operate the business and retains full control over the operations of the business while participating fully in the financial returns and risks of the business. Exhibit 2: Sole Proprietorship

\begin{center}
\includegraphics[max width=\textwidth]{2023_05_04_b5cfa4f1bc883752f121g-621}
\end{center}

An example of a sole proprietorship is a family-owned store. To start the business and run daily operations, the owner would likely use personal savings, credit card debt, and loans from banks or other family and friends. The owner retains full control over the business, including how it will operate and what products to sell at what price. If the business does well, the owner retains all return (profits), which are taxed as personal income. At the same time, the owner has unlimited liability and retains all risk associated with the business, meaning she can be held financially responsible for all debt the business owes.

While sole proprietorships are preferred for small-scale businesses given their simplicity and flexibility, the business is constrained by the owner's ability to access capital and assume risk.

In summary, key features of sole proprietorships include the following:

\begin{itemize}
  \item No legal identity; considered extension of owner

  \item Owner-operated business

  \item Owner retains all return and assumes all risk

  \item Profits from business taxed as personal income

  \item Operational simplicity and flexibility

  \item Financed informally through personal means

  \item Business growth is limited by owner's ability to finance and personal risk appetite

\end{itemize}

What if more resources are needed than can be provided by an individual owner?

\section{General Partnership}
A general partnership, shown in Exhibit 3, has two or more owners called partners whose roles and responsibilities in the business are outlined in a partnership agreement. General partnerships are like sole proprietorships with the important distinction that they allow for additional resources to be brought into the business along with the sharing of business risk among a larger group of individuals.

\section{Exhibit 3: General Partnership}
\begin{center}
\includegraphics[max width=\textwidth]{2023_05_04_b5cfa4f1bc883752f121g-622}
\end{center}

Examples of general partnerships are professional services businesses (e.g., law, accounting, medicine) and small financial or financial advisory firms. Such businesses have a small number of partners who establish the business by contributing equal amounts of capital. The partners bring complementary expertise, such as business development, financial acumen, operations, or legal/compliance, and share responsibility in running the business. All profits, losses, and risks of the business are collectively assumed and shared by the partners. If one partner is unable to pay their share of the business's debts, the remaining partners are fully liable. Like a sole proprietorship, potential for growth is limited by the partners' ability to source capital and expertise and their collective risk tolerance.

In summary, key features of general partnerships include the following:

\begin{itemize}
  \item No legal identity; partnership agreement sets ownership

  \item Partner-operated business

  \item Partners share all risk and business liability

  \item Partners share all return, with profits taxed as personal income

  \item Contributions of capital and expertise by partners

  \item Business growth is limited by partner resourcing capabilities and risk appetite

\end{itemize}

\section{Limited Partnership}
Exhibit 4 shows a special type of partnership called the limited partnership. A limited partnership must have at least one general partner with unlimited liability who is responsible for the management of the business. Remaining partners, called limited partners, have limited liability, meaning they can lose only up to the amount of their investment in the limited partnership. With limited liability, personal assets are considered separate to, and thus protected from, the liabilities of the business. All partners are entitled to a share of the profits, with general partners typically getting a larger portion given their management responsibility for the business.

An example of a limited partnership is a private equity fund, which operates with a general partner (GP) who assumes responsibility for business operations and liabilities. Remaining partners are called limited partners (LPs), who have limited liability to business risk but may provide capital or expertise. Limited partners have no control over the operation of the business and no way to replace the GP in the event the GP runs the business poorly or fails to act in the interest of the LPs.

\section{Exhibit 4: Limited Partnership}
\begin{center}
\includegraphics[max width=\textwidth]{2023_05_04_b5cfa4f1bc883752f121g-623}
\end{center}

Other examples of limited partnerships include real estate and professional services businesses (e.g., law, accounting, medicine), small financial firms, and hedge funds.

In summary, key features of limited partnerships include the following:

\begin{itemize}
  \item No legal identity; partnership agreement sets ownership

  \item GP operates the business, having unlimited liability

  \item LPs have limited liability but lack control over business operations

  \item All partners share in return, with profits taxed as personal income

  \item Contributions of capital and expertise by partners

  \item Business growth is limited by GP/LP financing capabilities and risk appetite and GP competence and integrity in running the business

\end{itemize}

In a limited partnership, while financial risk and reward are shared, such resources as capital and expertise are limited to what the partners can personally contribute. Limited partners ultimately grant control to the GP, which entails risk.

What if a business requires greater resources than can be provided by either an individual or small group of individuals?

\section{Corporation (Limited Companies)}
An evolved model of the limited partnership is the corporation, known as a limited liability company (LLC) in many countries or as a limited company in others, such as the United Kingdom. In the United States, while similar, an LLC and a corporation are not the same. The main difference being with a US LLC, taxation occurs at the personal level, whereas with a US corporation, taxation takes place at both the personal level (distributions to owners) and corporate level (profits).

Like a limited partnership, owners in a corporation (and US LLC) have limited liability; however, corporations have greater access to the capital and expertise required to fuel growth. As a result, the corporation is the preferred form for larger companies and the dominant business structure globally by revenues and asset values.

Examples of corporations are national or multinational conglomerates, global asset managers, and regional stock exchanges. As shown in Exhibit 5, the three main types of corporations are public for-profit, private for-profit, and nonprofit.

\section{Exhibit 5: Types of Corporations}
\begin{center}
\includegraphics[max width=\textwidth]{2023_05_04_b5cfa4f1bc883752f121g-624}
\end{center}

\section{Nonprofits}
While corporations are typically created to return profits, some corporations are formed with the specific purpose of promoting a public benefit, religious benefit, or charitable mission. These are known as nonprofit corporations (nonprofits) or nonprofit organizations and include private foundations. Like for-profit corporations, nonprofits have a board of directors and can have paid employees.

Unlike for-profit corporations, nonprofits do not have shareholders and do not distribute dividends. Nonprofits typically are exempt from paying taxes. If they are run well, nonprofits can generate profits; however, all profits must be reinvested in promoting the mission of the organization. Well-known nonprofits include Harvard University, Ascension (a large nonprofit private health care provider in the United States), and the Asian Development Bank (regional development bank based in the Philippines).

\section{For-Profits}
Most corporations are "for profit," or motivated to make money (profits) for the owners of the business. While corporations can engage in any kind of legal business, they are usually created with a profit motive and specific purpose at inception.

Prior Example: In the introduction, the two entrepreneurs formed Tesla with the belief they had the expertise to design and produce an electric sports car at a price that high-end consumers would pay. Over time, Tesla expanded its manufacturing to other electric models, including sedans and SUVs, and later entered the solar power market with a product line of solar panels and batteries for homes and businesses. As is often the case with companies, over time Tesla pursued growth unrelated to its original product and purpose - that of manufacturing a viable electric sports car for a targeted customer segment.

\section{For-Profits: Public vs. Private}
For-profit corporations can be public or private. Primary distinctions are the number of shareholders that exist and whether the company has a stock exchange listing. In some countries, such as the United Kingdom and Australia, if there are a large number of shareholders (usually greater than 50), the company is categorized as a public company and subject to more onerous regulatory requirements whether or not it is listed on a stock exchange. In numerous other countries like the United States, the distinction of being public is defined by a stock exchange listing. Terms and names for business entities vary at the local country level, such as the use of Plc, AG, or SA extensions for public limited liability companies, or GmbH, Pte/Pty Ltd, or SARL for private limited liability companies.

For remaining coverage, we focus on for-profit corporations given their overriding importance within the investor community.

\section{Legal Identity}
A corporation is formed through the filing of articles of incorporation with a regulatory authority. A corporation is therefore considered a legal entity separate and distinct from its owners. As far as the law is concerned, a corporation has many of the rights and responsibilities of an individual and can engage in many of the same activities. For example, a corporation can enter into contracts, hire employees, sue and be sued, borrow and lend money, make investments, and pay taxes.

Large corporations frequently have business operations in many different geographic regions and are subject to regulatory jurisdictions where either:

\begin{itemize}
  \item the company is incorporated,

  \item business is conducted, or

  \item the company's securities are listed

\end{itemize}

for such activities as:

\begin{itemize}
  \item registration (for public companies),

  \item financial and non-financial reporting and disclosure, or

  \item capital market activities (e.g., security issuance, trading, investment).

\end{itemize}

\section{Owner-Operator Separation}
A key feature of most corporations is the separation between those who own the business, the owners, and those who operate it, as represented by the board of directors and company management. In a corporation, owners are largely removed from the day-to-day operations of the business. This owner-operator separation of capital and business capabilities enables owners to create businesses by leveraging greater resources to run the business while allowing for shared business risk and return.

In a corporation, owners elect a board of directors to oversee business operations. The directors hire the CEO and senior leaders responsible for management and day-to-day operations of the company. Directors and officers have a responsibility to act in the best interest of the owners. The separation of operating control from ownership enables the corporation to finance itself from a larger universe of potential investors who are not required to have expertise in operating the business.

Should the board or management not conduct business in line with owner interests, owners have the ability to enact change through voting rights attached to their shares. In this way, owners can ensure those operating the business are aligned with owners' interests to maximize their return on investment in the company. The ability to influence or change operational control through the use of their voting rights as corporate owners is a key difference from the limited partnership model. Note, however, that shares can differ in their voting rights and not all shares have voting rights.

At the same time, corporations are also expected to consider the interests of other stakeholders, including employees, creditors, customers, suppliers, regulators, and members of the communities in which they operate and conduct business. While the appointed board of directors and company officers are obligated to act in the best interests of shareholders, conflicts of interest do occur when management acts to place their interests, or the interests of other stakeholders, above those of the owners. To prevent conflicts and mismanagement of the business, corporate governance policies and practices are in place to oversee business operations and ensure sound management practices.

\section{Business Liability}
In a corporation, risk is shared across all owners and owners have limited liability. The maximum amount owners can lose is what they invested in the company. Owners also share in the returns of the company through their equity claim as represented by their shares. No contractual obligation exists for the company to repay ownership capital. Instead, owners have a residual claim to the company's net assets after its liabilities have been paid. Exhibit 6 shows the relationship between owners and the corporation.

\section{Exhibit 6: Corporation}
\begin{center}
\includegraphics[max width=\textwidth]{2023_05_04_b5cfa4f1bc883752f121g-626}
\end{center}

\section{Capital Financing}
The separation between ownership and management allows corporations to access capital more easily than other business structures because capital is the only requirement for owners to join the business. While more expensive to form and operate than other business structures, the corporate structure is typically preferred when capital requirements for a business overwhelm what could be raised by an individual or limited number of individuals.

Corporations are able to raise the financing they need from capital providers, who include those individuals and entities willing to provide capital to the company in return for the corporation's issued securities, which may be equity securities (stocks) or debt securities (bonds). Other capital providers are financial institutions, such as banks who lend corporations capital in the form of loans.

Types of capital providers include the following:

\begin{itemize}
  \item Individuals

  \item Institutions

  \item Corporations

  \item Family offices

  \item Government

\end{itemize}

Corporations can raise two types of capital: ownership capital (equity), and borrowed capital (debt). Shareholders have exchanged their capital for issued equity securities, while bondholders have exchanged their capital for issued debt securities. Both are investors in the corporation's securities. Ownership capital, or equity, refers to money invested by the owners of the corporation. In return for capital provided, the company grants the equity investor ownership in the company. Owners are also called shareholders (or stockholders) because they are issued shares, with each share representing an ownership interest. The more shares an investor owns, the greater their ownership stake in the corporation.

Exhibit 7 highlights the exchange of capital and issued security by type between the corporation and investor.

Exhibit 7: Corporate Financing - Ownership Capital (Equity) vs. Borrowed Capital (Debt)

\begin{center}
\includegraphics[max width=\textwidth]{2023_05_04_b5cfa4f1bc883752f121g-627}
\end{center}

\section{Taxation}
While taxation for corporations can differ greatly from country to country, the corporation is ultimately subject to the tax authority and tax code governing the issuer's tax reporting, payment, and status. In most countries, corporations are taxed directly on their profits. In many countries, shareholders pay an additional tax on distributions (dividends) that are passed on to them. Economists refer to this as the double taxation of corporate profits. In some countries, shareholders do not pay a personal tax on dividends if the corporation has paid tax previously on the earnings distributed to shareholders; or, shareholders receive a personal tax credit for their proportional share of taxes paid by the corporation. Still in other countries, corporations pay no tax at all or may face different tax regimes within one country.

\section{KNOWLEDGE CHECK}
\section{Double Taxation of Corporate Profits}
\begin{enumerate}
  \item The French company Elo (previously known as Auchan Holding) generated operating income of $€ 838$ million and paid corporate taxes of $€ 264$ million. Investors in France also pay a $30 \%$ tax on dividends received. If Elo had distributed all of its after-tax income to investors as a dividend, what would have been the effective tax rate on each euro of operating earnings?
\end{enumerate}

\section{Solution:}
$\begin{array}{ll}\text { Corporate Taxes (31.5\%) } & € 264 \\ \text { After-Tax Income } & (€ 838-€ 264) \\ & =€ 574 \\ \text { Distributed Dividend } & € 574 \\ \text { Investor Dividend Tax (30\%) } & € 574 \times 0.3 \\ & =€ 172.2 \\ & (€ 264+€ 172.2) / € 838 \\ \text { Effective Tax Rate } & =52.1 \%\end{array}$

If the remaining after-tax income of $€ 574$ million was paid to investors as a dividend, investors would pay $€ 172.2$ million in taxes on the dividends received. Total taxes paid would be $€ 436.2$ million ( $€ 264$ million at the corporate level plus $€ 172.2$ million at the personal level), resulting in an effective tax rate of $52.1 \%$.

In many countries, a tax disadvantage is associated with the corporate business structure because shareholders must pay a tax on distributions that have already been taxed at the corporate level. Despite this disadvantage, the corporate business structure remains attractive because corporations have the potential capability to raise large amounts of capital from a disparate group of investors.

While shareholders may be taxed on distributions, owners in other business structures are taxed on profits, regardless of distribution. This difference makes the corporate structure attractive to businesses that require significant amounts of capital and, or, anticipate retaining earnings for future investment. In addition, where corporate tax rates are lower than personal income tax rates, it can be advantageous in some jurisdictions to "store" profits or capital in the business.

\section{Corporation Key Features Summary}
In summary, key features of corporations include the following:

\begin{itemize}
  \item Separate legal identity

  \item Owner-operator separation allowing for greater, more diverse resourcing with some risk control

  \item Business liability is shared across multiple, limited liability owners with claims to return and financial risk of their equity investment

  \item Shareholder tax disadvantage in countries with double taxation

  \item Distributions (dividends) taxed as personal income

  \item Unbounded access to capital and unlimited business potential

\end{itemize}

\section{KNOWLEDGE CHECK}
\begin{enumerate}
  \item Which of the following are shared similarities among the four major business structure types?
\end{enumerate}

A. Sole proprietorships and general partnerships lack legal identity.

B. Corporate shareholders and general partners have limited liability.

C. The taxation of sole proprietorships and limited partnerships is comparable.

Solution:

$\mathrm{A}$ and $\mathrm{C}$ are correct. Both sole proprietorships and general partnerships have no legal identity, with the business considered an extension of the owner in a sole proprietorship and the partnership agreement setting ownership in a general partnership. Both sole proprietorships and limited partnerships have similar tax structures, with all profits taxed as personal income. But in relation to liability, while general partners have unlimited liability, shareholders of corporations are granted limited liability.

\begin{enumerate}
  \setcounter{enumi}{1}
  \item State one condition that would make a corporation subject to a regulatory jurisdiction.
\end{enumerate}

\section{Solution:}
Large corporations frequently have business operations in many different geographic regions and are subject to regulatory jurisdictions where either:

\begin{itemize}
  \item the company is incorporated,

  \item business is conducted, or

  \item the company's securities are listed.

\end{itemize}

\begin{enumerate}
  \setcounter{enumi}{2}
  \item True or false: A primary advantage of the separation of ownership from control in corporations is that it improves management by preventing conflicts of interest.
\end{enumerate}

A. True.

B. False.

Solution:

$\mathrm{B}$ is correct; the statement is false.

A major benefit of the separation between ownership and management is that it allows corporations to access capital more easily than other business structures because capital is the only requirement for owners to join the business. While the appointed board of directors and company officers are obligated to act in the best interests of shareholders, that does not eliminate conflicts of interest, which occur when management acts to place their interests, or the interests of other stakeholders, above those of the owners. To prevent conflicts and mismanagement of the business, corporate governance policies and practices are in place to oversee business operations and ensure sound management practices.

\begin{enumerate}
  \setcounter{enumi}{3}
  \item What is the primary distinction between corporate bondholders and shareholders?
\end{enumerate}

\section{Solution:}
Both bondholders and shareholders are capital providers and thus investors in the corporation's securities. However, in exchanging their capital for issued equity securities, shareholders purchase an ownership stake that entitles them to a residual claim in the corporation. Bondholders, or debtholders, have exchanged their capital for issued debt securities and are lenders to the corporation with no ownership entitlement. PUBLIC AND PRIVATE CORPORATIONS

compare public and private companies

compare the financial claims and motivations of lenders and owners

The word "public" can be misleading because it typically implies government involvement. However, when it comes to corporations, "public" and "private" are typically defined by whether the company's equity is listed on a stock exchange, although in some countries whether a company is considered public or not may depend on its number of shareholders, irrespective of whether it is listed.

Shares of public companies are most often listed on a stock exchange and thereafter trade on the exchange in the open market, also known as the secondary market. Listed companies have undergone a "go public" event, such as an initial public offering (IPO) in the primary market. An IPO is a process in which shares in the company are offered to the public for the first time in an exchange listing. Primary differences between public and private companies relate to the following:

\begin{itemize}
  \item Exchange listing and share ownership transfer

  \item Share issuance

  \item Registration and disclosure requirements

\end{itemize}

We discuss each of these.

\section{Exchange Listing and Share Ownership Transfer}
In most cases, public companies have their shares listed and traded on an exchange. An exchange listing allows ownership to be more easily transferred because buyers and sellers transact directly with one another in the secondary market, on the exchange. An investor with a brokerage account can become a shareholder in a public company simply by executing a buy order. Similarly, a shareholder can reduce or liquidate her ownership position by executing a sell order. This can be done in a matter of seconds if the number of shares involved in the transaction is relatively small and trading of the stock is liquid (i.e., a large number of shares trade on a daily basis). It can take longer, however, if the investor is trying to buy or sell a large amount of stock in a company whose shares trade infrequently.

Each trade between buyer and seller can cause a change in the share price. By plotting share price over time, we can see how the company's value changes. We can also see how significant news, either about the company specifically or about the general state of the economy, impacts the value of the shares.

Exhibit 8 and Exhibit 9 highlight differences in share ownership transfer between public and private companies.

\section{Exhibit 8: Public Companies - Share Ownership Transfer}
\begin{center}
\includegraphics[max width=\textwidth]{2023_05_04_b5cfa4f1bc883752f121g-631(1)}
\end{center}

In contrast, private company shares do not trade on an exchange, so no visible valuation or price transparency exists for the company and shares are not easily bought and sold. This makes ownership transfers between seller and buyer much more difficult than for a public company. If an owner of a private company wants to sell shares, he must find a willing buyer and the two parties must agree on a price. Even then, the company may refuse the transfer of ownership. Shareholders in private companies must exercise patience. Their investment is usually locked up until the company is acquired for cash or shares by another company, or it goes public.

\section{Exhibit 9: Private Companies - Share Ownership Transfer}
\begin{center}
\includegraphics[max width=\textwidth]{2023_05_04_b5cfa4f1bc883752f121g-631}
\end{center}

So why invest in a private company if you cannot readily sell the shares when you want to?

Because the potential returns in private companies can be much larger than those earned from investing in public companies. This is because investors in private companies are usually joining early in the company's life cycle when there is little assurance of success. They are often buying shares in a company that has little more than a business plan. The investment risks are great, but the potential rewards can be great too. With often smaller numbers of shareholders in private companies, investors have greater control over management and there may be greater chance of ownership overlaps between management and shareholders.

\section{Market Capitalization and Enterprise Value}
Because shares in (most) public companies are traded on an exchange, it is also easy to determine what the company's equity is worth at any moment in time. By taking the most recent stock price and multiplying it by the number of shares outstanding, we can calculate the company's market capitalization. That is,

Market Capitalization $=$ Current Stock Price $\times$ Total Shares Outstanding.

In theory, this is what someone would have to pay to acquire ownership of the entire corporation. In reality, a premium over this amount would have to be offered to convince enough shareholders to agree to an acquisition.

While market capitalization represents the market value of the company's shares in aggregate, enterprise value represents the total market value of the corporation, net of cash held by the company (i.e., the sum of the market value of the equity and the market value of debt, net of cash). Acquirers are often more interested in enterprise value because it's a better representation of what it would cost to own the company free and clear of all debt.

Market Capitalization = Market Value of Shares

Enterprise Value = Market Value of Shares + Market Value of Debt - Cash

\section{KNOWLEDGE CHECK}
\section{Market Capitalization vs. Enterprise Value}
8Tera Therapeutics is a hypothetical, public company with 15.2 million shares outstanding, no short-term debt, and USD200 million of long-term debt. The company also has USD20 million in cash and a recent stock price of USD120 per share.

\begin{enumerate}
  \item What is 8Tera's market capitalization?
\end{enumerate}

\section{Solution:}
Market capitalization is the total market value of the equity, which is determined by multiplying the most recent stock price by the number of shares outstanding.

Market Capitalization $=($ USD120 per share $) \times(15.2$ million shares $)$

$=$ USD1.824 billion

\begin{enumerate}
  \setcounter{enumi}{1}
  \item What is 8 Tera's enterprise value?
\end{enumerate}

\section{Solution:}
Enterprise value (EV) is equal to market capitalization plus net debt.

$\mathbf{E V}=$ USD1.824 billion + USD200 million - USD20 million

$=\mathrm{USD} 2.004$ billion

EV tells us what it would cost to own the entire company free and clear of all debt.

\section{Share Issuance}
To raise more capital after listing, public companies may issue additional shares in the capital markets, typically raising very large amounts from many investors who may then actively trade shares among themselves in the secondary market. In contrast, private companies finance much smaller amounts in the primary market (private debt or equity) with far fewer investors who have much longer holding periods.

Exhibit 10 and Exhibit 11 illustrate differences in public vs. private company share issuance and relative size of capital accessed.

\section{Exhibit 10: Public Companies — Share Issuance and Capital Access}
\begin{center}
\includegraphics[max width=\textwidth]{2023_05_04_b5cfa4f1bc883752f121g-633(1)}
\end{center}

\section{Exhibit 11: Private Companies — Share Issuance and Capital Access}
\begin{center}
\includegraphics[max width=\textwidth]{2023_05_04_b5cfa4f1bc883752f121g-633}
\end{center}

Investors in private companies are typically invited to purchase shares in the company through a private placement whose terms are outlined in a legal document called a private placement memorandum (PPM). The PPM, also termed the offering memorandum, describes the business, the terms of the offering, and most importantly, the risks involved in making an investment in the company. Because private securities are generally not registered with a regulatory authority, investors may be restricted to accredited investors, also termed "eligible" or "professional" investors depending on the jurisdiction.

Accredited investors are those who are sophisticated enough to take greater risks and to have a reduced need for regulatory oversight and protection. To be considered accredited, an investor must have a certain level of income or net worth or possess a certain amount of professional experience or knowledge. Given the lower levels of disclosure required of private companies, regulators want to make sure that investors understand the associated risks and can afford the possibility of losing their entire investment.

\section{Registration and Disclosure Requirements}
Public companies are required to register with a regulatory authority. As a result, they are subject to greater compliance and reporting requirements. In the United States, for example, public companies must disclose certain kinds of financial information on a quarterly basis through the filing of documents with the Securities and Exchange Commission (SEC) on the system known as EDGAR (Electronic Data Gathering, Analysis, and Retrieval). In the European Union, listed companies must disclose on a semi-annual basis providing such information as their financial reports, major changes in the holding of voting rights, and other inside information that might be expected to affect security price.

Public companies must also disclose other kinds of information, such as any stock transactions made by officers and directors. These documents are made available to the general public, not just the investors in the company. The primary purpose of this kind of disclosure is to make it easier for investors and analysts to assess the risks that might impact the company's business strategy and its ability to generate profits or meet its financial obligations in the future.

In contrast, private companies are generally not subject to the same level of regulatory oversight. While many of the rules that pertain to the regulation of public companies (for example, prohibitions against fraud and the obligation to produce a corporate tax return) also apply to private companies, private companies have no obligation to disclose certain information to the public.

Of course, they willingly disclose pertinent information directly to their investors, especially if they hope to be able to raise additional capital in the future, but private companies are typically not required to file documents with the regulatory authority that oversees public companies. Exhibit 12 and Exhibit 13 compare typical entity relationships for public and private companies.

\section{Exhibit 12: Public Companies - Typical Entity Relationships}
\begin{center}
\includegraphics[max width=\textwidth]{2023_05_04_b5cfa4f1bc883752f121g-634}
\end{center}

\section{Exhibit 13: Private Companies - Typical Entity Relationships}
\begin{center}
\includegraphics[max width=\textwidth]{2023_05_04_b5cfa4f1bc883752f121g-634(1)}
\end{center}

\section{KNOWLEDGE CHECK}
\begin{enumerate}
  \item Match the applicable company characteristic with its most correct category (Publicly-held, Privately-held, Both).
\end{enumerate}

Company characteristic:

\begin{center}
\begin{tabular}{l}
\hline
\multicolumn{1}{l}{Publicly-held $\quad$ Privately-held $\quad$ Both} \\
\hline
Exchange listed \\
Shareholder- \\
management ownership \\
overlaps \\
Registered \\
Share liquidity \\
Non-financial disclosure \\
required \\
Negotiated sales of debt \\
and equity \\
\end{tabular}
\end{center}

\section{Solution:}
\begin{center}
\begin{tabular}{|c|c|c|c|}
\hline
 & Publicly-held & Privately-held & Both \\
\hline
Exchange listed & $\mathrm{X}$ &  &  \\
\hline
$\begin{array}{l}\text { Shareholder- } \\ \text { management ownership } \\ \text { overlaps }\end{array}$ &  & $\mathrm{X}$ &  \\
\hline
Registered & $\mathrm{X}$ &  &  \\
\hline
Share liquidity & $\mathrm{X}$ &  &  \\
\hline
$\begin{array}{l}\text { Non-financial disclosure } \\ \text { required }\end{array}$ &  &  & $\mathrm{X}$ \\
\hline
$\begin{array}{l}\text { Negotiated sales of debt } \\ \text { and equity }\end{array}$ &  & $\mathrm{X}$ &  \\
\hline
\end{tabular}
\end{center}

Publicly-held companies are most often listed on exchanges and required to register shares. Their shares are typically liquid with minor ownership overlap between management and shareholders. These companies must make both financial and non-financial disclosures, and both their debt and equity are typically traded on exchanges.

Privately-held companies are not exchange listed nor usually subject to registration requirements. Share issuance is smaller in nature, creating a greater chance of ownership overlap between management and shareholders.

Private company shares are illiquid. Generally, these companies are required to make only non-financial disclosures. The sale of their equity and debt is privately negotiated between company insiders and capital providers.

\section{Going Public from Private - IPO, Direct Listing, Acquisition}
So how exactly do private companies become public companies? A private company can go public in the following ways:

\begin{itemize}
  \item IPO - Direct listing

  \item Acquisition

  \item Special Purpose Acquisition Company

\end{itemize}

As noted earlier, many companies choose to use the IPO process to list on an exchange. To complete an IPO, companies must meet specific listing requirements required by the exchange. The IPO involves the participation of investment banks who underwrite, or guarantee, the offering or sale of new (or existing) shares. Proceeds from the sale of new shares go to the issuing corporation, which can use the proceeds to finance new investments. Once the IPO process is completed, the company is public and its shares begin trading on an exchange.

A private company can also go public through a direct listing (DL), which differs from an IPO in two key ways. A DL does not involve an underwriter, and no new capital is raised. Instead, the company is simply listed on an exchange and shares are sold by existing shareholders. Major benefits of a DL are the speed of going public and the lower costs involved.

Often, a company may go public through acquisition. For example, this may occur indirectly when the company is acquired by another company that is public. In such cases, the acquiring company is usually larger. As a result, a share in the combined entity might represent only a small interest in the company that was acquired.

Another means of acquisition is through a special purpose acquisition company (SPAC). A SPAC is a shell company, often called a "blank check" company, because it exists solely for the purpose of acquiring an unspecified private company sometime in the future.

SPACs raise capital through an IPO. Proceeds are placed in a trust account and can only be disbursed to complete the acquisition or for return back to investors. SPACs are publicly listed and may specialize in a particular industry. SPACs have a finite time period, such as 18 months, to complete a deal; otherwise, proceeds are returned to investors. While investors in a SPAC might not know with certainty what the SPAC will buy, they might make an educated guess based on the backgrounds of the SPAC executives or comments these individuals have made in the media.

Once the SPAC completes the purchase of a private company, that company becomes public. SPACs are replacing the formerly used reverse merger process of going public, which used a listed company shell with a previous business and trading history.

\section{KNOWLEDGE CHECK}
\begin{enumerate}
  \item Match the method by which a private company can go public with the most closely related term from corporate finance.
\end{enumerate}

Going Public Method:

\begin{center}
\begin{tabular}{llcl}
\hline
 & "Blank check" & $\begin{array}{c}\text { Existing } \\ \text { shareholder }\end{array}$ & Underwriter \\
\hline
IPO &  &  &  \\
DL &  &  &  \\
SPAC &  &  &  \\
\hline
\end{tabular}
\end{center}

Solution:

\begin{center}
\begin{tabular}{lccc}
\hline
 & "Blank check" & $\begin{array}{c}\text { Existing } \\ \text { shareholder }\end{array}$ & Underwriter \\
\hline
$\mathrm{IPO}$ & $\mathrm{X}$ &  &  \\
$\mathrm{DL}$ & $\mathrm{X}$ &  &  \\
$\mathrm{SPAC}$ &  &  &  \\
\hline
\end{tabular}
\end{center}

An IPO is facilitated by investment banks who underwrite, or guarantee, the offering. A direct listing does not involve an underwriter, and no new capital is raised. Instead, the company is simply listed on an exchange and shares are sold by existing shareholders. A SPAC is a shell company, often called a "blank check" company, because it exists solely for the purpose of acquiring an unspecified private company sometime in the future.

\begin{enumerate}
  \setcounter{enumi}{1}
  \item True or false: Accredited investors are the capital providers qualified by regulators to invest in public companies.
\end{enumerate}

A. True. Statement to support this option on why it's true.

B. False. Statement to support this option on why it's false.

\section{Solution:}
B is correct; the statement is false. Accredited investors are perceived by regulators to have the sophistication for understanding and assuming the risks that come with investing in private, not public, companies.

\section{Life Cycle of Corporations}
Whether a company is public or private often depends on where it is in its life cycle. As shown in Exhibit 14, companies begin life as start-ups, then enter growth, maturity, and lastly decline. During their life cycle, companies access different types of financing.

\section{Exhibit 14: Life Cycle and Financing}
\begin{center}
\includegraphics[max width=\textwidth]{2023_05_04_b5cfa4f1bc883752f121g-637}
\end{center}

\section{Start-Ups}
Start-ups are a little more than an idea and a business plan. They are initially funded by the founders. If more capital is needed, the founders might turn to friends and family, who may buy an ownership interest or make a loan to the company. At this stage, the company lacks revenues and cash flows. Business risk is extremely high, making financing challenging.

As the company grows, more capital will be needed. The founders might hire an investment banker who specializes in helping private companies raise capital from private equity or private debt investors. Early-stage equity investors are sometimes referred to as venture capitalists, or Series A investors.

\section{Growth}
As the company progresses through the growth stage, even greater amounts of capital will be needed. Most likely, while revenues and cash flows may be improving, the company is still not profitable; so, it cannot yet rely on internally generated earnings to fund growth. It might raise more capital through a Series B or even a Series C issuance (i.e., additional rounds of capital raises). This is also the time the company might consider "going public" in an IPO.

Despite the desire to go public, some companies might remain private for many years. In some cases, the founder(s) may not be willing to give up ownership or control, while in other cases, this may be partly due to stock exchange listing requirements. Depending on the exchange, to qualify for listing a company must be able to pay the listing fees, have a minimum number of shareholders, and meet a minimum valuation requirement. If the company can meet the listing requirements, the decision to go public may depend on its ability to access capital.

Rapidly growing companies may find that their capital needs are too great for private investors to meet. In such cases, going public could make sense because it is usually easier to raise large amounts of capital in the public markets than in the private markets. In many developed countries, however, it has become easier for companies to access needed capital in the private market and the number of companies choosing to go public has been decreasing.

\section{Maturity}
Once a company reaches the maturity stage of the life cycle, its external financing needs diminish and business risk is much less. Companies in this stage are usually profitable, cash flow generative, and can fund growth internally with retained earnings. In addition, these companies find it easier to borrow money at reasonable terms because their cash flows are more predictable with "business-as-usual" (BAU) operations. A mature company that is public can borrow money either in the public or private markets.

\section{Decline}
When a company is in decline it may have little need for additional financing. However, companies in this stage may try to reinvent themselves either by developing new lines of business or by acquiring other companies that are growing rapidly. Either way, additional financing can be useful in such circumstances, but companies may find the cost of financing increasingly expensive as their financials and cash flows deteriorate.

Exhibit 15 illustrates the sources of financing typically available to a company during its life cycle, given characteristics of its revenues, cash flows, and business risk.

\section{Exhibit 15: Type of Financing by Firm Life Cycle}
\begin{center}
\includegraphics[max width=\textwidth]{2023_05_04_b5cfa4f1bc883752f121g-639}
\end{center}

\section{Public to Private - $L B O, M B O$}
Public companies can also end up going private. This happens when an investor (or group of investors) acquires all of the company's shares and delists it from the exchange. This is done through a leveraged buyout (LBO) or management buyout (MBO). Both processes involve borrowing large amounts of money to finance the acquisition. An LBO occurs when the investors are not affiliated with the company they are buying. An MBO occurs when the investors are members of the company's current management team.

LBOs and MBOs are initiated when the investors believe the public market is undervaluing the shares and financing costs are sufficiently low and attractive. Even though they must pay a premium to convince shareholders to tender (sell) their shares, the investors believe the transaction is worth it due to synergies or cost savings they believe can be realized by taking the company private. These companies are often taken public again several years later if the investors believe they can get a good valuation price at that time.

Exhibit 16 shows the interchange that can occur between private and public companies. Private companies can go public by being acquired by a company that is already public or through an IPO, DL, or SPAC. A private company could also remain private if it is acquired by another private company. Public companies may be taken private and delisted in an LBO or MBO.

\section{Exhibit 16: Interchange between Private and Public Companies}
\begin{center}
\includegraphics[max width=\textwidth]{2023_05_04_b5cfa4f1bc883752f121g-640}
\end{center}

It is interesting to note the trends in public and private companies over time. In many emerging economies, the number of public companies is rising, while the opposite is happening in developed economies. Emerging economies are typically characterized by higher rates of growth and are transitioning from closed to open market structures. Therefore, it makes sense that the number of public companies would increase as these emerging economies grow larger.

What explains the decline in public companies in developed markets?

Mergers and acquisitions are partly responsible. When a public company is acquired by a private company or by another public company, one less public company exists. LBOs and MBOs are also responsible since they are structured to take public companies private. Still another factor is that many private companies simply choose to remain private. This is due, in part, to the greater ease of accessing capital in the private markets. Due to the burgeoning venture capital, private equity, and private debt markets, companies often find that they can get the capital they need in these markets while avoiding the regulatory burdens and associated compliance costs of going public. In addition, companies may choose to remain private to avoid the short-term focus many investors in listed companies have. Remaining private can also provide leadership with greater flexibility and responsiveness in decision making.

\section{LENDERS AND OWNERS}
compare the financial claims and motivations of lenders and owners

As shown in Exhibit 17, a key difference between the debt and equity financing used by corporations is the binding contract the company has with its debtholders, or lenders. The contract requires their claims be fully paid before the company can make distributions to equity owners. In other words, debtholders have a prior legal claim on the company's cash flows and assets over the claims of equity owners. Equityholders are therefore residual claimants to the company after all other stakeholders have been paid, including creditors (interest/principal), suppliers (accounts payable), government (taxes), and employees (wages).

\section{Exhibit 17: Debt vs. Equity Claim Difference}
\begin{center}
\includegraphics[max width=\textwidth]{2023_05_04_b5cfa4f1bc883752f121g-641}
\end{center}

Another difference is that interest payments to debtholders are generally treated as a tax-deductible expense for the company, while dividend payments to shareholders are not. Finally, equityholders represent a more permanent source of capital and have voting rights to elect the board of directors, which oversees the management of the company. In contrast, debt represents a cheaper financing source for companies and a lower risk for investors.

\section{KNOWLEDGE CHECK}
\begin{enumerate}
  \item Match/choose the applicable financing feature with its correct form (Debt or Equity).
\end{enumerate}

\begin{center}
\begin{tabular}{|c|c|c|}
\hline
 & Debt & Equity \\
\hline
\multicolumn{3}{|l|}{Legal repayment obligation} \\
\hline
\multicolumn{3}{|l|}{Residual asset claim} \\
\hline
\multicolumn{3}{|l|}{Discretionary payments} \\
\hline
\multicolumn{3}{|l|}{Tax-deductible expenses} \\
\hline
\multicolumn{3}{|l|}{Finite term commitments} \\
\hline
\multicolumn{3}{|l|}{Voting rights} \\
\hline
\multicolumn{3}{|l|}{Solution:} \\
\hline
 & Debt & Equity \\
\hline
Legal repayment obligation & $\mathrm{X}$ &  \\
\hline
Residual asset claim &  & $\mathrm{X}$ \\
\hline
Discretionary payments &  & $\mathrm{X}$ \\
\hline
Tax-deductible expenses & $\mathrm{X}$ &  \\
\hline
Finite term commitments & $\mathrm{X}$ &  \\
\hline
Voting rights &  & $\mathrm{X}$ \\
\hline
\end{tabular}
\end{center}

Similar to a loan agreement, debt involves a contractual obligation with priority for interest and principal payments. Equity has no contractual commitment and involves a residual claim to assets. Equity features discretionary payments like dividends, which are not tax deductible. Debt requires contractual interest and principal payments, with interest expense being tax deductible to the issuer. Debt has a stated, finite term with generally no voting rights, while equity has no finite term and includes voting rights.

\section{Equity and Debt Risk-Return Profiles}
Risk is an important issue to consider for both the issuing corporation and the investors. From an investor's perspective, stocks are riskier than bonds because shareholders are residual claimants on the firm. Thus, bonds having predictable coupon payments are less risky than a stock that may, or may not, receive dividends or experience a capital gain. The opposite is true from the corporate issuer's perspective. Bonds require the corporate issuer to have the funds available for the payments - or face default while payments to shareholders are at the discretion of the issuer's management team.

\section{Investor Perspective}
The maximum loss equity owners face is limited to the amount of their equity investment. However, an equity owner has the potential for significant upside gain, dependent on future share price increases. If the corporation is successful, there is theoretically no limit to how much equity owners could make from their investment. As residual claimants, after a profitable corporation meets its other obligations, shareholders are entitled to the full remaining value of assets and distributions.

Stocks are considered riskier for investors, however, because the company has no contractual obligation to distribute funds to shareholders or repay their capital investment. In the worst-case scenario, the company might go bankrupt and owners may lose their entire investment. Due to their limited liability with the corporation, however, shareholders cannot lose more than their investment.

Exhibit 18 shows this asymmetry in shareholder's downside losses versus potential upside gains. The value of equity is determined as the residual of the future value of the firm less the value of its debt. In theory, the market value of debt should be used; however, book value is often used as a proxy since market values for bonds are often unavailable or unreliable. Potential upside gains to shareholders are limited only by the future value of the firm, while shareholder losses are limited to their initial investment. If the value of the firm falls below the book value of debt, debtholders experience losses.

\section{Exhibit 18: Equityholder - Upside and Downside}
\begin{center}
\includegraphics[max width=\textwidth]{2023_05_04_b5cfa4f1bc883752f121g-643}
\end{center}

Equity owners, therefore, have an interest in the ongoing maximization of company value (net assets less liabilities), which directly corresponds to the value of their shareholder wealth. Cash flows to equityholders include such distributions as dividends, share repurchases and buybacks, and proceeds from the sale of the company.

\begin{center}
\begin{tabular}{lll}
\hline
Investor Perspective & Equity & Debt \\
\hline
Return potential & Unlimited & Capped \\
Maximum loss & Initial investment & Initial investment \\
Investment risk & Higher & Lower \\
Investment interest & Max (Net assets - Liabilities) & Timely repayment \\
\hline
\end{tabular}
\end{center}

With a fixed priority claim, bondholders are contractually promised priority in receiving their specified interest payment with return of principal. No matter how profitable the company becomes, however, bondholders will never receive more than their interest and principal repayment. When the company is financially healthy and able to service its debt commitments from cash flows, or has sufficient assets to serve as collateral to debt, debt offers predictable returns for investors. There is no residual claim value for bondholders, only a priority claim.

Exhibit 19 shows this asymmetry in downside losses versus potential upside gains to debtholders. Potential upside gains to debtholders are limited to interest plus principal repayment regardless of how high the future value of the firm rises. In contrast, if the value of the firm falls below the book value of debt, debtholders experience losses in direct relationship to the decrease in firm value.

\section{Exhibit 19: Debtholder - Upside and Downside}
\begin{center}
\includegraphics[max width=\textwidth]{2023_05_04_b5cfa4f1bc883752f121g-644}
\end{center}

Bondholders are thus interested in assessing the likelihood of timely debt repayment and the risk associated with the company's ability to meet its debt obligations. This assessment includes the following:

\begin{itemize}
  \item Assessing issuer cash flows and collateral/security

  \item Evaluating issuer creditworthiness and willingness to pay its debt

  \item Estimating the probability of default and amount of loss given a default

\end{itemize}

Downside risk increases to bondholders as a company takes on more debt. Debt becomes increasingly unattractive and risky to investors when the company's cash flows fail to comfortably cover its debt obligations.

If the company is struggling, unlike equity holders, bondholders do have some recourse. While bondholders could lose their entire investment as well, they receive priority in the event of financial distress. If necessary, bondholders can force the company through the contractual agreement in place to liquidate assets and return to them as much of their money as possible, reducing the likelihood that bondholders lose their entire investment. Additionally, the company cannot legally make dividend payments to shareholders until it first meets its bondholder obligations.

\section{KNOWLEDGE CHECK}
\begin{enumerate}
  \item Equity and debtholders share the same investor perspective in regard to
A. maximum loss.
B. investment risk.
C. return potential.
\end{enumerate}

\section{Solution:}
A is correct. For both equityholders and debtholders, their initial investment represents their maximum possible loss. The return potential is theoretically unlimited for equityholders while it is capped for debtholders. Equityholders are exposed to a higher level of investment risk given no contractual obligation exists between them and the company.

\begin{enumerate}
  \setcounter{enumi}{1}
  \item True or false: Debtholders, unlike equityholders, have symmetric potential downside losses and upside gains.
\end{enumerate}

A. True. Both are at risk of losing the full amount of their investment.

B. False. Debtholders' upside gains are capped in contrast to those of equityholders.

\section{Solution:}
B is correct; the statement is false. Both debtholders and equityholders have asymmetries between their downside losses and upside gains. For debtholders, potential upside gains are limited to interest plus principal repayment regardless of how high the future value of the firm rises. In contrast, if the value of the firm falls below the book value of debt, debtholders experience losses in direct relationship to the fall in firm value.

For equityholders, equity value is determined as the residual of the future value of the firm less the value of its debt. Potential upside gains to shareholders are limited only by the future value of the firm, while shareholder losses are limited to their initial investment.

\section{Issuer Perspective}
Because the returns to lenders are capped and because the cost of debt is lower than the cost of equity, corporations with predictable cash flows may prefer to borrow money rather than sell an ownership stake to raise the capital they need to finance their investments. This is because issuing more equity dilutes upside return for existing equity owners given that residual value must be shared across more owners.

From the issuer's perspective, bonds are riskier than stocks for the same reason bonds are safer than stocks for investors. Bonds increase risk to the corporation by increasing leverage. If the company is struggling and cannot meet its promised obligations to bondholders, bondholders have the legal standing to force certain actions upon the corporation, such as bankruptcy and liquidation.

\begin{center}
\begin{tabular}{lll}
\hline
Issuer Perspective & Equity & Debt \\
\hline
Capital cost & Higher & Lower \\
Attractiveness & Creates dilution, may be only & Preferred when issuer cash \\
 & $\begin{array}{l}\text { option when issuer cash flows } \\ \text { are absent or unpredictable }\end{array}$ & $\begin{array}{l}\text { flows are predictable }\end{array}$ \\
Investment risk & $\begin{array}{l}\text { Lower, holders cannot force } \\ \text { liquidation }\end{array}$ & Higher, adds leverage risk \\
Investment interest & Max (Net assets - Liabilities) & Debt repayment \\
\hline
\end{tabular}
\end{center}

In contrast, shareholders have no such contractual rights. Issuing equity dilutes ownership, but equity is less risky to the corporation than debt because shareholders do not have the same incentive to force the company into bankruptcy or liquidation proceedings. In fact, early-stage companies or companies with unpredictable cash flows may not be able to borrow even if they wanted to. If the company fails to meet its obligation to bondholders, it does have some options to try and avoid bankruptcy. For example, it can try to renegotiate more favorable terms with the bondholders or bank lenders. If the bondholders refuse, it can petition the courts for bankruptcy protection. In such cases, the company may be forced to suspend certain other payments, such as dividend payments on the stock. Eventually, however, the assets might have to be liquidated to raise as much money as possible to return to the bondholders. Alternatively, the business might be reorganized, with shareholders getting wiped out and bondholders becoming the new shareholders of the reorganized company.

\section{KNOWLEDGE CHECK}
\section{Biotech Startup}
A scientist discovers and patents a medical compound that shows promise for curing a debilitating disease. This scientist soon realizes that significant capital will be required to run the clinical trials demanded by the national public health authority before approval for the medication can be granted. If approval is granted, more money will be required for manufacturing, marketing, and distribution. Laboratory space and state-of-the-art equipment must be purchased or leased, contracts will have to be signed with suppliers, and other scientists and administrative personnel will have to be hired.

\begin{enumerate}
  \item What form of business structure is most appropriate for this business, and why?
\end{enumerate}

\section{Solution:}
The corporate form of business structure would be most suitable for the scientist's biotechnology business. Because the corporation is a legal entity, it can engage in all of the activities mentioned and is better suited to handle the financing of the anticipated growth needs of the business in addition to providing risk protection to the scientist and other future investors.

\begin{enumerate}
  \setcounter{enumi}{1}
  \item Will the scientist be able to borrow the money she needs to get the business started?
\end{enumerate}

\section{Solution:}
This early-stage start-up company has no revenues or profits; therefore, it will be almost impossible to borrow money. Lenders are more risk averse than equity investors and tend to avoid businesses that do not have sufficient cash flows to make the contractual payments on a loan. Furthermore, lenders have a limited upside payoff profile. No matter how successful the company becomes, lenders receive only a fixed return. This start-up company would seek to avoid debt at this stage as missing an interest payment could result in bankruptcy, putting an end to the business.

The more realistic solution for raising capital at this stage is to sell an ownership stake to others. Venture capitalists might find this start-up an attractive investment because they specialize in taking greater risks than lenders, in exchange for an unlimited upside payoff profile. By selling an ownership interest to equity investors, the scientist can raise the capital required to begin the process of running the trials and securing approval to manufacture and market the medication. The scientist, as founder of the company, starts off as the sole investor and full owner. She owns all of the equity, or shares, in the company. To raise significant amounts of capital for the clinical trials, she agrees to sell some of her ownership stake to others. With the help of skilled advisers, the business is valued at $¥ 100$ million based on its potential and shares are offered to investors.

To maintain control over the company, the scientist/founder decides to keep half the shares for herself and make the other half available to others. Suppose one investor invests $¥ 20$ million and three other investors invest $¥ 10$ million each.

\begin{enumerate}
  \item Who are the likely investors, and why?
\end{enumerate}

\section{Solution:}
In this case, the money will be raised in the private market. Potential investors are likely to be venture capitalists who specialize in investing in risky biotechnology startups.

\begin{enumerate}
  \setcounter{enumi}{1}
  \item What percentage ownership stake and share amount does each investor have after the equity raise if there are 10 shares in total?
\end{enumerate}

\section{Solution:}
Given the company has a valuation of $¥ 100$ million, $¥ 50$ million will have to be raised from other investors. After the capital is raised, the founder owns $50 \%$ ( $¥ 50$ million / $¥ 100$ million) and 5 shares of the company, one investor owns $20 \%$ ( $¥ 20$ million / $¥ 100$ million) and 2 shares, and the three remaining investors own $10 \%$ ( $¥ 10$ million / $¥ 100$ million) and 1 share each in the company.

Note that the number of shares is arbitrary and can change by splitting the shares or by raising more capital in the future. The more important issues are the overall valuation of the company and the proportional ownership stake each investor has.

\section{Equity vs. Debt Conflicts of Interest}
Potential conflict can occur between the interests of shareholders and bondholders. Because shareholders have limited downside liability, equal to the amount of their investment, and unlimited return potential, they prefer management to invest in projects that involve greater calculated risks and potential returns. At the extreme, shareholders would like the company to simply increase dividend payments and share repurchases with debt proceeds.

For bondholders, the upside return is capped or limited to the face value of debt plus the coupon. Bondholders receive no financial benefit or reward from issuer investment decisions that increase risk. As a result, bondholders prefer management to invest in less risky projects that increase cash flow certainty, even if those cash flows are relatively small, to increase the likelihood of timely interest and principal repayment by the company. Because bondholders do not have control over management decisions, they often rely on covenants to protect them against exploitive actions that compromise the safety of their investment.

\section{SUMMARY}
\begin{itemize}
  \item Common forms of business structures include sole proprietorships, general and limited partnerships, and corporations.

  \item Sole proprietorships and partnerships are considered extensions of their owner or partner(s). This largely means that profits are taxed at the individual's personal rates and individuals are fully liable for all of the business's debts.

  \item Limited partnerships and corporations allow for the specialization of expertise in operator roles, in addition to the re-distribution of risk and return sharing between owners, partners, and operators.

  \item The corporate form of business structure is preferred when capital requirements are greater than what could be raised through other business structures.

  \item A corporation is a legal entity separate and distinct from its owners. Owners have limited liability, meaning that only their investment is at risk of loss.

  \item Corporations raise capital by selling an ownership interest and by borrowing money. They issue stocks, or shares, to equity investors who are owners. Debt represents money borrowed from lenders. Long-term lenders are issued bonds.

  \item Nonprofit corporations are formed to promote a public benefit, religious benefit, or charitable mission. They do not have shareholders, they do not distribute dividends, and they generally do not pay taxes.

  \item For-profit corporations can be public or private.

  \item In many jurisdictions, corporate profits are taxed twice: once at the corporate level and again at the individual level when profits are distributed as dividends to the owners.

  \item Public corporations are usually listed on an exchange and ownership is easily transferable.

  \item Private corporations are not listed on an exchange and, therefore, have no observable stock price, making their valuation more challenging. Transactions between buyers and sellers are negotiated privately, and ownership transfer is much more difficult.

  \item The market capitalization of a public company is equal to share price multiplied by number of shares outstanding.

  \item Enterprise value represents the total value of the company and is equal to the sum of the market capitalization and the market value of net debt. (Net debt is debt less cash.)

  \item Public companies are subject to greater regulatory and disclosure requirements-most notably, the public disclosure of financial information through periodic filings with their regulator. Private companies are not required to make such disclosures to the public.

  \item Given greater risks, only accredited investors are permitted to invest in private companies.

  \item Corporations have a life cycle with four distinct stages: start-up, growth, maturity, and decline. - Although corporations begin as private companies, many eventually choose to go public or are acquired by public companies. IPOs typically occur in the growth phase and are usually driven by capital needs to fund growth.

  \item In many developed countries, it has become easier for private companies to access the capital they need without having to go public. As a result, the number of listed (public) companies in developed countries has been trending downwards. The number of listed companies in emerging economies continues to grow.

  \item Debt (bonds) represents a contractual obligation on the part of the issuing company. The corporation is obligated to make the promised interest payments to the debtholders and to return the principal. Equity (stocks) does not involve a contractual obligation.

  \item Interest payments on debt are typically a tax-deductible expense for the corporation. Dividend payments on equity are not tax deductible.

  \item Debtholders have claim priority, but they are entitled only to the interest payments and the return of principal. Equityholders have no priority in claims.

  \item Therefore, from the investor's perspective, investing in equity is riskier than investing in debt. Equityholders do have a residual claim, meaning that they are entitled to whatever firm value remains after paying off the priority claim holders, which grants them unlimited upside potential.

  \item From the corporation's perspective, issuing debt is riskier than issuing equity. A corporation that cannot meet its contractual obligations to the debtholders can be forced into bankruptcy and liquidation.

  \item Potential conflicts can occur between debtholders and equityholders. Debtholders would prefer the corporation to invest in safer projects that produce smaller, more certain cash flows that are large enough to service the debt. Equityholders would prefer riskier projects that have much larger return potential, which they do not share with the debtholders.

\end{itemize}

\section{PRACTICE PROBLEMS}
\begin{enumerate}
  \item Describe the process of going public by a private company.

  \item Describe the process of going private by a public company.

  \item Identify the true statement(s) about corporation types from among the following:

\end{enumerate}

A. Nonprofit corporations by definition cannot generate profits.

B. Transferring ownership from seller to buyer is more difficult for a private company than for a public company.

C. Companies are categorized as public when they have greater than a minimum number of shareholders.

\begin{enumerate}
  \setcounter{enumi}{3}
  \item From the corporate issuer's perspective, the risk level of bonds compared to stocks is
A. lower.
B. higher.
C. the same.

  \item True or false: Bondholders can become shareholders through non-market-based means. Justify your answer.
A. True.
B. False.

  \item Explain potential conflicts of interest between debtholders and equityholders.

  \item State a reason for the declining number of public companies in developed markets.

\end{enumerate}

\section{SOLUTIONS}
\begin{enumerate}
  \item Private companies can go public through a process known as the initial public offering (IPO). This means that the company is offering its shares to all investors for the first time. An investment bank (or group of investment banks) acts as the underwriter of the offering, meaning that they guarantee the sale of the shares for the issuer. Once the IPO process is completed, the shares are listed on an exchange and available for trading. Another way a private company can go public is through a direct listing (DL), which is a process that does not involve underwriters or the issuance of new shares.
\end{enumerate}

Private companies can also go public indirectly by being acquired by another company that is already public or through a special purpose acquisition company (SPAC). A SPAC is a publicly-listed holding company created for the sole purpose of acquiring a private company.

\begin{enumerate}
  \setcounter{enumi}{1}
  \item Two common ways for public companies to go private are the leverage buyout (LBO) and management buyout (MBO). An LBO occurs when an outside investor or group of investors borrows money to purchase all of the equity of the public company. A premium to the market price must be paid to convince all shareholders to agree to the LBO. The investors typically pledge the assets of the company against the loan.
\end{enumerate}

An MBO is similar, except that the investors are part of the company's management team. Another way for a public company to go private is to be acquired by a private company. Once a company goes private, its shares are no longer listed on an exchange. A public company can also be acquired by another public company. When this happens, shares of the acquired company are delisted from the exchange; however, shares of the acquiring company remain listed.

\begin{enumerate}
  \setcounter{enumi}{2}
  \item B and C are correct.
\end{enumerate}

A is incorrect. If they are run well, nonprofits can generate profits; however, all profits must be reinvested in promoting the mission of the organization.

$B$ is correct. In contrast to public companies, private company shares do not trade on an exchange, so no visible valuation or price transparency exists for the company. Private company shares are not liquid. This means that transferring ownership from seller to buyer is more difficult than it is for a public company. $\mathrm{C}$ is correct. In many countries, if there are a large number of shareholders (usually greater than 50), the company is categorized as a public company and subject to more onerous regulatory requirements whether or not it is listed on a stock exchange.

\begin{enumerate}
  \setcounter{enumi}{3}
  \item B is correct. From the issuer's perspective, bonds are riskier than stocks for the same reason bonds are safer than stocks for investors. Bonds increase risk to the corporation by increasing leverage. If the company is struggling and cannot meet its promised obligations to bondholders, bondholders have the legal standing to force certain actions upon the corporation, such as bankruptcy and liquidation.

  \item A is correct; the statement is true. If a company fails to meet its obligation to bondholders and ultimately needs to petition the courts for bankruptcy protection, a potential alternative to asset liquidation to maximize proceeds for debt repayment is business reorganization. Following that path through the legal process as opposed to transactions in private or public markets, the company can be reorganized with shareholders getting wiped out and bondholders becoming its new shareholders. 6. From an investor's perspective, debt is less risky than equity because the company has a contractual obligation to repay the debt but no obligation to repay equity capital. Furthermore, debtholders are entitled only to the promised interest payments and the return of principal. As a result, debtholders would prefer that the company invest in relatively safe projects that produce sufficient returns to service the debt. They see no added benefit of taking greater risks that might generate larger returns. The gains to equityholders, however, are unlimited. As a result, equityholders prefer that the company invest in projects that might be riskier but that have the potential to produce much larger returns.

  \item Reason 1. Mergers and acquisitions are partly responsible. When a public company is acquired by a private company or by another public company, there is one less public company.

\end{enumerate}

Reason 2. LBOs and MBOs are also responsible since they are structured to take public companies private.

Reason 3. Many private companies choose to remain private. Greater ease in accessing capital in private markets (venture capital, private equity, and private debt) has enabled companies to source the capital they might need and avoid the regulatory burden associated with being a public company.

\section{LEARNING MODULE 2}
\section{Introduction to Corporate Governance and Other ESG Considerations}
by Young Lee, CFA, JD, Assem Safieddine, PhD, Donna F. Anderson, CFA, Deborah S. Kidd, CFA, and Hardik Sanjay Shah, CFA.

Young Lee, CFA, JD, is at MacKay Shields (USA and Europe). Assem Safieddine, PhD, is at Suliman Olayan Business School, American University of Beirut (Lebanon). Donna F. Anderson, CFA (USA). Deborah S. Kidd, CFA, is at CFA Institute (USA). Hardik Sanjay Shah, CFA, is at GMO LLC (Singapore).

\section{LEARNING OUTCOME}
\begin{center}
\begin{tabular}{c|l}
Mastery & The candidate should be able to: \\
 & $\square$ \\
$\square$ & $\begin{array}{l}\text { describe a company's stakeholder groups and compare their interests } \\ \text { describe the principal-agent relationship and conflicts that may arise } \\ \text { between stakeholder groups } \\ \text { describe corporate governance and mechanisms to manage } \\ \text { stakeholder relationships and mitigate associated risks } \\ \text { describe both the potential risks of poor corporate governance and } \\ \text { stakeholder management and the benefits from effective corporate } \\ \text { governance and stakeholder management } \\ \text { describe environmental, social, and governance considerations in } \\ \text { investment analysis } \\ \text { describe environmental, social, and governance investment } \\ \text { approaches }\end{array}$ \\
$\square$ &  \\
\end{tabular}
\end{center}

\section{INTRODUCTION}
All companies operate in a complex ecosystem composed of interested stakeholder groups that are dependent on the company as well as each other for economic success. Key stakeholder groups include the company's capital providers, otherwise referred to as its debt and equity holders. In addition, companies have a number of other interested parties.

These stakeholder groups do not necessarily share the same goals for, nor seek the same ends from, the company. The interests of any one stakeholder group may diverge or conflict with that of others and, in some cases, with the interests of the company itself. A company's ability to maximize long-term value for shareholders and generate sufficient profitability to make its debt obligations is compromised if one stakeholder group is able to consistently extract benefits to the detriment of another group. Therefore, the controls and mechanisms to harmonize and safeguard the interests of the company's stakeholders are key areas of both interest and risk for financial analysts.

\section{2}
\section{STAKEHOLDER GROUPS}
describe a company's stakeholder groups and compare their interests

A stakeholder is any individual or group that has a vested interest in a company. The primary stakeholder groups and their roles are listed below.

In a typical company,

\begin{itemize}
  \item the shareholders and creditors provide the capital and financial resources to finance the company's activities;

  \item the board of directors serves as the steward of the company;

  \item the managers execute the strategy set by the board and run day-to-day operations;

  \item the employees provide the human capital for the company's day-to-day operations;

  \item the customers provide the demand for the company's products and services;

  \item the suppliers provide the raw inputs and the goods and services that cannot be efficiently generated internally, including functions that are outsourced; and

  \item the government and regulators dictate the rules and regulations governing the corporate entity.

\end{itemize}

Exhibit 1 illustrates the primary stakeholder groups and their interests, showing what each group provides and, in turn, receives in their relationship with the company.

\section{Exhibit 1: Key Stakeholder Groups}
\begin{center}
\includegraphics[max width=\textwidth]{2023_05_04_b5cfa4f1bc883752f121g-655}
\end{center}

Customers

\section{Shareholder vs. Stakeholder Theory}
In the traditional corporate governance framework, shareholders elect a board of directors, which in turn hires managers to serve the interests of shareholders. The interests of other stakeholders - such as creditors, employees, customers, and society more broadly - are considered only to the extent that they affect shareholder value. This is referred to as the "shareholder theory" model.

Advocates for what is known as "stakeholder theory" argue that corporate governance should consider the interests of all stakeholders, not just shareholders. Stakeholder theory gives more prominence to ESG (environmental, social, and governance) considerations by making them an explicit objective for the board of directors and management. Doing this, however, involves certain challenges:

\begin{itemize}
  \item the complexity of balancing multiple objectives

  \item a lack of clarity on how non-shareholder objectives are defined, measured, and balanced

  \item the challenges of competing globally when competitors may not face similar constraints

  \item the direct costs of adhering to higher ESG standards

\end{itemize}

Here we consider a corporation's primary stakeholder groups, such as its capital providers (e.g., shareholders and debtholders), as well as other groups. A discussion of principal-agent considerations and the conflicts that may arise among the groups also follows.

\section{Shareholders}
Shareholders, also called equity holders, own shares of stock, or equity, in a corporation. In terms of capital structure, they are residual claim holders, which means that if the company defaults or goes bankrupt, shareholders receive proceeds, if any, only after all other claims are paid. Shares typically entitle their owners to certain rights, including the exclusive right to vote on important matters, such as the composition of the board of directors, mergers, and the liquidation of assets.

\section{Introduction to Corporate Governance and Other ESG Considerations}
A shareholder's interests are typically focused on growth in corporate profitability that maximizes the value of the company's share price and dividends.

\section{Creditors/Debtholders}
\section{Banks and Private Lenders}
Banks and private lenders generally hold a company's debt to maturity. They typically have direct access to company management and non-public information about the company; in principle, this reduces information asymmetries that exist between the company and these groups. Since an individual bank or private lender can be a critical source of capital, particularly for a small- or mid-sized company, this type of lender has a degree of influence on the company. These lenders can decide to relax debt restrictions or extend more credit, or it can refuse to do either of these, which is typically more difficult in the case of public debt held by many parties.

With banks and private lenders, the general perspective of debtholders applies: less financial leverage implies less risk and is therefore preferred. Among private lenders, however, there is wide variation in their risk appetite, approach, behavior, and relationships with companies to whom they have provided capital. Some focus mainly on asset value, such as the collateral supporting their loan; some hold both debt and equity positions in the same company; some have an equity-like focus on the cash flows, value, and prospects of the business; and some lend to businesses they would be interested in owning if the borrowing company were to default on its payments.

\section{Public Debtholders}
Public debtholders (or bondholders) rely on public information and credit rating agency determinations to make their investment decisions. In return for their capital, debtholders expect to receive interest payments on a regular basis and a return of their principal at the maturity of the bond. Unlike shareholders, debtholders do not hold voting power, and they typically have limited influence over a company's day-to-day operations.

Debtholders generally seek to minimize downside risk, preferring stability in company operations and performance, which contrasts with the interests of shareholders, who are inclined to tolerate higher risks in return for higher return potential from strong company performance.

\section{Board of Directors}
A company's board of directors is elected by shareholders to protect shareholders' interests and provide strategic direction, taking into consideration the company's risk appetite, which it defines for the company. The board is also responsible for hiring the $\mathrm{CEO}$ and monitoring the performance of the company and management. A board is often composed of both inside and independent directors. Inside directors are major shareholders, founders, and senior managers, whereas independent directors do not have a material relationship with the company with regard to employment, ownership, or remuneration and are chosen because of their experience managing or directing other companies. (The terms remuneration and compensation are typically interchangeable, with compensation generally used in North America and remuneration generally used outside North America. In this reading, unless specifically identified with North America, we primarily use remuneration.)

There is no single or optimal board structure, and the number of directors may differ depending on the company's size, structure, and complexity of operations. However, most codes addressing corporate governance standards (known as "corporate governance codes") require that board members represent a diverse mix of expertise, backgrounds, and competencies, with best practice generally dictating at least one third of the board be independent. The duties of directors are mandated by law in many countries and vary across jurisdictions. Directors are required under law to display a high standard of prudence, care, and loyalty to the company.

\section{STAGGERED BOARDS}
The general practice for boards is that elections occur simultaneously and for specified terms (three years, for example). Some companies, however, have staggered boards, whereby directors are typically divided into three classes that are elected separately in consecutive years - that is, one class every year. Because shareholders would need several years to replace a full board, this election process limits their ability to effect a major change of control at the company. The positive aspect of a staggered board, though, is that it provides continuous implementation of strategy and oversight without constantly being reassessed by new board members, which otherwise risks bringing short-termism into company strategy. This practice is common in Australia and a number of European countries, including Belgium, France, the Netherlands, and Spain.

\section{Managers}
Managers, led by the chief executive officer of the company, are responsible for determining and implementing the corporation's strategy under the oversight of the board of directors. In addition, they are responsible for the smooth function of day-to-day operations.

Senior executives and other high-level managers are normally compensated through a base salary, a short-term bonus usually delivered in the form of cash, and a multi-year incentive plan delivered in one or more forms of equity (options, time-vested shares, and/or performance-vested shares). As a result, in addition to acting to protect their employment status, managers may be motivated to maximize the value of their total remuneration.

\section{Employees}
A company relies on the labor and skill, or human capital, of its employees to provide its goods and services. In return, employees typically seek fair remuneration, good working conditions, access to promotions, career development opportunities, training, job security, and a safe and healthy work environment.

Employees may have both a direct financial stake in their employer company through equity-oriented participation plans (such as profit-sharing, share purchase, or stock option plans) and a broader interest in the company's long-term stability, survival, and growth. For employees in most businesses, equity ownership is a minor part of compensation and thus less important than interest in the company's stability and growth.

\section{Customers}
Customers expect a company's products or services to satisfy their needs and provide appropriate benefits given the price paid, while meeting applicable standards of safety. Depending on the type of product or service and the duration of their relationship with the company, customers may desire ongoing support, product guarantees, and after-sale service. Given its potential correlation with sales revenue and profit, customer satisfaction is a key concern for most companies. Brand boycotts by consumers and shareholder actions in response to adverse environmental and social impacts and controversies caused by their products and services are also increasingly concerning for companies.

\section{Suppliers}
A company's suppliers are typically short-term creditors who have a primary interest in being paid as contracted in a timely manner for products or services delivered to the company. These include outsourced services provided to the company, such as information technology and payroll. When companies are in financial distress, the financial position of its suppliers may be affected, as might suppliers' willingness to extend additional credit to the company. However, suppliers' interest in the financial health of a customer company is often broad because suppliers often seek to build long-term relationships with companies for the benefit of both parties and because suppliers aim for these relationships to be fair and transparent. Like customers, suppliers typically have an interest in the company's long-term stability, particularly in the case when products are specialized and the customer or supplier has made a significant investment in the relationship - for example, through product design, retooling, training, or software customization.

\section{Governments}
Governments seek to protect the interests of the general public and ensure the well-being of their nations' economies. Because corporations have a significant effect on a nation's economic output, capital flows, employment, social welfare, and environment, for instance, regulators have an interest in ensuring that corporations behave in a manner that is consistent with applicable laws. Moreover, as the collector of tax revenues from companies and their employees, a government can also be considered one of the company's major stakeholders.

\section{STAKEHOLDERS IN NON-PROFIT ORGANIZATIONS}
The stakeholders of a non-profit organization tend to differ from those of for-profit companies. Non-profit organizations do not have shareholders. Their stakeholders most commonly include board directors or trustees, employees, clients, regulators, society, patrons of the organization, donors, and volunteers. The stakeholders of non-profit organizations are generally focused on ensuring that the organization is serving the intended cause and that donated funds are used as promised.

\section{EXAMPLE 1}
\section{Stakeholder Groups}
\begin{enumerate}
  \item Which stakeholders would most likely realize the greatest benefit from a significant increase in the market value of the company?
\end{enumerate}

A. Creditors

B. Customers

C. Shareholders

\section{Solution}
$\mathrm{C}$ is correct. Shareholders own shares of stock in the company, and their wealth is directly related to the market value of the company. A is incorrect because creditors are usually not entitled to any additional cash flows (beyond interest and debt repayment) if the company's value increases. B is incorrect because while customers may have an interest in the company's stability and long-term viability, they do not benefit directly from an increase in a company's value.

\section{PRINCIPAL-AGENT AND OTHER RELATIONSHIPS}
describe the principal-agent relationship and conflicts that may arise
between stakeholder groups

A principal-agent relationship (also known as an agency relationship) is created when a principal hires an agent to perform a particular task or service. Because the agent is expected to act in the best interests of the principal, the principal-agent relationship involves obligations, trust, and expectations of loyalty and diligence.

The relationship between shareholders and managers/directors is a classic example of a principal-agent relationship, whereby shareholders (the principal in this case) elect directors (an agent) who are expected to protect their interests by appointing senior managers (another agent) to run the company. Exhibit 2 illustrates the principal-agent relationship.

\section{Exhibit 2: Principal-Agent and Other Relationships}
\begin{center}
\includegraphics[max width=\textwidth]{2023_05_04_b5cfa4f1bc883752f121g-659}
\end{center}

Asymmetric information (an unequal distribution of information) arises from the fact that managers have more information about a company's performance and prospects (including future investment opportunities) than is made available to outsiders, such as owners and creditors. Such information asymmetry decreases the ability of shareholders to assess the true performance of manager and directors, thereby weakening their ability to vote out poor performers.

\section{Introduction to Corporate Governance and Other ESG Considerations}
Whereas all companies have a certain level of asymmetric information, companies with comparatively high asymmetry in information include those with complex products. These include high-tech companies, companies with little transparency in financial accounting information, and companies with lower levels of institutional ownership. Providers of both debt and equity capital demand higher risk premiums and returns from companies with higher asymmetry in information because there is greater potential for conflicts of interest.

Examples of principal-agent relationships and potential conflicts between the principal and agent are discussed in the following sections. Other conflicts among stakeholder groups not involving principal-agent relationships are also discussed.

\section{Shareholder and Manager/Director Relationships}
Conflicts arise where the interests of a principal and an agent diverge. In practice, compensation is the main tool used to create alignment of interests between management and board directors on the one hand and shareholders on the other. In principle, management compensation (which may include grants of shares and options to purchase shares in the company) is intended to motivate managers to work hard to maximize shareholder value. However, the alignment of interests between managers and shareholders is rarely perfect. If we consider the typical elements of management compensation, we can identify common examples of misalignment or conflicts.

\section{Entrenchment}
When the overall level of board director or manager compensation, or tenure, is excessive, the result may lead to the avoidance of risk motivated by a vested interest in keeping one's position. In such a scenario, directors may avoid speaking out against management in the interest of shareholders or other stakeholders.

\section{Empire building}
When director and management compensation are high and tied to the size of the business, it can also lead to "growth for growth's sake" in which managers are motivated to pursue acquisitions and expansion that might not increase shareholder value.

\section{Excessive risk taking}
A compensation package relying too heavily on stock grants and options can motivate risk-taking behavior by management, since option holders participate only in upside share price moves. Similarly, too little or no use of stock grants and options awarded to managers can lead to the opposite result. However, managers and directors without a meaningful equity stake in the company are typically more risk averse in their corporate decision making so they can better protect their long-term engagement by the company. This misalignment may be at odds with the company's value creation objective or with shareholder's desire for higher-risk, higher-reward endeavors.

\section{EXAMPLE 2}
\section{Shareholder and Manager/Director Relationships}
\begin{enumerate}
  \item A construction company has the opportunity to invest in a high-risk but high-reward capital infrastructure project. Which of the following could be a reason why the company decides not to pursue the project?
\end{enumerate}

A. The compensation of managers is closely tied to the size of the company's business. B. The directors receive excessive all-cash compensation.

C. The managers have recently been awarded a generous amount of options to purchase shares in the company.

\section{Solution}
B is correct. Where compensation, particularly if it is excessive, does not include an adequate amount of stock grants or options, the risk tolerance of directors and managers may be low because directors and managers may be inclined to give up taking risks that create value for the company so as to not jeopardize the compensation they have been receiving. Choice A is incorrect because this describes the "empire building" phenomenon that would likely result in the decision to grow the company at any cost in order to attempt to secure higher compensation. Choice $C$ would likely lead to an alignment of interests similar to that of shareholders and thus to a tolerance for risk.

Under the agency theory, managers are expected to undertake their duties with a central goal of serving shareholders' best interests and maximizing firm value. (Agency theory considers the problems that can arise in a business relationship when one person delegates decision-making authority to another. The traditional view in the investment community is that directors and managers are agents of shareholders. More recently, however, many legal experts have argued that in several countries, corporations are separate "legal persons"; thus, directors and managers are agents of the corporations rather than shareholders (or a subset of shareholders). See https:// \href{http://themoderncorporation.wordpress.com/company-law-memo.)}{themoderncorporation.wordpress.com/company-law-memo.)} The separation of ownership and control in corporations, however, creates diverging interests for shareholders and managers and gives rise to agency problems. Managers may attempt to exploit the firm's resources and direct its operations so as to maximize their personal benefits not limited to financial compensation, perquisites, job safety, and other benefits - to the detriment of shareholders' interests. Managers can do so through various means such as making value-destroying decisions, granting themselves excessive benefits, exploiting the company's assets and resources for their own gain, misappropriating funds, or committing other forms of fraudulent behavior.

Agency costs - the incremental costs arising from conflicts of interest when an agent makes decisions for a principal - are associated with the fact that all public companies and the larger private companies are managed by non-owners. In the context of a corporation, agency costs arise from conflicts of interest between managers, shareholders, and bondholders. "Perquisite consumption" refers to items that executives may legally authorize for themselves that have a cost to shareholders, such as subsidized dining, a corporate jet fleet, and chauffeured limousines.

The smaller the stake managers have in the company, the less their share in bearing the cost of excessive perquisite consumption - and, consequently, the less their desire to give their best efforts in running the company to maximize shareholder wealth. The costs arising from this conflict of interest have been called the agency costs of equity. Given that outside shareholders are aware of this conflict, they will monitor activities by management and the board that incur costs by taking actions such as requiring audited financial statements and holding annual meetings. Management will also take actions to assure owners they are working in the owners' best interest - actions such as using noncompete employment contracts and insurance to guarantee performance.

The better the company governance, the lower the agency costs. Good governance practices translate into higher shareholder value because managers' interests are better aligned with those of shareholders.

\section{Controlling and Minority Shareholder Relationships}
Corporate ownership structures are generally classified as dispersed, concentrated, or a hybrid of the two. Dispersed ownership reflects the existence of many shareholders, none of which have the ability to individually exercise control over the corporation. Exhibit 3 illustrates a scenario in which each of the four shareholders has $25 \%$ share ownership and voting rights and none has control.

\section{Exhibit 3: Dispersed Ownership}
\begin{center}
\includegraphics[max width=\textwidth]{2023_05_04_b5cfa4f1bc883752f121g-662}
\end{center}

In contrast, concentrated ownership reflects an individual shareholder or a group (called controlling shareholders) with the ability to exercise control over the corporation. In this context, a group is typically a family, another company (or companies), or a sovereign entity.

Share ownership alone may not necessarily reflect whether the control of a company is dispersed or concentrated. This is because controlling shareholders may be either majority shareholders (i.e., own more than $50 \%$ of a corporation's shares) or minority shareholders. (i.e., own less than $50 \%$ of shares). In some ownership structures, shareholders may have disproportionately high control of a corporation relative to their ownership stakes.

Exhibit 4 illustrates controlling majority versus minority shareholder scenarios. In the controlling majority shareholder scenario, Shareholder A holds $70 \%$ (greater than $50 \%$ ) share ownership and voting rights, and where decisions are based on a majority vote, Shareholder A will always control the outcome.

In the controlling minority shareholder scenario, Shareholder A has significant influence but not the same degree of control as in the controlling majority scenario where a majority vote is required. Nevertheless, in the controlling minority shareholder scenario, Shareholder A has control. With $40 \%$ (less than 50\%) share ownership and voting rights, Shareholder A only needs to ensure that it has the support of anything over $10 \%$ of the voting shareholders. Exhibit 4: Concentrated Ownership - Controlling Majority and Controlling Minority Shareholder

Concentrated majority shareholder (1 share $=1$ voting right)

\begin{center}
\includegraphics[max width=\textwidth]{2023_05_04_b5cfa4f1bc883752f121g-663(2)}
\end{center}

Share ownership (\%)

\begin{center}
\includegraphics[max width=\textwidth]{2023_05_04_b5cfa4f1bc883752f121g-663}
\end{center}

Shareholder $C$ Shareholder D Concentrated minority shareholder

( 1 share $=1$ voting right)

\begin{center}
\includegraphics[max width=\textwidth]{2023_05_04_b5cfa4f1bc883752f121g-663(1)}
\end{center}

Share ownership (\%)

Whareholder A Shareholder B

Shareholder $C$ Shareholder D

Shareholder E

Clearly, not all shareholders have identical financial interests, and where there are one or more large shareholders, the issuance or retirement of common shares affects voting control of the company.

In companies in which a particular shareholder holds a controlling stake, conflicts of interest may arise among the controlling and minority (without control) shareholders. In such ownership structures, the opinions of minority shareholders are often outweighed or overshadowed by the influence of the controlling shareholders. Minority shareholders often have limited or no control over management and limited or no voice in director appointments or in major transactions that could have a direct effect on the value of their shares.

For instance, in companies that adopt straight voting (i.e., one vote for each share owned), controlling shareholders clearly wield the most influence in board of director elections, leaving minority shareholders with much less representation on the board.

The decisions made by controlling shareholders, or their board representatives, could also have an effect on corporate performance and, consequently, on minority shareholders' wealth. Takeover transactions are notable situations in which controlling shareholders typically have greater influence than minority shareholders have with regard to the consideration received and other deal terms.

\section{QTEL CONSORTIUM ACQUISITION}
In 2007 Qtel, Qatar's largest telecommunications company, executed a deal with a select consortium of the shareholders of Wataniya (excluding from the deal shareholders that were not part of the consortium), Kuwait's telecommunications company, to acquire the consortium's shares in Wataniya (representing a $51 \%$ stake in the target). This consortium of Wataniya's shareholders sold their shares to Qtel at a premium of $48 \%$ on the stock price to the exclusion of minority shareholders. In addition, concentrated ownership may lead to any of the following situations:

\begin{itemize}
  \item Where controlling shareholders have multiple voting shares, they can pursue expansion strategies and acquisitions that may not increase shareholder value and can issue shares to do so, with no risk that minority shareholders or other stakeholders will block their actions. An example might be the acquisition of a sports team by a company in an unrelated business.

  \item A controlling shareholder (such as a private equity backer) seeking to sell its stake might take a very short-term view on financing - for example, resisting actions by the company to raise new capital by issuing new shares that might result in dilution of its equity stake.

  \item Conversely, some companies that are founder-led and founder-controlled benefit from the fact that founders often take a very long-term perspective on their business, while minority shareholders might not. Amazon is a high-profile example since it has consistently sacrificed short-term profitability in favor of long-term growth and market share. Founder Jeff Bezos has allowed his ownership stake to be diluted as the company has grown.

\end{itemize}

Generally speaking, corporations with publicly traded equity have a voting structure that involves one vote for each share; that is, any shareholder's voting power is equal to the percentage of the company's outstanding shares owned by that shareholder. When there are exceptions to this norm and economic ownership becomes separated from control, investors can face significant potential risks.

In a small number of markets, the local regulatory framework or exchange rules allows other alternative share-class structures, which are the most common way that voting power is decoupled from ownership. An equity structure with multiple share classes in which one class is non-voting or has limited voting rights creates a divergence between the ownership and control rights of different classes of shareholders.

\section{Dual-Share Classes}
Under a multiple-class structure (traditionally called a dual-class structure when there are two share classes), a common arrangement has one share class (for example, Class A) that carries one vote per share and those shares are publicly traded while another share class (for example, Class B) carries several votes per share and those shares are held exclusively by company insiders or family members. The company's founders, executives, and other key insiders control the company by virtue of ownership of a share class with superior voting powers.

In this situation, shareholders have unequal voting rights. For example while Shareholder A holds $55 \%$ of shares (a majority), this represents only $10 \%$ of the voting rights. Exhibit 5 illustrates this scenario of unequal voting rights where one share may have greater or less than one voting right attached to it.

\section{Exhibit 5: Unequal Voting Rights}
\begin{center}
\includegraphics[max width=\textwidth]{2023_05_04_b5cfa4f1bc883752f121g-665}
\end{center}

The multiple-class structure enables controlling shareholders, usually founders or insiders, to mitigate dilution of their voting power when new shares are issued and to continue to control board elections, strategic decisions, and all other significant voting matters for a long period - even once their ownership level declines to less than $50 \%$ of the company's shares. Examples of companies that have adopted multiple-class stock structures are Alibaba and Facebook (each with two share classes).

\section{EXAMPLE 3}
\section{Dual-Share Class Structure}
\begin{enumerate}
  \item Which of the following best describes dual-class share structures?
\end{enumerate}

A. Dual-class share structures can be easily changed over time.

B. Company insiders can maintain significant power over the organization.

C. Conflicts of interest between management and stakeholder groups are less likely than with single-share structures.

\section{Solution}
B is correct. Under dual-class share systems, company founders or insiders may control board elections, strategic decisions, and other significant voting matters. A is incorrect because dual-share systems are virtually impossible to dismantle once adopted. $\mathrm{C}$ is incorrect because conflicts of interest between management and stakeholders are more likely than with single-share structures because of the potential control element under dual systems.

\section{Manager and Board Relationships}
Conflicts of interest between the management and the board of directors can occur in cases where managers seek to extract private benefits or where they tend to increase information asymmetries by withholding relevant information from the board with the purpose of hindering its ability to exercise proper monitoring.

\section{Shareholder vs. Creditor (Debtholder) Interests}
The conflict between shareholders and creditors, or debtholders, reflects their different positions in the capital structure and the different structure of risks and returns for each. Debtholders have a contractual and prior claim to cash flows and firm assets over shareholders. For a holder of debt to maturity, the return or upside is prescribed and limited, while the risk, or downside, is not. In other words, the risk-return profile for debtholders is very asymmetric.

In contrast, shareholders typically have more downside risk but much higher upside return potential. As a result, debtholders favor decisions that reduce a company's leverage and financial risk, whereas shareholders often prefer higher leverage levels that offer them greater return potential. A divergence in risk tolerance regarding the company's investments thus exists between shareholders and debtholders. The potential debt/equity conflict is greater in the case of long-term rather than short-term debt because the passage of time exposes debtholders to possible changes in business conditions, strategy, and management behavior.

Debtholders may also find their interests jeopardized when the company attempts to increase its borrowings to a level that would increase default risk. If the company's operations and investments fail to generate sufficient returns required to repay the increased interest and debt obligations, creditors will be increasingly exposed to default risk. The distribution of excessive dividends to shareholders might also conflict with debtholders' interests if it impairs the company's ability to pay interest and principal. To prevent this, creditors may require covenants to limit increased leverage or dividend payouts.

The company's activities may also deviate from creditors' interests when it attempts to increase its borrowings to a level that would drive default risk upward. In some cases, the excessive borrowings might be driven by fraudulent behavior by shareholders or managers. In other cases, it might aim at financing risky operations or investments. If such investments fail to generate the returns required to pay the large interest and debt obligations, creditors will be exposed to default risk.

\section{EXAMPLE 4}
\section{Stakeholder Relationships}
\begin{enumerate}
  \item A controlling shareholder of XYZ Company owns $55 \%$ of XYZ's shares, and the remaining shares are spread among a large group of shareholders. In this situation, conflicts of interest are most likely to arise between:
\end{enumerate}

A. shareholders and bondholders.

B. the controlling shareholder and managers.

C. the controlling shareholder and minority shareholders.

\section{Solution}
$\mathrm{C}$ is correct. In this ownership structure, the controlling shareholder's power is likely more influential than that of minority shareholders. Thus, the controlling shareholder may be able to exploit its position to the detriment of the interests of the remaining shareholders. Choices $\mathrm{A}$ and $\mathrm{B}$ are incorrect because the ownership structure in and of itself is unlikely to create material conflicts between shareholders and regulators or shareholders and managers.

\section{EXAMPLE 5}
\section{Leverage and Other Stakeholders - Mainly Metal Boats, Inc.}
Mainly Metal Boats is a small, debt-free manufacturer of welded metal boats and is located in a remote small town. It sells both domestically and internationally in a market that is highly competitive and cyclical. Sales are conducted through a network of dealers, who typically sell three or four different boat brands. Of the company's 50 employees, about half are specialized aluminum welders, with most others in sales and management. The primary purchased input for Mainly's boat production is sheet aluminum. Mainly's sole aluminum supplier is a large multinational company that has many clients. Mainly has not paid dividends previously and now has substantial retained earnings that finance the company's working capital. Mainly's owners have recently decided to borrow heavily to finance working capital so they can pay a large, one-time dividend from the retained earnings.

\begin{enumerate}
  \item Which of the following stakeholder groups is likely to be most negatively affected by the increase in leverage?
\end{enumerate}

A. The welders employed by the company

B. The company's dealers

C. The supplier of aluminum to the business

\section{Solution}
A is correct. As employees, the welders could face loss of employment if the company were to become financially distressed with the increase in leverage, and since their skills are very specialized, they would probably have difficulty finding another job locally. In a small and remote town, employment opportunities are likely to be limited for specialized workers.

$\mathrm{B}$ is incorrect. The dealers might suffer lost sales if Mainly were to fail, but they could likely replace Mainly with a competing brand.

$\mathrm{C}$ is incorrect. The aluminum supplier would probably suffer the least impact since it is large and Mainly is not likely a large proportion of its sales

\begin{enumerate}
  \setcounter{enumi}{1}
  \item Is it likely that any of these groups would be impacted positively?
\end{enumerate}

\section{Solution}
In all cases, impacts are negative. Note that for modest borrowing, these effects would be very minor.

\section{CORPORATE GOVERNANCE AND MECHANISMS TO MANAGE STAKEHOLDER RISKS}
describe corporate governance and mechanisms to manage stakeholder relationships and mitigate associated risks Corporate governance practices differ among countries and jurisdictions, and even within countries different corporate governance systems may coexist. These differences reflect unique economic, political, social, legal, and other forces in each country and/ or region. Notwithstanding these variations or the lack of a common definition, there is universal consensus that effective corporate governance is essential for the proper functioning of a market economy. There is evidence that some movement toward a global convergence of corporate governance systems is underway. One trend is the increased acceptance and adoption of corporate governance regulations with similar principles from one jurisdiction to another.

Notably, the Organisation for Economic Co-operation and Development (OECD) affirmed that "The presence of an effective corporate governance system, within an individual company and across an economy as a whole, helps to provide a degree of confidence that is necessary for the proper functioning of a market economy. As a result, the cost of capital is lower and firms are encouraged to use resources more efficiently, thereby underpinning growth" (OECD 2004).

Corporate governance is the arrangement of checks, balances, and incentives that exists to manage conflicting interests among a company's management, board, shareholders, creditors, and other stakeholders. Sound corporate governance practices are essential to ensuring sound capital markets and the stability of the financial system; weak corporate governance is a common thread found in many company failures. The assessment of a company's corporate governance system, including consideration of conflicts of interest and transparency of operations, has increasingly become an essential factor in the investment decision-making process.

\section{EXAMPLE 6}
\section{Corporate Governance Overview}
\begin{enumerate}
  \item Which statement regarding corporate governance is most accurate?
\end{enumerate}

A. Most countries have similar corporate governance regulations.

B. A single definition of corporate governance is widely accepted in practice.

C. Both shareholder theory and stakeholder theory consider the needs of a company's shareholders.

\section{Solution}
$\mathrm{C}$ is correct. Both shareholder and stakeholder theories consider the needs of shareholders, with the latter extending to a broader group. A is incorrect because corporate governance regulations differ across countries, although there is a trend toward convergence. B is incorrect because a universally accepted definition of corporate governance remains elusive.

Given differences in the interests of stakeholder groups, a company's governance and stakeholder management practices attempt to manage these interests to minimize potential stakeholder conflict and risk to the organization. Stakeholder management involves identifying, prioritizing, and understanding the interests of stakeholder groups, and it aims at laying the framework through which they can exercise adequate levels of influence and control and protect their interests in the company. For such balance to be maintained, corporate governance sets upon an underlying legal, contractual, and organizational infrastructure that defines the rights, responsibilities, and powers of each group and employs specific stakeholder management mechanisms.

\section{Shareholder Mechanisms}
As residual owners of the company, shareholders have legal and contractual rights in the company (ownership rights) and are driven to secure those rights through a variety of mechanisms (control rights) to protect their ownership interest and control in the company.

A prescribed, or standard, set of rights and mechanisms does not exist across all companies, and the principles vary across countries and jurisdictions. Yet there are some basic ownership rights granted to shareholders and common control rights and mechanisms available to them in corporations around the globe. These are presented in this section but should not be viewed as comprehensive or standard.

\section{Corporate Reporting and Transparency}
Shareholders have access to a range of financial and non-financial information concerning the company, typically through annual reports, proxy statements, disclosures on the company's website, the investor relations department, and other means of communication (e.g., social media). This information may relate to the company's operations, its strategic direction or objectives, audited financial statements, governance structure, ownership structure, remuneration policies, related-party transactions, and risk factors. Such information is essential for shareholders to

\begin{itemize}
  \item reduce the extent of information asymmetry between shareholders and managers;

  \item assess the performance of the company and that of its directors and managers;

  \item make informed decisions in valuing the company and deciding to purchase, sell, or transfer shares; and

  \item vote on key corporate matters or changes.

\end{itemize}

\section{Shareholder Meetings}
General meetings, also termed annual general meetings (AGMs) because they are typically held once a year, enable shareholders to participate in discussions and vote on major corporate matters and transactions that are not delegated to the board of directors. An extraordinary general meeting (EGM) can be called by the company or shareholders throughout the year when other resolutions requiring shareholder approval are proposed. Companies are required to give shareholders the right to call a general meeting, subject to conditions of a specified minimum number of calling shareholders or a minimum proportion of outstanding stock held by these shareholders.

The matters present for shareholder vote vary across jurisdictions and companies, but the most basic and common ones relate to

\begin{itemize}
  \item election of board members,

  \item approval of annual financial statements,

  \item approval of proposed dividend distributions,

  \item approval of director compensations,

  \item approval of independent auditor and auditor compensation, and

  \item material corporate changes, such as the following:

  \item amendments to the company's by-laws and articles of association

  \item transactions, mergers and acquisitions, takeovers, sale of significant corporate assets

  \item increases in the company's capital - implementation of shareholder rights plans

  \item voluntary liquidation of the firm

\end{itemize}

Proxy voting is a process that enables shareholders who are unable to attend a meeting to authorize another individual to vote on their behalf. Proxy voting is the most common form of investor participation in general meetings. Although most resolutions at most companies pass without controversy, sometimes minority shareholders attempt to strengthen their influence at companies via proxy voting. Several shareholders might opt to use this process to collectively vote their shares in favor of or in opposition to a certain resolution.

Cumulative voting (as opposed to straight voting) enables shareholders to accumulate and vote all their shares for a single candidate in an election involving more than one director. This voting process raises the likelihood that minority shareholders are represented by at least one director on the board, but it may not be compatible with majority voting standards for director elections in which share ownership is widely dispersed. In terms of worldwide practice, the existence of cumulative voting varies; for example, it is mandated in Spain but not allowed in several countries, such as Germany, Japan, Singapore, and Turkey.

\section{EXAMPLE 7}
\section{Shareholder Meetings}
\begin{enumerate}
  \item Which of the statements about extraordinary general meetings (EGMs) of shareholders is true?
\end{enumerate}

A. The appointment of external auditors occurs during the EGM.

B. A corporation provides an overview of corporate performance at the EGM.

C. An amendment to a corporation's bylaws typically occurs during the EGM.

\section{Solution}
$\mathrm{C}$ is correct. An amendment to corporate bylaws would normally take place during an EGM, which covers significant changes to a company, such as bylaw amendments. A and B are incorrect because the appointment of external auditors and a corporate performance overview would typically take place during the AGM, not the EGM.

\section{Shareholder Activism}
Shareholder activism refers to strategies used by shareholders to attempt to compel a company to act in a desired manner. Although shareholder activism can focus on a range of issues, including those involving social, political, or environmental considerations, the primary motivation of activist shareholders is to increase shareholder value. Activism can also take the form of direct corporate engagement and stewardship to promote positive corporate action. Activist shareholders often pressure management through such tactics as initiating proxy battles (fights), proposing shareholder resolutions, and publicly raising awareness on issues of contention.

Hedge funds are among the most predominant shareholder activists. Unlike most traditional institutional investors, hedge funds have a fee structure that often provides a significant stake in the financial success of any activist campaign. Furthermore, unlike regulated investment entities, such as mutual funds, that are typically subject to restrictions on their investments (e.g., limitations on leverage or ownership of distressed or illiquid securities), hedge funds are more lightly regulated and can thus pursue a greater range of activist opportunities.

\section{EXAMPLE 8}
\section{Activist Shareholders}
\begin{enumerate}
  \item Which of the following best describes activist shareholders?
\end{enumerate}

A. Activist shareholders help stabilize a company's strategic direction.

B. Activist shareholders have little effect on the company's long-term investors.

C. Activist shareholders can alter the composition of a company's shareholder base.

\section{Solution}
$C$ is correct. The presence of activist shareholders can create substantial turnover in the company's shareholder composition. A is incorrect because the presence of activist shareholders can materially change a company's strategic direction. $B$ is incorrect because long-term investors in a company need to consider how activist shareholders affect the company.

\section{Shareholder Derivative Lawsuits}
Shareholder activists may pursue additional tactics, such as shareholder derivative lawsuits, which are legal proceedings initiated by one or more shareholders against board directors, management, and/or controlling shareholders of the company. The theory behind this type of lawsuit is that the plaintiff shareholder is deemed to be acting on behalf of the company in place of its directors and officers, who have failed to adequately act for the benefit of the company and its shareholders. With shareholder derivative lawsuits, minority shareholders have some protection in a dual-share class structure. However, in markets with dual-share classes where derivative lawsuits are illegal, minority shareholders have little protection.

In many countries, however, the law restricts shareholders from pursuing legal action via the court system - in some cases, by imposing thresholds that enable only shareholders with interests above a minimum amount to pursue legal actions or by denying legal action altogether.

\section{Corporate Takeovers}
The traditional view of the market for corporate control (often known as the takeover market) is one in which shareholders of a company hire and fire management to achieve better resource utilization. Corporate takeovers can be pursued in several different ways. One mechanism is the proxy contest (or proxy fight). In a proxy contest, shareholders are persuaded to vote for a group seeking a controlling position on a company's board of directors.

Managerial teams can also be displaced through tender offers and hostile takeovers, which seek to control a company through control of the board and thus management. A tender offer involves an offer to shareholders to sell their interests directly to the group seeking to gain control. A contest for corporate control may attract arbitrageurs and takeover specialists, who facilitate transfers of control by accumulating long positions from existing shareholders in the target company and later selling the positions to the highest bidder. A hostile takeover is an attempt to acquire the company without consent of the company's management. Preservation of their employment status serves as an incentive for board members and managers to focus on shareholder wealth maximization. This threat of removal, however, can also have negative implications for a company's corporate governance practices if the company chooses to adopt anti-takeover measures, such as a staggered board or a shareholder rights plan (also known as a poison pill) to reduce the likelihood of an unwanted takeover. Staggering director elections can dilute the value of shareholder voting rights by extending the term that each director serves and eliminating the ability of shareholders to replace the entire board at any given election. Shareholder rights plans enable shareholders to buy additional shares at a discount if another shareholder purchases a certain percentage of the company's shares. These plans are designed to increase the cost to any bidder seeking to take over a company.

\section{Creditor Mechanisms}
To protect their economic interests in a company, creditors have a number of available mechanisms. The rights of creditors are established by laws and according to contracts executed with the company. Laws vary by jurisdiction but commonly contain provisions to protect creditors' interests and provide legal recourse.

\section{Bond Indenture}
The rights of bondholders are established through contracts executed with the company. A bond indenture is a legal contract that describes the structure of a bond, the obligations of the company, and the rights of the bondholders. To limit bondholders' risk during the term of a bond (or loan), the bond indenture typically contains covenants, which are the terms and conditions of lending agreements, enabling creditors to specify the actions an issuer is obligated to perform or prohibited from performing. Affirmative, or positive, covenants require the company to perform certain actions or meet certain requirements, such as maintaining adequate levels of insurance. Restrictive, or negative, covenants require bond issuers to not perform certain actions, such as allowing the company's liquidity level to fall below a minimum coverage ratio. Collaterals are another tool often used by bondholders to guarantee repayment, representing assets or financial guarantees that are pledged by an issuer to secure its promise to repay its obligations.

\section{Corporate Reporting and Transparency}
To further protect their rights, bondholders usually require the company to provide periodic information (including financial statements) to ensure that covenants are not violated and thus potential default risk is not increased. Because it is usually impractical and costly for individual bondholders to fully scrutinize a bond issue, companies often hire a financial institution to act as a trustee and monitor the company on behalf of the bondholders.

\section{Creditor Committees}
In some countries, official creditor committees - particularly for unsecured bondholders - are established once a company files for bankruptcy. Such committees are expected to represent bondholders throughout the bankruptcy proceedings and protect bondholder interests in any restructuring or liquidation.

Where a company is struggling to meet its obligations under an indenture, ad-hoc committees may be formed by a group of bondholders to discuss with the company potential options to restructure their bonds. While members of the ad-hoc committees are not representative of the other bondholders as a whole, their interests are often aligned with the broader bondholder group.

\section{Board of Director and Management Mechanisms}
The board is a central component of a company's governance structure. In addition to ensuring the company has appropriate audit control and enterprise risk management systems in place, the board also has the responsibility to review any proposals for corporate transactions or changes - such as major capital acquisitions, divestments, mergers, and acquisitions - before they are referred to shareholders for approval, if applicable.

\section{Board Committees}
To fulfill their duties, boards establish committees to which they delegate specific functions within their areas of specialization. The committees are responsible for thoroughly considering, reviewing, monitoring, and following up on matters falling within their mandates, which may require specific expertise or independence. The committees provide recommendations and reporting to the board on a regular basis. When establishing committees, boards do not delegate their ultimate responsibility, nor are they discharged of their liabilities. The board is required to review, challenge, and assure the content of any reports raised to it by the committees and make the proper decisions or actions.

The most commonly established board committees are shown in Exhibit 6.

\section{Exhibit 6: Common Board Committees and Key Oversight Functions}
\begin{center}
\includegraphics[max width=\textwidth]{2023_05_04_b5cfa4f1bc883752f121g-673}
\end{center}

Audit Committee

The audit committee is the most widely required and most commonly established committee of the board. The audit committee plays a key role in overseeing the audit and control systems at the company and ensuring their effectiveness. Best practice for audit committee composition is for all members to be independent.

The audit committee monitors the financial reporting process, including the application of high standard accounting policies, and ensures the integrity of the financial statements. It supervises the internal audit function or department and ensures its independence and competence. It presents an annual audit plan to the board and monitors its implementation by the internal audit function. It examines an annual review of the audit and control systems, ensures their effectiveness, and proposes necessary actions. The audit committee is also responsible for recommending the appointment of a competent and independent external auditor and proposing its remuneration. It interacts and holds meetings with the external auditor. It receives the reports raised to it by the internal and external auditors, proposes remedial action for highlighted issues or matters, and follows up on them. In some cases, the audit committee may also oversee information technology security.

In summary, key oversight functions of the audit committee include

\begin{itemize}
  \item monitoring the financial reporting process;

  \item supervising the internal audit function, annual audit plan, and annual review; and

  \item appointing and interacting with an external auditor as well as implementing remediation per that auditor.

\end{itemize}

\section{Governance Committee}
The main role of the board's governance committee is to ensure that the company adopts good corporate governance structures and practices. For this purpose, it oversees the development of the governance policies at the company such as

\begin{itemize}
  \item the corporate governance code,

  \item the charter of the board and its committees,

  \item the code of ethics, and

  \item the conflict of interest policy, among others.

\end{itemize}

The committee reviews these policies on a regular basis to incorporate any new regulatory requirements or relevant developments in the field. Most importantly, it monitors implementation of the governance policies standards and compliance with the applicable laws and regulations throughout the firm and then recommends proper action if any flaws or breaches are identified.

Developing and implementing policies for related-party transactions and other conflicts of interest are increasingly common among companies. These policies establish the procedures for mitigating, managing, and disclosing such cases. Typically, directors and managers are required to disclose any actual or potential, or direct or indirect, conflict of interest they have with the company, as well as any material interests in a transaction that may affect the company.

In addition to overseeing the board election process, the governance committee may also oversee an annual evaluation of the board to ensure that its functioning and activities are aligned with the governance principles. In terms of the governance structure, the committee ensures that the company's organizational structure clarifies the distribution of responsibilities and authorities and that it allows for smoothness and efficiency in operations while maintaining adequate levels of control. The committee ensures that the composition of the board and its committees is aligned with the governance principles and that new board members receive appropriate training to fulfill their roles effectively.

In summary, key oversight functions of the governance committee include

\begin{itemize}
  \item developing and monitoring corporate governance policies and practices,

  \item ensuring organizational compliance and remediation with applicable laws and regulations, and

  \item aligning organizational structure with governance principles. Remuneration or Compensation Committee

\end{itemize}

The remuneration committee of the board specializes in compensation matters. It develops and proposes the remuneration policies for the directors and key executives. It may also be involved in handling the contracts of managers and directors, as well as in setting performance criteria and evaluating the performance of managers. The responsibilities of the remuneration committee may extend to setting the company's human resources policies, particularly those policies related to pay packages and compensations of employees.

\section{Remuneration Plans}
Executive remuneration plans have gained significant attention in the investment world, with a primary goal of aligning the interests of managers with those of shareholders. For this purpose, incentive plans increasingly include a variable component - typically profit sharing, stocks, or stock options - that is contingent on corporate or stock price performance. The granting of stock-based remuneration does not serve its purpose, however, if managers can improve their personal gains at the expense of the company while limiting their exposure to weak stock performance.

As a result, companies are increasingly designing incentive plans that discourage either "short-termism" or excessive risk taking by managers. Some incentive plans include granting shares, rather than options, to managers and restricting their vesting or sale for several years or until retirement. A long-term incentive plan delays the payment of remuneration, either partially or in total, until company strategic objectives (typically performance targets) are met. Overall, remuneration packages that are designed to prevent managers from chasing short-term profits are likely to be most effective and aligned with long-term shareholder interests.

Given the role of remuneration plans in aligning the interests of executives with those of shareholders, both regulators and companies are increasingly seeking shareholder views on pay. The concept of say on pay enables shareholders to vote on executive remuneration matters. By allowing shareholders to express their views on remuneration, companies can limit the discretion of directors and managers in granting themselves excessive or inadequate remuneration. Best practice is for the majority of remuneration committee members to be independent.

In summary key oversight functions of the remuneration committee include

\begin{itemize}
  \item developing director and executive remuneration policies,

  \item overseeing performance policy management and evaluation, and

  \item setting human resources (HR) policies relating to employee compensation.

\end{itemize}

\section{Nomination Committee}
The nomination committee is concerned with the nomination of directors to the board and the election process. This committee may also identify candidates for senior leadership roles, such as the CEO. It sets the nomination procedures and policies, including the criteria for board directorship, the search for and identification of qualified candidates for board directorships, and the election process by shareholders. Most importantly, it oversees the complete election process and ensures its transparency.

In designing its policies and nominating candidates to the board, the committee ensures that the structure of the board is balanced and maintains alignment with the governance principles. It recommends a definition for director independence and ensures on a continuous basis that the independent members of the board remain so.

In summary, key oversight functions of the nomination committee include

\begin{itemize}
  \item overseeing the director nomination and board election process,

  \item identifying senior leadership candidates, and

  \item maintaining board composition and independence.

\end{itemize}

\section{Risk Committee}
The risk committee assists the board in determining the risk profile and appetite of the company and in ensuring the company has an appropriate enterprise risk management system in place whereby risks are identified, mitigated, assessed, and managed appropriately. Accordingly, it oversees the setting of the risk policy and risk management annual plans and monitors their implementation. It supervises the risk management and control functions in the company, receives regular reports, and reports on its findings and recommendations to the board. In fulfilling its responsibilities, the committee ensures that the company's activities are aligned with its risk profile and that risks are mitigated or identified and well managed.

In summary, key oversight functions of the risk committee include

\begin{itemize}
  \item determining company risk profile,

  \item ensuring appropriate enterprise risk management, and

  \item aligning corporate activities with risk appetite.

\end{itemize}

\section{Investment Committee}
The investment committee of the board reviews the major investment opportunities proposed by the management and considers their viability. Such opportunities may include large projects, acquisitions, and expansion plans, as well as divestures or major asset disposals, among others. In studying an opportunity and making its recommendations to the board, the committee investigates and considers such factors as the projected financials and the expected value creation to the company, the alignment of the investment with the company's strategic direction and its risk profile, the underlying risks, the proposed financing of the project, and other quantitative and qualitative factors. In doing so, the committee challenges, where necessary, the management assumptions underlying the investment prospects.

In summary, key oversight functions of the investment committee include

\begin{itemize}
  \item assessing of major investment opportunities, and

  \item evaluating board investment recommendations.

\end{itemize}

\section{Employee Mechanisms}
By managing its relationships with its employees, a company seeks to comply with employees' rights and mitigate legal or reputational risks in violation of these rights. Managing employee relationships also helps ensure that employees are fulfilling their responsibilities toward the company and are qualified and motivated to act in the company's best interests.

\section{Labor Laws}
Employee rights are primarily secured through labor laws, which define the standards for employees' rights and responsibilities and cover such matters as labor hours, pension and retirement plans, hiring and firing, and vacation and leave. In most countries, employees have the right to create unions. Unions seek to influence certain matters affecting employees. Although this not a common practice in many other parts of the world, in some countries (e.g., Germany, Austria, and Luxembourg) employees are sometimes represented on the board of directors - or on supervisory boards - of companies meeting certain size or ownership criteria.

\section{Employment Contracts}
At the individual level, employment contracts specify an employee's various rights and responsibilities. Some companies have employee stock ownership plans (ESOPs) to help retain employees and further align their interests with that of the company. As part of an ESOP, a company establishes a fund consisting of cash and/or company shares. The shares, which have designated vesting periods, are granted to employees who, because of their compensation package, have economic interests similar to that of shareholders.

\section{Customer and Supplier Mechanisms}
A company's customers and suppliers enter into contractual agreements that specify the products and services underlying the relationship, the prices or fees and the payment terms, the rights and responsibilities of each party, the after-sale relationship, and any guarantees. Contracts also specify actions to be taken and recourse available if either party breaches the terms of the contract.

Social media has become a powerful tool that customers, owners, and other stakeholders have increasingly used to voice or protect their interests or to enhance their influence on corporate matters. For example, negative media attention can adversely affect the reputation or public perception of a company or its managers and directors. Through social media, these stakeholders can instantly broadcast information with little cost or effort and are thus better able to compete with company management in influencing public sentiment.

\section{Government Mechanisms}
\section{Regulations}
As part of their public service roles, governments and regulatory authorities develop laws that companies must follow and monitor companies' compliance with these laws. Such laws may address or protect the rights of a specific group, such as consumers or the environment. Industries or sectors whose services, products, or operations are more likely to endanger the public or specific stakeholders' interests are usually subject to a more rigorous regulatory framework. Examples of these industries are banks, food manufacturers, and health care companies.

\section{Corporate Governance Codes}
Many regulatory authorities have also adopted corporate governance codes that consist of guiding principles for publicly traded companies. These codes require companies to disclose their adoption of recommended corporate governance practices or explain why they have not done so. In some jurisdictions, companies are required to go beyond this "comply or explain" approach. In Japan, for example, companies with no outside directors must justify why appointing outside directors is not appropriate. Some jurisdictions do not have national corporate governance codes but make use of company law or regulation (e.g., Chile) or stock exchange listing requirements (e.g., India) to achieve similar objectives.

\section{EXAMPLE 9}
\section{Stakeholder Management}
\begin{enumerate}
  \item Which of the following is not typically used to protect creditors' rights?
\end{enumerate}

A. Proxy voting

B. Collateral to secure debt obligations

C. The imposition of a covenant to limit a company's debt level

\section{Solution}
A is correct. Proxy voting is a practice adopted by shareholders, not creditors. B and C are incorrect because both collateral and covenants are used by creditors to help mitigate the default risk of a company.

\section{EXAMPLE 10}
\section{Responsibilities of Board Committees}
\begin{enumerate}
  \item A primary responsibility of a board's audit committee does not include:
\end{enumerate}

A. the proper application of accounting policies.

B. the adoption of proper corporate governance.

C. the recommendation of remuneration for external auditors.

\section{Solution}
$\mathrm{B}$ is correct. The adoption of proper corporate governance is the responsibility of a corporation's governance committee. Both A and C are incorrect because proper application of accounting policies and the remuneration of external auditors fall under the domain of the audit committee.

\section{EXAMPLE 11}
\section{Shareholder Activism}
\begin{enumerate}
  \item Which of the following is true of shareholder activism?
\end{enumerate}

A. Shareholder activists rarely include hedge funds.

B. Regulators play a prominent role in shareholder activism.

C. A primary goal of shareholder activism is to increase shareholder value.

\section{Solution}
$\mathrm{C}$ is correct. Although the subject of shareholder activism may involve social and political issues, activist shareholders' primary motivation is to increase shareholder value. A is incorrect because hedge funds commonly serve as shareholder activists. B is incorrect because regulators play a prominent role in setting standards, not shareholder activism.

\section{Common Law vs. Civil Law Systems}
The legal environment in which a company operates can significantly influence the rights and remedies of stakeholders. Countries that have a common law system (such as the United Kingdom, the United States, India, and Canada) are generally considered to offer superior protection of the interests of shareholders and creditors relative to those that have adopted a civil law system (such as France, Germany, Italy, and Japan).

The key difference between the two systems lies in the ability of a judge to create laws. In civil law systems, laws are created primarily through statutes and codes enacted by the legislature. The role of judges is generally limited to rigidly applying the statutes and codes to the specific case brought before the court.

In contrast, in common law systems, laws are created both from statutes enacted by the legislature and by judges through judicial opinions. In common law systems, shareholders and creditors have the ability to appeal to a judge to rule against management actions and decisions that are not expressly forbidden by statute or code, whereas in civil law systems this option is generally not possible.

Regardless of a country's legal system, creditors are generally more successful in seeking remedies in court to enforce their rights than shareholders are because shareholder disputes often involve complex legal theories, such as whether a manager or director breached a duty owed to shareholders. In contrast, disputes involving creditors, such as whether the terms of an indenture or other debt contract have been breached, are more straightforward and therefore more easily determinable by a court.

\section{CORPORATE GOVERNANCE AND STAKEHOLDER MANAGEMENT RISKS AND BENEFITS}
describe both the potential risks of poor corporate governance and stakeholder management and the benefits from effective corporate governance and stakeholder management

Depending on their nature and magnitude, unmanaged conflicts of interest with weaknesses in stakeholder management mechanisms or the adoption of poor governance structures with weak control over a company's operations can create various risks for a company and its stakeholders. These include legal, regulatory, reputational, and financial default risks. A weak control environment can encourage misconduct and hinder the ability of the company to identify and manage its risks.

In contrast, the development of good governance practices can play a vital role in both aligning the interests of managers and the board of directors with those of shareholders and balancing the interests of the company's other stakeholders. By adopting effective guidelines and instituting adequate levels of control, corporate governance can be reflected in better company relationships, operational efficiency, improved control processes, better financial performance, and lower levels of risk.

\section{Operational Risks and Benefits}
\section{Weak Control Systems, Ineffective Decision-Making}
Organizations with higher inherent risks - financial institutions, for example - require stronger controls so that residual risks are low or moderate. In a company with weak control systems or inefficient monitoring tools, such as poor audit procedures or insufficient scrutiny by the board, one stakeholder group may benefit at the expense of the company or other stakeholders. This can result in an adverse effect on the company's resources, performance, and value.

When the quality and quantity of information available to managers are superior to those available to the board or shareholders and sufficient monitoring tools are absent, managers have an opportunity to make decisions that benefit themselves relative to the company or shareholders. Without proper scrutiny, such practices might go unnoticed. Deficient decisions could include managing the company with a lower risk profile relative to shareholders' tolerance, thus avoiding investment opportunities that could create value for the company. Conversely, manager overconfidence may result in poor investment decisions without proper examination of their effect on the company or on shareholders' wealth.

\section{EXAMPLE 12}
\section{Theranos Inc.}
The rise and fall of Theranos Inc., a US-based blood testing technology company, is illustrative of numerous corporate governance failures. The company and its founder and CEO, Elizabeth Holmes, hoped to revolutionize the health care industry with breakthrough technology that could cheaply and rapidly identify numerous human conditions based on a simple blood test.

In 2014 Theranos was valued at US $\$ 10$ billion. Its board was composed of highly recognizable and influential directors (including former US Secretaries of State and former directors of the US Centers for Disease Control and Prevention), giving the company and Holmes the credibility to raise hundreds of millions of dollars from investors.

In 2015, questions began to surface publicly around the company's blood testing technology. Whistleblowers came forward voicing concerns about questionable testing practices within the company. Soon after, it became clear that the technology and results promoted by the company and Holmes had serious issues. In 2018, the company, its CEO (Holmes), and its COO (Ramesh Balwani) were charged with "massive fraud" by the US Securities and Exchange Commission (SEC). Facing numerous criminal investigations and civil lawsuits, Theranos closed down.

The numerous corporate governance failures at Theranos largely relate to inadequate board composition and oversight.

\begin{itemize}
  \item While the board was composed of highly accomplished and well-known individuals, most had little to no knowledge of medical technology and were therefore not able to detect the fraud being perpetuated by Holmes.

  \item Given the lack of medical technology understanding, the board should have hired an independent consultant to validate the innovative technology being promoted, but it failed to do so.

  \item The board failed to raise conflict of interest concerns regarding both Holmes' romantic relationship with Balwani and his senior position at the company despite his lack of relevant industry experience.

  \item The board dismissed fraud allegations brought forward by whistleblowers and remained silent even after these whistleblowers were fired soon after making their allegations. In addition, the company failed to have individuals operating in crucial roles such as chief financial officer, global compliance officer, and other key positions for most of its life.

\end{itemize}

\section{(Adequate) Scrutiny and Control}
Strong governance practices institute more efficient procedures for scrutiny and control at all corporate levels, starting at the level of shareholders and moving up to management and the board of directors. These mechanisms allow for the mitigation of risk factors as well as fraudulent activities or for their identification and control at early stages, thus preventing them from hindering corporate performance and reputation. The control is enhanced by the proper functioning of the audit committee and the effectiveness of the audit systems in ensuring the accuracy and fairness of financial reporting at the company, promoting transparency, and resolving any matters of concern. By having procedures for monitoring compliance with internal and external policies and regulations and for reporting any violations, the firm can mitigate the risks of being exposed to regulatory questioning or legal proceedings and their associated costs.

Additionally, the adoption of formal procedures for dealing with conflict of interest and related party transactions allows the company to ensure fairness in the deal terms and avoid the hidden costs that could be associated with any preferential or unfair treatment in favor of the related party.

\section{(Improved) Operating Performance}
In a good governance setting, an organizational structure would clarify the delegation of authorities and reporting lines across the company and ensure that all employees have a clear understanding of their respective responsibilities. In addition, the governance, risk, and compliance (GRC) functions in the organization would work together to ensure an effective alignment of interests. These arrangements lead to smoother and improved decision-making processes and provide managers with the flexibility they need when responding to opportunities and challenges in a constantly changing environment.

Nonetheless, the adequate internal control mechanisms are an equally important pillar of the organizational and governance structures as they aim at ensuring that decisions and activities are properly monitored and controlled to prevent risks from arising and to circumvent misconduct or abusive behaviors. These mechanisms improve the operational efficiency of the company. Similarly, when the board exercises its role in defining the risk profile of the firm, setting its strategic direction, and supervising its implementation, then the managerial decisions and the firm operations will be better aligned with the interests of shareholders, thus paving the way for better operational results.

\section{Legal, Regulatory, or Reputational Risks and Benefits}
Compliance weaknesses in the implementation of regulatory requirements or lack of proper reporting practices may expose the company to legal, regulatory, or reputational risks. In such cases, the company may become subject to investigation by government or regulatory authorities for violation of applicable laws. A company could also be vulnerable to lawsuits filed by shareholders, employees, creditors, or other parties for breach of contractual agreements or company bylaws or for violation of stakeholders' legal rights. Improperly managed conflicts of interest or governance failures could bring reputational damage to the company, and its associated costs could be significant. Such risks are particularly acute for publicly listed companies subject to scrutiny by investors, analysts, and other market participants.

\section{EXAMPLE 13}
\section{Diesel Emissions ("Dieselgate")}
The diesel emissions scandal in 2014 involved a number of global car manufacturers, including Volkswagen and Audi, who were caught using "cheat" software. When the software detected that a car was being tested for compliance with nitrous oxide emissions requirements, it automatically provided a result that significantly underreported the actual level of pollutants emitted.

Whether company executives were aware of the cheat software codes remains unanswered. However, concerns about board independence, proper governance and oversight of the employees who installed the codes, and the blatant disregard for the health of the communities in which the cars were sold significantly damaged the reputation of the companies involved. At the time the scandal became public, these companies experienced significant stock price declines. A number of companies have since paid billions in fines and penalties.

The commitment to governance and the corporate initiatives to better balance the interests of its various stakeholders can be reflected in the company's reputation. Employees, creditors, customers, and suppliers would more likely strive to build long-term relationships with a company that has a reputation for respecting the rights of its constituents and stakeholders. This would likely improve the company's ability to attract talent, secure capital, improve sales, or reach better terms with suppliers. In addition, ethics education and training for key stakeholders help support good governance practices. Good governance enables a company to mitigate the various risks underlying conflict of interest and agency problems and to maintain stability in operations.

\section{Financial Risks and Benefits}
\section{Debt Default and Bankruptcy}
Poor corporate governance, including weak management of creditors' interests, can affect the company's financial position and hinder its ability to honor its debt obligations. To the extent that the deterioration of corporate performance results in a debt default, the company may be exposed to bankruptcy risk if creditors choose to take legal action. The adverse consequences of corporate failures are not limited to the company's shareholders; they extend to other stakeholders, such as managers and employees, creditors, and even society and the environment.

\section{(Lower) Default Risk and Cost of Debt}
Good corporate governance contributes to maximizing shareholder value, and investors reward those companies that have sound governance, social, and environmental practices. Additionally, good corporate governance is associated with lower levels of business and investment risk, which limit the likelihood of unfavorable incidents affecting the company and reduce their impact. Governance arrangements that seek to manage creditor conflicts of interest restrict those corporate actions that would hinder the company's ability to repay its debt and therefore reduce its default risk. Default risks are also mitigated by properly functioning audit systems, transparent and better reporting of earnings, and controlling information asymmetries between the company and its capital providers. Lower default risks are associated with better credit ratings for the company and lower costs of debt borrowing, given that creditors typically require a lower return when their funds are better secured and their rights protected.

\section{(Enhanced) Valuation and Stock Performance}
Governance mechanics practiced at shareholders' meetings and maintained by internal corporate mechanisms, such as the board of directors and its committees, grant investors the assurance as to protection of their capital. These mechanisms help ensure investors that their rights to participate in discussions, vote on important matters in general meetings, and to enjoy fair and equal treatment are well protected. Investor confidence and the company's credibility in the marketplace are also enhanced by the appropriate and timely disclosure of material information concerning operating, financial, and governance activities. The improved transparency, the integrity of financial reporting processes, and the independent audit promote shareholders' and market participants' trust in the quality of reported earnings and their fair representation of the firm's financial position. This enables investors to make educated investment decisions and to reduce their risk perception of well-governed firms and therefore of the return required on capital invested in these firms. Consequently, good governance enhances the attractiveness of firms to investors, improves their valuations and stock performance, and reduces their cost of equity. Studies have found that

\begin{itemize}
  \item improvements in corporate governance practices increase the likelihood of upgrading a company's credit rating from speculative to investment grade, resulting in a significant decrease in the cost of debt;

  \item a positive impact associated with experienced audit committees' possessing financial expertise whereby listed firms with such committees are more likely to have positive market performance during times of crisis; and

  \item board independence and diversity appear to be key factors in firm valuation - particularly for initial public offerings - and play an important role in value creation and value protection for firms.

\end{itemize}

\section{EXAMPLE 14}
\section{Benefits of Corporate Governance}
\begin{enumerate}
  \item Which of the following is not a benefit of an effective corporate governance structure?
\end{enumerate}

A. Operating performance can be improved.

B. A corporation's cost of debt can be reduced.

C. Corporate decisions and activities require less control.

\section{Solution}
$\mathrm{C}$ is correct. A benefit of an effective corporate governance structure is to enable adequate scrutiny and control over operations. $B$ is incorrect because an effective governance structure can reduce investors' perceived credit risk of a corporation, thus potentially lowering the corporation's cost of debt. A is incorrect because operating efficiency may indeed be a benefit of an effective corporate governance structure.

Key questions analysts should consider about a company's corporate governance or stakeholder management include the following:

\begin{itemize}
  \item What is the company's ownership and voting structure?
\end{itemize}

Analysts should evaluate whether separations exist between ownership and control or whether unusual structures are in place that favor certain shareholders, creating potential risks. - Do the skill sets and experience of the board representatives match the current and future needs of the company?

Analysts should look at aspects such as director independence, tenure, experience, board size, and board diversity for investment insights.

\begin{itemize}
  \item How closely does the management team's remuneration and incentive structure align with factors expected to drive overall company results?
\end{itemize}

While the availability and quality of information about executive remuneration plans vary widely across markets, analysts should assess whether misalignments in interests, which create risk, exist.

\begin{itemize}
  \item Who are the significant investors in the company?
\end{itemize}

Analysts should assess the composition and behavior of major investors in a company to identify limitations and catalysts with regard to future changes in the company.

\begin{itemize}
  \item Are shareholder rights at the company strong, weak, or average compared with peers?
\end{itemize}

Within a framework of regional regulations and corporate governance codes, analysts should evaluate how strong shareholder rights are relative to the company's competitors.

\begin{itemize}
  \item How effectively is the company managing long-term risks and strengthening long-term sustainability?
\end{itemize}

Management's consideration of, or response to, environmental risks, human capital, corporate transparency, treatment of investors, and other stakeholders can provide analysts with important information or insights.

These questions and an analysis of these areas - typically provided by a company's proxy statements, annual reports, and sustainability reports, if available - can provide important insights about the quality of management and sources of potential risk.

\section{EXAMPLE 15}
\section{Analyst Considerations}
\begin{enumerate}
  \item An investment analyst would likely be most concerned with an executive remuneration plan that:
\end{enumerate}

A. varies each year.

B. is consistent with a company's competitors.

C. is cash-based only, without an equity component.

\section{Solution}
$\mathrm{C}$ is correct. If an executive remuneration plan offers cash only, the incentives between management and investors and other stakeholders may be misaligned. A is incorrect because a plan that varies over time would typically be of less concern to an analyst compared with one that did not change. B is incorrect because an analyst would likely be concerned if a company's executives were excessively compensated relative to competitors.

\section{ESG CONSIDERATIONS IN INVESTMENT ANALYSIS}
describe environmental, social, and governance considerations in investment analysis

The inclusion of governance factors in investment analysis has evolved considerably. Management and accountability structures are relatively transparent, and information regarding them is widely available. Many performance indicators can help evaluate risks arising from governance issues, such as ownership structure, board independence and composition, and compensation. Also, the risks of poor corporate governance have long been understood by analysts and shareholders.

In contrast, the practice of considering environmental and social factors (which collectively with governance are known by the commonly used acronym ESG) has evolved more slowly. A large number of environmental and social issues exist, and identifying which factors are likely to affect company performance is not an easy task.

ESG considerations have become increasingly relevant for two key reasons. First, ESG issues are having more material financial impacts on a company's fair value. Many investors have suffered substantial losses due to mismanagement of ESG issues by corporations, which resulted in environmental disasters, social controversies, or governance deficiencies. Second, a greater number of younger investors are increasingly demanding that their inherited wealth or their pension contributions be managed using investment strategies that systematically consider material ESG risks as well as negative environmental and societal impacts of their portfolio investments.

Historically, environmental and social issues such as climate change, air pollution, and societal impacts of a company's products and services have been treated as negative externalities - ones whose costs are not borne by the concerned company. However, increased stakeholder awareness and strengthening regulations are internalizing environmental and societal costs into the company's income statement either explicitly or implicitly by responsible investors.

Although ESG factors were once regarded as intangible or qualitative information, refinements in the identification and analysis of such factors, as well as increased corporate disclosures, have resulted in increasingly quantifiable information.

\section{EXAMPLE 16}
\section{ESG Relevancy}
\begin{enumerate}
  \item ESG considerations have become increasingly relevant for which of the following reasons?
\end{enumerate}

A. Many in the new generation of investors are demanding that investment strategies incorporate ESG factors.

B. ESG issues are having more material financial impacts on a company's fair value.

C. Environmental and social issues are being treated as negative externalities.

\section{Solution}
$A$ and $B$ are correct.

A is correct because a greater number among the new generation of investors are increasingly demanding that their inherited wealth or their pension contributions be managed using investment strategies that systematically consider material ESG risks as well as negative environmental and societal impacts of their portfolio investments.

$B$ is correct because ESG issues are having more material impacts on a company's fair value. Many investors have suffered substantial losses due to mismanagement of ESG issues by corporations, which resulted in environmental disasters, social controversies, or governance deficiencies.

$\mathrm{C}$ is incorrect. Environmental and social issues are not being treated as negative externalities as much as they were previously. Historically, environmental and social issues such as climate change, air pollution, and societal impacts of a company's products and services have been treated as negative externalities - ones whose costs are not borne by the concerned company. However, increased stakeholder awareness and strengthening regulations are internalizing environmental and societal costs into the company's income statement either explicitly or implicitly by responsible investors.

\section{Introduction to Environmental and Social Factors}
The materiality of ESG factors, particularly environmental and social factors, in investment analysis often differs meaningfully among sectors. An ESG factor is considered to be material when that factor is believed to have an impact on a company's long-term business model. For example, environmental factors such as emissions and water usage will likely be significant for utilities or mining companies, yet these are relatively inconsequential for financial institutions.

Overall, environmental factors that are generally considered material in investment analysis include natural resource management, pollution prevention, water conservation, energy efficiency and reduced emissions, the existence of carbon assets, adherence to environmental safety and regulatory standards, and the humane treatment of animals.

A specific concern among investors of energy companies is also the existence of "stranded assets" - carbon-intensive assets that are at risk of no longer being economically viable because of changes in regulation or investor sentiment. Analysts may find it difficult to assess potentially significant financial risks of energy companies because of limited information on the existence of these companies' carbon assets and because of the difficulty in determining political and regulatory risks.

Material environmental effects can arise from strategic or operational decisions based on inadequate governance processes or errors in judgment. For example, oil spills, industrial waste contamination events, and local resource depletion can result from poor environmental standards, breaches in safety standards, or unsustainable business models. Such events can be costly in terms of regulatory fines, litigation, clean-up costs, reputational risk, and resource management.

\section{TOXIC EMISSIONS AND WASTE AS AN ENVIRONMENTAL RISK}
Environmental issues such as toxic emissions and waste have historically been treated as externalities and thus not fully provisioned for in a company's financial reporting. However, with growing awareness among stakeholders, including regulators, companies may face financial liabilities associated with pollution, contamination, and the emission of toxic or carcinogenic substances and therefore must manage these risks. Gross mismanagement of these risks could result not only in a permanent loss of a company's license to operate but also in severe financial penalties.

In 2019, the collapse of Dam I of the Córrego do Feijão Mine in Brumadinho, Brazil, resulted in the spillage of millions of tons of nontoxic mud. Two hundred seventy lives were lost, and the nearby Paraopeba River was contaminated. The mine was owned and operated by Vale, multinational Brazilian mining corporation. Vale has since been accused of hiding information about the dam's instability for years to avoid damaging its reputation. Several employees from the company, including its former $C E O$, and its auditor, TÜV SÜD, were charged with murder and environmental crimes. Vale was fined millions of dollars in addition to bearing clean-up costs.

Social factors considered in ESG implementation generally pertain to the management of a business's human capital, including human rights and welfare concerns in the workplace; product development; and, in some cases, community impact. Staff turnover, worker health, training and safety, employee morale, ethics policies, employee diversity, and supply chain management can all affect a company's ability to sustain its competitive advantage.

In addition, minimizing social risks can lower a company's costs (e.g., through higher employee productivity, lower employee turnover, and reduced litigation potential) and reduce its reputational risk.

\section{DATA PRIVACY AND SECURITY AS A SOCIAL RISK}
Data privacy and security focus on how companies gather, use, and secure personally identifiable information and other metadata collected from individuals. In some industries, such as internet software and services, this includes managing the risks associated with government requests that may result in violations of civil and political rights.

With the proliferation of the internet, more services are being offered online, and consumers of these services are leaving a large digital footprint behind, often unknowingly. Some of this information may be personally identifiable in nature, leaving users vulnerable in the case of theft or misuse. As per the 2019 Cost of a Data Breach Report released by IBM and the Ponemon Institute, the average cost of a data breach is US\$3.9 million. (IBM 2021) Given the large amounts of sensitive data being managed by some of the largest internet and financial services companies today, mismanagement of data privacy and security risk can have materially damaging consequences for both a company's business model and its financial performance. For example, lax cybersecurity measures at Equifax Inc. led to a data breach and the theft of identity and financial data belonging to more than 140 million US citizens in 2017. Equifax has incurred hundreds of millions of dollars in expenses resulting from the breach and faced numerous lawsuits and investigations. Another example is the scandal involving Facebook and Cambridge Analytica, in which the personal data of over 80 million Facebook users were allegedly shared without consent and used in influencing voters, leading to one of the largest US government fines (US\$5 billion) ever imposed in the technology sector and to a significant drop in user trust.

\section{Evaluating ESG-Related Risks and Opportunities}
A typical starting point for evaluating ESG-related risks and opportunities is the identification of material qualitative and quantitative ESG factors that pertain to a company or its industry. For this, a company's annual and sustainability reports can be good sources of information. An analyst may evaluate these factors on a historical basis and make appropriate forecasts, as well as evaluate these factors with respect to a specific company relative to its peers.

A non-exhaustive list of environmental, social, and governance factors is shown in Exhibit 7. Typically, a smaller set of ESG factors are material for each company, influenced by the business segments it operates in and its geography of operation. Exhibit 7: Examples of Environmental, Social, and Governance Factors

\begin{center}
\begin{tabular}{|c|c|c|}
\hline
Environmental & Social & Governance \\
\hline
$\begin{array}{l}\text { - Climate change and carbon } \\ \text { emissions } \\ \text { - Air and water pollution } \\ \text { - Biodiversity } \\ \text { - Deforestation } \\ \text { - Energy efficiency } \\ \text { - Waste management } \\ \text { - Water scarcity }\end{array}$ & $\begin{array}{l}\text { - Human rights } \\ \text { - Labor standards } \\ \text { - Data security and privacy } \\ \text { - Occupational health and } \\ \text { safety } \\ \text { - Customer satisfaction and } \\ \text { product responsibility } \\ \text { - Treatment of workers } \\ \text { - Equity and diversity } \\ \text { - Community relations and } \\ \text { charitable activities }\end{array}$ & $\begin{array}{ll}\text { - Bribery and corruption } \\ \text { - } & \text { Shareholder rights } \\ \text { - Board composition (inde- } \\ & \text { pendence and diversity) } \\ \text { - } & \text { Audit committee structure } \\ \text { - Executive compensation } \\ \text { - } & \text { Lobbying and political } \\ & \text { contributions } \\ \text { - Whistleblower schemes }\end{array}$ \\
\hline
\end{tabular}
\end{center}

From a risk/reward perspective, the use of qualitative and quantitative ESG factors in traditional security and industry analysis typically differs for equity and fixed-income (debt) analysis. In equity analysis, ESG considerations are used to both identify potential opportunities and mitigate downside risk, whereas in fixed-income analysis, ESG considerations are generally focused on mitigating downside risk.

The process of identifying and evaluating relevant ESG-related factors is reasonably similar for both equity and corporate credit analysis. However, ESG integration differs considerably between equities and fixed income with respect to valuation. In equity security analysis, ESG-related factors are often analyzed in the context of forecasting financial metrics and ratios, adjusting valuation model variables (e.g., discount rate), or using sensitivity and/or scenario analysis.

For example, an analyst might increase the forecast of a hotel company's operating costs because of the impacts of excessive employee turnover - lost productivity, reduced customer satisfaction, and increased expenses for employee searches, temporary workers, and training programs. As another example, an analyst might choose to lower the discount rate for a snack food company that is expected to gain a competitive advantage by transitioning to a sustainable source of a key ingredient in its products.

In credit analysis, ESG factors may be integrated by using internal credit assessments, forecasting financial ratios, and ranking the relative credit of companies (or governments). In terms of valuation, relative value, spread, duration, and sensitivity/ scenario analysis is often used. For example, an analyst may include the effect of lawsuits on the credit ratios, cash flow, or liquidity of a toy company. The same analyst may also estimate the potential for the credit spreads of the company's bonds to widen from these lawsuits.

Generally speaking, the effect on the credit spreads of an issuer's debt obligations or its credit default swaps (CDSs) may differ depending on maturity. As a different example, consider an analyst who believes that a coal company faces long-term risk from potential asset write-downs - that is, for assets that are obsolete or no longer economically viable, owing to changes in regulatory or government policy or shifts in demand. In this case, the analyst may believe that valuation of the coal company's 10-year maturity notes would be considerably more negatively affected than its 1-year maturity notes.

\section{ENVIRONMENTAL, SOCIAL, AND GOVERNANCE INVESTMENT APPROACHES}
describe environmental, social, and governance investment
approaches

There is a lack of consensus on ESG related terms used in the investment community. For the purposes of this reading, we define ESG investing terminologyas follows:

Responsible investing is the broadest (umbrella) term used in ESG investing coverage. Responsible investing incorporates ESG factors into investment decisions with the objective of mitigating risk and protecting asset value while avoiding negative environmental or social consequences.

Sustainable investing selects assets and companies based on their perceived ability to deliver value by advancing economic, environmental, and social sustainability. Sustainable investing seeks to promote positive ESG practices that may enhance returns.

Socially responsible investing (SRI) incorporates environmental and social factors into the investment decision-making process, selecting those investments and companies with favorable profiles or attributes based on the investor's social, moral, or faith-based beliefs.

ESG investing ranges from being value-based to values-based. The objective of value-based investing is to incorporate material ESG considerations and traditional financial metrics into investment analysis while mitigating risk. The objective of values-based investing is to select investments that express, or align with, the moral or faith-based beliefs of an investor.

\section{ESG Investment Approaches}
Below we discuss six common ESG investment approaches:

\begin{itemize}
  \item Negative screening

  \item Positive screening

  \item ESG integration

  \item Thematic investing

  \item Engagement/active ownership

  \item Impact investing

\end{itemize}

Negative screening excludes certain sectors, companies, or practices from investment using criteria based on the investor's values, ethics, or preferences. Examples of a negative screen are excluding from a portfolio the fossil fuel extraction/production sector or those companies underperforming globally accepted standards in areas such as human rights or environmental management. Companies producing controversial products such as weapons or tobacco can also be excluded as part of a negative screening approach. Many negative screens use a specific set of standards, such as the UN Global Compact's Ten Principles on human rights, labor, the environment, and corruption.

Positive screening includes certain sectors, companies, or practices for investment using criteria based on the investor's values, ethics, or preferences. A positive screening approach is typically implemented using an ESG ranking or scoring methodology. Positive screening targets investments in companies with well-managed material ESG risks relative to peers that rank highest according to the selected criteria. For example, a positive screening approach may include seeking companies that promote employee rights, exhibit diversity in the workplace and board rooms, or perform well in customer safety.

ESG integration entails the systematic consideration of material ESG factors in asset allocation, security selection, and portfolio construction decisions to achieve the product's or portfolio's stated investment objectives. ESG factors are explicitly included in the financial analysis of individual stocks or bonds as inputs into cash flow forecasts, credit/default risk forecasts, and/or cost-of-capital estimates. The focus of an ESG integration approach is to identify risks and opportunities arising from material ESG factors and to determine how well a company may be managing them.

Thematic investing refers to investing in assets related to ESG factors or themes, such as clean energy, green technology, sustainable agriculture, gender diversity, and affordable housing. This approach is often based on needs arising from economic or social trends. Two common investment themes focus on growing demand for energy and water and on the availability of alternative sources for each. Global economic development has increased the demand for energy at the same time that rising greenhouse gas emissions are widely believed to be negatively affecting Earth's climate. Similarly, increasing global living standards and industrial needs have created greater demands for water along with the need to prevent drought or increase access to clean drinking water in certain regions of the world. While these themes are based on trends related to environmental issues, social issues - such as access to affordable health care and nutrition, especially in the poorest countries in the world - are also of great interest to thematic investors.

Engagement/active ownership uses shareholder or bondholder rights and mechanisms to influence corporate behavior through direct corporate engagement (i.e., communicating with senior management and/or boards of companies), the filing or co-filing of shareholder proposals, and proxy voting directed by ESG guidelines. Engagement/active ownership seeks to achieve targeted social or environmental objectives in addition to financial returns. Engagement/active ownership can be implemented through various asset classes and investment vehicles and often through direct transactions, such as venture capital investing.

Collaborative engagement initiatives entail multiple investors collectively engaging with company management to influence positive action in managing their material ESG risks. Climate Action 100+, backed by more than 450 investors with over US $\$ 40$ trillion in assets collectively under management (as of July 2020), is one such widely supported initiative that aims to influence the world's largest corporate greenhouse gas emitters to take necessary action on climate change (Climate Action). Other key ESG initiatives include air pollution, plastic waste management, human and labor rights in the supply chain, and executive remuneration.

Impact investing, which represents a smaller segment of the sustainable and responsible investing market, is investing "with the intention to generate positive, measurable social and environmental impact alongside a financial return," according to the Global Impact Investing Network (GIIN). An example is investing in products or services that help achieve one (or more) of the 17 Sustainable Development Goals (SDGs) launched by the United Nations in 2015, such as SDG 6: Clean Water and Sanitation ("Ensure availability and sustainable management of water and sanitation for all") or SDG 11: Sustainable Cities and Communities ("Make cities and human settlements inclusive, safe, resilient, and sustainable").

\section{EXAMPLE 17}
\section{ESG Investment Approach}
\begin{enumerate}
  \item The ESG investment approach that is most associated with excluding certain sectors or companies is:
A. thematic investing.
B. negative screening.
C. positive screening.
\end{enumerate}

\section{Solution}
B is correct. Negative screening entails excluding certain companies or sectors, such as fossil fuel extraction, from a portfolio. A is incorrect because thematic investing typically focuses on investing in companies within a specific sector or following a specific theme, such as energy efficiency or climate change, as opposed to merely excluding a set of companies or industries from a portfolio. Likewise, $C$ is incorrect because positive screening focuses on including companies that rank (or score) most favorably compared to their peers with regard to ESG factors.

\section{GREEN FINANCE}
Green finance is a responsible investing approach that uses financial instruments that support a green economy. According to the Organisation for Economic Co-Operation and Development (OECD), green finance relates to "achieving economic growth while reducing pollution and greenhouse gas emissions, minimizing waste, and improving efficiency in the use of natural resources."

The primary investment vehicles used in green finance are green bonds, in which issuers earmark the proceeds toward environmental-related projects. To meet investor ESG requirements and thus increase issue appeal to a broader investor audience, companies often will issue their bonds in compliance with voluntary standards such as the Green Bond Principles developed by the International Capital Market Association (ICMA). In addition to green bonds, sustainability linked loans (including green loans) are also used in green finance investing. Sustainability linked loans are types of loan instruments and/or contingent facilities (such as bonding lines, guarantee lines, or letters of credit) that incentivize the borrower's achievement of ambitious, predetermined sustainability performance objectives.

A summary of the six investment approaches along with a mapping to other classifications is shown in Exhibit 8.

\section{Exhibit 8: ESG Investment Approaches}
\begin{center}
\begin{tabular}{|c|c|c|c|}
\hline
\multirow[b]{2}{*}{$\begin{array}{l}\text { ESG Investment } \\
\text { Approach }\end{array}$} & \multirow[b]{2}{*}{Description} & \multicolumn{2}{|c|}{Mapping to Other Classifications} \\
\hline
 &  & $\begin{array}{c}\text { Financial } \\ \text { Analysts } \\ \text { Journal }\end{array}$ & $\begin{array}{l}\text { Global Sustainable } \\ \text { Investment Review }\end{array}$ \\
\hline
$\begin{array}{l}\text { Negative } \\ \text { screening }\end{array}$ & $\begin{array}{l}\text { Excludes companies or sectors } \\ \text { based on business activities } \\ \text { or environmental or social } \\ \text { concerns }\end{array}$ & $\begin{array}{l}\text { Negative } \\ \text { screening }\end{array}$ & $\begin{array}{l}\text { Negative/ } \\ \text { Exclusionary } \\ \text { screening } \\ \text { Norms-based } \\ \text { screening }\end{array}$ \\
\hline
\end{tabular}
\end{center}

\begin{center}
\begin{tabular}{|c|c|c|c|}
\hline
\multirow[b]{2}{*}{$\begin{array}{c}\text { ESG Investment } \\
\text { Approach }\end{array}$} & \multirow[b]{2}{*}{Description} & \multicolumn{2}{|c|}{Mapping to Other Classifications} \\
\hline
 &  & $\begin{array}{l}\text { Financial } \\ \text { Analysts } \\ \text { Journal }\end{array}$ & $\begin{array}{l}\text { Global Sustainable } \\ \text { Investment Review }\end{array}$ \\
\hline
$\begin{array}{l}\text { Positive } \\ \text { screening }\end{array}$ & $\begin{array}{l}\text { Includes sectors or companies } \\ \text { based on specific ESG criteria, } \\ \text { typically ESG performance } \\ \text { relative to industry peers }\end{array}$ & $\begin{array}{l}\text { Positive } \\ \text { screening } \\ \text { Relative/ } \\ \text { Best-in-class } \\ \text { screening }\end{array}$ & $\begin{array}{c}\text { Positive/Best-in-class } \\ \text { screening }\end{array}$ \\
\hline
ESG integration & $\begin{array}{l}\text { Systematically considers } \\ \text { material ESG factors in asset } \\ \text { allocation, security selection, } \\ \text { and portfolio construction } \\ \text { decisions to achieve or exceed } \\ \text { the product's stated investment } \\ \text { objectives }\end{array}$ & $\begin{array}{c}\text { Full integration } \\ \text { Overlay/ } \\ \text { Portfolio tilt } \\ \text { Risk factor/Risk } \\ \text { premium }\end{array}$ & ESG integration \\
\hline
$\begin{array}{l}\text { Thematic } \\ \text { investing }\end{array}$ & $\begin{array}{l}\text { Invests in themes or assets } \\ \text { related to ESG factors }\end{array}$ & $\begin{array}{l}\text { Thematic } \\ \text { investing }\end{array}$ & $\begin{array}{c}\text { Sustainability-themed } \\ \text { investing }\end{array}$ \\
\hline
$\begin{array}{l}\text { Engagement/ } \\ \text { Active } \\ \text { ownership }\end{array}$ & $\begin{array}{l}\text { Uses shareholder power to } \\ \text { influence corporate behavior to } \\ \text { achieve targeted ESG objectives } \\ \text { along with financial returns }\end{array}$ & $\begin{array}{c}\text { Engagement/ } \\ \text { Active } \\ \text { ownership }\end{array}$ & $\begin{array}{l}\text { Corporate engage- } \\ \text { ment/Shareholder } \\ \text { action }\end{array}$ \\
\hline
Impact investing & $\begin{array}{l}\text { Invests with the intention to } \\ \text { generate positive, measurable } \\ \text { social and environmental } \\ \text { impact alongside a financial } \\ \text { return }\end{array}$ & $\mathrm{N} / \mathrm{A}$ & $\begin{array}{c}\text { Impact/Community } \\ \text { investing }\end{array}$ \\
\hline
\end{tabular}
\end{center}

Note: For information on the Financial Analysts Journal column under "Mapping to Other

Classifications," see \href{http://www.tandfonline.com/doi/full/10.2469/faj.v74.n3.2}{www.tandfonline.com/doi/full/10.2469/faj.v74.n3.2}. For the Global Sustainable

Investment Review column, see \href{http://www.gsi-alliance.org/wp-content/uploads/2021/08/GSIR-20201.pdf}{www.gsi-alliance.org/wp-content/uploads/2021/08/GSIR-20201.pdf}.

\section{ESG Market Overview}
Reflecting the growth of ESG-related information available, global assets dedicated to ESG investments have increased substantially. According to the Global Sustainable Investment Alliance (GSIA), a collaboration of organizations dedicated to advancing sustainable investing in the financial markets, Europe and the United States account for the vast majority of these assets. China and India are leading emerging markets for green bond issuance. Given differences in the way managers and investors define and implement sustainable and ESG mandates, however, determining the exact size of the ESG investment universe is challenging.

Increased interest in sustainable investing has also led to increased corporate disclosures of ESG issues and to a growing number of entities that collect and analyze ESG data. In addition to the GSIA, other organizations have formed to monitor and advance the mission of sustainable investing. These include

\begin{itemize}
  \item the Global Reporting Initiative (GRI), a non-profit organization that produces a sustainability reporting framework to measure and report sustainability-related issues and performance,

  \item a collaboration between the United Nations and a consortium of institutional investors that launched the Principles for Responsible Investment (PRI) Initiative, and - the Sustainability Accounting Standards Board (SASB), a non-profit organization that supports sustainability accounting standards for companies disclosing material ESG information.

\end{itemize}

In 2018, to help educate investors, the CFA Institute and the PRI published Guidance and Case Studies for ESG Integration: Equities and Fixed Income (www. \href{http://cfainstitute.org/-/media/documents/survey/guidance-case-studies-esg-integration}{cfainstitute.org/-/media/documents/survey/guidance-case-studies-esg-integration}. $\operatorname{ash} x)$

One area of continuing debate is whether the consideration of ESG factors is consistent with fiduciary duty - particularly in the oversight and management of pension fund assets. While pension fund regulation regarding ESG considerations varies globally, the PRI and the United Nations Environment Programme Finance Initiative (UNEP FI) promote the belief that ESG integration is a key part of investment analysis, stating, "Investors that fail to incorporate ESG issues are failing their fiduciary duties and are increasingly likely to be subject to legal challenge" (UNEPFI and PRI 2019).

\section{SUMMARY}
The investment community is increasingly recognizing and quantifying environmental and social considerations and the impacts of corporate governance in the investment process. Analysts who understand these considerations can better evaluate their associated implications and risks for an investment decision. The core concepts covered are listed here:

\begin{itemize}
  \item The primary stakeholder groups of a corporation consist of shareholders, creditors, the board of directors, managers and employees, customers, suppliers, and government/regulators.

  \item A principal-agent relationship (or agency relationship) entails a principal hiring an agent to perform a particular task or service. In a company, both the board of directors and management act in agent capacity to represent the interests of shareholder principals.

  \item Conflicts occur when the interests of various stakeholder groups diverge and when the interests of one group are compromised for the benefit of another.

  \item Stakeholder management involves identifying, prioritizing, and understanding the interests of stakeholder groups and managing the company's relationships with stakeholders.

  \item Mechanisms to mitigate shareholder risks include company reporting and transparency, general meetings, investor activism, derivative lawsuits, and corporate takeovers.

  \item Mechanisms to mitigate creditor risks include bond indenture(s), company reporting and transparency, and committee participation.

  \item Mechanisms to mitigate board risks include board/management meetings and board committees.

  \item Remaining mechanisms to mitigate risks for other stakeholder (employees, customers, suppliers, and regulators) include policies, laws, regulations, and codes.

  \item Executive (internal) directors are employed by the company and are typically members of senior management. Non-executive (external) directors have limited involvement in daily operations but serve an important oversight role. - Two primary duties of a board of directors are duty of care and duty of loyalty.

  \item A company's board of directors typically has several committees that are responsible for specific functions and report to the board. Although the types of committees may vary across organization, the most common are the audit committee, governance committee, remuneration (compensation) committee, nomination committee, risk committee, and investment committee.

  \item Shareholder activism encompasses a range of strategies that may be used by shareholders when seeking to compel a company to act in a desired manner.

  \item From a corporation's perspective, risks of poor governance include weak control systems; ineffective decision making; and legal, regulatory, reputational, and default risk. Benefits include better operational efficiency, control, and operating and financial performance, as well as lower default risk (or cost of debt), which enhances shareholder value.

  \item Key analyst considerations in corporate governance and stakeholder management include economic ownership and voting control, board of directors' representation, remuneration and company performance, investor composition, strength of shareholders' rights, and the management of long-term risks.

  \item Environmental and social issues, such as climate change, air pollution, and societal impacts of a company's products and services, have historically been treated as negative externalities. However, increased stakeholder awareness and strengthening regulations are internalizing environmental and societal costs into the company's income statement by responsible investors.

  \item ESG investment approaches are value-based or values-based. There are six common ESG investment approaches: negative screening, positive screening, ESG integration, thematic investing, engagement/active ownership, and impact investing.

\end{itemize}

\section{REFERENCES}
OECD "Green Finance and Investment." OECD iLibrary. \href{http://www.oecd-ilibrary.org/fr/environment/}{www.oecd-ilibrary.org/fr/environment/} green-finance-and-investment\_24090344.

OECD 2004. "OECD Principles of Corporate Governance." \href{https://legalinstruments.oecd.org/}{https://legalinstruments.oecd.org/} en/instruments/151.

IBM 2021. Cost of a Data Breach Report 2021(July): \href{http://www.ibm.com/security/data-breach}{www.ibm.com/security/data-breach}.

Climate Action 100+. "Initiative Snapshot" (\href{http://www.climateaction100.org}{www.climateaction100.org}).

GIIN “Impact Investing." Global Impact Investing Network. \href{https://thegiin.org/impact-investing}{https://thegiin.org/impact-investing}.

UNEPFI and PRI 2019. Fiduciary Duty in the 21st Century Final Report. \href{https://wedocs.unep}{https://wedocs.unep}. org/handle/20.500.11822/31658.

\section{PRACTICE PROBLEMS}
\begin{enumerate}
  \item Which group of company stakeholders would be least affected if the firm's financial position weakens?
\end{enumerate}

A. Suppliers

B. Customers

C. Managers and employees

\begin{enumerate}
  \setcounter{enumi}{1}
  \item Which of the following represents a principal-agent conflict between shareholders and management?
\end{enumerate}

A. Risk tolerance

B. Multiple share classes

C. Accounting and reporting practices

\begin{enumerate}
  \setcounter{enumi}{2}
  \item Which of the following statements regarding stakeholder management is most accurate?
\end{enumerate}

A. Company management ensures compliance with all applicable laws and regulations.

B. Directors are excluded from voting on transactions in which they hold material interest.

C. The use of variable incentive plans in executive remuneration is decreasing.

\begin{enumerate}
  \setcounter{enumi}{3}
  \item Which of the following issues discussed at a shareholders' general meeting would most likely require only a simple majority vote for approval?
\end{enumerate}

A. Voting on a merger

B. Election of directors

C. Amendments to bylaws

\begin{enumerate}
  \setcounter{enumi}{4}
  \item Which of the following statements about environmental, social, and governance (ESG) in investment analysis is correct?
\end{enumerate}

A. ESG factors are strictly intangible in nature.

B. ESG terminology is easily distinguishable among investors.

C. Environmental and social factors have been adopted in investment analysis more slowly than governance factors.

\begin{enumerate}
  \setcounter{enumi}{5}
  \item The existence of "stranded assets" is a specific concern among investors of:
\end{enumerate}

A. energy companies.

B. health care companies.

C. property companies.

\begin{enumerate}
  \setcounter{enumi}{6}
  \item An investor concerned about clean-up costs resulting from breaches in a publicly traded company's safety standards would most likely consider which factors in her investment analysis?
A. Social factors
B. Governance factors
C. Environmental factors

  \item investing is the umbrella term used to describe investment strategies that incorporate environmental, social, and governance (ESG) factors into their approaches.
A. ESG
B. Sustainable
C. Responsible

  \item An investor concerned about a publicly traded company's data privacy and security practices would most likely incorporate which type of ESG factors in an investment analysis?
A. Social
B. Governance
C. Environmental

  \item Which of the following statements regarding ESG investment approaches is most accurate?

\end{enumerate}

A. Negative screening excludes industries and companies that do not meet the investor's ESG criteria.

B. Thematic investing considers multiple factors.

C. Positive screening excludes industries with unfavorable ESG aspects.

\begin{enumerate}
  \setcounter{enumi}{10}
  \item Which of the following stakeholders are least likely to be positively affected by increasing the proportion of debt in the capital structure?
A. Senior management
B. Non-management employees
C. Shareholders

  \item Which statement correctly describes corporate governance?

\end{enumerate}

A. Corporate governance complies with a set of global standards.

B. Corporate governance is independent of both shareholder theory and stakeholder theory.

C. Corporate governance seeks to minimize and manage conflicting interests between insiders and external shareholders.

\begin{enumerate}
  \setcounter{enumi}{12}
  \item Which of the following represents a responsibility of a company's board of direc- tors?
A. Implementation of strategy
B. Enterprise risk management
C. Considering the interests of shareholders only

  \item Which of the following statements concerning the legal environment and shareholder protection is most accurate?

\end{enumerate}

A. A civil law system offers better protection of shareholder interests than does a common law system.

B. A common law system offers better protection of shareholder interests than does a civil law system.

C. Neither system offers an advantage over the other in the protection of shareholder interests.

\section{SOLUTIONS}
\begin{enumerate}
  \item B is correct. Compared with other stakeholder groups, customers tend to be less affected by or concerned with a company's financial performance.

  \item A is correct. Shareholder and manager interests can diverge with respect to risk tolerance. In some cases, shareholders with diversified investment portfolios can have a fairly high risk tolerances because specific company risk can be diversified away. Managers are typically more risk averse in their corporate decision making to better protect their employment status.

  \item B is correct. Often, policies on related-party transactions require that such transactions or matters be voted on by the board (or shareholders), excluding the director holding the interest.

  \item B is correct. The election of directors is considered an ordinary resolution and, therefore, requires only a simple majority of votes to be passed.

  \item C is correct. The risks of poor corporate governance have long been understood by analysts and shareholders. In contrast, the practice of considering environmental and social factors has been slower to take hold.

  \item A is correct. A specific concern among investors of energy companies is the existence of "stranded assets," which are carbon-intensive assets at risk of no longer being economically viable because of changes in regulation or investor sentiment.

  \item C is correct. Material environmental effects can arise from strategic or operational decisions based on inadequate governance processes or errors in judgment. For example, oil spills, industrial waste contamination events, and local resource depletion can result from poor environmental standards, breaches in safety standards, or unsustainable business models. Such events can be costly in terms of regulatory fines, litigation, clean-up costs, reputational risk, and resource management.

  \item C is correct. Responsible investing is the broadest (umbrella) term used to describe investment strategies that incorporate environmental, social, and governance (ESG) factors into their approaches.

  \item A is correct. Social factors considered in ESG implementation generally pertain to the management of the human capital of a business, including data privacy and security.

  \item A is correct. Negative screening refers to the practice of excluding certain sectors, companies, or practices that do not meet specific ESG criteria based on the investor's values, ethics, or preferences.

  \item B is correct. While leverage increases risk for all stakeholders, shareholders generally benefit through higher potential returns. Senior management typically benefits through equity-based compensation. For non-management employees, equity-based compensation is likely to be small to non-existent.

  \item $C$ is correct. Corporate governance is the arrangement of checks, balances, and incentives a company needs to minimize and manage the conflicting interests between insiders and external shareholders.

  \item B is correct. The board typically ensures that the company has an appropriate enterprise risk management system in place.

  \item B is correct. A common law system offers better protection of shareholder interests than does a civil law system.

\end{enumerate}

\textbackslash title\{

LEARNING MODULE

\begin{center}
\includegraphics[max width=\textwidth]{2023_05_04_b5cfa4f1bc883752f121g-701}
\end{center}

\section{Business Models \& Risks}
Glen D. Campbell, MBA (Canada).

\section{LEARNING OUTCOME}
\begin{center}
\begin{tabular}{c|l}
Mastery & The candidate should be able to: \\
\hline
$\square$ & Describe key features and types of business models. \\
$\square$ & $\begin{array}{l}\text { Describe expected relations between a company's external } \\ \text { environment, business model, and financing needs. } \\ \text { Explain and classify types of business and financial risks for a } \\ \text { company. }\end{array}$ \\
\end{tabular}
\end{center}

\section{INTRODUCTORY CONTEXT/MOTIVATION}
A clearly described business model helps the analyst understand a business: how it operates, its strategy, target customers, key partners, prospects, risks, and financial profile. Rather than rely on management's description of its business model, analysts should develop their own understanding.

Many firms have conventional business models that are easily understood and described in simple terms, such as manufacturer, wholesaler, retailer, professional firm, or restaurant chain. However, many business models are complex, specialized, or new. Digital technology in particular has enabled significant business model innovation, bringing business models into the spotlight. It has spawned new services and markets and has also changed the way most businesses operate. In many cases, technology has enabled the disruption of existing business models, allowing new players to win against large and well-established players who lack the capabilities or agility to respond.

WHAT IS A BUSINESS MODEL?

Describe key features and types of business models. Successful new businesses may be based on a new product or technology, but there are many success stories based on familiar products or services and a new business model. For example, IKEA successfully combines existing business concepts-low-cost self-assembly furniture, modernist Scandinavian design, and big box retailing-in a uniquely successful way. Similarly, Google did not invent online search but found a way to generate revenues through online advertising based on user search data.

Often, successful business models are not new or unique. Many businesses, such as wholesalers, retailers, law firms, building contractors, banks, and insurers, have conventional business models. Success for these firms hinges not on business model innovation but on superior execution, skill, proprietary technology, a strong brand, scale, or other factors.

So what is a business model? There is no precise definition, but a business model essentially describes how a business is organized to deliver value to its customers:

\begin{itemize}
  \item who its customers are,

  \item how the business serves them,

  \item key assets and suppliers, and

  \item the supporting business logic.

\end{itemize}

A business model makes it clear what the business does, how it operates, and how it generates revenue and profits, as well as how it differs in these respects from its competitors. It provides enough detail so that the basic relationships between the key elements are clear, but it does not provide a full description that we would expect to see in a business plan, such as detailed financial forecasts.

A business model should have a value proposition and a value chain as illustrated in Exhibit 1. In addition, an analyst must also assess the profitability and risk of the firm's business model.

Exhibit 1: Firm Business Model

\begin{center}
\includegraphics[max width=\textwidth]{2023_05_04_b5cfa4f1bc883752f121g-702}
\end{center}

\section{Business Model Features}
Key features of a public company's business model are often provided in annual reports or other disclosure documents. For example, the "Business Description" in Tesla's annual report starts as follows: We design, develop, manufacture, sell and lease high-performance fully electric vehicles and energy generation and storage systems, and offer services related to our products. We are the world's first vertically integrated sustainable energy company, offering end-to-end clean energy products, including generation, storage and consumption. We generally sell our products directly to customers, including through our website and retail locations. We also continue to grow our customer-facing infrastructure through a global network of vehicle service centers, Mobile Service technicians, body shops, Supercharger stations and Destination Chargers to accelerate the widespread adoption of our products. We emphasize performance, attractive styling and the safety of our users and workforce in the design and manufacture of our products, and are continuing to develop full self-driving technology for improved safety. We also strive to lower the cost of ownership for our customers through continuous efforts to reduce manufacturing costs and by offering financial services tailored to our vehicles. Our sustainable energy products, engineering expertise, intense focus to accelerate the world's transition to sustainable energy and achieve the benefits of autonomous driving, and business model differentiate us from other companies.

In this paragraph, the company describes numerous features, including its products and their attributes, key channels, and its emphasis on innovation and vertical integration. An analyst's focus is understanding a company's business model to evaluate how effectively it has been implemented and the related impact on the return and risk of the company. Let's explore the features further.

\section{Customers, Market: Who}
\begin{center}
\includegraphics[max width=\textwidth]{2023_05_04_b5cfa4f1bc883752f121g-703}
\end{center}

The business model should identify the firm's target customers:

\begin{itemize}
  \item What geographies will be served?

  \item What market segments will be served?

  \item What customer segments will be served? Is this a business (B2B) or consumer (B2C) market?

\end{itemize}

It is common in consumer markets to think of target demographic segments as defined by marketers (e.g., high-income suburban families). In many cases, segmentation is unique to the product or service category (e.g., affluent early adopters of technology with homes that support plug-in charging in countries with EV [electric vehicles] subsidies). Business opportunities often arise because established firms may not effectively serve (or even recognize) particular customer segments. At the same time, choices made concerning a firm's business model may introduce other related considerations or risks to the firm. For example, the firm might face high barriers to entry, changes in customer segment(s), or increased market competition.

In the Tesla example, the description is silent on which customer segments Tesla is targeting. The likely reason is that Tesla's target market is shifting over time toward the mass market, as costs and prices decline. As an analyst, this type of inference and evaluation can be used to guide your financial modeling and to produce forecasts that reflect key aspects of the firm's business model.

\section{Firm Offering: What}
\begin{center}
\includegraphics[max width=\textwidth]{2023_05_04_b5cfa4f1bc883752f121g-704}
\end{center}

The business model should define what the firm offers (what product or service), in terms that differentiate it from competitor offerings, and with reference to the needs of its target customers. This helps the analyst to understand the addressable market for the business and to identify key competitors and associated risks. For example, there may be high risk of imitation or substitution if no "moat," or barriers to competition, exists for the product or service offered by the firm and the level of differentiation is low, or there may be changes in target customer needs and preferences for the firm's offer.

Using the Tesla example, "electric car" is too broad a description of its product offering. Tesla's description of "high-performance fully electric vehicles and energy generation and storage systems" is more precise and useful to the analyst.

It is common for companies to use overly broad terms in describing their offerings or addressable markets, to overstate differentiation, or to reference platforms or networks that may be very weakly developed-all in an attempt to convince the analyst or investor of the value of the business. It is important for analysts to understand and assess the business independently.

\section{Channels: Where}
\begin{center}
\includegraphics[max width=\textwidth]{2023_05_04_b5cfa4f1bc883752f121g-704(1)}
\end{center}

A firm's channel strategy refers to "where" the firm is selling its offering; that is how it is reaching its customers. Channel strategy usually involves two main functions:

\begin{enumerate}
  \item Selling the firm's products and services

  \item Delivering them to customers

\end{enumerate}

In assessing a firm's channel strategy, it is important to distinguish the functions performed from the assets that might be involved and different firms that might be involved in performing those functions or owning those facilities. Exhibit 2 provides examples of the functions, assets and firms that may be part of the channel strategy for a firm.

\section{Exhibit 2: Channel Strategy: Functions vs. Assets vs. Firms}
\begin{center}
\includegraphics[max width=\textwidth]{2023_05_04_b5cfa4f1bc883752f121g-705(1)}
\end{center}

For "product" businesses, the traditional channel strategy is typically reflected in the flow of finished goods (e.g., from manufacturer to wholesaler, retailer, and end customer), each with its own physical facilities and with the product sold and purchased at each stage.

In some categories, manufacturers employ a direct sales strategy, selling directly to the end customer. Direct sales to the end customer bypass ("disintermediate") the distributor or retailer. Typically, this involves the company's own sales force, which in some cases represents a significant business investment and carrying cost for the firm. Large retailers often purchase directly from manufacturers, bypassing wholesalers. Direct sales is a common and longstanding channel strategy for complex or high-margin products or services, such as industrial equipment, pharmaceuticals, and life insurance. It is also a common strategy in B2B markets where the universe of potential customers is relatively small and easily reached. With e-commerce, however, direct sales have become a cost-effective strategy across many business and consumer markets.

Exhibit 3 contrasts the interactions of a traditional channel strategy versus a direct sales strategy.

Exhibit 3: Traditional Channel Strategy (Product) vs. Direct Sales Strategy

\begin{center}
\includegraphics[max width=\textwidth]{2023_05_04_b5cfa4f1bc883752f121g-705}
\end{center}

Where an intermediary is involved, that intermediary may work on an agency basis, earning commissions, rather than taking ownership of the goods. Examples include auctioneers, such as Sotheby's (fine art) and Ritchie Bros. (industrial equipment). Using an intermediary may be a good business model solution for the firm, but it requires the firm to give up a degree of control to the intermediary. In the e-commerce realm, the drop shipping model enables an online marketer to have goods delivered directly from manufacturer to end customer without taking the goods into inventory.

Often, channels are used in combination. With an omnichannel strategy, both digital and physical channels are used to complete a sale. For example, a customer might order an item online and pick it up in a store ("click and collect") or select an item in a store and have it delivered. The use of a digital channel strategy introduces potential cybersecurity and access risks, while having a physical location might introduce substantially greater financial risk.

\section{EXAMPLE 1}
\section{Adidas Business Model}
Adidas is the world's second largest sportswear brand. The following excerpt from the 2019 Adidas annual report (\href{http://www.adidas-group.com/media/filer_public/}{www.adidas-group.com/media/filer\_public/} a8/5c/a85c9b8e-865b-4237-8def-8574be243577/annual\_report\_gb-2019\_en.pdf) provides a brief summary of its channel strategy:

With more than 2,500 own-retail stores, more than 15,000 mono-branded franchise stores and more than 150,000 wholesale doors, we have an unrivaled network of consumer touchpoints within our industry. In addition, through our own e-commerce channel, our single biggest store available to consumers in over 40 countries, we are leveraging a consistent global framework.

\begin{enumerate}
  \item List and number the channels used by the Adidas business model.
\end{enumerate}

\section{Solution}
Retail, franchise, wholesale, e-commerce; 4.

The channels used by a business can influence a firm's revenues, cost structure, profitability, and sensitivity to internal and external risk factors.

For some businesses, channel strategy is of critical importance and a key competitive advantage. For example, large firms in such sectors as life insurance, pharmaceuticals, film/music production, and food/beverages have often made a large investment in their sales force or distribution network, one that is not easily replicated by competitors.

It is important to recognize how a firm's channel strategy differs from those of its competitors. For example, Tesla references its direct sales strategy, which differs from the franchised dealer model used by most automakers. Hyundai's Genesis luxury car division also uses a no-dealer model, making visits to the customer's home for test drive appointments and for after-sale service appointments.

\section{KNOWLEDGE CHECK}
\begin{enumerate}
  \item The Tesla and Hyundai no-dealer models are examples of an established distribution channel.
\end{enumerate}

\section{Solution}
The Tesla and Hyundai no-dealer models are examples of disintermediating (bypassing) an established distribution channel.

\begin{enumerate}
  \setcounter{enumi}{1}
  \item A customer looking to buy a new car might do research online and enter her contact details in order to obtain product information; those details are a "customer lead" that is forwarded to a nearby dealer who might offer a test drive. Identify the channel strategy used in this business model.
\end{enumerate}

A. A. "Bricks and mortar"

B. B. Third party

C. C. Omnichannel

Solution

$\mathrm{C}$ is correct. The business model employs an omnichannel strategy in which both digital and physical channels are used in combination to complete a sale.

\section{Pricing: How Much}
\begin{center}
\includegraphics[max width=\textwidth]{2023_05_04_b5cfa4f1bc883752f121g-707}
\end{center}

A business model needs to provide sufficient pricing detail so that the logic of the business is clear.

\begin{itemize}
  \item Does the firm price at a premium, parity, or discount relative to competitors?

  \item How is the firm's pricing justified in its business model?

\end{itemize}

For a producer of a commodity, pricing is usually not an essential part of the business model. Companies with little differentiation are "commodity" producers that must accept market prices dictated to them ("price taker"), whereas companies with high differentiation can command premium pricing ("price setter") and face much less pricing risk from competitors. It is important for an analyst to assess whether the firm requires access to specific capital, labor, or inputs to maintain its level of differentiation.

When a business lacks pricing power, demand is highly price-elastic, with small price changes by the company causing very large changes in demand. A business model in a commodity market is likely to emphasize other sources of value, such as a cost advantage. That said, a discounting strategy might be employed in order to build scale and a cost advantage. The retail sector provides ready examples of this, including Walmart, Carrefour, and Tesco.

More commonly, firms attempt to differentiate their offerings in some way, in order to achieve some degree of pricing power. Using our Tesla example, rather than providing details on price points, the annual report references "total cost of ownership," reflecting the importance of government subsidies and lower operating costs for electric vehicles as an offset to higher purchase prices. Total cost of ownership refers to the aggregate direct and indirect costs associated with owning an asset over its life span. Tesla's pricing could be in line with other high-volume luxury cars but with a lower "total cost of ownership." This value proposition (reduced total cost of ownership through some combination of product capabilities, reliability, ease of maintenance and operation, training, etc.) is common to business models in many sectors.

\section{Pricing and Revenue Models}
Pricing approaches are typically value or cost based. Value-based pricing attempts to set pricing based on the value received by the customer. As noted earlier, Tesla's cars command a premium price in part because they have low operating costs, which are a source of value to their owners. Cost-based pricing attempts to set pricing based on costs incurred. An environmental law firm might opt to charge clients by the hour, given the difficulties in estimating how many hours might be required or in measuring the benefit to the client. In contrast, a law firm handling personal injury claims can more easily assess these factors and might charge clients a percentage of the award received. The latter is an example of success-based or contingency billing, a type of value-based pricing.

\section{Price Discrimination}
Economists use the term "price discrimination" when firms charge different prices to different customers. In theory, the objective of price discrimination is to maximize revenues in a situation where different customers have different willingness to pay. It may also be less costly to serve certain customer segments, such as large-volume buyers. Common pricing strategies in this category include the following:

\begin{itemize}
  \item Tiered pricing charges different prices to different buyers, most commonly based on volume purchased. Note that volume discounts achieve a similar result.

  \item Dynamic pricing charges different prices at different times. Specific examples include off-peak pricing (e.g., for hotel rooms, advertising, airline tickets, electricity, or matinee movie tickets), "surge" pricing, and "congestion" pricing (e.g., for ride sharing and toll roads). Digital technology can make dynamic pricing models easier to implement and manage.

  \item Auction/reverse auction models establish prices through bidding (by sellers in the case of reverse auctions). Digital technology enables this process to be automated, making these models feasible in new categories (e.g., eBay for consumer merchandise and Google and Amazon for digital advertising).

\end{itemize}

\section{Pricing for Multiple Products}
Some pricing models are used by firms selling multiple or complex products:

\begin{itemize}
  \item Bundling refers to combining multiple products or services so that customers are incentivized, or required, to buy them together. Bundling can be effective, particularly for products that are complementary, with high incremental margins and high marketing costs relative to the cost of the product itself. Examples include hotel rooms with free breakfast; furnished rental apartments; cable TV and internet services; pre-packaged sets of toys, tools, or kitchen utensils; and cloud-based software combining an application, processing power, storage, and support services.

  \item Razors-and-blades pricing combines a low price on a piece of equipment (e.g., razor, printer, water purifier, or gaming console) and high-margin pricing on repeat-purchase consumables (blades, printer ink, filter cartridges, software).

  \item Optional product pricing applies when a customer buys additional services or product features, either at the time of purchase (e.g., a deluxe interior for a car or a side order with a restaurant meal) or afterward (e.g., change orders in a construction contract). A common strategy is to seek higher margins on the optional features or services, when the customer is "captive" (i.e., the initial purchase decision has already been made), although firms that take this strategy too far can damage customer goodwill and their reputation.

\end{itemize}

\section{Pricing for Rapid Growth}
\begin{itemize}
  \item Penetration pricing is an example of discount pricing and is used when a firm willingly sacrifices margins in order to build scale and market share. Examples include Netflix (subscription video), Huawei (telecom equipment), and Amazon (tablets, e-readers, Alexa speakers).
\end{itemize}

For digital businesses, growing a user base is often a critical objective, since the incremental costs associated with one more customer subscription are often minimal and the benefits can be enormous, including promotion through word of mouth and, potentially, network effects. Trial and adoption can be encouraged through pricing strategies, which include the following:

\begin{itemize}
  \item Freemium pricing allows customers a certain level of usage or functionality at no charge-for example, with news content, a software application, or a game. This model is widely used in digital content and services, such as periodicals, video games, software/apps, and cloud storage, where the provider stands to benefit from wide adoption (often via network effects).

  \item Hidden revenue business models provide services to users at no charge and generate revenues elsewhere. This is a common feature of both legacy and digital business models in the media sector, with "free" content and paid advertising. Examples can also be found in online marketplaces (e.g., where sellers pay and buyers do not) and financial services (e.g., free checking accounts, free stock trading).

\end{itemize}

\section{Alternatives to Ownership}
Some business models create value by providing an alternative to purchasing an asset or product, such as the following:

\begin{itemize}
  \item Recurring revenue/subscription pricing ("product as a service") enables customers to "rent" a product or service for as long as they need it. The subscription model is traditionally associated with the media sector, where firms provide access to standardized content that has a low marginal cost (e.g., TV channels, magazines, streaming services, consumer software). Subscription models can also tie revenues to customer needs or usage, as we see in utilities, telecommunication services, real estate, and many business services. The simplicity of electronic invoicing and payments and the value that businesses and investors place on predictable, recurring revenue streams have driven the introduction of subscription pricing models in new areas. Wealth management is a good example, with fees increasingly based on assets managed rather than commissions charged for each stock or bond trade. Subscription models have also been extended broadly to enterprise software ("software as a service"); computing power, storage, and other technology business; charitable giving; and even consumer staples, such as printer ink and disposable razors (Dollar Shave Club).

  \item Fractionalization creates value by selling an asset in smaller units or through the use of an asset at different times. Examples include web hosting, which enables sharing of server capacity, office sub-leasing/co-working (WeWork), vacation property time shares (Marriott), and private jets (NetJets).

  \item Leasing involves shifting the ownership of an asset from the firm using it to an entity that has lower costs for capital and maintenance. Common examples include real estate, automobiles, aircraft, and specialized equipment. - Licensing typically gives a firm access to intangible assets (e.g., a brand name or intellectual property, such as a film library, song, or patented formula) in return for royalty payments (often a percentage of revenues).

  \item Franchising is a more comprehensive form of licensing, in which the franchisor typically gives the franchisee the right to sell or distribute its product or service in a specified territory and to receive marketing and other support.

\end{itemize}

\section{Value Proposition (Who + What + Where + How Much)}
A firm's value proposition refers to the product or service attributes valued by a firm's target customer that lead those customers to prefer a firm's offering over those of its competitors, given relative pricing. Value propositions vary greatly and relate to

\begin{itemize}
  \item the product itself (e.g., capability, performance, features, style),

  \item service and support (e.g., "high-touch" or "low-touch" customer service, depending on the requirements of the customer and the type of product or service, access to repairs, spare parts, etc.),

  \item the sale process (e.g., purchasing convenience, no-hassle returns), and

  \item pricing relative to competitors.

\end{itemize}

Crafting a value proposition requires management to consider carefully which customers it is targeting and through which channels; their wants, needs, and "pain points" being addressed; and competitor product/service offerings and pricing.

Tesla's electric car value proposition emphasizes the benefits of its electric propulsion system: zero emissions and high performance (strong and silent acceleration) and technological sophistication (e.g., self-driving capabilities, frequent enhancements via software upgrades). Low operating costs result in a total cost of ownership below that of other luxury cars. Its direct sales approach helps keep unit costs down without creating inconvenience for customers or limiting sales volume.

\section{EXAMPLE 2}
\section{HD Tools Business Model}
Let's consider a fictional firm, HD Tools, that wants to sell common hand tools, such as wrenches, screwdrivers, and pliers. The firm is considering which one of two business models it should pursue in the market. Elements of the business models are presented below.

\begin{center}
\begin{tabular}{lll}
\hline
 & \multicolumn{1}{c}{Business Model A} & \multicolumn{1}{c}{Business Model B} \\
\hline
Customer Segment & $\begin{array}{l}\text { • apartment dwellers and } \\ \text { new homeowners }\end{array}$ & $\begin{array}{l}\text { • do-it-yourself/professional } \\ \text { trades market }\end{array}$ \\
Product & $\begin{array}{l}\text { a simple, low-priced set } \\ \text { (kit) of tools }\end{array}$ & $\begin{array}{l}\text { a full assortment of } \\ \text { high-quality tools for individ- } \\ \text { ual purchase }\end{array}$ \\
Channel & $\begin{array}{l}\text { • they do not shop at home } \\ \text { improvement stores and are } \\ \text { likely to buy a kit online or } \\ \text { from a mass retailer }\end{array}$ & $\begin{array}{l}\text { improvement stores and spe- } \\ \text { cialty trade distributers }\end{array}$ \\
 &  &  \\
\end{tabular}
\end{center}

\begin{center}
\begin{tabular}{|c|c|c|}
\hline
 & Business Model A & Business Model B \\
\hline
$\begin{array}{l}\text { Customer Profile, } \\ \text { Need }\end{array}$ & $\begin{array}{l}\text { - they may not know what } \\ \text { tools they need and may } \\ \text { have none to start with } \\ \text { - occasional use to do } \\ \text { everyday repairs } \\ \text { - tools do not have to be } \\ \text { heavy duty }\end{array}$ & $\begin{array}{l}\text { - their time is extremely } \\ \text { valuable } \\ \text { - know and differentiate } \\ \text { between tool brands } \\ \text { - heavy use, high-quality } \\ \text { requirement }\end{array}$ \\
\hline
Relative Pricing & $\begin{array}{l}\text { - low and affordable price } \\ \text { point }\end{array}$ & - premium priced tools \\
\hline
Customer Value & $\begin{array}{l}\text { - a kit that contains several } \\ \text { tools they are likely to need } \\ \text { - a simple and compact } \\ \text { toolbox given limited stor- } \\ \text { age space }\end{array}$ & $\begin{array}{l}\text { - will pay a premium for a } \\ \text { tool that they are confident } \\ \text { will be durable and will } \\ \text { perform }\end{array}$ \\
\hline
Service Expectations & $\begin{array}{l}\text { - after-sales support is not } \\ \text { critical, provided there is an } \\ \text { acceptable return policy }\end{array}$ & $\begin{array}{l}\text { - demand high-quality tools } \\ \text { that will perform flawlessly }\end{array}$ \\
\hline
\end{tabular}
\end{center}

Each of these business models is valid for its respective customer segment. Which one makes the most sense for HD Tools to pursue?

\section{Discussion}
Both require significant scale to gain distribution in large retailers. However, Model A (low-priced kits) could potentially be established at a smaller scale online.

With Model A, the challenge will be to have a low enough product cost, or cost of goods sold (COGS), to generate an acceptable margin for the business. Differentiation against other "kit" providers might be achievable through superior toolbox design or construction, a better selection of tools, or perhaps an extra feature not offered by competitors-for example, including a selection of common fasteners, aimed at maximizing customer convenience at a low price point. With Model B, the more critical factor is likely to be product cost and quality. The do-it-yourself (DIY)/professional market requires tools that are of demonstrably high quality so that their customers, who are knowledgeable, will choose them over established rival brands upon seeing them in the store. This might be difficult for a new brand, such as HD Tools. However, HD Tools might be able to overcome this, with objective statements about the quality of materials used in making the tools, their strength and durability, a lifetime warranty, or a "no questions" return policy.

Business models can be viewed in terms of level of product or service differentiation and potential for scale or scope. For example, businesses with "commodity" type offerings that are thus limited to no pricing power might face constraining factors that limit potential scale and growth, such as a regional (regulated) energy provider subject to price caps, or they might have few constraining factors and be relatively unconstrained in growth prospects, such as a mass-market consumer goods manufacturer.

Businesses with high levels of differentiation in their products or services and strong pricing power can set their desired prices given a unique value proposition. Similarly, these businesses might have specific market segments and face scale or scope constraints, such as a luxury watch brand specializing in limited edition time pieces for wealthy consumers, or they might have few constraints on scale or scope with much larger markets, such as a technology platform business. It is important to note that growth business models can also exist in mature markets or industries-for example, fintech firms.

\section{Business Organization, Capabilities: How}
Evaluating a firm's business model requires consideration of not only the value proposition (who it is targeting, what and where the firm is selling, and for how much) but also "how" the firm is structured to deliver that value:

\begin{itemize}
  \item What assets and capabilities (e.g., skilled personnel, technologies) does the firm require to execute on its business model?

  \item Will these be owned/insourced or rented/outsourced?

\end{itemize}

If a firm is renting or purchasing critical resources from other firms, those supplier relationships can become key elements of its strategy and a potential risk.

The auto business has traditionally been organized with automakers supplied by networks of parts manufacturers. Parts manufacturers might supply very specific components (e.g., tires) or complete assemblies (e.g., seats, engines). In some cases, automakers outsource vehicle production entirely. There is generally a very tight working relationship between automakers and their suppliers. Parts or assemblies are often custom designed for a particular vehicle, with the supplier involved from the design stage and throughout the production stage. In these cases, it becomes critical to the automaker to avoid quality issues or supply interruptions at its suppliers.

In contrast, Tesla's business model emphasizes vertical integration, with in-house development and production of key components, such as batteries and software/electronics. This requires substantial capital and effort and is therefore an unusual choice for a new company but is consistent with Tesla's strategy of maintaining a competitive advantage in product technology.

\section{Value Chain}
The "how" aspect of a business model is also referred to as a firm's value chain:

\begin{itemize}
  \item the systems and processes within a firm that create value for its customers.
\end{itemize}

A value chain includes only those functions performed by a single firm, which may be functions that are valued by customers but do not involve physical transformation or handling the product.

Note that a firm's value chain is different from a supply chain, which refers to the sequence of processes involved in the creation of a product, both within and external to a firm. A supply chain includes all the steps involved in producing and delivering a physical product to the end customer, regardless of whether those steps are performed by a single firm.

For some businesses, marketing and sales are strategically critical functions. The "value proposition" addresses what channels are used to reach customers and what value the firm might deliver through the sales and service functions; how to make that happen becomes an important business organization issue.

Tesla's vertical integration strategy extends to distribution, with its company-owned network of stores. It was a major undertaking for Tesla to build this network in multiple countries and to ensure that it achieved the level of sales support and service expected by luxury car buyers. In Tesla's value proposition, the "where" (distribution) element is less critical than the "what" (product features), but it has a significant impact on the company's economics and is a significant part of the "how" (business organization) part of its strategy. Value chain analysis provides a link between the firm's value proposition for customers and its profitability. It involves:

\begin{enumerate}
  \item identifying the specific activities carried out by the firm,

  \item estimating the value added and costs associated with each activity, and

  \item identifying opportunities for competitive advantage.

\end{enumerate}

Michael Porter's 1985 book Competitive Advantage defined five primary activities:

\begin{itemize}
  \item inbound logistics,

  \item operations,

  \item outbound logistics,

  \item marketing, and

  \item sales and service.

\end{itemize}

In addition, a firm's four primary "support" activities are procurement, human resources, technology development, and firm infrastructure. This is a useful starting point for an analyst evaluating the value chain of a company, although dramatic advances in digital technology have radically changed the way that some of these functions are carried out in many businesses.

\section{Profitability and Unit Economics}
When examined, a business model should also reveal how the firm expects to generate its profit. An analyst will want to examine margins, break-even points, and unit economics, which is expressing revenues and costs on a per-unit basis. For example:

\begin{itemize}
  \item A producer of bottle caps might sell its product at 2.5 cents per unit, with direct costs for material and labor of 2.0 cents per unit and a contribution margin (selling price per unit minus variable cost per unit) of 0.5 cents per unit. If the firm has fixed costs of USD500,000 per year, what is its unit break-even point?
\end{itemize}

Break-even point $($ unit) $=$ Fixed costs/Contribution margin

$=\mathrm{USD} 500,000 / 0.5$ cents

$=100$ million units.

\begin{itemize}
  \item A restaurant chain might have an average order of EUR50, with ingredient costs equal to $50 \%$ of sales. If fixed costs are EUR250,000 annually per outlet, what is the firm's unit break-even point and operating margin at 20,000 orders per year?
\end{itemize}

Break-even point (order) = Fixed costs/Contribution margin

$=\mathrm{EUR} 250,000 / \mathrm{EUR} 25$

$=10,000$ orders/year.

Operating margin $=20,000$ orders $\times$ EUR50

$=$ EUR $1,000,000$ revenues - EUR500,000 ingredient costs (

$=50 \%$ of sales) - EUR50,000 fixed costs

$=\mathrm{EUR} 250,000$ operating profit, or $=\mathrm{EUR} 250,000 / \mathrm{EUR} 1,000,000=25 \%$

\begin{itemize}
  \item A custom homebuilder might price its services at cost plus a $20 \%$ markup. In this case, there is no simple way to express output in terms of units, but it is not necessary to do so.
\end{itemize}

Tesla's business model is based on a decline over time in unit revenues and costs as volumes increase and technology improves. This would be expected to create a virtuous circle: Lower prices enable Tesla to expand its addressable market and its market share, while lower costs allow profits to rise and create a barrier to competition.

Exhibit 4 summarizes key business model features for analyst consideration.

Exhibit 4: Business Model Features for Analysis

\begin{center}
\includegraphics[max width=\textwidth]{2023_05_04_b5cfa4f1bc883752f121g-714}
\end{center}

\section{BUSINESS MODEL TYPES}
Each industry tends to have its own established business models. In goods-producing sectors, it is generally easy to classify firms based on how they fit into the supply chain, such as manufacturers, wholesalers, retailers, and various suppliers of raw materials, components, equipment, and services. Describing a firm as a "plumbing supplies wholesaler," for example, gives an analyst a good starting point in understanding the business model of a typical company in that business.

Service businesses are more diverse. Some target consumers (B2C); some sell to other businesses (B2B). Some involve physical products (e.g., importing, selling, testing, repairs); most do not. There are service business models specific to each sector. For example, the financial services sector includes many established business models, as shown in Exhibit 5.

\section*{Exhibit 5: Financial Services Business Models }
Sector: Financial services

Business model

\begin{itemize}
  \item Lending and deposit-taking

  \item Insurance

  \item Payment processing

  \item Brokerage

  \item Trading

  \item Merchant banking

  \item Asset management

  \item Combination of above

\end{itemize}

Some firms combine these ("universal banks"), while some specialize. The same is true for other service sectors, such as health care, transportation, real estate, lodging, and entertainment.

\section{Business Model Innovation}
Most discussion of business models focuses on innovation: how new business models can be introduced in conjunction with new businesses or adapted to existing markets. Digital technology has spawned many new businesses, such as

\begin{itemize}
  \item software,

  \item content,

  \item digital advertising,

  \item data and related services, and

  \item a wide array of internet-based communities and marketplaces.

\end{itemize}

These services are not entirely new. There was linear television before streaming video and classified advertising before online marketplaces. But digital technology has changed these services-and the economics of delivering them-almost unrecognizably.

More specifically, digital technology has transformed the "where" and the "how" elements of business models in established markets, by radically reducing the cost of communicating, exchanging information, and transacting between businesses.

\begin{itemize}
  \item Location matters less; digital communications and, in particular, e-commerce enable customers to shop and purchase more easily from firms having no local physical presence.

  \item Outsourcing is easier, for similar reasons.

  \item Digital marketing makes it easy and cost-effective to reach very specific groups of customers, regardless of location, and to engage more deeply with them than was possible with traditional advertising

  \item Network effects, discussed below, have become more powerful and accessible to more firms. The rate of innovation itself creates pressure on firms to be agile and adapt, to avoid disruption. In some cases, this can lead them to outsource, in order to focus on what they do best (e.g., apparel companies); in other cases, firms will "insource" in order to strengthen their competitive position, rather than relying on outside suppliers (think Apple with processor chips, Tesla with batteries). While large-scale business model innovation did not start with digital technology, the rapid and open-ended advance of digital technology has dramatically changed how businesses operate.

\end{itemize}

\section{Business Model Variations}
However, there are still many business model variations, such as the following:

\begin{itemize}
  \item Private label or "contract" manufacturers that produce goods to be marketed by others. This is an extremely common arrangement, particularly for offshore production.

  \item Licensing arrangements in which a company will produce a product using someone else's brand name in return for a royalty. This is common in toys and apparel, for example, when manufacturers might pay for the right to use the name of a famous film character, a sports team, or a brand that has become popular in a related category (e.g., sporting goods).

  \item Value added resellers that not only distribute a product but also handle more complex aspects of product installation, customization, service, or support. This is common with complex, service-intensive products, such as construction machinery, heating/air conditioning systems, and other specialized equipment.

  \item Franchise models in which distributers or retailers have a tightly defined and often exclusive relationship with the parent company. Franchising is typically used in multi-location service-intensive businesses, where the franchisee handles sales and service and uses the parent (franchisor) business model and brand. Compensation is usually via a royalty arrangement and/or a markup on products sold.

\end{itemize}

\section{E-Commerce Business Models}
E-commerce is a very broad category, encompassing a variety of internet-based direct sales models. The following are a few key business model variations within e-commerce:

\begin{itemize}
  \item Affiliate marketing generates commission revenues for sales generated on others' websites. It is a specific type of "performance marketing," which refers to an arrangement in which a marketer or agency is paid to achieve defined results (leads, clicks, or actual sales). Examples include CJ Affiliate, Awin, and Leadbit.

  \item Marketplace businesses create networks of buyers and sellers without taking ownership of the goods during the process. Examples include Alibaba, eBay, Mercado Libre, and Etsy.

  \item Aggregators are similar to marketplaces, but the aggregator re-markets products and services under its own brand. Examples include Uber and Spotify.

\end{itemize}

\section{Network Effects and Platform Business Models}
Network effects refer to the increase in value of a network to its users as more users join. Many internet-based businesses are built on network effects. For example, China's WeChat messaging and payments platform is valuable to its users in large part because it is used by so many people. Other examples can be found in such internet businesses as online classifieds, social media, and ride-sharing services.

Network effects are also at work in many older, non-internet businesses, such as telephone service, credit cards, real estate agencies, and stock exchanges. In some cases, network effects apply to two or more groups of users, such as buyers and sellers in an online marketplace, and can be described as "two-sided" or "multi-sided." "One-sided" network effects apply in the case when users are a single, homogeneous group.

A platform business is based on a network and can be distinguished from a traditional or "linear" business that adds value to something that is sold to customers in a linear supply chain. Software companies for example, often have an essentially linear business model. With a linear business, value is added by the firm; with a platform business, value is created in the network, outside the firm.

\section{Crowdsourcing Business Models}
Crowdsourcing business models enable users to contribute directly to a product, service, or online content. Examples include contests and competitions; online gaming; product development, such as open source software; knowledge aggregation, such as Wikipedia and Waze/Google Maps; fan or hobbyist clubs; and networks of tradespersons or professionals. Other examples include customer reviews and feedback, such as Amazon and Tripadvisor. Many of these examples involve "user communities" that enable voluntary collaboration between users of a product or service, typically with little to no oversight.

\section{Hybrid Business Models}
Hybrid models, combining platform and traditional "linear" businesses, are also common. For example, Amazon's core business has both traditional elements (goods distribution) and platform elements (online marketing and advertising). Tesla sells cars (via a linear model), but its customers benefit from an expanding network of charging stations. Intuit sells QuickBooks accounting software (a linear business) but benefits from the fact that many accountants know how to use it.

\section{EXAMPLE 3}
\section{Business Model Evolution in the Hotel and Travel Industry}
The hotel industry provides a good illustration of business model evolution over time and the emergence of different business models with varying financial characteristics.

The hotel industry has ancient origins, likely based on an early version of home sharing. The world's oldest operating hotel, Nishiyama Onsen Keiunkan in Japan, is over 1,000 years old and has been in the same family for 52 generations. Until the 20th century, business model evolution was slow. In response to increasing scale and specialization demanded by a growing market, hotels became larger and more numerous but with little change in the basic business model. The 20th century brought major changes to business models in the industry. The single-property hotel business remained a relevant business model due to the inherent uniqueness of hotel properties. However, with a much larger and highly mobile customer base, hotel operators saw an opportunity and a need to

\begin{itemize}
  \item serve a growing corporate travel market,

  \item increase the scale and footprint of their businesses,

  \item provide convenience and consistency, and

  \item increase operating efficiency.

\end{itemize}

The results of this change were the following:

Scale: The emergence of large hotel chains with multiple locations to serve highly mobile customers and brands to serve numerous market segments.

Example: InterContinental Hotels \& Resorts has more than 15 brands in approximately 6,000 locations whose markets range from basic to luxury and extended stay.

Specialization: The emergence of specialized lodging businesses to serve specific market segments.

Example: Resort hotels, vacation packages (bundled flights plus lodging plus meals), casinos, weekly/monthly accommodation (for out-of-town executives).

Franchising: The application of franchising to the hotel business.

Example: Hilton Hotels \& Resorts. The vast majority of Hilton properties are not owned or leased by Hilton but instead are operated by franchisees who pay fees.

Functional separation: A move to specialized businesses handling such functions as branding and marketing, property ownership, management, and development.

Example: REITs and property ownership. The largest hotel companies seldom own all of their hotels; some own none. Host Hotels \& Resorts, the world's largest hotel REIT and the only investment-grade rated lodging REIT, owns close to 100 hotel properties and is the largest third-party owner of Marriott and Hyatt hotels.

Fractionalized ownership: The introduction of fractionalized ownership in the form of time-sharing, creating a new lodging category between the hotel and the vacation home.

Example: Wyndham Destinations is the largest vacation ownership company. The company develops, sells, and manages time-share properties under various vacation ownership clubs.

Loyalty programs: The introduction of programs to increase brand loyalty among high-value, frequent travelers.

Example: Hotel loyalty programs first introduced by Holiday Inn and Marriott in 1983. Online travel agency (OTA): The emergence of the online travel agency business to address the complexities of travel planning and bookings. Digital technology has accelerated changes in hotel business models. The most important single change has been in the agency business, with the rise of the online travel agency business model. In both its traditional and online forms, the travel agency business is a two-sided network.

Like a traditional travel agent, the OTA assists travelers with research and planning, price comparisons, bookings, and logistics. For hotels, it provides exposure, leads, bookings, customer information, feedback, and competitive intelligence. Automating these functions and delivering them as a web-based service so greatly improved their convenience, speed, and efficiency that the business model was transformed, shifting the relationship and the balance of power between agencies and their hotel customers.

Examples of differing OTA business models are shown in the following table.

OTA Business

Model

Description

Example(s)

Price-comparison "aggregators"

These network-based businesses offer buyers and sellers within the hotel and travel community travel-related price comparison services and bookings.

Crowdsourcing

These platform businesses provide crowdsourced reviews and information on hotels and other travel services.

Home sharing/ These platform businesses have challenged and disrupted the traditional hotel model by supplying a variety of temporary, often unique, accommodations. \href{http://Booking.com}{Booking.com}, Expedia, Trip. com, eDreams

\href{http://Tripadvisor.com}{Tripadvisor.com}

Airbnb; Vrbo (acquired in 2006 by HomeAway, which, in turn, was acquired by Expedia in 2015)

Hotel operators have responded to these challenges by investing in their own websites, customer data, and direct booking capabilities. In response to competition from alternative accommodations, traditional hotel chains, which have historically emphasized consistency, have launched so-called soft brands, which are hotels that operate under their own name with a greater degree of local operator autonomy.

\section{BUSINESS MODELS: FINANCIAL IMPLICATIONS}
Describe expected relations between a company's external environment, business model, and financing needs.

Businesses have very different financing needs and risk profiles, driven by their business model and other factors. Whether capital is readily available to finance a business and what type of capital depends on these factors, which may be external or firm specific.

Lenders in particular look closely at the predictability and stability of demand, revenue and margins of the business (whether there are cash flows to support the servicing and repayment of the debt), and whether the assets in the business can be resold if the business fails (whether there is collateral to secure the debt). Equity analysts and investors also consider these factors but focus more on the long-term potential of the business.

\section{External Factors}
External factors include the following:

\begin{itemize}
  \item Economic conditions affect almost all businesses. In addition to GDP growth, other macroeconomic variables, such as exchange rates, interest rates, the credit environment, unemployment, and inflation, can also be important, depending on the sector. While security markets and the financial press focus on short-term movements in economic data, most businesses are concerned with longer-term trends. There are important variations by sector. For example, a consumer credit business is likely to be highly sensitive to fluctuations in unemployment and GDP growth, while the economics of a wind farm business are likely to be very sensitive to interest rates and the long-term outlook for alternative energy costs and less sensitive to overall GDP growth. In addition, a country's level of economic development is also important when assessing market potential.

  \item Demographic trends influence the overall economy, but in certain markets, they are important in their own right. Mature, urbanized economies are increasingly characterized by aging and, in some cases, declining populations, with growing labor shortages. In most emerging markets, notably in Latin America and Africa, population growth and labor availability remain high. Unlike economic forecasting, long-term demographic forecasting can be done with relatively high accuracy since the underlying drivers (birth, death, and immigration rates) change very slowly.

  \item Sector demand characteristics vary by industry. Some industries, such as consumer staples, have very stable and predictable demand, while others, such as industrial machinery, are more cyclical. The business cycle tends to have less effect on demand for products that are non-discretionary and consumed immediately than on demand for long-lived products or assets. For example, toothpaste and breakfast cereal would be classified as consumer staples, with less cyclicality than clothing, autos, housing, or the commodities that go into making autos or housing. The machinery used to make cars or other "capital goods" are also long-lived assets, with cyclical demand. For long-lived goods, the timing of their replacement is often somewhat discretionary, even if the goods themselves are not. - Industry cost characteristics, such as capital intensity and operating leverage, are also important. Some industries are inherently capital intensive, such as hotels, utilities, and airlines, while others tend to require very little capital, such as many internet-based and service businesses. In addition, the degree of operating leverage differs by business. Businesses with low variable costs and high contribution margins have high "operating leverage" and are said to be "scalable," with operating margins expanding rapidly as business revenues grow. Examples of operations that may scale in this manner include many software, media, and online marketplace businesses.

  \item The political, legal, and regulatory environment is also a key "external" factor for many businesses. It includes the institutions, laws, regulations, and policies that affect the business. Some aspects are background factors that affect business generally, with businesses preferring an environment in which laws and regulations are stable, predictable, and consistently applied and where contracts are easily enforceable. Some regulations create constraints or support for specific industries or businesses. Examples include licensing, environmental, product safety, and trade regulations. While businesses often balk at regulation, keep in mind that regulations that constrain businesses, such as the requirement to obtain a license or to adhere to product standards, can also create barriers to competition. Such barriers can protect a firm's profit margins and add value to the business.

  \item Social and political trends: Shifts in public opinion and tastes often precede changes in consumer buying behavior or the political/legal environment. Examples include consumer preferences for green products, the shift to renewable energy, healthy e-living, and remote learning/working. That said, it can be risky for analysts to generalize broadly from trends or headlines.

\end{itemize}

\section{Firm-Specific Factors}
Firm-specific factors include the following:

\begin{itemize}
  \item Firm maturity or stage of development of the business: A startup or early-stage business typically requires more capital, such as that needed to finance new facilities or for "investment spending" on product development, marketing/sales, working capital, and/or startup losses, and presents more business risk than a more mature business.

  \item Competitive position: A company with strong barriers to competition, also referred to as a "wide moat," will have lower business and financial risk than one that does not, other things being equal. Companies that are leaders in their markets often enjoy scale and brand advantages over smaller players.

  \item Business model: Some businesses are inherently capital intensive, requiring investment for facilities, productive capacity, and/or working capital, while others are more human capital intensive. A firm's decisions about which assets and resources to "own" rather than "rent" are often driven by its business model and will greatly affect its financial profile. For instance:

  \item Asset-light business models shift the ownership of high-cost assets to other firms. Examples can be found in traditionally capital-intensive businesses, such as hotels, restaurants, and retailing, where the physical assets are owned by the franchisee rather than the parent company.

  \item Lean startups extend this logic to human resources, outsourcing as many functions as possible. Technology companies frequently adopt this approach, to accelerate their development and to increase their agility. Note that Tesla has embraced the opposite approach: Whereas most automakers rely on networks of parts suppliers, Tesla has chosen to vertically integrate in order to control and accelerate product development. Apple has embraced a similar strategy.

  \item Pay-in-advance business models reduce or eliminate the need for working capital. Companies that can generate cash from sales before paying suppliers can operate with minimal or even negative working capital. As the business grows, working capital can become a source of cash. Examples include Amazon (e-commerce) and Berkshire Hathaway Specialty Insurance (insurance).

\end{itemize}

\section{EXAMPLE 4}
\section{Three Ways to Sell Cars}
In this example, we consider the business of selling used cars. While the underlying product is the same-used cars-we show that each business has a different underlying business model, resulting in different assets, financing requirements, and expected profitability.

Consider the three following businesses.

\begin{itemize}
  \item D Cars (D): A traditional dealership buys and resells vehicles, which are cleaned, inspected, and held as inventory on its balance sheet between purchase and resale.

  \item C Cars (C): A consignment dealer sells vehicles on behalf of their owners from its own premises but without taking ownership. The seller receives funds only when the car is sold to the new owner, and the consignment dealer, $C$, then receives a commission. In this transaction, $C$ is essentially a broker or agent and does not take ownership of the vehicle, and so the seller might receive a higher net price.

  \item O Cars $(\mathbf{O})$ : A two-sided online marketplace that allows owners to list vehicles for sale and has no physical premises. As with the consignment model, the transaction between seller and buyer does not directly involve $\mathrm{O}$, but $\mathrm{O}$ receives a listing fee from the seller. This is less convenient for both buyer and seller, but both should benefit financially by avoiding the dealer profit or commission.

\end{itemize}

These three models offer different value propositions for buyers and sellers of used vehicles. The two customer groups in this market are shown in the following table.

\section{Value Proposition Summary}
D Cars

C Cars

O Cars

\begin{center}
\begin{tabular}{lccr}
\hline
 & Traditional Dealer & Consignment Sales & Online Marketplace \\
\hline
Dealer pays seller & Buyer pays seller seller & $\begin{array}{c}\text { Seller pays dealer } \\ \text { commission }\end{array}$ &  \\
\hline
Buyer pays dealer & Seller pays dealer fee &  &  \\
\hline
\end{tabular}
\end{center}

Buyer Value Proposition

Vehicle selection

Good

Good

Better

Compare cars in one place

Yes

Yes

No

Vehicle inspected and cleaned

Yes

Yes

No

\begin{center}
\begin{tabular}{|c|c|c|c|}
\hline
 & D Cars & C Cars & O Cars \\
\hline
 & Traditional Dealer & Consignment Sales & Online Marketplace \\
\hline
 & Dealer pays seller & Buyer pays seller & Buyer pays seller \\
\hline
 & Buyer pays dealer & $\begin{array}{l}\text { Seller pays dealer } \\ \text { commission }\end{array}$ & Seller pays dealer fee \\
\hline
$\begin{array}{l}\text { Assistance with licensing and } \\ \text { paperwork }\end{array}$ & Yes & Yes & No \\
\hline
Price paid & Average & Better & Best \\
\hline
\multicolumn{4}{|l|}{Seller Value Proposition} \\
\hline
Convenience/one-stop shop & Yes & Yes & No \\
\hline
Time to transact & Fast & Slower & Slower \\
\hline
Price realized & Average & Better & Best \\
\hline
\end{tabular}
\end{center}

The three businesses also have differing assets and cost structures. The costs for $\mathrm{D}$ and $\mathrm{C}$ are fairly similar because both require a physical showroom/ lot to sell cars and involve on-site vehicle inspections, sales, and operations. C, however, does not have to finance or bear the risk associated with inventory.

O's business costs are significantly lower since sales occur online, with no physical premises or on-premise sales staff. D's assets include cars held as inventory between purchase and resale. Both $\mathrm{D}$ and $\mathrm{C}$ require physical premises with employees to sell cars, while $\mathrm{O}$ requires neither inventory nor a physical sales lot, reducing its asset base and financing need. It does, however, require a supported website.

Business Costs and Capital Summary

D Cars

C Cars

O Cars

\begin{center}
\begin{tabular}{|c|c|c|c|}
\hline
 & Traditional Dealer & Consignment Sales & Online Marketplace \\
\hline
 & Dealer pays seller & Buyer pays seller & Buyer pays seller \\
\hline
 & Buyer pays dealer & Seller pays commission & Seller pays fee \\
\hline
Business Cost Structure & High & High/medium & Low \\
\hline
Advertising & Yes & Yes & Yes \\
\hline
Vehicle inspection and cleaning & Yes & Yes & No \\
\hline
Showroom operations & Yes & Yes & No \\
\hline
On-premise sales force & Yes & Yes & No \\
\hline
Website maintenance & No & No & Yes \\
\hline
Capital Required & High & Medium & Low \\
\hline
Showroom/Lot & Yes & Yes & No \\
\hline
Inventory & Yes & No & No \\
\hline
\end{tabular}
\end{center}

Let's consider an example involving the sale of a vehicle with a standard retail value of 20,000 . D Cars: If the dealer earns a margin of $20 \%$ or 4,000 and incurs direct costs of 1,000 to inspect, clean, and advertise the vehicle, it would have a contribution margin of 3,000 to cover the cost of its facilities, the cost of sales staff, the cost of capital required by the business, and the risk that it has misjudged the market value of the car.

C Cars: While the process would be similar to D Cars, the seller in this case would set an asking price in consultation with $C$ to ensure that the car is marketable. The final price would be negotiated between seller and buyer, and $C$ would charge a commission on the final sale. The commission might be a flat fee, a percentage of the negotiated price, a cost-plus arrangement that reimburses $C$ for its direct costs, or some combination thereof. Since $C$ does not have to bear the inventory risk or provide capital on the transaction, it is likely to earn less than if it did do these things. If we assume a commission of 3,000, by dealing with $\mathrm{C}$ the seller receives 17,000 .

O Cars: With O Cars, the sale process is markedly different. The seller would list her vehicle for a fixed fee; prospective buyers would then arrange to meet to see it, test-drive it, and arrange for their own inspection. This process is more time consuming for both buyer and seller. The seller is likely to demand a higher net price (say, 18,000). If we assume a listing fee paid by the seller of, say, 100, then the buyer would pay less than by going to $\mathrm{C}$ or $\mathrm{D}$, compensating for the additional effort involved. For $\mathrm{O}$, the listing fee of 100 is a fraction of the 3,000 commission that $C$ would receive and the 4,000 gross margin that $D$ would receive; however, $\mathrm{O}$ has much lower operating costs and capital need than $\mathrm{C}$ or $\mathrm{D}$.

The following table summarizes this information for each business.

\begin{center}
\begin{tabular}{lccc}
\hline
 & D Cars & C Cars & O Cars \\
\hline
Buyer pays & 20,000 & 20,000 & 18,000 \\
Margin/commission/fee & 4,000 & 3,000 & 100 \\
Seller receives & 16,000 & 17,000 & 17,900 \\
\hline
\end{tabular}
\end{center}

Looking at the three business models:

\begin{itemize}
  \item With D Cars, the dealer adds more value and requires a higher gross profit per vehicle.

  \item With C Cars, the seller bears more risk but receives a higher net price.

  \item With O Cars, both seller and buyer receive a better price, reflecting the effort and inconvenience of having to transact without a dealer to facilitate. $\mathrm{O}$ earns a relatively low fee but has a very low cost structure.

\end{itemize}

In practice, the used vehicle market includes all three of these business models, which cater to different buyer and seller preferences. New car dealers are also active in the used car market. Like all marketplace businesses, buying and selling used cars requires a critical mass of buyers and sellers in order to succeed and become an effective two-sided network.

In the case of the $\mathrm{D}$ and $\mathrm{C}$ businesses operating from a single lot, this can be difficult to achieve since the addressable market includes only buyers and sellers who are within a convenient distance of their lot. The online model of $\mathrm{O}$ has a similar challenge: It may attract buyers and sellers from anywhere, but they will be willing to transact only if they are convenient to each other.

\section{BUSINESS RISKS}
\section{Summary}
\begin{itemize}
  \item Risk factors with a direct influence on the long-term financial viability of a business are macro risks, business risks, and financial risks.

  \item Business risk is the risk that the firm's operating results will be different from expectations, independently of how the business is financed, and includes both industry risk and company-specific risk.

  \item Main industry risk factors include cyclicality, industry structure and concentration, competitive intensity, competitive dynamics in the value chain, long-term growth, and demand outlook.

  \item Main company-specific risk factors include competitive position, product market risk, execution risk, and operating leverage.

  \item These risk factors and risks are cumulative and often multiplicative.

\end{itemize}

Financial analysts and investors need to consider risk factors that might cause investment returns to be different from expectations. Business risk encompasses factors related to the business itself and the industry in which it operates. We distinguish business risk from macro risk, which relates to the overall environment in which the business operates, and financial risk, which relates to how the business is financed.

The perspective of debt and equity investors is different but has many common elements. Debtholders expect the timely repayment of principal and interest, with interest paid timely at an agreed-on rate. They must consider the risk that the borrower will default and if that occurs, the likely magnitude of their loss. For equity investors, the potential for loss is more complicated. Their returns consist of dividends received and changes in the value of the business, which, in turn, hinge not only on how the business performs but also on changes in expectations for future performance. Both consider the cash flow-generating ability of the business as an indicator of future performance.

It is also worth noting that investors who buy and sell investments face risk from security price fluctuations. The price fluctuations reflect not only changes to the fundamentals of the business but also changes in investor sentiment, expectations, and market conditions.

The risk framework presented here classifies risks by type and source. Investors cannot influence the company's business, so their goals are to understand and evaluate those risks and underlying risk factors and risk drivers. Business risks are in the domain of management, who do influence the business and whose risk framework would typically distinguish between preventable risks (to be avoided), strategic risks (consciously taken to achieve business benefits), and external risks (outside management control, to be monitored and mitigated).

\section{MACRO RISK, BUSINESS RISK, AND FINANCIAL RISK}
Risk that management, debtholders, and equity investors consider arises from the economic environment in which a business operates (macro risk), the business itself (business risk), and the way the business is financed (financial risk), as Exhibit 6 shows.

\section{Exhibit 6: Risks Impacting the Business}
\begin{center}
\includegraphics[max width=\textwidth]{2023_05_04_b5cfa4f1bc883752f121g-726}
\end{center}

Macro risk refers to the risk from political, economic, legal, and other institutional risk factors that impact all businesses in an economy, a country, or a region. In most situations, the principal macro risk is the potential for a slowdown or a decline in economic activity (measured by changes in GDP) and associated demand. Depending on the geographic location of the business, other country risk factors, such as exchange rates, political instability, or gaps in the legal or financial framework, may also be important. Some firms sell to customers in multiple countries, which can reduce country risk (although not necessarily macro risk).

While macro risks apply to essentially all businesses in the country, some industries are relatively insensitive to economic activity levels (e.g., utilities, consumer staples), whereas others are more sensitive (e.g., capital goods and consumer discretionary goods, such as jewelry and vacation travel).

Business risk is the risk that the firm's operating results will be different from expectations, independently of how the business is financed. In accounting terms, business risk reflects the risk at the operating profit (EBIT) level. Business risk can be seen as the risk of a revenue shortfall and/or higher-than-expected costs, potentially magnified by operating leverage.

Business risk includes both industry risk and company-specific risk, as Exhibit 7 shows. Some risk frameworks would distinguish "external" and "internal" risk factors, with external risks including both macro and industry risks.

\section{Exhibit 7: Business Risk and Its Components}
\begin{center}
\includegraphics[max width=\textwidth]{2023_05_04_b5cfa4f1bc883752f121g-727}
\end{center}

Financial risk refers to the risk arising from a company's capital structure and, specifically, from the level of debt and debt-like obligations (such as leases and pension obligations) involving fixed contractual payments. These fixed financial charges cause net profit and cash flow to vary by more than operating profit, on both the upside and the downside-hence the term financial leverage. They also create the possibility that the firm will be unable to secure needed financing on competitive terms: "financing risk." Default risk is also a result of financial leverage. As we will see, even without a default, excessive financial leverage can lead a firm to operate in a sub-optimal manner, as it struggles to conserve cash, key customers, and employees.

Debtholders' perspective on risk differs from that of the company and its equity investors. Debt covenants and collateral reduce risk to the lender but increase risk for equityholders. Long-term fixed-coupon debt reduces interest rate and refinancing risk for the borrower but exposes the holder to risk from interest rate fluctuations (relative to short-term or floating-rate debt). Exchange rate risk arises for the borrower when debt and revenues are denominated in a different currency; for lenders, exchange rate risk can also arise when debt is in a different currency than their own reference currency. In addition, firms will often have multiple loans or debt issues outstanding, with differences in seniority, covenants, collateral, and other terms. These differences create differences in risk faced by the firm's debt investors.

\section{Risk Impacts Are Cumulative}
The impact of risks on a business and the returns debtholders and equity investors earn from the business is cumulative. While we make a distinction between macro risks and business risks, both affect a company's operating results. When we look to measure business risk, it will reflect the impact of macro risk. Likewise, we make a distinction between business risk and financial risk, but both (together with macro risk) affect the company's cash flow, net earnings, and ability to service debt. Financial risk will reflect the combined impacts of macro issues, business risks, and financial leverage.

\section{BUSINESS RISK: A CLOSER LOOK}
Business risk includes both industry risk and company-specific risk, which we now will examine.

\section{Industry Risks}
Industry risks apply to all competitors in the same industry and include risk factors likely to affect the overall level of demand, pricing, and profitability in the industry.

\begin{itemize}
  \item Cyclicality is a feature of many industries, particularly discretionary goods; housing; durable goods, such as autos and appliances; and capital equipment. These products are generally long lived, giving the buyer flexibility as to when to replace them. Inputs to those industries-including commodities, such as lumber, steel, and copper, and certain business services (e.g., heavy equipment rentals)-also face cyclical demand. For cyclical businesses, revenue fluctuations generally cause even larger fluctuations in operating profit, due to the impact of fixed costs.
\end{itemize}

Firms attempt to mitigate the impact of cyclicality in different ways. Some firms try to stabilize revenues by entering long-term customer or hedging contracts. Some attempt to minimize fixed operating costs-for example, through outsourcing or flexible contracts with workers and suppliers. Generally, firms with cyclical (or otherwise unpredictable) revenues will tend to have more conservative capital structure policies, with relatively little debt and therefore smaller fixed financing obligations, compared with less cyclical firms.

\begin{itemize}
  \item Industry structure has an impact on the overall risk of the industry. Lower concentration (i.e., the presence of many small competitors) is generally associated with a high degree of competitive intensity, although the presence of many small players is also a feature of sectors that are very service oriented, are local in nature, or have high product differentiation. Examples include law firms, electrical contracting, boutique hotels, and niche software providers.
\end{itemize}

A common measure of industry concentration is the Herfindahl-Hirschman Index (HHI), calculated as the sum of the squares of competitor market shares. The HHI would be 1.00 for a one-player market, 0.50 for a market with two equal players, 0.33 if there are three equal players, and so on. (Index values are different when competitors have unequal market shares, as is normally the case.)

\begin{itemize}
  \item Competitive intensity influences overall industry profitability. If the data are available, one can measure industry profitability directly. Return on invested capital (ROIC) and operating profit margins (EBIT/revenue) are common measures. Analysts look at the absolute levels for these measures and changes over time (correcting for cyclical variations). They also seek to understand the factors that influence competitive intensity for an industry and how those factors are changing.

  \item Competitive dynamics within the value chain-that is, potential profitability pressures from the interaction of buyers, suppliers, current and potential competitors, and suppliers of substitute goods. Michael Porter's "five forces" model (mentioned previously) describes these factors in detail.

  \item Long-term growth and demand outlook are more a determinant of industry attractiveness than of risk, but an unexpected falloff (or absolute decline) in growth can result in excess capacity and more aggressive competition. Long-term growth trends are determined by the level of innovation, the pervasive and dominating business models, and the age/maturity of the industry. Long-term growth can contribute to long-term profitability, although it can also attract more competition. - Other industry risks include regulatory and other potential external risks to industry demand and profitability.

\end{itemize}

\section{Industry Definition}
Care must be taken to define the industry appropriately when evaluating risk and more generally when analyzing a business. If the analyst defines an industry too broadly, conclusions about the market and competitive conditions may be too broad to be useful. Within the auto parts sector, for example, the markets for brakes, tires, and entertainment systems are very different, with different competitors. Within the lodging sector, there are wide variations between geographic markets (e.g., Paris versus Rio de Janeiro) and across segments (e.g., beach hotels versus corporate extended-stay lodging). In general, a more precise (i.e., narrower) industry definition is preferable, since it enables the analyst to focus on the specific factors affecting the firm. However, if the industry is defined too narrowly, it can become difficult to obtain accurate data on demand and competition and to give appropriate emphasis to trends, issues, or risks affecting the industry as a whole.

\section{Company-Specific Risks}
Company-specific risks vary based on the nature, scale, and maturity of the business. They are often closely related to the company's market position and business model. Generally speaking, these risks are likely to be greater for businesses that are smaller, less dominant in their industry, or at an early stage of development (i.e., when their profit potential or even the revenue potential is not yet proven). Company-specific risk categories include the following:

\begin{itemize}
  \item Competitive risk can be defined as the risk of a loss of market share or pricing power to competitors and often reflects a lack of competitive advantage. Pricing power is analogous to the price elasticity of demand: the likely decline in demand if the company were to raise its prices (e.g., by 10\%) and competitors did not follow. When market share and pricing power are low or declining, the level of competitive pressure and risk is generally high. Typically, companies with strong competitive positions in attractive industries tend to have high margins and, other things being equal, lower business risk.
\end{itemize}

Competitive risk also arises from the potential for disruption: when new or potential competitors using new technology or business models take market share, rather than known or established competitors using established business models.

\begin{itemize}
  \item Product market risk is the risk that the market for a new product or service will fall short of expectations. It is not generally a consideration for mature or well-established businesses. But for an early-stage startup, this is often the single largest source of business risk. Risk is typically very high for pre-revenue companies; it declines as the firm progresses from concept development through product testing and market testing (still with no revenues) to commercial launch, revenue generation, and then larger-scale roll-out. Analysts should also consider the possibility that product life cycles may be cut short, by such factors as changing consumer preferences, product obsolescence, or the end of patent protection. Product mix and diversity can reduce risk for large firms that sell multiple products to diverse customers and markets. At the same time, a broad product portfolio can reduce the ability of the business to address problems across its product portfolio.
\end{itemize}

Much has been written on the sources of competitive advantage. Companies can benefit from the following:

\begin{itemize}
  \item Cost advantages, which can be based on scale, a superior production process, proprietary technology, or other factors.

  \item Product or service differentiation, which creates value for customers. This could be based on traditional attributes, such as reliability, durability, and performance; on how the product is sold, bundled, or serviced; on branding; or on network effects.

  \item Network effects, which refer to the increase in utility for some services and products when they are widely adopted. Examples include social media networks, Airbnb, or companies marketing construction equipment that rely on having a trained network of qualified mechanics.

  \item Switching barriers are factors that make it more difficult or more costly to switch suppliers. These can derive from long-term supply contracts, the need for the customer to invest in training or new systems when changing suppliers, or other factors.

\end{itemize}

The larger and more durable these advantages, the lower the level of competitive risk. Note that these factors affect competitive risk for a particular firm; they are in addition to the factors discussed earlier that are applicable to the industry as a whole.

\begin{itemize}
  \item Execution risk arises from the possibility that management will be unable to do what is needed to deliver the expected results. Execution risk tends to be accentuated by other business risks: A small or early-stage business with a weak or deteriorating market position and/or in a highly competitive industry is likely to have higher operational risk than a large, established, stable business. For such a business, there is likely to be less room for management error and more work to do in order for the business to achieve its plan. Analysts should also look for specific issues that could create operational risk, such as reliance on key suppliers or personnel, challenging turnarounds or IT platform migrations, or high-risk new product launches.

  \item Capital investment risk is the potential for sub-optimal investment by a firm. This tends to be a bigger concern for mature businesses that generate cash flow but lack natural reinvestment opportunities in their current business and in firms where management has shown a propensity to make high-profile ego-driven investments. While some companies have diversified successfully, either organically or through mergers and acquisition, many have not.

  \item ESG risk traditionally focuses on governance risk: the potential misalignment of objectives between shareholders focused on value maximization and management (who might pursue growth for its own sake or resist necessary major changes to the business). It is impossible to eliminate governance risk completely, but analysts generally look for strong independent directors, separation of the CEO and chairman roles, executive compensation that properly incentivizes shareholder value creation, and significant share ownership by the board and senior management. Analysts must also consider the risks arising from potential failure of the business to meet society's expectations for environmental and social responsibility. - Operating leverage refers to the sensitivity of a firm's operating profit to a change in revenues. For businesses with low variable costs (e.g., software, media), operating leverage is high. For successful, growing businesses, high operating leverage can have a very positive influence on profitability and value. For businesses that are struggling or facing declining demand, high operating leverage can be problematic and is a source of business risk. We discuss the calculation of operating leverage below.

\end{itemize}

\section{FINANCIAL RISK}
Financial risk (as distinguished from business risk) refers to the risk arising from a company's capital structure and, specifically, from the level of debt (and other debt-like obligations, such as leases and pension obligations) involving fixed contractual payments. These fixed financial charges cause net profit and cash flow to vary by more than operating profit, on both the upside and the downside-hence the term "financial leverage." They also create the possibility that the firm will be unable to secure needed financing (financing risk) or that it will fail to meet its financial obligations (default risk). Default can be expensive and traumatic. But even without a default, excessive financial leverage impacts the firm's financial capacity and leads a firm to become financially distressed - to operate in a sub-optimal manner as it struggles to conserve cash, key customers, and employees.

Financial risk is closely related to the variability of profits and cash flows. These, in turn, depend on the predictability, or volatility, of the revenues and operating cash flow (i.e., business risk). Financial risk thus reflects the cumulative impacts of macro and business risk. Consider an example: Uncertainty about sales to a key customer would be considered a business risk, not a financial risk. But when we measurefinancial risk, we would normally capture all the sources of uncertainty about financial results, including risks to revenues.

Businesses vary widely in their ability and capacity to support debt, depending on their industry, competitive position, and stage of maturity. A business with a low level of business risk can typically support a high level of financial leverage and is generally one with demand that is predictable and stable, a strong and durable competitive position, and high operating margins and that does not require large amounts of investment to maintain its position. For example, an electric utility-a regulated monopoly with very stable revenues and operating cash flows-can typically maintain a high proportion of debt in its capital structure. Exhibit 8: Components of Leverage

\begin{center}
\includegraphics[max width=\textwidth]{2023_05_04_b5cfa4f1bc883752f121g-732}
\end{center}

\section{Measuring Operating and Financial Leverage}
Financial risk is closely related to the variability of profits and cash flows. These, in turn, depend on the predictability, or volatility, of the business revenues but also on the sensitivity of profits to changes in revenue. There are several different approaches we can use to quantify this risk. An approach is to consider profit sensitivity by looking at the change in net income in relation to change in revenues to capture total leverage of the business. Specifically, total leverage can be broken down to several components:

Total leverage $=$ Operating leverage $\times$ Financial leverage.

Here, operating leverage captures the sensitivity of operating profit, proxied by EBIT, to a change in revenues and can also be calculated as follows:

$$
\text { Operating leverage }=\text { Contribution/EBIT } \text {. }
$$

Financial leverage reflects the variability of profits introduced by interest charges-that is, the sensitivity of net profit to a change in operating profit. It can also be calculated as

Financial leverage $=$ EBIT $/$ EBT .

Note that this approach refers to EBT rather than net income, since taxes fluctuate, but would normally be a stable percentage of EBT. For this reason, we can also measure financial leverage as the percentage change in EBIT divided by the percentage change in earnings or EPS.

In accounting terms, "business risk" can be seen as the risk of a revenue shortfall, which is, in turn, magnified by operating leverage. (Note that when demand is weak, revenue shortfalls are also accompanied by pricing and margin weakness, which causes operating leverage to increase.) Financial risk is the magnification of business risk through financial leverage (i.e., the magnification of an EBIT shortfall at the EBT or EPS line).

\section{BUSINESS RISK IN THREE HOTEL BUSINESSES}
This example compares three typical but distinctly different hotel businesses and how their strategy impacts their business risk and consequently their financial risk. Hotels cater to essentially three different types of customers: corporate customers on business trips, leisure customers on holidays, and temporary accommodations. As such, hotels offer a broad range of services, at different prices, and different experiences. Economic factors tend to influence all hotel businesses in largely a similar manner.

\begin{itemize}
  \item MegaChain Hotel owns and operates a multi-hotel franchise and has a very well-established brand. It does not own any physical hotel properties outright but provides marketing, management, and booking services to all its franchisees, who pay $10 \%$ of revenues as a franchise fee. Operating expenses are about $75 \%$ of revenues. MegaChain Hotel has no debt.

  \item Hotel OwnCo, a MegaChain Hotel franchisee, owns five well-located large hotel properties catering to a mix of business and leisure costumers. The company has mortgages outstanding for about $40 \%$ of property value. Its operating expenses are about $50 \%$ of revenues.

  \item BCHZ Hotel is a small, family-run hotel catering to backpackers, surfers, and local families. It sits on a location comparable to other similar properties in the area: near a beach among other similar businesses. BCHZ is known for excellent food and service, and demand for its rooms is not sensitive to overall economic activity. Operating expenses are about $35 \%$ of revenues. The business owns the hotel property outright with no debt.

\end{itemize}

Comparing these three hotels, MegaChain has a high level of operating leverage and no financial leverage. Its franchisee, Hotel OwnCo has lower operating leverage but high financial leverage. $\mathrm{BCHZ}$ Hotel has low operating leverage and no financial leverage. BCHZ Hotel presents the lowest level of financial risk. However, it appears that its success is based on food quality and service, presumably reflecting the effort and skill of the family who runs it. This is a significant source of business-specific risk that is not present with MegaChain Hotel or Hotel OwnCo.

\section{SUMMARY}
\begin{itemize}
  \item A business model describes how a business is organized to deliver value to its customers:

  \item who its customers are,

  \item how the business serves them,

  \item key assets and suppliers, and

  \item the supporting business logic.

  \item The firm's "value proposition" refers to the product or service attributes valued by a firm's target customer that lead those customers to prefer a firm's offering over those of its competitors, given relative pricing.

  \item Channel strategy may be a key element of a business model, and it addresses how the firm is reaching its customers.

  \item Pricing is often a key element of the business model. Pricing approaches are typically value or cost based.

  \item In addition to the value proposition, a business model should address the "value chain" and "how" the firm is structured to deliver that value. - While many firms have conventional business models that are easily understood and described in simple terms, many business models are complex, specialized, or new.

  \item Digital technology has enabled significant business model innovation, often based on network effects.

  \item To understand the profitability of a business, the analyst should examine margins, break-even points and "unit economics."

  \item Businesses have very different financing needs and risk profiles, depending on both external and firm-specific factors, which will determine the ability of the firm to raise capital.

\end{itemize}

\section{PRACTICE PROBLEMS}
\begin{enumerate}
  \item Which of the following is least likely to be a key feature of a business model?
A. Unit economics
B. Channel strategy
C. Financial forecasts
D. Customer cost of ownership
E. Target customer identification

  \item When should an analyst expect a business model to employ premium pricing? When:
A. the company is a price taker.
B. the firm is small and returns are highly scale sensitive.
C. significant differentiation is possible in the product category.
D. the firm is a market leader and demand is very price sensitive.

  \item Which is the most accurate statement about a platform business?
A. A platform business is based on network effects.
B. A platform business can be a non-technology business.
C. Value creation for customers for a platform business occurs externally.
D. It can be difficult to attract users in the beginning to a platform business.
E. All of the above

  \item Which of the following businesses is least likely to have network effects?
A. A stock exchange
B. A telephone company
C. A classified advertising website
D. A price comparison website for travel airfares
E. A resume preparation service for online job seekers

  \item A flower shop has preferred supplier arrangements with an answering service, to take orders after hours, and a bicycle delivery service, to ensure that it can make deliveries quickly, reliably, and at a reasonable cost. Which of the following statements is most accurate for the flower shop?

\end{enumerate}

A. The answering service is part of its supply chain.

B. The bicycle delivery service is part of its value chain.

C. The bicycle delivery service is part of its supply chain. D. The bicycle delivery service is not a part of the value proposition for the flower shop.

\begin{enumerate}
  \setcounter{enumi}{5}
  \item Which of the following is the closest example of a one-sided network?
A. An online employment website
B. A dating website for men and women
C. A social network for model train collectors
D. A website for home improvement contractors

  \item Which of the following statements is not representative of unit costs?

\end{enumerate}

A. Unit costs generally exclude labor costs.

B. Business models generally consider unit costs.

C. Unit costs are used to calculate break-even points.

D. If a lemonade stand uses 5 cents worth of lemons, 2 cents worth of sugar, and a cup costing 3 cents for each glass of lemonade, it has a unit cost of 10 cents.

\begin{enumerate}
  \setcounter{enumi}{7}
  \item Which of the following companies would most likely have a high level of macro risk?
\end{enumerate}

A. A coffee plantation in Brazil

B. A Swedish mining equipment manufacturer

C. A call center outsourcing business based in India

\begin{enumerate}
  \setcounter{enumi}{8}
  \item Which of the following is most likely to have a high level of industry risk?
A. Toll road
B. Pest control services company
C. Oil well drilling service company

  \item For a newly launched clothing company in Japan that uses offshore production in Malaysia, classify each of the following impacts:

  \item Demand falls gradually due to a declining population

  \item Consumer tastes shift to favor locally manufactured apparel

  \item The company faces uncertainty about future demand as it hires a new chief designer and makes changes to its top-selling products

  \item Which of the following is an example of significant execution risk?

\end{enumerate}

A. A manufacturer replaces aging factory machinery with similar but more efficient equipment. A. Company-specific risk

B. Macro risk

C. Industry risk B. A marketer of high-fashion pet accessories tests the market to see if there is demand for glamourous dog harnesses made with faux fur.

C. A company with consistent operating margins of about $5 \%$ with stable market share of 5\% for swimming pool chemicals plans to double its margins and triple its market share over the next five years.

\begin{enumerate}
  \setcounter{enumi}{11}
  \item Which of the following is most likely to increase a business's operating leverage?
A. Reducing prices
B. Borrowing rather than issuing equity
C. Using casual labor rather than a salaried work force

  \item Which of the following is most likely to increase financial leverage?
A. Cutting prices
B. Replacing short-term debt with long-term debt
C. Entering a sale-leaseback transaction for the company's head office building

\end{enumerate}

\section{SOLUTIONS}
\begin{enumerate}
  \item $C$ is correct. Financial forecasts are normally part of a more detailed business plan. A business model should convey how the business makes money, so unit economics (i.e., per-unit revenue and costs) are a key element of a business model. Based on the product and market, the target market (who the business serves) the channel strategy (where they purchase), and the total cost of ownership, including maintenance after purchase, would also be key business model elements.

  \item $\mathrm{C}$ is correct. A business model that assumes premium pricing must address why customers will be willing to pay a premium, normally because of some type of differentiation. It is less likely (although not impossible) that a price premium could be sustained in a category where pricing is set in the market (A), where a small change in price causes a large change in demand (another way to describe the price taker scenario; D), or when a firm is trying to scale up to a competitive size (B).

  \item E is correct. All these statements are true, in most cases, for a platform business. A platform business is defined as a business based on network effects-that is, where the value of its service or product is enhanced by the addition of customers or users. While many think of platform businesses as being web-based or software-based, there are many older business models that qualify, such as brokerage and exchange businesses and transportation and communication networks. The value creation for a platform business is external to the company that created the product or service. When the business is launched, it has no customers, which can make the launch challenging-one reason why many platform startups employ a "freemium" pricing strategy to attract users quickly.

  \item $E$ is correct. The resume preparation service benefits from the network effects on various online job sites, but the service is not the source of those network effects. Each of the other businesses (A, B, C, and D) becomes more valuable to its customers as it attracts users. A stock exchange is valuable and worth joining because many securities trade on it. The telephone network is very useful because most people are on it. A classified advertising website becomes more useful as it attracts more listings. An airfare price comparison website is valuable to airlines because it has many shoppers and valuable to shoppers because it features prices for multiple airlines and routes.

  \item C is correct. A supply chain includes all the steps involved in producing and delivering a physical product to the end customer, regardless of whether those steps are performed by a single firm. A value chain includes only those functions performed by a single firm, but it also includes functions that are valuable to customers but may not involve physical transformation or handling of the product. The bicycle delivery service is a source of value to customers, so it is part of the flower shop's value proposition, but it is performed by a third party, so it is not part of the flower shop value chain but, rather, is part of its supply chain. The answering service is not a step in the physical goods flow, so it is not a part of the supply chain.

  \item $\mathrm{C}$ is correct. A social network for model train collectors involves a single group of users and thus is closest to a one-side network. The others involve two user groups: employers and job-seekers in A, men and women in B, and homeowners and contractors in D.

  \item A is correct. Unit costs normally include direct labor costs. A unit cost analysis should be considered in most business models, although in some cases, they will be close to zero (for example, digital media). If unit costs are non-zero, they must be taken into account when calculating the break-even point. In $\mathrm{D}$, there are no direct labor costs, so the unit cost calculation is reasonable. (The lemonade stand is staffed, and no extra labor is required to pour the lemonade.)

  \item B is correct. Macro risk is likely to be highest with a Swedish mining equipment manufacturer since product demand is very sensitive to the global economy. With the coffee plantation in Brazil, the call center outsourcing business based in India, and the Swedish mining equipment manufacturer, there is also exchange rate risk that could impact profitability and competitiveness.

  \item C is correct. An oil well drilling service company operates in a highly competitive industry, where demand is difficult to forecast and is very sensitive to macro risk. Industry risk is therefore likely to be high. The toll road and the pest control services company have recurring, predictable revenues with significant "moats" or barriers to competition.

  \item Option 1 (Demand falls gradually due to a declining population) matches with $B$ (Macro risk). That demand falls gradually due to a declining population is a macro risk because it impacts all business and economic activities.

\end{enumerate}

Option 2 (Consumer tastes shift to favor locally manufactured apparel) matches with C (Industry risk). That consumer tastes shift to favor locally manufactured apparel is an industry risk that impacts all apparel manufacturing business in similar fashion.

Option 3 (The company faces uncertainty about future demand as it hires a new chief designer and makes changes to its top-selling products) matches with A (Company-specific risk). That the company faces uncertainty about future demand as it hires a new chief designer and makes changes to its top-selling products is a company-specific risk because it is within the company management's direct control and does not impact other businesses.

\begin{enumerate}
  \setcounter{enumi}{10}
  \item C is correct. A company with consistent operating margins and a stable market share in a highly specialized business embarks on a significant and ambitious strategic change. Its success will depend entirely on how well management succeeds in delivering on its objective by improving margins (either by increasing prices or reducing costs) and taking market share from its competitors. Considering the relatively small size of the business, it may be difficult. Considering that many manufacturing businesses in the same industry typically operate around similar margins, any margin improvement may be difficult. That a manufacturer replaces aging factory machinery with similar but more efficient equipment is not an example of execution risk; it is part of regular improvement and capital investment. That a marketer of high-fashion pet accessories tests the market to see if there is demand for glamourous dog harnesses made with faux fur is a standard and common expansion of an existing product line with limited risk.

  \item A is correct. Reducing prices decreases the business's margin, and as such, it increases its sensitivity to changes in demand, revenue, and costs and its operating leverage. The choice between debt and equity financing has no bearing on operating leverage, although it should be noted that interest expenses on debt are contractually determined payments, while dividends are discretionary payments. Using casual labor rather than a salaried work force reduces the fixed employee expenses, which reduces operating leverage.

  \item $\mathrm{C}$ is correct. Entering a sale-leaseback transaction for the company's head office building increases financial leverage. The company sells assets with the obligation to repurchase the assets in the future as well as make lease payments. These transactions increase its financial leverage. Additionally, sale and leaseback transactions reduce the business's overall asset base, which, in turn, reduces its ability to add more debt should the company need to raise debt. Cutting prices reduces the profit margin for the business, thereby increasing operating leverage. Replacing short-term debt with long-term debt does not change financial leverage: Debt, irrespective of maturity, is simply debt.

\end{enumerate}

\section{LEARNING MODULE
4}
\section{Capital Investments}
by John D. Stowe, PhD, CFA, and Jacques R. Gagné, FSA, CFA, CIPM.

John D. Stowe, PhD, CFA, is at Ohio University (USA). Jacques R. Gagné, FSA, CFA, CIPM, is at ENAP (Canada).

\section{LEARNING OUTCOME}
\begin{center}
\begin{tabular}{c|l}
Mastery & The candidate should be able to: \\
$\square \square$ & $\begin{array}{l}\text { describe types of capital investments made by companies } \\ \text { describe the capital allocation process and basic principles of capital } \\ \text { allocation } \\ \text { demonstrate the use of net present value (NPV) and internal rate of } \\ \text { return (IRR) in allocating capital and describe the advantages and } \\ \text { disadvantages of each method } \\ \text { describe common capital allocation pitfalls } \\ \text { describe expected relations among a company's investments, } \\ \text { company value, and share price } \\ \text { describe types of real options relevant to capital investment }\end{array}$ \\
$\square$ &  \\
\end{tabular}
\end{center}

\section{INTRODUCTION}
Capital investments, also referred to here as capital projects, are investments with a life of one year or longer made by corporate issuers. Issuers make capital investments to generate value for their stakeholders by returning long-term benefits and future cash flows greater than the associated funding cost of the capital invested. How companies allocate capital between competing priorities and the resulting capital investment portfolio are central to a company's success and together constitute a fundamental area for analysts to understand. Given that corporate disclosure of capital investments is typically very high level and lacking in specifics, the evaluation of a company's capital investments is often challenging for analysts.

Capital investments describe a company's future prospects better than its working capital or capital structure, which are often similar for companies, and provide insight into the quality of management's decisions and how the company is creating value for stakeholders. While the focus of this coverage is on capital investments, it is important to note that companies also make other investments in increased working capital, information technology (IT), and/or human resources projects that might not be capitalized and therefore affect near-term operating profit, but that are made for similar longer-term benefit as capital investments.

\section{TYPES OF CAPITAL INVESTMENTS}
describe types of capital investments made by companies

The types of capital investments made by companies vary considerably and often span the full spectrum of risk and return. Some are less risky and fairly easy to evaluate, such as the replacement of depreciated equipment, while others, such as the development of a new product or an acquisition of another company, are riskier and far more complex.

Capital investments are undertaken for two primary purposes-to maintain the existing business and to grow it-and can generally be classified into four types of projects. The first two types,

\begin{enumerate}
  \item going concern (or maintenance) projects and

  \item regulatory/compliance projects,

  \item ensure business-as-usual continuity while the latter two investment types,

  \item expansion projects and

  \item other projects,

\end{enumerate}

are made to expand the business in some strategic manner. Each of these is highlighted in Exhibit 1 with a brief explanation and examples.

\section{Exhibit 1: Types of Capital Projects}
Business Maintenance

\section{Going concern}
Projects necessary to continue current operations and maintain existing size of the business or to improve business efficiencies Example:

machine replacement, infrastructure improvement

\section{Regulatory/Compliance}
Projects typically required by a third party, such as the government regulatory body, to meet specified safety and compliance standards

Example:

factory pollution control installation, performance bond posting to guarantee satisfactory project completion

\section{Business Growth}
\section{Expansion}
Projects that expand business size and typically involve greater degrees of risk and uncertainty than going concern projects

Example:

new product or service development, merger, acquisition

\section{Other}
Projects, which should include high-risk investments and new growth initiatives, that are outside the company's conventional business lines

\section{Example:}
exploration investment into a new innovation, business model, or idea

\section{Business Maintenance}
\section{Going Concern Projects}
Going concern projects are those investments needed to continue the company's current operations and maintain the existing size of the business. The most common going concern projects are replacements of assets that reach the end of their useful life and spending to maintain IT hardware and software. For example, a company might elect to replace older infrastructure in its production facilities (light fixtures, heating and cooling units, etc.) with more modern and efficient alternatives. Going concern projects do not typically yield incremental revenue but might benefit the company through improved efficiencies and cost savings over time.

Going concern projects are fairly easy for management to evaluate because their costs are typically small relative to the production or business interruption costs that could result from not making the investment. In addition, the analyses of going concern projects often benefit from having readily available data from existing business operations to use in the decision-making process. The time length for these projects can vary, with some projects expected to span several years.

To fund these projects, managers will often try to match the financing with the life-span of the asset. For example, a company might issue a 20-year bond to finance replacement equipment with an expected useful life of 20 \href{http://years.In}{years.In} matching the cash flows and maturity structures of their assets and liabilities, companies reduce risk. A company financing long-term assets with short-term obligations faces rollover risk, which could threaten profitability if short-term financing costs go up over the financing period.

Similarly, a company financing short-term assets with long-term financing beyond the term needed faces the risk that the company overpays in financing costs. Asset liability misalignment increases the risk of default and cost of capital for companies as capital suppliers demand higher returns in compensation.

Corporate issuers typically do not disclose the amount of capital spending associated with going concern projects, or any other type of capital investment, on financial statements or elsewhere. A common, but imperfect, estimate for going concern capital spending used by analysts is the amount of depreciation and amortization expense reported on the income statement. The accuracy of this estimate depends on how closely the accounting useful life of assets approximates the actual useful life and whether the historical cost of an asset approximates its replacement cost; both assumptions are likely to be more accurate for shorter-lived assets.

\section{Regulatory/Compliance Projects}
Unlike going concern and expansion projects, for which management has discretion in deciding whether or not to invest, regulatory and compliance projects are required by third parties, such as government regulatory bodies, to meet safety and regulatory compliance standards. These projects might be driven by public or private mandates, such as a government agency's newly enacted requirement to install pollution control technologies, the need to secure a surety performance bond to guarantee satisfactory completion of a project, or the requirement of financial institutions to meet capital adequacy requirements and perform stress tests to assess their ability to withstand economic shocks.

Regulatory/compliance projects might not generate revenue and might not otherwise be undertaken by a company but are required to continue operations. However, there can be a potential benefit to industry incumbents, as regulatory/compliance costs can serve as barriers to industry entry, thereby increasing or protecting incumbents' profitability. Additionally, in some instances when a company is able to work directly with regulators to develop the prescribed standards, those standards can be tailored to best suit the company's compliance capabilities, thereby reducing the compliance burden.

When a company is faced with having to invest in a new regulatory/compliance project-perhaps due to a new regulatory requirement-management will have to decide whether the economics of the underlying business are still favorable after consideration of the additional costs associated with the regulatory/compliance project. In many cases, the company will accept the required investment and continue to operate, passing the added cost on to consumers, in full or in part, in the form of higher prices. Occasionally, however, the cost of such projects is sufficiently high that the company would be better off ceasing to operate altogether or shutting down any part of the business that is related to the new regulatory/compliance project. "Stranded assets," or assets at risk of no longer being economically feasible because of changes in regulation (or investor sentiment), such as carbon-intensive assets, are an example of this; their "costs" of production, inclusive of greenhouse emissions, are seen as being higher than their realizable value.

\section{Business Growth}
\section{Expansion Projects}
An important value driver for companies is growth in profits, which can be achieved by companies expanding the scale of existing activities or extending their reach into new product or service categories and markets. Expansion projects are those that increase business size, usually by investing in the development of new products or services and/or acquiring other companies.

Expansion projects typically involve greater uncertainty, time, and amounts of capital than going concern projects. Some industries, such as pharmaceuticals and oil and gas, spend heavily on expansion projects. Companies in these two industries often invest over $10 \%$ of annual revenues per year in pursuit of new medications and energy reserves, respectively. Similarly, technology companies typically invest heavily in expansion projects to accelerate product development cycles, maintain competitiveness, and stay ahead of rivals.

When internal opportunities for expansion appear to be limited, or when a rapid integration of new capabilities is preferred to building from within the company, a common strategic response by management is to look to acquisitions. Acquisitions can take a variety of forms, with special tax, financial accounting, and market regulation considerations. Two serious risks, however, are the difficulty in integrating the business operations of the acquirer and the target and the risk of overpaying. Often, stakeholders are better off if management returns capital through dividends and share repurchases rather than expanding through acquisitions.

Analysts should carefully examine the issuer's level and trend of expansion capital investment overall, as well as by segment or line of business if it is disclosed, to analyze growth prospects, management's priorities, and the rates of return on the investment relative to alternatives. The level and trend of expansion capital spending can be estimated by subtracting non-expansion capital investment from total capital spending, often estimated using depreciation and amortization expense.

\section{Other Projects}
Sometimes, a company's management will decide to invest in a unique activity outside, or only minimally related to, the company's strategy. This happens more often with firms that are privately held or under the control of a founding owner or significant shareholder. Whether these projects are seen as special situations offering atypical growth or investment opportunities or innovation opportunities for the company's business or business model, these projects are likely to be at the riskier end of capital investments. In some cases, these might be projects driven by a founding owner or controlling shareholder who feels strongly about the investment. Such projects might be approved without going through the customary analysis normally undertaken for the other types of capital investments.

These other projects often have a venture capital element to them, such as investing some capital to explore a new technology or a business idea/model for sources of new business growth. The probable outcome might be a complete loss of investment, but the attraction is that the project could be highly profitable if successful.

\section{EXAMPLE 1}
\section{Types of Capital Investment}
Paladote Company is a mid-sized, financially sound, and profitable supplier of food service packaging, based in Canada. Paladote produces its products in a variety of new and established facilities throughout the world and distributes these to a globally diverse customer base. At Paladote's annual Investor and Analyst Day, management describes four capital investment priorities for the next one to three years:

\begin{enumerate}
  \item Investment to develop new customized packaging products for new customers in India.

  \item Funding of an extensive study on edible food packaging. During the presentation, the investment is positioned as an initiative that will reshape the industry. Similar efforts by several other companies have been unsuccessful to date.

  \item Investment to modify production processes in its Latin American production plant in response to a recent regional ban on the use of certain food contact substances that becomes effective next year.

  \item Investments in Paladote's older production facilities to lower production costs.

  \item Which project would most likely be classified as an "other" project?

\end{enumerate}

\section{Solution:}
Project 2. The funding of an extensive study on edible food packaging appears to be a project that is likely to be classified as an "other" project. While the project is unlikely to be successful, the company's management believes that making the investment could potentially reshape the food packaging industry.

\begin{enumerate}
  \setcounter{enumi}{1}
  \item Which project would most likely be classified as a going concern project?
\end{enumerate}

\section{Solution:}
Project 4. Investments in Paladote's older production facilities to lower production costs would be classified as a going concern project, which are projects generally focused on asset replacement, either to replace assets at the end of their useful lives or to replace older, inefficient assets with newer and more efficient assets. 3. Which project would most likely be classified as an expansion project?

\section{Solution:}
Project 1. An investment to develop new customized packaging products for new customers in India would be classified as an expansion project, which are those projects that increase business size, usually by investing in the development of new products or services and/or by the acquisition of other companies.

\begin{enumerate}
  \setcounter{enumi}{3}
  \item Which project would most likely be classified as a regulatory/compliance project?
\end{enumerate}

\section{Solution:}
Project 3. An investment to modify production processes in Paladote's Latin American production plant in response to a recent regional ban on the use of certain food contact substances would be classified as a regulatory/compliance project. The project would be required to comply with the new ban that goes into effect next year.

\section{THE CAPITAL ALLOCATION PROCESS}
$$
\begin{aligned}
& \text { describe the capital allocation process and basic principles of capital } \\
& \text { allocation }
\end{aligned}
$$

Capital allocation is the process used by an issuer's management to make capital investment decisions. Given that corporations exist to deliver competitive risk-adjusted returns for stakeholders, capital allocation is the most important job of management and is typically done by those at the executive and board level. In general, the process undertaken by a corporate issuer is substantially the same as the process undertaken by portfolio managers and analysts of an investment manager; both entities are allocating stakeholder capital in pursuit of competitive risk-adjusted returns. However, most corporate managers do not have a background in capital allocation; it is an uncommon skill that analysts must actively identify in corporate issuers' management.

As Warren Buffett, CEO of Berkshire Hathaway, said: "[CEOs] get to the top of a corporation in various ways. [They] may come through sales or engineering. And all of a sudden, [they're] in a different job.... because now [they're] involved in allocating capital. That's something many CEOs haven't done, and yet they're expected to do it and some of them think they can do it that aren't really very good at it."

The steps that management should take in the capital allocation process are shown in Exhibit 2.

\section{Exhibit 2: Steps in the Capital Allocation Process}
\begin{center}
\includegraphics[max width=\textwidth]{2023_05_04_b5cfa4f1bc883752f121g-747(2)}
\end{center}

Step 2: Investment Analysis The collection of information needed to forecast the investment's expected cash flows and to evaluate the investment's profitability.

\begin{center}
\includegraphics[max width=\textwidth]{2023_05_04_b5cfa4f1bc883752f121g-747}
\end{center}

Step 1: Idea Generation

The generation of investment ideas can originate from anywhere within the organization or from outside the company. Having good investment ideas for management consideration is the most important step in the capital allocation process Step 3: Capital Allocation Planning The organization of the profitable proposals that together best fit the company's strategy. Financial and real resource constraints mean the scheduling and prioritizing of capital investments are key considerations

\begin{center}
\includegraphics[max width=\textwidth]{2023_05_04_b5cfa4f1bc883752f121g-747(1)}
\end{center}

Step 4: Monitoring and Post-Audit

The comparison of actual results to

Following the generation of investment ideas, management should forecast the amount, timing, duration, and volatility of an investment's expected cash flows in addition to the probability of the cash flows' occurrence. Capital allocation planning then involves the selection and prioritization of profitable investment opportunities that, when considered together, are most value enhancing on a risk-adjusted return basis. Opportunities that fail to generate returns sufficient to cover their associated cost of funding should not be pursued. Additionally, some projects that look attractive in isolation might be undesirable strategically. Rather than investing in unprofitable or strategically unwise projects, management is better off returning capital to shareholders as dividends or share repurchases.

Post-auditing capital projects and monitoring an investment's realized revenues, expenses, and cash flows against expectations are important for several reasons. First, these steps help review assumptions that underlie the capital allocation process. Systematic errors, such as overly optimistic forecasts, become apparent. Second, they might help improve business operations. If sales or costs are out of line, the monitoring and post-auditing processes will focus management's attention on bringing performance closer to expectations, if at all possible. Finally, monitoring and post-auditing recent capital investments could produce concrete ideas for future investments. Managers should invest more heavily in profitable areas and scale down or dispose of assets in areas that are disappointing or are worth more in the hands of others. In practice, the post-auditing and monitoring of projects is difficult given the challenges in the accurate measurement of expected and realized benefits, costs, and revenues, data availability (which in some cases could take years), organizational politics, and so on. Like other business activities, the capital allocation activity is a cost-benefit exercise for the company. At the margin, the benefits from the improved decision making should exceed the costs of the capital allocation efforts. The primary principles underlying capital allocation are shown in Exhibit 3.

\section{Exhibit 3: Capital Allocation Principles}
\begin{center}
\includegraphics[max width=\textwidth]{2023_05_04_b5cfa4f1bc883752f121g-748}
\end{center}

In ignoring financing costs, we referred to the rate used in discounting the cash flows as the "required rate of return." The required rate of return is the discount rate that the issuer's suppliers of capital require given the riskiness of the project. This discount rate is frequently called the "opportunity cost of funds" or the "cost of capital." A company's weighted average cost of capital (WACC) is its cost of capital at the enterprise level, based on its average-risk investment and capital sources used to finance its assets.

If the company can invest elsewhere and earn a return of $r$, or if the company can repay its sources of capital and save a cost of $r$, then $r$ is the company's opportunity cost of funds. If the company cannot earn more than its opportunity cost of funds on an investment, it should not undertake that investment. Regardless of what it is called, an economically sound discount rate is essential for making capital allocation decisions.

\section{EXAMPLE 2}
\section{Required Rate of Return for Capital Investments}
At its Capital Markets Day, the management of Ørsted, a listed renewable energy company headquartered in Denmark, shared the following information concerning the company's required rate of return on capital investments.

Ørsted uses a WACC plus a 150-300 basis point spread as the required rate of return for each project. The WACC reflects the required rate of return of Ørsted's capital providers, while the 150-300 basis point spread, which varies by project, reflects risks associated with a specific project, such as technology (solar versus wind, hydrogen, etc.), its counterparties, and the country in which the project is located.

The achievement of rates of return at or above the WACC plus the $150-300$ basis point spread on investments depends on the achievement of after-tax project cash flows at or above the estimate used at the time of the project bid.

Although the principles of capital allocation are simple, they are easily confused in practice, leading to poor decisions made by companies. The following are important capital allocation principles.

\begin{itemize}
  \item A sunk cost is one that has already been incurred. One cannot change a sunk cost. Management's decisions should be based on current and future cash flows and should not be affected by prior, or sunk, costs.

  \item An incremental cash flow is the cash flow that is realized because of an investment decision: the cash flow with a decision minus the cash flow without that decision. Only incremental cash flows are relevant for capital allocation.

  \item An externality is the effect of an investment on things other than the investment itself. Frequently, an investment affects the cash flows of other parts of the company, and these externalities can be positive or negative. $A$ positive externality occurring within the company would be expected synergies with existing projects or business activities that result from making the investment. The production and sale of a new complementary product or service might increase demand or lower costs for an existing product or service the company offers. Cannibalization is an example of a negative externality occurring within the company. Cannibalization occurs when an investment takes customers and sales away from another part of the company. If possible, companies should consider externalities in the investment decision. Often the adjustments are subjective, such as the anticipated competitor response to a new product feature or price change, because the company cannot derive a precise estimate. Sometimes externalities occur outside the company that require evaluation of economic, social, or environmental considerations. An investment might benefit (or harm) other companies or society at large, yet the company is not compensated for these benefits (or charged for the costs).

  \item Conventional cash flow pattern versus nonconventional cash flow pattern: A conventional cash flow pattern is one with an initial outflow followed by a series of inflows. In a nonconventional cash flow pattern, the initial outflow is not followed by inflows only, but the cash flows can flip from being positive (inflows) to negative (outflows) again or possibly change signs several times. An investment that involved outlays (negative cash flows) for the first couple of years that were followed by positive cash flows would be considered to have a conventional pattern. If cash flows change signs once, the pattern is conventional. If cash flows change signs two or more times, the pattern is nonconventional. Exhibit 4 shows a conventional cash flow pattern followed by an unconventional cash flow pattern.

\end{itemize}

\section{Exhibit 4: Cash Flow Patterns}
\begin{center}
\includegraphics[max width=\textwidth]{2023_05_04_b5cfa4f1bc883752f121g-750}
\end{center}

B. Unconventional cash flow pattern

\begin{center}
\includegraphics[max width=\textwidth]{2023_05_04_b5cfa4f1bc883752f121g-750(1)}
\end{center}

For example, a crude oil refining facility might require an initial investment outlay that produces positive cash flows for several years followed by additional outlays to refurbish the facilities needed to produce additional positive cash flows. Or an investment in a nuclear power facility that generates electricity for a number of years might be followed by another cash outlay to decommission the power plant.

In assessing potential investment opportunities, several types of project interactions make the incremental cash flow analysis challenging for companies. The following are some of these interactions:

\begin{itemize}
  \item Independent projects versus mutually exclusive projects: Independent projects are capital investments whose cash flows are independent of each other. Mutually exclusive projects compete directly with each other. For example, if Projects A and B are mutually exclusive, the company can choose to invest in A or B but cannot choose both. Sometimes management might be presented with a group of several mutually exclusive projects and can choose only one project from the group. While some investments require relatively little management effort ("order the new plane"), others are extremely management intensive, require other resources that are very scarce (e.g., skilled IT personnel), or create a significant level of organization disruption that limits the company's ability to take on additional projects. Given that all management teams have limits on their execution capability, to help resolve these challenges, management will create and compare "menus" of projects ("we can do A, $\mathrm{C}$, and D or C, E, and F").

  \item Project sequencing: Many capital projects are sequenced over time, so that investing in a project creates the option to invest in future projects. For example, the company might invest in a project today and one year later invest in a second project if the financial results of the first project or new economic conditions are favorable. If the results of the first project or new economic conditions are not favorable, the company would not invest in the second project. Another example might be a strategically important investment, such as investment in a new capability, system, or platform, that enables follow-on investments. While these projects can be difficult to evaluate because their benefits could be hard to predict, they can be extremely valuable for companies.

  \item Management principles such as "fail fast," “minimum viable product," and prototyping are concepts associated with project sequencing. These are intended to take a potentially large project and break it up into smaller successive "projects" that enable management to make go/no-go decisions in a faster, more agile timeframe based on intelligence gleaned in early stages.

\end{itemize}

\section{INVESTMENT DECISION CRITERIA}
demonstrate the use of net present value (NPV) and internal rate of return (IRR) in allocating capital and describe the advantages and disadvantages of each method

Management uses several criteria to make capital investment decisions. The two most comprehensive measures to assess whether an investment is profitable or unprofitable are the net present value and internal rate of return. An analyst should understand the economic logic behind each of these investment decision criteria, as well as their strengths and limitations in practice.

\section{Net Present Value}
For a capital investment with one investment outlay, made initially, the net present value (NPV) is the present value of the future after-tax cash flows minus the investment outlay, or

$$
\mathrm{NPV}=\sum_{t=1}^{n} \frac{\mathrm{CF}_{t}}{(1+r)^{t}}-\text { Outlay }
$$

where

$\mathrm{CF}_{t}=$ After-tax cash flow at time $t$

$r=$ Required rate of return for the investment

Outlay $=$ Investment cash flow at time zero

To illustrate the NPV criterion, we will consider a simple example.

\section{EXAMPLE 3}
\section{Gerhardt Corporation NPV}
\begin{enumerate}
  \item Assume that Gerhardt Corporation is considering a capital investment of EUR50 million that will return after-tax cash flows of EUR16 million per year for the next four years, plus another EUR20 million in Year 5.
\end{enumerate}

\begin{center}
\includegraphics[max width=\textwidth]{2023_05_04_b5cfa4f1bc883752f121g-752}
\end{center}

If the company's required rate of return is $10 \%$, what is the associated NPV of this investment?

\section{Solution}
The NPV would be

$$
\begin{aligned}
\mathrm{NPV} & =\frac{16}{1.10^{1}}+\frac{16}{1.10^{2}}+\frac{16}{1.10^{3}}+\frac{16}{1.10^{4}}+\frac{20}{1.10^{5}}-50 . \\
\text { NPV } & =14.545+13.223+12.021+10.928+12.418-50 . \\
\text { NPV } & =63.135-50=€ 13.135 \text { million. }
\end{aligned}
$$

The investment has a total value, or present value of future cash flows, of EUR63.135 million. Because this investment can be acquired at a cost of EUR50 million, the company is giving up EUR50 million of its wealth in exchange for an investment worth EUR63.135 million. The investment increases the wealth of the company by a net amount of EUR13.135 million.

Because the NPV is the amount by which the company's wealth increases as a result of the investment, the decision rule for the NPV is as follows:

Invest if

$\mathrm{NPV}>0$.

Do not invest if

$\mathrm{NPV}<0$.

Positive-NPV investments are wealth increasing for the company and its shareholders, whereas negative-NPV investments are wealth decreasing for the company and its shareholders. In the rare case that NPV turns out to be zero, the project could be accepted because it meets the required rate of return. Keep in mind, however, that NPV analysis relies on estimated future cash flows. A zero-NPV project leaves no room for error.

Many investments have unconventional cash flow patterns in which outflows might occur not only at time zero but also at future dates. NPV is still calculated as the present value of all future cash inflows and outflows:

$$
\begin{aligned}
\mathrm{NPV} & =\mathrm{CF}_{0}+\frac{\mathrm{CF}_{1}}{(1+r)^{1}}+\frac{\mathrm{CF}_{2}}{(1+r)^{2}}+\cdots+\frac{\mathrm{CF}_{n}}{(1+r)^{n}}, \text { or } \\
\mathrm{NPV} & =\sum_{t=0}^{n} \frac{\mathrm{CF}_{t}}{(1+r)^{t}} .
\end{aligned}
$$

In Equation 2, the investment outlay, $\mathrm{CF}_{0}$, is simply a negative cash flow. Future cash flows can also be negative.

Microsoft Excel functions can be used to quickly solve for the NPV, regardless of the cash flow pattern. Two functions are available: NPV and XNPV.

NPV takes the form of $=N P V(r a t e$, values), where "rate" is the discount rate and "values" are the cash flows. By default, the NPV function assigns the first cash flow $t=1$, so a $t=0$ cash flow must be entered outside the NPV function and assumes equal periods between cash flows.

The XNPV function is more flexible and takes the form of $=X N P V$ (rate, values, dates), where "rate" is the discount rate, "values" are the cash flows, and "dates" are the dates of each of the cash flows, which must be in the form of dates (not periods like $0,1,2$, etc.).

Both NPV and XNPV functions assume a constant discount rate; for varying discount rates, the analyst must enter the discounting algebra manually. The spreadsheet shows the operation of the NPV and XNPV functions for the Gerhardt example.

NPV function

\begin{center}
\begin{tabular}{llcccccc}
\hline
 & $\mathbf{0}$ & $\mathbf{1}$ & $\mathbf{2}$ & $\mathbf{3}$ & $\mathbf{4}$ & $\mathbf{5}$ &  \\
\hline
 & $\begin{array}{l}\text { EUR Cash } \\ \text { flow }\end{array}$ & $€(50)$ & $€ 16$ & $€ 16$ & $€ 16$ & $€ 16$ & $€ 20$ \\
NPV & $€ 13.14$ &  &  &  &  &  &  \\
XNPV function &  &  &  &  &  &  &  \\
 & $\begin{array}{l}\text { assumed date } \\ \text { of CF }\end{array}$ & $01 / 01 / 00$ & $31 / 12 / 00$ & $31 / 12 / 01$ & $31 / 12 / 02$ & $30 / 12 / 03$ & $30 / 12 / 04$ \\
EUR cash flow & $€(50)$ & $€ 16$ & $€ 16$ & $€ 16$ & $€ 16$ & $€ 20$ &  \\
XNPV & $€ 13.14$ &  &  &  &  &  &  \\
\hline
\end{tabular}
\end{center}

\section{Internal Rate of Return}
The internal rate of return (IRR) is one of the most frequently used concepts in capital allocation and security analysis. For a capital investment with a conventional cash flow pattern, the IRR is the discount rate that makes the present value of the future after-tax cash flows equal to the investment outlay. Expressed in equation form, the IRR solves the following equation:

$$
\sum_{t=1}^{n} \frac{\mathrm{CF}_{t}}{(1+\mathrm{IRR})^{t}}=\text { Outlay }
$$

where IRR is the internal rate of return. The left-hand side of this equation is the present value of the capital investment's future cash flows, which, discounted at the IRR, equals the investment outlay. This equation will also be seen rearranged as

$$
\sum_{t=1}^{n} \frac{\mathrm{CF}_{t}}{(1+\mathrm{IRR})^{t}}-\text { Outlay }=0 .
$$

In this form, Equation 3 looks like the NPV equation, Equation 1, except that the discount rate is the IRR instead of $r$ (the required rate of return). Discounted at the IRR, the NPV is equal to zero.

It is also common to define the IRR as the discount rate that makes the present values of all cash flows sum to zero:

$$
\begin{aligned}
& \sum_{t=0}^{n} \frac{\mathrm{CF}_{t}}{\left(1+\mathrm{IRR}^{t}\right.}=0 \\
& 0=\text { Outlay }_{0}+\sum_{t=1}^{n} \frac{\mathrm{CF}_{t}}{\left(1+\mathrm{IRR}^{t}\right.}=\mathrm{CF}_{0}+\sum_{t=1}^{n} \frac{\mathrm{CF}_{t}}{(1+\mathrm{IRR})^{t}}=\sum_{t=0}^{n} \frac{\mathrm{CF}_{t}}{\left(1+\mathrm{IRR}^{t}\right.} \text { (new 3) }
\end{aligned}
$$

Equation 4 is a more general version of Equation 3 .

In the Gerhardt Corporation example, we want to find a discount rate that makes the total present value of all cash flows, the NPV, equal zero.

In equation form, the IRR is the discount rate that solves the following equation:

$$
-50+\frac{16}{(1+I R R)^{1}}+\frac{16}{(1+I R R)^{2}}+\frac{16}{(1+I R R)^{3}}+\frac{16}{(1+I R R)^{4}}+\frac{20}{(1+I R R)^{5}}=0 .
$$

Algebraically, this equation would be difficult to solve. Without the use of a financial calculator or functions, we would have to resort to trial and error, systematically choosing various discount rates until we find one, the IRR that satisfies the equation.

Fortunately, software such as Microsoft Excel can derive the IRR quickly using the IRR function, which takes the form of =IRR(values, guess), where "values" are the cash flows and "guess" is an optional user-specified guess that defaults to $10 \%$. Importantly, the Microsoft Excel IRR function assumes that cash flows are received, or paid, at the end of each period and that each period is evenly spaced.

The XIRR function, which takes the form of $=\operatorname{XIRR}$ (values, dates, guess) grants flexibility on the periods, with the "dates" arguments being a specification of the date of each cash flow. The following spreadsheet shows the IRR of the Gerhardt examples using both the IRR and XIRR functions, as well as the IRR using the XIRR function if the date of each cash inflow is moved forward by six months.

\section{IRR function}
\begin{center}
\begin{tabular}{llllllll}
\hline
$\mathbf{t}$ & $\mathbf{0}$ & $\mathbf{1}$ & $\mathbf{2}$ & $\mathbf{3}$ & $\mathbf{4}$ & $\mathbf{5}$ \\
\hline
EUR cash flow & $€(50)$ & $€ 16$ & $€ 16$ & $€ 16$ & $€ 16$ & $€ 20$ \\
IRR & $19.52 \%$ &  &  &  &  &  \\
\end{tabular}
\end{center}

\section{XIRR function}
\begin{center}
\begin{tabular}{llccccccc}
 & $\begin{array}{l}\text { Assumed date } \\ \text { of cash flow }\end{array}$ & $01 / 01 / 00$ & $12 / 31 / 00$ & $12 / 31 / 01$ & $12 / 31 / 02$ & $12 / 31 / 03$ & $12 / 31 / 04$ \\
 & EUR cash flow & $€(50)$ & $€ 16$ & $€ 16$ & $€ 16$ & $€ 16$ & $€ 20$ \\
XIRR & $19.52 \%$ &  &  &  &  &  &  \\
 &  &  &  &  &  &  &  \\
XIRR & $24.55 \%$ & $€ 1 / 01 / 00$ & $30 / 06 / 00$ & $30 / 06 / 01$ & $30 / 06 / 02$ & $30 / 06 / 03$ & $30 / 06 / 04$ \\
\hline
\end{tabular}
\end{center}

The decision rule for the IRR is to invest if the IRR exceeds the required rate of return for a capital investment: Invest if

IRR $>r$.

Do not invest if

IRR $<r$.

The required rate of return is often called the hurdle rate, the rate that a project's IRR must exceed for the project to be accepted by the company. In the unlikely event that the IRR is equal to $r$, the project is theoretically acceptable because it meets the required return. In fact, NPV equals zero when IRR equals $r$. In the Gerhardt example, because the IRR of $19.52 \%$ exceeds the project's required rate of return of $10 \%$, Gerhardt should invest.

NPV and IRR criteria will usually indicate the same investment decision for a given capital investment. In the case of mutually exclusive investment projects, a company could, on occasion, face the following situation: Project A might have a larger NPV than Project B, but Project B has a higher IRR than Project A. Because the company can invest in only one project, which should it be-Project A or Project B?

The correct choice is Project A, the one with the higher NPV. To understand why, consider this simple example. Suppose you could choose just one of two investments. The first allows you to double your initial outlay of USD100 in just one year. The second requires an investment of USD100,000, which will grow by $20 \%$ in a year. Which should you choose? The first investment gives you a profit of USD100. The second gives you a USD20,000 profit. Assuming you can access the required funds, the second choice is preferred. Even though it offers a smaller percentage return, it increases your wealth by much more.

When the choice is between two mutually exclusive projects and the NPV and IRR rank the two projects differently, the NPV criterion is strongly preferred. There are good reasons for this preference. The NPV shows the amount of gain, or wealth increase in company value, as a currency amount. The NPV assumes reinvestment of cash flows at the required rate of return, while the IRR assumes reinvestment at the IRR. In using the opportunity cost of funds, the reinvestment assumption of the NPV is the more economically realistic measure. Mathematically, whenever you discount a cash flow at a particular discount rate, you are implicitly assuming that you can reinvest a cash flow at that same discount rate. It is more realistic to assume reinvestment at a lower rate. The IRR does give a rate of return, but the IRR could be for a small investment size or for only a short period of time. As a practical matter, once a corporation has the data to calculate the NPV, it is fairly trivial to then calculate the IRR and other capital allocation criteria. However, the most appropriate and theoretically sound criterion is the NPV.

Another issue is that when the cash flows are nonconventional (i.e., they change sign more than once), there are multiple IRRs. We can illustrate this problem with the following nonconventional cash flow pattern:

$\begin{array}{lccc}\text { Time } & 0 & 1 & 2 \\ \text { Cash flow } & -1,000 & 5,000 & -6,000\end{array}$

The IRR for these cash flows satisfies this equation:

$$
-1,000+\frac{5,000}{(1+I R R)^{1}}+\frac{-6,000}{(1+I R R)^{2}}=0 .
$$

It turns out that two values of IRR satisfy the equation: IRR $=1=100 \%$ and IRR = $2=200 \%$. As a result, the IRR is not useful for nonconventional cash flow projects.

This brings up the following question: If NPV is always preferred over IRR for selecting projects, why do companies even bother with IRR? The answer is that many people find a rate of return easy to understand. If they know that the required return is $10 \%$, they can easily understand that a project returning more than $10 \%$ is desirable. If they are simply told the NPV amount, they might not find it as meaningful. In practice, most companies use both metrics. They typically use NPV to make the investment decisions, but they also report the IRR to help their audience understand.

\section{COMMON CAPITAL ALLOCATION PITFALLS}
describe common capital allocation pitfalls

Although the principles of capital allocation might be easy to understand, applying the principles to real-world investment opportunities can be challenging for companies. Some of the common capital allocation pitfalls, or mistakes, that companies make are listed here.

\begin{itemize}
  \item Inertia. In a study of more than 1,600 US listed companies, McKinsey research found a 0.92 correlation between capital investment in a business segment or unit from one year to the next. This is the result of management anchoring their capital investment budgets to prior year amounts.
\end{itemize}

Analysts can identify the presence of inertia by examining the level of capital investment in total, by segment, or by business line, if disclosed, and comparing it to the prior year level and the return on investment. If capital investment each year is static or increasing despite falling returns on investment, the analyst should question the issuer's justification for its capital investment and whether management should be considering alternative uses.

\begin{itemize}
  \item Source of capital bias. The primary source of capital for investments by corporate issuers is cash flows from operations. Many management teams behave as though this internally generated capital is "free" but scarce and allocate it according to a budget that is heavily anchored to prior period amounts. Externally raised capital, from debt or equity issuance, however, is treated differently: it is used less often, typically for larger investments such as acquisitions, and is treated as "expensive."
\end{itemize}

Instead, management teams should view all capital as having an opportunity cost, regardless of source, and use the capital allocation process for all capital investments.

This pitfall might be identified by analysts in examining the issuer's capital structure and its capital return history in addition to inquiring about the capital allocation process. If the issuer has significantly lower leverage than peers, a high cash balance, and stated return hurdles for acquisitions but not for internal investments, then management might potentially be affected by this bias.

\begin{itemize}
  \item Failing to consider investment alternatives or alternative states. While generating good investment ideas is the most basic step in the capital allocation process, many good alternatives are never even considered at some companies. Many companies also fail to consider differing states of the world, which can and should be incorporated through breakeven, scenario, and simulation analyses. This pitfall could be in effect if an issuer has had limited capital investment activity, such as not making divestitures or acquisitions, or not having a failed investment; while failure is obviously undesirable, the lack of having any failed investments over time could indicate a management team that is not taking enough risk.

  \item Pushing "pet" projects. Often, pet projects are selected at companies without undergoing normal capital allocation analysis. Or the pet project receives the analysis, but overly optimistic projections are used to inflate the investment's profitability. As Warren Buffett noted in the 1989 Berkshire Hathaway Shareholder Letter, "Any business craving of the leader, however foolish, will be quickly supported by detailed rate-of-return and strategic studies."

\end{itemize}

Observing pet projects, or management's penchant for them, is difficult to do quantitatively because financial statements are at a high level of aggregation and the projects might not meet the threshold of materiality. Instead, analysts should look to the corporate governance structure for warning signs that increase the chances of misallocation of capital: controlled companies or significant ownership concentration by a single individual or group, weak oversight by the board of directors, and executive compensation that is not aligned with stakeholders' interests.

\begin{itemize}
  \item Basing investment decisions on EPS, net income, or ROE. Companies sometimes have incentives to increase earnings per share, net income, or return on equity and to do so in the short run. Many capital investments, even those with high NPVs, do not increase these accounting numbers in the short run and often reduce them, as opposed to reducing costs or repurchasing shares, which can be done quickly and can positively affect those measures. Paying too much attention to short-run accounting numbers can result in a company choosing investments that are not in the long-run economic interests of its shareholders. Analysts can observe this behavior first by examining the direct financial incentives of management: the structure and composition of their compensation. Second, analysts can compare the level of capital spending to historical and peer levels to judge whether management has prioritized shorter-term, accounting-based measures. However, a decline in capital investment can be a sign of a lack of investment opportunities, in which case allocating capital to alternative uses is wise.

  \item Internal forecasting errors. In addition, companies might make errors in their internal forecasts, which could be difficult, if not impossible, for external analysts to identify. However, if significant enough, the incorrect or flawed analysis will ultimately manifest itself in failed, or underperforming, investment outcomes. These errors include incorrect cost or discount rate inputs. For example, overhead costs such as management time, IT support, and financial systems can be challenging to estimate. The incorrect treatment of sunk costs, missed real opportunity costs, and the use of the company's overall cost of capital, cost of debt, or cost of equity, rather than the investment's required rate of return, are also common mistakes by companies. Finally, companies often fail to incorporate market responses into the analysis of a planned investment.

\end{itemize}

\section{CORPORATE USE OF CAPITAL ALLOCATION}
describe expected relations among a company's investments,
company value, and share price

Analysts should understand the basic logic of the various capital allocation criteria as well as the practicalities involved in their use at corporations. The usefulness of any analytical tool always depends on the specific application.

If a company can make an investment that earns more than its opportunity cost of funds, then the investment is creating value for stakeholders and should be undertaken. Conversely, if a potential investment is expected to earn less than the company's opportunity cost of funds, or an existing investment is already underperforming the hurdle rate, the investment decreases stakeholder value, and alternative uses should be considered.

The return on invested capital (ROIC) is a measure of the profitability of a company or business segment relative to the amount of capital invested by the equityand debtholders. ROIC reflects how effectively a company's management is able to convert capital into after-tax operating profits. The ratio is calculated by dividing the after-tax operating profit by the average book value of invested capital: common equity, preferred equity, and debt, or

ROIC = After-Tax Operating Profit/Average Book Value of Invested Capital

The numerator does not expense the cost of financing (e.g., interest expense) because it represents a source of return for providers of debt capital, and the denominator includes sources of capital from all providers. Example 4 illustrates an example where ROIC might provide important insights into a company's situation.

\section{EXAMPLE 4}
\section{Valeant ROIC}
In September 2010, Valeant Pharmaceuticals International, Inc., a TSX and NYSE listed pharmaceutical company, embarked on an acquisitions-focused growth strategy. This strategy was quite different from the R\&D-driven strategies used by most pharmaceutical companies at the time. Valeant would selectively acquire companies with approved products and pursue opportunities to increase revenues, by launching in new geographies, and profitability, by reducing overhead and R\&D expenses. Valeant completed dozens of acquisitions between 2010 and 2015, using debt to finance a large portion of its purchases.

From September 2010 to mid-2015, Valeant's share price increased from CAD26 per share to over CAD330 per share, as the company increased revenues and adjusted net income rapidly, largely as a result of its acquisitions. However, looking at the company's ROIC told a different story. Calculating the measure from reported (US GAAP) figures showed that the company's ROIC was lackluster, never rising above $8 \%$. Even when using the company's adjusted (non GAAP) earnings figures, ROIC was in the low double-digit percentages. In most years, Valeant's free cash flow significantly lagged adjusted earnings significantly, casting doubt as to the relevance of its adjusted earnings figures. These measures are shown in the following chart.

\begin{center}
\includegraphics[max width=\textwidth]{2023_05_04_b5cfa4f1bc883752f121g-759}
\end{center}

In 2015, problems began to surface for Valeant when an activist investor published a report detailing significant problems with Valeant's business model and pointing to price increases as being responsible for a significant percentage of the company's growth. The report also alleged accounting misstatements of transactions with one of Valeant's subsidiaries.

In response to these damaging allegations, the company restated its financial statements, began a restructure of its business model, and made numerous divestments to ensure it would not default on its debt. Valeant's share price reaction was swift and severe; by year-end 2016, Valeant shares had fallen to CAD18 per share, down $95 \%$ from 18 months earlier.

While some of the problems with Valeant would not be visible in its ROIC, an important takeaway for analysts is to closely scrutinize issuer claims of value creation associated with acquisition activity when ROIC is low and falling. Secondly, it is important to prioritize cash flow measures such as free cash flow over accounting measures, especially adjusted accounting measures, in evaluating capital investments.

The ROIC measure is often compared with the associated company cost of capital (COC), the required return used in the NPV calculation, and the company's associated cost of funds. If the ROIC measure is higher than the COC, the company is generating a higher return for investors compared with the required return, thereby increasing the firm's value for shareholders. The inverse is true if the COC is higher than the ROIC.

If management has undertaken investments with a positive NPV and/or an IRR greater than its COC, the ROIC for the issuer overall will generally exceed the COC as well, and value will be created for shareholders.

Example 5 illustrates this scenario for a company making a new investment.

\section{EXAMPLE 5}
\section{NPVs and Stock Prices}
\begin{enumerate}
  \item Paladote is investing CAD600 million in distribution facilities. The present value of the future after-tax cash flows is estimated to be CAD850 million. Paladote has 200 million outstanding shares with a current market price of CAD32.00 per share. This investment is new information, and it is independent of other expectations about the company. What should be the investment's effect on the value of the company and the stock price?
\end{enumerate}

\section{Solution}
The NPV of the investment is CAD850 million - CAD600 million = CAD250 million. The total market value of the company prior to the investment is CAD32.00 $\times 200$ million shares $=$ CAD6,400 million. The value of the company should increase by CAD250 million, to CAD6,650 million. The price per share should increase by the NPV per share, or CAD250 million/200 million shares = CAD1.25 per share. The share price should increase from CAD32.00 to CAD33.25.

The effect of a capital investment's positive or negative NPV on share price is more complicated than in Example 3, in which the value of the stock increased by the investment's NPV. The value of a company is the value of its existing investments plus the NPVs of all its future investments, accounting for any externalities. If an analyst learns of an investment, the impact of that investment on the company's stock price will depend on whether the investment's profitability is more or less than expected.

For example, an analyst could learn of a positive-NPV project, but if the project's profitability is less than expected, the company's stock price might drop because of this news. Alternatively, news of a particular capital project might be considered a signal about other capital investments under way or in the future. An investment that by itself might add EUR0.25, for example, to the value of the stock might signal the existence of other profitable projects. News of this investment might increase the stock price by far more than EUR0.25.

Inflation affects a company's capital allocation analysis in several ways. The first is whether the investment analysis is done in "nominal" terms or in "real" terms. Nominal cash flows include the effects of inflation, whereas real cash flows are adjusted downward to remove the effects of inflation. Companies might choose to do the analysis in either nominal or real terms, and sound decisions can be made either way, but irrespective of choice, the cash flows and discount rate used by the company should be consistent. That is, nominal cash flows should be discounted at a nominal discount rate, and real cash flows should be discounted at a real rate.

Inflation reduces the value of depreciation tax savings to the company (unless the tax system adjusts depreciation for inflation), effectively increasing its real taxes. The effect of expected inflation is captured in the discounted cash flow (DCF) analysis. If inflation is higher than expected, the profitability of the investment is correspondingly lower than expected. Inflation essentially shifts wealth from the taxpayer (i.e., company) to the government. Conversely, lower-than-expected inflation reduces real taxes for the company (the depreciation tax shelters are more valuable than expected) and results in higher-than-expected profitability of the investment and a corresponding wealth increase for the company.

Finally, inflation does not affect all revenues and costs uniformly. The company's after-tax cash flows will be better or worse than expected depending on how particular sales outputs or cost inputs are affected.

\section{REAL OPTIONS}
$$
\text { describe types of real options relevant to capital investment }
$$

Real options are options that allow companies to make decisions in the future that alter the value of capital investment decisions made today. Instead of making all capital investment decisions now, at time zero, a company can wait and make additional decisions at future dates when these future decisions are contingent on future economic events or information. It is more reasonable to assume that a company is making decisions sequentially, some now and some in the future, rather than one-time decisions. A combination of optimal current and future decisions is what will maximize company value. Real options, by providing future decision-making flexibility to companies, can be an important piece of the value in many capital investments.

Real options are like financial options except that they deal with real assets instead of financial assets.

\section{EXAMPLE 6}
\section{Financial Option Similarity}
A simple financial option could be a call option on a share of stock. Suppose Paladote stock is selling for CAD50, you own a call option with an exercise (strike) price of CAD50, and the option expires in one year.

\begin{enumerate}
  \item What action should you take if Paladote's stock price increases to CAD60?
\end{enumerate}

\section{Solution}
Because Paladote's stock price (CAD60) is greater than the option exercise price (CAD50), you should exercise the option, which allows you to buy Paladote stock, worth CAD60, for a price of CAD50. In effect, you have a CAD10 gain less what you paid for the option.

\begin{enumerate}
  \setcounter{enumi}{1}
  \item What action should you take if Paladote's stock price falls to CAD40?
\end{enumerate}

\section{Solution}
Because Paladote's stock price (CAD40) is less than the option exercise price (CAD50), you should not exercise the option. After all, why would you pay CAD50 for Paladote when it is only worth CAD40? If you really want to own the stock, it would be cheaper to buy Paladote in the market for CAD40.

Real options, like financial options, grant companies the right to make a decision but do not impose an obligation to do so. A company should exercise a real option it holds only if it is value enhancing.

Just as financial options are contingent on an underlying asset, real options are contingent on future events for a company. The flexibility that real options give to companies can greatly enhance the NPV of the companies' capital investments. The following are four types of real options used by companies:

\begin{itemize}
  \item Timing options

  \item Sizing options

  \item Flexibility options

  \item Fundamental options

\end{itemize}

\section{Timing Options}
Instead of investing now, the company can choose to delay its investing decision. In doing so, the company hopes to obtain improved information for the NPV of the projects selected. Project sequencing options allow the company to defer the decision to invest in a future investment until the outcome of some or all of a current investment is known. Investments can be sequenced over time so that investing in a project creates the option to make future investments.

\section{Sizing Options}
An abandonment option allows a company to abandon the investment after it is undertaken, if the financial results are disappointing. At some future date, if the cash flow from abandoning an investment exceeds the present value of the cash flows from continuing the investment, the company should exercise the abandonment option. Conversely, if the company can make additional investments when future financial results are strong, the company has a growth option, or an expansion option. When estimating the cash flows from an expansion, the analyst must also be wary of cannibalization.

\section{Flexibility Options}
Companies might also have other options for operational flexibilities, besides abandonment or expansion, once an investment is made. For example, suppose the company finds itself in a situation where demand for its product or service exceeds capacity. Management might be able to exercise a price-setting option. By increasing prices, the company could benefit from the excess demand, which it cannot do by increasing production.

There are also production-flexibility options, which offer the company the operational flexibility to alter production when demand varies from what is forecast. The company can profit from working overtime or from adding shifts, which in this case makes economic sense, even though it is expensive.

\section{Fundamental Options}
In such cases as the aforementioned, options are embedded in a project that can raise its value. In other cases, the whole investment is essentially an option. The payoffs from the investment are contingent on an underlying asset, just like most financial options.

For example, the value of an oil well or refinery investment is contingent on the price of oil. The value of a gold mine is contingent on the price of gold. If oil prices were low, the company likely would not choose to drill a well. If oil prices were high, it would go ahead and drill. Many R\&D projects also look like options.

Companies use the following approaches in evaluating capital investments with real options:

\begin{enumerate}
  \item DCF analysis without considering options. If the $\mathrm{NPV}$ is positive to the company without considering real options and the project has real options that would add more value, the company should simply make the investment.

  \item Project NPV = NPV (based on DCF alone) - Cost of options + Value of options. In this approach, the company determines the project NPV based on expected cash flows, then subtracts the incremental cost of the real options and adds back their associated value. 3. Decision trees and option pricing models. Both of these can be used by companies to assess the value associated with future sequential decisions a company might have for its business.

\end{enumerate}

Example 7 illustrates a production-flexibility option for a company, in which an additional investment outlay gives the company an option to use alternative fuel sources.

\section{EXAMPLE 7}
\section{Production-Flexibility Option}
\begin{enumerate}
  \item Auvergne AquaFarms has estimated the NPV of the expected cash flows from a new processing plant to be -EUR0.40 million. Auvergne is evaluating an incremental investment of EUR0.30 million that would give the company the flexibility to switch among coal, natural gas, and oil as energy sources. The original plant relied only on coal. The option to switch to cheaper sources of energy when they are available has an estimated value of EUR1.20 million. What is the value of the new processing plant including this real option to use alternative energy sources?
\end{enumerate}

\section{Solution}
The NPV, including the real option, should be

Project NPV

= NPV (based on DCF alone) - Cost of options + Value of options.

Project NPV $=-0.40$ million -0.30 million +1.20 million

$=$ EUR0.50 million.

Without the flexibility offered by the real option, the plant is unprofitable. The real option to adapt to cheaper energy sources adds enough to the value of this investment to give it a positive NPV. The company should undertake the investment, which would add to its value.

\section{SUMMARY}
Capital investments-those investments with a life of one year or longer-are key in determining whether a company is profitable and generating value for its shareholders. Capital allocation is the process companies use to decide their capital investment activity. This reading introduces capital investments, basic principles underlying the capital allocation model, and the use of NPV and IRR decision criteria.

\begin{itemize}
  \item Companies invest for two reasons: to maintain their existing businesses and to grow them. Projects undertaken by companies to maintain a business including operating efficiencies are (1) going concern projects and (2) regulatory/compliance projects, while (3) expansion projects and (4) other projects are undertaken by companies to strategically expand or grow their operations.

  \item Capital allocation supports the most critical investments for many corporations - their investments in long-term assets. The principles of capital allocation are also relevant and can be applied to other corporate investing and financing decisions and to security analysis and portfolio management.

  \item The typical steps companies take in the capital allocation process are (1) idea generation, (2) investment analysis, (3) capital allocation planning, and (4) post-audit/monitoring.

  \item Companies should base their capital allocation decisions on the investment project's incremental after-tax cash flows discounted at the opportunity cost of funds. In addition, companies should ignore financing costs because both the cost of debt and the cost of other capital are captured in the discount rate used in the analysis.

  \item The NPV of an investment project is the present value of its after-tax cash flows (or the present value of its after-tax cash inflows minus the present value of its after-tax outflows) or

\end{itemize}

$\mathrm{NPV}=\sum_{t=0}^{n} \frac{\mathrm{CF}_{t}}{(1+r)^{t}}$

where the investment outlays are negative cash flows included in $\mathrm{CF}_{t}$ and $r$ is the required rate of return for the investment.

\begin{itemize}
  \item Microsoft Excel functions to solve for the NPV for both conventional and unconventional cash flow patterns are

  \item NPV or $=\mathrm{NPV}($ rate, values) and

  \item XNPV or $=$ XNPV(rate, values, dates),

\end{itemize}

where "rate" is the discount rate, "values" are the cash flows, and "dates" are the dates of each of the cash flows.

\begin{itemize}
  \item The IRR is the discount rate that makes the present value of all future cash flows of the project sum to zero. This equation can be solved for the IRR:
\end{itemize}

$\sum_{t=0}^{n} \frac{\mathrm{CF}_{t}}{\left(1+\mathrm{IRR}^{t}\right.}=0$.

\begin{itemize}
  \item Using Microsoft Excel functions to solve for IRR, the functions are

  \item $\quad$ IRR or $=I R R($ values, guess) and - XIRR or =XIRR(values, dates, guess),

\end{itemize}

where "values" are the cash flows, "guess" is an optional user-specified guess that defaults to $10 \%$, and "dates" are the dates of each cash flow.

\begin{itemize}
  \item Companies should invest in a project if the NPV $>0$ or if the IRR $>r$.

  \item For mutually exclusive investments that are ranked differently by the NPV and IRR, the NPV criterion is the more economically sound and the approach companies should use.

  \item The fact that projects with positive NPVs theoretically increase the value of the company and the value of its stock could explain the use and popularity of the NPV method by companies.

  \item Real options allow companies to make future decisions contingent on future economic information or events that change the value of capital investment decisions the company has made today. These can be classified as (1) timing options; (2) sizing options, which can be abandonment options or growth (expansion) options; (3) flexibility options, which can be price-setting options or production-flexibility options; and (4) fundamental options.

\end{itemize}

\section{PRACTICE PROBLEMS}
\begin{enumerate}
  \item With regard to capital allocation, an appropriate estimate of the incremental cash flows from an investment is least likely to include:
\end{enumerate}

A. externalities.

B. interest costs.

C. opportunity costs.

\begin{enumerate}
  \setcounter{enumi}{1}
  \item The NPV of an investment is equal to the sum of the expected cash flows discounted at the:
\end{enumerate}

A. internal rate of return.

B. risk-free rate.

C. opportunity COC.

\begin{enumerate}
  \setcounter{enumi}{2}
  \item A USD2.2 million investment will result in the following year-end cash flows:
\end{enumerate}

\begin{center}
\begin{tabular}{lclll}
\hline
Year & $\mathbf{1}$ & $\mathbf{2}$ & $\mathbf{3}$ & $\mathbf{4}$ \\
\hline
$\begin{array}{l}\text { Cash flow } \\ \text { (millions) }\end{array}$ & USD1.3 & USD1.6 & USD1.9 & USD0.8 \\
\hline
\end{tabular}
\end{center}

Using an 8\% opportunity COC, the investment's NPV is closest to:
A. USD2.47 million.
B. USD3.40 million.
C. USD4.67 million.

\begin{enumerate}
  \setcounter{enumi}{3}
  \item The IRR is best described as the:
A. opportunity COC.
B. time-weighted rate of return.
C. discount rate that makes the NPV equal to zero.

  \item A three-year investment requires an initial outlay of GBP1,000. It is expected to provide three year-end cash flows of GBP200 plus a net salvage value of GBP700 at the end of three years. Its IRR is closest to:
A. $10 \%$.
B. $11 \%$.
C. $20 \%$.

  \item Given the following cash flows for a capital investment, calculate the NPV and IRR. The required rate of return is $8 \%$.

\end{enumerate}

\begin{center}
\begin{tabular}{lcccccc}
\hline
Year & $\mathbf{0}$ & $\mathbf{1}$ & $\mathbf{2}$ & $\mathbf{3}$ & $\mathbf{4}$ & $\mathbf{5}$ \\
\hline
Cash flow & $-50,000$ & 15,000 & 15,000 & 20,000 & 10,000 & 5,000 \\
\hline
\end{tabular}
\end{center}

$\begin{array}{ccc} & \text { NPV } & \text { IRR } \\ \text { A } & \text { USD1,905 } & 10.9 \% \\ \text { B } & \text { USD1,905 } & 26.0 \% \\ \text { C } & \text { USD3,379 } & 10.9 \%\end{array}$

\begin{enumerate}
  \setcounter{enumi}{6}
  \item An investment of USD100 generates after-tax cash flows of USD40 in Year 1, USD80 in Year 2, and USD120 in Year 3. The required rate of return is $20 \%$. The NPV is closest to:
A. USD42.22.
B. USD58.33.
C. USD68.52.

  \item An investment of USD150,000 is expected to generate an after-tax cash flow of USD100,000 in one year and another USD120,000 in two years. The COC is $10 \%$. What is the IRR?
A. $28.39 \%$
B. $28.59 \%$
C. $28.79 \%$

  \item Kim Corporation is considering an investment of KRW750 million with expected after-tax cash inflows of KRW175 million per year for seven years. The required rate of return is $10 \%$. What is the investment's:
NPV? $\quad$ IRR?
A. KRW102 million $\quad 14.0 \%$
B. KRW157 million $\quad 23.3 \%$
C. $\quad$ KRW193 million $\quad 10.0 \%$

  \item Erin Chou is reviewing a profitable investment that has a conventional cash flow pattern. If the cash flows for the initial outlay and future after-tax cash flows all double, Chou would predict that the IRR would:
A. increase and the NPV would increase.
B. stay the same and the NPV would increase.
C. stay the same and the NPV would stay the same.

  \item Catherine Ndereba is an energy analyst tasked with evaluating a crude oil exploration and production company. The company previously announced that it plans to embark on a new project to drill for oil offshore. As a result of this announcement, the stock price increased by $10 \%$. After conducting her analysis, Ms. Ndereba concludes that the project does indeed have a positive NPV. Which statement is true? A. The stock price should remain where it is because Ms. Ndereba's analysis confirms that the recent run-up was justified.

\end{enumerate}

B. The stock price should go even higher now that an independent source has confirmed that the NPV is positive.

C. The stock price could remain steady, move higher, or move lower.

\begin{enumerate}
  \setcounter{enumi}{11}
  \item The Bearing Corp. invests only in positive-NPV projects. Which of the following statements is true?
\end{enumerate}

A. Bearing's ROIC is greater than its COC.

B. Bearing's COC is greater than its ROIC.

C. We cannot reach any conclusions about the relationship between the company's ROIC and COC.

\begin{enumerate}
  \setcounter{enumi}{12}
  \item Investments 1 and 2 have similar outlays, although the patterns of future cash flows are different. The cash flows, as well as the NPV and IRR, for the two investments are shown below. For both investments, the required rate of return is $10 \%$.
\end{enumerate}

\begin{center}
\begin{tabular}{lccccccc}
\hline
\multicolumn{7}{c}{Cash Flows} &  \\
\hline
Year & $\mathbf{0}$ & $\mathbf{1}$ & $\mathbf{2}$ & $\mathbf{3}$ & $\mathbf{4}$ & $\mathbf{N P V}$ & IRR (\%) \\
\hline
Investment 1 & -50 & 20 & 20 & 20 & 20 & 13.40 & 21.86 \\
Investment 2 & -50 & 0 & 0 & 0 & 100 & 18.30 & 18.92 \\
\hline
\end{tabular}
\end{center}

The two projects are mutually exclusive. What is the appropriate investment decision?

A. Invest in both investments.

B. Invest in Investment 1 because it has the higher IRR.

C. Invest in Investment 2 because it has the higher NPV.

\begin{enumerate}
  \setcounter{enumi}{13}
  \item Consider the two investments below. The cash flows, as well as the NPV and IRR, for the two investments are given. For both investments, the required rate of return is $10 \%$.
\end{enumerate}

\begin{center}
\begin{tabular}{lccccccc}
\hline
\multicolumn{7}{c}{Cash Flows} &  \\
\hline
Year & $\mathbf{0}$ & $\mathbf{1}$ & $\mathbf{2}$ & $\mathbf{3}$ & $\mathbf{4}$ & $\mathbf{N P V}$ & IRR (\%) \\
\hline
Investment 1 & -100 & 36 & 36 & 36 & 36 & 14.12 & 16.37 \\
Investment 2 & -100 & 0 & 0 & 0 & 175 & 19.53 & 15.02 \\
\hline
\end{tabular}
\end{center}

What discount rate would result in the same NPV for both investments?

A. A rate between $0.00 \%$ and $10.00 \%$

B. A rate between $10.00 \%$ and $15.02 \%$

C. A rate between $15.02 \%$ and $16.37 \%$ 15. Wilson Flannery is concerned that the following investment has multiple IRRs.

\begin{center}
\begin{tabular}{ccccc}
\hline
Year & $\mathbf{0}$ & $\mathbf{1}$ & $\mathbf{2}$ & $\mathbf{3 w}$ \\
\hline
Cash flows & -50 & 100 & 0 & -50 \\
\hline
\end{tabular}
\end{center}

How many discount rates produce a zero NPV for this investment?
A. One, a discount rate of $0 \%$
B. Two, discount rates of $0 \%$ and $32 \%$
C. Two, discount rates of $0 \%$ and $62 \%$

\begin{enumerate}
  \setcounter{enumi}{15}
  \item What type of project is most likely to yield new revenues for a company?
A. Regulatory/compliance
B. Going concern
C. Expansion
\end{enumerate}

\section{The following information relates to questions}
 17-19Bouchard Industries is a Canadian company that manufactures gutters for residential houses. Its management believes it has developed a new process that produces a superior product. The company must make an initial investment of CAD190 million to begin production. If demand is high, cash flows are expected to be CAD40 million per year. If demand is low, cash flows will be only CAD20 million per year. Management believes there is an equal chance that demand will be high or low. The investment, which has an investment horizon of ten years, also gives the company a production-flexibility option allowing the company to add shifts at the end of the first year if demand turns out to be high. If the company exercises this option, net cash flows would increase by an additional CAD5 million in Years 2-10. Bouchard's opportunity cost of funds is $10 \%$.

The internal auditor for Bouchard Industries has made two suggestions for improving capital allocation processes at the company. The internal auditor's suggestions are as follows:

Suggestion 1: "In order to treat all capital allocation proposals in a fair manner, the investments should all use the risk-free rate for the required rate of return."

Suggestion 2: "When rationing capital, it is better to choose the portfolio of investments that maximizes the company NPV than the portfolio that maximizes the company IRR."

\begin{enumerate}
  \setcounter{enumi}{16}
  \item What is the NPV (CAD millions) of the original project for Bouchard Industries without considering the production-flexibility option?
A. -CAD6.11 million
B. -CAD5.66 million
C. CAD2.33 million

  \item What is the NPV (CAD millions) of the optimal set of investment decisions for Bouchard Industries including the production-flexibility option?
A. -CAD6.34 million
B. CAD7.43 million
C. CAD31.03 million

  \item Should the capital allocation committee accept the internal auditor's suggestions?
A. No for Suggestions 1 and 2
B. No for Suggestion 1 and yes for Suggestion 2
C. Yes for Suggestion 1 and no for Suggestion 2

\end{enumerate}

\section{SOLUTIONS}
\begin{enumerate}
  \item B is correct. Costs to finance the investment are taken into account when the cash flows are discounted at the appropriate COC; including interest costs in the cash flows would result in double-counting the cost of debt.

  \item C is correct. The NPV sums the investment's expected cash flows (CF) discounted at the opportunity COC. The NPV calculation is

\end{enumerate}

$\mathrm{NPV}=\sum_{t=0}^{N} \frac{\mathrm{CF}_{t}}{(1+r)^{t}}$

where

$\mathrm{CF}_{t}=$ the expected net cash flow at time $t$

$N=$ the investment's projected life

$r=$ the discount rate or opportunity $\mathrm{COC}$

\begin{enumerate}
  \setcounter{enumi}{2}
  \item A is correct.
\end{enumerate}

$\mathrm{NPV}=-\$ 2.2+\frac{\$ 1.3}{(1.08)}+\frac{\$ 1.6}{(1.08)^{2}}+\frac{\$ 1.9}{(1.08)^{3}}+\frac{\$ 0.8}{(1.08)^{4}}=\$ 2.47$ million

\begin{enumerate}
  \setcounter{enumi}{3}
  \item $\mathrm{C}$ is correct. The IRR is computed by identifying all cash flows and solving for the rate that makes the NPV of those cash flows equal to zero.

  \item B is correct. Using either the IRR function in Excel or a financial calculator, IRR is determined by setting the NPV equal to zero for the cash flows shown in the following table.

\end{enumerate}

\begin{center}
\begin{tabular}{lcccc}
\hline
Year & 0 & $\mathbf{1}$ & $\mathbf{2}$ & $\mathbf{3}$ \\
\hline
Cash flow (GBP) & $-1,000$ & 200 & 200 & 900 \\
\hline
\end{tabular}
\end{center}

\begin{enumerate}
  \setcounter{enumi}{5}
  \item $\mathrm{C}$ is correct.
\end{enumerate}

$\mathrm{NPV}=-50,000+\frac{15,000}{1.08}+\frac{15,000}{1.08^{2}}+\frac{20,000}{1.08^{3}}+\frac{10,000}{1.08^{4}}+\frac{5,000}{1.08^{5}}$

$\mathrm{NPV}=-50,000+13,888.89+12,860.08+15,876.64+7,350.30$

$+3,402.92$

$\mathrm{NPV}=-50,000+53,378.83=3,378.83$.

Using either the IRR function in Excel or a financial calculator, the IRR is 10.88\%.

\begin{enumerate}
  \setcounter{enumi}{6}
  \item B is correct.
\end{enumerate}

$\mathrm{NPV}=\sum_{t=0}^{3} \frac{\mathrm{CF}_{t}}{(1+r)^{t}}=-100+\frac{40}{1.20}+\frac{80}{1.20^{2}}+\frac{120}{1.20^{3}}=\$ 58.33$.

\begin{enumerate}
  \setcounter{enumi}{7}
  \item C is correct. Using a financial calculator or the trial and error method, the IRR is $28.79 \%$. The COC, which is stated as $10 \%$, is not used to solve the problem.
\end{enumerate}

\begin{center}
\begin{tabular}{cccccc}
\hline
 & \multicolumn{4}{c}{Present Value} &  \\
\hline
Year & Cash Flow & $\mathbf{2 8 . 1 9 \%}$ & $\mathbf{2 8 . 3 9 \%}$ & $\mathbf{2 8 . 5 9 \%}$ & $\mathbf{2 8 . 7 9 \%}$ \\
\hline
0 & $-150,000$ & $-150,000$ & $-150,000$ & $-150,000$ & $-150,000$ \\
1 & 100,000 & 78,009 & 77,888 & 77,767 & 77,646 \\
\end{tabular}
\end{center}

\begin{center}
\begin{tabular}{|c|c|c|c|c|c|}
\hline
\multirow[b]{2}{*}{Year} & \multirow[b]{2}{*}{Cash Flow} & \multicolumn{4}{|c|}{Present Value} \\
\hline
 &  & $28.19 \%$ & $28.39 \%$ & $28.59 \%$ & $28.79 \%$ \\
\hline
2 & 120,000 & 73,025 & 72,798 & 72,572 & 72,346 \\
\hline
Total &  & 1,034 & 686 & 338 & -8 \\
\hline
Year &  & 0 &  &  & 2 \\
\hline
Cash fle &  & $-150,000$ & 1 & ) & 120,000 \\
\hline
\end{tabular}
\end{center}

Using the IRR function in Excel results in a more precise IRR of $28.7854 \%$ with a total present value closer to zero.

\begin{enumerate}
  \setcounter{enumi}{8}
  \item A is correct.
\end{enumerate}

$\mathrm{NPV}=-750+\sum_{t=1}^{7} \frac{175}{1.10^{t}}=-750+851.97=101.97$ million won.

Using either the IRR function in Excel or a financial calculator, the IRR is 14.02\%. Using a financial calculator, present value is $-750, N=7$, and PMT $=175$.

\begin{enumerate}
  \setcounter{enumi}{9}
  \item B is correct. The IRR would stay the same because both the initial outlay and the after-tax cash flows double, so the return on each dollar invested would remain the same. All the cash flows and their present values double. The difference between the total present value of the future cash flows and the initial outlay (the NPV) also doubles.

  \item $\mathrm{C}$ is correct. There are many factors that can affect the stock price, including whether Ms. Ndereba's analysis indicates that the project is more or less profitable than investors expected.

  \item A is correct. Because all Bearing's projects have a positive NPV, they are all providing a return that is greater than the opportunity COC. Therefore, the ROIC must be greater than the COC.

  \item C is correct. When valuing mutually exclusive investments, the decision should be made with the NPV method because this method uses the most realistic discount rate-namely, the opportunity cost of funds. In this example, the reinvestment rate for the NPV method (here, $10 \%$ ) is more realistic than the reinvestment rate for the IRR method (here, $21.86 \%$ or $18.92 \%$ ).

  \item B is correct. For these investments, a discount rate of $13.16 \%$ would yield the same NPV for both (an NPV of 6.73).

  \item $\mathrm{C}$ is correct. Discount rates of $0 \%$ and approximately $61.8 \%$ both give an NPV of zero.

\end{enumerate}

\begin{center}
\begin{tabular}{llllllll}
\hline
Rate & $\mathbf{0 \%}$ & $\mathbf{2 0 \%}$ & $\mathbf{4 0 \%}$ & $\mathbf{6 0 \%}$ & $\mathbf{6 1 . 8 \%}$ & $\mathbf{8 0 \%}$ & $\mathbf{1 0 0 \%}$ \\
\hline
NPV & 0.00 & 4.40 & 3.21 & 0.29 & 0.00 & -3.02 & -6.25 \\
\hline
\end{tabular}
\end{center}

\begin{enumerate}
  \setcounter{enumi}{15}
  \item C is correct. Expansion projects increase the scale of a firm's existing activities and/or extend a firm's reach into new product or service categories and markets, in the hopes of generating longer-term expected gains. Regulatory/compliance projects are required for the business to continue operations but otherwise might not be undertaken by a company. Going concern projects benefit the company through improved efficiencies and cost savings over time. 17. B is correct.
\end{enumerate}

If demand is "high," the NPV is as follows:

$\mathrm{NPV}=-190+\sum_{t=1}^{10} \frac{40}{1.10^{t}}=\mathrm{C} \$ 55.78$ million.

If demand is "low," the NPV is

$\mathrm{NPV}=-190+\sum_{t=1}^{10} \frac{20}{1.10^{t}}=-\mathrm{C} \$ 67.11$ million.

The expected NPV is $0.50(55.78)+0.50(-67.11)=-C \$ 5.66$ million.

\begin{enumerate}
  \setcounter{enumi}{17}
  \item B is correct. The additional NPV of adding shifts if demand is "high" is
\end{enumerate}

$\mathrm{NPV}=\sum_{t=2}^{10} \frac{5}{1.10^{t}}=\mathrm{C} \$ 26.18$ million.

If demand is "low," the production-flexibility option will not be exercised. The optimal decision is to add shifts only if demand is high.

Because the production-flexibility option is exercised only when demand is high, which happens $50 \%$ of the time, the expected present value of adding shifts is

$\mathrm{NPV}=0.50(26.18)=\mathrm{CAD} 3.09$ million .

The total NPV of the initial project and the production-flexibility option is

NPV $=-$ CAD5.66 million + CAD13.09 million $=$ CAD7.43 million.

The option to add shifts, handled optimally, adds sufficient value to make this a positive-NPV project.

\begin{enumerate}
  \setcounter{enumi}{18}
  \item B is correct. In valuing investments, expected cash flows should be discounted at required rates of return that reflect their risk, not at a risk-free rate that ignores risk. NPV is superior to IRR. Choosing projects based on IRR might cause the company to concentrate on short-term investments that do not maximize the company's NPV.
\end{enumerate}

\section{LEARNINGMODULE}
\begin{center}
\includegraphics[max width=\textwidth]{2023_05_04_b5cfa4f1bc883752f121g-775}
\end{center}

\section{Working Capital \& Liquidity}
\section{LEARNING OUTCOME}
\begin{center}
\begin{tabular}{c|l}
Mastery & The candidate should be able to: \\
$\square$ & compare methods to finance working capital \\
$\square$ & $\begin{array}{l}\text { explain expected relations between working capital, liquidity, and } \\ \text { short-term funding needs (NEW) } \\ \text { describe sources of primary and secondary liquidity and factors } \\ \text { affecting a company's liquidity position } \\ \text { compare a company's liquidity position with that of peers } \\ \square\end{array}$ \\
evaluate short-term funding choices available to a company &  \\
\end{tabular}
\end{center}

\section{INTRODUCTION}
Working capital (also called net working capital) is defined simply as current assets minus current liabilities or

(Net) Working capital $=$ Current assets - Current liabilities

and includes both operating assets and liabilities, such as accounts receivable, accounts payable, and inventory, as well as financial assets and liabilities, such as short-term investments and short-term debt. Working capital management is the management of a firm's short-term assets and liabilities and an important aspect of a firm's operations. The goal of working capital management is to ensure the company has adequate, ready access to funds necessary for day-to-day operations, while avoiding excess reserves that can be a costly drag on the business' profitability and returns. Having excess levels of working capital can have a harmful effect on shareholder returns. At the same time, insufficient levels of working capital can harm a company if it cannot meet its short-term obligations, leading to product shortages, sales slowdowns, and, in the extreme, bankruptcy.

An analyst should carefully evaluate the working capital position of the firm to make an informed decision about the firm's ability to meet its short-term needs as it works to implement its long-term plans. To assess whether a firm is operating at an optimal level of working capital, financed at the lowest possible cost, an analyst should begin by asking two fundamental questions:

\begin{itemize}
  \item What are the required investments in working capital for the firm? - How should those investments be financed?
\end{itemize}

Understanding this provides the analyst with a basis for sound valuation analysis.

\begin{center}
\includegraphics[max width=\textwidth]{2023_05_04_b5cfa4f1bc883752f121g-776}
\end{center}

FINANCING OPTIONS

compare methods to finance working capital

A firm typically has a number of options to finance itself, and how a firm chooses to finance itself can determine whether the firm stays liquid long enough to see its long-term plans materialize. Given the complex nature of a firm's assets and liabilities, understanding the financing options available to a firm is important. Companies make their decisions based on the costs and risks to the company and the returns and risks to the creditor or investor, in addition to market conditions, regulatory requirements, and the services of agents, brokers, dealers, and financial intermediaries working in the market.

It can be helpful to think of working capital from the perspective of uses and sources of funding for the firm. Cash, accounts receivable, inventory, and short-term investments, also termed marketable securities, represent uses of working capital because they must be financed.

Funds from operating activities such as accounts payable, credit lines, short-term loans, and short-term instruments sold (or issued) by the company, such as commercial paper, represent sources of short-term financing. Long-term debt and common equity issued by a company represent other sources of financing that can also be used to fund working capital needs.

Exhibit 1 shows a firm's financing alternatives by source and funding duration. A company can finance its working capital internally, through operating activities, or externally, through financial intermediaries and the capital markets. We discuss each of these in turn.

\section{Exhibit 1: Internal vs. External Financing Alternatives}
\begin{center}
\includegraphics[max width=\textwidth]{2023_05_04_b5cfa4f1bc883752f121g-776(1)}
\end{center}

\section{Internal Financing}
Companies of varying sizes and profitability can, in principle, generate internal financing and liquidity from their short-term operating activities in several ways. These include

\begin{itemize}
  \item generating more after-tax operating cash flow;

  \item increasing working capital efficiency, such as extending a company's payables period, reducing its receivables period, or shortening its cash conversion cycle; and

  \item converting liquid assets such as receivables, inventories, and marketable securities to cash.

\end{itemize}

\section{Operating Cash Flows}
Operating cash flows - which are the company's after-tax operating cash flows less interest and dividend payments (adjusted for taxes) - can be used to invest in assets and are equal to net income plus depreciation minus dividend payments. A company with higher, more predictable after-tax operating cash flows has greater ability to finance itself using internal means.

\section{Working Capital Efficiency: Accounts Payable and Accounts Receivable}
Accounts payable are amounts due to the company's suppliers of goods and services that have not been paid. They arise from trade credit, which is a spontaneous form of credit in which the company, as buyer of the goods or service, is financing its purchase by delaying the date on which payment is made.

Trade credit might involve a delay of payment with a discount for early payment. The terms of trade credit are generally stated in the discount form: A discount from the purchase price is given by the supplier if payment by the firm is received within a specified number of days; otherwise, the full amount is due by a specified date. For example, the terms " $2 / 10$, net 30 " indicate that a $2 \%$ discount is available if the account is paid within 10 days; otherwise, the full amount is due by the 30th day. While terms differ by industry-influenced by tradition within the industry, terms of competitors, and current interest rates-often accounts payable costs are high. Having market power, however, can help a company dictate more favorable terms for accounts payable, allowing it to delay payments without having to forgo discounts.

\section{EXAMPLE 1}
\section{Keown Corporation-Internal Financing Decision}
Keown Corp. is an established manufacturer of custom paddleboards for whitewater expeditions operating in the North American market. Keown, which operates its own manufacturing plant in Canada, sells its paddleboards exclusively through its website to avoid the overhead costs of retail locations.

The bulk of Keown's sales takes place during the North American summer season from May to August. Its customers want instant order fulfillment when they go to the site, so Keown must have substantial inventory on hand to start the summer season or risk losing sales to competitors who can provide prompt delivery if Keown cannot. Given the seasonality of the business, Keown is particularly focused on working capital management to ensure it can continue to operate and meet customers' needs. Keown has had a target capital structure of $60 \%$ equity and $40 \%$ debt. The business is profitable and has generated positive net income and free cash flow for the past five years. Keown's payout ratio is targeted at $40 \%$ of net income, allowing the firm to retain $60 \%$ for reinvestment. As the firm is considered mature by the market, the CEO feels it is critical to finance any working capital needs through the use of debt rather than equity to avoid diluting existing shareholders.

As the CFO of Keown, you are deciding whether you should draw down your external line of credit to take the discount offered by your suppliers (standard terms in the industry are $2 / 10$, net 30 ). The CEO has argued that she would rather not utilize the external line of credit for a $2 \%$ supplier discount. Keown can borrow through its line of credit at an effective annual rate of 7.7\%.

\begin{enumerate}
  \item Is the CEO correct? Should you forego the $2 \%$ supplier discount rather than drawing on the line of credit?
\end{enumerate}

\section{Solution}
No. Keown should not forego the $2 \%$ discount offered by its supplier and should instead use its external credit line for financing. This is because the effective annual rate (EAR) on the foregone trade credit is $44.6 \%$, or

$$
\begin{aligned}
& =\left[(1+(2 / 98))^{\wedge}(365 / 20)\right]-1 \\
& =\left[(1+0.02041)^{\wedge}(18.25)\right]-1 \\
& =44.6 \%,
\end{aligned}
$$

which is significantly higher than the $7.7 \%$ rate Keown would pay on proceeds drawn from the external credit line.

Accounts receivable are the opposite of accounts payable and represent amounts owed by the firm's customers. In general, businesses prefer to delay paying what they owe but prefer to receive what is owed to them as quickly as possible. One company's accounts payable represent another company's accounts receivable. The sooner a company can collect what it is owed, the lesser its need to finance its operations in some other way.

\section{Liquid Assets: Inventory and Marketable Securities}
Like accounts receivable, inventory is a current asset on the balance sheet. These include raw materials and work-in-progress in production as well as finished goods waiting to be sold. Investing in and holding inventory cost money. Companies would prefer not to put a lot of money into inventory when that money could be used for more productive means. The longer the inventory remains unsold, the longer that money is tied up and not usable for other purposes. There may also be costs associated with maintaining or insuring inventory, and inventory can become obsolete the longer it is held, resulting in costly write-downs. As a result, an efficient company holds as little inventory as is necessary and sells or turns over the inventory as quickly as possible. If a company holds too little inventory, however, it risks resulting shortfalls and lost sales. Companies must therefore manage the tradeoff between the costs of holding too much inventory and the benefits of avoiding inventory shortfalls. Once inventory is sold, the purchase amount moves into accounts receivable, and the cash becomes available when the customer makes payment.

Marketable securities are financial instruments, such as stocks and bonds, that can be quickly sold and converted to cash. They are listed as a current asset if the company intends to liquidate them within a year or if the security has less than a year in maturity, earmarking them for the company's working capital needs. These securities could be short-term debt that matures within a year, long-term debt, and even common stocks (which have no maturity date) that will be sold within a year. Companies often invest in marketable securities to earn a rate of return greater than what they would earn by holding cash. Companies can sell marketable securities if they need funds for any reason, such as when making a capital expenditure.

\section{External Financing: Financial Intermediaries}
Financial intermediaries, such as bank and non-bank lenders, can also be a means to finance working capital. The main types of short-term bank financing include

\begin{itemize}
  \item uncommitted bank lines of credit,

  \item committed bank lines of credit, and

  \item revolving credit agreements, or revolvers.

\end{itemize}

The latter two types can be unsecured or secured, depending on the company's financial strength and the general credit situation, which can vary from country to country. Uncommitted lines and revolvers are more common in the United States, whereas regular committed lines are more common in other parts of the world. Lines of credit represent flexible and immediate funding access.

\section{Lines of Credit}
Uncommitted lines of credit are, as the name suggests, the least reliable form of bank borrowing. A bank can offer an uncommitted line of credit but reserves the right to refuse to honor any request for use of the line. Given this, companies should not rely on uncommitted lines as the banks are often unwilling to lend at the time the lines are most needed, such as during times of financial market stress. The primary attraction of uncommitted lines is that they do not require the company to pay any compensation other than interest.

Committed (regular) lines of credit are the form of bank line of credit, up to a pre-specified amount, that most companies refer to as regular lines of credit. They are more reliable than uncommitted lines because of the bank's formal commitment, which can be verified through an acknowledgment letter as part of the annual financial audit and can be footnoted in the company's annual report. These lines of credit are in effect for 364 days. This term length benefits companies by minimizing amounts needed to meet bank capital requirements. For commitments of a year or longer, banks require more capital. This term length also effectively ensures that, when drawn, these are short-term liabilities, usually classified as notes payable or the equivalent.

Regular lines are unsecured and pre-payable without penalty. The borrowing rate paid by the company is negotiated, with the most common rate being the bank's prime rate or a money market rate plus a spread. The spread is dependent on the firm's creditworthiness, which is the perceived willingness and ability of the firm to pay its debt obligations in a timely manner and represents the ability of a company to withstand adverse impacts on its cash flows. Unlike uncommitted lines, regular lines require compensation, usually in the form of a commitment fee to the lender. The fee is typically a fractional percent (e.g., $0.50 \%$ ) of the full amount or the unused amount of the line, depending on bank-company negotiations.

Revolving credit agreements, also referred to as "revolvers" (or "operating lines of credit"), are the most reliable form of short-term bank borrowing. They involve formal legal agreements that define the aspects of the agreement. These agreements are similar to those for regular lines with respect to borrowing rates, compensation, and being unsecured. Revolvers differ in that they are in effect for multiple years and can have optional medium-term loan features. In addition, they are often used for larger amounts than a regular line would cover, and these larger amounts are spread out among more than one bank. With revolvers, companies draw down and pay back amounts periodically.

\section{Loans and Factoring}
Secured ("asset-based") loans are loans in which the lender requires the company to provide collateral in the form of an asset, such as a fixed asset that the company owns or high-quality receivables, inventory, or marketable securities. These assets are pledged against the loan, and the lender files a lien against them. This lien becomes part of the borrowing company's financial record and is shown on its credit report. Companies that lack sufficient credit quality to qualify for unsecured loans might arrange for secured loans.

A company can use its accounts receivable to generate cash flow through the assignment of accounts receivable, which is the use of these receivables as collateral for a loan. A company can securitize its receivables by selling its accounts receivable to a special purpose vehicle (SPV), which in turn issues a bond, backed by the receivables as collateral, that is sold to investors. A company can also sell its accounts receivable to a lender, called a factor, typically at a substantial discount. In an assignment arrangement, the company remains responsible for the collection of the accounts, whereas in a factoring arrangement, the company shifts the credit-granting and collection process to the lender or factor. The cost of this credit (i.e., the amount of the discount) depends on the credit quality of the accounts and the costs of collection. Similarly, inventory can be used in different ways as collateral for a loan.

Web-based lenders and non-bank lenders are recent innovations not typically used by larger companies. Web-based lenders operate primarily on the internet, offering loans in relatively small amounts, typically to small businesses in need of cash. Non-bank lenders also lend to businesses, but unlike typical banks, which make loans and take deposits, these lenders only make loans and may provide specific financial services to targeted consumers and firms such as mortgage services, lease financing, and venture capital.

Many short-term loans have historically been tied to Libor (London Interbank Offered Rate), such as Libor plus 0.50\% (50 basis points). However, in 2021, Libor and other similar global benchmark rates were replaced with alternative reference rates in the United States, Europe, Switzerland, Great Britain, and Japan. Both bank and non-bank lenders often charge companies commitment fees, or additional commissions and fees beyond the quoted interest rate.

\section{External Financing: Capital Markets}
\section{Short-Term Commercial Paper}
Commercial paper is a short-term, unsecured instrument typically issued by large, well-rated companies. A company sells, or issues, commercial paper (also referred to as promissory notes or bills of exchange) directly to investors or through dealers. To avoid registration costs with national regulatory agencies, maturities for commercial paper are typically a few days or up to 270 days. Although a significant amount of asset-backed commercial paper exists, most commercial paper is unsecured, with no specific collateral. Issuers of commercial paper are often required to have a backup line of credit. The short-term nature of commercial paper, along with the creditworthiness of the borrower and the backup line of credit, generally make commercial paper a low-risk investment for investors.

\section{Long-Term Debt and Equity}
Depending on a firm's particular risk objectives, it may choose to finance its working capital with longer-term securities. By their nature, these securities are generally more costly to the firm, but they can provide the firm with increased financial flexibility. Borrowing long-term capital, such as debt or equity, removes pressure on the firm to find ways to repeatedly refinance needs in the near term. In addition, most bonds only require semiannual interest payments until maturity, reducing cash needs for the firm in the interim.

\section{Long-Term Debt}
Long-term debt has a maturity of at least one year. Because of their long maturities, bonds are riskier than shorter-term notes or money market instruments from an interest rate and credit risk perspective. Hence, bond lenders (investors) and borrowers (companies) agree to bond covenants that are detailed contracts specifying the rights of the lender and restrictions on the borrower. These covenants regulate the company's use and nature of assets, restrict its ability to pay dividends, and limit the issuance of additional debt that might dilute the value of a bond.

Public debt is negotiable and approved for sale on open markets. In this context, a negotiable instrument is a written document describing the promise to pay that is transferable and can be sold to another party. Private debt can also be negotiable; however, private debt does not trade on a market and, therefore, is less liquid and more difficult for the holder to sell. While some private debt instruments, such as savings bonds and certificates of deposit, are not negotiable, private debt issued by businesses can usually be sold by one party to another.

\section{Common Equity}
Common equity, or common shares, represents ownership in a company and is considered a more permanent source of capital. Owners, or shareholders, have residual claim on the company's profits after its obligations under other contractual claims are satisfied. In some instances-more likely when debt is not available-companies may issue common equity to fund growth in working capital.

\section{EXAMPLE 2}
\section{Keown Corporation - External Financing Decision}
Keown Corp. is planning its financing for a substantial investment of C $\$ 30$ million next year. The assumptions for Keown's plan are the following:

\begin{itemize}
  \item The total investment of $C \$ 30$ million will be distributed as follows: C\$5 million in receivables, $C \$ 5$ million in inventory, and fixed capital investments of $C \$ 20$ million, including $C \$ 5$ million to replace depreciated equipment and $C \$ 15$ million of net new investments.

  \item Projections include a net income of $C \$ 10$ million, depreciation charges of $C \$ 5$ million, and dividend payments of $C \$ 4$ million.

  \item Short-term financing from accounts payable of C\$3 million is expected. The company will use receivables as collateral for another $C \$ 3$ million loan. The company will also issue a $C \$ 4$ million short-term note to a commercial bank.

  \item Any additional external financing needed can be raised from an increase in long-term bonds. If additional financing is not needed, any excess funds will be used to repurchase common shares. 1. Describe how Keown would determine its financing needs.

\end{itemize}

\section{Solution}
Keown should begin by considering its various sources of net cash to determine whether its net cash is sufficient to meet its investment needs. The facts state that the company will generate net income and will also have depreciation charges and make dividend payments. Because depreciation is a non-cash expense, it should be added to net income as a source of cash. Dividends, however, are paid from net income, are a use of cash, and reduce the net cash available for funding. Other sources of cash include the amounts generated from accounts payable, accounts receivable, and the short-term note from the bank.

\begin{enumerate}
  \setcounter{enumi}{1}
  \item How much, if any, does Keown need to issue in long-term bonds?
\end{enumerate}

A. Keown does not need to issue any bonds.

B. Keown will need to issue $\mathrm{C} \$ 4$ million of bonds.

C. Keown will need to issue $\mathrm{C} \$ 9$ million of bonds.

\section{Solution}
$\mathrm{C}$ is correct. Keown must issue $\mathrm{C} \$ 9$ million of bonds.

\begin{center}
\begin{tabular}{lc}
\hline
Source & Amount (C\$, millions) \\
\hline
Internal Financing & 11 \\
Operating cash flow (Net income + Depreciation &  \\
- Dividends) & 3 \\
Accounts payable & 3 \\
Short-Term Intermediary Financing & 4 \\
Bank loan against receivables & 21 \\
Short-term note & 21 \\
Total Sources &  \\
\hline
\end{tabular}
\end{center}

Since Keown requires $\mathrm{C} \$ 30$ million of financing and the planned sources total $\mathrm{C} \$ 21$ million, Keown will need to issue $\mathrm{C} \$ 9$ million of new bonds.

\section{WORKING CAPITAL, LIQUIDITY, AND SHORT-TERM FUNDING NEEDS}
explain expected relations between working capital, liquidity, and short-term funding needs (NEW)

Companies require sufficient resources to meet day-to-day obligations-such as paying suppliers and employees as well as meeting lease terms and other financial commitments - and to continue operations as a going concern. Successful companies seek to balance funds dedicated to current assets (most of which earn minimal returns or even decline in value as inventory becomes increasingly obsolete) against the risk of shortages in current assets negatively impacting day-to-day operations, while keeping focus on longer-term goals. Working capital requirements are often a function of a firm's particular business model. Some businesses require heavy investment in inventory and receivables, while others do not. Retail businesses-particularly those with physical ("brick and mortar") locations or significant inventory and those operating more heavily on credit (vs. cash)-require substantially more working capital to fund their day-to-day operations. In contrast, technology businesses-such as software companies or online-only businesses with large amounts of intangible assets and few physical assets-have far lower working capital needs. There are wide variations, however, in working capital needs even among firms in the same industry or segment.

Companies typically determine their required working capital investment by first identifying their optimal levels of inventory, receivables, and payables, as function of sales, and then modeling those assumptions forward for the business. In establishing what is "optimal," a company's management usually evaluates some trade-off between costs, both the cost of capital for the company and obsolescence risks with its inventory, and benefits, such as fewer inventory shortfalls (or stockouts) and more accommodative credit policies, which management believes will translate to higher sales or revenues.

After determining its working capital requirements, a firm then identifies the optimal mix of short- and long-term financing to acquire necessary current assets while providing sufficient financial flexibility in times of stress. The analyst's job is to measure the efficacy of the firm's current approach and to understand its inherent risk.

Companies take different approaches to working capital management, ranging from conservative to moderate to aggressive.

Conservative: In a conservative approach the firm holds a larger position in cash, receivables, and inventories, relative to sales. This provides the firm with the financial flexibility to manage and respond to unforeseen events that might disrupt supply chains and customer payments, such as during a regional or global crisis.

Aggressive: In an aggressive approach the firm has substantially less committed to current assets, thereby reducing the company's short-term financial flexibility in exchange for higher equity returns.

Moderate: In a moderate approach the firm holds a position somewhere between the two approaches.

\section{EXAMPLE 3}
\section{Keown Corporation - Approach to Working Capital Management}
As the CFO of Keown Corp., you would like recommendations on which approach to working capital (conservative, aggressive, moderate) makes the most sense for the business. You note that Keown has base levels of inventory, staffing, and receivables that are considered permanent current assets because they remain relatively constant during the year, while its higher levels of inventory and labor needed during the company's peak production and sales periods are considered variable current assets.

What factors should be considered and what approach should the company take?

\section{Scenario 1 (Conservative)}
Keown finances its current assets with long-term debt or equity financing.

\begin{center}
\includegraphics[max width=\textwidth]{2023_05_04_b5cfa4f1bc883752f121g-784}
\end{center}

What are the pros and cons of a conservative approach for Keown?

\begin{center}
\begin{tabular}{|c|c|}
\hline
Pros & Cons \\
\hline
$\begin{array}{l}\text { Stable, more permanent financing } \\ \text { that does not require regular refinancing; } \\ \text { reduced rollover risk }\end{array}$ & $\begin{array}{l}\text { Higher debt financing cost with an } \\ \text { upward-sloping yield curve }\end{array}$ \\
\hline
Financing costs are known upfront & High cost of equity \\
\hline
$\begin{array}{l}\text { Certainty of working capital needed } \\ \text { to purchase the necessary paddleboard } \\ \text { inventory }\end{array}$ & $\begin{array}{l}\text { Permanent financing dismisses the } \\ \text { opportunity to borrow only as needed } \\ \text { (increasing ongoing financing costs) }\end{array}$ \\
\hline
$\begin{array}{l}\text { Extended payment term reduces } \\ \text { short-term cash needs for debt service }\end{array}$ & $\begin{array}{l}\text { A longer lead time is required to } \\ \text { establish the financing position }\end{array}$ \\
\hline
$\begin{array}{l}\text { Improved flexibility during times of } \\ \text { stress, with excess liquidity in marketable } \\ \text { securities }\end{array}$ & $\begin{array}{c}\text { Long-term debt may require } \\ \text { more covenants that restrict business } \\ \text { operations }\end{array}$ \\
\hline
\end{tabular}
\end{center}

Choosing to finance Keown's current assets with fixed-rate long-term debt, given a normal (upward-sloping) yield curve, or equity will have a higher associated cost. At the same time, using longer-term financing will provide Keown with more stability in funding, even during periods of market stress, by reducing the firm's need to regularly roll its debt and reducing short-term debt service requirements.

Why might Keown choose this conservative funding approach?

Reasons Keown might choose a conservative approach include the following:

\begin{itemize}
  \item Reduced need to access the capital markets in times of stress

  \item Keown anticipates a flat to rising interest rate environment

  \item Preference for long-term cash flow stability over the risk inherent in available short-term, unsecured financing options

  \item Benefits of increased certainty and access to permanent sources of capital are perceived to offset the higher associated financing cost

  \item CEO's preference for long-term debt (over equity) to reduce shareholder dilution

\end{itemize}

\section{Scenario 2 (Aggressive)}
Keown finances the majority of its current assets with short-term debt or payables.

\begin{center}
\includegraphics[max width=\textwidth]{2023_05_04_b5cfa4f1bc883752f121g-784(1)}
\end{center}

What are the pros and cons of an aggressive approach for Keown?

\begin{center}
\begin{tabular}{|c|c|}
\hline
Pros & Cons \\
\hline
$\begin{array}{l}\text { Lower cost of financing than con- } \\ \text { servative approach, given a normal } \\ \text { upward-sloping yield curve }\end{array}$ & $\begin{array}{l}\text { Interest expense could fluctuate as } \\ \text { rates change on short-term financing }\end{array}$ \\
\hline
\end{tabular}
\end{center}

Pros

Short-term lines of credit provide the flexibility to access financing only when needed-particularly appropriate for seasonality-reducing overall interest expense

Short-term loans involve less rigorous credit analysis, as the lender has greater clarity as to the short-term operations of the firm

Flexibility to refinance if rates decline Cons

Higher levels of short-term cash may be needed to meet short-term debt maturities

Potential difficulty in rolling the short-term loans, thus increasing bankruptcy risk, particularly during times of stress

Greater reliance on trade credit (expensive financing) may be necessary if the business is unable to refinance at favorable terms

Tighter customer credit standards may be required, thereby reducing sales, if the business is unable to access the necessary financing to support credit terms to its customers

An increased reliance on short-term financing (short-term debt, accounts payable, accruals) would reduce the cost of financing relative to a conservative approach in exchange for greater risk if short-term financing sources dry up, which would make the rolling of short-term debt difficult, or even impossible, during economic downturns. Such a phenomenon did occur during the financial crisis in 2008-2009 and during the initial phases of the pandemic crisis in 2020. Why might Keown choose this aggressive funding approach?

Reasons Keown might choose an aggressive approach include the following:

\begin{itemize}
  \item Keown is confident in its ability to forecast upcoming sales and produce a detailed cash budget with a high degree of precision. Doing so would provide Keown with improved clarity and certainty as to its working capital needs.

  \item Keown anticipates a stable to falling interest rate environment.

  \item Keown anticipates shortening its cash conversion cycle, which would be achieved by reducing its account receivable and inventory period and lengthening its accounts payable period.

  \item In this case, moving to a model whereby customers pay cash at the time of order would be useful for Keown, allowing it to minimize its accounts receivable.

\end{itemize}

The challenge for Keown is that it must purchase its inventory well in advance. In this scenario, Keown is confident it can forecast near term sales and produce an accurate cash budget. Assuming this is the case, Keown would have a high degree of clarity and certainty as to its working capital needs to make the paddleboards, which requires a significant investment. Given the seasonal nature of the business, losing access to financing at the wrong time of the year could have a significantly negative impact for Keown.

\section{Scenario 3 (Moderate)}
Keown employs a moderate strategy, using short and long-term financing methods, focusing on a liability-matching approach.

\begin{center}
\includegraphics[max width=\textwidth]{2023_05_04_b5cfa4f1bc883752f121g-786}
\end{center}

In this approach, Keown uses long-term financing as a baseline for financing its permanent level of current assets, such as year-round staffing needs and demo models needed for trade shows, and uses short-term financing for variable current assets, such as increased inventory and staffing during its seasonally busy periods. What are the pros and cons of a moderate approach for Keown?

\begin{center}
\begin{tabular}{|c|c|}
\hline
Pros & Cons \\
\hline
$\begin{array}{l}\text { Lower cost of financing than conser- } \\ \text { vative approach }\end{array}$ & $\begin{array}{l}\text { Access to short-term capital may be } \\ \text { restricted when needed for inventory } \\ \text { build }\end{array}$ \\
\hline
$\begin{array}{l}\text { Flexibility to increase financing for } \\ \text { seasonal spikes while maintaining a base } \\ \text { level for ongoing needs }\end{array}$ & $\begin{array}{l}\text { Potential difficulty in rolling the } \\ \text { short-term loans, thus increasing bank- } \\ \text { ruptcy risk, particularly during times } \\ \text { of stress }\end{array}$ \\
\hline
\multirow[t]{2}{*}{$\begin{array}{l}\text { Diversifying sources of funding with } \\
\text { a more disciplined approach to balance } \\
\text { sheet management }\end{array}$} & $\begin{array}{l}\text { Greater reliance on trade credit } \\ \text { (expensive financing) may be necessary } \\ \text { if the business is unable to refinance at } \\ \text { favorable terms }\end{array}$ \\
\hline
 & $\begin{array}{l}\text { Tighter customer credit standards } \\ \text { may be required, thereby reducing } \\ \text { sales, if the business is unable to access } \\ \text { the necessary financing to support } \\ \text { credit terms to its customers }\end{array}$ \\
\hline
\end{tabular}
\end{center}

Why might Keown choose this moderate funding approach?

\section{Reasons Keown might choose a moderate approach include the following:}
\begin{itemize}
  \item Reduced risk with having to repeatedly access the equity markets for additional financing, given that Keown's core level of financing is established in accordance with its target capital structure

  \item Reduced financing costs relative to a conservative approach while taking less risk and receiving greater financial flexibility than the aggressive approach

  \item Preference to balance the benefits of lower-cost short-term financing with the increased stability and certainty of having its permanent working capital financed with long-term options

\end{itemize}

Given Keown's business, taking a moderate approach to working capital would likely be appropriate for Keown. Due to the firm's seasonality, having a baseline level of long-term funding available for inventory, receivables, and permanent staffing makes sense. Keown could then look to control financing costs by using short-term financing for its seasonal periods. Keown's approach, however, will likely vary over time and be based on the preferences of firm management at the time. For example, a more risk-averse manager would take a more conservative approach, opting for reduced risk and greater financial flexibility at the expense of higher financing costs, whereas a more aggressive manager, in relying more heavily on short-term financing solutions, would be choosing lower financing costs at the expense of greater risk and reduced financial flexibility.

One way to evaluate the financial impact of a firm's working capital approach is to use the DuPont equation where

Return on equity (ROE $)=\frac{\text { Net income }}{\text { Average shareholder's equity }}$

ROE $=($ Net profit margin $) \times($ Total asset turnover $) \times($ Leverage $)$

ROE $=\frac{\text { Net income }}{\text { Revenue }} \times \frac{\text { Revenue }}{\text { Average total assets }} \times \frac{\text { Average total assets }}{\text { Average shareholder's equity }}$

To assess the impact on returns, an analyst should focus on the total asset turnover component of the DuPont equation:

Total asset turnover $=\frac{\text { Revenue }}{\text { Average total assets }}$.

As a company reduces its investment in working capital, its total asset turnover will increase, driving higher returns on equity, all else being equal. A company can reduce its working capital, for example, by using just-in-time inventory management, allowing it to reduce its investment into inventory on hand. Similarly, by requiring faster payment from customers and accelerating cash collections, a company can reduce its working capital. To help ensure appropriate financing to meet day-to-day liabilities, companies may use daily or monthly cash budgets.

A company that can generate high levels of sales with minimal asset levels can provide outsized returns to shareholders. Examples of this are the rapid growth and strong returns of large global technology companies such as Alibaba, Apple, Google, Microsoft, and Tencent. A common characteristic of these "asset light" companies is their reduced level of investment in long-term assets, leading to higher margins and cash flow generation.

Exhibit 2 summarizes the relationship between financing requirements, costs, risks, and return on equity based on funding approach.

\section{Exhibit 2: Working Capital: Financing, Costs, and Returns}
\begin{center}
\includegraphics[max width=\textwidth]{2023_05_04_b5cfa4f1bc883752f121g-787}
\end{center}

Looking more closely at the components of working capital helps an analyst to better understand the liquidity position of the firm and its likelihood of financial distress. While cash, accounts receivable, inventory, and marketable securities have different liquidity characteristics for a firm, the commonality between them is that higher levels of each provide the firm with increased financial flexibility, reducing the risk of financial distress while simultaneously leading to lower equity returns for the firm.

Complicating the working capital decision further is the interaction between working capital policies and firm marketing efforts. Extending additional credit, or more generous credit terms, can lead to increased sales for the company. Such changes result in larger accounts receivable balances and increases in uncollectible receivables, both of which require additional financing by the firm. In this case, the increased borrowing cost must be weighed against the profit generated from the higher level of sales and the firm's ability to access incremental financing needed to support higher levels of current assets.

Effective management of liquidity is a core finance function in most firms. Even profitable companies can encounter financial difficulties by failing to ensure they have sufficient liquidity to meet current liabilities.

\section{EXAMPLE 4}
\section{A Change in Credit Policy}
\begin{enumerate}
  \item Keown Corp. is considering increasing the line of credit it offers to new customers because its sales manager believes this will lead to increased sales. What would be the expected impact on Keown's working capital if this change were made?
\end{enumerate}

A. The company would reduce its inventory levels.

B. The company would likely collect faster, reducing its receivables.

C. The company would have an increased need for working capital.

D. The company could pay its suppliers sooner, reducing its accounts payable.

E. The company would not see any impact to its net working capital needs as a result of the change.

\section{Solution}
$\mathrm{C}$ is correct. The company would likely need more working capital to support the expected increase in required inventory and accounts receivable resulting from an increase in sales.

Liquidity is the extent to which a company is able to meet its short-term obligations using cash flows and those assets that can be readily transformed into cash. Liquidity refers to the cash balances, borrowing capacity, and ability to convert other assets or extend other liabilities into cash, enabling the business to pay its short-term obligations when due and continue its operations. The liquidity of an asset can be evaluated along two dimensions:

\begin{itemize}
  \item the type of asset

  \item the speed at which the asset can be converted to cash, either by sale or financing

\end{itemize}

For companies that have large excesses of cash, liquidity is typically taken for granted, and their focus is on putting the excess liquidity to its most productive use. In the event that no productive use can be identified, shareholders will pressure companies to return this cash to owners in the form of share buybacks or dividends.

Liquidity management refers to the company's ability to generate cash when needed, at the lowest possible cost. For the most part, liquidity is associated with short-term assets and liabilities, yet longer-term assets such as marketable securities can be converted into cash to provide liquidity. In addition, companies can also renegotiate longer-term liabilities. These last two methods might be costly for a company because they tend to reduce the company's overall financial strength. When a company faces tighter financial situations, having effective liquidity management is important to ensure solvency.

The challenges of managing liquidity include developing, implementing, and maintaining a liquidity policy. To do this effectively, a company must manage its key sources of liquidity efficiently. While these sources vary by company, they generally include

\begin{itemize}
  \item primary sources of liquidity, such as cash balances

  \item secondary sources of liquidity, such as selling assets

\end{itemize}

\section{Primary Sources of Liquidity}
Primary sources of liquidity represent the most readily accessible resources available to the company. These can be cash held or near-cash securities and include the following:

\begin{itemize}
  \item Free cash flow, which is the firm's after-tax operating cash flow less planned short- and long-term investments. For a profitable firm, free cash flow provides substantial liquidity. A rapidly growing firm has less free cash flow because of the investments required to facilitate the firm's growth.

  \item Ready cash balances, which is cash available in bank accounts, resulting from the firm's payment collections, investment income, liquidation of near-cash securities (those with maturities of fewer than 90 days), and other cash flows.

  \item Short-term funds, which can include items such as trade credit, bank lines of credit, and short-term investment portfolios of the firm.

  \item Cash management, which is the company's effectiveness in its cash management system and practices and the degree of decentralization of the collections or payments processes. The more decentralized the system of collections, for example, the more likely the company will be to have cash tied up in the system and not available for use. The use of technology has enabled companies to make significant improvements in the timeliness, efficiency, and accuracy of their cash management functions.

\end{itemize}

These sources represent the liquidity that is typical for most companies and readily accessible at relatively low cost.

\section{Secondary Sources of Liquidity}
The main difference between primary and secondary sources of liquidity is that using a primary source is not likely to affect the normal operations of the company, whereas using a secondary source might result in a change in the company's financial and operating positions. Secondary sources used by companies include

\begin{itemize}
  \item negotiating debt contracts, such as relieving pressures from high interest payments or principal repayments, waiving debt covenants or suspending dividends, and negotiating contracts with customers and suppliers;

  \item liquidating assets, which depends on the degree to which short-term and/or long-term assets can be liquidated and converted into cash without substantial loss in value; and

  \item filing for bankruptcy protection and reorganization.

\end{itemize}

The use of secondary sources might signal a company's deteriorating financial health and provide liquidity at a high price-the cost of giving up a company asset to produce emergency cash. This last source, reorganization through bankruptcy, can also be considered a liquidity tool because a company under bankruptcy protection that generates operating cash will be liquid and generally able to continue business operations until a restructuring has been devised and approved. However, this option is likely to be at a significant cost, or disadvantage, to existing debt and equity holders, the company's employees, and other stakeholders.

Example 5 shows the net proceeds from the primary and secondary sources of liquidity for Keown Corp. in a liquidity crisis. It also shows the liquidation costs incurred by the company when these sources are used to raise funds. These costs can include the fees and commissions involved with the asset sale as well as any discount in asset value due to liquidity issues.

\section{EXAMPLE 5}
\section{Keown Corporation: Estimating Costs of Liquidity}
Keown Corp. is having a liquidity crisis. You have identified four potential actions that Keown could take to raise funds. Your estimates of fair value for Keown's assets and liquidation costs are shown below.

\begin{center}
\begin{tabular}{lcc}
\hline
\multicolumn{1}{c}{Source of Funds} & $\begin{array}{c}\text { Fair Value (C\$, } \\ \text { millions) }\end{array}$ & $\begin{array}{l}\text { Liquidation Costs (\%) }\end{array}$ \\
\hline
Sell short-term marketable securities & 10 & 0 \\
Sell select inventories and receivables & 20 & 10 \\
Sell excess real estate property & 50 & 15 \\
Sell a subsidiary of the firm & 30 & 20 \\
\hline
\end{tabular}
\end{center}

The liquidation costs include the fees and commissions of selling an asset as well as any reduction in the value of the asset because it is an illiquid asset being sold quickly. In this case, liquidation costs for marketable securities are rounded to $0 \%$.

\begin{enumerate}
  \item Net of liquidation costs, how much liquidity can Keown raise if all four sources of funds are used, and what are the total liquidation costs incurred by Keown? In local currency, these amounts are:
\end{enumerate}

A. 110 million, 9.5 million

B. 94.5 million, 15.5 million

\section{125.5 million, 15.5 million}
\section{Solution}
The costs and net funds raised are summarized in this table:

\begin{center}
\begin{tabular}{|c|c|c|c|c|}
\hline
\multirow[b]{2}{*}{Source of Funds} & \multirow{2}{*}{$\begin{array}{l}\text { Fair Value } \\
\text { (C\$, } \\
\text { millions) }\end{array}$} & \multicolumn{2}{|c|}{Liquidation Costs} & \multirow{2}{*}{$\begin{array}{c}\text { Net } \\
\text { Proceeds } \\
\text { (C\$, } \\
\text { millions) }\end{array}$} \\
\hline
 &  & $\%$ & (C\$, millions) \\
\hline
Marketable securities & 10 & 0 & 0 & 10 \\
\hline
$\begin{array}{l}\text { Inventories and } \\ \text { receivables }\end{array}$ & 20 & 10 & 2 & 18 \\
\hline
Real estate property & 50 & 15 & 7.5 & 42.5 \\
\hline
Subsidiary of the firm & 30 & 20 & 6 & 24 \\
\hline
Total &  &  & 15.5 & 94.5 \\
\hline
\end{tabular}
\end{center}

B is correct. Shown in the table above in local currency terms, the total net proceeds are $\mathrm{C} \$ 94.5$ million, and the total liquidation costs incurred are C\$15.5 million.

\section{Drags and Pulls on Liquidity}
A company's cash flow transactions - that is, cash receipts and disbursements-have significant effects on its liquidity position. These effects are known as drags and pulls on liquidity. A drag on liquidity is when receipts (inflows) lag, creating pressure from the decreased available funds; a pull on liquidity is when disbursements (outflows) are paid too quickly by the company or trade credit availability is limited, requiring companies to expend funds before they receive funds from sales that could cover the liability.

Major drags on receipts involve pressures from credit management and deterioration in other assets and include the following:

\begin{itemize}
  \item Uncollected receivables. The longer these are outstanding, the greater the risk that they will not be collected at all. They are indicated by the large number of days of receivables and high levels of bad debt expenses.

  \item Obsolete inventory. If inventory stands unused for long periods, it might be an indication that it is no longer usable.

  \item Tight credit. When economic conditions make capital scarcer, short-term debt becomes more expensive to access.

\end{itemize}

In many cases, drags can be reduced by stricter enforcement of credit and collection practices, but this can also lead to lower sales. (Customers may be unwilling or unable to make a purchase if cash payment is required, for example.)

Managing cash outflows is as important as managing the inflows. If suppliers and other vendors who offer credit terms perceive the company to be in a weakened financial position or are unfamiliar with the company, they might restrict payment terms so much that the company's liquidity reserves are stretched thin. Major pulls on payments include the following:

\begin{itemize}
  \item Making payments early. By paying vendors, employees, or others before the due dates, companies forgo the use of those funds. Effective payment management means not making early payments. - Reduced credit limits. If a company has a history of making late payments, suppliers might cut the amount of credit they will allow to be outstanding at any time.

  \item Limits on short-term lines of credit. If a company's bank reduces the line of credit it offers the company, a liquidity squeeze might result. Credit line restrictions can be government mandated, market related, or simply company specific.

  \item Low liquidity positions. Many companies face chronic liquidity shortages, often because of their industry or from their weaker financial position. This risk is enhanced when the firm takes an aggressive approach to working capital management.

\end{itemize}

In Example 6, you are an analyst trying to identify changes that are affecting Keown's liquidity position.

\section{EXAMPLE 6}
\section{Drags and Pulls on Liquidity}
Keown Corp. is experiencing liquidity challenges. Several things might be contributing to this. Three notable changes have been suggested as drags or pulls on liquidity:

\begin{enumerate}
  \item The increasing days in receivables is a drag on liquidity.

  \item Lower inventory turnover is a drag on liquidity.

  \item Increase in credit limits by lenders is a pull on liquidity.

  \item Which of these does not contribute to the firm's liquidity issue?

\end{enumerate}

A. The change in days in receivables

B. The change in inventory turnover

C. The change in credit limits

\section{Solution}
$\mathrm{C}$ is correct. The increase in credit limits is not a pull on liquidity but is in fact the opposite: it provides liquidity.

\section{MEASURING LIQUIDITY}
compare a company's liquidity position with that of peers

Liquidity contributes to a company's creditworthiness, and the latter allows the company to obtain lower borrowing costs and better trade credit terms, giving the company greater flexibility and enabling it to exploit profitable opportunities.

The less liquid the company, the greater the risk it will experience financial distress or, in the extreme, insolvency or bankruptcy. Because debt obligations are paid with cash, the company's cash flows ultimately determine solvency. Immediate sources of funds for paying bills are cash on hand, proceeds from the sale of marketable securities, and the collection of accounts receivable. Additional liquidity comes from inventory that can be sold and thus converted into cash, either directly through cash sales or indirectly through credit sales (i.e., accounts receivable).

At some point, however, a company might have too much invested in low- and non-earning assets. Cash, marketable securities, accounts receivable, and inventory represent a company's liquidity position, but these investments are low-earning relative to the long-term, capital investment opportunities that companies might have available.

Various financial ratios can be used to assess a company's liquidity as well as its management of assets over time. Here we look at a number of liquidity and activity ratios in more detail, which are summarized in Exhibit 3.

\section{Exhibit 3: Ratios Used for Assessing Company Liquidity}
\section{Liquidity Ratios}
Current ratio $=\frac{\text { Current assets }}{\text { Current liabilities }}$ Quick ratio $=\frac{\text { Cash }+ \text { Short-term marketable instruments }+ \text { Receivables }}{\text { Current liabilities }}$

Cash ratio $=\frac{\text { Cash }+ \text { Short-term marketable instruments }}{\text { Current liabilities }}$

\section{Activity Ratios}
Accounts receivable turnover $=\frac{\text { Credit sales }}{\text { Average receivables }}$

\includegraphics[max width=\textwidth, center]{2023_05_04_b5cfa4f1bc883752f121g-793}
$=\frac{\text { Average accounts receivable }}{\text { Sales on credit } / 365}$

Number of days of inventory ("Days inventory outstanding") $=\frac{\text { Average inventory }}{\text { Average day's cost of goods }}$ $=\frac{\text { Average inventory }}{\text { Cost of goods sold } / 365}$

Number of days of payables ("Days payable outstanding") $=\frac{\text { Average accounts payable }}{\text { Average day's purchases }}$ $=\frac{\text { Average accounts payable }}{\text { Purchases } / 365}$

Cash conversion cycle

$=$ Days of inventory + Days of receivables - Days of payables

We calculate liquidity ratios to measure a company's ability to meet short-term obligations to creditors as they mature or come due. This form of liquidity analysis focuses on the relationship between current assets and current liabilities and the rapidity with which receivables and inventory can be converted into cash during normal business operations. The levels of these ratios, and their trends, or changes over time, in addition to comparisons with competitors or the industry, are used to judge a firm's liquidity position. In addition to looking at the relationships among these balance sheet accounts, we can also estimate activity ratios, which measure how well key current assets and working capital are managed over time. These key activity ratios use information from the income statement and the balance sheet to help tell the story of how well a company is managing its liquid assets.

Some of the major applications of this type of analysis include performance evaluation, monitoring, creditworthiness assessment, and financial projections. But ratios are useful only when they can be compared. This should be done in two ways:

\begin{itemize}
  \item comparisons over time for the same company, and

  \item comparisons over time for the company compared with its peer group. Peer groups can include competitors from the same industry as well as other companies of comparable size with comparable financial situations.

\end{itemize}

Consider Daimler AG, a producer of cars, trucks, and vans. We can see the change in the company's current, quick, and cash ratios over a decade (2010-2019) in Panel A of Exhibit 4. Here, we see that the current ratio and the quick ratio increased over the time period. However, the cash ratio did not show the upward trend that the quick and current ratios exhibited over the decade.

We can see what is driving these trends in the calculation of the cash conversion cycle in Panel B of the exhibit. Over the 10 years, the cash conversion cycle increased by more than 40 days. The slight increase in days of payables outstanding would have decreased the cash conversion cycle slightly. The days of inventory increased by approximately nine days, and the days of receivables increased by approximately 37 days. Although the increase in inventory certainly puts a demand on liquidity, for Daimler the impact of the increase in receivables was more dramatic. Over the decade, however, based on the liquidity and activity ratios, this company seemed to have good control of its liquidity position.

Exhibit 4: Liquidity Analysis of Daimler AG, 10 Years Ending December 2019

\begin{center}
\begin{tabular}{|c|c|c|c|c|c|c|c|c|c|c|}
\hline
 & 2019 & 2018 & 2017 & 2016 & 2015 & 2014 & 2013 & 2012 & 2011 & 2010 \\
\hline
\multicolumn{11}{|c|}{Panel A: Current, Quick, and Cash Ratios, December 2010-2019} \\
\hline
Current ratio & 1.21 & 1.24 & 1.22 & 1.21 & 1.19 & 1.15 & 1.19 & 1.15 & 1.11 & 1.07 \\
\hline
Quick ratio & 0.93 & 0.94 & 0.93 & 0.91 & 0.88 & 0.84 & 0.90 & 0.85 & 0.80 & 0.80 \\
\hline
Cash ratio & 0.25 & 0.25 & 0.26 & 0.25 & 0.23 & 0.23 & 0.30 & 0.26 & 0.20 & 0.25 \\
\hline
\multicolumn{11}{|c|}{Panel B: Days of Inventory, Receivables, Payables, and Cash Conversion Cycle, December 2010-2019} \\
\hline
Days of inventory on hand & 75.2 & 74.9 & 71.6 & 73.9 & 69.1 & 68.5 & 69.2 & 71.5 & 71.2 & 66.6 \\
\hline
+ Days of receivables & 136.8 & 127.9 & 118.8 & 118.4 & 104.6 & 102.6 & 105.5 & 104.6 & 104.2 & 100.0 \\
\hline
$\begin{array}{l}\text { - Days of payables } \\ \text { outstanding }\end{array}$ & 34.1 & 35.2 & 33.6 & 32.8 & 31.3 & 33.4 & 35.5 & 37.4 & 37.5 & 31.6 \\
\hline
$=$ Cash conversion cycle & 177.9 & 167.6 & 156.8 & 159.5 & 142.4 & 137.7 & 139.2 & 138.7 & 137.9 & 135.0 \\
\hline
\end{tabular}
\end{center}

Now consider Walmart, Target Corporation, Kohl's Corporation, and Costco Wholesale Corporation. Selected ratios for these large US discount retailers are shown in Exhibit 5. The data are from the fiscal year ending January 2020, except for Costco, whose fiscal year ended in August 2019.

We see some differences among these four competitors. These differences can be explained, in part, by the retailers' different product mixes (e.g., Walmart and Costco have more sales from grocery lines than the other two) as well as by their different inventory management systems and different inventory suppliers. None of the four firms invests heavily in accounts receivable, and customers generally pay with credit cards. The different need for liquidity can also be explained, in part, by the companies' different operating cycles.

The most striking difference is for Kohl's. Walmart, Target, and Costco have investments in inventory that are largely matched and paid for by accounts payable. Kohl's, notably, has the highest investment in inventory, which is not financed by payables. Because it includes more of such items as clothing, Kohl's inventory is also more seasonal than that of the other companies. Hence, Kohl's has the highest current ratio because much of its inventory must be financed by non-current liabilities.

\section{Exhibit 5: Liquidity Ratios among Discount Retailers}
\begin{center}
\begin{tabular}{lcccc}
\hline
Ratio for January 2020 Fiscal Year & Walmart & Target & Kohl's & Costco \\
\hline
Current ratio & 0.79 & 0.89 & 1.68 & 1.01 \\
Quick ratio & 0.22 & 0.27 & 0.40 & 0.52 \\
Days of inventory on hand & 41.0 & 59.1 & 98.0 & 30.8 \\
Days of receivables & 4.4 & 4.5 & 0.4 & 3.8 \\
Days of payables outstanding & 43.5 & 56.2 & 33.3 & 31.4 \\
Cash conversion cycle & 1.9 & 7.4 & 65.1 & 3.2 \\
\hline
\end{tabular}
\end{center}

\section{EXAMPLE 7}
\section{Measuring Liquidity}
\begin{enumerate}
  \item Given the following ratios, how well has Company $\mathrm{X}$ been managing its liquidity?
\end{enumerate}

Average for the Previous Five Fiscal

Current Fiscal Year Years

\begin{center}
\begin{tabular}{|c|c|c|c|c|}
\hline
Ratio & Company X & Industry & Company $x$ & Industry \\
\hline
Current ratio & 1.9 & 2.5 & 1.1 & 2.3 \\
\hline
Quick ratio & 0.7 & 1.0 & 0.4 & 0.9 \\
\hline
Days of receivables & 39.0 & 34.0 & 44.0 & 32.5 \\
\hline
Days of inventory on hand & 41.0 & 30.3 & 45.0 & 27.4 \\
\hline
Days of payables outstanding & 34.3 & 36.0 & 29.4 & 35.5 \\
\hline
\end{tabular}
\end{center}

\section{Solution}
The ratios should be compared in two ways: for Company $\mathrm{X}$ over time (which would typically be examined) and for Company X over time vis-à-vis the trend in the industry. In all ratios shown here, the current year shows improvement over the previous years in terms of increased liquidity. In each case, however, Company $\mathrm{X}$ remains behind the industry average in terms of liquidity. A brief snapshot such as this example could be the starting point in assessing management's ability to deliver future improvement and reach or beat the industry standards.

\section{EVALUATING SHORT-TERM FINANCING CHOICES}
evaluate short-term funding choices available to a company

Companies may find themselves with liquidity issues if they fail to sufficiently explore their available options or take advantage of cost savings that some forms of financing, or borrowing, offer. If a company lacks a sound short-term financing strategy, it may find itself stuck in an uneconomical situation or, in the extreme, facing a crisis in which it cannot borrow from any source.

To avoid this, companies seek to implement a short-term financing strategy that achieves a number of objectives:

\begin{itemize}
  \item Ensure the company has sufficient funding capacity to handle peak cash needs.

  \item Maintain sufficient and diversified sources of credit to fund ongoing cash needs. While many companies use one alternative primarily, although often with more than one provider, , a company should ideally ensure it has adequate alternatives and is not overly reliant on one lender or form of lending, particularly if the amount of their borrowing is large.

  \item Ensure the rates paid for financing are competitive.

  \item Ensure both implicit and explicit funding costs are considered in the company's effective cost of borrowing. This becomes more challenging for complex instruments such as convertibles and other derivatives.

\end{itemize}

In addition, several other factors, such as the following, influence a company's short-term borrowing strategy:

\begin{itemize}
  \item Size and creditworthiness. A company's size can dictate the financing options available to it. Larger companies can take advantage of economies of scale to access commercial paper, banker's acceptances, and so on. The lender's size is also an important criterion, because larger banks have higher house or legal lending limits. The company's creditworthiness will determine the rate it will pay, the compensation amount, and even whether the loan will be approved by the lender.

  \item Legal and regulatory considerations. Some countries impose constraints on how much a company can borrow and the terms under which it can borrow. Such constraints are usually greater for companies operating in developed countries with well-defined legal systems than for those operating in countries with emerging economies. In developed countries, some industries are highly regulated. Companies in these industries, such as utilities and banks, might be restricted in how much they can borrow and the kind of borrowing they can engage in. Banks, additionally, are required to have minimum levels of equity capital.

  \item Asset nature. Depending on their business model, companies may have assets considered attractive as collateral for secured loans.

  \item Flexibility of financing options. Flexibility enables a company to manage its debt maturities more efficiently. Cash budgeting exercises can help companies avoid issues when tight credit markets restrict a firm's ability to roll a particular maturity. In addition, the proper spacing of debt maturities through effective maturity management can also be critical for companies.

\end{itemize}

\section{EXAMPLE 8}
\section{Evaluating Short-Term Choices}
\begin{enumerate}
  \item When contemplating choices for short-term financing, which of the following should a company consider?
\end{enumerate}

A. The cost of the funds borrowed

B. The flexibility offered by the source

C. The ease with which the funds can be accessed

D. Any legal or regulatory constraints that might favor one source over another

E. All of the above

\section{Solution}
The correct answer is $\mathrm{E}$. The cost of funds for a company is the most obvious item to consider, but it may choose to borrow at a slightly higher cost after taking all the other items into consideration.

\section{EXAMPLE 9}
\section{Meeting Short-Term Financing Need}
\begin{enumerate}
  \item Keown Corp. has accounts payable of $\mathrm{C} \$ 2$ million with terms of $2 / 10$, net 30. Accounts receivable also stands at $\mathrm{C} \$ 2$ million. In addition, the company has $\mathrm{C} \$ 5$ million in marketable securities. Keown has a short-term need of $\mathrm{C} \$ 200,000$ to meet payroll. Which of the following options makes the most sense for raising the $\mathrm{C} \$ 200,000$ ?
\end{enumerate}

A. The company should issue long-term debt.

B. The company should issue common stock.

C. The company should delay paying accounts payable and forgo the $2 \%$ discount.

D. The company should sell some of its accounts receivable to a factor at a $10 \%$ discount.

E. The company should sell some of its marketable securities at a $0.5 \%$ brokerage cost.

\section{Solution}
$A$ and B would not be appropriate for raising $C \$ 200,000$ for a short-term need. These options take time to arrange, and they are more appropriate for long-term capital needs and for much larger financing amounts.

C, D, and $\mathrm{E}$ are all appropriate options for meeting short-term financing needs. However, $\mathrm{C}$ and $\mathrm{D}$ are costly.

The options for raising $\mathrm{C} \$ 200,000$ are summarized in this table:

\begin{center}
\begin{tabular}{lccc}
\hline
 &  & \multicolumn{2}{c}{Liquidation Costs} \\
\cline { 2 - 4 }
Source of Funds & Action & $\%$ & $\mathbf{C \$}$ \\
\hline
$\begin{array}{c}\text { C. Accounts payable (2/10, } \\ \text { net 30) }\end{array}$ & $\begin{array}{c}\text { Delay } C \$ 200,000 \text { in } \\ \text { payment and forgo } 2 \% \\ \text { discount }\end{array}$ & 2.0 & 4,000 \\
\hline
\end{tabular}
\end{center}

\begin{center}
\begin{tabular}{lccc}
\hline
 &  & \multicolumn{2}{c}{Liquidation Costs} \\
\cline { 3 - 4 }
Source of Funds & Action & $\%$ & C\$ \\
\hline
D. Accounts receivable & $\begin{array}{c}\text { Sell } C \$ 222,222 \text { in value } \\ \text { at } 10 \% \text { discount to raise }\end{array}$ & 10.0 & 22,222 \\
E. Marketable securities & Sell $\mathrm{C} \$ 200,000$ in value & 0.0 & 1,000 \\
\hline
\end{tabular}
\end{center}

Choosing C means forgoing a $2 \%$ discount, which on $\mathrm{C} \$ 200,000$ amounts to a cost of $C \$ 4,000$. To net $C \$ 200,000$ using option $D$, the company would have to sell $\mathrm{C} \$ 222,222$ of accounts receivable to a factor, representing a cost of $\mathrm{C} \$ 22,222$. E appears to be the best choice. Marketable securities are liquid and can be easily sold for market value, less the relatively minor brokerage cost of $\mathrm{C} \$ 1,000$.

\section{SUMMARY}
Here we considered key aspects of short-term financial management: the choices available to fund a company's working capital needs and effective liquidity management. Both are critical in ensuring a company's day-to-day operations and ability to remain in business.

Key points of coverage included the following:

\begin{itemize}
  \item Internal and external sources available to finance working capital needs and considerations in their selection

  \item Working capital approaches, their considerations, and their impact on the funding needs of the company

  \item Primary and secondary sources of liquidity and factors that can enhance a company's liquidity position

  \item The evaluation of a company's liquidity position and comparison to peers

  \item The evaluation of short-term financing choices based on their characteristics and effective costs

\end{itemize}

\section{PRACTICE PROBLEMS}
\begin{enumerate}
  \item Two analysts are discussing the costs of external financing sources. The first states that the company's bonds have a known interest rate but that the interest rate on accounts payable and the interest rate on equity financing are not specified. They are implicitly zero. Upon hearing this, the second analyst advocates financing the firm with greater amounts of accounts payable and common shareholders equity. Is the second analyst correct in his analysis?
\end{enumerate}

A. He is correct in his analysis of accounts payable only.

B. He is correct in his analysis of common equity financing only.

C. He is not correct in his analysis of either accounts payable or equity financing.

\begin{enumerate}
  \setcounter{enumi}{1}
  \item A company has arranged a $\$ 20$ million line of credit with a bank, allowing the company the flexibility to borrow and repay any amount of funds as long as the balance does not exceed the line of credit. These arrangements are called:
A. convertibles.
B. factoring.
C. revolvers.

  \item The SOA Company needs to raise 75 million, in local currency, for substantial new investments next year. Specific details, all in local currency, are as follows:

\end{enumerate}

\begin{itemize}
  \item Investments of 10 million in receivables and 15 million in inventory will be made. Fixed capital investments of 50 million, including 10 million to replace depreciated equipment and 40 million of net new investments, will also be made.

  \item Net income is expected to be 30 million, and dividend payments will be 12 million. Depreciation charges will be 10 million.

  \item Short-term financing from accounts payable of 6 million is expected. The firm will use receivables as collateral for an 8 million loan. The firm will also issue a 14 million short-term note to a commercial bank.

  \item Any additional external financing needed can be raised from an increase in long-term bonds. If additional financing is not needed, any excess funds will be used to repurchase common shares.

\end{itemize}

What additional financing does SOA require?

A. SOA will need to issue 19 million of bonds.

B. SOA will need to issue 26 million of bonds.

C. SOA can repurchase 2 million of common shares.

\begin{enumerate}
  \setcounter{enumi}{3}
  \item XY1 Corporation's CFO has decided to pursue a moderate approach to funding the firm's working capital. Which of the following methods would best fit that particular approach?
\end{enumerate}

A. Finance permanent and variable current assets with long-term financing.

B. Finance permanent and variable current assets with short-term financing. C. Finance permanent current assets with long-term financing and variable current assets with short-term financing.

\begin{enumerate}
  \setcounter{enumi}{4}
  \item Kwam Solutions must raise $€ 120$ million. Kwam has two primary sources of liquidity: $€ 60$ million of marketable securities (which can be sold with minimal liquidation/brokerage costs) and $€ 30$ million of bonds (which can be sold with 3\% liquidation costs). Kwam can sell some or all of either of these portfolios. Kwam has a secondary source of liquidity, which would be to sell a large piece of real estate valued at $€ 70$ million (which would incur $10 \%$ liquidation costs). If Kwam sells the real estate, it must be sold entirely. (A fractional sale is not possible.) What is the lowest cost strategy for raising the needed $€ 120$ million?
\end{enumerate}

A. Sell $€ 60$ million of the marketable securities, $€ 30$ million of the bonds, and $€ 34.3$ million of the real estate property.

B. Sell the real estate property and $€ 50$ million of the marketable securities.

C. Sell the real estate property and $€ 57$ million of the marketable securities.

\begin{enumerate}
  \setcounter{enumi}{5}
  \item A company increasing its credit terms for customers from $1 / 10$, net 30 to $1 / 10$, net 60 will most likely experience:
\end{enumerate}

A. an increase in cash on hand.

B. a lower level of uncollectible accounts.

C. an increase in the average collection period.

\begin{enumerate}
  \setcounter{enumi}{6}
  \item Paloma Villarreal has received three suggestions from her staff about how to address her firm's liquidity problems.
\end{enumerate}

Suggestion 1. Reduce the firm's inventory turnover rate.

Suggestion 2. Reduce the average collection period on accounts receivable.

Suggestion 3. Accelerate the payments on accounts payable by paying invoices before their due dates.

Which suggestion should Villarreal employ to improve the firm's liquidity position?
A. Suggestion 1
B. Suggestion 2
C. Suggestion 3

\begin{enumerate}
  \setcounter{enumi}{7}
  \item Selected liquidity ratios for three firms in the leisure products industry are given in the table below. The most recent fiscal year ratio is shown along with the average of the previous five years.
\end{enumerate}

\begin{center}
\begin{tabular}{lccccccc}
\hline
 & \multicolumn{2}{c}{Company H} & \multicolumn{2}{c}{Company J} & \multicolumn{2}{c}{Company S} \\
\hline
 & $\begin{array}{cccccc}\text { Most } \\ \text { recent }\end{array}$ & $\begin{array}{c}\text { Five-year } \\ \text { average }\end{array}$ & $\begin{array}{c}\text { Most } \\ \text { recent }\end{array}$ & $\begin{array}{c}\text { Five-year } \\ \text { average }\end{array}$ & $\begin{array}{c}\text { Most } \\ \text { recent }\end{array}$ & $\begin{array}{c}\text { Five-year } \\ \text { average }\end{array}$ \\
\hline
Current ratio & 5.37 & 2.51 & 3.67 & 3.04 & 3.05 & 2.53 \\
Quick ratio & 5.01 & 2.19 & 2.60 & 2.01 & 1.78 & 1.44 \\
Cash ratio & 3.66 & 0.97 & 1.96 & 1.28 & 0.96 & 0.67 \\
\hline
\end{tabular}
\end{center}

Relative to its peers and relative to its own prior performance, which company is in the most liquid position?
A. Company $\mathrm{H}$
B. Company J
C. Company $\mathrm{S}$

\begin{enumerate}
  \setcounter{enumi}{8}
  \item An analyst is examining the cash conversion cycles and their components for three companies that she covers in the leisure products industry. She believes that changes in the investments in these working capital accounts can reveal liquidity stresses on a company.
\end{enumerate}

\begin{center}
\begin{tabular}{|c|c|c|c|c|c|c|}
\hline
 & 2021 & 2020 & 2019 & 2018 & 2017 & 2016 \\
\hline
\multicolumn{7}{|l|}{Company $\mathrm{H}$} \\
\hline
Days of inventory on hand & 68.4 & 70.5 & 60 & 57.8 & 59.8 & 59.8 \\
\hline
+ Days of receivables & 101.8 & 103.4 & 95.6 & 92.4 & 94.7 & 93.3 \\
\hline
$\begin{array}{l}\text { - Days of payables } \\ \text { outstanding }\end{array}$ & 52.1 & 54.6 & 48 & 41.9 & 36.8 & 35.9 \\
\hline
$=$ Cash conversion cycle & 118.1 & 119.3 & 107.6 & 108.3 & 117.7 & 117.2 \\
\hline
\multicolumn{7}{|l|}{Company J} \\
\hline
Days of inventory on hand & 105.6 & 101.4 & 96.3 & 105.2 & 103.2 & 101.4 \\
\hline
+ Days of receivables & 27.7 & 29.4 & 32.9 & 36.3 & 37.8 & 38 \\
\hline
$\begin{array}{l}\text { - Days of payables } \\ \text { outstanding }\end{array}$ & 36.6 & 38.5 & 35.3 & 39.3 & 37.8 & 40.2 \\
\hline
$=$ Cash conversion cycle & 96.7 & 92.3 & 93.9 & 102.2 & 103.2 & 99.2 \\
\hline
\multicolumn{7}{|l|}{Company $S$} \\
\hline
Days of inventory on hand & 135.8 & 131 & 118.9 & 69.2 & 63.4 & 81.7 \\
\hline
+ Days of receivables & 49.1 & 42.5 & 54.2 & 36.2 & 29.1 & 38.3 \\
\hline
$\begin{array}{l}\text { - Days of payables } \\ \text { outstanding }\end{array}$ & 30.9 & 27.9 & 34.6 & 29.8 & 31.8 & 35.9 \\
\hline
$=$ Cash conversion cycle & 154.0 & 145.6 & 138.5 & 75.6 & 60.7 & 84.1 \\
\hline
\end{tabular}
\end{center}

Which company's operating cycle appears to have caused the most liquidity stress?
A. Company H's
B. Company J's
C. Company S's

\begin{enumerate}
  \setcounter{enumi}{9}
  \item Which of the following are considered internal sources of financing for a company's working capital management?
A. Committed and uncommitted lines of credit
B. Accounts receivable and inventory
C. Accounts payable and accruals
\end{enumerate}

\section{SOLUTIONS}
\begin{enumerate}
  \item C is correct. Although accounts payable do not charge an explicit interest rate, the cost of accounts payable is reflected in the costs of the services or products purchased and in the costs of any discounts not taken. Accounts payable can have a very high implicit cost. Similarly, equity financing is not free. A required return is expected on shareholder financing just as on any other form of financing.

  \item $\mathrm{C}$ is correct. A revolver is a short-term borrowing facility in which a bank allows the firm to borrow and repay loans during the life of the line of credit.

  \item A is correct. SOA must issue 19 million of bonds.

\end{enumerate}

\begin{center}
\begin{tabular}{lc}
\hline
Source & $\begin{array}{c}\text { Amount } \\ \text { (local, millions) }\end{array}$ \\
\hline
Accounts payable & 6 \\
Bank loan against receivables & 8 \\
Short-term note & 14 \\
Net income + depreciation - dividends & 28 \\
Total sources & 56 \\
\hline
\end{tabular}
\end{center}

The firm requires 75 million of financing in local currency terms. Given that the planned sources (before bond financing or repurchases) total 56 million, SOA will need to issue 19 million of new bonds.

\begin{enumerate}
  \setcounter{enumi}{3}
  \item C is correct. In a moderate approach, XY1 would attempt to match the duration of the assets with the liabilities. This would allow the company to use long-term financing for permanent working capital needs while at the same time looking to minimize interest expense through the use of more flexible short-term financing on an as-needed basis.

  \item C is correct. Kwam must sell the entire real estate property because the two primary sources (marketable securities and bonds) will not raise the needed $€ 120$ million. A is incorrect because it assumes a fractional real estate sale. The real estate sale will raise a net of $€ 63$ million ( $€ 70$ million minus $10 \%$ liquidation expenses). To raise the rest of the funds needed ( $€ 120$ million $-€ 63$ million $=€ 57$ million), Kwam can sell $€ 57$ million of marketable securities, which have minimal liquidation/brokerage costs.

  \item C is correct. A longer average collection period will certainly occur. Higher cash balances and a lower level of uncollectible accounts will not occur.

  \item B is correct. Reducing the average collection period would speed up receipts and improve the firm's liquidity position. The other two suggestions would worsen the firm's liquidity position.

  \item A is correct. Relative to peers, Company $\mathrm{H}$ has the highest set of ratios. Relative to historical average ratios, Company H's recent ratios show the greatest increases. The cash ratio is the most relevant for judging liquidity, and Company H's cash ratio is quite high.

  \item C is correct. Company S's cash conversion cycle nearly doubled over recent years, while the cash conversion cycles for Companies $\mathrm{H}$ and $\mathrm{J}$ are nearly unchanged. The days of inventory on hand and days of receivables both increased substan- tially for Company S, and its days of payables outstanding decreased very slightly. The net effect was the large increase in the cash conversion cycle. Although changes occurred in the components of the cash conversion cycles for Companies $\mathrm{H}$ and $\mathrm{J}$, the net effect on their cash conversion cycles was small.

  \item $\mathrm{C}$ is correct. Accounts payable and accruals are internal and a source of cash as they are payments not yet made to suppliers, employees, or other related parties. Lines of credit are external sources of financing. Accounts receivable and inventory are internal uses of cash since a company must access financing to purchase inventory and lend to its customers.

\end{enumerate}

\end{document}